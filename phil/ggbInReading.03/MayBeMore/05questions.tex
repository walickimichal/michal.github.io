\chapter*{Questions and additions}
\addcontentsline{toc}{chapter}{Questions and additions}
\setcounter{PARAGRAPH}{0}

\section*{Notes}
\addcontentsline{toc}{section}{Notes/additions}

\ad{Quietism}
We saw what happened after Eckhart: Suso++... This kind of thinking may, indeed,
invite to quietism and even when it does not invite, it may lead to it...

\ad{distinction but no essence}
We can imagine a \co{distinction} without \co{ipseity} when we 
encounter something for the very first time: we do not re-cognize it, 
we do not know, nor even feel, what this thing, this feeling, this 
something might be. But we know that it is {\em something}, only that we 
do not know what, we encounter it as something different from everything else.

\ad{holiness, concepts}
Before you meet a holy man, you may have a very confused, if any, 
concept of holiness. But it is no goal to have all the concepts 
correct and ready in advance. The goal is only this: {\em when} you 
meet a holy man, do not let your concepts (or misconceptions) prevent 
you from recognizing his holiness, from admitting it. (Actually, concepts, no 
matter what the hermeneutics tell, make it harder to accept the 
testimony of new experiences.)


\ad{actuality confused with eternity}
One of the main effects common to all the drugs is that they reduce 
the world to the pure \herenow, they narrow the horison of perception 
and consideration to this very moment, without tomorrow, without 
yonder, to pure actuality so seductively resembling the mystical 
presenceof eternity.

\ad{Analytical philosophy}
says \wo{this is bubbling, incomprehensible rubbish}. Well, if I read 
only one page from Kant, it is incomprehensible, but it helps if I 
read more and think about it. When I read Bible for the first time I 
understood much less than when I read it again. But so was it when I 
read Carnap, and when I studied mathematics!

\ad{expression vs. expressed}
The question about truth has been all the time confused with the 
question about the formulation of truth; the absolute 
with the expression of the absolute. And finding no unique, 
unequivocal, verifiable formulation, one decided to abandon the quest.

There is a deep lack of respect, both for the truth and for the other 
in this. The truth need not -- {\em can not} -- be expressed in a 
single way like a mathematical definition. (It is abundant, 
overflowing, non-actual.) And the other can understand the intention beyond 
the partial, even misleading expressions.


\ad{reflection: USED}
The art of reflection is not to jump to a higher level of 
self-consciousness and observe everything, oneself included, from 
above and outside. On the contrary! It is to stay inside and from 
within to record the events, feelings passing by without externalizing 
them. It is to realize that there is nothing outside, that everything 
is and must be seen only from within. The art of reflection is not to 
oppose immediacy but to preserve it -- in spite of reflection.


\ad{Community-personality}
There is a bunch of thinkers who tend to consider \thi{the state of the
  culture}, \thi{the state of the civilisation} and to design their philosophy
as if in response to the observed crisis. Admirable as such pieces of sociology or
might be, they pass often for philosophy -- the more
warmly welcome and embraced, the more actualities they manage to address
explicitly.

The hero of this mixture of sociological analysis of the crises with the
pretensions to human, personal relevance is Nietzsche -- the depths of his
socio- and meta-socio- observations is the more striking, the easier one
recognizes in them the element of down to earth psychology. Now, we do not want
to criticise the analyses of culture and civilisation rooted in the description
of the involved personal values and characters. But we would limit them to
purely socio-, or meta-socio- dimension and delegate, perhaps, to some corner of
socio-philosophy of politics.\ftnt{The \wo{post} of
  modernism could be taken as an abbreviation for such a meta-socio- of a
  reflection dissolving individual in the play of market forces (just like the
  criticised capitalism does) or some other forces, or else for a socio-meta-, a
  sociology of 
   intellectualism incapable of anything more than games of meta-perspectives
   and meta-interpretations. \citf{As the horison of the century which will be
     Your century, there appears increase of complexity in most areas, including
   the area of various \wo{ways of life}, of daily life. And this dictates the
   decisive challenge: make sure that humankind will be able to accommodate to
   very complex means of perceiving, understanding and acting, which transcend
   beyond its expectations.}{Lyotard, Le Postmoderne expliqu\'{e} aux
     enfants, VIII [p.113]} How do you make sure anything like that? And what do
 I care if humankind is able to accommodate to anything? I have to accommodate
 to all kinds of things, not any humankind...} 
 The relevance of such thinking may be to tell an 
indivdual where he is, in what world he happens to live. So far so good. But
then, trying to philosophise, one tends to formulate programs, advice,
suggestions -- committing the crime of passing from \thi{being} to \thi{ought}
which one used to be the first to denounce.



\ad{pantheism vs. gnosticism}
Pantheism (and we observed its relationship to empiricism) recognizing value and
holines of everything turns their totality into god. Gnosticism is the opposite:
only the highest, inaccessible, ineffable divinity has any value, while all the
details of actual world and life are valueless, if not directly evil. This is,
indeed, the tendency of Heidegger's complaining about the forgetfulness of Being
and, at the same time, claiming the essential association of such a
forgetfulness with the openness of existence. This is also the tendency
discernible in Jung, the tendency which Gilson wanted to discern in most
influences of neo-platonism... Do not we also commit the same mistake?

Hopefully not. We have said: as below so above, works on earth do influence the
decisions in heaven; and although both ontological and eventual concrete
founding proceed from above, the transition from the former to the latter, and
the latter's particular form, are helped -- though never determined! -- by
everything happeing here below. 

\section*{Questions}
\addcontentsline{toc}{section}{Questions}
%% {\large{
\pa
Check if, verify, and so explain \equin. A is \equi\ B:
\begin{enumerate}
\item
then A can be explained using B, and B using A; 
or
\item
then A can not be explained without B, nor B without A; or
\item \ldots ?
\end{enumerate}
Perhaps, \thi{explaining} is too strong here \ldots

\pa One vs.  Many, Section \refsp{sub:OneMany} \ldots needs fixing: I
guess, on the one hand, as a \co{nexus} of \co{aspects} -- this is ok
(the whole, static); but, on the other hand, I did describe it as an
{\em emergence}, as a \thi{generation} -- this, probably, is but
\co{virtuality} (but isn't \co{virtuality} merely a word, then?)

\pa
Attention (non-reactive) vs. will (control) may need more precise elaboration
\refpp{pa:SchelerZ}

\pa
I guess, the \co{non-actual experience} of \co{non-actuals} (\co{I}, 
\co{world}, etc.) should be described more concretely \ldots as 
\co{experience} but not \co{an experience}\ldots

\pa
More specifically about \co{commands} and \co{inspirations} -- they may 
be felt, as \co{actual signs}, but what they reveal is completely 
contentless. That is, they may be slightly more specific \ldots So, 
is there {\em the} ultimate form of a \co{command} which is completely 
contentless? I guess, this coincides with \sch. 

\pa
Elaborate more on \thi{upward} movements. 1) As in memory, i.e., 
things going into virtual state, or else 2) by mere applications of 
reflective categories to higher levels, as in 
\refp{pa:dissociateUp} (chap.II, level-4: my-Self vs. Self)

Also, on construction of \co{complexes}, as in \refp{pa:lab}.f

\tsep{questions?}

\pa
On the one hand: asymmetry of Being, and on the other: \co{One} has no precedence,
appears only in the triunity of aspects
\co{existence}-\co{confrontation}-\co{One}. 

\ad{evil -- alienation, or not ... vs. death}
On the one hand: evil is \co{alienation}; then: death is not so bad, since it is
eventual dissolution in the \co{indistinctness}. So why isn't killing good? (Not
because death must be willed, which might then justify suicide.)

Perhaps: evil, i.e., \co{alienation} is that which inhibits \co{existence} in
\co{existing}, which prevents it from \co{confronting} its
\co{origin} (\refpp{alienX}). While dissolution in \co{indistinct}, death,
overcomes (possibility of) \co{alienation} only because it dissolves one of
the terms of \co{confrontation} -- the \co{existence} -- in the other.

It follows that death, as such, is not evil. It does not follow that it is good,
either. 

\pa
At the level of immediacy (e.g., \refpp{pa:levsSigns} and earlier), 
the sign coincides with the signified and yet, the object is given as 
transcendent -- external? (? \oss\ vs. \rss ?)

\ad{subjectivism, projections, atheism or ...}
Non-\co{incarnated} ideals are worthless -- haven't I made up such an 
ideal (of \co{spiritual choice} and \co{love})?

Even more significantly: is not this whole ``explanation by projection''? One,
God, invisibles, etc., all are but poles of relations of which other pole is
existence.

This, perhaps, is the difference -- a pole of a relation is not a
projection, even if its necessary condition is the other pole. ?..Existence is
relational..?

There is no subjectivism -- without existence, God would make no sense, but this
does not mean that God can be reduced to any actual dimension of existence like
thinking, feeling or the like. None of human \thi{faculties}, especially none of
those pertaining to the sphere of visibility, is ever capable of approaching,
let alone representing God...

Yet, it smells pure a-theism, Schoppenhauer if not Fichte, for this God is not
\thi{there, in-itself}, but only \thi{for-us}... Well...


\ad{spirituial choice vs. dualism}
It does remind about dualisms, religions of radical choice --
zoroastrianism, mitraism, manicheim and various gnostic trends with bogomilism
and katars as the closest relatives. Yet it is not... Ontologically, there is
One, and only that (it is not the most real among the reals, but just the
reality of everything); spiritually, \No\ is not of the same order, although at
the same level, for it does not reach the bottom, the origin

\pa
Is not distinguishing smth. \thi{in the world} {\em very} different 
from distinguishing smth. \thi{within myself}? (This probably goes 
away\ldots)

%% }} %\end \large