
\ad{distinction but no essence}
We can imagine a \co{distinction} without \co{ipseity} when we 
encounter something for the very first time: we do not re-cognize it, 
we do not know, nor even feel, what this thing, this feeling, this 
something might be. But we know that it is {\em something}, only that we 
do not know what, we encounter it as something different from everything else.

\ad{holiness, concepts}
Before you meet a holy man, you may have a very confused, if any, 
concept of holiness. But it is no goal to have all the concepts 
correct and ready in advance. The goal is only this: {\em when} you 
meet a holy man, do not let your concepts (or misconceptions) prevent 
you from recognizing his holiness, from admitting it. (Actually, concepts, no 
matter what the hermeneutics tell, make it harder to accept the 
testimony of new experiences.)

\ad{holiness, the Other}
Holiness is never experienced as the Other, the {\em absolutely 
Other} (unless one denies it). On the contrary, it embraces 
everything around itself, gives an unmistakeable feeling that 
\thi{also I am a part of it}.

\ad{actuality confused with eternity}
One of the main effects common to all the drugs is that they reduce 
the world to the pure \herenow, they narrow the horison of perception 
and consideration to this very moment, without tomorrow, without 
yonder, to pure actuality so seductively resembling the mystical 
presence of eternity.

\ad{Analytical philosophy}
says \wo{this is bubbling, incomprehensible rubbish}. Well, if I read 
only one page from Kant, it is incomprehensible, but it helps if I 
read more and think about it. When I read Bible for the first time I 
understood much less than when I read it again. But so was it when I 
read Carnap, and when I studied mathematics!

\ad{expression vs. expressed}
The question about truth has been all the time confused with the 
question about the formulation of truth; the absolute 
with the expression of the absolute. And finding no unique, 
unequivocal, verifiable formulation, one decided to abandon the quest.

There is a deep lack of respect, both for the truth and for the other 
in this. The truth need not -- {\em can not} -- be expressed in a 
single way like a mathematical definition. (It is abundant, 
overflowing, non-actual.) And the other can understand the intention beyond 
the partial, even misleading expressions.


\ad{reflection}
The art of reflection is not to jump to a higher level of 
self-consciousness and observe everything, oneself included, from 
above and outside. On the contrary! It is to stay inside and from 
within to record the events, feelings passing by without externalizing 
them. It is to realize that there is nothing outside, that everything 
is and must be seen only from within. The art of reflection is not to 
oppose immediacy but to preserve it -- in spite of reflection.


\ad{Community-personality}

\ad{faith-ateism...}
The alternative, the contradiction \wo{God is} vs. \wo{There is no God} is
possible only after one has reduced the supposed \thi{being of God} to the
level of \co{actual experience}, to the level of \thi{being a thing}. It helps
nothing to claim that this was not the intention, for this is, in fact, the
result; and when the results are clear, the intentions do not matter. Assuming
that such an alternative is at all possible, one has already falsified the
meaning of the most well intended and positive answer to the 
question \wo{Is there God?}.

A better question would be \wo{What does \wo{God} mean?}, or even \wo{What does
  God mean?}, though this, obviously, involves one in the matters which even the
most prominent theologians eventually had to give up. Any answer could be accused
of arbitrariness -- but this should not worry
The advantage is that the previous alternative (\wo{He is} vs. \wo{He is not})
now becomes \wo{He means nothing} vs. \wo{He means something}. The first one
will quickly declare Him to be non-existent. But this is now fine because now
it is also more clear that it is only inability, perhaps even only \co{my}
inability, to find any meaning in Him (in it?), which leads \co{me} to this
conclusion.

If there is any God, He can be only in my life, and if He is not there then,
indeed, He is not...




