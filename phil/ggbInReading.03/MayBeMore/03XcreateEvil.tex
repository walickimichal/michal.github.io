
\kom {\bf Thirst}

Even De Sade thirsts...

Conscience is a manifestation of \co{thirst} (it can be ignored just as
\co{thirst} can be); it is a \co{reflection} of \gre{gnosis} (the knowledge
of the \co{absolute}), reminder; and also the other side of the fact that sin is
its own punishment, because it is just \co{alienation} from the good: \citt{the
sinful delights which entice men are the very instruments of God's
punishment}{Innocent III, On the Misery of Human Condition [The Many Faces of
Evil, p.79]} for \citt{that wherewithal a man sinneth, by the same also shall he
be punished.}{Wisd.(om of Solomon) XI:16}

Observing and emphasizing the \thi{universal suffering} is, too, an expression
of \co{thirst}... (Budda, Schopenhauer)

\co{Thirst} for what... Being, indefinite, reality (sickness to \thi{unreality})

Augustine -- \la{esse uelle}, the thirst for being, the ontological thirst for
the true being which humans experience all the time, according to Augustine, as a
consequence of the original sin or, as we might say, of the separation from the
\co{orgin}. [cf. Bog-Nicosc, J.Miernowski, p.91, asurekcja Bovelles, analogia
Tomasza, neoplatonism...]

Nothing \co{visible} can quench it, though we use \co{visible} to suppress
it. 

\kom
whoever sins against a small one, sins against the One; every
communion \herenow\ adds to the communnion \co{above}...

[`as below so above']


Do good things... --
\begin{itemize}
\item
`following' from above are good
\item
Morality is founded -- if there is no God, everything is allowed
\item
in order to rise above: as below so
above (sedimentation of virtuality);

\citt{Kazda cnota sprawiedliwego rodzi Boga}{Eckhart [Otto,
Mistyka...p.226]} -- as below so above
\item
good acts are preparations for \yes, eventually, for \co{grace}
\end{itemize}



Evil is always limited -- it is this or that... Seeing the world as evil, that
total amount of evil outweights the total amount of good, or whatever funny
formulations one may invent for the existential disappointment, are expressions
of thirst and obtain their univerrsal character from it...

Evil is evil but evil person is, in a sense, innocent, for he only reacts, 
\refp{whoisevil} This innocence does not mean lack of responsibility -- we are
responsible for much more than what we choose to do; first of all, we are
responsible for what we are. We are responsible for all the help evil receives
from us. Since the amount of this help is seldom known, we seldom know the
amount of our resposibility. 

 Evil develops 
\thi{bottom-up}, by accumulation of disappointments and negative experiences
until these sum up, high \co{above}, into \co{malum negativum} or even
\co{activum}.
There are souls which seem to be marked by one
form of evil from the very beginning, that is, from the moments when anybody
remembers them doing anything (and this is, of course, very late after the
beginning).  But our claim (unverifiable as it may be) is that they, too, have
been exposed to \thi{misunderstanding}, have reacted to experiences which were
interpreted as evil, even if nobody around could ever imagine that they
would be so interpreted. We do not know, do not have full control over what
turns \co{thirst} into lack, what possibly might give rise to the experience of
evil. Even if we have some general ideas, some rough impressions, these are only
that -- in every individual these things get settled anew, as he brings his
distinctions into the world. 

It may be pain, a catastrophy, a tragedy, always an \co{actual} event, which may
pollute one's mind with the idea of evil. Evil originates in the \co{actuality}
and grows by moving higher and higher up in the \co{soul}, until it 


\tsep{justification}
%%%%%%%%%%%%%%%%%%


\pa
\yes\ relieves \co{reflection} from the dependency on \co{actuality}. 
It is still involved in it, but it has now lost its absolute validity 
as the only \ldots


Evil is impossibility of justification -- and 
justification = concrete founding (in One after \yes). The former divides and
dissociates the later gathers the distinguished

\ad{Jealousy/envy}
Envy -- as well as jelaousy -- is only an \co{actual} expression of a deeper
defeat, of a conviction (whether conscious or not) of impossibility:
impossibility of filling a lack, of achieving the ideal state, the object of
one's dreams, in short, of quenching the {thirst}. As the \co{actual object} (of
envy, jelaousy) begins to fill the whole horison, so the feeling itself
penetrates deeper and deeper, enslaving gradually the wider scopes of
personality until it reaches the fundament of one's whole being.

Trivialising: impotence is founded on impossibility, but here impossibility
amounts to a search for the ultimate quietude in something \co{visible}, in
possessing the object of one's envy, in \thi{possesing} the person one is
jealous about. The order of founding is: \inv impossibility; \mine impotence;
\act enslavement.

This deep and usually unrealized conviction of impossibility is the same as the
lack of justification, the \co{alienation} from the \co{origin}...


\pa
I can't justify myself: only by smth. else (below me = outside: rationalization,
necessity...; or above me -- also another person may justify/personal love, too)

Indeed, justification refers one to the community and, eventually, to the
communion because, in the last instance, only \yes\ justifies. Levels....

\act -- achievements, realized goals

\mine -- others. There is a tremendous force in others' presence. A group of
friends justifies every memeber, helps to dissolve any doubts. I may be
uncertain about everything, whether I should think this or that, do this or
that, whether what I am actually doing makes any sense. Such doubts simply
dissolve when confronted with others. A simple \wo{Yes} or \wo{No} received from
another person has completely different power than \co{my} own, perhaps deep,
but still only personal \wo{Yes} or \wo{No}: the latter is always exposed to
possible doubt and dissolution while the former, once pronounced begins to
affect me as an objective fact. Others act -- psychologically -- as a
solidifcation of the subjective flux

Personal love, in a much deeper sense than a community, provides also the
experience of justification

\inv -- it is only \Yes; 

\pa
Justification is the eventual answer to the questions about meaning
\begin{itemize}
  \item asking
for the later we look for the former, which is but concrete founding in
One.
\item Meaning is only one: absolute participation. It is self-confirming and
self-justifying -- unlike e.g. participation in evil (unintensional and seldom
fully volitional), it does not look for any explanations/justification of itself 
\item
meaning, good, freedom -- absolute nexus, concrete One = personal God
\end{itemize}
In short, such questions do not have any precise answers because they concern
the concreteness of existence, its unique confrontation with One. Their answers
rest in invisible depths and no psychology, socio-technology, not to mention
science or technique will ever help to answer them in ...

Evil disappears....



\newpage
\subsubi{? [here?] creation of evil: Not needed any more....}

whatever we say, we are saying something about ourselves, therefore silence is
the ultimate \co{sign} and praise... (\wo{Silence is praise to Thee} does not
occur in Ps. LXV(65):2, as Maimonides claims...)



God creates by admitting, just as He inspires simply by being desired,
\co{thirsted} for...

%\noo{ ? Evil God ?
\kom \citt{For it is said (Is. 45:5,7): "I am the Lord, and there is no other
  God, forming the light, and creating darkness, making peace, and creating
  evil. [I the Lord do all these things.]" And Amos 3:6, "Shall there be evil in
  a city, which the Lord hath not done?"}{Aquinas, Summa Theologiae, First Part
  Q49 A2 Obj. 1 Para. 1/1}
Though Aquinas dismisses these as refering to \la{malum poenae} (evil of
  penalty) and not \la{malum culpae}. Is this distinction doing the job...?

\citt{For I will at this time send all my plagues upon thine heart,
        and upon thy servants, and upon thy people; that thou mayest
        know that there is none like me in all the earth.
For now I will stretch out my hand, that I may smite thee and
        thy people with pestilence; and thou shalt be cut off from the
        earth.}{2 Exodus, IX:14-15}

And then \citeti{he hath torn, and
      he will heal us; he hath smitten, and he will bind us up.}{Hosea}{ VI:1}
      
Job come of course to mind, too. Nonce of these cases seem to suggest any prior
guilt: it is only a moralistic bias of the theologian which will impute it.
    
\kom
God acts/is personal/... only through/in me, only by being desired (thirsted
for) -- being founded \simu 
\co{confrontation} = Imago Dei;  \co{concrete founding} \simu Similitudo Dei.
[Rilke, Eckhart]

There is no God without His people;

\kom
Evil is born between men but not by them; God does not create without us...

You distinguish evil, or rather call smth. \wo{evil}. (We are not attempting any
classification for socio-forensic purposes, which are always and by necessity
deficient and incomplete.)

\citt{Bog wszystko stworzyl przeze mnie, gdy znajdowalem sie w niezglebionej
podstawie Boga [...] Gdyby mnie nie bylo, nie byloby Boga. Rozumiec tego nie ma
potrzeby.}{Mistyka Wschodu i Zachodu, p.121} + Rilke... 

\citt{the evil which
consists in the defect of action is always caused by the defect of the agent.
But in God there is no defect, but the highest perfection, as was shown above
(Q4, A1). Hence, the evil which consists in defect of action, or which is caused
by defect of the agent, is not reduced to God as to its cause.}
{Aquinas, Summa Theologiae, I:q49.a2 Body Para. 1/2}

\citt{But the evil which consists in the corruption of some things is reduced
to God as the cause. And this appears as regards both natural things and
voluntary things. For it was said (A1) that some agent inasmuch as it produces
by its power a form to which follows corruption and defect, causes by its power
that corruption and defect. But it is manifest that the form which God chiefly
intends in things created is the good of the order of the universe. Now, the
order of the universe requires, as was said above (Q22, A2, ad 2; Q48, A2), that
there should be some things that can, and do sometimes, fail. And thus God, by
causing in things the good of the order of the universe, consequently and as it
were by accident, causes the corruptions of things, according to 1 Kgs. 2:6:
"The Lord killeth and maketh alive." But when we read that "God hath not made
death" (Wis. 1:13), the sense is that God does not will death for its own sake.
Nevertheless the order of justice belongs to the order of the universe; and this
requires that penalty should be dealt out to sinners. And so God is the author
of the evil which is penalty, but not of the evil which is fault, by reason of
what is said above.}{Aquinas, Summa Theologiae, First Part Q49 A2 Body Para. 2/2}

Perhaps, we do not distinguish \la{malum culpae} (evil of fault) from \la{malum
  poenae} (evil of penalty)... Even if every fault will be punished by some
  evil, it still does not follow that every evil is a punishment...

We are not asking for explanations, and punishment here sounds like an attempt
to construct one. But punishment by one who punishes not quite consistently
(according to our logic) is not much better an explanation than saying simply
``shit happens''. (Job was one example...) The arbitrariness 
was rather obvious, so one had to distinguish further the divine causality: God
causes some things by generating them while others by not impeding their
generation by secondary, as often the case may be, deficient causes. In the
latter case, it was still causation because, after all, God might have prevented
the effect but he did not.

Now all this dissolves or, perhaps, we have only divine non-impediment. He is
certainly not responsible for my problems and sufferings. As a matter of
fact, unless I can identify, or at least reasonably postulate, a potential
perpetrator of some evil, then nobody is. Looking for a responsible one in cases
when none can be identified is simply an expression of bitternes which, if
unrestrained, will embrace the whole world and, eventually, consider it evil...

But it still leaves everything pretty arbitrary. So, at least, it appears in the 
\co{experience} of \co{this world}. Postulating God's punishment beyond and
behind all `shit' has, again, the effect we agree on -- \co{humble} acceptance
of things which are not within our power. The difference is only minor and lies
in the answer to the question \wo{Why?}. We do not try to construct a theory
providing such an answer, we simply say: because it is the best thing one can
do. We do not intend to convince those who disagree...
      

%%%%%%%%%%%%%%

%\subsubi{God, person, evil}


\tsep{Existence (not Being), as such, is good...}

The activity of evil and its self-strengthening may seem to oppose the tradition
announcing any being, in so far as it simply is, for good. 

%Thus, for instance, \citt{evil hath no being, nor any inherence in things that
%  have being. Evil is nowhere \la{qua} evil; and it arises not through any power
%  but through weakness. Even the devils derive their existence from the Good,
%  and their mere existence is good.}{Pseudo-Dionysius, The Divine Names, IV:34
%  [p.129]}
%used above in concentration camp is not good...

%\citt{evil can exist only in good as in its subject} {Aquinas, Summa Theologiae,
%  I:q49.a3 (q48.a3)}
% used
Existence, yeah...

\citet{It must be said that every evil in some way has a cause.
For evil is the absence of the good, which is natural and due to a thing. But
that anything fail from its natural and due disposition can come only from some
cause drawing it out of its proper disposition. For a heavy thing is not moved
upwards except by some impelling force; nor does an agent fail in its action
except from some impediment. But only good can be a cause; because nothing can
be a cause except inasmuch as it is a being, and every being, as such, is good.}
{SumTh}{ I:q49.a1}

We have, of course, replaced the language of \wo{causes}, especially the
primary, as we might say, vertical causes which became the hyposthases, down to
the \co{actualization}, of the \co{virtual nexus} of \co{existence}. The center
of \co{existence}, the \co{confrontation} with the \co{origin} is not only the
good in itself, but is also the true beginning from which proceed all the
\co{distinctions}. In this sense, we may agree on the last sentence: every
\co{existence}, as such, is good and nothing can be the beginning (rather than a
cause) except inasmuch as it is an \co{existence}. 

Evil is any form of \co{alienation} which breakes the continuity with the
\co{origin} -- it is the absence of good. The \wo{cause drawing a thing out of
  its proper disposition}, which Thomas insists must be good, is in our language
\co{thirst} which gets misunderstood -- for it \co{thirsts} for the \co{absolute
  nothing}, but one tries to quench it with the relative, with \co{idols}. And
so, indeed, a conglomeration of good causes may sum up to an evil effect

There is only one highest principle -- existence as good of confrontation, and
so evil has no highest principle....

\say Everything is good -- for me! It is not so that I meet evils which are evil
only for me but, eventually, in some totalitarian accounting books, will turn
out good for the whole. No! It is a bad excuse... But, true, I should learn to
see good -- for me! -- where I otherwise would tend to see evil directed, or in
any case acting, against me. This may be a lesson of true humility, and if it is
worth undertaking is up to every person to judge...

\tsep{Human nature: good or evil...}

\kom
It seems that it is the capacity for acting and {\em
  being} evil which is our true differentia specifica, which truly distinguishes
  us from all other animals 


  
\ad{Morality?}\label{morals}
side-effect

And finally, 
is human nature good or evil... (from the OLD) -- \co{thirst} itself says...
\& ... Evil be thou my {\em good}! Scheler: good $>>$ evil ...

\pa\label{kom:noEvil}
Evil-good, in particular looking for the signs of this distinction, is already a
moralism, attachment. Love does not abolish evil which happens around, but it
does not see it ... as evil. Evil ceases to be evil, becomes a
misunderstanding. In particular, love never looks for the guilty ones, for
anybody to blame.

\citet{It does not matter whether that which should be defeated is described as
  weakness or defect, or else as destructive powers; that against which
  one is fighting is always evil, and that in the name of which one is fighting
  is always good. Certainty of clear contradiction between good and evil will is
  indispensable and infallible mark of heroic world view.}{Klagges}{\citaft{Gnostics}{, 
IV:12.8 [p.297]}}
%  \btit{Geschichtunterricht}, after E.Hieronimus, 
%\btit{Dossier: Dualismus und Gnosis in der v\"{o}lkischen Bewegung} [Der
%Gnostiker: der Traum von der Selbsterl\"{o}sung des Menschen,
%Micha Brumlik, Vito von Eichborn GmbH \& Co Verlag KG, Frankfurt am Main, 1992,
So much Klagges, the minister of culture in Brunshwig in 1933, later officer SS,
  one of those responsible for granting Hitler German citizenship in 1932.

The \thi{heroic world view} shares with the healthy attitude the ability to
confront obstacles; but it 
does not see them merely as obstacles and problems to be solved, but as evil;
and evil offends, is inadmissible, must be destroyed.
The difference may seem a subtelty but is the most important one -- the crucial
division lines are invisibly thin.

In the \co{spiritual} matters (and let us emphasize once more, these are matters
concerned exclusively with \co{nothing}, in fact, it is only one single matter),
not making any explicit choices, opposing the temptation to assume a heroic attitude, is
a much better choice, is often a \co{sign} of \yes. 



\subsection{...concluding...}
So, there are two fundamental faces of God, reflecting the respective spiritual
attitudes. Analogically, acceptance of a respective image of God, may lead into
the direction of the respective spiritual attitude.


But as the \sch\ is not a \co{visible} event, one can hardly be sure where one's
\co{spirit} is wandering. Job is said to have been \citet{perfect and upright,
  and one that feared God, and eschewed evil.}{Job}{I:1} His perfection seems to
consisted in the conscientous observance of the laws and honest fear of God. Yet
even he encounters what appears as the tremendous wrath of God: \citet{Let him
  take his rod away from me, and let not his fear terrify me: Then would I
  speak, and not fear him; but it is not so with me. [...]  Withdraw thine hand
  far from me: and let not thy dread make me afraid.}{Job}{IX:34-35; XIII:21}

      Otto: both elements constitute \la{sacrum}...
      
chciane/poszukiwane -- so not from the mere fear

Abelard rightly notices that \citet{God pays attention only to the mind in
  rewarding good and evil, not to the results of the deeds.}{AbelardEthics}{(90)} But
  \citet{human beings do not judge about what is hidden but about what is
  plain}{AbelardEthics}{(82)) and so \wo{sometimes a penalty is reasonably exacted from
  one in whom no fault has occurred.} Trying to figure out the reasons for Job's
  sufferings would be like trying to model God's judment on the plain human
  justice of \co{visible} penalties for \co[visible} crimes. It would be like
  asking again \wo{Why?} instead of accepting the obvious fact \co{that} -- to
  cope with it, one need not understand the reasons...

\wo{God pays attention only to the mind} has been variously misinterpreted,
primarily, as a penalty for any \thi{unclean thought}. Maimonides' prophet may
serve as a good example of such an intellectual image of a perfectly pure and
unadultered mind... But mind is not thoughts, in any case, not only thoughts. It
is much more what one does with the thoughts which invade one... Saints are
tempted, too, the difference from common folk is what they do with the tempting
thoughts... 

but chciane, not the tremendum as such, but its august aspect -- it is possible
to separate them so that tremendum ceases to frighten as the doomsday, but
becomes weakened....



