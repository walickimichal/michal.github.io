\NOWY{Why?}\plan{Why}

\pa
A word is an actual sign which points to something invisible, lying
beyond the horizon of actuality. And since it transcends this horizon,
it can never be fully present within its limits. What you can do is to
try to apply -- and more often invent -- words which point towards
things you intend in the way in which you apprehend them; to create a
system of signs which convey your feeling of life. It will never be
unique, the only possible, final. But if you manage to convey this
feeling -- which unlike its expression is much more universal and
common -- then what else would you like to do with your signs?

\pa
Reductionism is a project of scientific consciousness. Perhaps it is
true, perhaps everything we experience can be reduced to some basic
level of physical or chemical reactions. But the point is -- so what?
How can this ``wisdom'' affect my life and actions? Is it at all
relevant whether this is so or not? Suppose that somebody produces a
definite proof that it is so. And? Shall I change anything in my life
because of such a proof?

The only way such an ideology might become relevant would be to
produce some controlling mechanisms.  Perhaps, having traced everything
to such a basic level will enable us to control everything which is
above -- and as some maintain, independent from -- it. Then it might
be a different story. But so far, the very meagerness of such results
makes the whole project irrelevant. Let those who believe in it, work
on it. In the meantime, let me tell you a different story.

\pa
If I announce some problem uninteresting it is not because it is
illegitimate for some reasons, because it should not be asked, or 
should be asked within 
ontology and not within ethics. I reject it because I do not find 
it relevant for my thinking, because it lies beyond my world.

All problems disappearing from the philosophical scene do so not because they 
have been solved but because people got tired of them -- they become 
irrelevant.

\pa
Wszystkie osoby i sytuacje przedstawione tutaj sa\; rzeczywiste, lecz
ich ewentualna zbie\.zno\'s\'c z sytuacjami rzeczywistymi jest czysto przypadkowa.

\NOWY{Philosophy}\plan{Philosophy}

\pa
\wo{Inventive}, when applied to a philosopher, means almost the same as
\wo{petty} -- the only difference is the additional element of \wo{smartness},
which is not contained in every \wo{pettiness}.

\pa
Philosophy does not create -- it reconstructs.

\pa
Philosophy is the art of appreciating simplifications.

\pa
Nothing new is ever said; there is no thought which has not been
thought before. 

What is new may be interesting for a scientist, for an intellectual --
never for a human being. Novelty is an idol.  

Typically, what is old is noble. But it is not noble because it is
old. On the contrary, it is old because it is noble.  Otherwise, it
would die as so many fashions and transient distractions. 

Philosophy
is a series of signatures put by different thinkers under thoughts
expressed earlier by somebody else. And so on, indefinitely, back into
the individual childhood or common, mythical prehistory.  But this
recurrence is its strength -- not a weakness against which it should
be defended.

\com{
Traditionally, one would say that philospohy 
does not deal with \wo{concrete} things but with their \co{essences}. 
It does not speak about
\thi{father} or \thi{table} but about \co{fatherhood} and \co{table-ness}.

This view is still biased towards concrete {\em things}. But in truth, it deals with
something which is \wo{above} the things and this world. Only that this something is
not the essence of things -- it is a different order of the world. It is the order where
quite different categories play the significant role, and where concrete things of this
world are merely their possible manifestations, incarnations. And, of course, in different
contexts very different things and very different relations between things may express
the same aspects of this \HH\ order.
}

\pa
``Any good text has many meanings''. (Hemingway)
Would not this be the difference between philosophy and literature.

\pa
Life makes \HH\ present; philosophy makes it \LL.

\pa A person is a way of living and feeling. 
A philosophy, any philosophy, is a way of speaking. 
``Mouth speaks from the richness of the heart.''  

\pa
It isn't so, as Russell would like it, that philosophy, starting with 
some obvious truths, attempts to arrive at astonishing paradoxes. 
(This is an attitude of a scientist or an arrogant intellectual and 
it seems that Russell was both. A philosopher shouldn't be either.) 

On the contrary, philosophy attempts to arrive at the most trivial 
conclusions known to all people and, hopefully, to the philosopher 
himself. The challenge, and not the least one, lies in finding an 
appropriate set of symbols -- words -- which is capable of expressing 
these conclusions as lucidly and coherently as possible; in making 
the \HH, which is known intuitively, \LL.

\pa
Contradictions are inventions of the separating activity of reason.
There are no two things in life which couldn't go together.

\qpa{
Each difference is only a difference of degree. It is only within the 
horizon of actuality, where things appear under distinct signs, that 
one can claim absolute, essential, definite, etc. difference between 
two things. But what one is actually separating in this way are not 
the things but their actual signs.}

\upa{
Say, roughly: epistemology inquires into the mechanisms of acquiring 
knowledge about what is there, ontology into what is there, 
and ethics into how we should relate to what is there. (It does not 
matter if ``what is there'' isn't really, but only appears as being 
there. The point is that the phrase has some meaning to us.)

Now, an inquiry into what is there is a way of relating to it, so 
epistemology is a part of ethics. Then in order to relate to what is 
there, we have to have some access to it, so ethics is a part of 
ontology. And finally, we cannot speak about what is there without 
relating to it in an understanding way, so ontology is a part of 
epistemology. 

Well, I do not 
object to the existence of {\em purely} epistemological, etc. 
questions. They may be meaningful in their narrow sense. 
And, of course, one can produce all kinds of distinctions and specific 
methods distinguishing these disciplines. But nevertheless, from 
the very beginning, the division is unfounded if we are interested in 
ako. totality of our being. Separating these things and then trying to 
build bridges joining them together is a weird procedure:
first one tries to emphasize the distinguishing features, and 
then to ignore them. I do not 
think that it is a fruitful point of departure for philosophical 
thinking which attempts to create a general picture of man's life and 
being in the world.
}

\pa\label{onethough}
Each great philosophy expresses (at least) one deep intuition. 
Typically, it has become great by blowing up this single thought to the 
universal and only principle of everything.

\pa
An analytical philosopher is a guy who will ask you, when after a concert 
you say ``It was a beautiful piece of music'', ``What do you mean by 
`beautiful'?'', ``What do you mean by `a piece of music'?'' \\
(It is the same usurpation as in \refp{onethough}.)

\pa
Analytical ``philosophers'' have to put almost all philosophy in the 
quotation marks. Well, I prefer to save some ink by putting only their 
``philosophy'' into such marks.

\pa
In fact, they are no better than all these earlier fractions and thinkers 
whom they accuse of producing only unverifiable Weltanschaungen. They are 
much worse -- to the degree that their Weltanschaung does not concern any 
world but only a method. In their pretty out-of-date fascination with 
science they forbid themselves to talk about views and matters that 
matter, and are allowed to talk only about what philosophy should and 
should not be.

\pa
Yet, analytical philosophy is one of the most respectable forms of this 
great complex of the modern intellectuals: to get closer to people, to the
``man on the street'', to the ``concrete, real world''.

\pa
It often strikes me that if we take any text on transcendental philosophy
(whether Kant, Husserl, or whoever) and replace the word ``transcendetal''
with ``unconcious'', we get equally meaningful yet much more comprehensible
text.

\pa
\thi{Brain in the box} -- silly. What does it show? That, perhpas, the word
is not the way the brain imagines it to be. And that's it! It does not show
that, perhaps, the word is not there, only that it may be different. For it
has to assume that the brain is connected to something outside itself. And
futhermore, it does not show that the brain couldn't come up with the idea
that, perhaps, the word it `sees' is not the way it is, that, perhaps, it
itself is only a brain in the box.

\pa
An academic attitude forces a philosopher to relate to and take into 
account any possible deisinction. A personal attitude is to stop 
distinguishing at some point.

\NOWY{Words}\plan{Words}

\pa
A text, the celebrated \thi{text}, is but a limited medium of 
communication. Limited by the absence of the author -- if the author 
could be present, the texts would rpobably be much shorter.

Those who make \thi{text} into an independent, absolute event of 
cultural world forget or, as the matter may be, want to eliminate the 
fact that any text actually does or did have an author. Even if we 
know nothing about him except that he was the author of the 
\thi{text} we are actually reading, it is already more than enough. 
The \thi{text} is but a medium of communication -- if not from the 
author to the reader so, in any case, from one human being to 
another. And even if \wo{author's intentions} may be an unfortunate 
term, what we try to find in a \thi{text} can be so labeled -- it is 
some human meaning, something one human being might want to say to 
another.

\pa
In a way, everything is matter of speaking, of saying things. Words 
try to reflect our experiences but, equally, influence the way we see 
and act.

\pa
The world is but a sign.

\pa
Expressions come before words; words before concepts. (Individual names 
before the common ones.)

\pa
Word -- a sign -- is a prereflective synthesis which precedes any 
analytical understanding.

\pa
A word whose meaning can be fully analyzed is redundant -- it
functions merely as a convenient abstraction increasing efficiency of the 
system.

\upa{
\thi{Big words}, or better \thi{high words}, do not force their meaning
upon us. They only hint at something not fully expressible, which we
are free to model and interpret -- they leave us freedom, exactly
because they do not have a unique, precise meaning.

As such, they are the opposite of arguments which, in honesty or
arrogance, always attempt to force the other to accept them.}

\pa
Words never embrace the whole reality -- they are mere signs,
pointers. If you do not understand what is being said, perhaps, you do
not know what the talk is about. And if you know, you need not the
absolute precision of expression -- a mere sign will suffice.

\pa
The difference between \thi{words} and \thi{mere words} is exactly 
this: the latter fail to make anything present, while the former do 
reveal, that is, make the other realize what they are pointing to. 

\pa 
Communication means sharing the \wo{object} of discourse. With 
somebody we know very well, we can talk in very imprecise terms and 
still be certain that we are actually communicating. Often, because, 
when we have only begun to utter a sentence, the other already knows 
what we are trying to say. With strangers, we usually have to be much 
more precise.

The art of speaking consists in such a \co{making the \wo{object} 
present and accessible} to others. It has nothing to do with the 
precision and definiteness -- these become necessary when 
communication is threatened.

\pa
``The primitive mind ascribes conscious life to all natural objects,
to nature in general.'' The modern mind ascribes to it nothing at all
or, perhaps, some crude functionalism. In either case, it is a matter
of ascribing.
\pa
To some extend words are irrelevant. Manicheism is as good as Augustine's
Christianity. The difference concerns the implied attitude.

But it is not at all the question of a subjective image of God. It is
a question of being, of existential realization of faith, which has
infinite variety of expressions. The words should merely point towards
such expressions.

\pa
A dialog is to look at the world also from another point of view.
Dialog with oneself is no contradiction (rather with one's Self).

\pa
Are there thoughts I did not think? Certainly. It doesn't bother
me. What bothers me sometimes is that there are thoughts which I {\em
cannot} think because there are things I have not seen.

\pa
The power of language consists in that we can say and talk about things
which we cannot understand (define).

\qpa{I say more than I mean.

I do not mean everything I say but I say everything I mean.}

\pa
{\em Self}-expression is an expression with bad conscience, an 
expression which knows and admits: \wo{this isn't really so, it is so 
only {\em for me}}.

\pa
Not so seldom, wrong signs lead to right conclusions.

\pa
Nothing is ever definite except definitions (concepts, decisions).

\pa
What can and what cannot be communicated doesn't depend on the subject matter
but only on the people involved. Given a wrong audience it may be equally impossible
for a mathematician to explain a most clear proof as for an artist his impression or 
for a saint his vision. That higher things are harder to explain means only that it
is harder to find people who can understand them.

\pa That something can't be described. defined, said does not mean that it is not.

\qpa{Symbol (conscious/reason) {\em is} the object -- only in a different context than
its perception or experience.}

\pa
When I say ``soul'' I am not committed to admit the existence of sould. When
I say ``soul'' I admit only that this is a reasonable way of speaking. All
our committements are epistemic -- the ontological ones are expressions of
either limited perspective or intellectual arrogance.

\pa
Often, and in the spiritual sphere always, the invisibly thin lines are the most
crucial boundaries. 

\NOWY{Work}\plan{Work}

\upa{
The goals must remain hidden until we reach them.
}
\pa
Only distance makes a relation possible -- there is no closeness without distance.
\pa
Whatever one is, another can be as well. But one cannot be the other.

No human being possesses a quality which another human being couldn't possess.
\pa
Work brings you closer. So does suffering. They are necessary
expressions of the distance. (Until it ceases to matter.)
\pa
Distance is finite -- and absolute. (Work has no end.)
\pa
The distance from \LL\ to \HH\ is infinitely long. But once you have
crossed the border it disappears.
\pa
You will find yourself in everything you do. But only there?

\qpa{
Success is an indication of a lack of content -- at least a lack of depth.}

\upa{
Man gains depth realizing that some questions should not be answered
-- even if they have to be asked.
}

\pa
Concrete is what is close, not what is precise.
\pa
Work is what keeps heaven and earth together.
\pa
Work is the (a) process through which spirit gains control over the ``world''.

\pa
Most significant changes (affecting the \HH) are effected by long series of 
acts and works none of which has in itself any apparent effects.

\pa
To create is to convince - first oneself, then others.

\pa
Rationality is a movement upwards under the spell of visibility: with the
pre-given intuition of unity, an attempt to re-construct it as a totality
(without ever surpassing the boundary of the visible).

\pa
What hides behind is what lies ahead -- we do not know exactly. Or, slightly
differently: 
``Know what is in front of your face, and what is hidden from
you will be disclosed to you.'' [Gospel of Thomas, 5]

\pa Some scholars are like men who insist the more on their ability to swim, the
closer they are to drowning.

\NOWY{Evil and/or pain}\plan{Evil}

\qpa{
There is no evil without suffering.
}

\pa
The only basis of spiritual power is the ability to suffer, that is humility.

\pa
\wo{Passivity increases our chances for evil, for becoming evil.} Is 
it a triviality? Yes -- but only in an evil world.

\pa
Sin is its own punishment - it immediately perverts one's being, diminishes 
humility, openness, and thus, the enjoyment of the world.

This, however, does not mean that it is immediately \LL. It may take
time before the sin is punished in a visible way. Indeed, it may 
happen that it never will. 
\pa
``Chciwo\'s\'c (avaritia) jest s{\l}u\.zeniem ba{\l}wanom. 
Avarice is serving the idols.

\pa
Only those looking for heaven get themselves occasionally into hell. The two
places are terrifyingly close to each other, separated only by a hair-breadth.

(Piek{\l}o jest wybrukowane dobrymi checiami.)

\NOWY{Feel, Love}\plan{Feel, Love}

\pa
The first form of love is the need to be loved. Disregarding it we sow the
seeds of quiet suffering which becomes evil.
\pa
To feel is to be vulnerable.
\pa
Not returning love is to kill it.
\pa
``I hate you''. Is it cruel? Not at all -- it can be changed. ``I do
not love you'' does not hurt more than honesty should do. But what is
terrifying, disclosing the cruel irreversibility of time is to hear
``I loved you ... ooh how I loved you.''
\pa
What is true is like love -- it lasts by enriching everything it
encounters; it does not negate anything. It is humble.

But it is not defensive, saying ``it is good for {\em me} but,
perhaps, not for you''. Defensiveness always hides some untruth.

\pa
Mi{\l}o\'s\'c rozgrzesza.

\pa
Perfect, all inclusive egoism: Whatever you do, you do for your own 
sake -- because you find yourself in everything and everything in 
yourself.

\upa{
One regrets the lost intensity of feelings which created the impression of
life. Shouldn't one then try to listen to the most weak and feeble `feelings'
which, lasting but not overwhelming, are the matter of whole life?
}

\upa{
There are things which enter our experience only as longing and it is a great
mistake to think that it is longing for something. A true, original longing
is not longing for ... -- it is simply longing, without any object, which
discloses reality beyond objects.
}

\pa\label{qpa:lin}
Love is more than a feeling. It is readiness to meet with humility whatever
might come. And be thankful for it. (cf.\refp{qpa:in}).

\pa
Love: those people offering each other their uncertainities, and then rising
stone and rock solid houses out of them.


\NOWY{God, Spirit, religion}\plan{God, Spirit, religion}

\pa
Eternity is the \co{presence} of the \co{absolute} in the temporal. 
Immortality is \co{concreteness} of this \co{presence}, that is, when this
\co{presence} not only surrounds but also permeats \co{actuality}; not when it
is felt (it never is), but when it is known and \co{recognized}. It can be
\co{experienced} but is never \co{an experience}. Perhaps, we
could say that the moment of death is the only possibility of \co{an experience}
of immortality, the only moment when the \co{presence} of the \co{absolute}
becomes fully \co{actual}. (But do we want to say that?)

\upa{
Man is an animal with ontological hunger. If you like -- a religious animal.}

\pa
Hunger of being is a hunger of state -- of something that is, not
something that becomes. 

\qpa{
And state is something which isn't merely actual -- it must extend 
beyond the horizon of actuality. State is the first image of 
something that lasts.}

\pa
What do you want to find looking for God beyond coincidences? Isn't He
the One whose ways are veiled and therefore look like coincidences?
What more do you want than just to accept the coincidences ...?

\pa
Religion does not address the paradoxical. It addresses the paradoxical {\em in life}.
\pa
Can you say anything meaningful which is beyond you control? Yes, you
can. And beyond your spirit?

The limit of your spirit is where your words cease to have meaning. 
\qpa{
The limit of your intellect is where your words cease to have
meaning. The spirit exists where the dreams begin.}

\pa
It is the spirit that has me -- not I who have spirit.

\upa{
Our vocation is to listen -- not to talk.}

\pa
Spiritual acts are directed onto \HH, even if their object is an
actual thing. Communion consists in co-performing and co-understanding
such acts. What distinguishes people from animals is not that animals
do not have spirit (how could we know?), but that we can experience
full communion with people only. With animals we can at best communicate.

\pa
Can a life void of prayer be dedicated to God?

\pa
Questions ``What is God?'' or ``What is He like?'' make no sense. The
only meaningful question about Him is ``How should I relate to
Him?''. Then His attributes cease to be properties and turn out to be
mere analogies implying appropriate existential attitude.

\pa\label{ugood} (cf.~\re{happylot})
Balance and harmony are impossible as long as one nourishes the
conviction that the world is not ultimately good.

Admitting its indifference may be an act of intellectual honesty but,
equally, it is an act of existential despair.

\pa
To be is to participate in something which transcends me. Being is participation.

\pa
Everything that is is {\em in-between}. Even solitude and
understanding are just forms of communication.

\pa
Faith looks for understanding. Well, what does not? 

(Life is to make \HH\ present, even visible.)


\NOWY{Life, Person}\plan{Life, Person}

\pa
When every casual gesture, charming laughter, every innocent sign carrying
unknowingly a promise of the
possible happiness is taken as falshood, as a deceitful promise of an
impossibility -- then one's life is broken. 

\pa
Wise man thinks about future, but does not make plans.

\upa{
To hope is to have no expectations. Hope is the infinite patience of 
waiting for the \co{invisible} whic is, even though it never happens. 
\citf{For we are saved by hope: but hope that is seen is not hope:
for what a man seeth, why doth he yet hope for?
But if we hope for that we see not, then do we with patience
wait for it.}{Romans VIII, 24-25}
}

\upa{
Man is a borderline between the \LL\ and the \HH.
}

\upa{
His life is making the \HH\ present. It's meaning is to bring and keep heaven and earth
together %(cf.~\refp{pa:heavenearth}).
}

\pa
You will never have anything which you earlier did not sacrifice. (Not 
even yourself,)

Never desire anything you cannot accept not having.

\pa
Mistyka normalno\'sci. Mistyka racjonalizmu to choroba
niemo\.zno\'sci; racjonalizm mistyki -- nie.

\upa{
This is the great adventure of meeting new people -- they can make
impossible look obvious and natural.
}

\pa
Life unfolds through extremes -- but persists through mediocrity.
\qpa{
Cz{\l}owiek jest tym, co po sobie zostawia.}
\pa
We have not decided what we are. But we are responsible for it -- for
what we have not created.
\pa
We are not masters of our being.

Yet, we have the power of acceptance and denial, the power to strive
or withdraw, of rejecting or staying faithful (to what we could be).

\pa
Every explanation must stop somewhere. It tells you a great deal about a person if you notice where his explanations stop.
\pa
Every doubt can be resolved if one forgets it.

\pa
You want something -- try to get it. And if you cannot - try to forget
it. If you cannot forget it, not immediately but after months and
years, then it becomes yours.

The real is what we cannot live without. Also what we cannot stop
believing in and longing for.


\pa
To want something is to immortalize it. [Like to hope; to wait; ...]

\upa{
The question which the ideal theory of personal (subject's) identity
must answer in negative is ``Is the disintegration of personal
identity, the loss of one's inner source of being, motivation,
consistency, continuity possible?'' But this answer is obviously wrong!
}

\pa\label{happylot} (cf.~\re{ugood})
It is the prerequisite of a happy lot -- to turn disasters and misery
into advantageous experiences. (And it is something very different from
``learning from experience.'') 

\pa\label{goodevil}
God creates from nothing. 
Human creativity is to derive good from evil. [Tischner]

\qpa
{What is beautiful is impossible -- and what is possible isn't interesting.}
\pa
Never forget -- we walk on hell, gazing at flowers.

\pa
The ultimate consequence is typically the same as the bitter end. \\
Extreme boredom becomes a bit funny.\\
(Problemem prznikliwo\'sci nie jest, \.ze nie idzie do ko\'nca, lecz \.ze 
go mija.)

\pa
A man who has no expectations is unable to hate. Yet, he can be able 
to love.

\pa
Jeg er ikke useri{\o}s -- jeg bare ikke tar seri{\o}sitet s{\aa} alvorlig.

\pa
In principle, everything is possible but in practice, almost nothing 
happens.

\pa 
The phrase ``commonly accepted rules (of conduct)'' means nothing more 
than predictability.

\pa
Principles which some people claim to follow or, at least, to have, 
are completely different from the principles and laws of science: the latter
are designed so that almost all phenomena conform to and agree with them -- the
former are so that almost none do.

\upa{ 
Decision, being an act of reflexive consciousness, externalizes
its object, that is fixes/freezes it.  Thus it may last, independently
from our feeling s and state of mind, supported only by further acts
of will.
}

\upa{
What makes it difficult to accept (hard) truth is the faith, the hope that 
it is false. It isn't a lack of realism.
}

\pa
Do not defend it, do not fight for it, do not try to reach it -- just do it.

\upa{
Modesty depends on one's standards. If they are too high, others will 
rather see ambition and pride.
}

\pa
Nothing helps more for self confidence than to ignore the possibility of 
making mistake.

\pa
\wo{Two weeks in Prague with my girlfriend}, although an individual 
event, was common -- nay, {\em shared} between the two of us. Life, 
when it is shared, is neither yours nor mine. I do not share my life 
with you, you do not share your life with me -- we share life, {\em 
the} life.

\pa
Arrogance is the usual end of false humility.

\pa
A man is a dream -- his invisible dream. Don't say it -- dream it!

\upa{
Do not search -- find! If you start searching, you will hardly find.
}

\pa
If you think that \co{separation} has, essentially, some negative
connotation, then consider that it is the only event signifying
transcendence, otherness, that is, the only event enabling relation,
abolishing loneliness. If everything is mine -- or myself, if I am not
separated from anything, then I am alone.

\NOWY{``Morality''}\plan{``Morality''}
\pa
It is impossible to reatin any self-value (self-respect?) maintaining
nihilism towards the rest of the world. For nihilism is rejection of
committment, while it is only through committement that I can acquire -- that
I acquire -- a value, become worthy. To committ onself, to take up
responsibility is to say: ``I am worthy of carrying this responsibility.''
Committement is the only way of (living) recognition of the (value of a)
thing, person.

\pa
Responsibility is the stronger, the more we feel that the thing, the 
person, 
whatever, for which we are feeling responsible is not us, is not ours. 
On the one extreme, it is a \wo{moralistic} attitude of freezing, 
impersonal responsibility. But, on the other extreme, it is just an 
aspect of accepting a thing as us, as ours, and with it, the 
responsibility for it which is the same as the responsibility for 
ourselves. Here, the word \wo{responsibility} is no longer adequate. 
There is no longer \thi{something to be responsible for}, only 
something which we take as seriously as ourselves, an aspect of 
ourselves deserving the same care, or lack thereof.

\NOWY{Time}\plan{Time}

\pa
For the young days pass quickly but years slowly; for the old days pass slowly
but years quickly.

When you have seen or experienced years of friendship, long
lasting love which gradually, over the years, deteriorates and forgets itself,
when you have seen people being born and die -- what possibly could then happen
during one day, one hour? The moments lose their weight as the significant
experiences stretch over months and years, when the \thi{essence} of last 20-30
years gets condensed to a few observations, perhaps, just a single impression...

\pa
Time is the means by which we can value things. [The only way to endow things
with value is to devote them one's time.]

\pa
I do not have enough time to be in a hurry.

\pa
Time never goes back. It moves on and on, forever -- until you stop it.

\pa
A moment devoid of awaiting for eternity becomes a moment of desperate awaiting for
the next moment. \\
(And because a single moment does not bring anything, the hope fades away.)

\pa
``One can think that authentic time is originally an ecstasis; yet one buys
oneself a watch.'' [Levinas]

\pa
A man who never dwells on the past in order to always move forward, never
moves an inch.

\pa
You can't remove your past -- the only thing you can do is to change it.

\pa
\co{Sign} is the result of a duration (continuity) exisiting through
\co{actuality} (ontological perspective) -- every \co{sign} is a trace...

\pa
It is no big art to realize that one is a fool, but it is a true art 
to stop being one. 

We do not cure a disease by pronouncing the name of a medicine, but by 
taking it.

\citf{No one can be made perfect in one day.}{Theologia Germanica XIII}

\NOWY{Freedom, Others, Solitude}\plan{Freedom, Others, Solitude}

\pa
``Humility is a way of being oneself.''

\pa
Silent emptiness is not when nobody is speaking but when nobody is 
listening.

\pa
``Bo ja tam ko\'ncze\; sie\;, gdzie mo\.zno\'s\'c moja'' [Norwid]

\refp{goodevil}

\pa
You aren't free unless you can say ``the world is mine''.

Not ``this is mine'' or ``that is mine'' but ``the world is mine''. It
is mine because it is the way it should be, because I accept it from
the bottom of my heart. To have something means to accept it the way
it is from the bottom of one's heart.

\pa
The opposite of solitude is not sociality; it is personal contact with
other human beings, friends, family. The oposition solitude-sociality is
misuderstanding coming from socilogical opposition private-public life.
Society can never solve the problem of loneliness.

\upa{
``One day man will go mad to prove that he is free.'' [Dostojewski]
}

\upa{
To be nothing is to owe everything (to others). Everybody, also ultimate freedom
owes most it has. 

One remembers the unhappy events and days of childhood, complaining about family
and relatives who did not do their due... About the society which did not and
does not function in the way promoting personal happinness... And about others, met then
and now, who take aways one's spare time, money, possibilties of enoyment, one's
life. All such complains my seem justified but their only work is: enslavement. 

Nobody owes anything to freedom.
}


\upa{\label{flirt2}
Thirsting for the woods [origin, begining], we rise cities [construct, reason]. 
%}

%\upa{\label{flirt}
We flirt with time, yearning for eternity. %\refp{flirt2}
%}

%\upa{
Thirsting for eternity, we flirt with time. %\ref{flirt}
}

\pa You are threatened only to the extent you have something to
protect.  And you are the more threatened, the more responsibility you
assume for protecting what is yours.  [Exaggerated responsibility is
an arrogance.]

\NOWY{Understanding}\plan{Understanding}

\pa
We believe to understand other persons mostly by imputing them motives
which would be acceptable to us. 

\pa
Being \thi{ahead of one's time} involves being out of one's time. Society prises
such people (at least those of them, who happened to say something fitting the
{\em current} situation) but one should rather look at their lives...(and what?)

\pa
Fundamental principles (of philosophy) never explain details!

\pa
Which distinctions one is able to make and consider tells something 
about one's intellect. Which distinctions he does not consider tells 
more about himself.

The more we think, the greater the need for distinctions. We begin to 
be once we stop distinguishing.

\pa
Unity is the horror of the intellect.

\pa
Only illiteracy allows one to nourish the hope that {\em the} truth
has been written down somewhere (else).

\pa
The fall begins with explanations, the belief that everybody should,
or at least could, understand.

Except at the universities, all popular explanatory writings should be forbidden.

\upa{
Arguments never convince.

Argumentation is a sign of lacking respect - one tries to convince by explaining to
another what he apparently is unable to understand. If he only could, he would
accept our conclusion.}

\pa
Never accept an attitude or a view which you find implausible 
only because you cannot find a counterargument for 
not accepting it.

\upa{
There is nothing unreal. How could there be? It takes a lot of
disappointment to rise suspicion, and a lot of suspicion to claim that
the reality consists of two parts: ``real'' and ``unreal''.
}

\upa{
``Reality'' was invented as a medicine against insecurity and
uncertainty, as a reaction against one's failure. And immediately,
there followed the search for its infallible criteria.
}

Thus epistemology was born out of suspicions against ontology which
were nothing else than suspicions of one's own failures.

\pa\label{anymisunderstood}
There is nothing that could not be misunderstood. It is just another
side of the fact that it is so difficult to say something which makes
absolutely no sense.

\pa It is better to be not understood than misunderstood.  

\pa
Wisdom is the art of hiding -- and simplifying.

\pa
Wisdom does not make saints.

\pa
To be wise is not to know what one does not know but to know what
cannot be know.

\pa
Create order: 
simplify the complicated and complicate the simple.

\pa
``Life is simple but man insists on making it complicated.'' [Confucius]

\pa
``He that hath ears to hear, let him hear''.[Mat.11:15]

\pa
Idealism is an unrealized illusion -- or an unfulfilled reality.

\pa
Explanation is a reduction. This is what makes it so interesting and, 
from the existential point of view, so irrelevant.

\pa
In order to say something understandable we have to create a world.
\qpa{To speak is to create.}

\pa
The role of distinctions isn't to separate but to unify.

\pa
Among all the things which can be distinguished almost none can be defined.

\pa
Where is the borderline separating facts from feelings? But does the fact 
that it cannot be sharply defined mean that it does not exist?

\pa
Experience is the best reason to change a view.

\pa
Do not try to explain to yourself -- and especially to others -- what and why you are
doing. For the most, you do not know. And almost inevitable misunderstanding will lead you
inspiration astray. You'll understand when you have done it -- when you have become mature.

\upa{
To explain is to reduce. To understand is something very different.
}

\pa
The problem of spiritual understanding is not that it is hard to understand
but that it is so easy to forget what one understood.

\pa 
Truth is agreement with the origin. (At each level, it may be the local 
origin of the sign for the local correlate.)

\pa
It is one of the charms -- that contradictions which reflection so vehemntly
forbids to be and not to be at the same time and place, co-\co{exist}, often
quite peacefully, sometimes a bit violently...

\NOWY{Certainty}\plan{Certainty}

\pa
Only the fear -- the fear of the unexpected -- makes us obsessed with certainty.

\pa 
It is a common fallcay: the possibility of non-$A$ proves non-necessity of $A$ and
is immediately taken for $A$'s non-reality.

\pa
You want certainty? -- you have to lose even before you start playing.\\
You want to win? -- you have to forget that already at the start you lost.

When the heart weeps for what it has lost, the spirit laughs for what it has
found. [anonymous Sufi aphorism, after Huxley's Perennial Philosophy, p.106]

\pa
Defeat is the only source of strength. For to be strong is to be unafraid of
defeat (even of death). You are strong when you have learnt that it is
impossible to lose, no matter what defeat you suffer. 

\pa 
Anything that can be achieved by means of (operational)
understanding and (rational) control is suspicious -- that is,
uncertain.  For if one can arrange circumstances so that it appears,
then somebody else may arrange them differently so that it disappears. 
The more things are liable to manipulation, the more relative, 
changeable and, eventually, uncertain they become.

The only certain thing is the one which cannot be obtained by manipulation: grace --
the undeserved, incomprehensible gift.

\pa
Commitments come first -- responsibility is only a consequence.

\qpa{
Kiedy pojawiaja\;\ sie\;\ pytania, na kt\'ore nie ma odpowiedzi, 
oznacza to, \.ze nasta{\;}pi{\l} kryzys. [R.~Kapu\'sci\'nski]}

\pa
Never look for contradictions -- they do not exist.
\pa
The most subtle art of contradictions.


\NOWY{Meaning}

\pa
To live is to perceive things as meaningful.

\pa
All or nothing: either everything has a meaning and significance or nothing does -- 
eventually.

\pa
It isn't true that when you have a goal, perhaps even many goals, your life
acquires meaning. Often the doubt and questions about the meaning appear exactly
in the middle of hectic activity stimulated by several goals. Goals can, at best,
make you not feel meaninglessness.

\pa
Spirit is the meaning of things. I do not create it -- it is given to me ... or not.

\pa
``Everything is possible'' means ``the very best, highest can happen''.

\pa
Even things are expressions; indirect and imperfect but nevertheless
significant -- perhaps, the only expressions we have. This is what makes them meaningful.

\pa
Movement and meaning of the world concentrates in things -- dying out. (zamieraj{a}c)

\NOWY{Levels}

\pa
\HH\ in \LL: metaphor, contradiction, unclarity; \LL\ on \HH: confusion, 
misapprehension, contradiction.

\pa
The higher the spiritual level, the fewer possible things. Not because they 
are impossible but because they became unacceptable.

\pa\label{qpa:in}
Higher (or deeper) than \co{things} and concepts are feelings. But still
higher are things which we hardly can call feelings, things which we merely
-- and simply -- know (\co{invisibles}, cf. \refp{qpa:lin})

\upa{
The relation between higher and lower, invisible and visible is not that of
the general and the particular, that of instantiation and specialisation. It
is the relation of expression, of incarnation. 

Framing it in terms of generality and specificity, the tradition has
completely disregarded its nature, casting it in merely conceptual terms.
}

\upa{
It was the assumption that the higher, eventually the highest, somewhat
`contains' everything lower, that forced one to double things with ideas or
else, to make God responsible for all the details of this world.
}

\NOWY{Truth}

\pa
Often, most false assumptions may lead to most true conclusions, just 
like usually the true assumptions will yield most unbearable 
conclusions if only arrived at with sufficient degree of correct 
analysis and unbearable patience of reason \ldots 

\pa
There is a major difference between the possibility of \thi{me being mistaken}
and \thi{the thing being something in itself}. I can be mistaken because the
thing can be seen from a different perspective, considered in a different
context, used for another purpose. This, in turn, is very different from
\thi{thing being only the perspectives on it}.

The whole problem is with the transition between the \co{actual}, \co{visible}
and the \co{invisible}; in the more mundane words, between what is consciously
chosen and what is given for consciousness.

Thing is only a limit of distinctions, it is indeed relative to the
distinguishing being, an existence, but this \co{existence} is more than our
beloved \co{I}, not to mention \co{subject}. Thing is constituted
transcendentally, only that this transcendentalism is of a pretty concrete, not
to say empirical, kind. It involves my/human brain and perception mechanisms as
much as, or rather more than, my voluntary choices. \thi{Perspectives} are not
chosen -- they are given. 

They are given as much as the thing itself is given to consciousness. \co{I} can
not choose another perspective on it than those which are given with it, which
are inscribed into the distinctions constituting the thing. Indeed, things are
given us to the extend they are not underlied our control, to the extend they
keep something away. And as we dissolve more and more things, as we dissolve
more and more the apparently solid world around us, we ask more and more
intensely what really is there, where is this whole reality we have though so
solidly given and determined?



\pa
If we were to define it, we would say that truth is (agreement) with
\co{origin}. (At each level, \co{origin} can be understood differently.)

\pa
We do encounter examples of genuinely emotional reactions as well as of
highly detached mental attitudes. This, however, does not mean that there is
any line separating feeling from thinking, that they do not relate to the
same reality.

\pa
Truth is confusing. I mean, search for truth and only truth breeds
confusion. For even if not everything is truth, everything is at least an
aspect of truth, and one is soon drawning in the proliferation of true
observations, true remarks, true statements. The question is not ``What is
truth?'', not even ``What is true?'' but ``Which truth?'' (shall I live).

\pa
If you want to feel thinking, or think feeling, Being, try to think 
something which can be dissociated but which is not, some aspects 
which, although in principle separable, are in fact inseparable -- 
not because of any necessity, not because they {\em must} come 
together, but simply because they do.

Insight of identity of two things which we have considered distinct is 
but a meagre analogy of that \ldots

\NOWY{Women/Men}

\pa
Men attempt not to fail. (The least success is the lack of failure.) Women
try not to miss the truth. Or else: men are satisfied when they don't fail,
women -- when there is nothing missing.

\pa
Being concretely present means being close to the chaos; not overwhelmed and
defeated by it, but close to it, constantly threatened by it. For chaos --
the uncontrollable flux of reality -- is the ultimate, concrete content of
life and of each situation, suspending the separation of the external from
the internal. Therefore women, yielding to the insecurity of the fluctuations
of their moods and feelings, are much more concretely present than men are.

This, however, is only statistical and general truth, so to speak,
concerning the given nature.  In exceptional cases, it is typically a
man who can be \co{present} with the irresistible force of ultimate
\co{concreteness}.  Such \co{presence} is \co{founded} beyond
\co{chaos} of mere feelings, it is a rare \co{spiritual} event.

\NOWY{Letter of Count (hrabia) Joseph de Maistre}

to Natalia Wiaziemcow (fiancee of his son, Rudolph de Maistre), Spring 1813.
[Tygodnik Powszechny, No.43, 27.10.2002,  ``List Hrabiego'' w: Apokryf na
75-lecie Leszka Kolakowskiego] 

\pa
Masy ludowe szanuja rzad tylko dlatego, ze nie jest ich dzielem. Ulegaja
zwierzchnosci, bo czuja, ze to cos swietego, czego nie moga ani stworzyc, ani
zniszczyc.

\pa
... jest przeznaczeniem zwierzchnosci karanie zla i z tego powodu wladza jest
nieomylna w kazdej dziedzinie, w ktorej zlo sie ukazuje, a ukazuje sie w
kazdej.

\pa
Cala wielkosc, cala potega, cala dyscyplina spoczywa na kacie: on jest
postrachem i wiezia spolecznosci ludzkiej. Zabierzcie swaitu ow niepojety
element, a w jednej chwili lad ustapi miejsca chaosowi, trony upadna i
spoleczenitwo zniknie.

\pa
Zdolnosc do samoograniczenia sie jest dowodem pojmowania dystansu. A to z
poczucia dystansu, z {\em ironii} wlasnie, ktora tak pogardzasz, rodzi sie
usmiech [\'{s}niony mlodzieniec do hrabiego]
%\newpage

\NOWY{St.~Augustin}

\citt{Ojciec karci syna, handlarz schlebia niewolnikowi. Gdybys obydwa czyny --
kare i delikatnosc -- dal do wyboru, ktoz by nie wybral delikatnosci i nie
odrzucil bicia. Niejeden czyn wyglada na zewnetrznie dobry, choc nie wyrasta z
korzenia milosci. Podobnie i kwiaty maja ciernie. Niejeden czyn wyglada na
surowy i twardy, choc dokonuje sie go dla wychowania, z pobudki milosci. Dlatego
polecam ci jedno krotkie zdanie: Kochaj i czyn, co chcesz!}{Homilia na 1.List
sw.Jana (hom. 7,8) [Love and do what you want!]}

\la{Qui bene amat, bene castigat} (Kto kocha wlasciwie, ten nalezycie karci.)

\NOWY{Dostojewski, Biesy}

\pa
Byla to mysl cyniczna, lecz ludzie o duszach wznioslych czasem nawet maja
tendencje do mysli cynicznych, chociazby tylko dzieki wszechstronnosci  swego
rozwoju. [I.1.iv] [Stiepan Wierchowienski, p.22]

\pa
Wolnosc nastapi wtedy dopiero, gdy juz bedzie wszystko jedno: zyc czy
umierac. Oto jedyny cel. [I.3.viii] [Kirillow, p.119]

\pa
Od prawdziwego cierpienia madrzeli nieraz glupsi, nie na dlugo,
oczywiscie. Cierpienie ma juz taka wlasciwosc. [I.5.vii] [ad Stiepan
Wierch. p.303]

\pa
Sa mysli wieczne, ktore nagle staja sie nowymi. [II.1.v] ] [Kirillow, p.237]

\pa
Tak juz wszystko jest nudne, ze nie ma powodu przebierac w rozrywkach, byleby to
bylo cos ciekawego. [II.5.ii] [Jedna z panius, ktora chce tez zobaczyc cialo
samobojcy, p.325]


\upa{
Nihilistami w istocie rzeczy bylismy my, wieczni poszukiwacze idei
nadrzednej. Teraz rozmnozyly sie badz zobojetniale miernoty, badz
mnichowie. [Dostojewski w notatnikach, p.725 [18]]
}

\pa
\fre{Libert\'{e}, \'{e}galit\'{e}, fraternit\'{e}}

\pa
Jezeli nie ma Boga, to ja jestem bogiem. [III.6.ii] [Kirillow, p.610]

Uswiadomic sobie, ze nie ma Boga, i nie uswiadomic sobie jednoczesnie, ze sie
samemu jest bogiem -- to nonsens. [III.6.ii] [Kirillow, p.612]

Jesli jest Bog, to i ja jestem niesmiertelny. [III.7.iii] [chory Stiepan, p.656]

\pa
Wiedzialem o tym od dawna, lecz widze dopiero teraz. [chory Stiepan, p.645]


\upa{
Tutaj najmniej lubilem mieszkac. Ale nawet tutaj niczego nie moglem
nienawidzic. [...] Moge zechciec zrobic dobry uczynek i robi mi to
przyjemnosc. Ale juz zaraz pragne zlego i tak samo czuje przyjemnosc. I to, i
tamto uczucie jest jak zawsze zbyt plytkie, a nigdy nie chce bardzo. 
[III.8.epilog] [Mikolaj Stawrogin, p.666-7; ad. Sade]
}

\pa
Wszelkie sytuacje niezwykle hanbiace, nadmiernie ponizajace i, co najwazniejsze,
osmieszajace, w jakich sie w moim zyciu znalazlem, zawsze wzbudzaly we mnie na
rowni z bezmiernym gniewem, niewiarygodna wprost rozkosz. [U Tichona]
[Stawrogin, p.682 - cf. Rousseau [9]; ad. Sade, silne przezycia]

\upa{
Wiem, ze powinienem zabic siebie, zmiesc siebie z powierzchni ziemi jak
szkodliwego owada. Lecz boje sie samobojstwa, gdyz boje sie okazac
wspanialomyslnosc. [III.8.epilog: Stawrogin, p.667; ad Th.Germ. self-despise]
}

\pa
Pragne, aby bylo wiadome, ze ani we wplywie srodowiska, ani w chorobie nie
zamierzam szukac usprawiedliwienia dla moich zbrodni. [U Tichona] [p.683]

\upa{
Kazdy czlowiek grzeszac, grzeszy przeciwko wszystkim ludziom i kazdy czlowiek w
jakims stopniu winny jest cudzego grzechu. [U Tichona] [Tichon, p.699] 
}

\pa
Tu wydarzylo sie to, o czym pisze Ewangelista Lukasz [.. Lk.VIII:32-36..]
Biesy wyszly z czlowieka rosyjskiego i weszly w stado swin, to znaczy
Nieczajewow i innych. Potonely one lub potona na pewno, a uzdrowiony czlowiek, z
ktorego wyszly biesy, siedzi u nog Jezusowych. Tak i byc powinno. Rosja
zwymiotowala te ohyde, ktora ja karmiono, a w owych zwymiotowanych nikczemnikach
nie ma juz, oczywiscie, nic rosyjskiego. I niech pan zapamieta sobie,
przyjacielu: kto traci swoj narod i narodowosc, ten traci i wiare ojczysta, i
Boga. Jesli pragnie pan wiedziec -- wlasnie to jest tematem mojej
powiesci. Nazywa sie one Biesy is stanowi opis tego, jak biesy weszly w stado
swin. [Dostojewski: list to Apo{\l}{\l}ona Makowa z Drezna, 9/21.10.1870]


Lk.VIII: 
[32] And there was there an herd of many swine feeding on the mountain: 
and they besought him that he would suffer them to enter into them. And he 
suffered them.
[33] Then went the devils out of the man, and entered into the swine: and 
the herd ran violently down a steep place into the lake, and were choked.
[34] When they that fed them saw what was done, they fled, and went and 
told it in the city and in the country.
[35] Then they went out to see what was done; and came to Jesus, and found 
the man, out of whom the devils were departed, sitting at the feet of 
Jesus, clothed, and in his right mind: and they were afraid.
[36] They also which saw it told them by what means he that was possessed 
of the devils was healed.


\NOWY{Herbert, \btit{Kr\'{o}l mr\'{o}wek}}


\pa
\wo{chtoniczny} = \gre{chtoni\'{o}s}, zrodzony z ziemi, podziemny

\pa
Los i lokaj losu -- przypadek. (Fate and its lackey -- chance.)

\ad{Narcyz}
...poszukuja daremnie sensu \.{z}ycia na dnie pustej duszy... [p.63]

Jednakze brutalno\'{s}\'{c} i g{\l}upota potrzebuja spoiwa, trzeciego
pierwiastka, by stworzy\'{c} trwala moleku{\l}e charakteru. Tym pierwiastkiem
jest najcze\'{s}ciej sentymentalizm. Wiec i on zakocha{\l} sie. [p.63]

Uroki opieku\'{n}stwa zblad{\l}y. [p.64]

\ad{Kleomedes}
Ale  katastrofy sa przecie\.{z} niewinne. [p.55]

Poni\.{z}enie daje pewno\'{s}\'{c}. [p.57]


\upa{(ad. Contradictions)
\begin{quote}
Odi et amo. Quare id faciam, fortasse requiris. \\
Nescio, sed fieri sentio et excrucior.\\
(Kocham i nienawidze. Jakze sie to dzieje? Spytasz.\\
Nie wiem, lecz czuje, cierpie i szaleje.)
\end{quote}
[Katullus, \btit{Poezje, Piesni:85}]
}