\subsubnonr{Reality}

\found{Acceptance of the world (through which it becomes `mine')}
    {sensualism/hedonism - here-and-now}
    {possess/control}
    {power (world may be evil)}
 {accept/love}
    {understand/feel}
    {care/respect}
    {enjoy}


%\addcontentsline{toc}{subsection}{\ \ \ The founding of reality}
Everything that is is real and so, since everything is, everything is real.
Thus, there is nothing to speak about unless we bring in some finer distinctions
into this all-embracing sphere. We could divide it, for instance, according to
the scope of relativity (\co{distinctions} may be relative to a particular
\co{existence}, particular kind of \o{existence}, community, etc.). But this
would be still too \co{objectivistic}. Reality, as everything else, happens on
the line of \co{confrontation} between \co{existence} and \co{One}; as
everything else, it concerns the character of this \co{confrontation}.
Granted the ever \co{present transcendence} of \co{nothingness}, 
 we no longer distinguish experience of reality and experienced reality.

Speaking \co{objectively}, reality is some \co{totality} of \co{external
objectivities}, which \co{subject} can endow with various meanings and
values. As we have dispensed both with \co{totalities} and \co{objectivity}
\thi{in itself}, we can no longer \co{dissociate} reality from the meaningfulness
with which it is encountered. Meaninglessness is a particular modification of
the experienced reality

Meaningfulness has little to do with meaning (as
we described it in Book I). Meaning of signs is some nexus of
denotation-connotation-extension-intension which does not concern us any
longer. Meaning, on the other hand, is not any extrinsic correlate but the
inherent value; we could almost say: meaning is value.


Reality can be taken as the \co{objectified} pole of the relation of value and
meaning, of which the other pole is \co{subject}. But such schema is a result of
\co{dissociation} of the \co{actual aspects} which, primordially, are not even
formally distinct. 
%Reality is inseparable from the meaningfulness and worth

Reality is what you cannot live without. 
The degrees of reality amount to the degrees of importance and relevance and, as
we will see, the degree of individuality (its inseparability from the
\thi{essence}) which diminishes as we go down...

%Or are we talking here only about the \thi{sense of reality}...? \co{concrete}...


\ad{\inv The foundation}
And thus, God is the \co{one} who has nothing but the proper name, who is most
genuinely named in the recognition of \co{indisitinctness}, in the resignation
from any search for an adequate linguistic expression, an \co{actual}
name. The meaning of any \co{actual} name is exhausted by the \co{acts} of
naming. The question is not which names might be adequate for God but, on the contrary, 
what aspects of reality can be adequately named
\wo{God}.\ftnt{Cf.~\citeauthor*{Schelling}} 
%The question is rather: what does the name \wo{God} name?
The two standard kinds of answers were:
\begin{itemize}\MyLPar
  \item[1.] God is the {\em only} reality, or
  \item[2.] God is the {\em highest} reality.\vspace*{-1ex}
\end{itemize}    
The first one may be associated more with the Eastern philosophy and
religiosity, but also with all Western movements negating the value of \co{this
  world}. (The exclusive reality of unchangeable Being and complete unreality of
all changeable phenomena according to Eleates; the \gre{charismos} of Plato's
cave, the extreme dualisms of gnostics, of 
bogomils and khatars, the exaggerated 
ascetic tendencies of monasticism...)

The latter characterizes roughly many forms of Western mysticism, which did not
accept Parmenidean proofs that nothing can ever emerge and, in particular, no
multiplicity can arise from the unity of Being: starting with Plotinus, followed
by most of neo-platonism and Christian mysticism from Pseudo-Dionysius til
Eckhart and beyond. (There is also the answer that God is the \co{totality} of
all realities, but we are done with pantheism.)  As usual, the borderlines
between such rough distinctions are not very sharp, and they become even less so
when we distinguish the latter answer from yet another possibility, namely,
that:
\begin{itemize}\MyLPar
  \item[3.] God is the {\em foundation} of everything (all reality).\vspace*{-1ex}
\end{itemize}
\co{Foundation} can be taken here in the double sense of \gre{arche}: as the
\co{origin} from which everything emerges {\em and} as the omnipresent
principle, not so much organizing and determining the detailed course of all
action, but surrounding everything as its ultimate and ever present horison.

Both answers 1. and 2. diminish the value (or reality) of the \co{actuality} and
refer one to a sphere beyond, either as an imperative goal or else only a
direction of striving. The third answer recognizes this \co{transcendence} too,
but its intension is to discern its \co{presence} in all the reality, without
dividing it into more and less real. Everything which is \co{distinguished} is,
and everything which is, is real. The \co{original virtuality} is neither more
nor less real than any \co{actual distinction}, although it is the prior
\co{foundation} of the latter. 

\pa
\co{Eternity} is the \co{presence} of the trans-temporal in time, the 
\co{immanence} of the \co{transcendent} God in all \co{actuality}. But in terms of
\co{actual experiences}, it is \co{nothing}.
%
% this may be a moment in the sense of Kierkegaard, a
% moment of the ultimate \co{confrontation}, but it may also be much less dramatic
% sense of \co{participation} which, announcing the \co{one}, announces
% everything: 
% \citet{Kto by posiadl umiejetnosc i moc, azeby czas i wszystko, co sie dzialo
%   kiedykolwiek w ciagu szesciu tysiecy lat lub mialoby sie jeszcze zdarzyc do
%   konca swiata -- to wszystko wbrew nauce wprowadzic w jakas obecna chwile, to
%   bylaby pelnia czasu. To jest chwila wiecznosci, wtedy dusza poznaje wszystkie
%   rzeczy w Bogu jako nowe i swierze...}{Eckhart}{\citaft{Mistyka}{, p.87}}
%
God is the \co{eternal} dimension of life, of every \co{actuality}. He is not
more \co{present} there than he is here, for he \citet{exists as a whole at every time
  and in every place.}{ReplyG}{1}\noo{His eternity persists altogether simple
  and incorporeal. Therefore, he i snowhere locally (that is, enclosed by a
  location) and yet everywhere (that is, he's said to be both {\em in} all
  places and {\em around} all places through the power of his
  operating).{AbelardPJC, II:341,p.128}} His \co{presence}, however, is not given, it
is not a matter of any particular \co{experience}, he is \citet{within me and
  around me and I do not have any experience of [him].}{Proslogion}{16} This
\co{presence} is \la{a priori} because it reflects the ontological
\co{foundation} in Godhead, because it is always and everywhere possible, but
its \co{concreteness} -- not \co{actuality}! -- depends on the \co{openness} of
the \co{existence}. And if 
\co{I} say \No, if \co{I} die -- ``What will you do, God?  I am worried.''

\tsep{}

The real is what you can not live without; and (consequently, but only
accidentally) most real is also what
is most deeply \co{shared} -- the \co{origin}. 

Tischner


\ad{\mine }
UN: \co{Thirst} is the deepest manifestation of reality, is the way the ultimate
enters \hoa\ when this is not \co{concretely founded}. 
Happiness is but a bad image of what we \co{thirst} for.

-- motives (relative to me, but world does not care for me, so: Stranger in the world.


FOUND: Indeed, happiness may
be a form of the consequence, but never more than that...


\ad{\act}
UN: That which gives a resistance; the project of power and control. Need of
definite \thi{sense of reality}, whose lack is indeed a mortifying suffering.

-- goals

FOOUND: Care and respect for everything


you can not live without these most detailed objects and things; but! their
individuality plays almost no role! You can live without {\em this} chair, but
you need something to sit on; you can live without any particular thing, but not
without any thing at all; Individuality of particular things is not real unless
one endows them with more meaning than is usual... They start to mean...

\ad{\imm}
UN: only \herenow; -- hedonism, estetic, 

FOUND: enjoy


\tsep{}


\ad{Nothing \& Reality -- Meaning}
When they wanted to humilate one of the Merovings, they cut his hair.
And indeed, he had to yield...

When I want to keep the memory of my mother, I keep her picture on my
desk... \co{Visible signs} not only \co{actualize} but also, by
\co{externalizing} \co{objectify}. They are (at least, can be) tokens of the
higher which itself remains \co{invisible}.

\co{Signs} are \co{manifestations} of the \co{invisibles}.
When \co{dissociated} from this relation, when reduced to the merely
\co{visible} content, they 
may still point, denote, announce, signify.... but they cease to mean. 

Meaning (and meanigfulness) is a crucial aspect of concrete founding... For
meaning emerges as the result of acceptance, and only in such a way
\citet{What strikes you was not there to miss you; what misses you was not there
  to strike you.}{Al-Ghazali}{99-100 \citaft{EvilFaces}{, p.55}} It does not
mean any defetism, 
determinism, pre-ordained fatum, although, truly, it may look like that. But the
point is not the question \wo{How the world is organized in-itself?}, but only
\wo{How do I take what confronts me?} This kind of fateful manner of speaking
is only an expression of the underlying theme of \co{humble} acceptance, of not
arguing about details nor, for that matter, about important matters, to accept
one's limits and never desire nor even imagine to transgress them. A brick
falling on my head may be called \wo{an accident}, \wo{a sign}, \wo{fate}... But
the only thing which matters is whether such a name expresses the \co{humble}
acceptance, the \co{non-attachment} of the one using it, or else his fundamental
disappointment, perhaps even \co{detachment}...





\noo{
\subsubnonr{Know, Being - theories}

\levs{10}{Know}
 {sense/perceive}
 {think}
 {feel}
 {\co{know} (intuit)}

[Jung, MHS, p.49]
\aretabb{rl}
  {sensation: & that something exists}
  {thinking: & what it is}
  {feeling: & its value (agreeable or not)}
  {intuition: & whence it comes and where it is going}

\noo{
\small{
\begin{tabular}{ll}
\parbox[t]{7.5cm}{ \levels{Beings} wrt. \\
\begin{tabular}[t]{rl|l|l}
        &  power & communic & with us  \\ \cline{2-4}
        1. & finite & closed & limited \\
        2. & finite & open & limited \\
        3. & finite & open & unlimited \\
        4. & infinite & open & unlimited 
 \end{tabular} \\[1ex]
1.things, 2.world,biological,\\ 3.social,personal, 
4.absolute,religious}
}

\small{
\parbox[t]{7.5cm}{ 
%\levels{Social activities}
%\are{craft-science-technology}
%    {private life, institutions}
%    {community life, morals-ethics}
%    {religion} \\
\levels{Quantity} (time/space)
\are{point - pure here \& now}
    {finite \& limited}
    {finite \& unlimited}
    {infinite \& unlimited} 
%\end{tabular}
}}
}% end \noo{ inner!!!

\levels{Being/positive and negation} (cf. Eriugena)
\are{this here -- not \herenow}
     {things' now -- not yet/not any more}
     {world's now, experienced/human -- unworldly/non-experienced}
     {God/One -- visibles/below}
In addition (since life/all is one): at-this-level: is -- below/above: is-not

} % end \noo{ around the whole \subsubnonr{Being...

