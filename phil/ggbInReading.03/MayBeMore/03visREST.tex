
\section{Visible to Invisible}
\secQQQ{8}{When you see your likeness, you are happy. But when you see
your images that came into being before you and that neither die nor become
visible, how much you will have to bear!}{The Gospel of Thomas, 84}
\plan{The attitude of Man}
%
%\say
%Ontological founding -- `abstract'; `true', `axiological' founding \ldots

\pa 
Our vocation is to listen -- not to talk; to listen to the silent 
\co{presence} of \co{invisible} movements which direct, if not 
change, our life. But what does such a listening mean?

\citt{The wind isn't a good listener! The wind wants to  speak, and we know how
  to listen. My father always told me that an Eskimo is a listener. We have
  survived here because we know how to listen. The white people in the lower
  forty-eight talk. They are like wind; they sweep over everything.}{Anonymous
  Eskimo, Native American Testimony [p.432]}

We think that emptiness, nothingness confronts us when we
desperately try to listen to the void from which no voice reaches our
ear and when \citt{[t]he eternal silence of these infinite spaces frightens 
me.}{Pascal, Pensees, III:206}
We think that emptiness is when we can not hear anything, when
no voice speaks to us.  Our vocation is, after all, to listen -- not
to talk.  But such a listening, such a search for words, voices, for
\co{visible signs} is not listening but demanding.  True emptiness is
not when nobody speaks but when nobody listens, when whatever we have
to say ends in a void, or else when the void which we find within 
makes us mute. For to listen is to express, to listen is to
fill one's heart with the abundance from which the mouth could speak. 
To listen is to be full of the \co{presence} of \co{invisibles} which
speak without saying, which \co{command} without directing, which are
\co{present} without being \co{actual}.  
To hear the \co{invisible} is to transform it into \co{visible 
expression}.

The \co{invisibles} are there, they are \co{present}, and you do not
have to look for any \co{visible signs}.  All this is there and you do
not have to spend time looking for it, for if ``you are looking for
it, you have already found it'' or else, what amounts to the same, the
best way to loose it is to look for it.  You are not only
\co{yourself} but also \co{your Self}.  All this is there and, in
fact, that art thou.


\pa All this is there and that art thou.  But what is that, what is
this \thi{all}?  What is there?  \co{Nothing}.  There is \co{nothing}
there to be known, there is \co{nothing} there to be found.  Such
questions ask for a \co{visible sign}, for an answer which would
elucidate, perhaps, explain something.  Man gains depth when he
realizes that some questions should not be answered -- even if they
should be asked.  To understand something is to know its limits
(\refpp{pa:understand=limit}), is to know what \co{distinguishes} this
something from other somethings.  What \co{distinguishes} \co{the One}
is the lack of \co{distinctions} and it \co{distinguishes} it from
everything which we possibly might understand (\refpp{nothingtoknow}). 
In the face of that, multiplying \co{distinctions} is not necessarily
a profitable process -- the fact that a \co{distinction} is possible
does not mean that it should be made or, as Ouspensky aid once
\citt{if we could get rid of half of our words perhaps we should have
a better chance of a certain understanding.}{P.D.Ouspensky, {\em The
Psychology of Man's Possible Evolution}, Lecture V [originally, in
{\em Six Psychological Lectures}, Historico-Psychological Society,
London, 1940]}


\pa
This understanding is now our starting point. More specifically:
\begin{enumerate}
    \item There is \co{nothing} to be known about the \co{One}, except
    that it is \co{One}, i.e., the undifferentiated \co{origin}, the
    ultimately \co{invisible} source of all \co{disitinctions}.  
    \item
    The \co{invisible transcendence} of this \co{origin} is what is
    most intimately \co{present}, although it forever remains 
    \co{invisible}, i.e., outside the\co{immanence} of \co{actual 
    distinctions}.
\end{enumerate}

\pa Now, knowledge is a function of being.  The very fact of being is 
in itself an event of knowledge, of knowledge in a very generous sense 
as the involvement in the world of \co{distinctions}. 
The things \co{I} know,
the way \co{I} understand the world, depend on
%, if are not entirely determined by 
the kind of person \co{I} am.  Being, on the other hand,
is not a simple function of knowing. 
In particular \co{my} being, \co{founded} by 
\co{birth}, is a \co{separation} which precedes any possibility of 
knowing.
The \co{distinctions} in which 
life involves me by far \co{transcend} what we would, in a more specific 
and \co{precise} sense, call knowledge. 
My knowledge, my \co{reflection}
can exercise some power over the \co{actual signs}, but not
directly over my being.  It may direct my \co{acts} and
\co{actions} but it does not determine who \co{I} am.  However,
\co{actuality} is the only place \co{I} can start, this is the only
place over which \co{I}, to some extent, may have some control.  It is
the field where \co{my} will can unfold, because the details of
\co{actuality} are exactly what is not determined by the
\co{invisibles}.  The two fields, \co{visible}
and \co{invisble}, meet in man who is a border between them.  Our
life is making the \co{invisible present}, \wo{as in heaven, so in
earth}, while the goal of our \co{acts} and \co{actions} is to keep
heaven and earth together, for \wo{whatsoever thou shalt bind on
earth shall be bound in heaven: and whatsoever thou shalt loose on
earth shall be loosed in heaven.}%{Mat.  XVI, 19}
Although there is \co{nothing} to know about the \co{origin}, there is
something to do.  And although whatever we do takes place within the
\hoa, it affects the whole hierarchy of Being, the whole \co{rest},
which is \co{present} in every \co{actual} instant.
Indeed, to cure a sickness we do not pronounce the name of a 
medicine, but we take it. And it is not difficult to realize that one 
is a fool but the art is to cease being one. 
\citt{No one can be made perfect in a day}{Theologia Germanica, XIII} 
because becoming perfect concerns what we are and this is never 
limited to the mere \co{actuality}.

\say
Explain here the relation and difference between \thi{ontological 
founding} = \co{presence}, and \thi{axiological one} = \co{incarnation}.

\pa
Keeping heaven and earth together, listening through \co{expressing}, 
is thus a matter of \co{actual} and \co{visible} actions. But what is 
most important in such actions is not their \co{actual object}, their 
intentional content but, on the contrary, their \co{invisible rest} --
in a more direct way, we might say, the attitude and disposition of 
heart and feeling which accompany these actions.

\noo{
\pa There is \co{nothing} to be known about \co{the One}, no name is
adequate for Godhead.   The attributes which one has
always been tempted to ascribe to God are truly the attributes of our
attitude; not necessarily of our conscious and self-conscious 
attitude, certainly not of our intentions and objectives, but of the 
\co{rest} accompanying these actions, the \co{rest} which \co{manifests} 
the form and quality of the contact with the \co{invisibles} and their 
\co{presence}, which 
\co{manifests} the deepest site where 
\co{One} becomes Many, where Godhead becomes God igniting 
the \wo{spark of the soul}. 
}
There is \co{nothing} to know about the \co{origin}, but there is
something to do. 
What \co{I} can do, is limited to the \hoa, and what \co{I} can do 
while writing, is merely to try to \co{express} the understanding 
which should accompany whatever \co{I} do.

\pa
\co{Reflection} can not secure that what happens \thi{up there}, in 
the \co{other world}, 
happens the way it should. It can only have a \co{vague} idea -- 
\co{vague} and \co{clear} -- of what should happen, of what would be 
desirable and agreeable. And it not only can have such an idea but 
it actually has it. 

\subsection{The search}

\citt{Prospera in aversis desidero [in adversity I wish success]}{Augustin,
The Confessions, X:28 [but also ``adversa in prosperis timeo'' [in prosperity I
am afraid of adversity], X:28]}

%%%%%%%%%
\subsub{The calls}\label{sub:calls}

\pa \co{Reflection} finds itself always involved in the interpaly of
\co{invisibles} but this interplay is precisely that -- an interplay,
a mixture, affected once by this, once by that, sometimes tending in
this direction, sometimes in that.  Life without \co{spiritual}
comittements may bind various levels, may keep heaven and earth
together only as if by accident.  But it may also, and usually it
does, pass in a constant discordance between various levels and
\co{experiences}.  
\citt{Let both grow together
until the harvest: and in the time of harvest I will say to the
reapers, Gather ye together first the tares, and bind them in bundles
to burn them: but gather the wheat into my barn.}{Mat.  XIII:30/Lk. 
III:17}
%\citt{Let both grow together until the harvest.  At
%that time I will tell the harvesters: First collect the weeds and tie
%them in bundles to be burned; then gather the wheat and bring it into
%my barn.}{Mat.:XIII.30, Lk.:III.}

\pa We can on some occasions \co{experience} moments of rare elation,
moments which show us the joy of love and unity.  \co{An experience}
of beauty, or else a simple \co{act} of charity, generousness,
compassion, when performed in due humility, may fill the \co{soul}
with the feelings of unity and \co{participation}, in fact, can
\co{manifest} the \co{presence} of \co{love}.  One does not have to
\co{attentively reflect} over it.  A life which one \co{reflects} over
and summarizes as good and satisfying, with a clear conscience (which,
of course, glosses over some small mistakes and minor sins), with a
genuine conviction of its gratifying value, for oneself as well as for
others, may give a feeling of a \co{vague}, almost mystical, harmony. 
 

\pa
%Jung's `mystical unity'\ldots
\citt{From a low hill in this broad savanna a magnificent prospect 
opened out to us. To the very brink of the horizon we saw gigantic 
herds of animals: gazelle, antelope, gnu, zebra, warthog, and so on. 
Grazing, heads nodding, the herds moved forward like slow rivers. 
There was scarcely any sound save the melancholy cry of a bird prey. 
This was the stillness of the eternal beginning, the world as it had 
always been, in the state of non-being; for until then no one had been 
present to know that it was this world. I walked away from my 
companions until I had put them out of sight, and savored the feeling 
of being entirely alone. There I was now, the first human being to 
recognize that this was the world, but who did not know that in this 
moment he had first really created it.}{C. G. Jung, {\em Memories, 
Dreams, Reflections}, p.251 [Creation of Concsiousness, p.14]}

\pa Such \co{an experience} may be a rare \co{gift}, although the
history, especially of religion, and literature record quite a number
of them.  Yet, it is only \co{an experience}, the feelings and
thoughts which it provokes are nothing more than feelings and
thoughts, intense and deep but still only \co{vague} intutions,
perhaps, \co{vague} \co{inspirations} to continue the life so that
they will recur.  If one does, and if the \co{invisible} currents of
\co{virtual} unfoldings favour one, one may indeed remain blissfully
satisfied and content.  But one may also encounter moments of
confrontation, moments when one realizes that it is not merely a
matter of \thi{things and life going this way}, but of one's attitude. 
There is an extensive literature on experiences which enter one's life
as genuine \co{calls} -- sometimes, but not neccesarily, calls to do
something, yet never coming from any specific agent, but always from a
strange power which is as tremendous as it is contentless,
non-\co{objectifiable}.\footnote{\label{ft:mysticexp}Job's Book, or
St.~Paul's vision on the way to Damascus are well known.  W. James'
{\em The Variety of Religious Experience}, or its XXth century
analogue, {\em The Perennial Philosophy} by Aldous Huxley, contain
plentiful examples presented in unusally lucid and accessible forms. 
Eliade's writings are an excellent source, too, and even Rudolf Otto's
{\em Das Heilige}, whose conceptual structure reflects and 
systematizes the aspects
of concrete experiences.  Kierkegaard's {\em The Concept of
Anxiety} is a classic analysis.  Almost any text of the mystics
contains its own, even if often unbearably elaborate, examples.}
%
Such moments may come at most unexpected time, but when they come,
they come as a thunder, an ultimate confrontation with the
unconditional \co{invisibility}.

\pa
\citt{One night I was seized on entering bed with a rigor, such as 
Swedenborg describes as coming over him with a sense of holiness, but 
over me with a sense of guilt. During that whole night I lay under 
the influence of the rigor, and from its inception I felt that I was 
under the curse of God.}{Quoted by W. James, Varieties of Religious 
Experience, IX: footnote 3, p.198 [Scholars identify several 
\thi{quotations} used by James as coming actually from himself. This 
might be one of them.]} 

perhaps, some better examples ???


\pa\label{pa:exAbsChoice}
Let's observe an important distinction. One will easily recognize rare 
moments of confrontation with some existential choice in which one 
has to make a decision known to be of utmost importance for one's life, 
perhaps the whole life. Important as they are, they are still concerned 
only with \co{my life}. We are not talking about that. 

% neither do we speak about any specific choice\ldots

We are talking about moments of confrontation with the \co{absolute
nothingness}, perhaps, with the ultimate Void, perhaps, with the
ultimate Power, with Being \thi{in the state of non-being}.  Whatever
the form of the \co{actual experience}, that is, whatever its concrete
content (and its variations may be as numerous as the known examples,
e.g., those from footnote~\ref{ft:mysticexp}), it confronts me with an
\co{absolute} objectivity, definitely \co{not-me}.  It is
\co{absolute} because it does not concern any thing but is immediately
{\em known} to apply to everything, to the {\em essence} of existence,
which happens to be also \co{my} existence.  We may easily call it a
\wo{feeling}, because in its complete lack of the \co{objective}
element, it is unique and unitary -- it does not come {\em from}
\co{One}, it is \co{One}.  But it is equally knowledge: when we
recognize its \co{commanding} power, we know that beyond this unity
there is nothing to be \co{distinguished}, there is no need for any
more \co{precise concepts}, which would unfold its \co{complexity}.

\subpa Do we really know this to be so?  The worst thing one can do is
to try to bring the \co{objectless}, contentless quality of such
\co{an experience} to the level of \co{precise} and understandable
thoughts, to try to grasp and explain to oneself what is the matter. 
\co{Nothing} is the matter. 
Angst, the Angst as Kierkegaard describes it but also in a more 
mundane sense, is a natural reaction of
one confronting the unconditional for the first time.  It has as 
repulsive as attractive power, like an abyss which \thi{draws} a 
person with high-anxiety into itself. 
A better
prepared person may feel awe or overpowering intensity, perhaps, the
presence of God in all his venegance and anger which all, too, invite
to an escape.  The attempts to bring it down to the level of
\co{visibly} understandable is a possible way of tampering it.  Thus,
we can obliterate its meaning.  But even if we do so, we will still
{\em know} that we have done just that -- denied it the attenion it
required.  We can imagine that with time, this knowledge may be
forgotten, especially if we encounter no more \co{experiences} of this
kind.  But I doubt that anybody with such an experience will forget it
-- one may only try to ignore it.  This is a possible, albeit rather
passive, way of saying \No.

\pa\label{pa:contNothing} There is a much more important distinction
to be made than that from \refp{pa:exAbsChoice}.  Often, one will
focus on the specific content in such \co{an experience} and follow
its \co{call} only in respect to this content.  If the experience
announces wrath, one may create a wrathful and venegant God; if it
announces unbearable emptiness, one may feel left and abandoned by God
and act accordingly; if it announces peaceful acceptance, one may
close one's eyes to pain and see only rosy cheerfulness.  Different
gods of polytheism and henotheism often took on forms corresponding
not only to lower levels of manifestation but to different variants of
such ultimate experience.  The psychological variations are
innumerable, and they may result in as many \co{idols} as in genuine,
\co{concrete expressions}.  In order to be the latter, the quality of
\co{the experience}, its \co{rest} which here is essential, must not
be confused with its \thi{what} -- the \co{invisible
nothingness}.\footnote{Rudolf Otto calls such experiences of
\la{numinosum} focusing on the content \wo{preparatory stages}.  They
must not be confused with the \la{numinosum} which is first of all 
\wo{something entirely different}, \wo{something more}.  For instance, \citf{[f]rom
demonic fear no cult could emerge, unless in the form of
\wo{gre{apaiteisthai}} and \wo{\gre{apotrepein}}, i.e., in the form of
repentance and \he{przeb{\l}agania}, of calming and redirecting the
anger.}{Das Heilige, VI, p.44}}

Variations in the \co{actual} contents notwithstanding, such \co{an
experience} is \co{an experience} of the \co{One}.  The \co{One} which
does not appear as \thi{it is in-itself}, but which nevertheless
appears -- not in the content of the experience but through its very
{\em intensity}.  It is the irresistible force of \co{nothingness},
the undeniable objectivity of the \co{objectless} which only exercises
its power through the \co{actual} contents of the experience.  The
ineffability of such experiences refers to this intensity, to thier
\co{nothingness}, not to their psychological content and character
which, to a degree, may be described and communicated.

\pa
The \co{command} of such an experience is to forget yourself or, 
what comes to the same, to tame the terrifying, to accept that it does 
not threaten, to embrace it.\footnote{If this sounds like a 
psychological contradiction then it probably is. \citf{We worship 
holiness with fear, but we do not escape from it, but even more are 
attracted to it.}{M.Luther, Speech about good deeds \fre{\'{a} propos} 
first commandment of second tablet [fourth commandment], in Otto VI, 
p.43}}
Only when \citt{you strip without being ashamed, and you take your
clothes and put them under your feet like little children and trample
them, that you will see the son of the living one and you will not be
afraid.}{The Gospel of Thomas, 37} This is never something which
merely \thi{happens to one}, this is something in which one must
actively take part, something one must actively, or as I say,
\co{reflectively} recognize and admit.  This active recognition is the
\sch\ of \yes.


\pa
\co{Experiences} in life may only provide an impulse for
\co{spiritual} search -- they never determine its result, nor even the
initial attitude because, to the extent such an attitude is
\co{spiritual}, it has already liberated itself from the
determinations of \co{this world} and life.

\co{I} might have
received particular \co{gifts} which make \co{me} more
amenable to choosing \yes\ rather than \No, which make \co{my} general
experience of life gentle and agreeable rather than harsh, severe,
even cruel. 
But, as we know, extreme situations,
extreme pain and suffering, may be exactly the means of finding a way
of \co{spiritual} acceptance, while all the successes and joys of
\co{this world} may never fill the void of longing and may leave the
soul \co{thirsting} as \citt{the desire for the bliss, which she had
lost, remained with her even after the Fall.}{John Scottus Eriugena,
`Periphyseon' IV 777C-D [88]}
 Lasting suffering may destroy the 
soul, but it may as well be a way to discover the need for and the way 
of \co{humble opennes} and \co{thankfulness} for the very fact of 
living, whatever form this life takes. And 
\citt{those who have been persecuted in their
hearts: they are the ones who have truly come to know the 
Father.}{The Gospel of Thomas, 69} 

Or else, I might have been lucky enough to have parents, teachers,
surroundings which did impair on me the general ability to be both
strong and weak, both forceful and humble in ways which effectively
bar me from the violence of a sudden encounter with the
\co{invisible}.  Starbuck, quoted by James, focuses on one \citt{who
is born, as I was, into a family where the religion is simple and
rational; who is trained in the theory of such a religion, so that he
never knows, for an hour, what these religious or irreligious
struggles are.  I always knew that God loved me, and I was always
grateful to him for the world he placed me in.}{Starbuck, Psychology
of Religion [in James, The Varieties of Religious Experience, IV,
p.82]} This may sound (and easily be) too much of a simple, complacent
ideality to ask for, but it may as well be an expression of a \sch\
which has been made in a blissfully tranquil manner.  A serious 
religion and 
religious upbringing, emphasizing and promoting such possibility, will never
entirely ignore the less \thi{rational} forms of manifestations of
Godhead.  Then, I might have been lucky enough to grow up in
a religious tradition which in \co{clear} though completely
\co{vague}, incomprehensible and irrational words taught me things
which I only now, in the moment of conforontation which arrives
\citt{like a thief in the night}{I Thess. 5:2}, begin to be able to understand.

\pa
But all such gifts of \co{experience}, of upbringing and tradition, 
can give, at best, only \co{vague spiritual} dispositions.  In the 
confrontation with the \sch, they play
no, or in any case only relative role, because it is \co{I} and only
\co{I} {\em alone} who have to \co{choose}.\footnote{This
is probably the deepest insight of Kierkegaard -- the triviality of
the fact that every existential choice is \thi{private}, combined with
the universal character of this eventual \co{choice}, which spans over
all generations.}
%No matter what kinds of such \co{gifts} \co{I} have received, 
If the gifts were of positive nature, the \co{choice} is still
required of \co{me} beyond and \co{above} anything which I might have
obtained earlier.  If they were of a negative nature, the \co{act} of
\sch\ simply suspends them, makes them for the first time irrelevant,
or rather, only relatively relevant.  The \co{choice} amounts, as
Kierkegaard says, to a leap, a Spring which surpasses all familiar
reasons and categories and goes right to the center of Being, to the
\co{nothingness}.  Such a \co{choice} is not from \co{this world}, is
not motivated by it, is not grounded in \co{my} understanding and
experience of it.  To the extent it is a \sch\ of \yes, it has already
traversed \co{this world} and renounced it.  To the extent it is a
\No, \co{I} simply shrink in the face of it and go back to \co{my
world}.

%\tsep{end from `The choice or the gift'}


\say
I guess, something about Hell comes here (impossibility of leaving -- mere
feeling of it!); though real Hell should also go with Evil...

\subsub{Opening (the doors?)}
\begin{tabular}{|r@{\dotfill}r@{\ \ \ --\ \ \ }l@{\dotfill}l|}
\hline
\multicolumn{4}{|c|}{Shifting, confused, \ldots} \\ \hline
trust/security,quiet    & {\bf hope}    & {\bf despair} & suspicion/insecurity,restless \\
and can be obtained     & {\bf asking}  & {\bf taking}  & only what can be taken \\
patience, even for ever & {\bf awaiting}        & {\bf impatience}      & I want it Now \\
within limits (stop disting)    &\ \ {\bf modesty}      & {\bf insatiability}\ \ \  
                      & avarice (more distinctions)\\
%preparedness for anything & {\bf strength}     & {\bf weakness}        &
%false security, unexpected \\
\hline
\end{tabular}

%\input{032way}
\ad{What do you hope for?}
What can \co{I} hope for? Well, to some extent \co{I} know what it is 
reasonable to expect, what things are within my reach, sometimes even 
how \co{I} possibly might achieve them. And so \co{I} say that \wo{I 
hope to get this job} or \wo{I hope that there will be nice wheather 
tomorrow}. We call an \wo{optimist} somebody who hopes for the best, 
although, \thi{the best} is included in \thi{hoping} because nobody 
hopes for the worst, one can only expect, or be afraid of it. 

\subpa
All such things have nothing to do with \co{hope}. To \co{hope} is to have no 
expectations. If I start calculating I have stopped \co{hoping}, if I 
start forming images and ideas, I have started calculating. 
\citt{[H]ope that is seen is not hope:
for what a man seeth, why doth he yet hope for?}{Romans VIII, 24}
Sure, I can see the possibility of fulfillment but still hope it to 
be \co{actualized}. But if I see this possibility and I see the 
fulfillment then these things have already become \co{visible} and, as 
such, removed from the sphere of \co{hope}, which is a \co{spiritual} 
attitude and not an \co{act} of expectation. Whatever I see, I can 
expect and, moreover, since I see it, I can look for ways getting 
there. If I want to get there, I even {\em should} look for ways of getting 
there, rather than sit passively and nourish empty expectations. And 
if I do not decide to get there, if I leave it in the sphere of 
possibility, saying \wo{it would be nice if it came out this way, but 
I do not have capacity for doing anything about it} then I, in fact, 
have no need for expecting anything, since I have left it to the 
discretion of fate or, as we might prefer to call it, of the course of 
the world. 

But when I do not see the fulfillment or when I see something which I 
can not grasp, then\ldots? Often, what makes it difficult to accept 
\thi{the hard truth}, \thi{the hard facts} is not any lack of realism, 
but the hope that they are not as hard as they seem, the hope for 
something which is impossible to oppose to these \co{actual} 
\thi{facts}, and yet the hope which is much  stronger and more
\co{clearly present} than all the \thi{reality} of \thi{facts}.

\subpa It is not often that we are able to nourish such a \co{hope}.  It
is not often that we feel any need for it because, given the facts and
matter-at-hands, we act in accordance with our goals and try to make
the best of it.  And if we do not act, we are often unable to renounce
the thoughts about it and are left with expectations.  For even if we
do not do anything, it still would be nice if we could benefit from
the development.  Unable or unwilling to \co{hope}, we admit that
\co{the world} is an overpowering complexity and, often, we simply
have to put up with the coincidences which we had no possibility to
influence.  We should notice here a subtle transition between two
levels of speaking.  \wo{This situation is unclear and I hope it will
turn out right} -- the word \wo{hope} indicates the complete vagueness
and indeterminacy of the possible outcome which, however, should in
some sense be positive.  We use \wo{expectation} when the thing we are
expecting for is a more definite, more precisely defined result.  In
the phrases like \wo{The world is insecure and we can only hope for
the best}, the word \wo{hope} can hardly be repalced by \wo{expect}. 
This \co{general thought} does not have any specific content except
expressing the general suspicion and warning.  It is, perhaps,
unreasonable to maintain such thoughts but we actually do express them
on various occasions.  

\subpa
In the extreme cases, in the states following
long suffering, insecurity, angst or disappointments which have
accumulated to the level of unbearable strain, we may actually end up
with the general and all-embracing \co{experiences} which do not even
leave any space for \co{hope}: \wo{The world is evil and the life is
a damnation}.  \co{Despair} is the extreme case opposed to \co{hope},
is the place where one no longer is able to believe that not
everything must be bad, that pain and suffering, of which one has 
experienced so much, do not necessarily constitute the only principle 
of the universe.  \citt{Also let a man mark, when he is in this
hell, nothing may console him; and he cannot believe that he shall
ever be released or comforted.}{Theologia Germanica, XI} It is a place
from which there is no exit, because being there, one is unable to
\co{hope}, that is, to \co{hope} for \co{nothing}, while no
\co{visible} doors can be discerned.


\ad{What do you await?} 
To \co{hope} is to have no expectations. 
Thus, to \co{hope} is to await for \co{nothing}.  But still, it is
awaiting, it is suspension where things are not exactly the way they
should be and where there is something to look for.  But this
something is not this or that but \co{nothing}. \co{Hope} does not 
wait for anything. And if it does, then it waits for something 
indefinitely \co{invisible}, for something it does not know what is, 
it waits with the infinite \co{patience}. Infinite, because it does 
not know what it is waiting for, how it might look, because it does not 
expect it to come tomorrow nor, maybe, next week. It waits, 
\co{patiently}, for something \co{invisible}, and is prepared to wait 
for ever, even if \co{nothing} were ever to happen.\footnote{See the 
beautiful cermons of Kierkegaard ``Om aa bevare sin Sjel i 
t{\aa}lmodighet'', in ``Oppbyggelige Taler'' \ldots}
In this 
infinite \co{patience}, the soul finds tranquility because, as a 
matter of fact, tranquility is nothing else than this \co{patience}.
\citt{For we are saved by hope: but hope that is seen is not hope:
for what a man seeth, why doth he yet hope for?
But if we hope for that we see not, then do we with patience
wait for it.}{Romans VIII, 24-25}

\subpa
If we substitute some image for the \co{nothingness} awaited in 
\co{hope}, we may still retain some of the \co{patience}. We may 
await for \co{vague} happiness, for indefinite relief, for some 
feeling of fulfillment which all are equally indeterminate and 
ineffable, but which, nevertheless, are \co{present} with a 
\co{clarity} amenable to an expression in words or other 
\co{signs}. We may know that we do not wait for the world war, nor 
for the next child but for \ldots something else. Such \co{vague} 
images can still be awaited in \co{patience}. As we move lower in the 
hierarchy of Being and await specific \co{things} or events, patience 
will have to yield to rational considerations -- \wo{Haven't I waited 
long enough?}, \wo{Is it still reasonable to expect that he may 
change?}. Relativization of patience is but a reflection of the 
relativization of the object of hope, that is, expectation. Patience 
with respect to \co{visible} things is a great virtue but, as we 
know, it may sometimes be difficult to distinguish it from stubborness. 

\subpa 
It is not always we have enough patience with the things of
\co{this world}.  Sometimes, because they do not deserve more
patience, and sometimes because {\em we} are simply becoming
impatient.  It is hard, though not impossible, to tell if it is the
one or the other.  But we also know people who are almost thoroughly
impatient, who do not accept the fact that things take time; people
who not only all the time want something but also always want it
\co{now}!  Such an impatience, such restlessness of soul, is not -- in
any case, not necessarily -- the same as energy capable of
simultaneous activity in various directions, of continuous goal
directed expression although, as always, the \co{precise} borders may
be almost impossible to draw. The fundamental thing is that impatience 
is possible only because one is thoroughly involved in the \co{visible 
world}, in the incessant play of a multitude of \co{distinctions}, 
which all seem to offer something others do not. Impatience signifies 
this total turn towards \co{this world}, where every moment, being 
devoid of the awaiting for eternity, becomes a deperate awaiting for 
the next moment.

\subpa 
In time, of course, even the greatest impatience calms down. 
Older people are seldom impatient, and even if they are, their
impatience is rather closer to irritability than to the impatience of
younger people.  With time, one does learn that there is no hurry;
perhpas, even that things one so impatiently awaited in earlier days,
are not worth so much and do not give so much more than things one
already had.  One regrets the lost intensity of feelings which created
the impression of life.  Shouldn't one then try to listen to the most
weak and feeble voices which, lasting but not overwhelming, are the
matter of whole life?

\subpa
In the extreme cases, the impatience which ends in hell, 
ceases completely. It ceases because there is no longer anything one 
might expect, there is no longer anything worth awaiting. The emptiness 
surrounds the whole world when all the things have lost their 
meaning and value and, thus, become detestable. The 
detached contempt for \co{the world}, for every this and that, may 
sometimes look like a spiritual liberation but, in fact, is the 
extreme form of slavery.


\ad{What do you ask for?}
Asking is different from \co{hoping}. Even if both are directed 
beyond any this or that, \co{hope} is an 
attitude, while asking an \co{act}. As an \co{act} it is involved into 
the categories of \co{actuality}, in this case, it has an \co{object} 
and an adresee. Both may still be completely \co{vague} and 
indeterminate, but they appear as the aspects of the \co{act}.

\subpa 
A perfect \co{spiritual love} does not ask for anything.  But
when \co{I} \co{hope} and yet do not know for what, when \co{I} await
in infinite \co{patience}, \co{I} may ask for this unidentifiable
something to occur.  In a sense, \co{hoping} is already such an
asking, but we distinguish the two for the reasons just mentioned. 
\co{I} ask, but since \co{I} do not know whom or for what, \co{I} do
it in full \co{humility}, \co{I} know that \co{I} am not entitled to
obtain.  It is like \wo{I wish I had \ldots} but \co{I} am fully
prepared not to have -- \co{I} do not know what.  The value of such an
asking does not depend on whether \co{I} am asking for this for
\co{myself} or for somebody else.

Asking for \co{things} and \co{objects}, for lucky outcomes of petty
or important situations is a completely different thing.  Even if, to 
the extent it is asking, it expresses some form of humility, it is 
first of all en expression of, on the one hand, my attachement to 
things and, on the other hand, my feeling of impotence. 
I can not resist mentioning the ridiculous rituals of common prayers 
for a new refigerator, a new car, an amount of money, for this or 
that handicapped soul from the audience. The lower in the hierarchy of 
Being the object of asking is located, the \co{clearer} the indication 
that asking is not a \co{thankful humility} but only an arrogant demand. 

\subpa Such asking may easily be also an expression of the \co{thirst} of
soul which has degenerated into insatiability, whether in form of
avarice, greed, gluttony or envy.  All these forms may be accompanied
by apparent asking, asking for this or for that, for \co{more} for
\co{myself} or for less for another, but as I said, this is not asking
but demanding, merely accompanied by the recognition of one's
impotency.  Low asking assumes basic passivity of the person and may 
often appear as 
humility -- I do not have to say, false humility.

The extreme form of insatiability is no longer an \co{act}
but an attitude of such a constant demand for \co{more} and, in fact, 
\co{more} for \co{myself}. Such a blindness to the \co{vertical} 
aspect of \co{transcendence} might be, perhaps, considered another 
dimension of hell. But hell in the sense indicated above, would 
rather amount to the complete sickness of will, to the inability of 
asking for anything whatsoever, since nothing appears there as worthy 
of having. Thus hell might appear as an opposite of insatiable demand 
but, in fact, it is but its eventual form, when the unsatisfiable 
thirst for \co{more} reaches the limit of its possibilities and 
realizes that all things it has craved for are not able to cure its 
sickness. 

\pa \citt{Ask, and it shall be given you; seek, and ye shall find;
knock, and it shall be opened unto you: For every one that asketh
receiveth; and he that seeketh findeth; and to him that knocketh it
shall be opened.}{Mat.  VII:7-8} But, to receive, you have to be
careful what you are asking for.  Ask only for the impossible, and it
shall be given to you; ask only for the \co{invisible} and you will
receive it, ask for \co{nothing}, in short, {\em ask}, do not calculate.  Because such an asking
is already its own answer, in such a search, you have already found,
and when \citt{it is found it is not he who searches but He who is
sought [\ldots] Who find it.}{Periphyseon, II 572A-B, John Scottus
Eriugena}

\sep


\say Shall we comment here on \co{inversions} \ldots ?


%%%%%% \subsection{The spiritual choice}
%\input{031choiceActs}
\noo{This subsection on \sch\ should be made into:
\begin{enumerate}
\item Possible \co{experiences} calling to make the choice: call\\
   -- intensity, not content
\item The choice itself\\
    -+- Yes - No\\
    -+- choice vs. gift\\
    --- not a projection\\
    --- end event, result of a long process
\item Analogues\\
   -- list\\
   -- not projections either
\end{enumerate}

%%%%%%%
\tsep{}
}

\subsection{The spiritual choice}\label{sub:sch}
\secQQQ{7}{But let your communication be, Yea, yea; Nay, nay: for
               whatsoever is more than these cometh of evil.}{Mat. 
               V:37}

%% here were \subsub{The calls} (went back to 2.1.2

%\subsub{The \sch}

\pa
Every \co{reflection} involves a choice; if not of aything else, than 
at least of its \co{object}. But an \co{object} is but a limit of 
\co{distinctions}, so this choice is also the choice of the way of 
viewing the \co{object}, is a choice of viewing the \co{object} in 
this rather than that way. 

The fundamental choice confronting \co{reflection}
concerns the reality of the sphere of \co{invisibles}, eventually, the
significance of the \co{origin}. Do I accept the reality, the 
relevance of such a sphere or not? Eventually, do I recognize the 
significance of \co{the One} as the \co{origin} or 
not? These are, perhaps, different questions. One can can answer 
positively the question about the general sphere of \co{invisibles} 
and yet deny the significance of the \co{origin}. I would say that 
this involves a confused understanding of the \co{invisibles} 
which are, in such a case, interpreted in the terms of \co{visibility} 
with a focus on our ability to actually \co{distinguish} them and 
attempts to make these \co{distinctions} rigid and \co{precise}. A 
proper understanding of the role of \co{invisibles} implies and is 
implied by a proper understanding of the \co{origin}. 
The fundamental \co{choice}, the \co{spiritual act} concerns 
only the latter.


\pa The higher the level of \co{experience}, the less reactive it is,
\refpp{nonreactive} ff.  There is no such thing as \co{an experience}
{\em of} the \co{origin} but still the \co{origin} is
\co{experienced}; it is a most constant element which is \co{present}
in, which underlies, every \co{experience}.  This \co{presence} is a
part of every \co{experience}.  However, this aspect of 
\co{experience}, or let me say, the \co{experience} at 
this level, is purely \co{spiritual} -- entirely contentless, entirely
non-reactive and entirely dependent on the attention devoted to it.

The \co{experiences} mentioned in \ref{sub:calls} may indeed confront
one with the necessity of \sch.  But they need not do so, nor are they
necessary for making such a \co{choice}.  As we said in
\refp{pa:contNothing}, one can and often one will try to
\thi{understand}, even \thi{rationalize} such \co{experiences} by
focusing on their content, and this may result in the emergence of
\co{idols} rather than of genuine, \co{concrete expressions}.

In order to be the latter, the quality of
the experience must not be confused with its \thi{what} -- the
\co{invisible nothingness}.  The \co{choice} is \co{spiritual} only if
it \co{transcends} any possible contents and is directed exclusively
towards this \co{nothingness}.  And thus, such a \co{choice} does not
really require any specific \co{experience}, it may be performed by a
person who never was close to any feeling of \thi{mystical union} or
\thi{God's venegance}.  

In any case, the \sch, although an \co{act}, is hardly ever the matter
of a single moment or day.  Unless, as is seldom the case, it is a
happy realization of something already accomplished, it is a moment of
climax, a fulfillment of a long and often painful process.
%[e.g. James, Varieties, p.212, through apathy\ldots]
Leaving \co{myself} may easily involve long-lasting apathy and
indifference (\co{I} was, after all, what gave the \thi{meaning} to all
\co{my} actions); it may involve the feelings of vanity of \co{this
world} and \co{my life} as they lose the \thi{meaning} they used to
have; feelings of guilt and corruption as the prior involvement into
\co{this world} seems a sin against \co{another world}.  Regeneration,
or in Jungian terminology individuation, is hardly ever a simple and
cheerful event, but we are not dealing here with the psychological
particulars, possibilities and impossibilities.


\pa
As the ultimate \co{invisiblity}, the \co{origin} is not
prone to any \co{precise} expressions in terms of \co{actuality}, in terms
familiar to \co{reflection}.  The most logical, and scientifically
correct attitude is to dispense with such a thing, is to deny its
meaning, perhaps, even its reality.  But I am not a scientist, and
even a scientist is also a human being.

The \co{origin} is just that -- the \co{invisible} source \co{founding}
everything we ever meet.  If you do not accept this, nothing that
follows will be of interest to you.  But since you have read so far, I
suppose you do recognize some truth in it.  The first choice is the 
\co{choice} in the face of \co{nothingness}, the choice not 
confronted with any \co{distinctions}, not to say \co{objects}, but 
the unconditional, \co{absolute} choice. 
This \co{choice}
-- the \sch\ and, in fact, the only \co{spiritual act} -- to a
high extent determines, that is \co{founds concretely} the contents 
of the lower levels.  And
there is only one, exclusive alternative, the absolute either-or,
\yes\ or \No.  Faced with this \co{choice}, it is impossible to suspend or
avoid it, because not choosing amounts here to the same as choosing,
perhaps only passively, \No.  \citt{He that is not
with me is against me; and he that gathereth not with me
scattereth.}{Mat.  XII:30/Lk.  XI:23} 
%%%
%\\[2ex]
%%%
%\begin{tabular}{|r@{\dotfill}r@{\ \ \ --\ \ \ }l@{\dotfill}l|}
%\hline
%\multicolumn{4}{|c|}{The opposites in Man's relation to \HH}\\
%\multicolumn{2}{|c}{\G\ \ } &\multicolumn{2}{c|}{\ \  \B} \\
%\hline\hline
%\multicolumn{4}{|c|}{Attitude} \\ \hline
%I might have not been  & {\bf love}            & {\bf self-centredness}        &  \\
%\multicolumn{2}{|r@{\ \ \ --\ \ \ }}{{\small not to world but for being (in the world)}}                       & hate          & why am I that way \\
%%%\multicolumn{1}{|c}{}        &               & hate          & why am I that way \\
%we are not masters     & {\bf humility}        & {\bf pride}   & I decide and make  \\
%everything is a gift   & {\bf thankfulness}  & {\bf 
%ingratitude}\ \ \ \    & I owe nothing, nobody \\
%anything may happen    & {\bf openness}        & {\bf closed}  & nothing happens There \\
%\hline
%\end{tabular}%\caption{Below to Above}
%\label{ta:BtoA}

\subsubsection{\yes}
%
\noo{
The \co{choice} of \yes\ means to recognize \co{nothingness} as the
\co{One}, the ultimate \co{invisibility} as the \co{origin}.  It is
the same as recognizing \co{myself} as \co{nothingness}, admitting not
only that \co{I am not the master} but that, in fact, \co{I am
nothing}.  If you think that you are anything, whatever, wise or
not-wise, good or bad, rich or poor, you have not chosen \yes.  If you use
any names, not only for the \co{invisible}, but also for \co{yourself},
you think that you are something, and thus you have not chosen \yes. 
To \co{choose} \yes\ is to say \co{I am nothing} and \co{nothingness} 
is the \co{origin}.

In such a recognition, \co{I} admit several things. 

\pa\label{humility}
Firstly, that \co{I am not the master}. The \co{distinctions} among 
which \co{I} live, and some of which \co{I} actually can control, originate 
beyond the sphere of \co{visibility}. My understanding is limited to 
this latter sphere, but it can only \co{vaguely} \co{reflect} the 
working of \co{invisibles}. \co{I} understand \co{myself} when \co{I} 
understand \co{my} limits, but the very fundamental issue is first to 
recognize that such limits at all exist, that \co{I} end where 
\co{this world} ends, end that beyond there is something which, from 
the perspective of \co{this world} is but \co{nothingness}. However, 
the positive answer to this first question says that this 
\co{nothingness} is not emptiness, is not lack of reality but, on the 
contrary, is the most real source of whatever is encountered in 
\co{this world}. All this amounts to \co{humility}.

Notice that this \co{humility} does not mean that \co{I} recognize any 
particular master who is govering me. \co{Humility} in this 
\co{spiritual} form is not a submission to any definite power. Even 
more, it does not even need any awe inspiring, ineffably powerful 
\la{tremendum sacrum}. Indeed, encounter with such powers may lead to 
humiliation but not to \co{humility}. Humility which is a 
reaction to anything specific, which is caused by no matter how 
\co{vague}, but still a particular cause, is not a \co{spiritual 
humility}.  

\pa\label{thankfulnes}
Secondly, \co{I} admit that \co{the world} in which \co{I} live, 
the field of \co{visible distinctions} is given to \co{me}, is the 
result of a process which might have involved \co{my Self} but 
certainly not \co{myself}. This does not mean that no single 
thing in \co{this world} is a result of \co{my activity}, only that, 
eventually, all such things are grounded in, or originate from the 
\co{transcedent} sphere of \co{invisibility}. \co{The world} is a 
\co{gift}, \co{my life} is a \co{gift} and everything that \co{I} ever 
encounter is a \co{gift}. \co{I am} while there is no sufficient 
reason for \co{me} being here. \co{I am} while \co{I} might not be. 
Recognition of \co{the world} and \co{my life} as a \co{gift} of 
\co{transcendence} amounts to \co{thankfulness}.

Notice that it is essential for \co{thankfulness} that we recognize
the \co{gift} as arbitrary, as having no sufficient reason --
\co{creation} is a mystery.  The recognition of the \co{One} as the
generous source admits only that it is a necessary but not a
sufficient condition.  Any search for sufficient reasons, any attempt
to explain the necessity of this \co{gift} amounts to explaining it
away and to renouncing the attitude of \co{thankfulness}.  The
arbitrariness of the \co{gift} is what \co{founds} the \co{spiritual 
thankfulness}. Its \co{spiritual} charaqcter means just that it
 is not thankfulness for anything specific; it is
\co{thankfulness} for \co{nothing}, that is, for everything.  If only
\co{I} start looking for reasons for being thankful, for any positive
things worth gratitude, \co{I} renounce the \co{spiritual} dimension of
\co{thankfulness}.

\pa\label{openness} 
Thirdly, the arbitrariness of this \co{gift} makes me realize that
it might have not taken place or else that it might have been entirely
different.  Instead of \thi{this} \co{I} might have gotten \thi{that},
instead of being \thi{this} \co{I} might have been \thi{that}.  No
matter what, in more \co{precise}, \co{visible} terms \co{I} have
obtained, deserves the same \co{thankfulness}.  Anything that \co{I
am} or encounter is but a particular instance of the fundamental
generosity.  This amounts to \co{openness}.

Again, it is not \co{openness} to this or that, to anything specific, 
it is not \thi{identification} with \co{my} things, \co{my} 
pleasures, \co{my} friends -- 
but unreserved, unrestricted acceptance, \co{spiritual openness} to
\co{nothing}, that is, to everything.

\pa These three aspects -- \co{humility}, \co{thankfulness} and
\co{openness} -- I call jointly \wo{\co{love}}
or, to avoid confusion, the \wo{\co{spiritual love}}.\footnote{\wo{Charity}
and, since I mostly relay on the Christian tradition, \wo{faith}, or
even \wo{obedience}, in the sense used by the Church fathers and
mystics, may be here equally good -- in fact, synonymous!  -- words. 
Let's only recall that the Greek \gre{pistis}, translated as \wo{faith} 
or \wo{belief}, 
can be, and often should be, rendered as \wo{fidelity} or \wo{loyalty}.}  
The crucial point is that \co{love} does not \thi{consist of} 
\co{humility}, \co{thankfulness} and \co{openness}; it is the unified 
and \co{indistinct} attitude and these three 
%(and, perhaps, other) 
are but \co{aspects} of the fundamental \yes, are
essentially coextensional and coinciding.\footnote{According to
Dionysius Areopagite, the first, closest to Godhead shpere of the
celestial hierarchy comprises three kinds of Angles: The Seraphim --
the fiery, purifying, inspiring source, The Cherubim -- the
illuminating light, and The Thrones -- the perfecting and receptive
openness.} 

I
have earlier made occassional references to love which, however, was
usually something different.  It was personal love, love at the level
of \co{minness}.  I will return to this distinction later on, but here
we should emphasize that the two are not to be confused.  \co{Love} as
\co{charity}, \co{spiritual love} has no object, it is \co{love} of
\co{nothing}, that is, of everything.  It is \co{spiritual} because it
is the attitude towards the \co{origin} as \co{origin}, in its full,
\co{indistinct invisibility}.  As such, it is not restricted to any
particular region of Being but penetrates the whole of it and is
\co{present} in every encounter with any particular being.  Personal
love is, of course, only a restricted form of such \co{love}.

\pa
Multitude of forms\ldots yet one and the same\ldots

\pa
\yes\ relieves \co{reflection} from the dependency on \co{actuality}. 
It is still involved in it, but it has now lost its absolute validity 
as the only \ldots
} %/noo{...



\subsubsection{\No}
%
The \co{choice} of \No\ may be twofold. It may be an active and 
declared refusal, or even just inability, to \co{choose} \yes, a refusal to admit that \co{I am 
nothing} which directly condemns and despises the possibility of 
another alternative. Or it may be more passive, a mere avoidance to 
confront the \co{choice}, a mere flight which may arise Angst in 
confrontation with the \co{invisible presence}. 
It may be even a mere unconsciousness that there is 
any \co{choice} to be made. Indeed, there is a multitude 
of possible ways of saying \No, and these various ways may imply 
variations in the involved attitudes. Nevertheless, they all share 
some underlying themes, some common aspects implied by the refusal
to admit that \co{I am nothing}.

GONE to new %\pa\label{pride}
%\pa\label{ingratitude}
%\pa\label{closed}

\sep

\pa
The absoluteness of the either-or, the \co{clarity} of the oppostion 
\yes-\No, emerges only for and through 
the voluntary consent of \co{reflective} choice. In a sense, it is 
only the \co{reflective} need for \co{precision}, the grounding of 
\co{reflection} in \co{actuality}, which creates the 
absolute \co{clarity} of the opposition of \yes\ and \No, and with 
it, the need for the \sch.

The \co{spiritual love}, and its opposite, \co{self-centredness}, is
an event of \co{virtual incarnation}, that is, it does not signify any
collection, any totality of elements and aspects.  It signifies an
infnite potential for innumerable forms of \co{manifestation} and
\co{actualization}.  The above aspects may appear so -- as separate
aspects -- only for the \co{reflective} understanding.  (A thoroughly
\co{attentive} and intense \co{reflection} can even posit them as as
completely independent things and believe that they can be studied 
(sic!) each
\thi{in-itself}, independently from others.)  In fact, I have listed
only these which seem most central to me and I am far from claiming
that these are the only one might reasonably mention.


%%%%%%%%%%%%%%
%\newpage

\subsub{The choice and the gift}

%\titp{The {choice}}
\pa
\co{Spirit} is the unity of -- not only a borderline between -- flesh
and soul, \co{visible} and \co{invisible}.  
\citt{Damn the flesh that depends on the soul. Damn the soul that
depends on the flesh.}{The Gospel of Thomas, 112}
It is the continuous flow of 
\co{love} from \co{above} to \co{below} and from \co{below} to 
\co{above}, embracing the \co{soul}, the whole 
\co{soul}, with the unmistakable recognition (not necessarily the  
feeling) of \co{participation}, of \co{soul}'s and the whole \co{visible 
world}'s \co{participation} in the \co{invisible love}.\footnote{We 
do not have to study the caricatures of spirit, whose name is 
\wo{plenty}, \co{dissociated} into \thi{immaterial} as opposed to 
\thi{material}, \thi{spiritual} as opposed to \thi{bodily}, \thi{mental} 
as opposed to \thi{sensous}, with all the degenerations of the word 
which should apply to science and art, but not to physical work, to 
intellectual and cultural achievements but not to events of individual 
life, to static contemplation but not natural inclinations, etc., etc., etc.. \citf{Spirit is to be found as a spark in 
everybody's life, and from the life of the most vital persons it 
erupts as a flame. [\ldots] There is no spirit other than that which 
is nourished by the unity of life and the unity of the world.}{M. 
Buber, {\em Das Problem des Menschen}, II:3.11}}

\noo{
\tsep{moved to `Calls'}

\pa
\co{Reflection} finds itself always involved in the
interpaly of \co{invisibles} but this interplay is precisely that --
an interplay, a mixture, affected once by this, once by that,
sometimes tending in this direction, sometimes in that.
Life without
\co{spiritual} comittements may bind various levels, may keep heaven
and earth together only as if by accident.  But it may also, and
usually it does, pass in a constant discordance between various levels 
and \co{experiences}. 

\pa We can on some occasions \co{experience} moments of rare elation,
moments which show us the joy of love and unity.  \co{An experience}
of beauty, or else a simple \co{act} of charity, generousness,
compassion, when performed in due humility, may fill the \co{soul}
with the feelings of unity and \co{participation}, in fact, can
\co{manifest} the \co{presence} of \co{love}.  One does not have to
\co{attentively reflect} over it.  A life which one \co{reflects} over
and summarizes as good and satisfying, with a clear conscience (which,
of course, glosses over some small mistakes and minor sins), with a
genuine conviction of its gratifying value, for oneself as well as for
others, may give a feeling of a \co{vague}, almost mystical, harmony. 
Mystical experiences, also those in most unexpectedly daily form, 
have been described and classified so many times that there is no need 
to repeat that here. (James' {\em The Variety of Religious Experience} 
is an excellent collection of such examples.)

\pa But all such feelings are nothing more than feelings, \co{vague}
intutions, perhaps, \co{vague} \co{inspirations} to continuing the
life in the same way.  If one does, and if the \co{invisible} currents
of \co{virtual} unfoldings favour one, one may indeed remain
blissfully satisfied and content.  But one may also encounter a moment
of confrontation, a moment when one realizes that it is not merely a
matter of \thi{things and life going this way}, but of one's attitude. 
Such moments may come at most unexpected time, \wo{like a thief at 
night}, but when they come,
they come as thunder, an ultimate confrontation with the unconditional
\co{invisibility}.

\subpa
Let's make this a bit more clear. One will easily recognize rare 
moments of confrontation with some existential choice in which one 
has to make a decision known to be of utmost importance for one's life, 
perhaps the whole life. Important as they are, they are still concerned 
only with \co{my life}. We are not talking about that. 

We are talking about moments of confrontation with the \co{absolute
nothingness}, perhaps, with the ultimate Void, perhaps, with the
ultimate Power.  Whatever the form of the \co{actual experience}, that
is, whatever its concrete content (or lack thereof), it confronts me
with an \co{absolute} objectivity, definitely \co{not-me}.  It is
\co{absolute} because it does not concern any thing but is immediately
{\em known} to apply to everything.  We may easily call it a
\wo{feeling}, because in its complete lack of the \co{objective}
element, it is unique and unitary -- it does not come {\em from}
\co{One}, it is \co{One}.  But it is equally knowledge because we also
recognize the definite \co{commanding} power, we know that beyond this
unity, there is nothing to be \co{distinguished}, there is no need for
any more \co{precise concepts}, which would unfold its
\co{complexity}.

\subpa Do we really know this to be so?  The worst thing one can do is
to try to bring the \co{objectless}, contentless quality of such
\co{an experience} to the level of \co{precise} and understandable
thoughts, to try to grasp and explain to oneself what is the matter. 
\co{Nothing} is the matter. 
Angst, the Angst as Kierkegaard describes it but also in a more 
mundane sense, is a natural reaction of
one confronting the unconditional for the first time.  It has as 
repulsive as attractive power, like an abyss which \thi{draws} a 
person with high-anxiety into itself. 
A better
prepared person may feel awe or overpowering intensity, perhaps, the
presence of God in all his venegance and anger which all, too, invite
to an escape.  The attempts to bring it down to the level of
\co{visibly} understandable is a possible way of tampering it.  Thus,
we can obliterate its meaning.  But even if we do so, we will still
{\em know} that we have done just that -- denied it the attenion it
required.  We can imagine that with time, this knowledge may be
forgotten, especially if we encounter no more \co{experiences} of this
kind.  But I doubt that anybody with such an experience will forget it
-- one may only try to ignore it.  This is a possible, albeit rather
passive, way of saying \No.

\pa The \co{command} of such an experience is to forget yourself or, 
what comes to the same, to tame the terrifying, to accept that it does 
not threaten, in short, to embrace it.\footnote{If this sounds like a 
psychological contradiction then it probably is. \citf{We worship 
holiness with fear, but we do not escape from it, but even more are 
attracted to it.}{M.Luther, Speech about good deeds \fre{\'{a} propos} 
first commandment of second tablet [fourth commandment], in Otto VI, 
p.43}}
Only when \citt{you strip without being ashamed, and you take your
clothes and put them under your feet like little children and trample
them, that you will see the son of the living one and you will not be
afraid.}{The Gospel of Thomas, 37} This is never something which
merely \thi{happens to one}, this is something in which one must
actively take part, something one must actively, or as I say,
\co{reflectively} recognize and admit.  This active recognition is the
\sch\ of \yes.


\pa
\co{Experiences} in life may only provide an impulse for
\co{spiritual} search -- they never determine its result, nor even the
initial attitude because, to the extent such an attitude is
\co{spiritual}, it has already liberated itself from the
determinations of \co{this world} and life.

\co{I} might have
received particular \co{gifts} which make \co{me} more
amenable to choosing \yes\ rather than \No, which make \co{my} general
experience of life gentle and agreeable rather than harsh, severe,
even cruel. 
But, as we know, extreme situations,
extreme pain and suffering, may be exactly the means of finding a way
of \co{spiritual} acceptance, while all the successes and joys of
\co{this world} may never fill the void of longing and may leave the
soul \co{thirsting} as \citt{the desire for the bliss, which she had
lost, remained with her even after the Fall.}{John Scottus Eriugena,
`Periphyseon' IV 777C-D [88]}
 Lasting suffering may destroy the 
soul, but it may as well be a way to discover the need for and the way 
of \co{humble opennes} and \co{thankfulness} for the very fact of 
living, whatever form this life takes. And 
\citt{those who have been persecuted in their
hearts: they are the ones who have truly come to know the 
Father.}{The Gospel of Thomas, 69} 

Or else, I might have been lucky enough to have parents, teachers,
surroundings which did impair on me the general ability to be both
strong and weak, both forceful and humble in ways which effectively
bar me from the violence of a sudden encounter with the
\co{invisible}.  Starbuck, quoted by James, focuses on one \citt{who
is born, as I was, into a family where the religion is simple and
rational; who is trained in the theory of such a religion, so that he
never knows, for an hour, what these religious or irreligious
struggles are.  I always knew that God loved me, and I was always
grateful to him for the world he placed me in.}{Starbuck, Psychology
of Religion [in James, The Varieties of Religious Experience, IV,
p.82]} This is, probably, too much of a simple, complacent ideality to
ask for. A serious religious upbringing will, emphasizing such
aspects, never entirely ignore the possibilities of less rational
manifestations of Godhead.  But even then, I might have been lucky
enough to grow up in a religious tradition which in \co{clear} though
completely \co{vague}, incomprehensible and irrational words taught me
things which I only now begin to be able to understand.  

\pa
But all such \co{gifts} concern only the \co{visible} aspects of life
and give, at best, only \co{vague spiritual} dispositions.  In the 
confrontation with the \sch, they play
no, or in any case only relative role, because it is \co{I} and only
\co{I} {\em alone} who have to \co{choose}.\footnote{This
is probably the deepest insight of Kierkegaard -- the triviality of
the fact that every existential choice is \thi{private}, combined with
the universal character of this eventual \co{choice}, which spans over
all generations.}
%No matter what kinds of such \co{gifts} \co{I} have received, 
If the gifts were of positive nature, the \co{choice} is still
required of \co{me} beyond and \co{above} anything which I might have
collected.  If they were of negative nature, the \co{act} of \sch\
simply suspends them, makes them for the first time irrelevant.  It
amounts, so to speak, to a leap, to a Kierkegaardean Spring which
surpasses all of them and goes right to the center of Being, to the
\co{nothingness}.  Such a \co{choice} is not from \co{this world}, is
not motivated by it, is not grounded in \co{my} understanding and
experience of it.  To the extent it is a \sch\ of \yes, it has already
traversed \co{this world} and renounced it.  To the extent it is a
\No, \co{I} simply shrink in the face of it and go back to \co{my
world}.

\tsep{end moved to `Calls'}
} %%% end \noo{...

\pa 
%The necessary step for this unity is a \co{reflection} of the
%\co{origin}.  
Unlike \No, which can be passively chosen by merely not \co{choosing},
there is no passive \co{choice} of \yes.  When confronted with the
\co{choice}, when confronted with the ultimate \co{experience}, which,
sometimes, may take the form of what Jaspers called the limit
experience, one is unable to avoid an active, \co{spiritual choice}. 
The \thi{passivity} of not choosing is then, as a matter of fact, not
thoroughly passive -- it is an active withdrawal, a panical escape. 
(This, of course, may again take thousands of forms which it would be 
too much to classify here.) The \co{choice} of
\yes\ is an \co{active} \co{reflection} of the \co{origin}, an \co{act} which,
moreover, may need to be repeated.  For such a \co{choice}, in the
repeated moments of \co{reflection}, does not mean the
immediate reality of the \co{spiritual love}.  It is only a necessary
but not sufficient condition, but it is {\em the only} condition which
\co{I} can fulfill on my own.

%\tsep{ gift }

%\titp{The \co{gift}} 
\pa
So, even though there is no \co{spiritual love}
without an \co{active choice}, the \co{choice} of \yes\ in a moment of
\co{reflection} does not, by itself, bring the immediate reality of
the \co{spiritual love}.  It brings \co{clarity} but not necessarily
\co{love}.

Indeed, \co{love} is the result of \co{incarnation} which, happening
in the sphere of \co{invisible}, is not controllable by \co{actual
acts}.  The \sch, in its most peaceful form, may be a mere 
recognition of the fact, a consent \co{I} give to such an event,
the consent which is a mere acceptance of the \co{incarnation} which
\co{I} \co{experience} has found, or is about to find the place.  Or
else it may be the impossible \co{hope} for it, a yearning, a mystical
spell which \co{I} cast but which can only \co{humbly} invoke and ask, 
not bring about.  Once \co{love} becomes \co{present}, it is
not an effect of \co{my} invocations but an
undeserved \co{gift} of \co{transcendence} (which theologians call 
\wo{grace}); \co{I} find \co{myself},
as well as the world, surrounded and permeated by it but \co{I} know
that it was not \co{me} who brough it about.

\pa \co{I} can not bring it about because as long as there is anything
like \co{myself}, as long as \co{I} still consider \co{myself} and
think that \co{I am anything}, that \co{I} can do something, that
\co{I} count, all these opninions, thoughts and wishes fill \co{my 
soul} and make no place for \co{nothingness}, that is, for
\co{love}.  \citt{He that loveth his life shall lose it; and
he that hateth his life in this world shall keep it unto life
eternal.}{John XII:25} But, of course, although \co{I} am not 
sufficient, yet \co{I} am the necessary condition of 
\co{incarnation} -- \co{I} have to be here first, to have anything to 
renounce. 

The \sch\ is an \co{act} of \co{reflection}, of \co{myself}, and as
long as \co{I} focus on this \co{act}, \co{I} remain {\em the} \co{active}
agent, {\em the} protagonist.  In a sense, any performance of this
\co{act} is its failure.  It is a necessary precondition but as long
as \co{I} stay with it, as long as \co{I} peform the \co{act}, it has
not yielded its result.


\pa The \sch, if performed in a perfect, complete form, is indeed an
\co{incarnation} of \co{love}.  But to reach this point is already to
transcend it.  The mere wishes, attempts and
\co{acts}, not to say, the mere words, the most intense and 
\co{active} exercises, all
involve \co{myself}, the \co{reflective
actuality}, and thus never reach the desired goal.  One may feel and
believe that one has the propper attitude and, at the same time,
constantly despair over the sensed implausibility of such believes. 
Trying to imagine, to understand, to analyse what else might be
needed, one never makes a step forward, for there is \co{nothing} to
imagine.  The goals remain hidden until they are reached.  Once \co{I}
cease to count (and that means, in particular, cease to imagine,
analyse, cherish opinions about the \co{spiritual} goal and, above
all, about \co{myself} and \co{my} way towards it), the goal becomes
the most obvious \co{presence} of something which, in fact, has been
\co{present} all the time.  For then \co{I} do not discover anything
new but only renounce part of something old -- \co{myself}.

%Thomas Aquinas
\pa
\citt{Every preparation in man must be by the help of God moving the soul
to good. And thus even the good movement of free choice, whereby anyone is
prepared for receiving the gift of grace, is an act of free choice moved
by God.}{Aquinas, Summa Theologiae, I:q22-art.2-reply obj.1 [Jones p.282]}
%???: \citt{Man by himself can in no way rise from sin without help of grace.}
%{Aquinas, ST Pt.I-II:q109-art.4-ans [Jones, 281,ftnt.25]}
% what means `by himself'!!!
This drastic conclusion is easily considered an offence to human dignity and
self-determination. If man can not on his own reach the ultimate goal (one
will like to say here \wo{happiness}), then he is not free, then he is
but a puppet in the play of some hidden forces, of some despotic god, isn't he?

The \co{act} of \sch, as most \co{acts}, is free.
It is free because nobody is forced to perform it, 
it is not caused by anything particular. Causality, along with
necessity, are categories of \co{actuality} which simply have nothing to
do with the \co{spiritual}. But free does not mean arbitrar; it rather means
to be \co{motivated} by, that is, originating from the eventual depth of my being
where it was \co{founded} long before \co{I} decided to choose. 
Such \co{founding}, such \co{motives}, as well as other, more \co{visible}
dependencies, are however at most necessary but never sufficient
conditions. And so, it is indeed \wo{an act of free choice moved
by God.}

Although the \co{choice} is free, the \co{concrete presence} and
reality of \co{love} is not something \co{I} can obtain at will. It is
the ultimate \co{gift}, the \co{gift} of grace. If the possibility of receiving
\co{gifts} offends somebody's sense of self-value, then he is welcome
to ignore it and pretend that his \thi{happiness} is not only entire
but also entirely his own work.


\pa Although the \sch\ of \yes\ does not effect the \co{presence} of
\co{love}, so it does contain all its aspects in a comprised and
\co{actual}, that is, separate form.  It is the beginning.  The
\co{love} is but the end, its living version, when \co{humility},
\co{thankfulness}, \co{openess} do not any more appear as separate nor
even separable, when \co{I myself} become as insignificant as \co{I}
have so far considered \co{myself} significant.  \co{I} am significant
in so far as \co{I} become a possible place of an \co{active
expression}, of the \co{manifestation} of \co{spiritual love}.  But
this \co{manifestation} is possible only when \co{I} accept \co{my}
insignificance, the insignificance of \co{myself} and anything
\co{mine}, the insignificance which makes \co{me nothing}, not even a
receptive agent, but a mere \co{openess}, a mere place where \co{love}
makes itself \co{present}.

In this sense, one might call it \wo{unity}, the unity of a person, 
the unity of Being.  
But \co{I} do not dissappear in it, \co{my actuality} is founded in
the ontological order and it is not dissolved.  It is only
\co{my attachment}, the insistence on \co{my} significance which
disappears and thus opens the field of true \co{participation} in the
\co{gift} of \co{invisiblity}.
%even if he is an agent of \co{active expressesion}.

\say{also that gift -- grace -- is here ONLY the name for the fact that
  sometimes one does leave hell and enters paradise; that all doors were closed,
  it was impossible to imagine leaving the place and ... somewhat it
  happened. \wo{Grace} is a name for THAT, for this very fact -- it is not its
  explanation!!! (We have at least 2 steps: choosing \yes\ (exiting hell,
  entering purgatory) -- this is sometimes pure Grace; and then incarnation of
  love -- which is pure Grace...
  }

%%%%%%%%%%%
\subsub{The concrete founding}

\pa In Book I, we had only to do with ontological \co{founding:} there
would be no \co{experiences} without \co{experience}, there would be
no \co{experience} without the \co{chaos} of \co{distinctions}, and
there would be no \co{distinctions} without something to
\co{distinguish}, or as it may be, to \co{distinguish} from.  In Book
II we saw its epistemological couterpart: there would be no
\co{reflection} without the \co{experience} to \co{reflect}, there
would be no \co{experience} without the \co{awarness}, that is,
without \co{chaos}, and there would be no \co{chaos} without the
underlying \co{non-actuality} of \co{invisibles}, eventually, without
the \co{indistinct One}.

These two hierarchies of \co{founding} are, in fact, the same and differ 
only by emphasis one puts either on the \thi{objective} or the 
\thi{subjective} pole of being; an emphasis which may be put only 
when viewing both, so to speak, \thi{bottom up}, from the assumed 
primacy of the \co{distinction} into \co{subject} and \co{object}. 
But viewed \thi{top down}, as the hierarchies of \co{founding}, they are 
one and the same. 

\pa This \co{founding} is, as abstract ontology or epistemology in
general, perhaps curiously interesting but existentially at best
helpful and at bottom irrelevant -- it happens and works according to
its structure, no matter what we do, and even if its understanding may
reward \co{my} curiosity, it does not really affect \co{me}.  We have
several times observed that events at different levels may happen
relatively independently from those at other levels because, in
\co{concrete} terms, this \co{founding} is not effective.  It merely
gives a general form and character to the events at each level but
does not relate \co{concretely} such events at different levels to
each other.

\pa Thus, although the hierarchy does procede from the unity of the
\co{One}, it is not \co{experienced} as such in \co{concrete}
terms.  In particular, \co{reflection} stays attached to the level of
\co{actuality} and, looking for \co{precision}, hardly ever finds
\co{clarity} of the \co{concrete} relations between the levels.  And
even if it finds such a \co{clarity}, it knows not what to do with it,
since under the \co{reflective} look \co{clarity} immediately slips
out of its \co{precise} grasp and dissolves in \co{vagueness}. 
What is \co{concrete} is personal and such a relation, if it is
to take the form of \co{concrete} acts, feelings, comittments and
insights, is a matter of personal choice and attitude.  \ldots

\pa The \co{concrete participation} is not merely \co{founded} -- in
the abstract, indifferent sense -- in the order of Being.  If it is to
be \co{concrete}, felt and \co{experienced}, it must be \co{founded}
in the \co{concrete} recognition that the lower things, the particular
\co{distinctions} of the lower levels, originate from those at the
higher levels\ldots This is not something that simply \thi{is that
way}, that simply is granted by the hidden but universal order of
things which \co{I} must only realize -- without \co{love}, without
\co{my love}, it actually is not\ldots I can find gaiety, joy, fun
in small things of \co{this world}, but unless this fun
\co{participates concretely} in the higher mirth, and the mirth is
surrounded by happiness of my whole \co{soul}, and by tranquility of
\co{spirit}, the fun itself may easily be disturbed by the
\co{rest}, and become only an escape towards \co{more} fun. 
\citt{Fun I love but too much Fun is of all things the most
loathsom.  Mirth is better than Fun \& Happiness is better than
Mirth -- I feel that a Man may be happy in This World.}{W. Blake,
Letter to Rev.  Dr.  Trusler, 23.08.1799 [used in II.2.5]}
%

\citt{[H]oliness is never the mere \la{numinosum}, even at its highest 
level, but is something which is always in a perfect way permeated and 
saturated with rational, purposeful, personal and ethical 
elements.}{Otto, Das Heilige, XV, p.131}
\wo{\co{Incarnation}} is the name I will use for the \co{concrete
founding} and most of the remaining sections of this Book will be
concerned with it.

%\say
%\thi{axiological founding} = \co{incarnation} happens through \sch, vs. 
%\thi{ontological founding}

The first stage of the \co{concrete founding} may find its 
\co{expression} in \co{analogues}.

\subsub{The \co{analogues}}
\secQQQ{8}{Why dost thou prate of God?  Whatever
thou sayest of Him is untrue.}{Eckhart}

%\input{030noNames}
\pa One has often emphasized the \thi{human need to speak about
God}. Well, there may be such a need, and it may be human, but these
are not sufficient excuses.  The need can arise from the
\co{reflective} attitude which, enscribed within the \hoa, can not
escape the spell of \co{objective} way of speaking.  I do not think it
is necessary to speak about God and when such speaking begins to
ascribe Him all kinds of attributes, it may actually be harmful. Yet\ldots


%\ad{A brief recapitualtion}
\pa
Section \refsp{sub:OneMany} explained the bareness of the notion of 
the \co{One} as the ontological \co{origin}. The 
\co{spiritual founding}, however, is a \co{concrete founding} 
establishing a relation 
between the \co{concrete} contents of all the levels, a relation to 
the \co{origin} which 
\co{I} can assume and live. 
\co{Love} recognizes the \co{One} not as a formal \thi{principle}, a
mere factual origin, but as a \co{presence} which permeats the world,
 which \co{founds} \co{concrete} ways of encounter and 
\co{experience} (we will see it in the following Sections).

The \co{concrete clarity} of the \co{experience} of such a
\co{founding} may seem to imply the possibility of a more
\co{concrete} description than that admitted for the \co{One} in
\refsp{sub:OneMany}.  However, the \co{conreteness} of
\co{spirituality} is still founded on the bareness of
\co{nothingness}.  The following few paragraphs are essentially only a
repetition of Section \ref{sub:OneMany}.  But while there we were
concerned with the merely ontological meaning of the \co{One}, now I
want to emphasize that also with respect to the \co{spiritual
presence} of \co{love}, equipping God with all kinds of attributes,
whether in essence, in fact or only in name is still \ldots

\pa \citt{All creatures have existed eternally in the divine essence,
as in their exemplar.  So far as they conform to the divine idea, all
beings were, before their creation, one thing with the essence of
God.}{Suso} The language of Platonic examplars, combined with the
Christian need to emphasize God's goodness and other positive
qualities, have made it almost impossible to think of Godhead
otherwise than as a collection of some definite, yet always
mysterious, essences which, in an equally mysterious way, are meshed
into one.  On the other hand, it was precisely the assumption that the
higher, eventually the highest, somewhat \thi{contains} everything
lower, as a box contains sand or as genus contains species, that
forced one to double things with exemplary ideas and, eventually, to
make Godhead responsible for all the details of the \co{visible
world}.  \co{Virtuality} does not \thi{contain} any \co{distinctions}
which flow from it except as their \co{indistinct} \co{origin}. 
\co{Substantiality} of self-identical, independent entities has been
discussed earlier in, hopefully, sufficient detail.  Application of
this category to the \co{invisibles} leads unavoidably to antinomies. 
But such an application is by no means necessary, even though similar
examples of modeling \co{invisibility} of the \co{origin} on the
Platonic ideas superimposed on the Christian intuitions, could be
multiplied \la{ad nauseam}.

\pa \citt{ `That which is perfect' is a Being, who hath comprehended
and included all things in Himself and His own Substance, and without
whom, and beside whom, there is no true Substance, and in whom all
things have their Substance.  For He is the Substance of all things,
and is in Himself unchangeable and immoveable, and changeth and moveth
all things else.  But `that which is in part', or the Imperfect, is
that which hath its source in, or springeth from the
Perfect.}{Theologia Germanica, I} This, in fact, might be entirely
acceptable if only we do not take that to \wo{comprehend and include
all things in Himself} in this Platonic sense of the pre-exisiting,
ready-made archetypes, and if we are careful to distinguish \wo{His
own Substance} from all the other
\wo{Substances}.\footnote{\thi{Substance} is understood by the author
\citf{not as a work fulfilled, but as [substance or] well-spring}{Th.Germ. 
XXXII} which we can easily interpret in terms of \co{virtuality}.}


\pa \citt{The Godhead gave all things up to God.  The Godhead is poor,
naked and empty as though it were not; it has not, wills not, wants
not, works not, gets not.  It is God who has the treasure and the
bride in him, the Godhead is as void as though it were not.}{Eckhart}
Eckhart is, as usual, most \co{clear}.  Godhead comes probably closest
to the \co{nothingness} -- the lack of any determinations, which
appear first in God.

\pa \citt{One of the greatest favours bestowed on the soul transiently
in this life is to enable it to see distinctly and to feel so
profoundly that it cannot comprehend God at all.  [...]  they who know
Him most perfectly perceive most clearly that He is infnitely
incomprehensible. [\ldots] those who have the
less clear vision do not peceive so clearly as do these others how
greatly He transcends their vision.}{St.  John of the Cross}

Let's only remember that such an `incomprehensibility' is not
any lack of our faculties, but the only thing to be known. 
This \co{transcending} is not any convergence 
towards an inaccessible limit of fullness, but simply the lack of 
anything \thi{comprehensible}. \co{The One} \co{transcends} 
everything because the lack of \co{distinctions} is simply  
beyond any \co{distinctions}. 

\pa In XX-th century, Karl Rahner advanced forcefully the thesis of
the positive content of \thi{incomprehensibility} as the primary name
for God.  Indeed, one may be tempted to repeat, after
Pseudo-Dionysius, the credo of negative theology (as Wittgenstein,
probably, unknowingly did): \citt{and the hidden Mysteries which lie
beyond our view we have honoured by silence}{The Celestial Hierarchy}
But, when it comes to the ultimate \co{origin}, there is a big
difference between \thi{honouring by silence} because one does not
understand what possibly might be hidding behind the veil, or just
because what is behind the veil provides a very poor ground for
speaking being completely undifferentiated.  The adjective
\thi{hidden} is therfore useful only in a metaphorical sense -- the
\co{origin}, the God or, let's follow Eckhart, the Godhead\footnote{Or
who was there first?  Perhaps, Dionysius the Areopagite, with his
distinction between the Undifferenced Godhead and its Differentiated
nature [{\em The Divine Names}, II:3ff], which for him, as the obedient
and dedicated Christian writing, as there are all reasons to assume,
{\em after} the Council of Nicaea, could only mean co-substantiality
of Father and Son, and in fact, the Triunity of persons?  In any case
we are on the border of orthodoxy, both with Eckhart who was accused,
but rehabilitated, and with the great inspiration of St.~Thomas and
medieval mystics, Dionysius, who was all deadly correct and orthodox
as long as he was St.~Paul's convert and pupil, first bishop of
Athens, and what not, until critics of XVIIth century pointed out that 
he could not possibly have been (he probably wrote
in the end of Vth century).  Since then he has been
known as Pseudo-Dionysius and earlier vague doubts as to his orthodoxy 
turned into serious and explicit accusations.  Just a funny story\ldots}
%
-- is hidden
because no \co{visible} categories of understanding, based on
\co{distinctions}, are applicable to the undifferentiated \co{One}. 
For \co{reflection}, however, the difference is crucial because we
thus limit the understanding of the \co{origin}, \la{Deus 
absconditus}\footnote{Is. IVV:15 \wo{Thou art a God that hidest 
thyself.}}, merely against all
differentiation, without endowing it with any contents, not to mention
\co{objective} contents.


\sep

\pa\label{GodheadGod} The \sch\ is directed towards \co{nothing}.  The
resulting attitude, whether \yes\ or \No, is purely \co{spiritual},
i.e., it has no direct relation to any particular region of Being, it
has no \co{objective} correlate%, it is aimed at \co{nothingness} 
--
yet, it is also \co{experienced} by \co{myself}.
In any \co{actual} moment of \co{experience},
\co{reflection} may refer to the \co{foundation} of this attitude
which \co{transcends} all \co{actuality}.  In terms of \co{actuality}
it can thus be ascribed an \thi{objective} character which one may
attempt to indicate in its own terms.

The \co{One} as the ontological \co{origin} is but 
the \co{indisitnct} \co{nothingness}, remote, ineffable and indifferent. 
The \sch\ affects this \co{nothingness} in the most fundametal way -- the
\co{love} \co{experiences} it as the \co{origin} and thus as fullness,
while \co{self-centredness} as a mere void, at best, as an 
indifferent principle of ultimate transcendence.  As \co{reflection} 
always tends to posit an \co{object}, any 
\co{experienced non-actuality} tends for it to assume 
a form of subsisting entity. Although this is wrong, it may also be legitimate 
to the extent that \co{reflection} stays aware of that, that is, does 
not identify the \thi{objectified} \co{analogue} with its original source 
or, perhaps, does not ascribe its {objectivity} to any \co{object}. 
%
The uncritical \co{analogues} arise from \co{positing} an \co{object}
to which they are attached, which easily leads to all kinds of 
antinomies.  In a genuine sense, the
\co{analogues} are but expressions of the most 
\co{concrete}, underlying \co{experiences} -- 
 expressions of the radiating, all-embracing \co{love} 
which is not \co{mine} but, on the contrary, which dawns on \co{me} 
from \co{above} as a \co{transcendent gift}, a \co{gift} in which \co{I} 
only \co{participate} and which affects the whole Being.

\pa Thus, various aspects of the respective attitudes may be expressed
in the \thi{objective}, or better, \thi{objectified} terms as
properties of something.  What this something is, does not matter much
and, indeed, must remain for \co{actual reflection} completely
\co{vague} and indeterminate.  Yet, the \yes, the \yes\ to the
\co{invisible} Godhead, is also a \yes\ to the world, while \No\ is
the \No\ to Godhead which is groundend in the wish to say yes to the
world.  \co{Reflection} may, and in fact it does, take recourse to an
\thi{analogical way of speaking}, projecting the aspects of the
\co{spiritual} attitude as properties, as \co{analogues}.  As Eckhart
says \wo{The Godhead gave all things up to God.}.

\pa
The \co{One}, the ultimate \co{transcendence}, is the most intimate, 
ever \co{present} aspect of all \co{experience}, is the deepest, all 
pervading \co{immanence}. The \sch, bringing these two together into 
the \co{presence} of all \co{experiences}, will thus encouter the 
\co{analogues} as permeating all 
\co{experience}. The \thi{analogical way of speaking}, the \thi{more 
than} of Eriugena, the \thi{omni-} preceding all superlative 
attributes of God, are but expressions of 
the \co{spiritual experience} of \co{incarnated presence}, of 
 the thorough 
\co{presence} of the ultimate \co{transcendence} in the \co{immanence} 
of all \co{experience}.\footnote{The term 
\wo{\co{analogy}} is used in
%a sense which is an inversion of its traditional use.
the way Aquinas would only partly accept.  \co{Love},
\co{humility}, etc. are not only genuine \co{aspects} of \co{spirit},
as he would say, of God beyond our comprehension -- they are also
adequately, absolutely and non-analogically predicated about it.
(This is possible because  \co{spiritual} relation to \co{nothingness} is
not something absolutely transcending human experience but, on the
contrary, the most \co{immanently present}, whether \co{concretely} or not,
\co{aspect} of such experience.) 
On the other hand, they not only belong merely \co{analogically} to
the aspects of \co{visible} experience, but are also predicated
\co{analogically} about it.  For this experience, even if prior with
respect to the \co{reflective} knowledge, is actually \co{founded} in
the \co{invisibles}. It is from there that words like \wo{love}, \wo{humility},
\wo{presence}, etc. obtain their genuine meaning, which is only 
\co{analogically} applied to the \co{visible} \co{analogues} of the
\co{spiritual love}.}


\subsubsection{\yes}

\pa\label{adhumility} The \co{humility} of \co{love} means
\co{recognition} that \co{I am not the master}, \co{recognition} of
the \co{origin} as the ultimate power which is the power of the
source.  Even if nothing \co{actually} is the way it was at the
beginning, when it emerged \la{in illo tempore} from the
\co{virutality} of the \co{origin}, so without this source, nothing
would be there.  Without the indefferentiated ground, no
\co{distinction} would be possible.  This -- not the ability to
determine every minute detail of \co{this world} -- is the
\co{omnipotence} of God. 
Without Him there would be nothing, which
is very different from saying that everything is the way He
has made it. 
\noo{
It is only in a lowest sense that power (potence?)  is the ability of
\co{actualization} (in the traditional language, of 
actualizing/realizing the potentiality). 

Buber, Das Problem des Menschen, p.55: Great man [\ldots] is strong 
[\ldots] but he does not desire power. That which he desires is 
realization of his intension: realization of spirit. For this 
realization he needs, of course, power because power -- if we clean 
this notion of the dytyrambic pathos in which Nietzsche `spowil 
je' -- means nothing else but simple the ability to realize that which 
one dsires to realize.
}

\subpa In a much less geniune sense (meaning, leading to, or even
perhaps dispalying, a misunderstood \co{objectification}), one speaks
\co{analogically} about \thi{His will} and \co{my} obedience or
disobedience to it.  Now, the \wo{Godhead is poor, naked and empty as
though it were not; it has not, wills not, wants not, works not, gets
not}.  It is only \co{I}, only a human being who may will.  \thi{God's
will} is but an expressions conveying what \co{I} should will -- to
say \yes.  There is no more content in \thi{God's will} than the
salvation through \co{love}.  \citt{Sin is nothing else than that the
creature willeth otherwise than God willeth, and contrary to Him.}{Th. 
Germ.  XXXVI} Indeed, it is.  But this \thi{will of God} which, in a
deep confusion one may attempt to look for in \co{visible signs}, is
nothing else than that \co{I} do not sin, that \co{I} do not make
\co{idols} of \co{this world} but, instead, find the \co{absolute
love}.  For \citt{as long as a man is seeking his own good, he does
not seek what is best for him, and will never find it.}{Th.  Germ. 
XXXIV} If \co{I} want to find it, then \co{I} am fine; if not, then it
is \co{my} private buisness, though not willing the highest good of
\co{love} is its own punishment, just like \citt{blessedness is not
the reward of virtue, but virtue itself [\ldots]}{Spinoza, Ethics, 
V:XLII}

\subpa Similarly, one speaks about \thi{God's works}.  But He
\wo{works not}. Even \citt{[g]race does not perform any works; it is
too subtle for that and is as far from performing any works as heaven
is from earth.}{Meister Eckhart.  ``German Sermons'', DW 38, W 29,
p.~118.} The \co{invisibile incarnation} precedes all
\co{distinctions} of \co{actual} kind which would allow any works. 
And all such works happen only through the being which makes the
respective \co{distinctions}.  \co{I} do the works, \co{you} do the 
works, \co{he} does the works. Like \thi{God's will} so also \thi{His
works} are expressions, not very fortunate ones, of the
\co{experienced} \co{transcendence} which \co{inspires} \co{me} to do
what \co{I} do, and which hardly ever determines the particular contents of
these works.

\pa\label{adthankfulness} \co{Thankfulness} of \co{love} amounts to
\co{recognition} of the \co{origin} as \co{good} or, which amounts to
the same, as the source of \co{goodness}.  \co{Thankfulness} is but a
\co{reflection} of the acceptance of the \co{origin}, of the
\co{recognition} of its \co{goodness}.  This \co{goodness}, if taken
in itself, is empty and impossible to characterize.  It does not mean
{\em anything else} except the attitude of \co{thankfulness} and
\co{acceptance}, nothing except the fact that \co{I} \co{recognize}
the value of everything which \co{I} encounter and am willing to
accept it with the underlying \co{love}.
It is not the
attitude of \yes\ which is \co{good}, it is not some God who is
\co{good}.  \co{Goodness} is the \co{experience} of \co{thankfulness}
rendered in terms of \co{actuality}, is an \co{analogue} of the latter. 
Nobody, who does not feel this \co{thankfulness} for being can ever
understand, not to say experience, the \co{goodness} of God. 
\footnote{If you
feel like that, you are welcome to draw from that the consequences for
the apparent and celebrated \thi{problem} of how possibly a good and
omnipotent God might have created all the evil in this world. We will 
return to this issue.}

\subpa Again, in a less genuine sense, one speaks about \thi{God's
love}, \thi{God's benevolence}, etc.  Misleading as such expressions
may be, they stand for the purity of \co{thankfulness} which is its
own reward.  It \citt{is not chosen in order to serve any end, or to
get anything by it, but for the love of its nobleness, and because God
loveth and esteemeth it so greatly.}{Th.  Germ.  XXXVIII} There is no
being, no non-being, no \cross{being} -- if you insist, no God --
sitting there and loving or esteeming anything.  This love and esteem
is but the value, the \wo{nobleness} such a \co{love} and
\co{thankfulness} have in themselves, opening \co{me} to the 
\co{transcendent gift} which gives the ultimate value to \co{my
life}.

\pa\label{adopenness} The \co{openness}, we can say, is openness of
the heart for all gifts of the \co{origin}.  In more concrete terms,
it can be taken as preparedness to meet with the open heart everything
and everybody but for the moment we are relating it only to the
perception of the \co{One}.  \co{Openness} means that we
\co{recognize} it as omnipresent, that everything is encountered with
the fundamental, implicit or explicit understanding of this being a
\co{gift} of the \co{origin}, of its being a hierophany and, hence, of
the \co{origin} being \co{present} in this.

\subpa\label{selfomniscient}
\co{Self-awareness} is an aspect of every \co{actual} encounter,
which makes \co{me} always, even if only implicitly, aware not only of
the thing, the \co{object}, the situation \co{I} am confronting, but
also of the fact of this confrontation, as well as, of its being
anchored in the midst of the field of \co{experience}
\co{transcending} the limits of \co{actuality}.  Although formally we
can say that it is \co{my} \co{self-awareness}, yet it does not
\thi{belong} to \co{me}, it is not something \co{I} determine and
control -- it accompanies \co{me} as \co{my} associate, not as 
\co{my} attribute.  
In \co{my} focusing on the \co{actual} contentes of
\co{experience}, it witnesses to the \co{presence} of something that
\co{transcends} it. Feeble and dependent on \co{my} recognition, on 
\co{my} 
acceptance of its voice, in the context of \co{love} it founds the 
\co{analogue} we might call \wo{God's \co{omniscience}}. 

It may take 
the form of a voice of conscience which discloses the \co{spiritual} 
context of \co{my action}, it may be a mere \co{awareness} of the
\co{presence} which \co{transcends} \co{actuality} and also, in the 
most figurative sense, modified by the \co{reflective} attention directed 
towards it, it may appear as the feeling of \thi{somebody looking 
at me}. 

\sep


\pa We could refine the description of \co{love} and continue along
the same lines but it shouldn't be necessary to play the game of
deriving all possible names of God.  If one got the first, one also
got the last.  Beyond being the \co{origin}, the \co{One} is but the
\co{One}.  There is \co{nothing} to be known about it and no names
are necesarry even if almost all are possible,   
\citt{the Godhead is nameless, and all naming is alien to Him}{Eckhart}.

Whether \la{El} (\wo{Mighty}) or \la{El Shaddai} (\wo{Allmighty}),
whether \la{Adonai} (\wo{Lord}) or \la{Logos} (\wo{Word}), whether
\la{YHWH} (\wo{I am who I am}) or \la{Elohim} (\wo{Might One}), whichever
of the 99 names one chooses, one could always invent a new one,
because God can be named by any name, and every name is approrpiate if
only pronounced in the humble attitude of \co{love}.  All the disputed
and undisputed names and attributes are but the attributes arising
from the attitude -- the \co{spiritual} attitude -- of the one who
pronounces them.  The question is not what names are and what are not
appropriate, but what attitude makes a name appropriate.\footnote{I
once saw a Professor of Philosophy and Logic performing a formal proof
(in first-order logic) that a God about whom we predicate
\thi{Goodness} and \thi{Allmight} could not exist.  The proof was
correct and the Professor seemed quite satisfied with himself.  I did
not ask if he meant it seriously but I suspect that he might.  In a
particular case, it may be hard to say if rigidity is an expression of
pride, of insecurity, or of both, but it is easy to notice that one
can be both intelligent and stupid.} \citt{[T]he form of God is itself
the joy with which it is recognized}{Visvanatha}

\subpa Indeed, if one wants to imagine the need to speak about God, 
the \co{spiritual love} will direct one towards
silence and, perhaps sometimes, lend one the words.  \citt{Just say
whatever is given you at the time, for it is not you speaking, but the
Holy Spirit.}{Mrk. XIII:11} If such \co{love} is absent, any naming 
will be wrong.
%
The questions \wo{Shall we say that God is?}, \wo{Shall
we say that He is not?}, \wo{That He is something}, \wo{That He is
nothing?}, \wo{That He is not something?}, \wo{That He is not
nothing?} may all be answered in positive because they are all
questions about a mere way of speaking.  
Their insistent presence in the history is but a result of the 
need -- not so much the need to speak about God, as the need to 
speak, eventually, the need to reduce everything to the sphere of 
\co{visibility} to which one remains incurably \co{attached}. 

\subpa
But words are only words, they are always only
\co{signs}.  They are significant to the extent that they should help 
to draw right \co{distinctions}, 
to grasp the meaning and should not be misused to confuse or cover up
the lack of meaning.  But if one understands what is being said, then
one knows that words do not matter that much as some sometimes would
like to think.  And if one does not understand, then one need no 
name but a hint how to get to the place where names begin, that is, 
cease to mean.
Any name of God, if taken as a final, absolutely
correct \co{sign} coinciding with anything real is wrong.  But if
taken as a mere \co{sign}, always incomplete and inadequate, always
incommensurable with its intended meaning of \co{concrete love}, then
most names will be adequate.  Words are only \co{signs}, thoughts are
only \co{signs}, ideas are only \co{signs}, things are only
\co{signs}, everything \co{actual} is but a \co{sign} of something
which matters, which transcedning the \hoa, is always \co{present}.
It is only \co{love} that can reach that far, the \co{spiritual}, that 
is, unconditional \co{openess}.  
\citt{Whoever blasphemes against the Father will be forgiven, and
whoever blasphemes against the son will be forgiven, but whoever blasphemes
against the holy spirit will not be forgiven, either on earth or in 
heaven.}{The Gospel of Thomas, 44}


\subsubsection{\No}
\pa
The attitude of \co{self-centeredness}, negating the efficiacy of the
\co{One}, refusing to accept that \co{I am nothing}, does not lend its characteristics to possible descriptions
of God. The \thi{objective} \co{analogue} of this attitude amounts to the 
simple conflation of \co{nothingness} and void -- beyond the 
\co{visible world}, there is emptiness. Yet, the \co{objectified} 
characteristics of the respective aspects -- of \co{pride}, 
\co{ingratitude} and \co{closedness} -- carry a lot of strength, 
even if one is unable to say to what they possibly could be 
ascribed -- to the world? to the life? to my life? to the proclaimed 
void of nothingness?  

Most importantly, this attitude opposes the thought that it is an
attitude resulting from a choice, from a \sch.  There is nothing
there, and so there has never been anything inviting to, not to say,
forcing me to choose anything.  The \No\ is occupied almost
exclusively with finding \co{visible} reasons serving as its
justifications -- in the world, in life, in my life.  The lack of
\co{visible} correlate is sometimes compensated by an idol and, most
often, by a shifting focus on various elements from this list.

\pa\label{adpride} \co{Pride} is not necessarily a personal pride, an
individual attitude of superiority over others.  \co{Pride} is merely
an attitude which does not \co{recognize} any higher power, any
\co{origin} beyond \co{visibility}.  The \co{analogues}, the
\thi{objectified} expressions of that embrace so many idols that one
can hardly attempt to enumerate them all.  The \co{objectivistic
illusion} from \ref{objectivisticillusion}, the assumption or
conviction that everything consists of \co{things}, eventually, of
\co{objects}, is an important example.  It underlies all kinds of
intellectual arrogance, naive or sophisticated scientism, exclusive
worship of causality and \thi{hard facts}.  As has been frequently
observed, humanism is another field providing a host of examples. 
\citt{But how can anyone judge or love what he does not know?}{Pico
Della Mirandola, Oration on the Dignity of Man} -- this is, perhaps,
the most concise summary of The Oration.  The real question, however,
is what one understands by \wo{knowing} and what is there worth such
\wo{knowing}.  In a sense, all I am doing is nothing else; this text
is an attempt to elucidate some fundamental issues of our life -- it
is humanistic.  But the adjective rings wrongly because the human
nature is not so simply human as humanists would often like
it.\footnote{Although the tone of The Oration is certainly of the kind
which could motivate the later development of humanistic \co{pride},
one should remember that it does not call for the simple reduction of
the world to the merely \co{visible}, human, all too purely human,
dimension.
% MOVED \citf{Let us disdain things of earth, hold as little worth
% even the astral orders and, putting behind us all the things of this
% world, hasten to that court beyond the world, closest to the most
% exalted Godhead.  There, as the sacred mysteries tell us, the
% Seraphim, Cherubim and Thrones occupy the first places; but, unable to
% yield to them, and impatient of any second place, let us emulate their
% dignity and glory.  And, if we will it, we shall be inferior to them
% in nothing.}{Pico Della Mirandola, Oration on the Dignity of Man}
According to Pico, the most highly human vocation is to transcend the
purely human level to which much of the humanism would like to limit
humanity.  The Oration primarily emphasizes that such a transcending is
possible.  In fact, the
later in history we move, the more \co{pride} can be discerned in
whatever calls itself \wo{humanism} at the moment.  Eventually, the
category of \thi{humanism} has been appropriated by forces and
movements which seem to have very little to do with its original form. 
Therefore, the above reservations against the use of the word.}

\pa\label{adingratitude} The \thi{objectified} \co{analogue} of
\co{ingratitude} is the image of life and world as, if not basically
and essentially so, in any case, to large extent, bad, mischevious,
perhaps, even evil.  In the world we meet many things and situations
and most of them require an attitude of suspicion and scrutiny.  One
can easily see that such a project can hardly fail and the field for
performing Voltairean dances is indeed inexhaustible.  One will always
find many serious examples which can be used as strong reasons
justifying ungrateful attitude, indeed, ridiculing any idea of
\co{greatfulness}.  And, in fact, in many situations one better stay
alert.  I am not saying that there are no healthy forms of alertness. 
But there is a great, a fundamental difference between seeing a danger
in the particulars of a situation, and seeing danger everywhere,
between being wary of a person who creates an impression of dishonesty
and being wary of all people, perhaps, even of all people in general. 
The suspicious alertness is the fundamental modus of ungratefulness,
reflecting and originating in the general idea of the world rendered
in terms of harm and reward; the world which, moreover, unless \co{I}
prevent it, will do me some harm rather than reward me.


\pa\label{adclosedness} The \thi{objectified} \co{analogue} of
\co{closedness} is the simple statement \wo{The world is as it is}. 
The facts are there, true, \thi{objective}, irresistible, and the only
thing we can do is to conform to them, possibly, to manipulate them so
as to achieve our goals.  The apparent activity of such an attitude of
smartness towards the givens is underlied by the fundamental,
\co{spiritual} passivity, the all pervading resigned acceptance of
givens as givens -- the world, after all, is as it is.  It becomes
stiff and rigid, not necessarily because it is how science is forced
to see it, but because it has been raised to the status of the highest and
only reality governed, as Stoics would have it, by the irresistible 
laws of necessity.

\subsub{Experiences or projections?}

\pa I expect one may ask a natural question.  Does not \sch\ amount to
a projection?  Do I not say that the \co{indistinct} and unknowable
\co{One} has to be endowed with the qualities of the source, goodness,
power and what not?  The answer is no, and if you see this, you may
safely skip this section.

Indeed, 
there \citt{can be no greater incongruity than [for a disiple of 
Spencer] to proclaim with one breath that the substance of things is 
unknowable, and with the next that the thought of it should inspire 
us with awe, reverence, and a willingness to add our co-operative push 
in the direction toward which its manifestations seem to be 
drifiting.}{W. James, Essays in Pragmatism, I, p.19}

The \sch\ does not call for any awe, reverence, or any willingness to
add this or that to this or that. It does not call for any attitude -- 
it helps create it. The \co{actual} attitude will in any case be 
determined also by \co{my} character, \co{my} intelligence, \co{my} 
physiology, by all \co{visible} and hardly \co{visible}, conscious 
and 
unconscious aspects of \co{my} being. In \co{precise} terms, the \co{choice} 
does nothing and the \co{love} merely adds a gentle touch, an 
\co{invisible rest}, which is as important, deep and decisive as it 
may seem insignificant.

It is the \co{choice} between the only
two alternatives offered -- not by the \thi{unknowable}, but by the
\co{indistinct} in so far as it can be known, that is, as \co{One
above distinctions}.  What I have done with, or rather {\em out of}
this concept in the first Book, can be taken as a mere illustration of
the grounds which might incline \co{me} towards seeing it as a true,
\co{virtual origin}, that is, towards saying \yes.  The
point is, that there is no necessity, no sufficient reasons which
might force one to make this, rather than the opposite choice. But 
the \co{choice} itself is thoroughly real, it is a choice between two 
real alternatives. 

\pa We should carefully distinguish the \co{choice} from mere
psychological effects.  According to James, \citt{to find religion is
only one out of many ways of reaching unity [\ldots] In judging of the
religious types of regeneration [\ldots] it is important to recognize
that they are only one species of a genus that contains other types as
well.  For example, the new birth may be away from religion into
incredulity; or it may be from moral scrupulosity into freedom and
license; or it may be produced by the irruption into the individual's
life of some new stimulus or passion, such as love, ambition,
cupidity, revenge, or partiotic devotion.  In all these instances we
have precisely {\em the same psychological form} of event, -- a
firmness, stability, and equilibrium succeeding a period of storm and
stress and inconsistency.}{W. James, The Varieties of Religious
Experience, VIII, p.176} 

One can form hierarchies of genuses and species as one finds
appropriate but if these have \thi{the same psychological form} (which
here I would tend to take as \thi{psychologically indisinguishable}),
then thank you very much for the psychological contribution -- here
our ways part with psychology.  Indeed, having only \co{actual
experiences} of a \thi{subjective} psyche as the basis of
distinctions, all such states may happen to end up in the same sack. 
Yet, even James does not include these later instances in his
treatement of the religious experience.  So, after all, they are
disinguishable?  The sense of purpose, of direction and goal, of
mission, or else of finding a valuable sphere of \co{experience} may
indeed, especially if taken as absolute, give firmness and stability. 
All \co{idols} can, and many minor matters can.  But the unity
\co{founded} by the \sch\ is not derived from any sense of goal,
direction or mission -- the goal is \co{nothing}, the direction is
\thi{anywhere}, and the mission is \thi{\co{love}, and do what you
will}.\footnote{As much as I owe to and have for pragmatism when
applied to the phenomena of the \co{visible}, here my ways part
definitely also with its ways.}


\subpa \co{One} can \co{manifest} itself in innumerable ways which may
be psychologically as different as trembling and adoration, as fear
and attraction.  I think that these differences may often be traced
behind different \co{visible} characters of various religions.  

\co{An experience} of God's presence will almost inevitably have
tremendous influence on one's life, and the form of this influence may
depend heavilty on the character of the experience.  But it is not its
character, its content which may account for its influence -- it is
the lack thereof, \co{expressed} as the tremendous force, 
\la{majestas}.  Psychologically distinguishable content plays
its part but what is constitutive for such \co{an experience} is what
this content reveals -- the ultimate, \co{absolute} force which groans
into one's face without showing its own.  It is the {\em intesity} of
such \co{an experience}, its irresistible power, which is its
essential content, not the form under which this content appears.  And
this power is \co{objectless} and contentless, it has no agent, it is
the power of \co{nothingness}, but it {\em is}. There is, 
consequently, nothing to be projected, there is only something to be 
recognized -- in the most mundane sense, \co{that I am not the master}. 

\subpa
\Sch\ is not a matter of \co{any experience} just like religiosity is
never reducible to \co{any experiences} which, perhaps (though I do 
not believe even in this \thi{perhaps}), may be
psychologically indistinguishable from a sudden attack of fear on a
neurotic or even a healthy person, or else from the attacks of 
infantilism or \he{senility}. It happens in a meeting with the 
\co{absolute nothingness} which involves {\em full and active
participation} of the invaded.  \Sch\ may take place long after the
actual experience and also when no such experience ever found place.

A meeting with \co{absolute} \thi{objectivity} 
does not require any specific context or experience, even if
specific experiences play usually (psychologically) the role of
motivating factors.
%
Such experiences are possible \co{actualizations}, as Otto says,
\thi{schematizations} of the \la{a priori} ground of all
\co{experience}.  To the extent the \co{presence} of \la{numinosum} is
recognized \co{beyond} their content, 
they themselves are \la{a priori} -- irreducible to any \co{visible
experiences} nor categories, concepts or feelings of \co{mine}, but
grounded in the ultimate \co{invisibility} of our being.

%
The fact that \co{absolute} may (in fact, always does) invade only one
person and not another is such an argument for its \thi{subjectivity}
as it would be against the objectivity of Japan that some people were
there while others were not.  That it is unverifiable?  Excuse me! 
What is?  We have read Popper, have we not?  It is as unverifiable as
the accusation of its being a \thi{subjective projection} is
self-confirming.  For as long as one insists on such a
characterization, one is simply unable to get any meaning whatsowever
of its nature and value.  \citt{\la{A priori} recognitions are not the
ones which everybody has but ones which everybody may have.}{Rudolf
Otto, Das Heilige, XXII, p.195 [\wo{Recognition} translates here the
word \wo{Erkentnis}.]}


\pa \Sch\ is not a choice of love, of morality, of charity, of
unselfishness, or whatever.  It does not aim at any such nor any other
things.  It is the pure and bare \yes\ (or \No).  It does not choose
any specific content which it might try to project \thi{outside} of
its \thi{subjectivity}.  It is a choice of the fundamental attitude. 
As it happens, the choice has tremendous consequences, but these are 
consequences, not projections.
%%% Here I had something more to say


\sep


\pa Consequently, neither the genuine \co{analogues} are any kind of
projections.  They are certainly not any psychological tricks played
by subconsciousness on one \thi{who does not know what he is {\em
really} speaking about}.
%-- they may be fully conscious \co{expressions}.  
Such a critique \fre{\'{a} la} Feurbach applies to the uncritical use
of \co{analogues}, the use which \co{posits} some indefinite
\co{object}, a projection of \co{reflective} understanding, and endows
it with some attributes.  But when used in the genuine sense, they are
\co{reflections} of the underlying \co{experience}, of the
\co{incarnation} of \co{love} -- they are \co{expressions}.  Since
this \co{experience}, once it comes, confronts \co{me} with the
ultimate \co{transcendence} and its \co{gift}, there is no way of
seeing it as a mere projection of human, all too \thi{subjective},
psyche, unless one is unable to renounce the absolutism and narrownes
of the distinction \thi{subjective}-\thi{objective}.  Saying that
\wo{God is but a projection of myself} is to conflate completely
\co{myself}, \co{my Self} and \co{Self}, and to confuse the
\co{transcendence} of the latter met in the \co{experience} of the
former.\footnote{Instead of \wo{projection}, one might perhaps find
another word for the unity of \co{spiritual aspects} as applied to
soul or world.  Irenaeus called the process whereby the passions of
Sophia are materialized and turned into substance
\wo{\gre{enthymesis}}.  \citf{[\ldots] who would not expend all that
he possessed, if only he might learn in return, that from her tears of
the \gre{enthymesis} of the Aeon involved in passion, seas, and
fountains, and rivers, and every liquid substance derived its origin;
that light burst forth from her smile [etc.]}{Irenaeus, Against
Heresies, I:4.3 [in Roberts and Donaldson, Ante-Nicene Fathers, vol. 
I], [Creation of Consciousness, p.72]} But who will want to see behind
such a mere \thi{projection} a {\em genuine} process of \ger{Beseelung},
\thi{ensoulment}?  If one thinks that projections are something
discovered only in XXth century, one might consult again Gnostics and
neo-Platonists, with their \gre{probol\'{e}}, which usually translated
as \thi{emanation}, may be equally well rendered as \thi{projection}. 
[Creation of Consciousness, p.76] }


\pa As said in \refpp{GodheadGod}, the \co{analogues} are only
reflections of the respective aspects of the underlying attitudes of
\yes\ or \No.  But the word \wo{\co{analogue}} does not mean that
these are some artificial constructions.  They are, in fact, the
possible qualities of \co{experiences}, and the tendencies, as well as 
one's attitude, can be \co{experienced} precisely through such
\co{analogues}.
%The main difference is that while the latter can be \co{actual 
%experienced}, the former \ldots?

\pa If one experiences world, life, things, the indefinite everything
as fullness, as good, as powerfully meaningful, one has perhaps not
said \yes, but one has a good chance to do so.  Such \co{experience},
or such qualities of some \co{experiences}, are ways of \co{presence}
of \co{invisibles} even if the \co{act} of \sch\ has not taken place. 
\co{An experience} \co{that it is} can be embraced by an aura of
intense beauty and wonderful mystery, even if the \co{actual object}
has no such qualities.  (Does any object {\em have}?)  Such
\co{experiences} are \co{gifts} of the \co{origin}, \co{gifts} of
\co{presence} which need no active or volitional participation on our
part.  But also the \co{acts}, the \co{acts} of gratitude,
benevolence, compassion, generosity, friendship, etc., which
presuppose the \co{open} attitude at least toward the \co{actually}
concerned object, situation or person, will result in a definite and
\co{clear} \co{experience} of \co{participation} in something which is
not limited to the boundaries of \co{miness}.

\pa In other situations, one may experience everything as basically
void and meaningless, or, perhaps, simply as coldly indifferent. 
\co{An experience} \co{that it is} may present one with a ridiculously
meaningless \thi{this}, dissociated from any context, an offence to
any sense of purpose and meaning.  It may be filled with disgust and
nausea, so thoroughly studied by Sartre and other epigons of
existentialism.  Such \co{experiences}, too, are \co{gifts} of
\co{presence}, perhaps, reflecting the hidden currents of
\co{invisible} \No, perhaps, signaling the stage of purgatory
transition.  Taking such \co{experiences}, or such qualities of
\co{experiences}, as the definite and last word about the world, one
will look for reasons, explanations, excuses, eventually indeed
\thi{projecting} them on the world and its essential nature.  

Experiences of this kind hurt to the bottom of one's soul which tries
desperately to escape them.  Projections, trying to \thi{explain} and
\thi{justify}, that is, bring release by re-establishing unity, by
making me and the world friends again, paradoxically have the opposite
effect -- they only \thi{objectify} the painful contents of the
\co{experience}.  The \thi{objective} reasons may be innumerable -- it
may be some \co{recognized} mischieves, some petty or imporatnt evils,
eventually, the vicious, because indifferent, emptiness of the world
in general.  If I would agree on calling them \wo{projections}, it is
because they, genuine and \co{actually experienced} as they are, do
not \co{express} the ultimate end and have to substitute for it
\thi{the nature}, \thi{the world} or whatever they choose as that
which \thi{actually and objectively is out there}.  Indeed, the
\co{choice} of \No, \co{attached} to the distinctions of \co{mine} vs. 
\co{not-mine}, \co{subjective} vs.  \co{objective}, seems bound to the
language of \wo{projections} (in addition to the common
\thi{receptions}, i.e., all forms of world's influence on \co{me}),
because even in its luckier and more happy forms, it still can not
avoid blurring these distinctions, it still can not distinguish
\co{precisely} between world and \co{my world}, between \co{my world}
and \co{my life}.


\pa Whether of the positive or negative character, such
\co{experiences} of the indefinable \thi{totality}, the
\co{experiences} which seem to announce the deepest essence of all
things, accompany our life.  And they do it not only as occasional
\co{actual experiences} but also as \co{qualities} of \co{our life}
which happen to be also \co{qualities} of \co{this world}.  As
long as they are not referred back to the \co{spiritual} \co{origin},
they are most \co{clearly experienced} only through the \hoa; first,
as the indefinable \co{rest} of \co{acts} and situations and then,
when we \co{reflect} over them, as qualities, as properties of
something.  

\pa 
Conversely, the \sch\ will found \co{experiences}
reflecting the attitude towards the \co{origin}, \yes\ will found
\co{actual experiences} of fullness and goodness, while \No\ of
meaningless, perhaps even evil, emptiness.

As long as \co{I} remain centered on the sphere of \co{visibility},
%as long as \co{I} do not perform the \sch, 
\co{I} do not see that \co{experiences} of totality of the world and
life are only \co{analogues} of \co{my} attitude because, as a matter
of fact, \co{I} have not as yet assumed any attitude.  They are
scattered pieces which come and go as they like.  \co{I} might have
heard some \co{vague} \co{calls}, might have had some \co{vague
intuitions}, might have had some particular \co{experiences}, but
these have not been solidifed.  As long as the \sch\ has not been
made, such \co{experiences} will easily shift from one extreme to
another.  \citt{Those by the way side are they that hear; then cometh
the devil, and taketh away the word out of their hearts, lest they
should believe and be saved.  [\ldots] these have no root, which for a
while believe, and in time of temptation fall away.}{Lk.  VIII:12-13}
%
Both wheat and chaff, both \yes\ and \No\ tendencies
\citt{grow together
until the harvest: and in the time of harvest I will say to the
reapers, Gather ye together first the tares, and bind them in bundles
to burn them: but gather the wheat into my barn.}{Mat.  XIII:30/Lk. 
III:17}


\pa Unlike any choice based on some definite, particular
\co{experiences}, on thoughts or feelings, the \sch\ is \co{absolute},
that is, not relative to any particular being or region of Being. 
Every choice can be made to suspend the relevance of subsequent
feelings and thoughts; as an \co{act} of \co{my will} it may simply
say: \wo{I choose this, no matter what might come}.  However, any
choice related to particular aspects of \co{experience}, any choice
motivated by and based on such particulars, will also continue being
involved in them.  Eventually, changes in their configuration may
render sticking to the original choice the matter of pure dogmatism,
inflexibility, stubborness. Every choice is a projection, a projection 
of its \co{actual} decision into the future, every \co{act} and 
\co{action} is a projection (Heidegger would say, a \wo{project}) 
saying \wo{I want this thing to be {\em so}}.

The \sch, too, is an \co{actual act}, and thus, in a somewhat similar
way, it too \co{externalizes} its content.  But this
\co{externalization} does not result in any particular \co{object}, in
any \co{dissociated} entity, nor in any quality ascribed to something
which, subsequently, might turn out not to possess it.  It is the
\co{act} of the recognition of ultimate \co{transcendence}, say, the
ultimate \thi{objectivity}, which is not dependent on the form or
quality of any possible and \co{actual experiences}.  Through this
\co{act} \co{reflection} admits the insufficiency of its modus which,
in the extreme case, is but the \co{objectivistic illusion}.  It
recovers the constant, underlying all \co{experiences} \co{presence},
which \co{reflection} always knows, if only dimly, through the
\co{self-awareness}, that is \co{awareness} of the \co{transcendence}. 
The \co{truth} of this \co{act}, the conformance to the \co{origin},
is thus \co{recognized} as lifted \co{above} and lasting beyond and
independently from \co{this world}.  \citt{Therefore whosoever heareth
these sayings of mine, and doeth them, I will liken him unto a wise
man, which built his house upon a rock: And the rain descended, and
the floods came, and the winds blew, and beat upon that house; and it
fell not: for it was founded upon a rock.}{Mat.  VII:24-25/Lk. 
VI:47-48}

\pa The \sch\ of \yes\ does not involve any feelings or impressions,
does not involve any specific thoughts or contents which might be
projected.  It is a response to the \co{command} of \co{nothingness},
a \co{reflection} of the \co{origin}.  It leaves all feelings and
thoughts, all \co{visible} \co{signs} and particulars \co{below},
centering around the essentially, not only potentially \co{invisible
One}.  \citt{The person is like a wise fisherman who cast his net into
the sea and drew it up from the sea full of little fish.  Among them
the wise fisherman discovered a fine large fish.  He threw all the
little fish back into the sea, and easily chose the large fish. 
Anyone here with two good ears had better listen!}{The Gospel of
Thomas, 8}

\pa This \co{choice} is \co{absolute} also in the sense that it
effects a final division, a definite separation of \wo{wheat from
chaff} which until now have been mixed together.  It takes the spell
of \co{closedness} from \co{this world} and \co{opens} it to the
\co{inspiration} from \co{another}.  But this happens only through an
\co{absolute} renounciation of \co{this world} -- any pretensions on
its part to the \co{absolute} validity and importance, to the 
ultimate and all-determining 
\thi{objectivity} are removed.  \citt{Think not that I am come to send
peace on earth: I came not to send peace, but a sword.  For I am come
to set a man at variance against his father, and the daughter against
her mother, and the daughter in law against her mother in law.  And a
man's foes shall be they of his own household.}{Mat.  X:34-36}


It is the \co{choice} of the attitude which does transform the world;
not, however, the \thi{objective world} which is what it is \thi{in-itself},
but the the world which is, as it always has been, \co{my experience}
of the world, \co{my life}.  If somebody wants to call this
\wo{projection}, it is his choice.


\newpage

\subsub{Concrete acts \ldots preparations}

\pa
As we have said, upbringing and tradition may serve as invaluable 
means of preparation for the possibility of encountering experiences 
like those leading one to \sch. They do not in any way condition such 
an experience but only serve as guidance once it occurs. They are 
actually the acts which themselves may be concrete forms of such a 
\co{choice}, and \ldots 

\titp{Prayer}
\pa
It is certainly not the proper place for reviewing the Catholic or
other taxonomies of prayers. Interested may consult relevant sources. 
I only want to emphasize the basic version which boils down to 
the prayer of Sant Augustine: 
\citt{Grant what you command and command what
you will.} {[Da quod iubes et iube quod vis], Saint Augustine,
"Confessions," X:29 40 CCL 27, 176} 
%cf "De Gratia el Libero Arbitrio," XV: PL 44 899.} 

\pa Prayer is an \co{act} of submission.  A genuine prayer never asks
for this or that\ldots it may express hope of obtaining the
\co{invisible gift}, but that is as far as it may go. For,
to begin with, \citt{we know not what we should pray for as we ought; but
the Spirit himself asketh for us with unspeakable
groanings.}{Rom:VIII.26}
Eventually, a \co{spiritual prayer} takes mostly on a 
doxological form and praising Godhead, praises His  
elevation \co{above} \co{me}, \co{my life}, world, that is, 
\co{expresses} thankfulness, openness, submission.

There is no such thing as an abstract prayer or a precise prayer --
prayer is thoroughly concrete, or it is not a prayer at all.  Reciting
a text may be called a \wo{prayer} only because we know that this text
is meant for, or is usually used as a prayer.  But prayer is an
\co{act} which, hopefully, \co{expresses} an attitude.  An attitude of
prayer, like any other, is not limited to an \co{act}; it only finds
an \co{expression} in the \co{act}.

Eventually, life is a prayer\ldots

%\pa
%Thus, prayer may be as well \co{an experience}. [Jung?]

\pa
Why is it a \sch\ldots?

\citt{all that which we feel as painful is really {\em giving} -- something
that our fellows are better for, even though we cannot trace 
it.}{James Hilton, The Mystery of Pain}

%\titp{Trembling}

\titp{Acceptance}

\pa
Acceptance -- willing, not resigned

as long as it does not spring from any reasons

This is what bothers us with Stoics -- they tried to explain {\em why}
one should accept fate, laws of the universe.  They were seeking for
reasons.  (Again, James gives excellent phenomenological descriptions 
and examples.) But there is no \thi{why}, except what you find worth 
doing -- it is your \co{choice}. 

This has multitude of forms: compassion, forgiveness, benevolence, all 
acts which 

\titp{Repentance}

\pa
Reue und Neugeburt

\titp{Work}
Cf. \refpp{pa:hardwork} (in Spirit=no Spirit; Non-attachment)

\sep

\pa
All such \co{acts} may be performed before one is confronted with 
the \sch. But as all \thi{rationalists}, and not only, know, they are 
meaningless, or else, as some may attempt to explain, they have also 
some inherent value and should be performed for their own sake. 

But such explanations, reasonable as they may be, always miss the mark
because these acts are as justifiable by some \co{visible} reasons as
unjustifiable by others.  A man who believes that everything is
understandable and resists everything which does not conform to his
state of thinking (or we should rather say, believing) is at least as
poor a wretch as one who resists attempts to understand.  The goals
remain hidden until they are reached, and the words remain
incomprehensible until they are confronted with the experiences to
which they refer.  True, we do not read Dostoyevsky to four years old
children.  Acts of submission, in prayer, in \ldots, become fully
meaningful {\em only} in the face of \sch\ and experiences which may
confront one with it.  Until then, they can not possibly have the full
meaning to the one who performs them.  But they are the most
advantageous and the only possible preparations for the moment when
such a \co{choice} will confront one.  And thus, although we do not
read Dostoyevsky to our kids, we may teach them to pray (as, in fact,
some still do!).  If for nothing else, then just in case.  But, as a
matter of fact, also because they may thus gain an attitude which will
make the right \sch\ easier and less violent than it otherwise easily
may be.  Gifts are neither necessary nor sufficient, but they may be
helpful.

A tremendous experience of confrontation \citt{at first appears like 
a confused chaos, but then those parts are selected which bear the 
nearest resemblance to such particular steps as are insisted on; and 
these are dwelt upon in their thoughts, and spoken of from time to 
time, till they grow more and more conspicuous in their view, and 
other parts which are neglected grow more and more obscure. Thus what 
they have experienced is insensibly strained, so as to bring it to an 
exact conformity to the scheme already established in their 
minds.}{Jonathan Edwards, quoted by W. 
James, The Varieties of Religious Experience, IX, footnote 5, p.200}


Both wheat and chaff, both \yes\ and \No\ tendencies
\citt{grow together
until the harvest: and in the time of harvest I will say to the
reapers, Gather ye together first the tares, and bind them in bundles
to burn them: but gather the wheat into my barn.}{Mat.  XIII:30/Lk. 
III:17}



Making the confrontation less dangerous by establishing a system, if 
not of techniques, so in any case of concepts and signifactions 
describing its forms and stages  \citt{was the purpose of rite and 
dogma: they were dams and walls to keep back the dangers of the 
unconscious, the \thi{perils of the soul}.}{Jung, The Archetypes and 
the Collective Unconscious, I:1.47}

%end \input{031choiceActs}



%%%%%% \section{The attitude towards visible}
%\input{032non-attach}
\section{The attitude towards \co{visible}}
\secQQQ{7}{God being, as He is, inaccessible, does not rest in the consideration
of objects perceptible to the senses and comprehended by the understanding. This
is to be content with less than God}{St. John of the Cross}

% \secQQQ{7}{For God shall be all in all, and every creature shall be
%   overshadowed, that is, converted to God, as the stars when the sun
%   arises.}{JSE, `Peiphyseon' III 689A [98]}

\pa Living in time, we \co{thirst} for eterninty, flirting with
\co{visibles}, we \co{thirst} for Being.\ftnt{\citf{Two loves in this life fight
with each other in each temptation: love of time and love of God.}{St.Augustine,
Sermones CCCLIV:1 [Heid. Fenomenologia Zycia Religijnego, (notes of Oskar
Becker) p.271]}} \Sch\ is an \co{act}, but it founds a state, a permanent
home-place, a \wo{house on the rock}, not a \co{visible} dwelling place but
\la{axis mundi}.  To \co{participate} in `\ldots' is to be \co{founded} by
`\ldots'.  Every being {\em is} only because it is \co{distinguished} from the
\co{One} and this \co{founds} its \co{participation}.  However, as we said, the
higher levels do not \co{found} the lower ones, except for the ontological
process of \co{actualization} of Being. It is only \sch, as an \co{expression}
of \co{spiritual love}, which \co{founds concretely} \co{actuality} in its
\co{invisible origin}.


\noindent
\begin{tabular}{|r@{\dotfill}r@{\ \ \ --\ \ \ }l@{\dotfill}l|}
\hline
\multicolumn{4}{|c|}{Consequences}\\
\multicolumn{2}{|c}{\G\ \ } &\multicolumn{2}{c|}{\ \  \B} \\
\hline\hline
\multicolumn{2}{|r@{\ \ \ --\ \ \ }}{{\bf non-attachment}} & 
\multicolumn{2}{l|}{\bf attachement - detachment} \\ 
this world is below     & \ \ {\bf ``above the world''} & {\bf ``in the
world''}\ \     &  nothing above \\
I can, if I will \G     & {\bf freedom to}      & {\bf freedom from}
& limits in the ``world'' \\
preparedness & {\bf strength}   & {\bf weakness}        &
hardness, false security \\
        & alertness     & false security        & unexpected\\
the order of things & {respect} & {disrespect} & use-and-throw \\
wyrozumia{\l}o{\'s}{\'c}        & {\bf forgiveness}     & {\bf
hardness}       & absolutizing the finite \\
\multicolumn{2}{|r@{\ \ \ --\ }}{{\small not judging anth. beyond one's compreh}} 
& \multicolumn{2}{l|}{{\small calling good/evil - moralistic arrogance}} \\
\hline
\end{tabular}%\caption{Below to Above}

\pa
One has observed that traditional representations, whether in 
painting, sculpture or literature, of paradise are 
unbearably dull and monotonous as compared to the  fascinatingly 
complex and eventful representations of hell. 

Because \yes\ is merely a \co{reflection} of the \co{invisible}, it is 
impossible to create any typography of \yes{-attitudes}; every such an 
attitude is a thoroughly personal event. One can write biographies of 
saints and sages but, to the extent such a biography captures anything 
significant, it will have to relate \co{concrete} situations, 
particular \co{acts}, specific behaviours of this particular person. 
It can be stated in more general terms (as I am doing it now), but 
such terms are merely abstract  \co{signs} -- they do not convey the 
\co{concrete} specificity of \yes. 

The \No{-attitude}, on the contrary, involves one immediately into the
\co{complexity} of the \co{visible world}.  Hence, the kinds and
variations are not only innumerable but also easier to grasp and
express.  The great literature of the last couple of centuries is
dominated by the descriptions and analyses of problems, sufferings,
mischieves which, indeed, seem to offer much richer and more
interesting material than the possible examples of tranquility and
sainthood.  Not least, this tendency is strengthened by the appeals to
\thi{descriptive objectivity} and \thi{relevance for the readers} 
(which simply means, impersonal masses) -- analysing
the ways in which things may go wrong is, after all, but a honest
description, is just staying true to what most people do experience in
the world.

\pa
I won't focus so much on the aspects of the \No{-attitude} towards 
\co{visibles} because it might easily involve us into indefinite 
elaborations and classifications. Its characteristics will arise 
naturally as the opposites of the respective aspects of the 
\yes{-attitude}. The following remarks are rather brief because they 
comprise only what is related to our general exposition. Almost any 
text of mystics, Christian or Muslim, of Buddhists masters,
of Eckhart, St.~John~of~the~Cross, Jakob~Boehme discusses these issues 
in detail which it therefore isn't necessary to repeat here.

\sep

\pa
As said in \refpp{notfounding}, the higher levels
\co{found} ontologically the lower ones, only by lending the 
latter a particular \thi{aura}, the \co{rest} which, on a closer 
attention, turns out to be much more important than the specific 
contents of the lower levels themselves, but which nevertheless may 
and for the most does escape the \co{reflection}.

It is only the \sch\ and the associated \co{gift} of \co{spiritual
love} which establishes an unbroken an unbreakeable continuity of
Being across all the levels of \co{experience}.  It penetrates \co{my
life} and \co{the world} with its light or, as may be the case of \No,
casts an all embracing shadow on everything.  It \co{founds 
concretely}, that is, \co{founds} a particular
way of \co{experience}, in which the attitude towards the \co{origin}
permeates the attitude towards everything because, as a matter of
fact, the two become the same.

\pa 
%They talk about two natures of Christ and wonder how such a 
%contradiction might be possible. 
The \co{spiritual} unity of \co{soul} and \co{body}, of {higher} 
and {lower}, modifies but does not change man's existential situation. 
Man is a borderline between what is \co{below} and what is \co{above}
-- \co{visible} is just the other side of \co{invisble}.  Any
attitude towards the one is, at the same time, an attitude towards the
other.

Yet, except for the \sch, no direct, intensional attitude towards the
\co{origin} is needed, if at all possible.  Such an attitude is
already an indication of a mistake -- the \co{One} can not be made
correlate of our intensions or \co{acts}, unless it is reduced to some
\co{objective} from; \co{analogues} can be \co{experienced} but they
are not -- or in any case, should not be turned into -- independent
entities.  Intensional \co{acts} find place only in the sphere of
\co{visible} contents, \co{disitinctions} which are sharp enough to be
turned into \co{reflective objects}, \co{objects} of \co{action} and,
secondarily, into \co{objects} of \co{reflection}.  The \sch\ is a
\co{reflective act}, but the resulting attitude is not \co{expressed}
by mere words and declarations, no matter how deep and true; the
reality of this \co{choice}, the reality of the \co{incarnated love},
is witnessed by its \co{manifestations} in our \co{experiences} of and
our dealings with the \co{visible world}.


%It marks the ultimate division of wheat from chaff, bringing
%the ultimate \co{clarity} into one's being.
%, even if one's life may still remain ... confused?
%
%\subpa 
%Recognition of the the world as a \co{gift} requires the
%\co{non-attachment}, renounciation of \co{myself}, and thus it is not
%relative to \co{my} thinking, wishes, goals.  This \co{communion} is
%consumated \co{above}, it is \co{spiritual}, which means,
%unconditional.  But 
%


\subsection{\co{Non-attachment}}\label{sub:nonattach}
\secQQQ{10}{Put not with God other gods, or thou wilt sit despised and 
forsaken.\\
  Thy Lord has decreed that ye shall not serve other than Him}{Kor. 
  XVII/22-23}

\say
The moods of silence are never satisfied by anything \co{visible}. 
The \co{thirst} is not a thirst for anything \co{visible} but for 
\co{concrete presence}. This is not to say that to quench it, one has 
to deny all the \co{visible world} -- only that \co{this world} does 
not \thi{fill the soul}, is not enough, since it contains all and only 
answers, it will never give {\em the} answer \ldots

\pa
Mystics and sages have always spoken about self-denial and denial
of \co{this world}.  This, however, is only a possible part of
\co{spiritual} exercises leading to \yes\ and, as the means, should
not be confused with the goal.  In a sense, \co{spiritual love} --
living and lived \co{spiritual love}, the 
union of which mystics speak -- 
\co{inspired} by the ultimate reality and importance of the \co{One},
is exactly such a denial. 

However, they at the same time speak always about the need for
constant alertness, presence of mind, active attention to the
\co{actual} situation.  This may seem to contradict the supposed peace
of the union with God based on absolute self-denial.  There is,
however, no contradiction because, as a matter of fact, the
\co{spiritual love} is but the \thi{second birth}, is \thi{re-birth}
not only of soul but of flesh, is resurrection of the body, that is,
of \co{this world} as much as the \co{other one}.  The difference is
that before, \co{this world} was only ontologically grounded in the
\co{other world} and thus there was not a real, that is,
\co{spiritual} unity of the two.  Resurrection is the \co{spiritual}
event which brings the two worlds together, which makes \co{visible}
not only a mere \co{actualization}, but a true \co{manifestation} of
the \co{invisible}, \wo{on earth, as in heaven}. 
\citt{Die st\"{a}rkeste und tiefste Wirklichkeit ist, wo alles ins 
Wirken eingeht, der ganze Mensch ohne R\"{u}ckhalt und der 
allumfassende Gott, das geeinte Ich und das schrankenlose Du.}{Buber, 
Ich und Du, III, p.90}

The attitude towards \co{visible} which reflects the \co{clear} 
univocal \co{love}, is equally \co{clear} and univocal
\co{non-attachement}. This name, however, requires some closer 
remarks. 

\pa The \thi{death to this world} means that \co{visibility} loses its
absolute importance, that it is seen now in \co{non-attachement}
\la{sub specie aeternitas}, with, as St.  Francois de Sales called it,
\thi{holy indifference}.  \co{I} remain \co{myself} as \co{I} have
always been, but this \co{miness} is no longer the axis of the world. 
Indeed, it is now seen and \co{experienced} only as an accident of the
\co{origin}, as only the \co{actual}, only one of its possible
\co{gifts}.  \citt{[A] man should so stand free, being quit of
himself, that is, of his I, and Me, and Self, and Mine, and the like,
that in all things, he should no more seek or regard himself, than if
he did not exist, and should take as little account of himself as if
he were not, and another had done all his works.  Likewise he should
count all the creatures for nothing.}{Theologia Germanica, XV}

This, indeed, is the
\co{reflective} attitude conditioning \co{spiritual love}.  But all
this \wo{counting for nothing} expresses only \co{non-attachment} to
the \co{visible} things, the consciousness that, in their \co{actual}
existence, they should not make \co{me} absolutely, that is 
unconditionally, dependent on them.  It is not
denial of their existence, nor of their possible relevance; it is only
denial of their value as \co{absolutes}.  \co{I} can \co{act} on them,
but \co{my life} is not exhausted by such \co{actions}, \co{I} can try
to attain them, but \co{I} do not crave them, \co{I} can enjoy them,
but \co{I} do not worship them.  And if \co{I} fail, if \co{I} do not
attain them, if \co{I} do not enjoy them, then \ldots it does not
matter.  
\co{My life} is never exhausted by them, it always carries the 
\co{rest}, the inexhaustible potential. This \co{rest} contains 
\co{thankfulness} even for \co{my} failures. 
For all these actions, attainments and enjoyments where
themselves only \co{visible} things of only relative value.  \co{I}
can try again or \co{I} can let it go -- \co{I} do not know what
\co{I} will do, this will turn out in the proper time and \co{I} may
be vastly surpirsed.  \co{I} can not have anything which \co{I} am not
prepared to lose.  \co{I} can not have anything which \co{I} have not
already lost. 
As Ibsen says, \wo{One possesses eternally only what one has lost.} 
For if I have not lost \co{this} whole \co{world},
\co{I} will never truly and deeply enjoy any single thing.  The \yes\
directed to \co{nothingness}, to Godhead, the apparent denial of and
\co{non-attachment} to \co{this world}, turn out as the \yes\ to all the
\co{visible} things.
%This is the ressurection of the flesh.

\pa 
\citt{Do not strive to seek after the true, only cease to cherish
opinions.}{A Zen master} \co{Idols} are not \co{visible} things, but
\co{visible} things considered as all important which, eventually means,
raised to the level of absolutes.\footnote{We do not have to go as far 
as the iconoclasts of IX-th century did in considering any representations as 
idolatry. The question, as always, is about the attitude towards 
things and enjoyment of artistic expressions, of Rublov or Dante, is 
something very different from \co{idolatry}.}  \thi{Worshipping images}, or
\co{idolatry} is exactly that -- to take as \co{absolutely} important something
that is not. 
\thi{Cherishing opinions} may be so much, 
and may be nourished by so many mechanisms. (\thi{Being entitled}, 
often \thi{entitled to 
one's own opinion}, and even \thi{entitled to be heard} are quite 
common forms.) At the bottom it is to
make an \co{idol} of \co{miness}, is to think that something
\co{visible} is worth cherishing an opinion about, and that \co{I} am
entitled to cherishing such an opinion.  Cherishing an opinion, \co{I}
cherish \co{myself}.

Again, all this does not mean that \co{I} can not mean anything about
anything.  I not only can -- I am bound to.  I will have opinions about
things, \co{I} will participate in arrangements of things, in research,
in work, in all kinds of activities of \co{this world}. Moreover, I 
will accept all these things as \co{my} part, as relative, yet
\co{absolutely} real, though not as \co{absolute} reality.  In the
moment \co{I} start cherishing them, \co{I} become
\co{attached}, that is, \co{I} start worshipping \co{idols}.

\pa \co{Idols} are what \thi{possess} man, \thi{being possessed}
consisting precisely in making the relative \thi{absolute}.  This is
by its very nature unconscious which justifies the phrase \wo{being
possessed}.  Even if I have and clearly see all the good reasons for
adhering unreservedly to a given opinion, my being possessed by
it consists in the {\em unreservedness}, in the perhaps unintended,
but therefore the stronger and more effective, turning it into an
\thi{absolute}.

Rationalism, defined as acceptance of \co{actual} statement or
position with the recognition of its limited validity (and in the best
case, also of its actual limits), is the opposite of being
possessed.  In this respect, it coincides with innocence which is 
just that -- being pure, that is, not being possessed. 

An \thi{-ism} is not a variant of a genuinely religious attitude but,
on the contrary, indicates being possessed, of \thi{absolutizing} some
relative sphere or expression.  One's intense materialism or idealism,
atheism or theism, liberalism or dogmatism, Protestantism or
Catholicism, intellectualism, existentialism or what not, testifies
against one's innocence.  In fact, one can become possessed even by
rationalism itself, which may take form of agnosticism, relativism,
scepticism or just dry rigidity.

%This does not mean that the only thing to do is 
Giving up \co{idols} does not mean merely to replace the \thi{object}
of such worship, to exchange the relative for the absolute,but still
retaining one's attitude.  {\em What} is being worshipped determines
the crucial aspects of the attitude.  Worship of patriotism {\em is}
different from worship of communism, worhsip of scientism {\em is} 
different from worship of money.  Yet, they are the same in so far
as \co{idolatry} is concern.  To cease worhipping \co{idols} is to
recognize the unique character of the \co{absolute}, of the \co{One}
which is unconditionally \co{above} the world of \co{distinctions}.

%%%%%%%%%
%\input{noSpirit}
\subsubsection{Spirit = no Spirit}

\pa
\co{Idolatory} is but an expression of the \co{spiritual} (which here
means the same as ontological) \co{thirst},
the \co{thirst} for the \co{absolute}. It is yet another example of
\thi{mixing the levels}, of lifiting \co{visible} categories -- in this case,
things, objects and ideas -- to the level of \co{absolute}. 
%
But \co{spirituality} is first of all \co{humility}
towards the \co{spiritual}, towards the \co{invisible origin}. 
%used later\citt{Know what is in front of your face, and what is hidden from you
%will be disclosed to you.  For there is nothing hidden that will not
%be revealed.}{The Gospel of Thomas, 5}
By this very token, the
\co{spiritual} attitude directs its attention exclusively
towards \co{this world}, because \co{this world} is no longer 
separated from \co{another}.  
If we view the ontological \co{founding} (Book I) as the descent of
which mystics and some philosophers spoke, while its lived and understood
reflection in the
levels of Being (Book II), as the corresponding ascent, then the 
\co{incarnation} of \co{love} marks the final and definite 
return.\footnote{This seems to be the natural way to interpret much 
of the mystical ascent though, of course, there are other possibilities. 
Hermes Trismegistus is probably one of the cleares examples 
emphasizing the element of return, \wo{as above so below}.} 
%
It does not end in \la{amor Dei intellectualis}, in ecstatic
contemplation, in constant wish for repetition of the \co{experiences}
of mystical union. It does not live in \co{another world}, but it 
does not have to descent into \co{this world} either -- there is only 
one world, and \co{actuality} becomes the scene of constant, 
\co{concrete presence}.
%%% used later  \citt{there is nothing better, than that a
%%man should rejoice in his own works; for that is his portion.}{Eccl. 
%% III:22 }

\pa
\co{Spirit}, we said, is the unity (not merely a totality, and not even
a borderline between) the \co{visible} and \co{invisible}. In what
does this unity consists, \co{concretely}? \co{Concretely}, in
\co{nothing} particular. \co{Spirit} is also 
directedness towards \co{nothingness}. But how can
that be? Simply, as directedness towards what is \co{visible}, for
\citt{there is nothing better, than that
      a man should rejoice in his own works; for that is his
      portion: for who shall bring him to see what shall be after
      him?}{Eccl. III:22 [We won't confuse these remarks with stoicism. 
\citf{We must make the best
use that we can of the things which are in our power, and use the
rest according to their nature. What is their nature then? As God
may please.}{Epictetus, The Discourses, I.1} This says the same as 
what has been said above, but the stoical endurance is a matter of 
resignation and surrender to the world which overgoes one's powers, 
not of \co{thankfulness} for its \co{gift}.]}    
Or, we might add, to {\em see} what is \co{above} him?

\pa The \co{non-attachment} means that the \co{love} is truly devoted
to the \co{origin} but also, and by the same token, to the \co{visible world}. 
I \co{love} things of \co{this world}, unconditionally, 
because they are \co{manifestations} of their \co{origin}.
The 
\co{origin}, however, is not distinct and separate -- my 
\co{spiritual love} and attitude towards it is nothing else except my 
attitutde towards the \co{visible world}. \co{This world} is not {\em only}
a \co{sign} of the \co{other world} -- it is {\em the only} \co{sign}, the
only form of \co{invisible presence}.

Such a \co{presence}, like verything \co{invisible}, is indeed
expressed through the \co{actual acts} and situations, but it is
entirely different from \co{actualization} of anything spiritual.
\co{Non-attachment}, founded on the \co{invisible incarnation} of
\co{love} and the \co{reflective} \sch, comprises the \co{qualities}
which are, so to say, on the edge of the sphere of \co{visibility}.
They reside in the \co{rest} of \co{acts}, as possible
\co{motivations}, \co{vague} hopes, \co{inspirations}...  But as
everything else, they may be \co{posited} as correlates of \co{my}
intensions, goals of \co{my acts}, turned into kind of \co{objects}.
Such a reduction amounts imply to losing them.


\tsep{restlessness}

\pa An \co{act} aiming at, consciously intending \thi{goodness} is not
good.  It need not be evil, nor wicked, nor malicious; it may be
thoroughly well-intendend but it is not good. The intension of
\thi{being good} pollutes every \co{act} unless it is withdrawn from
the sphere of \co{actuality}, unless it is only an \co{invisible
  rest}. But this means exactly that it is not any intension, that it
does not in any way enter into \co{my} considerations.  \thi{Being
  good} emerges, as it were, only as a side-effect of \co{acts} which
themselves are occupied only with their \co{actual} \co{object} and
\co{visible} relations.

This applies to all higher things which one might possibly \co{posit} as
one's intensions, even goals.
 An \co{act}  
whose main goal is {\em to be} compassionate, is not 
compassionate, just like an \co{act} by which \co{I} try to prove and 
show my freedom is not free. 
%\citt{One day man will go mad to show that he is free}{Dostoyevsky}
A person focused on making always \thi{right}
decisions may, indeed, happen to make them \thi{right}. But he will spend time
in constant worry about doing just that. And since \thi{right} is entirely
\co{vague} category, one never rests. A person focused on his salvation may
happen to do a lot of good things, but his focus on such a thing will always
bother him: \wo{Has it already happened or not yet?} A person focused on
morality may, indeed, happen to perform a lot of good \co{acts} and \co{actions},
but his focus will always ask in the background: \wo{Is it morally perfect?},
and if not, then \wo{Is it moral enough?}

\pa There is of course a huge difference between striving for
perfection, or spiritual realization and for mere possession, fame,
money. But in so far as it is striving after completing some lack, it
witnesses to and \co{manifests} restlessness which makes this lack
more \co{clear}; Perfection with respect to \co{visible} things and
situations is, perhaps, just to \co{act} as good as possible. But how
much is possible, {\em what} is possible?  And so, suspecting ever
other possibilities, we keep striving after some \co{rest} which we do
not know what is, but all the time feel {\em that} it is.
\co{Thirsting} for eternity, we flirt with time. 

\tsep{Forget}

\pa \co{Spirit}'s directedness towards \co{nothing} means that it
\co{rests} with the mere \thi{that}.\footnote{You may notice that the
verb \wo{\co{rest}} here (and \wo{\co{rest}} as \thi{tranquility}) is,
obviously, something different from the noun \wo{\co{rest}} as
\thi{reminder }which we have been using earlier. The homonymity,
however, serves us perfectly because the equivocation is thoroughly
intensional. To \co{rest} is to admit, to allow for, to accept the
\co{rest}.} As there is nothing more to do about \co{nothingness} than
saying {\em that} it is, \co{spirit} is the \co{restful} return to
\co{this world}. To \co{rest} is to accept the
\co{invisible rest} -- to give up all attempts at
making it \co{visible} -- and in this sense, to \co{forget} it. 

\thesis{\co{Spirit} is \co{forgetfulness} of the spiritual.}\label{pa:forget}

\co{Spirit}, as contentless relation to \co{nothingness} is purity and
poorness. It does not aim at the spiritual, does not seek it, does not
worry about it. This is the only way of its \co{concrete presence}.
\citt{Blessed are the poor in spirit}{Mt. V:3; Lk. VI:20}. 
Walking the spiritual paths may be an expression of a genuine
spiritual \co{thirst} -- but this only means, the absence of
\co{spirit}.  \citt{Hold up my goings in thy paths, that my footsteps
slip not.}{Psalms. XVII:5} Any strife, any search, whether spiritual
or not, is but a lack. The more spiritual such a search is, the more
it circles around the \co{vague} and indefinable \thi{that} and the
less it is satisfied with any \thi{what}.  But spiritual search is not
\co{spirit}. \co{Spirit} either is, either is thoroughly,
\co{concretely} and \co{absolutely present}, or it is not at all, is
only some unidentifiable and ever missing \co{rest}, which one may
desperately try to replace by some \co{visible} surrogates or, as the case
may be, by the insatiability of \co{more}.


\tsep{... but remember -- paradox}

\pa \co{Spirit} transforms \co{actuality} suspending its absolutized
validity; it \co{incarnates} when one has ceased the attempts to
\co{actualize} everything. 
But by its very nakedness, \co{nothingness}, \co{spirit} grants \co{actuality}
all the validity it possesses as the only place of our \co{acts} and works. 

As soon as something more \co{precise} gets involved, a distinct
thought, a specific feeling, the \co{spirit} seems to evaporate, to
lose \co{actuality} giving 
place to the \co{flesh} -- perhaps, to \co{myself}, perhaps, to
\co{my} \co{Ego} and \co{body}.
But this withdrawal is its only true, \co{concrete presence}.
Being \co{invisible}, it can never become \co{actual}, but it can be
\co{present} around and \co{above actuality} which, in fact, means in
the very midst of it.
If you look to the left, you won't find
it, if you look to the right, you won't find it, if you look forward
or backward, in past or in future, you won't find it.  Because, when
you look for it, you have already found it, you only have to stop
looking.  \citt{[The kingdom] will not come by watching for it.  It
will not be said, 'Look, here!'  or 'Look, there!'  Rather, the
Father's kingdom is spread out upon the earth, and people don't see
it.}{The Gospel of Thomas, 113}
But \thi{to stop looking for it} is as difficult as it sounds easy.


\pa There is a great difference, which may appear as a paradox,
between \co{forgetfulness} and forgetfulness, or perhaps, between
\co{forgetfulness} and denial. \co{Forgetfulness} of the spiritual is,
as a matter of fact, the deepest remembrance of
\co{nothingness}. Remembrance, however, not in the form of constant,
\co{actual} remembering, of incessant focus on the desired, even if
impossible, \co{actuality} of the spiritual. It is remebrance which,
for the first, remembers only \co{nothing}, only \thi{that} it is, but
does not worry constantly about \thi{what} it is; and, for the second,
remembrance which itself is not \co{actual} but thoroughly
\co{invisible}, which does not enter the sphere of \co{actual}
considerations and intensions and does not try to bring the \co{invisible rest}
into explicit \co{actuality} of \herenow. 
It is forgetfulness as far as 
the \co{actual} occupations of the \co{subject}, even of \co{myself}, are
concerned, for these deal only with \co{visible} things. But as far as
\co{my} being is concerned, it is the remembrance which \co{I} have become,
the \co{Self} which is no longer overshadowed by \co{my Self}, not to mention,
by \co{myself}. 

\pa To have some particular thing is to have already lost it, to agree
that it is not \co{mine}, that \co{I} do not control it. Only then can
\co{I} truly have it.\footnote{We have here obviously abandoned
Marcel's distinction between \thi{being} and \thi{having}.}  Having
already lost it is simply to admit its fragility, which only makes my
appreciation greater.  Expectation of its possible loss may certainly
cause \co{me} some worry.  If, and when, \co{I} \co{actually} lose it,
this may certainly cause sorrow and pain.  \co{Spirit} does not
abolish all such negative moods, thoughts, feelings. On the contrary,
it actually \co{opens} \co{me} for their thorough and deep
experience. This happens because such worries and sorrows, really and
deeply felt as they may be, are only relative, concern the things
which they concern, which are not ultimate and so do not affect the
tranquil unity of \co{spirit}.

We are not saying that spiritual unity is a
kind of an alternative, a tranquilizer, a placebo used against finite
failures and \co{actual} dissatisfactions. \co{Spirit}, directed
towards \co{nothingness}, is fully directed towards such finite and
\co{visible} things and events, it does not supersede them. It only
makes me worry for the things of \co{this world} without worrying
about the ultimate things, without absolutizing the \co{visible}, that
is, without turning it into \co{idols}.

It makes \co{me} care for all finite things because, having
\co{founded} \co{my} being in the only \co{absolute} of
\co{nothingness}, it allows \co{me} to recognize their fundamental
fragility. A thing which \co{I} could not {\em possibly} lose (if such a
thing existed) would be, or in any case would turn with time
... worthless. An eternal life, imagined vulgarly as merely temporally
infinite, would be, if not unbearable, then eventually boring. And
boredom would not come from the fact that there were no new things to
encounter. It would come exactly from the fact that there would be
nothing else to encounter than mere novelties.
Death is the complete return to \co{indistinctness}. And it is the
knowledge of this ultimate \co{nothingness}, of the fragility of all
\co{visiblility}, which makes life so valuable. However, life occupied
{\em exclusively} with the maintainace of itself, forgetting \thi{that},
i.e., that there is something more worthy than it, perhaps even something
for which it could be sacrificed, becomes a mere social, even a mere
biological phenomenon -- deindividualized, impersonal, eventually,
meaningless. Although it is impossible to live fully such an idea, it is
nevertheless possible to accept and believe it. 

\pa
Eventually, there are only \co{visible} things of \co{this world}
which are given to us, so that we \citt{have dominion over the fish of
the sea, and over the fowl of the air, and over the cattle, and over
all the earth, and over every creeping thing that creepeth upon the
earth.}{Gen. I:26} But reducing Being, and what then follows, our
being, to such things only, \co{dissociating} them from the
\co{nothingness} of their \co{origin} (as is typically the case in
the attempts to see, \co{re-cognize} and admit the value which they do
possess), turns them into
dead and empty \co{objects}. It makes us forget so that we do not remember.
\co{Forgetfulness}, too, directs \co{me} towards \co{visible} things but not
as the only and \co{absolute} form of Being. \co{Forgetfulness} makes
\co{me} remember that they are only \co{signs}, but also that they are
{\em the only}, true and real, \co{signs} of the \co{invisible}. 
\citt{Know what is in front of your face, and what is hidden from
you will be disclosed to you.}{The Gospel of Thomas, 5}

%\pa - perhaps, this should go elsewhere [love?]

%Trust, not faith, and not unfounded trust. Trust is unfounded when
% I choose to trust -- the very decision, the very consciousness of
% having decided makes me know the possibility of betrayal. This is,
% perhaps, the only form of trust we know from \co{this visible world}.
% Trust, which is comittment and unconditional reliance, is possible
% only in the face of the \co{absolute}.\footnote{Again, let us remind that
% the Greek \gre{pistis}, translated usually as \thi{faith}, is often
% better rendered as \thi{trustfulness}...}

\tsep{back to non-attachment}
\pa\label{pa:hardwork}
And so, \co{non-attachment} is completely \co{concrete presence} in the
midst of \co{this world}. 
But it is not the goal to make me so concerned with \co{this world} -- it is
only the effect. To achieve it, one has to renounce it. Paradoxically as this
may sound, there is not a slightest dose of paradox in
it.\footnote{\citf{For whoever wants to save his life will lose it, but whoever
               loses his life for me will save it.}{Lk. IX:24, XVII:33,
Mt. X:39, Jh. XII:25}}

We can care for things because we value them to the degree which we do
not even realize, we may do good works because our boss, our spouse,
other people expect that or because our hidden inhibitions force us to
do so. All this has nothing to do with the \co{spirit}, even if the
\co{externally visible} results may be exactly the same. For
\co{external} results, as we know, do not count for us as much, and do
not give us as deep a satisfaction, as we often would like to
believe. They may be necessary but are never sufficient.  But we can
also care for things and do good works because \co{spirit} does not
distinguish that from the ultimate \co{humility} -- in fact, because
it does not leave us anything else to do.  Care for finite things,
work carried out with conscientousness, respect and \co{humility}, do
keep heaven and earth together. Work -- hard, tiring, exhaustive work
-- which has engaged fully body and mind, makes \co{me
forget}. \co{Forgetfulness} finds the \co{expression} as respect for
\thi{the order of things}. At the same time, it is the deepest form of
remebrance, \co{forgetfulness} of \co{spirit}.  Sloth is a cardinal
sin because there is no such thing as \thi{pure spirit}, disembodied,
non-incarnated. There is only living, \co{concrete spirit}, which
unfolds itself in the body, in \co{this world}.

Dedication and thoroughness, hard work and conscientousness are not,
in any case, not necessarily signs of \co{attachment}. More often than not,
they are signs of \co{spirit}. And as all \co{acts} which are \co{expressions}
of \co{spirit}, they also strengthen it or, as is often the case, prepare for
it.  \citt{Commit thy works unto
the Lord, and thy thoughts shall be established.}{Prov. XVI:3}


\pa To be sure, \co{concrete presence} of \co{spirit} is nothing
common. Perhaps, it is even extremely rare, though it seems to me that
it is less rare than we want to admit or are able to realize.  But the
fact that no statistical investigation may ever give a slightest
indication of it means only that it is the most real, that is, the
most individual and personal possibility -- unrepeatable, not because
of varying \co{visible} conditions but because thoroughly
\co{concrete}; and always the same, because consumated in the same
existential situation, in the face of \co{One nothingness}. Statistics
can never say anything about anything individual.

\noo{
--------

\citt{Thou canst not see my face: for there shall no
        man see me, and live.}{Ex. XXXIII:20}

\citt{blessed are those who have not seen and yet have
               believed}{Jh. XX:29}
--------
}
%%%%%%%%%
\ad{Inversions}\label{pa:inversion}
%Let's give a few more concrete examples of the \co{aspects} of
%\co{non-attachment}.
\label{pa:allaspects} \co{Love} has unlimited number and forms of
\co{incarnation} which are always purely personal.  Every \co{concrete
  manifestation} of \co{love}, although it may seem to express only one or few
of its \co{aspects}, is always a full expression of all of them.  Say, modesty
may seem a natural example of \co{humility}, but it involves equally
\co{thankfulness} and \co{openness}. For modesty, that is, full modesty
expressing \co{love}, is not a servile admission of one's inferiority. It is
a humble greatefulness which does not argue (whether with another person
or with the world) about qualities and conditions of the \co{gift} -- one's own
achievements and labor being, too, \citt{nothing more than the finding and
  collecting of God's gifts}{M.Luther, Luther's Works 45:p.327 [Concordia
  Publishing House and Fortress Press, 1955-], [Theol.of M.L., p.109]}.
%being \co{gifts} like anything else.
And it is greateful for
whatever one has obtained, for a person which is now modest and now not, is
simply not modest but only behaves modestly in some
situations.

\pa Now, all the \co{aspects} of \co{love} are predicated
adequately about the \co{spiritual} attitude, and only analogically about
anything within the \co{visible world} Together with the \co{unity} of all
\co{aspects} in every expression of \co{spiritual love}, this may easily give
raise to apparent \co{inversions}. Roughly, \co{inversion} is a \co{manifestation}
through something which appears as the opposite of the manifested. It is only
appearance, usually a misreading of the situation, which \co{inverts} its
meaning. 

\co{Spirit} is \co{forgetfulness} of the spiritual,
\refp{pa:forget}.  The apparent lack of the \co{spirit} may be its true
\co{manifestation}, it may become \co{present} through an \co{inverted} form.
On the other hand, the total absence of \co{spirit} is, too, its total
forgetfulness. On the surface the two extremes may be indistinguishable, for
what separates \co{forgetfulness} from forgetfulness, \co{presence} from
absence, is an \co{invisibly} thin line.

\subpa Modesty is to do everything one can. Althoug this is all, it may need an
explanation, so let us add: and knowing that one can not do more.  My own
achievements are also \co{gifts}, only ones which I can influence. Waiting
resigned for a miraculous gift from heaven has nothing to do with modesty;
perhaps with laziness or sloth. (So, an achievement may be an \co{inverted} form
of a \co{gift}.)  Modesty works with full dedication, it employs all abilities
and potential for achievement of its goals. Only having done everything, that
is, only meeting one's limits one becomes modest. And when one has done
everything one could, one knows it -- for knowing that one can not do more {\em
  is the same as} having done everything one could. The addition of \thi{knowing
  that...}  does not add anything; it only seduces [deludes] us to think of
\thi{knowing} merely as explicit, \co{actual} and fully \co{reflective} knowing.
(It even seduces us to think that what we said may be self-satisfied and detached
\wo{I am done with it ('cos I can't do anything else).} Modesty is never done
with anything, for it knows that no matter what it has done, more could be done,
only that it can not do that.)

Now, a person trying actively to accomplish some task may spend a lot of time
and effort in this direction.  He may become a highly skilled expert with very
high professional standards.  From outside, and seen only in abstract terms, it
may easily look like he is only craving for reputation, recognition or just for
professional achievements of which he could be proud. Although often this may be
the case, it certainly does not have to be.\ftnt{I believe that Leonardo da
  Vinci, Albert Einstein and many other great scientists and artists might
  provide examples but the issue would require more personal aquaintance than I
  could claim.} Modesty depends on one's capacities and standards one applies to
{\em oneself} -- if these are exceptionally high, others will rather see
ambition and pride.  But this does not mean anything.  The person may -- though
only may -- in fact, be full of \co{openness} and modesty, and what is judged as
craving and striving may be but dedication, energy and ... true \co{humility}.

\subpa
The \co{non-attachment} brings me truly \thi{above the world}, not however, in
any specific sense of despising \co{this world}, of contempt for human
weaknesses, etc., but only in the sense of not accepting anything \co{visible}
as \co{absolute}.  If one wants to maintain that \co{this world} is everything
that is and there is nothing \co{above} it, such an attitude may easily be
labeled as \wo{detachement} and \wo{pride}, as being above all the matters which
concern us, normal humans. It has been attempted to construe \co{love} and
\co{humility} bordering on holiness as simple egoism, exclusive preoccupation
with one's own self, one's own salvation. The very fact that such a blindness is
at all possible, illustrates well the meaning of \co{inversion}. 

In short, what appears as arrogance may, in fact, be \co{thankfulness}; what
appears as preocuppation with one's little world may, in fact, be \co{openness};
what appears as pride may, in fact, be \co{humility} -- and one does wisely
suspending one's judgment about others in such matters.

\say
We are not after any Lutheran stuff here. God may, indeed, hide in the opposites
of what one considers His attributes or forms of presence, but He may equally be
present in all that is not such opposites. If He is everywhere, there is no need
to insist that He is only in what seems to contradict His nature. \co{Inversion}
is a frequent possibility but still only a possibility. 

Keeping the phenomenon of \co{inversion} in mind, let us review a few more
concrete aspects of \co{non-attachment}. 

\ad{Freedom}\label{pa:invfreedom}
We have said in \refp{pride} that
freedom is an aspect of \co{pride}.  This certainly needs some
qualifications.  The \yes, suspending the presumed \co{absoluteness}
of co{this world} and anchoring \co{my} being \co{above} it, makes me
completely free in relation to it.  This freedom is precisely the
content of \co{non-attachment}, of the ability to not worshipping
\co{idols}.  Of course, I am dependent on various things, I have to
eat and sleep, etc.  I am involved into causal realtions of \co{this
  world} but the point is that \ldots it does not matter, it in no way
contradicts my freedom.

The idea of freedom as independence from such relations is a
ridiculous invention of the \co{attachement} to \co{this world} which
tries to \co{detach} itself from it, which tries to liberate itself by
rising above \co{this world} and \ldots still stays in \co{this
world}, because above it, it finds only emptiness.  Causality (which
is much overemphasized notion) and, more significantly, the physical
existence, the body, the physiology, in short, the whole sphere of
most \co{actualized} contents, does not in the slightest oppose the
freedom. But \co{attachment}, reducing all that is to what is \co{visible},
 can not but oppose the two, since it is bound to look for everything
 exclusively within the sphere of \co{visibility}.   All
the lower concepts of freedom, as indicated in \refsp{sub:freedom},
are expressions of such an \co{attachment} in that they all carry with
them the idea of a liberating \thi{freedom from ...}

\pa \co{Non-attachment} is freedom from \co{attachment} but it is not
a freedom {\em from} \co{this world}. On the contrary, it is precisely
the freedom to live and act {\em in} \co{this world}.  Because \co{I}
no longer value any of the \co{visible} things as \co{absolutes},
\co{I} can easily accept them, whether they are one way or another.
\co{I} can accept them and respect them.  In the most general sense,
they are the way they are not because \co{I} or somebody else have
made them so, but because this is the way they are.  Except for the
narrow horison of things close-at-hand which \co{I} can control, the
most of natural as well as human creatures have to be left in peace
the way they are.  To try to bring them in conformance with \co{my}
wishes and whims, ok, with \co{my} true, meaningful and deep goals, is
not only a futile \co{attachment} -- it is an idea which may only lead
to disappointment, even despair.

Freedom makes me accept things, which is very different from making 
me surrender to them. acceptance means here the same as respect.
\co{I} do not need explanations and \co{I}
do not need reasons or arguments which is precisely to say: \co{I}
respect them.  They run their course, they may have their logic and it
may be highly rewarding to study their ways.  My freedom is the
freedom to do this.  Arranging them to my wishes and likings are but
petty consequences which may be useful but which have nothing to do
with my freedom.  Freedom, true freedom, is freedom to respect
\thi{the order of things} -- \citt{no Thing is contrary to God; no
creature nor creature's work, nor anything that we can name or think
of is contrary to God or displeasing to Him, but only disobedience and
the disobedient man.}{Theologia Germanica, XVI} 

\pa Any attempt to escape from this order, any attempt to liberate
\co{oneself} from it is already an expression of \co{attachement}, of
an involvement which makes \co{this world} the only reference frame,
of the underlying feeling of enslavement which sees its only
alternative in negation, in \co{detachement}.\footnote{Unfortunately,
  I must classify here Shestov's powerful opposition of Jerusalem to
  the reason of Athen's.  It isn't, of course, so simple, and his
  texts do contain more than freedom as a mere negation of laws of
  nature.  Freedom is also freedom to \co{hope}, although such a
  freedom may still be involved into negation of \co{this world}.
}  \thi{Use-and-throw} attitudes, \thi{things are for \co{me} and
  \co{I} do what \co{I} want with them}, all forms of disrespectfull
arrogance are expressions of un-freedom.  Also, an \co{inverted}
attitude, the toical resigned \thi{acceptance of the world}, the
realization that \co{I} can not oppose \thi{the whole world} and that
therefore it is wiser not to fight against it but humbly accept
whatever it brings \co{me}, is an expression not of freedom and wisdom
but of defeat and surrender. It may look like respect but, typically,
it will be a mere servility, a mere observance of all \co{precise}
rules, regulations, customs.  Although there is nothing wrong with all
that in itself, it often carries at its bottom the feeling of
unfreedom when it is a mere \co{act} within \co{this world}, a mere
defeat in the face of \co{visibility}, that is, when it is not
\co{founded} on the freedom flowing from the \co{invisible world}.

\pa
Eventually, freedom is the freedom to follow the \co{command} and 
\co{inspiration}, it is freedom to recognize them and to find the 
ways of \co{expressing} them. In their complete contentlessness, they 
make me realize fully my freedom. On the other opposite, we have the 
ultimate freedom from \ldots which faces the emptiness. 
There is no such thing as \thi{the problem of freedom in itself}. 
Freedom is unbreakeably bound with the sense of meaningfulness and 
the lack of meaning is also the lack of freedom, just like the lack 
of respect is.


If you like, you can see here yet another example of \co{inversion}:
the \yes\ to Godhead, turns out as the deepest \yes\ to \co{this
world}, and the \co{humility} and submission to the higher
\co{commands} is the fullest form of freedom.

\pa%freewill
Thus freedom is not a fact, it is not given, it is not something everybody 
has or, as the case may be, does not have.
Everybody can achieve it, but this is a fundamental difference. Much 
of the discussions about freedom were carried on under the spell of 
objectivity: \wo{{\em Are} we, or {\em are} we not free?}, \wo{{\em Is} 
our will free or {\em is} it not?} It is no wonder that with such 
questions, one was led to the level of what was considered the 
\thi{real}, the physcial world, causality, and natural laws -- these 
gave at least a context for speaking about that which {\em is}, and 
from which one might be free.

\subpa
We have, perhaps, liberated ourselves from this mode of speaking, but
still, do we not hear, occasionally, one or another talking about the
\thi{problem of free-will}?  It can, indeed, be made into a tremendous
problem.  Here \co{I} am, in \co{this world}, determined by the laws
of nature, and yet \co{I} can choose to do this or that, \co{I} do
have a definite feeling that \co{I} have free will.  But do \co{I}? 
How is it possible in the world, of which \co{I} am a part, which is
just a clockwork.  Ok, we do not believe in this clockwork any more
but we haven't got anything else to believe in which would replace it. 
So, perhaps, the question gets a more definite turn: \wo{The world
does not matter, but {\em am} \co{I} free, do \co{I} have free will?}
The degeneration of human being implied by such questions makes me
embarrased to comment on them at all.  What do I care about freedom or
not of my will?  
%? as long as it is not dtermined by any visible factors/others?
Do I really want to reduce my freedom to deciding
whether I will have an ice cream or a chocolate?  Is this freedom
we want to speak about? Is this freedom?

\subpa Will is the main aspect of \co{myself} and, as \co{I} am \co{my
  world}, \co{I} am involved into the interplay of all \co{visible
  distinctions}, and all possible relations between them.  Replacing
natural, necessary laws by some stochastic processes, by indeterminate
laws of social interactions, by Heisenberg's principle, does not
change a slightest thing, no matter how much one would like to believe
that indeterminacy of the world is more pleasing, than its necessity
would be, to my freedom.  Meaninglessness is an aspect of unfreedom
and increasing the indeterminacy of things does rather the opposite
than what one would like to believe.  Do you, reading Camus or Sartre,
get the feeling of freedom, of true freedom?  But the will seems to be
perfectly free, and even associated with the resoluteness to do
whatever one decided to do, all the way through.  So, perhaps, reading
Nietzsche? The need, the intense need of \co{attachment} to overcome
something, is but the need to deny it, is but an expression of will
strained by disrespect.\footnote{You will be able not to misinterpret
  this. Of course, \co{I} can act in \co{this world} for various
  purposes, \co{I} can try to achieve various goals and this does not,
  by itself, mean that \co{I} am not free. What makes me unfree is
  when something becomes unbearable so that \co{I} {\em have to}
  achieve some goals, when something {\em has to} be changed, when the
  arrangements of \co{visible things} are \co{experienced} as all and
  only ground determining my freedom. Freedom is not the same as mere
  arrangement of things which would be comfortable and pleasing to
  \co{my} will.}
%
\co{Detachment}, or any attempt in its direction, is but a form 
(\co{inversion}) of \co{attachment}, a despair capable of nothing 
more than negation.
In fact, \co{I} would probably feel much
more unfree in a completely chaotic world, facing the \wo{certitude of 
the abyss}, than in the world of Newton, or even Laplace.

\pa
Freedom comes from the \co{other world} but it is freedom to
\co{this world}, freedom to accept and respect \co{visible} things
among which we live.
%
It does not amount to overcoming causality or other possible
determinations of our \co{acts}. Various \co{acts} in various
situations may indeed be performed under some coercing factors. But
this does not change the fact that most human \co{acts} have no
discernible, \co{visible} causes because causality, as Kant teaches,
is a category of mere \co{actuality}. \co{Experience} of free choice
is irrefutable, and so determinism, possible perhaps as it might be,
remains since millenia a mere postulate -- the postulate to find out all
the determining causes.

Lack of (ok, let's say only, \co{experience} of) determinism does not
mean complete indeterminism. As all contradictions, this is only the
case for thinking bound to \co{actual dissociations}. The lack of
\co{discernible} causes does not preclude the \co{presence} of
\co{motivations} and, eventually, \co{inspirations}. For these do not
cause anything, they do not determine any specific \co{acts}; they
only indicate \co{vaguely} a possible direction.\footnote{We tried to
  acsribe this distinction to Duns Scotus with his distinction between \la{ordo
    eminentiae} and \la{ordo dependentiae},
  \refpp{pa:foundingCausing}. Similarly, we might try to recast in
  these terms the Thomistic \thi{double causality}, as well as general
  Scholastic distinctions of various \thi{orders of primacy} (e.g.,
  with respect to knowledge, dignity, time, causality, etc.).
  Eriugena's \thi{primordial causes} come probably closest to our
  \co{invisible founding} of \co{actual} events, with his emphasis on
  the \co{invisible} unity and continuous \co{presence} of such
  \thi{causes}. Certainly, the word \wo{cause} is particularly unlucky
  and much confusion seems to arise from conflating it to the level of
  \co{actual} causality.}
%
\thi{Good intension} may be considered a
cause of a particular good \co{act}.  But as we said, such an
intension already pollutes the \co{act} and diverts it from its
\co{actual object} by attempting to \co{actualize} something
\co{invisible}. In any case, being \co{visible}, it does not \co{found}
the goodness of the \co{act}. \co{Invisibility} of a good being, however,
does not cause any particular \co{acts} -- it only \co{founds} the \co{rest}
\co{present} in such \co{acts}. 

%
This freedom to \co{express} the \co{invisible}, this \co{presence} of
\co{transcendence} in the midst of \co{immanence}, does not so much
found the freedom of lower levels, but rather abolishes the need for
looking for it there.  To be free is to \co{forget} freedom. (Let us
only remind that \co{forgetfulness} is not the same as renounciation
or denial, \refp{pa:forget}.) This freedom founds only the thorough
\co{experience} of freedom, which permeates \co{my} whole being, and
which is not contradicted by any problems, obstacles, restrictions at
the lower levels. In fact, all such obstacles may then be taken as
mere challanges to exercise of freedom (although this phrase might
wrongly suggest that there is any point in \thi{exercising freedom}).

\pa The question is not if \co{I}, by a universal decree of human
nature, by a solid and undeniable, natural or unnatural law, am free
or not, if my will chooses the ice cream because of my upbringing,
social dependencies and digestive problems or else just because it
chooses so, in a complete indeterminacy of emptiness.  The question is
 if \co{I}, by the power of my \co{spirit}, am able to become
worthy of receiving freedom from \co{above}.  Indeed, freedom is but a
side-effect, an aspect of \co{love}, of submission to the \co{command}
and the resulting \co{non-attachment}.  If this sounds like a
contradiction then I am pleased -- as far as freedom is concerned,
this will have to suffice.

\ad{Alertness and strength} 
The respect for things and people is but
the \co{presence} of \co{openness}, an \co{expression} of 
God's \co{omnipresence}.
Every meeting, with a person,
with a situation, with a problem, is a \co{gift}; sometimes, a
challenge, sometimes, but a pleasant confirmation, sometimes a plain
disaster.  No matter what the specific character of this meeting,
\co{I} should be \co{thankful} for it because, at the bottom of it,
the very fact of being able to meet something, deserves deepes
\co{gratitude}, and because every such meeting is a meeting with 
%an aspect of 
the \co{origin}.  

This \co{thankfulness}, however, does not mean that \co{I} am to fall
flat and thank God for bestowing on me yet another disastrous gift. 
Being annoyed, being displeased, being disgusted are impressions and
feelings one need not get rid of -- they are feelings of human saints
as much as of human wretches.  To be thankfull for particulars, is to
meet them with all the respect and to try place them on the right
shelf in \thi{the order of things}.\footnote{If you are getting tired
of this phrase and wonder what it means, then I won't tell you.  I do
not know!  I think a person who pretends to know it -- in general --
is confused, perhaps, even dangerous. It is probably close to Homer's
\citf{golden chain}{Iliad, VIII:18-27}, or to \thi{the true value of things} in
the following: \citf{Virtue is to be able to
render the true value to the things among which we move and in which
we live.}{Lucilius [in My Sister and I, XII:17, p.226]}} And if \co{I}
have no clue where it belongs, then it can stay where it is, at least for
the time being.  Valuing thigs we also value our life and \co{express}
our \co{gratitude}.  The alertness and presence of mind is just the
steady preparedness to meet things with such an attitude.  It is
\co{founded} in \co{transcendence}, by \co{openness}, but it concerns 
all \co{immanent}, particular things.

\pa
Strength isn't much more than this attitude. It is 
not strength of will,it is  not strength of abilities, of powers but just 
that -- strength, preparedness to meet everything with equal 
tranquility and \co{openness}, to face things and be ready to handle them. 
It has nothing to do with hardness, with the defensive, 
self-protective shell one can, sometimes with ingenous inventiveness, 
build around in order to be able to handle all kinds of situations. 
Hardness assumes that situations one has to be prepared to meet are 
something one has to protect oneself against, are potentially 
harmfull, dangerous. Strength sees the possible dangers, too, but its 
main purpose in such meetings is not to protect itself but to put it 
on the right shelf in \thi{the order of things}. 

Hardness is but an extreme case (an \co{inversion}) of false security
which has spent years on designing schemes and laws of things making everything
fit neatly here and there, on the right or on the wrong side, and
which, eventually, realizes that the whole scheme was but a
construction; security which, in the most unexpected moment, in the
moment of uttermost security and complacency, is suddenly
surpirsed, and that means defeated, to the bottom of its scheme.  The
fear of unexpected, natural as it might be, and which we might
call insecurity, is founded in false security, in \co{closedness} of
\No, which tries to build walls, houses, cities and yet, all the time,
knows {\em that} there still may be something it did not take into account,
although it has no idea {\em what} it might be.
Rising cities, we \co{thirst} for the woods.


\ad{Compassion and forgiveness}\label{compassion} \co{Openness},
the \co{recognition} of God's \co{omnipresence}, breeds respect and love
for things.  The \co{openness} means thus also to embrace and
invigorate, to let everything grow (as Aristoteles would say, to let
it achieve its ultimate Good, to actualize its essence) and to take
care for things which are within the horison of our \co{action}.  The
care we take for things is not grounded in our infinite love for their
\co{absolute} value, but in our love for them as \co{conrete}
\co{gifts} of the \co{origin}; it is love in \co{analogical} sense,
love which is but an \co{expression} of \co{spiritual love}.  As
always, telling one from another, telling such love from
\co{idolatry}, is not a matter of any rules and laws, but of
\co{concrete}, personal presence.  What is the utlimate Good of this
or that thing?  Fortunately, there is no general answer, because if
there were, your life would be pretty boring.  Taking care of a farm,
a buisness, a house may be one or another, and we do wisely suspeding
our general judgments in such matters.  It is only by \co{concrete}
presence in every meeting, by alert sensitivity to every situation
that we may, possibly, figure out what can and should be done, how to
take care of things and how to support the people.

\subpa Compassion is but the extreme form of this \co{openness} and
respect, of \thi{wishing everytbody the best} and \thi{regretting all
misfortunes} others might encounter.  

It is all embracing because it does not divide people into those who
reached the adequate spiritual level and deserve understanding and
those who did not and therefore have only themselves to thank for
their problems.  In a given situation, compassion is directed
exclusively towards the person and yet, in a sense, it is completely
impersonal since it embraces all who are in need.  It is not, however,
the same as \co{love} and need not extend to the whole of world.  Such
a universality of compassion, as can be found for instance in Unamuno,
assumes that the whole world is, indeed, in a need for it, that the
whole world is a scene of one all embracing misery and the life has
only tragic sense.  Such an exaggerated compassion is but an
exaggerated feeling; it comes closer to patronizing, lacking the basis
of \co{thankfulness}.  Compassion does not need any pity, for pity
hides some lack of respect, we could perhaps say, pity is compassion
without respect.

Compassion does not pity the tragic sens of life, the unbearable
and unavoidable ivolvement into the evil of the world, the corruption
of one's soul.  Because compassion knows that, ultimately, everything
is a \co{gift}, only that some of these \co{gifts} may be harder to
carry than others. Those who are in need of help and compassion need
not be pitied, for they, too, have been \co{born}, they, too, have a hidden pact
with God and, eventually, their salvation is the matter of 
their life. They may only need some help, here and now, with this or that.
Every problem is a particular, \co{visible} thing. 

\subpa Compassion is the ultimate form of openness towards \co{this
world} because it has no sign of reproach or rebuke.  It may criticize
one for doing wrong things but not for being wrong person, for it
knows that others, too, have been \co{born}, they, too, have a hidden
pact with God. Thus if it judges, it \wo{judges the act not the 
person.} This is forgiveness -- the ability to distinguish \co{visible} 
from \co{invisible}, the \co{act} from the person, and to 
\citt{render to Caesar the
               things that are Caesar's, and to God the things that are
               God's.}{Mrk. XII:17}

As always, forgiveness, too, can sometimes manifest itself in
\co{inverted} form.  Insisting on a just punishment may, in some
cases, be an expression forgiveness  (although we are perhaps stretching
now the meaning of words).  It need not signify the
bitter feeling of hummiliation but may be a \co{sign} of respect and
concern which can so easily be taken as proud self-righteousness.  It
is hard to imagine, but it is possible for a person to be full of
forgiveness and still insist on some form of retribution.  One should,
however, be very wary of such \co{acts} because the border separating them
from the feelings of \co{myself} being offended and seeking revenge
is \co{invisibly} thin.  
Perfect \co{love} can not be offended because there is no \co{me} 
who would matter enough to be offended.
The search for receiving satisfaction for an
offence is a \co{sign} of \co{attachment}, of involvement into 
\co{miness}. It is only a milder form of venegance and this, in turn, is the
exact opposite of forgiveness -- the hatred which is directed no
longer against mere \co{acts} but against the person himself. But
\citt{vengeance is mine; I will repay, saith the Lord}{Romans XII:19}
               
\say
We will see more examples of this \thi{axiological founding} in the 
next section~\ref{se:morals}.


\subsection{Work and \co{acts}\ldots (also of \sch) ?}
\pa
The \sch\ as desribed earlier was only an abstract characterization. 
Every \co{act} performed with \co{love} is, in fact, an \co{act} of 
\sch, an \co{act} which repeats unconditional \yes.


\pa\label{actmodes}
All acts performed in \G\ keep heaven and earth together.

Founding \ldots

\pa (???)
But specially:  work, wisdom, prayer, patience, humility, 
observance of rituals -- which aren't acts, but complexes, attitudes, etc.

\pa
Yet, all these acts can be performed in emptiness -- without the inspiration
from \HH. Then they only help to reach the goal: the state where they flow
from the \HH\ not as obedience but as the natural consequence.


%end \input{032non-attach}

%%%%%%%%%%%%%
\subsection{\co{Concrete founding}: examples}
\secQ{The loveliness that is in the sense-realm is an index of
the nobleness of the Intellectual sphere, displaying its power and its
goodness alike: and all things are for ever linked;}{Plotinus, IV:8:6}


\pa The \co{spiritual} attitude, the attitude towards Godhead, 
\co{nothingness}, is actually \ldots  morality.  It does not matter if
one calls it \co{One}, God, \co{origin} or, perhaps, does not call it
at all.  What matters is, whether one \co{recognizes} something 
\co{invisible above}
which abolishes validity of \co{my} \co{visible} goals and wishes and,
at the same time, opens for \co{me} \co{the world} as a field of
\co{incarnation}, of continuous \co{expression}.
\citt{Neither may a man who is made a partaker of the divine nature, 
oppress or grieve any one. That is, it never entereth into his 
thoughts, or intents, or wishes, to cause pain or distress to any, 
either by deed or neglect, by speach or silence.}{Th. Germ. XXXIII}
How come? Why can't he oppress or grieve any one? Do not we have 
enough counter-examples?

\pa ``Love and do whatever you want.'' Whatever? Is this focusing
on a mere attitude justifiable? We know it, and we know that, as an
ethical rule, it does not work well. If I love, does it mean that I
can steal, kill, and whatever?  In a sense, yes.  ``Whatever'' means
whatever.  But, if you steal and kill like a crazy criminal, then you
do not love.  Following Augustin, the Free Spirits of XIV-th century
said too: \wo{Love God, and do as you like.} Eckhart's reservation was
immediate: \citt{Yes; but as long as you like anything contrary to
God's will, you do not love Him.}{Eckhart}

It looks like a circle, doesn't it?  Not, however, a
vicious circle but a virtuous one.  \citt{[B]y their fruits ye shall
know them}{Math.  VII:20} The \co{spiritual love} \co{founds} the whole
attitude and, by this token, makes some acts impossible.  As this
founding, this \co{incarnation} happens in the \co{invisible} center
of \co{virtuality}, it contains things which are inextricably bound
together and which only rational \co{reflection} can dissociate.  For
a person who truly \co{loves}, the ``whatever'' means something very
different than for a person without such a \co{love}.  The circle seems
vicious only if one has \co{dissociated} the \co{spiritual} from the 
\co{visible} and then tries to find \co{visible} criteria for \co{love} which
one does not possess,to determine, univocally and
\co{precisely}, what should count as \co{love} and what should not.

\pa
The focusing on the attitude is justifiable by the claim that the 
fruits of the right attitude are right, that what is right 
\co{above}, what is \co{holy},  
will also be right \co{below}, will be morally right. \citt{As above, 
so below}{The Emerald Tablet of Hermes Trismegistus [Although the 
original is unknown, this formula is reasonably documented as one of 
its central teachings (in cnojunction with \wo{and as below, so above}.]}
Consequently, what is wrong 
\co{below} reflects also something wrong \co{above}. 
%I have to dismiss 
%nthe questions about \co{precise} criteria, because I do not believe in 
%them. 
And no, there is no circle\ldots

\pa
The \sch\ of \yes\ founds a new order which, however, is only a 
\co{reflection} of the ontological order of Being -- it finds 
everything flowing from, and immersed in the \co{original}
source. The result is a state -- not, however, of body or mind, not of 
consiousness or feelings, but a \co{spiritual} state of thorough 
participation. This state \co{manifests} itself, down to the lowest 
level of \co{actualization}, through a gradual dissociation of 
elements, through separation of what \co{virtuially} can not be 
separated. And thus. lower levels become founded in the higher ones, 
smaller and more \co{precise}, \co{actual} things get embraced by the 
unity -- not toality -- of the underlyiing \yes. 
%Let us, for the sake of illustration, consider the example of love.

\pa It is hard to accept because it is so rare\ldots But this does not
mean that it is unreal.  \citt{All things excellent are as difficult
as they are rare.}{Spinoza, last sentence of {\em Ethics}} 
%and even
%if, in principle, they are the most prone to be shared, they go often
%unrecognized by the impersonal structures capable only of appreciating
%\co{visible} causes and effects.
\co{Any experience} of love which is not a fully \co{incarnated
spiritual love} is distorted and aware of its imperfection.  Most
\co{experiences} are of this kind.  But \co{love} is not, for this
reason, an unattainable ideal, a regulative idea, an inaccessible
goal.  Such goals, such postulates of inaccessible ideas are either
illusory constructions or else projections of the vertical transcendence
onto its horisontal dimension: results of \co{attachment} which,
yearning for the highest things, is unable, from its narrowly
\co{actual} perspective, to grasp and re-construct their
\co{invisible} possibility.  Indeed, {love} which is not
\co{incarnated} and fully \co{present}, or at least, which one is
unable to perceive as possibly \co{incarnated}, remains an empty idea. 
But a \co{concrete love} is not a mere state of mind, is not an 
image, \co{vague} and \co{unclear}, 
of something ideal and in its mere desirability completely
\thi{unreal}; it is a thoroughly \co{concrete manifestation}
penetrating to the lowest levels of Being, through thoughts and feelings,
\co{actions} and goals, to the most \co{immediate} \co{acts}.
 
\pa A few subsections that follow will give examples of \co{spiritual
founding}.  In each case, I will describe 
their \co{incarnated} modifications resulting from such a 
\co{founding}, and confront them with possible 
variations of the respective \co{experiences} which are not 
\co{concretely founded} 
but are limited to the current level, as if cutting the hierarchy at 
this level and unable to reach beyond it.



\subsub{The founding of love}

\ldots\footnote{In spite of many differences, 
the hierarchy here may be found to conform closely to Ibn`Arabi's in his {\em 
The Treatise on Love}, in particular, II/13}

\found{Love}
  {sensous : instantaneous pleasures (body)}
  {egotic (Ego) : of things/this world/humanity}
  {personal : of people/{\co myself}}
{charity (spirit) : of \co{nothing}}
  {of {\em the person}/soul}
  {respect and enjoy: things}
  {pleasure}
  
[cf. Huxley, pp.83]


\pa \inv 
\co{Love} at the level of \co{invisible}, the \co{spiritual love} was 
described in section \ref{sub:sch}. It is a unified attitude of a 
whole person towards the \co{invisibility} of the \co{origin}, the 
unity of \co{humility}, \co{thankfulness}, \co{openness} and, possibly, other 
aspects which one can call by various names and which never exhaust 
its reality. By \co{analogy} (and 
only by \co{analogy}!) it founds the \co{experiences} of
\co{omnipotence}, \co{goodness} and \co{omnipresence} -- of 
\co{noithingness}, the 
\co{One}, the \co{origin}, Godhead. 
All these aspects are not related to any particular, 
\co{visible} region of Being but found the unbroken unity 
throughout \co{this} and \co{another world}. Above all, they are {\em 
only} aspects, some aspects, of a one unified attitude (which I have 
to list here for the sake of, at least some, \co{concreteness} of the 
presentation).

\pa\label{pa:persloveB} \mine
At the level of \co{minness}, such a \co{love} will find an 
\co{expression} as a living love 
with which the \co{soul} embraces the world or, perhaps, some more 
particular region of Being. 
Whether it becomes love of the 
world, of all the people, of life, it still carries the \co{humility} 
which prevents it from focusing on \co{oneself}, which, founding it, 
makes it a true love and not a mere appearance of love.

\subpa
The personal love, 
love of another person, may have many degenerate forms, but in its 
true form it is never a separate focusing on this only person with the 
exclusion of everything and everybody else. A true love of another 
person is impossible without the \co{presence} of the underlying 
\co{love}. Love between two people is always immersed into something 
bigger, something which, apparently, brings the lovers out of 
\co{this world}. It is not the social law and context offended by 
their love, it is 
not the king Mark, the husband of Iseult, but the 
wood of Morois, to which she and Tristan flee from \co{this world}. 


\subpa 
Love between two people
unfolds always against the background of \co{presence} which has
acquired a friendly, perhaps even exctatic character.  And yet, this
very \co{presence} may have an oddly \co{inverted} \co{expression}
when judged by the standards and customs of \co{this world}.  
\citt{Think not that I am come to send peace on earth: I came not to
send peace, but a sword.  For I am come to set a man at variance
against his father, and the daughter against her mother, and the
daughter in law against her mother in law.  And a man's foes shall be
they of his own household.  He that loveth father or mother more than
me is not worthy of me: and he that loveth son or daughter more than
me is not worthy of me.}{Math.  X:34-37} 
Such an \co{inversion}
does not mark a break, an immorality, but at most, an offence which is
so perceived only by those who can not judge otherwise than
according to the ethical -- and in cases of such conflicts, this always 
means, pharisean -- prescriptions.


\subpa  
Meeting with another person in the face of a common \co{foundation}, 
personal love becomes a true \co{communion}, the \co{communion} of 
\co{sharing} the \co{origin}. 
This love is now meeting 
with a third, with something \co{above} us both, which lends its 
meaning and depth to our mutual relation. And in all our 
\co{sharing}, of life, that is of the world, of time, of works and 
days, of joys and sorrows, this \co{founding} element, this 
indefinable \co{rest} is always \co{present} with unmistakable 
\co{clarity}.

\subpa
The \co{concreteness} of such a personal love may 
 involve fascination with this or that feature, this or that 
 characteristic of another 
person, but all such features are but attractive accidents -- they 
may be needed for \co{me} to fall in love, but they do not constitute 
the exclusive \co{foundation}  of this love. This love is directed 
towards the whole person, which means, towards the person as 
\co{transcending} the particular features, ways of being and 
behaving, the person as the center and site of \co{invisible 
incarnation}. \co{I} do not divide the loved person into aspects and 
traits and decide to love her because of $a,b,c$ and $d$. If \co{I} can 
tell why \co{I} love a person, then \co{I} do not love. Sure, \co{I} 
can list a long series of agreaable and wonderful features of this 
person, but if this list exhaustes the resons, then this is not love. 

To love a person is to love the whole person, the 
person seen as the source of \co{incarnation}. In this respect 
\citt{the loved one is impeccable 
%stainless, spotless, immaculate, unimpechable
in his vesture
%apparel
at the very beginning of being, because nothing lowers nor
stains him in the first moment of his revelation and  
being.}{Ibn`Arabi, {\em The Treatise on Love} V/88}
\co{I} may, if not at once then with time, see all the negative sides 
of the loved person, but to the extent \co{I} love the person, these 
are but lower aspects, possible failures which, as a matter of fact, 
may be charming too. 

\pa %--\mine
And now, all this is quite different from a love which is not 
\co{founded} in \co{love}, but which stops at the level of 
\co{miness}. 

The lower we descend into \co{this world}, the more strength of
will may be needed to stay true to the \co{inspirations} and to
nourish the constant intensity of feelings.  However, the strength of
will is needed only to the extent the \co{original commands} get
clouded by the the lower aspects.  Antigone says ``I know that I
please where I am most bound to please.''  and this does not involve
any conflict of her will.\footnote{And I do not think that this happens
merely because, being Greek, she does not have the concept. The whole 
tragedy is built around the choice which we may easily term \wo{a 
conflict of comittment, or will}. 
}
The very attempts at nourishing and
keeping the intensity of the beginnings are already 
expressions of a loss, that is, expressions of an \co{attachment}, an 
\co{attachment} to the past. Whether \co{I} insist on
\co{my} feelings, \co{my} expectations, \co{my} goals it is all 
\co{attachment} to the \co{visibility} of the past -- whether by 
attempts to preserve it or negate it -- 
which has separated me from the \co{invisible} source of \co{love}. 
Such an \co{attachment} actually \thi{divides} the loved person, 
puts a \thi{+} at $a,b,c,d$ and \thi{--} at $f,g,h$, and when the 
calculus of \thi{+}s and \thi{--}s yields a negative result, \co{I} 
become disappointed \ldots with the person. The disappointments 
reflect only the fact that \co{my} love was not directed towards the 
whole person -- it was cultivated and maintanined not for the sake of 
the loved person, but for \co{my} own the sake.\footnote{A simple 
example is a love through which \co{I} merely seek a compensation of 
some fundamental lack on \co{my} part. Not (necessarily) a lack of strength or 
intelligence or success, but a {\em fundamental} lack which \co{I vaguely} 
feel -- the emptiness at the bottom of \co{my soul} which the other 
person would fill, the uncanny loneliness which the other person would 
cure, the undefinable dissatisfaction with \co{my life} which the 
other person would calm.
}

\subpa
There are no disappointments if one, instead of nourishing
expectations, nourishes \co{hope}.  \co{Love} knows that the
particulars (whether the particular ways and reactions of another
person, or particular things and situations) need
leniency, respectful openness and acceptance. 
But such a true patience and care for things and people are not 
ontological gifts of the origin. They are founded
only in the deepest \co{humility} and \co{openness} -- if not,
patience and respect will, sooner or later, reach the end and
only laziness may prevent them from jumping to new conclusions.



\pa \act Personal love, which at the level of \co{actuality} and Ego
may also be expressed through infatuation, embraces \co{this world}
and its things and lend them the character of enchantment and
agreeable vitality.  This may be a mere feeling, a series of
\co{impressions} which change and pass as soon as infatuation goes
away.  But if it is not a mere infatuation but a lasting love, the
things, being \co{shared} with the loved one, will, too, last in this
state of rewarding and peaceful \co{presence}.  The wonderful traces
of another's personality, expressed in \co{actual} situations, 
transform them into a joy of \co{particpation}. Even situations which 
\co{I} might otherwise find inattrative, even repulsive, acquire this 
character through the presence of the loved one -- \wo{It is better to 
lose with the loved one, than to find with the hated one.}

\subpa
The \co{spiritual love} is a constant \co{inspiration} for the lower
levels, an \co{inspiration} to embrace, strengthen and invigorate --
whether the loved person or the things towards which it turns at a
given moment.  It is manifested through care and respect for things,
as well as for the particulars of the behaviors, feelings and
reactions of the loved person.\footnote{This is perhaps obvious, but
let me emphasize that this care and respect are quite different from
Heidegger's care -- \ger{Sorge}.  \ger{Sorge}, that is, \citf{the
Being of Dasein ahead-of-itself-Being-already-in-(the-world) as
Being-alongside (entities encountered within-the-world) [\ldots] is
used in a purely ontologico-existential manner.  From this
signification every tendency of Being which one might have in mind
ontically, such as worry or carefreeness, is ruled out.}{SuZ, 192}
This corresponds more closely to our horison of \co{experience}, with
its \co{spatio-temporal actuality} confronted with the
\co{non-actual}, which precedes the constitution of time and separate
\co{experiences}.  Care and respect we are talking about here are, to
use Heidegger's terminology, precisely \thi{ontical tendencies} which
Dasein may or may not realize.} This care and respect need not, of
course, mean unconditional acceptance of everything the loved one
does.  But all particulars which \co{I} find blameable are placed at
the level to which they belong, at the level of \co{actual} failures,
and in no way diminish \co{my} love.  As we know, the blindness of
love may not see, and if it sees will excuse, many things which others
find inexcusable.  Shall we say that this blindness is what I have
called an \co{inspiration} from \co{above}? Not necessarily because a 
mere infatuation may have similar effect. But it does exemplify the 
general way of \co{concrete founding}, that is, transformation of the 
 events and their perception at the lower 
levels by the \co{concrete} events at the higher ones.
%%
%%\subpa Sure, the lower we descend into \co{this world}, the more
%%\co{distinctions} and more possible choices we face.  The most
%%dramatic ones are those which involve a choice between lower and
%%higher.  Antigone {\em has to} burry her brother, 
%%%Polyneices, 
%%in spite
%%of Creon's interdiction and popular fear, shared even by her sister: ``I will
%%bury him: well for me to die in doing that.  I shall rest, a loved one
%%with him whom I have loved, sinless in my crime; for I owe a longer
%%allegiance to the dead than to the living: in that world I shall abide
%%for ever.  But if thou wilt, be guilty of dishonouring laws which the
%%gods have stablished in honour.''  But Sophocles does not leave any
%%doubt -- when the higher \co{command} is \co{expressed} with such a
%%\co{clarity}, the choice is obvious.

\pa %-\act
Love at the level of \co{actuality} which is not \co{spiritually 
founded}, may be directed at things which \co{I} want to possess, or 
else may find \co{my Ego} as the highest value. It is hard to 
recognize any true love in narcistic self-idolatry, but even such 
extreme forms of \co{egotism} may hide themselves behind appearances 
of love. 
Not recognizing anything 
higher than the \co{actuality} of \co{Ego} and \co{visibility} of its 
\co{objects}, one can still yearn for \co{love} and this yearning may 
easily find occasional \co{expressions} of less \co{egotic} 
character. But these will only be occasional \co{expressions}, 
constantly confused by the tyrany of \co{egotic} impulses.

Another person is no longer loved only for \co{my} sake, honestly 
though confusedly, but for the sake of some particular thing. \wo{\co{I} 
love her smile}, \wo{\co{I} love her meekness}, \wo{\co{I} love her 
determinacy}, no matter what particulars happen to arise the reaction, 
it is only a reaction, it is only a response to an \co{actual} 
fascination which is cherished for the sake of satisfaction it gives 
\co{me}. 


\pa \imm At the lowest level of \co{immediacy}, love, like anything
else, finds only the most momentanous \co{expressions}.  Sex may
provide a very good example, since the infinite gap separating the
purely carnal, \co{unfounded} but \co{actual} sex from the event of
making love to a loved person will be perfectly clear to anybody who
has experienced both extremes.  The sensous pleasure is not
necessarily \co{spiritually founded} in any higher order of things,
but it is tremendously modified if such a \co{founding} has taken place. 
On the one extreme, it may be a mere moment of escape from the
unbearable suffering, a moment of sudden meeting with eternity in the
midst of confusion and evil, like is typically experienced by the war
time lovers of Remarque.  It may be even more desparate attempt to
convince oneself that, after all, there are good things in life,
things which, in the brief moments of pleasure let one forget about
the constant despair and meaninglessnes.  All such moments do provide
the pleasure they promise, but the pleasure turns out to be
insufficient to calm the soul.  On the other extreme, a sensous
pleasure of a moment may be peacefully embraced by the context of
mutual respect and understanding, and, at the deepest level, of the
ineffable \co{love} which, by a lucky coincidence, found an
\co{incarnation} in the other person, like an inexplicable, arbitrary
\co{gift} which never ceases to surpirse and please by its mere
presence.

\pa %--\imm
We should not try to reduce such considerations to a mere
contextuality of \co{actual experiences}.  The context of \co{an
experience} is a \co{more} of other \co{experiences}, and a
well experienced and well paid hore can create such a context.  One
can only buy moments, \co{actualities}, separate, scattered parts. 
But such parts, and contexts which are their \co{complexes},
especially if they are merely bought, will never sum up to
\co{an experience} which in a single moment traverses the whole
hierarchy of Being and reaches smoothly to the deepest intimacy of
\co{transcendent presence}.  There is no way to reach the \co{origin}
by taking, one after another, the same step at a time, even though
taking one step at a time is all we can do.  But the steps may be
wrong while they must be right, and nobody can take them for me.  In
all pleasant delusions, which \co{I} bought or aarranged, which \co{I}
find purposeful and satisfying, \co{I} always also know -- if only
\co{I} do not ask too intensely -- that they are but momentaneous
pleasures, simple delights of a hedonist, real and \co{actual} and, at
the same time, empty and urewarding.

\pa There is no such thing as \co{a single experience} which is not
permeated by the whole Being, that is, \co{my} being and being of
\co{my Self}.  There is no such thing as a moment of love, unless this
moment is immersed in the texture (not context) of body, \co{Ego},
\co{soul} and \co{spirit}, which all together agree on the
\co{humility} of \co{love} -- in an agreement which goes beyond the
bottom of one's soul and heart, where the \co{spiritual love 
incarnates}.  Only
this whole texture lends the \co{actual} moment its full meaning, only
it makes up its \la{quiddity}, makes it \thi{this \co{concrete
experience}} rather than \thi{that}. No \co{actual} rules can ever 
grasp this distinction with desired adequacy. And no ethical rules or 
principles will ever grasp the moral value of an \co{act} which, to 
the extent it is moral, descends from the \co{invisible} heights and 
only as if by accident takes this particular form rather than that.


\subsub{The founding of communion} 

\pa
\co{Communion} is \co{sharing}, that which is 
\co{shared} determines the character of the \co{communion}. 
\co{Sharing} means recognizing what is being \co{shared}
as being \co{mine} and yours, which as a matter of
fact means, being neither \co{mine} nor yours.  Nature and the
physical world is not ours, we \co{share} it with all the physical
things and living organisms, and this is the basis of a form of
\co{communion}.  We \co{share} more with the animals than with the
dead things, and more with the higher animals than with the lower
ones. 

\pa
Communication is a possible \co{visible} \co{manifestation}
\co{founded} on \co{sharing}; the more we \co{share} with others, the
easier and more complete is the possible communication. 
We \co{share} 
quite a lot with all other people, but there are always special 
people, friends, family, the loved ones with whom we \co{share} much 
more than with the mass of anonymous individuals. And our 
communication with the former ones will be much more adequate and 
complete than with the latter ones. 

Of course, communication is not limited to language and verbal 
expression. Gestures, looks, \co{acts} and \co{actions}, all we do and 
even think are forms of communication. With whom? -- it depends on who 
is interested in and receptive for the message. A simple statement may be 
entirely obscure to a stranger, while it is perfectly clear to a 
person who knows \co{me}. The \co{shared} background, what in some 
contexts might be called the \thi{common reference frame}, determines 
the readability of a communication. The less these frames 
overlap, the less \co{sharing} we can assume and the less adequate 
will the communication be. 

\pa 
We should not, however, confuse that which is \co{shared} with that
which is merely \thi{common}. 
{\thi{Common}, or universal, is the flat, 
\co{objective} version to which \co{sharing} reduces when seen only 
from the perspective of \co{actuality}.
\co{My} religious \co{experience} of 
God may be entirely different from yours, \co{my} understanding and 
solution of a situation may have close to nothing in common with 
yours; but all such differences in no way imply the impossibility or 
even the lessening of \co{sharing}. This becomes impossible only when 
the \co{visible} differences become \co{absolutized}. Looking for 
\thi{common} features and traits which, as one thinks, would promote 
communication, leads only to reducing the communication to the least 
common denominator, to that which being most universal is also lowest. 
What do I have in common with a stranger whom I have never met? 
Certainly something, but if I want to base my communication with him 
on that, it will be the same as communicating with any other 
stranger, with an anonymous individual. Sure, we all have body and 
natural drives, we all get hungry and thirsty, but it is more or less 
as far as it goes. Universality is an attempt to capture \co{more}, is 
an expression of \co{horisontal transcendence} -- it may be useful, 
but never \co{concrete}.

\pa
\co{Sharing}, on the other hand, refers always to a \co{vertical 
transcendence}, what is \co{shared} is always \co{above} \co{me}.
In a particular context of communication, it may be simply a \thi{good 
will} of listening to another; in the hermeneutical theory of 
interpretation, it is the initial assumption that the text has 
something meaningful to communicate, eventually, the 
\thi{principle of charity} in reading.




\found{Levels of communion}
    {atomic / no relations}
    {common goals/useful persons: formal-institutional}
    {friendship, personal engagement}
{\co{participation}:communion -- That art thou.}
    {friendship, personal engagement}
    {helpfulness, emphaty}
    {shared moments}
 
\say
Communication involves certainly transfer of some meaning, but such a 
transfer is not what \co{founds} nor exhaust the event of 
communication. Communication is \co{founded} on communion, and without 
it, without anything \co{shared}, it simply does not happen. 

\citt{Each word means something slightly different to each person, even 
among those who share the same cultural background.}{Jung, MHS, p.28} 
This, however, in no way makes communication impossible, in fact, 
it makes it even more interesting. Talking to a friend I say \wo{I am 
deeply disturbed; I do not know if it is because of the last events or 
else if it is a sign of something going wrong with myself.} Almost 
every word here may be understood differently by any two people, and 
yet the statement does communicate something\ldots

-- not only context (because it comprises only visibles)

\pa \inv
%All the above forms of \co{sharing} are dominated by the category of 
%\co{miness} and hence, \co{visibility} and, no matter how unselfish in 
%appearance, by \co{self-centeredness}. No mattter how potentially 
%useful or acceptable, they are in no way \co{founded} in the 
%unconditional \co{community}, in the recognition of the ultimate 
%\co{communion} of \co{love}.
%
The highest form of communication is \co{founded} on \co{sharing} 
the \co{origin}, and the respective \co{communion}, the \co{communion} 
of \co{love} is but \co{participation}.
In what?  In \co{nothing}, that is, in everything.  It embraces and
pemeats the whole world, all the people, all \co{visible} and
\co{invisible} things.  Mysticism 
%(if this is how you would like to call what I am writing) 
is not a \co{detachment} from \co{this world}.  On the contrary, it is
the deepest form of \co{communion} which sees \co{this world} 
\la{sub specie aeternitas}, 
in the light of eternity, 
as a \co{manifestation} of \co{another world},
eventually, as a \co{manifestation} of common \co{origin}.  The
\co{other world} is not a result of \co{distinctions}, in any case not
of \co{objectivisation} and \co{externalisation}, but is the world
which, by its very nature, is a \co{gift}.  It is \co{shared} -- with
whom?  With nobody, with \co{nothing}, that is, with everybody. 
Recognising its nature of a \co{gift}, \co{I} actually \co{experience}
it as not merely \co{mine} but as a \co{gift}
of \co{transcedence} which \co{shares} it with \co{me}.  This
\co{participation}, being the most personal event of \co{spiritual}
life, is also the most intense \co{communion}. \citt{Jede konkrete 
Stunde mit ihrem Welt- und Schicksalsgehalt, die der Person zugeteilt 
wird, ist dem Aufmerkenden Sprache.}{Buber, Zwiesprache I:Verantwortung, 
in Das dialogische Prinzip, p.161}

But it is also \co{communion} with others, because, as was observed 
in \refp{???}, the higher things are also these which admit of more 
unconditional \co{sharing}, which are less dimnished by being 
\co{shared} among more. Since the \co{origin} is common to all of 
us, it is most intimately \co{shared} by all, we all \co{participate} 
in one and the same source from which our lifes and worlds originate. 
The \co{spiritual sharing} is not sharing of this or that, but is 
\co{sharing} of the \co{origin}, and to the extent it also shares this 
or that, it does so in the light of the \co{invisible communion}. 

\noo{
\subpa Recognition of the the world as a \co{gift} requires the
\co{non-attachment}, renounciation of \co{myself}, and thus it is not
relative to \co{my} thinking, wishes, goals.  This \co{communion} is
consumated \co{above}, it is \co{spiritual}, which means,
unconditional.  But \co{spirituality} is first of all \co{humility}
towards the \co{spiritual}, towards the \co{invisible} things. 
\citt{Know what is in front of your face, and what is hidden from you
will be disclosed to you.  For there is nothing hidden that will not
be revealed.}{The Gospel of Thomas, 5} By this very token, the
\co{spiritual} attitude directs its attention almost exclusively
towards \co{this world}.  \citt{[T]here is nothing better, than that a
man should rejoice in his own works; for that is his portion.}{Eccl. 
III:22 [We won't confuse these remarks with stoicism. 
\citf{We must make the best
use that we can of the things which are in our power, and use the
rest according to their nature. What is their nature then? As God
may please.}{Epictetus, The Discourses, I.1} This says the same as 
what has been said above, but the stoical endurance is a matter of 
resignation and surrender to the world which overgoes one's powers, 
not of \co{thankfulness} for its \co{gift}.]}
}

\pa\mine The \co{spiritual} renounciation of \co{myself} does not mean
that \co{I} disappear.  \co{I} am still present as this person, but
this person's privacy and selfishness have dissolved in a \co{higher},
\co{invisible} \co{communion}.  \co{I} do not any longer oppose
\co{myself} to \thi{the world}, to \thi{others}, because we all
\co{share} in the common \co{origin}.
\citt{And just as the same town when seen from different sides will
seem quite different, and as it were multiplied {\em perspectivally},
the same thing happens here: because of the infinite multitude of
simple substances it is as if there were as many different universes;
but they are all perspectives on the same one, according to the
different {\em point of view} of each monad.}{Leibniz, Monadology, 57}


\subpa Personal love from \refpp{pa:persloveB}, love which in all its
private intimacy is but a humble \co{sharing} of the \co{origin}, the 
\co{community} of \co{participation}, 
 is certainly an example. The 
\co{communion} of friendship is another. In the same way as personal 
love, it is \co{founded} on the recognition and apprehension of the whole 
person before it begins to list his admirable qualities. And just 
like love does not necessarily mean the unconditional acceptance of 
everything the loved one does, so too a friend can point out what 
it perceives as mistakes or failures on \co{my} part. 
Such a pointing out, in full friendship and appreciacion of \co{my} 
person is possible only because it is not reduced to the level of 
\co{my} \co{actual} deeds, thoughts and acts, but is \co{founded} in 
the personal recognition of \co{my} value.
There is often 
much more friendship in saying things which \co{I} do not want to 
hear, than in a flat acceptance of everything \co{I} do. 

\subpa
\thi{Recognition and apprehension of the whole person} is exactly to 
see this person as \co{sharing} the \co{origin}, as having an 
\co{invisible} pact with God, just like the one \co{I} have. 
\citt{Human life touches the absolute  throught its \thi{dialogical 
character}; [\ldots] man can become whole not through a relation to 
his own self but only through a relation to another self. This other self 
may be equally limited and conditioned as he is, but in being 
together one experiences that which is unlimited and 
unconditional.}{Buber, {\em Das Problem des Menschen}, II:1.5}
This formulation seems to set the \co{communion} with another \thi{self} 
as a precondition which it, indeed, oten may happen to be in practice. 
But it is not a precondition in the sense that it is only \thi{being 
whole} which \co{founds concretely}, that is, unmistakenably, 
\co{communion} which is its natural \co{expression}.

In more \co{concrete} terms this will, of course, mean a more 
specific acquitance with the particular traces and expressions which 
this pact finds in the being of the other person. 
Establishing \thi{private codes} of communication, words, gestures, 
expressions which carry the full meaning only to those initiated in 
the \co{community} is characteristic for lovers as well as close 
friends. In the extreme cases, secret organizations will establish 
such \thi{codes} equally for the purpose of hidding its secrets as 
for the confirmation of the identity of their community.

Disregarding such extremes (which more often than not are signs of 
sickness), the \thi{codes} are only an \co{expression} of the 
\co{communion}, and its \co{founding} element. They are 
\co{expressions} of \co{sharing} for which, as Kierkegaard says, 
there is no direct communication. In its \co{presence}, any words 
might be used, while in its absence no words will be able to 
communicate it. The most intimate communication is \co{founded} on the 
most intimate \co{communion} -- it is not univocal \co{precision}, 
not universal adequacy of the used 
\co{signs} (which are here always inadequate), but the \co{shared} 
understanding which makes communication possible. 
% ??? diff in words' understanding ??
We might say, communication of \co{love} happens always via way of
God; I understand your message only because I already \co{participate}
in what this message concerns.  \citt{But in simple substances this
influence of one monad over another is only {\em ideal}, and it can
have its effect only through the intervention of God}{Leibniz,
Monadology, 51 [With all possible reservations against the particular
ways in which Leibniz imagined this \thi{intervention}.  The intuition
behind the fragment seems, however, to be the same.]}.  Or better
still, as Buber says it: \citt{Oben und unten sind aneinander
gebunden.  Wer mit den Menschen reden will, ohne mit Gott zu reden,
dessen Wort vollendet sich nicht\ldots}{Buber, Zwiesprache I:Oben und
unten, in Das dialogische Prinzip, p.160 [the sentence continues in
the Buberian way: \wo{aber wer mit Gott reden will, ohne mit den
Menschen zu reden, dessen Wort geht in die Irre.} \co{Concrete
founding} of \co{communion} means that, indeed, existentially the
\co{spiritual} \yes\ and the true \co{communnion} with others (\co{love}, 
freedom, morality,\ldots) are
indistinguishable, or better, co-extensional -- wherever is the one,
the other is too. But I am trying to be a bit more pedantic than Buber -- 
\co{concrete founding} is still \co{founding}, and it \co{founds} a 
whole \nexus\ of \co{aspects} of which \co{communion} is but one.]}


\subpa A very strong form of \co{communion} is \co{expressed} by the
consciousness of belonging to a group.  Not, however, 
 to a group of mere common interests, to an \co{actual}
group which shares a hobby or, perhaps, a problem.  I heard once a
native American saying to his children \wo{White people have been here
for 500 years, {\em we} have been here for 15000 years.  They make
choices based on what seems cool and advantegous to them but this is
not how {\em we} make our choices.  {\em We} have got this land and we
have to care for it for future generations. Our private
wishes are not what counts most.} Belonging to a tribe is to belong to
the world which is far greater than \co{me}, is to be but a member of
a community which transcends the sphere of \co{my life}.  The
respect which native Americans, as many other peoples, used
to show for their land and tradition is an \co{expression} of 
\co{sharing} something which does not belong to anybody, which is 
greater than \co{me}, than you, than him.

\subpa
For Westerners it may still sometimes be the consciousness of
belonging to the nation.  No matter what group is concerned (as long
as it is not a merely \co{actual} group), the recognition of the sound
values of this group, which override \co{my} private preferences, and
healthy dedication to their realization is a possible, \co{visible
expression} of a \co{communion} at the level of \co{miness}.  At this
level, it will be opposed to \thi{others}, \thi{them} but this
distinction need not carry a negative meaning.  For the first, it is
not so much \co{my} group as opposed to \co{not my} group -- it is
{\em this} group as opposed to {\em that}, and it just so happens that
\co{I} belong to this one.  And furthermore, the distinction between
the groups need not mean conflict, in any case, not a fundamental
enemity.  The groups just happen to live different traditions and
different values, occupy different regions of \co{visible world}.  The
\wo{healthy dedication to its values} means their recognition as {\em
only} one -- as opposed to {\em the only} one -- possible
\co{manifestation} of the \co{invisible}.  The capacity to recognize
also other's values as such a \co{manifestation} (and this does not
mean their acceptance, but a deep understanding and thorough respect)
is the condition for being able to live one's own values in a 
\wo{healthy} manner. 

\pa %--\mine
%\label{pa:personallove}
\co{Unfounded communion} at the level of \co{miness} is not \co{sharing} 
somethig higher but sharing \co{myself}, 
whatever \co{I} happen to understand by this at the moment. \co{I} can 
dedicate \co{my} activities to a common good, to a beneficial work for 
the society, \co{I} can become personally engaged -- but in all that 
\co{I} share \co{myeself}.
Focusing on the 
categories of \co{miness}, \co{I} will tend to oppose selfishens to 
unselfishness, the circle of \co{my} private life and interests to 
the good and interests of others, and the engagement into the latter 
will easily assume a charater of a sacrifice on \co{my} part. No 
matter how possibly useful and socially valuable, such an attitude, 
called by Kierkegaard ``ethical'', finds itself in a constant conflict 
which it is unable to resolve otherwise than by negating it or 
turning \co{my attachment} into an \co{idol} of \co{my} generousness, 
\co{my} benevolence, \co{my} self-sacrifice.

%In more intimate context, \co{I} may find some persons valuable and
%enjoy their values, life style, or even life, because these are also
%\co{mine} and so \co{I} am willing to \co{share} them.  A form of
%friendship or even personal love can be expressions of such
%communion.
\subpa 
A possible form of that is when \co{I} \thi{idenitfy} with \co{my} 
community, in the extreme cases of communal or, perhaps, communistic 
consciousness, when \co{I} see the abstractly 
universal good and interest of this community as the highest value.  
When \co{my} community -- tribe, nation, class, religion -- is the 
only source of truth then, indeed,  nationalism or tribal 
consciousness acquire unhealthy form of dogmatism. Such an 
absolutization of a relative sphere of Being is possible only because 
one does not recognize its \co{founding} in a deeper, higher 
sphere -- here it ends, there is nothing \co{above}, and this is the 
last, absolute truth, its final expression. 

\pa\label{pa:relativism}
Relativism is but a variation on this theme, is but an inability to 
recognize something higher combined with the inability to sign the 
narrow-minded doom on all otherness; and the inability to recognize 
something higher means, in fact, that one's own or one's own 
community's values are not so convincing as one would like 
them to be. The recognition of other values 
which \co{I} do not share, is founded on the \co{sharing} of their 
source. If such a source is not found while, at the same time, one 
feels uneasiness with absolutizing the \co{mine} with all its charater 
of historical and social contingency, then, indeed, the only 
possibility is to state the relativity of all \co{visible signs}. 
There is a close connection between relativism and negtive theology. 
The problem is not that they are too extreme but, on the 
contrary, that they do not go far enough.

\co{I} may be proud of belonging to \co{my} nation, \co{I} may be 
even willing to sacrifice \co{my} life for it, but if this nation
is the deepest value which \co{I} am capable to recognize then it 
will easily end in \co{pride} and, probably, nationalism of a doubious 
shade. \co{I} may be proud of that and, at the same time, recognize 
the possibility of others' being proud of belonging to their nations 
and even of some not bothering about such a thing at all. 
The conflict between one ethos and another may be of fundamental 
character but for the most it is a conflict resting on the 
absolutization of \co{visibile expressions}, of the \co{signs} which 
merely announce, always only in one particular form, the 
\co{invisible presence}.

\co{I} can live thoroughly the values of \co{my} cultural 
formation (as long as \co{I} find in them the \co{spiritual 
foundation}) and, at the same time, recognize equally thorough 
validity of other values. Because all 
\co{visibility} is \co{founded} in the \co{invisible}, \co{I} simply 
recognize that any \co{particpation} in the \co{absolute} is 
intervowen into the matter of \co{this world}. There is no other way 
of \co{participation} than through some form of tradition, historical 
consciousness and involvement into the \co{actual world}. Only through this 
\co{visibility} is the \co{invisible} \co{present}. My task is to 
recognize how this \co{presence} \co{manifests} itself in the world in 
which \co{I} live. The multiplicity of religions is but an expression 
of the unavoidable \co{incarnation} of the \co{invisible} in the 
\co{visible}. And 
to the extent these are true religions, that is, to the extent 
they are based on the recognition of \co{invisible origin above this 
world}, they all provide means for finding a way to rebirth and 
salvation. The fact that somebody born in Tibet does it on the way of 
Buddhism, while somebody born in Europe on the way of Christianity, 
does not in any way diminish the possibility of the ultimate \co{sharing}.
It takes an analphabet to believe that the Truth is written 
somewhere, but it takes a literate superficiality to believe that it 
is written somewhere else.


\pa\act
The highest \co{communion} will hardly found social
institutions, organizations, committees.  But it will neither oppose
them, to the extent they do not oppose the \co{spiritual} foundations. 
\co{Non-attachment} is not withrawal but, on the contrary, an
\co{expression} of ultimate \co{participation}.  In the \co{visible}
terms, this \co{participation} will easily, though not necessarily,
find an \co{expression} as an involvement in the life of the
community.  This will depend on the person but, in principle, there is
not a slightest conflict between the personal attitude of
\co{non-attachment} and participation in the social life.

At the level of \co{actuality} it will again find most personal
\co{expressions} in form of compassion, helpfulness \ldots, in the
\co{actual} \co{sharing} of other person's problems, sorrows, and
joys.  But it is not a mere emphaty, a mere sharing in what is
other's.  The other's problems and achievments are not somethig which
belongs to other's, just like mine are not merely \co{mine} -- they
are just that: problems, achievments, sorrows, joys. When \co{I}
meet them, they are simply there and arise my reaction by becoming
\co{mine}.  Meeting a smile, a joyfull spark in the eyes, a happy
moment in another's life, \co{I} do not \co{participate} in it by observing it,
concluding that it belongs to another, and then deciding to
partake in it.  To the extent \co{I} \co{participate} in it, to
the extent \co{I} \co{share} it, \co{I} do not see that it is
another's and not \co{mine}.  And although \co{my reflection} will
tell me that there is a \co{distinction}, \co{I} know that it is
not telling the whole truth.


\subpa
The level, or the depth of \co{communion} depends on the possibility of 
\co{sharing}. It is only at the personal level, in the face of 
\co{invisible}, that the full \co{communion} is possible. 
The less we can \co{share}, the lesser the \co{communion}. 
We do not 
\co{share} so much with crabs and ants, not to mention inanimate 
things. 
We do not communicate that well with
bacteria, ants or butterflies.  They have quite different structure of
experience; their world has few, if any, common points with ours. 
When some basic aspects of experience are common, there is the issue
 of \co{recognized} contents.  There is probably close
to none overlapp between ours and butterflies.  We communicate a bit
better with cats or dogs; we perceive the same things as obstacles, we
also find more advanced expressions of \thi{feelings} than can be found
in ants and butterflies.  Their \co{experiences} \co{cut} the
background along the lines sufficiently similar to ours and provide
them with a lot of things which we too \co{distinguish} and
\co{recognise.}

Yet, all lower organisms and animals too belong to the common
\co{origin} and although being \co{below} \co{me}, invite to the
\co{experience} of \co{sharing}.  This may be \co{expressed} as care
and respect for their being, as a recognition of the value which is
inherent in this being, too, of which St.~Francis provided such a
powerful example and which Buddhists try to live so literally. 
It is always \co{founded} in the \co{participation} in the
\co{origin}, in the \co{openness} which welcomes all things as its
\co{manifestations}, as if telling \citt{Arise and drink your bliss,
for every thing that lives is holy!}{W.Blake, {\em Visions of the
Daughters of Albion}}



\pa %--\act
In the \co{unfounded communion} of \co{actuality}, one recognizes the
context of one's functioning, of the fact that goals and projects can
be shared between people, that one may meaningfully, and here it
means, to one's own advantage, engage in common activities and pay due
attention to less private issues.  What can be \co{shared} here are
goals and objectives which can be common to several people, perhaps,
to whole parties and communities.  As far as the \co{experience} is
concerned, these are formal, institutional, \co{objective}, kind of
bloodless goals which one only recognizes as serving one's purposes. 
The creed \wo{By contributing to the common goals we also increase our
own well-being} is as true as it is flat.  It has nothing to do with
true \co{communion}, morality nor, for that matter, with anything
which might appeal to a human person.  That many theories of society,
of justice and utility, focus exactly on this form of community does
not seem to show anything except that it is indeed possible to
\co{dissociate} such aspects from the \co{concreteness} of being in an
apparently meaningful way.

\subpa
Communication with others will then easily be reduced to an 
insistence on some \co{visibly} common basis which, as we learn 
from many attempts, it is impossible to specify \co{precisely}, nor 
even \co{concretely}. The 
worship of \thi{rational argumentation} is a common example. Trying to 
convince oneself and others that we are all first of all rational 
beings (whatever that means), one postulates then some ideal goal of 
rational morality consisting in the unreserved acceptance of rational 
arguments. One may even insist that it 
recognizes the dignity of humans paying all due respect to their 
value -- which happens to be the same as the value of their rationality.

Let us ignore the fact that there is hardly anything, hardly any
action or attitude, which could not be supported by plausible
arguments.  In some contexts (of which the academia may serve as the
paramount example), openess to other's arguments is certainly a matter
of professional ethics.  In life, one can also occasionally learn
something from listening to other's arguments.  But when rised to the
level of the fundamental principle it becomes a caricature of genuine
communication.  Have you ever been convinced by an argument?  I mean,
convinced not in some petty matter of this or that, not in the common
attempts to come up with a solution to some problem, but in a matter
of significance, in a matter which you recognise as having existential
relevance.  If one believes in God, is it because of an argument?  If
one does not, is it because of an argument?  I doubt it; in the latter
case, it may rather be because one does not find any argument, and
rests satisfied with one's \thi{rationality}.  But there are no
argumants, or rather, no sufficient reasons; at best, there are only
clarifications of meanings, accounts of experience.  When arguments
are applied beyond the sphere of \co{precisely} defined, \co{actual}
problems, they either become an intellectual game or, when taken
seriously, boil down to one thing: \wo{Either you are stupid since you
do not see that this is right, or else you are respectably rational
and accept it.} Argumentation and persuation are much closer to brute
force than they are willing to admit.\footnote{We won't mention by
name those who see in liberal democracy \thi{the best
political system} known from history since it appeals to arguments and
not force.  How can one call the majority vote, where a vote of a
professor counts as much as that of a farmer, argumentation? How can 
one compare shows of political sophistry and demagogy to rational discourse?
[e.g. Habermas, {\em Strukturwandel der \"{O}ffentlichkeit}, V-VI] Blindness may be hard to distinguish from confusion.  Did not Hegel
made similar claims about the highest possible value of the Prussian
state in {\em his} time?}
%
Appeals to some \thi{ideals}, like \thi{communicative reason and 
rationality}, \thi{uniterrupted communication}, \thi{tolerance}, \thi{solidarity}, etc., are unable
to cover up the underlying disrespect for the human being -- the {\em
whole} human being.  Calls to assuming a respectful attituide towards
the opponents become necessary, because it has to be {\em added on the
top} of all the arguments, like a meek tablecloth coverig a dirty
table.

\pa\imm
At the lowest level of \co{immediate experiences}, \co{communion} 
amounts to \co{sharing} \ldots the moments. 
With whom? It may be the loved person, or else people who happen to be 
present. But it need not be anybody in particular, nobody may be 
\co{actually} present. 
\co{Thankfulness} is \co{sharing} through \co{participation}, and 
every moment, even if lived in loneliness, is but a \co{gift} of the 
\co{transcendence}. Recognition of this \co{gift} in a single moment 
is the same as \co{sharing} it with others -- whether \co{actually} 
present or not. 

\pa %--\imm
Restricted to the level of \co{immediacy} without
\co{concrete founding}, there is hardly any possibility of \co{sharing}
anything.  The things and \co{objects} viewed from this level appear as
arbitrary events of pure \co{immediacy}, and consequently, all kinds
of relations between them are as if purely nominal, unreal,
indifferent.  In terms of relations between people, this amounts to
extreme atomicity, of positing every individual as totally
independent, dissociated from any context and influences from
\thi{outside}.  Everybody has his private goals and life which are not
in any meaningful way shared.  \citt{All creatures are born isolated
and have no need of one another.} {Marquis de Sade, Aliene et Valcour}
Although it is impossible to live such an attitude completely, it is possible to
maintain it mentally, and even to \co{experience} the surroundings and people
around in this way.


\subsub{Reality}
Or are we talking here only about the \thi{sense of reality}...? \co{concrete}...

The real is what you can not live without; and most real is also what
is most deeply \co{shared} -- the \co{origin}. 

\co{Thirst} is the deepest manifestation of reality...

Happiness is but a bad image of what we \co{thirst} for. Indeed, happiness may
be a form of the consequence, but never more than that...

Tischner

??? \found{Acceptance of the world (through which it becomes `mine')}
    {sensualism/hedonism - here-and-now}
    {possess/control}
    {power (world may be evil)}
 {accept/love}
    {understand/feel}
    {care/respect}
    {enjoy}

\subsub{The levels of moral values}
\pa
 The values, and the attitudes which 
we would tend to call ``moral'', can be recognized on different 
levels of Being.\footnote{This hierarchy is probably as close as 
possible to Max Scheler's, e.g., {\em Formalismus in der Ethik und 
die Materiale Wertethik}}

\levs{12}{Levels of moral values}
    {pleasant-unpleasant: counts only Here\& Now (aesteticism)}
    {useful-useless: formal-institutional-traditional}
    {good-evil in-itself, idea beyond forms}
    {holy-unholy: as religious attitude} 

    \pa\inv The values of the \co{invisible} order have only two
modifications -- these are values of either holiness or unholiness. I 
call them \co{spiritual} values. 
Notice that I do not oppose evil (which is a negative value of
\co{miness}) to holiness.  Holiness corresponds to the \co{incarnated
love} of \yes. Its opposite, the \No, does not in itself imply evil 
\co{soul}, but often just a passive \No, a lacking recognition of the 
Godhead. Of course, a consious, \co{active} negation of the ultimate 
\co{invisible} carries a big potential for evil, but it does not 
imply it with necessity.

These values are not relative to any particular region of Being and
their recognition happens only through \co{spiritual}, which in
particular means, thoroughly personal experience.  Holiness is not
expressed through any particular feelings, it does not involve any 
specific forms of thought or action. But any meeting with it results
in, or rather is equivalent with, submission, adoration, reverence, awe.
Unholiness, in its extreme
form of active denial and negation, is associated with despair and feeling of
damnation.  But it is, probably, as rare as holiness itself.  In most 
general form, meeting
with unholiness does not have any particular form. 
Almost everybody we meet is unholy, and the character of the meeting,
as well as our response, are determined then exclusively by the lower
aspects of the situation. This, in fact, is the general 
characterization of unholiness -- the mere absence of holiness.


\pa\label{pa:unholygood}
Now, in itself (that is, except for the ontological order of
Being), the higher levels do not found the lower ones.  In particular,
unholiness, which in its mild and passive form is the most common
thing in the world, and which is a merely passive negation of \No, just a
lacking recognition of \yes, does not found any particular attitudes
at the lower levels. It thus does not exclude any particular 
attitudes at the lower levels -- it only excludes its only opposite, 
holiness. A person may be unholy, even deliberately and actively so, 
and yet be smart, just, perhaps, even good towards his companions and 
friends, and may enjoy all the positive values of the lower levels. Of 
course, the opposite may also happen, and one is almost guaranteed 
that, at one point or another, something opposite will take place.

\pa
You may sometimes meet a person who seems perfect -- wise, just, 
honest, whatever positive predicates you want to use. You know that 
perfection does not exist, and you suspect that this person is not 
perfect either. But the time goes on and on, and the person does not 
do one wrong thing. Eventually, you have to conclude: he is, indeed, 
perfect. And then comes a moment, not necessarily of any deep 
significance, a moment when a single word sounds somewhat inadequate, 
a simple gesture seems kind of inapropriate. 
A minor thing, a slip of the tongue. Nothing wrong has happened, but 
your problem is that this single thing does not fit into the overall 
picture of perfection. You ignore it, but it recurs nagging and 
disturbing your peace of mind. The more you dwell on it, the clearer 
becomes the idea and significance of this single thing. In fact, 
it discloses a side of the person which you did not expect and which 
horrifies you when you think what it is.\footnote{I do not 
necessarily have in mind a peaceful, kind burgoise, a perfect father 
of a family who, in his working hours, runs a concentration camp. I 
do not necessarily have in mind a kind, succesfull, loved by all 
Dorian Grey, whose picture rottens in the locked room of his soul. 
But these might be exaples, too.} A single moment can disclose the 
unexpected depth of corruption or mere misunderstanding and 
imperfection, which has passed unnoticed for months and years. And, 
even though through all these months and years the person did appear 
perfect, you have to admitt that it was an illusion, a deception. If 
you want, you can conclude from such examples that, as a matter of 
fact, all holiness and perfection is only an appearance hidding 
something from the eyes of the world. Often, it may be the case. But 
I am not inclined to draw statistical conclusions. Holiness is a 
state which excludes such moments because it does not hide anything. 

\pa Holiness is always preceded by a \sch\ and,
consequently, does found an attitude.  It is not, however, 
any specific attitude which chooses some particular things rather than
others.  It is an attitude of \co{love}, of \co{humility},
\co{thankfulness} and \co{openess}, with its more specific aspects as
described in section \ref{sub:nonattach} on \co{non-attachment}.  Just
as the \yes\ is merely the renounciation of \co{oneself}, so the
resulting attitude can be best characterized negatively by what it
excludes.  It says, perhaps, ``You shalt love your neighbour'', but
this has little content for one who looks for \co{precise}
characterizations.  ``You shalt not kill'' is probably more specific,
but any such command or prohibition is only an expression of the
intended direction of the \co{soul} -- not a detailed, \co{precise}
prescription excluding any possibility of an exception. 

\pa The attitude founded in holiness will not be occupied with doing
good and right things.  Least of all will it preoccupied with being
holy.  It will be occupied {\em exclusively} with not doing anything
wrong or evil, with avoiding the possible mistakes.  The true and
humble fear of doing wrong things, which is a constant aspect of the
\co{openess} of \co{incarnated love}, is perhaps not a necessary proof
of the impossibility of any wrongs, but it is, in fact, the most
genuine and certain guarantee that one will not committ them.

\pa What \co{precisely} consititutes a mistake and what does not, what
\co{precisely} is wrong and what is not, is, fortunately, impossible
to say in general, and it is something every person has to determine
in a given situation.  Philosophers, and ethical philosophers in
particular, have long despaired that the same questions, the same
problems, have been posed over centuries without ever finding
answers.\footnote{\citf{[A]after more than two thousand years the same
discussions continue, philosophers are still ranged under the same
contending banners, and neither thinkers nor mankind at large seem
nearer to being unanimous on the subject[\ldots]}{J.S.Mill,
Utilitarianism} Kan, whoever\ldots} But this is the reason to despair
only if one assumes that there is an answer, {\em the} answer.  In a
sense, there is, because if only one avoids exaggerated
\co{precision}, things can be said.  But, of course, this means that
there is no answer, because avoiding such a \co{precision} means that
the answer has to be formulated anew again and again.  The attitude of
\co{openess} and \co{love} pays much more respect to the universal
\co{concreteness} of life, than ethical systematizations of
all possibilites in tables of goods and wrongs.

\pa\mine The values of the level of \co{miness} are relative to
\co{myself}, to the \co{I}.  As such, they transcend the horison of
\co{actual visibility} and express \co{general ideas}.  Things may be
beautiful or ugly, good and friendly or evil and malicious, right or
wrong.  Above all, I can perceive a person as good or evil, which is
quite different from perceiving the actions of the person as just or
unjust.  These values are recognized through soul's attitude expressed
in feelings like preference, appreciation, admiration, sympathy, love
or their opposites.  
%I call them the \co{eudaimonistic} values, or
%else the values of \co{soul} or of \co{miness}.

\subpa What we said about relativism in \refp{pa:relativism} can be
repeated here.  To the extent such values are not \co{founded} in
\co{love}, they appear as matters of private choice and individual
preference.  They are relative to \co{myself}, and {\em only} to
\co{myself}, to the individual who chooses them, or happens to live
them. Theories of ethics built on the identification of the good with 
the fullfilment of personal preferences have hard time with excluding 
terrorism and homicide carried out precisely in the name of such 
preferences.

Without \co{foundation} in \co{holiness}, they are indeed arbitrary 
and purely private. Being, on the other hand, \co{founded} on the 
higher level, they are no longer arbitrary, they can only be an 
\co{expression} of the underlying \co{love}. They may still aim at 
\co{my} happiness, \co{my} good life, but these are no longer 
dissociated from the all embracing attitude of \co{humility} and 
\co{openess}. In fact, they are merely \co{expressions} of this 
attitude, that is, \co{my} goals are merely goals of making this 
attitude \co{visible}. 
Recognition of the \co{shared origin} breeds recognition of the value 
and dignity of other people; beauty and insight are values 
\co{expressing} \co{openess}; compassion and benevolence are forms of 
\co{humility}; etc. These are not mere results of \co{my} arbitrary 
choice which can be dissolved in the relativity of other, even opposite 
choices. These are the \co{expressions} of the \co{absolute}, of the 
\co{invisible love}, and their innumerable variations are only 
\co{expressions} of its inexhaustible potential.


\pa \act The level of \co{actuality} involves us into \co{complexes}
of things and situations, and correlated values are, in general, the
values of \co{usefulness}.  These may be relative to the whole body
or, perhaps, the living organism.  They are not as punctual,
localized and \co{immediate} as the \co{sensous} values of 
\co{immediacy}, but they are neither the values of my whole person.  Instead,
they may be values serving health, increased sense of vitality, 
aesthetical enjoyment.  In a
broader context of social structure, these are values of social
usefulness and purposefulness.  A just, honest, intelligent, able
 person represents a value for a society -- is \co{useful},
in a way in which a person with opposite characteristics is not. A 
thing can have a value for \co{me} in so far as it is useful for \co{my} 
purposes, in so far as it serves \co{my} promotion, \co{my} carrier, 
\co{my} fame, 
or achievement of more particular goals, in short, \co{my} 
\co{Ego}. 
Reactive responses to such values are \co{impressions} like joy or 
 rage, feelings of courage and strength or fear and impotence.
I call these values \co{Egotic} (which, of course, must not be 
confused with egoistic) or the values of 
\co{usefulness}. 
%(To have a Greek word for this case, too, we might call 
%them \gre{aretetic} values, as they are indeed closest to the Greek 
%\gre{arete}, understood more as \thi{excellence} than as 
%\thi{virtue}.)

\subpa
Clearly, valued in dissociation from the higher levels, they will 
easily lead to egoism and promotion of \co{my} private interests 
without any consideration for other things or people, in a way 
simialr to Kierkegaard's aesthetical stage. 
But when \co{founded} 
in higher \co{openess} and \co{humility}, they will reflect the respect 
for things and people\ldots

\pa \imm The \co{immediate} values are related to the most \co{actual
experiences}.  When \co{unfounded}, they find their full expression in
a single moment, without any reference to higher levels.  The most
natural, perhaps the only, example would be sensous pleasures.  They
are relative exclusively to the general sensous nature.  A pleasant
thing will arise a sensous lust and, perhaps, an impulse directed at
this thing -- it embodies the value of the pleasant.  Unpleasant
thing, causing a bodily pain, or in any case a displeasure, will
cause, sometimes deliberate, but typically purely reactive and
instinctive, avoidance, withdrawal.  Extreme forms of hedonism, of
fascination with the senses and pleasure, in the deviant forms, with
pain, are examples of such values considered for \thi{their own sake},
as goods \thi{in themselves}.  What distinguishes them from the
\co{spiritually founded} values of this level is the exclusive
character of the project aiming merely and only at achievement of
pleasure.  But it is more of a mere possibility, in any case a rare
phenomenon, because one will almost always find oneself involved in the
levels of \co{actuality} and \co{miness}, which involvement will
modify the character of values of \co{immediacy}, will, so to say,
restrict their claims to absoluteness.  Such restrictions will not
differ much from the restrictions originating in the \co{spiritual
foundation}.
%I call these the \co{sensous} or
%\co{hedonistic} values.



\section{Evil}

\pa
Evil -- how could God create it? 



\subsection{Pain, suffering, evil\ldots}
\pa
Pain, suffering and evil are all different things\ldots

\levs{10}{of pain}
{(immediate pain) bodily}
{(actual pain) humiliation, prolonged bodily (Ego)}
{(personal) = \co{suffering}, tragedy}
{(spiritual) damnation}


\pa The deeper joy and the deeper satisfaction with life, the stronger
\co{communion}, the stronger sense of \co{participation}.  Pain, on
the other hand, \co{alienates}.  It doesn't individuate -- but
idividualizes, it does not turn \co{me} into a person, but reduces
\co{me} to a lonely atom unable to relate to anything except the pain
\co{I} am feeling.  Pain, more than anything else, spreads upwards the
hierarchy of Being.  Eventually, it may corrupt the soul\ldots It may
make it extremely difficult to say \yes.

\pa \imm There is the \co{immediate} pain, what we most naturally
associate with the word, a momentaneous sensation.  Any uneasy
sensation in the body, from slight uneasiness to extreme distress, 
proceeding from a derangement of functions, disease, or
injury.  Pain may occur in
any part of the body where sensory nerves are distributed, and it is
always due to some kind of stimulation of them.  The sensation is
generally referred to the peripheral end of the nerve, and in its 
\co{immediacy} it simply engages my nervous system into reaction of 
avoidance. 

\pa \act
Pain need not be a mere sensation. A prolonged bodily pain, like 
under torture, affects more than a bodily organ. It can affect \co{my} 
whole body, which is reduced because some of its organs are. 
\co{Actual} pain 
can also affect \co{my Ego} as, for instance, when \co{my Ego} is humiliated, 
when it suffers because it has not received due recognition or 
share, when, as we say, it has been inflicted pain by ingratitude or 
disrespect.

\pa \mine
Being exposed to a prolonged torture, \co{I} can actually cease to 
exist, \co{I} can renounce whatever was dear and true to \co{me} and 
accept whatever the torturer wants me to accept. Torture may break a 
person. In a more mundane context, any personal tragedy inflicts 
pain. A loss of a loved person is not an event of one day but a 
lasting suffering which may for years take away all joy of life and 
ability to enjoy its gifts. 

\pa \inv

\pa I will use the word \wo{\co{pain}} in the broad sense -- not merely as an
\co{immediate}, sensous pain, not merely as an \co{actual} pain of the
body or \co{Ego}, but as the general event which leads to the dimished
sense of \co{participation} which, again in the much broader than the
traditional sense, I call \wo{\co{alienation}}.  Suffering, personal
tragedy, the deep seated feeling of damnation are, too, forms of
\co{pain}.  It thus says itself that there are no \co{objective}
criteria for classifying what is and what is not \co{pain} in this
sense.  Certainly, sensous or bodily pain are undeniable and most
easily recognizable forms of \co{pain}, but focusing on them with the
exclusion of other forms is but an expression of \co{attachment} to
\co{visiblility}.


\sep

\pa
Pain is a fact. But it is important to ask: the fact of what? Sure, 
we encounter it in the world, sometimes, undeservedly exposed to it. 
But it is a 
fact of sentient being, of being capable of feeling. Feeling is -- we 
could say, by definition -- to be vulnerable. To be able to feel, is 
to be able to feel what \co{I} have not created and intended, is to be 
able to feel something which does not originate in \co{me}. Pain is
an extreme expression of this vulnerability. 

And so, we should be careful with condemning all forms of pain because
it always is also a witness of \co{my} involvement in something more
than \co{myself}, it is also a reminder that \co{I} am not alone,
eventually, \co{that I am}.  Certainly, masochism which seeks pain as 
the only, or in any case, the most intense and full form of
confirmation of its existence is a deviation.  But recognizing that pain is
such a reminder, \co{I} can easier discern something greater and
deeper than the pain \co{I} am \co{actually} feeling, \co{I} can
easier maintain the bond with the world in spite of the pain which
\co{I} wish \co{I} wasn't exposed to.

Pain may cause \co{evil} but also, like with Job, it may have 
purifying role. Just as it is natural to avoid it, one may sometimes 
discover that \co{pain} one had \co{experienced} was a source of 
\co{spiritual} development and, instead of crushing the \co{soul}, 
enriched it.


\subsection{Evil and this world}\label{sub:evilworld}

\pa One may be tempted to say: the very existence of \co{pain} and
suffering is evil.  But this is not a good way of speaking.  For the
first, I certainly do not share this view.  But more importantly, it
turns evil into an impersonal force, into an inexplicable presence of
some devilish, hidden power which perverts the things.  
It makes evil into some form of a basic, permanent, if not the first 
and fundamental principle. \citt{But evil is unstable}{Dionysius the 
Aeropagite, {\em The Divine Names}. IV:23}, even \citt{det demoniske 
er det plutselige.}{Kierkegaard} Evil is a break in the continuity 
of Being, is a break \co{alinating} one from the possibility of 
\co{participation}, from the continuity of \co{love}. It does arise 
from \co{this world}, but it hardly reaches the \co{other}.

\pa Certainly, \co{pain} may be an undeserved result of some
impersonal forces of the world, whether natural, physiological,
political.  And forces creating \co{pain} will be justly considered
undesirable.  But there is a fundamental difference between
undesirability and evil.  Facing something painful, undesirable, we
retain, at least in principle, the ability to oppose it -- we
recognize its inherently \co{visible} character, even if it
\co{actually} exceeds the power we feel we posses to overcome it. 
Calling it \wo{evil}, we not only acsribe it a power of some
metaphysical dimension, we also judge it, for evil means inherently
evil, evil to the bones.  Meeting numerous \thi{evil} things and
forces, we thus cast a shadow on the world and, eventually, may be
tempted to say \wo{the world is evil}.  All talents of Schoppenhauer
did not prevent him from (or, perhaps, inclined him towards?)  making 
this shadow his basic principle.


\pa It is hard to oppose it because, being a \co{general
idea}, neither it nor its opposite has any proof, any final,
convincing argument supporting or contradicting it.  \thi{The evil
world} is the world to which \co{I} do not belong or, in any case, do
not want to belong.  And looking for goodness other places, among
people, in the nature of man, or wherever is of little use.  As
Mi{\l}osz says it \citt{The divinization of Man, when one abhors the
order of the world as essentially evil, is a risky and
self-contradictory venture.}{Mi{\l}osz, (last sentence of)
``Dostoyevsky and Swedenborg'', in ``Emperor of the Earth''} It is a
statement which effects an ultimate alienation and, by the same token,
does not leave much space for forgiveness.

\subpa To be sure, I wouldn't object to calling the nazi state, or the
communism of Stalin, Pol Pot or others evil.  But all such
pronounciations have always been made from the perspective of another
aspect {\em of the world}, whether the world before, or the world
outside, which was not evil.  They were not expressions of the
\co{general idea} of \thi{the evil world}, but only of a particular
political system and its servants, emboding the evil tendencies of its
rulers.  As such, they never invited to a gnostic reconcilliation with
the inherent evil of the world but, on the contrary, were always
permeated by the opposition and understanding that this is something
which, sooner or later, has to change.\footnote{Bulhakov's {\em Master
and Margarita} is an excellent analysis of the evil inherent in
Stalin's communism which, however impersonal and undefeatable, is
actually an effect of a monstrous misunderstanding and alienation,
described with a great amount of irony and humor.}

\pa
\wo{Passivity increases our chances for evil, for becoming evil.} Is 
it a triviality? Yes, of course --  if we only assumed that \thi{the 
world is evil}.

\pa
\co{This world} is evil, but only to the extent we exclude from it 
completely the consideration of the \co{other world}. I do not follow 
Gnostics and Khatars in the simplistic identification.

\co{This world} is good\ldots

In \refp{compassion} I said \wo{let everything grow} - really, 
everything?

\subsection{Evil}

\pa \co{Pain} in itself is not \co{alienation}.  And I do not mean
some sick masochism which searches for painful experiences.  \co{Pain}
is an event, is a momentaneous or lasting \co{experience}, is
something \co{I} feel or recognize, in short, it is something
necessarily \co{visible}.  Even damnation is a merely \co{visible}
conclusion \co{I} draw, and perhaps, live, from a series of deeply 
\co{painful} \co{experiences}.  \co{Alientaion}, on the other hand, is
a state of negative separation, of the lacking sense of
\co{participation}.


\pa \co{Evil} is whatever causes \co{aliention}, whatever diminishes
the sense of \co{participation}.  In the derivative sense, by the
general way I just characterized \co{pain}, \co{evil} is anything that
causes \co{pain}.  But remember the objections, made in section
\ref{sub:evilworld}, against calling the world, or perhaps just some
event in the world, \co{evil}.  \co{Evil} is a \co{spiritual}
phenomenon, or rather, the state of \co{spiritual} destitution in
which \co{soul} becomes \co{alienated}, becomes a \thi{stranger in this
world}.  Although this last phrase may be, in some sense, applicable,
its underlying tone of \co{alienation} from \co{this world} makes it a
very poor expression.  \co{I} can indeed feel \thi{stranger}, \co{I}
can find \co{this world} the place of corruption and disappointment, 
whether as the Platonic cave or as the Baudelairean captive albatross.
\co{I} can even go as far as seeing it merely and simply as \co{evil}. 
But such \co{experiences} are already expressions of \co{evil} -- of
\co{evil} which devoured \co{my soul}, which made the world, the
richness of life's joys, feelings, and tasks an unbearable burden, a
place which is if not unihabitable then in any case hardly acceptable. 
In short, it made \co{participation} impossible.\footnote{One might
easily quote a long tradition of Empedocles, manicheism, gnosticism,
bogomilism, kathars, and also, in a more ambivalent manner, of
neo-Platonism, of Stoicism, of exaggerated ascetism, of all the
directions which, taking the \co{spiritual} renounciation of \co{this
world} in a one-sided fashion and considering \co{this world} as
essentially evil and laden only with encumbrances to salvation, were bound
to despise the joys and pleasures of human life.  Nietzsche, his
one-sidedness in the opposite direction notwithstanding, ceratinly
deserves the credit for showing possible mechanisms of sick
renounciation.
% Rom. VIII:12-13 - renounc. flesh (Paul)
\ccom{gullibility of Christians and trickstery of the priests
 see Lucian (from Samosata, 120AD), with cynic-turned-christian
 ``expert'' priest Peregrinus Proteus; p.26;
 also Celsus, The True Word [Doctrine]
 also Porphyry, Against the Christians (denouncing self-hatred and
 abnegation of flesh). Resentment seems to appear in Porphyry's view 
 of Christians:  ``they want riches and glory ... they are renegades 
 seeking to take control'', they `themselves were the worst sort of 
 people, moral invalids who (cf. Celsus) found security in their 
 common weakness' (this is by J.Hoffman)
 }
} 
Just like \co{participation} is an opening which makes the \co{soul} grow 
and find joy in all things, \co{evil} makes it shrink and withdraw -- 
not in a purifying step towards \co{humble love} as \co{pain} may 
sometimes do, but in a cold and, yes, always senseless fashion which 
says \wo{this is the end}, the end in itself.



\pa \co{Evil} may be, partially, caused by \co{pain}, it may arise as
a reaction to \co{pain}, it is, in fact, a common response to
suffering, but it is not the same as \co{pain}.  Often, the very lack
of \co{pain} may cause \co{paint} and breed \co{alienation}, the very
lack of opposition and challenge may breed corruption of the
\co{soul}.  The bordeds are \co{invisible}, and there is hardly a
thing or an attitude which could not, given a particular sensitivity,
cause \co{pain}.

\co{Evil} may be born between man and nature, it may be born between
men, but it is born {\em into} men.  The legends are full of \co{evil}
residing in the persons of hunchbacks, cripples, dwarfts, in all kinds
of not merely ugly, but unnaturally deformed people and creatures. 
Their deformity is -- symbolically -- an obvious reason, a clear sign of suffering and
\co{alienation} and, consequently, a natural place for the growth of
\co{evil}.

\pa\label{pa:noevilthings}
Things are certainly not \co{evil}, though they may be employed to 
\co{evil} purposes. Institutions and human 
arrangements may be \co{evil} to the extent they sow \co{evil} in humans. Just 
like prolonged \co{pain} may infect the \co{soul}, so also \co{evil} 
breeds \co{evil} and in this sense it has highly \thi{positive} 
character which the philosophers of goodness of all Being denied to 
recognize. Causing \co{pain} it enters the \co{soul} and \co{evil 
soul} can not avoid spreading the plague.
\co{Evil} is something that prevents \co{soul} from  
\co{participation}, and some \co{souls} may be so affected by 
the contingency and indifference of nature. But we do not learn 
\co{evil} from nature -- we learn it from other people.
It is primarily one person who can inflict \co{pain} on another in a 
way which appears not merely as an accident of nature but as 
something grounded deeper, in some \co{vague} principle of devilish 
flavor.

\co{Evil} is an attribute of a \co{soul}, of a person.  The
degree of \co{evil} is not dependent on the kind of \co{pain} one
causes, but on the level from which the motivation for it emerges -- 
shall we say, on the degree to which the \co{soul} is corrupted?


\pa\label{pa:ultevil} \inv \co{Evil}, in the strong sense, is a \co{soul} which
wishes pain for its own sake.  \co{Pain} separates and \co{alienates}, and
causing suffering does help to turn separation into the feeling of
\co{alienation}, of opposition, of defenceless.  Such a \co{soul} has to be
itself \co{alientated}, it has to pronounce \No\ with the full force -- as
Milton's Satan \citt{Evil, be thou my good.}{J. Milton, Paradise Lost IV:110} --
in an \co{act} of despair (exmpl\ldots), in an \co{act} of \co{pround} arrogance
(exmpl\ldots).  It is a \co{soul} which has accepted its own damnation, its
eventual loneliness, and which tries to derive from it a possible pleasure of a
lower order, to convince others, but primarily itself, that this is the ultimate
truth of the world and life.


\pa \mine 
\co{Evil}, in a weaker sense, is a \co{soul} which, causing
\co{pain} and suffering, does not necessarily do it for their own sake, but
which nevertheless accepts the necessity of suffering following from
its projects and \co{actions}.  One can have various ideas and ideals
which, sort of unintensionally, cause suffering.  I do not know the
people, but I suspect that even Hitler or Stalin might have been of
this sort -- \co{souls} full of misunderstanding and \co{pride},
\co{souls} full of worship for idols which they put in the place of
the \co{invisible}.

\thi{Ends justify the means} is an example of that.  Power and riches,
giving the influence over the \co{complexity} of
\co{this world}, to the extent they may require impersonal decisions
based on statistical expectations, on average effects, on calculations
of possible goods and wrongs, of ``the net overbalancing sum total of
pleasure over pain'', result almost (but only almost) inevitably in
some \co{evil}.  If \co{my} considerations and \co{reflections} do not lead
to the certainty that the \co{action} will not cause any \co{pain}, if
\co{I} remain in doubt and end up calculating possible goods against
possible wrongs, then \co{I} act above my moral abilities, \co{I} act
with \co{pride}, even if \co{I} assume all apparent humility of
insecurity, doubt and mere hopes for the eventual good.  There is no
golden rule for the excercise of power.  But the more power, the more
chances for mistakes and for \co{evil}.  Its difficulty and responsibility
will cause nothing but fear in a moral \co{soul}.  \citt{For unto
whomsoever much is given, of him shall be much required: and to whom
men have committed much, of him they will ask the more.}{Lk.  XII:48}

There are people, whose character had acquired features making 
them almost instinctively wicked, who \thi{naturally} \co{act} in the ways 
causing \co{pain}. 


\pa \act
\co{Evil}, in a still weaker sense, is a \co{soul} which causes \co{pain}, so to 
speak, by carelesness or cowardice. It may recognize that it causes
\co{pain} which it shouldn't cause, but, because of 
its position, because of its obligations, because of the expectations 
of the surroundings, it feels forced to follow a course of action which, it 
knows, will cause \co{pain}. Indeed, this may 
easily be, at the bottom of it, a good \co{soul} which only makes a 
wrong choice and causes \co{pain}.

\pa \imm
In the extreme case, we might perhaps call even stupidy \co{evil}. 
Stupidity which simply does not see what it does, does not see that it 
causes \co{pain}, does not realize that its actions are harmful to others. 
Children may display this form of \co{evil}, half-consciouss, 
half-innocent and, if it wasn't blamelessly unintended, it would be 
entirely rotten, unimaginably inventive, malicious satisfaction of meaningless pain.
In the case of absolute debility, such a \co{soul} may be entirely 
innocent. And yet, if devoted to the pattern for harmful actions 
which, perhaps, even bring it a sick satisfaction, it is, in spite of 
its debile innocency, \co{evil}.

\say \co{Evil} \citt{is rooted in the whole earthly condition of men;
it is the weakness and error of the spirit parted from its
origin.}{after Harnack, {\em The History of Dogma}, p.~365, with
reference to Origen's view of evil.  [in Edinger, Psyche in Antiquity
II, p.~102]} It \thi{is rooted in} this condition, not equivalent to
it. The \co{attachment} to \co{this world} may be interpreted, 
following Origen, mystics and, eventually, even Heidegger as the 
\thi{fall}, as \thi{original sin}, but such an interpretation must 
keep in mind that \co{attachment} is not an ontological state of 
being but only a possibility of \co{spiritual} absence -- eventually, 
it is only a possible attitude of an individual and \co{evil} is only 
a measure of the degree to which this attitude consents to 
\co{alienation}\ldots This consent has little to do with coonscious 
choice \ldots


\subsub{The emergence of evil}
\pa
Is human nature {\em intrinsically} good, or {\em intrinsically}
evil? 
If we accept \co{evil} as an opposite of \co{holiness}, as a mere 
alternative at the same level, at the same depth of human being, then 
the question may have no answer. 
But if we do not, if we consider the \co{choice} of \No\ as a 
privation, an inability to say \yes\ (\refp{pa:notno}),  
then we see that it does not reach the eventual depth of human being. 


\pa\label{pa:noevil}
I am not sure, if \co{souls} of ultimate \co{evil}, \co{souls} like 
indicated in \refp{pa:ultevil}, exist. And even if they do, it is 
wise to observe their unintensional involvement into the ultimate 
unity of life. The devil, Woland, in Bulhakov's {\em Master and 
Margarita}, causing all the pain and confusion in the \co{alienated} 
city and state is, indeed, almost an acceptable personality. Goethe's 
Mephistofeles is \citt{the force which always wishes evil and does 
good.}{Faust} In the deepest confusion and disappointment, one is 
hardly capable of an \co{act} of ultimate despair, that is, of 
admitting that one is not worth of {\em any} \co{participation}.
One will create \co{idols} and endow them with 
\co{absolute} validity and meaning, but this very endowment witnesses 
to one's, no matter how confused and desperate, search for the 
tranquility of \co{participation}, for something ultimately 
worthy and good. 

% was...... History knows Billy the Kid, Nero, ....

\subpa Let's be frank.  No matter how sick and degenerate these (and
even more, other) statements of de Sade's may seem, there is no doubt
some truth in it.  One does not have to be a sadist to recognize the
intensity of experiences furnished by the perverse acts Marquise
describes.  Sensations, and sensations of pain in particular, do
provide moments of extremely intense \co{experiences} --
\co{experiences}, that is, moments of a meeting with something we are
not, something which we encounter, something we find as \thi{being
there}, \thi{coming to us from out there}.\footnote{At this point, the
diffference between masochism and sadism is not crucial.  Although we
might feel that masochism is \thi{less evil}, it is equally sick
attitude motivated by the intensity of pain which, for the reason of
its unmistakable intensity, becomes a \thi{good in itself}.} And this is the whole point.  Where does the thirst for the
``keen and active sensations'' come from?

The ``strong
feelings'' arise in the moments of exceptional intensity, whose
shadowy origins envelop the \co{soul}.  The search for this intensity
of an \co{actual} moment, the thirst for the unmistakable \thi{truth}
of such \co{experiences} is grounded in the \co{thirst} for
\co{participation}, albeit it has been reversed in the underlying
feeling of ultimate \co{alientation}:
% all born alone...
But although \co{alientaed} from the world and others,
although unable to see the value of the \co{spiritual}
\co{participation}, it still looks for a justification, for a form of
participation in some \thi{truth}:
% was no guiltier than Nile...
But the reference to the \thi{nature of things} is, in this
case, a very feeble argument which even the person pronouncing it
knows to be but an excuse.
% I do not care for others who do not approve of my manner of thinking



\pa 
This \co{attachment} to \co{myself} is here but another side of that 
\co{attachment} to the \thi{nature} which is brought to justify and 
explain that this is \wo{the way I am made}.
It is the fundamental thirst for the \thi{truth}, 
for something which would overpower \co{me} with its unmistakable, 
undeniable presence and intensity. It is most real, as real as any 
misunderstanding may be.
De Sade exemplifies a sick form of searching for this \thi{truth}, a 
form which has found some \co{visible} means of convincing oneself of a 
contact with it, of participation in it. 

\pa
More commonly, this \co{thirst} is just the \co{thirst} for \co{participation}
which, in the most \co{virtual} form of love, is simply the \co{thirst} for
being loved.  The first form of goodness is the \co{thirst} for
\co{experience} of goodness, the first form of acceptance is the
\co{thirst} for being accepted.  One may call it \wo{selfish}, but only if
one dismisses the distinction between \co{myself} and \co{my Self},
only if one isists on viewing the world as a pure dichotomy of
\thi{subject} and \thi{object}.  Such a \co{thirst}, even as
\co{experienced} by \co{myself}, is a \co{thirst} for something entirely
different than what the \co{visible world} of \co{miness} can offer. 
Such a \co{thirst} \co{reflects} the underlying intuition of the reality of
that for which it \co{thirsts}, an \co{inspiring presence} of something
\co{invisible} and which, viewed merely in the categories of
\co{miness}, disappears as \co{nothingness} and, at most, 
searches for some \co{visible} \co{manifestations}.  For most people,
it will be expressed as \co{thirst} for
\thi{goodness}, \thi{happiness}, \thi{love}, or whatever \co{vague}
and indeterminate words we might use for it.  The humble
defencelessness and dependency of small children tells us the same
story, even -- or, in fact, especially -- when we do not try to
psychologize it in any way.

\thesis{The very \co{thirst} for \thi{happiness}, for \thi{ultimate
peace}, for \thi{love} and for \thi{being loved}, for \thi{acceptance
and understanding}, for \thi{deep joy of life} is an expression of the
\co{virtual} goodness.} 

\pa
Nietzsche's confession, in his unfinished
and posthumously published autobiography, provides both an
illustration of this and a fundamental clue to his thinking:
\begin{quote}
\citt{But since there was no love in my age or in my private life, I 
could not conceive of any cosmic Love rooted in man's members, as 
Empedocles put it, and the cosmic conflict between love and strife 
which harmonized itself in the process of dynamic living, became for 
me strife alone, the sheer brutality of social Darwinism. It was Lou 
Salome who punded away at her Tolstoyan thesis of love's hegemony 
over hate, a thesis that Empedocles himself expounded and in which I 
lost faith when I was exposed as a child to the frost-bitten 
Puritanism of Naumburg with its chilling atmosphere of prudery and 
decorum.

In her arms I could well believe with Empedocles that cosmic love was 
rooted in my own members and vouched for itself \ldots

The legend-makers saw Empedocles plunging into the belching flames of 
Aetna, but this fate was reserved not for the great pre-Socratic, but 
for me alone. Having been separated from the love of my life, the love 
that made me human, I made my desperate plunge into the fires of 
madness, hoping like Zarathustra to snatch faith in myself by going 
out of my mind and entering a higher region of sanity -- the sanity 
of the raving lunatic, the normal madness of the damned!}{Nietzsche, {\em My Sister and I}, p.113 ???}
\end{quote}
%
\pa
If you suspect that you are thoroughly evil person, ask
yourself if you, somewhere -- past the bottom of your heart, past the 
limits of the thoughts you hardly ever allow yourself, in the
\co{invisible} and suppressed depths of your \co{soul} -- do not feel
such a \co{thirst}. And if you do not then, if you wouldn't like to feel it\ldots
Before you call somebody else evil, ask yourself the same question about 
him.
\citt{To our consciousness there is usually a residuum of worth left 
over after our sins and errors have been told off -- our capacity of 
acknowleding and regretting the is the germ of a better self \la{in 
posse} at least. But the world deals with us \la{in actu} and not 
\la{in posse}.}{W. James, The Varieties of Religious Experience, 
VI\&VII, footnote 7, p.138}

I am far from calling de Sade a good person, but I am also unable to 
ignore the importance of this \co{virtual presence}, of this \co{thirst} 
which reaches the \co{visible} surface from the bottom of one's 
\co{soul}. The fact that it may find sick expressions does not 
diminish its fundamental character.



\pa This \co{thirst}, in all its variants, has a passive character, it is a
\co{thirst} for \thi{an experience of \ldots}, for \thi{the state in which
\ldots}, a \co{thirst} for \thi{being given \ldots}.  This apparent
passivity, which makes \co{attachment} turn away from it with pitying 
disgust, may be, however, also the deepest \co{spiritual} activity, the
deepest \co{spiritual} presence of mind directed towards something
\co{invisibly} intense, impossibly \co{hopeful}, \co{absolutely}
valuable.  It is a \co{thirst} for a state, for being, or rather Being, for
something which does not deny the temporality of \co{this world}, and
yet, is \co{above} it.\footnote{According to a long tradition this is,
in fact, {\em the} way in which God \thi{moves} all things -- by being
desired. (Heidegger might say that this is the \thi{deficient mode} of God's
presence.)  This tradition certainly involves Plato and neo-Platonics,
but the exclusiveness of this form of \thi{moving} as God's only
action in the world, comes probably more clearly forth in Aristotle,
St.~Augustine, St.~Thomas.  I prefer the word \wo{\co{thirst}}, becuase
\wo{desire} brings in this context implausibly active and volitional
connotations. Areopagite's formulation \citf{[E]vil cannot even in any 
wise exist, if it act as evil upon itself. And unless it do so act, 
evil is not wholly evil, but hath some portion of the Good whereby it 
can exist at all.}{The Divine Names, IV:19}}

We do not have any general rules; we do not know all the possible times
when and all the possible reasons why such a \co{thirst} may die, if it
dies, but we will hardly deny its fundamental \co{presence}.

\pa
Certainly, the gradual engagement into the \co{visible world} may, if
not quench it, then disperse it and make one ignore it. 
\co{Attachment} will simply see it as a \thi{dream}, an \thi{unreal
dream}.  But the engagement into \co{this world}, under passive or
active \No, does not annihilate it nor, in and by itself, forces one
to renounce it.  I think that it never disappears, until it is
satisfied.  \citt{[T]he desire for the bliss, which [the soul] had
lost, remained with her even after the Fall.}{John Scottus Eriugena,
`Periphyseon' IV 777C-D [88]} But the ways by which one may omit it, 
the multitude of sick formations which can arise from its denial are 
inumerable. 

%%[ removed 230-233]

\pa One has invented many cures against \co{pain}.  De Sade's active
acceptance and worship is an extreme possibility.  A resigned
acceptance, recasting the whole world in terms of hardship and
suffering, \la{a l\'{a}} Shopenhauer, is also a possibility.  Yet
another, is a resigned acceptance which does not blame the world, a
stoical acceptance, which merely calls forth the need to put up with
\co{pain}.  Madness, too, is often a cure against \co{pain}.  As
Dostoyevsky says \citt{One day, man will go mad to prove that he is
free.}{???} Apparently, it concerns freedom, but the need to prove it
is but the need to liberate itself, it arises from the deep, long
lasting \co{pain}.

\wo{If nobody cares what will happen to me, why should I care?} 
\wo{If the world is not good, as I expected, then I have to find an 
isolated place for myself, away from the world}, all forms of 
disappontment, in the moment they are \co{experienced} as factual, 
\thi{objective} truths about the world, breed an \co{alientation} 
which may lead to an acceptance, since one does not want to oppose 
the \thi{truth}.

Psychoanalysis studies details of processes which, starting with
\co{painful} experiences, infect the \co{soul} of a child.  Sometimes,
it may be helpful to uncover the forgotten sources of hidden
\co{pain}.  But if the \co{pain} has settled so deeply in the
\co{soul} that it actually permeats the whole world, it may be
necessary to \co{experience} this world anew, to encounter people
and situations which not only set the old experiences in a new mental
dressing, but which actually yield a new experience, which show a new
side of the world, a side which has remained not only hidden but 
buried under the unhealed wounds.


\pa\label{pa:evilpoints}
I have not used so much space for talking about \co{pain} in order to
arrive at psychoanalysis or other psychological tricks for avoiding
\co{pain}.  The point was: 
\begin{itemize}
\item \co{pain} is, or at least begins with, 
\co{visible} events.  
\item Nature, and lower levels of Being may cause
\co{pain}, but its deepest form -- the form reaching most directly the
very \co{soul} of the affected person -- is the \co{pain} arising 
from interactions beteen people, sometimes other people, most often, 
other people and the person experiencing \co{pain}.
\item The growth of \co{pain} (through suppression, because of its 
natural causes, because of its frequency, because its depth) may, eventually, 
dominate the whole \co{soul}.
\item The reaction to \co{pain} is withdrawal, distancing, avoidance, in the most 
abstract sense, \co{alienation}. Such a form of \co{experience}, however, is 
just a passive \No.
\end{itemize}

\pa
Thus, in general, \co{evil}, which is a possible reaction to \co{pain}
may arise from the natural grounds, from physical suffering, sickness,
deformity.  But primarily, \co{evil} is born between men, and it is
born into them.

Eventually, it is the question of how \co{I} comport with \co{pain}, 
where \co{I} choose to see its source, to whom or to what \co{I} 
ascribe its cause, and how much intensionality \co{I} want to see 
behind it. An \co{evil} person will necessarily see a lot of 
\co{evil} in the world, because he himself experiences his attitutde 
as a reaction, as a well motivated and even justified response. In this 
lies the essential unfreedom of ultimate \co{evil}, that it is bound 
to see itself as a mere reaction. The only cure agains such a feeling 
of unfreedom, of \co{alienation} from the world and 
its metaphysical \co{evil}, seems to be to accept this \co{evil} as one's 
ultimate truth and goal. Let's emphasize again -- even that 
witnesses to the deep need of \co{participation}, misconstrued as it 
might be.

\pa True, the points from \refp{pa:evilpoints} do not necessarily
cause evil in the common sense of the word (whatever it is).  A person
may become defensive and closed, insecure and scared, but not
necessarily evil.  This, in itself, is \co{evil} which happened to
this person, but it does not mean that the person became \co{evil}. 
But I do not think it is worthwhile to try to draw all to precise
distinctions.  We never know if the \co{pain} we are causing will
result in \co{evil} person, only \co{evil} experienced by the person
or, perhaps no \co{evil} at all.  But I find \co{pain} which we,
intensionally or not, consciously or not, cause, as well as \co{pain}
which is caused by other, non-human factors, to be the only fundament
on which \co{evil} possibly may develop.  We never know \co{precise}
effects, and the very \co{vague imprecision} of our knowledge in this
respect should make us cautious.  The only thing we can and should do
is to try, in any \co{concrete} situation, to avoid mistakes and to
communicate all the \co{love} of which we should be capable.  This may
be \co{nothing} in terms of \co{visible} criteria, but it carries all
the weight and is capable of perfect recognition by the involved
persons.


\subsub{God, creation and evil}

\pa \co{Evil} is the event of \co{visible world}, it emerges only as a
result of a passive or active \No, of not merely ontological
\co{separation} but of axiological dissociation from the attitude of
\co{love}, as the \co{alienation} in response to \co{pain}.

The more 
active and \co{reflected} the denial, the more intensely and deeply 
the \co{evil} can permeat the \co{soul}. The experience of \co{evil} is always 
\co{an experience} of something evil, of something particular which, 
causing \co{pain}, makes us withdraw, perhaps even escape. Although 
arising from the sphere of \co{visibility}, \co{pain} has the tendency 
to spread and infect the \co{soul} which, eventually, may make it feel 
that \co{participation} is impossible. 


\pa
\co{Pain} is a fact, while \co{evil} is but the failure of
\co{incarnation}, is but the absence of \co{love}, is just the
helplesness of a human being who, searching for the \co{visible}
reasons, remains blind to his \co{orign}.  
This helplesness consists
in the \co{attachment} to the \co{visibility} of \co{pain}, and then
rasing the \co{pain} to the level of a principle, of \co{evil}.

The clearest expression of this \co{attachment} is the 
question ``Who is responsible?''. Such a question not only has accepted 
\co{pain} as the basic reality; it has also assumed that it is 
insufferable, perhaps, incommensurable with human dignity; then, that 
it is hardly distinguishable from \co{evil}; and, most importantly, 
that there must be somebody, something responsible for it.
The very idea of such a responsibility, of a sufficient reason for the 
audacious and unacceptable \co{actuality} of \co{evil}, is an express 
statement of \co{attachment}. If \co{pain}, even \co{evil}, is 
present all over the world, then God, who created this world, is the 
one eventually responsible for it.

\pa
Shall we remind what has been said earlier about creation?
Every \co{birth} is a creation -- a creation of a new world, 
which proceeds only through me. The \co{origin} is the source of 
everything which emerges in the world, but all that emerges only 
through me. Godhead \co{incarnates} and then creates, but only through us. 
God's \co{omnipotence} consists in being the \co{origin}, not a 
\co{precise}, detrmining, sufficient reason of every \co{visible} 
thing. The {\em only} action of God in \co{this world} is to fill the 
\co{soul} with \co{thirst} for \co{participation}, is to \wo{be desired}. 
To find this \co{participation}, to respond actively to the 
\co{invisible incarnation}, is the ultimate good man can achieve -- 
\co{evil} may emerge only as a failure of \co{visibility} to find the 
\co{spiritual} ground of its being.


\pa
The ontological founding of the \co{visible world} involves also 
\co{pain}. But this is not the full \co{incarnation} of Godhead. 
\citt{You should know [God] without image, unmediated and without likeness. 
But if I am to know God without mediation in such a way, then "I" must
become "he", and "he" must become "I".  More precisely I say: God must
become me and I must become God, so entirely one that "he" and this
"I" become one "is" and act in this "isness" as one [\ldots] 
%for this "he" and
%this "I", that is God and the soul, are very fruitful.
}{Eckhart, (p.  238)}
%
It 
is only the \sch\ and the \co{spiritual love} which make Godhead 
\co{present}. 
\citt{Gott ist die Liebe, und wer in der Liebe wohnt, der wohnt in
Gott und Gott in ihm.  Gott wohnt in der Seele mit allem dem, was er
ist, und alle Kreatur.  Darum: wo die Seele ist, da ist Gott, denn die
Seele ist in Gott. Darum ist auch die Seele, wo Gott ist, es sei denn,
da{\ss} die Schrift l\"{u}ge.  Wo meine Seele ist, da ist Gott, und wo Gott
ist, da ist auch meine Seele, und das ist so wahr als Gott Gott ist.}
{Eckhart, {\em In dem Pers\"{o}nlichen Wesen}.}


\pa God's \co{presence} is a feeble affair of the feeble human
\co{soul} and \co{spirit}.  \citt{Man could not without God, and God
should not without man.}{Theologia Germanica, III} Indeed, what would
God be without the human \co{soul}?  \citt{Nay, if there ought not to
be, and were not this and that -- works, and a world full of real
things, and the like, -- what were God Himself, and what had He to do,
and whose God would He be?  Here we must turn and stop, or we might
follow this matter and grope along until we knew not where we were,
nor how we should find our way out again.}{The.  Germ.  XXXI}
Everything seems to indicate that the \thi{Anonymous Teutonic Friend
of God} actually found his way out, but the \co{reflective} insistence
on some form of \co{objectified} being might not.  Following this
theme from Eckhart,\footnote{Surely, not only Eckhart.  There are many 
examples like \citf{He
became man, that we might be made God.}{St.  Athanase, Orat.  de
Incarn.  Verbi, I}, which find rather meager justification in a single 
passage from the scripture \citf{Ye are gods; and all of you are children of
the most High.}{Ps.  XIX:82:6}.} Rilke writes
{\begin{verse}{\small{\em Was wirst du tun, Gott, wenn ich sterbe?  \\
Ich bin dein Krug (wenn ich zerscherbe?) \\
Ich bin dein Trank (wenn ich verderbe?) \\
Bin dein Gewand und dein Gewerbe,\\
mit mir verlierst du deinen Sinn.

Nach mir hast du kein Haus, darin\\
dich Worte, nah und warm, begr\"{u}ssen.\\
Es f\"{a}llt von deinen m\"{u}den F\"{u}ssen\\
die Samtsandale, die ich bin.

Dein grosser Mantel l\"{a}sst dich los.\\
Dein Blick, den ich mit meiner Wange\\
warm, wie mit einem Pf\"{u}hl, empfange,\\
wird kommen, wird mich suchen, lange --\\
und legt beim Sonnenuntergange\\
sich fremden Steinen in den Schooss.

Was wirst du tun, Gott? Ich bin bange.\footnote{Rilke, {\em Das Stunden-Buch}}
}}\end{verse}
%
God is \co{incarnated} or dead, and he lives \co{incarnated} {\em
only} in the human \co{soul}.  \citt{So wird die Seele eine himmlische
Behausung der ewigen Gottheit.  So da{\ss} er seine g\"{o}ttlichen
Werke nun in ihr vollbringt}{Eckhart} This living \co{presence} of
God's in the \co{soul} is not a fact, an \co{objective} truth -- it is
only the possibility of \yes.  Without Godhead's \co{nothingness},
there would be no \co{me} and no world.  But without \co{me}, without
the place where \co{love} can \co{incarnate}, there would be \ldots no
God, or else, God would have to remain \co{One}, a mere principle,
perhaps, a \co{reflective} abstarction, \thi{the first mover} or
\thi{the ultimate cause}.  It is said about God \citt{I love them that
love me}{Prv.  VIII:17}, but, in fact, God's life is {\em
nothing else} but this \co{love}.  If \co{I} deny this \co{incarnation}, if
\co{I} oppose it, then what can God do?  His \co{command} leaves
\co{me} free, it always leaves the place for saying \No.  And if
\co{I} say \No, if \co{I} die -- ``What will you do, God?  I am
worried.''

\pa God's \co{transcendent presence} is the \co{virtuality} of the
\co{origin} where everything begins.  This \co{virtuality}, being the
most \co{immanent}, most \co{concrete} center of a person is, at the
same time, the ultimate \co{transcendence} which, permeating all
\co{actuality}, is never embraced by it.  The rest of creation
proceeds, however, through the stages of gradually increasing
\co{visibility}, through \co{my Self}, \co{myself}, down to the lowest
levels of Being, down to the most \co{actual} \co{distinctions}.  The
\co{birth}, the very seed of creation, does not determine these lower
levels, it only conditions their possibility.

When \co{I} return to God, He is no longere a mere Being, a mere 
\co{One}, because such a return happens only through the thoroughly 
personal \co{humility},
\co{thankfulness} and \co{openess}. It is not a \co{truth} which can 
be proved but an \co{active} acceptance. Nobody has to do it and 
nobody can force anybody to do it. But if \co{I} do it, then God 
becomes the constant event of \co{my spiritual} life, the 
\co{incarnated}, \co{immanent} \co{presence} of \co{love}. 
Up
there, in the \co{concreteness} of the utmost \co{immanence}, in the
depth of the \co{soul}, there is no \co{evil},
there is as yet no \co{evil}.  At worst, there is the \co{thirst}.
\co{Evil} is precisely the opposite of
this \co{concreteness}, is precisely the denial of \co{incarnation},
is the absolutization of the \co{attachment} to \co{visibility},  
when it \co{proudly} pronounces its \co{self-sufficiency}.  It
may, perhaps, happen through an active \No, but for all we know, it is
but a \co{visible} misunderstanding, or a misunderstanding of the
\co{visible}, an unfulfilled \co{incarnation}, an accident, a possible
side-effect of \co{pain}, in short, a mistake on our part.


\subsection{The attitude}

\pa \co{Love} abolishes \co{evil}, though not necessarily \co{pain}. 
It simply can not see any \co{evil}, it can only see \co{painful}
misunderstandings, between people, between people and the world, 
between a person and himself.  The
\co{pain} which it encounters, is no longer a \thi{bad thing}, an
\thi{unacceptable offense}, but is also an aspect of Being which has
to be met with \co{humble openness}, that is, the \co{openess} which 
does not draw \co{distinctions} between \co{love} and \co{the world}. \citt{If all was right with you,
your sufferings would no longer be suffering, but love and
comfort.}{Eckhart} Forgiveness, an aspect of \co{openness}, is the
ability to suffer without polluting one's \co{soul} with hatred. 
It does not remove suffering, only the danger of \co{alienation} which
lies in it.  To forgive means to see no \co{evil}.

\citt{A good man out of the good treasure of the heart bringeth forth
good things: and an evil man out of the evil treasure bringeth forth
evil things.}{Math. XII:35}
If we want to keep using the religious language, God is the 
\co{incarnated love} which knows \co{pain} but not anything \co{evil}. 
But this \co{incarnation} happens {\em only} in the \co{soul}, in 
human beings. 


\pa It is the attitude to \co{pain} which, to a high extent,
determines the reality of \co{love}.  \citt{True suffering is a mother
of all the virtues.}{Eckhart} We need not, perhaps, go that far, but
since \co{pain} is the \co{visible} fact, then the \co{spiritual}
strength, consisting in the ability to keep the deep sense of
\co{participation}, has to be prepared for suffering, for \wo{true
suffering}.  The \wo{true suffering} is precisely the ability to
suffer in full \co{humility}, without blaming the \co{invisible}
causes, without searching for sufficient reasons, for \co{evil}
agents, without polluting one's \co{soul} with hatred.  This
\co{humility} does not even mean that we have to accept the suffering,
that we have to agree to it.  No, we can try to abolish it, we can try
to work against it, we can try to change it.  But in all these
endavours, we only try to change the constellation of the \co{visible
world}, to re-arrange the \co{visible} things, so that \co{pain}, if
not disappears, then at least becomes bearable.  Abolishion of
\co{pain} provides only a possible \co{visible} goal for our
\co{actions}.  Whether the \co{actions} are succesfull or not, this in
no way affects the \co{incarnation} of \co{love}, the \co{invisible
incarnation} of God. 
\co{Love}, spending its time on avoiding mistakes which 
might result in \co{pain}, is beyond good and \co{evil}, because this 
opposition, this \co{distinction} does not apply in its world. 

\ccom{\citt{Was Wesen hat, Zeit oder Raum, das geh\"{o}rt nicht zu Gott, er ist 
\"{u}ber
dasselbe; was er in allen Kreaturen ist, das ist er doch dar\"{u}ber; was
da in vielen Dingen eins ist, das muss notwendig \"{u}ber den Dingen sein.}
{Eckhart, {\em Was ist Gott?}.}
}

\pa\label{pa:noYouShalt}
\thi{Beyond good and evil} means that there is no
\thi{ought}, no \thi{You shalt}.  All such ethical imperatives seem
needed only when one accepts the reality of evil in the world and,
in particular, in human nature and behaviour, and thus finds oneself
opposing the world \thi{as it is} to how it \thi{ought to be}. One may 
genuinely try to follow ethical principles of \thi{ought} in one's life, 
and yet, the principles themselves are always only 
solemn summons, sermons to the fellow humans. 

There is no \thi{ought} in the attitude \co{founded} on \co{love}
because what I do then is entirely and thoroughly for \co{my} own sake
-- I do not need any external imperatives, any constant reminders, any
\thi{You shalt not}.  I assume such an attitude not because I
\thi{ought to} but simply because I know that this is the best thing I
can do.  And, as it happens, whatever I do for \co{my} own sake is
then indistinguishable from the sake of the \co{communion} with others
and the world. 

Sure, others may not see the value and reward which such an attitude
carries in itself.  But for them (as well as for me) all ethical
summons and principles, all \thi{ought}s remain mere \co{inconcrete}
abstractions.  No matter how plausible they may be, no matter how much
rational consent one may give to them, they never become \co{mine} but
remain foreign; they may suggest a way of \co{action} (althoough 
hardly any ethical principles actually do that) but, being
ultimately justified exclusively by \thi{ought}, they will never
affect \co{my} being.



\pa
Thus, I would not go so far and say so precisely as Aeropagite that 
\co{evil} is \citt{a deficiency and lack of the perfection of our 
proper virtues}{The Divine Names, IV:24}, that for a being \citt{evil 
is the destruction of its nature, the weakness and deficiency of its 
natural qualities, activities and powers.}{Ibid, IV:25.} 
Indeed, a reduction of one's qualities, perhaps, as a 
consequence of an accident, is not an \co{evil} in itself.
Such things may 
be sources of \co{pain}, but in the age of genetic manipulation, the 
idea of \thi{natural qualities} and \thi{proper virtues} may be hard to 
defend, even if recognition of such qualities may, indeed, be an expression 
of respect for \thi{the order of things}.\footnote{What is that? I do 
not know! \citf{Now there
is only one way of taking care of things, and this is to give to each the
food and motion which are natural to it.}{Plato, Timeaus, my:III.48}}

And, on the other hand, it does not sound entirely correct that 
\citt{every entity, even if it is a defective one, in so far as it is an 
entity, is good. In so far as it is defective, it is evil.}{St. 
Augustine, Enchiridion, IV} I can hardly see anything good in the 
\thi{entity}, that is, in the mere existence of, for instance, a 
concentration camp, even if its buildings, in principle, might be 
used for other purposes. 

\co{Evil} and good are all too \co{vague} categories to classify things; 
first of all, with respect to things, they do not exhaust all 
possibilities. The assumption of their \co{absoluteness} smells all 
too narrow ethicism.  
\co{Evil}, in the 
definite sense, is only \co{aliention} from the \co{origin}. What, 
\co{precisely}, leads to such an \co{alientation} is the matter of 
personal \co{experience} and sensitivity, that is, has to be 
determined anew in every \co{concrete} situation.

\ccom{cf. 1st sentence of \refp{pa:noevilthings}}
%%%

\say
Godhead creates from \co{nothing} bringing forth the \co{virtual} 
contents, the first, \co{invisible} \co{distinctions}. When left for 
themselves, they may find any kind of \co{actualizations} in \co{my} 
life and actions -- they may turn good or bad, advantageous or not, 
all depending on the accidents of \co{my visible} goals and wishes. 
But being far \co{above} the \co{visible world}, they are also 
\thi{beyond good and evil}.

Human creativity, unlike God's, is to create something good, in the 
extreme cases, to create good from evil. This is the attitude of 
\co{thankful} \co{non-attachment} which \co{humbly} accepts the 
\co{gifts} of the \co{origin}. But this attitude is not a fact of our 
being -- it is a challenge. When taken up, it shows the goodness of 
God because, as a matter of fact, everything which emerges from 
\co{nothingness} can be turned into good. This, however, is no longer 
the creation of God Himself, but of His \co{incarnated presence}.

\citt{Although the divine incarnation is a cosmic and absolute event, 
it only manifests empirically in those relatively few individuals 
capable of enough consciousness to make ethical decisions, i.e., to 
decide for the Good. Therefore God can be called good only inasmuch as 
He is able to manifest His goodness in individuals. His moral quality 
depends upon individuals. That is why He incarnates. Individuation 
and individual existence are indispensable for the transformation of 
God the Creator.}{{\em C.G.Jung Letters}, eds. G.Adler, A.Jaff\'{e}, 
Bollingen Series XCV, Princeton University Press, 1975, vol.2, 
p.314; letter to Elined Kotschnig, [The Creation of Consciousness, p.93]}

\subsub{Responsibility}

\pa Responsibility is always responsibility for \ldots, for this or
that, for something.  ``Who is responsible for this horrible state of
affairs, for this outrageous mess?''  Nowadays, it is usually some
committee, behind which it is impossible to find a responsible person. 
But ``the outrageous mess and horrible state of affairs'', no matter how complicated and
involved its mechanisms, is something we \co{actually} see.  The very
talk about responsibility, responsibility for this or that wrong,
requires a \co{visible} state of affairs.  Any talk about
responsibility implies a \co{visible} object, and if the object
happens to be invisible or, perhaps, simply non-present, it brings it
down to the categories of \co{visibility}.  The question about
responsibility for \co{evil} is yet another example of the antinomy
(section \refsp{sub:antinomies}, chap.~I) -- of first \co{positing}
the totality of \co{evil}, and then applying the category of
\co{visible} cause, the responsible agent, to this posited totality.

\pa
Who is responsible for \co{evil}?  If
you ask the question, you are, if I ask the question, I am. The 
person asking this question is responsible. He is responsble for 
\co{positing} the totality of \co{evil}, and for positing a hidden agent, 
the evil spirit, who must have caused all this \co{evil}. 
\citt{He [God] shall not be questioned concerning what He does, but they shall
be questioned.}{Kor. XXI/23}
The person 
asking this question is responsible for seeing the \co{evil}. 

\pa
So, perhaps, ``Who is responsible for \co{pain}?''.  This, indeed, is
better.  What \co{pain}?  Which \co{pain}?  Wherever you see some
\co{pain}, you may be entitled to look for its causes, whether this
particular person who did this particular wrong, or else a course of
nature which resulted in this particular \co{pain}.  But if you mean
some totality, perhaps, \thi{the totality of pain}, then you are the 
only one responsible for this \co{positing}, for this 
\co{posited} \thi{totality of pain} which, perhaps, you want to see as 
\thi{the totality of \co{evil}}.

\pa There is no such thing as \thi{pain in general}, there is no
\thi{totality of pain}.  There is only this particular \co{pain}, 
the \co{pain} which \co{I} or another person is \co{actually} feeling. 
Sometimes, finding its cause may be helpful.  But often it is much
more helpful to ask what \co{I} can do to help.  If the \co{thirst} for
love has died, we may be unable to say exactly when and why.  But we
may offer the love which, opening the world, fills the emptiness
of the soul.  The love which not only avoids pain but which also 
knows how to accept it.


\pa \co{Love} does not see \co{evil}, even when it experiences
\co{pain}.  And this \co{pain} is always not only \co{conrete}, but
also \co{actual}, it is \co{visible}.  It has no measure, because we
do not know what it may cause, we do not know, how little \co{pain}
may cause how much \co{evil}, how little things may grow into
monsters, nor, for that matter, how great monsters may turn out to be
harmless, even advantageous.  There is no calculus of \co{pain}. 
\co{Love} embraces \co{pain} because it,
too, like everything else, is a \co{gift} which can be accepted with
\co{humble openess}.  Only Godhead creates from \co{nothing}.  Human
freedom, the freedom of \co{love}, is to create something good not only from
something else, but also from something \co{painful}, even \co{evil}. 
This is the highest form of human creativity, which art only 
imperfectly sometimes imitates.

\pa
The perfect \co{love} does not cause \co{pain}. But who is perfect? 
As we said, \co{pain} may arise from sources which we not only do not 
intend, but sometimes do not even suspect. Perhaps, it is impossible 
not to cause \co{pain}. \citt{There is none good but one, that is 
God}{Mat. XIX:17, Mrk. X:18} 
We want to think that we know what we do, but we do not know.
We gather some recollections from a 
distant 
past, and we realize how much \co{pain} our actions, unintensionally, 
have caused. We realize that, without knowing it, we hurt somebody or 
else, that what we did might have been understood completely 
differently, and seen from this angle, might have caused \co{pain}. 
We have considered all visible and foreseeable consequences of our 
action and then it turns out that somewhere, in the most unexpected 
corner of our calculations, a person who was not supposed to be 
there, got hurt. We do not know what we do. 

\pa And thus, we are responsible for things we have not done.  We are
responsible for all the \co{pain} we have not caused, for all the
\co{evil} we have not intended, for all the things we have not done. 
We want to say that we are responsible only for ourselves, but did we
make ourselves?  \co{I am not the master}, not only of the world, but
not even of \co{myself}. \co{I} find \co{myself} with the habits 
\co{I} acquired, with the patterns of thinking and acting \co{I} 
hardly chose by any deliberate act. \co{I} have not intended many 
things \co{I} have done, and yet \co{I} am responsible for them.
{\em This} is responsibility, the rest is hair-splitting, in constant 
attempts to find a compromise, to excuse \co{myself}, to convince 
\co{myself} that, at the bottom of it, \co{I} am a good person. But 
who says \co{I} am not? At the bottom of it, \co{I} only have a 
\co{vague} dream, an impossible \co{hope}, a \co{thirst} which can be 
dismissed, but which nevertheless surfaces whenever \co{I} want to 
emphasize that \co{I} am a good person. If \co{I} am concerned with  
morality -- and this, unless it is arrogance, means exclusively \co{my} 
morality -- \co{I} can only actively acknowledge and 
cherish this \co{thirst}. The rest, all the compassion, openness, 
helpfulness, tolerance and, not least, firm steadiness, are but 
consequences \co{I} will have to draw from this recognition.



\subsub{Forgiveness}


\pa Sin, \co{pain} inflicted on another is its own punishment -- it
immediately perverts one's being, diminishes \co{humility} and
\co{openness}, and thus, makes the \co{participation}, the tranquil 
\co{thankfulness} for the world and enjoyment of life lesser. In 
fact, it does not {\em lead} to such a lessening -- it {\em is} such 
a lessening. 
\co{Self-awarness} does not necessarily tell \co{me}, in a clear voice 
of disturbed conscience, \wo{Now you have done wrong!}. Its feebleness 
is precisely \co{my} possibility to ignore it and, in a longer run, to 
completely cloud such voices. But such a clouding and ignoring is but 
that -- clouding and ignoring something which only retreats deeper and 
deeper into the \co{virtuality} from which it \co{originally} 
emerged. At the bottom of \co{my soul}, there persists the 
\co{self-awareness}, if not of \co{me} doing wrong, then of something 
going wrong; the \co{self-awareness} of a break and a lack, of a 
\co{thirst} and
a removal from some \co{unclear} site of hoped for but, now 
gradually more and more unimaginable, satisfaction. 

This, of course, does not mean that it is immediately \co{visible}. It may take
time before the sin is punished in a \co{visible} way. Indeed, it may 
happen that it never will be. This  crux of all moralism and ethicism 
is irresolvable if we want to explain it, or explain it away, in the 
categories of \co{visibility}. If this is all we have, then not only 
infliciting \co{pain} may turn advantageous but, indeed, everything is 
allowed.

\pa If salvation of \co{my soul} matters to \co{me}, if \co{I}
recognize the deepest truth of meaningful \co{participation}, then
\co{I} know that this is the best one can \co{hope} for and the best
one can obtain.  If \co{I} am able to do that, then \co{I} do not need
any arguments to convince others that they, too, should live this way. 
\co{I} only hope for their own sake that they will.  In this respect
-- and only in this \co{vague}, entirely \co{imprecise} and thoroughly
\co{clear} respect -- what is best for \co{me} is also best for
others.  \co{I} wish them all the good \co{I} wish \co{myself}, and if
somebody fails, \co{I} can regret it is much as when \co{I myself}
fail.  There is no rebuke, there is not a slightest sign of
condemnation -- there is only a possible failure.  And when \co{I}
know that such a failure is its own punishment, \co{I} do not look for
more.  Above all, \co{I} do not ask how others might be made better,
what rules they should follow, or what rules we should apply to
distinguish the good guys from the bad ones, because \co{I} have more
than enough to do with \co{my} own imperfection. \co{I} should do all 
\co{I} can, but all \co{I} can do is to improve \co{myself}, not 
others. As far as others are concerned, \co{I} can only grieve their 
failures, as much as \co{I} can grieve \co{mine}.
I daresay, this is the {\em only} moral attitude.

\pa %\inv 
True forgiveness is simply to have nothing to forgive.  \co{Love} does
not see \co{evil}, it does not see \co{itself} either and hence, as it
cannot be offended or being done \co{evil} to, it has nothing to
forgive.  In any harm it may \co{experience} it sees either a
challenge to put up with it, perhaps, to repair it or prevent it from
happening again, or a failure on the part of the perpetrator.  This
failure is the more grave, the deeper from the \co{soul} it springs,
but it is always already its own punishment.

\pa %\mine 
Although in appearance it may be indistinguishable from another
form of forgiveness, it is entirely different.  Another form is a
forgiveness in a more mundane sense of the word, the recognition that
\co{I} have been treated unjustly, that \co{I} have been done some
wrong, but by an effort of \co{my will}, by \co{my} moral conviction 
and principle, \co{I} am able to
forgive.  This is the only form of forgiveness a moralist captured
into the \co{distinction} between good and evil can imagine. 
Accepting this \co{distinction}, however, it is impossible to say why
actually \co{I} should forgive something which everybody sees is a
wrong done to \co{me}.  One may argue in a pragmatic way by observing 
that, in a long run, such a forgiveness actually may pay off, may 
actually be advantageous. This is true and, equally, a misconstrued,  
pharisean form of forgiveness. This may be used for constructing 
ethical arguments, but it has nothing to do with human morality.

\pa
For morality has nothing to do with good vs. evil. Being \co{founded} 
in \co{love}, it is beyond this \co{distinction}, in the sphere where 
there is no \co{evil}. Therefore it shouldn't, perhaps, be called 
\wo{morality} since we are so accustomed to associate it precisely 
with this \co{distinction}. However, I only wanted to point out the 
force of \co{founding} which, rendered in the easier categories of 
\co{visibility} does amount to what we tend to call \wo{goodness} as 
opposed to \wo{evil} or \wo{wickedness}. 
\co{Love} \co{founds} true forgiveness and in this sense \co{I}, who 
still am unable to renounce the image of wrongs and evils done to 
\co{me}, can call it \wo{good}, \wo{morally good}. 
In 
itself, however, it does not rest on this \co{distinction} at all and 
knowing nothing \co{evil}, it does not know anything which could be 
opposed to it either.


%%%%%%%%%%%%%%%%
%%%%
%\input{035vsEthics}
\subsection{Morality vs. ethics}\label{se:morals}
%no moral rules
\secQQQ{8.5}{If there is no God, then everything is allowed}{ater 
Dostoyevsky, {\em Brothers Karamazov}}
%
\pa But we still want some rules, we do not want to set up an absolute
opposition between life and mechanism, between lawless spontaneity of
unrestrained freedom and rigidity of exceptionless system.  Ethical
principles and secondary rules may be useful and there need not be
anything wrong with them.  But they are useful for purposes which are
of very different order than those addressed here.

Moral can be a person and only a person; ethical can be an act, an
enterpirse, an intension.  Morality has no \co{precise} rules, but is
a matter of personal, that is, ultimately \co{conrete} attitude. 
Ethics, on the other hand, has nothing but rules which always try to
capture the ineffable content of moral attitude in \co{visible} terms. 
Defending utilitarianism Mill says that \citt{utilitarian moralists
have gone beyond almost all others in affirming that the motive has
nothing to do with the morality of the action, though much with the
the worth of the agent.}{Mill, Utilitarianism, I [The quotations from 
this work must not be interpreted as any agreement with it. Except 
for a few formulations, as far as morality is concerned, I simply 
find it one of the most detestable examples.]} It is
certainly a sound principle for any ethics to restrain from the
attempts to determine the critaria for deciding the \thi{worth of the
agent}.

\pa Ethics is preoccupied with \thi{judgments}; morality knows nothing
about rules and principles, about criteria and judgments.  It is but 
an \co{aspect} of \yes, in the best case, of
 \co{holiness}, and is occupied exclusively with \co{myself},
with \co{my} ability to avoid mistakes -- it only tells how \co{I} can
try to determine \co{my} worth as a person.  It will never allow
itself to judge another person.  It may get involved into ethical,
that is, social context of judging according to the accepted system of
laws and customs, but if only such a system serves its purpose, it
will be mainly a question of applying the laws.  (A system where any
case is unpredictable and reduced to the cunning smartness of the
lawers certainly does not fulfill its function.)  The moral attitude
founded in \co{love} sees only a person, that is, a site of
\co{incarnation}.  The \co{acts} and \co{actions} of this person
may deserve blame and reproach -- the person himself is
beyond any reproach.\footnote{This, by the way, is the aspect which
Nietzsche grasped well in his image of the master beyond good and evil
of ethical systems, of an aristocratic indifference to the actions and
opinions of others when following one's own course.  What he missed
completely is the depth of the positive aspect of the renounciation of
\co{oneself} and \co{this world}, which not only brings the \co{soul}
back to \co{the world}, but also unite the two, with genuine morality 
being but an \co{aspect} of this 
attitude.  Blind to this fact, his master had to remain in a pure
sbjectivity of \thi{anything goes} -- antyhing, that is, which \co{I}
find worht pursuing.  }

\pa
Being occupied exclusively with \thi{principles}, with \co{visible} 
criteria and formulations, 
ethics is of minimal, if any, value for personal conduct -- it may be of 
forensic interest, of relevance for designing the laws and determining 
punishments, so
I leave such problems to those interested in them. Here I will 
only comment briefly on the inadequacy, if not danger, of confusing 
ethics with morals.

%%%%%
\noo{
\pa
\co{Humility} -- do not oppose / \co{I} do not make good, only amiss
\co{Thankfulness} -- do not hunt for your own
\co{Openness} -- let everything grow

\begin{tabular}{lll@{\ \ |\ \ }ll}
&   &  Ethics &      & Morals \\ \hline
1.& -- & actions &   -- & persons \\
2.& -- &relative  &  --& absolute \\
  &    & (need not be God-given) && \\
3.& -- & arbitrary  & --& founded \\
  &    & (need to assume smth. good) && \\
  && why am I to be ethical? & \\
  && to accept {\em this} principle? & & \\
4.&-- & ought, shalt & -- & see\ldots\\
5.&-- & 1/100-th of situations,  & -- & always\\
  &   & so exactly when? && \\
6.&-- & moral vs. immoral & -- & true vs. confused\\
  & & only force/ & & respectful compassion, help\ldots 
   pity \\ 
  &  & /external sanctions     && and if he doesn't get it -- I can do little \\
7.&-- & good of fellow-creatures & -- & I=others  
\end{tabular}

Mill'a lucidity makes it, if not a worthy reading, then at least not 
a total waste of time\ldots
}
%%%%% end \no{...}

\sep

\pa Ethics has since times immemorial struggled with the ghost of
arbitrariness.  \citt{What is its sanction?  what are the motives to
obey it?  or more specifically, what is the source of its obligation? 
whence does it derive its binding force?}{Mill, Utilitarianism, III}
Let's notice at once the underlying theme of \thi{forcing},
\thi{obliging}, \thi{binding force}.  The despair of an ethicist is 
that he can not manage to \thi{force} anybody. 
Indeed, not only the question \wo{What forces me to accept 
this rather than that principle}, but above all \wo{Why should I be 
ethical at all?} remains in the sphere of assumptions which one has 
to take for granted before one at all starts an ethical dispute. 

\pa
An easy answer is the possibility of punishment. If we reduce ethics 
to its proper place, punishment is a factor of social system which is 
perhaps not sufficient, but the only avilable one to \thi{force}. And 
here we should stop.

However, ethical \thi{principles} are defended as having some
foundation and justification which, however, need not any appeal to
\co{transcendence}.  One sets up all kinds of systems, rules and laws
avoiding any \co{transcendent foundation} which, eventually, means
avoiding any \co{foundation}.  \wo{I just deny to care about this and
that principle}, \wo{I just deny to care about ethics}.  This may
sound extremely irrational, but \thi{forcing} should be effective
primarily for such cases -- those who accept some ethical standards
need not any \thi{forcing}.

\pa Eventually, one has to make an appeal to \thi{conscience}, to the
moral decency, yeah, moral intuition, and say someting like \citt{The
ultimate sanction, therefore, of all morality (external motives
apart) [is] a subjective feeling in our own minds [\ldots]}{Mill,
Utilitarianism, III} Captured into the \co{objectivistic} opposition
between \thi{objective} and \thi{subjective}, between \thi{transcendent} 
and \thi{immanently rational}, the \thi{conscience} is
indeed a thoroughly and exclusively \thi{subjective feeling}.  Thus,
\wo{If you do not have conscience, we can not talk to you}.  The two
exclusive groups -- the \thi{ethical} and the \thi{unethical} -- are
as definite as they are indeterminable.  And so ethicists keep talking
to themselves\ldots 

\pa Appealing to the ethical standards (which at least they possess),
they talk about \thi{ought}, \thi{You shalt}, \thi{You shalt not}. 
Having no sanction, no \co{transcendent founding} in the
\co{concreteness} of a person, the \thi{ought} has no relation to the
reality of Being, except that of the celebrated opposition to
\thi{is}.  \wo{It is what you should do, you should be ethical, and if
you are not you are not \ldots worthy, cultivated, etc.} Meaning, I am
not \thi{ethical}?  Well, it would sound better if we used another
adjective, for instance, \thi{rational human}.  \wo{You should act
this way because it is what humans, all rational humans agree on!} And
if I do not agree I am not a rational human?  Perhaps it fits me
better.  Perhaps I enjoy it with all rational and even plausible
arguments one may produce for such an attitude.  Sure, it puts me
outside the ethical discourse, but as said before, it is hardly the
goal of such a discourse to establish a boys-club which could
internally agree on what it decided.  Consensus is a poor substitute
for \thi{rationality}, and appeals to majority are here,
as elsewhere, meagre expressions of the the total lack of anything
convincing to say.  Arbitrarines\ldots

\pa
There is no such thing as \thi{should} opposed to \thi{is}, or to put 
it differently, there is no such thing as a \thi{subjective should}. 
To the extent I feel that \thi{I should} it meets me as a 
\co{command}, as an \thi{objective} \co{command}, even if nobody else 
can hear it.
 True, 
\co{commands} \co{reveal} the \co{invisible} order of things which 
may not conform to their \co{actual} state. But they do \co{reveal} 
something that, in the most genuine and intimate sense, {\em is}.
%
If the privacy of its \co{commanding} character makes me label such a
\thi{should} as \thi{subjective}, it loses all its potential force and
relevance, it becomes a figment.  And in many cases it better does. 
It is as easy to ignore \co{commands} as it is difficult to avoid
hearing all too many commands which one would like to be
\thi{objective}.  For having split the world into \thi{is} and
\thi{should}, we crave through the resulting arbitrariness for their
unity.  By the former we simply reduce the world to \co{actuality},
while the latter will often demand \co{positing} of some inaccessible
\thi{ideal}.

\pa The arbitrariness of ethics \co{dissociated} from the
\co{foundation} in the unity of a person and Being, is also well
illustrated by the fact that most situations have no ethical import --
they are ethically indifferent, irrelevant.  So one has to either
admitt this partiality of the interest, or else assume the posture of
a preacher directing solemn summons to the people of unethical
conduct, \co{absolutizing} the opposition of \thi{good} and
\thi{evil}, of \thi{right} and \thi{wrong}, \thi{just} and
\thi{unjust}.  Infrequent and sick as such an attitude may be, it does
bear witness to the definiciency in relating ethical thinking to the
being of a person.



\pa
\citt{[Y]our ideas of virtue you have got from mere human authority, 
and on human authority too its obligations rest: hence your system of 
practical morality is defincient.}{Tertullian, Apology, 45 [Please, 
keep in mind, that the \thi{merely human} is not opposed to 
\thi{inhuman} or \thi{transcedent} in some mysterious sense of 
comming from beyond the human being, but to the \thi{wholly human}.]}

But, but, one can do good things without being filled with
\co{spiritual love}, without any \co{spiritual foundation}, without
anything \co{transcending} one's human sense and rationality.  This is
the subject of ethics.

Indeed, what follows from the difference between \co{ontological} and
\co{concrete founding}, in particular \refpp{pa:unholygood}, is that the lower
values, in themselves, are not founded in the higher ones and may be realized
without the respective higher values being realized. Taking any set of (lower)
values as the starting point, one can both find examples of people realizing
them and design systems of ethics centered around them. One can even descend
down the levels of the hierarchy, starting from hapiness or the good life which
all desire, and point to possible connections between these and lower values.
What one will not achieve in this way, however, is the \co{concreteness} of
morality, the \co{concrete foundation} of morals in very depth of a person.

\pa Tables and classifications may be useful for temporary purposes,
even if they always contain holes and never can list all possible
cases.  Given a cultural context, especially a stable one, it may be
possible to list the behaviours, situations and attitudes which count
as good or wrong.  Such lists, however, are of purely social
relevance.  Designing a political or juridical system, one is of
course forced to make such lists.  But this should not be confused
with the \co{spiritually founded} morality, that is, \co{transcending}
distinctions of good and evil.  A saint, a wise sage may, on
occasions, act in a way obnoxious to the observers of the ethical
order grounded in the lists of acceptable behaviours, because he aims
at the personal center beyond any regulations.  The rules, any rules,
may at best approximate this center, but never capture it.  For rules
are only expressions of statistical observations while \co{holiness}
which \co{founds}, and in fact is co-extensional with true goodness,
resides only in the most \co{concrete} center where person touches the
\co{invisible} Godhead.

\pa What good can a person do without \co{spiritual love}?  The list
may, indeed, be indefinitely long.  But it will be only a list -- an 
arbitrary list, with each item preceded by the arbitrary \thi{You 
shalt}!  A
mere listing of possible situations and cases is hardly manageable, so
it will have to take some principle, \co{visible} formulation, and
elaborate it.  The more \co{precise} the formulation is, the more
applause can the corresponding list gather.  But then, also, the more
holes it will have.  
%%%
\noo{
It shouldn't be necessary to analyse various
alternatives.  Let utilitarianism provide a simple example.  Let us
ignore the fact that it
reduces moral value to some lower level of the hierarchy, 
be it pleasure, happiness, welfare, esthetic experiences, or whatever of the kind.
%
Bentham's principle -- \citt{an act is 
          right if
          and only if it tends to maximise the net
          overbalancing sum total of pleasure over pain for all
          parties concerned}{Jeremy Bentham, Introduction to the Principles of 
          Morals and Legislation, 1789.}
-- addresses an \co{act}, though other variants of utilitarianism may
substitute for it a rule, an attitude, a course of action.  Indeed,
the inescapable problem is exactly {\em what} is supposed to be
judged by such principles -- acts, actions, attitudes, rules,
persons\ldots?  An atomic event has to be posited somehow, and much
of the history of utilitarianism involves attempts to specify it,
whether with respect to the events judged, or consequences to be taken
into account.  Whichever variant we choose, it will be difficult to
apply the principle to a person, understood as a unity.
%What are the goods?  Let it
%pass, for any definition will smell a totalitarian attitude.  
For then it would declare a sincere, rich merchant, making generous
donations to charity benefitting thousands of people, who only
occasionally exposes his wife and children to mental torture, a better
person than St.~Francis who, if he has ever done anything specifically
good to any people, the value of this good (computed in any kind of
the felicific calculus) was probably negligible.  This is but a
typical example of \thi{ethical rules} which, useful as they may be
for the design of laws and legislation, are only capable to address a
\thi{majority}, the average, statistical cases, but never the
\co{concreteness} of the situation.  Even applied to single \co{acts},
the principle is useless -- not only because it merely allows to judge
an \co{act} \la{post factum} but does not help to decide what to do,
but because the involved felicific calculus has to be based on some
uniform quantification of goods.  Again, the history of utilitarianism
yields numerous proposals of lesser or greater ingenuity, but their
very multiplicty witnesses to the alternative, personal attitudes
which defy subsumption under a common rule.\footnote{Restrictive,
indirect and ideal utilitarianism (G.E.Moore, H.Rashdall), even the
variant which is willing to consider the very occurrence of an act as
a good to be counted (a consequence of an act of loyalty is also that
such an act occured -- and this counts) are variants which indeed try
to fill the gap of impersonality, and as such move in the direction of
moral, and not merely ethical theory.  }
}
%%%%% end \no{...}

%\pa
The lists, principles and regulations, which attempt to find a 
\co{visible} expression, even definition, of the good they try to 
capture, are the more desired, the more general attitude is dominated 
by the \co{attachment} to the \co{visible world}, that is, the more 
definitely it distances itself from the \co{invisible commands}. But 
any list, attempting an \co{objective} determination of goods and 
wrongs, will necessarily have holes when confronted with the 
\co{concreteness} of life. 
%Moreover, these holes will only grow 
%bigger with time. If everything which is not prohibited is allowed, 
%then, eventually, everything is allowed, for any list of particular
%prohibitions, in so far as these concern only \co{visible} aspects of 
%situations, consists only of infinite large holes with respect to 
%\co{concrete} actions, not to mention, personal attitudes. 
And the greatest hole of all is the arbitrariness which can not but
appeal to \thi{subjective conscience}.  

If goodness of actions is not grounded in the \co{transcendence} of
\co{invisiblity}, in the personal attitude of \co{humility} of
\co{communion} in the face of Godhead, then, in fact, everything
becomes allowed.  As many a skilfull lawyer has shown and as everybody 
knows, any rule has
exceptions or can be re-interpreted and tailored to the current needs. 
Without Godhead there is 
%no place for \co{holiness}, 
no reason for a \co{spiritual} attitude of \co{absolute} personal
comittment, for an attitude which is not relative to any particular
things, to the possibilities and impossibilites of the context, to the
local laws and regulations, to the goods and values of
\thi{subjective} arbitrariness, but which engages the whole person
only because the person recognizes its unconditional applications to
everything and its \co{absolute}, not only \thi{subjective} value for 
oneself.  One can choose one's comittments freely, one can dedicate
one's life to being a respectable citizen, a righteous person, a
charitable activist.  But these values, being grounded in a free,
which here means arbitrary, choice, can as well be changed to others,
whenever the reasons for their acceptance change, diminish and are
confronted with new reasons.  If the choice of values is not founded
on the \co{absolute} reality of the \co{origin}, if it is not founded
on the unconditional \co{love} and submission, it is, indeed, a free
choice which leaves selection from an infinity of alternatives to the
arbitrariness of \co{my} private considerations.  \wo{If there is no
God, then everything is allowed.}

%and this, eventually, means in the personal fear of 
%comitting, at any, most unexpected moment, a mistake (and, shall we 
%perhaps say, a moral mistake against others is a sin of 
%\co{ingratitude}, \co{pride} or \co{closedness}, that is, a sin
%against Godhead), 

%The categorical imperative? Good, but only if we really take it 
%personally, that is, try to interpret the possible intensions lying 
%behind it, and not take it literally. 

\pa There are {\em no} rules, no absolute rules -- only persons and
\co{concrete} situations.  In general, we might say \wo{Thou shalt not
kill} but it is not difficult to come up with examples where sticking
rigidly to this commandment might seem less praiseworthy than opposing
it.  We do praise people who attempted to assasinate Hitler, do not
we?  And, if not them personally so, in any case, the thing they tried
to do.  We do debate on the capital punishment, on euthanasia. We do 
accept killing during the wars. 
%We even 
%wonder how it was possible for so many Germans, Austrians, and others 
%to participate in the atrocities of the nazism without opposing it in 
%any way, perhaps, exactly without attempting to kill Hitler. 
Imagine two people lost in a desert, with no reasonable chances of
meeting anobody for days, of getting anywhere close to water, and with
one of them getting wounded, falling sick unable to go further, and
the other unable to carry him.  The first man asks the other to kill
him, not to let him stay there, perhaps, for days, not to let him die
of hunger, thirts, heat\ldots You may, in advance, decide \thi{I will
never, under any circumstances, kill} and this may provide you with an
answer in such a situation.  But you can not deny that one can
reasonably argue also for the opposite action.  \la{In abstracto} one 
can always argue, and seldom do anything else. One can always argue because
there are no rules -- there are only people, \co{concrete} people and,
eventually, either \co{open}, \co{loving}, compassionate or else
rigid, \co{closed} and \co{proud} attitudes.\footnote{As a person 
fluent in Hebrew pointed out, the  
commandment ``Thou shalt not kill'' [\ger{loh taroh}],  
in a more accurate translation says ``Thou shalt not murder'' 
[\ger{loh tirtzach}, Ex.XX:13].}

Making lists of \co{visible signs} and categories may simplify and 
help, eventually dupe, but it won't do.

%%%
\noo{
\sep

\pa The criteria and rules ethics purports to establish try usually to
capture some of the moral intuitions shared by the community.  The
more \co{precise} they become, the more \co{concrete} relevance they
lack.  Ethics and morality represent two enirely different orders;
they have entirely different purposes and, consequently, have to apply
quite different rules.  

\pa
We have lived with abstract ethical principles for a while until we 
realized that they do not fulfill satisfactorily their function 
(whatever that was supposed to be). So we should rather start 
thinking in terms of applied ethics. This, however, was still 
dominated by the idea of an abstract principle, with the only 
difference that emphasis has been now put on its applications. In a 
short time, thinkers like, for instance MacIntyre, realized that this 
won't do. Ethical principles seemed to be much more deeply rooted in 
the specificity of particular communities than what an abstract, even if 
applied approach, could account for. Social sciences and observations 
were called for help and this may be, indeed, a sign of healthy 
departure from the philosophical universalism. Emergence of fields 
like medical ethics, scientific ethics, bioethics witness well to this 
development.

\pa From there, however, there is only one step to acknowledge the
importance of the \co{conrete}, whether in form of a personal
character, intensions of the agents, context of action.  Unsatisfied
with the lists of \co{objective acts}, we start adding intensions of
the agents, personal character, possible motivations, impossible
excuses -- and the more satisfied we become with the apparent
\co{concreteness} of the covered cases, the more unmanageable the
lists become, the more unpredictable scenarios can be constructed from
the amalgamation of specific, yet hardly transparent, subcases.

\pa
But we want to design ethics, perhaps, even an ethical theory. We do 
not pretend to capture moral attitude and so the personal aspect of
morality, impossible as it is to grasp in anything reminding of a
universally intersubjective, not to say scientific language, can not be
let in all too far.  Let us, as an example, take the context which
seems to be quite a popular pet.  What is a context?  I am a
philosopher, so you have to excuse me.  It is this particular room in
this particular hospital where the surgeon Tom performs this
particular operation on the patient Bob, it is this particular
crossroad where the accident -- between Mary's and John's cars --
found place, it is this particular office where Jack offended Jim and
spilt his coffee all over Jim's suit.  Is it?  Isn't it also the fact
that Tom has marital problems, that Mary was in a hurry, and that Jim
was nasty to Jack's fiancee last week?  But isn't it also that Tom had
a peaceful and cheerful childhood but was bad in football, that John's
father was a sadist, that \ldots This will take us too far, that is,
this won't take us anywhere, right?  Sure, you want to take into
account only the {\em relevant} aspects.  But what is relevant?  Who,
in short defines the context?  Who and how decides, as in the case of
utilitarianism, what is the atomic event, even if now this atomic
event isn't any more a pure atom but a context, that is, a
\co{complex} thereof?  Say that a context a finite set of atomic, and
only relevant events (whatever we might mean by any of these phrases). 
The collection of contexts becomes thus a power set of the collection
of atoms.  As we know, even if the set of atoms is countable, even if
it is recursively enumerable and, as such, gives a vague promise of a
possible tractability, its power set becomes a continuum exploding all
the frames of (ok, computationally) manageable entities.

\pa The argument may not appeal to a social animal but I said
-- I am a philosopher and beg your pardon.  Chasting \co{concretenes}
and morality, ethics may propse new and more appealing models, but
this only for the price of impossible complexity.  The ghost of
atomistic empiricism haunts any attempt to approach the \co{concrete}
by means of \co{complexes} of definite, seggregated, \co{precise} terms. 
Social consensus can establish the demarcation lines only 
approximately, only on average. Eventually, these lines are drawn 
anew in every \co{actual} situation, and morality or immorality of the 
involved persons slips through the holes of the network of ethical 
concepts.

}
%%%% end \noo{...}

\pa Do not get me wrong.  I am not saying that it is not worth trying. 
Probably, it is even necessary.  We need try to design ethics of
operation hall, ethics of biolgical research, of dentists, of car
mechanics, of Ford specialists, of Mercedes specialists, of \ldots ,
and such things may be used for employing or firing people, for making
political or economic decisions, and who knows what.  There is some
truth in most proposals, there is some agreement between morality and
most ethical principles.  But it is always only \thi{some}.  The
concreteness of these principles is only apparent and, often,
misleading.  Except for the \co{vague} inspiration, it simply has {\em
nothing} to do with morality, because morality concerns \co{my} being,
who \co{I} am, while rules and principles only rough and generalized
guidelines for \co{actions} and behavior.  If we push this process of
\thi{conretization} of ethics far enough, it ceases to have anything
to do with morality also in a more common and vague, not only this
proposed here, sense of the word.

Designing ethics is, indeed, an impossible and unavoidable task, and as such it
must be left to the continuous attempts and revisions -- not necessarily by
philosophers, but by those active in the area where ethics is relevant:
sociologists, lawers, statesmen.  Ethical systems are matters of exclusively
social relevance and it seems that Lycurgos and Solon were much better at such a
work than Plato.  Philosophers have rather poor record in striking a right
balance between heterogenous interests and groups.  But of course, converting
the issues of moral character into ethical form and language opens an unlimited
field for academic discussions of irrefutable social relevance. What else might
an academician wish\ldots

\pa Eventually, no matter what ethics the society uses as the basis
for its legal code, the institution of a judge (or, sometimes, jury),
a person who is present and participates in the detailed analysis of
the case and who, sometimes, is the one eventually deciding it, is
indispensable.  It could be probably defined as the institution
serving the resolution of the unavoidable discrepancy between ethics
and morality.  (In fact, how a society chooses its judges, and what is
a judge's authority will tell one quite a lot about the way the
society perceives and resolves this conflict and, through it, about
the society's ethos.)

\pa Ethical attitude is the attitude of respect for the law -- the
most immoral person is capable of that.  Teaching ethics, one attempts
to promote desirable attitudes; desirable, that is to say, socially
\co{useful}.  Morality would be nice to have, but it can not be
determined by any courses or social institutions, so it has to remain in
the sphere of pure \thi{subjectivity}.
%Courses in ethics, the more frequent courses in ethical 
%behaviour in a given profession, may serve this purpose very well. 

%\subpa
As witnessed by numerous examples, Jesus being the paradigmatic one,
the morally praiseworhty attitude can be highly unethical, that is,
socially unacceptable, even damaging.  It is not a necessary conflict,
but it is a conflict, the possibility of which, one always has to take
into account.  Moral education is possible only in one way -- by
example, by a living, preferably, the teacher's own
example.  Besides that, the \co{conrete} stories about people we
recognize as moral exemplars, the stories of their particular actions,
as well as their lives, are probably the only means of communicating
the moral standards.  
\citt{Children are educated by what the grown-up {\em is} and not by 
what he {\em says}. The popular faith in words is a veritable disease 
of the mind, for a superstition of this sort always leads farther and 
farther away from man's foundations and seduces people into a 
disastrous identification of the personality with whatever slogan 
might be in vogue.}{Jung, The Archetypes and the Collective 
Unconscious, IV:293}
To choose such a \co{concrete} course, however, one has to know what 
oneself is and what one has to offer. Moreover, 
one has to be
interested in morality and not only in ethics, that is, to admit that
the most universal principle of human life, the most common thing
which brings people together, is the mere fact of everybody being an
individual, thoroughly \co{concrete}, human being, with his own
freedom of denial or acceptance of the \co{invisible} pact with
Godhead which alone \co{founds} the \co{communion} with
others.\footnote{Such a trend, though not necessarily grounded in the
same fundament, in ethics education seems to have appeared recently. 
It carries, of course, bad connotations for the ethical philosophy,
which can hardly discern any role for itself in such a pure casuistry,
a mere ``case method''.}


\sep

\pa The differences should be clear.  There is no \thi{binding}
imperative which, in any way whatsoever, could \thi{force} somebody to
be moral.  And any ethical system is bound to appeal, at some point or
other, to morality.  The fact that most people recognize validity of
some basic moral intuitions arises directly from the \co{awarness} of
\co{ontological founding}, from the \co{awareness} that \co{I am not
the master}.  If one chooses to follow this up and strive for
\co{concrete founding}, one needs no other reasons than the most
obvious and natural one -- \co{my} being is the primary issue of
\co{my life}, or else, \co{my} salvation is what concerns \co{me}
most.  This, in itself, might be met with the accusations of
selfishness but, as we have explained, it has nothing to do with it,
because \co{my} salvation is grounded exclusively in the renounciation
of \co{myself}, and love of \co{myself} has nothing to do with \co{Self
  love}. The \co{concrete founding} following such a 
renounciation is hardly distinguishable from \ldots morality.  It
becomes distinguishable only if we first dissociate \co{visible} from 
\co{invisible}, and then being from
becoming, \thi{is} from \thi{ought}, happiness from freedom, and all
kinds of things which can be distinguished but never should be
dissociated.  
\co{I} choose \co{founding} of life in the
\co{invisibility} of its \co{origin}, \co{I} choose \yes\ not because
there is any \thi{ought} but simply because \co{I} want to and see
that this is the best thing one can do.  Designing rules for people
who do not see it is as a challenging exercise for shrewdness and
inventiveness as it is futile.  It always has the taste of the fruits from the 
tree of good and evil which, as we were told, is somewhat removed from 
the center of the garden where the tree of life grows. 
Ethics, without \co{foundation} in the
thoroughly \co{concrete} morality of a person, may keep designing
rules and regulations which forever will appear as arbitrary as they
have always been.  Showing the way is not the same as guiding an
excursion.
%end 035vsEthics

%%%%%%
\input{038summary}


%%%%%%%%%%%%%%%

\newpage 
\subsection{\G\ and \B\ as the existential options (?at the Absolute level)}
%
\plan{The final level does not involve both \G\ and \B}
\planA{but is Absolute ``surrender'' to \HH\ and power over \LL}
%
\pa
Balance and harmony are impossible as long as one is not convinced that
the world is ultimately good.
\pa
If you are satisfied (with \co{your Self}) you do not blame anybody who might have done
you wrong -- you pity him. You only ask ``Why?'' and do not expect any 
answer \ldots

If you trust \co{your Self} you do not blame the world for anything which might have gone wrong.

\levels{Acceptance of the world} (through which it becomes `mine')
\label{lev:accept}
\are{sensualism/hedonism - of here-and-now}
    {possess}
    {understand/feel}
    {accept/love}

Referring to the table on page~\pageref{ta:BtoA}. \nopagebreak

\noindent
A. In \G :  \nopagebreak
\begin{enumerate}\MyLPar
\item  good and bad ``zlewaj\c{a} si\c{e}'' in one total attitude \\
        -- beyond good and evil \\
        $\bullet$ beyond the world (of distinctions) \\
        $\bullet$ makes all things appear Good 
        (``dobry cz{\l}owiek wybiera dobre rzeczy z dobrego skarbca...'')
\item   However, the world does not disappear -- is only perceived 
        ``sub specie eternitas''
\item   So, things are still there but no longer with the capacity for intrinsic evil --
        Forgiveness!!
\end{enumerate}
B. Refining A.1-2 :
\begin{enumerate}\MyLPar
\item The basic experience - of just being here confronted with there -- is \G!!! I might
have not been! (cf. happy life consisting exclusively of unhappy moments)
\item When you experience something wrong -- even evil -- it is already {\em something},
 that is finite and definite. 
\item In this experience you should not forget 1. -- even if (you judge that) 2. is bad.
\item Since it (2.) is something (definite) it is Below and you can 
        (try to) do something about it.
\end{enumerate}
\upa{
Simply: \G\ is acceptance of \HH, \B\ its denial.
}
\upa{
\G, ``yes'' to God, is also ``yes'' to the world. \B\  is ``no'' to God inspired by the
wish of saying ``yes'' to the world.
}
\pa
\G\ is not the question of hope -- hope is before it.
It is the question of accepting -- without hope or fear, in full humility and trust.
\\[3ex]

\pa \ldots
\begin{itemize}\MyLPar
\item what is external can be controlled -- the rest not; \\
 {\small rather than: Visible can be controlled -- Invisible cannot;}
\item not observing the higher level $\mapsto$ illusion of power and security;
\item[?] responsibility for what is Visible towards what is Invisible ?
\end{itemize}
\pa
The 4-th level is not what is invisible in principle but the \HH\ as such, as a constant
aspect of our being.
\pa
The existential situation is: between \LL-\HH. The absolute relation is the relation
of the 4-th level: relation to \HH\ as \ldots

