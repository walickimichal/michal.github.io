
\subsubnonr{Death?} 
Death is return to the \co{original indistinctness}.

Life's as we \lin Find it -- death too. \lin A parting poem? \lin Why insist?
[Daie-Soko, in Zen Poetry, p.17]

\woo{Free man thinks least of all about death}{Spinoza}

For a person completely \co{attached} to \co{this world} it is impossible
-- an unacceptable offense to human aspirations and dignity. Such people spend
much of their life escaping from death, simulating youthfulness, constantly
worrying about their growing age which they can not accept.

\co{Attachment} is a sign of loneliness (and \co{alienation}). 

It is harder, the more attached we are to some persons -- it is much easier to
accept the fact of death for a single person, than for one surrounded by family
and children. However, this comparison is superficial. A person
surrounded by the loving ones, loses apparently more having to part with
them. But we should not assume that, for this reason, such a person is unable to
die quietly consoled. A harmonious life in the community of the family is more
often than not an expression of continuity of being founded in the true origin,
and death may be then more easily accepted. Fear of death invades rather the
lonely ones, and only in the extreme loneliness
%of \co{alienation}
it grows exaggerated to the limits of obsession.
It will be quite common in decadent societies -- like, for instance, the late
XIV-th century, when the Danse Macabre and ghoulish fascination
with death and its physical details
emerged and then flourished through the Renaissance (the first of its kind in the
cult of death was the effigy of Guillaume 
de Harsigny, at Laon, from 1393; poetry of Villon is a well known
example),\noo{B.Tuchman, A distant Mirror, p.502}\noo{as Henri Matisse remarked:
  \wo{Renaissance -- it is decadence}}
or today's West -- societies where the dissolution of the common ethos results
in atomization and alienation.

Death is always of another...

%%%%%%%%%%%%%%%%%



\subsubnonr{The moral values}
%\addcontentsline{toc}{subsection}{\ \ \ The founding of moral values}
No ethical rules or principles will ever grasp the moral value of an
\co{act} which, to the extent it is moral, descends from the \co{invisible}
heights and only as if by accident takes this particular form rather than that.

\pa The \co{spiritual} attitude, the attitude towards Godhead, 
\co{nothingness}, is actually \ldots  morality.  It does not matter if
one calls it \co{One}, God, \co{origin} or, perhaps, does not call it
at all.  What matters is, whether one \co{recognizes} something 
\co{invisible above}
which abolishes validity of \co{my} \co{visible} goals and wishes and,
at the same time, opens for \co{me} \co{the world} as a field of
\co{incarnation}, of continuous \co{expression}.
\citet{Neither may a man who is made a partaker of the divine nature, 
oppress or grieve any one. That is, it never entereth into his 
thoughts, or intents, or wishes, to cause pain or distress to any, 
either by deed or neglect, by speach or silence.}{TheolGerm}{XXXIII}
How come? Why can't he oppress or grieve any one? Do not we have 
enough counter-examples?

\pa ``Love and do whatever you want.'' Whatever? Is this focusing
on a mere attitude justifiable? We know it, and we know that, as an
ethical rule, it does not work well. If I love, does it mean that I
can steal, kill, and whatever?  In a sense, yes.  ``Whatever'' means
whatever.  But, if you steal and kill like a crazy criminal, then you
do not love.  Following Augustin, the Free Spirits of XIV-th century
said too: \wo{Love God, and do as you like.} Eckhart's reservation was
immediate: \citet{Yes; but as long as you like anything contrary to
God's will, you do not love Him.}{Eckhart}{}

It looks like a circle, doesn't it?  Not, however, a
vicious circle but a virtuous one.  \citet{[B]y their fruits ye shall
know them}{Matt}{VII:20} The \co{spiritual love} \co{founds} the whole
attitude and, by this token, makes some acts impossible.  As this
founding, this \co{incarnation} happens in the \co{invisible} center
of \co{virtuality}, it contains things which are inextricably bound
together and which only rational \co{reflection} can dissociate.  For
a person who truly \co{loves}, the ``whatever'' means something very
different than for a person without such a \co{love}.  The circle seems
vicious only if one has \co{dissociated} the \co{spiritual} from the 
\co{visible} and then tries to find \co{visible} criteria for \co{love} which
one does not possess,to determine, univocally and
\co{precisely}, what should count as \co{love} and what should not.

\pa
The focusing on the attitude is justifiable by the claim that the 
fruits of the right attitude are right, that what is right 
\co{above}, what is \co{holy},  
will also be right \co{below}, will be morally right. \citet{As above, 
so below}{Emerald}{[Although the 
original is unknown, this formula is reasonably documented as one of 
its central teachings (in cnojunction with \wo{and as below, so above}.]}
Consequently, what is wrong 
\co{below} reflects also something wrong \co{above}. 
%I have to dismiss 
%nthe questions about \co{precise} criteria, because I do not believe in 
%them. 
And no, there is no circle\ldots

\kom
\citet{For a good tree bringeth not forth corrupt fruit; neither doth a 
corrupt tree bring forth good fruit.
 For every tree is known by his own fruit.}{Lk}{VI:43-44; Mt.VII:17-18}

\citet{The thoughts of the righteous are right: but the counsels of
      the wicked are deceit.}{Prov}{XII:5} 
-- a bit too definite specialization, but founding ok...


\kom
(Evil divides, scatters, eventually, alienates --  good gathers,
unifies. But this is only(??) \co{spiritual} -- it has no necessary, \co{actual}
consequences, no \co{visible} proofs... A good person can oppose, object, just
as one affected by evil can agree and be amiable...)

\tsep{}


\pa
 The values, and the attitudes which 
we would tend to call ``moral'', can be recognized on different 
levels of Being.\footnote{This hierarchy is probably as close as 
possible to Max Scheler's, e.g., in \citeauthor*{MaxForm}.}

\levs{12}{Levels of moral values}
    {pleasant-unpleasant: counts only Here\& Now (aesteticism)}
    {useful-useless: formal-institutional-traditional}
    {good-evil in-itself, idea beyond forms}
    {holy-unholy: as religious attitude} 

\pa [-- moved from Freedom --] 
Lack of (ok, let's say only, the \co{experience} of) determinism does not
mean complete indeterminism. As all contradictions, this is only the
case for thinking bound to \co{actual dissociations}. The lack of
\co{discernible} causes does not preclude the \co{presence} of
\co{motivations} and, eventually, \co{inspirations}. For these do not
cause anything, they do not determine any specific \co{acts}; they
only indicate \co{vaguely} a possible direction.\footnote{We tried to
  acsribe this distinction to Duns Scotus with his distinction between \la{ordo
    eminentiae} and \la{ordo dependentiae},
  \refpp{pa:foundingCausing}. Similarly, we might try to recast in
  these terms the Thomistic \thi{double causality}, as well as general
  Scholastic distinctions of various \thi{orders of primacy} (e.g.,
  with respect to knowledge, dignity, time, causality, etc.).
  Eriugena's \thi{primordial causes} come probably closest to our
  \co{invisible founding} of \co{actual} events, with his emphasis on
  the \co{invisible} unity and continuous \co{presence} of such
  \thi{causes}. Certainly, the word \wo{cause} is particularly unlucky
  and much confusion seems to arise from conflating it to the level of
  \co{actual} causality.}
%
\thi{Good intension} may be considered a
cause of a particular good \co{act}.  But as we said, such an
intension already pollutes the \co{act} and diverts it from its
\co{actual object} by attempting to \co{actualize} something
\co{invisible}. In any case, being \co{visible}, it does not \co{found}
the goodness of the \co{act}. \co{Invisibility} of a good being, however,
does not cause any particular \co{acts} -- it only \co{founds} the \co{rest}
\co{present} in such \co{acts}. 


\pa\inv The values of the \co{invisible} order have only two
modifications -- these are the \co{spiritual} values of either holiness or unholiness.
 Holiness corresponds to the \co{incarnated
love} of \yes. Its opposite, the \No, whether in an actively evil 
\co{soul}, or just as a passive \No, a lacking recognition of the 
Godhead, is its own evil: it may lead to inflicting evil on others, it may also
result in one's own suffering, but it is by its very nature its own punishment,
as \No\ is the \co{alienation} itself.

These values are not relative to any particular region of Being and
their recognition happens only through \co{spiritual}, which in
particular means, thoroughly personal experience.  Holiness is not
expressed through any particular feelings, it does not involve any 
specific forms of thought or action. But any meeting with it results
in, or rather is equivalent with, submission, adoration, reverence, awe.
Unholiness, in its extreme
form of active denial and negation, is associated with despair and feeling of
damnation.  But it is, probably, as rare as holiness itself.  In most 
general form, meeting
with unholiness does not have any particular form. 
Almost everybody we meet is unholy, and the character of the meeting,
as well as our response, are determined then exclusively by the lower
aspects of the situation. This, in fact, is the general 
characterization of unholiness -- the mere absence of holiness.


\pa\label{pa:unholygood}
Now, in itself (that is, except for the ontological order of
Being), the higher levels do not found the lower ones.  In particular,
unholiness, which in its mild and passive form is the most common
thing in the world, and which is a merely passive negation of \No, just a
lacking recognition of \yes, does not found any particular attitudes
at the lower levels. It thus does not exclude any particular 
attitudes at the lower levels -- it only excludes its only opposite, 
holiness. A person may be unholy, even deliberately and actively so, 
and yet be smart, just, perhaps, even good towards his companions and 
friends, and may enjoy all the positive values of the lower levels. Of 
course, the opposite may also happen, and one is almost guaranteed 
that, at one point or another, something opposite will take place.

\pa
You may sometimes meet a person who seems perfect -- wise, just, 
honest, whatever positive predicates you want to use. You know that 
perfection does not exist, and you suspect that this person is not 
perfect either. But the time goes on and on, and the person does not 
do one wrong thing. Eventually, you have to conclude: he is, indeed, 
perfect. And then comes a moment, not necessarily of any deep 
significance, a moment when a single word sounds somewhat inadequate, 
a simple gesture seems kind of inapropriate. 
A minor thing, a slip of the tongue. Nothing wrong has happened, but 
your problem is that this single thing does not fit into the overall 
picture of perfection. You ignore it, but it recurs nagging and 
disturbing your peace of mind. The more you dwell on it, the clearer 
becomes the idea and significance of this single thing. In fact, 
it discloses a side of the person which you did not expect and which 
horrifies you when you think what it is.\footnote{I do not 
necessarily have in mind a peaceful, kind burgoise, a perfect father 
of a family who, in his working hours, runs a concentration camp. I 
do not necessarily have in mind a kind, succesfull, loved by all 
Dorian Grey, whose picture rottens in the locked room of his soul. 
But these might be exaples, too.} A single moment can disclose the 
unexpected depth of corruption or mere misunderstanding and 
imperfection, which has passed unnoticed for months and years. And, 
even though through all these months and years the person did appear 
perfect, you have to admitt that it was an illusion, a deception. If 
you want, you can conclude from such examples that, as a matter of 
fact, all holiness and perfection is only an appearance hidding 
something from the eyes of the world. Often, it may be the case. But 
I am not inclined to draw statistical conclusions. Holiness is a 
state which excludes such moments because it does not hide anything. 

\pa Holiness is always preceded by a \sch\ and,
consequently, does found an attitude.  It is not, however, 
any specific attitude which chooses some particular things rather than
others.  It is an attitude of \co{love}, of \co{humility},
\co{thankfulness} and \co{openess}, with its more specific aspects as
described in section \ref{sub:nonattach} on \co{non-attachment}.  Just
as the \yes\ is merely the renounciation of \co{oneself}, so the
resulting attitude can be best characterized negatively by what it
excludes.  It says, perhaps, ``You shalt love your neighbour'', but
this has little content for one who looks for \co{precise}
characterizations.  ``You shalt not kill'' is probably more specific,
but any such command or prohibition is only an expression of the
intended direction of the \co{soul} -- not a detailed, \co{precise}
prescription excluding any possibility of an exception. 

\pa The attitude founded in holiness will not be occupied with doing
good and right things.  Least of all will it preoccupied with being
holy.  It will be occupied {\em exclusively} with not doing anything
wrong or evil, with avoiding the possible mistakes.  The true and
humble fear of doing wrong things, which is a constant aspect of the
\co{openess} of \co{incarnated love}, is perhaps not a necessary proof
of the impossibility of any wrongs, but it is, in fact, the most
genuine and certain guarantee that one will not committ them.

\pa What \co{precisely} consititutes a mistake and what does not, what
\co{precisely} is wrong and what is not, is, fortunately, impossible
to say in general, and it is something every person has to determine
in a given situation.  Philosophers, and ethical philosophers in
particular, have long despaired that the same questions, the same
problems, have been posed over centuries without ever finding
answers.\footnote{\citef{[A]after more than two thousand years the same
discussions continue, philosophers are still ranged under the same
contending banners, and neither thinkers nor mankind at large seem
nearer to being unanimous on the subject[\ldots]}{MillUtil}
 Kan, whoever\ldots} But this is the reason to despair
only if one assumes that there is an answer, {\em the} answer.  In a
sense, there is, because if only one avoids exaggerated
\co{precision}, things can be said.  But, of course, this means that
there is no answer, because avoiding such a \co{precision} means that
the answer has to be formulated anew again and again.  The attitude of
\co{openess} and \co{love} pays much more respect to the universal
\co{concreteness} of life, than ethical systematizations of
all possibilites in tables of goods and wrongs.

\pa\mine The values of the level of \co{miness} are relative to
\co{myself}, to the \co{I}.  As such, they transcend the horison of
\co{actual visibility} and express \co{general ideas}.  Things may be
beautiful or ugly, good and friendly or evil and malicious, right or
wrong.  Above all, I can perceive a person as good or evil, which is
quite different from perceiving the actions of the person as just or
unjust.  These values are recognized through soul's attitude expressed
in feelings like preference, appreciation, admiration, sympathy, love
or their opposites.  
%I call them the \co{eudaimonistic} values, or
%else the values of \co{soul} or of \co{miness}.

\subpa What we said about relativism in \refp{pa:relativism} can be
repeated here.  To the extent such values are not \co{founded} in
\co{love}, they appear as matters of private choice and individual
preference.  They are relative to \co{myself}, and {\em only} to
\co{myself}, to the individual who chooses them, or happens to live
them. Theories of ethics built on the identification of the good with 
the fullfilment of personal preferences have hard time with excluding 
terrorism and homicide carried out precisely in the name of such 
preferences.

Without \co{foundation} in \co{holiness}, they are indeed arbitrary 
and purely private. Being, on the other hand, \co{founded} on the 
higher level, they are no longer arbitrary, they can only be an 
\co{expression} of the underlying \co{love}. They may still aim at 
\co{my} happiness, \co{my} good life, but these are no longer 
dissociated from the all embracing attitude of \co{humility} and 
\co{openess}. In fact, they are merely \co{expressions} of this 
attitude, that is, \co{my} goals are merely goals of making this 
attitude \co{visible}. 
Recognition of the \co{shared origin} breeds recognition of the value 
and dignity of other people; beauty and insight are values 
\co{expressing} \co{openess}; compassion and benevolence are forms of 
\co{humility}; etc. These are not mere results of \co{my} arbitrary 
choice which can be dissolved in the relativity of other, even opposite 
choices. These are the \co{expressions} of the \co{absolute}, of the 
\co{invisible love}, and their innumerable variations are only 
\co{expressions} of its inexhaustible potential.


\pa \act The level of \co{actuality} involves us into \co{complexes}
of things and situations, and correlated values are, in general, the
values of \co{usefulness}.  These may be relative to the whole body
or, perhaps, the living organism.  They are not as punctual,
localized and \co{immediate} as the \co{sensous} values of 
\co{immediacy}, but they are neither the values of my whole person.  Instead,
they may be values serving health, increased sense of vitality, 
aesthetical enjoyment.  In a
broader context of social structure, these are values of social
usefulness and purposefulness.  A just, honest, intelligent, able
 person represents a value for a society -- is \co{useful},
in a way in which a person with opposite characteristics is not. A 
thing can have a value for \co{me} in so far as it is useful for \co{my} 
purposes, in so far as it serves \co{my} promotion, \co{my} carrier, 
\co{my} fame, 
or achievement of more particular goals, in short, \co{my} 
\co{Ego}. 
Reactive responses to such values are \co{impressions} like joy or 
 rage, feelings of courage and strength or fear and impotence.
I call these values \co{Egotic} (which, of course, must not be 
confused with egoistic) or the values of 
\co{usefulness}. 
%(To have a Greek word for this case, too, we might call 
%them \gre{aretetic} values, as they are indeed closest to the Greek 
%\gre{arete}, understood more as \thi{excellence} than as 
%\thi{virtue}.)

\subpa
Clearly, valued in dissociation from the higher levels, they will 
easily lead to egoism and promotion of \co{my} private interests 
without any consideration for other things or people, in a way 
simialr to Kierkegaard's aesthetical stage. 
But when \co{founded} 
in higher \co{openess} and \co{humility}, they will reflect the respect 
for things and people\ldots

\pa \imm The \co{immediate} values are related to the most \co{actual
experiences}.  When \co{unfounded}, they find their full expression in
a single moment, without any reference to higher levels.  The most
natural, perhaps the only, example would be sensous pleasures.  They
are relative exclusively to the general sensous nature.  A pleasant
thing will arise a sensous lust and, perhaps, an impulse directed at
this thing -- it embodies the value of the pleasant.  Unpleasant
thing, causing a bodily pain, or in any case a displeasure, will
cause, sometimes deliberate, but typically purely reactive and
instinctive, avoidance, withdrawal.  Extreme forms of hedonism, of
fascination with the senses and pleasure, in the deviant forms, with
pain, are examples of such values considered for \thi{their own sake},
as goods \thi{in themselves}.  What distinguishes them from the
\co{spiritually founded} values of this level is the exclusive
character of the project aiming merely and only at achievement of
pleasure.  But it is more of a mere possibility, in any case a rare
phenomenon, because one will almost always find oneself involved in the
levels of \co{actuality} and \co{miness}, which involvement will
modify the character of values of \co{immediacy}, will, so to say,
restrict their claims to absoluteness.  Such restrictions will not
differ much from the restrictions originating in the \co{spiritual
foundation}.
%I call these the \co{sensous} or
%\co{hedonistic} values.



%%%%%%%%%%%% was 1.5.1 in Book II: ``Some examples of distinct levels''


\subsubnonr{Creativity ?}
     
\pa Like freedom, and many other supposed \thi{faculties} of a soul,
creativity is not any \thi{faculty} which could be applied
indiscriminately to this or that, whatever the actual content might
be.  One can hear talk about \wo{promoting creativity}, \wo{creative
persons}, and the like.  What Valery noticed in the beginning of
XX-th century, can be still observed, pehraps even more often: today 
we do not speak about God's creation -- 
%not only a reporter, a story-teller but 
even a cook may happen to be \thi{creative}.

Creativity is always relative to a particular 
sphere of Being and is thoroughly related to {\em this}
sphere. An inventive, ok, creative mathematician may be entirely 
boring and uncreative in personal relations or any other contexts. 
The following points do not distinguish kinds of creativity 
with respect to any quality of the created 
contents, but only to the sphere in which it finds its expression. 
Also, they do not imply that the level of engagement and intensity of 
the creative process is limited to the level of expression -- the most 
trivial invention may consume the whole person, although this would 
probably be some form of psychosis.
%\citt{even cook\ldots}{Valery}

\pa
\inv The highest form of creativity is \co{creation}, emergence of 
\co{invisibles} from \co{nothingness}. It is not an affair of human 
\thi{subjectivity}, even if it takes place in the center of personal 
being. Only Godhead 
creates from \co{nothing} and as \co{creation} remains a mystery, 
there isn't much more to say about that.

\pa \mine The highest form of creativity of which man is capable is to
create good from evil.  This happens only at the personal level
concerning either \co{myself} or another person.  Although involving
\co{concrete acts}, it is not a matter of \thi{producing} anything
\co{visible}.  It domesticates \co{the world}, brings order into chaos
creating an inhabitable sphere for human being.  We will say more on
this aspect of creativity in the last Book.\footnote{I owe the thought
that only God creates from nothingness while humans can, at best,
create good from evil, to the philosopher, poet and priest J\'{o}zef
Tischner.}


\pa \act In the sphere of \co{actual} creations, one can bring
something \co{non-actual}, eventually even \co{invisible}, to
\co{actuality}.  This form is exemplified by a genuine artistic
creativity through which \co{vague} contents are given \co{actual}
form.  (Creative thinkers and some researchers, at least in some
areas, would also fall into this category.)  A genuine work of art is
an essentially multivalent \co{expression} amenable to polysemous
interpretations.  But each and all such interpretations are never able
to capture the \thi{essence} of the truth revealed by the interepreted
work of art.  To the extent the latter \co{expresses}, it overflows
with meanings for which the \co{actual} categories of \co{reflection}
are always too poor and narrow.  A genuine work of art makes \co{present}
something which simply can not be fully and unambiguously interpreted
-- the ressentment and animosity many artists feel towards the art
critics are easily understandable phenomena.

\pa \imm Finally, there is a mere conversion of one \co{actual}
content into another.  Bad art often does that, but the more genuine
examples would involve scientific discoveries and technological
inventions.  \wo{Invent} is a proper word for this form of creativity,
which derives simply from \la{invenire} -- \wo{to find}.  I am not
trying to dismiss the great deal of intelligence and work which may be
needed.  This, however, does not change the fact that 
this kind of \thi{creativity} is only transformation of \co{actual}
contents, finding a new constellation of given, even if invented,
elements.  Unlike the artistic creativity, it dwells exclusively in
the \co{external}, \co{objective} contents.  An artist, too, will
often be able to repeat after Picasso \wo{I do not search, I find.},
but this \thi{find} is of a very different order -- an artist finds
something which, although \co{manifest} in the found \co{expression},
still remains beyond or \co{above} it, while a scientific (not to
mention technololgical) invention is a fully \co{actualized complex}.

%\levs{10}{Creativity}
% {\co{actual} from \co{actual} (science, technlgy): invent -- \la{invenire} = `find' [MHS, p.69]}
% {\co{actual} from \co{non-actual} (artist, thinker)}
% {good/cosmos from evil/chaos (man)}
% {something from nothing (Godhead)}
%     
%\pa Creativity is \co{expression}.

%conformance to the origin
%
%actualization of the non-actual (only signs)
 

%\subsubnonr{Reality ?}
%\input{0338real}
