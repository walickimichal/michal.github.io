\chapter*{Preface} \pagenumbering{arabic}
\addcontentsline{toc}{chapter}{Preface}
%
What is my goal? I do not have any. It certainly is not: to strengthen faith, to
weaken faith, to strengthen faith in reason, to weaken faith in reason, to
defend science, to attack science, to promote investigations into some
direction, to encumber investigations in some direction, to embrace ecstatically
words and defeat representations, to restrain the cacophony and to promote
controlled representations\ldots What is my problem? I do not have any.
If I had, I would try to solve it rather than write a book, and if I wrote a
book to solve my problem I would not tell you that. What new do I have to offer?
Nothing -- I am neither a scientist nor a news reporter. (As a matter of fact, I
expect you to know most things I have to tell.)

Finding answers to such questions in the introduction means today that one can
do well without reading the book. Its detailed table of contents can be
summarised briefly: 1.~The great (or a lesser) challenge, 2.~The hero facing the
challenge,  
3.~The final victory, 4.~Concluding remarks (whose apparent
modesty equals only the arrogance of the underlying attitude). Such books,
produced at the rate exceeding the production of films with the same manuscript, 
though usually more vulgar contents,
sprinkled with broad knowledge of facts and expressions of acquaintance
with the recent events at certain cultural level, or even (in the braver cases)
at some lower level,\noo{though then written with a clearly squint eye),} nourish 
the intelligent mob (not only in the country where most of them are
written), increasing its fascination even faster than its general confusion
and dissatisfaction. For all that is very smart and intelligent, very smart and
intelligent indeed, offering breaking discoveries and final explanations and
sounding, if not like the trumps of the doom so, at least, like the honks of
applications for research grants or of ending tenure tracks. Good is no longer
enough. Spectacular, breath-taking, amazing, wonderful or -- on the other but in
fact the same note -- useful, applicable, pragmatic, advantageous,
beneficient\ldots One reads, one learns a lot and, in the end of the day, one knows as
little, or even less, than in the beginning. 

Are you not tired? I am. {\em This} is a useless rumination offering nothing new
and proving absolutely nothing -- at best, suggesting only some thoughts over
things known for millenia.  

\tsep{...}

Not seeing differences witnesses to dilettantish simplifications. But is not
simplicity also the ultimate goal of scholarly thoroughness: to simplify the complicated,
to arrive at the unity which -- perhaps in a different sense but still --
removes contradictions discovered and produced by the
legions afraid of being called simple-minded? Every simplicity can be
accused of simplification, while many simplifications can be misunderstood as
simplicity. Where is the line separating the two? Isn't it, at least to some
extent, in the ear and eye of the beholder? You tell me. Avoiding complications
may, so it seems, end up as one or as other...


\tsep{???}

%\subsubnonr{anthropology, account, summa}
%\chapter*{Preface} \pagenumbering{arabic}
\addcontentsline{toc}{chapter}{Preface}
%
What is my goal? I do not have any. It certainly is not: to strengthen faith, to
weaken faith, to strengthen faith in reason, to weaken faith in reason, to
defend science, to attack science, to promote investigations into some
direction, to encumber investigations in some direction, to embrace ecstatically
words and defeat representations, to restrain the cacophony and to promote
controlled representations\ldots What is my problem? I do not have any.
If I had, I would try to solve it rather than write a book, and if I wrote a
book to solve my problem I would not tell you that. What new do I have to offer?
Nothing -- I am neither a scientist nor a news reporter. (As a matter of fact, I
expect you to know most things I have to tell.)

Finding answers to such questions in the introduction means today that one can
do well without reading the book. Its detailed table of contents can be
summarised briefly: 1.~The great (or a lesser) challenge, 2.~The hero facing the
challenge,  
3.~The final victory, 4.~Concluding remarks (whose apparent
modesty equals only the arrogance of the underlying attitude). Such books,
produced at the rate exceeding the production of films with the same manuscript, 
though usually more vulgar contents,
sprinkled with broad knowledge of facts and expressions of acquaintance
with the recent events at certain cultural level, or even (in the braver cases)
at some lower level,\noo{though then written with a clearly squint eye),} nourish 
the intelligent mob (not only in the country where most of them are
written), increasing its fascination even faster than its general confusion
and dissatisfaction. For all that is very smart and intelligent, very smart and
intelligent indeed, offering breaking discoveries and final explanations and
sounding, if not like the trumps of the doom so, at least, like the honks of
applications for research grants or of ending tenure tracks. Good is no longer
enough. Spectacular, breath-taking, amazing, wonderful or -- on the other but in
fact the same note -- useful, applicable, pragmatic, advantageous,
beneficient\ldots One reads, one learns a lot and, in the end of the day, one knows as
little, or even less, than in the beginning. 

Are you not tired? I am. {\em This} is a useless rumination offering nothing new
and proving absolutely nothing -- at best, suggesting only some thoughts over
things known for millenia.  

\tsep{...}

Not seeing differences witnesses to dilettantish simplifications. But is not
simplicity also the ultimate goal of scholarly thoroughness: to simplify the complicated,
to arrive at the unity which -- perhaps in a different sense but still --
removes contradictions discovered and produced by the
legions afraid of being called simple-minded? Every simplicity can be
accused of simplification, while many simplifications can be misunderstood as
simplicity. Where is the line separating the two? Isn't it, at least to some
extent, in the ear and eye of the beholder? You tell me. Avoiding complications
may, so it seems, end up as one or as other...


\tsep{???}

%\subsubnonr{anthropology, account, summa}
%\chapter*{Preface} \pagenumbering{arabic}
\addcontentsline{toc}{chapter}{Preface}
%
What is my goal? I do not have any. It certainly is not: to strengthen faith, to
weaken faith, to strengthen faith in reason, to weaken faith in reason, to
defend science, to attack science, to promote investigations into some
direction, to encumber investigations in some direction, to embrace ecstatically
words and defeat representations, to restrain the cacophony and to promote
controlled representations\ldots What is my problem? I do not have any.
If I had, I would try to solve it rather than write a book, and if I wrote a
book to solve my problem I would not tell you that. What new do I have to offer?
Nothing -- I am neither a scientist nor a news reporter. (As a matter of fact, I
expect you to know most things I have to tell.)

Finding answers to such questions in the introduction means today that one can
do well without reading the book. Its detailed table of contents can be
summarised briefly: 1.~The great (or a lesser) challenge, 2.~The hero facing the
challenge,  
3.~The final victory, 4.~Concluding remarks (whose apparent
modesty equals only the arrogance of the underlying attitude). Such books,
produced at the rate exceeding the production of films with the same manuscript, 
though usually more vulgar contents,
sprinkled with broad knowledge of facts and expressions of acquaintance
with the recent events at certain cultural level, or even (in the braver cases)
at some lower level,\noo{though then written with a clearly squint eye),} nourish 
the intelligent mob (not only in the country where most of them are
written), increasing its fascination even faster than its general confusion
and dissatisfaction. For all that is very smart and intelligent, very smart and
intelligent indeed, offering breaking discoveries and final explanations and
sounding, if not like the trumps of the doom so, at least, like the honks of
applications for research grants or of ending tenure tracks. Good is no longer
enough. Spectacular, breath-taking, amazing, wonderful or -- on the other but in
fact the same note -- useful, applicable, pragmatic, advantageous,
beneficient\ldots One reads, one learns a lot and, in the end of the day, one knows as
little, or even less, than in the beginning. 

Are you not tired? I am. {\em This} is a useless rumination offering nothing new
and proving absolutely nothing -- at best, suggesting only some thoughts over
things known for millenia.  

\tsep{...}

Not seeing differences witnesses to dilettantish simplifications. But is not
simplicity also the ultimate goal of scholarly thoroughness: to simplify the complicated,
to arrive at the unity which -- perhaps in a different sense but still --
removes contradictions discovered and produced by the
legions afraid of being called simple-minded? Every simplicity can be
accused of simplification, while many simplifications can be misunderstood as
simplicity. Where is the line separating the two? Isn't it, at least to some
extent, in the ear and eye of the beholder? You tell me. Avoiding complications
may, so it seems, end up as one or as other...


\tsep{???}

%\subsubnonr{anthropology, account, summa}
%\chapter*{Preface} \pagenumbering{arabic}
\addcontentsline{toc}{chapter}{Preface}
%
What is my goal? I do not have any. It certainly is not: to strengthen faith, to
weaken faith, to strengthen faith in reason, to weaken faith in reason, to
defend science, to attack science, to promote investigations into some
direction, to encumber investigations in some direction, to embrace ecstatically
words and defeat representations, to restrain the cacophony and to promote
controlled representations\ldots What is my problem? I do not have any.
If I had, I would try to solve it rather than write a book, and if I wrote a
book to solve my problem I would not tell you that. What new do I have to offer?
Nothing -- I am neither a scientist nor a news reporter. (As a matter of fact, I
expect you to know most things I have to tell.)

Finding answers to such questions in the introduction means today that one can
do well without reading the book. Its detailed table of contents can be
summarised briefly: 1.~The great (or a lesser) challenge, 2.~The hero facing the
challenge,  
3.~The final victory, 4.~Concluding remarks (whose apparent
modesty equals only the arrogance of the underlying attitude). Such books,
produced at the rate exceeding the production of films with the same manuscript, 
though usually more vulgar contents,
sprinkled with broad knowledge of facts and expressions of acquaintance
with the recent events at certain cultural level, or even (in the braver cases)
at some lower level,\noo{though then written with a clearly squint eye),} nourish 
the intelligent mob (not only in the country where most of them are
written), increasing its fascination even faster than its general confusion
and dissatisfaction. For all that is very smart and intelligent, very smart and
intelligent indeed, offering breaking discoveries and final explanations and
sounding, if not like the trumps of the doom so, at least, like the honks of
applications for research grants or of ending tenure tracks. Good is no longer
enough. Spectacular, breath-taking, amazing, wonderful or -- on the other but in
fact the same note -- useful, applicable, pragmatic, advantageous,
beneficient\ldots One reads, one learns a lot and, in the end of the day, one knows as
little, or even less, than in the beginning. 

Are you not tired? I am. {\em This} is a useless rumination offering nothing new
and proving absolutely nothing -- at best, suggesting only some thoughts over
things known for millenia.  

\tsep{...}

Not seeing differences witnesses to dilettantish simplifications. But is not
simplicity also the ultimate goal of scholarly thoroughness: to simplify the complicated,
to arrive at the unity which -- perhaps in a different sense but still --
removes contradictions discovered and produced by the
legions afraid of being called simple-minded? Every simplicity can be
accused of simplification, while many simplifications can be misunderstood as
simplicity. Where is the line separating the two? Isn't it, at least to some
extent, in the ear and eye of the beholder? You tell me. Avoiding complications
may, so it seems, end up as one or as other...


\tsep{???}

%\subsubnonr{anthropology, account, summa}
%\input{00pref}
But if such a useless nothing, why \la{summa}? 
What could possibly make one write a \la{summa}, of any kind, nowadays?
Complexity -- not only of the incomprehensible totality of the world, but even
of every single issue -- makes it look pretentious. Things fall apart
and observing the dissolution or, as many a one pretends, praising it as the
openness unto plurality, is the only reasonable attitude, in fact, the only
politically correct way of distancing oneself from the trauma and danger of monism,
that is, totalitarianism.

Yet, philosophy which does not try to capture any unity ends up as a catalogue
of particular cases, particular concepts or just words, which may be as
elaborate, intelligent and intricate as it is uninteresting and useless.
Philosophy which distances itself from any attempt at reaching some wisdom, and
that means in particular, thinking and relating to the ultimate questions, 
ends up calling its impotency for modesty and glorifying the incessant
questioning, that is, public scratching one's head with the emanating
self-assurance that all genuine issues should be left to those who are
unintelligent enough to expect any answers. After all, it takes an analphabet to
believe that truth is written some place.
% But receiving guidance does not exclude autonomy and it appears to us quite
% unfortunate that
%

Philosophy attempts to think what one knows. For to know means much more than
what Plato explains in \btit{Timaeus}, or what others have tried (and are still
trying) to specify with
respect to explicit, reflective thoughts, \gre{episteme}. Sure, I know the
pythagorean theorem and the way from my house to my work. But I also know that
I will die, I know my love of my mother, I know how she loves me, I know what my
best friend likes and of what my girl-friend is afraid. I know that I am the
same person I was yesterday and I know that, from behind the invisible forces,
the hidden eye of God's  watches everything. Most of these things I am not
actually able to think, not to mention, to express precisely and explain with
any degree of stringency.
%
We know what philosophy is, don't we? We know what is {\em not} philosophy and
we know what is, at least, when we encounter it. We can even tell good
philosophy from bad one. And yet, can anybody tell what we thus know? Can
anybody think this knowledge clearly enough to express it precisely to
everybody's, or even only to one's own, satisfaction? 
%

Knowing, we might perhaps say, is what allows me to
consent to some thoughts and to reject others. For when I content to the truth of
a thought, I do it in the light of something I know. Only in very special cases
this knowledge consists of other, explicit thoughts.
%
To think what one knows is a challenge, just like it is 
a challenge to become what one is. Only to a self-satisfied rationalism such challenges
may appear as paradoxes, and only to a narrow-minded pedantry as contradictions.

The attempt to think what one knows need not (as it seems, can not any longer)
rely on necessary principles and forcing arguments. Those who try
\citet{withholding their consent from any proposition that has not been
  proved}{CiceroGods}{ I:1. [In the translation of F.~Brooks: \wo{refusing to
    make positive assertions upon uncertain data}.]} end up with absolute 
certainty -- about nothing. The rigidity of irrefutable argumentation which
attempts to force its conclusions and tries to dispense with everything which
escapes such attempts seems, indeed, to empty every phenomenon of its
concreteness yielding only residual site whose necessity equals its hollowness.
But the lack of the universally forcing arguments is not the same as the lack of
any truth, the lack of the necessary laws is not the same as the lack of any
order. Besides (or instead of) the effcient causes there may be many necessary
ones, besides (or instead of) the sufficient reasons there may be many
insufficient -- but, as Weber would say, favourable and supportive -- ones. Just
like explanations try to capture the former, so a more modest account may try to
identify only some of the latter.

\citeti{I say the following about the Whole ... Man is that which we all
  know.}{Democritus}{ DK 68B165}
Philosophical anthropology tries to give an account of human life. This does not
mean isolating this specific \thi{object} and ignoring all the rest. It amounts
only to seeing all the rest in a specific perspective: not as a set of
inviolable mechanisms but as the field of unfolding of human existence.  The
dualism of deteminism confronting freedon, of \thi{objectivity} confronting
\thi{subjectivity} is only a 
result of a particular way of thinking  the existential
confrontation which precedes the understanding -- and dissociation -- of these
two (and other) aspects.  The two, posited as the ultimate poles, can never not
only agree but even meet and one is constantly forced to the impossible choice
of the one or the other. Above and across this impossibility one may try to draw
the border between the two in ever new ways.  For even if the number of the
elements, say, of the constitutive dimensions of existence, is limited, the concrete
borders between them may be drawn in an unlimited variety of ways. Every human
being draws these borders anew and the abstractness of the fact that somebody
else did it earlier in a similar way does not in the slightest diminish the
concrete need to do it ever anew.\ftnt{Besides, \thi{limited} can be, in all practical
  respects, as good (or bad) as infinite. On how many different points can any
  two philosophical systems differ?  Ridiculous question, but to play the game
  only for a while, let us say some number which certainly is lower than it is
  in fact, say: 100. Simplifying further, let us say that on each of these
  points one has only a binary choice, yes or no, + or --. How many different
  \thi{systems} do we get? Well, $2^{100}$.\noo{This is, if we were to believe
    such calculations, approximately the square (perhaps cubic) root of the
    estimated number of atoms in the universe.} If humankind produced one such
  system every second, it would take some $2^{75}$ or, rounding off, $10^{20}$
  years to merely produce them all. This is a bit more than twice the estimated
  age of the universe.\noo{10^9 And merely producing them is not even the
  beginning of anything.} Thus, even if the 
  number (of possible forms of existence, of possible philosophical systems)
  were finite, the simple combinatorics is on the side of 
  unrepeatability -- not principal, however, not in some ideal infinite limit,
  where indeed the finite number of possibilities would have to be reapeated infinitely
  many times, but only in practice, that is, in fact. (If one wanted now to
  reduce the number 100 to, say, 20 such points,\noo{ at 
  which two systems might differ} it still leaves over 1.000.000
  possibilities.)} 
%
Everything has been said before -- {\la{mundus senescit}} (\wo{the world has
  aged}) noticed St.~Gregory of Tours in the VI-th century and some 1500 years
before him a preacher observed \citeti{[t]he thing that hath been, it is that
  which shall be; and that which is done is that which shall be done: and there
  is no new thing under the sun.}{Eccl.}{I:9} This may sound a bit depressing
but only to an intellectual capable of dealing exclusively with abstractions, or to
the petty mentality occupied with even pettier novelties, one \citeti{consoled
  by a mere trifle, as it is distressed by a mere trifle,}{after
  \citeauthor*{Pensees}}{II:{136}} for which everything beyond the yesterday's
scandals and today's news is a boring repetition. But \citeti{what is right can
  well be uttered even twice.}{Empedocles}{ DK
  31B25\noo{\citaft{FirstPhil}{: 58, p.165}}} Moreover, to understand something
  said before one often must say it 
oneself. Every existence says anew something said before, but saying this
concretely amounts to drawing anew the distinctions in the matter of life, the
distinctions between abstractions like \thi{love} and \thi{indifference},
\thi{indifference} and \thi{impotence}, \thi{impotence} and \thi{thirst},
\thi{thirst} and \thi{lack}, \thi{lack} and \thi{illusion}, \thi{illusion} and
\thi{lie}, \thi{lie} and \thi{truth}\ldots The infinite concreteness lies in
such distinctions and the lack of final definitions is only another side of this
concreteness. And the lack of rigid definitions does not mean the lack of
significant distinctions.

Giving an account of -- rather than explaining -- human existence, we are not
very concerned with many accepted distinctions. We certainly have to keep
various points of the discourse at approximately the same level of abstraction
but this need not prevent us from addressing also issues which the
administration of academic life places at different
departments. \citet{Philosophy is first of all a science about human being,
  about integral human being and of integral human
  being.}{Bier}{I:2\kilde{p.21}} 
% The advantage of philosophical anthropology is that no posited
% objectivities-in-themselves need to disturb us.
Asking a question, we will often rest satisfied with relating it only to
the place it occupies in the field of existence. Whether the question happens to
sound (or traditionally even {\em is}) theological, psychological, mythological or
astrological is of no importance, as long as it addresses an aspect of this
integrity. \wo{\la{Summa}} from the title refers to such a summary, to the 
attempt to 
gather quite different, sometimes even disparate, aspects into one unity and not
to the ultimate summit, 
to the vanity of collecting all relevant (and irrelevant) details in a systematic
and scientific totality.

%\input{hee}

\tsep{}

One of such questions, perhaps the only question of all philosophy, is:
%
%\equ{What is true?\label{questA}}
\begin{center}What is true?\end{center}
%
Some try to answer it in the way it is asked, that is, in
dissociation from any person asking it. Such an answer amounts always to
specifying what one, everybody, you and I {\em should} accept as true.
The \thi{should} is usually surrounded by the arguments and proofs which should
convince everybody. 
Unfortunately, any respect expressed by imputations of rationality, accusations of
irrationality, expectations of a direct and infallible communication is, 
 at best, merely the respect for some \thi{rationality}. As far as 
a personal meeting with the reader, that is, as far as the person of the reader
is concerned, it shows disrespect equal only to that displayed by the strangely popular
reluctance to express one's meanings in an understandable form. The difference
of proceeding results only in that sometimes the latter, but never the former,
can be excused on the assumption of incapacity. 

The best, in fact, the only thing one can do is to answer this (as any other)
question for oneself or, what amounts to the same, to accept some known answer.
The answer may be communicated to others but only as {\em my} answer -- I may be
convinced that you should accept this answer, too, but I do not even try to
convince you about that. What you do with my answer is your sake, in fact, to
understand it you have most probably already known it, even if you did not think
it yourself in the same way. 

\citet{Gradually it has become clear to me what every great philosophy
so far has been: namely, the personal confession of its author and a
kind of involuntary and unconscious memoir [\ldots] In the philosopher
[\ldots] there is nothing impersonal.}{BeyondGE}{ I:6} Moreover, the
\citet{beliefs to which we most strongly adhere are those of which we should
  find it most diffcult to give an account.}{BergTime}{ II p.135}
% \citt{It is almost incredible that men who are themselves working 
% philosophers should pretend that any philosophy can be, or ever has 
% been, constructed without the help of personal preference, belief, or 
% divination.}{W. James, Essays in Pragmatism, I, p.24}
Thus, since this, like every other question, is asked -- and answered -- only by a
concrete person, we can take it to be the same as:
%
%\equ{What shall I accept as true?\label{questB}}
\begin{center}What shall I accept as true?\end{center}
%
Here one might object: one senses a difference, the element of subjectivity, or
perhaps relativity, in the latter which does not disturb the former. But the
difference is only apparent: every answer to any of these questions, can be used
also to answer the other. If something is true then, sure, I shall accept it as
such.  On the other hand, if I accept something as true it is because it is --
as far as I can see, imagine, feel, understand, speculate -- true.  Unlike the
former, this later implication has been judged problematic, mainly because one
imagines some voluntary act of acceptance which one's subjectivity could decide
to perform as it wished. The meaning of this implication will be our main theme
and its validity will rest on the fundamental difference between relativity and
subjectivity.

{Relativity of every observation and conclusion to the
  subject making this observation and drawing this conclusion seems to be the
  most obvious thing in the world. However, one can draw quite the contrary
  conclusions from this observation. On the one hand, one can conclude that this
  relativity for ever prevents us from gaining an insight into the true nature
  of the world, into the true nature of things as they are \thi{in themselves}.
  The underlying assumption is that things indeed are something specific \thi{in
    themselves}, and even that they, at least in principle, can be described
  \thi{in themselves}, as if independently from the view of the one making the
  description. The contradictory nature of such a project seems hard to accept.
  So, on the other hand, one can draw the conclusion that there are no things
  \thi{in themselves} and, consequently, no measure of truth -- in short, that
  relativity amounts to complete relativism and truth to mere (perhaps even
  merely volitional) subjectivity. But does the fact that the specificity of
  things \thi{in themselves} is only ideal and posited, show the ultimate void?
  Does relativity of particular truths exclude absolute truth? 
Both these interpretations start from the assumed duality. The
intuition that the standard of truth carries an element of transcendence is
opposed to the observation that absolute transcendence, dissociated and remote
from any immanent view, is inaccessible and hence, at least in the latter case,
irrelevant. We will instead take relativity as the basic phenomenon which
  precedes and underlies various dualities like subject vs. object,
  immanence vs. transcendence, acceptance as X vs. being an X, etc.}

\ins{ [Reality -- is between] "Now, the quantum postulate implies that any observation of atomic phenomena will involve an interaction with the agency of observation not to be neglected. An independent reality, in the ordinary physical sense, can neither be ascribed to the phenomena nor to the agencies of observation..."

Bohr, N. (1934). Atomic Theory and the Description of Nature. Cambridge, Cambridge University Press (p. 53).

 N. Bohr, The Philosophical Writings of Niels Bohr, p. 54, quoted in
 T. J. McFarlane, Quantum Mechanics and Reality, 1995 (quotes also: 

All things -- from Brahma the creator down to a single blade of grass --
are. . .simply appearances and not real. 
[Shankara, Crest Jewel of Discrimination, tr. Chris Isherwood, p. 97])

apparently well-menaing and also well-thinking people associate the two...
}

Taken contra-positively, the implication from my acceptance of X as true to X's
truth says that I should not accept any untruth.  Emphasizing this direction
amounts thus to a more modest project: not necessarily finding {\em the} truth
but rather, or at least, simply avoiding untruth. In the able hands of a
quibbler this modesty might reduce to a mere scepticism but we will hopefully
manage to say also something meaningful.  We should only keep in mind that we
are not following the fashion of {\em reducing} the first question to the second
(which results in the interminable quest for \thi{subjective} criteria of
\thi{objective truth}). We claim simply that the two questions are genuinely
equivalent, that everything \citet{that is known, is comprehended not according
  to its own force, but rather according to the nature of those who know
  it.}{Console}{ V:4.25\noo{154}} \noo{\citet{everything which is cognizable, is
    cognized not according to its own power, but according to the powers of the
    cognizing.}{Console}{ V:4.25 \noo{Filozofia Sredniowiecza,p.106}} }

Giving only an account -- rather than explaining -- we will ignore many
distinctions. We will not, for instance, treat the above question as {\em the}
question, focusing on which demands subordination and dissociation of
 all the others. Every genuine problem of human
existence invovles necessarily all the others. One can frequently meet attempts
to address a specific \thi{philosophical problem}, elaborate in the scope of a
single paper the problem of free will, the problem of meaning, the problem of
truth.  Interesting and perhaps even to some extent legitimate as such attempts
may be, they reveal the prejudice that such a division is at all possible, that
distinct problems indeed can be treated in a relative independence; eventually,
that only the closest scrutiny of the most minute distinctions is able to give an
adequate description of any single issue. But the problem is that we not only do
not quite know how one problem could or should be addressed and approached
-- we do not quite know {\em what} any particular problem is. Attempting to isolate any
particular problem for a separate treatement, we fight first with circumscribing it in
any reasonable way which, however, {\em never} reaches the goal of becoming
entirely satisfactory. In this process of 
partial circumscription we invest the problem with all the relations and
implications it carries to all the other problems. 
If one tries to address, say, the issue of freedom without at the same time
illuminating the meaning of subjectivity and openness to truth, the sense of
meaningfulness, the presence or absence of the absolute, in short, without
addressing the intergrity of whole existence, one ends up with the
distinctions one started with and keeps opposing arbitrariness to determinism,
subjectivity to objectivity, spontaneity of feelings to the rationality of some 
inviolable laws, etc.  Every (not only fundamental) issue is the sum of what it
excludes, is the border contracting the tension between this issue and others
into which it is interwoven. In a bit strange (but, in fact, quite understandable)
dialectics, the tradition which had marked the XX-th century with the missionary
zeal of dissociating all the issues and bringing them under objective, systematic and
separate analysis, ends up with the holistic and coherentist postulates, whether
with respect to language, meaning or truth.  \noo{(Wittgenstein,
  Quine, Davidson)} Even though one can not forget the idealistic origins of all such
potulates, one would still like to deny their 
idealistic connections. And one almost manages that, at least, 
as long as one keeps dissociating, as long as one sticks to dissecting one 
particular issue at a time.

We will not try to establish any totality which, in the presence of all too many
accepted distinctions, would indeed be a vanity. We will therefore ignore many possible
distinctions -- not because they would necessarily destroy any unity we might
wish to find but because they would (tend to) completely obscure it. 
We hope to avoid the accusations of relating the unrelated by
making at least plausible that the intimate affinity and kinship of vague yet
distinct aspects, their genuine unity preceds more rigid dissociations, and that
the latter mark only the end -- or perhaps the middle, but certainly not the
beginning -- of the road.
%
\noo{We will certainly try to avoid pitfalls of pantheism (of which most such holisms
are examples), but only because unity can be found above it and not, as some
also trying to avoid the same claim, because it does not obtain at all.
}
%

% \addcontentsline{toc}{section}{Sources, references and conventions}
\input{00sc}


%%%%%% END
\noo{%END
In fact, most if not all meta-discussions in modern philosophy,
arise as a consequence of elaborating the aspects discernible in (\ref{questB})
but not in (\ref{questA}): the idealism versus realism, subjectivism or
perspectivism versus
objectivism, correspondance versus verificationism or pragmatism, tradition
inspired by phenomenology versus analytical scientism, existential orientation
versus linguistic analyses,...

We do not dismiss all these discussions as completely irrelevant but our first
goal is not to be drown in the methodological and conceptual meta-perspectives. After
all, they too are, at least originally, motivated by the interest in an answer
to the first question.

An important aspect of the second question, not present in the first, concerns
the suggested need of justification. If I am to accept anything, it better be
sufficiently justified. This element has overshadowed philosophy if not since
its Platonic beginnings, so since Descartes. It has been used to distinguish
philosophy suggesting, in fact, that it is an existential difference which, in
case of a philosopher, makes him rely exclusively on \thi{reason} while, for
instance in case of a theologian on some \thi{faith}, while in case of an
average man on \thi{casual opinions of common-sense}.  This differentiation,
reflecting already preoccupation with the second question, has of course nothing
to do with the answer to the first question. But philosophers tend to make it
relevant by claiming that only their \thi{reason} can serve in answering the
first question. Various accussations of
\thi{irrationality} follow. Unfortunately, more often than not, justifications of 
claims to \thi{rationality} or even \thi{true rationality}, and in any case the
results of following them, amount to selecting only 
some truths which, as it happens, can be assessed by the \thi{reason}, or what
is actually meant by it.

Such differentiations are, as a matter of fact, political issues, issues of
delegation of competences, allocation of educational responsibilities, division of
faculties and departments or even, as it happened in
post-Cartesian Repubic of Unified Provinces, of national conflict between the
conservatists and liberal republicans.\ftnt{Cartesians supported liberation of
  philosophy from its subordination to theology, while ortodox Calvinists
  wished not only such a dependence but a state underlied religious
  goverment. As any issue with a theological element was almost bound in the
  Republic of that time to turn into a political one, so did this one.}
We do not dismiss all such discussions as completely irrelevant, but we think
that they truly belong to politics. Who is responsible for what? and Who is
entitled to what kind of questions? -- is it reasonable to believe that
answering such questions may help answering (\ref{questA})? 
The problem is that they did not
manage to produce any certain measure of truth which could be used in evaluating
the proposed answers to the first question. And, in fact, if they were to
produce such a measure, it could arise only as a consequence of answering, at some
point or another, the first question. Of course, I shall accept as true only
what is true. The attempts to elaborate on the second question do not bring us
any closer. The difference between the two can be seen by comparing them to the
respective questions below:
\equ{What is important?\label{questAA}}
and
\equ{How should I figure out what is important?\label{questBB}}
We can iterate the meta-appications past (\ref{questB}) -- \wo{On what basis
  shall I accept something as true?} -- just like past
(\ref{questBB}). The infinite regress is very much like the infinite
regress of formal reflection. And, in fact, just like the formal regress of \wo{I think
  that I think that...} adds nothing substantial to the first thought, and only
enmeshes it in a  formality of a childish mechanism, so the formal regress
of our meta-applications does not clarify anything with respect to the first
question, but only multiplies the possible issues, doubts and problems one may
explicitly rise and discuss.

One may easily object that, as a matter of fact, when the answer to
(\ref{questAA}) is unknown and hard to find, one could get some help from
answering first (\ref{questBB}). Formally, it may indeed seem so, but it is a
pure formality.  Notice that questions (\ref{questA}) and (\ref{questAA}) do not
concern any particulars which 
might happen to be special cases of more general laws, possibly useful to know
in treating the special cases. No, these questions have uncanny level of
generality which it seems futile to generalize further.  To such an objection we
can only answer: Sorry, disagree completely, for any answer to (\ref{questBB})
will, in fact if not in principle, 
require at least a partial answer to (\ref{questAA}). It may certainly help to ask
another, and related question, but approaching an issue from a different
perspective is not the same as approaching it from above. The only things that
may happen in the latter case are that one either falls down or flies away.


\subsubnonr{vs. arguments; Different tempers}

Yet, philosophy is only a particular expression of the fundamental
differences of human existence:... Simplifying to the extreme
opposition, the two tendencies are those of one vs. many, Plato vs. Aristotle,
Aquinas vs. Ockham, Husserl vs. James, unity vs. plurality. 

Different tempers...


\sep

What makes Anaximander the first philosopher is that he tried to argue for and
justify his claims, he used arguments and not mere statements. This is, at
least, the general view of ... philosophers. The \thi{love of wisdom} with
which one started, has long ago become the love of argument, to the extent that
wisdom without argument goes for simplicity if not stupidity.  One insists on
the arguments the more, the more one feels threatened, that is, the less one
feels certain of own self-identity.

Thus one has managed to separate itself from
literature with its usual lack of and occasional distaste for intellectualism.
It was a bit more difficult to cut off theology but here, too, the fact that
some aspects were assumed to be indisputable helped. Unfortunately, along with
that all the concern for the divine presence in the world was removed too.
Indeed, how would one argue about God, immortality and other invisibles? Their
invisibility amounts more or less exactly to the inadequacy of any arguments;
after one got tired of proofs of God's existence, the whole \thi{topic} became
highly inadeaquate, because incommpatible with the image of precise  logical
argumentation.

The strange thing that so happened was the emergence of the perfectly univocal
definition of valid arguments and mathematical precision in their study. Formal
logic, and its subfield of comutability and recursion theory, put the final
period after the millenia of developing Aristotle's syllogism, Leibniz's
calculus of reason and the ideas of mechnical reasoning. We now know that, in
itself, it yields absolutely no insights and, moreover, that no mechnical
procedure can ever be found for generating valid results...



\sep

The title is used by others....
I could have found another title but this does capture the
essential feature of existential situation which is the
only object of this book


\wo{Philosophical antropology} is but another way of saying
\wo{description of the existential situation}.


Ignore what is not experienced; what is really, objectively, in-itself is as
little relevant for this description as the answer to the question whether the
universe will continue expanding or starts at some point receding.

We study human experience and how its various aspects are constituted. This is
the opposite, in fact, antagonistic approach to the one which starts with some
ready constituted elements and tries to explain the construction, emergence and
functioning of a mechanism. One never knows if the given building blocsk are
really the eventual ones, nor even if they are appropriate for the task one
tries to perform. We leave such exercises to the empirical project. Whether it
succeedes or not won't change one fundamental thing: the human experience. We
know that it is Earth rotating around the Sun, and yet we can not stop talking
about -- and in fact, feeling and living as if it was -- the sun rising up at
the horison. This is how things look from our standing point and no amount of
empirical or logical proofs and arguments, no amount of bad or good science,
will ever be able to change that, unless, perhaps, one starts messing up with
the human beings themselves (and we may still hope that genetics won't go that
far).

We are not interested in mathematics or physics, in biology or sociology -- at
most, in how such forms of modelling the world emerge within the horison of our
experience; we are not interested in objective time or space -- at most, in how
such forms emerge and penetrate our experience; we are not interested in what is
objective and what is not -- only in how such distinctions may be relevant for
us; we are not interested in life -- only in the feeling of life, or better, in
living. 

This looks pretty bad, right? Pretty empiristic, subjectivisitc,
phenomenological, even idealistic, or perhaps just existentialistic! As a matter
of fact, it is not any of these, at least I hope it isn't.
But it is not any of their traditional opposites, either. It is only something
in the direction of 

\ad{Anthropology}
\citet{What is typical allows one to retain cold blood, only individually
  conceived matters cause nervous shock. In this consists the peacefulness of
  science.}{Faustus}{XLV\noo{p.622}}

The only object -- human being and his confrontation with the
world. We can easily imagine the world without human beings, but
for philosophical antropology such a world has no relevance.
As we will try to show, such a world is only apparently imaginable
-- in fact, it is totally unthinkable or else, to the degree it is,
it is totally indistinct.
(This has nothing to do with its objectivity or subjectivity, its
reality or ideality.)

Thus philosophy which tries to present the world as it is in-itself,
as it is sort of independently from human being, may be perhaps
interesting, sometimes even enlightening, but it will never provide 
a completely satisfactory account. It may reduce its sphere to some
particular area and problems and study that. But its concepts and
results are then founded outside its scope...


Yet, not explaining but describing = choosing what matters and
placing it in relation to other aspects which matter = giving account;
which leads to a unity of understanding ....

\wo{To give account}...
\begin{itemize}\MyLPar
\item
  First of all: try to determine \co{that} before even asking \thi{what}.
  And then ask thi{what} before even thinking \thi{how}. If sometimes it happens
  that one answers \thi{how}, then in any case never ask \thi{why}...
\item
  Only (some) necessary conditions, hardly ever any sufficient ones -- they
  do not obtain. We are not after any explanations; perhaps we are after
  something which possibly might be attempted explained later on, but that is
  not so either; we are after something for which explanation is inadequate
  category...
\item
  give only necessary conditions; the sufficient ones do not obtain any
  way. Does it mean that I believe in miracles? Well, as you like it. I have
  never seen a sufficient reason for almost anything, and I would say that those
  who insist on them must actually {\em believe} that they obtain. And as they
  do not obtain (ok, except in mathematics, or some simple physical analogies),
  saying that we can see necessary conditions but not sufficient ones 
  seems to me a matter of simply conforming to what is experienced, not believing
  in anything. There is much less believing here than in the belief in
  sufficient reasons. But sure, if you want to call the transition from the
  given necessary 
  conditions for some X to the X itself for \wo{a miracle} then I
  do not {\em believe} in miracles -- I see them all the time.
\end{itemize}
\wo{Irrationalism!!!} may exclaim some others, but let them remain calm, too.
Nobody knows -- and those who claim to know can not agree -- what rationalism
is. It seems that, at least, it requires (do we notice a necessary condition?)
not being carried away, some sobriety. Or in less prosaic terms, rationalism
requires that one assumes a position only with the admittance of the possible
limits of its validity, if not with the actual knowledge of such limits.  If one
accepts this rather generous critical rationalism then we should be able to
complain, all the way. Even to the point where one asks about the limits of
validity of this very position...  (After all, this admittance of partiality
squares so nicely with the etymology of \la{ratio}! A bit worse with the
pretentions to universality...)

\sep

It is not for all...

\sep

Many more specific issues are only touched upon and one may wonder why at
all. But we are not trying to resolve all the issues, to come up with a definite
and final answer. In most cases such answers simply do not exist and we prefer
to say too little rather than too much. Multiplying distinctions and
perspectives may be as rewarding academically as it is existentially futile. 


\subsection*{The main points ...}
{\small{ \begin{enumerate}
\item \co{There is} $\sim$ the One -- the particulars are only 
``perspectives'', modifications of the \co{Is} (incommensurable for 
frog and man) \\
 It is the fundament of all experience and \ldots the fundamental 
 experience. (``Objectivity'' cannot be reduced to any experience because 
 it comes before. ``Externality'' can.)
\item \co{Distinction} is the begining; \co{point} --  \co{pure 
distinction} -- is its \co{reflection}
\item There is nothing beyond experience.
\item The levels of \thi{being} and transcendence coincide with the time-spans
 \begin{itemize}
 \item \co{transcendence} is essentially something present which is
 not exhausted in this presence, something \wo{more than it is}, the overpowering
 \item
 this \co{more} can be relative to various aspects but, primarily, it is related to the
 \co{horizon of actuality}
 \end{itemize}
\item The ``fourth level'' is not a level ``above'' the other three but
 concerns the ``essential structure'' of division into \LL\ and \HH. (It is not
merely ``formal''.)
\item The world ``consists of'' \LL\ and \HH\ --
 Man is a borderline between \LL\ (lower) and \HH\ (higher).
 \begin{itemize}\MyLPar
 \item the \wo{totality} of his being is not \wo{constructed out of points} but 
 precedes the \co{distinctions}.
 \item Before you construct out of pieces, the pieces must be there. \\
  \begin{tabular}{l|l|l|l|l|l}
    \HH & (phon)emic & whole (Gestalt) & mind & concrete & quality \\ 
    \hline
    \LL & (phon)etic & part & body & precise & quantity 
  \end{tabular}
 \end{itemize}
\item The basic existential mode is determined by \G\ (openness, acceptance) or \B\ 
 (denial, refusal) of the mastery of \HH.
\item Important are \co{nexuses} [tie, connection?! -- vs. \la{religare}] 
of \co{aspects} which can be reflectively dissociated but which 
function meaningfully only in the \co{nexus}
\end{enumerate}

\subsubsection{{...and defs}}
\begin{enumerate}
\item \co{Experience} -- whatever leaves a mark vs. the source of unpredictability.
\item \co{Reality} -- what you cannot live without.
\item ``natural attitude'' to external world -- possibility of new, unexpected, 
      not-from-me.
\item \co{Sharing} -- the same \HH.
\item \co{relevant} -- telling what to do in the face of transcendence.
\end{enumerate}
}} % end \small

\section{Preface}


Well, perhaps, I was not quite honest.  I want to give an account of a
possible understanding of the unity of \ldots Yeah, of what?  Of a
human being, of a person but, perhaps eventually, only of myself. 

\citt{Gradually it has become clear to me what every great philosophy
so far has been: namely, the personal confession of its author and a
kind of involuntary and unconscious memoir [\ldots] In the philosopher
[\ldots] there is nothing impersonal;}{Nietzsche, {\em Beyond Good and
Evil}, I:6} 
\citt{It is almost incredible that men who are themselves working 
philosophers should pretend that any philosophy can be, or ever has 
been, constructed without the help of personal preference, belief, or 
divination.}{W. James, Essays in Pragmatism, I, p.24}
The very same words may be deeply meaningful to some and ridiculously
superficial, even stupid to others. 
This book is not meant to {\em convince} anybody about
anything; I hope it will be found interesting by a few, and it is
addressed to these few.
% You find pretty quickly, by just starting reading, if you are among them.

What is my method?  I do not have any. 
\wo{Philosophy does not so much tell {\em what} to think as {\em how} to 
think.} All kinds of methodological postulates have polluted philosophy 
since it started to consider itself, or rather started 
{\em to try} to 
consider itself a science in the modern sense of the word\ldots
Positivism, pragmatism, phenomenology, all \thi{methodologies} of 
philosophical inquiry tried first of all to ape 
science, to achieve the same level of precision\ldots

But \thi{how} is only, and only at its best, a pale reflection of
\thi{what}, a method is at best only a desiccated sediment of a
particular way of understanding, which is dictated, if not determined,
by the particular objects or sphere of experienced addressed. 
Scholastics knew it very well, but we do not like scholastics very
much nowadays, do we?  Sure, \thi{how}, having extracted some resdiual
sediments of \thi{what}, turns easily into a socio-political
phenomenon, a \thi{school} or a \thi{party}, which pretends to know
{\em the} axioms, and in any case at least {\em the} rules of the
game.  But the game goes on and the only rules are to be
understandable, and then to have something worth understanding. 

\thi{How} a person thinks is merely an expression of \thi{what} he
thinks -- uncovering, perhaps, some hidden or unconscious assumptions,
but still only assumptions about the \thi{what}. To dissociate the two is a
violence against concrete thinking. Useful and justifiable as it may 
be in a social context, determined always by the overwhelming 
majority of mediocricy, it is a violence against concrete thinking.

It is
\thi{what} and only \thi{what} that interests me \ldots The \thi{how} 
only follows the suit \ldots 





\ad{Philosophy searches for the Absolute Reality --} 
whatever this might 
mean. Since epistemology, this changed to the search for the Absolute 
Knowledge of Reality

A universal temptation underlies most of philosophy, the temptation 
to go after the absolute, the \he{undubitable} truth, to 
construct things, matters, the world once for all on a secure rock of 
\he{incontestable} reasons and matters of fact. The keyword of this 
temptation is ``security''. And neither the poor results nor even the 
schizofrenic split of the intellectual personality between `is' and `ought' 
are able to eradicate it. 

\begin{enumerate}
\item Certainty -- justification:
 \begin{enumerate}
  \item fear of the unexpected (reason vs. feeling)
  \item flirt with science
  \item[\isimp] unchangable
  \item[\isimp] necessary
 \end{enumerate}
\item Necessity (apriori) \impl irrelevant \\
what must be (irrelevant) vs. what is/can be
\end{enumerate}

\pa
The praise of arguments is but a side-effect of that. But what is an 
argument good for? Have you ever changed your mind concerning some 
fundamental matter because you have been exposed to an irrefutable 
argument? Have you ever accepted a view because somebody managed to 
produce a \thi{proof} of it? I do not think you have, but it is not 
only because
\citt{[q]uestions of ultimate ends are not amenable to direct
proof.}{Mill, Utilit.  ch.I} An argument need not be \thi{direct 
proof}, it may be just an argument, although it always tries to force 
its conclusion, to convince. It is, indeed, a great field open for 
invebntivness and shrewdness of intellect. But \ldots if I do not find 
the cnclusion plausible, then all the shrewdness of the argument does 
not help a least thing.

Philosophy can do better than produce arguments -- it can try to 
describe experiences or, perhaps, experience. If you find Kierkegaard 
worth reading, I doubt it is because of the excellency of his 
arguments. It is because you find something worth paying attention to, 
it is because you sense an attempt to communicate an experience which, 
being an experience of another human being, may turn out to be most 
relevant for yourself, too.

\pa We do have a lot of philosophy which occupies itself with
inventing new arguments for old truths and, on the way, with
rearranging the language in order to give the truths, as well as the
arguments, the apparent look of novelty.  Although the reasons for
such a search remain hidden in the obscurity of academic vanity,
it may be, perhaps, worthwhile. It is not my objective.

\ad{We do not believe in Absolute Knowledge,} 
final justification. 
Reason, in the broad and traditional sense, implies, and since Nietzsche 
means, control; control over 
chaos and anarchy which threaten our finite being. The 
anarchic element is just the opposite of reason, is the unreasonable, the 
uncontrollable. Again, in a broad and traditional sense, it has often been identified
with emotions and feelings.
The conflict between the two is thus not only a platitude of a second rate 
literature but also an analytic triviality -- it follows by definition. 
This looks like a great starting point! Does anything sell better than 
trivial platitudes? And the sale numbers are just reflection of the 
commonest, not to say the meanest, interest.

Reason, having gradually turned into 
rationalism, did not give up its absolute claims expressed in its
controlling function. But becoming {\em ratio}, it turned partial; 
after 
all, ``ratio'' refers to reason as much as to a relative value. Order needs 
proportions which bring divisions.
 Again by definition, reason loses the possibility of 
full control. But this only increases the tension -- the attempts to 
regain control become the more desperate.  

\begin{enumerate}\setcounter{enumi}{2}
\item Dissolution of subject (one of the guarants of infallibility)
 \begin{enumerate}
  \item society, culture, epistemic authority, selfish gene, discourse ...
 \end{enumerate}
\item Dissolution of concepts and distinctions \impl processes 
 (evolutionary epistemology)
 \begin{enumerate}
  \item empiricism-rationalism-idealism-pragmatism (only rough distinctions)
  \item Gestalt, holism (hermeneutics) ...
  \item empirical turn (Dennett's I, dynamic systems ...)
  \item sociological turn (soc. of knowledge: Kuhn, Rorty ...)
  \item analytic-synthetic (Quine), ...
 \end{enumerate}
But this only emphasises continuity between the old conceptual extremes.
\end{enumerate}

\pa
We do not believe in absolute knowledge because we have lost grasp on 
objects. The empirical turn of much of philosophy is but an attempt 
to re-confirm our intimate involvement into the \thi{objective world}, 
the unbroken relation with the field of our experience, which seems to 
disappear dissolved in the arbitrariness of discourse or whatever one 
wants to install in its place. 

Dissolution of objects\ldots great! It will be an important point.

\ad{No Absolute Knowledge \impl no Absolute}\label{pa:attitudes}
Because
epistemology got us used to identify reality with our knowledge thereof, it
is often hard to see if renouncing Absolute Knowledge we do not also renounce 
Absolute Reality. ``Wer spricht \"{u}ber siegen? \"{U}berleben ist alles.''

Inquiring into possibly ultimate dimensions of human existence is not popular.
Knowledge, even if not absolute, seems still a relevant question. But its relevance
and reality haunt in the background.
\begin{enumerate}
\item agoraphobic: ``Wovon man nicht sprechen kann, dar\"{u}ber muss man schweigen'' 
   (Witt. I = Witt. II) -- liguistic-analytic turn (despair, for intuition knows
that language \isnt reality)
\item ecstatic: catch the ineffable, use ``new'' language 'cos this is reality 
 (Heidegger, Derrida...)
\end{enumerate}
These two avoid distinction language \isnt reality, still under the spell of
epistemology; remove thing-in-itself, the transcendent
\begin{enumerate}
\item[3] Stay on the edge: moralism (Levinas, Rorty...); critical rationalism
\end{enumerate}
 
\subpa
It should be observed that the three
attitudes from \refp{pa:attitudes} are by no means specific to 
postmodernism. They 
are present in any encounter with the transcendence, in any encouter of a 
finite being with something that is greater than it. The character of the 
transcendence, that is, what is perceieved as transcendent and in what way, 
leads to various specific manifestations of these three basic modi: unreserved 
acceptance, claustrophobic refusal or sober confrontation.

They have thus always been present in philosophy. I am now about to commit
horrible simplifications and indefensible inadequacies. So I won't defense them nor 
claim that one 
couldn't arrange the following examples in a very different way. 
Each philosophy contains all the three moments of this tension. But each philosophy has
also its specific flavour which, apart from all the technicalities,
distinguishes it 
from others. I hope 
that you may recognize the flavours which made me classify each triple in this
way.

\begin{tabular}{rlllll}
tragic:  &  sophists & Hume &      nominalism        & Kant  & Husserl  \\
comic:   & Plato  &    Spinoza &   extreme realism   & Hegel & Bradley, Heidegger \\
pragmatic:& Aristot. &  Descartes & Abelards' realism  &     &  James, Scheler
\end{tabular}


\subsubsection{What is philosophy?}

It does not appeal merely to the intellect (lest it gets sterile)!


\ad{Reach reality} 
In general -- the temptation to reach reality; to be relevant; 

-- Cope with transcendence; {The greatest fear}
of unexpected, new, unpredictable...


\pa{Thing-in-itself,} objective world, etc. \co{There is}, yes, but what? 
I do know that \co{there is}, from the very beginning, from the virtual 
\co{signification} to the most advanced concepts, I know that there is 
something beyond them. But this something merely is ... At each level it 
may be the level below, or above, it, which is not accessible, like 
experience is not accessible to reflection, sensation is not accessible to 
concepts, etc. -- every layer has its level of transcendence. And at the 
bottom \co{there is} \co{nothingness}. 

Replace things-in-themselves by One, \co{there is}; and then something 
``objective out there'' by something inexhaustible

[murmur of being]

[the experience of objectivity (in what sense?) is not built up, 
construed, it is not reducible to the experience I had yesterday after 
lunch or may have tomorrow -- it is the fundament of experience and, by the 
same token, the most fundamental experience]

[it shouldn't be confused with the questions ``is this objective?'', ``is 
this objectively true?'', ``what is there objectively?'']


\pa
Reality degenerated into givenness\ldots

and philosophy thus degenerated into attempts to provide a universal 
description of thjis static givenness. Universal, which means 
primarily, incontestable, applicable to and recognizable by all. The 
descriptive bias made it hard to recognize that the basic, most 
important aspect of human reality is actually concerned with the 
choice, fundamental, spiritual choice; that facticity and possibility 
of this choice is what, in the deepest sense, constitutes human reality. 

-- ethics tries to take care of this but only in letter. Living as a 
rather poor relative, on the outskirts of ontology and metaphysics, 
it is heavily influenced by the general frame of mind. It degenerates 
into detailed analysis of possible acts and actions, particular 
choices and, sometimes, attempts to formulate general, abstract forms of 
moral imperatives, dissociated from the results of the ontological and 
epistemological investiagtions.

\ad{One --} Not science; not unity of sciences; not based on 
sciences; not concepts and their logic... but addressed to a whole human being.

Not ``how'' and ``why'' are the primary questions -- they belong to sicence -- 
but ``what''. (Eckhart's living ``without why''.)


\ad{Origin}
As One it isn't a science and shouldn't envy sciences. Historically, it was their 
origin.

-- Psychologism is detestable -- because psychology begun to take over some 
problems philosophers were occupied with. But it isn't if we are 
interested in human life rather than in being philosophers and the 
associated definition and  status.
(Unfortunately, psychology addresses concrete human condition and, negating
psychologism, philosophy tends to neglect it.)

-- History, 
A variant of the absolute is `unchangeable' -- no absolute = only history.
\\
History is irrelevant -- just one of the temptations to become more 
concrete and closer to the real world. I still read Heraclitus and Bible...
\\
Like in Kuhn's theory of quiet tides of science rising through periods of stability
until reaching the revolutionary height and turning in a new direction -- history, and with 
it philosophy (?), presents us with such a picture. A turning tide, a confrontation
with transcendence in a new form, brings chaos and disorder. 

-- Sociology

\ad{Rational}
One used to think of rationality in terms of justification, 
which then is something like argumentation. However, we do not believe in the 
ultimate justification any more, do we? 
There is at most a difference of degree (and often not even that) between
philosophy focusing on the quality of the arguments on the one hand and
rhetorics and sophisms -- disciplines, perhaps venerable, but hardly kept in high
respect nowadays. And to the general public, this is what philosophy often means
-- sophisticated and convincing arguments for the most ridiculus and unconvincing
theses.

The arguments function the better, the more petty matters are at stake -- 
``Shall we eat indian or mexican?''... The more important matters are concerned, 
the less imperative the arguments become, because then they are merely attempts
at justifying what we already are convinced about. 
Eventually, arguments never convince.
In the matters of real importance, 
argumentation is just a sign of lacking respect - one tries to convince by 
explaining to
another what he apparently is unable to understand. If he only could, if he only 
saw all the valid reasons which we see, he would accept our conclusion.


\thi{Big words}, or better \thi{high words}, on the other hand, 
do not force their meaning upon us. Their vagueness invites to most personal
interpretations and misunderstandings. Yet, they are not for this reason arbitray
and incomprehensible. 
They only hint at something not fully expressible, which we
are free to model and interpret -- they leave us freedom, exactly
because they do not have a unique, precise meaning.
As such, they are the opposite of arguments which, in honesty or
arrogance, always attempt to force the other to accept them.


\pa
The important thing is that what is being said is said as clearly and understandably
as possible. For such a purpose, arguments may, occasionally, have a value, just as
examples do. They are not, however, applied to force any conclusion, but merely to 
illustrate the connections between various aspects of the discourse -- in particular,
those which seem more acceptable and those which seem less so. 
But to identify them with the whole importance of rationalism is to turn it into
sophism.

Now, forgeting justification and keeping in mind the inherent partiality of reason, 
I would formulate the thesis of rationalism as follows
\thesis{\label{th:panrat}
A rational acceptance of a statement is the one accepting the limits 
of its applicability/validity.} 
%(Critisism concerns these limits.)
% Science is relative to a context...of possible falsification/criticism 
% which is exactly what limits the scope of scientific theories.
I may not know what these limits are but, at least, I am open for the 
possibility of their existence. And as far as I am able to, I try
to specify them. 

\pa
This thesis is not limited to the theory of science but 
can be taken as a quite general fundament of a philosophical project. For 
the first, even the generality and absolutism need not be dogmatic -- I am trying
to communicate some experiences, perhaps the ultimate ones, but I admit that
my formulations may be unfortunate, may be unclear, may be improved.
For the second, it is self-applicable: many statements, not only the statements of
the absolutistic philosophy, can be accepted without limits. Saying ``I 
love you'' one would like to give it an absolute value -- no temporal 
or contextual factors should limit its validity. This is perhaps the most 
irrational meaning one can give to this most irrational statement and I do 
not think that anybody who has ever made it this way would like it to 
have been made with all rational reservations. However, 
even the irrational statements and attitudes can be treated in a rational 
way according to \refp{th:panrat}. Sure, they lose then their magical air 
of actual existential commitments but philosophy is, at best, a reflection 
of life and never the life itself. It may invite to making some 
commitments or choosing a particular way, but it is never such a commitment 
or choice itself.

\pa
From the existential point of view, thesis \refp{th:panrat} implies something 
like the following attitude: I keep my convictions and comittments as long as
I do not encounter the contexts (situations, arguments, attitudes) 
which ivalidate them, which make tham impractical, 
unviable or unacceptable (in whatever sense). In such situations I do not have
to renounce them entirely -- typically, I will merely introduce more reservations
concerning their applicability. More importantly, it does not imply
that all such convictions and comittments are kept rationally -- the deepest, 
most significant aspects of my being and acting are, probably, those which 
have not as yet got the chance to be explicitly limited. Their source does not
lie in any critical examination but beyond. Still, the attitude makes me, at least
in principle, open for the possibility that they too may need such a limitation 
at some time. But I may nourish implicit and explicit convictions of various
degrees of constancy, some being easily disposed of, other being of fundamental
importance so that only the most extreme circumstances may shatter them.
\com{
Perhaps, knowledge in a stricter sense, involves a rational acceptance with the
explicit statement of such limits.}

\pa
{\em Explicit concepts} -- unlike literature or poetry. (lists of 
distinctions; perhaps, just enough words for the vague concepts)
and thesis~\ref{th:panrat}

-- don't multiply concepts and distinctions. 
They never match experience, while one thinks 
that, more and more disitinctions will bring one closer. Nothing more 
illusory!

\pa
Of course, all cannot be embraced with a scientific precision ...

-- {\em Proper abstractions}! ... create = ignore $A$, assign weight to $B$; 
economy of signs.


\ad{Is vs. Ought}
Ontology vs. axiology; the world as it is vs. as it should be. There is no 
doubt that we can speak about our wishes as to what or how the world should 
be. That we can speak about how it actually, really is, looks much more 
like a wished for assumption. And taking into account the 
rather poor results of the attempts at establishing any form of consensus 
in this matter, it even looks like a possibly wrong assumption. 
It pushes ethics on the side -- for who can doubt the primacy of the real 
reality over mere \thi{ought} -- while, at the same time, leaves theoreticians 
in an eternal despair over the lost connection with life. 
The separation of theory and practice, engendered by the conviction that they 
address the same reality, makes theory irrelevant and practice unreliable. 
It separates man from the world he lives in, that is, from himself. 

In the middle of the pretensions to know how the world is, perhpas even how 
it must be, and -- what 
is usually even more annoying -- moralising advice on how it should be, 
the crucial question asking {\em how it can be} appears highly
\he{reproachable}. Avoidings both other questions, it promises a mere 
\la{Weltanschaung}. But avoiding these respectable questions, it 
also avoids the schizofrenia they caused. And furthermore, it avoids the almost 
inevitable arrogance of the attempts to answer them and claims to having 
done so. And who would daresay that, on the final account, and in spite of 
all the claims to the contrary, any philosophy -- and any of its 
opponents -- offers anything more than 
a \la{Weltanshaung}, a statement ``look, you {\em can} see the world this 
way''? One may try to hide behind claims to objectivity and absolute truth 
but then, if not anybody else then eventually the history will fetch one 
from this hiding place.

\pa\label{pa:tomee}
It all may be wrong. It may be, however, only wrong for you. It may also be 
wrong for many others while it may seem right to you. It is so to me. 


\pa
This is, roughly, the history of Western philosophy, perhaps, its short 
version for the lazy students. This is also the theme of this book -- for 
those who, in their perplexity, find simply living the life somehow ...
unsatisfactory. I doubt that anybody can learn from it anything which he 
otherwise does not know. It is only, a sort of, giving an account. And giving an account is 
like paying a bill -- it brings the satisfaction of fulfilling an obligation. 
But since the smartness and skills of a self-made one have reached the
 sky on the stock market,
getting away without paying the bills is a reason to pride. Paying them is 
stupid. \citt{For the people of this world are more shrewd
in dealing with their own kind than are the people of the 
light.}{Luke, 16:8}

\pa
Philosophy attempts to derive something obvious from something acceptable. It
isn't an art of discovery but of appreciation.

\noindent\dotfill

The problem of the existence of the external world comes from
\begin{enumerate}\MyLPar
\item\label{sub-ob}
thinking of our being in terms of the idealized, timless subject-object
relationship, and 
\item\label{dedworld}
the attempts to construct or deduce the world from within such a horizon 
of pure actuality.
\end{enumerate}
It is just one of the problems arising from this perspective, totality of my being, 
and phenomenon of time being other well-known examples. If, on the other hand, 
I think of myself as being capable of continuous experience and take time as
a fundamental, rather than possible, notion, the existence of the world beyond
my experience ceases to be a problem and becomes a fact of experience.

\pa
Let me emphasise again, but for the last time, that I am not looking for
necessary, irrefutanle truths. there is no {\em necessity} of such an
admission. One may let the all-powerful \co{objectivistic attitude} prevail,
reducing the world to a never reachable totality of independent objects. But
I do not aim at a \co{re-construction} of a pre-reflective truth, which is
what it is and cannot be otherwise -- which is necessary. Such a truth may be
of interest to a scientist but, as I said before, it cannot satisfy
reflection.  Perhaps it is true, perhaps everything we experience can be
reduced to some basic, pre-reflective facts, even to some necessary physical
or chemical reactions. But then a naive question arises -- so what?  How can
this ``wisdom'' affect my life and actions? Is it at all relevant that
something, which cannot be otherwise, is as it is? 

Suppose that somebody produced a definite proof that it is so. And? Shall I
change anything in my life because of such a proof?  Shall I stop feeling the
way I do and choosing as I do? The only way such a proof might become
relevant would be to produce some controlling mechanisms.  Perhaps, having
traced everything to such a basic level will enable us to control everything
which is above -- and as some maintain, independent from -- it. Then it might
be a different story, but only to the extent, that our feelings and choices
could span over a different, perhaps wider range.  So far, the very
meagerness of such results makes the whole project as irrelevant for
philosphical reflection, as it is relevant for the \co{objectivistic
attitude} of science and controll. Let those who believe in it, work on it.

Necessary truth is what it is and {\em cannot} be otherwise -- do what you
want.  Thus, although it determines me, it does not affect my reflective
attitude -- it must be so, and there is nothing I can do about
it. Philosophical flirts with necessity do not interest me not because one
cannot produce cunning arguments in its favor but because they make the whole
endeavor existentially irrelevant. Firmness of existence arises from the
fragility of its background, not from the necessity of its assumptions.  Even
if at the bottom everything is involved in a play of deterministic forces,
this, certainly, is not the reality of our \co{experience}. In particular, there
is an element of freedom in reflection, if not in anything else, then at
least in the choice of its object. Reflection may admit that something has
relevance but that it cannot or does not want to focus on it and chooses to
focus on something else. Appeal to necessity of considering this rather than
that is a simple act of disrespect. The point is not to demonstrate necessary
truths but truths which are of interest, not to \co{re-construct} but, yes,
to construct.  What matters for me is not an inescapable result of necessary
processes, but the question  how and, above all, towards {\em what} direct
my reflective attitude.

\pa
What I see as the aim of philosophical enterprise is not to build a theory about 
how things really are, even less to build a theory which simply isn't inconsistent. 
Rather, I see it as some \thi{economy of language}, \thi{economy of words} which
should be spoken in as purposeful manner as possible.
I do not think that many
(if any) proponents of philosophical theories which we would classify 
as implausible ment really ridiculous things -- but their language sometimes makes 
us feel as if they did. ``Reincarnation'', ``immorality'', ``deeper Self'', 
``God'', ``external world'' etc. may all happen to refer to some things 
one wants to convey in a way unmatched by any other words. They may make 
speaking simpler, thus appearing most economical and
far better than ``circulation of matter'', ``immortal fame'', ``super Ego'', 
``opium for people'', ``transcendentally ideal world''. 
We have a tendency to believe that the latter express some concepts and (therefore?)
are more precise:
they reduce the former to something apparently more understandable, more
adequate for a reasonable discourse. But I am not sure that a discourse is
``reasonable'' when it excludes or simply distorts matters unclear and
never given as well defined objects, only because they
cannot be stated in unambiguous terms, preferably with the precision 
approaching that of mathematics.

\pa
The goal of reflection is not to re-construct the original, pre-reflective
truth. The origin of reflection -- experience -- harbours the possibility of
reflection and the material for it. But is made of a different
matter which cannot be fully and completely re-created with the categories of
reflection. The concrete material of pre-reflective experience can be, at
best, mimicked, symbolised but not re-constructed by reflection.

The only thing reflection can (and should) do is to construct its own
world. Its own, because reflection, like any other activity, can operate only
with its own categories. But this world should be constructed in such a way
that its original truth can live through and within it. One should strive
after the concepts which not only do not falsify or oppose the pre-reflective
feelings and understanding, but which actually open for their presence, allow
them to enter the world of reflection and unfold therein.

\ad{Novelty}
New insights -- rather inspiring new branches, 
history, sociology, psychology...

-- But otherwise, human-wise, \la{nihil novi}, 

-- [used in chap. III, Intro] \citt{\la{Mundus senescit}}{Gregory of Tours, {\em History of the 
Franks}, \~594AD} ``The world ages''

So WHY?

-- ``One can't construct a system which captures (the whole) reality.'' But, in
fact, more is true -- one cannot capture reality into words! Kierkegaard's
criticism of a system does not (really) address systematic analysis but the
intellectual arrogance believing that it does manage that...

-- A new philosophical book is but an expression of understanding, or 
misunderstanding, the old books. There are no new thoughts, no 
thoughts which were not thought before. The goal is not to think new 
thoughts, but to understand the old ones. 

Heidegger recommended return to Greeks, to Heraclitus, if possible, 
even farther back. So did Rousseau\ldots There is no kernel, no 
historical site -- it is everywhere, and nowhere\ldots

\subsection{We have \ldots, and shall\ldots}

\ad{Actuality}
We have heard the trumpeths!  
We have noticed the importance of distinguishing the beings from 
their Being, we have noticed the metaphysical preoccupation with
presence or, as I shall call it, with \co{actuality}. 
The barbarians have entered the gates of Rome and \ldots have found 
their place there. 
The Derridean fear of being accomodated, 
appropriated, embraced by the philosophical discourse was well 
founded because it was based on equally one-sided, as it was deep, 
understanding of this discourse.

After the initial fascination, seduced by vague promises of something 
new and different, even general public gets quickly tired of 
\wo{writing otherwise}, 
\wo{speaking otherwise}, all these \wo{otherwise} which, eventually, 
turn out to be nothing else than the other, ignored side of the discourse 
against which they tried to protest or, often, just a new way of 
saying the old things.
%%% `thinking otherwise' is in Book II. level of actuality

\pa
The barbarians have won, they upset the Rome, its order and its life. 
But there is a great danger in a great victory. 
And now comes the time of defeat, of settling down, of accomodating to 
the culture which lasted some thousands years not by a 
mere accident, not by a mere misunderstanding,
but because it was founded on something real. The barbarians 
have won because they had a good point to make or, in any case, because 
Rome did not have anything to counter them. But the noice of 
victorious entrance is but a passing news, and the horses on which 
one makes one's tryumph will soon be weary and have to be replaced. 

One thinks that conquering Rome everything has to be renewed, started 
anew; one lies down new foundations, challanges to new ways of 
building, living \ldots yes, thinking. And what? A short time passes 
and it turns out that people think as they used to, that people live 
as they used to, that the same emotions and passions steer their 
lives. The only things which, possibly (but even that not necessarily) 
have changed are the cloths. 

I do not believe in \wo{thinking otherwise} and, to a large extent, I 
abhore most of \wo{otherwise writing}. The barbarians overlook, in 
their tryumph, that the conquered traditions had developed the ways 
for accomodating all aspects of experience and that it did it, 
exactly, in constant confrontation with life. They were not 
neglected, they might only have been invisible at the places which 
attracted the eyes of newcommers. 

\pa
The power of the \wo{metaphysics of actuality}\footnote{I am 
reserving the word \wo{presence} for something else, so I have to 
translate this phrase in the above way.} needed undeniably a correction. 
But this can only be a correction of attitude. As long as one 
attempts \wo{to think}, one will do it in familiar categories. And it 
is hardly to be expected that, because of the fall of Rome, one will 
stop thinking. The difference may concern the significance one attaches to 
one's thinking and its results. And, not least, to the results of 
other's thiking. If everything is merely a question of interpretation 
then, certainly, most of the texts which one would like to subsume 
under the \wo{metaphysics of actuality} can, actually, be interpreted 
in different direction. They certainly are involved into thinking in 
terms of actuality but the point is -- what thinking is not? It 
remains to be seen what the hoped for \wo{thinking otherwise} will 
have to offer. The answer is: \la{nihil novi}.

\pa
But point taken. Reflection has to watch its steps, that is, it has to 
carefully observe what it is reflecting over. It has to always keep 
the clear distance to its object, in order not to confuse it with its 
reality, in order not to confuse itself with the omnipotent power of 
control. This is, exactly, the difference of attitude. And, for that, 
possibly a nonexistent one because, although some earlier thinkers might 
certainly be accused of nourishing such a belief, many could not. 
Just like the texts enscribe to an extent the conclusions, so does the 
choice of focus determine, first, the choice of texts and, then, the way 
of reading them. The attitude may be only assumed in the other -- it 
can be controlled, if at all, only in onself. Especially, when one 
reads. Discovering an attitude of \wo{fascination with the actual} in 
all the texts, the possible question is about the attitude of the 
reader. 
In short, it is not a question of \wo{thinking, speaking, writing 
otherwise} but rather of {\em reading} otherwise or, perhaps, reading 
other texts. \wo{Reading otherwise} would simply mean to look for what 
one cannot find, to look for the traces of thoughts which one thinks 
are not there. Eventually, one will find them, but this requires a 
different attitude of reading.
\thi{Metaphysics of 
actuality} is not all of the traditional metaphysics, in any case, 
not all of the traditional philosophy.


\pa
Critique of \thi{metaphysics of actuality} (Derrida, Heidegger) wants 
to show something more, \co{non-actual}. Critique of objectivistic 
system of thought of Hegel by Kierkegaard opposes to it the experience 
of an individual, not just a common experience, but the fundamental 
experiences, elaborated through all the depth psychology. 

My point: the two are the same! The feeling one may get of 
existential relevance of Derrida, and certainly of Heidegger, is 
based exactly on that the personal/individual/etc. comes into the 
world of \co{actualities} through the \co{non-actual}. 


\pa
Follow Heidegger's, Derrida's critique of \thi{the metaphysics of 
actuality}\footnote{I will reserve the word \wo{presence} for 
something different.} but not his aggressiveness. There is nothing 
wrong with actuality, it is our way of conceiving the world. There is 
only something wrong with positing it as the only way of treating 
everything, all aspects of experience\ldots 

We are not scared of concepts as Heidegger, and later post-modernists, became.
We consider this fear unnecessary, in fact, impossible in length, sometimes
comical, sometimes hurtful...

\ad{\thi{The linguisting turn}\ldots} 
\ldots or shall we rather say, \wo{a linguistic twist}?  
One has always known that words and language
can not capture reality.  Not only there is no final, most adequate
and true name of God.  Even trying to communnicate most simple facts
and situations, when using words, we are forced to distribute the
responsibility between the one who is speaking -- to express himself
as clearly and understandably as possible -- and the one who is
hearing -- assuming that he not only knows the language, but also has
the sufficient background of experience to grasp the meaning of what
one is saying.  This sounds, perhaps, a bit exclusive, even elitistic,
and so it is supposed to sound.  \citt{Who hath ears to hear, let him
hear.}{Mat.  XIII:9} Success measured exclusively by publicity, by the
sheer number of receipents is always -- {\em always}!  -- inversely
proporitional to the value of the thing.  In the domain of spirit,
there governs a law opposite to that of mass culture, namely, the
more, the worse.  Even deep things, to the extend they are spread
around and received by many, become superficial, flat and low.

\wo{The sky was blue and the sun was shining.} is as
inadequate a phrase, if we try to measure it by some some objective
conformance to reality, as any name has ever been for expressing the
essence of God. When we speak, and even more when we write, we assume 
that the reader fulfills some basic conditions necessary for 
receiving the message. If, on the other hand, we do not know to whom 
we are speaking, if we, perhaps, assume that we are speaking to 
everybody, irrespectively of his knowledge, background, level of 
intelligence, and what-not, if we, for instance, are merely interested 
in selling the book to indefinitely large and, consequently, also 
indefinitely amorphous masses, we can not avoid getting confused. In 
particular, if we forget that any utterance, and also any text, is an 
address from somebody to somebody, if we abstract the text from this 
fundamental relation of intension and meaning, we can not avoid 
getting confused. 

\pa
One feels proud having recently overcome all these difficulties by
the \thi{discovery} that, as a matter of fact, there is no reality 
beyond the words, there are no human subjects beyond the text. As any 
thesis, so also this one can find innumerable reasons and 
justifications. I certainly won't spend time on reviewing them all. 
This inhuman abstraction of \thi{textuality}, this invention of 
intellectual despair over the impossibility to ensnare reality into 
words, does not deserve much attention. Yet, its wide 
popularity makes it hard to simply ignore it. The barbarians entered 
Rome and settled down, but their ways do have reprecussions\ldots

\pa
Reasoning, and understanding in general, is a way of using, inventing 
and arranging \co{signs}. Words are not the only possible \co{signs}, 
but let us be charitable and take all this \thi{linguisticity} in a 
very general, probably over-general, sense of semiotics. Without an 
adequate system of \co{signs}, there is no understanding. Ok, but 
there is no such thing as \thi{understanding}; there is only and 
always only understanding {\em of something}. What this something is, 
is highly vague, and the \thi{linguistic turn} is an attemt to 
substitute for our lacking, or in any case \co{vague}, understanding 
of what this is that we are trying to understand, something which 
apparently is much more precise, well-defined and hence manageable -- 
the very system of \co{signs} itself. Cetainly, a smart intellectual 
twist. However, the fact that something is \co{vague}, that we can 
not put the finger on it, does not mean that it does not exist. 

\pa Many \thi{linguistically twisted} will agree.  There exists
something but our way of speaking, the language and words we are used
to, are inadequate to capture it.  Therefore, we must \thi{speak
otherwise}, \thi{write otherwise}, invent a new language.  The ghost
of capturing the reality into words is peeping in through all the
holes.  What else, after all, can an intellectual do than to market
his words?  
%Among the most prominent (but by no means the only) examplars
%of such \thi{writing otherwise} one can probably mention late
%Heidegger and Derrida, perhaps, but only perhaps, late Wittgenstein. 
%These are, as usual, accompanied by a tremendous herd of noisy epigons
%who \thi{have seen the light} and start \thi{bubbling otherwise}.

%\pa 
If the reality which one senses beyond the inadequate words is
really a fundamental aspect of human existence, then is it reasonable
to assume that we had to wait thousands of years, until the day today,
to discover the need for \thi{speaking otherwise}?  Sure, one can
always patronize the past generations pointing out their mistakes. 
But such patronizing is but an aspect of a positivistic faith in the
absolute character of progress which we are not willing to take
without any reservations any more, if we at all are willing to accept
it at except, perhaps, for the sphere of technology and civilisation. 
Perhaps, the men of Reneissance, or of Enlightment invented
particularly obscure ways of speaking about some aspects of human
existence, ways which, due to their low value, became particularly
popular.  But there were wise men, too, in those times, weren't there? 
OK, they were all led astray.  What about the men of Middle Ages?  Of
early Middle Ages?  Heidegger would say, and what about the Greeks? 
Indeed, what about them?  But we do not have to follow the
idiosyncratic, not to say comical, attempts to revitalize a dead
language.  What about Indians, Jews, Chineese?  I do not want to boast
with false modesty, but I find it hard to belive that to address the
most fundamental issues we need an entirely new way of \thi{speaking
otherwise}.

\pa If all history of human culture, in any case, of written human
culture, has been perverted by the inadequate language, then what
makes us, some of us, today read Plato, Aristotles, Lao Tse, Bible,
St.  Augustine?  For my part, I can say for sure that it is not a need
to find their mistakes but, on the contrary, the conviction, and I
should be able to say -- the experience, that they do have something
important to tell me.  I doubt that many others read them only for the
sake of intellectual curiosity, deconstructivistic exercise or
academic career.

\pa
OK, so old texts may, sometimes, contain some valuable insights. Or, 
shall we only say, some valuable words? Do these texts contain 
anything \thi{in-themselves}, anything which is not read into them by 
the reader?

If text is entirely open to the arbitrariness of the reader and the
indeterminacy of possible interpretations, if it has no intension, no
truth it tries to communicate, if any system of signs is but an 
arbitrary invention with but an arbitrary relation to what possibly 
might be lying outside it, then one might almost agree that
writing should not involve any attempt to make it understandable and
accessible -- {\em what}, in such a case, should one try to make 
understandable?  Probably, one does even better resisting any such
attempts and we have been exposed to many examples of such a 
resistance. 

Unfortunately, or rather fortunately, all texts have been written by 
humans with some intensions. They may communicate these intensions in 
better or worse ways, in more or less adequate form, in more or less 
understandable and plain manner, but intensions are there. And if one 
does not like the word \wo{intensions}, in particular, any 
\wo{intensions of the author}, then it will still suffice that, having been 
written by humans, they contain human expressions of human experience 
and life. This, at least, is more than sufficient for me and if it 
isn't for you, then you are free to keep dissolving words, expressions 
and, eventually, insights in the mud of \thi{intertextuality}.



No private language -- so, no private thinking?

language -- changing, ok, but still not changing and translatable

New problems? New solutions? What? -- check it out, it has all been 
there\ldots

\pa A completely different, even opposite, dimension of the
\thi{linguisitic twist} is analytical philosophy.  A name which seems
more adequate to me would be something like \wo{logical analysis of
linguistic behaviour} because, as a matter of fact, I have found
extremely little and also extremely meagre philosophy in this camp. 
This, however, is due only to my biased and misunderstood conception
of philosophy, so let us not quarrel about the names.

The strength and, indeed, the visible vital force of this camp is
based on the fact that it tries to focus on more or less identifiable
and well-defined problems.  The resulting insights and proposals have
often much intrinsic value and I am last to deny that.  I am a bit
more unsure what this value is and to what purposes it could be used,
except for possible applications in cognitive science, artificial
intelligence, linguistics and, perhaps, legal theory and new design of
dictionaries.  Even if only of pragmatic -- shall we say, scientific? 
-- relevance, so, being usable, they also represent some kind of
value.  I want to say this with all due emphasis because, having said
that, I think we can leave this camp for itself.  Hopefully, with
time, it will, as any scientific community earlier in history,
establish its identity independently from philosophy and then continue
resolving all its specific problems and particular issues with the
undisturbed peace of mind to the best of whoever will want to use
them.
%
\sep 
%
\pa This is as much I have to say about all the linguistic twisting
and I leave writing of the voluminous binds of history and analyses of
the involved \thi{problems}, \thi{questions} and -- yes!  --
\thi{solutions} to the competent scholars.

\ad{Genealogy}
Not in time, but from virtual to concrete $\sim$ gradual 
differentiation

Then: refinement of systems of things/concepts; these consist of 
simultaneous $\sim$ equiprimordial \ldots correlates/aspects (find a good 
word)


\ad{Notation}
\begin{itemize}
\item ``\herenow'' \impl ``here and now'': dissociation of aspects
\item ``\co{recognition}'' \impl ``\co{re-cognition}'': rozpad into 
more detailed parts.
\end{itemize}
ntuitions should be preserved -- for more detailed understantding, one 
should first keep in mind the possible presence of another notational 
variant, and then check it \ldots




} % end the main \noo{%END

But if such a useless nothing, why \la{summa}? 
What could possibly make one write a \la{summa}, of any kind, nowadays?
Complexity -- not only of the incomprehensible totality of the world, but even
of every single issue -- makes it look pretentious. Things fall apart
and observing the dissolution or, as many a one pretends, praising it as the
openness unto plurality, is the only reasonable attitude, in fact, the only
politically correct way of distancing oneself from the trauma and danger of monism,
that is, totalitarianism.

Yet, philosophy which does not try to capture any unity ends up as a catalogue
of particular cases, particular concepts or just words, which may be as
elaborate, intelligent and intricate as it is uninteresting and useless.
Philosophy which distances itself from any attempt at reaching some wisdom, and
that means in particular, thinking and relating to the ultimate questions, 
ends up calling its impotency for modesty and glorifying the incessant
questioning, that is, public scratching one's head with the emanating
self-assurance that all genuine issues should be left to those who are
unintelligent enough to expect any answers. After all, it takes an analphabet to
believe that truth is written some place.
% But receiving guidance does not exclude autonomy and it appears to us quite
% unfortunate that
%

Philosophy attempts to think what one knows. For to know means much more than
what Plato explains in \btit{Timaeus}, or what others have tried (and are still
trying) to specify with
respect to explicit, reflective thoughts, \gre{episteme}. Sure, I know the
pythagorean theorem and the way from my house to my work. But I also know that
I will die, I know my love of my mother, I know how she loves me, I know what my
best friend likes and of what my girl-friend is afraid. I know that I am the
same person I was yesterday and I know that, from behind the invisible forces,
the hidden eye of God's  watches everything. Most of these things I am not
actually able to think, not to mention, to express precisely and explain with
any degree of stringency.
%
We know what philosophy is, don't we? We know what is {\em not} philosophy and
we know what is, at least, when we encounter it. We can even tell good
philosophy from bad one. And yet, can anybody tell what we thus know? Can
anybody think this knowledge clearly enough to express it precisely to
everybody's, or even only to one's own, satisfaction? 
%

Knowing, we might perhaps say, is what allows me to
consent to some thoughts and to reject others. For when I content to the truth of
a thought, I do it in the light of something I know. Only in very special cases
this knowledge consists of other, explicit thoughts.
%
To think what one knows is a challenge, just like it is 
a challenge to become what one is. Only to a self-satisfied rationalism such challenges
may appear as paradoxes, and only to a narrow-minded pedantry as contradictions.

The attempt to think what one knows need not (as it seems, can not any longer)
rely on necessary principles and forcing arguments. Those who try
\citet{withholding their consent from any proposition that has not been
  proved}{CiceroGods}{ I:1. [In the translation of F.~Brooks: \wo{refusing to
    make positive assertions upon uncertain data}.]} end up with absolute 
certainty -- about nothing. The rigidity of irrefutable argumentation which
attempts to force its conclusions and tries to dispense with everything which
escapes such attempts seems, indeed, to empty every phenomenon of its
concreteness yielding only residual site whose necessity equals its hollowness.
But the lack of the universally forcing arguments is not the same as the lack of
any truth, the lack of the necessary laws is not the same as the lack of any
order. Besides (or instead of) the effcient causes there may be many necessary
ones, besides (or instead of) the sufficient reasons there may be many
insufficient -- but, as Weber would say, favourable and supportive -- ones. Just
like explanations try to capture the former, so a more modest account may try to
identify only some of the latter.

\citeti{I say the following about the Whole ... Man is that which we all
  know.}{Democritus}{ DK 68B165}
Philosophical anthropology tries to give an account of human life. This does not
mean isolating this specific \thi{object} and ignoring all the rest. It amounts
only to seeing all the rest in a specific perspective: not as a set of
inviolable mechanisms but as the field of unfolding of human existence.  The
dualism of deteminism confronting freedon, of \thi{objectivity} confronting
\thi{subjectivity} is only a 
result of a particular way of thinking  the existential
confrontation which precedes the understanding -- and dissociation -- of these
two (and other) aspects.  The two, posited as the ultimate poles, can never not
only agree but even meet and one is constantly forced to the impossible choice
of the one or the other. Above and across this impossibility one may try to draw
the border between the two in ever new ways.  For even if the number of the
elements, say, of the constitutive dimensions of existence, is limited, the concrete
borders between them may be drawn in an unlimited variety of ways. Every human
being draws these borders anew and the abstractness of the fact that somebody
else did it earlier in a similar way does not in the slightest diminish the
concrete need to do it ever anew.\ftnt{Besides, \thi{limited} can be, in all practical
  respects, as good (or bad) as infinite. On how many different points can any
  two philosophical systems differ?  Ridiculous question, but to play the game
  only for a while, let us say some number which certainly is lower than it is
  in fact, say: 100. Simplifying further, let us say that on each of these
  points one has only a binary choice, yes or no, + or --. How many different
  \thi{systems} do we get? Well, $2^{100}$.\noo{This is, if we were to believe
    such calculations, approximately the square (perhaps cubic) root of the
    estimated number of atoms in the universe.} If humankind produced one such
  system every second, it would take some $2^{75}$ or, rounding off, $10^{20}$
  years to merely produce them all. This is a bit more than twice the estimated
  age of the universe.\noo{10^9 And merely producing them is not even the
  beginning of anything.} Thus, even if the 
  number (of possible forms of existence, of possible philosophical systems)
  were finite, the simple combinatorics is on the side of 
  unrepeatability -- not principal, however, not in some ideal infinite limit,
  where indeed the finite number of possibilities would have to be reapeated infinitely
  many times, but only in practice, that is, in fact. (If one wanted now to
  reduce the number 100 to, say, 20 such points,\noo{ at 
  which two systems might differ} it still leaves over 1.000.000
  possibilities.)} 
%
Everything has been said before -- {\la{mundus senescit}} (\wo{the world has
  aged}) noticed St.~Gregory of Tours in the VI-th century and some 1500 years
before him a preacher observed \citeti{[t]he thing that hath been, it is that
  which shall be; and that which is done is that which shall be done: and there
  is no new thing under the sun.}{Eccl.}{I:9} This may sound a bit depressing
but only to an intellectual capable of dealing exclusively with abstractions, or to
the petty mentality occupied with even pettier novelties, one \citeti{consoled
  by a mere trifle, as it is distressed by a mere trifle,}{after
  \citeauthor*{Pensees}}{II:{136}} for which everything beyond the yesterday's
scandals and today's news is a boring repetition. But \citeti{what is right can
  well be uttered even twice.}{Empedocles}{ DK
  31B25\noo{\citaft{FirstPhil}{: 58, p.165}}} Moreover, to understand something
  said before one often must say it 
oneself. Every existence says anew something said before, but saying this
concretely amounts to drawing anew the distinctions in the matter of life, the
distinctions between abstractions like \thi{love} and \thi{indifference},
\thi{indifference} and \thi{impotence}, \thi{impotence} and \thi{thirst},
\thi{thirst} and \thi{lack}, \thi{lack} and \thi{illusion}, \thi{illusion} and
\thi{lie}, \thi{lie} and \thi{truth}\ldots The infinite concreteness lies in
such distinctions and the lack of final definitions is only another side of this
concreteness. And the lack of rigid definitions does not mean the lack of
significant distinctions.

Giving an account of -- rather than explaining -- human existence, we are not
very concerned with many accepted distinctions. We certainly have to keep
various points of the discourse at approximately the same level of abstraction
but this need not prevent us from addressing also issues which the
administration of academic life places at different
departments. \citet{Philosophy is first of all a science about human being,
  about integral human being and of integral human
  being.}{Bier}{I:2\kilde{p.21}} 
% The advantage of philosophical anthropology is that no posited
% objectivities-in-themselves need to disturb us.
Asking a question, we will often rest satisfied with relating it only to
the place it occupies in the field of existence. Whether the question happens to
sound (or traditionally even {\em is}) theological, psychological, mythological or
astrological is of no importance, as long as it addresses an aspect of this
integrity. \wo{\la{Summa}} from the title refers to such a summary, to the 
attempt to 
gather quite different, sometimes even disparate, aspects into one unity and not
to the ultimate summit, 
to the vanity of collecting all relevant (and irrelevant) details in a systematic
and scientific totality.

%\tsep{hee}

All relevant (and irrelevant) details can be abstractly thought as a graph with
various edges (of relevance, dependence, association, etc.) connecting various points.
Traversal of a graph can be, in general, perfomed in two different ways: depth
first ({\sc df}) or breadth first ({\sc bf}). {\sc df} starts in the actual node
and follows one path -- an edge to a neighbour node, then to some neighbour of
the first neighbour, and so on. (Encountering a previously visited node, it
backtracks and tries another path.) {\sc bf}, on the other hand, vists first all the
neighbours of the actual node, before proceeding to all the neighbours of all
first neighbours, and so on level by level. 
A scholar is a {\sc df}. We do not have equally general, and certainly no
equally respectable name for the {\sc bf}, but it could be associated with a
dilettante -- knowing a little bit about everything which concerns him in one
way or another.

Of course, the world is not a graph. If we insisted on the analogy, we would have to
extend the idea at least by suggesting that the graph is unlimited, if
not actually infinite, and that in at least twofold way: every node has
infinitely many immediate neighbours and, from every node, there is an infinite
number of infinite paths (which never enter a cycle). A scholar diggs thus
further and furhter away from his home, trying to get to the end of an infinite
path, and hoping that it will bring him back home. A dilettante, on the
contrary, circles around, always in a safe distance
trying only to cover the infinite circle 
surrounding the house. Neither ever completes the road, both seem \citeti{to join
together diverse peaks of thought,\lin And not complete one road that has no
turn.}{Empedocles}{DK 31B24 [translation after \citeauthor*{Emped} 24.]} Scholar
ends up knowing everything about nothing, a 
dilettante knowing nothing about everything.

But this is unfair to the scholars! How can one compare them, put them on equal
lines with dilettantes?! Probably, one should not, but \wo{dilettante} is only a
name, the better of which seems hard to find. So let us ask the scholars...

Does the world -- eternally returning in the cycles oscillating between Love and
Strife -- exist twice (on the way from Love to Strife {\em and} on the way
back), or only once? And if once, then why, when and how?
\kilde{Szczerba,p.63-65}The diverging opinions may be the
consequence of the lacking sources which might have possibly contained the
answers of Empedocles himself. But, as a matter of fact, the question might have
never been asked and the answer never intended. Perhaps, all that was meant, was
to point to \wo{The world-wide warfare of the eternal Two}, the Love pushing to
unity and Strife to separation?  Perhaps, all the cosmogonies and cosmologies,
reflecting only the human understanding which, eventually, is always only
understanding of oneself, are but images never meant to be studied in and for
themselves. But sure, the questions can be asked, and so constructions can start
spinning...

What is Plotinus rejecting in V:5.1, claiming that the intelligibles, perceived by the
intellect, are not \thi{propositions}, \thi{axiomata} or \thi{sayables}? Does he
claim that the intellect's \thi{knowledge} is non-propositional or only
non-formalisable, non-expressible or only non-representational? Indeed, one may 
ask and keep answering, and many distinctions can arise from such diputes. But,
the question is, what shall we do with all these distinctions? Is it reasonable
to assume that, although the text does not say anything clearer, the intended
ideas were nevertheless so much more precise? Or, perhaps, they were not but
they have become so in the course of history?
Certainly, scholars shall sort out what Plotinus actually said
and meant and what he did not -- but preferably only as far as it goes. The fact
that questions can be asked with respect to a text, does not mean that the text
(and related texts, and texts related to texts related to...) contains the
answer, nor even that the answer can be at all 
meaningfully given.\kilde{EmilssonIntellect,p.29-30} We are rather clearly
\citet{told that the Intellectual-Principle and
the Intellectual Objects are linked in a standing unity}{Plotinus}{ V:5.1} and
even if \wo{we demand the description of this unity}, we should not dismiss the
possibility that what we have just been told is all we can actually get to know,
yeah, all that is worth knowing! Of course, 
to admit that, one would first have to learn cherishing and being satisfied with
perhaps clear and understandable but still vague expressions; expressions which
admit their limitation and put some trust in the reader who, hopefully, is the
same kind of being as the writer. All this, however, might amount to ceasing being a
scholar and turning~... a dilettante? 


There may be, and almost always there are, many meanings and one seldom can be
\co{precise} enough to narrow the expression, not to mention the thought, to
only one of them.  In this sense, interpretation is usually over-interpretation
and thus mis-interpretation. It is not so that we understand when we have made
the possible meaning most possibly, that is, impossibly \co{precise}. Dissecting
it into all too specific and detailed alternatives brings perhaps everything
under our control but only by making it impotent, by denying it life which our
thinking can only address, but never imitate... One may, perhaps, do it
sometimes legitimately in the name of scholarship, but positing it as the
universal aim of philosophy, and even thinking, sounds neither convincing nor
even plausible to many ears. Precising things beyond their limits results in
dissociating them and what is once dissociated can not be put together unless
one rediscovers the unity which preceded all the dissociated precisions. This
unity is not any combination, any coherence, any consistency of the elements
because it does not as yet know of any elements. \noo{This smells vagueness, if
  not directly misticism, so it may be a good point to begin.}

\tsep{???}


\tsep{}

One of such questions, perhaps the only question of all philosophy, is:
%
%\equ{What is true?\label{questA}}
\begin{center}What is true?\end{center}
%
Some try to answer it in the way it is asked, that is, in
dissociation from any person asking it. Such an answer amounts always to
specifying what one, everybody, you and I {\em should} accept as true.
The \thi{should} is usually surrounded by the arguments and proofs which should
convince everybody. 
Unfortunately, any respect expressed by imputations of rationality, accusations of
irrationality, expectations of a direct and infallible communication is, 
 at best, merely the respect for some \thi{rationality}. As far as 
a personal meeting with the reader, that is, as far as the person of the reader
is concerned, it shows disrespect equal only to that displayed by the strangely popular
reluctance to express one's meanings in an understandable form. The difference
of proceeding results only in that sometimes the latter, but never the former,
can be excused on the assumption of incapacity. 

The best, in fact, the only thing one can do is to answer this (as any other)
question for oneself or, what amounts to the same, to accept some known answer.
The answer may be communicated to others but only as {\em my} answer -- I may be
convinced that you should accept this answer, too, but I do not even try to
convince you about that. What you do with my answer is your sake, in fact, to
understand it you have most probably already known it, even if you did not think
it yourself in the same way. 

\citet{Gradually it has become clear to me what every great philosophy
so far has been: namely, the personal confession of its author and a
kind of involuntary and unconscious memoir [\ldots] In the philosopher
[\ldots] there is nothing impersonal.}{BeyondGE}{ I:6} Moreover, the
\citet{beliefs to which we most strongly adhere are those of which we should
  find it most diffcult to give an account.}{BergTime}{ II p.135}
% \citt{It is almost incredible that men who are themselves working 
% philosophers should pretend that any philosophy can be, or ever has 
% been, constructed without the help of personal preference, belief, or 
% divination.}{W. James, Essays in Pragmatism, I, p.24}
Thus, since this, like every other question, is asked -- and answered -- only by a
concrete person, we can take it to be the same as:
%
%\equ{What shall I accept as true?\label{questB}}
\begin{center}What shall I accept as true?\end{center}
%
Here one might object: one senses a difference, the element of subjectivity, or
perhaps relativity, in the latter which does not disturb the former. But the
difference is only apparent: every answer to any of these questions, can be used
also to answer the other. If something is true then, sure, I shall accept it as
such.  On the other hand, if I accept something as true it is because it is --
as far as I can see, imagine, feel, understand, speculate -- true.  Unlike the
former, this later implication has been judged problematic, mainly because one
imagines some voluntary act of acceptance which one's subjectivity could decide
to perform as it wished. The meaning of this implication will be our main theme
and its validity will rest on the fundamental difference between relativity and
subjectivity.

{Relativity of every observation and conclusion to the
  subject making this observation and drawing this conclusion seems to be the
  most obvious thing in the world. However, one can draw quite the contrary
  conclusions from this observation. On the one hand, one can conclude that this
  relativity for ever prevents us from gaining an insight into the true nature
  of the world, into the true nature of things as they are \thi{in themselves}.
  The underlying assumption is that things indeed are something specific \thi{in
    themselves}, and even that they, at least in principle, can be described
  \thi{in themselves}, as if independently from the view of the one making the
  description. The contradictory nature of such a project seems hard to accept.
  So, on the other hand, one can draw the conclusion that there are no things
  \thi{in themselves} and, consequently, no measure of truth -- in short, that
  relativity amounts to complete relativism and truth to mere (perhaps even
  merely volitional) subjectivity. But does the fact that the specificity of
  things \thi{in themselves} is only ideal and posited, show the ultimate void?
  Does relativity of particular truths exclude absolute truth? 
Both these interpretations start from the assumed duality. The
intuition that the standard of truth carries an element of transcendence is
opposed to the observation that absolute transcendence, dissociated and remote
from any immanent view, is inaccessible and hence, at least in the latter case,
irrelevant. We will instead take relativity as the basic phenomenon which
  precedes and underlies various dualities like subject vs. object,
  immanence vs. transcendence, acceptance as X vs. being an X, etc.}

\ins{ [Reality -- is between] "Now, the quantum postulate implies that any observation of atomic phenomena will involve an interaction with the agency of observation not to be neglected. An independent reality, in the ordinary physical sense, can neither be ascribed to the phenomena nor to the agencies of observation..."

Bohr, N. (1934). Atomic Theory and the Description of Nature. Cambridge, Cambridge University Press (p. 53).

 N. Bohr, The Philosophical Writings of Niels Bohr, p. 54, quoted in
 T. J. McFarlane, Quantum Mechanics and Reality, 1995 (quotes also: 

All things -- from Brahma the creator down to a single blade of grass --
are. . .simply appearances and not real. 
[Shankara, Crest Jewel of Discrimination, tr. Chris Isherwood, p. 97])

apparently well-menaing and also well-thinking people associate the two...
}

Taken contra-positively, the implication from my acceptance of X as true to X's
truth says that I should not accept any untruth.  Emphasizing this direction
amounts thus to a more modest project: not necessarily finding {\em the} truth
but rather, or at least, simply avoiding untruth. In the able hands of a
quibbler this modesty might reduce to a mere scepticism but we will hopefully
manage to say also something meaningful.  We should only keep in mind that we
are not following the fashion of {\em reducing} the first question to the second
(which results in the interminable quest for \thi{subjective} criteria of
\thi{objective truth}). We claim simply that the two questions are genuinely
equivalent, that everything \citet{that is known, is comprehended not according
  to its own force, but rather according to the nature of those who know
  it.}{Console}{ V:4.25\noo{154}} \noo{\citet{everything which is cognizable, is
    cognized not according to its own power, but according to the powers of the
    cognizing.}{Console}{ V:4.25 \noo{Filozofia Sredniowiecza,p.106}} }

Giving only an account -- rather than explaining -- we will ignore many
distinctions. We will not, for instance, treat the above question as {\em the}
question, focusing on which demands subordination and dissociation of
 all the others. Every genuine problem of human
existence invovles necessarily all the others. One can frequently meet attempts
to address a specific \thi{philosophical problem}, elaborate in the scope of a
single paper the problem of free will, the problem of meaning, the problem of
truth.  Interesting and perhaps even to some extent legitimate as such attempts
may be, they reveal the prejudice that such a division is at all possible, that
distinct problems indeed can be treated in a relative independence; eventually,
that only the closest scrutiny of the most minute distinctions is able to give an
adequate description of any single issue. But the problem is that we not only do
not quite know how one problem could or should be addressed and approached
-- we do not quite know {\em what} any particular problem is. Attempting to isolate any
particular problem for a separate treatement, we fight first with circumscribing it in
any reasonable way which, however, {\em never} reaches the goal of becoming
entirely satisfactory. In this process of 
partial circumscription we invest the problem with all the relations and
implications it carries to all the other problems. 
If one tries to address, say, the issue of freedom without at the same time
illuminating the meaning of subjectivity and openness to truth, the sense of
meaningfulness, the presence or absence of the absolute, in short, without
addressing the intergrity of whole existence, one ends up with the
distinctions one started with and keeps opposing arbitrariness to determinism,
subjectivity to objectivity, spontaneity of feelings to the rationality of some 
inviolable laws, etc.  Every (not only fundamental) issue is the sum of what it
excludes, is the border contracting the tension between this issue and others
into which it is interwoven. In a bit strange (but, in fact, quite understandable)
dialectics, the tradition which had marked the XX-th century with the missionary
zeal of dissociating all the issues and bringing them under objective, systematic and
separate analysis, ends up with the holistic and coherentist postulates, whether
with respect to language, meaning or truth.  \noo{(Wittgenstein,
  Quine, Davidson)} Even though one can not forget the idealistic origins of all such
potulates, one would still like to deny their 
idealistic connections. And one almost manages that, at least, 
as long as one keeps dissociating, as long as one sticks to dissecting one 
particular issue at a time.

We will not try to establish any totality which, in the presence of all too many
accepted distinctions, would indeed be a vanity. We will therefore ignore many possible
distinctions -- not because they would necessarily destroy any unity we might
wish to find but because they would (tend to) completely obscure it. 
We hope to avoid the accusations of relating the unrelated by
making at least plausible that the intimate affinity and kinship of vague yet
distinct aspects, their genuine unity preceds more rigid dissociations, and that
the latter mark only the end -- or perhaps the middle, but certainly not the
beginning -- of the road.
%
\noo{We will certainly try to avoid pitfalls of pantheism (of which most such holisms
are examples), but only because unity can be found above it and not, as some
also trying to avoid the same claim, because it does not obtain at all.
}
%

% \addcontentsline{toc}{section}{Sources, references and conventions}
\subnonr{Sources and references}
I quote rather extensively and from rather different traditions.  However, I
never go into exegesis of the texts or analysis of the thoughts of others.  An
attempt to do so would make finishing this work impossible.  On few occasions I
make more detailed statements in order to illustrate differences which also
should clarify my meanings.  The variety of sources and inspirations makes me
even limit the quotations to the most succinct statements which, I think, express
some essential idea.  Although the basic rules of conscientious exegesis may be
thus violated, and some quotations might have even been not only drawn out of
their context but even adjusted to fit the present one, the intention is never
to violate the meaning of the quoted text. (Besides, exegesis is not our
objective.)
% Asked by a true Irishman \wo{Are you a drinking man or are you a fighting man?}
% I could whole-heartedly confirm the former.  I like company, and

Variety of traditions suggests that we should focus on affinities and often even
only vague similarities rather than differences and oppositions.  Was
St.~Augustine entitled to claim the presence of Christian truths in the
neo-Platonic texts, as he did in the much disputed and controverted passage in
\btit{Confessions} VII:9? Was St.~Clement of Alexandria right in the
similar claims of the affinity of the Greek philosophy and literature with the
Christian revelation? Was Philo Judaeus right claiming not only the similarities
between but even the direct dependence of Greek thought on the Biblical
tradition? Scholars might prove that they were all wrong pointing out
significant differences making the two views different and even incompatible.
The Greek spirit was, after all, completely different from the Christian one.
Perhaps, but this depends on how one draws the borders around the intuitions
like \thi{Greek spirit} or \thi{Christian spirit}.\noo{(Let us also notice that
  such abstracts, useful as they sometimes may be in philosophy, are primarily
  only of historical and sociological character.)}  One can always find
differences separating two views -- the question is at what level, and then,
what value one will attach to them as opposed to the similarities. (After all,
the neo-Platonic culmination of Greek spirit, with its severe critiques of the
emerging Christianity, provided the foundation for the depth of Christian
mysticism.) Opposing, say, Greek spirit and Christian spirit, one should never
forget that in both cases one is speaking about spirit which, incarnated in
opposing socio-historical and political constellations, remains at the bottom
human spirit. It takes some wisdom to recognise concrete unity behind actual
differences and to stop distinguishing when everything worth saying has been
said -- the problem of perspicacious thoroughness, as La Rochefoucauld observed,
is not that it does not reach the end but that it goes beyond it.
%only blathering and babbling has no limit while it 
We will for the most focus on the similarities and it is up to you
to decide whether they are only due to the negligence in observing the important
distinctions or, perhaps, they are justified because the possible distinctions
are of negligible importance.

\noo{I quote others because I find myself to be their friend, even if they might not
always be my friends. If I did not have the quotation, I would write something
similar myself.  Although one might use this to construct the accusations of
eclectism, I would consider such accusations as a proof of a lack of even
minimal good will if not also of intelligence; and so I will rather get rid of
the bad company than of the good one.  The context in which the quotation is
used indicates, hopefully clearly enough, the interpretation I have in mind. I
hardly ever subscribe to the totality of the quoted author's or source's ideas;
it is only the thought behind the quoted piece which I want to bring forth.
(This, I guess, is one of the reasons why I bother to write.)}
%In all cases (with so few exceptions, that they are not worth
%mentioning here), a quotation indicates my more or less full agreement
%with the quoted author 
%I mean that there is enough unity of thought precluding the judgment
%of this work as eclectic, but if you do I could probably understand
%the reasons.


There are a few special sources which deserve a comment.  The authorship
of {\em My Sister and I} is the matter of dispute and scholars can not
tell for sure (perhaps, rather seriously doubt)
that it is indeed, as is also claimed, autobiography written
by Nietzsche himself. %I could hardly care less, since finding
The authorship of relevant thoughts should not be that important.
%whether they were written by Nietzsche or a skillful and competent forger.
However, in an academic context the issue may become a bit sensitive, especially
when the claimed author is Nietzsche.  (It might be so, in particular, if one
wanted to relate the contents of this autobiography to his other works which,
however, I am not doing.)

%For me, it is Nietzsche, for e
Even if it were not Nietzsche, it certainly could be, though 
%As somebody said, it is \wo{how you imagined Nietzsche would sound if 
%you got him drunk}. 
the author might also have been more Nietzschean than Nietzsche himself. Facing
the lack of any decisive proofs or disproofs of purely textual, linguistic or
medical nature, we are left with the text which looks like it might have been
written, if not carefully re-read and edited, by Nietzsche.  The voice for or
against his authorship depends then on one's view of his thought -- whether this
text \thi{fits} into the image one has of his whole thinking and, not least,
personality.  For me, there is a perfect match with the image I had formed
before I found this book. (Possible objections against the portrait arising from it,
should be confronted with less extreme, yet by no means incompatible, impressions
of the close friend in \citeauthor*{LouN}.) \citet{In the end,
  \btit{My Sister and I} reminds me of a true story.}{Sirens}{} Having made this
reservation, I will quote the text as if Nietzsche was its author.

Another referenced text, hopefully of much less dubious value, is a collection
of early Freiburg lectures by \citeauthor*{PhenomReligio} [\btit{Phenomenologie
  des religi\"{o}sen Lebens}, Gesamtausgabe, vol.~60]. Some of these have been
reconstructed almost exclusively from the notes of the students. Thus the reader
should be warned that the quoted formulations, although reflecting hopefully the
intentions, are hardly Heidegger's. (In any case, they are translated by me into
English, and that mostly from the Polish translation of the German text. Well...)

Likewise, \citeauthor*{Celsus}, is only reconstructed from the extensive
fragments quoted and criticized in \citeauthor*{aCelsus}. In this case, however,
the breadth and details of Origen's response give reasonable confidence into the
authenticity of the reconstruction. Much worse is the case of
\citeauthor*{Porphyry} where even the attribution of authorship may be disputed
as the work is reconstructed mainly from the \btit{Apocriticus} of Macarius
Magnes which need not reflect the philosophy of Porphyry. These works are quoted
as if they were written by the authors to whom they are attributed by the
general (though not universal) scholarly opinion. For investigating the
associated doubts and controversies the reader may start by consulting the
referenced editions.

Two distinct editions of \citeauthor*{Periphyseon} have been used. The critical
edition (started by late I.~P.~Sheldon-Williams and continued by
\'{E}.~A.~Jeauneau) of volumes I, II and IV is referenced as just done, with the
number+letter identifying the page number and the manuscript as in the edition.
Volumes III and V are from the abbreviated translation by M.~L.~Uhlfelder and
are referenced to in the same way, \citeauthor*{Periphy}, with only page numbers
in this single volume edition. In either case, the volume number identifies
uniquely the referenced edition. 


\sep
%
One encounters sometimes cases when, in an English text, quotations and longer
passages are given in French, German or some other language of the original --
sometimes even Latin or 
Greek. Although this may serve as an indication that the text is addressed to a
particular audience, it is no more pleasing than any other form of intellectual
snobbery.  It is perhaps a good tone to know German, French, Italian, Latin and
Greek, but few people do and I am not one of them. Since I have used extensively
sources in other languages, I have attempted to access -- and if I did not
succeed then to translate -- all the quotations into English. (A few exceptions
concern passages of German poetry which I did not dare to attempt translating.)
Sometimes, I ended thus translating back into English texts translated
originally from English into another language in which I read them. Such cases
are marked as \thi{my {\bf re}translation...}.  Hopefully, this will not cause
any serious confusion -- to fix it, I have to find some time with nothing better
to do.

\subnonr{Some conventions}

All the works are referred by the English title, even if I used the source in
another language; this is then indicated in the Bibliography at the end of the
text. (A few exceptions are made when the original source is referred after
another author, as is often the case with collected works or fragments.)

The references to all the works look uniformly as
\begin{center}
  Author, \btit{Title} XI:1.5\ldots
\end{center}  
where the part before `:', typically a Roman numeral, refers to the main part
into which the source is divided (e.g., book, part, chapter), and the numerals
after `:' to the nested subparts.  The references to the Bible have no `Source',
thus `Matt. X:5' refers to \btit{The Gospel of Matthew}, chapter X, verse 5. 
(I have used primarily King James Version and commented occasional usage of
other translations in the footnotes.) 
Likewise, the references to pre-Socractics are usually given without any source by
merely specifying the author and the Diels-Kranz number, e.g., `Heraclitus, DK
22B45', where the number identifying the philosopher (here 22) is taken from the
fifth edition of Diels, \btit{Fragmente der Vorsokratiker}.

Identifying quotations by page numbers might have been reasonable in times when
most books existed only in one edition.  I have tried to avoid such references
but in a few cases, where the structuring and numbering of the text happens to
be very poor, I had to use this form. This is also sometimes the case with the
quotations borrowed from others which I did not verify (the source is then given
in the square braces ``[after...]'' following the reference).  The pagination
follows then at the end of the reference as `Author, \btit{Title}
XI:1.5\ldots;p.21', where the numbers indicating part and subparts usually
involve only the main part (i.e., only `XI;p.21'), and may be totally absent, if
no such division of the work is given.  The edition is identified in the
Bibliography.  Occasionally, the subparts may have a letter, as e.g.,
`II:d7.q1.a2'. These are only auxiliary and their meaning depends on the source.
Typically, these are used with the medieval authors and the reference above
might be to the {\bf d}istinction 7, {\bf q}uestion 1, {\bf a}nswer 2, in the
second, II, volume/book.

In few cases I do not know the origin of the quotation, or else I only (believe
to) know its author. I chose to indicate such incomplete pieces of information,
rather than skipping them all together. I have likewise indicated the use of
unauthorized, or in any case unedited, versions of the texts found on the
interned for which no bibliographical data except for the title and the author
are given in the Bibliography. (For some, certainly very pragmatic reasons,
books printed in the USA do not carry explicitly the year of publication but only
the year of copyright. Consequently, the bibliographical information for such
books refers usually to this date.)

\sep
Words which are given some more specific, technical meaning are 
written with \co{slanted font}. \wo{Quotation marks} are used for 
words and quotations. \thi{Shudder-quotes} indicate, 
typically, either the referent of the word in the quotes, or else a 
concept or expression which is not given a technical meaning in the 
text but which is borrowed from somewhere else or even is only 
assumed to have some technical sense. Thus, for instance:
\begin{itemize}
\item 
\co{subject} -- is the subject in the technical sense introduced in
the text;
\item 
\thi{subject} -- is subject in some, possibly technical sense of
somebody else; it may often indicate a slight irony over only apparently
precise meaning one might believe the word \wo{subject} to have;
\item 
\wo{subject} -- refers to the word itself (quotations are also given
in the quotation marks);
\item 
subject -- this is just subject, with full ambiguity and with whatever meaning
the common usage might associate with it at the moment. 
\end{itemize}
I have tried to place more technical details in the footnotes which therefore
can be, for the most, skipped at first or casual reading. They are not, however,
addressed specifically to the scholars.\noo{We have not only no exegetic but
neither any scholarly ambitions.} Sometimes they elaborate the text but in
general will be useful only for those 
 who find some ideas interesting enough to follow them in other
authors.\noo{The footnotes contain, for the most, more details and references to the
possible starting points for such (re)search paths.}




%%%%%% END
\noo{%END
In fact, most if not all meta-discussions in modern philosophy,
arise as a consequence of elaborating the aspects discernible in (\ref{questB})
but not in (\ref{questA}): the idealism versus realism, subjectivism or
perspectivism versus
objectivism, correspondance versus verificationism or pragmatism, tradition
inspired by phenomenology versus analytical scientism, existential orientation
versus linguistic analyses,...

We do not dismiss all these discussions as completely irrelevant but our first
goal is not to be drown in the methodological and conceptual meta-perspectives. After
all, they too are, at least originally, motivated by the interest in an answer
to the first question.

An important aspect of the second question, not present in the first, concerns
the suggested need of justification. If I am to accept anything, it better be
sufficiently justified. This element has overshadowed philosophy if not since
its Platonic beginnings, so since Descartes. It has been used to distinguish
philosophy suggesting, in fact, that it is an existential difference which, in
case of a philosopher, makes him rely exclusively on \thi{reason} while, for
instance in case of a theologian on some \thi{faith}, while in case of an
average man on \thi{casual opinions of common-sense}.  This differentiation,
reflecting already preoccupation with the second question, has of course nothing
to do with the answer to the first question. But philosophers tend to make it
relevant by claiming that only their \thi{reason} can serve in answering the
first question. Various accussations of
\thi{irrationality} follow. Unfortunately, more often than not, justifications of 
claims to \thi{rationality} or even \thi{true rationality}, and in any case the
results of following them, amount to selecting only 
some truths which, as it happens, can be assessed by the \thi{reason}, or what
is actually meant by it.

Such differentiations are, as a matter of fact, political issues, issues of
delegation of competences, allocation of educational responsibilities, division of
faculties and departments or even, as it happened in
post-Cartesian Repubic of Unified Provinces, of national conflict between the
conservatists and liberal republicans.\ftnt{Cartesians supported liberation of
  philosophy from its subordination to theology, while ortodox Calvinists
  wished not only such a dependence but a state underlied religious
  goverment. As any issue with a theological element was almost bound in the
  Republic of that time to turn into a political one, so did this one.}
We do not dismiss all such discussions as completely irrelevant, but we think
that they truly belong to politics. Who is responsible for what? and Who is
entitled to what kind of questions? -- is it reasonable to believe that
answering such questions may help answering (\ref{questA})? 
The problem is that they did not
manage to produce any certain measure of truth which could be used in evaluating
the proposed answers to the first question. And, in fact, if they were to
produce such a measure, it could arise only as a consequence of answering, at some
point or another, the first question. Of course, I shall accept as true only
what is true. The attempts to elaborate on the second question do not bring us
any closer. The difference between the two can be seen by comparing them to the
respective questions below:
\equ{What is important?\label{questAA}}
and
\equ{How should I figure out what is important?\label{questBB}}
We can iterate the meta-appications past (\ref{questB}) -- \wo{On what basis
  shall I accept something as true?} -- just like past
(\ref{questBB}). The infinite regress is very much like the infinite
regress of formal reflection. And, in fact, just like the formal regress of \wo{I think
  that I think that...} adds nothing substantial to the first thought, and only
enmeshes it in a  formality of a childish mechanism, so the formal regress
of our meta-applications does not clarify anything with respect to the first
question, but only multiplies the possible issues, doubts and problems one may
explicitly rise and discuss.

One may easily object that, as a matter of fact, when the answer to
(\ref{questAA}) is unknown and hard to find, one could get some help from
answering first (\ref{questBB}). Formally, it may indeed seem so, but it is a
pure formality.  Notice that questions (\ref{questA}) and (\ref{questAA}) do not
concern any particulars which 
might happen to be special cases of more general laws, possibly useful to know
in treating the special cases. No, these questions have uncanny level of
generality which it seems futile to generalize further.  To such an objection we
can only answer: Sorry, disagree completely, for any answer to (\ref{questBB})
will, in fact if not in principle, 
require at least a partial answer to (\ref{questAA}). It may certainly help to ask
another, and related question, but approaching an issue from a different
perspective is not the same as approaching it from above. The only things that
may happen in the latter case are that one either falls down or flies away.


\subsubnonr{vs. arguments; Different tempers}

Yet, philosophy is only a particular expression of the fundamental
differences of human existence:... Simplifying to the extreme
opposition, the two tendencies are those of one vs. many, Plato vs. Aristotle,
Aquinas vs. Ockham, Husserl vs. James, unity vs. plurality. 

Different tempers...


\sep

What makes Anaximander the first philosopher is that he tried to argue for and
justify his claims, he used arguments and not mere statements. This is, at
least, the general view of ... philosophers. The \thi{love of wisdom} with
which one started, has long ago become the love of argument, to the extent that
wisdom without argument goes for simplicity if not stupidity.  One insists on
the arguments the more, the more one feels threatened, that is, the less one
feels certain of own self-identity.

Thus one has managed to separate itself from
literature with its usual lack of and occasional distaste for intellectualism.
It was a bit more difficult to cut off theology but here, too, the fact that
some aspects were assumed to be indisputable helped. Unfortunately, along with
that all the concern for the divine presence in the world was removed too.
Indeed, how would one argue about God, immortality and other invisibles? Their
invisibility amounts more or less exactly to the inadequacy of any arguments;
after one got tired of proofs of God's existence, the whole \thi{topic} became
highly inadeaquate, because incommpatible with the image of precise  logical
argumentation.

The strange thing that so happened was the emergence of the perfectly univocal
definition of valid arguments and mathematical precision in their study. Formal
logic, and its subfield of comutability and recursion theory, put the final
period after the millenia of developing Aristotle's syllogism, Leibniz's
calculus of reason and the ideas of mechnical reasoning. We now know that, in
itself, it yields absolutely no insights and, moreover, that no mechnical
procedure can ever be found for generating valid results...



\sep

The title is used by others....
I could have found another title but this does capture the
essential feature of existential situation which is the
only object of this book


\wo{Philosophical antropology} is but another way of saying
\wo{description of the existential situation}.


Ignore what is not experienced; what is really, objectively, in-itself is as
little relevant for this description as the answer to the question whether the
universe will continue expanding or starts at some point receding.

We study human experience and how its various aspects are constituted. This is
the opposite, in fact, antagonistic approach to the one which starts with some
ready constituted elements and tries to explain the construction, emergence and
functioning of a mechanism. One never knows if the given building blocsk are
really the eventual ones, nor even if they are appropriate for the task one
tries to perform. We leave such exercises to the empirical project. Whether it
succeedes or not won't change one fundamental thing: the human experience. We
know that it is Earth rotating around the Sun, and yet we can not stop talking
about -- and in fact, feeling and living as if it was -- the sun rising up at
the horison. This is how things look from our standing point and no amount of
empirical or logical proofs and arguments, no amount of bad or good science,
will ever be able to change that, unless, perhaps, one starts messing up with
the human beings themselves (and we may still hope that genetics won't go that
far).

We are not interested in mathematics or physics, in biology or sociology -- at
most, in how such forms of modelling the world emerge within the horison of our
experience; we are not interested in objective time or space -- at most, in how
such forms emerge and penetrate our experience; we are not interested in what is
objective and what is not -- only in how such distinctions may be relevant for
us; we are not interested in life -- only in the feeling of life, or better, in
living. 

This looks pretty bad, right? Pretty empiristic, subjectivisitc,
phenomenological, even idealistic, or perhaps just existentialistic! As a matter
of fact, it is not any of these, at least I hope it isn't.
But it is not any of their traditional opposites, either. It is only something
in the direction of 

\ad{Anthropology}
\citet{What is typical allows one to retain cold blood, only individually
  conceived matters cause nervous shock. In this consists the peacefulness of
  science.}{Faustus}{XLV\noo{p.622}}

The only object -- human being and his confrontation with the
world. We can easily imagine the world without human beings, but
for philosophical antropology such a world has no relevance.
As we will try to show, such a world is only apparently imaginable
-- in fact, it is totally unthinkable or else, to the degree it is,
it is totally indistinct.
(This has nothing to do with its objectivity or subjectivity, its
reality or ideality.)

Thus philosophy which tries to present the world as it is in-itself,
as it is sort of independently from human being, may be perhaps
interesting, sometimes even enlightening, but it will never provide 
a completely satisfactory account. It may reduce its sphere to some
particular area and problems and study that. But its concepts and
results are then founded outside its scope...


Yet, not explaining but describing = choosing what matters and
placing it in relation to other aspects which matter = giving account;
which leads to a unity of understanding ....

\wo{To give account}...
\begin{itemize}\MyLPar
\item
  First of all: try to determine \co{that} before even asking \thi{what}.
  And then ask thi{what} before even thinking \thi{how}. If sometimes it happens
  that one answers \thi{how}, then in any case never ask \thi{why}...
\item
  Only (some) necessary conditions, hardly ever any sufficient ones -- they
  do not obtain. We are not after any explanations; perhaps we are after
  something which possibly might be attempted explained later on, but that is
  not so either; we are after something for which explanation is inadequate
  category...
\item
  give only necessary conditions; the sufficient ones do not obtain any
  way. Does it mean that I believe in miracles? Well, as you like it. I have
  never seen a sufficient reason for almost anything, and I would say that those
  who insist on them must actually {\em believe} that they obtain. And as they
  do not obtain (ok, except in mathematics, or some simple physical analogies),
  saying that we can see necessary conditions but not sufficient ones 
  seems to me a matter of simply conforming to what is experienced, not believing
  in anything. There is much less believing here than in the belief in
  sufficient reasons. But sure, if you want to call the transition from the
  given necessary 
  conditions for some X to the X itself for \wo{a miracle} then I
  do not {\em believe} in miracles -- I see them all the time.
\end{itemize}
\wo{Irrationalism!!!} may exclaim some others, but let them remain calm, too.
Nobody knows -- and those who claim to know can not agree -- what rationalism
is. It seems that, at least, it requires (do we notice a necessary condition?)
not being carried away, some sobriety. Or in less prosaic terms, rationalism
requires that one assumes a position only with the admittance of the possible
limits of its validity, if not with the actual knowledge of such limits.  If one
accepts this rather generous critical rationalism then we should be able to
complain, all the way. Even to the point where one asks about the limits of
validity of this very position...  (After all, this admittance of partiality
squares so nicely with the etymology of \la{ratio}! A bit worse with the
pretentions to universality...)

\sep

It is not for all...

\sep

Many more specific issues are only touched upon and one may wonder why at
all. But we are not trying to resolve all the issues, to come up with a definite
and final answer. In most cases such answers simply do not exist and we prefer
to say too little rather than too much. Multiplying distinctions and
perspectives may be as rewarding academically as it is existentially futile. 


\subsection*{The main points ...}
{\small{ \begin{enumerate}
\item \co{There is} $\sim$ the One -- the particulars are only 
``perspectives'', modifications of the \co{Is} (incommensurable for 
frog and man) \\
 It is the fundament of all experience and \ldots the fundamental 
 experience. (``Objectivity'' cannot be reduced to any experience because 
 it comes before. ``Externality'' can.)
\item \co{Distinction} is the begining; \co{point} --  \co{pure 
distinction} -- is its \co{reflection}
\item There is nothing beyond experience.
\item The levels of \thi{being} and transcendence coincide with the time-spans
 \begin{itemize}
 \item \co{transcendence} is essentially something present which is
 not exhausted in this presence, something \wo{more than it is}, the overpowering
 \item
 this \co{more} can be relative to various aspects but, primarily, it is related to the
 \co{horizon of actuality}
 \end{itemize}
\item The ``fourth level'' is not a level ``above'' the other three but
 concerns the ``essential structure'' of division into \LL\ and \HH. (It is not
merely ``formal''.)
\item The world ``consists of'' \LL\ and \HH\ --
 Man is a borderline between \LL\ (lower) and \HH\ (higher).
 \begin{itemize}\MyLPar
 \item the \wo{totality} of his being is not \wo{constructed out of points} but 
 precedes the \co{distinctions}.
 \item Before you construct out of pieces, the pieces must be there. \\
  \begin{tabular}{l|l|l|l|l|l}
    \HH & (phon)emic & whole (Gestalt) & mind & concrete & quality \\ 
    \hline
    \LL & (phon)etic & part & body & precise & quantity 
  \end{tabular}
 \end{itemize}
\item The basic existential mode is determined by \G\ (openness, acceptance) or \B\ 
 (denial, refusal) of the mastery of \HH.
\item Important are \co{nexuses} [tie, connection?! -- vs. \la{religare}] 
of \co{aspects} which can be reflectively dissociated but which 
function meaningfully only in the \co{nexus}
\end{enumerate}

\subsubsection{{...and defs}}
\begin{enumerate}
\item \co{Experience} -- whatever leaves a mark vs. the source of unpredictability.
\item \co{Reality} -- what you cannot live without.
\item ``natural attitude'' to external world -- possibility of new, unexpected, 
      not-from-me.
\item \co{Sharing} -- the same \HH.
\item \co{relevant} -- telling what to do in the face of transcendence.
\end{enumerate}
}} % end \small

\section{Preface}


Well, perhaps, I was not quite honest.  I want to give an account of a
possible understanding of the unity of \ldots Yeah, of what?  Of a
human being, of a person but, perhaps eventually, only of myself. 

\citt{Gradually it has become clear to me what every great philosophy
so far has been: namely, the personal confession of its author and a
kind of involuntary and unconscious memoir [\ldots] In the philosopher
[\ldots] there is nothing impersonal;}{Nietzsche, {\em Beyond Good and
Evil}, I:6} 
\citt{It is almost incredible that men who are themselves working 
philosophers should pretend that any philosophy can be, or ever has 
been, constructed without the help of personal preference, belief, or 
divination.}{W. James, Essays in Pragmatism, I, p.24}
The very same words may be deeply meaningful to some and ridiculously
superficial, even stupid to others. 
This book is not meant to {\em convince} anybody about
anything; I hope it will be found interesting by a few, and it is
addressed to these few.
% You find pretty quickly, by just starting reading, if you are among them.

What is my method?  I do not have any. 
\wo{Philosophy does not so much tell {\em what} to think as {\em how} to 
think.} All kinds of methodological postulates have polluted philosophy 
since it started to consider itself, or rather started 
{\em to try} to 
consider itself a science in the modern sense of the word\ldots
Positivism, pragmatism, phenomenology, all \thi{methodologies} of 
philosophical inquiry tried first of all to ape 
science, to achieve the same level of precision\ldots

But \thi{how} is only, and only at its best, a pale reflection of
\thi{what}, a method is at best only a desiccated sediment of a
particular way of understanding, which is dictated, if not determined,
by the particular objects or sphere of experienced addressed. 
Scholastics knew it very well, but we do not like scholastics very
much nowadays, do we?  Sure, \thi{how}, having extracted some resdiual
sediments of \thi{what}, turns easily into a socio-political
phenomenon, a \thi{school} or a \thi{party}, which pretends to know
{\em the} axioms, and in any case at least {\em the} rules of the
game.  But the game goes on and the only rules are to be
understandable, and then to have something worth understanding. 

\thi{How} a person thinks is merely an expression of \thi{what} he
thinks -- uncovering, perhaps, some hidden or unconscious assumptions,
but still only assumptions about the \thi{what}. To dissociate the two is a
violence against concrete thinking. Useful and justifiable as it may 
be in a social context, determined always by the overwhelming 
majority of mediocricy, it is a violence against concrete thinking.

It is
\thi{what} and only \thi{what} that interests me \ldots The \thi{how} 
only follows the suit \ldots 





\ad{Philosophy searches for the Absolute Reality --} 
whatever this might 
mean. Since epistemology, this changed to the search for the Absolute 
Knowledge of Reality

A universal temptation underlies most of philosophy, the temptation 
to go after the absolute, the \he{undubitable} truth, to 
construct things, matters, the world once for all on a secure rock of 
\he{incontestable} reasons and matters of fact. The keyword of this 
temptation is ``security''. And neither the poor results nor even the 
schizofrenic split of the intellectual personality between `is' and `ought' 
are able to eradicate it. 

\begin{enumerate}
\item Certainty -- justification:
 \begin{enumerate}
  \item fear of the unexpected (reason vs. feeling)
  \item flirt with science
  \item[\isimp] unchangable
  \item[\isimp] necessary
 \end{enumerate}
\item Necessity (apriori) \impl irrelevant \\
what must be (irrelevant) vs. what is/can be
\end{enumerate}

\pa
The praise of arguments is but a side-effect of that. But what is an 
argument good for? Have you ever changed your mind concerning some 
fundamental matter because you have been exposed to an irrefutable 
argument? Have you ever accepted a view because somebody managed to 
produce a \thi{proof} of it? I do not think you have, but it is not 
only because
\citt{[q]uestions of ultimate ends are not amenable to direct
proof.}{Mill, Utilit.  ch.I} An argument need not be \thi{direct 
proof}, it may be just an argument, although it always tries to force 
its conclusion, to convince. It is, indeed, a great field open for 
invebntivness and shrewdness of intellect. But \ldots if I do not find 
the cnclusion plausible, then all the shrewdness of the argument does 
not help a least thing.

Philosophy can do better than produce arguments -- it can try to 
describe experiences or, perhaps, experience. If you find Kierkegaard 
worth reading, I doubt it is because of the excellency of his 
arguments. It is because you find something worth paying attention to, 
it is because you sense an attempt to communicate an experience which, 
being an experience of another human being, may turn out to be most 
relevant for yourself, too.

\pa We do have a lot of philosophy which occupies itself with
inventing new arguments for old truths and, on the way, with
rearranging the language in order to give the truths, as well as the
arguments, the apparent look of novelty.  Although the reasons for
such a search remain hidden in the obscurity of academic vanity,
it may be, perhaps, worthwhile. It is not my objective.

\ad{We do not believe in Absolute Knowledge,} 
final justification. 
Reason, in the broad and traditional sense, implies, and since Nietzsche 
means, control; control over 
chaos and anarchy which threaten our finite being. The 
anarchic element is just the opposite of reason, is the unreasonable, the 
uncontrollable. Again, in a broad and traditional sense, it has often been identified
with emotions and feelings.
The conflict between the two is thus not only a platitude of a second rate 
literature but also an analytic triviality -- it follows by definition. 
This looks like a great starting point! Does anything sell better than 
trivial platitudes? And the sale numbers are just reflection of the 
commonest, not to say the meanest, interest.

Reason, having gradually turned into 
rationalism, did not give up its absolute claims expressed in its
controlling function. But becoming {\em ratio}, it turned partial; 
after 
all, ``ratio'' refers to reason as much as to a relative value. Order needs 
proportions which bring divisions.
 Again by definition, reason loses the possibility of 
full control. But this only increases the tension -- the attempts to 
regain control become the more desperate.  

\begin{enumerate}\setcounter{enumi}{2}
\item Dissolution of subject (one of the guarants of infallibility)
 \begin{enumerate}
  \item society, culture, epistemic authority, selfish gene, discourse ...
 \end{enumerate}
\item Dissolution of concepts and distinctions \impl processes 
 (evolutionary epistemology)
 \begin{enumerate}
  \item empiricism-rationalism-idealism-pragmatism (only rough distinctions)
  \item Gestalt, holism (hermeneutics) ...
  \item empirical turn (Dennett's I, dynamic systems ...)
  \item sociological turn (soc. of knowledge: Kuhn, Rorty ...)
  \item analytic-synthetic (Quine), ...
 \end{enumerate}
But this only emphasises continuity between the old conceptual extremes.
\end{enumerate}

\pa
We do not believe in absolute knowledge because we have lost grasp on 
objects. The empirical turn of much of philosophy is but an attempt 
to re-confirm our intimate involvement into the \thi{objective world}, 
the unbroken relation with the field of our experience, which seems to 
disappear dissolved in the arbitrariness of discourse or whatever one 
wants to install in its place. 

Dissolution of objects\ldots great! It will be an important point.

\ad{No Absolute Knowledge \impl no Absolute}\label{pa:attitudes}
Because
epistemology got us used to identify reality with our knowledge thereof, it
is often hard to see if renouncing Absolute Knowledge we do not also renounce 
Absolute Reality. ``Wer spricht \"{u}ber siegen? \"{U}berleben ist alles.''

Inquiring into possibly ultimate dimensions of human existence is not popular.
Knowledge, even if not absolute, seems still a relevant question. But its relevance
and reality haunt in the background.
\begin{enumerate}
\item agoraphobic: ``Wovon man nicht sprechen kann, dar\"{u}ber muss man schweigen'' 
   (Witt. I = Witt. II) -- liguistic-analytic turn (despair, for intuition knows
that language \isnt reality)
\item ecstatic: catch the ineffable, use ``new'' language 'cos this is reality 
 (Heidegger, Derrida...)
\end{enumerate}
These two avoid distinction language \isnt reality, still under the spell of
epistemology; remove thing-in-itself, the transcendent
\begin{enumerate}
\item[3] Stay on the edge: moralism (Levinas, Rorty...); critical rationalism
\end{enumerate}
 
\subpa
It should be observed that the three
attitudes from \refp{pa:attitudes} are by no means specific to 
postmodernism. They 
are present in any encounter with the transcendence, in any encouter of a 
finite being with something that is greater than it. The character of the 
transcendence, that is, what is perceieved as transcendent and in what way, 
leads to various specific manifestations of these three basic modi: unreserved 
acceptance, claustrophobic refusal or sober confrontation.

They have thus always been present in philosophy. I am now about to commit
horrible simplifications and indefensible inadequacies. So I won't defense them nor 
claim that one 
couldn't arrange the following examples in a very different way. 
Each philosophy contains all the three moments of this tension. But each philosophy has
also its specific flavour which, apart from all the technicalities,
distinguishes it 
from others. I hope 
that you may recognize the flavours which made me classify each triple in this
way.

\begin{tabular}{rlllll}
tragic:  &  sophists & Hume &      nominalism        & Kant  & Husserl  \\
comic:   & Plato  &    Spinoza &   extreme realism   & Hegel & Bradley, Heidegger \\
pragmatic:& Aristot. &  Descartes & Abelards' realism  &     &  James, Scheler
\end{tabular}


\subsubsection{What is philosophy?}

It does not appeal merely to the intellect (lest it gets sterile)!


\ad{Reach reality} 
In general -- the temptation to reach reality; to be relevant; 

-- Cope with transcendence; {The greatest fear}
of unexpected, new, unpredictable...


\pa{Thing-in-itself,} objective world, etc. \co{There is}, yes, but what? 
I do know that \co{there is}, from the very beginning, from the virtual 
\co{signification} to the most advanced concepts, I know that there is 
something beyond them. But this something merely is ... At each level it 
may be the level below, or above, it, which is not accessible, like 
experience is not accessible to reflection, sensation is not accessible to 
concepts, etc. -- every layer has its level of transcendence. And at the 
bottom \co{there is} \co{nothingness}. 

Replace things-in-themselves by One, \co{there is}; and then something 
``objective out there'' by something inexhaustible

[murmur of being]

[the experience of objectivity (in what sense?) is not built up, 
construed, it is not reducible to the experience I had yesterday after 
lunch or may have tomorrow -- it is the fundament of experience and, by the 
same token, the most fundamental experience]

[it shouldn't be confused with the questions ``is this objective?'', ``is 
this objectively true?'', ``what is there objectively?'']


\pa
Reality degenerated into givenness\ldots

and philosophy thus degenerated into attempts to provide a universal 
description of thjis static givenness. Universal, which means 
primarily, incontestable, applicable to and recognizable by all. The 
descriptive bias made it hard to recognize that the basic, most 
important aspect of human reality is actually concerned with the 
choice, fundamental, spiritual choice; that facticity and possibility 
of this choice is what, in the deepest sense, constitutes human reality. 

-- ethics tries to take care of this but only in letter. Living as a 
rather poor relative, on the outskirts of ontology and metaphysics, 
it is heavily influenced by the general frame of mind. It degenerates 
into detailed analysis of possible acts and actions, particular 
choices and, sometimes, attempts to formulate general, abstract forms of 
moral imperatives, dissociated from the results of the ontological and 
epistemological investiagtions.

\ad{One --} Not science; not unity of sciences; not based on 
sciences; not concepts and their logic... but addressed to a whole human being.

Not ``how'' and ``why'' are the primary questions -- they belong to sicence -- 
but ``what''. (Eckhart's living ``without why''.)


\ad{Origin}
As One it isn't a science and shouldn't envy sciences. Historically, it was their 
origin.

-- Psychologism is detestable -- because psychology begun to take over some 
problems philosophers were occupied with. But it isn't if we are 
interested in human life rather than in being philosophers and the 
associated definition and  status.
(Unfortunately, psychology addresses concrete human condition and, negating
psychologism, philosophy tends to neglect it.)

-- History, 
A variant of the absolute is `unchangeable' -- no absolute = only history.
\\
History is irrelevant -- just one of the temptations to become more 
concrete and closer to the real world. I still read Heraclitus and Bible...
\\
Like in Kuhn's theory of quiet tides of science rising through periods of stability
until reaching the revolutionary height and turning in a new direction -- history, and with 
it philosophy (?), presents us with such a picture. A turning tide, a confrontation
with transcendence in a new form, brings chaos and disorder. 

-- Sociology

\ad{Rational}
One used to think of rationality in terms of justification, 
which then is something like argumentation. However, we do not believe in the 
ultimate justification any more, do we? 
There is at most a difference of degree (and often not even that) between
philosophy focusing on the quality of the arguments on the one hand and
rhetorics and sophisms -- disciplines, perhaps venerable, but hardly kept in high
respect nowadays. And to the general public, this is what philosophy often means
-- sophisticated and convincing arguments for the most ridiculus and unconvincing
theses.

The arguments function the better, the more petty matters are at stake -- 
``Shall we eat indian or mexican?''... The more important matters are concerned, 
the less imperative the arguments become, because then they are merely attempts
at justifying what we already are convinced about. 
Eventually, arguments never convince.
In the matters of real importance, 
argumentation is just a sign of lacking respect - one tries to convince by 
explaining to
another what he apparently is unable to understand. If he only could, if he only 
saw all the valid reasons which we see, he would accept our conclusion.


\thi{Big words}, or better \thi{high words}, on the other hand, 
do not force their meaning upon us. Their vagueness invites to most personal
interpretations and misunderstandings. Yet, they are not for this reason arbitray
and incomprehensible. 
They only hint at something not fully expressible, which we
are free to model and interpret -- they leave us freedom, exactly
because they do not have a unique, precise meaning.
As such, they are the opposite of arguments which, in honesty or
arrogance, always attempt to force the other to accept them.


\pa
The important thing is that what is being said is said as clearly and understandably
as possible. For such a purpose, arguments may, occasionally, have a value, just as
examples do. They are not, however, applied to force any conclusion, but merely to 
illustrate the connections between various aspects of the discourse -- in particular,
those which seem more acceptable and those which seem less so. 
But to identify them with the whole importance of rationalism is to turn it into
sophism.

Now, forgeting justification and keeping in mind the inherent partiality of reason, 
I would formulate the thesis of rationalism as follows
\thesis{\label{th:panrat}
A rational acceptance of a statement is the one accepting the limits 
of its applicability/validity.} 
%(Critisism concerns these limits.)
% Science is relative to a context...of possible falsification/criticism 
% which is exactly what limits the scope of scientific theories.
I may not know what these limits are but, at least, I am open for the 
possibility of their existence. And as far as I am able to, I try
to specify them. 

\pa
This thesis is not limited to the theory of science but 
can be taken as a quite general fundament of a philosophical project. For 
the first, even the generality and absolutism need not be dogmatic -- I am trying
to communicate some experiences, perhaps the ultimate ones, but I admit that
my formulations may be unfortunate, may be unclear, may be improved.
For the second, it is self-applicable: many statements, not only the statements of
the absolutistic philosophy, can be accepted without limits. Saying ``I 
love you'' one would like to give it an absolute value -- no temporal 
or contextual factors should limit its validity. This is perhaps the most 
irrational meaning one can give to this most irrational statement and I do 
not think that anybody who has ever made it this way would like it to 
have been made with all rational reservations. However, 
even the irrational statements and attitudes can be treated in a rational 
way according to \refp{th:panrat}. Sure, they lose then their magical air 
of actual existential commitments but philosophy is, at best, a reflection 
of life and never the life itself. It may invite to making some 
commitments or choosing a particular way, but it is never such a commitment 
or choice itself.

\pa
From the existential point of view, thesis \refp{th:panrat} implies something 
like the following attitude: I keep my convictions and comittments as long as
I do not encounter the contexts (situations, arguments, attitudes) 
which ivalidate them, which make tham impractical, 
unviable or unacceptable (in whatever sense). In such situations I do not have
to renounce them entirely -- typically, I will merely introduce more reservations
concerning their applicability. More importantly, it does not imply
that all such convictions and comittments are kept rationally -- the deepest, 
most significant aspects of my being and acting are, probably, those which 
have not as yet got the chance to be explicitly limited. Their source does not
lie in any critical examination but beyond. Still, the attitude makes me, at least
in principle, open for the possibility that they too may need such a limitation 
at some time. But I may nourish implicit and explicit convictions of various
degrees of constancy, some being easily disposed of, other being of fundamental
importance so that only the most extreme circumstances may shatter them.
\com{
Perhaps, knowledge in a stricter sense, involves a rational acceptance with the
explicit statement of such limits.}

\pa
{\em Explicit concepts} -- unlike literature or poetry. (lists of 
distinctions; perhaps, just enough words for the vague concepts)
and thesis~\ref{th:panrat}

-- don't multiply concepts and distinctions. 
They never match experience, while one thinks 
that, more and more disitinctions will bring one closer. Nothing more 
illusory!

\pa
Of course, all cannot be embraced with a scientific precision ...

-- {\em Proper abstractions}! ... create = ignore $A$, assign weight to $B$; 
economy of signs.


\ad{Is vs. Ought}
Ontology vs. axiology; the world as it is vs. as it should be. There is no 
doubt that we can speak about our wishes as to what or how the world should 
be. That we can speak about how it actually, really is, looks much more 
like a wished for assumption. And taking into account the 
rather poor results of the attempts at establishing any form of consensus 
in this matter, it even looks like a possibly wrong assumption. 
It pushes ethics on the side -- for who can doubt the primacy of the real 
reality over mere \thi{ought} -- while, at the same time, leaves theoreticians 
in an eternal despair over the lost connection with life. 
The separation of theory and practice, engendered by the conviction that they 
address the same reality, makes theory irrelevant and practice unreliable. 
It separates man from the world he lives in, that is, from himself. 

In the middle of the pretensions to know how the world is, perhpas even how 
it must be, and -- what 
is usually even more annoying -- moralising advice on how it should be, 
the crucial question asking {\em how it can be} appears highly
\he{reproachable}. Avoidings both other questions, it promises a mere 
\la{Weltanschaung}. But avoiding these respectable questions, it 
also avoids the schizofrenia they caused. And furthermore, it avoids the almost 
inevitable arrogance of the attempts to answer them and claims to having 
done so. And who would daresay that, on the final account, and in spite of 
all the claims to the contrary, any philosophy -- and any of its 
opponents -- offers anything more than 
a \la{Weltanshaung}, a statement ``look, you {\em can} see the world this 
way''? One may try to hide behind claims to objectivity and absolute truth 
but then, if not anybody else then eventually the history will fetch one 
from this hiding place.

\pa\label{pa:tomee}
It all may be wrong. It may be, however, only wrong for you. It may also be 
wrong for many others while it may seem right to you. It is so to me. 


\pa
This is, roughly, the history of Western philosophy, perhaps, its short 
version for the lazy students. This is also the theme of this book -- for 
those who, in their perplexity, find simply living the life somehow ...
unsatisfactory. I doubt that anybody can learn from it anything which he 
otherwise does not know. It is only, a sort of, giving an account. And giving an account is 
like paying a bill -- it brings the satisfaction of fulfilling an obligation. 
But since the smartness and skills of a self-made one have reached the
 sky on the stock market,
getting away without paying the bills is a reason to pride. Paying them is 
stupid. \citt{For the people of this world are more shrewd
in dealing with their own kind than are the people of the 
light.}{Luke, 16:8}

\pa
Philosophy attempts to derive something obvious from something acceptable. It
isn't an art of discovery but of appreciation.

\noindent\dotfill

The problem of the existence of the external world comes from
\begin{enumerate}\MyLPar
\item\label{sub-ob}
thinking of our being in terms of the idealized, timless subject-object
relationship, and 
\item\label{dedworld}
the attempts to construct or deduce the world from within such a horizon 
of pure actuality.
\end{enumerate}
It is just one of the problems arising from this perspective, totality of my being, 
and phenomenon of time being other well-known examples. If, on the other hand, 
I think of myself as being capable of continuous experience and take time as
a fundamental, rather than possible, notion, the existence of the world beyond
my experience ceases to be a problem and becomes a fact of experience.

\pa
Let me emphasise again, but for the last time, that I am not looking for
necessary, irrefutanle truths. there is no {\em necessity} of such an
admission. One may let the all-powerful \co{objectivistic attitude} prevail,
reducing the world to a never reachable totality of independent objects. But
I do not aim at a \co{re-construction} of a pre-reflective truth, which is
what it is and cannot be otherwise -- which is necessary. Such a truth may be
of interest to a scientist but, as I said before, it cannot satisfy
reflection.  Perhaps it is true, perhaps everything we experience can be
reduced to some basic, pre-reflective facts, even to some necessary physical
or chemical reactions. But then a naive question arises -- so what?  How can
this ``wisdom'' affect my life and actions? Is it at all relevant that
something, which cannot be otherwise, is as it is? 

Suppose that somebody produced a definite proof that it is so. And? Shall I
change anything in my life because of such a proof?  Shall I stop feeling the
way I do and choosing as I do? The only way such a proof might become
relevant would be to produce some controlling mechanisms.  Perhaps, having
traced everything to such a basic level will enable us to control everything
which is above -- and as some maintain, independent from -- it. Then it might
be a different story, but only to the extent, that our feelings and choices
could span over a different, perhaps wider range.  So far, the very
meagerness of such results makes the whole project as irrelevant for
philosphical reflection, as it is relevant for the \co{objectivistic
attitude} of science and controll. Let those who believe in it, work on it.

Necessary truth is what it is and {\em cannot} be otherwise -- do what you
want.  Thus, although it determines me, it does not affect my reflective
attitude -- it must be so, and there is nothing I can do about
it. Philosophical flirts with necessity do not interest me not because one
cannot produce cunning arguments in its favor but because they make the whole
endeavor existentially irrelevant. Firmness of existence arises from the
fragility of its background, not from the necessity of its assumptions.  Even
if at the bottom everything is involved in a play of deterministic forces,
this, certainly, is not the reality of our \co{experience}. In particular, there
is an element of freedom in reflection, if not in anything else, then at
least in the choice of its object. Reflection may admit that something has
relevance but that it cannot or does not want to focus on it and chooses to
focus on something else. Appeal to necessity of considering this rather than
that is a simple act of disrespect. The point is not to demonstrate necessary
truths but truths which are of interest, not to \co{re-construct} but, yes,
to construct.  What matters for me is not an inescapable result of necessary
processes, but the question  how and, above all, towards {\em what} direct
my reflective attitude.

\pa
What I see as the aim of philosophical enterprise is not to build a theory about 
how things really are, even less to build a theory which simply isn't inconsistent. 
Rather, I see it as some \thi{economy of language}, \thi{economy of words} which
should be spoken in as purposeful manner as possible.
I do not think that many
(if any) proponents of philosophical theories which we would classify 
as implausible ment really ridiculous things -- but their language sometimes makes 
us feel as if they did. ``Reincarnation'', ``immorality'', ``deeper Self'', 
``God'', ``external world'' etc. may all happen to refer to some things 
one wants to convey in a way unmatched by any other words. They may make 
speaking simpler, thus appearing most economical and
far better than ``circulation of matter'', ``immortal fame'', ``super Ego'', 
``opium for people'', ``transcendentally ideal world''. 
We have a tendency to believe that the latter express some concepts and (therefore?)
are more precise:
they reduce the former to something apparently more understandable, more
adequate for a reasonable discourse. But I am not sure that a discourse is
``reasonable'' when it excludes or simply distorts matters unclear and
never given as well defined objects, only because they
cannot be stated in unambiguous terms, preferably with the precision 
approaching that of mathematics.

\pa
The goal of reflection is not to re-construct the original, pre-reflective
truth. The origin of reflection -- experience -- harbours the possibility of
reflection and the material for it. But is made of a different
matter which cannot be fully and completely re-created with the categories of
reflection. The concrete material of pre-reflective experience can be, at
best, mimicked, symbolised but not re-constructed by reflection.

The only thing reflection can (and should) do is to construct its own
world. Its own, because reflection, like any other activity, can operate only
with its own categories. But this world should be constructed in such a way
that its original truth can live through and within it. One should strive
after the concepts which not only do not falsify or oppose the pre-reflective
feelings and understanding, but which actually open for their presence, allow
them to enter the world of reflection and unfold therein.

\ad{Novelty}
New insights -- rather inspiring new branches, 
history, sociology, psychology...

-- But otherwise, human-wise, \la{nihil novi}, 

-- [used in chap. III, Intro] \citt{\la{Mundus senescit}}{Gregory of Tours, {\em History of the 
Franks}, \~594AD} ``The world ages''

So WHY?

-- ``One can't construct a system which captures (the whole) reality.'' But, in
fact, more is true -- one cannot capture reality into words! Kierkegaard's
criticism of a system does not (really) address systematic analysis but the
intellectual arrogance believing that it does manage that...

-- A new philosophical book is but an expression of understanding, or 
misunderstanding, the old books. There are no new thoughts, no 
thoughts which were not thought before. The goal is not to think new 
thoughts, but to understand the old ones. 

Heidegger recommended return to Greeks, to Heraclitus, if possible, 
even farther back. So did Rousseau\ldots There is no kernel, no 
historical site -- it is everywhere, and nowhere\ldots

\subsection{We have \ldots, and shall\ldots}

\ad{Actuality}
We have heard the trumpeths!  
We have noticed the importance of distinguishing the beings from 
their Being, we have noticed the metaphysical preoccupation with
presence or, as I shall call it, with \co{actuality}. 
The barbarians have entered the gates of Rome and \ldots have found 
their place there. 
The Derridean fear of being accomodated, 
appropriated, embraced by the philosophical discourse was well 
founded because it was based on equally one-sided, as it was deep, 
understanding of this discourse.

After the initial fascination, seduced by vague promises of something 
new and different, even general public gets quickly tired of 
\wo{writing otherwise}, 
\wo{speaking otherwise}, all these \wo{otherwise} which, eventually, 
turn out to be nothing else than the other, ignored side of the discourse 
against which they tried to protest or, often, just a new way of 
saying the old things.
%%% `thinking otherwise' is in Book II. level of actuality

\pa
The barbarians have won, they upset the Rome, its order and its life. 
But there is a great danger in a great victory. 
And now comes the time of defeat, of settling down, of accomodating to 
the culture which lasted some thousands years not by a 
mere accident, not by a mere misunderstanding,
but because it was founded on something real. The barbarians 
have won because they had a good point to make or, in any case, because 
Rome did not have anything to counter them. But the noice of 
victorious entrance is but a passing news, and the horses on which 
one makes one's tryumph will soon be weary and have to be replaced. 

One thinks that conquering Rome everything has to be renewed, started 
anew; one lies down new foundations, challanges to new ways of 
building, living \ldots yes, thinking. And what? A short time passes 
and it turns out that people think as they used to, that people live 
as they used to, that the same emotions and passions steer their 
lives. The only things which, possibly (but even that not necessarily) 
have changed are the cloths. 

I do not believe in \wo{thinking otherwise} and, to a large extent, I 
abhore most of \wo{otherwise writing}. The barbarians overlook, in 
their tryumph, that the conquered traditions had developed the ways 
for accomodating all aspects of experience and that it did it, 
exactly, in constant confrontation with life. They were not 
neglected, they might only have been invisible at the places which 
attracted the eyes of newcommers. 

\pa
The power of the \wo{metaphysics of actuality}\footnote{I am 
reserving the word \wo{presence} for something else, so I have to 
translate this phrase in the above way.} needed undeniably a correction. 
But this can only be a correction of attitude. As long as one 
attempts \wo{to think}, one will do it in familiar categories. And it 
is hardly to be expected that, because of the fall of Rome, one will 
stop thinking. The difference may concern the significance one attaches to 
one's thinking and its results. And, not least, to the results of 
other's thiking. If everything is merely a question of interpretation 
then, certainly, most of the texts which one would like to subsume 
under the \wo{metaphysics of actuality} can, actually, be interpreted 
in different direction. They certainly are involved into thinking in 
terms of actuality but the point is -- what thinking is not? It 
remains to be seen what the hoped for \wo{thinking otherwise} will 
have to offer. The answer is: \la{nihil novi}.

\pa
But point taken. Reflection has to watch its steps, that is, it has to 
carefully observe what it is reflecting over. It has to always keep 
the clear distance to its object, in order not to confuse it with its 
reality, in order not to confuse itself with the omnipotent power of 
control. This is, exactly, the difference of attitude. And, for that, 
possibly a nonexistent one because, although some earlier thinkers might 
certainly be accused of nourishing such a belief, many could not. 
Just like the texts enscribe to an extent the conclusions, so does the 
choice of focus determine, first, the choice of texts and, then, the way 
of reading them. The attitude may be only assumed in the other -- it 
can be controlled, if at all, only in onself. Especially, when one 
reads. Discovering an attitude of \wo{fascination with the actual} in 
all the texts, the possible question is about the attitude of the 
reader. 
In short, it is not a question of \wo{thinking, speaking, writing 
otherwise} but rather of {\em reading} otherwise or, perhaps, reading 
other texts. \wo{Reading otherwise} would simply mean to look for what 
one cannot find, to look for the traces of thoughts which one thinks 
are not there. Eventually, one will find them, but this requires a 
different attitude of reading.
\thi{Metaphysics of 
actuality} is not all of the traditional metaphysics, in any case, 
not all of the traditional philosophy.


\pa
Critique of \thi{metaphysics of actuality} (Derrida, Heidegger) wants 
to show something more, \co{non-actual}. Critique of objectivistic 
system of thought of Hegel by Kierkegaard opposes to it the experience 
of an individual, not just a common experience, but the fundamental 
experiences, elaborated through all the depth psychology. 

My point: the two are the same! The feeling one may get of 
existential relevance of Derrida, and certainly of Heidegger, is 
based exactly on that the personal/individual/etc. comes into the 
world of \co{actualities} through the \co{non-actual}. 


\pa
Follow Heidegger's, Derrida's critique of \thi{the metaphysics of 
actuality}\footnote{I will reserve the word \wo{presence} for 
something different.} but not his aggressiveness. There is nothing 
wrong with actuality, it is our way of conceiving the world. There is 
only something wrong with positing it as the only way of treating 
everything, all aspects of experience\ldots 

We are not scared of concepts as Heidegger, and later post-modernists, became.
We consider this fear unnecessary, in fact, impossible in length, sometimes
comical, sometimes hurtful...

\ad{\thi{The linguisting turn}\ldots} 
\ldots or shall we rather say, \wo{a linguistic twist}?  
One has always known that words and language
can not capture reality.  Not only there is no final, most adequate
and true name of God.  Even trying to communnicate most simple facts
and situations, when using words, we are forced to distribute the
responsibility between the one who is speaking -- to express himself
as clearly and understandably as possible -- and the one who is
hearing -- assuming that he not only knows the language, but also has
the sufficient background of experience to grasp the meaning of what
one is saying.  This sounds, perhaps, a bit exclusive, even elitistic,
and so it is supposed to sound.  \citt{Who hath ears to hear, let him
hear.}{Mat.  XIII:9} Success measured exclusively by publicity, by the
sheer number of receipents is always -- {\em always}!  -- inversely
proporitional to the value of the thing.  In the domain of spirit,
there governs a law opposite to that of mass culture, namely, the
more, the worse.  Even deep things, to the extend they are spread
around and received by many, become superficial, flat and low.

\wo{The sky was blue and the sun was shining.} is as
inadequate a phrase, if we try to measure it by some some objective
conformance to reality, as any name has ever been for expressing the
essence of God. When we speak, and even more when we write, we assume 
that the reader fulfills some basic conditions necessary for 
receiving the message. If, on the other hand, we do not know to whom 
we are speaking, if we, perhaps, assume that we are speaking to 
everybody, irrespectively of his knowledge, background, level of 
intelligence, and what-not, if we, for instance, are merely interested 
in selling the book to indefinitely large and, consequently, also 
indefinitely amorphous masses, we can not avoid getting confused. In 
particular, if we forget that any utterance, and also any text, is an 
address from somebody to somebody, if we abstract the text from this 
fundamental relation of intension and meaning, we can not avoid 
getting confused. 

\pa
One feels proud having recently overcome all these difficulties by
the \thi{discovery} that, as a matter of fact, there is no reality 
beyond the words, there are no human subjects beyond the text. As any 
thesis, so also this one can find innumerable reasons and 
justifications. I certainly won't spend time on reviewing them all. 
This inhuman abstraction of \thi{textuality}, this invention of 
intellectual despair over the impossibility to ensnare reality into 
words, does not deserve much attention. Yet, its wide 
popularity makes it hard to simply ignore it. The barbarians entered 
Rome and settled down, but their ways do have reprecussions\ldots

\pa
Reasoning, and understanding in general, is a way of using, inventing 
and arranging \co{signs}. Words are not the only possible \co{signs}, 
but let us be charitable and take all this \thi{linguisticity} in a 
very general, probably over-general, sense of semiotics. Without an 
adequate system of \co{signs}, there is no understanding. Ok, but 
there is no such thing as \thi{understanding}; there is only and 
always only understanding {\em of something}. What this something is, 
is highly vague, and the \thi{linguistic turn} is an attemt to 
substitute for our lacking, or in any case \co{vague}, understanding 
of what this is that we are trying to understand, something which 
apparently is much more precise, well-defined and hence manageable -- 
the very system of \co{signs} itself. Cetainly, a smart intellectual 
twist. However, the fact that something is \co{vague}, that we can 
not put the finger on it, does not mean that it does not exist. 

\pa Many \thi{linguistically twisted} will agree.  There exists
something but our way of speaking, the language and words we are used
to, are inadequate to capture it.  Therefore, we must \thi{speak
otherwise}, \thi{write otherwise}, invent a new language.  The ghost
of capturing the reality into words is peeping in through all the
holes.  What else, after all, can an intellectual do than to market
his words?  
%Among the most prominent (but by no means the only) examplars
%of such \thi{writing otherwise} one can probably mention late
%Heidegger and Derrida, perhaps, but only perhaps, late Wittgenstein. 
%These are, as usual, accompanied by a tremendous herd of noisy epigons
%who \thi{have seen the light} and start \thi{bubbling otherwise}.

%\pa 
If the reality which one senses beyond the inadequate words is
really a fundamental aspect of human existence, then is it reasonable
to assume that we had to wait thousands of years, until the day today,
to discover the need for \thi{speaking otherwise}?  Sure, one can
always patronize the past generations pointing out their mistakes. 
But such patronizing is but an aspect of a positivistic faith in the
absolute character of progress which we are not willing to take
without any reservations any more, if we at all are willing to accept
it at except, perhaps, for the sphere of technology and civilisation. 
Perhaps, the men of Reneissance, or of Enlightment invented
particularly obscure ways of speaking about some aspects of human
existence, ways which, due to their low value, became particularly
popular.  But there were wise men, too, in those times, weren't there? 
OK, they were all led astray.  What about the men of Middle Ages?  Of
early Middle Ages?  Heidegger would say, and what about the Greeks? 
Indeed, what about them?  But we do not have to follow the
idiosyncratic, not to say comical, attempts to revitalize a dead
language.  What about Indians, Jews, Chineese?  I do not want to boast
with false modesty, but I find it hard to belive that to address the
most fundamental issues we need an entirely new way of \thi{speaking
otherwise}.

\pa If all history of human culture, in any case, of written human
culture, has been perverted by the inadequate language, then what
makes us, some of us, today read Plato, Aristotles, Lao Tse, Bible,
St.  Augustine?  For my part, I can say for sure that it is not a need
to find their mistakes but, on the contrary, the conviction, and I
should be able to say -- the experience, that they do have something
important to tell me.  I doubt that many others read them only for the
sake of intellectual curiosity, deconstructivistic exercise or
academic career.

\pa
OK, so old texts may, sometimes, contain some valuable insights. Or, 
shall we only say, some valuable words? Do these texts contain 
anything \thi{in-themselves}, anything which is not read into them by 
the reader?

If text is entirely open to the arbitrariness of the reader and the
indeterminacy of possible interpretations, if it has no intension, no
truth it tries to communicate, if any system of signs is but an 
arbitrary invention with but an arbitrary relation to what possibly 
might be lying outside it, then one might almost agree that
writing should not involve any attempt to make it understandable and
accessible -- {\em what}, in such a case, should one try to make 
understandable?  Probably, one does even better resisting any such
attempts and we have been exposed to many examples of such a 
resistance. 

Unfortunately, or rather fortunately, all texts have been written by 
humans with some intensions. They may communicate these intensions in 
better or worse ways, in more or less adequate form, in more or less 
understandable and plain manner, but intensions are there. And if one 
does not like the word \wo{intensions}, in particular, any 
\wo{intensions of the author}, then it will still suffice that, having been 
written by humans, they contain human expressions of human experience 
and life. This, at least, is more than sufficient for me and if it 
isn't for you, then you are free to keep dissolving words, expressions 
and, eventually, insights in the mud of \thi{intertextuality}.



No private language -- so, no private thinking?

language -- changing, ok, but still not changing and translatable

New problems? New solutions? What? -- check it out, it has all been 
there\ldots

\pa A completely different, even opposite, dimension of the
\thi{linguisitic twist} is analytical philosophy.  A name which seems
more adequate to me would be something like \wo{logical analysis of
linguistic behaviour} because, as a matter of fact, I have found
extremely little and also extremely meagre philosophy in this camp. 
This, however, is due only to my biased and misunderstood conception
of philosophy, so let us not quarrel about the names.

The strength and, indeed, the visible vital force of this camp is
based on the fact that it tries to focus on more or less identifiable
and well-defined problems.  The resulting insights and proposals have
often much intrinsic value and I am last to deny that.  I am a bit
more unsure what this value is and to what purposes it could be used,
except for possible applications in cognitive science, artificial
intelligence, linguistics and, perhaps, legal theory and new design of
dictionaries.  Even if only of pragmatic -- shall we say, scientific? 
-- relevance, so, being usable, they also represent some kind of
value.  I want to say this with all due emphasis because, having said
that, I think we can leave this camp for itself.  Hopefully, with
time, it will, as any scientific community earlier in history,
establish its identity independently from philosophy and then continue
resolving all its specific problems and particular issues with the
undisturbed peace of mind to the best of whoever will want to use
them.
%
\sep 
%
\pa This is as much I have to say about all the linguistic twisting
and I leave writing of the voluminous binds of history and analyses of
the involved \thi{problems}, \thi{questions} and -- yes!  --
\thi{solutions} to the competent scholars.

\ad{Genealogy}
Not in time, but from virtual to concrete $\sim$ gradual 
differentiation

Then: refinement of systems of things/concepts; these consist of 
simultaneous $\sim$ equiprimordial \ldots correlates/aspects (find a good 
word)


\ad{Notation}
\begin{itemize}
\item ``\herenow'' \impl ``here and now'': dissociation of aspects
\item ``\co{recognition}'' \impl ``\co{re-cognition}'': rozpad into 
more detailed parts.
\end{itemize}
ntuitions should be preserved -- for more detailed understantding, one 
should first keep in mind the possible presence of another notational 
variant, and then check it \ldots




} % end the main \noo{%END

But if such a useless nothing, why \la{summa}? 
What could possibly make one write a \la{summa}, of any kind, nowadays?
Complexity -- not only of the incomprehensible totality of the world, but even
of every single issue -- makes it look pretentious. Things fall apart
and observing the dissolution or, as many a one pretends, praising it as the
openness unto plurality, is the only reasonable attitude, in fact, the only
politically correct way of distancing oneself from the trauma and danger of monism,
that is, totalitarianism.

Yet, philosophy which does not try to capture any unity ends up as a catalogue
of particular cases, particular concepts or just words, which may be as
elaborate, intelligent and intricate as it is uninteresting and useless.
Philosophy which distances itself from any attempt at reaching some wisdom, and
that means in particular, thinking and relating to the ultimate questions, 
ends up calling its impotency for modesty and glorifying the incessant
questioning, that is, public scratching one's head with the emanating
self-assurance that all genuine issues should be left to those who are
unintelligent enough to expect any answers. After all, it takes an analphabet to
believe that truth is written some place.
% But receiving guidance does not exclude autonomy and it appears to us quite
% unfortunate that
%

Philosophy attempts to think what one knows. For to know means much more than
what Plato explains in \btit{Timaeus}, or what others have tried (and are still
trying) to specify with
respect to explicit, reflective thoughts, \gre{episteme}. Sure, I know the
pythagorean theorem and the way from my house to my work. But I also know that
I will die, I know my love of my mother, I know how she loves me, I know what my
best friend likes and of what my girl-friend is afraid. I know that I am the
same person I was yesterday and I know that, from behind the invisible forces,
the hidden eye of God's  watches everything. Most of these things I am not
actually able to think, not to mention, to express precisely and explain with
any degree of stringency.
%
We know what philosophy is, don't we? We know what is {\em not} philosophy and
we know what is, at least, when we encounter it. We can even tell good
philosophy from bad one. And yet, can anybody tell what we thus know? Can
anybody think this knowledge clearly enough to express it precisely to
everybody's, or even only to one's own, satisfaction? 
%

Knowing, we might perhaps say, is what allows me to
consent to some thoughts and to reject others. For when I content to the truth of
a thought, I do it in the light of something I know. Only in very special cases
this knowledge consists of other, explicit thoughts.
%
To think what one knows is a challenge, just like it is 
a challenge to become what one is. Only to a self-satisfied rationalism such challenges
may appear as paradoxes, and only to a narrow-minded pedantry as contradictions.

The attempt to think what one knows need not (as it seems, can not any longer)
rely on necessary principles and forcing arguments. Those who try
\citet{withholding their consent from any proposition that has not been
  proved}{CiceroGods}{ I:1. [In the translation of F.~Brooks: \wo{refusing to
    make positive assertions upon uncertain data}.]} end up with absolute 
certainty -- about nothing. The rigidity of irrefutable argumentation which
attempts to force its conclusions and tries to dispense with everything which
escapes such attempts seems, indeed, to empty every phenomenon of its
concreteness yielding only residual site whose necessity equals its hollowness.
But the lack of the universally forcing arguments is not the same as the lack of
any truth, the lack of the necessary laws is not the same as the lack of any
order. Besides (or instead of) the effcient causes there may be many necessary
ones, besides (or instead of) the sufficient reasons there may be many
insufficient -- but, as Weber would say, favourable and supportive -- ones. Just
like explanations try to capture the former, so a more modest account may try to
identify only some of the latter.

\citeti{I say the following about the Whole ... Man is that which we all
  know.}{Democritus}{ DK 68B165}
Philosophical anthropology tries to give an account of human life. This does not
mean isolating this specific \thi{object} and ignoring all the rest. It amounts
only to seeing all the rest in a specific perspective: not as a set of
inviolable mechanisms but as the field of unfolding of human existence.  The
dualism of deteminism confronting freedon, of \thi{objectivity} confronting
\thi{subjectivity} is only a 
result of a particular way of thinking  the existential
confrontation which precedes the understanding -- and dissociation -- of these
two (and other) aspects.  The two, posited as the ultimate poles, can never not
only agree but even meet and one is constantly forced to the impossible choice
of the one or the other. Above and across this impossibility one may try to draw
the border between the two in ever new ways.  For even if the number of the
elements, say, of the constitutive dimensions of existence, is limited, the concrete
borders between them may be drawn in an unlimited variety of ways. Every human
being draws these borders anew and the abstractness of the fact that somebody
else did it earlier in a similar way does not in the slightest diminish the
concrete need to do it ever anew.\ftnt{Besides, \thi{limited} can be, in all practical
  respects, as good (or bad) as infinite. On how many different points can any
  two philosophical systems differ?  Ridiculous question, but to play the game
  only for a while, let us say some number which certainly is lower than it is
  in fact, say: 100. Simplifying further, let us say that on each of these
  points one has only a binary choice, yes or no, + or --. How many different
  \thi{systems} do we get? Well, $2^{100}$.\noo{This is, if we were to believe
    such calculations, approximately the square (perhaps cubic) root of the
    estimated number of atoms in the universe.} If humankind produced one such
  system every second, it would take some $2^{75}$ or, rounding off, $10^{20}$
  years to merely produce them all. This is a bit more than twice the estimated
  age of the universe.\noo{10^9 And merely producing them is not even the
  beginning of anything.} Thus, even if the 
  number (of possible forms of existence, of possible philosophical systems)
  were finite, the simple combinatorics is on the side of 
  unrepeatability -- not principal, however, not in some ideal infinite limit,
  where indeed the finite number of possibilities would have to be reapeated infinitely
  many times, but only in practice, that is, in fact. (If one wanted now to
  reduce the number 100 to, say, 20 such points,\noo{ at 
  which two systems might differ} it still leaves over 1.000.000
  possibilities.)} 
%
Everything has been said before -- {\la{mundus senescit}} (\wo{the world has
  aged}) noticed St.~Gregory of Tours in the VI-th century and some 1500 years
before him a preacher observed \citeti{[t]he thing that hath been, it is that
  which shall be; and that which is done is that which shall be done: and there
  is no new thing under the sun.}{Eccl.}{I:9} This may sound a bit depressing
but only to an intellectual capable of dealing exclusively with abstractions, or to
the petty mentality occupied with even pettier novelties, one \citeti{consoled
  by a mere trifle, as it is distressed by a mere trifle,}{after
  \citeauthor*{Pensees}}{II:{136}} for which everything beyond the yesterday's
scandals and today's news is a boring repetition. But \citeti{what is right can
  well be uttered even twice.}{Empedocles}{ DK
  31B25\noo{\citaft{FirstPhil}{: 58, p.165}}} Moreover, to understand something
  said before one often must say it 
oneself. Every existence says anew something said before, but saying this
concretely amounts to drawing anew the distinctions in the matter of life, the
distinctions between abstractions like \thi{love} and \thi{indifference},
\thi{indifference} and \thi{impotence}, \thi{impotence} and \thi{thirst},
\thi{thirst} and \thi{lack}, \thi{lack} and \thi{illusion}, \thi{illusion} and
\thi{lie}, \thi{lie} and \thi{truth}\ldots The infinite concreteness lies in
such distinctions and the lack of final definitions is only another side of this
concreteness. And the lack of rigid definitions does not mean the lack of
significant distinctions.

Giving an account of -- rather than explaining -- human existence, we are not
very concerned with many accepted distinctions. We certainly have to keep
various points of the discourse at approximately the same level of abstraction
but this need not prevent us from addressing also issues which the
administration of academic life places at different
departments. \citet{Philosophy is first of all a science about human being,
  about integral human being and of integral human
  being.}{Bier}{I:2\kilde{p.21}} 
% The advantage of philosophical anthropology is that no posited
% objectivities-in-themselves need to disturb us.
Asking a question, we will often rest satisfied with relating it only to
the place it occupies in the field of existence. Whether the question happens to
sound (or traditionally even {\em is}) theological, psychological, mythological or
astrological is of no importance, as long as it addresses an aspect of this
integrity. \wo{\la{Summa}} from the title refers to such a summary, to the 
attempt to 
gather quite different, sometimes even disparate, aspects into one unity and not
to the ultimate summit, 
to the vanity of collecting all relevant (and irrelevant) details in a systematic
and scientific totality.

%\tsep{hee}

All relevant (and irrelevant) details can be abstractly thought as a graph with
various edges (of relevance, dependence, association, etc.) connecting various points.
Traversal of a graph can be, in general, perfomed in two different ways: depth
first ({\sc df}) or breadth first ({\sc bf}). {\sc df} starts in the actual node
and follows one path -- an edge to a neighbour node, then to some neighbour of
the first neighbour, and so on. (Encountering a previously visited node, it
backtracks and tries another path.) {\sc bf}, on the other hand, vists first all the
neighbours of the actual node, before proceeding to all the neighbours of all
first neighbours, and so on level by level. 
A scholar is a {\sc df}. We do not have equally general, and certainly no
equally respectable name for the {\sc bf}, but it could be associated with a
dilettante -- knowing a little bit about everything which concerns him in one
way or another.

Of course, the world is not a graph. If we insisted on the analogy, we would have to
extend the idea at least by suggesting that the graph is unlimited, if
not actually infinite, and that in at least twofold way: every node has
infinitely many immediate neighbours and, from every node, there is an infinite
number of infinite paths (which never enter a cycle). A scholar diggs thus
further and furhter away from his home, trying to get to the end of an infinite
path, and hoping that it will bring him back home. A dilettante, on the
contrary, circles around, always in a safe distance
trying only to cover the infinite circle 
surrounding the house. Neither ever completes the road, both seem \citeti{to join
together diverse peaks of thought,\lin And not complete one road that has no
turn.}{Empedocles}{DK 31B24 [translation after \citeauthor*{Emped} 24.]} Scholar
ends up knowing everything about nothing, a 
dilettante knowing nothing about everything.

But this is unfair to the scholars! How can one compare them, put them on equal
lines with dilettantes?! Probably, one should not, but \wo{dilettante} is only a
name, the better of which seems hard to find. So let us ask the scholars...

Does the world -- eternally returning in the cycles oscillating between Love and
Strife -- exist twice (on the way from Love to Strife {\em and} on the way
back), or only once? And if once, then why, when and how?
\kilde{Szczerba,p.63-65}The diverging opinions may be the
consequence of the lacking sources which might have possibly contained the
answers of Empedocles himself. But, as a matter of fact, the question might have
never been asked and the answer never intended. Perhaps, all that was meant, was
to point to \wo{The world-wide warfare of the eternal Two}, the Love pushing to
unity and Strife to separation?  Perhaps, all the cosmogonies and cosmologies,
reflecting only the human understanding which, eventually, is always only
understanding of oneself, are but images never meant to be studied in and for
themselves. But sure, the questions can be asked, and so constructions can start
spinning...

What is Plotinus rejecting in V:5.1, claiming that the intelligibles, perceived by the
intellect, are not \thi{propositions}, \thi{axiomata} or \thi{sayables}? Does he
claim that the intellect's \thi{knowledge} is non-propositional or only
non-formalisable, non-expressible or only non-representational? Indeed, one may 
ask and keep answering, and many distinctions can arise from such diputes. But,
the question is, what shall we do with all these distinctions? Is it reasonable
to assume that, although the text does not say anything clearer, the intended
ideas were nevertheless so much more precise? Or, perhaps, they were not but
they have become so in the course of history?
Certainly, scholars shall sort out what Plotinus actually said
and meant and what he did not -- but preferably only as far as it goes. The fact
that questions can be asked with respect to a text, does not mean that the text
(and related texts, and texts related to texts related to...) contains the
answer, nor even that the answer can be at all 
meaningfully given.\kilde{EmilssonIntellect,p.29-30} We are rather clearly
\citet{told that the Intellectual-Principle and
the Intellectual Objects are linked in a standing unity}{Plotinus}{ V:5.1} and
even if \wo{we demand the description of this unity}, we should not dismiss the
possibility that what we have just been told is all we can actually get to know,
yeah, all that is worth knowing! Of course, 
to admit that, one would first have to learn cherishing and being satisfied with
perhaps clear and understandable but still vague expressions; expressions which
admit their limitation and put some trust in the reader who, hopefully, is the
same kind of being as the writer. All this, however, might amount to ceasing being a
scholar and turning~... a dilettante? 


There may be, and almost always there are, many meanings and one seldom can be
\co{precise} enough to narrow the expression, not to mention the thought, to
only one of them.  In this sense, interpretation is usually over-interpretation
and thus mis-interpretation. It is not so that we understand when we have made
the possible meaning most possibly, that is, impossibly \co{precise}. Dissecting
it into all too specific and detailed alternatives brings perhaps everything
under our control but only by making it impotent, by denying it life which our
thinking can only address, but never imitate... One may, perhaps, do it
sometimes legitimately in the name of scholarship, but positing it as the
universal aim of philosophy, and even thinking, sounds neither convincing nor
even plausible to many ears. Precising things beyond their limits results in
dissociating them and what is once dissociated can not be put together unless
one rediscovers the unity which preceded all the dissociated precisions. This
unity is not any combination, any coherence, any consistency of the elements
because it does not as yet know of any elements. \noo{This smells vagueness, if
  not directly misticism, so it may be a good point to begin.}

\tsep{???}


\tsep{}

One of such questions, perhaps the only question of all philosophy, is:
%
%\equ{What is true?\label{questA}}
\begin{center}What is true?\end{center}
%
Some try to answer it in the way it is asked, that is, in
dissociation from any person asking it. Such an answer amounts always to
specifying what one, everybody, you and I {\em should} accept as true.
The \thi{should} is usually surrounded by the arguments and proofs which should
convince everybody. 
Unfortunately, any respect expressed by imputations of rationality, accusations of
irrationality, expectations of a direct and infallible communication is, 
 at best, merely the respect for some \thi{rationality}. As far as 
a personal meeting with the reader, that is, as far as the person of the reader
is concerned, it shows disrespect equal only to that displayed by the strangely popular
reluctance to express one's meanings in an understandable form. The difference
of proceeding results only in that sometimes the latter, but never the former,
can be excused on the assumption of incapacity. 

The best, in fact, the only thing one can do is to answer this (as any other)
question for oneself or, what amounts to the same, to accept some known answer.
The answer may be communicated to others but only as {\em my} answer -- I may be
convinced that you should accept this answer, too, but I do not even try to
convince you about that. What you do with my answer is your sake, in fact, to
understand it you have most probably already known it, even if you did not think
it yourself in the same way. 

\citet{Gradually it has become clear to me what every great philosophy
so far has been: namely, the personal confession of its author and a
kind of involuntary and unconscious memoir [\ldots] In the philosopher
[\ldots] there is nothing impersonal.}{BeyondGE}{ I:6} Moreover, the
\citet{beliefs to which we most strongly adhere are those of which we should
  find it most diffcult to give an account.}{BergTime}{ II p.135}
% \citt{It is almost incredible that men who are themselves working 
% philosophers should pretend that any philosophy can be, or ever has 
% been, constructed without the help of personal preference, belief, or 
% divination.}{W. James, Essays in Pragmatism, I, p.24}
Thus, since this, like every other question, is asked -- and answered -- only by a
concrete person, we can take it to be the same as:
%
%\equ{What shall I accept as true?\label{questB}}
\begin{center}What shall I accept as true?\end{center}
%
Here one might object: one senses a difference, the element of subjectivity, or
perhaps relativity, in the latter which does not disturb the former. But the
difference is only apparent: every answer to any of these questions, can be used
also to answer the other. If something is true then, sure, I shall accept it as
such.  On the other hand, if I accept something as true it is because it is --
as far as I can see, imagine, feel, understand, speculate -- true.  Unlike the
former, this later implication has been judged problematic, mainly because one
imagines some voluntary act of acceptance which one's subjectivity could decide
to perform as it wished. The meaning of this implication will be our main theme
and its validity will rest on the fundamental difference between relativity and
subjectivity.

{Relativity of every observation and conclusion to the
  subject making this observation and drawing this conclusion seems to be the
  most obvious thing in the world. However, one can draw quite the contrary
  conclusions from this observation. On the one hand, one can conclude that this
  relativity for ever prevents us from gaining an insight into the true nature
  of the world, into the true nature of things as they are \thi{in themselves}.
  The underlying assumption is that things indeed are something specific \thi{in
    themselves}, and even that they, at least in principle, can be described
  \thi{in themselves}, as if independently from the view of the one making the
  description. The contradictory nature of such a project seems hard to accept.
  So, on the other hand, one can draw the conclusion that there are no things
  \thi{in themselves} and, consequently, no measure of truth -- in short, that
  relativity amounts to complete relativism and truth to mere (perhaps even
  merely volitional) subjectivity. But does the fact that the specificity of
  things \thi{in themselves} is only ideal and posited, show the ultimate void?
  Does relativity of particular truths exclude absolute truth? 
Both these interpretations start from the assumed duality. The
intuition that the standard of truth carries an element of transcendence is
opposed to the observation that absolute transcendence, dissociated and remote
from any immanent view, is inaccessible and hence, at least in the latter case,
irrelevant. We will instead take relativity as the basic phenomenon which
  precedes and underlies various dualities like subject vs. object,
  immanence vs. transcendence, acceptance as X vs. being an X, etc.}

\ins{ [Reality -- is between] "Now, the quantum postulate implies that any observation of atomic phenomena will involve an interaction with the agency of observation not to be neglected. An independent reality, in the ordinary physical sense, can neither be ascribed to the phenomena nor to the agencies of observation..."

Bohr, N. (1934). Atomic Theory and the Description of Nature. Cambridge, Cambridge University Press (p. 53).

 N. Bohr, The Philosophical Writings of Niels Bohr, p. 54, quoted in
 T. J. McFarlane, Quantum Mechanics and Reality, 1995 (quotes also: 

All things -- from Brahma the creator down to a single blade of grass --
are. . .simply appearances and not real. 
[Shankara, Crest Jewel of Discrimination, tr. Chris Isherwood, p. 97])

apparently well-menaing and also well-thinking people associate the two...
}

Taken contra-positively, the implication from my acceptance of X as true to X's
truth says that I should not accept any untruth.  Emphasizing this direction
amounts thus to a more modest project: not necessarily finding {\em the} truth
but rather, or at least, simply avoiding untruth. In the able hands of a
quibbler this modesty might reduce to a mere scepticism but we will hopefully
manage to say also something meaningful.  We should only keep in mind that we
are not following the fashion of {\em reducing} the first question to the second
(which results in the interminable quest for \thi{subjective} criteria of
\thi{objective truth}). We claim simply that the two questions are genuinely
equivalent, that everything \citet{that is known, is comprehended not according
  to its own force, but rather according to the nature of those who know
  it.}{Console}{ V:4.25\noo{154}} \noo{\citet{everything which is cognizable, is
    cognized not according to its own power, but according to the powers of the
    cognizing.}{Console}{ V:4.25 \noo{Filozofia Sredniowiecza,p.106}} }

Giving only an account -- rather than explaining -- we will ignore many
distinctions. We will not, for instance, treat the above question as {\em the}
question, focusing on which demands subordination and dissociation of
 all the others. Every genuine problem of human
existence invovles necessarily all the others. One can frequently meet attempts
to address a specific \thi{philosophical problem}, elaborate in the scope of a
single paper the problem of free will, the problem of meaning, the problem of
truth.  Interesting and perhaps even to some extent legitimate as such attempts
may be, they reveal the prejudice that such a division is at all possible, that
distinct problems indeed can be treated in a relative independence; eventually,
that only the closest scrutiny of the most minute distinctions is able to give an
adequate description of any single issue. But the problem is that we not only do
not quite know how one problem could or should be addressed and approached
-- we do not quite know {\em what} any particular problem is. Attempting to isolate any
particular problem for a separate treatement, we fight first with circumscribing it in
any reasonable way which, however, {\em never} reaches the goal of becoming
entirely satisfactory. In this process of 
partial circumscription we invest the problem with all the relations and
implications it carries to all the other problems. 
If one tries to address, say, the issue of freedom without at the same time
illuminating the meaning of subjectivity and openness to truth, the sense of
meaningfulness, the presence or absence of the absolute, in short, without
addressing the intergrity of whole existence, one ends up with the
distinctions one started with and keeps opposing arbitrariness to determinism,
subjectivity to objectivity, spontaneity of feelings to the rationality of some 
inviolable laws, etc.  Every (not only fundamental) issue is the sum of what it
excludes, is the border contracting the tension between this issue and others
into which it is interwoven. In a bit strange (but, in fact, quite understandable)
dialectics, the tradition which had marked the XX-th century with the missionary
zeal of dissociating all the issues and bringing them under objective, systematic and
separate analysis, ends up with the holistic and coherentist postulates, whether
with respect to language, meaning or truth.  \noo{(Wittgenstein,
  Quine, Davidson)} Even though one can not forget the idealistic origins of all such
potulates, one would still like to deny their 
idealistic connections. And one almost manages that, at least, 
as long as one keeps dissociating, as long as one sticks to dissecting one 
particular issue at a time.

We will not try to establish any totality which, in the presence of all too many
accepted distinctions, would indeed be a vanity. We will therefore ignore many possible
distinctions -- not because they would necessarily destroy any unity we might
wish to find but because they would (tend to) completely obscure it. 
We hope to avoid the accusations of relating the unrelated by
making at least plausible that the intimate affinity and kinship of vague yet
distinct aspects, their genuine unity preceds more rigid dissociations, and that
the latter mark only the end -- or perhaps the middle, but certainly not the
beginning -- of the road.
%
\noo{We will certainly try to avoid pitfalls of pantheism (of which most such holisms
are examples), but only because unity can be found above it and not, as some
also trying to avoid the same claim, because it does not obtain at all.
}
%

% \addcontentsline{toc}{section}{Sources, references and conventions}
\subnonr{Sources and references}
I quote rather extensively and from rather different traditions.  However, I
never go into exegesis of the texts or analysis of the thoughts of others.  An
attempt to do so would make finishing this work impossible.  On few occasions I
make more detailed statements in order to illustrate differences which also
should clarify my meanings.  The variety of sources and inspirations makes me
even limit the quotations to the most succinct statements which, I think, express
some essential idea.  Although the basic rules of conscientious exegesis may be
thus violated, and some quotations might have even been not only drawn out of
their context but even adjusted to fit the present one, the intention is never
to violate the meaning of the quoted text. (Besides, exegesis is not our
objective.)
% Asked by a true Irishman \wo{Are you a drinking man or are you a fighting man?}
% I could whole-heartedly confirm the former.  I like company, and

Variety of traditions suggests that we should focus on affinities and often even
only vague similarities rather than differences and oppositions.  Was
St.~Augustine entitled to claim the presence of Christian truths in the
neo-Platonic texts, as he did in the much disputed and controverted passage in
\btit{Confessions} VII:9? Was St.~Clement of Alexandria right in the
similar claims of the affinity of the Greek philosophy and literature with the
Christian revelation? Was Philo Judaeus right claiming not only the similarities
between but even the direct dependence of Greek thought on the Biblical
tradition? Scholars might prove that they were all wrong pointing out
significant differences making the two views different and even incompatible.
The Greek spirit was, after all, completely different from the Christian one.
Perhaps, but this depends on how one draws the borders around the intuitions
like \thi{Greek spirit} or \thi{Christian spirit}.\noo{(Let us also notice that
  such abstracts, useful as they sometimes may be in philosophy, are primarily
  only of historical and sociological character.)}  One can always find
differences separating two views -- the question is at what level, and then,
what value one will attach to them as opposed to the similarities. (After all,
the neo-Platonic culmination of Greek spirit, with its severe critiques of the
emerging Christianity, provided the foundation for the depth of Christian
mysticism.) Opposing, say, Greek spirit and Christian spirit, one should never
forget that in both cases one is speaking about spirit which, incarnated in
opposing socio-historical and political constellations, remains at the bottom
human spirit. It takes some wisdom to recognise concrete unity behind actual
differences and to stop distinguishing when everything worth saying has been
said -- the problem of perspicacious thoroughness, as La Rochefoucauld observed,
is not that it does not reach the end but that it goes beyond it.
%only blathering and babbling has no limit while it 
We will for the most focus on the similarities and it is up to you
to decide whether they are only due to the negligence in observing the important
distinctions or, perhaps, they are justified because the possible distinctions
are of negligible importance.

\noo{I quote others because I find myself to be their friend, even if they might not
always be my friends. If I did not have the quotation, I would write something
similar myself.  Although one might use this to construct the accusations of
eclectism, I would consider such accusations as a proof of a lack of even
minimal good will if not also of intelligence; and so I will rather get rid of
the bad company than of the good one.  The context in which the quotation is
used indicates, hopefully clearly enough, the interpretation I have in mind. I
hardly ever subscribe to the totality of the quoted author's or source's ideas;
it is only the thought behind the quoted piece which I want to bring forth.
(This, I guess, is one of the reasons why I bother to write.)}
%In all cases (with so few exceptions, that they are not worth
%mentioning here), a quotation indicates my more or less full agreement
%with the quoted author 
%I mean that there is enough unity of thought precluding the judgment
%of this work as eclectic, but if you do I could probably understand
%the reasons.


There are a few special sources which deserve a comment.  The authorship
of {\em My Sister and I} is the matter of dispute and scholars can not
tell for sure (perhaps, rather seriously doubt)
that it is indeed, as is also claimed, autobiography written
by Nietzsche himself. %I could hardly care less, since finding
The authorship of relevant thoughts should not be that important.
%whether they were written by Nietzsche or a skillful and competent forger.
However, in an academic context the issue may become a bit sensitive, especially
when the claimed author is Nietzsche.  (It might be so, in particular, if one
wanted to relate the contents of this autobiography to his other works which,
however, I am not doing.)

%For me, it is Nietzsche, for e
Even if it were not Nietzsche, it certainly could be, though 
%As somebody said, it is \wo{how you imagined Nietzsche would sound if 
%you got him drunk}. 
the author might also have been more Nietzschean than Nietzsche himself. Facing
the lack of any decisive proofs or disproofs of purely textual, linguistic or
medical nature, we are left with the text which looks like it might have been
written, if not carefully re-read and edited, by Nietzsche.  The voice for or
against his authorship depends then on one's view of his thought -- whether this
text \thi{fits} into the image one has of his whole thinking and, not least,
personality.  For me, there is a perfect match with the image I had formed
before I found this book. (Possible objections against the portrait arising from it,
should be confronted with less extreme, yet by no means incompatible, impressions
of the close friend in \citeauthor*{LouN}.) \citet{In the end,
  \btit{My Sister and I} reminds me of a true story.}{Sirens}{} Having made this
reservation, I will quote the text as if Nietzsche was its author.

Another referenced text, hopefully of much less dubious value, is a collection
of early Freiburg lectures by \citeauthor*{PhenomReligio} [\btit{Phenomenologie
  des religi\"{o}sen Lebens}, Gesamtausgabe, vol.~60]. Some of these have been
reconstructed almost exclusively from the notes of the students. Thus the reader
should be warned that the quoted formulations, although reflecting hopefully the
intentions, are hardly Heidegger's. (In any case, they are translated by me into
English, and that mostly from the Polish translation of the German text. Well...)

Likewise, \citeauthor*{Celsus}, is only reconstructed from the extensive
fragments quoted and criticized in \citeauthor*{aCelsus}. In this case, however,
the breadth and details of Origen's response give reasonable confidence into the
authenticity of the reconstruction. Much worse is the case of
\citeauthor*{Porphyry} where even the attribution of authorship may be disputed
as the work is reconstructed mainly from the \btit{Apocriticus} of Macarius
Magnes which need not reflect the philosophy of Porphyry. These works are quoted
as if they were written by the authors to whom they are attributed by the
general (though not universal) scholarly opinion. For investigating the
associated doubts and controversies the reader may start by consulting the
referenced editions.

Two distinct editions of \citeauthor*{Periphyseon} have been used. The critical
edition (started by late I.~P.~Sheldon-Williams and continued by
\'{E}.~A.~Jeauneau) of volumes I, II and IV is referenced as just done, with the
number+letter identifying the page number and the manuscript as in the edition.
Volumes III and V are from the abbreviated translation by M.~L.~Uhlfelder and
are referenced to in the same way, \citeauthor*{Periphy}, with only page numbers
in this single volume edition. In either case, the volume number identifies
uniquely the referenced edition. 


\sep
%
One encounters sometimes cases when, in an English text, quotations and longer
passages are given in French, German or some other language of the original --
sometimes even Latin or 
Greek. Although this may serve as an indication that the text is addressed to a
particular audience, it is no more pleasing than any other form of intellectual
snobbery.  It is perhaps a good tone to know German, French, Italian, Latin and
Greek, but few people do and I am not one of them. Since I have used extensively
sources in other languages, I have attempted to access -- and if I did not
succeed then to translate -- all the quotations into English. (A few exceptions
concern passages of German poetry which I did not dare to attempt translating.)
Sometimes, I ended thus translating back into English texts translated
originally from English into another language in which I read them. Such cases
are marked as \thi{my {\bf re}translation...}.  Hopefully, this will not cause
any serious confusion -- to fix it, I have to find some time with nothing better
to do.

\subnonr{Some conventions}

All the works are referred by the English title, even if I used the source in
another language; this is then indicated in the Bibliography at the end of the
text. (A few exceptions are made when the original source is referred after
another author, as is often the case with collected works or fragments.)

The references to all the works look uniformly as
\begin{center}
  Author, \btit{Title} XI:1.5\ldots
\end{center}  
where the part before `:', typically a Roman numeral, refers to the main part
into which the source is divided (e.g., book, part, chapter), and the numerals
after `:' to the nested subparts.  The references to the Bible have no `Source',
thus `Matt. X:5' refers to \btit{The Gospel of Matthew}, chapter X, verse 5. 
(I have used primarily King James Version and commented occasional usage of
other translations in the footnotes.) 
Likewise, the references to pre-Socractics are usually given without any source by
merely specifying the author and the Diels-Kranz number, e.g., `Heraclitus, DK
22B45', where the number identifying the philosopher (here 22) is taken from the
fifth edition of Diels, \btit{Fragmente der Vorsokratiker}.

Identifying quotations by page numbers might have been reasonable in times when
most books existed only in one edition.  I have tried to avoid such references
but in a few cases, where the structuring and numbering of the text happens to
be very poor, I had to use this form. This is also sometimes the case with the
quotations borrowed from others which I did not verify (the source is then given
in the square braces ``[after...]'' following the reference).  The pagination
follows then at the end of the reference as `Author, \btit{Title}
XI:1.5\ldots;p.21', where the numbers indicating part and subparts usually
involve only the main part (i.e., only `XI;p.21'), and may be totally absent, if
no such division of the work is given.  The edition is identified in the
Bibliography.  Occasionally, the subparts may have a letter, as e.g.,
`II:d7.q1.a2'. These are only auxiliary and their meaning depends on the source.
Typically, these are used with the medieval authors and the reference above
might be to the {\bf d}istinction 7, {\bf q}uestion 1, {\bf a}nswer 2, in the
second, II, volume/book.

In few cases I do not know the origin of the quotation, or else I only (believe
to) know its author. I chose to indicate such incomplete pieces of information,
rather than skipping them all together. I have likewise indicated the use of
unauthorized, or in any case unedited, versions of the texts found on the
interned for which no bibliographical data except for the title and the author
are given in the Bibliography. (For some, certainly very pragmatic reasons,
books printed in the USA do not carry explicitly the year of publication but only
the year of copyright. Consequently, the bibliographical information for such
books refers usually to this date.)

\sep
Words which are given some more specific, technical meaning are 
written with \co{slanted font}. \wo{Quotation marks} are used for 
words and quotations. \thi{Shudder-quotes} indicate, 
typically, either the referent of the word in the quotes, or else a 
concept or expression which is not given a technical meaning in the 
text but which is borrowed from somewhere else or even is only 
assumed to have some technical sense. Thus, for instance:
\begin{itemize}
\item 
\co{subject} -- is the subject in the technical sense introduced in
the text;
\item 
\thi{subject} -- is subject in some, possibly technical sense of
somebody else; it may often indicate a slight irony over only apparently
precise meaning one might believe the word \wo{subject} to have;
\item 
\wo{subject} -- refers to the word itself (quotations are also given
in the quotation marks);
\item 
subject -- this is just subject, with full ambiguity and with whatever meaning
the common usage might associate with it at the moment. 
\end{itemize}
I have tried to place more technical details in the footnotes which therefore
can be, for the most, skipped at first or casual reading. They are not, however,
addressed specifically to the scholars.\noo{We have not only no exegetic but
neither any scholarly ambitions.} Sometimes they elaborate the text but in
general will be useful only for those 
 who find some ideas interesting enough to follow them in other
authors.\noo{The footnotes contain, for the most, more details and references to the
possible starting points for such (re)search paths.}




%%%%%% END
\noo{%END
In fact, most if not all meta-discussions in modern philosophy,
arise as a consequence of elaborating the aspects discernible in (\ref{questB})
but not in (\ref{questA}): the idealism versus realism, subjectivism or
perspectivism versus
objectivism, correspondance versus verificationism or pragmatism, tradition
inspired by phenomenology versus analytical scientism, existential orientation
versus linguistic analyses,...

We do not dismiss all these discussions as completely irrelevant but our first
goal is not to be drown in the methodological and conceptual meta-perspectives. After
all, they too are, at least originally, motivated by the interest in an answer
to the first question.

An important aspect of the second question, not present in the first, concerns
the suggested need of justification. If I am to accept anything, it better be
sufficiently justified. This element has overshadowed philosophy if not since
its Platonic beginnings, so since Descartes. It has been used to distinguish
philosophy suggesting, in fact, that it is an existential difference which, in
case of a philosopher, makes him rely exclusively on \thi{reason} while, for
instance in case of a theologian on some \thi{faith}, while in case of an
average man on \thi{casual opinions of common-sense}.  This differentiation,
reflecting already preoccupation with the second question, has of course nothing
to do with the answer to the first question. But philosophers tend to make it
relevant by claiming that only their \thi{reason} can serve in answering the
first question. Various accussations of
\thi{irrationality} follow. Unfortunately, more often than not, justifications of 
claims to \thi{rationality} or even \thi{true rationality}, and in any case the
results of following them, amount to selecting only 
some truths which, as it happens, can be assessed by the \thi{reason}, or what
is actually meant by it.

Such differentiations are, as a matter of fact, political issues, issues of
delegation of competences, allocation of educational responsibilities, division of
faculties and departments or even, as it happened in
post-Cartesian Repubic of Unified Provinces, of national conflict between the
conservatists and liberal republicans.\ftnt{Cartesians supported liberation of
  philosophy from its subordination to theology, while ortodox Calvinists
  wished not only such a dependence but a state underlied religious
  goverment. As any issue with a theological element was almost bound in the
  Republic of that time to turn into a political one, so did this one.}
We do not dismiss all such discussions as completely irrelevant, but we think
that they truly belong to politics. Who is responsible for what? and Who is
entitled to what kind of questions? -- is it reasonable to believe that
answering such questions may help answering (\ref{questA})? 
The problem is that they did not
manage to produce any certain measure of truth which could be used in evaluating
the proposed answers to the first question. And, in fact, if they were to
produce such a measure, it could arise only as a consequence of answering, at some
point or another, the first question. Of course, I shall accept as true only
what is true. The attempts to elaborate on the second question do not bring us
any closer. The difference between the two can be seen by comparing them to the
respective questions below:
\equ{What is important?\label{questAA}}
and
\equ{How should I figure out what is important?\label{questBB}}
We can iterate the meta-appications past (\ref{questB}) -- \wo{On what basis
  shall I accept something as true?} -- just like past
(\ref{questBB}). The infinite regress is very much like the infinite
regress of formal reflection. And, in fact, just like the formal regress of \wo{I think
  that I think that...} adds nothing substantial to the first thought, and only
enmeshes it in a  formality of a childish mechanism, so the formal regress
of our meta-applications does not clarify anything with respect to the first
question, but only multiplies the possible issues, doubts and problems one may
explicitly rise and discuss.

One may easily object that, as a matter of fact, when the answer to
(\ref{questAA}) is unknown and hard to find, one could get some help from
answering first (\ref{questBB}). Formally, it may indeed seem so, but it is a
pure formality.  Notice that questions (\ref{questA}) and (\ref{questAA}) do not
concern any particulars which 
might happen to be special cases of more general laws, possibly useful to know
in treating the special cases. No, these questions have uncanny level of
generality which it seems futile to generalize further.  To such an objection we
can only answer: Sorry, disagree completely, for any answer to (\ref{questBB})
will, in fact if not in principle, 
require at least a partial answer to (\ref{questAA}). It may certainly help to ask
another, and related question, but approaching an issue from a different
perspective is not the same as approaching it from above. The only things that
may happen in the latter case are that one either falls down or flies away.


\subsubnonr{vs. arguments; Different tempers}

Yet, philosophy is only a particular expression of the fundamental
differences of human existence:... Simplifying to the extreme
opposition, the two tendencies are those of one vs. many, Plato vs. Aristotle,
Aquinas vs. Ockham, Husserl vs. James, unity vs. plurality. 

Different tempers...


\sep

What makes Anaximander the first philosopher is that he tried to argue for and
justify his claims, he used arguments and not mere statements. This is, at
least, the general view of ... philosophers. The \thi{love of wisdom} with
which one started, has long ago become the love of argument, to the extent that
wisdom without argument goes for simplicity if not stupidity.  One insists on
the arguments the more, the more one feels threatened, that is, the less one
feels certain of own self-identity.

Thus one has managed to separate itself from
literature with its usual lack of and occasional distaste for intellectualism.
It was a bit more difficult to cut off theology but here, too, the fact that
some aspects were assumed to be indisputable helped. Unfortunately, along with
that all the concern for the divine presence in the world was removed too.
Indeed, how would one argue about God, immortality and other invisibles? Their
invisibility amounts more or less exactly to the inadequacy of any arguments;
after one got tired of proofs of God's existence, the whole \thi{topic} became
highly inadeaquate, because incommpatible with the image of precise  logical
argumentation.

The strange thing that so happened was the emergence of the perfectly univocal
definition of valid arguments and mathematical precision in their study. Formal
logic, and its subfield of comutability and recursion theory, put the final
period after the millenia of developing Aristotle's syllogism, Leibniz's
calculus of reason and the ideas of mechnical reasoning. We now know that, in
itself, it yields absolutely no insights and, moreover, that no mechnical
procedure can ever be found for generating valid results...



\sep

The title is used by others....
I could have found another title but this does capture the
essential feature of existential situation which is the
only object of this book


\wo{Philosophical antropology} is but another way of saying
\wo{description of the existential situation}.


Ignore what is not experienced; what is really, objectively, in-itself is as
little relevant for this description as the answer to the question whether the
universe will continue expanding or starts at some point receding.

We study human experience and how its various aspects are constituted. This is
the opposite, in fact, antagonistic approach to the one which starts with some
ready constituted elements and tries to explain the construction, emergence and
functioning of a mechanism. One never knows if the given building blocsk are
really the eventual ones, nor even if they are appropriate for the task one
tries to perform. We leave such exercises to the empirical project. Whether it
succeedes or not won't change one fundamental thing: the human experience. We
know that it is Earth rotating around the Sun, and yet we can not stop talking
about -- and in fact, feeling and living as if it was -- the sun rising up at
the horison. This is how things look from our standing point and no amount of
empirical or logical proofs and arguments, no amount of bad or good science,
will ever be able to change that, unless, perhaps, one starts messing up with
the human beings themselves (and we may still hope that genetics won't go that
far).

We are not interested in mathematics or physics, in biology or sociology -- at
most, in how such forms of modelling the world emerge within the horison of our
experience; we are not interested in objective time or space -- at most, in how
such forms emerge and penetrate our experience; we are not interested in what is
objective and what is not -- only in how such distinctions may be relevant for
us; we are not interested in life -- only in the feeling of life, or better, in
living. 

This looks pretty bad, right? Pretty empiristic, subjectivisitc,
phenomenological, even idealistic, or perhaps just existentialistic! As a matter
of fact, it is not any of these, at least I hope it isn't.
But it is not any of their traditional opposites, either. It is only something
in the direction of 

\ad{Anthropology}
\citet{What is typical allows one to retain cold blood, only individually
  conceived matters cause nervous shock. In this consists the peacefulness of
  science.}{Faustus}{XLV\noo{p.622}}

The only object -- human being and his confrontation with the
world. We can easily imagine the world without human beings, but
for philosophical antropology such a world has no relevance.
As we will try to show, such a world is only apparently imaginable
-- in fact, it is totally unthinkable or else, to the degree it is,
it is totally indistinct.
(This has nothing to do with its objectivity or subjectivity, its
reality or ideality.)

Thus philosophy which tries to present the world as it is in-itself,
as it is sort of independently from human being, may be perhaps
interesting, sometimes even enlightening, but it will never provide 
a completely satisfactory account. It may reduce its sphere to some
particular area and problems and study that. But its concepts and
results are then founded outside its scope...


Yet, not explaining but describing = choosing what matters and
placing it in relation to other aspects which matter = giving account;
which leads to a unity of understanding ....

\wo{To give account}...
\begin{itemize}\MyLPar
\item
  First of all: try to determine \co{that} before even asking \thi{what}.
  And then ask thi{what} before even thinking \thi{how}. If sometimes it happens
  that one answers \thi{how}, then in any case never ask \thi{why}...
\item
  Only (some) necessary conditions, hardly ever any sufficient ones -- they
  do not obtain. We are not after any explanations; perhaps we are after
  something which possibly might be attempted explained later on, but that is
  not so either; we are after something for which explanation is inadequate
  category...
\item
  give only necessary conditions; the sufficient ones do not obtain any
  way. Does it mean that I believe in miracles? Well, as you like it. I have
  never seen a sufficient reason for almost anything, and I would say that those
  who insist on them must actually {\em believe} that they obtain. And as they
  do not obtain (ok, except in mathematics, or some simple physical analogies),
  saying that we can see necessary conditions but not sufficient ones 
  seems to me a matter of simply conforming to what is experienced, not believing
  in anything. There is much less believing here than in the belief in
  sufficient reasons. But sure, if you want to call the transition from the
  given necessary 
  conditions for some X to the X itself for \wo{a miracle} then I
  do not {\em believe} in miracles -- I see them all the time.
\end{itemize}
\wo{Irrationalism!!!} may exclaim some others, but let them remain calm, too.
Nobody knows -- and those who claim to know can not agree -- what rationalism
is. It seems that, at least, it requires (do we notice a necessary condition?)
not being carried away, some sobriety. Or in less prosaic terms, rationalism
requires that one assumes a position only with the admittance of the possible
limits of its validity, if not with the actual knowledge of such limits.  If one
accepts this rather generous critical rationalism then we should be able to
complain, all the way. Even to the point where one asks about the limits of
validity of this very position...  (After all, this admittance of partiality
squares so nicely with the etymology of \la{ratio}! A bit worse with the
pretentions to universality...)

\sep

It is not for all...

\sep

Many more specific issues are only touched upon and one may wonder why at
all. But we are not trying to resolve all the issues, to come up with a definite
and final answer. In most cases such answers simply do not exist and we prefer
to say too little rather than too much. Multiplying distinctions and
perspectives may be as rewarding academically as it is existentially futile. 


\subsection*{The main points ...}
{\small{ \begin{enumerate}
\item \co{There is} $\sim$ the One -- the particulars are only 
``perspectives'', modifications of the \co{Is} (incommensurable for 
frog and man) \\
 It is the fundament of all experience and \ldots the fundamental 
 experience. (``Objectivity'' cannot be reduced to any experience because 
 it comes before. ``Externality'' can.)
\item \co{Distinction} is the begining; \co{point} --  \co{pure 
distinction} -- is its \co{reflection}
\item There is nothing beyond experience.
\item The levels of \thi{being} and transcendence coincide with the time-spans
 \begin{itemize}
 \item \co{transcendence} is essentially something present which is
 not exhausted in this presence, something \wo{more than it is}, the overpowering
 \item
 this \co{more} can be relative to various aspects but, primarily, it is related to the
 \co{horizon of actuality}
 \end{itemize}
\item The ``fourth level'' is not a level ``above'' the other three but
 concerns the ``essential structure'' of division into \LL\ and \HH. (It is not
merely ``formal''.)
\item The world ``consists of'' \LL\ and \HH\ --
 Man is a borderline between \LL\ (lower) and \HH\ (higher).
 \begin{itemize}\MyLPar
 \item the \wo{totality} of his being is not \wo{constructed out of points} but 
 precedes the \co{distinctions}.
 \item Before you construct out of pieces, the pieces must be there. \\
  \begin{tabular}{l|l|l|l|l|l}
    \HH & (phon)emic & whole (Gestalt) & mind & concrete & quality \\ 
    \hline
    \LL & (phon)etic & part & body & precise & quantity 
  \end{tabular}
 \end{itemize}
\item The basic existential mode is determined by \G\ (openness, acceptance) or \B\ 
 (denial, refusal) of the mastery of \HH.
\item Important are \co{nexuses} [tie, connection?! -- vs. \la{religare}] 
of \co{aspects} which can be reflectively dissociated but which 
function meaningfully only in the \co{nexus}
\end{enumerate}

\subsubsection{{...and defs}}
\begin{enumerate}
\item \co{Experience} -- whatever leaves a mark vs. the source of unpredictability.
\item \co{Reality} -- what you cannot live without.
\item ``natural attitude'' to external world -- possibility of new, unexpected, 
      not-from-me.
\item \co{Sharing} -- the same \HH.
\item \co{relevant} -- telling what to do in the face of transcendence.
\end{enumerate}
}} % end \small

\section{Preface}


Well, perhaps, I was not quite honest.  I want to give an account of a
possible understanding of the unity of \ldots Yeah, of what?  Of a
human being, of a person but, perhaps eventually, only of myself. 

\citt{Gradually it has become clear to me what every great philosophy
so far has been: namely, the personal confession of its author and a
kind of involuntary and unconscious memoir [\ldots] In the philosopher
[\ldots] there is nothing impersonal;}{Nietzsche, {\em Beyond Good and
Evil}, I:6} 
\citt{It is almost incredible that men who are themselves working 
philosophers should pretend that any philosophy can be, or ever has 
been, constructed without the help of personal preference, belief, or 
divination.}{W. James, Essays in Pragmatism, I, p.24}
The very same words may be deeply meaningful to some and ridiculously
superficial, even stupid to others. 
This book is not meant to {\em convince} anybody about
anything; I hope it will be found interesting by a few, and it is
addressed to these few.
% You find pretty quickly, by just starting reading, if you are among them.

What is my method?  I do not have any. 
\wo{Philosophy does not so much tell {\em what} to think as {\em how} to 
think.} All kinds of methodological postulates have polluted philosophy 
since it started to consider itself, or rather started 
{\em to try} to 
consider itself a science in the modern sense of the word\ldots
Positivism, pragmatism, phenomenology, all \thi{methodologies} of 
philosophical inquiry tried first of all to ape 
science, to achieve the same level of precision\ldots

But \thi{how} is only, and only at its best, a pale reflection of
\thi{what}, a method is at best only a desiccated sediment of a
particular way of understanding, which is dictated, if not determined,
by the particular objects or sphere of experienced addressed. 
Scholastics knew it very well, but we do not like scholastics very
much nowadays, do we?  Sure, \thi{how}, having extracted some resdiual
sediments of \thi{what}, turns easily into a socio-political
phenomenon, a \thi{school} or a \thi{party}, which pretends to know
{\em the} axioms, and in any case at least {\em the} rules of the
game.  But the game goes on and the only rules are to be
understandable, and then to have something worth understanding. 

\thi{How} a person thinks is merely an expression of \thi{what} he
thinks -- uncovering, perhaps, some hidden or unconscious assumptions,
but still only assumptions about the \thi{what}. To dissociate the two is a
violence against concrete thinking. Useful and justifiable as it may 
be in a social context, determined always by the overwhelming 
majority of mediocricy, it is a violence against concrete thinking.

It is
\thi{what} and only \thi{what} that interests me \ldots The \thi{how} 
only follows the suit \ldots 





\ad{Philosophy searches for the Absolute Reality --} 
whatever this might 
mean. Since epistemology, this changed to the search for the Absolute 
Knowledge of Reality

A universal temptation underlies most of philosophy, the temptation 
to go after the absolute, the \he{undubitable} truth, to 
construct things, matters, the world once for all on a secure rock of 
\he{incontestable} reasons and matters of fact. The keyword of this 
temptation is ``security''. And neither the poor results nor even the 
schizofrenic split of the intellectual personality between `is' and `ought' 
are able to eradicate it. 

\begin{enumerate}
\item Certainty -- justification:
 \begin{enumerate}
  \item fear of the unexpected (reason vs. feeling)
  \item flirt with science
  \item[\isimp] unchangable
  \item[\isimp] necessary
 \end{enumerate}
\item Necessity (apriori) \impl irrelevant \\
what must be (irrelevant) vs. what is/can be
\end{enumerate}

\pa
The praise of arguments is but a side-effect of that. But what is an 
argument good for? Have you ever changed your mind concerning some 
fundamental matter because you have been exposed to an irrefutable 
argument? Have you ever accepted a view because somebody managed to 
produce a \thi{proof} of it? I do not think you have, but it is not 
only because
\citt{[q]uestions of ultimate ends are not amenable to direct
proof.}{Mill, Utilit.  ch.I} An argument need not be \thi{direct 
proof}, it may be just an argument, although it always tries to force 
its conclusion, to convince. It is, indeed, a great field open for 
invebntivness and shrewdness of intellect. But \ldots if I do not find 
the cnclusion plausible, then all the shrewdness of the argument does 
not help a least thing.

Philosophy can do better than produce arguments -- it can try to 
describe experiences or, perhaps, experience. If you find Kierkegaard 
worth reading, I doubt it is because of the excellency of his 
arguments. It is because you find something worth paying attention to, 
it is because you sense an attempt to communicate an experience which, 
being an experience of another human being, may turn out to be most 
relevant for yourself, too.

\pa We do have a lot of philosophy which occupies itself with
inventing new arguments for old truths and, on the way, with
rearranging the language in order to give the truths, as well as the
arguments, the apparent look of novelty.  Although the reasons for
such a search remain hidden in the obscurity of academic vanity,
it may be, perhaps, worthwhile. It is not my objective.

\ad{We do not believe in Absolute Knowledge,} 
final justification. 
Reason, in the broad and traditional sense, implies, and since Nietzsche 
means, control; control over 
chaos and anarchy which threaten our finite being. The 
anarchic element is just the opposite of reason, is the unreasonable, the 
uncontrollable. Again, in a broad and traditional sense, it has often been identified
with emotions and feelings.
The conflict between the two is thus not only a platitude of a second rate 
literature but also an analytic triviality -- it follows by definition. 
This looks like a great starting point! Does anything sell better than 
trivial platitudes? And the sale numbers are just reflection of the 
commonest, not to say the meanest, interest.

Reason, having gradually turned into 
rationalism, did not give up its absolute claims expressed in its
controlling function. But becoming {\em ratio}, it turned partial; 
after 
all, ``ratio'' refers to reason as much as to a relative value. Order needs 
proportions which bring divisions.
 Again by definition, reason loses the possibility of 
full control. But this only increases the tension -- the attempts to 
regain control become the more desperate.  

\begin{enumerate}\setcounter{enumi}{2}
\item Dissolution of subject (one of the guarants of infallibility)
 \begin{enumerate}
  \item society, culture, epistemic authority, selfish gene, discourse ...
 \end{enumerate}
\item Dissolution of concepts and distinctions \impl processes 
 (evolutionary epistemology)
 \begin{enumerate}
  \item empiricism-rationalism-idealism-pragmatism (only rough distinctions)
  \item Gestalt, holism (hermeneutics) ...
  \item empirical turn (Dennett's I, dynamic systems ...)
  \item sociological turn (soc. of knowledge: Kuhn, Rorty ...)
  \item analytic-synthetic (Quine), ...
 \end{enumerate}
But this only emphasises continuity between the old conceptual extremes.
\end{enumerate}

\pa
We do not believe in absolute knowledge because we have lost grasp on 
objects. The empirical turn of much of philosophy is but an attempt 
to re-confirm our intimate involvement into the \thi{objective world}, 
the unbroken relation with the field of our experience, which seems to 
disappear dissolved in the arbitrariness of discourse or whatever one 
wants to install in its place. 

Dissolution of objects\ldots great! It will be an important point.

\ad{No Absolute Knowledge \impl no Absolute}\label{pa:attitudes}
Because
epistemology got us used to identify reality with our knowledge thereof, it
is often hard to see if renouncing Absolute Knowledge we do not also renounce 
Absolute Reality. ``Wer spricht \"{u}ber siegen? \"{U}berleben ist alles.''

Inquiring into possibly ultimate dimensions of human existence is not popular.
Knowledge, even if not absolute, seems still a relevant question. But its relevance
and reality haunt in the background.
\begin{enumerate}
\item agoraphobic: ``Wovon man nicht sprechen kann, dar\"{u}ber muss man schweigen'' 
   (Witt. I = Witt. II) -- liguistic-analytic turn (despair, for intuition knows
that language \isnt reality)
\item ecstatic: catch the ineffable, use ``new'' language 'cos this is reality 
 (Heidegger, Derrida...)
\end{enumerate}
These two avoid distinction language \isnt reality, still under the spell of
epistemology; remove thing-in-itself, the transcendent
\begin{enumerate}
\item[3] Stay on the edge: moralism (Levinas, Rorty...); critical rationalism
\end{enumerate}
 
\subpa
It should be observed that the three
attitudes from \refp{pa:attitudes} are by no means specific to 
postmodernism. They 
are present in any encounter with the transcendence, in any encouter of a 
finite being with something that is greater than it. The character of the 
transcendence, that is, what is perceieved as transcendent and in what way, 
leads to various specific manifestations of these three basic modi: unreserved 
acceptance, claustrophobic refusal or sober confrontation.

They have thus always been present in philosophy. I am now about to commit
horrible simplifications and indefensible inadequacies. So I won't defense them nor 
claim that one 
couldn't arrange the following examples in a very different way. 
Each philosophy contains all the three moments of this tension. But each philosophy has
also its specific flavour which, apart from all the technicalities,
distinguishes it 
from others. I hope 
that you may recognize the flavours which made me classify each triple in this
way.

\begin{tabular}{rlllll}
tragic:  &  sophists & Hume &      nominalism        & Kant  & Husserl  \\
comic:   & Plato  &    Spinoza &   extreme realism   & Hegel & Bradley, Heidegger \\
pragmatic:& Aristot. &  Descartes & Abelards' realism  &     &  James, Scheler
\end{tabular}


\subsubsection{What is philosophy?}

It does not appeal merely to the intellect (lest it gets sterile)!


\ad{Reach reality} 
In general -- the temptation to reach reality; to be relevant; 

-- Cope with transcendence; {The greatest fear}
of unexpected, new, unpredictable...


\pa{Thing-in-itself,} objective world, etc. \co{There is}, yes, but what? 
I do know that \co{there is}, from the very beginning, from the virtual 
\co{signification} to the most advanced concepts, I know that there is 
something beyond them. But this something merely is ... At each level it 
may be the level below, or above, it, which is not accessible, like 
experience is not accessible to reflection, sensation is not accessible to 
concepts, etc. -- every layer has its level of transcendence. And at the 
bottom \co{there is} \co{nothingness}. 

Replace things-in-themselves by One, \co{there is}; and then something 
``objective out there'' by something inexhaustible

[murmur of being]

[the experience of objectivity (in what sense?) is not built up, 
construed, it is not reducible to the experience I had yesterday after 
lunch or may have tomorrow -- it is the fundament of experience and, by the 
same token, the most fundamental experience]

[it shouldn't be confused with the questions ``is this objective?'', ``is 
this objectively true?'', ``what is there objectively?'']


\pa
Reality degenerated into givenness\ldots

and philosophy thus degenerated into attempts to provide a universal 
description of thjis static givenness. Universal, which means 
primarily, incontestable, applicable to and recognizable by all. The 
descriptive bias made it hard to recognize that the basic, most 
important aspect of human reality is actually concerned with the 
choice, fundamental, spiritual choice; that facticity and possibility 
of this choice is what, in the deepest sense, constitutes human reality. 

-- ethics tries to take care of this but only in letter. Living as a 
rather poor relative, on the outskirts of ontology and metaphysics, 
it is heavily influenced by the general frame of mind. It degenerates 
into detailed analysis of possible acts and actions, particular 
choices and, sometimes, attempts to formulate general, abstract forms of 
moral imperatives, dissociated from the results of the ontological and 
epistemological investiagtions.

\ad{One --} Not science; not unity of sciences; not based on 
sciences; not concepts and their logic... but addressed to a whole human being.

Not ``how'' and ``why'' are the primary questions -- they belong to sicence -- 
but ``what''. (Eckhart's living ``without why''.)


\ad{Origin}
As One it isn't a science and shouldn't envy sciences. Historically, it was their 
origin.

-- Psychologism is detestable -- because psychology begun to take over some 
problems philosophers were occupied with. But it isn't if we are 
interested in human life rather than in being philosophers and the 
associated definition and  status.
(Unfortunately, psychology addresses concrete human condition and, negating
psychologism, philosophy tends to neglect it.)

-- History, 
A variant of the absolute is `unchangeable' -- no absolute = only history.
\\
History is irrelevant -- just one of the temptations to become more 
concrete and closer to the real world. I still read Heraclitus and Bible...
\\
Like in Kuhn's theory of quiet tides of science rising through periods of stability
until reaching the revolutionary height and turning in a new direction -- history, and with 
it philosophy (?), presents us with such a picture. A turning tide, a confrontation
with transcendence in a new form, brings chaos and disorder. 

-- Sociology

\ad{Rational}
One used to think of rationality in terms of justification, 
which then is something like argumentation. However, we do not believe in the 
ultimate justification any more, do we? 
There is at most a difference of degree (and often not even that) between
philosophy focusing on the quality of the arguments on the one hand and
rhetorics and sophisms -- disciplines, perhaps venerable, but hardly kept in high
respect nowadays. And to the general public, this is what philosophy often means
-- sophisticated and convincing arguments for the most ridiculus and unconvincing
theses.

The arguments function the better, the more petty matters are at stake -- 
``Shall we eat indian or mexican?''... The more important matters are concerned, 
the less imperative the arguments become, because then they are merely attempts
at justifying what we already are convinced about. 
Eventually, arguments never convince.
In the matters of real importance, 
argumentation is just a sign of lacking respect - one tries to convince by 
explaining to
another what he apparently is unable to understand. If he only could, if he only 
saw all the valid reasons which we see, he would accept our conclusion.


\thi{Big words}, or better \thi{high words}, on the other hand, 
do not force their meaning upon us. Their vagueness invites to most personal
interpretations and misunderstandings. Yet, they are not for this reason arbitray
and incomprehensible. 
They only hint at something not fully expressible, which we
are free to model and interpret -- they leave us freedom, exactly
because they do not have a unique, precise meaning.
As such, they are the opposite of arguments which, in honesty or
arrogance, always attempt to force the other to accept them.


\pa
The important thing is that what is being said is said as clearly and understandably
as possible. For such a purpose, arguments may, occasionally, have a value, just as
examples do. They are not, however, applied to force any conclusion, but merely to 
illustrate the connections between various aspects of the discourse -- in particular,
those which seem more acceptable and those which seem less so. 
But to identify them with the whole importance of rationalism is to turn it into
sophism.

Now, forgeting justification and keeping in mind the inherent partiality of reason, 
I would formulate the thesis of rationalism as follows
\thesis{\label{th:panrat}
A rational acceptance of a statement is the one accepting the limits 
of its applicability/validity.} 
%(Critisism concerns these limits.)
% Science is relative to a context...of possible falsification/criticism 
% which is exactly what limits the scope of scientific theories.
I may not know what these limits are but, at least, I am open for the 
possibility of their existence. And as far as I am able to, I try
to specify them. 

\pa
This thesis is not limited to the theory of science but 
can be taken as a quite general fundament of a philosophical project. For 
the first, even the generality and absolutism need not be dogmatic -- I am trying
to communicate some experiences, perhaps the ultimate ones, but I admit that
my formulations may be unfortunate, may be unclear, may be improved.
For the second, it is self-applicable: many statements, not only the statements of
the absolutistic philosophy, can be accepted without limits. Saying ``I 
love you'' one would like to give it an absolute value -- no temporal 
or contextual factors should limit its validity. This is perhaps the most 
irrational meaning one can give to this most irrational statement and I do 
not think that anybody who has ever made it this way would like it to 
have been made with all rational reservations. However, 
even the irrational statements and attitudes can be treated in a rational 
way according to \refp{th:panrat}. Sure, they lose then their magical air 
of actual existential commitments but philosophy is, at best, a reflection 
of life and never the life itself. It may invite to making some 
commitments or choosing a particular way, but it is never such a commitment 
or choice itself.

\pa
From the existential point of view, thesis \refp{th:panrat} implies something 
like the following attitude: I keep my convictions and comittments as long as
I do not encounter the contexts (situations, arguments, attitudes) 
which ivalidate them, which make tham impractical, 
unviable or unacceptable (in whatever sense). In such situations I do not have
to renounce them entirely -- typically, I will merely introduce more reservations
concerning their applicability. More importantly, it does not imply
that all such convictions and comittments are kept rationally -- the deepest, 
most significant aspects of my being and acting are, probably, those which 
have not as yet got the chance to be explicitly limited. Their source does not
lie in any critical examination but beyond. Still, the attitude makes me, at least
in principle, open for the possibility that they too may need such a limitation 
at some time. But I may nourish implicit and explicit convictions of various
degrees of constancy, some being easily disposed of, other being of fundamental
importance so that only the most extreme circumstances may shatter them.
\com{
Perhaps, knowledge in a stricter sense, involves a rational acceptance with the
explicit statement of such limits.}

\pa
{\em Explicit concepts} -- unlike literature or poetry. (lists of 
distinctions; perhaps, just enough words for the vague concepts)
and thesis~\ref{th:panrat}

-- don't multiply concepts and distinctions. 
They never match experience, while one thinks 
that, more and more disitinctions will bring one closer. Nothing more 
illusory!

\pa
Of course, all cannot be embraced with a scientific precision ...

-- {\em Proper abstractions}! ... create = ignore $A$, assign weight to $B$; 
economy of signs.


\ad{Is vs. Ought}
Ontology vs. axiology; the world as it is vs. as it should be. There is no 
doubt that we can speak about our wishes as to what or how the world should 
be. That we can speak about how it actually, really is, looks much more 
like a wished for assumption. And taking into account the 
rather poor results of the attempts at establishing any form of consensus 
in this matter, it even looks like a possibly wrong assumption. 
It pushes ethics on the side -- for who can doubt the primacy of the real 
reality over mere \thi{ought} -- while, at the same time, leaves theoreticians 
in an eternal despair over the lost connection with life. 
The separation of theory and practice, engendered by the conviction that they 
address the same reality, makes theory irrelevant and practice unreliable. 
It separates man from the world he lives in, that is, from himself. 

In the middle of the pretensions to know how the world is, perhpas even how 
it must be, and -- what 
is usually even more annoying -- moralising advice on how it should be, 
the crucial question asking {\em how it can be} appears highly
\he{reproachable}. Avoidings both other questions, it promises a mere 
\la{Weltanschaung}. But avoiding these respectable questions, it 
also avoids the schizofrenia they caused. And furthermore, it avoids the almost 
inevitable arrogance of the attempts to answer them and claims to having 
done so. And who would daresay that, on the final account, and in spite of 
all the claims to the contrary, any philosophy -- and any of its 
opponents -- offers anything more than 
a \la{Weltanshaung}, a statement ``look, you {\em can} see the world this 
way''? One may try to hide behind claims to objectivity and absolute truth 
but then, if not anybody else then eventually the history will fetch one 
from this hiding place.

\pa\label{pa:tomee}
It all may be wrong. It may be, however, only wrong for you. It may also be 
wrong for many others while it may seem right to you. It is so to me. 


\pa
This is, roughly, the history of Western philosophy, perhaps, its short 
version for the lazy students. This is also the theme of this book -- for 
those who, in their perplexity, find simply living the life somehow ...
unsatisfactory. I doubt that anybody can learn from it anything which he 
otherwise does not know. It is only, a sort of, giving an account. And giving an account is 
like paying a bill -- it brings the satisfaction of fulfilling an obligation. 
But since the smartness and skills of a self-made one have reached the
 sky on the stock market,
getting away without paying the bills is a reason to pride. Paying them is 
stupid. \citt{For the people of this world are more shrewd
in dealing with their own kind than are the people of the 
light.}{Luke, 16:8}

\pa
Philosophy attempts to derive something obvious from something acceptable. It
isn't an art of discovery but of appreciation.

\noindent\dotfill

The problem of the existence of the external world comes from
\begin{enumerate}\MyLPar
\item\label{sub-ob}
thinking of our being in terms of the idealized, timless subject-object
relationship, and 
\item\label{dedworld}
the attempts to construct or deduce the world from within such a horizon 
of pure actuality.
\end{enumerate}
It is just one of the problems arising from this perspective, totality of my being, 
and phenomenon of time being other well-known examples. If, on the other hand, 
I think of myself as being capable of continuous experience and take time as
a fundamental, rather than possible, notion, the existence of the world beyond
my experience ceases to be a problem and becomes a fact of experience.

\pa
Let me emphasise again, but for the last time, that I am not looking for
necessary, irrefutanle truths. there is no {\em necessity} of such an
admission. One may let the all-powerful \co{objectivistic attitude} prevail,
reducing the world to a never reachable totality of independent objects. But
I do not aim at a \co{re-construction} of a pre-reflective truth, which is
what it is and cannot be otherwise -- which is necessary. Such a truth may be
of interest to a scientist but, as I said before, it cannot satisfy
reflection.  Perhaps it is true, perhaps everything we experience can be
reduced to some basic, pre-reflective facts, even to some necessary physical
or chemical reactions. But then a naive question arises -- so what?  How can
this ``wisdom'' affect my life and actions? Is it at all relevant that
something, which cannot be otherwise, is as it is? 

Suppose that somebody produced a definite proof that it is so. And? Shall I
change anything in my life because of such a proof?  Shall I stop feeling the
way I do and choosing as I do? The only way such a proof might become
relevant would be to produce some controlling mechanisms.  Perhaps, having
traced everything to such a basic level will enable us to control everything
which is above -- and as some maintain, independent from -- it. Then it might
be a different story, but only to the extent, that our feelings and choices
could span over a different, perhaps wider range.  So far, the very
meagerness of such results makes the whole project as irrelevant for
philosphical reflection, as it is relevant for the \co{objectivistic
attitude} of science and controll. Let those who believe in it, work on it.

Necessary truth is what it is and {\em cannot} be otherwise -- do what you
want.  Thus, although it determines me, it does not affect my reflective
attitude -- it must be so, and there is nothing I can do about
it. Philosophical flirts with necessity do not interest me not because one
cannot produce cunning arguments in its favor but because they make the whole
endeavor existentially irrelevant. Firmness of existence arises from the
fragility of its background, not from the necessity of its assumptions.  Even
if at the bottom everything is involved in a play of deterministic forces,
this, certainly, is not the reality of our \co{experience}. In particular, there
is an element of freedom in reflection, if not in anything else, then at
least in the choice of its object. Reflection may admit that something has
relevance but that it cannot or does not want to focus on it and chooses to
focus on something else. Appeal to necessity of considering this rather than
that is a simple act of disrespect. The point is not to demonstrate necessary
truths but truths which are of interest, not to \co{re-construct} but, yes,
to construct.  What matters for me is not an inescapable result of necessary
processes, but the question  how and, above all, towards {\em what} direct
my reflective attitude.

\pa
What I see as the aim of philosophical enterprise is not to build a theory about 
how things really are, even less to build a theory which simply isn't inconsistent. 
Rather, I see it as some \thi{economy of language}, \thi{economy of words} which
should be spoken in as purposeful manner as possible.
I do not think that many
(if any) proponents of philosophical theories which we would classify 
as implausible ment really ridiculous things -- but their language sometimes makes 
us feel as if they did. ``Reincarnation'', ``immorality'', ``deeper Self'', 
``God'', ``external world'' etc. may all happen to refer to some things 
one wants to convey in a way unmatched by any other words. They may make 
speaking simpler, thus appearing most economical and
far better than ``circulation of matter'', ``immortal fame'', ``super Ego'', 
``opium for people'', ``transcendentally ideal world''. 
We have a tendency to believe that the latter express some concepts and (therefore?)
are more precise:
they reduce the former to something apparently more understandable, more
adequate for a reasonable discourse. But I am not sure that a discourse is
``reasonable'' when it excludes or simply distorts matters unclear and
never given as well defined objects, only because they
cannot be stated in unambiguous terms, preferably with the precision 
approaching that of mathematics.

\pa
The goal of reflection is not to re-construct the original, pre-reflective
truth. The origin of reflection -- experience -- harbours the possibility of
reflection and the material for it. But is made of a different
matter which cannot be fully and completely re-created with the categories of
reflection. The concrete material of pre-reflective experience can be, at
best, mimicked, symbolised but not re-constructed by reflection.

The only thing reflection can (and should) do is to construct its own
world. Its own, because reflection, like any other activity, can operate only
with its own categories. But this world should be constructed in such a way
that its original truth can live through and within it. One should strive
after the concepts which not only do not falsify or oppose the pre-reflective
feelings and understanding, but which actually open for their presence, allow
them to enter the world of reflection and unfold therein.

\ad{Novelty}
New insights -- rather inspiring new branches, 
history, sociology, psychology...

-- But otherwise, human-wise, \la{nihil novi}, 

-- [used in chap. III, Intro] \citt{\la{Mundus senescit}}{Gregory of Tours, {\em History of the 
Franks}, \~594AD} ``The world ages''

So WHY?

-- ``One can't construct a system which captures (the whole) reality.'' But, in
fact, more is true -- one cannot capture reality into words! Kierkegaard's
criticism of a system does not (really) address systematic analysis but the
intellectual arrogance believing that it does manage that...

-- A new philosophical book is but an expression of understanding, or 
misunderstanding, the old books. There are no new thoughts, no 
thoughts which were not thought before. The goal is not to think new 
thoughts, but to understand the old ones. 

Heidegger recommended return to Greeks, to Heraclitus, if possible, 
even farther back. So did Rousseau\ldots There is no kernel, no 
historical site -- it is everywhere, and nowhere\ldots

\subsection{We have \ldots, and shall\ldots}

\ad{Actuality}
We have heard the trumpeths!  
We have noticed the importance of distinguishing the beings from 
their Being, we have noticed the metaphysical preoccupation with
presence or, as I shall call it, with \co{actuality}. 
The barbarians have entered the gates of Rome and \ldots have found 
their place there. 
The Derridean fear of being accomodated, 
appropriated, embraced by the philosophical discourse was well 
founded because it was based on equally one-sided, as it was deep, 
understanding of this discourse.

After the initial fascination, seduced by vague promises of something 
new and different, even general public gets quickly tired of 
\wo{writing otherwise}, 
\wo{speaking otherwise}, all these \wo{otherwise} which, eventually, 
turn out to be nothing else than the other, ignored side of the discourse 
against which they tried to protest or, often, just a new way of 
saying the old things.
%%% `thinking otherwise' is in Book II. level of actuality

\pa
The barbarians have won, they upset the Rome, its order and its life. 
But there is a great danger in a great victory. 
And now comes the time of defeat, of settling down, of accomodating to 
the culture which lasted some thousands years not by a 
mere accident, not by a mere misunderstanding,
but because it was founded on something real. The barbarians 
have won because they had a good point to make or, in any case, because 
Rome did not have anything to counter them. But the noice of 
victorious entrance is but a passing news, and the horses on which 
one makes one's tryumph will soon be weary and have to be replaced. 

One thinks that conquering Rome everything has to be renewed, started 
anew; one lies down new foundations, challanges to new ways of 
building, living \ldots yes, thinking. And what? A short time passes 
and it turns out that people think as they used to, that people live 
as they used to, that the same emotions and passions steer their 
lives. The only things which, possibly (but even that not necessarily) 
have changed are the cloths. 

I do not believe in \wo{thinking otherwise} and, to a large extent, I 
abhore most of \wo{otherwise writing}. The barbarians overlook, in 
their tryumph, that the conquered traditions had developed the ways 
for accomodating all aspects of experience and that it did it, 
exactly, in constant confrontation with life. They were not 
neglected, they might only have been invisible at the places which 
attracted the eyes of newcommers. 

\pa
The power of the \wo{metaphysics of actuality}\footnote{I am 
reserving the word \wo{presence} for something else, so I have to 
translate this phrase in the above way.} needed undeniably a correction. 
But this can only be a correction of attitude. As long as one 
attempts \wo{to think}, one will do it in familiar categories. And it 
is hardly to be expected that, because of the fall of Rome, one will 
stop thinking. The difference may concern the significance one attaches to 
one's thinking and its results. And, not least, to the results of 
other's thiking. If everything is merely a question of interpretation 
then, certainly, most of the texts which one would like to subsume 
under the \wo{metaphysics of actuality} can, actually, be interpreted 
in different direction. They certainly are involved into thinking in 
terms of actuality but the point is -- what thinking is not? It 
remains to be seen what the hoped for \wo{thinking otherwise} will 
have to offer. The answer is: \la{nihil novi}.

\pa
But point taken. Reflection has to watch its steps, that is, it has to 
carefully observe what it is reflecting over. It has to always keep 
the clear distance to its object, in order not to confuse it with its 
reality, in order not to confuse itself with the omnipotent power of 
control. This is, exactly, the difference of attitude. And, for that, 
possibly a nonexistent one because, although some earlier thinkers might 
certainly be accused of nourishing such a belief, many could not. 
Just like the texts enscribe to an extent the conclusions, so does the 
choice of focus determine, first, the choice of texts and, then, the way 
of reading them. The attitude may be only assumed in the other -- it 
can be controlled, if at all, only in onself. Especially, when one 
reads. Discovering an attitude of \wo{fascination with the actual} in 
all the texts, the possible question is about the attitude of the 
reader. 
In short, it is not a question of \wo{thinking, speaking, writing 
otherwise} but rather of {\em reading} otherwise or, perhaps, reading 
other texts. \wo{Reading otherwise} would simply mean to look for what 
one cannot find, to look for the traces of thoughts which one thinks 
are not there. Eventually, one will find them, but this requires a 
different attitude of reading.
\thi{Metaphysics of 
actuality} is not all of the traditional metaphysics, in any case, 
not all of the traditional philosophy.


\pa
Critique of \thi{metaphysics of actuality} (Derrida, Heidegger) wants 
to show something more, \co{non-actual}. Critique of objectivistic 
system of thought of Hegel by Kierkegaard opposes to it the experience 
of an individual, not just a common experience, but the fundamental 
experiences, elaborated through all the depth psychology. 

My point: the two are the same! The feeling one may get of 
existential relevance of Derrida, and certainly of Heidegger, is 
based exactly on that the personal/individual/etc. comes into the 
world of \co{actualities} through the \co{non-actual}. 


\pa
Follow Heidegger's, Derrida's critique of \thi{the metaphysics of 
actuality}\footnote{I will reserve the word \wo{presence} for 
something different.} but not his aggressiveness. There is nothing 
wrong with actuality, it is our way of conceiving the world. There is 
only something wrong with positing it as the only way of treating 
everything, all aspects of experience\ldots 

We are not scared of concepts as Heidegger, and later post-modernists, became.
We consider this fear unnecessary, in fact, impossible in length, sometimes
comical, sometimes hurtful...

\ad{\thi{The linguisting turn}\ldots} 
\ldots or shall we rather say, \wo{a linguistic twist}?  
One has always known that words and language
can not capture reality.  Not only there is no final, most adequate
and true name of God.  Even trying to communnicate most simple facts
and situations, when using words, we are forced to distribute the
responsibility between the one who is speaking -- to express himself
as clearly and understandably as possible -- and the one who is
hearing -- assuming that he not only knows the language, but also has
the sufficient background of experience to grasp the meaning of what
one is saying.  This sounds, perhaps, a bit exclusive, even elitistic,
and so it is supposed to sound.  \citt{Who hath ears to hear, let him
hear.}{Mat.  XIII:9} Success measured exclusively by publicity, by the
sheer number of receipents is always -- {\em always}!  -- inversely
proporitional to the value of the thing.  In the domain of spirit,
there governs a law opposite to that of mass culture, namely, the
more, the worse.  Even deep things, to the extend they are spread
around and received by many, become superficial, flat and low.

\wo{The sky was blue and the sun was shining.} is as
inadequate a phrase, if we try to measure it by some some objective
conformance to reality, as any name has ever been for expressing the
essence of God. When we speak, and even more when we write, we assume 
that the reader fulfills some basic conditions necessary for 
receiving the message. If, on the other hand, we do not know to whom 
we are speaking, if we, perhaps, assume that we are speaking to 
everybody, irrespectively of his knowledge, background, level of 
intelligence, and what-not, if we, for instance, are merely interested 
in selling the book to indefinitely large and, consequently, also 
indefinitely amorphous masses, we can not avoid getting confused. In 
particular, if we forget that any utterance, and also any text, is an 
address from somebody to somebody, if we abstract the text from this 
fundamental relation of intension and meaning, we can not avoid 
getting confused. 

\pa
One feels proud having recently overcome all these difficulties by
the \thi{discovery} that, as a matter of fact, there is no reality 
beyond the words, there are no human subjects beyond the text. As any 
thesis, so also this one can find innumerable reasons and 
justifications. I certainly won't spend time on reviewing them all. 
This inhuman abstraction of \thi{textuality}, this invention of 
intellectual despair over the impossibility to ensnare reality into 
words, does not deserve much attention. Yet, its wide 
popularity makes it hard to simply ignore it. The barbarians entered 
Rome and settled down, but their ways do have reprecussions\ldots

\pa
Reasoning, and understanding in general, is a way of using, inventing 
and arranging \co{signs}. Words are not the only possible \co{signs}, 
but let us be charitable and take all this \thi{linguisticity} in a 
very general, probably over-general, sense of semiotics. Without an 
adequate system of \co{signs}, there is no understanding. Ok, but 
there is no such thing as \thi{understanding}; there is only and 
always only understanding {\em of something}. What this something is, 
is highly vague, and the \thi{linguistic turn} is an attemt to 
substitute for our lacking, or in any case \co{vague}, understanding 
of what this is that we are trying to understand, something which 
apparently is much more precise, well-defined and hence manageable -- 
the very system of \co{signs} itself. Cetainly, a smart intellectual 
twist. However, the fact that something is \co{vague}, that we can 
not put the finger on it, does not mean that it does not exist. 

\pa Many \thi{linguistically twisted} will agree.  There exists
something but our way of speaking, the language and words we are used
to, are inadequate to capture it.  Therefore, we must \thi{speak
otherwise}, \thi{write otherwise}, invent a new language.  The ghost
of capturing the reality into words is peeping in through all the
holes.  What else, after all, can an intellectual do than to market
his words?  
%Among the most prominent (but by no means the only) examplars
%of such \thi{writing otherwise} one can probably mention late
%Heidegger and Derrida, perhaps, but only perhaps, late Wittgenstein. 
%These are, as usual, accompanied by a tremendous herd of noisy epigons
%who \thi{have seen the light} and start \thi{bubbling otherwise}.

%\pa 
If the reality which one senses beyond the inadequate words is
really a fundamental aspect of human existence, then is it reasonable
to assume that we had to wait thousands of years, until the day today,
to discover the need for \thi{speaking otherwise}?  Sure, one can
always patronize the past generations pointing out their mistakes. 
But such patronizing is but an aspect of a positivistic faith in the
absolute character of progress which we are not willing to take
without any reservations any more, if we at all are willing to accept
it at except, perhaps, for the sphere of technology and civilisation. 
Perhaps, the men of Reneissance, or of Enlightment invented
particularly obscure ways of speaking about some aspects of human
existence, ways which, due to their low value, became particularly
popular.  But there were wise men, too, in those times, weren't there? 
OK, they were all led astray.  What about the men of Middle Ages?  Of
early Middle Ages?  Heidegger would say, and what about the Greeks? 
Indeed, what about them?  But we do not have to follow the
idiosyncratic, not to say comical, attempts to revitalize a dead
language.  What about Indians, Jews, Chineese?  I do not want to boast
with false modesty, but I find it hard to belive that to address the
most fundamental issues we need an entirely new way of \thi{speaking
otherwise}.

\pa If all history of human culture, in any case, of written human
culture, has been perverted by the inadequate language, then what
makes us, some of us, today read Plato, Aristotles, Lao Tse, Bible,
St.  Augustine?  For my part, I can say for sure that it is not a need
to find their mistakes but, on the contrary, the conviction, and I
should be able to say -- the experience, that they do have something
important to tell me.  I doubt that many others read them only for the
sake of intellectual curiosity, deconstructivistic exercise or
academic career.

\pa
OK, so old texts may, sometimes, contain some valuable insights. Or, 
shall we only say, some valuable words? Do these texts contain 
anything \thi{in-themselves}, anything which is not read into them by 
the reader?

If text is entirely open to the arbitrariness of the reader and the
indeterminacy of possible interpretations, if it has no intension, no
truth it tries to communicate, if any system of signs is but an 
arbitrary invention with but an arbitrary relation to what possibly 
might be lying outside it, then one might almost agree that
writing should not involve any attempt to make it understandable and
accessible -- {\em what}, in such a case, should one try to make 
understandable?  Probably, one does even better resisting any such
attempts and we have been exposed to many examples of such a 
resistance. 

Unfortunately, or rather fortunately, all texts have been written by 
humans with some intensions. They may communicate these intensions in 
better or worse ways, in more or less adequate form, in more or less 
understandable and plain manner, but intensions are there. And if one 
does not like the word \wo{intensions}, in particular, any 
\wo{intensions of the author}, then it will still suffice that, having been 
written by humans, they contain human expressions of human experience 
and life. This, at least, is more than sufficient for me and if it 
isn't for you, then you are free to keep dissolving words, expressions 
and, eventually, insights in the mud of \thi{intertextuality}.



No private language -- so, no private thinking?

language -- changing, ok, but still not changing and translatable

New problems? New solutions? What? -- check it out, it has all been 
there\ldots

\pa A completely different, even opposite, dimension of the
\thi{linguisitic twist} is analytical philosophy.  A name which seems
more adequate to me would be something like \wo{logical analysis of
linguistic behaviour} because, as a matter of fact, I have found
extremely little and also extremely meagre philosophy in this camp. 
This, however, is due only to my biased and misunderstood conception
of philosophy, so let us not quarrel about the names.

The strength and, indeed, the visible vital force of this camp is
based on the fact that it tries to focus on more or less identifiable
and well-defined problems.  The resulting insights and proposals have
often much intrinsic value and I am last to deny that.  I am a bit
more unsure what this value is and to what purposes it could be used,
except for possible applications in cognitive science, artificial
intelligence, linguistics and, perhaps, legal theory and new design of
dictionaries.  Even if only of pragmatic -- shall we say, scientific? 
-- relevance, so, being usable, they also represent some kind of
value.  I want to say this with all due emphasis because, having said
that, I think we can leave this camp for itself.  Hopefully, with
time, it will, as any scientific community earlier in history,
establish its identity independently from philosophy and then continue
resolving all its specific problems and particular issues with the
undisturbed peace of mind to the best of whoever will want to use
them.
%
\sep 
%
\pa This is as much I have to say about all the linguistic twisting
and I leave writing of the voluminous binds of history and analyses of
the involved \thi{problems}, \thi{questions} and -- yes!  --
\thi{solutions} to the competent scholars.

\ad{Genealogy}
Not in time, but from virtual to concrete $\sim$ gradual 
differentiation

Then: refinement of systems of things/concepts; these consist of 
simultaneous $\sim$ equiprimordial \ldots correlates/aspects (find a good 
word)


\ad{Notation}
\begin{itemize}
\item ``\herenow'' \impl ``here and now'': dissociation of aspects
\item ``\co{recognition}'' \impl ``\co{re-cognition}'': rozpad into 
more detailed parts.
\end{itemize}
ntuitions should be preserved -- for more detailed understantding, one 
should first keep in mind the possible presence of another notational 
variant, and then check it \ldots




} % end the main \noo{%END

But if such a useless nothing, why \la{summa}? 
What could possibly make one write a \la{summa}, of any kind, nowadays?
Complexity -- not only of the incomprehensible totality of the world, but even
of every single issue -- makes it look pretentious. Things fall apart
and observing the dissolution or, as many a one pretends, praising it as the
openness unto plurality, is the only reasonable attitude, in fact, the only
politically correct way of distancing oneself from the trauma and danger of monism,
that is, totalitarianism.

Yet, philosophy which does not try to capture any unity ends up as a catalogue
of particular cases, particular concepts or just words, which may be as
elaborate, intelligent and intricate as it is uninteresting and useless.
Philosophy which distances itself from any attempt at reaching some wisdom, and
that means in particular, thinking and relating to the ultimate questions, 
ends up calling its impotency for modesty and glorifying the incessant
questioning, that is, public scratching one's head with the emanating
self-assurance that all genuine issues should be left to those who are
unintelligent enough to expect any answers. After all, it takes an analphabet to
believe that truth is written some place.
% But receiving guidance does not exclude autonomy and it appears to us quite
% unfortunate that
%

Philosophy attempts to think what one knows. For to know means much more than
what Plato explains in \btit{Timaeus}, or what others have tried (and are still
trying) to specify with
respect to explicit, reflective thoughts, \gre{episteme}. Sure, I know the
pythagorean theorem and the way from my house to my work. But I also know that
I will die, I know my love of my mother, I know how she loves me, I know what my
best friend likes and of what my girl-friend is afraid. I know that I am the
same person I was yesterday and I know that, from behind the invisible forces,
the hidden eye of God's  watches everything. Most of these things I am not
actually able to think, not to mention, to express precisely and explain with
any degree of stringency.
%
We know what philosophy is, don't we? We know what is {\em not} philosophy and
we know what is, at least, when we encounter it. We can even tell good
philosophy from bad one. And yet, can anybody tell what we thus know? Can
anybody think this knowledge clearly enough to express it precisely to
everybody's, or even only to one's own, satisfaction? 
%

Knowing, we might perhaps say, is what allows me to
consent to some thoughts and to reject others. For when I content to the truth of
a thought, I do it in the light of something I know. Only in very special cases
this knowledge consists of other, explicit thoughts.
%
To think what one knows is a challenge, just like it is 
a challenge to become what one is. Only to a self-satisfied rationalism such challenges
may appear as paradoxes, and only to a narrow-minded pedantry as contradictions.

The attempt to think what one knows need not (as it seems, can not any longer)
rely on necessary principles and forcing arguments. Those who try
\citet{withholding their consent from any proposition that has not been
  proved}{CiceroGods}{ I:1. [In the translation of F.~Brooks: \wo{refusing to
    make positive assertions upon uncertain data}.]} end up with absolute 
certainty -- about nothing. The rigidity of irrefutable argumentation which
attempts to force its conclusions and tries to dispense with everything which
escapes such attempts seems, indeed, to empty every phenomenon of its
concreteness yielding only residual site whose necessity equals its hollowness.
But the lack of the universally forcing arguments is not the same as the lack of
any truth, the lack of the necessary laws is not the same as the lack of any
order. Besides (or instead of) the effcient causes there may be many necessary
ones, besides (or instead of) the sufficient reasons there may be many
insufficient -- but, as Weber would say, favourable and supportive -- ones. Just
like explanations try to capture the former, so a more modest account may try to
identify only some of the latter.

\citeti{I say the following about the Whole ... Man is that which we all
  know.}{Democritus}{ DK 68B165}
Philosophical anthropology tries to give an account of human life. This does not
mean isolating this specific \thi{object} and ignoring all the rest. It amounts
only to seeing all the rest in a specific perspective: not as a set of
inviolable mechanisms but as the field of unfolding of human existence.  The
dualism of deteminism confronting freedon, of \thi{objectivity} confronting
\thi{subjectivity} is only a 
result of a particular way of thinking  the existential
confrontation which precedes the understanding -- and dissociation -- of these
two (and other) aspects.  The two, posited as the ultimate poles, can never not
only agree but even meet and one is constantly forced to the impossible choice
of the one or the other. Above and across this impossibility one may try to draw
the border between the two in ever new ways.  For even if the number of the
elements, say, of the constitutive dimensions of existence, is limited, the concrete
borders between them may be drawn in an unlimited variety of ways. Every human
being draws these borders anew and the abstractness of the fact that somebody
else did it earlier in a similar way does not in the slightest diminish the
concrete need to do it ever anew.\ftnt{Besides, \thi{limited} can be, in all practical
  respects, as good (or bad) as infinite. On how many different points can any
  two philosophical systems differ?  Ridiculous question, but to play the game
  only for a while, let us say some number which certainly is lower than it is
  in fact, say: 100. Simplifying further, let us say that on each of these
  points one has only a binary choice, yes or no, + or --. How many different
  \thi{systems} do we get? Well, $2^{100}$.\noo{This is, if we were to believe
    such calculations, approximately the square (perhaps cubic) root of the
    estimated number of atoms in the universe.} If humankind produced one such
  system every second, it would take some $2^{75}$ or, rounding off, $10^{20}$
  years to merely produce them all. This is a bit more than twice the estimated
  age of the universe.\noo{10^9 And merely producing them is not even the
  beginning of anything.} Thus, even if the 
  number (of possible forms of existence, of possible philosophical systems)
  were finite, the simple combinatorics is on the side of 
  unrepeatability -- not principal, however, not in some ideal infinite limit,
  where indeed the finite number of possibilities would have to be reapeated infinitely
  many times, but only in practice, that is, in fact. (If one wanted now to
  reduce the number 100 to, say, 20 such points,\noo{ at 
  which two systems might differ} it still leaves over 1.000.000
  possibilities.)} 
%
Everything has been said before -- {\la{mundus senescit}} (\wo{the world has
  aged}) noticed St.~Gregory of Tours in the VI-th century and some 1500 years
before him a preacher observed \citeti{[t]he thing that hath been, it is that
  which shall be; and that which is done is that which shall be done: and there
  is no new thing under the sun.}{Eccl.}{I:9} This may sound a bit depressing
but only to an intellectual capable of dealing exclusively with abstractions, or to
the petty mentality occupied with even pettier novelties, one \citeti{consoled
  by a mere trifle, as it is distressed by a mere trifle,}{after
  \citeauthor*{Pensees}}{II:{136}} for which everything beyond the yesterday's
scandals and today's news is a boring repetition. But \citeti{what is right can
  well be uttered even twice.}{Empedocles}{ DK
  31B25\noo{\citaft{FirstPhil}{: 58, p.165}}} Moreover, to understand something
  said before one often must say it 
oneself. Every existence says anew something said before, but saying this
concretely amounts to drawing anew the distinctions in the matter of life, the
distinctions between abstractions like \thi{love} and \thi{indifference},
\thi{indifference} and \thi{impotence}, \thi{impotence} and \thi{thirst},
\thi{thirst} and \thi{lack}, \thi{lack} and \thi{illusion}, \thi{illusion} and
\thi{lie}, \thi{lie} and \thi{truth}\ldots The infinite concreteness lies in
such distinctions and the lack of final definitions is only another side of this
concreteness. And the lack of rigid definitions does not mean the lack of
significant distinctions.

Giving an account of -- rather than explaining -- human existence, we are not
very concerned with many accepted distinctions. We certainly have to keep
various points of the discourse at approximately the same level of abstraction
but this need not prevent us from addressing also issues which the
administration of academic life places at different
departments. \citet{Philosophy is first of all a science about human being,
  about integral human being and of integral human
  being.}{Bier}{I:2\kilde{p.21}} 
% The advantage of philosophical anthropology is that no posited
% objectivities-in-themselves need to disturb us.
Asking a question, we will often rest satisfied with relating it only to
the place it occupies in the field of existence. Whether the question happens to
sound (or traditionally even {\em is}) theological, psychological, mythological or
astrological is of no importance, as long as it addresses an aspect of this
integrity. \wo{\la{Summa}} from the title refers to such a summary, to the 
attempt to 
gather quite different, sometimes even disparate, aspects into one unity and not
to the ultimate summit, 
to the vanity of collecting all relevant (and irrelevant) details in a systematic
and scientific totality.

%\tsep{hee}

All relevant (and irrelevant) details can be abstractly thought as a graph with
various edges (of relevance, dependence, association, etc.) connecting various points.
Traversal of a graph can be, in general, perfomed in two different ways: depth
first ({\sc df}) or breadth first ({\sc bf}). {\sc df} starts in the actual node
and follows one path -- an edge to a neighbour node, then to some neighbour of
the first neighbour, and so on. (Encountering a previously visited node, it
backtracks and tries another path.) {\sc bf}, on the other hand, vists first all the
neighbours of the actual node, before proceeding to all the neighbours of all
first neighbours, and so on level by level. 
A scholar is a {\sc df}. We do not have equally general, and certainly no
equally respectable name for the {\sc bf}, but it could be associated with a
dilettante -- knowing a little bit about everything which concerns him in one
way or another.

Of course, the world is not a graph. If we insisted on the analogy, we would have to
extend the idea at least by suggesting that the graph is unlimited, if
not actually infinite, and that in at least twofold way: every node has
infinitely many immediate neighbours and, from every node, there is an infinite
number of infinite paths (which never enter a cycle). A scholar diggs thus
further and furhter away from his home, trying to get to the end of an infinite
path, and hoping that it will bring him back home. A dilettante, on the
contrary, circles around, always in a safe distance
trying only to cover the infinite circle 
surrounding the house. Neither ever completes the road, both seem \citeti{to join
together diverse peaks of thought,\lin And not complete one road that has no
turn.}{Empedocles}{DK 31B24 [translation after \citeauthor*{Emped} 24.]} Scholar
ends up knowing everything about nothing, a 
dilettante knowing nothing about everything.

But this is unfair to the scholars! How can one compare them, put them on equal
lines with dilettantes?! Probably, one should not, but \wo{dilettante} is only a
name, the better of which seems hard to find. So let us ask the scholars...

Does the world -- eternally returning in the cycles oscillating between Love and
Strife -- exist twice (on the way from Love to Strife {\em and} on the way
back), or only once? And if once, then why, when and how?
\kilde{Szczerba,p.63-65}The diverging opinions may be the
consequence of the lacking sources which might have possibly contained the
answers of Empedocles himself. But, as a matter of fact, the question might have
never been asked and the answer never intended. Perhaps, all that was meant, was
to point to \wo{The world-wide warfare of the eternal Two}, the Love pushing to
unity and Strife to separation?  Perhaps, all the cosmogonies and cosmologies,
reflecting only the human understanding which, eventually, is always only
understanding of oneself, are but images never meant to be studied in and for
themselves. But sure, the questions can be asked, and so constructions can start
spinning...

What is Plotinus rejecting in V:5.1, claiming that the intelligibles, perceived by the
intellect, are not \thi{propositions}, \thi{axiomata} or \thi{sayables}? Does he
claim that the intellect's \thi{knowledge} is non-propositional or only
non-formalisable, non-expressible or only non-representational? Indeed, one may 
ask and keep answering, and many distinctions can arise from such diputes. But,
the question is, what shall we do with all these distinctions? Is it reasonable
to assume that, although the text does not say anything clearer, the intended
ideas were nevertheless so much more precise? Or, perhaps, they were not but
they have become so in the course of history?
Certainly, scholars shall sort out what Plotinus actually said
and meant and what he did not -- but preferably only as far as it goes. The fact
that questions can be asked with respect to a text, does not mean that the text
(and related texts, and texts related to texts related to...) contains the
answer, nor even that the answer can be at all 
meaningfully given.\kilde{EmilssonIntellect,p.29-30} We are rather clearly
\citet{told that the Intellectual-Principle and
the Intellectual Objects are linked in a standing unity}{Plotinus}{ V:5.1} and
even if \wo{we demand the description of this unity}, we should not dismiss the
possibility that what we have just been told is all we can actually get to know,
yeah, all that is worth knowing! Of course, 
to admit that, one would first have to learn cherishing and being satisfied with
perhaps clear and understandable but still vague expressions; expressions which
admit their limitation and put some trust in the reader who, hopefully, is the
same kind of being as the writer. All this, however, might amount to ceasing being a
scholar and turning~... a dilettante? 


There may be, and almost always there are, many meanings and one seldom can be
\co{precise} enough to narrow the expression, not to mention the thought, to
only one of them.  In this sense, interpretation is usually over-interpretation
and thus mis-interpretation. It is not so that we understand when we have made
the possible meaning most possibly, that is, impossibly \co{precise}. Dissecting
it into all too specific and detailed alternatives brings perhaps everything
under our control but only by making it impotent, by denying it life which our
thinking can only address, but never imitate... One may, perhaps, do it
sometimes legitimately in the name of scholarship, but positing it as the
universal aim of philosophy, and even thinking, sounds neither convincing nor
even plausible to many ears. Precising things beyond their limits results in
dissociating them and what is once dissociated can not be put together unless
one rediscovers the unity which preceded all the dissociated precisions. This
unity is not any combination, any coherence, any consistency of the elements
because it does not as yet know of any elements. \noo{This smells vagueness, if
  not directly misticism, so it may be a good point to begin.}

\tsep{???}


\tsep{}

One of such questions, perhaps the only question of all philosophy, is:
%
%\equ{What is true?\label{questA}}
\begin{center}What is true?\end{center}
%
Some try to answer it in the way it is asked, that is, in
dissociation from any person asking it. Such an answer amounts always to
specifying what one, everybody, you and I {\em should} accept as true.
The \thi{should} is usually surrounded by the arguments and proofs which should
convince everybody. 
Unfortunately, any respect expressed by imputations of rationality, accusations of
irrationality, expectations of a direct and infallible communication is, 
 at best, merely the respect for some \thi{rationality}. As far as 
a personal meeting with the reader, that is, as far as the person of the reader
is concerned, it shows disrespect equal only to that displayed by the strangely popular
reluctance to express one's meanings in an understandable form. The difference
of proceeding results only in that sometimes the latter, but never the former,
can be excused on the assumption of incapacity. 

The best, in fact, the only thing one can do is to answer this (as any other)
question for oneself or, what amounts to the same, to accept some known answer.
The answer may be communicated to others but only as {\em my} answer -- I may be
convinced that you should accept this answer, too, but I do not even try to
convince you about that. What you do with my answer is your sake, in fact, to
understand it you have most probably already known it, even if you did not think
it yourself in the same way. 

\citet{Gradually it has become clear to me what every great philosophy
so far has been: namely, the personal confession of its author and a
kind of involuntary and unconscious memoir [\ldots] In the philosopher
[\ldots] there is nothing impersonal.}{BeyondGE}{ I:6} Moreover, the
\citet{beliefs to which we most strongly adhere are those of which we should
  find it most diffcult to give an account.}{BergTime}{ II p.135}
% \citt{It is almost incredible that men who are themselves working 
% philosophers should pretend that any philosophy can be, or ever has 
% been, constructed without the help of personal preference, belief, or 
% divination.}{W. James, Essays in Pragmatism, I, p.24}
Thus, since this, like every other question, is asked -- and answered -- only by a
concrete person, we can take it to be the same as:
%
%\equ{What shall I accept as true?\label{questB}}
\begin{center}What shall I accept as true?\end{center}
%
Here one might object: one senses a difference, the element of subjectivity, or
perhaps relativity, in the latter which does not disturb the former. But the
difference is only apparent: every answer to any of these questions, can be used
also to answer the other. If something is true then, sure, I shall accept it as
such.  On the other hand, if I accept something as true it is because it is --
as far as I can see, imagine, feel, understand, speculate -- true.  Unlike the
former, this later implication has been judged problematic, mainly because one
imagines some voluntary act of acceptance which one's subjectivity could decide
to perform as it wished. The meaning of this implication will be our main theme
and its validity will rest on the fundamental difference between relativity and
subjectivity.

{Relativity of every observation and conclusion to the
  subject making this observation and drawing this conclusion seems to be the
  most obvious thing in the world. However, one can draw quite the contrary
  conclusions from this observation. On the one hand, one can conclude that this
  relativity for ever prevents us from gaining an insight into the true nature
  of the world, into the true nature of things as they are \thi{in themselves}.
  The underlying assumption is that things indeed are something specific \thi{in
    themselves}, and even that they, at least in principle, can be described
  \thi{in themselves}, as if independently from the view of the one making the
  description. The contradictory nature of such a project seems hard to accept.
  So, on the other hand, one can draw the conclusion that there are no things
  \thi{in themselves} and, consequently, no measure of truth -- in short, that
  relativity amounts to complete relativism and truth to mere (perhaps even
  merely volitional) subjectivity. But does the fact that the specificity of
  things \thi{in themselves} is only ideal and posited, show the ultimate void?
  Does relativity of particular truths exclude absolute truth? 
Both these interpretations start from the assumed duality. The
intuition that the standard of truth carries an element of transcendence is
opposed to the observation that absolute transcendence, dissociated and remote
from any immanent view, is inaccessible and hence, at least in the latter case,
irrelevant. We will instead take relativity as the basic phenomenon which
  precedes and underlies various dualities like subject vs. object,
  immanence vs. transcendence, acceptance as X vs. being an X, etc.}

\ins{ [Reality -- is between] "Now, the quantum postulate implies that any observation of atomic phenomena will involve an interaction with the agency of observation not to be neglected. An independent reality, in the ordinary physical sense, can neither be ascribed to the phenomena nor to the agencies of observation..."

Bohr, N. (1934). Atomic Theory and the Description of Nature. Cambridge, Cambridge University Press (p. 53).

 N. Bohr, The Philosophical Writings of Niels Bohr, p. 54, quoted in
 T. J. McFarlane, Quantum Mechanics and Reality, 1995 (quotes also: 

All things -- from Brahma the creator down to a single blade of grass --
are. . .simply appearances and not real. 
[Shankara, Crest Jewel of Discrimination, tr. Chris Isherwood, p. 97])

apparently well-menaing and also well-thinking people associate the two...
}

Taken contra-positively, the implication from my acceptance of X as true to X's
truth says that I should not accept any untruth.  Emphasizing this direction
amounts thus to a more modest project: not necessarily finding {\em the} truth
but rather, or at least, simply avoiding untruth. In the able hands of a
quibbler this modesty might reduce to a mere scepticism but we will hopefully
manage to say also something meaningful.  We should only keep in mind that we
are not following the fashion of {\em reducing} the first question to the second
(which results in the interminable quest for \thi{subjective} criteria of
\thi{objective truth}). We claim simply that the two questions are genuinely
equivalent, that everything \citet{that is known, is comprehended not according
  to its own force, but rather according to the nature of those who know
  it.}{Console}{ V:4.25\noo{154}} \noo{\citet{everything which is cognizable, is
    cognized not according to its own power, but according to the powers of the
    cognizing.}{Console}{ V:4.25 \noo{Filozofia Sredniowiecza,p.106}} }

Giving only an account -- rather than explaining -- we will ignore many
distinctions. We will not, for instance, treat the above question as {\em the}
question, focusing on which demands subordination and dissociation of
 all the others. Every genuine problem of human
existence invovles necessarily all the others. One can frequently meet attempts
to address a specific \thi{philosophical problem}, elaborate in the scope of a
single paper the problem of free will, the problem of meaning, the problem of
truth.  Interesting and perhaps even to some extent legitimate as such attempts
may be, they reveal the prejudice that such a division is at all possible, that
distinct problems indeed can be treated in a relative independence; eventually,
that only the closest scrutiny of the most minute distinctions is able to give an
adequate description of any single issue. But the problem is that we not only do
not quite know how one problem could or should be addressed and approached
-- we do not quite know {\em what} any particular problem is. Attempting to isolate any
particular problem for a separate treatement, we fight first with circumscribing it in
any reasonable way which, however, {\em never} reaches the goal of becoming
entirely satisfactory. In this process of 
partial circumscription we invest the problem with all the relations and
implications it carries to all the other problems. 
If one tries to address, say, the issue of freedom without at the same time
illuminating the meaning of subjectivity and openness to truth, the sense of
meaningfulness, the presence or absence of the absolute, in short, without
addressing the intergrity of whole existence, one ends up with the
distinctions one started with and keeps opposing arbitrariness to determinism,
subjectivity to objectivity, spontaneity of feelings to the rationality of some 
inviolable laws, etc.  Every (not only fundamental) issue is the sum of what it
excludes, is the border contracting the tension between this issue and others
into which it is interwoven. In a bit strange (but, in fact, quite understandable)
dialectics, the tradition which had marked the XX-th century with the missionary
zeal of dissociating all the issues and bringing them under objective, systematic and
separate analysis, ends up with the holistic and coherentist postulates, whether
with respect to language, meaning or truth.  \noo{(Wittgenstein,
  Quine, Davidson)} Even though one can not forget the idealistic origins of all such
potulates, one would still like to deny their 
idealistic connections. And one almost manages that, at least, 
as long as one keeps dissociating, as long as one sticks to dissecting one 
particular issue at a time.

We will not try to establish any totality which, in the presence of all too many
accepted distinctions, would indeed be a vanity. We will therefore ignore many possible
distinctions -- not because they would necessarily destroy any unity we might
wish to find but because they would (tend to) completely obscure it. 
We hope to avoid the accusations of relating the unrelated by
making at least plausible that the intimate affinity and kinship of vague yet
distinct aspects, their genuine unity preceds more rigid dissociations, and that
the latter mark only the end -- or perhaps the middle, but certainly not the
beginning -- of the road.
%
\noo{We will certainly try to avoid pitfalls of pantheism (of which most such holisms
are examples), but only because unity can be found above it and not, as some
also trying to avoid the same claim, because it does not obtain at all.
}
%

% \addcontentsline{toc}{section}{Sources, references and conventions}
\subnonr{Sources and references}
I quote rather extensively and from rather different traditions.  However, I
never go into exegesis of the texts or analysis of the thoughts of others.  An
attempt to do so would make finishing this work impossible.  On few occasions I
make more detailed statements in order to illustrate differences which also
should clarify my meanings.  The variety of sources and inspirations makes me
even limit the quotations to the most succinct statements which, I think, express
some essential idea.  Although the basic rules of conscientious exegesis may be
thus violated, and some quotations might have even been not only drawn out of
their context but even adjusted to fit the present one, the intention is never
to violate the meaning of the quoted text. (Besides, exegesis is not our
objective.)
% Asked by a true Irishman \wo{Are you a drinking man or are you a fighting man?}
% I could whole-heartedly confirm the former.  I like company, and

Variety of traditions suggests that we should focus on affinities and often even
only vague similarities rather than differences and oppositions.  Was
St.~Augustine entitled to claim the presence of Christian truths in the
neo-Platonic texts, as he did in the much disputed and controverted passage in
\btit{Confessions} VII:9? Was St.~Clement of Alexandria right in the
similar claims of the affinity of the Greek philosophy and literature with the
Christian revelation? Was Philo Judaeus right claiming not only the similarities
between but even the direct dependence of Greek thought on the Biblical
tradition? Scholars might prove that they were all wrong pointing out
significant differences making the two views different and even incompatible.
The Greek spirit was, after all, completely different from the Christian one.
Perhaps, but this depends on how one draws the borders around the intuitions
like \thi{Greek spirit} or \thi{Christian spirit}.\noo{(Let us also notice that
  such abstracts, useful as they sometimes may be in philosophy, are primarily
  only of historical and sociological character.)}  One can always find
differences separating two views -- the question is at what level, and then,
what value one will attach to them as opposed to the similarities. (After all,
the neo-Platonic culmination of Greek spirit, with its severe critiques of the
emerging Christianity, provided the foundation for the depth of Christian
mysticism.) Opposing, say, Greek spirit and Christian spirit, one should never
forget that in both cases one is speaking about spirit which, incarnated in
opposing socio-historical and political constellations, remains at the bottom
human spirit. It takes some wisdom to recognise concrete unity behind actual
differences and to stop distinguishing when everything worth saying has been
said -- the problem of perspicacious thoroughness, as La Rochefoucauld observed,
is not that it does not reach the end but that it goes beyond it.
%only blathering and babbling has no limit while it 
We will for the most focus on the similarities and it is up to you
to decide whether they are only due to the negligence in observing the important
distinctions or, perhaps, they are justified because the possible distinctions
are of negligible importance.

\noo{I quote others because I find myself to be their friend, even if they might not
always be my friends. If I did not have the quotation, I would write something
similar myself.  Although one might use this to construct the accusations of
eclectism, I would consider such accusations as a proof of a lack of even
minimal good will if not also of intelligence; and so I will rather get rid of
the bad company than of the good one.  The context in which the quotation is
used indicates, hopefully clearly enough, the interpretation I have in mind. I
hardly ever subscribe to the totality of the quoted author's or source's ideas;
it is only the thought behind the quoted piece which I want to bring forth.
(This, I guess, is one of the reasons why I bother to write.)}
%In all cases (with so few exceptions, that they are not worth
%mentioning here), a quotation indicates my more or less full agreement
%with the quoted author 
%I mean that there is enough unity of thought precluding the judgment
%of this work as eclectic, but if you do I could probably understand
%the reasons.


There are a few special sources which deserve a comment.  The authorship
of {\em My Sister and I} is the matter of dispute and scholars can not
tell for sure (perhaps, rather seriously doubt)
that it is indeed, as is also claimed, autobiography written
by Nietzsche himself. %I could hardly care less, since finding
The authorship of relevant thoughts should not be that important.
%whether they were written by Nietzsche or a skillful and competent forger.
However, in an academic context the issue may become a bit sensitive, especially
when the claimed author is Nietzsche.  (It might be so, in particular, if one
wanted to relate the contents of this autobiography to his other works which,
however, I am not doing.)

%For me, it is Nietzsche, for e
Even if it were not Nietzsche, it certainly could be, though 
%As somebody said, it is \wo{how you imagined Nietzsche would sound if 
%you got him drunk}. 
the author might also have been more Nietzschean than Nietzsche himself. Facing
the lack of any decisive proofs or disproofs of purely textual, linguistic or
medical nature, we are left with the text which looks like it might have been
written, if not carefully re-read and edited, by Nietzsche.  The voice for or
against his authorship depends then on one's view of his thought -- whether this
text \thi{fits} into the image one has of his whole thinking and, not least,
personality.  For me, there is a perfect match with the image I had formed
before I found this book. (Possible objections against the portrait arising from it,
should be confronted with less extreme, yet by no means incompatible, impressions
of the close friend in \citeauthor*{LouN}.) \citet{In the end,
  \btit{My Sister and I} reminds me of a true story.}{Sirens}{} Having made this
reservation, I will quote the text as if Nietzsche was its author.

Another referenced text, hopefully of much less dubious value, is a collection
of early Freiburg lectures by \citeauthor*{PhenomReligio} [\btit{Phenomenologie
  des religi\"{o}sen Lebens}, Gesamtausgabe, vol.~60]. Some of these have been
reconstructed almost exclusively from the notes of the students. Thus the reader
should be warned that the quoted formulations, although reflecting hopefully the
intentions, are hardly Heidegger's. (In any case, they are translated by me into
English, and that mostly from the Polish translation of the German text. Well...)

Likewise, \citeauthor*{Celsus}, is only reconstructed from the extensive
fragments quoted and criticized in \citeauthor*{aCelsus}. In this case, however,
the breadth and details of Origen's response give reasonable confidence into the
authenticity of the reconstruction. Much worse is the case of
\citeauthor*{Porphyry} where even the attribution of authorship may be disputed
as the work is reconstructed mainly from the \btit{Apocriticus} of Macarius
Magnes which need not reflect the philosophy of Porphyry. These works are quoted
as if they were written by the authors to whom they are attributed by the
general (though not universal) scholarly opinion. For investigating the
associated doubts and controversies the reader may start by consulting the
referenced editions.

Two distinct editions of \citeauthor*{Periphyseon} have been used. The critical
edition (started by late I.~P.~Sheldon-Williams and continued by
\'{E}.~A.~Jeauneau) of volumes I, II and IV is referenced as just done, with the
number+letter identifying the page number and the manuscript as in the edition.
Volumes III and V are from the abbreviated translation by M.~L.~Uhlfelder and
are referenced to in the same way, \citeauthor*{Periphy}, with only page numbers
in this single volume edition. In either case, the volume number identifies
uniquely the referenced edition. 


\sep
%
One encounters sometimes cases when, in an English text, quotations and longer
passages are given in French, German or some other language of the original --
sometimes even Latin or 
Greek. Although this may serve as an indication that the text is addressed to a
particular audience, it is no more pleasing than any other form of intellectual
snobbery.  It is perhaps a good tone to know German, French, Italian, Latin and
Greek, but few people do and I am not one of them. Since I have used extensively
sources in other languages, I have attempted to access -- and if I did not
succeed then to translate -- all the quotations into English. (A few exceptions
concern passages of German poetry which I did not dare to attempt translating.)
Sometimes, I ended thus translating back into English texts translated
originally from English into another language in which I read them. Such cases
are marked as \thi{my {\bf re}translation...}.  Hopefully, this will not cause
any serious confusion -- to fix it, I have to find some time with nothing better
to do.

\subnonr{Some conventions}

All the works are referred by the English title, even if I used the source in
another language; this is then indicated in the Bibliography at the end of the
text. (A few exceptions are made when the original source is referred after
another author, as is often the case with collected works or fragments.)

The references to all the works look uniformly as
\begin{center}
  Author, \btit{Title} XI:1.5\ldots
\end{center}  
where the part before `:', typically a Roman numeral, refers to the main part
into which the source is divided (e.g., book, part, chapter), and the numerals
after `:' to the nested subparts.  The references to the Bible have no `Source',
thus `Matt. X:5' refers to \btit{The Gospel of Matthew}, chapter X, verse 5. 
(I have used primarily King James Version and commented occasional usage of
other translations in the footnotes.) 
Likewise, the references to pre-Socractics are usually given without any source by
merely specifying the author and the Diels-Kranz number, e.g., `Heraclitus, DK
22B45', where the number identifying the philosopher (here 22) is taken from the
fifth edition of Diels, \btit{Fragmente der Vorsokratiker}.

Identifying quotations by page numbers might have been reasonable in times when
most books existed only in one edition.  I have tried to avoid such references
but in a few cases, where the structuring and numbering of the text happens to
be very poor, I had to use this form. This is also sometimes the case with the
quotations borrowed from others which I did not verify (the source is then given
in the square braces ``[after...]'' following the reference).  The pagination
follows then at the end of the reference as `Author, \btit{Title}
XI:1.5\ldots;p.21', where the numbers indicating part and subparts usually
involve only the main part (i.e., only `XI;p.21'), and may be totally absent, if
no such division of the work is given.  The edition is identified in the
Bibliography.  Occasionally, the subparts may have a letter, as e.g.,
`II:d7.q1.a2'. These are only auxiliary and their meaning depends on the source.
Typically, these are used with the medieval authors and the reference above
might be to the {\bf d}istinction 7, {\bf q}uestion 1, {\bf a}nswer 2, in the
second, II, volume/book.

In few cases I do not know the origin of the quotation, or else I only (believe
to) know its author. I chose to indicate such incomplete pieces of information,
rather than skipping them all together. I have likewise indicated the use of
unauthorized, or in any case unedited, versions of the texts found on the
interned for which no bibliographical data except for the title and the author
are given in the Bibliography. (For some, certainly very pragmatic reasons,
books printed in the USA do not carry explicitly the year of publication but only
the year of copyright. Consequently, the bibliographical information for such
books refers usually to this date.)

\sep
Words which are given some more specific, technical meaning are 
written with \co{slanted font}. \wo{Quotation marks} are used for 
words and quotations. \thi{Shudder-quotes} indicate, 
typically, either the referent of the word in the quotes, or else a 
concept or expression which is not given a technical meaning in the 
text but which is borrowed from somewhere else or even is only 
assumed to have some technical sense. Thus, for instance:
\begin{itemize}
\item 
\co{subject} -- is the subject in the technical sense introduced in
the text;
\item 
\thi{subject} -- is subject in some, possibly technical sense of
somebody else; it may often indicate a slight irony over only apparently
precise meaning one might believe the word \wo{subject} to have;
\item 
\wo{subject} -- refers to the word itself (quotations are also given
in the quotation marks);
\item 
subject -- this is just subject, with full ambiguity and with whatever meaning
the common usage might associate with it at the moment. 
\end{itemize}
I have tried to place more technical details in the footnotes which therefore
can be, for the most, skipped at first or casual reading. They are not, however,
addressed specifically to the scholars.\noo{We have not only no exegetic but
neither any scholarly ambitions.} Sometimes they elaborate the text but in
general will be useful only for those 
 who find some ideas interesting enough to follow them in other
authors.\noo{The footnotes contain, for the most, more details and references to the
possible starting points for such (re)search paths.}




%%%%%% END
\noo{%END
In fact, most if not all meta-discussions in modern philosophy,
arise as a consequence of elaborating the aspects discernible in (\ref{questB})
but not in (\ref{questA}): the idealism versus realism, subjectivism or
perspectivism versus
objectivism, correspondance versus verificationism or pragmatism, tradition
inspired by phenomenology versus analytical scientism, existential orientation
versus linguistic analyses,...

We do not dismiss all these discussions as completely irrelevant but our first
goal is not to be drown in the methodological and conceptual meta-perspectives. After
all, they too are, at least originally, motivated by the interest in an answer
to the first question.

An important aspect of the second question, not present in the first, concerns
the suggested need of justification. If I am to accept anything, it better be
sufficiently justified. This element has overshadowed philosophy if not since
its Platonic beginnings, so since Descartes. It has been used to distinguish
philosophy suggesting, in fact, that it is an existential difference which, in
case of a philosopher, makes him rely exclusively on \thi{reason} while, for
instance in case of a theologian on some \thi{faith}, while in case of an
average man on \thi{casual opinions of common-sense}.  This differentiation,
reflecting already preoccupation with the second question, has of course nothing
to do with the answer to the first question. But philosophers tend to make it
relevant by claiming that only their \thi{reason} can serve in answering the
first question. Various accussations of
\thi{irrationality} follow. Unfortunately, more often than not, justifications of 
claims to \thi{rationality} or even \thi{true rationality}, and in any case the
results of following them, amount to selecting only 
some truths which, as it happens, can be assessed by the \thi{reason}, or what
is actually meant by it.

Such differentiations are, as a matter of fact, political issues, issues of
delegation of competences, allocation of educational responsibilities, division of
faculties and departments or even, as it happened in
post-Cartesian Repubic of Unified Provinces, of national conflict between the
conservatists and liberal republicans.\ftnt{Cartesians supported liberation of
  philosophy from its subordination to theology, while ortodox Calvinists
  wished not only such a dependence but a state underlied religious
  goverment. As any issue with a theological element was almost bound in the
  Republic of that time to turn into a political one, so did this one.}
We do not dismiss all such discussions as completely irrelevant, but we think
that they truly belong to politics. Who is responsible for what? and Who is
entitled to what kind of questions? -- is it reasonable to believe that
answering such questions may help answering (\ref{questA})? 
The problem is that they did not
manage to produce any certain measure of truth which could be used in evaluating
the proposed answers to the first question. And, in fact, if they were to
produce such a measure, it could arise only as a consequence of answering, at some
point or another, the first question. Of course, I shall accept as true only
what is true. The attempts to elaborate on the second question do not bring us
any closer. The difference between the two can be seen by comparing them to the
respective questions below:
\equ{What is important?\label{questAA}}
and
\equ{How should I figure out what is important?\label{questBB}}
We can iterate the meta-appications past (\ref{questB}) -- \wo{On what basis
  shall I accept something as true?} -- just like past
(\ref{questBB}). The infinite regress is very much like the infinite
regress of formal reflection. And, in fact, just like the formal regress of \wo{I think
  that I think that...} adds nothing substantial to the first thought, and only
enmeshes it in a  formality of a childish mechanism, so the formal regress
of our meta-applications does not clarify anything with respect to the first
question, but only multiplies the possible issues, doubts and problems one may
explicitly rise and discuss.

One may easily object that, as a matter of fact, when the answer to
(\ref{questAA}) is unknown and hard to find, one could get some help from
answering first (\ref{questBB}). Formally, it may indeed seem so, but it is a
pure formality.  Notice that questions (\ref{questA}) and (\ref{questAA}) do not
concern any particulars which 
might happen to be special cases of more general laws, possibly useful to know
in treating the special cases. No, these questions have uncanny level of
generality which it seems futile to generalize further.  To such an objection we
can only answer: Sorry, disagree completely, for any answer to (\ref{questBB})
will, in fact if not in principle, 
require at least a partial answer to (\ref{questAA}). It may certainly help to ask
another, and related question, but approaching an issue from a different
perspective is not the same as approaching it from above. The only things that
may happen in the latter case are that one either falls down or flies away.


\subsubnonr{vs. arguments; Different tempers}

Yet, philosophy is only a particular expression of the fundamental
differences of human existence:... Simplifying to the extreme
opposition, the two tendencies are those of one vs. many, Plato vs. Aristotle,
Aquinas vs. Ockham, Husserl vs. James, unity vs. plurality. 

Different tempers...


\sep

What makes Anaximander the first philosopher is that he tried to argue for and
justify his claims, he used arguments and not mere statements. This is, at
least, the general view of ... philosophers. The \thi{love of wisdom} with
which one started, has long ago become the love of argument, to the extent that
wisdom without argument goes for simplicity if not stupidity.  One insists on
the arguments the more, the more one feels threatened, that is, the less one
feels certain of own self-identity.

Thus one has managed to separate itself from
literature with its usual lack of and occasional distaste for intellectualism.
It was a bit more difficult to cut off theology but here, too, the fact that
some aspects were assumed to be indisputable helped. Unfortunately, along with
that all the concern for the divine presence in the world was removed too.
Indeed, how would one argue about God, immortality and other invisibles? Their
invisibility amounts more or less exactly to the inadequacy of any arguments;
after one got tired of proofs of God's existence, the whole \thi{topic} became
highly inadeaquate, because incommpatible with the image of precise  logical
argumentation.

The strange thing that so happened was the emergence of the perfectly univocal
definition of valid arguments and mathematical precision in their study. Formal
logic, and its subfield of comutability and recursion theory, put the final
period after the millenia of developing Aristotle's syllogism, Leibniz's
calculus of reason and the ideas of mechnical reasoning. We now know that, in
itself, it yields absolutely no insights and, moreover, that no mechnical
procedure can ever be found for generating valid results...



\sep

The title is used by others....
I could have found another title but this does capture the
essential feature of existential situation which is the
only object of this book


\wo{Philosophical antropology} is but another way of saying
\wo{description of the existential situation}.


Ignore what is not experienced; what is really, objectively, in-itself is as
little relevant for this description as the answer to the question whether the
universe will continue expanding or starts at some point receding.

We study human experience and how its various aspects are constituted. This is
the opposite, in fact, antagonistic approach to the one which starts with some
ready constituted elements and tries to explain the construction, emergence and
functioning of a mechanism. One never knows if the given building blocsk are
really the eventual ones, nor even if they are appropriate for the task one
tries to perform. We leave such exercises to the empirical project. Whether it
succeedes or not won't change one fundamental thing: the human experience. We
know that it is Earth rotating around the Sun, and yet we can not stop talking
about -- and in fact, feeling and living as if it was -- the sun rising up at
the horison. This is how things look from our standing point and no amount of
empirical or logical proofs and arguments, no amount of bad or good science,
will ever be able to change that, unless, perhaps, one starts messing up with
the human beings themselves (and we may still hope that genetics won't go that
far).

We are not interested in mathematics or physics, in biology or sociology -- at
most, in how such forms of modelling the world emerge within the horison of our
experience; we are not interested in objective time or space -- at most, in how
such forms emerge and penetrate our experience; we are not interested in what is
objective and what is not -- only in how such distinctions may be relevant for
us; we are not interested in life -- only in the feeling of life, or better, in
living. 

This looks pretty bad, right? Pretty empiristic, subjectivisitc,
phenomenological, even idealistic, or perhaps just existentialistic! As a matter
of fact, it is not any of these, at least I hope it isn't.
But it is not any of their traditional opposites, either. It is only something
in the direction of 

\ad{Anthropology}
\citet{What is typical allows one to retain cold blood, only individually
  conceived matters cause nervous shock. In this consists the peacefulness of
  science.}{Faustus}{XLV\noo{p.622}}

The only object -- human being and his confrontation with the
world. We can easily imagine the world without human beings, but
for philosophical antropology such a world has no relevance.
As we will try to show, such a world is only apparently imaginable
-- in fact, it is totally unthinkable or else, to the degree it is,
it is totally indistinct.
(This has nothing to do with its objectivity or subjectivity, its
reality or ideality.)

Thus philosophy which tries to present the world as it is in-itself,
as it is sort of independently from human being, may be perhaps
interesting, sometimes even enlightening, but it will never provide 
a completely satisfactory account. It may reduce its sphere to some
particular area and problems and study that. But its concepts and
results are then founded outside its scope...


Yet, not explaining but describing = choosing what matters and
placing it in relation to other aspects which matter = giving account;
which leads to a unity of understanding ....

\wo{To give account}...
\begin{itemize}\MyLPar
\item
  First of all: try to determine \co{that} before even asking \thi{what}.
  And then ask thi{what} before even thinking \thi{how}. If sometimes it happens
  that one answers \thi{how}, then in any case never ask \thi{why}...
\item
  Only (some) necessary conditions, hardly ever any sufficient ones -- they
  do not obtain. We are not after any explanations; perhaps we are after
  something which possibly might be attempted explained later on, but that is
  not so either; we are after something for which explanation is inadequate
  category...
\item
  give only necessary conditions; the sufficient ones do not obtain any
  way. Does it mean that I believe in miracles? Well, as you like it. I have
  never seen a sufficient reason for almost anything, and I would say that those
  who insist on them must actually {\em believe} that they obtain. And as they
  do not obtain (ok, except in mathematics, or some simple physical analogies),
  saying that we can see necessary conditions but not sufficient ones 
  seems to me a matter of simply conforming to what is experienced, not believing
  in anything. There is much less believing here than in the belief in
  sufficient reasons. But sure, if you want to call the transition from the
  given necessary 
  conditions for some X to the X itself for \wo{a miracle} then I
  do not {\em believe} in miracles -- I see them all the time.
\end{itemize}
\wo{Irrationalism!!!} may exclaim some others, but let them remain calm, too.
Nobody knows -- and those who claim to know can not agree -- what rationalism
is. It seems that, at least, it requires (do we notice a necessary condition?)
not being carried away, some sobriety. Or in less prosaic terms, rationalism
requires that one assumes a position only with the admittance of the possible
limits of its validity, if not with the actual knowledge of such limits.  If one
accepts this rather generous critical rationalism then we should be able to
complain, all the way. Even to the point where one asks about the limits of
validity of this very position...  (After all, this admittance of partiality
squares so nicely with the etymology of \la{ratio}! A bit worse with the
pretentions to universality...)

\sep

It is not for all...

\sep

Many more specific issues are only touched upon and one may wonder why at
all. But we are not trying to resolve all the issues, to come up with a definite
and final answer. In most cases such answers simply do not exist and we prefer
to say too little rather than too much. Multiplying distinctions and
perspectives may be as rewarding academically as it is existentially futile. 


\subsection*{The main points ...}
{\small{ \begin{enumerate}
\item \co{There is} $\sim$ the One -- the particulars are only 
``perspectives'', modifications of the \co{Is} (incommensurable for 
frog and man) \\
 It is the fundament of all experience and \ldots the fundamental 
 experience. (``Objectivity'' cannot be reduced to any experience because 
 it comes before. ``Externality'' can.)
\item \co{Distinction} is the begining; \co{point} --  \co{pure 
distinction} -- is its \co{reflection}
\item There is nothing beyond experience.
\item The levels of \thi{being} and transcendence coincide with the time-spans
 \begin{itemize}
 \item \co{transcendence} is essentially something present which is
 not exhausted in this presence, something \wo{more than it is}, the overpowering
 \item
 this \co{more} can be relative to various aspects but, primarily, it is related to the
 \co{horizon of actuality}
 \end{itemize}
\item The ``fourth level'' is not a level ``above'' the other three but
 concerns the ``essential structure'' of division into \LL\ and \HH. (It is not
merely ``formal''.)
\item The world ``consists of'' \LL\ and \HH\ --
 Man is a borderline between \LL\ (lower) and \HH\ (higher).
 \begin{itemize}\MyLPar
 \item the \wo{totality} of his being is not \wo{constructed out of points} but 
 precedes the \co{distinctions}.
 \item Before you construct out of pieces, the pieces must be there. \\
  \begin{tabular}{l|l|l|l|l|l}
    \HH & (phon)emic & whole (Gestalt) & mind & concrete & quality \\ 
    \hline
    \LL & (phon)etic & part & body & precise & quantity 
  \end{tabular}
 \end{itemize}
\item The basic existential mode is determined by \G\ (openness, acceptance) or \B\ 
 (denial, refusal) of the mastery of \HH.
\item Important are \co{nexuses} [tie, connection?! -- vs. \la{religare}] 
of \co{aspects} which can be reflectively dissociated but which 
function meaningfully only in the \co{nexus}
\end{enumerate}

\subsubsection{{...and defs}}
\begin{enumerate}
\item \co{Experience} -- whatever leaves a mark vs. the source of unpredictability.
\item \co{Reality} -- what you cannot live without.
\item ``natural attitude'' to external world -- possibility of new, unexpected, 
      not-from-me.
\item \co{Sharing} -- the same \HH.
\item \co{relevant} -- telling what to do in the face of transcendence.
\end{enumerate}
}} % end \small

\section{Preface}


Well, perhaps, I was not quite honest.  I want to give an account of a
possible understanding of the unity of \ldots Yeah, of what?  Of a
human being, of a person but, perhaps eventually, only of myself. 

\citt{Gradually it has become clear to me what every great philosophy
so far has been: namely, the personal confession of its author and a
kind of involuntary and unconscious memoir [\ldots] In the philosopher
[\ldots] there is nothing impersonal;}{Nietzsche, {\em Beyond Good and
Evil}, I:6} 
\citt{It is almost incredible that men who are themselves working 
philosophers should pretend that any philosophy can be, or ever has 
been, constructed without the help of personal preference, belief, or 
divination.}{W. James, Essays in Pragmatism, I, p.24}
The very same words may be deeply meaningful to some and ridiculously
superficial, even stupid to others. 
This book is not meant to {\em convince} anybody about
anything; I hope it will be found interesting by a few, and it is
addressed to these few.
% You find pretty quickly, by just starting reading, if you are among them.

What is my method?  I do not have any. 
\wo{Philosophy does not so much tell {\em what} to think as {\em how} to 
think.} All kinds of methodological postulates have polluted philosophy 
since it started to consider itself, or rather started 
{\em to try} to 
consider itself a science in the modern sense of the word\ldots
Positivism, pragmatism, phenomenology, all \thi{methodologies} of 
philosophical inquiry tried first of all to ape 
science, to achieve the same level of precision\ldots

But \thi{how} is only, and only at its best, a pale reflection of
\thi{what}, a method is at best only a desiccated sediment of a
particular way of understanding, which is dictated, if not determined,
by the particular objects or sphere of experienced addressed. 
Scholastics knew it very well, but we do not like scholastics very
much nowadays, do we?  Sure, \thi{how}, having extracted some resdiual
sediments of \thi{what}, turns easily into a socio-political
phenomenon, a \thi{school} or a \thi{party}, which pretends to know
{\em the} axioms, and in any case at least {\em the} rules of the
game.  But the game goes on and the only rules are to be
understandable, and then to have something worth understanding. 

\thi{How} a person thinks is merely an expression of \thi{what} he
thinks -- uncovering, perhaps, some hidden or unconscious assumptions,
but still only assumptions about the \thi{what}. To dissociate the two is a
violence against concrete thinking. Useful and justifiable as it may 
be in a social context, determined always by the overwhelming 
majority of mediocricy, it is a violence against concrete thinking.

It is
\thi{what} and only \thi{what} that interests me \ldots The \thi{how} 
only follows the suit \ldots 





\ad{Philosophy searches for the Absolute Reality --} 
whatever this might 
mean. Since epistemology, this changed to the search for the Absolute 
Knowledge of Reality

A universal temptation underlies most of philosophy, the temptation 
to go after the absolute, the \he{undubitable} truth, to 
construct things, matters, the world once for all on a secure rock of 
\he{incontestable} reasons and matters of fact. The keyword of this 
temptation is ``security''. And neither the poor results nor even the 
schizofrenic split of the intellectual personality between `is' and `ought' 
are able to eradicate it. 

\begin{enumerate}
\item Certainty -- justification:
 \begin{enumerate}
  \item fear of the unexpected (reason vs. feeling)
  \item flirt with science
  \item[\isimp] unchangable
  \item[\isimp] necessary
 \end{enumerate}
\item Necessity (apriori) \impl irrelevant \\
what must be (irrelevant) vs. what is/can be
\end{enumerate}

\pa
The praise of arguments is but a side-effect of that. But what is an 
argument good for? Have you ever changed your mind concerning some 
fundamental matter because you have been exposed to an irrefutable 
argument? Have you ever accepted a view because somebody managed to 
produce a \thi{proof} of it? I do not think you have, but it is not 
only because
\citt{[q]uestions of ultimate ends are not amenable to direct
proof.}{Mill, Utilit.  ch.I} An argument need not be \thi{direct 
proof}, it may be just an argument, although it always tries to force 
its conclusion, to convince. It is, indeed, a great field open for 
invebntivness and shrewdness of intellect. But \ldots if I do not find 
the cnclusion plausible, then all the shrewdness of the argument does 
not help a least thing.

Philosophy can do better than produce arguments -- it can try to 
describe experiences or, perhaps, experience. If you find Kierkegaard 
worth reading, I doubt it is because of the excellency of his 
arguments. It is because you find something worth paying attention to, 
it is because you sense an attempt to communicate an experience which, 
being an experience of another human being, may turn out to be most 
relevant for yourself, too.

\pa We do have a lot of philosophy which occupies itself with
inventing new arguments for old truths and, on the way, with
rearranging the language in order to give the truths, as well as the
arguments, the apparent look of novelty.  Although the reasons for
such a search remain hidden in the obscurity of academic vanity,
it may be, perhaps, worthwhile. It is not my objective.

\ad{We do not believe in Absolute Knowledge,} 
final justification. 
Reason, in the broad and traditional sense, implies, and since Nietzsche 
means, control; control over 
chaos and anarchy which threaten our finite being. The 
anarchic element is just the opposite of reason, is the unreasonable, the 
uncontrollable. Again, in a broad and traditional sense, it has often been identified
with emotions and feelings.
The conflict between the two is thus not only a platitude of a second rate 
literature but also an analytic triviality -- it follows by definition. 
This looks like a great starting point! Does anything sell better than 
trivial platitudes? And the sale numbers are just reflection of the 
commonest, not to say the meanest, interest.

Reason, having gradually turned into 
rationalism, did not give up its absolute claims expressed in its
controlling function. But becoming {\em ratio}, it turned partial; 
after 
all, ``ratio'' refers to reason as much as to a relative value. Order needs 
proportions which bring divisions.
 Again by definition, reason loses the possibility of 
full control. But this only increases the tension -- the attempts to 
regain control become the more desperate.  

\begin{enumerate}\setcounter{enumi}{2}
\item Dissolution of subject (one of the guarants of infallibility)
 \begin{enumerate}
  \item society, culture, epistemic authority, selfish gene, discourse ...
 \end{enumerate}
\item Dissolution of concepts and distinctions \impl processes 
 (evolutionary epistemology)
 \begin{enumerate}
  \item empiricism-rationalism-idealism-pragmatism (only rough distinctions)
  \item Gestalt, holism (hermeneutics) ...
  \item empirical turn (Dennett's I, dynamic systems ...)
  \item sociological turn (soc. of knowledge: Kuhn, Rorty ...)
  \item analytic-synthetic (Quine), ...
 \end{enumerate}
But this only emphasises continuity between the old conceptual extremes.
\end{enumerate}

\pa
We do not believe in absolute knowledge because we have lost grasp on 
objects. The empirical turn of much of philosophy is but an attempt 
to re-confirm our intimate involvement into the \thi{objective world}, 
the unbroken relation with the field of our experience, which seems to 
disappear dissolved in the arbitrariness of discourse or whatever one 
wants to install in its place. 

Dissolution of objects\ldots great! It will be an important point.

\ad{No Absolute Knowledge \impl no Absolute}\label{pa:attitudes}
Because
epistemology got us used to identify reality with our knowledge thereof, it
is often hard to see if renouncing Absolute Knowledge we do not also renounce 
Absolute Reality. ``Wer spricht \"{u}ber siegen? \"{U}berleben ist alles.''

Inquiring into possibly ultimate dimensions of human existence is not popular.
Knowledge, even if not absolute, seems still a relevant question. But its relevance
and reality haunt in the background.
\begin{enumerate}
\item agoraphobic: ``Wovon man nicht sprechen kann, dar\"{u}ber muss man schweigen'' 
   (Witt. I = Witt. II) -- liguistic-analytic turn (despair, for intuition knows
that language \isnt reality)
\item ecstatic: catch the ineffable, use ``new'' language 'cos this is reality 
 (Heidegger, Derrida...)
\end{enumerate}
These two avoid distinction language \isnt reality, still under the spell of
epistemology; remove thing-in-itself, the transcendent
\begin{enumerate}
\item[3] Stay on the edge: moralism (Levinas, Rorty...); critical rationalism
\end{enumerate}
 
\subpa
It should be observed that the three
attitudes from \refp{pa:attitudes} are by no means specific to 
postmodernism. They 
are present in any encounter with the transcendence, in any encouter of a 
finite being with something that is greater than it. The character of the 
transcendence, that is, what is perceieved as transcendent and in what way, 
leads to various specific manifestations of these three basic modi: unreserved 
acceptance, claustrophobic refusal or sober confrontation.

They have thus always been present in philosophy. I am now about to commit
horrible simplifications and indefensible inadequacies. So I won't defense them nor 
claim that one 
couldn't arrange the following examples in a very different way. 
Each philosophy contains all the three moments of this tension. But each philosophy has
also its specific flavour which, apart from all the technicalities,
distinguishes it 
from others. I hope 
that you may recognize the flavours which made me classify each triple in this
way.

\begin{tabular}{rlllll}
tragic:  &  sophists & Hume &      nominalism        & Kant  & Husserl  \\
comic:   & Plato  &    Spinoza &   extreme realism   & Hegel & Bradley, Heidegger \\
pragmatic:& Aristot. &  Descartes & Abelards' realism  &     &  James, Scheler
\end{tabular}


\subsubsection{What is philosophy?}

It does not appeal merely to the intellect (lest it gets sterile)!


\ad{Reach reality} 
In general -- the temptation to reach reality; to be relevant; 

-- Cope with transcendence; {The greatest fear}
of unexpected, new, unpredictable...


\pa{Thing-in-itself,} objective world, etc. \co{There is}, yes, but what? 
I do know that \co{there is}, from the very beginning, from the virtual 
\co{signification} to the most advanced concepts, I know that there is 
something beyond them. But this something merely is ... At each level it 
may be the level below, or above, it, which is not accessible, like 
experience is not accessible to reflection, sensation is not accessible to 
concepts, etc. -- every layer has its level of transcendence. And at the 
bottom \co{there is} \co{nothingness}. 

Replace things-in-themselves by One, \co{there is}; and then something 
``objective out there'' by something inexhaustible

[murmur of being]

[the experience of objectivity (in what sense?) is not built up, 
construed, it is not reducible to the experience I had yesterday after 
lunch or may have tomorrow -- it is the fundament of experience and, by the 
same token, the most fundamental experience]

[it shouldn't be confused with the questions ``is this objective?'', ``is 
this objectively true?'', ``what is there objectively?'']


\pa
Reality degenerated into givenness\ldots

and philosophy thus degenerated into attempts to provide a universal 
description of thjis static givenness. Universal, which means 
primarily, incontestable, applicable to and recognizable by all. The 
descriptive bias made it hard to recognize that the basic, most 
important aspect of human reality is actually concerned with the 
choice, fundamental, spiritual choice; that facticity and possibility 
of this choice is what, in the deepest sense, constitutes human reality. 

-- ethics tries to take care of this but only in letter. Living as a 
rather poor relative, on the outskirts of ontology and metaphysics, 
it is heavily influenced by the general frame of mind. It degenerates 
into detailed analysis of possible acts and actions, particular 
choices and, sometimes, attempts to formulate general, abstract forms of 
moral imperatives, dissociated from the results of the ontological and 
epistemological investiagtions.

\ad{One --} Not science; not unity of sciences; not based on 
sciences; not concepts and their logic... but addressed to a whole human being.

Not ``how'' and ``why'' are the primary questions -- they belong to sicence -- 
but ``what''. (Eckhart's living ``without why''.)


\ad{Origin}
As One it isn't a science and shouldn't envy sciences. Historically, it was their 
origin.

-- Psychologism is detestable -- because psychology begun to take over some 
problems philosophers were occupied with. But it isn't if we are 
interested in human life rather than in being philosophers and the 
associated definition and  status.
(Unfortunately, psychology addresses concrete human condition and, negating
psychologism, philosophy tends to neglect it.)

-- History, 
A variant of the absolute is `unchangeable' -- no absolute = only history.
\\
History is irrelevant -- just one of the temptations to become more 
concrete and closer to the real world. I still read Heraclitus and Bible...
\\
Like in Kuhn's theory of quiet tides of science rising through periods of stability
until reaching the revolutionary height and turning in a new direction -- history, and with 
it philosophy (?), presents us with such a picture. A turning tide, a confrontation
with transcendence in a new form, brings chaos and disorder. 

-- Sociology

\ad{Rational}
One used to think of rationality in terms of justification, 
which then is something like argumentation. However, we do not believe in the 
ultimate justification any more, do we? 
There is at most a difference of degree (and often not even that) between
philosophy focusing on the quality of the arguments on the one hand and
rhetorics and sophisms -- disciplines, perhaps venerable, but hardly kept in high
respect nowadays. And to the general public, this is what philosophy often means
-- sophisticated and convincing arguments for the most ridiculus and unconvincing
theses.

The arguments function the better, the more petty matters are at stake -- 
``Shall we eat indian or mexican?''... The more important matters are concerned, 
the less imperative the arguments become, because then they are merely attempts
at justifying what we already are convinced about. 
Eventually, arguments never convince.
In the matters of real importance, 
argumentation is just a sign of lacking respect - one tries to convince by 
explaining to
another what he apparently is unable to understand. If he only could, if he only 
saw all the valid reasons which we see, he would accept our conclusion.


\thi{Big words}, or better \thi{high words}, on the other hand, 
do not force their meaning upon us. Their vagueness invites to most personal
interpretations and misunderstandings. Yet, they are not for this reason arbitray
and incomprehensible. 
They only hint at something not fully expressible, which we
are free to model and interpret -- they leave us freedom, exactly
because they do not have a unique, precise meaning.
As such, they are the opposite of arguments which, in honesty or
arrogance, always attempt to force the other to accept them.


\pa
The important thing is that what is being said is said as clearly and understandably
as possible. For such a purpose, arguments may, occasionally, have a value, just as
examples do. They are not, however, applied to force any conclusion, but merely to 
illustrate the connections between various aspects of the discourse -- in particular,
those which seem more acceptable and those which seem less so. 
But to identify them with the whole importance of rationalism is to turn it into
sophism.

Now, forgeting justification and keeping in mind the inherent partiality of reason, 
I would formulate the thesis of rationalism as follows
\thesis{\label{th:panrat}
A rational acceptance of a statement is the one accepting the limits 
of its applicability/validity.} 
%(Critisism concerns these limits.)
% Science is relative to a context...of possible falsification/criticism 
% which is exactly what limits the scope of scientific theories.
I may not know what these limits are but, at least, I am open for the 
possibility of their existence. And as far as I am able to, I try
to specify them. 

\pa
This thesis is not limited to the theory of science but 
can be taken as a quite general fundament of a philosophical project. For 
the first, even the generality and absolutism need not be dogmatic -- I am trying
to communicate some experiences, perhaps the ultimate ones, but I admit that
my formulations may be unfortunate, may be unclear, may be improved.
For the second, it is self-applicable: many statements, not only the statements of
the absolutistic philosophy, can be accepted without limits. Saying ``I 
love you'' one would like to give it an absolute value -- no temporal 
or contextual factors should limit its validity. This is perhaps the most 
irrational meaning one can give to this most irrational statement and I do 
not think that anybody who has ever made it this way would like it to 
have been made with all rational reservations. However, 
even the irrational statements and attitudes can be treated in a rational 
way according to \refp{th:panrat}. Sure, they lose then their magical air 
of actual existential commitments but philosophy is, at best, a reflection 
of life and never the life itself. It may invite to making some 
commitments or choosing a particular way, but it is never such a commitment 
or choice itself.

\pa
From the existential point of view, thesis \refp{th:panrat} implies something 
like the following attitude: I keep my convictions and comittments as long as
I do not encounter the contexts (situations, arguments, attitudes) 
which ivalidate them, which make tham impractical, 
unviable or unacceptable (in whatever sense). In such situations I do not have
to renounce them entirely -- typically, I will merely introduce more reservations
concerning their applicability. More importantly, it does not imply
that all such convictions and comittments are kept rationally -- the deepest, 
most significant aspects of my being and acting are, probably, those which 
have not as yet got the chance to be explicitly limited. Their source does not
lie in any critical examination but beyond. Still, the attitude makes me, at least
in principle, open for the possibility that they too may need such a limitation 
at some time. But I may nourish implicit and explicit convictions of various
degrees of constancy, some being easily disposed of, other being of fundamental
importance so that only the most extreme circumstances may shatter them.
\com{
Perhaps, knowledge in a stricter sense, involves a rational acceptance with the
explicit statement of such limits.}

\pa
{\em Explicit concepts} -- unlike literature or poetry. (lists of 
distinctions; perhaps, just enough words for the vague concepts)
and thesis~\ref{th:panrat}

-- don't multiply concepts and distinctions. 
They never match experience, while one thinks 
that, more and more disitinctions will bring one closer. Nothing more 
illusory!

\pa
Of course, all cannot be embraced with a scientific precision ...

-- {\em Proper abstractions}! ... create = ignore $A$, assign weight to $B$; 
economy of signs.


\ad{Is vs. Ought}
Ontology vs. axiology; the world as it is vs. as it should be. There is no 
doubt that we can speak about our wishes as to what or how the world should 
be. That we can speak about how it actually, really is, looks much more 
like a wished for assumption. And taking into account the 
rather poor results of the attempts at establishing any form of consensus 
in this matter, it even looks like a possibly wrong assumption. 
It pushes ethics on the side -- for who can doubt the primacy of the real 
reality over mere \thi{ought} -- while, at the same time, leaves theoreticians 
in an eternal despair over the lost connection with life. 
The separation of theory and practice, engendered by the conviction that they 
address the same reality, makes theory irrelevant and practice unreliable. 
It separates man from the world he lives in, that is, from himself. 

In the middle of the pretensions to know how the world is, perhpas even how 
it must be, and -- what 
is usually even more annoying -- moralising advice on how it should be, 
the crucial question asking {\em how it can be} appears highly
\he{reproachable}. Avoidings both other questions, it promises a mere 
\la{Weltanschaung}. But avoiding these respectable questions, it 
also avoids the schizofrenia they caused. And furthermore, it avoids the almost 
inevitable arrogance of the attempts to answer them and claims to having 
done so. And who would daresay that, on the final account, and in spite of 
all the claims to the contrary, any philosophy -- and any of its 
opponents -- offers anything more than 
a \la{Weltanshaung}, a statement ``look, you {\em can} see the world this 
way''? One may try to hide behind claims to objectivity and absolute truth 
but then, if not anybody else then eventually the history will fetch one 
from this hiding place.

\pa\label{pa:tomee}
It all may be wrong. It may be, however, only wrong for you. It may also be 
wrong for many others while it may seem right to you. It is so to me. 


\pa
This is, roughly, the history of Western philosophy, perhaps, its short 
version for the lazy students. This is also the theme of this book -- for 
those who, in their perplexity, find simply living the life somehow ...
unsatisfactory. I doubt that anybody can learn from it anything which he 
otherwise does not know. It is only, a sort of, giving an account. And giving an account is 
like paying a bill -- it brings the satisfaction of fulfilling an obligation. 
But since the smartness and skills of a self-made one have reached the
 sky on the stock market,
getting away without paying the bills is a reason to pride. Paying them is 
stupid. \citt{For the people of this world are more shrewd
in dealing with their own kind than are the people of the 
light.}{Luke, 16:8}

\pa
Philosophy attempts to derive something obvious from something acceptable. It
isn't an art of discovery but of appreciation.

\noindent\dotfill

The problem of the existence of the external world comes from
\begin{enumerate}\MyLPar
\item\label{sub-ob}
thinking of our being in terms of the idealized, timless subject-object
relationship, and 
\item\label{dedworld}
the attempts to construct or deduce the world from within such a horizon 
of pure actuality.
\end{enumerate}
It is just one of the problems arising from this perspective, totality of my being, 
and phenomenon of time being other well-known examples. If, on the other hand, 
I think of myself as being capable of continuous experience and take time as
a fundamental, rather than possible, notion, the existence of the world beyond
my experience ceases to be a problem and becomes a fact of experience.

\pa
Let me emphasise again, but for the last time, that I am not looking for
necessary, irrefutanle truths. there is no {\em necessity} of such an
admission. One may let the all-powerful \co{objectivistic attitude} prevail,
reducing the world to a never reachable totality of independent objects. But
I do not aim at a \co{re-construction} of a pre-reflective truth, which is
what it is and cannot be otherwise -- which is necessary. Such a truth may be
of interest to a scientist but, as I said before, it cannot satisfy
reflection.  Perhaps it is true, perhaps everything we experience can be
reduced to some basic, pre-reflective facts, even to some necessary physical
or chemical reactions. But then a naive question arises -- so what?  How can
this ``wisdom'' affect my life and actions? Is it at all relevant that
something, which cannot be otherwise, is as it is? 

Suppose that somebody produced a definite proof that it is so. And? Shall I
change anything in my life because of such a proof?  Shall I stop feeling the
way I do and choosing as I do? The only way such a proof might become
relevant would be to produce some controlling mechanisms.  Perhaps, having
traced everything to such a basic level will enable us to control everything
which is above -- and as some maintain, independent from -- it. Then it might
be a different story, but only to the extent, that our feelings and choices
could span over a different, perhaps wider range.  So far, the very
meagerness of such results makes the whole project as irrelevant for
philosphical reflection, as it is relevant for the \co{objectivistic
attitude} of science and controll. Let those who believe in it, work on it.

Necessary truth is what it is and {\em cannot} be otherwise -- do what you
want.  Thus, although it determines me, it does not affect my reflective
attitude -- it must be so, and there is nothing I can do about
it. Philosophical flirts with necessity do not interest me not because one
cannot produce cunning arguments in its favor but because they make the whole
endeavor existentially irrelevant. Firmness of existence arises from the
fragility of its background, not from the necessity of its assumptions.  Even
if at the bottom everything is involved in a play of deterministic forces,
this, certainly, is not the reality of our \co{experience}. In particular, there
is an element of freedom in reflection, if not in anything else, then at
least in the choice of its object. Reflection may admit that something has
relevance but that it cannot or does not want to focus on it and chooses to
focus on something else. Appeal to necessity of considering this rather than
that is a simple act of disrespect. The point is not to demonstrate necessary
truths but truths which are of interest, not to \co{re-construct} but, yes,
to construct.  What matters for me is not an inescapable result of necessary
processes, but the question  how and, above all, towards {\em what} direct
my reflective attitude.

\pa
What I see as the aim of philosophical enterprise is not to build a theory about 
how things really are, even less to build a theory which simply isn't inconsistent. 
Rather, I see it as some \thi{economy of language}, \thi{economy of words} which
should be spoken in as purposeful manner as possible.
I do not think that many
(if any) proponents of philosophical theories which we would classify 
as implausible ment really ridiculous things -- but their language sometimes makes 
us feel as if they did. ``Reincarnation'', ``immorality'', ``deeper Self'', 
``God'', ``external world'' etc. may all happen to refer to some things 
one wants to convey in a way unmatched by any other words. They may make 
speaking simpler, thus appearing most economical and
far better than ``circulation of matter'', ``immortal fame'', ``super Ego'', 
``opium for people'', ``transcendentally ideal world''. 
We have a tendency to believe that the latter express some concepts and (therefore?)
are more precise:
they reduce the former to something apparently more understandable, more
adequate for a reasonable discourse. But I am not sure that a discourse is
``reasonable'' when it excludes or simply distorts matters unclear and
never given as well defined objects, only because they
cannot be stated in unambiguous terms, preferably with the precision 
approaching that of mathematics.

\pa
The goal of reflection is not to re-construct the original, pre-reflective
truth. The origin of reflection -- experience -- harbours the possibility of
reflection and the material for it. But is made of a different
matter which cannot be fully and completely re-created with the categories of
reflection. The concrete material of pre-reflective experience can be, at
best, mimicked, symbolised but not re-constructed by reflection.

The only thing reflection can (and should) do is to construct its own
world. Its own, because reflection, like any other activity, can operate only
with its own categories. But this world should be constructed in such a way
that its original truth can live through and within it. One should strive
after the concepts which not only do not falsify or oppose the pre-reflective
feelings and understanding, but which actually open for their presence, allow
them to enter the world of reflection and unfold therein.

\ad{Novelty}
New insights -- rather inspiring new branches, 
history, sociology, psychology...

-- But otherwise, human-wise, \la{nihil novi}, 

-- [used in chap. III, Intro] \citt{\la{Mundus senescit}}{Gregory of Tours, {\em History of the 
Franks}, \~594AD} ``The world ages''

So WHY?

-- ``One can't construct a system which captures (the whole) reality.'' But, in
fact, more is true -- one cannot capture reality into words! Kierkegaard's
criticism of a system does not (really) address systematic analysis but the
intellectual arrogance believing that it does manage that...

-- A new philosophical book is but an expression of understanding, or 
misunderstanding, the old books. There are no new thoughts, no 
thoughts which were not thought before. The goal is not to think new 
thoughts, but to understand the old ones. 

Heidegger recommended return to Greeks, to Heraclitus, if possible, 
even farther back. So did Rousseau\ldots There is no kernel, no 
historical site -- it is everywhere, and nowhere\ldots

\subsection{We have \ldots, and shall\ldots}

\ad{Actuality}
We have heard the trumpeths!  
We have noticed the importance of distinguishing the beings from 
their Being, we have noticed the metaphysical preoccupation with
presence or, as I shall call it, with \co{actuality}. 
The barbarians have entered the gates of Rome and \ldots have found 
their place there. 
The Derridean fear of being accomodated, 
appropriated, embraced by the philosophical discourse was well 
founded because it was based on equally one-sided, as it was deep, 
understanding of this discourse.

After the initial fascination, seduced by vague promises of something 
new and different, even general public gets quickly tired of 
\wo{writing otherwise}, 
\wo{speaking otherwise}, all these \wo{otherwise} which, eventually, 
turn out to be nothing else than the other, ignored side of the discourse 
against which they tried to protest or, often, just a new way of 
saying the old things.
%%% `thinking otherwise' is in Book II. level of actuality

\pa
The barbarians have won, they upset the Rome, its order and its life. 
But there is a great danger in a great victory. 
And now comes the time of defeat, of settling down, of accomodating to 
the culture which lasted some thousands years not by a 
mere accident, not by a mere misunderstanding,
but because it was founded on something real. The barbarians 
have won because they had a good point to make or, in any case, because 
Rome did not have anything to counter them. But the noice of 
victorious entrance is but a passing news, and the horses on which 
one makes one's tryumph will soon be weary and have to be replaced. 

One thinks that conquering Rome everything has to be renewed, started 
anew; one lies down new foundations, challanges to new ways of 
building, living \ldots yes, thinking. And what? A short time passes 
and it turns out that people think as they used to, that people live 
as they used to, that the same emotions and passions steer their 
lives. The only things which, possibly (but even that not necessarily) 
have changed are the cloths. 

I do not believe in \wo{thinking otherwise} and, to a large extent, I 
abhore most of \wo{otherwise writing}. The barbarians overlook, in 
their tryumph, that the conquered traditions had developed the ways 
for accomodating all aspects of experience and that it did it, 
exactly, in constant confrontation with life. They were not 
neglected, they might only have been invisible at the places which 
attracted the eyes of newcommers. 

\pa
The power of the \wo{metaphysics of actuality}\footnote{I am 
reserving the word \wo{presence} for something else, so I have to 
translate this phrase in the above way.} needed undeniably a correction. 
But this can only be a correction of attitude. As long as one 
attempts \wo{to think}, one will do it in familiar categories. And it 
is hardly to be expected that, because of the fall of Rome, one will 
stop thinking. The difference may concern the significance one attaches to 
one's thinking and its results. And, not least, to the results of 
other's thiking. If everything is merely a question of interpretation 
then, certainly, most of the texts which one would like to subsume 
under the \wo{metaphysics of actuality} can, actually, be interpreted 
in different direction. They certainly are involved into thinking in 
terms of actuality but the point is -- what thinking is not? It 
remains to be seen what the hoped for \wo{thinking otherwise} will 
have to offer. The answer is: \la{nihil novi}.

\pa
But point taken. Reflection has to watch its steps, that is, it has to 
carefully observe what it is reflecting over. It has to always keep 
the clear distance to its object, in order not to confuse it with its 
reality, in order not to confuse itself with the omnipotent power of 
control. This is, exactly, the difference of attitude. And, for that, 
possibly a nonexistent one because, although some earlier thinkers might 
certainly be accused of nourishing such a belief, many could not. 
Just like the texts enscribe to an extent the conclusions, so does the 
choice of focus determine, first, the choice of texts and, then, the way 
of reading them. The attitude may be only assumed in the other -- it 
can be controlled, if at all, only in onself. Especially, when one 
reads. Discovering an attitude of \wo{fascination with the actual} in 
all the texts, the possible question is about the attitude of the 
reader. 
In short, it is not a question of \wo{thinking, speaking, writing 
otherwise} but rather of {\em reading} otherwise or, perhaps, reading 
other texts. \wo{Reading otherwise} would simply mean to look for what 
one cannot find, to look for the traces of thoughts which one thinks 
are not there. Eventually, one will find them, but this requires a 
different attitude of reading.
\thi{Metaphysics of 
actuality} is not all of the traditional metaphysics, in any case, 
not all of the traditional philosophy.


\pa
Critique of \thi{metaphysics of actuality} (Derrida, Heidegger) wants 
to show something more, \co{non-actual}. Critique of objectivistic 
system of thought of Hegel by Kierkegaard opposes to it the experience 
of an individual, not just a common experience, but the fundamental 
experiences, elaborated through all the depth psychology. 

My point: the two are the same! The feeling one may get of 
existential relevance of Derrida, and certainly of Heidegger, is 
based exactly on that the personal/individual/etc. comes into the 
world of \co{actualities} through the \co{non-actual}. 


\pa
Follow Heidegger's, Derrida's critique of \thi{the metaphysics of 
actuality}\footnote{I will reserve the word \wo{presence} for 
something different.} but not his aggressiveness. There is nothing 
wrong with actuality, it is our way of conceiving the world. There is 
only something wrong with positing it as the only way of treating 
everything, all aspects of experience\ldots 

We are not scared of concepts as Heidegger, and later post-modernists, became.
We consider this fear unnecessary, in fact, impossible in length, sometimes
comical, sometimes hurtful...

\ad{\thi{The linguisting turn}\ldots} 
\ldots or shall we rather say, \wo{a linguistic twist}?  
One has always known that words and language
can not capture reality.  Not only there is no final, most adequate
and true name of God.  Even trying to communnicate most simple facts
and situations, when using words, we are forced to distribute the
responsibility between the one who is speaking -- to express himself
as clearly and understandably as possible -- and the one who is
hearing -- assuming that he not only knows the language, but also has
the sufficient background of experience to grasp the meaning of what
one is saying.  This sounds, perhaps, a bit exclusive, even elitistic,
and so it is supposed to sound.  \citt{Who hath ears to hear, let him
hear.}{Mat.  XIII:9} Success measured exclusively by publicity, by the
sheer number of receipents is always -- {\em always}!  -- inversely
proporitional to the value of the thing.  In the domain of spirit,
there governs a law opposite to that of mass culture, namely, the
more, the worse.  Even deep things, to the extend they are spread
around and received by many, become superficial, flat and low.

\wo{The sky was blue and the sun was shining.} is as
inadequate a phrase, if we try to measure it by some some objective
conformance to reality, as any name has ever been for expressing the
essence of God. When we speak, and even more when we write, we assume 
that the reader fulfills some basic conditions necessary for 
receiving the message. If, on the other hand, we do not know to whom 
we are speaking, if we, perhaps, assume that we are speaking to 
everybody, irrespectively of his knowledge, background, level of 
intelligence, and what-not, if we, for instance, are merely interested 
in selling the book to indefinitely large and, consequently, also 
indefinitely amorphous masses, we can not avoid getting confused. In 
particular, if we forget that any utterance, and also any text, is an 
address from somebody to somebody, if we abstract the text from this 
fundamental relation of intension and meaning, we can not avoid 
getting confused. 

\pa
One feels proud having recently overcome all these difficulties by
the \thi{discovery} that, as a matter of fact, there is no reality 
beyond the words, there are no human subjects beyond the text. As any 
thesis, so also this one can find innumerable reasons and 
justifications. I certainly won't spend time on reviewing them all. 
This inhuman abstraction of \thi{textuality}, this invention of 
intellectual despair over the impossibility to ensnare reality into 
words, does not deserve much attention. Yet, its wide 
popularity makes it hard to simply ignore it. The barbarians entered 
Rome and settled down, but their ways do have reprecussions\ldots

\pa
Reasoning, and understanding in general, is a way of using, inventing 
and arranging \co{signs}. Words are not the only possible \co{signs}, 
but let us be charitable and take all this \thi{linguisticity} in a 
very general, probably over-general, sense of semiotics. Without an 
adequate system of \co{signs}, there is no understanding. Ok, but 
there is no such thing as \thi{understanding}; there is only and 
always only understanding {\em of something}. What this something is, 
is highly vague, and the \thi{linguistic turn} is an attemt to 
substitute for our lacking, or in any case \co{vague}, understanding 
of what this is that we are trying to understand, something which 
apparently is much more precise, well-defined and hence manageable -- 
the very system of \co{signs} itself. Cetainly, a smart intellectual 
twist. However, the fact that something is \co{vague}, that we can 
not put the finger on it, does not mean that it does not exist. 

\pa Many \thi{linguistically twisted} will agree.  There exists
something but our way of speaking, the language and words we are used
to, are inadequate to capture it.  Therefore, we must \thi{speak
otherwise}, \thi{write otherwise}, invent a new language.  The ghost
of capturing the reality into words is peeping in through all the
holes.  What else, after all, can an intellectual do than to market
his words?  
%Among the most prominent (but by no means the only) examplars
%of such \thi{writing otherwise} one can probably mention late
%Heidegger and Derrida, perhaps, but only perhaps, late Wittgenstein. 
%These are, as usual, accompanied by a tremendous herd of noisy epigons
%who \thi{have seen the light} and start \thi{bubbling otherwise}.

%\pa 
If the reality which one senses beyond the inadequate words is
really a fundamental aspect of human existence, then is it reasonable
to assume that we had to wait thousands of years, until the day today,
to discover the need for \thi{speaking otherwise}?  Sure, one can
always patronize the past generations pointing out their mistakes. 
But such patronizing is but an aspect of a positivistic faith in the
absolute character of progress which we are not willing to take
without any reservations any more, if we at all are willing to accept
it at except, perhaps, for the sphere of technology and civilisation. 
Perhaps, the men of Reneissance, or of Enlightment invented
particularly obscure ways of speaking about some aspects of human
existence, ways which, due to their low value, became particularly
popular.  But there were wise men, too, in those times, weren't there? 
OK, they were all led astray.  What about the men of Middle Ages?  Of
early Middle Ages?  Heidegger would say, and what about the Greeks? 
Indeed, what about them?  But we do not have to follow the
idiosyncratic, not to say comical, attempts to revitalize a dead
language.  What about Indians, Jews, Chineese?  I do not want to boast
with false modesty, but I find it hard to belive that to address the
most fundamental issues we need an entirely new way of \thi{speaking
otherwise}.

\pa If all history of human culture, in any case, of written human
culture, has been perverted by the inadequate language, then what
makes us, some of us, today read Plato, Aristotles, Lao Tse, Bible,
St.  Augustine?  For my part, I can say for sure that it is not a need
to find their mistakes but, on the contrary, the conviction, and I
should be able to say -- the experience, that they do have something
important to tell me.  I doubt that many others read them only for the
sake of intellectual curiosity, deconstructivistic exercise or
academic career.

\pa
OK, so old texts may, sometimes, contain some valuable insights. Or, 
shall we only say, some valuable words? Do these texts contain 
anything \thi{in-themselves}, anything which is not read into them by 
the reader?

If text is entirely open to the arbitrariness of the reader and the
indeterminacy of possible interpretations, if it has no intension, no
truth it tries to communicate, if any system of signs is but an 
arbitrary invention with but an arbitrary relation to what possibly 
might be lying outside it, then one might almost agree that
writing should not involve any attempt to make it understandable and
accessible -- {\em what}, in such a case, should one try to make 
understandable?  Probably, one does even better resisting any such
attempts and we have been exposed to many examples of such a 
resistance. 

Unfortunately, or rather fortunately, all texts have been written by 
humans with some intensions. They may communicate these intensions in 
better or worse ways, in more or less adequate form, in more or less 
understandable and plain manner, but intensions are there. And if one 
does not like the word \wo{intensions}, in particular, any 
\wo{intensions of the author}, then it will still suffice that, having been 
written by humans, they contain human expressions of human experience 
and life. This, at least, is more than sufficient for me and if it 
isn't for you, then you are free to keep dissolving words, expressions 
and, eventually, insights in the mud of \thi{intertextuality}.



No private language -- so, no private thinking?

language -- changing, ok, but still not changing and translatable

New problems? New solutions? What? -- check it out, it has all been 
there\ldots

\pa A completely different, even opposite, dimension of the
\thi{linguisitic twist} is analytical philosophy.  A name which seems
more adequate to me would be something like \wo{logical analysis of
linguistic behaviour} because, as a matter of fact, I have found
extremely little and also extremely meagre philosophy in this camp. 
This, however, is due only to my biased and misunderstood conception
of philosophy, so let us not quarrel about the names.

The strength and, indeed, the visible vital force of this camp is
based on the fact that it tries to focus on more or less identifiable
and well-defined problems.  The resulting insights and proposals have
often much intrinsic value and I am last to deny that.  I am a bit
more unsure what this value is and to what purposes it could be used,
except for possible applications in cognitive science, artificial
intelligence, linguistics and, perhaps, legal theory and new design of
dictionaries.  Even if only of pragmatic -- shall we say, scientific? 
-- relevance, so, being usable, they also represent some kind of
value.  I want to say this with all due emphasis because, having said
that, I think we can leave this camp for itself.  Hopefully, with
time, it will, as any scientific community earlier in history,
establish its identity independently from philosophy and then continue
resolving all its specific problems and particular issues with the
undisturbed peace of mind to the best of whoever will want to use
them.
%
\sep 
%
\pa This is as much I have to say about all the linguistic twisting
and I leave writing of the voluminous binds of history and analyses of
the involved \thi{problems}, \thi{questions} and -- yes!  --
\thi{solutions} to the competent scholars.

\ad{Genealogy}
Not in time, but from virtual to concrete $\sim$ gradual 
differentiation

Then: refinement of systems of things/concepts; these consist of 
simultaneous $\sim$ equiprimordial \ldots correlates/aspects (find a good 
word)


\ad{Notation}
\begin{itemize}
\item ``\herenow'' \impl ``here and now'': dissociation of aspects
\item ``\co{recognition}'' \impl ``\co{re-cognition}'': rozpad into 
more detailed parts.
\end{itemize}
ntuitions should be preserved -- for more detailed understantding, one 
should first keep in mind the possible presence of another notational 
variant, and then check it \ldots




} % end the main \noo{%END
