
\section{A Few Conclusions}

\subsection{generals...}

\pa We have heard.
God is dead!  Subject is dead!  Representational thinking is dead! 
Metaphysics is dead!  And philosophy is dying dissolving in the interminable
analyses of scattered thoughts.  Does not our culture,
and philosophical understanding in particular, become all too
overfilled cementeries?
Perhaps a bit surpirsingly, \thi{philosophers} themselves seem
quite alive, although some dared to warn even them:
\citet{the end of the philosophers who did not honor God was
nothing other than to perish in their vanities.}{SeekGod}{ III:40} Too many
deaths in vicinity may eventually affect the landlord, too.


But are we not all too fast in conflating the death of our
misunderstandings with the death of the things themselves?  Do we not,
in all the hullabaloo about the power of language and its identity with reality, forget
the persistent and insistent presence of some fundamental words which
irresponsibly refuse to die; the refusal which then should mean that
the reality of these words, after all, has not died either?  For my
part, I have seen many terror acts, many an attempt of assasination,
some fewer attempts of open killing (even with a hammer!) but, as yet,
no actual casualty.  Can it be, perhaps, because I do not follow the
news?  Or is it because I did not study thoroughly enough, did not pay
sufficient attention to the subtle distinctions which, in fact, make
the whole difference? 

\pa Deaths of idols shatter the foundations but do not affect the
origin. God, or let's just say, the ultimate
reality of the invisible origin, is not merely something to be left
for abstract considerations, even if words of a text
can never do much more then, communicating such abstractions, suggest 
that there is something beyond them.  It is, in
fact, the most immanently present element of life which not only does
not contradict its concreteness but, in fact, is its very
foundation. Unfortunately, or rather fortunately, this concreteness can be found
only by an individual, everybody has to find it anew himself -- when lifted to
the level of socio-cultural phenomena it is bound to dissolve, especially when
these are almost by definition dissociated from the religious dimension of
cultural life. And neglecting or just missing this presence, all the
search for the concrete facts of actual life finds only sterile
products of its own conceptions. 
%This presence has always found an expression in human being and 
%conduct of human affairs, as well as in language.
Neglecting it, we simply reduce the sphere of our experience. Pretending that it
is not there, or else that it {\em really} is something different, perhaps
simple, perhaps complicated, but explicable and possible to grasp in the
plainly visible concepts and categories, is to thoroughly falsify this presence, is to
veil and confuse the quality of its experience. The greatest sins against
humanity are comitted in the name of humanity. And likewise, the greatest
mistakes are comitted in the search of truth.
%

It seems that all the casualties were announced with reference to the
culture, by sociologists or philosophers with a strong sense of the cultural, if
not directly sociological context. Eventually, even existentialism turned out to be
nothing but an expression of a time of depersonalization and destabilization...
Strangely enough, we have been so infected by historicism that one can hardly
imagine that a philosopher is anything more than an expression of his times,
that the world has been old since its beginning.\noo{\la{Mundus senescit},
  \wo{The world has aged}, says Gregory of Tours in the VI-th century, and
  likewise, some 1000-1500 years before him, the preacher observes: \citefi{The
    thing that hath been, it is that which shall be; and that which is done is
    that which shall be done: and there is no new thing under the sun.}{ Eccl.}{
    I:9}} Our exposition may seem rather antique and certainly out of touch with
\thi{our times}, with \thi{our historical context}, with \thi{our current
  issues}.  But strangely enough, surrounded by all that is \thi{ours} and
specific to \thi{our times} (and such \thi{ours} involve always the element of
\thi{{\em only} ours}), we still happen to read Heraclitus, Parmenides, Plato,
Bible, Vedas... The celebrated historicism suffers from a thoroughly false sense
of history. Only superficiality viewing history as a mere progress can restrict
itself to what has been and no longer is and to what is now but has not been
before. If the sway of history were so total, there would be no history but only
the current now, the mere actuality, dissociated from the incommensurable past.
The mere fact that we can understand somethig from the writings of Herodotus,
Caesar or Josephus Flavius is itself the evidence that, all social and
technological changes notwithstanding, not everything is so completely new. 
Underneath all social systems, behind all the differences of habits, customs and
norms, throughout all the historical epochs, one can not fail to discern ever
the same aspects of human existence. That is, one can not fail unless all one
does is trying to catch up with one's time. 

  
\pa But was {\em this} a philosophy?  Hardly, with all these mystics,
quotations from the Scriptures, with God, or rather only some one. 
Theology then?  Not really, for would it be Protestant or Catholic? 
Hardly {\em any} orthodoxy to attach it to.  A mystical hymn?  A tribute to
life, the {\em whole} life?
\noo{I am,
indeed, after the mystical value and quality of life, of {\em whole}
life, might it never find definite expression in an actual experience. 
But no, too many concepts and too little of particular way, advice.  What
is it then?}  Perhaps, just an ecclectic collection, a bit of everything? 
Perhaps.  It is, in fact, entirely up to you to decide.   
%
\noo{ I am not a
mystic.  \wo{Mystical} are called \citt{those supernatural acts or
states which no effort or labour on our part can succeed in producing,
even in the slightest degree or for a single instant.}{Catholic
Encyclopedia} I am not after any \thi{acts or states}, I am not after
any form of -- be it incomplete or complete, full, exstatic or
deifying -- union.  The absence of anything about the transition from
\sch\ to actual \co{incarnation} of \co{love} should indicate clearly
that, even if I have my ways, I do not consider them to be of any
general interest.  }
%
\noo{
\pa God, misconstrued as a deistic concept, as the First Cause of all
the calamities of this world, as a potent, all controlling boss is,
indeed, dead.  (But was he ever alive?)  God who has no relevance for
human life is, fortunately, dead.

The subject, the petty Ego, or else the subject as the all
determining and self-constituting axis of the world is dead, too.  The
subject, whose transcendental omnipotence or else rational impotence
lost all contact with the actual experience of subjectivity is,
fortunately, dead.

Philosophy based on such concepts, philosophy without least appeal to
human reality is also dead.  Cross over its grave and solemn Amen.

\pa Consequently, the subject, the {\em human} subject, too, is pretty
alive, even if vastly confused and not quite well.  Deaths of idols
shatter the foundations but do not affect the origin.  Open Jung on
almost any page and you will read about the inflation of such a
subject which is swallowed by the confrontation with the archetype;
the inflation, however, which is exclusively the result of subject's
inability to cope, its inability to integrate the contents of
experience it can not control.  Confrontation with chaos, if it is not
to end in cacophonic confusion, babbling and, sometimes, insanity,
requires an attitude -- a mere description does not help the least any
more, because the event brings man face to face with his Self.  It is
the most concrete, the most personal event which has been lived by
numberless persons since the dawn of humanity.  \citt{Although the
account is common, most men live as if they had a private
understanding.}{Heraclitus, B2 [This may, and probably does, refer to
the patheistic ever-living fire, rather than human \thi{account}.]} In
the current philosophical context, this only means the final, healthy
death of the idol of scientism, {\em pseudo}-rationalism and
impersonal objectivism.  Death which, starting with Nietzsche and
Kierkegaard, has taken some 100 years.  The worthy work of
deconstruction has been carried to its successful end, even if for
many deconstructionists themselves it is still unclear what exactly
has been deconstructed, and even less clear what possibly might be
constructed from the scattered pieces of the dismembered corpse.
}
%

If philosophy is \thi{love of wisdom}, then it
does not live up to its name by glossing over the fundamental aspects
of human life in order to deal with more palpable and petty
matters of most actual problems.  In fact, by such a glossing over, by
insisting on some undefinable \thi{reason} with its capacity for
definitions, 
%(which \thi{reason}, by the way, is much more mysterious
%entity than the invisible presence) 
it ceases to be philosophy and
becomes a \ldots science?  a field for academic dispute?  a reference
frame for quarrels, that is, \thi{rational} discourse?
%
\citet{A quarrel between philosophers should be taken as seriously as 
a philosophical argument between two bricklayers.}{SisterI}{ IV:39} 
%Shall we dare to say that, as a matter of 
%fact, it actually {\em is} taken as seriously by all, also 
%philosophers, except the quarreling ones \ldots 


\sep


We have heard that: from neoplatonism, through Greek Fathers (and heresiarches),
through Eckhart and Suso, Molinos and their pietistic counterparts..., it leads
to subjectivism and quietism, to mere internal development in alienation from
the others and even from the Church...

\noo{We said that it is not a book for everybody. Perhaps even it is only book for
some less healthy ones, who are no longer able to rest satsfied with the average
amount of direction and advice, buit need a more personal relation to that which
underlies also the common truth of the church.
}

We recognize indispensability of \co{visible} means and signs as we claim the
co-presence of all levels of Being in every \co{existence}. Nobody can do
completely without any of them, but some will need or emphasize one more than
other. Indispensability does not mean primacy and we are merely maintaining that
the higher and deeper things allow quite a freedom among the lower ones. 


\subsection{Directions...tactless}

\pa
So it was yet another tactless attempt to tell others how to live...?  But what
makes such attempts tactless? The imputed sense of superiority, I assume. But
superiority is tactless only for one who feels inferior, who meeting any clear
suggestion or command, shrinks immediately into his own defensive insecurity.
Inferiority, just like \citet{shame is pride's cloke.}{BlakeMarr}{ Proverbs of
hell \citaft{EvilFaces}{, p.201}} A healthy (and this means, \co{open} and
\co{humble}) man meets only his equals -- superiority may only concern this or
that \co{aspect}, but never the value of the whole person.

So, perhaps, it is tactless because the ability is always assumed but never justified,
because as a matter of fact nobody actually can tell anybody anything meaningful
about life. I leave this assumptions to those who want to insist on the
tactlessness of my enterprise.

It seems to me that philosophy is worth only as much as it tells something about
life, as it is existentially relevant. But telling anything about life can only
in special cases be distinguished from telling how to live. (The special cases
are exactly all the uninteresting ones, for instance biology, but also sociology
or psychology -- all more or less failed attempts at getting scientific about
life.) There is however a difference between telling how to live and preaching,
and I hope that I did not overstepped the border between these two. A preacher
admonishes, lists the reasons and tries to convince, himself convinced (or
perhaps not?) that he possesses all the sufficient reasons for his
admonishons. Telling how to live may be preaching, but hopefully it may also be
showing images -- not proofs but images, not sufficient but necessary
reasons, not forcing arguments but appealing stories


\pa On the other hand, one might perhaps hear: \citet{\ `But give me
  directions.' Why should I give you directions? has not 
Zeus given you directions?}{Epictet}{ I:25}
Imprecise, avoiding many crucial problems -- that is, idstinctions. Indeed...

We do not, people
know how to run their lifes and arrange all the details. Looking for detailed
plans for ethical etc. acting in all situations is as ridiculuous as it is petty
burgeoise...


\pa
\Yes-\No\ -- isn't it an abstract, in any case exaggerated opposition? Most
people are somewhere inbetween. So what is it good for?

These are not abstract impossibilities! We do describe but, indeed, we do not
care about numbers and statistics -- they do not count! The extreme alternatives
may look improbable, but they are possible. The fact that only few persons
realize them in such an extreme degree does not deprive them of any reality...
They are the eternal dimensions of every human existence...

Desiring good, \co{actually} willing it is only a \co{reflection} of the genuine
goodness, a \co{concrete} expression of the ontological \co{founding} in the
\co{origin}. But it has few specific rules which could be formulated once and
applied everywhere. In the last instance, most things belong on the context --
the only thing which need not is the genuine desire of good. Abstract and
contentless as it may sound, it is in fact the most \co{concrete} possibility of
\co{existence} because it is not the mere \co{actual} declaration of a voluntary
intention, but the admission of one's own insufficiency and fallibility. In more
\co{precise} terms, it for the most means only: try to avoid evil...

Avoiding evils is more difficult than one would expect... \citt{If a man shuns
  them when he sees them from afar, before he is entangled in them, it is by
  God's wisdom and forethought that he is protected from them}{Hermetica,
  (ascribed to) Hermes Trismegistus, Letter II to Asclepius [The Many Faces of
  Evil, p.24]} In fact, man can shun them also when he does not see them, though
this shunning may be simply the fear -- one would say, irrational and ungrounded
-- of getting caught and entangled by them...

\noo{
We do not defend irrationalism, but we do not deify reason either. In our
hierarchy reason is -- if we identify it with the reflective analysis, the level
of actuality and immediacy -- the lowest stage, furthest removed from the divine
origin. We thus do oppose all rational totalitarianism which, starting with
Aristotles, if not Plato, has been polluting most of philosophy claiming that
contemplation and, in fact, thinking, dwelling on concepts and ideas, brings one
closer to God or, as those who think they can do without wish, happiness. The
usual tendency is rather the 
opposite, and we point out that it is not only due to misunderstanding but also
due to the nature of conceptual reasoning as ... exactly the remotest from the
origin -- remotest, but the mere immediacy -- layer of Being. But we do not just throw
reason away, together with the thinking subject, and what not. On the contrary,
we recognize its indispensability and also its useful function. We only
emphasize that this function and indispensability have virtually no existential
relevance.
}

% It is misunderstood individualism of protestant falvour that some matters are of
% purely private character and that their public discussion is an offence, shame. 

\subsection{???}

\pa
Book I descended in the order of ontological founding; Book II ascendend in a
way analogous to neoplatonic ascent, although in a less mystical
manner. But admiring the neoplatonic games of negativity, we find the idea of
ascent an incomplete truth.  
For true return is not the return to the \co{origin} but {\em from} it, the
return from the mount Carmel down to earth. As already 
Trismegistus taught, the soul's ascent must be followed by the renewed descend
back to the world it has left. And so, admiring
Cusanus and his reinterpretation of Psudo-Dionysius, we can not quite accept the
scepticism towards the idea of analogy which is one of the most genuine
philosophical reflections of this ancient insight.\footnote{J.~Hopkins comments
  on Cusanus' dislike of \thi{analogy} in the  
  introduction  to his edition of De docta ignorantia, The Arthur J. Banning
  Press, Minneapolis, 1981 [after Bog-Nicosc, J.Miernowski, p.68[ftnt.31]]} 
\co{Concrete founding}, is analogy re-cast in our existential context or, if one
prefers, the ultimate return from Hell and Heaven to Earth, where is our place. 


\pa
Our insistence on the fact that \co{I am not the master} might give some
associations with 
what nowadays goes under the name of the \wo{postmodern} turn to the other,
otherness, (\citt{The real face of postmodernity ... is the face of the
  other}{D. Tracy, Theology and the Many Faces of Postmodernity, in Theology
  Today, vol.~51, no.~1, April 1994}) We certainly would not like to be
associated with this word which carries a lot of other connotations (no pun
intended -- what for?). But in this respect, there seems to be a point of
agreement, albeit a very slight one. Subject, this epitome of moderninty
according to the preachers of the postmodern gospel, is equally central in our
exposition and, we would maintain, ineradicable. \wo{One may believe that the
  original time is ecstatic, yet one buys oneself a watch.} We may maintain that
the depth of our being hides \co{invisible} powers which never become
\co{actual}, and yet we may still insist on the necessity of remaining a sober
subject occupied with the \co{actual} tasks and problems.
...
The otherness of postmoders makes an impression of the Cartesian stick which one
bends unreasonably far to the left, only because before it was bent unreasonably
far to the right. But to at all confront otherness, there must be somebody doing
that, and dissolution of a subject (or as we'd rather say, of a person)
eradicates otherness as well -- without the same, there is no other. 

The attempts to posit otherness as the new site of true substantiality (the
language is, of course, very different, but it is the role of otherness),
square very badly with the proclaimed dissolution of all substantiality, all
identity. Of course, nobody wants to keep the identity modeled in the
\co{objective} manner of \co{actual objects} and things. But relativity of such
an identity has been a fact for quite a while. It does not, however, make it
unreal or unnecessary (it always will be) -- only relative. 



\noo{

\pa Perhaps, it becomes a method?  I guess, one could abstract aspects
even of this work and try to call them a \wo{method}. The overall 
intuition is that it is not true that \citt{the only way to mediate 
between diversity and unity is to class the diverse items as cases of 
a common essence which you discover in them.}{W. James, Essays in 
Pragmatism, I. The Sentiment of Rationality, p.6} On the contrary, 
unity is only expression of \co{virtual} orgin\ldots
\begin{enumerate}
\item Identify \equi\ \co{aspects} -- things which not necessarily,
but factually, form a \nexus\ and can not be dissociated from each
other 
\\ 
\thi{being an \co{aspect} of} is, often, a symmetric
relation!  -- sometimes, we may have a word for the
\nexus, the unity of \co{aspects}, and then the relation is
asymmetric.  
\\
distinctions arising from a \nexus\ are `circular' [I said to say it 
somewhere]
%
\item Identify how new \co{aspects} emerge from the
nucleus of earlier ones 
\\ 
Many \thi{natural}, \thi{common sense}
concepts -- given by limits, perhaps at different levels 
%
\item \thi{inversions}
\item
No sufficient conditions/reasons, only necessary ones \ldots
\item
Everything belongs to one of four (perhaps more) levels. Elements of 
existential relevance can be usually viewed from any of them.
\\
It is only at the highest level that all such elements may form a unity.
\end{enumerate}
But all this is no method -- it is just because we see everything as 
\co{founded} on the \co{virtuality} of the \co{origin}. I find this 
way of thinking not only helpful but also gratifying. 
%But I would be as vastly surprised if everybody else did, as I would
%be pleased if at least some did.

\thi{Method} is but a reflection of some form of understanding, a way
of expression and organization of {\em this} understanding. 
But understanding only seldom, and deep, genuine understanding 
never results from an application
of any \thi{method}.
} %\end \noo{...
%

\tit{Rationalism}

\pa Yeah\ldots Meaning what?  If sufficient reasons,
then I am certainly out in the blue of irrationality.  But, but! 
There need be no conflict with anything written here, if only we take
rationalism in a bit more generous sense of \thi{knowing that
everything has limits of its validity} and, in some cases,
\thi{knowing where such limits go}\ldots It has little to do with
causality, resons, and nothing with sufficient reasons which simply do
not obtain, except in the narrow sphere of actual experimentation. 

I know people who do not like me.  They are stupid! -- I am quite
likeable and friendly guy.  But do not they have reasons for not
liking me?  Sure they do \ldots only that these are not {\em good}
reasons.  If we reduce rationality to reasons, then everybody is
rational; if to sufficient reasons, then nobody is. And so, 
\thi{rationalists} keep trying to say what should and what should not 
count as \thi{good} reasons. 

One may certainly try to narrow down the notion of rationalism, make 
it so precise that
it comes into conflict with everything one does not like, but are we
impressed by such attempts at essentialistic definitions?  Rationalism
has only to do with reason's finitude, i.e., its partiality,
\la{ratio}, which means simply a slight \ldots humility.

The most rational attitude is not to absolutize this
partiality, not to relay unreservedly on one's reason, that is, on
one's finite and explicit understanding as the only source of
judgments and actions. 
For there is no such thing as \thi{good reasons} {\em in general}, 
\la{in abstracto}, albeit there may be many goo dreasons in any 
particular situation. 
\citt{[I]t is well to acknowledge the demons
formally only when reason dictates, and reason may not dictate our
doing so in every case.}{Celsus, On the True Doctrine, X
[reconstructed translation by R.J.Hoffmann]} Isn't it a perfectly
rational statement?  Sure, dealing with matters \thi{rationalism}
would most gladly label \wo{irrational}? But he explicitly links them 
to the judgment of reason. Was his reason {\em so} different from mine? 
Do you really mean that
demons are unreal?  Because they are \thi{projections} of a
\thi{subjective psyche} not worth rational attention?  That, instead, we 
have obsessions, suggestions,
projections, mental disorders, schizophrenia, pathological
depressions, neuroses, crowd hysterias, and what not.  We have as much
\thi{belief}, or \thi{disbelief}, in therapy, as primitives had in
their shamans and medicine-men; statistically, there is probably
little difference either with respect to \thi{beliefs} or to the
results of treatement.  The main difference is that we are doing it
\thi{rationally}; another, that a bushman who \thi{lost his soul} was
terrified, while a person labelled with a diagnosis becomes pacified. 
Less violent outbursts? Even if, violence went somewhere else, but it 
still stays with us. 

Surely, there must be some limits to the irrationality of
\thi{rationalism}.  The creed of our rationalism would be:
%
\begin{quote}
to accept any statement or position admitting only its
limited validity (and, whenever possible, recognizing also its actual
limits) -- and this is self-applicable, i.e., even this very statement
has only limited validity.  
\end{quote}
%
Rationalist can always only say \wo{I mean this but I may be wrong.}
Yet, it is one of the most common and prevalent need \thi{to be right}
-- no \thi{rationalist} ever managed to avoid falling prey to it, for
he gets scared by the arbitrariness and relativism inherent in his
choice of what counts as \thi{good reasons}.  And the more absolutely
right and \thi{rational} I am trying to be about something about which
I may be wrong, the more absolutely wrong I am -- worshipping idols I
have created.

The limit of rationalism is the sphere of the \co{spiritual} relation
to the \co{indistinct absolute}, to \co{nothingness}.  It is not
opposed to empiricism, albeit empiricism devoid of its ultimate atoms;
it is not opposed to pragmaticism (this American variant of German 
phenomenology), albeit pragmatism extended with the
{\em constant} reference to the unity of human life; it
cherishes varieties of religious experience, but not only as some
alternative scattered bits and pieces but as so many unavoidably
different concrete realizations of the same.  It is not opposed to
Platonism, albeit Platonism with ideas clearly separated from concepts
and abstract universals.  It is not even opposed to psychoanalysis,
albeit only its Jungian version.  It is not opposed to theology,
albeit theology in the generous sense of interest in and fundamental
relevance of matters of spirit, which are the same as the matters of
human relations to the invisible Godhead.  If all these \thi{albeit}
seem all too grevious, then it is indeed opposed to all.  But in fact
it is opposed only to narrow-minded \thi{rationalism} which,
distributing the labels \wo{rational} and \wo{irrational}, opposes
itself to anything that does not fit into its flat language of
causality, dissociated categories and plainly visible distinctions.

\tsep{Pantheism -- went to end of Book I}

\pa I have many times used expressions like \wo{\co{nothing}, that is,
everything}.  Since \co{distinctions} are made in, that is, from
\co{nothingness}, what we find has already been there.  Isn't it
pantheism, then? 

No, it is not.  There is a seed of truth in pantheism, too, for it
only tries to express the fact that every thing is holy.  This,
however, does not mean that every
thing is holiness.  Holines of every thing is \co{concretely founded}
in holiness of your attitude.  If you are not holy, such an expression
may have some sentimental appeal, but no real truth content.

\wo{Everything} is but the expression we find for \co{nothingness} in
the differentiated world of feelings and concepts, eventually, of
\co{reflection}.  But this \thi{everything} is not the totality of
things, is not the sum of all things.  \co{Nothingness} is \co{above}
them, it in no sense \wo{belongs to this world} but is, in fact,
thoroughly \co{transcendent}.  That our \co{distinctions} derive their
\co{objectivity} from the \co{One}, does not in any way mean that they
sum up to the \co{One} -- they never do, because they never sum up to
anything.  Pantheism would say that they do sum up. Personally, I do not know 
of any text which, if only taken seriously, without all too 
simplifying yearnings of \co{reflective precision}, without all too 
many \thi{if you say this, you {\em must} also mean that}, is 
patheistic in this naive form.

The \co{spiritual} contact with the ultimately \co{transcendent One}
happens at entirely different level than does the \co{reflective}
contact with things, objects and facts.  And yet, it does \co{found
concretely} the \co{immanence} of the \co{One}, its thorough
\co{presence} in every thing.  It recovers the inspiring unity in the
midst of the charming, and sometimes cruel, manifold.  There may be
some who will shrugg at this apparent paradox, at this unacceptable
contradiction, at the mere \thi{subjectivity} of this \la{coincidentia 
oppositorum}, but they will shrugg at most other things I have
said, so let them keep shrugging.


\tit{Truth (and pragmatism)}

\pa
We are left with ... \co{nothing}. So? Do what you want? Of course,
not. \co{Nothing} means everything. Yet, we are left with a kind of missing
foundation, for except for the \sch\ which happens \co{above} us anyway, the
rest seems completely undetermined and open. So, no foundation for science,
objective truth, truth...?

Indeed, with respect to all the \co{visible} things, except for the imperative
of \co{non-attachment}, we are left with a kind of open pragmatism: try and hope
for the best. Indeed, there are no \co{precise}, \co{visible} rules ensuring
anything -- there are no sufficient conditions. Every field has its own local
conditions and there is no greater reason to look for the common basis for
quantum physics, brain biology, computer programming and the most recent trend
in literary criticism, than there is for trying to force people from different
cultures to live next door to each other and share the same social code. 

But does it mean that there is no truth or, perhaps, only some machiavello-pragmatic
criterion of appropriateness for the achievements of the goals? Of course, not,
for the whole issue concerns the choice of the goals and even if most are chosen
as if semi-automatically, sub-consciously, the issue nevertheless remains and
delegates the whole pragmatism to a lower level of reflection.

Truth is not \thi{what works}, it is rather a corrective. We can stay for the
moment in a field of Wittgenstein's practice, rather than James' pluralistic
universe. We have never heard the last word concerning what following a rule
consists in, but here we do not need it. (It must probably remain a mystery,
just as creation.) A rule, or following a rule (in the way
Wittgenstein analyses it) is a good approximation to what we mean by a
\wo{corrective}. It is that which remains silent as long as everything is ok and
which protests in the moment something is not. It is not the fact, or event, of
being corrected, it is not the content of the correction but it is that whence
the correction originates.

\imm In the trivial case of facts and situations, these constitute the basis for
correction (provided that all agree on what counts as facts). Not respecting
facts is, well, stupid because it goes against their truth.

\act Education and uprising, acquired social norms, small techniques of playing
games and solving minor problems, all that may fall under the name of rutine, or
as Wittgenstein would say, following the rules, may function as corrective in
our \co{actions}. It is \co{ego} and its luggage...

With respect to scientifc theories, different things act as correctives. As Kuhn
has shown, there is a large element of sociological mechnisms influencing the
success of a theory. But of course, at the bottom of it, there are facts of
experience, the results of experiments which are accepted as valid correctives
-- theory is judged true if it conforms to their truth. There is, of course, an
abyss of indeterminacy with which the champions of narrow-mided rationalism and 
even more narrow-minded scientism will keep fighting for ever. 

\mine 
Agreement as to the truth at the lower levels is conditioned by sharing some of
the correctives at this level. Which facts count as facts, which among them as
relevant ones, what is significant and what not. All such questions at the lower
levels may lead to insurmountable conflicts if the personal correctives of the
involved people show sufficient divergence.

Which values do I accept... and which do I live? 

Conscience may act as the corrective, but it must be sensitive. One may silence it

\inv It is only the truth of the \co{existence} which may be a corrective at this
level: being \co{confronted} and hence \co{not being the
master}. This is the last and only \co{absolute} truth, which \co{founds} the
idea of truth at the lower levels, which makes it impossible to discredit the
idea of truth for the sake of some fashionable perspectivisms, sophisms,
relativisms, and the like. But as we have seen, even this truth can be ignored
and replaced by falsehood of \No. 

\pa
Thus truth, being a corrective, is indeed involved in our activity but not exactly in
the way pragmatism would like to have it. Thinking, also theorizing, may be our
activity which needs a corrective like anything else. It does not show only
\la{post factum}, as that which turned out to be the most successful. It will in
most cases announce its objections before, or at least, during our actions. And if
we do not take them into account then\ldots well, who knows? We may still live,
perhaps even be successful and satisfied -- for truth can be ignored. A stupid
one can ignore even facts, perhaps not for long, but for the moment. A smart one
can ignore his conscience, and everybody can ignore \co{nothingness}. But
truth is that which wins, eventually. This is not any deep insight but merely
the consequence of the definition. The corrective which objects in the momemnt
something is done wrong, will keep objecting until it is satisfied. Thus,
apparently, we might agree again with the pragamtism, but it would require a
closer distinction. We would agree that truth is that which wins -- eventually!
But according to a possible interpretation, \wo{that which works} might win by
definition and thus turn out to be truth. In such a case, we disagree completely
because truth, although recognizable as a corrective, is not a purely
operational term with which we could dispense. Our truth is there before, or in
any case, during the process, not only at its end. Only therefore it can provide
a corrective. 


\tit{Language}

\pa
It is, in a sense, only a question about language. 
For the most, we do agree on the kinds and qualities of various 
experiences, we only sometimes call them by different names.

However, language may reveal or confuse.  Calling things and
experiences we place them in the context of the unity, of our 
being, of Being,  we
assign them a place there.  Language is a meeting place of tradition
and novelty, of past and future.  Drawing its structure and
significance from itself, that is, from the shadow of the past, it
always reflects or tries to reflect the actuality, before it begins to
cast its shadow on the future. The shadow of the past meanings is the 
most valuable possession of the language. But as Jung says, integration of 
the shadow is an indispensable aspect of the process of individuation, 
and this happens by frist \thi{dismembering} the given meanings and then 
\thi{integrating} the scattered parts into a new unity, which is but a 
new form of the old one. (This last qualification is, of course, my 
addition and does not seem to apply in the Jungian individuation 
process.)


Speaking about \thi{God's venegance}, \thi{His will}, \thi{works},
etc., one may be easily accused of inventing invisible, irrational
causes for the visible effects which, moreover, should be and
sometimes even are explainable by visible means.  In many cases they
are.  But are they always?  No, but what we do not know today, we will
eventually learn.  Really?  And even if, what does it help me --
today?  That instead of an invisible source of something I posit an
ideal limit, that instead of anchoring my being in the midst of
eternity I reduce it to a mere stage on a way towards something I can
not even imagine, that instead of giving my indivuduality an account
compatible with the experience of my life, world, and the ultimate, I
become only an accident of a history of progress.  A lot of
intellectual confusion, scientific idolatry or sociological bias is
needed for anybody to take it seriously.

\pa
The language is there and there is no need for inventing a new one. 
If one wants to {\em invent} one, one better consider the fate of
\la{Esperanto}.  The only problem is to appreciate the depth which the
language we have can hide, when most we hear around are factual 
reports of more or less intelligent attempts to mix water with flour 
in order to get glue.

And so the linguists will have something to do for all the future
(until, of course, one gets bored).  For just like repetition is the
only way of Being, of being onself -- by repeating the invisible
foundation in a concrete, that is, ever new, personal and unrepeatable
form, so language follows after this reality of life.  Allowing for
expressing in actual form invisible presence, it is one of the
greatests gifts of \ldots of what?  I want to say, of the origin.  But
sure, of evolution, of the historical and social processes which led
to the development of language.  But the fact that it emerged only
through some evolution which we can vaguely -- and always only vaguely
and even unclearly -- posit, does not change anything.  If we had a
complete knowledge of its development then, perhaps, we could do this
and that.  But this is a contrafactual \thi{if} of such dimensions
that it shouldn't be necessary to worry about it.  The hidden, deep or
shallow structures of the language will never be confined to precise
and univocal rules, for language not only developes but also lives --
it lives the life of those who speak it.  And like with all life, we
can try to improve it, adjust it, change it but, at the bottom of it,
we can only try because we have no clue.  The only thing we can do
certainly and deliberately, entirely in accordance with our will, is
to destroy it, to kill it.  This, we {\em can} do and this we know
{\em how} to do.  There is not much more we can and know for sure.



\pa Everything can be understood differently. 
It may be debatable whether \citt{there is nothing so absurd that it 
has not been said by some philosopher.}{Cicero, De Divinatione, II:58} 
But there is hardly anything so absurd which could not be, or even which 
had not been, heard some philosopher saying. 

One can say that
various concepts, like \co{actuality}, \co{distinction}, etc.  do not
capture what they are supposed to express; that {\em actually}, what
belongs to \co{\ldots} is also this and that, and not only what I
included and listed under it.  And I may agree -- there are no absolute
places where distinctions have to or must not be drawn in an
unconditional manner.  But if one draws them very differently, one will
obtain a different philosophy.
%-- if only it allows me to recognize
%some aspects, preferably vague intuitions, which appear valuable and
%relevant, I am more than willing to listen to.  
I am, however, afraid
that in order to do that, one would still have to 
{\em conceptualize} the unity of life and experience. This may certainly 
be done differently, because concepts are only actual signs of 
the underlying intuitions.


\tsep{? perhaps, move to II.Some consequences ?}


\tit{Reality}

\pa
And what about reality? I objected so strongly against the labels 
\thi{real} and \thi{unreal} because there is nothing except 
\co{dissociations} of disappointment which might justify this 
distinction. But like everything else so reality, too, has degrees. 

Real is what you can not live without.  Not what you need or what you
would like but what you {\em can not} live without.  Breath, air,
water \ldots Sure. 
If one can not live without the pink linen in one's luxurious bedroom, 
the linen threatens with becoming all too real, perhaps even obsessively so. 
If you go for years with the uncessant thirst
for peace, fulfillment, for something which you start calling
\wo{happines} or \wo{love} or \wo{success}, but which you also know
can not be named -- it is real, oh, very real.  If you can not get rid
of a terrible image from a devastating experience you went through
long time ago, it is more real than all the nicietis you may find
around you.  And when you finally manage to overcome it, it becomes
\ldots unreal, you can breath freely again and live without it.

%what you would not live without

Real is what can not be replaced, what can not be substituted by
something else.  That is, as always, something is {\em the more} real,
{\em the less} it is possible to replace it.  The lack of the
\thi{sense of reality} may lead to an attachment to this or that, to
my car or my house, to my dog or all the world's animals.  And all
such attachments are real and make things to which I am attached highly
real.  But it is good to keep some sense of proportions.  In the
social context, a great scientist or a great political leader is much
more real than a shoemaker (today, we should probably say, a
programmist, or a cleaning lady).  The most real is a person -- a 
person's birth and pact with God make it irreplaceable because
non-repeatable.  \co{My life} is so real not just because I
cherish it so much and because it is \co{mine}, but because I know 
that it, too, is unique.

Real is what is shared, and the more real that which can be shared
with less diminunition.  The most real is the \co{origin} shared by
all and everybody.  Love is more real than hate, although the latter
is still more real than bricks, commodities, money or other things it
may destroy, and which may be replaced.  Sharing gives the deepest
sense of reality and it is utter loneliness which forces one to look
for it somewhere else.

And so, real is what is true, and what is true is real. One could 
not live without being born from the origin, no idol can relplace the 
origin and the origin is what is most deeply and intimately shared by 
all. The more 
truly I reflect the \co{origin}, the more real I and my life become. 
Eventually, only life can be real just as only life can be true -- 
\citt{the way and the truth and the life.}{John XIV:6}

\tit{Concept summary...}

\begin{enumerate}
\item \co{One} \simu \co{nothing} \simu \co{origin} (not everything, not yet and
  not ever; but in everything...)
\item \begin{tabular}[t]{lccc}
    being = & distinction & + & pariticpation  \\
            & immanence   & + & transcendence \\
            & actuality   & + & non-actuality\\
            & visible     & + & invisible
     \end{tabular}
\item Erkentnis (appropriation) \simu actualization (not Aristotelian!):
  transforming/translating into actuality
  \begin{itemize}
   \item act or several acts
    \item act(ual) insight / getting the concept, idea
   \item using it -- in relation to other actualities, also tacit knowledge
   \item also understanding = knowing the limits does actualize....
  \end{itemize}
\item Truth \simu that which is signified by every sign (that which
  every sign is a sign of), eventually \co{One}
  \\
  False -- ???
\item Good \simu whatever keeps Heaven and Earth together
  \\
  Evil -- alienating, separating, dissociating...
\item Real \simu that without which one can not live\\
  Unreal -- dispensable, unnecessary... (it seems too strong!)
\end{enumerate}  

\noo{
\pa When you know -- not have a vague feeling, but perhaps only have
clear experiences, not believe, not think, but {\em know} -- that
there is nothing that should be and remains to be changed, then there
opens up an infinity of things which can, may, and perhaps even will
be done, an unlimited horison of what is, through eternal becoming,
through you\ldots 

Talleyrand used to say that everything worth saying can be written on
one page and his clerks in the Ministry of Foreign Affairs had to write
concise, well-pointed reports.  I really wanted to write only a
short piece, but I did not manage; I can still try:
%
\begin{verse}
Not earth and water, for two came from one\\
And many are lands, while depths hide the eye,\\
Till the end of time\ldots

The higher you climb, the lower you descend\\
The road of many is wide, but the door only one,\\
Entering, one returns\ldots

%Life is strife, strife -- death. No borders; stop seeking\\
%Then you will find, when two become one\ldots\\
%What is, becomes, 

Life is strife, strife is death. No borders. \\
Stop seeking, then you will find, \\
When two become one.

What is, becomes, and passing -- remains.\\
The dead live unless the living die. %among the living. 
Everything is yours\\
`Cos nothing can be owned.

\end{verse}


}
