\tsep{hee}

All relevant (and irrelevant) details can be abstractly thought as a graph with
various edges (of relevance, dependence, association, etc.) connecting various points.
Traversal of a graph can be, in general, perfomed in two different ways: depth
first ({\sc df}) or breadth first ({\sc bf}). {\sc df} starts in the actual node
and follows one path -- an edge to a neighbour node, then to some neighbour of
the first neighbour, and so on. (Encountering a previously visited node, it
backtracks and tries another path.) {\sc bf}, on the other hand, vists first all the
neighbours of the actual node, before proceeding to all the neighbours of all
first neighbours, and so on level by level. 
A scholar is a {\sc df}. We do not have equally general, and certainly no
equally respectable name for the {\sc bf}, but it could be associated with a
dilettante -- knowing a little bit about everything which concerns him in one
way or another.

Of course, the world is not a graph. If we insisted on the analogy, we would have to
extend the idea at least by suggesting that the graph is unlimited, if
not actually infinite, and that in at least twofold way: every node has
infinitely many immediate neighbours and, from every node, there is an infinite
number of infinite paths (which never enter a cycle). A scholar diggs thus
further and furhter away from his home, trying to get to the end of an infinite
path, and hoping that it will bring him back home. A dilettante, on the
contrary, circles around, always in a safe distance
trying only to cover the infinite circle 
surrounding the house. Neither ever completes the road, both seem \citeti{to join
together diverse peaks of thought,\lin And not complete one road that has no
turn.}{Empedocles}{DK 31B24 [translation after \citeauthor*{Emped} 24.]} Scholar
ends up knowing everything about nothing, a 
dilettante knowing nothing about everything.

But this is unfair to the scholars! How can one compare them, put them on equal
lines with dilettantes?! Probably, one should not, but \wo{dilettante} is only a
name, the better of which seems hard to find. So let us ask the scholars...

Does the world -- eternally returning in the cycles oscillating between Love and
Strife -- exist twice (on the way from Love to Strife {\em and} on the way
back), or only once? And if once, then why, when and how?
\kilde{Szczerba,p.63-65}The diverging opinions may be the
consequence of the lacking sources which might have possibly contained the
answers of Empedocles himself. But, as a matter of fact, the question might have
never been asked and the answer never intended. Perhaps, all that was meant, was
to point to \wo{The world-wide warfare of the eternal Two}, the Love pushing to
unity and Strife to separation?  Perhaps, all the cosmogonies and cosmologies,
reflecting only the human understanding which, eventually, is always only
understanding of oneself, are but images never meant to be studied in and for
themselves. But sure, the questions can be asked, and so constructions can start
spinning...

What is Plotinus rejecting in V:5.1, claiming that the intelligibles, perceived by the
intellect, are not \thi{propositions}, \thi{axiomata} or \thi{sayables}? Does he
claim that the intellect's \thi{knowledge} is non-propositional or only
non-formalisable, non-expressible or only non-representational? Indeed, one may 
ask and keep answering, and many distinctions can arise from such diputes. But,
the question is, what shall we do with all these distinctions? Is it reasonable
to assume that, although the text does not say anything clearer, the intended
ideas were nevertheless so much more precise? Or, perhaps, they were not but
they have become so in the course of history?
Certainly, scholars shall sort out what Plotinus actually said
and meant and what he did not -- but preferably only as far as it goes. The fact
that questions can be asked with respect to a text, does not mean that the text
(and related texts, and texts related to texts related to...) contains the
answer, nor even that the answer can be at all 
meaningfully given.\kilde{EmilssonIntellect,p.29-30} We are rather clearly
\citet{told that the Intellectual-Principle and
the Intellectual Objects are linked in a standing unity}{Plotinus}{ V:5.1} and
even if \wo{we demand the description of this unity}, we should not dismiss the
possibility that what we have just been told is all we can actually get to know,
yeah, all that is worth knowing! Of course, 
to admit that, one would first have to learn cherishing and being satisfied with
perhaps clear and understandable but still vague expressions; expressions which
admit their limitation and put some trust in the reader who, hopefully, is the
same kind of being as the writer. All this, however, might amount to ceasing being a
scholar and turning~... a dilettante? 


There may be, and almost always there are, many meanings and one seldom can be
\co{precise} enough to narrow the expression, not to mention the thought, to
only one of them.  In this sense, interpretation is usually over-interpretation
and thus mis-interpretation. It is not so that we understand when we have made
the possible meaning most possibly, that is, impossibly \co{precise}. Dissecting
it into all too specific and detailed alternatives brings perhaps everything
under our control but only by making it impotent, by denying it life which our
thinking can only address, but never imitate... One may, perhaps, do it
sometimes legitimately in the name of scholarship, but positing it as the
universal aim of philosophy, and even thinking, sounds neither convincing nor
even plausible to many ears. Precising things beyond their limits results in
dissociating them and what is once dissociated can not be put together unless
one rediscovers the unity which preceded all the dissociated precisions. This
unity is not any combination, any coherence, any consistency of the elements
because it does not as yet know of any elements. \noo{This smells vagueness, if
  not directly misticism, so it may be a good point to begin.}

\tsep{???}
