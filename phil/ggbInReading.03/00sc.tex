\subnonr{Sources and references}
I quote rather extensively and from rather different traditions.  However, I
never go into exegesis of the texts or analysis of the thoughts of others.  An
attempt to do so would make finishing this work impossible.  On few occasions I
make more detailed statements in order to illustrate differences which also
should clarify my meanings.  The variety of sources and inspirations makes me
even limit the quotations to the most succinct statements which, I think, express
some essential idea.  Although the basic rules of conscientious exegesis may be
thus violated, and some quotations might have even been not only drawn out of
their context but even adjusted to fit the present one, the intention is never
to violate the meaning of the quoted text. (Besides, exegesis is not our
objective.)
% Asked by a true Irishman \wo{Are you a drinking man or are you a fighting man?}
% I could whole-heartedly confirm the former.  I like company, and

Variety of traditions suggests that we should focus on affinities and often even
only vague similarities rather than differences and oppositions.  Was
St.~Augustine entitled to claim the presence of Christian truths in the
neo-Platonic texts, as he did in the much disputed and controverted passage in
\btit{Confessions} VII:9? Was St.~Clement of Alexandria right in the
similar claims of the affinity of the Greek philosophy and literature with the
Christian revelation? Was Philo Judaeus right claiming not only the similarities
between but even the direct dependence of Greek thought on the Biblical
tradition? Scholars might prove that they were all wrong pointing out
significant differences making the two views different and even incompatible.
The Greek spirit was, after all, completely different from the Christian one.
Perhaps, but this depends on how one draws the borders around the intuitions
like \thi{Greek spirit} or \thi{Christian spirit}.\noo{(Let us also notice that
  such abstracts, useful as they sometimes may be in philosophy, are primarily
  only of historical and sociological character.)}  One can always find
differences separating two views -- the question is at what level, and then,
what value one will attach to them as opposed to the similarities. (After all,
the neo-Platonic culmination of Greek spirit, with its severe critiques of the
emerging Christianity, provided the foundation for the depth of Christian
mysticism.) Opposing, say, Greek spirit and Christian spirit, one should never
forget that in both cases one is speaking about spirit which, incarnated in
opposing socio-historical and political constellations, remains at the bottom
human spirit. It takes some wisdom to recognise concrete unity behind actual
differences and to stop distinguishing when everything worth saying has been
said -- the problem of perspicacious thoroughness, as La Rochefoucauld observed,
is not that it does not reach the end but that it goes beyond it.
%only blathering and babbling has no limit while it 
We will for the most focus on the similarities and it is up to you
to decide whether they are only due to the negligence in observing the important
distinctions or, perhaps, they are justified because the possible distinctions
are of negligible importance.

\noo{I quote others because I find myself to be their friend, even if they might not
always be my friends. If I did not have the quotation, I would write something
similar myself.  Although one might use this to construct the accusations of
eclectism, I would consider such accusations as a proof of a lack of even
minimal good will if not also of intelligence; and so I will rather get rid of
the bad company than of the good one.  The context in which the quotation is
used indicates, hopefully clearly enough, the interpretation I have in mind. I
hardly ever subscribe to the totality of the quoted author's or source's ideas;
it is only the thought behind the quoted piece which I want to bring forth.
(This, I guess, is one of the reasons why I bother to write.)}
%In all cases (with so few exceptions, that they are not worth
%mentioning here), a quotation indicates my more or less full agreement
%with the quoted author 
%I mean that there is enough unity of thought precluding the judgment
%of this work as eclectic, but if you do I could probably understand
%the reasons.


There are a few special sources which deserve a comment.  The authorship
of {\em My Sister and I} is the matter of dispute and scholars can not
tell for sure (perhaps, rather seriously doubt)
that it is indeed, as is also claimed, autobiography written
by Nietzsche himself. %I could hardly care less, since finding
The authorship of relevant thoughts should not be that important.
%whether they were written by Nietzsche or a skillful and competent forger.
However, in an academic context the issue may become a bit sensitive, especially
when the claimed author is Nietzsche.  (It might be so, in particular, if one
wanted to relate the contents of this autobiography to his other works which,
however, I am not doing.)

%For me, it is Nietzsche, for e
Even if it were not Nietzsche, it certainly could be, though 
%As somebody said, it is \wo{how you imagined Nietzsche would sound if 
%you got him drunk}. 
the author might also have been more Nietzschean than Nietzsche himself. Facing
the lack of any decisive proofs or disproofs of purely textual, linguistic or
medical nature, we are left with the text which looks like it might have been
written, if not carefully re-read and edited, by Nietzsche.  The voice for or
against his authorship depends then on one's view of his thought -- whether this
text \thi{fits} into the image one has of his whole thinking and, not least,
personality.  For me, there is a perfect match with the image I had formed
before I found this book. (Possible objections against the portrait arising from it,
should be confronted with less extreme, yet by no means incompatible, impressions
of the close friend in \citeauthor*{LouN}.) \citet{In the end,
  \btit{My Sister and I} reminds me of a true story.}{Sirens}{} Having made this
reservation, I will quote the text as if Nietzsche was its author.

Another referenced text, hopefully of much less dubious value, is a collection
of early Freiburg lectures by \citeauthor*{PhenomReligio} [\btit{Phenomenologie
  des religi\"{o}sen Lebens}, Gesamtausgabe, vol.~60]. Some of these have been
reconstructed almost exclusively from the notes of the students. Thus the reader
should be warned that the quoted formulations, although reflecting hopefully the
intentions, are hardly Heidegger's. (In any case, they are translated by me into
English, and that mostly from the Polish translation of the German text. Well...)

Likewise, \citeauthor*{Celsus}, is only reconstructed from the extensive
fragments quoted and criticized in \citeauthor*{aCelsus}. In this case, however,
the breadth and details of Origen's response give reasonable confidence into the
authenticity of the reconstruction. Much worse is the case of
\citeauthor*{Porphyry} where even the attribution of authorship may be disputed
as the work is reconstructed mainly from the \btit{Apocriticus} of Macarius
Magnes which need not reflect the philosophy of Porphyry. These works are quoted
as if they were written by the authors to whom they are attributed by the
general (though not universal) scholarly opinion. For investigating the
associated doubts and controversies the reader may start by consulting the
referenced editions.

Two distinct editions of \citeauthor*{Periphyseon} have been used. The critical
edition (started by late I.~P.~Sheldon-Williams and continued by
\'{E}.~A.~Jeauneau) of volumes I, II and IV is referenced as just done, with the
number+letter identifying the page number and the manuscript as in the edition.
Volumes III and V are from the abbreviated translation by M.~L.~Uhlfelder and
are referenced to in the same way, \citeauthor*{Periphy}, with only page numbers
in this single volume edition. In either case, the volume number identifies
uniquely the referenced edition. 


\sep
%
One encounters sometimes cases when, in an English text, quotations and longer
passages are given in French, German or some other language of the original --
sometimes even Latin or 
Greek. Although this may serve as an indication that the text is addressed to a
particular audience, it is no more pleasing than any other form of intellectual
snobbery.  It is perhaps a good tone to know German, French, Italian, Latin and
Greek, but few people do and I am not one of them. Since I have used extensively
sources in other languages, I have attempted to access -- and if I did not
succeed then to translate -- all the quotations into English. (A few exceptions
concern passages of German poetry which I did not dare to attempt translating.)
Sometimes, I ended thus translating back into English texts translated
originally from English into another language in which I read them. Such cases
are marked as \thi{my {\bf re}translation...}.  Hopefully, this will not cause
any serious confusion -- to fix it, I have to find some time with nothing better
to do.

\subnonr{Some conventions}

All the works are referred by the English title, even if I used the source in
another language; this is then indicated in the Bibliography at the end of the
text. (A few exceptions are made when the original source is referred after
another author, as is often the case with collected works or fragments.)

The references to all the works look uniformly as
\begin{center}
  Author, \btit{Title} XI:1.5\ldots
\end{center}  
where the part before `:', typically a Roman numeral, refers to the main part
into which the source is divided (e.g., book, part, chapter), and the numerals
after `:' to the nested subparts.  The references to the Bible have no `Source',
thus `Matt. X:5' refers to \btit{The Gospel of Matthew}, chapter X, verse 5. 
(I have used primarily King James Version and commented occasional usage of
other translations in the footnotes.) 
Likewise, the references to pre-Socractics are usually given without any source by
merely specifying the author and the Diels-Kranz number, e.g., `Heraclitus, DK
22B45', where the number identifying the philosopher (here 22) is taken from the
fifth edition of Diels, \btit{Fragmente der Vorsokratiker}.

Identifying quotations by page numbers might have been reasonable in times when
most books existed only in one edition.  I have tried to avoid such references
but in a few cases, where the structuring and numbering of the text happens to
be very poor, I had to use this form. This is also sometimes the case with the
quotations borrowed from others which I did not verify (the source is then given
in the square braces ``[after...]'' following the reference).  The pagination
follows then at the end of the reference as `Author, \btit{Title}
XI:1.5\ldots;p.21', where the numbers indicating part and subparts usually
involve only the main part (i.e., only `XI;p.21'), and may be totally absent, if
no such division of the work is given.  The edition is identified in the
Bibliography.  Occasionally, the subparts may have a letter, as e.g.,
`II:d7.q1.a2'. These are only auxiliary and their meaning depends on the source.
Typically, these are used with the medieval authors and the reference above
might be to the {\bf d}istinction 7, {\bf q}uestion 1, {\bf a}nswer 2, in the
second, II, volume/book.

In few cases I do not know the origin of the quotation, or else I only (believe
to) know its author. I chose to indicate such incomplete pieces of information,
rather than skipping them all together. I have likewise indicated the use of
unauthorized, or in any case unedited, versions of the texts found on the
interned for which no bibliographical data except for the title and the author
are given in the Bibliography. (For some, certainly very pragmatic reasons,
books printed in the USA do not carry explicitly the year of publication but only
the year of copyright. Consequently, the bibliographical information for such
books refers usually to this date.)

\sep
Words which are given some more specific, technical meaning are 
written with \co{slanted font}. \wo{Quotation marks} are used for 
words and quotations. \thi{Shudder-quotes} indicate, 
typically, either the referent of the word in the quotes, or else a 
concept or expression which is not given a technical meaning in the 
text but which is borrowed from somewhere else or even is only 
assumed to have some technical sense. Thus, for instance:
\begin{itemize}
\item 
\co{subject} -- is the subject in the technical sense introduced in
the text;
\item 
\thi{subject} -- is subject in some, possibly technical sense of
somebody else; it may often indicate a slight irony over only apparently
precise meaning one might believe the word \wo{subject} to have;
\item 
\wo{subject} -- refers to the word itself (quotations are also given
in the quotation marks);
\item 
subject -- this is just subject, with full ambiguity and with whatever meaning
the common usage might associate with it at the moment. 
\end{itemize}
I have tried to place more technical details in the footnotes which therefore
can be, for the most, skipped at first or casual reading. They are not, however,
addressed specifically to the scholars.\noo{We have not only no exegetic but
neither any scholarly ambitions.} Sometimes they elaborate the text but in
general will be useful only for those 
 who find some ideas interesting enough to follow them in other
authors.\noo{The footnotes contain, for the most, more details and references to the
possible starting points for such (re)search paths.}

