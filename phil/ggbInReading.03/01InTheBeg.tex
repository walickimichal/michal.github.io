
%%\chapter{In the beginning ...}\label{chap1}
%%\chapQ{Nothing that is has a nature, But only mixing and parting of the 
%%mixed, And nature is but a name given them by man.}{Empedocles, in Aristotle,
%%Metaphysics, V,4, DK 31B8}

\section{there was Nothingness}\label{se:nothing} 
\secQQ{Even the transcendental subject had to be born.}{}

\pa \wo{Why is there something rather than nothing?} What makes one ask?  Just
because we have the intuition, if not of nothingness, then at least of a sheer
possibility of nothingness, its empty intention? But an empty intention, a sheer
possibility -- isn't it just the idea of actual impossibility?

Why? Because you {\em are}, because you were \co{born}.  \co{Birth} is the
\co{separation} from the \co{origin} and \co{separation} results in a
confrontation of the separated poles.  \co{Birth} establishes the
\co{confrontation} of the emerging being with the \co{transcendence}. Such a
\co{confrontation} is \co{existence}.  \citet{By \co{existence} we do not mean
  here the existence in the sense of an occurrence and \thi{being there}
  (being-at-hand) of some being.  Neither does \wo{\co{existence}} mean here the
  existential worries about body and soul underlying man's moral care about
  himself. [...] \co{ex-occurence} is the \co{confrontation} in the openness of
  Being as such.}{vWW}{ IV \orig{\wo{Existenz} hei{\ss}t hier nicht existentia
    im Sinne des Vorkommens und \wo{Daseins} (Vorhandenseins) eines Seienden.
    \wo{Existenz} bedeutet hier aber nicht \wo{existenziell} die auf eine
    leiblich-seelische Verfassung gebaute sittliche Bem\"{u}hung des Menschen um
    sein Selbst. [...] Ek-sistenz ist die Aus-setzung in die Entborgenheit des
    Seienden als eines solchen.}}  Something is there because we \co{exist} and
without us, or other \co{existing} beings, everything would sink back into the
indistinct waters of \co{original nothingness}.

%Nothingness
\pa Before any experience, there was \co{nothingness}; no \thi{what}, not even a
\thi{that} which is not yet any \thi{what} -- but just \co{nothing}.

One designed notions of mere nothingness {\em for} consciousness, or {\em of} our
finitude.  What we cannot grasp, what we cannot see and embrace may seem to be
nothing.  And there is a lot of psychological plausibility in such notions.  But
\co{nothingness}, the hardly imaginable \co{indistinct} homogeneity, the {\em
  lack} of any objects, concepts, distinctions, is not nothingness of myself, of
a subject, for \co{nothingness} does not know of any subject, it is there long
before a subject appears.  It is \co{absolute}.  There is no access to it, it is
\citeti{above anything which even in thought or name could be a mere image or
  phantom of differentiation, in it vanishes every definiteness and
  property.}{Eckhart \citaft{Mistyka}{ A:I.3.b}}{ \kilde{p.22}} There can be no
experience of \co{nothingness}, for an experience requires a \co{distinction} --
\co{nothingness} is exactly a total lack thereof.

%confrontation - image
\pa\label{pa:imago} \co{Birth} is not an experience, it is the ontological
event. It precedes any \co{distinctions} and so no one remembers own \co{birth}.
\co{Birth} brings forth an \co{existence}, that is, a \co{confrontation}.
\co{Confrontation} is not a relation, it is a meeting. Only when seen as if
\thi{from outside} it can be reduced to a relation between two dissociated
entities, but to be \co{confronted} means to encounter \co{transcendence}, whose
ultimate form is \co{nothingness} of the beginning.

In \co{confrontation} the \co{separated} poles reflect each other. Not in the
sense of one being somehow \thi{similar} to another, but because they together,
and only together, constitute the uniqueness of the event.  \co{Confrontation},
in all its later and more specific forms, can be likened to a fight or a game in
which one opponent {\em reflects} the other; responding to the other's moves
or punches, he is in fact an \la{imago} from which one could reconstruct the
moves of the other.\ftnt{As all analogies, this one is not perfect either.
  Primarily, \co{confrontation} lacks the symmetry of a fight or interaction.}
In this sense, \co{existence} is \la{imago} of \co{nothingness} (and one would
be tempted to say, \la{imago Dei}).

%absolute beginning
\pa The \co{original} \co{confrontation} takes place in the midst of
\co{nothingness} -- it is \co{absolute}. It is not relative to any particular
being, because \co{nothingness} is the total lack of any particular beings, the
total lack of \co{distinctions}.

You were \co{born}, and there was time when you didn't exist.  But then there
was \co{nothing} -- no \co{distinctions} which now fill the world you are living
in.  To put it differently, if there always has been something then no beginning
has ever taken place.  Beginning, true beginning means precisely this --
something emerges from \co{nothing}.  If it emerges from something else, it is a
beginning only in a derived, analogous sense given to the word in the practical
context of daily experience.  If it emerges from something else, it is not new,
it is not unique -- it is a repetition, no matter how different it may be from
everything which preceded it.  \co{Absolute} beginning, creation from
\co{nothingness}, is the only way a unique individual, something that isn't a
repetition, can emerge.  \co{Birth} is such a beginning and so no \co{existence}
can be repeated.

%self-understanding/individuality
\pa To be \co{confronted} means to be confronted with \co{transcendence}, and
such an encounter implies also immediate self-understanding. On the one hand, it
implies \citet{being in such a way that one has an understanding of Being,}{SuZ}{
  Intro.1.4 (H12) \kilde{p.32}} that one has the understanding of the very fact
of \co{confrontation}. But to begin with this \thi{understanding} is nothing
else but the very \co{confrontation} itself, \thi{knowing} oneself to be an
\la{imago} is neither more not less than simply being \co{confronted}, that is,
\co{existing}.  Therefore the \citetib{question of existence never gets
  straightened out except through existing itself.}{SuZ}{\kilde{33}}

\co{Existence} is \citetib{in each case mine [...], delivered over to its own
  Being}{SuZ}{I:1.1 (H42) \kilde{p.67}}.  Heidegger does not, of course, mean
any solipsistic \thi{mineness}, and his emphasis on \thi{mineness} refers rather
to the unique individuality, \la{haecceitas}.\ftnt{We will borrow this term --
  though not the concept itself -- from Duns Scotus. (The \thi{concrete, i.e.,
    autonomous, lowest eidetic singularities} from \citeauthor*{IdeasI}{
    I:1.1.\para 15}, seem to be exact counterpart of Duns Scotus' concept.) We
  do not intend any univocal analogy. E.g., the concept applied to all things,
  while we apply it only to \co{existence}.  However, the emphasis it put on the
  fundamental character of individuality which is not conceptual
  (\la{quidditative}) and is really coinciding with, yet formally distinct from,
  the actual individual, gives very adequate associations.}  Such an
individuality is nothing more (nor less) than the event of \co{confrontation}.
It is only that which \thi{I am not} which delivers me over to myself, which
throws me back to myself and makes my own being an issue.  For a solipsist there
can be no \thi{mineness} for that, as Fichte maintained, arises only in a
\co{confrontation} with \thi{not-mine}. The all-embracing immanence of some
postulated spiritual unity is void of any \thi{mineness} except, perhaps, for
the one which reminds it that it is not so all-embracing and immanent as it
would like to believe. It is only \co{confrontation}, encounter with
\co{transcendence}, which constitutes anything that can have a character of
\thi{mineness} -- \la{haecceitas} is involved in the notion of \co{existence}
only because it is an \co{aspect} of \co{confrontation}.
% only because it is confronted with something which \thi{is not mine},
%  which \thi{it is not}.
  
\pa The \co{original} \co{confrontation} takes place in the midst of
\co{nothingness} -- as the \co{separation} by \co{birth}. Although it has many
\co{actual} analogues, it does not belong to phenomenology -- happening
\co{above} any \co{visible} contents, it never constitutes a \thi{phenomenon}.
Perhaps, it only underlies all phenomena, surrounding everything that appears
{\em for} ..., with an \co{invisible} trans-phenomenal \co{rest}.

We witness many \co{births}, of people, of animals, even beginnings of things.
Reasonably enough, we see the analogy and think that our \co{birth} was of the
same kind.  It was -- when seen from outside!  If you reduce {yourself} to this
mode of thinking, if you try to \thi{jump out of your skin} and pretend that you
are not here, only \thi{out there}, you will never be able to appreciate the
meaning of \co{your birth}, and hence neither of any \co{birth}.  For uniqueness
of every person is also what is {\em the same} in every person -- those who like
paradoxical formulations might say: every \co{existence} is a repetition of the
unrepeatability of the beginning \ldots\noo{Derrida, Introduction to Husserl's
  Origin of Geometry: the irreplaceable fact of origin (which can never be
  repeated) still has its invariance (the `essence of the first time') repeated,
  and then ``sense is indissociable from being, ... the de facto is
  indissociable from the de jure'', though it is not quite clear why... }

One can consider one's \co{birth} exclusively in the order of causality and
dependence, whether natural, biological, physical, or whatever, in the
\co{objective} categories of \co{externality}. Just like one can consider one's
life in such categories. But can one, really? And even if one could, would one
like to? One can not doubt that many events preceded one's \co{birth}. But this
is something one has to realise, something which is not among the first things
one learns. One has to develop the whole understanding of the world and even if
such a development does not amount to an idealistic constitution, it amounts, at
least, to a discovery. This discovery, which we will follow, begins with the
trans-phenomenal \co{nothingness}.

\noo{ \pa And then, imagining a world, the same world you know and live in, in
  which you happen not to exist isn't difficult.  But only because such an
  imagination is performed with the {\em assumed somebody} who does live in this
  world and, in fact, lives very much like you do.  You may subtract this
  somebody only by an act of reflective abstraction.  But in order for your
  image to remain valid, he must be still there, because you imagine -- see --
  that world where you are not with your eyes, which he has borrowed.
%you lended to him.
}

\sep

\pa One may easily claim that a (human) being never has been in such a
\thi{state of \co{nothingness}}: before experience there had been simple
feelings and sensations which we do not remember but can identify in innocent
tests on new-borns, even fetuses.  And on embryos?  If there was a time when I
didn't exist, if there was a time when no humans, no living beings existed, then
there was a time when there was \co{nothing}.  And sure, (human) being was never
in the state of \co{nothingness} -- for then there was \co{nothing}, in
particular, not this being.

In our daily life we are surrounded by all kinds of objects which we can, more
or less precisely, distinguish from each other.  The table in front of me is
obviously different from the chair on which I am sitting: they have different
properties, occupy different regions of space, one can be moved without
affecting the other, and so on.  However, the further we look into the past of
our personal being the less we find there, the fewer definite objects and
experiences.  And it is not simply our memory which should be blamed.  There
{\em were} fewer objects and less diversity.  It is only in the process of
growing and education that we learn to distinguish things and experiences which
were previously fused with an indistinct \thi{background}.  It takes time before
a child learns that a chair and a table are two separate things.  It takes time
before it learns that a chair and a table are things at all, before they emerge
from the \co{indistinct} background as two independent entities.  And when that
happens it happens because they are \co{distinguished} from the background and
from each other, because they emerge as \co{distinct} things.

Once we begin to distinguish sharply and precisely, it is difficult to recall
this original, almost magical power of the surrounding which has not yet fallen
apart, where parts have not yet been estranged from the background and acquired
independent existence of their own.  Perhaps,
% the {Aha!}-experiences may remind us of it when we suddenly solve a problem
% realising some crucial distinction or meaning which has evaded us so far.  We
we can sometimes experience a similar situation when we are placed in an
entirely new and unknown surroundings.  We do recognise individual objects (this
ability, once acquired, hardly ever gets lost) but the whole world appears
chaotic, perhaps, meaningless.  There are no indications as to which things or
observations are significant, which mean something and carry relevant
information and which do not.  We experience a chaotic variety which -- due to
the lack of meanings and significance -- appears as an undifferentiated,
homogeneous totality.  Only after some time we are able to pull some objects out
of this background, to distinguish the relevant from the irrelevant.

These, however, are only imperfect analogies.


%\tsep{Kant...}

\pa The \citet{appearances are not things in themselves; they are only
  representations, which in turn have their object -- an object which cannot
  itself be intuited by us, and which may, therefore, be named the
  non-empirical, that is, transcendental object $= x$.  The pure concept of this
  transcendental object, {\em which in reality throughout all our knowledge is
    always one and the same}, is what can alone confer upon all our empirical
  concepts in general relation to an object, that is, objective reality.}{CrPR}{
  B137} The emphasized phrase is the point from which \noo{, following the
  post-Kantian \ger{Identit\"{a}tsphilosophie},} we would start remodelling Kant
to fit our purposes.  He speaks here only about the {pure concept} of such a
transcendental object, not about the object itself.  Yet, there isn't much which
could distinguish the two, except for the presumed conviction that the two
should be distinguished.  Allow us therefore to think them the same: emptiness
of the \thi{pure concept of $x$} is but a reflection of the pure
\co{nothingness}, \co{indistinctness} of $x$.

There is hardly anything in Kant's Critiques which would justify a {\em
  multiplicity} of things-in-themselves.  The concept is always one and the same
and the whole Kantian exposition might be carried without much (if any) changes
if we allowed, equally, only \co{one} thing-in-itself -- inaccessible to the
categories of understanding because \ldots entirely \co{indistinct}.  Different
things-in-themselves are equally empty, contentless and transcendental --
offering no grounds for being distinguished, they should better remain one and
the same.  This would make even identity and distinctness of different things of
experience a mere \thi{appearance} in the Kantian sense but, with all
reservations to be made on the way, we are going to do precisely that.
% In particular, the relativity of any \co{distinctions} to the
% distinguishing \co{existence} does not turn them into \thi{mere appearance for
% ...}. Relativity excludes absolutness but not objectivity. 

So, no things-in-themselves but thing-in-itself and, as a matter of fact, not
even thing-in-itself, but just \co{indistinct nothingness}, or the \co{one}.  As
we will also see, this \co{one} does not arise from but, on the contrary, is the
\co{foundation} of \co{experience}, and then also of any specific experience of
\thi{objective reality}.


\pa \co{Nothingness} is void of {\em any} experience.  But it is as well the
simple \co{one}, the \co{origin}, since everything in the world originates
beyond world's boundaries, comes from what embraces it, from the entirely other
-- in the world's language, from nothingness. \citet{Sacred ignorance teaches me
  that which seems to the intellect to be nothing is the incomprehensible
  Maximum}{DDI}{I:17.51} But we do not want to posit too much into
\co{nothingness} which would force us to find it there later through \thi{sacred
  ignorance}.  \co{Nothingness} is not any \thi{Maximum}, and even \la{ens
  realissimum} has too much of \la{ens} to be adequate; it is the \co{virtual},
not \co{actual} \co{origin}, the germ from which everything arises, perhaps a
\la{summum}, but not a sum containing everything {\em within} itself -- being
\co{indistinct}, it does not \thi{contain} anything.  It is the background from
which and against which anything that is appears.  \citet{All things proceed
  from the Nothing, and are borne towards the Infinite.}{Pensees}{II:72} And it
is the background which, once the world appears, continues to encircle it.


\pa \co{Nothingness} is what precedes the world, not so much in the temporal
order, but in the ontological order of \co{foundation}. It is the state before
things and the world emerged, when \citeti{the earth was without form, and void;
  and darkness was upon the face of the deep.}{Gen.}{I:2 [Septuagint has
  \wo{invisible and non-composite} instead of \wo{without form and void}.
  \kilde{Periphyseon, II, nt.214}]}

Things arise only from this formless homogeneity, from \thi{the dark and
  indistinct waters} which embraced everything before the creation of the world.
Whether it is \thi{creation from \co{nothing}} or else \thi{emanation from the
  \co{one}} is only a manner of speaking.  It has no adequate names not because
our knowledge is finite and imperfect but because the lack of any
\co{distinctions} makes every name inadequate.

\noo{\co{Nothingness} is what we find in the beginning, and as there can not be
  two \co{nothingnesses}, it is \co{one}. Once \co{distinctions} begin to
  emerge, it can be opposed to them as the \co{indistinct origin},
  \co{invisible} source.  \citet{For everything that is understood and sensed is
    nothing else but the apparition of what is not apparent, the manifestation
    of the hidden, the affirmation of the negated, the comprehension of the
    incomprehensible, the utterance of the unutterable, the body of the
    bodiless, the essence of the superessential, the form of the formless, the
    measure of the measureless, the number of the unnumbered, the weight of the
    weightless, the materialisation of the spiritual, the visibility of the
    invisible.}{Periphyseon}{III:633A, 678C} }

The mystery is not how the mind forms, out of the diversity of perceptions and
atomic properties, the idea of a sustaining, self-identical object, nor how the
objective atoms \thi{compose} to form the experienced unities. Such questions
address secondary constructions and can be asked and attempted answered with
full \co{visibility} of their objects and contents.  On the contrary, the
mystery is how the original uniformity passes to the multiplicity of independent
individuals, how the \co{one} becomes \thi{many}, or how God creates the world
from \co{nothingness}. \citet{The simplex [absolute] requires no derivation; but
  any manifold, or any dual, must be dependent.}{Plotinus}{V:6.4}
% \citet{Know that all beings have their birth in
%   this. I am the origin of all this world and its dissolution as
%   well.}{Bhagavad}{VII:6 (Indian p.127)}

\pa We will not disturb the tranquility of mysteries attempting to answer any
\thi{How?} We only notice that what marks the end of \co{indistinct nothingness}
and a transition towards the world of experience, is that \citeti{God divided
  the light from the darkness}{Gen.}{I:4}, is \co{distinction}.  The primordial
act of creation is an act of \co{distinction}, turning the \co{indistinct}
\co{nothingness} into something, pulling this something out of \co{nothingness}
and letting it come forth, letting it appear.  This happens still in \la{illo
  tempore}, against the background of mere \co{nothingness}, before we can talk
about any person or subject.  We could say, it is \co{birth} which is the first
\co{distinction} in that a new being is \co{separated} from the \co{origin}. But
it is immediately accompanied by a multitude of further \co{distinctions}, which
will concern us most.

\co{Distinction} breaks the original unity.  \co{Nothingness} withdraws and
becomes a mere background, a mere stage for the performance of the richness of
the world.  Every being will now carry within itself the element of the
\co{original nothingness} from which it emerged. Or else, as the Pythagoreans
could say, the limit \noo{$\pi\varepsilon\rho\alpha\zeta$ {fix this, \tau\'{o}
    and some more accents}} introduced into the {eternal and ageless} indefinite
(\gre{apeiron}) \noo{${\alpha}\pi\varepsilon\iota\rho o\nu$ {fix this, \tau\'{o}
    and some more accents}} results in the limited cosmos (of \co{distinctions})
which \wo{inhale} the surrounding air, the boundless (\gre{apeiron})
{encompassing all the worlds.}\kilde{Copleston, p.35, but quotes from there p.25
  - Anaximander} \co{Distinction} does not merely \co{distinguish} something and
brings it forth.  Primarily, it introduces the difference between the
\co{distinguished} and the \co{non-distinguished}, between the \thi{many} of
actual \co{distinctions} and the \co{one}, or \co{nothingness} of the
\co{indistinct origin}. Then it is indeed \citet{most difficult to apprehend the
  mind's invisible measure\lin Which alone holds the boundaries of all
  things.}{Stromata}{V:12 [quoting Solon]} Whether one says that the mind's
measure is in the mind or outside, that the beginning was in the mind or not,
one puts it wrongly, for (the indefinite and boundless) \co{nothingness} knows
of no \co{distinctions}, and hence neither of any \thi{either-or} nor of any
\thi{outside}.

\co{Confrontation} is the constant circumscription of the boundaries of all
things, limiting the unlimited, \co{distinguishing} the \co{indistinct}. We
might thus characterise \co{existence} equivalently as the being which makes the
difference and for which things make difference, which \co{distinguishes} and
hence for which there is not only the \co{indistinct} but also
\co{distinctions}; in short, the being which is not merely enveloped by
\co{nothingness} but which encounters something.  \noo{We won't, however, prove
  anything from our definitions -- we will unfold them.}


\section{In the begining there was only Chaos, the Abyss}\label{se:chaos}
%Hesiod, Theogony 116
\pa
The \co{distinctions} emerge only gradually as we grow up and learn.  As we
become more sophisticated and mature, the world becomes richer and more
diversified.  How and when does the first \co{distinction} occur?  Perhaps, at
the moment of birth, perhaps earlier.  We do not know.  We do not know and it
does not matter.  What matters is {\em that} we did begin to \co{distinguish}.
The indistinct waters embracing everything in the beginning withdrew as the
primordial act of creation brought the first \co{distinctions} to the surface.
But we do not remember.  We do not know and creation remains a mysterious gift
of the \co{origin}.

The first \co{distinction} does not occur alone. Strictly speaking, there is
nothing like {\em the first distinction} -- only a transition from the state of
undifferentiated \co{unity} to the multiplicity of \co{distinctions}. And
naturally, one \co{distinction} may encompass other ones, one \co{distinction}
may be a gathering place of many other.  Creation does not merely bring forth a
single object but a whole world. We do not merely \co{distinguish} pain from a
formless background but at the same time from hunger and satisfaction, we
\co{distinguish} this light from that darkness, one person from another, mother
from father, then a chair from a table ...  A \co{distinction} occurs only in
the midst of other distinctions.  The gradual emergence of the world amounts
only to the gradual refinement and adjustment of the distinctions. At every
stage there is always unlimited number of distinctions, in fact, a chaos
exceeding our possibilities to embrace it in a single act.

\co{Chaos} -- the limitless manifold, the ever too big, the ever exceeding
our capacities number of \co{distinctions} -- is not \co{nothingness} any more.
\co{Nothingness} has no \co{distinctions}, in particular, no subjective pole.
But \co{chaos} exceeds {\em somebody}'s power.  It has a subjective pole -- the
\co{actuality} of our finite being which emerges in its midst. For this
\co{actuality}, \co{chaos} is the first, differentiated analogy of the
\co{origin}.
%For this actuality, chaos is a scary fullness rather than nothingness. 
The \co{actuality} confronted with \co{chaos} appears powerless.  The
\co{proto-experience} of the limitless \gre{apeiron} is {\em the same} as the
\co{proto-experience} of the finitude of \co{actuality}. It is the experience of
the impossibility to embrace everything within the \hoa.  The \co{confrontation}
with the limitless reveals not \co{nothingness} but the limited; the limited
whose fragility dissolves in the overpowering.  This fragility, the finite
reflection, \la{imago} of \co{chaos}, is the site of \co{actuality} or --
\co{proto-consciousness}; the limitless is its primordial correlate.


\pa \wo{{\em Why} is there something rather than \co{nothing}?} and \wo{{\em How}
did \co{nothingness} become something?} are perhaps questions introducing to
metaphysics. But there is a grave danger that metaphysics attempting to answer such
questions, instead of ending up with saying something -- no matter how vague and
imprecise, but still {\em something} -- ends up merely glorifying its impotency and
inability to say anything.

The celebrated question is certainly different from many others but,
nevertheless, shares with them some common ground. For instance, ``{\em How} can
the subject be certain that its knowledge represents adequately the external
reality?''  Differences notwithstanding, both questions ask for an {\em
  explanation}, perhaps, even for a demonstration.  \wo{Why} and \wo{how} are
questions more or less successfully addressed by sciences and common sense --
the agents seeking {\em explanations}. But when directed towards the
\co{origins} preceding the world, they can, at best, produce conceptual analyses
of dubious value, transcendental illusions and, sometimes, transcendental
dogmas. The task of philosophy is not to explain.

If we don't want to end up in a pseudo-science of conceptual analysis, if we
renounce the \thi{whys} and \thi{hows}, we are left with the simple and
audacious \thi{what}, or even worse, with the mere \co{that}.  \thi{Why} and
\thi{how} something happens are already involved in the differentiated world of
experiences and concepts.  But {\em what} experiences, {\em what} concepts?  The
philosophical challange is to see and say \thi{that} something happens and
\thi{what} this is. \thi{Whys} and \thi{hows} can only sometimes, and only
possibly, help to clarify this \thi{what}.  The challange is not to explain,
perhaps not even to understand but to gather and give an account.

\pa\label{pureexp} If there was a time when you didn't exist, then there was a
time when there was \co{nothing}. \noo{(That I, this particular person and field
  of experience, have always existed and was not created may be a tempting
  thought but I resist this temptation.)}  And as something is there now, it
must have started to appear some time.  That much one can say.  But when? How?
In the first hour, first minute after your birth? Before? When you were an
embryo? Just afer conception? We do not know and it is not important to us.
\noo{We can leave it to those worried by contraception and abortion.}  Some
time, some things begun to emerge but these somethings were not things, forms,
people, etc. as we know them. In the beginning \wo{the earth was without form
  and void, and darkness was upon the face of the deep}.  Then \citeti{God said:
  `Let there be light' [...] and God divided light from darkness}{Gen.}{I:3-4},
%then ``Let there be a firmament in the midst of the waters, and let it divide
%the waters from the waters.''
then \citeti{God made the firmament and divided the waters which were under the
  firmament from the waters which were above the firmament}{Gen.}{I:7}.  Not
before the third day the \citeti{grass, the herb yielding seed and the fruit
  tree}{Gen.}{I:11} emerged.

\citet{\ `Pure experience' is the name which [one] gave to the immediate flux of
  life which furnishes the material to our later reflection with its conceptual
  categories.  Only new-born babies, or men in semi-coma from sleep, drugs,
  illnesses, or blows, may be assumed to have an experience pure in the literal
  sense of a {\em that} which is not yet any definite {\em what}, tho' ready to
  be all sorts of whats; full both of oneness and manyness, but in respects that
  don't appear; changing thoroughout, yet so confusedly that its phases
  interpenetrate and no points, either of distinction or of identity, can be
  caught.}{ThingRel}{ [This, by the way, seems to be one of the carrying ideas
  of the late XIX-th century, a crucial aspect of the \ger{Zeitgeist}.  Not only
  the calculus of inifinities of Cantor and rigorous treatment of continuity
  and real numbers emerging from the work of people like Weierstrass and
  Dedekind, but likewise \fre{dur\'{e}e} of Bergson's, \thi{stream of
    consciousness} of Joyce's and psychoanalysis, the Absolute as experience
  with distinctions but without relations of Bradley are all variations over
  this theme.  We will rather continue with our presentation, leaving the
  comparison of the similarities and differences to the interested reader.  Our
  \co{proto-experience} is, probably, closest to James'.  Here the fundamental
  difference consists in that he very quickly passes from it to another meaning,
  which is also the cornerstone of his \thi{radical empiricism}, namely, a
  \thi{pure experience} as \co{actual}, that is, limited to a \thi{now}.
  \co{Proto-experience} knows as yet nothing about time and such a limitation.]}
 %His \thi{radical empiricism} requires to refer all claims to such
 %\co{actual} experiences at some definite point of time.


\pa\label{pa:toBeDist} If we try to imagine -- and we can hardly do anything
more than {\em imagine} -- a \pexp, it is like a continuous, irreflective flux
of \thi{somethings}; a chaos of \co{pure distinctions} not only without any
mutual relations, but without any sameness. One should not focus here on an
object, on \thi{this pen on the table}, because such an act involves already
fixation and \co{recognition}.  I may turn off my reflection and just stare at
\thi{this pen here}. It is probably as close as I can get, but it is not a
\pexp, because there is nothing like \thi{a pen} in \pexp.  \Pexp\ is not an
experience of something; it isn't even an experience of nothing -- it just isn't
an experience of anything.

\Pexp\ does not involve any thing.  It does not even involve a \thi{that} which
is too much suggesting some kind of a definite entity.\ftnt{The reader manages
  hopefully to distinguish the cases when (like here) \thi{that} is meant as a
  pronoun, and the more usual ones when it is a conjunctive, especially when
  written \co{that}.}  But it is not \co{nothingness}; something begun to emerge
at the edge of \co{nothingness} which soon will become the edge of the world. It
is like a pure heterogenity, merely opposed to \co{nothingness} but not yet
incarnated in any definite \thi{thats}.

It is not \co{nothingness} because it involves \co{distinctions}. And whatever
is \co{distinguished} already is. We can never find anything about which we
couldn't, in one sense or another, say that it is. The universal equivocity of
the word ``is'' reflects this fact that to be is to be
\co{distinguished}.\noo{Various postulates of some \thi{being} are all very nice
  in so far as they emphasize that what is there is much more than the things,
  or substances, to which, since Aristotle, we have used to limit the meaning of
  the copula. But they become unclear as soon as they try to say something more
  specific about this \thi{being} -- not only because talk about higher things
  necessarily loosens academic precision, but because they assume this
  \thi{being} of theirs, which is not \co{nothing}, to admit closer
  determination. \citef{Permanent, always identical, already-there, given -- all
    mean fundamentally the same: enduring presence, \la{on} as
    \la{ousia}.}{IntroMeta}{\noo{~p.202}} For any of these determinations it is
  not difficult to find something that is without sharing it. I can't see how
  enduring presence follows from or means the first four determinations and, in
  particular, to whom it is present. One can certainly proceed with this kind of
  definitions but it would help if one were less obsessed with proving their
  indispensable reality, and more with explaining their meaning.}  Yet these are
only \co{pure distinctions}, like mere facts of mere differences possessing no
sameness, no self-identity; it is a flux, a light which isn't darkness any more
but where still there are no \thi{thats}, no somethings at which one could stop
and point.
 
\pa \Pexp\ is not divided into \thi{now} and \thi{then}, it is timeless, that
is, \thi{objectively} it may last one second as well as one day.  Approaching it
phenomenologically, one will easily use images like \citet{a flow of continuous
  change with the absurdity that it flows just as it flows and can flow neither
  faster nor slower.}{Zeit}{ A:3.\para 35 \orig{einen Flu{\ss} stetiger
    \thi{Ver\"{a}nderung}, und diese Ver\"{a}nderung hat das Absurde, da{\ss}
    sie genau so l\"{a}uft, wie sie l\"{a}uft, und weder \thi{schneller} noch
    \thi{langsamer} laufen kan.}} The \wo{flow} is, however, not a name for any
flow of time but rather for an \thi{overflow}, in the sense of each content
flowing over into every other, static co-presence of \co{distinctions} without
any distinguished objects, mutual interpenetration of \co{vaguely} distinct
contents which, however, possess no inherent identities.  \citetib{Thus any
  changing object is here missing; and in so far as in every process
  \thi{something} is happening, here no process is involved. There is nothing
  there which is changing and hence one can not speak meaningfully about
  anything which is lasting. [...] It is the {\em absolute subjectivity} and has
  the absolute property of an image as \thi{flow} to the signified [...] For all
  that we lack names.}{Zeit}{\para 35-36.\verify{} \orig{Sodann fehlt hier jedes Objekt,
    das sich ver\"{a}ndert; und sofern in jedem Vorgang \thi{etwas} vorgeht,
    handelt es sich hier um keinen Vorgang. Es ist nichts da, das sich
    ver\"{a}ndert, und darum kann auch von etwas, das dauert, sinnvoll keine
    Rede sein. [...] Es ist die {\em absolute Subjektivit\"{a}t} und hat die
    absoluten Eigenschaften eines Bilde als \thi{Flu{\ss}} zu Bezeichnenden
    [...]  F\"{u}r all das fehlen uns die Namen.}} Being timeless it is also
spaceless, not divided into \thi{here} and \thi{there}. It is like an
\thi{absolute place}, that is, one not relative to any other place, one beyond
which no other place exists, and yet a \co{concrete} place and not the whole
abstract universe. Finally, this \co{concreteness} does not involve any subject,
there is no subject of \pexp\ just like there is no object.  There is nothing of
the sort because there are no somethings, because \co{pure distinction} is not a
distinction between this and that.  Rather, it is a \co{distinction} without
content.  From the point of view of later \co{reflection}, we might also say
that it is the {mere {\em fact} of distinguishing}.\ftnt{The word
  \wo{\co{distinction}} will be, intentionally, used in the equivocal sense: as
  the act of distinguishing and as that which is distinguished. So far, there
  are no acts nor things distinguished, and distinguishing between the two would
  be misleading.  Derrida's \thi{differance} is probably a close analogue of
  this concept.  However, we wouldn't dare to attempt any more detailed review
  of what {\em he} possibly might mean, so this is only some hunch that both
  point to the same intuition.}
 % Heidegger emphasizes that the word \citf{\thi{experience} means: 1. the
 %   experiencing activity, 2. something that is  experienced through it. We
 %   intentionally use the word in its double meaning in order to indicate the
 %   fundamental feature of factual life experience: that the experiencing and the
 %   experienced are not dissociated as two different things.}{Heidegger,
 %   Introduction to Phenomenology of Religion (early Frieburg lecture, winter
 %   semester 1920/21), pp.3 [p.12]}
 %
The \thi{mere fact} also because the only thing we can say about it is \co{that}
it happens -- not \thi{what} was \co{distinguished}, only that what happened to
be, also could be \co{distinguished}; above all, there is nothing which could
entitle us to say that it was true or necessary to \co{distinguish} this rather
than that.

\pa All these lacking distinctions are what distinguishes \pexp\ from
\co{experience}, in particular, any experience empiricists ever managed to
suggest.  \noo{ Empiricists' experience is always founded on some basic,
  ultimate \thi{atoms}, whether sensations or perceptions, which assume a \hoa\ 
  within which the \thi{atomic data} are grasped. Caught into the impossibility
  of defining objectively the scope of this horizon and, consequently, of
  specifying these \thi{indivisible somethings}, they divide forever,
  incessantly looking for the ultimate atoms.  Nothing of the sort appears in a
  \pexp, nothing comes forth.  But, although it is still nothing as far as
  understanding is concerned, it is already marked by the heterogenous manifold,
  \co{chaotic} variety.  }
%
Using words like \wo{\pexp} and \wo{\co{chaos}}, we should keep in mind that
there is no {experience} {\em of} \co{chaos}.  \Pexp\ is not \co{an experience}.
It is \co{chaos} which was at the beginning, after the darkness of
\co{nothingness} was separated from the light, but before the world and anybody
who could experience anything emerged. It is not accessible to any
\co{reflective} introspection.  \citet{Born as we are out of chaos, why can we
  never establish contact with it?  No sooner do we look at it than order,
  pattern, shape is born under our eyes.}{Cosmos}{II\kilde{p.31}} As Husserl
used to emphasize and as we will emphasize in what follows, the fact that
something is not (an object of) \co{an experience} does not mean that it is not
experienced! We could say that it is co-experienced. As \co{nothingness} and
\co{chaos} withdraw beyond the horizon of \co{experience}, they do not
disappear. They constitute an integral part of \co{experiencing} as well as of
\co{any experience} and so are given along with it. They are only never given as
\co{objects} of any particular \co{experiences}.

\pa Except for being differentiated, \pexp\ does not offer anything. It is
properly continuous, not in the sense of a successive flux of distinct
\thi{nows} and \thi{thats}, but as timeless, without any \thi{now} and {then}.
It is \co{chaos}, but not a chaos of objects (which is secondary) but just
\co{chaos} -- of \co{pure distinctions}, \wo{without number or multitude},
 %Eckhart after Otto, Mistyka Wschodu i Zachodu, p.85
of \thi{thats} which are not \thi{whats} and do not yet pretend to possibly
possessing any meaning.  It is no longer \co{indistinct} Parmenidean \thi{is},
but rather Anaximander's \gre{apeiron}, with full ambiguity of the term meaning
{\em both} indefinite {\em and} infinite, unlimited and limitless. This second
hypostasis precedes any later differentiation -- \citeti{it is neither water nor
  any other of the so-called elements, but some other \gre{apeiron}, from which
  came into being all the heavens and the worlds in them.}{Anaximander, DK 12A9,
  Theophrastus' account \citaft{Vorsokr}{ III\kilde{p.106}}}{ Plotinus about the
  limitless: \citef{what is known as the flux of the unlimited is not to be
    understood as local change; nor does any other sort of recognisable motion
    belong to it in itself; therefore the limitless cannot move: neither can it
    be at rest: in what, since all place is later? Its movement means little
    more than that it is not fixed in rest.  Is it, then, suspended at some one
    point, or rocking to and fro?  No; any such poising, with or without side
    motion, could be known only by place [which Matter precedes].  How, then,
    are we to form any conception of its being?  We must fasten on the bare
    notion and take what that gives us -- opposites that still are not
    opposed.}{Plotinus}{VI:6.3} We would be willing to discern the same theme in
  many forms: in the chaos of Hesiod, in \la{materia informis} of neo-Platonism
  or \la{materia confusa} of alchemists; likewise, in the symbolism of water, as
  both the source of life {\em and} the confused and hardly differentiated
  principle: \la{hydor theion}, divine water, the indistinct image of fertile,
  life giving, first \gre{arche} of Thales; the world-encircling \gre{Okeanos},
  \la{aqua vita} of the alchemists.} For the moment, there are still no
\thi{elements}, no things, there are no \co{distinguished} objects, as there is
no subject making any \co{distinctions}.


\subsection{Spatiality/simultaneity}\label{sub:spatialityBeforeTemporality}
\pa\label{pa:spatialityBeforeTemporality} There is nothing like \thi{the first
  \co{distinction}}; emerging \co{distinctions} mark simply the second
hypostasis in which the primordial \co{nothingness} turns into \co{chaos} -- a
manifold of heterogenity. However, the time has not yet begun to flow, there is
as yet no \co{distinction} between the \co{actual} and \co{non-actual}, not to
speak about any succession. Another word for this may be \wo{heterogenity} or
\wo{simultaneity} -- all \co{distinctions} are simultaneous, not because they
were comprised into a simultaneity, because all \thi{before} and \thi{after} have
been abstracted away, but because there is, as yet, no \thi{before}, no
\thi{after}.  Likewise, there are as yet no entities which could be
distinguished through their properties.  \noo{The \co{chaos} of
  \co{distinctions} is a pure heterogenity.}  There is no sign of distance, nor
any measure, because \co{distinctions} are merely distinct -- not so that an $x$
may be more distinct from $y$ than from $z$, but only so that $x$, $y$ and $z$
are simply mutually distinct, and are not even any identifiable $x$, $y$, $z$.

\noo{There is here no talk about homogenous, infinitely divisible space.  In
  principle, \co{chaos} can be \co{divided} indefinitely, but only in principle
  -- in fact, it has not been so divided.  At each stage of gradual
  differentiation there is a \co{chaos} of \co{distinctions} which {\em have
    been made}. The question of possible continuation of this process, i.e., of
  infinite divisibility, can only be asked as if \thi{from outside}, but it has
  no relevance at this stage.  }

This feature -- mutuality which means also simultaneity of \co{distinctions} --
can be taken as the fundamental characteristic of \co{spatiality}.  Thus
\co{spatiality} (not space, but mere simultaneity of \co{distinctions}) is
somehow prior to temporality.  It is a mere expression of the fact that
\co{distinctions} do not arise one after another\ftnt{This happens too, but it
  is a completely different -- factual, and not, as here, virtual -- process.},
that the level of \co{chaos} involves immediately a whole range of mutually
different, heterogenous elements. \co{Chaos} is the \co{virtual} co-presence of
a manifold of \co{distinctions}.\ftnt{Looking for an intuition of \co{chaos} in
  our \co{reflective experience}, we are naturally bound to recognise in it an
  element of duration. Bergson's \fre{dur\'{e}e} comes very close to our
  \co{chaos}, if only we are willing to subtract from his descriptions the
  phenomenological element of observing consciousness.  \citef{Pure duration is
    the form which the succession of our conscious states assumes when our ego
    lets itself {\em live}, when it refrains from separating its present state
    from its former states. [...] We can thus conceive of succession without
    distinction, and think of it as a mutual penetration, an interconnexion and
    organisation of elements, each one of which represents the whole, and cannot
    be distinguished or isolated from it except by abstract thought. [...] In a
    word, pure duration might well be nothing but a succession of qualitative
    changes, which melt into and permeate each another, without precise
    outlines, without any tendency to externalise themselves in relation to one
    another, without any affiliation with number: it would be pure
    heterogenity.}{BergTime}{ II;p.100-101-104}
  
  We notice here the crucial ambiguity which penetrates Bergson's
  \fre{dur\'{e}e}: it is succession but, on the other hand, it is a pure
  heterogenity, simultaneity.  \citef{Bergson's duration is eventually defined
    less by succession than by co-existence.}{DeleuzeBergson}{ III\noo{p.57}}
  Duration may be imagined and posited \thi{in itself}, as merely happening
  \thi{out there} without any witnesses. As such, it can be imagined and posited
  as not requiring any simultaneity, as merely flowing. But, of course, one has
  posited it, that is, gathered all this duration in the simultaneity of one
  act. Only extreme objectivistic abstraction can allow itself, while
  considering duration, to dispense with the one actually apprehending it.  Any
  form of apprehending duration requires simultaneity, requires juxtaposition of
  some earlier and some later moment in one act.  This aspect of simultaneity
  and constancy is more emphasized by later Bergson.  \citef{A {\em moving
      continuity} is given to us, in which everything changes and yet
    remains.}{MattMem}{ IV:3;p.197} The dissociation of the experience into
  \thi{inner} and \thi{outer} happens to the distinctions which are originally
  present in one unified field: \citefib{Psychologists who have studied infancy
    are well aware that our representation is at first impersonal. [... I]f you
    start from my body, as is usually done, you will never make me understand
    how impressions which concern that body alone, are able to become for me
    independent objects and form an external world. But if, on the contrary,
    {\em all images are posited at the outset}, my body will necessarily end by
    standing out in the midst of them as a distinct thing, since they change
    unceasingly, and it does not vary.}{MattMem}{ I;p.47 [my emph.]}
  
  \fre{Dur\'{e}e} can be taken as a close analogue of \co{chaos} if we emphasize
  this aspect of simultaneity rather than succession, and then dispense with any
  attempts of identifying any particular contents.
%\co{chaos} is an impersonal field of \co{distinctions} (not yet any images)
}


\subsection{Distinction}
\pa\label{th:cut} To \co{distinguish} means to {cut} out of the formless
background of the \pexp. To ``connect'' means to connect distinct somethings; it
presupposes distinct \thi{thats}, and the difference between {now} and
{then}, between {here} and {there}.  In order to connect we first have
to \co{distinguish}.

To \co{distinguish} means, in the primordial sense, to encounter {\em for the
  first time} -- and only once; to encounter the ... entirely new, the ... never
encountered before. But this ... isn't anything specific, it isn't \thi{this
  something} as distinguished from \thi{that something}. \co{Distinction} does
not involve relation, it does not distinguish \thi{this} from \thi{that}.
\co{Distinction} is anything that makes a difference; but making a difference
does not require being noticed, making a difference does not require being
perceived. It \co{distinguishes} ... from the background and thus makes it
appear -- not {\em for} anybody, not {\em for} so beloved consciousness which
can hardly be postulated at this stage, but just {\em appear}: in the middle of
\co{indistinct nothingness}.

\thesisnonr{ Before anything \co{recognisable} emerges, something must be first
  \co{distinguished} from the background. \co{Distinction} is a {cut} from
  the \co{indistinct}: formless -- and hence also timeless -- background.  }
 
And how and when did the first \co{distinction} appear? Ridiculous question.  As
far as I am concerned, I do not remember. If there was a time when I did not
\co{distinguish} then, \la{a fortiori}, I couldn't remember this time. Because
to remember means to remember {\em something} and when there is \co{nothing}
then there is nothing to remember either.\ftnt{Hence, by the way, we got it
  other way around than those who imply that distinction presupposes memory.
  Not at this level, at least. \co{Distinction} and \co{recognition} are
  different things and the latter presuppose the former.}

\pa\label{sub:stomach} Psychologists, however, biologists or brain researchers
can say something more definite on this \thi{how}. Although objective
\thi{whens} and \thi{hows} do not concern us so much, let us, nevertheless, make
a short detour into infant psychology since it may provide some concrete
illustrations.

Neonates have preference for sweet taste but, for instance, \citet{weak saline
  solutions are not found aversive, because they have been experienced before
  birth, as aminotic fluid.}{ID1}{\noo{p.20}} There is a vast experimental
evidence on a wide spectrum of innate abilities to distinguish various stimuli,
many of which are probably present before birth and which develop rapidly during
the first months of life.

For instance, during the first weeks, the intake of food is regulated by stomach
distension irrespectively of the amount of calories in the food. \thi{Full
  stomach} is a distinction which leads to the reaction \thi{stop sucking}. One
shouldn't claim that it is distinguished {\em from} \thi{empty stomach} -- these
are just two \co{distinctions}, originating in the same organ but, otherwise,
unrelated to each other. It doesn't seem plausible to assume that a cry of a
hungry infant -- shall we say ``of an empty stomach''? -- in the first weeks of
life has any intention of filling the stomach.

At around 6 weeks\ftnt{Perhaps earlier -- this is the age for which
  experimental data have been gathered.} much more sophisticated regulatory
system emerges based on monitoring blood sugar and other, more peripheral
mechanisms (like gastric-emptying rate). After the age of 7 weeks infant's
system can no longer ``be fooled'' by low calories food which is now compensated
either by increasing consumption or frequency of feeding until the energy needs
are met. These are, of course, the terms in which we describe the observations
but they do indicate that the system attunes and reacts to new \co{distinctions}
which weren't registered before.

\noo{ Certainly, more sophisticated abilities to \co{distinguish} emerge very
  quickly from the reactive, autonomous ones. But it seems reasonable to assume
  that the latter provide the initial impulse and ground for the appearance of
  first \co{distinctions}.  }
 
\pa\label{pa:distwhat} And now, more interestingly, what is it that can so be
\co{distinguished} from the background? One would say, a thing, \thi{this table
  here}. Perhaps. And things too are {cuts} from the background. But
\thi{this table here} seems rather too complicated -- or better, too specific
and precise -- an object. A \thi{that} would be better, but the question
concerns exactly what, from our later perspective, such a \thi{that} might be.
Possibly, it can be a sensation, especially an intense one, heat or warmth,
pricking. One only should keep in mind that it is not {\em my} sensation -- it
is a sensation of a newly born (or, perhaps, very primitive) being.  It is an
autonomous sensation of a peripheral organ provoking possibly, but not
necessarily, a reaction which need not be mediated through the central nervous
system.

\label{sub:psychexa}
But original \co{distinctions} can be much more than just sensations.

\begin{enumerate}\MyLPar
\item Vision acuity in the neonates is somewhere between 10 to 30 times poorer
  than in the adult.\incitt{ID9}{p.224} It develops rapidly and from about 6
  months it is at near-adult level but initially it prevents the infants from
  making all too fine visual \co{distinctions}.
\item 2-day-olds shown a square in different slants, on the test trials
  preferred to look at a novel shape, trapezium, even if the slants might have
  given the appearance of a square.\incitt{ID2}{p.33+} Similar experiments with
  cubes of various sizes indicate that, to begin with, infants prefer stimulus
  giving the largest retinal image size. However, when desentisised to the
  changes in the distance (and hence retinal size) of a constant-size cubes,
  they strongly preferred different-sized cubes on the subsequent trials. The
  experiments suggest that infants as young as two days, can distinguish both
  slant and retinal size but also have the ability to perceive objective, real
  shape and size.
\item 2-day-olds show preferences for moving, rather than stationary stimuli. At
  about 6 weeks they can distinguish coherent, biomechanical motion and prefer
  it to a random one. These tests did not present any moving objects but merely
  point-light displays produced by filming a person in the dark who has points
  of light attached to his major joints.\incittib{ID2}{p.43} The general
  \co{distinction} of organised motion can be thus thought as an \thi{innate},
  or original, 
  category of \co{distinction} which need not evolve from the subsequent
  observations of moving objects.
\item Few-days olds fail to recognise a three-dimensional object after having
  being shown its photograph which indicates that the \co{distinction} between
  two- and three-dimensional objects is more significant than their possible
  similarities.\incitt{ID9}{p.226}
\item 4-day-olds seem to have a preference for human speech above music, and
  music above noise.\incitt{ID13}{p.312+} A one week old girls can
  \co{distinguish} a baby cry from a background of general noise of a similar
  volume.
\item 1-day-olds seem to discriminate visual stimuli according to the degree of
  contrast rather than to the actual pattern.\incitt{ID9}{p.230} Still, given
  approximately the same contrast, face or face-like pattern will be preferred
  to other patterns. Within few hours after birth, neonate's reaction to
  mother's face shows that the face is \co{distinguished}, perhaps even
  \co{recognised}.
\item\label{it:comm} There is some evidence that infants are attuned not only to
  face-like patters but also to the emotional expressions, including facial,
  vocal and gestural movements of others.\incitt{ID12}{p.293+} 1- to 2-day-olds
  can focus on and imitate -- in a voluntary and not merely reflexive effort --
  a wide range of expressions, like sad, happy and
  surprise.\ftnt{\citeauthor*{ID10},\noo{p.249+} \citeauthor*{Meltz83,Meltz89}.}
  
  One study shows that two to four days old girls spend twice as long as boys
  maintaining eye contact with a silent adult and also look longer than boys
  when the adult is talking. Also, baby girls are more easily comforted by
  soothing words and singing -- even before they can understand language, they
  seem to identify the emotional content of
  speech.\ftnt{\citeauthor*{MaleFemale, BrainSex}}
  
  Few-weeks old infants show clear preferences for happy rather than sad, as
  well as attractive faces.\noo{ID9,p.241} They also seem to discriminate faces
  according to gender.\incitt{ID9}{p.239}.
 % Eduard Seidler, Der Neugeborenversuch Friedrichs II von Hohenstaufen, in
 % Deutsche \"{A}rztblatt, No.39/26.11.1964; also Ernst Kantorowicz, Kaiser
 % Friedrich der Zweite, 2 Bde., Dusseldorf, 1963.]}
  
  In a famous experiment\incitt{TVproto}{also IDp.281+} a 2-months-old and
  mother were sat in front of a television monitors showing the other's face.
  When the images were transmitted in real time, the two engaged in an apparent
  proto-conversation, initiating communication and responding to the other's
  signals. When then video was replayed with a delay of 30 seconds, and thus
  mother's expressions did not represent any adequate responses, infant showed
  considerable distress, turning away from and darting brief looks back at the
  mother's face.\ftnt{Similar distress was observed in infants confronted with
    mothers assuming indifferent \thi{blank face} posture or affected by
    postpartum depression, \citeauthor*{IndifMom,IndifMom2}.
    
    The infamous experiment of Friedrich II (in which new born infants, deprived
    of any emotional and verbal contact with adults, died all together) was
    interpreted already by the ckontemporary chronicler in the same way as we
    would tend to interpret it today (if only the experiment was recorded
    adequately; as it is, one could easily imagine all kinds of reasons of the
    deaths other than the lack of contact).  E.g. \citeauthor*{Hohen}, also
    \citeauthor*{HohenKantor}.}
\end{enumerate}
 
The intention is neither to give an overview of this research nor to suggest
that it constitutes a proof of anything. \noo{(The best a philosophy looking for
  proofs can generate are ingenious arguments -- unfortunately only of logical,
  not real value.)}  But it provides illustrations, if not sufficient grounds
for some of the following claims.
 
\pa\label{pa:norigid} \co{Distinctions} emerge gradually, in a top-down fashion,
from the general and diffuse ones, they gradually become more acute and precise.
They involve initially only some rough, vague categories rather than sharp,
specific differences.  Every \co{distinction} is, on the one hand, `real' or
`true' in that it arises from the background, it pulls something out of the
undifferentiated homogenity of the \co{one}. On the other hand, however, it is
`uncertain' or `unsharp', it does not draw an absolute, definite border between
$x$ and not-$x$, it merely sketches the \co{distinguished} pole. It is like the
adjacent stripes of the rainbow, mutually distinct but without any definite
boundary separating them from each other.  \co{Distinctions} are like waves:
here is one, there another, and there yet another, but where one ends and the
next begins, nobody can tell. We only can point to the peaks and thus be sure
that there are, indeed, different waves.\ftnt{Analogous phenomena are observed
  in experiments (with adults) concerning span of apprehension. When multiple
  letters are briefly exposed, only some are identified although all are
  certainly seen. The subjects insist that they were there, clear and distinct,
  can even tell their number but can not identify them.}

The so called \wo{paradoxes} of Heap or Sorites\ftnt{One stone is not a heap,
  neither are 2, nor 3 stones, not 4, nor... So when do we get a heap? Likewise,
  removing the stones, one by one, from a heap, when exactly does the heap cease
  to be a heap? Man with no hair is bald, and so is man having only 1 hair, or
  10 hairs, etc.  But when exactly does man cease to be bald?}  appear as
paradoxes only under the assumption that \co{concepts} draw rigid
\co{distinctions} with uniquely identifiable boundaries and, as a consequence,
uniquely determined negation. But although this may be the case as long as one
plays with the \co{concepts} obtained by \co{distinctions} and constructions
within the sphere of prior \co{reflective dissociations}, it does not apply
  generally to
the contents \co{experience}.\ftnt{These formulations may
  sound a bit cryptic at this point.  Section \ref{se:reflection} below should
  provide enough clarification; \co{concepts} are then discussed in
  II:\ref{sec:levelB}.}  But primarily \gre{panta rei} and even later, in the
world of words, \co{concepts} and apparently rigid \co{distinctions}, everything
still flows into another, the imperceptible shades of meaning attached by
different people to the same understandings make them unexpectedly drift apart,
as the differences come forth and drag one and the same thing in opposite
directions.  Moreover, a \co{distinction} can be always refined, made more
\co{precise}. Yet, although never reaching the final, definite, rigid form, the
\co{distinctions} exhaust the content of the world, for as Dr. Johnson observed,
the fact of twilight does not mean that one cannot tell day from night.
%in Anscombe, War and Murder [The Many Faces of Evil, p.301]


\pa From the very beginning \co{distinctions} aren't limited to sensations but
concern structures and objects which, according to traditional empiricism, would
have to be ``constructed'' from the material of minute sensations.  Furthermore,
a {thing}, understood as a definite, well-defined object, is by no means a
fundamental component of our \co{experience}. What is \co{distinguished} from the
formless background is pretty accidental and it may be just ... anything.  It
may be an \thi{abstract} property, like a shape, a size, or a colour; it may be
warmth or movement, an emotional expression of another or a feeling.

The original \co{distinctions} do not discriminate between different {\em kinds}
of objects because one thing is not distinguished from another but from the
background. Everything counts equally: properties and relations, some actual
{things}, sensations, changes, motions, continuous processes not composed of
any parts, feelings, emotions.  No things are more fundamental than others.
Before we can begin to \co{experience}, we have to first make enough
\co{distinctions}, from which the later \co{experience} may be built.
\thesisnonr{There is no hierarchy of original \co{distinctions}; no
  \co{distinctions} are more fundamental than others.}

Sure, something which later will be called a ``thing'' can be
\co{distinguished}, too. This table can be \co{distinguished} too. To begin
with, it is nothing, or else, as an integral part of the background, it is not
at all. There may be a play of lines and shades which run indiscriminately
through the table, the wall, the windows. In the psychological experiments one
always attempts to make sure that infants are presented with a distinct stimuli
not to be confused with other unintended elements of the environment. But even
such a complex thing as a table, when somebody pushes it aside thus effecting a
\co{distinction} of the sub-chaos of lines, forms, shades and colours from the
surrounding chaos, and, at the same time, giving them totality, may give this
sub-chaos, which we call table, a new status -- of something \co{distinct} from
the rest.\ftnt{Common motion is used, for instance, to test infants for object
  permanence.}  \thesisnonr{ Originally, things, like other \co{distinctions},
  are just {cuts} from the indistinct background of \pexp.  }

\pa This lack of any organisation of the first \co{distinctions} involves, in
particular, lack of any temporal discrimination. There is no time, and a
\co{distinguished} {thing} counts equally with a \co{distinguished} sensation
or a \co{distinguished} emotion.

It remains to be seen if the \thi{square shape} is \co{distinguished} on the
basis of generalisation or innate predispositions.  But in any case, {\em if}
this is what is \co{distinguished}, then it does not involve only a purely
\co{actual} object but something, well, universal.

Hunger is something that does not appear just like that. It increases gradually.
When it eventually hits the barrier at which I say ``I am hungry'' and an infant
begins to cry, it involves not just this moment now but the continuity of the
whole development, of its gradual increase. It is never so that I am not hungry
in one moment and then, in the next, I suddenly am. \co{Experience} of hunger
involves something which is not, seen \thi{objectively}, purely \co{actual}.

Fear aroused by possibly very different circumstances, the atmosphere of love
and acceptance not connected with any specific person or actions, security or
insecurity, all kinds of emotions which, unlike sensations, cannot, in general,
be classified as arising and occuring in a specific moment, are among things
which can be \co{distinguished} along with colour, shape, size, motion.
\thi{Objectively} speaking, these \co{experiences} require more time to occur,
but since the time has not yet begun to flow, they all are equally just
{cuts} from the indistinct background. \co{Proto-experience} itself is
timeless and knows not only no difference between \thi{this} and \thi{that}, but
neither any between 1 second, 20 minutes and 5 days.  Consequently, something
which is later determined as an object can be \co{distinguished} in the same
way, on the same footing, and with the same status, as something we later will
call a ``property'', a ``complex'', a ``process'', a ``feeling'', a
``conjunctive relation''.
%
\thesisnonr{ The primordial \co{distinctions} are not limited to \co{objects}
  given within a \herenow; they may bring forth something actual, like this
  colour, or this table, as well as changes, processes and emotions which span
  over \thi{long periods of time}.  }
%
\noo{ analogy in normal experience: 1. \co{recognising} for the first time
  something I didn't see before; suddenly a crucial distinction, idea;
  \co{recognising} means {\em for the first time}; 2. recognition is a secondary
  act based on memory.
  
  \pa Reactive, autonomous, but not only; unrelated, not one-from-another }


\subsection{Signification}
\pa\label{pa:signA} In spite of its indeterminate, timeless and spaceless
character, despite its reactive and objectless character, despite its entire
lack of relations, \co{distinction} involves a virtual \co{signification}. For
the moment, not in the sense of one thing signifying some other thing, but
merely in the sense of {cutting} off the \co{actual} \co{distinction} from
the \co{rest}, from the background. The former, except for being
\co{distinguished}, or better, precisely by being \co{distinguished}, involves
also an immediate reference to the background from which it emerged. In this
sense it is a \co{sign}, a \co{sign} of all the \co{rest}, of all that was left
behind when the \co{distinction} has been made.

This \co{aspect} of a \co{sign} in every \co{distinction} expresses merely the
mutuality of the two poles, the fact that \co{distinction} arises only from
something which, only from now on, can be properly called background. It is the
seed of two later poles of \co{actuality} and \co{non-actuality}. Everything
\co{actual} will always be interpenetrated by the \co{non-actual}, every \heno\ 
by \thth. And this {signifying} reference is not the result of abstraction or
successive \co{experiences} but the very beginning of \co{experience}. As a bare
reference to the indefinite and indefinable \thi{something more}, \thi{all the
  rest}, it will be later involved in all life, consciousness and, in a
derivative form, in all specific signs and representations.

\pa\label{pa:externalWorld} The primordial \co{signification}, as the
\co{aspect} of the first modification of the \co{original confrontation},
\co{founds} the permanent and indissoluble awareness of \thi{something being out
  there}, expressed by the common uneasiness with all kinds of solipsism and
subjectivism, as well as by Samuel Johnson's refutation of idealism.  Every
\co{actual} object and situation, every \co{actual experience} is haunted by the
all-permeating shadow, the \co{non-actual rest}. But \co{experience} is only
``haunted'' by it, because \co{experience} is always directed to something more
specific, never to this indefinite \woo{murmur of being.}{E.~Levinas}

We do not have \co{any experience} of the objectivity of the world, because this
objectivity is rooted in \co{proto-experience}, is something preceding \co{any
  experience}, and thus more primordial than the world. We do not have any
specific experience of it, only a sense of it -- as imperishable as it is
ineffable, as \co{clear} as it is \co{vague}.  The \thi{out there} comes before
any \thi{something out there}; the {separation} of light from darkness comes
before any particulars. 


\section{In the beginning was Word}\label{se:begWord}   
\secQpar{5.5}{6.5}{when all things were in disorder God created in each
thing in relation to itself, and in all things in relation to each other,
all the measures and harmonies which they could possibly 
receive.}{Plato, \btit{Timaeus}, III:37}
%
\pa \co{Chaos}, though created from \co{nothingness} and thus, in a way, opposed
to it, is not yet the world.  It is like the \la{materia prima}, or even
\la{confusa}, from which world can be created.  In the world there are no
\co{pure distinctions} but things -- things which may disappear and then return
because they have some identity, some {sameness}; things which can be seen and
thought because they can recur, that is, be \co{recognised}.

To \co{recognise} is to {signify} and \co{recognition} is a \co{sign}. This
structure of \co{sign} is no longer a mere \co{signification}. It means that
something \co{actual} carries with it the burden of \co{non-actuality}, it
points towards something that is not entirely \co{here-and-now}.

It points from \herenow\ to somewhere else or sometime else, \thi{outside} of
\herenow.  A \co{sign} is something which points from the \hoa\ somewhere
\thi{outside} it, which makes something \co{transcending actuality} present.
\noo{What this \thi{pointing} itself is, or what it consists of, is another
  matter.  There are many ways of establishing such ({pointing}) connections
  between appearances, things, concepts.  And there are different kinds of
  {pointing}, or relationships between things.  What matters here is that all
  such kinds of relating are secondary and occur only after \co{distinctions}
  have been introduced making available things which one, subsequently, can
  attempt to relate.}  Things are \co{signs}, words are \co{signs} and
\co{signs} are what make the world emerge from \co{chaos}.

\pa\label{pa:speculation} The exposition of \pexp\ can be summarised thus: in
\pexp\ something is \co{distinguished} but nothing is \co{recognised}.

We have to speculate to the extent that we do not \co{recognise} and do not
remember. {Experience is \co{pure}, is \co{chaos}, to the extent it does not
  involve \co{recognition}.} Any talk about it is thus bound to be a
speculation.  Or, if you prefer, it is a mystery how God created the world from
\co{nothingness}, how He {\em divided} light from darkness and the waters under
the firmament from the waters above it. But it is also a mystery how He, having
separated these \co{virtual} elements, created the things which we
\co{recognise}. But our concern is not with \thi{how} but with \co{that} and
\thi{what}.

\co{Pure distinction} does not distinguish $A$ from $B$, but only brings forth
$A$ against the \co{indistinct} background. \co{Recognition} is to \pexp\ what
\co{distinction} is to \co{nothingness}: it {cuts} off what is
\co{recognised} from the \co{chaos}, from the formless horizon of \pexp. It
brings forth not only a \co{pure distinction}, an unrelated and unconnected
\thi{in itself}, but an appearance, that is, an appearance of {\em something}.
To appear is the same as to be \co{recognised}; nobody would like to speak about
unrecognised appearances (although this, like most expressions, can also be
given some meaning). But also only appearances bring forth \thi{somethings}.
Properly speaking, only from now on the word \wo{something} can mean something
which is not a mere reflex but a \thi{this}, which has some character and
sameness.

To be sure, \co{recognition} is not \co{reflective}, it is not made by \co{you},
but it is what will make \co{reflection} and \co{yourself} possible.


\subsection{Sign and recognition}
\pa\label{pa:signB} \co{Recognition}, the \co{separation} of \co{actuality} from
\co{non-actuality}, \co{founds} a \co{sign}: not any more a merely \co{virtual}
\co{signification} of \thi{all the rest} as in \refp{pa:signA}, but a \co{sign}
of {\em something}.  \co{Recognition} refers \thi{this here and now} to
\thi{that then and there}, brings forth something \co{actual} {\em as} something
else.  Any connection between \co{distinctions} involves a \co{sign} in this
elementary sense that an \co{actual} appearance signifies another,
non-\co{actual} or even \co{non-actual} one.\ftnt{We will later distinguish
  between the non-\co{actual} things 
  which could be \co{actual} but just happen not to be \herenow, and the
  \co{non-actual} ones which never can be fully \co{actualised}. Non-\co{actual}
contents presuppose \co{non-actuality} and so, occasionally, we may omit the
latter designation leaving it implicit.}
As yet no \co{sign} appears as a sign -- \co{sign} means 
here just the \thi{re-} of the immediate \co{recognition}. It is a \co{sign} in
the sense that something \co{actual} points to, or just is continued in {\em
  something} else, something non-\co{actual} or \co{non-actual}, even if the two
are immediately 
merged into one and thus the \co{sign} is entirely transparent.

And how and when did the first \co{recognition} occur? Ridiculous question,
indeed. As far as I am concerned, I do not remember. If there was a time when I
did not \co{recognised} anything then I couldn't remember that time. Because to
remember means to remember {\em something} and when there is \co{nothing} then
there is nothing to remember either. We let the child psychologists say a few
more words concerning these \thi{hows} and \thi{whens}.

\pa A confusion might have arisen with respect to the word \wo{preference} when
relating the experiments on neonates in \refp{pa:distwhat}. We said that
\co{pure distinction} {cuts} something off the background and not from
another something, while one prefers something to something else. Infants'
\wo{preferences} is the name given by the psychologists to what is measured in
the quoted experiments: the reaction to or time span of attention to given
stimuli, the longer time span indicating the greater ``preference''. It is
certainly disputable whether this is what such a time span reveals. For our part
it is sufficient to assume that different reactions indicate the fact of
\co{distinguishing} the given stimuli.

More significantly, we wouldn't claim that \co{distinguishing} necessarily
precedes \co{recognising}, that they are really dissociated and temporally
ordered events. The two may overlap and while some things are \co{recognised} at
some stage, others may be \co{distinguished} only later.  The very fact that the
experiments involve, first, a habituation stage, when infants are familiarised,
possibly, desentisised, to some stimuli, which are then used in the trial tests
may be taken as an indicator that from the very first hours infants also
\co{recognise} various stimuli. There seems, however, to be sufficient
differences in the character of the stimuli to which infants respond, to
classify them into \co{distinctions} and \co{recognitions}.

\noo{Reactive:actual vs. $\_\_\_$:permanent}

\pa\label{sub:psychexb} One of the most significant changes concerns object
permanence observable, for instance, in that infants search for hidden objects.
According to Piaget\incitt{PiagetReality}{also ID5,p.111} such a search that is
guided by a representation of the hidden object develops in the final stage of
infancy, after 18 months when infants fist show capacity for symbolic
representation. The recent research, and techniques of habituation, indicate
that, whether guided by representation or not, perception from very early age
has an objective character.\ftnt{Our \co{recognition} has little to do with
  \co{representation} which appears at yet later level of experience.}
\begin{enumerate}\MyLPar
\item 4-month-olds were habituated to a rod which moved back and forth behind an
  occluder, $A$, so that only the top and the bottom of the rod was visible.
%\begin{figure}[hbt]\refstepcounter{FIG}
%%% this should work on Mac: With \input epsf in whole.tex!
%- but it doesn't
\begin{center}
\epsfxsize=8cm
%MAC: \epsfbox{Pschrods.eps} 
\epsffile{Pschrods.eps} %UNIX
\end{center}
% \caption{Object completion}
% \end{figure}
% \noindent
On subsequent trials the babies were shown two test displays without the
occluder, one, $B$, being a complete rod, the other, $C$, being the top and the
bottom parts, with a gap where the occluder had been. The babies spent more time
looking at the two rod pieces.\incitt{ID2}{p.46++} One is more than willing to
interpret it in the obvious way: the original common motion of an occluded rod
leads to object completion -- perception of one moving object; two unoccluded
separate pieces are then a kind of surprise to a four month old infant.  Babies
younger than 4-months, however, perceive the complete rod as novel.
\item It is important to emphasize that the object permanence does not seem to
  be a matter of a definitive either-or.\incitt{ID5}{p.125+} It is suggested
  that permanence, as well as representation, develop gradually, from briefly
  permanent objects to solid and varying representations.
\item 3- to 4-month-olds perceive subjective contours.
%
% \begin{figure}[hbt]\refstepcounter{FIG}
\begin{center}
\epsfxsize=12cm
%MAC: \epsfbox{Pschcont.eps} 
\epsffile{Pschcont.eps} %UNIX
\end{center}
% \caption{Subjective contours}
% \end{figure}
% \noindent
Infants familiarised to pattern $A$ discriminated in the subsequent trials this
pattern from those, like $B$-$D$, not containing subjective contour.  Infants
familiarised to $B$, however, did not discriminate between it and other patterns
without subjective contours. The difference between patterns with and without
subjective contours seems to be greater than the difference between patterns
without such contours.\incitt{ID2}{p.39++}
\item At about 4 months one can observe \co{distinction} of perceptual
  categories. Familiarised to a number of stripped patterns which differed in
  their orientation but were all oblique, the infants on the test trials
  ignored other, new and differently oriented oblique patterns and preferred a
  novel vertical pattern.\incittib{ID2}{p.41+} Similarly, shown distorted versions
  of some prototype shape (square, triangle), on the test trials infants treated
  the never before seen ideal shape as familiar and directed their attention to
  a distorted, even if seen before, version.
\item If 3-month-olds are shown a series of pictures of horses, within a few
  trials they will form a perceptual category of horses that excludes zebras and
  various other animals.\incitt{ID7}{p.165++} Interestingly, 7 to 11 months olds
  showed categorisation at a \thi{general level}, differentiating animals from
  vehicles, even birds from aeroplanes, without, by the same time, categorising
  at a \thi{more basic level}, i.e. without differentiating dogs from fish or
  dogs from rabbits.\incittib{ID7}{p.178++}
\item Again, as in the case of \co{distinctions}, infants seem much earlier
  attuned to communication with others.  Not later than at six to eight weeks,
  this attunement is not limited to the purely \co{actual} situation.  Following
  disruptions in the other person's communicative behaviour, 6- to 8-week-olds
  attempt a series of other-directed acts. If the adult is unresponsive the
  infant may increase the intensity of the proto-conversational expressions,
  like vocalisation, arm movements. When the other is looking elsewhere, infants
  develop ways of ``calling'' the other, for instance, with a shrill
  vocalisation with a pleasant expression and gaze on the other's
  face.\ftnt{\citeauthor*{ID10,ID12}}
%\incitt{ID10}{p.251++} \citt{ID}{p.294++}
\end{enumerate}

\pa As there was no order and hierarchy among the emerging \co{distinctions}, so
there is no such inherent order in \co{recognitions}. A rod can be
\co{recognised} as well as a shape, a perceptual category as well as a feeling,
a friendly and coherent response of another person, as well as an object.

\co{Recognition} of object's permanence testifies to the emerging
\co{distinction} between the \co{actual} and the non-\co{actual}, even
\co{non-actual}, where the 
\co{actual} functions as the \co{sign} for the \co{non-actual}, but also for the
unity of both aspects. This is the stage where we can properly start talking
about signs in the more common sense of the word.


\pa\label{pa:intentional} What distinguishes the above examples
%\refp{sub:psychexb}
from those concerning \co{distinctions} in
\refp{sub:psychexa} is the non-\co{actual} or \co{non-actual} element
discernible in the infants' 
reactions.  Infant sees two separate parts but fills this \co{actual} stimuli
with the missing part and perceives one rod.  The disrupted communication with
the other makes the infant attempt to attract the other's attention.  Shall we
call it ``intentionality''?  Why not?  Thus understood, intentionality is just
the presence of the non-\co{actual} or \co{non-actual} in and through the
\co{actual}, the pressure it exercises on 
the \co{actuality}.  It is a more crisp, more definite modification of the virtual
\co{signification} from \refp{pa:signA}.  But the word ``\co{sign}'' might be
preferable -- it captures better this relation of the \co{actual} functioning
here as a \co{sign} of something which is not \co{actually} given.  The objects
appear only to the extent they transcend the pure \hoa, or better, this is what
gives an appearance an objective character.  An appearance, a \co{sign} is
\co{actual} through and through, is exhausted within the \hoa.  An object, on
the other hand, bears the dual character consisting of the \co{actual} and the
non-\co{actual} moments.  This non-\co{actual} aspect is what gives it the
objective character surpassing the subjectivity of mere appearance.  Pure
\co{subjectivity} means pure \co{actuality}; \co{subjective} is only what is
exhausted in an \co{actual} experience and does not hide, does not keep anything
for itself, like the passing feelings and immediate sensations.

The presence of the non-\co{actual} and \co{non-actual} in all \co{experience}, its penetration
through the \co{actuality}, may be, perhaps, called ``intentionality''.  Even in
the least reflective moments of \co{experience} one hardly ever focuses
exclusively on the merely \co{immediate} sensations, everything is always
interwoven into the texture of other, non-\co{actual} experiences and eventually
of \co{non-actuality}.  The
transcendental subject is what, in the philosophy of pure \co{actuality}, plays the
role of the \co{non-actual}; it is responsible for endowing the \co{actual}
contents with the noumenal identity, that is, inaccessible source of all
possible contents.  For idealism this constitution, like everything else, must
happen instantaneously, \thi{now}.  But intentionality need not be so
\co{dissociated} from the rest of \co{experience}; it is only its \co{actual}
moment, the focus which gathers various threads into the totality of the
\co{actual} phenomenon.  \thi{Intentionality} is a phenomenological expression
of the co-presence in \co{any experience} of the \co{non-actual aspect} which,
nevertheless, is viewed exclusively from the perspective of \co{actuality}.
What constitutes the unity of these two \co{aspects} is just the fact that it is
gathered, appears -- true, itself, but only through a \co{sign} -- within the
\hoa.  Even if two parts, and not one rod are present, the whole one rod is now
perceived, and if, in the next moment, it turns out to be two after all, it is a
surprise.  Life is full of such surprises and no transcendental constitution of
objects of adequate intuition can force it to always conform to our
expectations, sorry, intentions.

\pa\label{pa:languageA} We have not yet attained a full \co{dissociation} of
\co{actuality} from 
non-\co{actuality} and 
\co{non-actuality} which still remain in the unity of a \co{virtual} \nexus. But
they have begun to be \co{distinguished}, to play the role of two \co{distinct
  aspects}. It is therefore too early to speak about abstract \co{signs} which we
will encounter in the following section. Yet words, or at least some vocal
signs, begin to appear and function as \co{signs} at this early stage. It is,
moreover, exactly the relative lack of \co{dissociation} of such a \co{sign}
from its meaning (characteristic for the current level) which accounts for its
{\em creative} role.

A child learning its first language (or languages) is not in the position of an
adult who recognises different contents and only has to attach to them
appropriate linguistic expressions. For a child the words are tools, as many
others, of drawing the \co{distinctions} in the matter of \co{experience}.
Acquisition of the first language proceeds along with the process of
differentiation in which no difference is given between the \co{actual}
\co{sign} -- the word -- and the distinguished, signified content. Both emerge
simultaneously and words are not merely \thi{attached} to things but are the
\co{signs} which bring things forth.  Providing the means of drawing the
\co{distinctions} (the most elementary example might be recurrence of particular
words in situations from which, eventually, the child distinguishes the element
signified by the recurring words) and organising the \co{chaos} of
\co{distinctions}, the first language contributes to the creation of the
world.
% given to the word in the Apocalypse. Sure, one may \la{ad infinitum} stress that
% the word there is God's word and not the common word, etc. In the view of the
% above, however, I do not see such a pressing need to distinguish the two.

It is thus also understandable why learning the first language is so natural and
easy while later learning a foreign language so difficult.  Learning the first
language is learning the world, is the emergence of the world for the first
time. It takes some years, but it also happens without conscious effort -- the
language comes to a child as naturally as the world does.  For one who has
already acquired the world, learning a new language implies almost always
translation of his world to the other world of the foreign language. This is the
most challenging part of foreign language acquisition which is responsible for
the associated difficulty. A person speaking fluently another language will
never translate it into one's own -- he will live in the world offered by that
other language as naturally as he does in his original world. What may happen
here in terms of fusion of different worlds, and how the difficulty of learning
a foreign language increases with the linguistic distance between the two, are
certainly interesting topics, but would bring us too far off our main track.

\noo{ \co{Distinction} and \co{recognition} coexist later: entirely new
  \co{distinctions} may emerge at any time but also \co{recognitions} may lead
  to new \co{distinctions} (abstraction -?- new \co{distinctions} through
  \co{recognition}).
  
  A \co{distinction}, a \co{separation} from the background is not a mechanic
  process. Mechanisms, machines, too, can ``distinguish'' and ``react''. The
  difference is that any machine I know of ``distinguishes'' $A$ from $B$,
  \thi{too high temperature} from \thi{too low}, a sphere from a square, etc.
  Some more advanced programs can even, to some degree, simulate learning by
  abstraction. But all this happens within a framework of some basic
  distinctions predefined by the designers. Abstraction, even if it leads to
  introduction of new distinctions, is a different process, opposite to
  \co{distinction} which differentiates \co{nothingness}...
} %end \noo


\subsection{Actuality}\label{se:temp}% spatio-temporality
\pa The lack of the spatio-temporal aspect in the \co{chaos} of
\co{distinctions} is based on the lack of any proper difference between the
\co{actual} and the \co{non-actual}. There is only simultaneity of
\co{distinctions} and the \co{virtual} \co{signification} which refers a
\co{distinction} to the \co{indistinct rest} but not to any specific,
non-\co{actual} or 
\co{non-actual} something. Before \co{recognition}, \co{pure experience} is
heterogenous but continuous or, if you prefer, simultaneous -- in short,
\co{spatial} (in very rudimentary sense) but timeless, even non-temporal.

\pa Like \co{distinctions}, \co{recognitions} are not limited to minute
\co{immediacies}. Although an element of temporality has emerged through the
fusion of \co{actual} and \co{non-actual}, \co{recognitions} are just {cuts}
from the \co{chaos} with no \thi{objective} time-stamp on them. Whether it is a
missing part of an object, a general schema of several instances, a lack of
other's attention or a lasting feeling of satisfaction -- the \thi{objective}
duration does not matter for the event of \co{recognition}.

Something born at one moment and dead two hours later wouldn't be able to
\co{recognise} \thi{a day}. There is no specific moment -- no single \co{act} --
when we encounter \thi{a day}. Day, by its very nature, lasts, i.e., cannot be
embraced by a single \co{act} within the \hoa. (If it were explained to this
something what \thi{a day} is, it might understand it, perhaps, acquire a
concept of \thi{a day}, but this would require development of the understanding
of the objective world.) What takes time is not
to develop a \co{concept} of \thi{a day} but to have enough \co{experience} to
be able to {cut} from its background a unit which is denoted by this word.
And initially, {\em in illo tempore}, what is \co{recognised} as such units may
be anything which only later \co{reflection} will classify as single things or
\co{complexes}, as \co{immediate} sensations or anything endowed with temporal
duration. But temporality has not yet entered the stage; a \co{sign} involves
only a primordial separation of its \co{actuality}, on the one hand, and its
meaning, on the other; the meaning which may embrace \co{distinctions} not only
not \co{actual} at the moment but genuinely and essentially \co{non-actual}.

Thus, although objectively speaking \co{recognition} requires some passage of
time, so from the point of view of \co{experience}, there is no time before
something has been \co{recognised}. \co{Recognition} is not a repeated {\em
  earlier} cognition, it does not juxtapose two separate images. It merely
fixates an \co{actual} \co{sign} as something involving also \co{non-actuality}.
It {cuts} off the \co{recognised} {something} from the \co{chaos} of \co{pure
  distinctions}. This {separation} brings forth -- in fact, is -- the
{separation} of \herenow\ from there-and-then, or better, of \herenow\ from
not-\herenow. It \co{founds} \co{actuality}, where what is \co{recognised}
appears, and which is distinguished from the \co{non-actuality}, from the rest
of the -- first now, only potential -- things.\ftnt{\label{ftnt:NowDuration}As
  always, we have a \nexus\ of \co{aspects} which here involves emergence of
  \co{actuality}, \co{recognition} of some content, and several others to be
  observed in the rest of this section. With respect to the \hoa, one might be
  tempted to say something like: \citef{The original time field is obviously
    limited, exactly as perception is. In general, one might well dare the claim
    that the time field always has the same extension. It kind of slides over
    the perceived and freshly remembered movement and its objective time, like
    the field of vision moves over the objective space.}{Zeit}{ A:2.\para 11
    \orig{Das origin\"{a}re Zeitfeld ist offenbar begrenzt, genau wie bei
      Wahrnemung. Ja, im gro{\ss}en und ganzen wird man wohl die Behauptung
      wagen d\"{u}rfen, da{\ss} das Zeitfeld immer dieselbe Extension hat. Er
      verschiebt sich gleichsam \"{u}ber die wahrgenommene und frisch erinnerte
      Bewegung und ihre objektive Zeit, \"{a}hnlich wie das Gesichtsfeld
      \"{u}ber den objektiven Raum.}} Of course, the expression \wo{the same
    extension} can hardly mean \thi{definite and fixed in minutes or seconds
    objective duration}. It may rather intend the fact that \co{actuality}
  retains its character of a horizon, its identity as constitutive for a new
  level of being, also across the variations in its objective duration. But it
  is not constitutive in isolation, for it is only one \co{aspect} of a \nexus\ 
  and \thi{now} is relative to other \co{aspects}, e.g., \citefib{In the ideal
    sense, a perception (impression) would be a phase of consciousness which
    constitutes a pure now, [...] perception constitutes actuality. In that a
    now as such emerges before my eye, I must be perceiving.}{Zeit}{A:2.\para
    16-\para 27 \orig{Im idealen Sinne w\"{a}re Wahrnehmung (Impression) die
      Bewu{\ss}tseinsphase, die das reine Jetzt konstituiert, [...] Wahrnehmung
      konstituiert Gegenwart. Damit ein Jetzt als solches mir von Augen steht,
      mu{\ss} ich wahrnehmen.}}}

\pa \co{Recognition} of something from the \co{chaos} establishes the \hoa\ as
distinct from the background, which now becomes a not-\herenow.

The name ``\herenow'' should emphasize that we are talking here about elementary
\hoa, not any kind of time.  It is equally spatial and what we call
``non-\co{actual}'' (and even more so ``\co{non-actual}'') means as much
\thi{there} as \thi{then}, or rather, 
\thi{not-here} and \thi{not-now}.  \co{Here-and-now} is like a site, a location,
a designated point in the midst of its surrounding; it is not yet differentiated
into space and time.  But the further breaking of the horizon of \co{experience}
into temporal and spatial dimensions is based on this \nexus\ of the two
primordial \co{aspects}: the \co{actual} and the \co{non-actual}.  The \hoa\ 
marks \co{actuality} but there is as yet no ordering, no past or future, no
mutual relations between \co{recognitions} except that of being distinct and
that of a \co{sign}: this \co{actual} vs. that non-\co{actual} or even
\co{non-actual}. These two 
aspects mark only the first modification of the \co{spatiality} from
\refp{pa:spatialityBeforeTemporality}. Their tension will later give rise to
\co{temporality} and its directedness, but here the \thi{not yet} is still
indistinguishable from the \thi{already not}, the thirst of an expectation is
not yet different from the remembrance of a loss.

The \hoa\ encircles the simultaneity of \co{actual} \co{recognitions} which, in
turn, carry within the \co{distinction} between the \co{actual} and the
\co{non-actual}.  It isn't any longer a mere simultaneity; it is a simultaneity
which is, so to speak, doubled, followed by a shadow of \co{non-actuality}.  In
so far as it involves simultaneity, it will give rise to space; in so far, as it
involves element of non-\co{actuality} and \co{non-actuality}, it will give rise
to time.  But it 
precedes both space and time, is their \co{spatio-temporal} \nexus, the nucleus
from which the two \co{aspects} will be \co{dissociated} achieving their
eventual crispness.
%
\thesis{\label{th:temp} It is \co{recognition} -- in which something is
  \co{distinguished} from something else -- which gives this something
  \co{actuality}. \co{Recognition} means a kind of proto-focus, a
  \co{distinction} of the bare \co{actuality} from \co{non-actuality}. This \hoa
  is the \co{spatio-temporal} \nexus\ of the subsequent distinctions of space
  and time.  }
\newpage
Any attempts to relate temporality to a succession of objects or perceptions are
concerned with secondary notions, time and space or spatio-temporality of, so to
speak, higher order. They are projections of \co{reflection} and \co{reflective
  experience} of time. But by the time when we reach the reflective stage, the
\co{spatio-temporality}, interwoven into the primordial \co{recognitions}, has
sunk into the depths of proto-conscious life. We can, perhaps, reach it by
imagination but hardly by introspection.

\pa If one prefers another kind of story, it was time (Cronos, if only we are
allowed to identify him with Chronos or, as we say, \co{spatio-temporality})
who, castrating his heavenly father, Uranus, separated heaven and earth, Gaea.
Before the appearance of time heaven and earth were married, but time separated
them from each other, bringing forth multitude of distinct things on earth or --
as the myth has it -- the war with its heavenly father. Then, time keeps
devouring its own children but, eventually, just like the highest and first
gods, heaven and earth, had to give place to a more earthly time, so also time
itself, Cronos, \noo{spatio-temporality} having entered the stage at the very
beginning, has to yield its place to his son, Zeus, who does not any more rule
over the heaven but over the sky, weather, thunder and other lesser gods. Cronos
is from then on inaccessible to the earthlings, either ruling Elysium, the
Golden Age of the origins, or imprisoned by Zeus in the very depth, in Tartarus
or even in a cavity behind it.


\subsection{{Awareness} and {self-awareness}}\label{sub:selfAware}

\pan Everything we are taking about
are still primordial -- 
still only \co{virtual} -- matters.  But these \co{virtualities} constitute the
nuclei on which their more advanced forms, once developed, will rest
and to which they will always try to return. 


\pa
The notion of consciousness originates from our reflective experience, where we can 
easily differentiate between moments of reflective -- in the common language, 
yes, just conscious -- attention given to something, and the greater part 
of our \co{experience} which goes without such a particular attention. And  all 
that empiricism ever managed to say on the subject concerns 
this commonly understandable notion.

Yet, although the reflective \co{experience} is what concerns us most, we do not
have to throw away the baby with the bath water. If we weren't immediately aware
of ourselves and our activities, we could hardly pause to reflect over them.
Beholding a view I can be completely absorbed in it or, as one says, unconscious
of it. Yet if interrupted and asked ``What are you doing?'', I can immediately
answer ``I am beholding this beautiful fiord.'' The answer involves an act of
reflection, but I can give it only because I have already been \co{aware} of
what I was doing. Calling this awareness for (self-)consciousness is probably
too optimistic, but it is what philosophy of consciousness used to do.

It is here that the confusion arises and it concerns the impossibility to
discern the intended meaning of the word ``consciousness'' -- the word simply
refuses to be completely dissociated from its common meaning. No matter how
transcendental and primordial consciousness becomes, it always bears the marks
of reflection. Although one claims to be talking about consciousness which is
{\em not} reflection, the reader may be at any moment exposed to a transition in
which something follows about consciousness because it can be justifiably said
about reflection.

\pa The principle of intentionality may serve as a good example.  The principle
postulates an intentional object, that is, a definite correlate of
consciousness. Sure, what characterises \co{reflective attention} is exactly its
focus on some particular object. This break of continuity, \co{dissociation} of
a particular object and narrowing the horizon of attention to it with exclusion
of everything else, is what distinguishes \co{reflection} from the
\co{experience} otherwise.  \co{Reflection} conforms perfectly well to the
principle of intentionality and it may be a reason for its great popularity. It
makes the \co{reflective act} the paradigm for our whole being.

A lot of abstracting effort is needed to bring it down to the level of
\co{experience}, because most of \co{experience} does not conform to it.  A
feeling of restlessness is an experience which does not provide me with any
object, any intention.  It may be, but also may be not, aroused by some specific
events, but once it appears it is not directed towards anything.  In fact, it
can be characterised as a search for an object which could calm it down.  Once
found, it terminates the experience of restlessness.  Kierkegaard's experience of
\thi{Angst} made some phenomenologist invent \thi{nothingness} as its intentional
correlate.  With all respect for the ingenuity and justification of the
analyses, the whole point was that \thi{Angst} does not allow me to get hold on
its source, that no possible intention can account for the
experience.\ftnt{Heidegger, in his considerations of the moods was close to
  this realisation.  But it seems that it was Levinas who was the first to point
  out the limitations of the principle to the \co{actuality} of conscious
  experience.} It may be methodologically pleasing to substitute the experience
of \thi{nothingness} for the experience of \thi{Angst}.  But it is a straight
way to disregarding the content and the character of such an experience.  The
former tries -- in spite of all denials and explicit statements to the contrary
-- to posit an object while the ``of'' of the latter does not refer to any
object.  It does not refer to anything!  It is equally adequate to say ``an
anxious experience'' as ``an experience of anxiety'' and in both cases it is
clear that neither \thi{anxiety} nor anything else is the object of the
experience but that the phrases designate its quality, in particular, its
objectlessness.\ftnt{In fact, we can discern here a slight transition
  expressed by the two phrases.  \wo{An anxious experience} stays still close
  to the objectless quality, focuses on the unity of \co{the experience}. \wo{An
    experience of anxiety} marks already a more \co{reflective}
  \co{dissociation}, an attitude of a more distanced \thi{experiencing subject}
  to the \thi{object} of this experience which arises anxiety.  We will follow
  this transition in the subsequent sections.}  Calling nothingness the
``intentional correlate'' of the experience of \thi{Angst} is to pay a lip
service to the \noo{over-generalised} methodological postulates.
%% One 
%% may be as concerned as Heidegger was with emphasizing that nothingness is not 
%% an object. But if one also admits that the experience of \thi{Angst} is 
%% something which haunts and embraces my whole being, one has already 
%% admitted it does not conform to the Husserlian principle.

\pa\label{th:protoconsciousness}\label{th:protoawareness}
What used to be called \wo{consciousness}, perhaps, \wo{immediate
  consciousness}, \wo{non-thetic consciousness} or \wo{apperception}, we will
call \wo{\co{awareness}}.  The following may be taken as a merely normative
definition, not of consciousness in its common sense, but of its germ haunting
the post-Kantian idealists:
\thesisnonr{ \co{Proto-awareness} is \co{actuality}.
}

It is not founded upon, it does not emerge from, it is not involved in -- it
{\em is} the \hoa, the 
horizon within which all contents, all \co{recognitions} have to be inscribed in
order to become \co{actual}.  It is as much the place, the \thi{here} defined by
the position of the body and the reach of the perceptual field, as the \thi{now}
of the immediate presence.

\label{pa:simul}
All these aspects: \co{recognition}, \co{actuality} and \co{sign} are 
\equi\ \co{aspects}
of one \nexus\ of \co{experience}. \Equi, that is, 
simultaneous and irreducible to each other. Trying to account for one of
them, involves immediately the other, and that irrespectively of which
is taken as the starting point.\noo{There is, of course, no necessity involved.
  \Nexus\ can be thought differently, with some of its \co{aspects dissociated}. In
  fact, \eguin\ means just that \co{aspects} only happen to be mutually
  conditioned, not that they could not possibly be thought otherwise. For the
  most, they even get naturally \co{dissociated}  as we proceed towards
more and more \co{immediate}, more and more \co{precise} determinations.}  In
this rudimentary sense, 
\co{proto-awareness} serves merely as an abbreviation for this \equin, 
simultaneity and 
interplay, of these \co{aspects} centered around the \hoa.  And {\em nothing}
more!  No subject-object relation, no consciousness-of, no appearance-for.  It
is merely an emergence of mutually distinct, \co{recognisable} contents, whether
sensations, {things}, moods or feelings.  A play of shadows can fill the \hoa\ 
equally well as a pen or an anxious feeling.

\pa
When the \co{actual} contents emerging in \co{proto-awareness} become
\co{recognised}, we may with greater confidence speak about \co{awareness}.
There is no sharp border separating \co{proto-awareness} from \co{awareness}. As
\co{recognitions} emerge gradually from the \co{chaos} of \co{distinctions}, so
does \co{consciousness} emerge gradually from the pure \co{actuality} of
\co{proto-awareness}. The former is a modification of the latter effected by the
sufficient degree of the non-\co{actual} aspect in its contents. \co{Awareness}
is still \co{actual} but only in the sense that it is fully absorbed in the
\co{actual} \co{sign}.  This \co{sign}, however, carries now with itself an
element of non-\co{actuality} which is sufficient to indicate that it extends
beyond the pure \hoa.

\thesis{\co{Awareness} is the difference between the \co{actual} and the
non-\co{actual}.}
%
\label{pa:objects}\label{pa:distanceA}
An \co{object} is not a necessary correlate of \co{awareness}, it is only one
possibility.  (Unless we want to use the name ``objects'' for moods, feelings,
vague intuitions and the like.)  Properly speaking, we shouldn't even use the
word ``correlate'' here, unless by this word we mean an \co{aspect}, that
is, one among several elements of a \nexus, none of which has any priority above
others.  \co{Awareness} {\em is} \co{actuality}, the \hoa\ in which all kinds of
contents may emerge: some of them as \co{vague} as the original intuitions of
\co{chaos} and \co{nothingness}, as apprehension of holiness or intangible evil,
of meaning or meaninglessness; some of them more specific but still indefinite,
without any identifiable essence, as feelings and moods; yet other quite precise
and, although containing the \co{non-actual} element, emerging in an unveiled
and full \co{actuality} of a transparent \co{sign} like things and concepts or
minute sensations -- simple \co{objects} -- which are eligible to a complete
grasp by the \co{acts} of \co{reflection}. Appearances are \co{actual},
everything that appears does so only within the \hoa. For most contents which
themselves can not be fully fitted within this \co{horizon} this means that they
appear exclusively through their \co{actual} \co{signs}.  What constitutes the
formal difference between different kinds of contents is the \co{distance} --
the \co{experienced distance} -- which separates the \co{signs} from the
\co{distinguished} contents.

\pa In all cases, the objective aspect of \co{awareness} emerges through the
\co{recognition} of something \co{actual} as something non-\co{actual}.  The
sharper presence of the non-\co{actual} element signals the sharper modification
of \co{temporality} from \refp{th:temp}.

In a \co{recognition} of, say, the room I am in now as the room I left
yesterday, the sameness of the \co{actual} object is its coincidence
with the memory of it.  But we are not, as yet, reached the past and
future dimensions of \co{temporality}.  \co{Recognising} one rod
behind an occluder, ``filling in'', as a phenomenologist would say,
the missing part between the two, synchronously moving ends,
isn't exactly like an invasion of the past into the presence. 
Perhaps, no rod has ever been seen and there is no ground for
speaking about recognition of something past.  \co{Recognition} is not
the same as \co{re-cognition}.  \thi{Filling in} may be of any
character: it may be filling in of something known from the past but,
equally, it may be a mere, unjustified and unfounded expectation, a
wish to find something non-\co{actual} there.  The past is being
accumulated but there is yet no experienced difference between
something which receded into the \co{non-actuality} of the past and
something which awaits in the \co{non-actuality} of the future.

\pa\label{pa:selfaware} \co{Awareness}, the \co{distance} separating the
\co{sign} from its content, the \co{actual} from the non-\co{actual},
is of course the same as the \co{distance} separating the content from the
\co{actual sign}.  Being \co{aware} of \thi{\ldots} is the same as to
be \co{aware} of the \co{distance} separating this \thi{\ldots} from the
\co{actuality} of \co{awareness}.  But this is the same as being
\co{aware} of the very \co{awareness} itself, of the very fact of
being \co{aware}.  \co{Self-awareness} is an \co{aspect} of
\co{awareness}.  It is even an \equi\ \co{aspect}, for
\co{self-awareness} is nothing more than \co{awareness} of being
\co{aware}, which is again just the distance separating the \co{sign}
from its content -- it is always and only consummated in the event of
being \co{aware} of \thi{\ldots}.
Every event within the \hoa\ is the event of \co{awareness} and 
\co{self-awareness}. (We only have to keep in mind that these 
\co{aspects} do not mean any \co{reflective} consciousness or 
introspection.)

In the jargon of Fichte: Ego is equiprimordial with Non-Ego, 
positing non-Ego is also self-positing, while self-positing is only 
positing of Ego against non-Ego.
Sartre would say that consciousness is equivalent to
self-consciousness. Any consciousness, being a consciousness of 
{\ldots}, is the consciousness of {\ldots} being different 
from the consciousness itself, i.e., is self-consciousness; and 
vice versa, any self-consciousness is only consciousness of itself 
being different from some {\ldots}, i.e., is
consciousness.\ftnt{\citef{[C]onsciousness is aware of itself {\em in so far
      as it is consciousness of a transcendent object}.}{TEgo}{ I:a} \wo{[E]very
    unreflected consciousness [is] non-thetic consciousness of itself.}{[Ibid.}{
    I:b]}\noo{p.40/46} The two are \co{equipollent}: \citef{the necessary and
    sufficient condition for a 
    knowing consciousness to be knowledge {\em of} its object, is that it be
    consciousness of itself being that knowledge. This is a necessary condition,
  for if my consciousness were not consciousness of being consciousness of the
  table, it would then be consciousness of that table without consciousness of
  being so. In other words, it would be a consciousness ignorant of itself,
  an unconscious -- which is absurd. This is a sufficient condition, for my
  being conscious of being conscious of that table suffices in fact for me to be
conscious of it.}{BandN}{ Introduction:3}}
%
Awareness of this \equin\ precedes the philosophy of consciousness.  As Aristotle
says, it is \citet{by sight that one perceives that one sees}{AristAnima}{
  III:2.425b12} -- seeing, like any other \co{act} consummated within the \hoa,
is an event of \co{actual awareness} and, {\em by the same token}, of
\co{self-awareness}. Proclus: \citet{Every intellect apprehends itself. [...]
  Every intellect in its act knows that it apprehends. Intellect whose feature
  is to apprehend is not different from that which apprehends that it
  apprehends.}{Proclus}{\para\para 167-168 [Although Proclus' concept of intellect (which
  translates \gre{nous}) cannot be identified with our 
  concept of \co{awareness}, it seems that \wo{intellect in its act} can
  be. (Justification makes it quite clear \wo{[...] since it sees that it
    apprehends, and knows that it sees, it knows that it is intellect in act [...]})
  In any case, it is legitimate to claim that the
  quote refers to the same \equin\ we are considering here.]}


If we subtract the differences in the vocabulary
and concepts, all these formulations say the same: \co{awareness} and
\co{self-awareness} appear simultaneously or not at all, they are \equi\
\co{aspects} of the same \nexus\ of \co{experience} and \co{recognition}. 

\pa \co{Awareness} is not any faculty of a \co{subject} -- it precedes
\co{subjectivity} of any experience. Neither is it any quality, property which
accompanies \co{experience} -- the two are \equi\ \co{aspects}.  It is not so
that you (or an ant, or a bat) can have \co{an experience} without also being
\co{aware}: to have \co{experience} is to be \co{aware}.  It does not mean that
you have to be \co{reflectively} conscious {\em of} what this experience
consists of, what it presents, etc.  As Nagel says it, an organism has
consciousness (\co{awareness}) \citet{if and only if there is something it is
  like to {\em be} that organism -- something it is like {\em for} the
  organism.}{NagelBat}{} This famous \thi{to be like} is fine, only that there
seems to be no need to distinguish it from \co{experience} -- \co{experiencing}
things may be taken precisely as that which \thi{it is like} to have \ldots this
form of \co{experience}.  It does not require any \thi{mental states}, any
concepts, any introspective consciousness but just that: a particular way of
\co{experiencing}.  \noo{This seems to be Nagel's main point -- to emphasize the
  character of \co{experience} which may vary between different organisms.
  Consciousness, in the sense of \co{awareness}, gets naturally involved but I am
  not sure if it is not all too generous interpretation of his \thi{conscious
    mental states}, etc.  Certainly, something \thi{it is like to be} human does
  not require any more \co{attentive}, \co{reflective} consciousness, even if
  such consciousness often accompanies our \co{experience}.}

The crucial thing is the \equin\ of \co{awareness} and \co{self-awareness}.  In
the following Section, we will see more and more sharp \co{distinction} between
\co{actual} and non-\co{actual}, the \co{sign} and the content, eventually, the
\co{subject} and the \co{object}.  This will be associated with the gradual
transition of \co{awareness} and \co{self-awareness} towards \co{reflection} and
\co{self-reflection} -- and the respective \co{dissociation} of the two
\co{aspects}.

Consciousness is anything between these two extremes -- it, too, is a matter of
degree, which on this scale corresponds to the degree of \co{precision}. In the
rest of the book, we will use \wo{consciousness} in a non-technical sense but
one may always exchange it with \wo{\co{awareness}} or else \wo{\co{reflection}}
-- the results will hardly ever be incorrect, though usually different, as these
represent only the limiting cases. 

%\input{014refl}
% Floats get lost in THIS \section !!!
\section{Reflection}\label{se:reflection}
\plan{It is an act}
\planA{hence comprises \HH}
\planA{and externalises it as a concept/object}

\pa We have thus arrived at some structure of the concept of
\co{experience}, of 
\co{recognition} within the \hoa, following the stages of
\co{nothingness} and \co{chaos}.  The latter, although they do not
constitute separate \co{experiences}, form always \co{present} background accompanying
any experience.
The ultimate \co{nothingness} is the outermost horizon of \co{experience} -- not
only in the logical order, but also in the sense that it is the deepest \co{aspect} of
any \co{experience}.  \co{Experience}, and any particular experience is always
surrounded by this ultimate homogenous background.  Its first modification is
the \co{chaos} of \co{distinctions}.  The virtual \co{signification} of a
\co{distinction}, in the midst of this chaotic modification, refers to the
underlying \co{nothingness}.  And finally, in the midst of \thi{all the rest},
within the \hoa\ surrounded by \thi{something more}, there emerge
\co{recognitions}, \co{signs} which not only refer to their original background
but which carry \co{non-actuality} within themselves, confronting \co{awareness}
with contents exceeding its \hoa\ and, by this very token, constituting also
\co{self-awareness}.


\pa \co{Experience} is an inexhaustible source of novelty and surprise, the
source of ever new \co{recognitions} offered by the \co{chaos} and, eventually,
\co{nothingness}. Thus we might think that the only thing to do is to study
\co{experience}, to ask how it emerges, how it is multiplied, developed,
refined. In the extreme case, we might assume the {objectivistic attitude}
toward \co{experience} and, miming the attempts of sciences, try to
\co{re-construct} it, build its model from \co{actual} concepts.


But even if we do not go that far, a study of \co{experience} is seldom what it
pretends to be -- instead, it is a study of \co{experiences}.  Conceiving
\co{experience} as a series, a totality of \co{actual experiences}, splitting
\co{experience} into \co{dissociated} \co{experiences}, such an activity marks a
new mode of being which, emerging from and, so to speak, within or into
\co{experience}, places itself outside \co{experience}.

This is achieved through \co{reflection} which is to \co{experience} what
\co{recognition} is to \co{chaos} and \co{distinction} to \co{nothingness}:
further and sharper differentiation.  It
is a \co{re-cognition}, but of second order; it is a \co{distinction}
abstracting something which, in \co{experience}, has already been
\co{distinguished} and \co{recognised}. Now, this is \co{dissociated} from the
\co{experience}, \co{externalised} as an independent \co{object} of
\co{reflection}.

Acts of pure \co{reflective} consciousness involve a mere registration \thi{that~...}, that something is, that it is so-and-so.  This is the abstract
characterisation of \co{reflection}: the mere observation \thi{that~...} But the
underlying theme of such \thi{that~...} is precisely the \co{dissociation} of
\thi{...} -- the \thi{that~...}  points specifically to \thi{...}, focusing on
this particular \thi{...}  rather than another.  The conjunctive \wo{that ...}
expresses but this fact of isolating, {cutting} this particular \thi{...}
from the context of \co{experience}, which now becomes its surrounding.


\pa\label{pa:attentive} \co{Reflection} amounts to splitting \co{experience}
into \co{experiences}. We may \co{reflect} over the whole experience as such,
but such a \co{reflection} would require \co{distinguishing} \thi{experience},
opposing it to something else. Consequently, it either can not become an
\co{object} of \co{reflection} or else becomes such an \co{object} only ceasing
to be itself.  Primarily, \co{reflection} focuses on a particular situation,
particular context, a particular thing. When it does not and tries to capture
some greater totality, it turns whatever it is \co{reflecting} over into its
\co{actual object}. In either case, one ends with a particular (\co{object} or 
situation) \co{posited} as an independent entity -- independent because
\co{dissociated} from the surrounding of \co{experience}.
% One should always remember what kind of
% experience one is reflecting over, that is, what character the correlate of
% \co{reflection} has before it has been made into its \co{actual} \co{object}.

\co{An experience} -- a particular, limited totality of \co{distinguished} and
\co{recognised} contents -- is a correlate of a \co{reflective cut} through
\co{experience}.\ftnt{As always, we do not search for any causes, we do not ask
  which element yields which but consider them as \co{aspects} of one \nexus\ --
  here, the \nexus\ of \co{reflective experience}. Although the word
  \wo{correlate} emphasizes the element of \co{dissociation} and opposition, it
  is not meant to abstract from this fact.} It need not be an act of
deep thoughtfulness; any, most common \co{act} of focusing on this rather
than that, is \co{an experience}, a conscious experience, an \co{act} of
\co{reflection} in this sense.  \thi{Reflection} in the more common sense, an
\co{attentive reflection} is but \co{reflection} carried to its extreme.  It
brings perhaps a new quality to \co{experience} but it does not bring anything
else which is new -- it only \co{dissociates} further and more definitely, fixates
and freezes the contents offered to it in the \co{reflective
  experience}.\ftnt{Sartre's \thi{positional consciousness} is a good
  expression denoting the same as our \co{reflection}.  The \thi{positional}
  aspect is just the effect of \co{dissociating} the \co{object} from its
  background, \thi{positing} it as the only correlate for the \co{actual}
  thought.}  This ultimate possibility of \co{reflection} arises when the
\co{reflectively} isolated \thi{\ldots} becomes completely \co{dissociated},
that is, posited as thoroughly independent, self-subsisting entity, when
\co{distinguishable} is seen as \co{dissociated}, when separable becomes
separated, when, following Hume one admits \citet{that all our distinct
  perceptions are distinct existences, and that the mind never perceives any
  real connextion among distinct existences.}{Hume}{Appendix [to be inserted in
  Book I:3.14 (p.161) after the words ``any idea of power'', p.636]} This leads
to the experience of \co{objectivity} which we will encounter pretty
soon.

\pa \co{Distinctions} make \co{nothingness} into \co{chaos} and
\co{recognitions} make \co{chaos} into \co{experience}. But \co{nothingness} did
not disappear under \co{chaos}, and \co{chaos} did not disappear under
\co{experience} -- they only withdrew beyond the horizon to stay and surround
the \co{experience}. A \co{distinction} is the virtual \co{signification},
contains a reference to \co{nothingness}.  Similarly, every \co{recognition},
besides the reference to something \co{non-actual}, contains also the reference
to \co{chaos} by which it is surrounded. And \co{reflection}, having \co{posited}
 its \co{actual object}, contains always also a
reference to \co{experience} -- which surrounds it.


These references, these inherent \co{significations} are not appropriations.  On
the contrary, \co{nothingness} is inaccessible through the \co{distinctions} and
\co{chaos} is inaccessible through the \co{recognitions} -- precisely because
the latter are just what transform the former, what change them into something
else. Now \co{reflection} changes the \co{experience} into \co{an experience},
into \thi{experience diversified into separate experiences}. From this
perspective, \co{experience} remains an inaccessible horizon, surrounding the
\co{reflection} with the perpetual intention to integrate what it has
\co{dissociated} back into the continuous texture of \co{experience}. An
attentive, positional, truly thoughtful and reflective \thi{reflection over
  experience} is but a more intense, sharper modification of this primary
structure of \co{reflective experience}.

\pa
Inaccessibility deserves a short remark. It does not mean that 
\co{reflection} is entirely unaware of \co{experience} or that 
\co{experience} has no contact with \co{chaos}. To say this would be 
to abstract, to \co{dissociate}. All these are \co{aspects} of an individual
being, which  
\co{experiences} as it \co{reflects}, which is immersed in \co{chaos} 
as it \co{experiences} and which touches \co{nothingness} through 
the \co{chaos} of its depth. Inaccessibility means the impossibility 
of recovering the mode of being, the quality of the higher level, using
exclusively the categories of the  lower one. It can be attempted reconstructed but all 
such attempts are bound to dwell in and apply the categories 
characteristic for the level from which they are undertaken. The 
problem is not to forget them, to erase them, to jump to the higher 
level, but to accommodate these categories so that they do not gain 
exclusive power and thus break the continuity which underneath 
the increased dissociation of the lower levels leaves the
\co{traces} of the higher ones and connects the different levels of one
being.

\pa To be sure, \co{recognitions} effected already diversification of
\co{experience} into various sub-totalities of \co{signs} and objects.  But this
means only that \co{experience} is not a homogenous background nor a pure
\co{chaos}, it is not an indistinct but a differentiated flux of
heterogenous variety and manifold.  A particular sub-totality of this variety is
what, traditionally, one would call \wo{an experience}.  However, we have to pay
attention not only to what is \co{distinguished} but also {\em from what}.
\co{Recognitions}, emerging from \co{chaos}, constituted the horizon of
\co{experience}, the field from which various \co{experiences} can be later
\co{dissociated}.

\co{Experience} is not a \co{totality} of \co{experiences}; on the contrary, it is a
mode, a level of being which, {preceding}, \co{founds} particular
\co{experiences}.  \co{Recognitions} differentiate \co{experience} but do not
\co{posit} separate \co{experiences} -- these are distinguished but not made
independent from each other.  This happens first through \co{reflection}.  The
basic r\^{o}le or, if you prefer, the functional definition of \co{reflection}
is just that: \co{dissociation} of \co{experience} into \co{experiences},
emergence of abstract \co{signs} and of the \co{dissociated, external objects}.

Thus \co{reflection} is a {\em new} mode of being which
\co{dissociates} from the flux of \co{experience} a particular
totality, \co{an experience}.  Any further, attentive discourse about
\co{experience} or about \co{an experience} involves a prior
\co{reflective dissociation} of this aspect from the horizon of
\co{experience}.  \citet{When we speak of different experiences, we can
refer only to the various perceptions, all of which, as such, belong
to one and the same general experience.}{CrPR}{B-138 [We would
not, of course, restrict the differentiation to \thi{various
perceptions} only since, unlike Kant, we do not identify the
\thi{matter of experience} with sensations. This and other conceptual
differences notwithstanding, the quoted formulation and its underlying intuition
fit the present context.]} \co{An experience} emerges as a part of
\co{experience} only through an \co{act} of \co{dissociation} --
\co{reflective} focusing on this particular aspect of \co{experience},
\co{positing} it as the \co{actual object}. 

\noo{This new mode of being does not mean that once it arises it replaces the
previous stages, pushes them aside and takes over the whole stage.  To begin
with, it is more of a mere potentiality which only occasionally gets realised.
One \citet{can see in every child how hesitantly and slowly ego-consciousness
  evolves out of a fragmentary consciousness lasting for single moments only,
  and how these islands gradually emerge from the total darkness of mere
  instinctuality.}{JungArche}{VI:501} It is only after a few years, perhaps
about the time when a child reaches school age, that these islands begin to form
a totality which gains autonomy of a conscious \co{ego}. This aspect will be
considered in Book II. 
}

\pa Just like the earlier processes of \co{distinction} and \co{recognition} so,
too, \co{reflection} can bring forth and fixate {\em anything} from
\co{experience}. Which particular \thi{...} is \co{dissociated} into a given
\co{experience}, is the matter of this particular \co{experience} and
\co{reflection}. Just as before, so for \co{reflection} there are no universal
principles defining what is basic and what is secondary, what is first and what
last. The only general rule is that \co{reflection}, confronted with the {\em
  excess} of \co{experience}, like \co{recognition} was confronted with the
excess of \co{chaos}, proceeds gradually from indefinite and vague towards more
specific and precise. It also has the strong tendency, though it is only a
tendency, to focus on the \co{distinctions} and their configurations which can
be embraced within the \hoa, which can be naturally \co{represented}.


\subsection{Representation}\label{sub:represent}% repetition and externalisation
\pa \co{Distinction} introduces the primordial \co{signification}, underlying
and all embracing reference to the ultimate \thi{outside}, \co{nothingness}.
\co{Recognition} happens through a \co{sign} -- an \co{actual distinction} which
merges into some non-\co{actual} ones, an \co{actual} reference to something
non-\co{actual} and, by the same token to the \co{chaotic} background.
\co{Reflection} brings in a \co{representation} -- a \co{sign} but not any more
an immediate and transparent one but a \co{sign} which is given \co{as a sign}.

\label{pa:refrepet} 
\co{Representation}, \co{sign as a sign}, is the first form of \co{repetition},
the \co{repetition} of the \co{recognised} \co{experience} as a \co{dissociated}
\co{experience}, as an isolated totality which through this \co{act} of
\co{reflection} becomes \co{repeated}, i.e., acquires the character of an
independent event, no longer merged with the background of \co{experience} but
merely {\em related} to other \co{experiences}. The word
\wo{\co{re-presentation}} expresses this double perspective on the same -- as a
moment of the unity of \co{experience} versus as an entity extracted from it.
The \co{reflected} experience \co{repeats} the unreflected
\co{recognition}. And this very \co{repetition} is also \co{dissociating} the
repeated aspect from its background. Thus, 
\co{representation} is not any new thing, any copy, any miraculous internal
duplication.  It is just a next level of differentiation, it is a part of
\co{experience} which has been more sharply isolated, a \co{dissociated} part
{cut} out of the whole.\ftnt{In Hegelian or Bradleyan terms, a particular is an
  abstraction from the totality. In Bergsonian terms, a perception is a part of
  the perceived, is the very object of perception stripped of some, at the moment
  irrelevant or overlooked, aspects.}


% The \co{sign as a sign} of \co{reflection} is the primordial
% \co{repetition} which happens within the \hoa.

I am in a room and catch myself focusing, staring at one piece of furniture. I
stare at this cupboard and as I do it, it loses its earlier character of being
just one, indifferent aspect of the whole room. It gains importance of being on
its own, of being in the focus.  Sure, its surroundings, the whole room, are
still present here, but the cupboard has been pulled out, {cut out} of the
room and is experienced in a new way. It has been doubled: I \co{experience} the
fact of its being merged with the background, of being there but, on the other
hand -- and simultaneously -- I \co{re-cognise} its particular status of a
\co{dissociated} entity, which my \co{reflection} found there (in its form,
perspective, colour, solidity, what not...), but found there only through its very
\co{act} of \co{dissociation}.  The two are the same but also the latter
\co{repeats} the former, is the continuity of the former \co{represented} in the
discontinuity of a single \co{act}.\ftnt{There are several places in
  \citeauthor*{DeleuzeDiff} indicating that this might be a legitimate
  interpretation of his notion of \thi{repetition}. Fichte makes the following
  observation concerning the meaning of copula: \citef{In the statement A=B, A
    denotes that which is now posited; B that which is encountered as already
    posited. -- {\em Is} expresses the transition of Self from positing to the
    reflection over the posited.}{FichteWL}{I:1, F.11 [<<So bezeichnet im Satze
    A=B, A das, was jetzt gesetzt wird; B dasjenige, was als gesetzt, schon
    angetroffen wird. -- {\em Ist} dr\"{u}kt den Uebergang des Ich vom Setzen
    zur Reflexion \"{u}ber das gesetzte aus.>>]} \wo{Reflexion} does not refer to
  philosophical reflection but to the structure of immediate experience. If we
  replace B with another A, and then read \co{recognition} for \wo{Setzen} (positing), we
  obtain a possible description of our \co{repetition} (without, of course,
  claiming that this was all Fichte wanted to communicate in this passage).}
\co{Representation} \co{repeats} its \co{object} by merely drawing a contour
around it, a border which not merely \co{distinguishes} but also
\co{dissociates} it from the surrounding
\co{experience}.

%\pa%\label{pa:refrepet}
Thus it is not a repetition in the common sense of \thi{recurrence of the same
  for second time}. Yet, this common \thi{repetition as recurrence} is founded
  on the possibilities opened up by the 
primordial \co{reflective} \co{repetition}. The latter is not confronted by the
problem of \wo{how do I know that this is really a repetition of {\em the same}
  thing?} Indeed, starting with the ready made things, with the \co{objects}
\co{dissociated} within the \hoa\ by \co{reflective experience}, the possibility
of repetition presents a mystery. And one need not go as far as the possibly
infinite series of repetitions -- a single repetition, recurrence of {\em one
  and the same} thing only {\em twice}, is already something mysteriously ideal.
But the original \co{repetition} is merely a \co{dissociation} from
differentiated \co{experience} of its particular aspect; it is an emergence of a
\co{sign as a sign}, of the difference between the {repeating} and the
{repeated}, between the \co{sign} and the signified -- but this whole event
happens within the \hoa. Thus there is, as yet, no way to talk about \thi{second
  time}, there is no question about how one knows that the two are the same --
it is the same \co{experienced} simultaneously from two different angles, as if
in two different contexts.  Representation in the more common sense of the word
is but a sharpened version of this initial \co{representation}. It is a more
explicit repetition -- it presupposes something {\em of which} it is a
representation as already given. To be represented, this something must be
already more or less definitely and \co{precisely dissociated} from other
objects. Our \co{representation} is the event of this primal \co{dissociation}.
Thus \co{dissociated} units \co{found} then also the possibility (in
fact, the need) of representation in the more common sense, of a one
\co{dissociated} thing or image by another, in short, of abstract \co{signs}.
(We will return to the abstract \co{signs} in a moment, \refp{pa:signTrace}, 
and to the question of \thi{repetition as
  recurrence} when discussing identity in II:\ref{sub:Identity}.)


\noo{\pa\label{pa:Rexternal}
Reflection \co{externalises}, that is, it posits its object as something distinct. 
It has the \co{intentional structure}, that is, it not only grasps an object which
is distinct from it but, at the same instant, it {\em knows} this object to be distinct;
it grasps something it itself is not. Reflection, like consciousness, is
pure \co{actuality}, while objects, \co{recognised} and then re-cognised,
contain always \co{non-actual} element. The phrase ``reflection posits its 
object'' means precisely that this object not merely appears but that it appears 
{\em as object}.
}

\pa\label{pa:doubleSep}\label{pa:Rexternal}
The \co{reflective repetition} can also be called \wo{double \co{dissociation}}.
The \co{act} of \co{reflection} \co{dissociates} this cupboard from the rest of
\co{experience} as an independent \co{object}.  It singles out a unit which can be
contained within the \hoa, and which is contained there with the exclusion of
everything else.  \citet{The mark of the mind is that there do not arise more
  acts of knowledge than one at a time.}{NayaSutra}{I:16} The
object acquires thus a special status as compared to all the rest of
\co{experience} which is, for the moment, ignored by \co{reflection}.  (This
implicit reference to \thi{all the rest} is the earlier mentioned
\co{signification} common to all \co{signs}.)

On the other hand, the object of \co{reflection} has been already
\co{experienced}, it is something which has already been \co{recognised}.  In
fact, the more intense my attempts to grasp and embrace the cupboard by my
attentive look, the more it loses its real presence, its reality withdraws and
gives place to the domination of my \co{reflective} attention, becomes a mere
\co{representation}.  The original \co{representation} is just a
\co{dissociation} of \co{an experience} from \co{experience}, a sharp
\co{dissociation} of an \co{object} from the background and, by the same token,
from itself.  An \co{object} appears by being so \co{dissociated} -- an \co{act}
we call \wo{\co{positing}}. \co{Dissociation} is definite, it actually
\thi{tears the object out of the context} and, desiccating it, lends it independent
identity. As a consequence, the object no longer coincides with its \co{sign}, it
has been \co{externalised} and the \co{sign} appears \co{as a sign} -- it does
not coincide with the signified whose intended \co{experience} has been
\co{dissociated} and turned into ... the \co{object} of \co{reflection}.  In so
far as it is a \co{sign}, it indicates the background; in so far as it does it
\co{as a sign}, it makes also clear the \co{distance} separating it from this
background. This \co{distance} is now also the distance separating the
\co{reflective sign} from its \co{object},
%is also the distance separating {\em this} \co{object} from other \co{objects}.
which is only another way of saying that \co{sign} appears \co{as a sign}.

We will call this double \co{dissociation} \wo{\co{externalisation}}.
\co{Reflection} \co{externalises}. This is the characteristic feature
distinguishing \co{reflection} from the earlier processes of \co{distinction} and
\co{recognition}.\ftnt{\co{Recognitions}, too, involved the element of
  \co{non-actuality} and were made against other \co{recognitions} as well as
  against the background.  But this double \co{dissociation} -- of something
  from the background and, at the same time, not only from another something but
  also from itself -- becomes fully constituted only at this \co{reflective}
  stage.}


\subsubnonr{Signs and meaning}%\label{sub:signsMean}

\pa\label{pas:signsMean}
\co{Distinction} is indiscriminately the event of distinguishing {\em and} the
distinguished content. Similarly
the \co{signs} of \co{recognition} coincide with the signified -- if not in
fact (that is, not in so far as the \co{actuality} of the \co{sign} might have
been only a limited aspect of the possibly \co{non-actual distinctions}), so in
any case in \co{experience}, in so far as \co{sign} does not appear as distinct
from the signified. \co{Reflective signs} do not do it any more; the \co{sign as
  a sign} is constituted exactly by the \co{dissociation} of the \co{sign} from the
signified.

The background has been now diversified into a variety of \co{representations}
-- \co{representations} which parade as if they were the objects, the \co{signs
  as signs} which, precisely by the fact of being mere \co{signs}, make it
possible to embrace within the \hoa\ more objects (that is, \co{signs}) than if
we were to keep there the objects themselves.  \co{Sign as a sign} means: to
\co{represent} a non-\co{actual} (possibly also \co{non-actual}, or else only
\co{external}) object by means of a 
\co{sign} which (i) is \co{dissociated} from this object and which (ii) fits
completely within the \hoa.

\label{pa:distance}
The point (ii) is particularly relevant here and might be even taken as
conditioning (i). It applies namely even if the object itself could not be
comprised within the \hoa. \wo{This} is perhaps the paramount case of a
\co{sign} which refers to something so \co{immediate} that it often escapes
all more specific means of linguistic identification. On the
other hand, \co{reflection} over pleasures or pains is bound to use mere words
\wo{pleasure} or \wo{pain} with only approximate and never finally determined
meaning. As the object of discourse becomes more and more remote (life, world,
love, God), the \co{distance} separating it from the \co{actual signs}
becomes longer and more \co{clear}.  The \co{distance} in general separates
  \co{actuality} from \co{non-actuality}, \refp{pa:distanceA}.  In the current
  case of \co{reflective 
    signs}, it amounts to their inherent inadequacy, the impossibility to
  capture the signified. (This inadequacy is, in fact, if not the same as their
  \co{meaning}, so in any case a witness to their involvement into 
  more than merely \co{actual} relations.) With respect to \co{actual objects},
  it may be their mere 
  \co{externality}; it increases with respect to the \co{signification}
  underlying all \co{distinctions} and becomes virtually infinite with respect
  to the \co{invisible origin}.\ftnt{A possible experience of the \co{distance}
  coincides with the experience of its increase or diminution, for instance,
  when somebody unexpectedly formulates an association revealing a deep insight,
  makes a very clear expression of a thought which seems to be as final and
  adequate in its revealing content as it is open for future and more specific
  interpretations. A moment of insight, bringing a sense of communion,
  diminishes the \co{distance}, and when the insight is provoked by (or in any
  other way genuinely \co{shared} with) another, it diminishes also the distance
  between the persons.  }
% (The \wo{re{\em presentation}} indicates also that the object is presented
% \thi{as if} it were \co{actual}.)

\pa\label{pa:signTrace} The appearance of \co{distance} amounts to a new
discovery: the \co{sign} need not be an \co{aspect} of the signified, the two
are \co{dissociated} and so may be {\em put into a relation} to each other.
Thus emerges the possibility of abstract (or conventional) \co{signs}.  Abstract
\co{signs} are the ones which have been completely \co{dissociated} from their
meaning.  Artificial and conventional signs provide the most typical examples.
\thi{Smoke}, as a sign of fire, is still only a \co{sign}, in so far as it means
only that: fire. When used for the purpose of signaling it becomes an abstract
\co{sign}.  Thus appears also the \co{trace} which the prior unity leaves among
the \co{actually dissociated aspects}, the relation between the \co{sign} and
the signified.\ftnt{We will encounter \co{traces} many times, so here let us
  only observe a close relationship to the Derridean traces: an \co{actual sign}
  may carry a \co{trace} of its \co{origin} and, to this extent, also of its
  \co{actual} opposite(s) \co{dissociated} from the same \nexus. Typically,
  \co{traces} emerge as various relations between the \co{dissociated} entities.
  They come always \thi{from above} and are not reducible to the given context
  where they terminate as \co{actual signs}. The following is perhaps a bit
  mystified but nevertheless adequate expression of meaning also of our
  \co{traces}: \citef{As rigorously as possible we must permit to
    appear/disappear the trace of what exceeds the truth of Being. The trace (of
    that) which can never be presented, the trace which itself can never be
    presented: that is, appear and manifest itself, as such, in its phenomenon.
    The trace beyond that which profoundly links fundamental ontology and
    phenomenology.}{Differance}{ p.23}\label{ftnt:traceA}}
%
We call this relation \wo{\co{meaning}}.  It is
the bond which still keeps the \co{actual signs as signs} and the possibly
\co{non-actual} or \co{externalised} \co{distinctions} together. Or else, it is
the means allowing the use of abstract \co{signs} for actually drawing some
\co{distinctions}. But the bond notwithstanding, the
effected \co{dissociation} and emerging abstract \co{signs} allow now for much
free play with all three elements.

%\co{Representations}, \co{re-cognitions} are memories of \co{objects}.

%\say Somehow: sign-meaning-signified/denoted ???

%\pa
\label{pa:signsMean}
A \co{sign} is {\em the} means of comprising (possibly \co{non-actual})
\co{distinctions} within the \hoa. \co{Sign as a sign} is, in addition, given as
distinct from its \co{meaning} and, appearing within the \hoa, most typically
makes present something which \co{transcends} it. Most words provide the
examples. \wo{Red} or \wo{perseverance} do not bring in all possible
aspects of \thi{redness} or \thi{perseverance}. But they do draw enough
\co{distinctions} in the \co{actual} context to make \thi{redness} or
\thi{perseverance} ... well, \co{actually} present or relevant, to bring them
into the \co{actual} focus. We say, the \co{signs} \co{actualise} the respective
aspects, they draw the respective \co{distinctions}.
%A sign is an actual form of some (possibly non-actual) distinction(s).&  = \\
A \co{sign} is a way of \co{actually} drawing some (possibly \co{non-actual})
\co{distinctions}, is a form under which such \co{distinctions} may enter the
\hoa. The \co{distinctions} drawn thus by a \co{sign} constitute its
\co{meaning}.\ftnt{The \co{actually} drawn \co{distinctions} comprise quite a
  lot. \wo{It is sunny} means certainly that it is sunny. However, pronounced on
  a sunny day to a stranger, it might mean insecurity as to the stranger's
  intentions, an attempt to start a conversation. It could mean \wo{Are you
    interested...? In talking...} These, too, are \co{distinctions} which the statement
  may make in the \co{actual} situation.  As hermeneutics never ceased to repeat, the
  \co{meaning} of the initial \wo{It is sunny} need not be given uniquely and to
  the extend it is, it may be as vague as a mere wish to say something. However,
  a feature very specific to language is that words, written or pronounced, are
  always already inscribed in the context of inter-human communication and carry
  their residual meanings independently from any \co{actual} situation. Thus,
whenever encountered, they can hardly fail to produce some \co{distinctions}
beyond the trivial \co{distinction} of their mere presence (which is a
\co{distinction} made by everything that is).}

\co{Meaning}, as an expression of the \co{distance}, is also the \co{trace} of
\co{transcendence} (of the signified non-\co{actual} or even \co{non-actual}
with respect to the 
\co{actuality} of the \co{sign} and abstraction of \co{sign as a sign}). This is
illustrated also by the fact that a word whose meaning could be fully analyzed
(if there were such a thing) would be redundant. It might function merely as a
convenient abstraction, a normative definition, increasing efficiency of the
system.  The difference between {words} and \thi{mere words} is exactly this:
the latter fail to make anything present, while the former do reveal.  The
\thi{mere words} turn out to be \co{signs} which try to ignore the
\co{distance}, and trying that end up without any. But where there is no
\co{distance} there is nothing to reveal either.  The power of language seems to
lie also in our ability to say -- and communicate -- something very
\co{distant}, something deeply meaningful which we however do not quite grasp,
which we can not make \co{precise}.  What makes \co{signs} and words meaningful
is the fact that they never embrace the whole (reality) which they intend --
they merely hint at it, are mere \co{signs}, pointers. If you do not understand
what is being said, perhaps, you do not know what the talk is about. And if you
know, you need not the absolute univocity and precision of the expression -- a
mere indication, a vague sign will suffice.

% \begin{tabular}{ll}
% \parbox{14cm}{A sign is a means of comprising a (possibly non-actual) distinction(s)
%   within the \hoa.}  &      =  \\
% A sign is an actual form of some (possibly non-actual) distinction(s).&  = \\
% A sign is a way of actually drawing some (possibly non-actual) distinction(s) &
%   $\Rightarrow$ \\
% The distinctions drawn by a sign is what we call its \wo{meaning}
% \end{tabular}

\noo{ %
If needed, we may use the notation $\mean s =d$ to indicate that \co{meaning} of s is
$d$, e.g., $\mean{\wo{snow is white}}=$ snow is white or, to avoid ambiguities,
putting the actual meaning into single quotes: $\mean{\wo{snow is white}}=$
`snow is white'.\ftnt{In practice, $m$ is not any function. If we wanted it to
  be, it would take some more arguments for disambiguating the abstract signs
  (e.g., the context of their utterance, etc.). But we do not even want it to be
  any function. Firstly, it may happen to act as a function for vague and
  imprecise \co{signs} by assigning to them equally vague and imprecise
  \co{distinctions}. Moreover, there is in general much fewer \co{signs} than
  there are possible \co{distinctions} -- in some contexts a \co{sign} may be
  ambiguous without calling for any disambiguation.
  
  On the other hand, \co{distinction} is a truly capacious notion. \thi{It is
    snowing} may be taken as (signifying) a \co{distinction}, and so may be
  \thi{It is sunny}. But then, \thi{It is snowing or it is sunny} may be viewed
  as an under-determined \co{sign} or else as a fully determinate one which only
    signifies the under-determined \co{distinction}. (The
  \co{distinctions} are closed under most imaginable operators, in particular,
  union (\wo{and}), alternative (\wo{or}), refinement.) The difference between
    these two will 
  depend on the context and is for us as inessential as most issues concerning
  the abstract \co{signs} -- we are interested in the \co{distinctions} and not
  means of drawing, let alone expressing, them.} The argument of $\mean\_$ is
    primarily an abstract 
\co{sign} -- when it is a natural \co{sign}, the result is usually identity. The
\co{actual} (limit of) \co{distinctions}, e.g., the tree I see in its totality
in front of me is, according to our terminology, a \co{sign}.  Unless we want to
perform some rather intricate interpretations (that it means the presence of
air, earth, some minerals necessary for growth, etc.), this \co{sign} means
nothing more than itself. (Thus $\mean{\wo{this tree}}=$ \thi{this tree}, but
then $\mean{\thi{this tree}}=$ \thi{this tree}.) In another context, of course,
this tree may happen to be a \co{complex} signified by a series of complicated
\co{signs}. Which \co{signs} coincide with their \co{meaning} depends on the level to
which the \co{actual distinctions} are drawn. I may focus on my impression of
this tree, on the sensations it causes, and then these become the \co{immediate
  signs}. When analysed \thi{in themselves}, however, they may be \co{dissociated}
from their possible meanings and established as \thi{pure givens}, \co{signs}
without any external meaning.\ftnt{The external meaning of a \co{sign} need not
  have anything to do with the \co{externality} of an \co{actual object}. It
  indicates only that the \co{distinctions} signified by a \co{sign}
  \co{transcend} its \co{immediacy}.}  A word as a mere string of letters, a
meaningless symbol, is an abstraction obtained by \co{dissociating} the
\co{aspects} of a \co{sign} and treating it as a mere \co{sign}, or rather, as
    no longer a \co{sign}.
%
} %end \noo, but...    

\noo{%2\subsubnonr{Distance}
\pa
Word is the paradigmatic example of \co{sign as a sign}, a \co{sign} which is
immediately given as distinct from what it signifies.
Words signify something which they are not,
something which transcends them and, most often, which transcends \co{actuality}.
 As \co{signs}, words appear within the \hoa\ making something 
which \co{transcends} it present.

The intricacies of how signs give rise to the identity of objects, and what role
words may possibly play in such a process will not occupy us.\ftnt{Entities
  (not merely objects, but all that can be \co{recognised}) do not arise in a
  uniform way. Some arise in the (original) process of distinguishing, before
  any \thi{building blocks} from which one could construct anything are
  available. Others, in particular those of \co{this world}, are
  \co{re-constructed} from the sediments of earlier \co{distinctions}. Some
  remarks on these issues appear in Book II, where also identity is addressed in
  \ref{sub:Identity}.}

\pa In the former case (of a \co{sign} pointing to something \co{actual}) one
can speak of and hope for an adequacy: a system of \co{signs} may quite
adequately capture a system of \co{actual objects} and \co{distinctions}. In the
later case we have only inadequate \co{signs}, \co{signs} which never capture
the signified contents. This inadequacy of a \co{sign} we call
\wo{\co{distance}}. In case of \co{immediate objects}, the \co{distance} finds
an expression as the mere \co{externality}, the fact that \co{sign as a sign},
no matter how adequate, does not coincide with the signified, remains in the
perhaps enchanted circle of the \co{reflective act}, but is unable to touch the
enchanting \co{object}. The \co{actual sign} is separated by a \co{distance}, 
whether from the non-\co{actuality} of its content or from the
\co{non-actuality} of its background. This 
\co{distance} becomes even larger with respect to the \co{signification} of the
background underlying every \co{distinction}. It becomes infinite with respect
to the \co{invisibility} of the \co{origin}.  We can experience the
\co{distance}, for instance, when somebody unexpectedly formulates an
association revealing a deep insight, makes a very clear and adequate expression
of a thought which seems to be as final and definite in its deep and revealing
content as it is open for future and more specific interpretations. A moment of
insight, bringing a sense of communion, diminishes the \co{distance}, and when
the insight is provoked by (or in any other way genuinely \co{shared} with)
another, it diminishes also the distance between the persons.

In short, \co{distance} is a \co{reflection} of the length of the convolution
which started with the \co{indistinct} and ended with the most \co{precise} and
\co{immediate distinctions}.  With respect to every \co{actuality}, and
\co{actual sign}, \co{distance} testifies to the vertical aspect of
\co{transcendence} which, so to speak, does not surround it but stays \co{above}
it. 
}% end \noo{2\subsub{Distance}

\pa \co{Meaning} arises in the context of \co{sign}'s application, in some
\co{actual} situations.  Unused \co{sign} is almost a \la{contradictio in
  adiecto}, unless one wants to refer to the potentiality of being used as a
\co{sign} which, however, can be ascribed to every thing.  And to be used as a
\co{sign} means to be applied in an \co{actual} situation to make some
\co{distinctions}, to make a difference. \citet{The arrow points only in the
  application that a living being makes of it.}{WittPI}{I:454} The context of
use
% (whether an \co{actual} situation or else a general context like, for instance,
% a book, it always ivolves the/an intended recipient)
usually disambiguates the abstract signs making their meaning clear --
\wo{Danger!!!}  means something different from \wo{Danger?}, and both can mean
quite different things depending on the context of usage. A \wo{sweet danger} is
so different from a \wo{terrible danger}, that we would never attempt any
definite assignment of fixed meanings.\ftnt{This can remind of Wittgenstein's
  \thi{meaning as use} and indeed\noo{, although we do not share his linguistic
    empiricism, we can appreciate} his observations on language are highly
  relevant to us. (i)~\citef{Every sign {\em by itself} seems dead. {\em What}
    gives it life? In use it is alive.}{WittPI}{I:432} Together with the general
  idea of language games as pertaining to a specific form of life, we could
  accept that \citefib{the meaning of a piece is its role in the
    game}{WittPI}{I:563}, reworded as: the meaning of a word is its place in a
  life (form).  Such a place might be identified with our \co{distinctions} made
  when using the word, or better, with the {\em potential} \co{distinctions}
  which could be made using the word.  The ambiguity of \wo{use} as such a {\em
    potential} for use versus as a (posited) totality of all \co{actual} cases
  of use creates some tension.  (ii)~Yet, our \co{signs} arise as the result of
  the \co{distance} between the \co{actuality} and \co{non-actuality} and only
  lowest, \co{reflective signs} become \co{dissociated} abstracts endowed with
  lasting meaning (potential for use) primarily by convention. This convention,
  however, is \co{founded} on the prior \co{distinctions}. \citefib{The
    agreement, the harmony, of thought and reality consists in this: if I say
    falsely that something is {\em red}, even the {\em red} is what it
    isn't.}{WittPI}{I:429} For {\em red} is distinguished anyway and, for
  instance sensations are also \co{signs}, although not yet conventional,
  \co{dissociated signs}.\noo{There is no problem with having private \co{signs}
    not supported by public (linguistic or other) conventions and behaviors, for
    one is not a solipsistic subject...} Words and language constitute only a
  very specific and limited subset of all \co{signs}, a subset of conventional
  \co{signs}.  Although their inter-subjective and conventional structure could
  hardly sustain without a linguistic community, they could hardly arise and
  sustain without individuals capable of making \co{actual distinctions}
  independently from the use of language, that is, capable of recognising
  \co{meaning} through \co{signs}.\noo{The \thi{argument} against \thi{private
      language} is an argument against solipsism: a solipsistic subject could
    not possibly ensure the consistency in application of conventional signs
    where, by the unpronounced definition, consistent application amounts to a
    conformance to inter-subjective conventions.}\noo{The involvement of
    \co{signs} into \co{non-actuality} removes also (as we will see later on)
    the element of nominalism which can be discerned in Wittgenstein, primarily,
    when \wo{use} is taken as merely all possible situations of application, of
    which the \thi{family resemblance} is only a consequence.} \citef{Human
    conventions are useless if they are not connected with the motives that
    impel people to observe them.}{WeilWaiting}{The Forms of Implicite Love of
    God:The love of religious practices;p.121} (iii) There is thus a difference
  to the extent that Wittgenstein's \thi{use}, embedded in the \thi{rules of the
    language game}, seems opposed to any other form of sedimented \thi{meaning}.
  (\citef{There cannot be a question whether these or other rules are correct
    ones for the use of <<not>>, (I mean, whether they accord with its meaning.)
    For without these rules the word has as yet no meaning; and if we change the
    rules, it now has another meaning.}{WittPI}{548(b)} One can hardly disagree,
  if only we read this as saying that the meaning is not reduced to the
  use/rules but only captured, reflected by them, so that we could say:
  \citefib{Let the use of words teach you their
    meaning.}{WittPI}{II:xi\kilde{p.187}}, but not that it {\em is} their
  meaning. Indeed, changing the rules of using \wo{not} would change its meaning
  -- but not merely because it would be combined differently with other words,
  but because it would thus lead to other \co{distinctions} whenever
  used.\noo{Linguisticism would rather like to eliminate this latter case. It
    says: We are human community and communicate because we have language, while
    it is the opposite, we have language only because we have community and are
    humans able to communicate. Communication without language would be
    different, but is perfectly possible -- like with babies.}  We are
  interested exclusively in {\em what} is thus captured, while Wittgenstein only
  in {\em how} this capturing happens in language.) Such a disposal with meaning
  which could be intended beyond or before \co{actual} use, is typical for
  thinking which, encountering the impossibility of any objective fixation of
  well-defined \thi{objects} or \thi{states of affairs} (which, one used to
  assume, provide the meaning), replaces objectivity with its substitute,
  socio-cultural relations. As far as the {\em constancy} of meaning of
  conventional signs and {\em inter-subjective verifiability} of this constancy
  is concerned, the association with the rules of use is hardly disputable. But
  as an explanation, or even a mere suggestion of the {\em meaning actually
    carried} by the signs, it must refer to deeper, not only extra-linguistic
  but also extra-social aspects (which indeed happens in \btit{Philosophical
    Investigations}, for instance, \thi{life form}, \thi{image}.)
%   But if meaning is use then also use is impossible without meaning
%   (previous footnote), and the intended relation between these two elements
%   remains rather unclear in Wittgenstein.
  Our \co{meanings} are the possible and effected \co{distinctions} --
  \co{distinctions} of a communication process which is neither \co{founded} in
nor exhausted by the linguistic interaction. Only {\em sedimentation} of
meanings and their constancy in the abstract signs of language are aspects
relative to a community (which we do not really address).  We can nevertheless
observe the close correlation (even \equin) of \thi{use} and \thi{meaning}, if
we view them as the result of \co{dissociating} the two \co{aspects} of
\co{distinction}: the fact or event of distinguishing (corresponding to the
pragmatics of \thi{use}, the {illocutionary force} of Austin's, or
speaker-meaning of Grice) and the effected distinction (corresponding to actual
meaning, {locutionary content} or sentence-meaning). Recognising this close
kinship, the origin in the common \nexus, we are far from reducing any
\co{aspect} to the other.}

\noo{{private language} W. combats solipsism. The crusade against private
  language (of sensations, but whatever) is motivated by the observation, which
  we too share, that a subject closed within its representations (sensations, or
  whatever), will never manage to get out -- unless it from the start is attuned
  to the reality which is not merely his, he will never get attuned to it by
  means of internal signs and their likewise internal meanings.
  
  The argument is based, however, on the givenness of conventions which a
  solipsistic subject could not possibly establish and maintain, and even if it
  managed that, it could not simply have any reliable criteria ensuring him
  about that. The crux of the matter is the reliable criteria. Apart from the
  fact that no ultimately certain criteria exist, with W., reliability is almost
  by definition associated with public means of verification, eventually,
  conventions maintained by the community and dictating its members rules of
  behaviour/using-language.  Thus, the argument amounts to the statement: a
  solipsistic subject (could not speak to itself because it) would have no
  access to lasting public conventions.  It is as hard to disagree as see what
  is the use for such a statement. Well argued irreducibility of language to
  supposed mental processes or internal images, does not in any way exclude the
  genuine anchoring of language in a deeper layer of signs which are not as
  precise and actual. \citet{What happens when we make an effort -- say in
    writing a letter -- to find the right expression for our thoughts? [...] Now
    if it were asked: <<Do you have thought before finding the expression?>>
    what would one have to reply? And what, to the question: <<What did the
    thought consist in, as it existed before its expression?>>}{WittPI}{I:335}
  The ghost of early W. comes creeping back, the ghost of empirical dissociation
  and inability, or in any case reluctance, to admit into the considerations
  anything beyond the plainly visible items. We can only refer back to the
  miserable artist who must need have the ready-made and detailed plan in his
  head before commencing the work. As to the shape of the thought before its
  expression, we commented enough on the \co{actualisation} of \co{virtual}
  \nexuss... AHEAD: \refp{pa:contActVirt} \tsep{private language}}

Besides the context of application, there is also another aspect of the
indeterminacy of sign's \co{meaning}.  There is no \co{precise} border
separating definitely the \co{meaning}(s) of one word from possible
\co{meanings} of many other words, and its possibilities of expressing various
meanings depend just on where such borders are drawn. {The \co{meanings} of
  \co{signs as signs} arise as \co{traces} of the earlier \nexuss\ of \co{signs}
  and of \co{signification}, and this amounts to their inherent interrelations:
  it is always a {\em system} of \co{signs} which jointly circumscribes the
  \co{distinctions} effected by any single \co{sign}. \wo{Tree} means {tree}
  also because \wo{bush} means {bush} and \wo{wood} means {wood} -- in the
  absence of these latter words the former would probably mean something
  slightly different.}  Words are only \co{signs}, that is, tools for
actualising \co{distinctions}. No \co{distinction} comes alone, and neither does
any \co{sign}. There are no rigid \co{distinctions} and things are but their
limits. Consequently, not only abstract \co{signs} can be associated with
different meanings by various conventions, but even \co{meanings} themselves can
move their boundaries. Neither \co{signs} nor abstract \co{signs} have any
unique and final \co{meanings}.  The stability of \co{signs} reflects mostly
only the stability of the corresponding \co{distinctions}.  But \co{dissociated
  signs} acquire abstract stability, independent from their \co{meaning} and
thus, the constancy and consistency of words can even strengthen the stability
of the signified \co{distinctions}.  \noo{But they may have stability
  \co{reflecting}, and sometimes even strengthening, the stability.}

\pa
The turn towards the most specific context of \co{actual} usage has an opposite
effect tending 
to dissolve any \co{non-actual} sense of \co{signs} in general, and of words,
in particular. 
Nevertheless, the specificity of words is that, having been \co{dissociated as
  signs}, they are {\em always} signs, they always carry if not a specific
\co{meaning}, so at least its promise. A word is not a mere string of letters, 
a completely \co{dissociated}, that is, meaningless entity. But 
it is precisely the \co{dissociation} of the
event of distinguishing from its content, of \thi{use} from \thi{meaning}, which
makes the one \co{dissociated} pole always carry the promise of the other, which
makes every word and sentence pregnant with meaning, and every \co{actual}
meaning dependent on the used words. 
% (Any attempt at reducing one to the other is a reflection of a deeper
% reductionism, nowadays, usually of the \co{non-actual} to \co{actuality}.)

Encountering a single word written on a stone in the middle of the woods, I know
that it is a sign, perhaps only a joke, but still a \co{sign}, a message. This
is the dependency characteristic for abstract \co{signs}: I
understand in general what the word means, it carries some meaning prior to the
\co{actual} encounter and independently of the context of its use. 
Only this makes it possible for me to understand something from the
written word at all and, possibly,
even some more specific intentions of the author.\ftnt{Of course, the two need
  not agree, and one can often encounter words one knows used in a (at least
  slightly) new meaning.  But that encountered words are messages {\em from other
    humans} is as obvious as it is fundamental; even if no recipient were ever
  intended, an author has been there and this is a part of the \co{meaning} of
  every text.}  If the written word is \wo{Danger}, I will almost for sure look
suspiciously around for, on average and for the most \wo{danger} means danger.\noo{Just like \thi{is true} can be used, according to Quine or rather Tarski, to
disquote propositional sentences, so \thi{means} can be used to disquote almost
everything.} The world is different from what it would be if the word were
\wo{Quiet}, or \wo{Abzdangh}. The lack of any more specific information about
the context and the author makes, perhaps, the message imprecise and unclear,
but it remains something understandable (and to misunderstand is, also, to
understand).
The ultimate specificity of the most \co{immediate} contexts, of the \co{actual}
situations of use may be, indeed, needed to acquire the understanding of the
language and the meanings of words. But once the words, the abstract \co{signs}
acquired some meaning, they will carry it in relative independence from the
context of use. \citet{May we not, for example, be affected with the promise
  of a good thing, though we have not an idea of what it is? Or is not the being
  threatened with danger sufficient to excite a dread, though we think not of
  any particular evil likely to befall us, nor yet frame to ourselves an idea of
  danger in abstract?}{BerkeleyPrinc}{Introduction \para 20. Here lies a
  difference from most approaches addressing the issue of the genesis of
  linguistic meaning.  Fluctuations of use may provide basis for the subtleties
  and variations of the meaning, but this process of constitution does not
  change the fact that some residual, even if undefinable meanings of most words
  are being constituted -- most people do have an idea of what \wo{Danger}
  written on the stone could mean, irrespectively of the fact that it might have
  been meant in different ways.  For us, the issue of language is not so much
  that of how it possibly might have arisen in the human history and how it
  possibly may function in the society but, primarily, how it is encountered by
  an individual. In this respect, it certainly has an element of
  \thi{givenness}, of some meanings which are encountered and not constituted.}

{\co{Meaning}, in an actual use of an abstract sign, may involve all aspects of
  the actual situation. If I suspect that you want to cheat on me, the meaning of
  your 
  \wo{Danger!} may include my theories about the possible ways of you cheating on
  me by saying just that. Thus, (almost) every situation of using a \co{sign}
  results in some meaning which, being dependent on the context, is unique. Such
  an extreme nominalism forgets, however, that the possibility of using a sign in a given
  situation is conditioned by its meaning prior to this situation. \citet{Here
  one might speak of a \thi{primary} and \thi{secondary} sense of a word. It is
  only if the word has the primary sense for you that you use it in the
  secondary one.}{WittPI}{II:xi\kilde{184}} 
  One can scream \wo{Danger!} as a joke or to cheat others, i.e.,
use the expression for purposes which, for the most, do not go well with its
meaning. But one could make such a joke {\em only because} \wo{danger}
means something prior to its actual usage.
Thus, the meaning of a \co{sign} has a twofold aspect: on the one hand,
every \co{actual} use effecting some particular \co{distinctions} in a given
context and, on the other hand, the potential for making various
\co{distinctions} in various contexts, a floating and eventually undefinable
kernel (in empiricist's terms: a family resemblance) with which some 
philosophers would like to endow each \co{sign} in its \co{dissociation} from
the rest of the world or the language. The ability -- and purposefulness -- of
such an endowment depend only on the degree of the attempted \co{dissociation}
and may vary from quite useful ones (like in the dictionaries), to hardly plausible
postulates of some metaphysical relation between an abstract sign and its meaning.
}

\pa
There are much more \co{signs} than there are words, and much more
\co{distinctions} than \co{signs}. We are sceptical to all forms of reductionism
and we are not interested in \co{signs} as such, let alone abstract
\co{signs}. We will therefore stick to 
the just mentioned disquotational schema of \co{meaning} which simply says that
\co{meaning} of the linguistic \co{signs} is, typically, not determined and not
definable by purely linguistic means.  The \co{meaning} of a word, the
\co{distinctions} it can actualise, transcend usually possibilities of the
language simply because they are of non-linguistic kind. To know the meaning of
\wo{blue}, no amount of linguistic or other explanations will ever suffice. One
just has to know what blue is. That its use will be related to and mutually
dependent on the use of \wo{green}, \wo{red}, etc. is only a reflection of the
fact that blue is \co{distinguished} relatively to green, red and other colors.
\wo{Horse -- what it is like, everybody can see.} said one version of Larousse
dictionary. Uninformative as this may be, it is perfectly sufficient with
respect to most trivialities.  One might think that the story with trivialities
like \wo{blue} and \wo{horse}, for which we have obvious, \co{immediately} given
\co{distinctions} of other than linguistic kind which this words signify, does
not generalise. But why should the story with any other words, like
\wo{perseverance}, \wo{hate}, \wo{eternity},... be any different?  Because one
assumes that the only reality is \la{hic et nunc}, is the pure \co{actuality}
and everything which extends beyond its horizon is something mental, uncertain,
suspicious. We have already started to oppose this assumption and will continue
doing so. There is a difference between the way in which \wo{blue} means blue
and \wo{hate} means hate. But this difference is simply the difference between
blue and hate. The shortest \co{distance} separating hate from an \co{actual}
pronouncement of \wo{hate} is incommensurably longer than the longest
\co{distance} possibly separating an instance of blue from the \co{actually}
spoken word \wo{blue}. We will have more to say about this difference,
especially, in Book II, which could hardly ever be characterised by lengthy
comparisons of possible and impossible contexts of usage of the words \wo{blue}
and \wo{hate}.  

% Such meanings -- one might wish to say: original or primordial meanings -- are
% not something added to the ready-made \co{signs} but, on the contrary, are
% constitutive for the \co{signs} being what they are.

\pa 
\label{tripleDissoc}
\noo{We should probably be a bit careful here because we put several things
  under one hat of the \co{reflective dissociation} of \co{sign as a sign} and
  signified. \co{Signs}, as described in Section \ref{pa:signB}, do not
  literally coincide with what they denote -- the former are \co{actual}, the
  latter non-\co{actual} or \co{non-actual}. But they coincide in that the
  \co{dissociation} does 
  not enter the \hoa, the signified is given \thi{in person}, although only
  through the \co{sign}. (The object completion and perception of subjective
  contours, \refp{sub:psychexb}, provide good examples.) The possibility of
  falsehood lies already here (like completing the common motion of two parts of
  a rod behind an occluder into one rod which, in the subsequent stage, turns
  out to have been wrong), and it is this possibility which signals the
  \co{reflective dissociation}.  The \co{dissociation} of \co{sign} and
  signified, the emergence of abstract \co{signs}, opens the possibility of
  \co{actualising} the meanings which do not belong to the \co{actual}
  background, the possibility of wrongly using the \co{signs}. (There are of
  course other forms of wrong usage of signs, as when one uses a sign with some
  private meaning not shared by the community to which it is addressed, but this
  does not concern us here.)  }
%
\co{Reflection} \co{externalising} its contents gives rise to 
\co{signs as signs}, to the \co{distance} separating \co{actuality} of the
\co{sign} from the drawn \co{distinctions}.  The \co{dissociation} of the
\nexuss\ of \co{sign} and the more primordial \co{signification} results in
at least three elements: a \co{sign} (which has now become abstract), its
\co{meaning} -- the \co{distinctions} it actualises (or, in general vagueness,
which it possibly can actualise), and the \co{actual} situation, the background
of the addressed \co{distinctions}.\noo{Of course, much more detailed analyses
  are possible. The background is determined as much by the situation, the
  \co{actual} context of the discourse, the participating persons, as by the
  used \co{signs}.}  \co{Reflection} might be now taken simply as the sphere
allowing these three \co{aspects} to function in a relative independence.

This independence is embodied in the structure of \thi{as} (and is related to
the abstract or conventional character of the new level.) \thi{As} in \co{sign
  as a sign} signals the \co{dissociation} of \co{sign} from what it signifies
(the \co{object} of \co{reflection}, \ref{sub:subobj}). But it comes in various
concrete forms. Seeing something {\em as} something, $x$ as $y$, is \co{founded}
in the fact that $x$ has been \co{dissociated} from its \co{actual} presentation
and, on other occasions, could also be seen as something else.  In the deepest
sense, \thi{as} is a \co{reflection} of a variety of \thi{aspects} of one
\nexus. One can view love \thi{as} enslavement and \thi{as} liberation,
friendship \thi{as} obligation and \thi{as} gratification and \thi{as}...  It is
no coincidence that all such \thi{aspects}, contrary as they might appear, are
joined by \wo{and} which represents the fact that they are only possible
\co{actual} manifestations of a unitary \nexus. In more mundane examples, one can
see a duck-rabbit drawing \thi{as} a duck or \thi{as} a
rabbit\ftnt{\citeauthor*{WittPI}{ II:xi\kilde{p.166}}}, one can see the
drawing\raisebox{1.2ex}{\cube{0.8cm}} 
\thi{as} a glass cube or \thi{as} a solid angle or \thi{as} a wire frame or...
(And, of course, the context, the background against which the drawing appears,
can contribute significantly to how one will see it.)  Here we notice the
difference: various \thi{as...} are now joined by \wo{or} for, indeed, one 
cannot see it as {\em both} glass cube and solid angle. This difference signals
a new status acquired by a \co{sign}. In a sense, it has become itself an
object, it has become independent from its signification and can now
\co{represent} different objects, depending only on \thi{as} {\em what} one sees 
it. It is this \co{dissociation}, where not only one \nexus\ happens to have
different \co{aspects} and \co{actual} manifestations, but where also one
\co{sign} can \co{represent} different objects, which marks sharply the level of
\co{reflective representation}.\noo{\citefib{The question now arises: Could
    there be human beings lacking in the capacity to see something {\em as
      something} -- and what would that be like? [...] The \thi{aspect-blind}
    will have an altogether different relationship to [\co{actual}] pictures
    from ours.}{WittPI}{\kilde{p.182}} Detailed empiricism might like to
  distinguish a purely \thi{symbolic} representation from the ability to
  understand the orders of the form \wo{Bring me something that looks like {\em
      this.}} which could be executed on the basis of some assumed
  associationism. But to us it seems that \thi{seeing something as something} is
  \equi\ with the use of \co{signs} which are known \co{as signs}, that is, with
  \co{reflection}. Inability to handle \thi{as} amounts to the inability to
  handle \co{signs as signs}.\noo{Even if the case of such an \thi{aspect-blind
      existence} were imaginable, it would hardly be human existence and would
    rather remind of a machine.} }

\subsubnonr{Distinctions in the (same) indistinct}

\pa Just to anticipate a possible worry which we will address in more details in
the last section of this Book, in particular, \ref{se:toBeDist}.

\co{Signs} are \co{actual} tokens of \co{distinctions} -- \co{distinctions}
which are drawn and made {\em in} the current situation, {\em in} the world but,
eventually, in the same \co{indistinct}. Every \co{distinction} makes a
\co{distinction} in the \co{indistinct}, and so does (\co{meaning} of) every
\co{sign}.  There is nothing peculiarly \thi{mental} about the meanings of
\co{signs}.\ftnt{We do not know what \wo{mental} means, unless it is, perhaps,
  just the meanings of \co{signs}.}
% \wo{Red} or \wo{house} mean the \co{distinctions} in the \co{indistinct}, not
% anything remaining fully and exclusively \thi{within one's head}.
Meanings typically \co{transcend} the \hoa\ and, in any case, may remain fully
\co{external}.  Hearing you saying \wo{There is a danger around the corner!}
changes the \co{actual} situation, makes a difference. 
It does not happen \thi{only in my head} -- it
modifies the world in which I actually am. (Of course, it does not modify it
\thi{physically}, but the material, physical things constitute only a minor
element of the world which only seldom concerns us.) In this sense every
linguistic \co{sign}, every utterance, is a true \thi{speech act}: it effects
some \co{distinctions} in the matter of the world.

Asking \wo{How does a thought act?} Frege gives the immediate answer: \citet{By
  being apprehended and taken to be true.}{FregeTruth}{\kilde{[p.104]}}
\co{Distinction}, we could say, acts by merely being apprehended, although this
would here mean simply effecting some \co{distinction}. (And as we have remarked
several times, often the mere fact of a triviality being uttered, introduces
\co{distinctions} far beyond the mere fact of the utterance.) \co{Distinction}
can not be \co{dissociated} from its meaning because every \co{distinction} is
its own meaning, is a \co{distinction} only in so far as it makes a difference,
even if no practical and observable consequences follow. A \co{sign}, and in any
case a \co{sign as a sign}, has its being independent from its possible
\co{meanings}, but it is a \co{sign} only to the degree in which it effects some
\co{distinctions}, i.e., has some \co{meaning}.

This crucial point seems to go counter much of common-sense but, at the bottom,
it complies with it. \co{Recognition} of a \co{sign}'s \co{meaning} is not
something which happens \thi{in my head} as opposed to some mysterious
\thi{reality outside}: a \co{sign} is a \co{sign} only to the extent it is
\co{recognised}, and its \co{recognition} amounts to drawing some
\co{distinctions}. And \co{distinctions} are not something \thi{in my head} but
in the world, eventually, in the \co{indistinct}. The \co{meaning} of the
exclamation \wo{There is a danger around the corner!} is the set of
\co{distinctions} it draws, the way it changes the world.
%
\co{Sign} is not any copy of anything, and neither
is its meaning. \co{Sign} is originally an \co{actual aspect} of a wider
experience and, eventually, of the whole \co{experience}. Its \co{meaning} is
not any picture one carries and can recreate \thi{in one's head} -- it is the
way in which it can affect the world, a set of \co{distinctions} carved,
eventually as all \co{distinctions}, in the \co{indistinct}.
% (through the intermediary \co{distinctions} constituting the world, i.e., our
% \co{signs} do not all the time address the \co{indistinct} directly, but are
% rather introduced into the texture of other \co{distinctions}\ftnt{\wo{The
%     danger around the corner} may effect different \co{distinctions} when
%   addressed to various persons in different situations.  We do not believe in
%   any \thi{propositions} or, as Frege could also say, \thi{complete thoughts},
%   which stay somewhere between the \thi{inner} and the \thi{outer} world. There
%   is some amount of consensus concerning the \co{distinctions} meant by this
%   \co{sign}, but this does not require it to have any static and unique
%   \thi{essence}, any fixed meaning. In every situation the meaning of this
%   \co{sign} may be slightly modified. But just like the lack of any rigid
%   \co{distinctions} does not mean that there are no \co{distinctions}, so the
%   lack of any \co{precise} \thi{propositions} (and eventually of any
%   \thi{essences}) does not mean that \co{signs} have no meaning.})

\noo{
The \co{sign}
\wo{the Eiffel tower} makes me \thi{see} the Eiffel tower, carves it on the wall
at which I am actually looking, or else in the darkness of my closed eyes. It
may also create the mood of my last visit in Paris when I encountered something
special at the Eiffel tower. In either case, this \co{sign} does not carve the
tower fully, the \co{actualised distinctions} do not match the \co{distinctions}
made when I am actually standing in front of the Eiffel tower, the element of
\co{actual externality} is lacking, but ultimately these \co{distinctions} have
the same status as those -- in the latter case, there are only some others, not
present in the former case.
}

What kind of \co{distinctions} a \co{sign} effects, and if they are sufficient
for, say, me deciding to act according to them is a completely another question.
It may not draw them in the final and sufficiently definitive way. Hearing about
the danger around the corner, I may wish some further \co{distinctions} to be
made which should carve the world along the same lines.  To be convinced that
there is \co{actually} a danger, I may need more information, perhaps, to know
whence you know it, what danger it is, perhaps, to \co{actually} see it myself.
All such steps are but collecting the \co{distinctions} which may or may not
modify the world in the same way as the original announcement did. At some
point, I may indeed stop further verification and conclude that what you said
was true.  If there were no distinction between meaning and using, I could
hardly wonder about that -- at most, I could wonder what you mean by saying
\wo{Danger!} when I see none.  But truth, although itself an adventure of
meaning, is a further story, which we postpone for the moment
(II:\ref{sub:truth}).\ftnt{If this looks like a repetition of Wittgenstein's
  \wo{language is a tool}, then it probably is. Wittgenstein seems happy with
  the fact that the tool works and analyses {\em how} it happens. The question
  one might ask is: {\em what} makes the tool work?  The question could be
  declared illegitimate if the tool were inherent part of the game but, as it
  seems, humankind played the game once without it and, moreover, every baby
  begins to play it without.  {\em What} makes it possible to include language
  into the whole game, to turn game into a language-game? It seems to be the
  fact that words do have \co{meanings} and that these \co{meanings} get woven
  into the rest of \co{distinctions}, i.e., both are \co{distinctions} in the
  same matter of life. Putting no more (and perhaps less) charity than necessary
  into the reading of the private language argument ({\em if} such an argument
  was given and {\em if} it was an argument -- we leave clarifications to
  scholars), it says: a solipsistic subject could not possibly maintain any
  reliable criteria of correct usage of his \co{dissociated} signs, of the
  consistency in obeying conventions of their usage. Why not? Because the only
  reliable criteria, if any, of following a rule are other players. We can
  accept critique of \thi{mental states} and \co{dissociated} sensations but
  have problems with understanding why seeing Eiffel Tower twice, one needs any
  conventions to \co{recognise} it as one and the same. If this can happen, one
  should be also able to associate with these two \co{experiences} the same
  abstract sign, say, \wo{Eiffel Tower} (or \#!*), and use it consistently. (Of
  course, our whole development suggests that we maintain the possibility of an
  individual to actually {\em establish the very relation} sign-signified, not
  only to utilise such a relation established socially for defining new and
  private special cases of it.)  Sure, private convention is not public
  convention and no social verification of correctness can be ensured. But,
  although public verification often {\em may} be more reliable, at bottom it is
  neither better nor different than one's private ability to \co{recognise}
  Eiffel Tower for the third time as \wo{Eiffel Tower} (or \#!*).  Language is
  essentially social in that it accumulates sedimented meanings with which each
  individual is confronted. But if one did not have the private ability to
  establish the relation of \co{meaning}, the mere rules of use could hardly
  guarantee anything of significance except, perhaps, a general consensus. (Such
  a consensus maintains only the sedimented \co{meanings}.) This private ability
  with respect to most \co{actual} \co{experiences} and \co{signs} is
  \co{founded} in the deeper layer of \co{signs} and \co{recognitions} which are
  not \co{dissociated} as abstract signs are.  Eventually, private consistency
  can be maintained because \co{existence} is not a solipsistic subject in need
  of some \thi{internal} verification criteria but, on the contrary, is
  \co{confronted} with something it is not, whence criteria for lower levels
  arise from the higher ones. (The possibility of imagining, after Max Scheler,
  a \thi{lifelong Crusoe}, who grows up alone and nevertheless establishes and
  uses consistently a system of private signs, seems to be a critique of the
  \thi{argument}, e.g., in \citeauthor*{LanguageRules}. But if we were willing
  to see the \thi{argument} as directed against solipsism in general, not only
  against linguistic solipsism (e.g., \citeauthor*{LanguageRulesMalcolm}), there
  need not be any genuine conflict between these two interpretation schools.)}
%   in the light Viewing the \thi{argument} as an point against solipsism, it does not
%   really matter whether one arrives at the necessity of others or of external
%   world. In this respect, there seems to be no reason to argue between the
%   \thi{community view} (which maintains that the point was to show that language
%   is essentially social) and the \thi{lifelong Crusoe view} (according to which, a single
%   person could, too, develop some language independently from others. We seem to
%   agree with the latter but not necessarily in its possible opposition to the
%   argument itself.)}


\noo{But it is never the question of some unimaginably irrefutable proof. No.
\citet{We are satisfied that the earth is round. The reasonable man does {\em
    not have} certain doubts.}{Certain}{\para 299, \para 220}
}

\subsection{Subject-object}\label{sub:subobj}
\pa
To \co{externalise} means to objectify. And vice versa, to \co{dissociate} an object 
means to posit it as distinct from the positing consciousness, as \co{external}. 
\co{Externality} is what constitutes an \co{object} in the strict sense. 
\co{Recognitions} of \co{awareness} contain the germ of this 
\co{externality} since they appear through \co{signs}, involving an element 
of non-\co{actuality} or \co{non-actuality}. But it is only when I pause to
\co{reflectively}  
consider an \co{object}, when it is pulled out of the horizon of
\co{experience} and enclosed within the \hoa,
that it appears as an \co{external}
totality. Objects of \co{awareness}, of \co{experience}, are not 
\co{external} in this strong sense; they are \co{distinguished} but 
their \co{non-actuality} is only a germ of \co{externality}.
In this strict sense, an \co{object} is only an object of \co{reflection} 
and only \co{reflective} being is confronted with \co{objects}.

Since \co{externalisation} involves double \co{dissociation},
\refp{pa:doubleSep}, in particular, the \co{dissociation} of its \co{object}
from itself, the respective \co{sign as a sign} has also more definite direction
towards its \co{object} than it was the case before.  \co{Distinctions} and
\co{signs} of \co{recognitions} referred primarily to the background.  A
\co{sign} of \co{reflection}, appearing \co{as a sign}, is by the same token
explicitly directed towards an \co{external} \co{object}.  \co{Object} is more
sharply distinguished from its background than it was in \co{recognition};
\co{sign as a sign} focuses more sharply on its \co{object} than a simple
\co{sign} does.

\pa \co{Objectivity}, like most other things, is a matter of degree.  In this
Book, we are drawing \co{distinctions} separating different levels of
ontological \co{founding}.  Different levels mark, indeed, distinctions in
nature.  But the distinction in nature between levels is but a distinction of
degree which has been drawn so far, which has been so intensified, that it
caused emergence of new \co{aspects}.  Thus, for instance, there is a
distinction in nature between \co{recognition} and \co{representation}, in that
the latter introduces \co{signs} which are no longer mere \co{signs} but are
\co{signs as signs} (we may say, not only natural but also abstract signs).
Yet, \co{awareness} is a nucleus from which \co{reflection} develops by a
process of further differentiation.  In this sense, they are on the same,
continuous line, on the same \co{trace}, and all differences are only
differences of degree.

This applies also to \co{objectivity}.  There are degrees of being an
\co{object}, that is, degrees of the potentiality for being captured as a
\co{sign} within the \hoa.  This pen here can easily be pointed at and made into
an \co{object}.  \thi{The whole world} can, too, be \co{posited} as an
\co{object} of \co{reflection}, but we feel easily that there is a significant
difference between the two. I can isolate myself as an independent entity, as an
\co{objective} totality and think of my, then only \co{external}, relations to
the world and other people. I can then also think about \thi{my time}, the time
of my life, as different, perhaps, even independent from the time of others. But
doing this I know immediately that it is not the whole truth about myself, that
the isolation went too far, and that this being I have just {objectified} stands
in more intimate relations to its surroundings than those I can discover
treating myself as an \co{objective} entity.

\pa \co{Objects} are sharply distinguished, \co{externalised} contents.  This
sharp \co{precision} is possible due to the \co{dissociation} of an \co{object},
the fact that it is entirely inscribed within the \hoa\ from which it has
suppressed all competitors.  The fact that \co{objects} carry with themselves the
aspect of \co{non-actuality} escapes easily \co{reflection} and appears for it
only as the fact of their \co{externality}.  \co{Externality} is the distance
separating \co{reflection} from its \co{object}, which is just another side of
the \co{distance} separating it from its origin.  \co{Reflection} is a mode of
perception and understanding, a hypostasis of being, farthest removed from the
\co{virtuality} of the \co{original} \co{nothingness}.
%Perhaps, one can imagine some further dissociation and differentiation
%To this distance there corresponds \co{externality} -- the distance
%separating it from its \co{object}.

\pa
%The two emerge simultaneously: 
As contents become more \co{precisely} \co{dissociated} from their background
and turn, eventually, into \co{objects} of \co{actual reflection}, so does
emerge their counterpart within the unity of \hoa, the \co{actual} \co{subject}.
The inseparable relation of \co{subject}-\co{object}, always thought in terms of
pure \co{actuality}, is an event of \co{reflection}: an isolated, purely
\co{actual} \co{object} and equally \co{actual}, instantaneous \co{subject} --
this is how the tradition, whether its idealistic, rationalistic or empirical
branch, used to see it.  Of course, in order to obtain a purely \co{actual}
\co{subject} which, nevertheless, has something to do with at least a shadow of
the real world, one has to invent a lot of transcendental machinery,
constitutions, and what-not.  And, on the other hand, starting with a purely
\co{actual} \co{objects} which, nevertheless, should appear for at least a
shadow of a real being, one has to perform a lot of constructions, sensations,
associations, amalgamations, juxtapositions, shortly -- desperate, even if
ingenious sewing, before one is forced to unwillingly give up.

For us, the \co{objects} are the final hypostases, the \co{actual} limits
(usually, only provisional and never necessary) of a process of
\co{distinguishing}.\ftnt{Virtually the same idea -- that \co{objects} are
  results of {objectification} relative to the \co{existence} -- can be found
  in \citeauthor*{Bier}, in particular, II.}  Some of these \co{objects} fit
better into our sensuous and perceptual capacities, into our scope of \hoa, and
these are the most common \co{objects}.  Others do not fit equally well, always
immediately announcing the inadequacy and insufficiency of the \co{objective
  representation}, and these are termed more \wo{subjective}.  In either case,
the \co{actual} \co{subject} is nothing more than the fact of the \co{reflective
  sign} appearing \co{as a sign}, of its non-coincidence with the signified
\co{experience} -- \co{actual subject} is the place, or better, the event of
this \thi{non-coincidence}.

\pa\label{pa:KnasterTarski} As in \refpp{pa:selfaware}, the distance separating
\co{object} from \co{subject} is the same as the distance separating
\co{subject} from \co{object} (as we will see in
\ref{sub:timeSpace}.\refp{pa:afterIsBefore}, \co{before} is the same relation as
\co{after}).  The sharper separation of the \co{object}, the sharper separation
of the \co{subject}; the more \co{external}, independent the former becomes, the
more precisely and definitely the latter is delineated from the background of
\co{experience}.  The two are \equi\ \co{aspects} of \co{reflection}.

And yet, the \equin\ of \co{awareness} and \co{self-awareness} does not go over
into the \equin\ of \co{reflection} and \co{self-reflection}. Further
\co{dissociation} has taken place and \co{self-reflection} is no longer a
necessary \co{aspect} of the \co{actual subject}. In fact, \co{reflection} and
\co{self-reflection} are incommensurable because they represent two different
\co{acts} which are hardly ever performed jointly.  \co{Actual subject} is
directed exclusively towards the \co{actual object}, it is exhausted within the
\hoa.  \woo{The mark of the mind is that there do not arise more acts of
  knowledge than one at a time.}{NayaSutra, quoted earlier} \co{Reflection} is
continually focused on its \co{objects}, in spite of their \co{externality} it
almost \thi{identifies} itself with them.  In this process, that is, in the
series of \co{reflective acts}, \co{self-reflection} can (although it seldom
does) arise as one of them.  Limited to the \hoa, \co{reflection} can occupy
itself with the \co{objects} only to the extent it forgets itself, only to the
extent it does not \co{reflect} over itself.  \co{Reflection} forgets itself and
in order to catch a glimpse of the \co{self-awareness} which underlies its
fascination with the \co{objects}, it has to actively gather itself to perform
another \co{act}, an \co{act} of \co{self-reflection}.  This is the site of
infinite regress. The \thi{I} {objectified} in an \co{act} of
\co{self-reflection}, being an \co{object}, is always distinct from the
\co{reflecting subject}. To make the two coincide, one has to \co{posit} an
infinite chain of such acts and claim the existence of the fix point obtained as
its ideal limit.\ftnt{After $\omega$ iterations of that-ing one gets to the
  point \thi{$f=$ that$_1$(that$_2$(that$_3$(...}, i.e., a fix point where no
  more that's add anything new, so that $that(f)=f$. Knaster-Tarski theorem is a
  mathematical reflection of this coincidence, specifying also simple conditions
  on the operation (here, that-ing) under which $\omega$ iterations indeed yield
  a fix point.\label{ftnt:KnasterTarski}} This ideal construction is as much as
\co{reflective dissociation} can do to re-construct the intuited \co{unity} of
\co{awareness} and \co{self-awareness}.

Thus, although \co{object} and \co{subject} are \equi\ \co{aspects} 
of \co{a reflective experience}, \co{reflection} and 
\co{self-reflection} are not such \co{aspects} -- they are two 
different modifications of a \co{reflective act}. If they ever find 
place together, in a simultaneity of the \hoa, this can happen only by 
a considerable effort of will and attention.

%One should take \co{subject} here very seriously, not as a mere
%noumenal entity.  Awareness of a \co{sign as a sign}, that is, of a
%\co{sign as a sign} of ..., in short, of an \co{object}, is at the
%same time, and by its very nature, awareness of the \co{subject}, of
%itself.  There is nothing like \thi{a subject}, pure and simple, which
%later, in some magical way, acquires self-awareness.  To be confronted
%with an \co{object} means to be aware of it being \co{external},
%distinct from ...  {\em me}, that is, to be self-aware. 
%\co{Reflective consciousness} implies self-consciousness.  And on the
%other hand, to be self-conscious means to be aware of oneself, of
%oneself as distinct from ...  something else, from something
%\co{external}.  Self-consciousness implies \co{reflective
%consciousness}.


\pa
\co{Externality} has nothing to do with \co{spatiality}. It is just an 
aspect of \co{subjectivity}, of \co{reflection} 
which is \co{aware} of the distance separating it from its \co{object}. If 
my \co{reflection} chooses as its \co{object} my sensation of pain, a 
particular memory, a particular feeling, these will appear as 
\co{external} in the same sense as other \co{objects}. Thus 
understood, \co{externality} is opposed to something like \thi{internality}, 
to that which has not been \co{dissociated} as an independent \co{object}. What
this \thi{internality} comprises is a perplexing question because any answer
proposed by \co{reflection} posits immediately an \co{external object}.
By opposition to \co{objectivity}, one terms it \wo{\co{subjective}}. We will
return to this opposition, but for the moment it should be admissible to see
\thi{inner} life in the flux of \co{experience}, of heterogenous 
\co{distinctions} and \co{recognitions}, but only as long as one lets 
them flow, only as long as one does not isolate them as independent \co{objects}. 
In the moment one does it, they leave one's \thi{interior} and appear as 
\co{external}, even if not \co{spatial}, \co{objects}.

\subsection{Time and space}\label{sub:timeSpace}
%
\co{Externality} is something different from the 
three-dimensional {extensionality}, and time of the \co{objects} and their
changes (not to mention \co{temporality} of the flow of
\co{experience}) is different from the linear and objective time. Yet they are
both \co{aspects} of the \nexus\ of \co{subjectivity}-\co{objectivity}. Also,
they are steps in the process of emergence of the eventual objective time and space
and we will now follow this process. 

The \co{spatio-temporality} from \ref{se:temp} involved merely the
\co{distinction} between the simultaneous \co{aspects} of \co{actuality} and
non-\co{actuality}.  \co{Representation}, the \co{reflective} \co{repetition},
\co{dissociating} an \co{object}, establishes its identity. As an experienced
identity it underlies the emerging \co{experience} of time and space. As the
identity pushed to its ideal limit of the residual point (or \thi{substance}),
it gives also rise to the abstract (\thi{objective}) structure of both \co{aspects}:
linearity of time and homogenous space. Let us consider the former aspect first.

\subsub{Time}\label{sub:time}
%
\pa \co{Reflection} comes always \thi{too late}, it \co{represents}
something which \thi{has already been} \co{recognised} in the flux of
\co{experience}. And as any act involves its whole structure within itself, no
additional step is needed to establish the experience of \thi{\co{after}} -- it
is the distance separating the \co{representing sign} from what it
\co{represents}. It does not matter that, objectively speaking, this distance
may take time 0 (if such a thing exists ???). It is there, in the structure of
the \co{reflective experience} and hence also in the experience itself:
\co{reflection} \co{repeats} what it \co{reflects}, and this basic
\co{repetition} is the same as the primordial {\co{after}}.\ftnt{We are
playing here on the possible (and intended) conflation of the proposed notion of
\co{reflection} and reflection in the more common sense of the word.  As
suggested in \refp{pa:attentive}, the two are almost the same -- they have no
structural differences but the difference of degree.\noo{ If this is not convincing
enough, please, tell them apart.}}

This \co{reflective after} is not that of one \co{actuality} coming
after another. We are still \thi{within} the scope of the \hoa, where the first
\co{after} finds its place. We can view it simply as the Husserlian retention,
as the \ger{prim\"{a}re, frische Erinnerung}. The withdrawal of the
just-perceived-object into the immediate past amounts just to the impossibility
of \co{actually} grasping and retaining the \co{object} in the unity of the
\co{reflective act}. 

The \co{after} is the \co{distinction} between the \co{actual sign} and the
signified thing which it \co{dissociates} and \co{after} which it
comes.
\co{After} is the \co{trace} of the \co{nexus} of \co{experience} which
has been \co{dissociated} into the \co{objective} content and \co{subjective
sign} -- \co{after} is the \co{distance} between the two. This is also the
\co{distance} separating the ever fleeting \thi{now} from \co{reflection} which
always comes \co{after} it.  This \co{distance} is
experienced simultaneously with the poles it \co{dissociates}, all elements of the
relation \co{after} are experienced simultaneously within the \hoa.  The
\co{experience} \co{dissociated} by \co{reflection} plays for it the role of
simultaneous, \co{immediately} present virtuality.\noo{ (very much like the virtuality
of Bergsonian memories).} Its withdrawal into the background, effected by the
\co{reflective} \co{representation}, establishes the \co{after} which from now
on permeates the whole life of \co{reflection}.

We only want to claim that this event is not something that merely happens to
\co{reflective} consciousness, but is one of its constitutive
\co{aspects}. (Ideally, its description should be free from references to the
passing time because it is what constitutes the very \co{experience} of time.)
\co{Reflective experience} is \equi\ with the \co{experience} of time. The more
definitely \co{reflection} approaches \co{dissociation}, the more
\thi{objective} becomes the time of the experience.


\pa James, speaking as always only in terms of \co{actual experiences}
\citet{explained the continuous identity of each personal consciousness as a
  name for the practical fact that new experiences come which look back on the
  old ones, find them \thi{warm}, and greet and appropriate them as \thi{mine}.
  The pen, realised in this retrospective [\co{reflective}] way as my percept,
  thus figures as a fact of \thi{conscious} life.  But it does so only so far as
  \thi{appropriation} has occurred; and appropriation is {\em part of the content
    of a later experience} wholly additional to the original \thi{pure}
  pen.}{Radical}{IV:2;p.129 \noo{One can take this as a picturesque description of
  Husserl's retentions.}} Such \thi{appropriations}, pragmatically or
phenomenologically convincing as they may be, can be used for explaining the
emergence of a {\em conscious} \co{ego}, which \citetib{is a part of the content
  of the world experienced}{Radical}{VI:footnote;p.168} and which indeed seems
the only form of personal unity pragmatism is capable to account for.  But since
such \thi{appropriations} are \co{actual experiences} and the whole explanation
happens from the perspective of \co{actuality}, it must need to presuppose
continuity in time, or as we would say in temporality, which is more primordial
than any contents of \co{actual experiences}.  Such continuity does not pertain
to the \co{actual subject} except, perhaps, for the \co{experiences} of
\thi{appropriations}.  Continuity in time is \co{founded} in the sphere of
\co{experience} preceding time, and \co{reflection} finds itself always
perplexed by this continuity, since it is irresistibly separated from itself, as
it is from its \co{object}, by the \co{after}.\ftnt{We are not trying to counter
  James' excellent, often phenomenological, descriptions.  But they cannot
  suffice when we do not believe that everything can be reduced to and explained
  in terms of \co{actual experiences}.  For instance, one of the first
  conditions for \thi{new experiences coming and looking back on the old ones}
  is \citefib{that the new experience has past time for its \thi{content}, and in
    that time a pen [or whatever] that \thi{was}.}{Radical}{IV:2;p.129}
  Adequate as it is, it does assume \thi{past} which is given in experience.
  Indeed, it is.  One can rest satisfied with that, with describing \citefib{what
    can be experienced at {\em some definite time} by some experient [\ldots] in
    some concrete kind of experience that can be {\em definitely} pointed
    out.}{Radical}{VI;p.160; my emphasis} But we do not believe that everything
  can be found in such \thi{definite} moments.}


\pa The \thi{pure past}, the past which not only isn't merely a collection of
past \co{actualities}, but which even never had been an \co{actuality}, is the
way in which the horizon of \co{experience} can be thought by the \co{attentive
  reflection}.\ftnt{\citef{This table bears traces of my past life, for I have
    carved my initials on it and spilt ink on it. But these traces in themselves
    do not refer to the past: they are present; and, in so far as I find in them
    signs of some \thi{previous} event, it is because I derive my sense of the
    past from elsewhere, because I carry this particular significance within
    myself.}{PontyPerc}{III:2\kilde{p.413}} Bergson is probably the source from
  which Merleau-Ponty borrows the idea which later appears also in
  \citeauthor*{Differance}.} \co{Experience} is inaccessible to \co{reflection}
and this inaccessibility finds its very clear expression in \co{after}. Any act
of \co{attentive reflection} is immediately self-aware of having arrived at the
scene \co{after} its \co{object}. But this \co{after} is merely an expression of
the change of level, of the \co{distance} separating \co{reflection} from the
\co{experience} which never has been, and never will be, reduced to
\co{reflective actuality}. One has to emphasise the \thi{purity} in the
expressions like \wo{pure past} exactly in order not to confuse it with a
collection of other. though now past, \co{actualities}. \co{Experience} is not a
collection, not 
even a \co{totality} of \co{experiences}; it is what precedes \co{experiences}
and makes them possible. Preceding the differentiation into \co{experiences}, it
also precedes time understood as succession, in particular, the possibility of
past \co{experiences}, of \co{actualities} which are not \co{actual} now but
were so some other time.  Past thought of as past \co{actuality} is \co{founded}
upon the \co{experienced} duration and the transition from this \co{experience} to
\co{reflection}. This \co{foundation} remains around \co{reflection} as the
\thi{pure past}, which alone makes it possible for \co{actual experiences} to
recede into past and thus turn into past \co{actualities}. 

\pa\label{pa:afterIsBefore}
This, we could say, establishes perhaps the dimension of the past, but what
about the future? The future is, at least at this level, no different from the
past. \co{After}, as the \co{trace} of the \co{dissociated} \nexus\ of
\co{experience}, as the relation connecting the \co{actually} given \co{object}
and the background from which it emerged (or its \equi\ \co{aspects}: the
\co{subjective sign} and the \co{objective} content), is asymmetric and is
experienced as such.  The \co{actual sign} is \co{distinct} from what it
signifies and, furthermore, it comes \co{after}.  The \co{actuality}, this
\citet{strange crest of the time series}{Zeit}{B:2.\para 26}, appears as the
point into which all \co{experience} converges, to use Bergson's image, as the
tip of the cone of the whole past.  This \co{after} means not only the
\co{distance} {separating} the \co{reflective sign} from the \co{experience} but
also its impassability -- \co{reflection} can never re-capture the
\co{experience}, because it always comes \wo{post factum}. This asymmetricity
gives the time arrow its direction.

The rest is uniformity by analogy -- \co{after} is asymmetric: 1) \co{objects}
are what is experienced through the \co{actual signs} of \co{reflection} which
involves them in the relation {\co{after}}, but 2) \co{after} {\em is the same}
relation as \co{before} -- \co{reflection} coming \co{after} $x$ means the same
as $x$ coming \co{before} \co{reflection}); 3) in a sense, \co{reflection} is
the future of its \co{object} which is always past and \co{after} which it arrives;
more abstractly, 4) future -- and now it is the future of \co{reflection}! -- is
just what is \co{after} the \co{actual reflection}. It is to the \co{actual
sign}, what this \co{sign} is to what it signifies, i.e., just like \thi{now} of
\co{reflection} is \co{after} what it signifies, future is \co{after now} of the
\co{reflection} -- it is a point of \co{reflection} over the \co{actual
experience} or, as the case may be with an \co{attentive reflection}, the point
of \co{reflection} over actual \co{reflection}.

This future which lies \co{before} is, of course, indeterminate, unlike the past
\co{after} which \co{reflection} relates to a particular, definite experience.
Past is something \co{actually reflected} and in this lies its definite,
determined character. Future, established by mere analogy, has only the
character of potentiality, of a possible \co{reflection}, it is a
\co{reflection} which has not happened yet.  This analogy by asymmetry
determines the dimension of the future. It can be found in the \co{immediacy} of
an \co{act} in the form of protention, \ger{anschauliche Erwartung}, which
presents (an aspect of) the object in some definite (as expected) form
augmented, however, with a sign of indeterminacy, the possibility of
unfullfilment, or else protention which anticipates the \co{immediate} action,
like the electric potential which can be measured over the entire scalp a
fraction of a second before a finger movement which one has already decided to
perform.  But future is not limited to the (affectively presented)
\co{immediacy} of expectation. It 
can be found in the general sense of openness of the future of one's
life and, eventually, in the abstract \thi{future of the world} in the objective
time.


\subsubi{{Objective} time}\label{sub:objectiveTime}

\pa\label{pa:temporality}
We have thus entered the dimension of \co{temporality}. 
But so far, \co{temporality} is not yet the objective 
time -- it has the dimensions of \co{actuality}, \co{after} and 
\co{before}, that is, of present, past and future, but these are, so 
to speak, subjectively localised, centered around the \herenow, which 
has become \thi{now}. 
Also, we have not established a uniform, global time. There may still be 
many different futures as there may be many different pasts and they 
may be only loosely (if at all) connected with each other. Even if 
they all pass through the unique \thi{now}, \co{temporality} still allows 
multiplicity of time paths. 

\co{Temporality} is objective in the sense of being an \co{aspect} of the
experience of \co{externalised objects}, but it isn't yet the objective time
\co{dissociated} from subjectivity and its apprehension of things.  It is still
time of \co{an experience} of \co{objects}, that is, still \co{temporality} with
a designated \co{actuality}, ``the present time'' of \herenow.  It shouldn't
sound too implausible, if we said that such a \co{temporality} pertains to any
being which has reached the level of discerning independent \co{objects}. A dog
bringing me a ball and looking expectingly into my eyes, waiting for me to throw
it away is, too, involved into \co{temporality}, just as it is when looking in
the bushes for the ball just thrown. 

\pa\label{pa:twoTimesA} Husserl describes two kinds of time consciousness: the
consciousness of time 
as it unfolds in the \co{actual experience} along the axis of retentions and
protentions, and another, \ger{uneigentliches 
  Zeitbewu{\ss}tsein}, which relates to the time of remote past and of lifeless
recollections.
\citet{We could say:
    temporality stands against the inauthentic representation of time, of
    infinite time, time and time relations which are not recognised in
    experience.}{Zeit}{A:1.1 \para {6} \orig{Wir k\"{o}nnen auch sagen: der
  Zeitanschauung steht gegen\"{u}ber die uneigentliche Zeitvorstellung, die
  Vorstellung der unendlichen Zeit, der Zeiten and Zeitverh\"{a}ltnisse, die
  nicht anschaulich realisiert sind.} \noo{The sentence refers to the criticised
  theory of Brentano's but it seems to express Husserl's point.}}
The dichotomy is quite significant, so let us summarise briefly
the main points. 

The former is the
time of immediate presence, of the \co{actual}, fresh retention (\ger{prim\"{a}re
  Erinnerung}) and the equally \co{actual} protention, the expectation of the
immediate continuation (\ger{anschauliche Erwartung}). The retention is aptly
illustrated by the famous figure:%~\ref{fig:retention}.
%\begin{figure}[hbt]\refstepcounter{FIG}
\[\xymatrix@R=0.2cm@C=0.4cm{
A_0 &&A_1 &&A_2 && A_3 \\
\bullet \ar@{-}[rrrrrr]\ar@{.>}[rrrrrrrr] \ar@{-}[rrrrrrddd]
  && \ar@{-}[d] && \ar@{-}[dd]  && \circ \ar@{-}[ddd] && \\
  && A_1'\ar@{.}[rrrr] &&  &&   && \\  
  &&  &&A_2'\ar@{.}[rr]&&   &&\\
  &&  &&  && A_3'\ar@{.>}[rrd]&& \\
  &&  &&  &&   &&
}\]\label{fig:retention}  
%\caption{Retention}
%\end{figure}
$A_0$ marks the initial point of the \co{actual experience}, the
\ger{Urimpression} of, say, an \co{object} $A$. The horizontal line indicates
the objective time in which the \co{object} may undergo some continuous changes,
indicated by the points $A_1,A_2,A_3$. (The discrete points are, of course, only
means of suggesting the genuine continuity of the process.) $A_1'$ represents
the \co{actual} impression of $A$ at the time-point $1$, $A_2'$ at the
time-point $2$.  The point 3 may here represent the idealised \co{immediacy} of
\thi{now} in which the impression $A_3'$ corresponds to the actual appearance of
the \co{object} $A_3$. The whole idea is that this impression relates not only
to the \co{immediacy} of the \co{object}, $A_3$, but also to its immediate past.
In a sense, it keeps and contains the whole line $A_0-A_3'$ with the
intermediary impression-points $A_1',A_2'$, etc., as indicated by the horizontal
dotted lines. The same happens at $A_2'$, which keeps and contains the past
$A_1'$, etc., so that \citet{each passing now retains retentionally all earlier layers.}{Zeit}{A:II.Beilage vi. \orig{Jedes vergangene
  Jetzt birgt retentional in sich alle fr\"{u}heren Stufen.} Some 15 years earlier James makes essentially the same observations:
  \citef{If recently the brain-tract $a$ was vividly excited, and then $b$, and
    now vividly $c$, the total present consciousness is not produced simply by
    $c$'s excitement, but also by the dying vibrations of $a$ and $b$ as
    well. If we want to represent the brain-process we must write it thus:
    $_ab^c$...}{PrincPsych}{I:9.3} 
  In a footnote, he remarks, concerning not only the retentional impressions,
  but the unity of a \thi{now} circumscribed by a horizon of gradually dissolving
  clarity: \wo{The most explicit
    acknowledgment I have anywhere found of all this is in a buried and
    forgotten paper by the Rev. Jas.~Wills, on \btit{Accidental Association}, in
  the Transactions of the Royal Irish Academy, vol.~XXI, part I
  (1846). Mr.~Wills writes: At every instant of conscious thought there is a
  certain sum of perceptions, or reflections, or both together, present, and
  together constituting one whole state of apprehension. Of this some definite
  portion may be far more distinct than all the rest; and the rest be in
  consequence proportionably vague, even to the limit of obliteration. But
  still, within this limit, the most dim shade of perception enters into, and in
some infinitesimal degree modifies, the whole existing state.}}

\pa
But now, there is also 
\citetib{inauthentic consciousness of time: a part of a perceived melody drained
  off a longer time ago.}{Zeit}{B:II.\para 27. \orig{Uneigentliches Zeitbewu{\ss}tsein: vor l\"{a}ngerer Zeit abgeflossene
  Teile einer wahrgenommenen Melodie.}}
\citetib{We say, that of which I am retentionally conscious is abolutely
  certain. How does it now stand with remoter past?}{Zeit}{A:I.2.\para
  22. \orig{Was ich retentional bewu{\ss}t habe, so sahen wir, das ist absolut
  gewi{\ss}. Wie steht es nun mit der ferneren Vergangenheit?}} Analysing a continuous experience, like that of listening to a melody, one
may still keep, towards its end, some living memory of its beginning, so that
\citetib{when I re-enact [the tone] $c,d$, this reproductive representation of
  succession finds its fulfillment in the yet living earlier
  succession.}{Zeit}{\orig{Wenn ich wiederhole [Tone] $c$, $d$, so findet diese
  reproduktive Vorstellung der Sukzession ihre Erf\"{u}llung in der noch eben
  lebendigen fr\"{u}heren Sukzession.}}
%
\noo{But we can reach also further into the
past. \citetib{Es kommt oft vor, da{\ss}, w\"{a}hrend noch die Retention von eben
  Vergangenem lebendig ist, ein reproduktives Bild von desselben auftaucht [...]
  (Dies Ph\"{a}nomen zeigt zugleich, da{\ss} zur Sph\"{a}re der prim\"{a}ren
  Erinnerung nebem dem intuitiven ein leerer Teil geh\"{o}rt, der sehr weiter
  reicht. W\"{a}hrend wir ein Gewesenes noch in der frischen, obschon leeren
  Erinnerung haben, kann zugleich ein \thi{Bild} davon auftauchen.)}{Zeit}{
  A:I.2.\para 30} We see how the phenomenological conscientiousness tries to keep
everything within the unbroken continuity of {\em one} experience.}
%
%  
But consciousness of time stretches much further than that.  \co{Attentatively},
we usually recall things which are not in any \co{actually} recognisable
(\ger{anschaulich}) continuity with the \thi{now}. Such a recollection intends
the original (now past) \thi{now} of the recollected experience or object and is
possible \citetib{thereby that against the flux of temporal withdrawal and of
  modifications of consciousness, there remains the object in its absolute
  apperceptive identity even as it appears to withdraw, the object actually
  experienced as \thi{this}. [...] It belongs to the essence of the modifying
  flux that this time point remains necessarily identical. The \thi{now} as the
  actual \thi{now} is the givenness of the actuality of a time point. As the
  phenomenon retires into the past, this \thi{now} retains the character of a
  past \thi{now}, but it remains the same \thi{now}, only that it emerges in the
  relation to the actual \thi{now} and temporary new \thi{now} as past.}
{Zeit}{A:I.2.\para 31. \orig{dadurch, da{\ss} gegen\"{u}ber dem Flu{\ss} der
    zeitlichen Zur\"{u}ckschiebung, dem Flu{\ss} von
    Bewu{\ss}tseinmodifikationen, das Objekt, das zur\"{u}ckgeschoben erscheint,
    eben in absoluter Identit\"{a}t apperzeptiv erhalten bleib, und zwar das
    Objekt mitsamt der im Jetzpunkt erfahrenen Setzung als \thi{dies}. [...] Zum
    Wesen des modifizierenden Flusses geh\"{o}rt es, da{\ss} diese Zeitstelle
    identisch und als notwendig identisch dasteht. Das Jetz als aktuelles Jetzt
    ist die Gegenwartsgegebenheit der Zeitstelle. R\"{u}ckt das Ph\"{a}nomen in
    die Vergangenheit, so erh\"{a}lt das Jetzt den Character des vergangenen
    Jetzt, aber es bleibt dasselbe Jetzt, nur da{\ss} es in Relation zum
    jeweilig aktuellen und zeitlich neuen Jetzt als vergangen dasteht.}}

In addition to this \co{externality}, objectified identity of single
\co{objects},  one last element seems indispensable
to constitute the consciousness of the fully uniform and homogenous time of the
objective world.  \citetib{For the
  emergence of this time consciousness, reproductive recollection (intuitive as
  well as in the form of empty intentions) plays important
  role.}{Zeit}{A:I.2.\para 32. \orig{F\"{u}r das Zustandekommen dieses
  Zeitbewu{\ss}tseins spielt die reproduktive Erinnerung (als anschauliche wie
  in der Form leerer Intentionen) eine wichtige Rolle.}}  \citetib{Only in
  recollection I can re-enact an identical time object, and I can also state in
  remembrance that what was earlier perceived is the same as what is later
  recollected.}{Zeit}{A:II.Beilage iv. \orig{Nur in der Wiedererinnerung kan ich einen
  identischen Zeitgegenstand wiederholt haben, und ich kann auch in der
  Erinnerung konstatieren, da{\ss} das fr\"{u}her Wahrgenommene dasselbe ist wie
  das nachher Wiedererinnerte.}} The reproductive recollection does not have the
capacity to \co{actually} bring the original object or experience to life
(\ger{Anschaung}). It can only intend it, as if \co{positing} the objective
identity across the time which broke the continuity of the experience of the
object. Thus, the flow of time becomes a rather abstract succession of
time-points which can be imagined as extending indefinitely.  \citetib{The
  reproduced time field reaches farther than the \co{actual}. If we pick there a past
  point, the reproduction yields, through an overlap with the time field in which
  this point was \thi{now}, further withdrawal into the past, and so on. This
  process is obviously to be thought as unlimited, although the actual
  recollection fails in practice.}{Zeit}{A:I.2.\para 32. \orig{Das reproduzierte
  Zeitfeld reicht weiter als das aktuell gegenw\"{a}rtige. Nehmen wir darin
  einen Vergangenheitspunkt, so ergibt die Reproduktion durch \"{U}berschiebunng
  mit dem Zeitfeld, in dem dieser Punkt das Jetzt war, einen weiteren
  R\"{u}ckgang in die Vergangenheit usw. Dieser Proze{\ss} ist
  evidenterma{\ss}en als unbegrenzt fortsetzbar zu denken, obwohl die aktuelle
  Erinnerung praktisch bald versagen ist.}}


%\bpa{Must be one time}
\pa\label{pa:twoTimes}
We are far from questioning the ingenuity and adequacy of these 
phenomenological descriptions of both (or rather, as can be gathered even from the
few included quotations, of several) levels/kinds of time experience.
\noo{
On the one hand, we have cases when \wo{[d]er Ton wird in der Phantasie
  \thi{erneuert}(\thi{wiedervergegenw\"{a}rtigt}, reproduziert).} And, on the
other hand, cases when \citetib{[d]er Ton is eben verklungen, erscheint aber nicht
  in der Weise eines Phantasma, einer \thi{Reproduktion}. Trotzdem habe ich ihn
  \thi{eben geh\"{o}rt}, habe noch ein \thi{Bewu{\ss}tsein} davon. Die Intention
  auf ihn dauert noch fort, ohne da{\ss} die Kontinuit\"{a}t des Meinens
  unterbrochen gewesen sein m\"{u}{\ss}te.
  
  Das ist doch ein wesentlicher Unterschied!}{Zeit}{B:I.10}
}
%
An \co{actual} {object} or event retires gradually
into the past, 
\noo{in die Vergangenheit r\"{u}cken} dissolving eventually in the horizon, that
is, {\em disappearing beyond} the \hoa. Once that happened, we can no longer
make it \co{actually} 
alive; we can only reproduce it, as if recalling it from beyond the grave. This
broken continuity makes the two kinds of experiences so fundamentally different
that one might perhaps legitimately ask what makes them both experiences of the
same time? What does the time of retentional \co{actuality} and fresh remebrance
have to do with the
time of remote, dead and only revived recollections?

The persisting identity of an \co{object} may help to understand the continuous
uniformity of the objective time but not its unity with the time of
\co{immediate experience}. Husserl answers the question by refering to the
double intentionality of time consciousness which, at every point, intends not
only its (lasting, changing or even disappearing) object but also the very
experience of this object.  \citetib{It belongs to the essence of the
  phenomenological situation that each past can be transformed reproductively
  into a \verify{}reproducing \thi{now}, which itself has some past. This is the
  phenomenological foundation of all laws of time.}{Zeit}{B:II.\para
  23. \orig{Zum Wesen der ph\"{a}nomenologischen Sachlage
  geh\"{o}rt, da{\ss} jedes \thi{Vergangen} reproduktiv in ein reproduktives
  \thi{Jetzt} verwandelt werden kann, das selbs wieder ein Vergang hat. Und das
  is das ph\"{a}nomenologische Fundament aller Zeitgesetze.}} \citetib{We have
in the flow of consciousness double intentionality. Either we consider the
content of the flow with its form of a flow. [...] Or we direct the view to the
intentional unity, to that which in the \verify{}stream of the flow is
intentionally given as unity: then emerges for us the objectivity of the
objective time, the authentic time field against the time field of the stream of
experience.}{Zeit}{A:II.Beilage viii. \orig{Wir haben im Bewu{\ss}tseinsstrom
  eine doppelte 
  Intentionalit\"{a}t. Entweder wir betrachten die Inhalt des Flusses mit seiner
  Flu{\ss}form. [...] Oder wir lenken den Blick auf die intentionalen Einheiten,
  auf das, was im Hinstr\"{o}men des Flusses intentional als Einheitliches
  bewu{\ss}t ist: dann steht f\"{u}r uns da eine Objektivit\"{a}t in der
  objektiven Zeit, das eigentliche Zeitfeld gegen\"{u}ber dem Zeitfled des
  Erlebnisstromes.}} In terms of the 
figure from \refp{fig:retention}, this says that the moment $A_3'$ involves both the
actual apprehension $A_3$ of the object $A$ (typically, with its temporal
character) {\em and} the process of its continuous apprehension represented by
the line $A_0-A_3'$. Consciousness of a temporal object involves also, by its
very nature, the consciousness of the very stream of consciousness.
\citetib{This is the one, unitary stream of consciousness in which there is
  constituted the immanent temporal unity of a tone as well as the unity of the
  stream of consciousness itself. Obnoxious (if not contradictory) as the fact
  that the stream of consciousness constitutes its own unity appears, it is
  nevertheless so.}{Zeit}{A:I.3.\para 39. \orig{Es
  ist der eine, einzige Bewu{\ss}tseinsflu{\ss}, in dem sich die immanente
  zeitliche Einheit des Tons konstituiert und zugleich die Einheit des
  Bewu{\ss}tseinsflusses selbst. So anst\"{o}{\ss}ig (wo nicht anfangs sogar
  wiedersinnig) es erscheint, da{\ss} der Bewu{\ss}tseinsflu{\ss} seine eigene
  Einheit konstituiert, so ist es doch so.} [As the stream
  of consciousness already here denotes the absolute subject (and will turn into
  the idealistic subject even more during later phases of Husserl's thought), we
  could really recognise here an \co{aspect} of the \nexus\ of \co{awareness}
  which is \equi\ with \co{self-awareness}, \refp{pa:selfaware}.]}

\pa\label{pa:unityTime} This answer remains satisfactory only as long as we are
willing to accept 
some compromises. For the first, we have to accept the phenomenological view and
treat memories not as factually coming from the factual past, but as merely
actual phenomena carrying a peculiar past time-stamp on
them. This aspect of \la{epoche} seems particularly unpleasant to us involving
the fundamental reduction to \co{actuality}.\ftnt{Husserl specifies that
  phenomenology does not aim at the ideal time-point. \citefib{That all reality
    lies in an indivisible now point, that in phenomenology everything is to be
    reduced to this point, these are downright fictions leading to absurdities. In
    phenomenology we do not deal with objective time but with the givenness of
    adequate \verify{}perceptions.}{Zeit}{
    B:I.12.\noo{p.169} \orig{Da{\ss} alle
    Realit\"{a}t in dem unteilbaren Jetzpunkt liegt, da{\ss} in der
    Ph\"{a}nomenologie alles auf diesen Punkt reduziert werden sollte, das sind
    lauter Fiktionen und f\"{u}hrt zu Absurdit\"{a}ten. In der
    Ph\"{a}nomenologie haben wir es nicht mit der objektiven Zeit, sondern mit
    Gegebenheiten der ad\"{a}quaten Wahrnehmung zu tun.}}
  And it is just the event of such a \thi{givenness} within the \hoa,
  \citefib{an \co{act} which forms the \co{actual now}, ``now'' in the sense of
  the crest of the actual time field.}{Zeit}{B:II.26.\noo{p.206} \orig{einen Akt,
    der ein jetzt Gegenw\"{a}rtiges erfa{\ss}t, ``jetzt'' im 
    Sinne des Gipfelpunktes der jeweiligen Zeitreihe}} 
  which is an \co{act} of \co{dissociation}, that is, \co{reflection}.
  Phenomenology does not attempt reduction to an ideal time-point, but it is
  thoroughly reductionistic in that everything must be expressed in terms of
  \co{an actual experience} grasped in the unity of a \co{reflective act}.}

A closely connected issue concerns the two related, but also essentially
different, aspects of the involved objectivity.  Moving within the \hoa\ one
addresses only the unity of the aspects from figure~\refp{fig:retention}, of the
objective time of the \co{actually} apprehended \co{object} and of the time of
its apprehension stretching between retentions and protentions which all belong
to the \co{actuality}, an \co{actual} \citetib{covering of the
  \verify{}reproductive with the retentional process.}{Zeit}{A:I.2.\para 22.
  \orig{eine Deckung des reproduktiven mit einem retentionalen Verlauf.}} But
the break to which we 
referred arises between the unity (or totality) of such an \co{actual experience}
and the one which has completely disappeared from the \hoa, between the time of the
\co{actual object} and the time of perhaps the same object as it is remembered from a year
ago.  The unity obtained here is only the unity of the \co{actual experience},
of the temporality of the \co{act} of remembrance ($A_0-A_3'$) {\em and} the
\co{actual}, that is, actually represented temporality of the content of this
\co{act}, of the object as actually recollected (which is now the \co{actual
  object}, i.e., $A_0-A_3$). It is not the unity which lets the (remotely) past
time flow into the actual (experience of) time, but only one which lets the
actual recollection of the past time be unified with the actually flowing
time.\ftnt{A:I.2.\para 23, concerning covering of the intended past by the 
  reproductive \thi{now}, achieves just that. The \thi{relation} between the
  actual and the past \thi{now} emerges as \wo{Gegenbild der zeitkonstituierenden
    Intention\"{a}litat} which, as far as we have seen, is always an event of
  pure \co{actuality}. More significantly, the two streams whose unification is
  supposed to yield the covering of the recollected by the recollection are the
  stream of reproductive modifications (recollections, $A_0-A_3'$) and the
  parallel stream of the recollected moments ($A_0-A_3$). The peculiarity of the
  situation consists in that the later stream is itself a repetition of an
  earlier one -- the experienced object is a past object/event with its past
  duration. Thus, we could expand the figure~\refp{fig:retention}, with an
  additional objective stream $B_0-B_3$, as if hidding behind $A_0-A_3$ (which,
  by the way, may also reflect some of the meaning of the expression \wo{double
    intentionality}).
%
\noo{ does not work in footnotes
\xymatrix@R=0.1cm@C=0.2cm{
B_0 \ar@{-}[rrrrrrr] &&B_1 &&B_2 && B_3 && \\
A_0 \ar@{-}[rrrrrrr] \ar@{-}[rrrrrrddd] %\ar@{.>}[rrrrrrrr]
  && A_1 \ar@{-}[d] && A_2 \ar@{-}[dd]  && A_3 \ar@{-}[ddd] && \\
  && A_1'\ar@{.}[rrrr] &&  &&   && \\  
  &&  &&A_2'\ar@{.}[rr]&&   &&\\
  &&  &&  && A_3' %\ar@{.>}[rrd]&& \\
%  &&  &&  &&   &&
}}
%
The double intentionality effects here, as elsewhere, the covering of the
represented stream ($A_0-A_3$) by the representing one ($A_0-A_3'$). But it does
not effect any unification with the stream $B_0-B_3$, i.e., with the stream of
the time of the original past event.  This stream is now only
represented/reflected/intended by/by the way of/behind $A_0-A_3$, and it remains
beyond the reach of the \co{actual} grasp, beyond any continuity and living
contact with the \hoa.}

The answer, however, need not be wrong, just because it does not fill all the
imaginable gaps. It is, probably, as good and specific as the phenomenological
method can allow; the unity can at most concern the \co{actually} given aspects,
here: the \co{actual} consciousness of the time of a recollected event and the
consciousness of the \co{actually} passing time.  Thus, even if intended with
respect to the totality of experience, the unity of the stream of consciousness
remains confined to the limits of the \hoa. And even when so confined, 
\citetib{this intention is unclear [impossible to fulfill, imperceptible], is an
  \thi{empty} intention, and its correlate is 
the objective time series of events, which is the dim surrounding of the actual
recollection.}{Zeit}{A:I.2.\para 25. \orig{...diese Intention 
  ist eine unanschauliche, eine \thi{leere} Intention, und ihr
  Gegenst\"{a}ndliches ist die objektive Zeitreihe von Ereignissen, und dies ist
  die dunkle Umgebung des aktuell Wiedererinnerten.}}
The \thi{empty intention} is the phenomenological way of taking into account
things which can not be taken into phenomenological account. Often, like here,
it refers to something which, 
although ingraspable and unverifiable in the \co{actuality} of any phenomenon,
appears nevertheless entirely evident. Here, we would be tempted to say:
something which is evident {\em exactly because} it for ever avoids any \co{actual}
determinations, any appearance as a mere phenomenon. The dim surrounding of the
actual recollection is, in fact, the dim surrounding of every \co{actuality}
into which disappear also retentions on their way toward the remote and dead past. 

\pa Founding the unity of time (experience) in the empty intention of the unity
of consciousness is, perhaps, the only phenomenological possibility. It carries
the germs of idealism of later Husserl which we certainly do not intend to
share. For the moment, let us sketch our view of the unity of the two aspects of
time -- the immediate and the remote (past) -- which phenomenologically have
turned out to have so little in common.\ftnt{Asking thus about the unity between
  phenomena so dramatically dissociated in the phenomenological description, we
  are not opposing its results. We are only filling in the \thi{empty
    intention}, we are here, as most places elsewhere, constructing and not
  reconstructing.}

What seems a bit disturbing in the figure from~\refp{fig:retention} is the origin, the
point $A$, the \ger{Urimpression}. Such impressions appear spontaneously, like
everything else in the stream of consciousness, but with the special role and
effect of marking a new \thi{now}. Husserl notices that, as a matter of fact,
even without any new \ger{Urimpression} one experiences the flow of time; even
to the point that the very lack of any new impression may become a new
\ger{Urimpression} marking a new \thi{now}. But \thi{now} has no beginning, no
particular point at which it becomes a new \thi{now}, as opposed to the (or
rather, {\em a}) previous \thi{now}. I notice a pricking which has become so
intense that I feel the difference between the moment now and a few minutes ago
when no such pricking was felt or, in fact, even present. But once it has become
irritating, I also realise that it has been there for a while before I noticed
it, as if interleaved with its absence, its irrelevancy. The \ger{Urimpression}
is here, so it seems, only the peek which marks a new quality, but which
radiates its gradual presence into the surrounding field of its increasing
absence. \thi{Now} does not begin, it is \herenow, especially when I
\co{reflectively} notice it, but it only arises constantly from the past, that
is, from the just past \thi{now}. It is, as Husserl always emphasized, a
continuous process. \citet{I can only define \thi{continuous} as that which is
  without breach, crack, or division}{PrincPsych}{I:9.3} -- but not without
\co{distinctions}. Continuity is like that of the waves; we can point
to one and to another but never to where, exactly, the one ends and the other
begins. Continuity means only that there are no sharp beginnings, for every
beginning, except for the \co{original} one, is but a continuation of what has
been before. \thi{Nows} are only points of marked intensity. Whether a hammer
which misses the nail (and hits the finger instead!), malfunctioning tools which
call for the attentive reflection, or else a beautiful view which makes us stop
and gaze -- the \co{attentively} registered \thi{nows} arise breaking the
continuity of the flow of \co{experience}.
%\begin{figure}[hbt]\refstepcounter{FIG}
\[\xymatrix@R=0.2cm@C=0.4cm{
\ar@{.}[rr]&\ar@{-}[rrrrrrrrr] &  && && && && \ar@{.>}[rr]&& \\ 
  &&  &&  && && &&\ar@{-}[ul] && \\
  &\circ\ar@{.}[ul]\ar@{-}[uur]&  &\bullet\ar@{-}[ull]\ar@{-}[uurr]&
\bullet\ar@{-}[uull]\ar@{-}[uur] && \bullet\ar@{-}[uulll]\ar@{-}[uurrr]&
\bullet\ar@{-}[uul]\ar@{-}[uurr]& &\bullet
\ar@{-}[uull]\ar@{-}[ur]\ar@{.}[urur]& &\circ\ar@{.}[uull]& \\   
}\]\label{fig:nows}  
%\caption{The flow of now's}
%\end{figure}
James suggests: \citetib{Let us call the resting-places [the $\bullet$'s]
  \wo{substantive parts}, and the places of flight [between them] the
  \wo{transitive parts}, of the stream of thought. It then appears that the main
  end of our thinking is at all times the attainment of some other substantive
  part than the one from which we have just been dislodged. And we may say that
  the main use of the transitive parts is to lead us from one substantive
  conclusion to another.}{PrincPsych}{\noo{I:9.3}[The color-phi phenomenon gives a good example. If two small spots in a
  close visual distance are briefly lit in rapid succession, a single spot seems
  to move. (This phi-phenomenon was originally studied in
  \citeauthor*{ColorPhiWert,ColorPhi}.)  This is, of course, what makes the
  movies move.  If now the two illuminated spots are different in color, the
  spot seems begin moving and then change the color abruptly {\em in the middle}
  of its illusory passage towards the second location.
  (\citeauthor*{ColorPhiKolers,DennTime}.)  The effect depends, of course, on
  timing and many gradations of the conscious reactions are possible. Describing
  the phenomenon in terms of objective time and idealised now-points, one would
  tend to impute the subjects projecting the resulting \thi{change of color}
  back in time. In this language one would say that later \thi{now} can modify
  the immediately prior retentional image(s). But it seems much more
  satisfactory to dispense with any ideal now-points. As argued in
  \citeauthor*{DennTime}, the change of color and its location is {\em really}
  perceived, just like movement is in the cinema. The temporal separation of the
  two events, if at all possible, happens far below the threshold of
  consciousness, at the time scale of cellular brain reactions but not of the
  consciously identifiable \thi{nows}. As the authors suggest, the time
  separating the two may happen to be too short even for the brain to bother to
  notice any difference -- the phenomenon it constructs, which is so consciously
  perceived, is that of a moving point changing its color. A \thi{now}, as in the
  figure above, can thus span several \thi{objectively} distinguishable events
  and points. In such \thi{objective} terms, the \co{dissociation} of \thi{nows}
  from \co{experience} can -- probably, must -- be considered, as in the whole
  tradition of 
  empiricism, to be {\em really} an association gathering the impressions
  dispersed across the impossibly minute time-points into unified and lasting
  (if only briefly) wholes.]}

Indeed, such breakpoints, such \thi{substantive parts}, mark only the
particularly intense and \co{reflectively attended nows}.  The \co{rest}, on the
last drawing %\ref{fig:nows}
the lines leading to these points, are also part of
the \co{experience}. We can even think of registered moments which do {\em not}
constitute any \thi{now}.
%\begin{figure}[hbt]\refstepcounter{FIG}
\[\xymatrix@R=0.05cm@C=0.7cm{
\ar@{.}[r]&\ar@{-}[rrrrrr] && && && \ar@{.>}[r]& \\
 & \ar@{-}[rrrddd] &  \ar@{-}[rrdddd] & & & & & \\
 & & & & & & \ar@{-}[rdd] & \\
 &\ar@{-}[rruu] & & ^x & & & & \\
% & & & & \drop{\circ}\ar@{-}[ruuu] & & & \\
 & & & & {\circ}\ar@{-}[rruuu] & & & \\ 
 & & & & \bullet\ar@{-}[rrruuu] &&&
}\]\label{fig:disNow}  
% \caption{An overtaken now}
% \end{figure}
Sitting quietly and strolling with my eyes around the room I am beginning to
anticipate that in a moment the quietude will turn into boredom which will annoy
me. I am on the way towards an \co{actual} culmination, a point $\circ$ where
boredom would become marked and registered. However, before that happens, at
the point $x$, the spider I have just caught in the edge of my eye enters
the horizon (attracts my eye), so that the anticipated moment $\circ$ does not
occur. The expectation is overtaken by the new {leitmotif} and $\circ$ --
which might have marked a new \thi{now} -- is suppressed underneath the quality
of the actually emerged \thi{now}, the dance of the spider in its web, the
registered $\bullet$. One {leitmotif} 
merges with and, gradually, replaces another, but even this continuity is only
apparent for it is already cut with various \co{distinctions}.

It is, however, continuity for the level of \co{reflective dissociations}.  
The \thi{substantive parts} only break this (apparent) continuity further and more
definitely, yielding the (apparently) discrete $\bullet$ after $\bullet$. But
they do not completely veil the underlying matter of 
\co{experience}; they are not any unfortunate accidents, any falsifications of
the flow of \thi{true temporality}. They are \co{aspects} of the new level of
\co{experience}, namely, of the \co{reflected experience}. Splitting of
\co{experience} into multiplicity of \co{experiences}, splitting of
\fre{dur\'{e}e} of \co{temporality} into a succession of \thi{nows} is a
necessary element, an \co{aspect} of the emergence of \co{reflection}.
\co{Reflection} is the exact opposite of continuity, if one likes, it is the attempt
to stop the flow, by extracting from it \thi{substantial parts}. As such,
\co{reflective dissociation} is also what turns the flow into a succession and
marks the new level of 
experience at which \citet{[t]he mark of the mind is that there do not arise
  more acts of knowledge than one at a time.}{NayaSutra}{I:1.16\kilde{p.360}}
The \thi{one act of knowledge}, the \co{object} of a \co{representation} -- in
its \co{dissociated} and lonely unity -- is the \co{aspect} constituting
\thi{the scope} of the \thi{now}.  Thinking two things involves, analytically, a
succession, an \co{after}.\ftnt{So much about the time as the dimension of
  inner experience (if only we take the latter phrase a bit seriously). Time
  thought objectively as a line is the dimension of \co{representation} and only
  equating \co{representation} with inner experience could suggest the idea that
  such a linear time has so much to do with it.} This does not exclude the
possibility of the \co{actuality} of several things. Just as one can see several
things simultaneously, one can also have them before the mind's eye, one can think
them. Indeed, only that then the several things turn into a {\em collection of}
several things, which is the \co{object}, albeit a bit \co{complex}, of one's 
\co{actual} attention.\ftnt{As observed in footnote~\ref{ftnt:NowDuration}, it
  would be tempting to assign to the \hoa\ (and here to a \thi{now}) some
  objective time duration. But it helps little to measure brain processes and
  subliminal reactions, even if for most normal persons some average limits
  might be drawn. A person waking up after 3 years in coma learns that 3 years
  have passed but they are, in fact, only a single \thi{now}.}

\pa
The analyses of the celebrated stream of consciousness, supposingly flowing
uninterrupted in the all embracing unity of one flux, stop at the
limits of one \thi{now}. The limits which are, as we have just admitted,
impossible to draw 
\co{precisely}, the limits dissolving gradually in the \ger{Umgebung} and
disappearing in the
fringe which marks equally a transition to another \thi{now}. Yet, the limits of
phenomenological analyses are there and the analyses stop, shall we say, rather
disgracefully, at these limits. One may rightly
claim that \citet{[t]he transition between the thought of one object and the
thought of another is no more a break in the {\em thought} than a joint in a
bamboo is a break in the wood. It is a part of the {\em consciousness} as much
as the joint is a part of the {\em bamboo}.}{PrincPsych}{I:9.3} It certainly is but this
does not change the fact that it still is a joint. The unity of the flow across
such joints is certainly felt and experienced but it is of different character
than the unity of the flow discernible by a purely phenomenological analysis
within a single \thi{now}. This is the difference between retention and
recollection, between \ger{prim\"{a}rer und sekund\"{a}rer Erinnerung}. 

Recollection, fetching its content as if from a bottomless well of the \thi{pure
  past} (or, phenomenologically speaking, from nowhere -- for it is not a
phenomenological question whence the contents of consciousness might arise),
opens an unlimited horizon.  The process of positing earlier \thi{nows}, \wo{is
  obviously to be thought as unlimited, although the actual recollection fails
  in practice.}\noo{quoted earlier:{Zeit}{A:I.2.\para 32}} Thus, one would like
to continue the line $A-A_3$ from figure~\refp{fig:retention} not only
indefinitely into the future beyond $A_3$ but also into the past, to the left
and beyond $A$.\ftnt{Let us observe that \wo{unbegrenzt} may mean indefinitely
  as much as infinitely. In the former sense, at least, it can be easily viewed
  as an analogy of the indefinite extension of the experienced time into the
  past towards -- but never reaching -- one's \co{birth}.} This would dissolve
everything into a single line (or two parallel ones) and might greatly please
the pupils of Cusanus but does not seem quite satisfactory to us.  Instead, we
would draw the whole ({\em the whole!})  process as in figure below,
%~\ref{fig:onetime},
as a spiral emerging from the \co{origin}...
%\begin{figure}[hbt]\refstepcounter{FIG}
\begin{center}
\epsfxsize=6cm
\epsffile{spiral.eps} %UNIX
\end{center}\label{fig:onetime}  
% \caption{The unity of time experience}
% \end{figure}
The figure can be thought of as an enfolding of the original figure
from~\refp{fig:retention} 
with the point $A_0$ of \ger{Urimpression} collapsed to the origin of the
spiral.\ftnt{We ignore $r$ and $\phi$ -- they are included only for the sake of
the footnote~\ref{ftnt:timedistance}.}
An \co{actual} point is anywhere on the spiral, and the lines linking such points
to the \co{origin} correspond to the vertical lines $A_i'-A_i$ from
figure in~\refp{fig:retention}. The spiral traversed backwards, say, from $B$ past
$A$ towards the origin, corresponds to the line $A_3'-A_0$ of the collected
past. An \co{actual experience} comprises a small segment of the spiral, say
$A-B$. At $B$ the \ger{frische Erinnerung} of $A$ is still present.
% Both may be directed towards different \co{actual objects} though, eventually,
% they both address the \co{origin}.
As the \thi{now} of $B$ moves forward leaving $A$ behind, it loses gradually the
later from its view. At some point, the line connecting the \thi{now} (e.g., at
$C$ but in fact much earlier) with the past $A$ must cross the inner part of the
spiral. One could take this as representing the point when $A$ definitely left
the span of retentional presence -- from now on, it can only enter \thi{now} as a
reproductive recollection.\ftnt{One would like to allow for something more: the
  living memories arising, like with Proust, not as mere images but as
  revitalised moments. The current abstraction does not make such fine
  distinctions.} As we pass through more and more rotations of the spiral, the
earlier points become screened from the view by ...  the memories of the earlier
ones. The point where the line from $C$ crosses the dotted half-circle marked
$\phi$ is inaccessible for direct introspection from $C$, it can be reached only
through the memory of an earlier point -- the memory represented by the line
from $C$ crossing the earlier rotation of the spiral.  This happens with all
events but, in particular, after the first rotation the \co{origin} becomes
inaccessible getting gradually immersed under ever new and wider rotations. Thus
memories emerge in the same process as retentions, one might say, are \thi{long
  distance retentions}, but this very length of the distance makes also a
fundamental change of the nature.\ftnt{We will discuss memory in
  II:\ref{sub:Identity}.}

One will ask, of course, what happened to the \thi{objective} line $A_0-A_3$. It
seems that we have retained only the primed points of impressions. Almost. All
\thi{objective points} collapsed to the one point of \co{origin}, the only
\ger{Urimpression} $\bullet$.  This may certainly seem worrying, though our
development so far should have made it less so. One possibility would be to say
that {\em any} point circumscribed by and \thi{within} the spiral so far, any
point between the current \thi{now} and the origin, as well as the whole spiral
can be taken to represent possible objects. But we prefer to say yes, in a
sense, there is only \co{one}. All we ever do is to \co{distinguish} and thus,
even if only indirectly, address the \co{indistinct}. Yet every point on the
spiral is a distinct perspective from which the \co{one} is experienced and,
moreover, is involved into different, steadily accumulating past which modifies
or even screens earlier experiences.\ftnt{\label{ftnt:timedistance}My \co{birth}
  is not accessible to my memories, yet there is no particular point which marks
  the beginning of memories. One may legitimately insist on the possibility of
  indefinite extension of time experience into the past, where indefinite is not
  to be confused with infinite. As a simple model, we would first postulate some
  formula for a spiral, say, the Archimedes' spiral described (in the polar
  coordinates) by the equation $r=c\cdot\phi$, where $c>0$ is some constant,
  $\phi$ is the angle growing indefinitely (i.e., one rotation is $360^\circ$,
  while two $720^\circ$, etc.), and $r$ is the radius from the origin at the
  angle $\phi$. The distance, or a time step, between some earlier point
  $\phi_1$ and a later one $\phi_2$, is then relative to the angle and
  approaches infinity as $\phi$ approaches $0$.  E.g., letting this distance be
  $d=\frac{1}{\phi_1}-\frac{1}{\phi_2}$, would make any point $\phi_2>0^\circ$
  infinitely far away from the origin $\phi_1=0^\circ$.  Also, the older one
  gets, the smaller becomes the distance between, \thi{objectively speaking},
  equidistant events. If one day corresponds, \thi{objectively}, to the angle of
  $10^\circ$, than for one who is one day old it will correspond to the distance
  $\frac{1}{10}-\frac{1}{20}=\frac{1}{20}$, while for one who is $100^\circ$ old
  to $\frac{1}{100}-\frac{1}{110}=\frac{1}{1100}$. Although it may be true that
  days are quite long for children (yet always end too soon) and short for old
  people (yet often end too late), we do not want to misuse such images.}


%{Objective time with us...}
\pa\label{pa:objectiveTime}
Every single \co{object} is \co{distinguished} from and in the \co{indistinct}
and, as the limit of \co{distinctions}, is endowed with its (relative) identity
and relative time. 
This \co{dissociation} of independent \co{objects}
endowed with some residual and lasting identity is \equi\ with the experience of
the time of these \co{objects}, the time in which the \thi{now} of \co{reflective
  actuality} becomes confronted with its own past and future, as well as the
past and future of the \co{objects}. But even this is not yet the ultimately
objective time. That time -- the \thi{time of the world} -- appears through
further abstraction of \co{attentive reflection}.  First, one has to dissociate
the relation \co{after} from its context of \co{representation} and allow it to
connect arbitrary objects.  And second, one has to remove the designated
\co{actuality} of \thi{now}, the \thi{now} which is {\em my} \co{actuality}.
Purely objective time emerges as a next stage of differentiation, as a
consequence of \thi{abstracting oneself away}, of recovering the uniform time of
the world of \co{externalised} \co{objects} from the \co{unity} of the
\co{experience} of temporal \co{existence}.  This happens with \co{positing} the
\co{totality} of \co{objects} as the \co{actual object}, \thi{the world}.

\noo{This final
\co{externalisation}, this detachment of the \thi{world's time} from the 
\co{temporality}, will not occupy much place in what follows but I 
should, probably, at least sketch the process.

This is a process of further \co{reflection}, that is, further 
separation and eventually \co{dissociation}. Its fundamental correlate is the 
constitution of \co{objects} as \co{separate}, that is independent, 
entities. \co{Representation} of \co{objects} involves \co{reflection} 
into \co{after}, but this relation remains as if in the background, it 
is there but only implicitly -- the explicit focus is on the 
\co{objects}, primarily \co{objects} are the only things for 
\co{reflection}. It is thus easy to apply the relation of \co{after} 
to, now arbitrary, \co{objects}. 

There is here the problem 
with applying it to \co{objects} which appear simultaneously, which 
are perceived simultaneously, and not merely thought, and hence which cannot be 
linearly ordered in time.
}

Although \co{reflection} is determined by \co{representing} one \co{object} at a
time, it is, of course, conscious of other \co{objects} and, not least, of
perceptions -- it is involved into \co{experience}. Likewise, the one \co{actual
  object} may be a \co{complex} involving several \co{objects}.
%This problem disappears for \co{reflection} with introduction of the \thi{time
%of the world}.
Any simultaneity, and such a co-presence in particular, is \co{spatiality}.  It
\co{founds} also the image of \thi{the whole world}, namely, simultaneity of all
\co{objects} \co{posited} itself as an \co{object}. (Although such an
\co{object} is rather vague and ideal and, according to relativity theory, even
impossible, there is nothing impossible with \co{positing} is as an
\co{object}.) Combined with the idea of \co{actuality}, it yields something like
\thi{the totality of the whole world at this particular point of time} -- the
\thi{now} of the world.
The relation \co{after} applied now to this \co{object} -- the whole 
world -- leads to the time of the world. 

\pa
It is isolation of {\em one} \co{object} which leads to the total, linear order of
time. And it is the totalisation, \co{positing} everything under one \co{sign},
the 
\co{representation} of the ideal \thi{whole world} as one \co{object}, which
yields the one uniform and linear time of this world. 

\co{Temporality} has many pasts and many futures.  This is so because
it unfolds surrounded by \co{transcendence}, by the possibility (lived and
experienced) of something else, something more, something different. My
\co{temporality} is interwoven into the \co{temporality} of all things and other
people. But if something is considered as an independent \thi{whole}, as an
isolated \co{object}, that is, if we, so to speak, suspend the
\co{transcendence}, then there is nothing which can bring in the variation of
multiple futures. The future of an isolated \co{object} may still be
indeterminate but it will be unique.  There may be internal changes and states
of this \co{object}, but not a multiplicity of other, alternative \co{objects}
and their time paths.  Such an abstract \thi{now} -- actuality of an isolated
\co{object} -- has only one, unique \co{before} and only one, unique \co{after}:
these are just stages of the isolated \co{object} which, being one and alone,
can only be in one stage at a time. In case of the \thi{world time}, what is
\co{posited} as an independent \co{object} is the postulated, ideal
\co{totality} of \co{objects}. It then \wo{includes} the times of all the
objects it \wo{contains}, as particular intervals, projections of its own,
global, objective time.

\label{pa:mytime}
Linearisation of time would take place if we \co{dissociated} any single being 
from its relations with \co{transcendence}. In particular, in order to 
think \thi{my time} as linear, I do not have to think of myself as 
an \co{objective} being, a thing in the world -- this would never 
give \thi{{\em my} time}. On the contrary, I have to isolate myself, so to speak, 
relativise everything to myself. As we know, this is an abstraction -- 
I have multiple pasts, depending on the contexts in which and depth to which I consider 
myself and, in the same way, I have multiple futures, all of which are in
addition indeterminate.
%

As the final step, after \co{positing} the \co{totality} of \co{objects} as one
\thi{world} and endowing it with its own \thi{now}, one can perform the final
\co{dissociation}, that is, the ultimate abstraction with respect to time.  It
took quite some time before European thought arrived at the idea of empty
\thi{time in itself}, flowing independently from any things and
events.\ftnt{Although Zeno's arguments assimilated time to a geometrical line,
  it was still relational time of events, the \wo{numerical aspect of motion
    with respect to its successive parts}. (Besides, Eleatic Being was timeless
  anyway.) Nicolas Bonnet in the XIV-th century, Bernardino Telesio in the
  XVI-th, Francisco Su\'{a}rez, all involved still in one way or another into
  Aristotelian physics or cosmogony, postulated true mathematical time in one
  form or another. The immediate predecessors of Newton, proposing independent
  time not requiring motion or any objects, were Pierre Gassendi and Newton's
  tutor Isaac Barrow.}
It appears as the ultimate abstraction and, as it seems, even modern science
does not need it any more, and so we will not be occupied with it at all.

\pa%{brief summary...}
Summarising briefly: \co{dissociation} of an
\co{object} involves \co{reflection} into the relation of being \co{after} the
\co{object} and, as a matter of fact, \co{after} the whole \co{experience}. The
\co{reflective} project of \co{dissociating} things is thus the same as the
project of stretching across the \co{distance} of \co{after} which separates
\co{reflection} from its \co{object}. It is thus the project of \thi{freezing}
the \co{objects} in the \co{immediacy} of \co{reflective acts}; \thi{freezing}
which, because it never finally succeeds, makes the flow of time the more
transparent. Perhaps a bit paradoxically, the \co{foundation} of the experience
of time marks also, at the same time and by its very nature, the attempt to
erase time, the thirst for the ever escaping entities \thi{beyond time}. 

Objective time, the time of the whole world, arises as the ultimate abstraction
of this \co{reflective} process. As Bergson constantly repeated, this objective,
\thi{spatialised} time, is only an image of the genuine \co{temporality} of
\co{existence}. But we would not, for this reason, consider the one authentic
and the other not, the one \ger{eigentliche} and the other not, the one
legitimate and genuine while the other only a result of inauthentic mode of
existence or of tradition \citet{engulfing all [the] delicate idiosyncrasies in
  its monotonous sound.}{PrincPsych}{I:9.3} We would not consider the time of
the world as a mistaken redundancy falsifying the true temporality. We only
observe the difference in the matter of \co{experience}, the difference between
the lived \co{existential temporality} and the dead time of the \co{objective}
world. The latter is an \co{aspect} of recognition of \co{objects} and of
\co{reflective experience} -- the reason for diminishing its importance are as
many as for making it the only measure of absolute truth, that is, none.  In
fact, the identities of the \co{objects} and the \co{posited} \co{objects}, like
the \co{totality} of the \thi{world}, contribute significantly to the
\co{reflective experience}.  Establishment of the objective time (and objective
world; not only a single \co{object}) is what extends the horizon of our
\co{experience} beyond the mere lived \co{actuality}, beyond the mere horizon of
retentions and protentions, beyond the unity of a single \co{act} which reaches
its end in the same moment in which it leaves its origin.  Relations to the
world and life are not exhausted by the contents of \co{immediate experiences},
by the merely \co{actually} given, and the \co{actually} remembered and
expected. Restricting them to such \co{actualities} amounts to a reduction,
perhaps, to the level of animal experience of time which, true and genuine as it
certainly is, does not probably reach the long term memories and abstract
recollections of forgotten past. This reduction, like every other (in
particular, also the reduction to mere objectivity), is an impoverishment of
life.  The objective time and world are the reminders, the \co{traces} of the
\co{original unity}, and then also of the \co{unity} of \co{existential
  confrontation}, retained in the midst of \co{reflective dissociations}.



\noo{
\subsubnonr{Existentialist time}
\say Husserl's anschauliche Erwartung assumes an object, ...

\pa
Although the experience of \co{after} is 
really different from the experience of \co{before}, although looking 
into the past is very different from looking into the future, so, 
nevertheless, the two supposed \wo{dimensions} of time are essentially one and 
the same thing, namely its directedness. There is, of course, one 
important difference. Past, as emerging from the \co{after} of a 
\co{reflected} \co{experience} is a past of some specific content. Future, 
as the inverse of the past, as the \co{after} of \co{actuality}, is 
not. This undetermined character of the future may, indeed, lead one to 
think of it as the primordial dimension of time. 

\pa
We should, however, 
be careful with the order of \co{founding}. The claim is that the elements 
proposed by (some) existentialists as the founding experiences of the 
future -- feelings or acts of hoping, expecting, projecting, etc. -- in 
fact presuppose the dimension of future rather than found it. To 
engage in a project with some goal, one is already involved into the 
experience of time, of the \co{object} not-being-there-{\em yet}. 
This \thi{not-yet} is not a mysterious quality of \ger{Lebenswelt} -- 
it is the \co{temporal} dimension without which no \co{object} would appear. 

The experiences of hoping, expecting, etc. may precede explicit 
thinking about time. But this does not mean that they precede time 
itself, not to say that they found it.
There are no such things as hoping, expecting, 
projecting. There is only hoping {\em for something}, expecting {\em 
something}, projecting {\em something}.\ftnt{This, certainly, 
applies to Heidegger's \thi{projects} which, after all, originate from 
the phenomenological perspective.} These somethings are then 
always, no matter how unclearly, \co{represented}, that is, always appear 
as \co{objects}. All these hopes and expectations find place in 
already differentiated and highly organised world, in a world which, 
because it contains \co{objects}, is also involved in time. 
}

\subsub{Space}

\pa\label{pa:complementaryDuality} And now, what about space? As mentioned in
\refp{pa:spatialityBeforeTemporality}, the element of \co{spatiality} -- as
simultaneity -- emerged already in \co{chaos}.  \co{Temporality} is like
stretching out this simultaneity along the dimension \co{before}-\co{after}.
But the emergence of \co{temporality} amounts to a sharper \co{distinction} of
\co{spatiality}, too. In fact, only isolating from the \herenow\ the element of
\co{after} allows the element of simultaneity (that is, neither \co{after} nor
\co{before}) to be isolated as well.  The latter remains as the residual rest,
as the simultaneity which remains from the \co{virtual spatio-temporality} of
\herenow, after things started to enter also the \co{temporal} dimension.
\co{Spatiality} amounts then to \co{distinguishing} the \co{here} -- from
\herenow -- as the place distinct from other, but simultaneous places, just like
\thi{now} has been \co{distinguished} -- from \herenow -- as the place (one
would, probably, prefer to say the \wo{point of time}) distinct from the places
which come \co{before} and \co{after} it.

There is thus a complementary duality: things have the \co{spatial} aspect to
the extent they are seen as simultaneous, or else the \co{temporal} aspect to
the extent they appear \co{after} each other.  The final dissociation of
\co{spatiality} from \co{temporality} happens when this complementary duality
gets \co{distinguished} into exclusive \thi{either ... or ...}, when we begin to
conceive things separately either as simultaneous or as ordered along the
\co{before}-\co{after}.

\pa
Analogous process to the one from \refp{pa:objectiveTime} 
leads to objective space. First, \co{spatiality} must become the 
spatiality of (arbitrary) \co{objects}, and then lose its designated 
\co{here}. The first thing happens naturally, since \co{reflection} 
sees the world as a collection of \co{objects}. In principle, any of 
them might be given simultaneously. 

One has thus to conceive an abstract \co{here} 
of \thi{the whole world}, an abstract simultaneity of all 
\co{objects}. Such a \co{here} does not any longer stand in relation 
to others, it becomes an abstract, that is, \co{dissociated} and isolated 
\thi{here}. (Yet, the 
questions creating the first antinomy, like ``What is outside the 
space?'', are most naturally asked, indicating precisely that 
the \thi{objective} space arises from a \thi{subjective}, i.e., limited and situated 
place, \co{here}.) This lack of \thi{outside}, of any transcendence 
is, in fact, just the opposite side of the idea of its emptiness. 
Consequently, its spatiality is exhausted by the 
spatiality \wo{within it}, the spatiality of the \co{objects} which have only 
been abstractly gathered in the totality of world's \thi{here}. 

%\subpa
The \co{spatiality} centered around an \co{actual} \co{here} has infinitely many
dimensions: any \co{object} marks a possible dimension (if you prefer, a
direction for a course of action).  Things like below, above, in front, behind,
etc., are already further abstractions.  The celebrated three dimensions of
space are but a further, highly convenient abstraction. But the fact that
localisation in objective space can be \co{represented} by a choice of a
reference point and three coordinates seems a very bad reason to postulate them
as the original truth of ontology, epistemology, perception, apperception or
whatever.  They are just that: a convenient \co{representation}.  In and by
themselves, they do not follow from the original character of \co{spatiality} or
even of the objective space. As we well know, completely different systems of
coordinates may be used which might be much more appropriate for animals with
different sensuous mechanisms.\ftnt{A vision system, like that of cattle,
  allowing one to see (almost) $360^{\circ}$ might naturally lead to the use of polar
  coordinates with the additional indication of hight and distance from the
  reference point.\noo{ A system which allowed one to see in all directions at
    once, would do well with one spherical coordinate and the distance.}}

\pa We have thus removed the designated \co{here} and established objective
space. But, one may wonder, where is the extensionality?  This, after all, is
taken to be a constitutive aspect of spatiality; \co{objects} in space are
exactly the ones which have extension. The answer is: extension is precisely
what we have termed simultaneity, once the space has been \co{dissociated} and
then re-filled with the \co{objects}. It is not something that {\em explains}
possibility of co-existence, of simultaneous presence of distinct \co{objects}
-- it {\em is} this very simultaneity. An extended \co{object} is one which has
some (more or less sharp) boundaries separating it from the surrounding.  These
boundaries are, in fact, {cut} from the \co{object} itself,
\co{distinctions} made within (or around) the \co{distinguished object} itself.
The extension of an \co{object} is the very simultaneity of its boundaries
(left, right, lower, upper, etc.) Distance is just another way of saying
extensionality. It only depends on where we draw boundaries, how we make the
{cuts}. We want to focus on an independent \co{object} -- the simultaneity of
its aspects is called ``extensionality''; the simultaneity of different
\co{objects} is called the ``distance'' between them.

If we imagine the boundaries of an \co{object} collapse, we obtain a 
point. A point has no extension. Is it in space? Yes, but only if we 
imagine it there, that is, only if we imagine it co-existing with 
other points (or system of coordinates, or its surrounding, or any 
other things posited as co-existing along with it). It may be a bit 
too advanced a gymnastics of imagination to try to think a single 
point, but it is possible. Such a point is then not in space, 
it has no spatial aura -- precisely because it is thought in complete 
isolation, without any simultaneous counterparts.

\pa
Finally, there remains the idea of homogeneity which is the same as 
infinite divisibility. It applies equally to space and to 
time.\ftnt{Bergson attributed it exclusively to space and was 
talking about \wo{spatialised time}, a degenerated duration, in order 
to account for this. It should be clear that, although our development 
is very intimately related to his, space and time are for us 
equiprimordial and develop in parallel from the virtualities of 
pre-temporal simultaneity and spatio-temporality.}
Homogeneity results from the two steps of the constitution of objective 
time and space: applying the respective relations (\co{after} and 
simultaneity) to arbitrary 
\co{objects} and then removing the \co{actual} \thi{now}, 
respectively, \co{here}. 

The first \thi{fills} the whole (time or space) with homogenous 
\co{distinctions}, which, although in themselves highly unlike and 
heterogenous, by the fact of having been viewed as mere 
\co{objects} acquired also the homogenous character of isolated, 
independent \co{actualities}. In the extreme, most abstract sense, an 
\co{object} is a mere indication of \thi{independence}, of an isolated, 
substantial entity, of a mere fact of its being, in short, a point. 
(This abstraction of a point, however, like the other abstractions we are
addressing at the moment, is not something which requires a conscious 
effort. It is given along with \co{pure distinction}. Conscious effort 
is needed only to bring it to \co{actual} consciousness, to establish it as an
explicit \co{representation}.) 

The second step removes the designated point of reference thus 
effecting a true uniformity, \thi{equivalence} of all points spread 
along the time line, respectively, in space.


The idea of infinite divisibility emerges now quite naturally. On the 
one hand, there is the \co{experience} of divisibility, the potential 
of making always new \co{distinctions}. This, however, does not in 
itself account for infinite divisibility of objective time and space. 
At every stage, one has made only such and such, so and so many 
\co{distinctions}, and one lives through these -- not through the 
possibility of making more. The lived process is a process of 
\co{distinguishing} but not of infinite \co{distinguishability}. 
With infinity of objective time and space we are by far in the realm of ideality. 
Their very foundations -- the {objectified} \co{totality} of the world 
of \co{objects}, its 
postulated \thi{here} and \thi{now}, the homogenous points filling 
them -- all these are \co{posited} abstractions, that is, not 
\co{representations} of lived \co{experiences} but their ideal limits. 
Infinite divisibility is just the equally ideal limit 
of \co{distinguishing}, posited for the homogenous totalities of 
objective time and space.

\noo{\pa
We have equated simultaneity with \co{spatiality} and this may provoke 
various natural questions. 
What about simultaneous \co{representations}? I can think two 
things simultaneously, can't I? I can't. \citt{The mark of mind is 
that there arise only one perception at a time}{check: Indian 
Philosophy} The \co{object} of a \co{representation} in 
its unity is the correlate, the aspect constituting \thi{the scope} of 
the \hoa. Thinking two things involves, analytically, a succession, 
an \co{after}.\ftnt{So much about the time as the dimension of 
inner experience. It has close to nothing to do with inner 
experience, if only we take the latter phrase a bit seriously. Time 
is the dimension of \co{representation} and only equating the latter 
with inner experience, one could come up with the idea that time has 
so much to do with it.} This does not exclude the possibility of the
\co{actuality} of several things. Just as I can see several things
simultaneously, I can also have them before my mind's eye. Indeed, only that then the
several things turn into a {\em collection of} several things, which is the
\co{object}, albeit a bit \co{complex}, of my \co{actual} attention.  

OK, but then what about the location of the
\co{representations}. If the tree is there, {\em where} is my perception of 
it? Such questions spring to mind because, first, we turn 
\co{representation} into an \co{object} and, second, we think that it 
(this \co{representation} as \co{object}) is simultaneous with the \co{object} 
of the original \co{representation}. As we saw, \co{representation} 
is the very point where \co{after} enters our \co{experience}, it is 
the very basis of temporality and hence, by the complementary duality 
from \refp{pa:complementaryDuality}, it never enters the \co{spatial} 
relation of simultaneity. The question, as has been often pointed out, 
has no meaning.\ftnt{Again, let those interested in such 
questions investigate whether representations arise in this or that 
part of the nervous system, in this or that part of the brain or else whether
representation of this in this part and of that in that. As with most
reductionistic projects, while 
curiosity and eagerness overshadow the relevance, the
promises keep eclipsing the results.}

One may try further: what about two \co{representations} occurring 
simultaneously for me and for you? But are they really simultaneous? 
In my or in your \co{spatiality}? In my or in your \co{temporality}? 
Well, of course, they may be simultaneous in the objective time. This, 
then, is the answer. Having made \co{representations} into 
\co{objects}, and having drawn them into the objective time and space, 
you may only thank yourself for the insoluble questions you may run into. 
Although common sense may have tendencies in this direction, one should 
be careful with taking the common sense as the source of only 
legitimate questions.

}

\subsub{Objective or constituted?}\label{sub:ObjConst}

\pa\label{pa:objecttemp} It is identity, solidified and sedimented in
\co{objects}, that transforms the original \co{temporality} into time and
\co{spatiality} into space.  But we should state clearly: time and space are not
the conditions of possibility of the \co{objects}, nor other way around.  They
are \equi.  There are no \co{objects} without space and time.  But neither could
we arrive at time and space, if we didn't also reach the
 \co{representation} of \co{objects}.  Instead of conditions of possibility we
rather speak about the order of \co{founding}, and there it is the continuity of
\pexp, timeless as it is, which precedes both spatio-temporality of \hoa\ and
\co{recognitions}, and which, in turn, precede space, time and \co{objects}.

To be sure: we are not doing here the impossible, we are not constructing
\co{objective} time nor space -- only a \co{representation} of objective time or
space.  More precisely, we are constructing a \co{representation} of
\co{spatiality}, that is, of simultaneity of different objects and of their
\co{temporality}.  \co{Spatiality} and \co{temporality} are still \co{aspects}
of un\co{dissociated experience} and thus can be concretely experienced in the
simultaneity and flow of \co{distinctions}. They can not, however, be reduced to
any concept. When we attempt to \co{represent} them, we arrive at the objective
time and space which, in terms of \co{experience}, are indeed only empty
concepts of empty containers.  These, conversely, can not be \co{experienced}
but only constructed, these {\em are} constructions.  Flow and
simultaneity are 
parts of \co{experience}; successive ordering of world's (or any \co{object}'s)
stages and the simultaneity of the \co{totality} of all \co{objects} are
conceptual constructions of \co{reflective} thinking.

\pa So, after all, we have not obtained any objective time or space but merely
\thi{subjective} \co{representations}?  For, do we not reduce the {objective}
time to its phenomenal constitution, that is, do we no strip it of its
\thi{objectivity}?  Does not, after all, the whole process of
\co{distinguishing} and gradual emergence of time and space happen already
within time and space, within \thi{objective} time and space?

Well, we certainly want to emphasize that the time as we experience and
understand it is relative to our \ldots experience and understanding.
Constructions need not be false or unreal because they are constructed -- but
they {\em are} only to the extent they are constructed. The shortest meaningful
unit of time is relative to the minuteness of objects which we are able to
distinguish and relate.  It is conceivable that a consciousness \citet{could
  live so slow and lazy a life as to take in the whole path of a heavenly body
  in a single perception, just as we do when we perceive the successive
  positions of a shooting star as one line of fire.}{BergTime}{ III;p.195} The
world, and the time of such a consciousness would be expressed in very different
way than ours.  A being living for only a fraction of a second, whose whole life
consisted of a single event, say a division in two beings, might have an
extremely poor experience of time.

But one would say that the differences here concern only different time-scale,
not the time itself. All these beings can be considered as living in the same,
\thi{objective} time. Indeed, they can but to the extent they are so considered
they are placed within not so much my or your \co{experience} as in, well,
\thi{objective} time. And every \co{object}, with the most abstract and
\co{posited} \thi{objectivity} included, assumes and requires an \co{existence}
which \co{distinguishes} it. And so is it with time. \citet{When I say that the
  day before yesterday 
  the glacier produced the water which is passing at this moment, I am tacitly
  assuming the existence of a witness tied to a certain spot in the world, and I
  am comparing his successive views: he was there when the snows melted and
  followed the water down [...] The \thi{events} are shapes cut out by a finite
  observer from the spatio-temporal totality of the objective
  world,}{PontyPerc}{III:2\kilde{p.411}} eventually, from the \co{unity} of the
\co{indistinct}. \wo{Time presupposes a view of time.}

\pa But is not this last claim an (intended and idealistic, if not merely
unfortunate) inversion of the famous phrase, according to which exactly the
opposite is the case, namely that \citet{perception of succession presupposes
  succession of perceptions}{Zeit}{B:II.20}?  This observation could be twisted
into a claim about some \thi{objective} time, as if a view of time presupposed
time.  But this would be a misinterpretation. The phrase (like the whole
paragraph, and the whole book) concerns the unity of {\em consciousness} of
time, with its double intentionality in which reproduction of a past event in
the present \thi{now} \noo{(perception of succession)} is itself involved into
the flow of time involving this very \thi{now}, \noo{(and thus assuming the
  succession of perceptions)} \refpto{pa:twoTimes}{pa:unityTime}. The
\thi{succession of perceptions}, which might be misunderstood as meaning some
\thi{objective succession}, refers only to the transcendental level which
constitutes the actual consciousness of succession.

% -- first (Husserl) pointing to the unity of the subject (double intentionality:
%    consc.~=~self-consc.) 
% -- then (Riceour) to the \thi{objective} time: discovered, not constituted

We have suggested that reducing everything to the mere \co{actuality} of
\co{reflective acts} seems unsatisfactory to us and that we would rather avoid
the danger of ending in idealism (which is equally imminent in phenomenology as
in empiricistic nominalism) with its subjectivistic flavour. But we do recognise
relativity of all \co{distinctions} to the \co{existence}. Where is the
difference? Only in \co{one}. We do not constitute anything, we
\co{distinguish}, which in the last instance means: discover. True, what we
discover is only our view and perception of the world, our ways of
\co{distinguishing} the \co{indistinct}, but this is also what for ever keeps
the hammer of some indefinable \thi{objectivity} over all sorts of
subjectivistic reductions.  We did not constitute {objective} time -- only its
\co{representation}. And this \co{representation} is \thi{true} because it
actually constitutes its own object, because objective time is nothing more than
objectified temporal \co{experience}, than succession viewed in abstraction from
the \co{experiencing existence}, which eventually leads to succession without
anything successive. Having once arrived at this objectivisation,
it is impossible to turn around and pretend that it is not there. Any
\co{distinction}, once made, remains forever -- it is what lies in its nature of
making a difference. Time is there, as a necessary \co{aspect} of the experience
of \thi{objective world}. It is an \co{aspect} of the conscious \co{actuality}
which, emerging \co{after} its \co{dissociated objects}, discovers in this very
\co{act} both its temporal relation to these \co{objects} and their temporal
character.

The fact that an experience is relative to the experiencing being does not in
any way diminish its \thi{objectivity}, here, the \thi{objectivity} of time.
Every \co{distinction} is relative to the \co{distinguishing} being, but it is a
\co{distinction} {\em in} the homogeneity of the \co{indistinct}, drawn {\em
through} or {\em from} the heterogenity of the background \co{chaos}.  As such,
a \co{distinction} made by you is as \thi{objective} as a \co{distinction} made
by an ant.  The human experience of time is as \thi{objective} as the experience
of an ant, even though the latter probably does not go as far as experiencing
the {objectivity} of time.  But objective experience does not require \co{an
experience} {\em of} this very objectivity. Experience {\em of} the
{objectivity} of time requires a \co{reflective dissociation} of the
\co{experience} into \co{external objects}, and ants probably do not reach this
level.  Yet their \co{experience} involves \co{distinctions} and time which are
equally \thi{objective} as ours.

Experience of {objective} time, as of \co{objectivity} in general, arises
through \co{externalisation}, through gradual abstraction from the relativity
to the experiencing being. The ultimate objectivisation would thus abolish
all \co{distinctions} and to prevent such a collapse into \co{indistinct}, some
elements must remain recognisable. Objectified succession and co-existence, 
time and space emerge as objective since they are not relative to
any {\em particular} human being or existence. But speaking about time or space
without any \co{existence} and its ability to differentiate the \co{indistinct},
is to project \co{distinctions} into the \co{indistinct}, is to forget
differentiating \co{existence} in
the very moment of making the claim of its irrelevance. 
Experience of {objective} time not so much presupposes \thi{objective} time as
reveals it, brings it forth, just like any \co{distinction} brings forth
whatever it \co{distinguishes}. And it is \co{founded} in the ultimate
\co{unity} of \co{existence} which precedes both the \co{temporality} of
\co{experience} and the experience of time. The experience of time, like every
other experience, is {\em both} a discovery {\em and} a creation. Neither
  is possible without the other; every \citet{[a]pprehension is not only a
    reflection but also a creative transformation.}{Bier}{II:1\kilde{p.29}}
This experience involves much more than mere registration of the `objective passage of
time' -- there are modes, as well as levels, of \co{experience} which do not
involve {objective} time and which, so to speak, suspend the validity of its
flow.  Thus, even if the whole setting can remind about Kantian forms of
intuition, the analogy is restricted to the level of \co{reflective experiences}
(which, discovering \thi{objective} time is already involved into \co{temporal
  experience}).
%
% Certainly, we need the \co{distinguishing} being, in fact, a being
% \co{distinguishing} very much as we do, to get the structure of experience and
% the world, with the {objective} time and space, the way we find them.  But my
% only stipulation concerning the \thi{mind}, or rather the \thi{subject}, is its
% ability to \co{distinguish} -- this is the only fundamental \la{a priori}.
% Sure, from the level of \co{reflective} understanding, everything \co{above} it,
% space and time included, appear equally \la{a priori}, but it is not so with
% respect to the whole of our being.
%
In fact, the \la{a priori} of our \co{existence} reaches deeper than the actual
flow of time to the mere fact of \co{distinguishing} and, eventually, of the
\co{confrontation} with the \co{indistinct}.  We \thi{discover} {objective} time
but, of course, this \thi{discovery} is made possible by the structure of our
being which brings the \co{original nothingness} and \co{chaos} to the level of
\co{reflective dissociation}.  It is also this fact, that we discover and not
merely constitute, which accounts for the natural and obvious interweaving of our
\co{experience} of time with the objective time. These are not \co{experienced}
as two different times -- simply, because they are not two. On the contrary, the
\co{temporal experience}, when arriving at the experience of objective time,
finds itself already not only in the prior \co{temporality} but also \thi{in}
this, just discovered, objective time. Our \thi{constitution} of time is not
transcendental in which case one is immediately \citet{referred back to the
  crucial problem, that of time of transcendental constitution. According to
  which time does it take place? Is it a time itself constituted by an atemporal
  subject? Is the subject itself temporal?}{DerGen}{I:2\kilde{p.20}} Our order
of \co{founding}, once the objective time has been discovered, is seen to have
evolved in this objective time, because what has been constituted is not this
time but only its \co{reflective experience}.

\noo{ is only because I discover -- and not constitute -- that the time (as
  everything else) I experience, melts immediately with the objective one; the
  time of my life is, in fact, the time which flew before my birth and will flow
  after my death. Somebody must be there to experience it for this to be/appear,
  but there is no objective time beyond somebody's thought/experience of time...
}

\noo{But is not objective time and world something much more than that? Is not
  objective world something which exists exactly in total independence from any
  existence and its experience? Is not objective time something which flows
  exactly in total independence from any existence and its experience?  Here, we
  will comment only on the later question which will also suggest the treatment
  of the more general issue raised by the former to be addressed at the end of
  this Book.
  
  -- We can only say if one set, one kind of distinctions is in conformance with
  another.
  
  This does not mean any ultimate subjectivity. For we start with the \co{one},
  and all distinctions being made in it. Every distinction reveals something of
  the \co{one}. Thus, we did not constituted \thi{objective} time -- we have
  discovered, we have seen how it is \co{reflected} in our \co{experience}.
  
  -- Why things are experienced in time, not simultaneously? This belongs to the
  very nature of being (an object of) an experience [Bergson]
  
  How is it possible to experience passage of time: a as before b, in one single
  moment/act? [Husserl] This shows reduction -- to the mere actually given. He
  says: a is experienced as-past, while b as-present, i.e., the Auffassung of
  the object itself changes, and temporal determination belongs to this
  Auffassung. Sure, but this is not an answer to the question, it is only its
  explication (phenomenologically, perhaps, all that is needed, but also all
  that is at all, phenomenologically, possible).
} %end \noo{But is not...


\section{Reflection and Experience}
%\subsection{Words}
Words, the paradigmatic \co{signs as signs}, the \co{signs} of \co{reflective
  dissociation}, make something transcending \co{actuality} present.
Nevertheless, the constitutive feature of \co{reflection} is \co{dissociation}
of its \co{actual object}, \co{positing} it in its isolated independence from
the non-\co{actual} surrounding, from the wider context of \co{experience},
eventually, from all the \co{non-actual rest}.  Thus \co{reflection}, nourishing
itself on the \co{experience} and, in particular, its \co{non-actual aspects},
performs its function in an apparent opposition and, in the extreme cases,
perfects its function in a direct opposition to it.  The present section is
devoted to this tension and to suggesting some of its possible consequences
which we will try to avoid later on.

\subsection{Actual and non-actual}\label{sub:actnonPower}
%\subsubnonr{The creative power}
\noo{
??? (or not???), ad St. John: Hebrew has no word for \thi{thing}?
\begin{tabular}{l@{\ :\ }l@{\ or\ }l}
dabar & speak & something \\ \hline
amar -- the root & (he) said \\
imrea & word, phrase, sentence & something said \\ \hline
mila & (one) word & --
\end{tabular}

{See Levi-Strauss ``Spojrzenie z oddal'' VIII, IX, X.}
}

\pa\label{pa:wordsPowerA} In the Hebrew language (of Old Testament) one did not
distinguish clearly between word and thing. From the primitive root \wo{{amar}}
(\gre{rma}), meaning \thi{to speak} or \thi{to say}, there derives the word
\wo{{`imrah}} (\gre{hrma}), meaning \thi{word}, \thi{speech} and, in particular,
\thi{word of God} as a command and what is commanded.
%\thi{phrase} as well as \thi{that which is said};
From the primitive root \wo{{dabar}} (\gre{rbd}), meaning \thi{to speak} but
also \thi{to converse}, \thi{command}, \thi{promise} or \thi{warn}, there
derives the noun \wo{dabar} (\gre{rbd}) which means \thi{word}, \thi{speaking}
as well as \thi{something} (spoken of), \thi{thing}, \thi{act}.\ftnt{There are
  numerous examples -- like that in Gen. XV:1, \wo{After these {\em things} the
    {\em word} of the Lord...}, Gen.~XXVII:42, \wo{And these {\em words} of
    Esau...}, Gen.~XXX:31, \wo{And Jacob said, Thou shalt not give me any {\em
      thing}: if you wilt do this {\em thing} for me...} -- where both
  \wo{thing} and \wo{word} translate \wo{dabar}.}  The creative power of the
Word which was in the beginning need not be taken so
literally.\ftnt{Especially, considering that \gre{logos}\noo{$\lambda o\gamma
    o\sigma$} of St.~John seems to carry enough of the influences from Philo to
  be taken the way the tradition has taken it, that is, in a much more Greek
  sense of, say, providential reason, soul of the universe.}  Nevertheless,
words do \thi{create}, and it is creativity of \co{reflection}. They
\thi{create} by fixing in an \co{actual} -- and that means, in particular,
graspable and repeatable -- form of a \co{sign} the flux of \co{experience} and
of the \co{experienced}.  Word has the power of \thi{freezing} something which,
if unsaid, might pass almost unnoticed.  As long as I am engaged in an
undisturbed (though not necessarily uneventful and indifferent) course of
\co{experience} without talking about it, I am actually engaged in a flux where
things, although identified and recognised, do not stand out sharply from the
background.  To \co{experience} is to \co{participate} in this flux.  But if I
pause and observe, saying \wo{Look at {\em this}! It is so-and-so but also a bit
  like that, consider this, reflect over this...}, I am giving it a more
definite shape, I am \co{dissociating} it in order to bring it to my or other's
attention. Such an \co{act} may give more intensity to this \co{actual
  experience} but at the same time, almost paradoxically, it also diminishes the
quiet sense of \co{experiencing}: by isolating {\em this} one element, it
removes it from the \co{rest} in which it lives. Of course, this
\co{dissociation} needs no words but words make it sharper, they express (the
possibility of) a definite \co{dissociation} in which the \co{sign as a sign}
points ostentatiously towards \thi{\ldots}.  This \thi{\ldots} towards which it
is pointing may still be \co{imprecise} and not well defined.  Yet, the very
\co{act} of pointing and the very \co{actual}, precisely limited \co{sign}
create the context where something has been definitely \co{dissociated}, even if
not clearly identified.  An \co{objective} center, an axis around which
attention may rotate is established -- an \co{act} of \co{reflective} {cut} from
\co{experience} has found place, the flux has been frozen leaving the sediment
of the \co{actual} content, an \co{object} or \co{objective} constellation.

Words bring forth something which has already been \co{experienced} and
\co{recognised}. But, in addition, they give it a special status, a more
definite form, which makes up a qualitatively new character of \co{an
  experience}. Even if they create only by focusing, they still create; they
bring a new order into \co{experience}.
\citet{It would be odd to say: <<A process looks different when it happens and
  when it doesn't happen.>> Or <<A red patch looks different when it is
  there and when it isn't there -- but language abstracts from this difference,
  for it speaks of red patch whether it is there or not.>>}{WittPI}{I:446}
Words put the definitive end to the
uncontrolled flux of \co{experience} providing a system of \co{signs} which, by
their nature, stand beyond and above this flux.  Quite a long-termed linguistic
analysis is needed to establish the changes and flux of the language itself.
Such changes may involve mere \thi{sliding} of the semantical fields of various
words but also, and perhaps even typically, their gradual differentiation
resulting in more \co{precise} meanings.\ftnt{Multiple examples of such
  refinements exist. One can mention German \wo{\ger{weil}} which, in the XVI-th
  century, was used rather indiscriminately for \wo{because} {\em and} for
  \wo{so long as}.  A famous -- because of the resulting controversies -- case
  is found in the Torgau declaration: \citefi{... we have at all times taught
    that one should accept and uphold the validity of temporal laws in what
    concerns them \ger{weil} the Gospel does not teach anything
    contrary...}{}{after \citeauthor*{LutherResist}{ p.185}} The idea of
  causality (in the modern, post-Galilean or perhaps even post-Newtonian sense)
  and the idea of, say, accidental co-occurence \wo{created} mutually each other
  (at least as far as the German vernacular is concerned) out of the \nexus\ of
  \ger{weil}. Multiple examples can be found in anthropological literature.
  E.g., quoting Holmes, Mauss mentions Papuan and Melanesian dialects which have
  \citef{one single term to designate buying and selling, lending and
    borrowing.}{Mauss}{II:2}\noo{p.41} The exchange of goods is still involved
  in the \nexus\ of \wo{total services}, the distinct aspects of which have not
  yet become dissociated. As another example, he notes that \citefib{the farther
    one goes back in Antiquity the more the meaning of the word
    \wo{\la{familia}} denotes the \thi{\la{res}} that are part of it, even going
    so far as to include food and the family's means of subsistence. The best
    etymology of the word \wo{\la{familia}} is without doubt that which compares
    it to the Sanskrit \wo{\la{dhaman}}, \thi{house}.}{Mauss}{III:1}\noo{p.63}}
But in the \co{experience} of an individual, words are \co{signs} which by their
external, extra-temporal character provide the means of sedimentation and
identification, of \co{dissociation} of \co{experience} into \co{experiences}.
Learning a language amounts to adopting (and adapting!)  the distinctions and
identities stored in its words and ways of using them. It is a unique (albeit
not the only) entrance into the world of \co{reflective experience}.

%\newpa

\pa \co{Reflective dissociation} means setting the limits, definitely and
\co{precisely cutting} off and thus enhancing (if not establishing) the identity
of whatever is named or denoted. \thi{Freezing} endows thing with a permanence,
by dragging it out of the \co{chaos} and \co{experience} it establishes it as an
independent -- because isolated and permanent -- element.  As the expression of
establishing the identity (proper names being the ultimate examples), words
\thi{give souls} to things, like Adam who not only arranges but in fact enlivens
all the things and animals by giving them names.  Naming used to have a magical
character and pronouncing a name could amount to contacting the transcendent
dimension of the soul of the named person or spirit. The God of The Old
Testament is quite busy with giving names (or new names) to his people
expressing their (new) identity.\ftnt{\citefi{[N]either shall thy name any
    more be called Abram, but thy name shall be Abraham; [...] As for Sarai thy
    wife, thou shalt not call her name Sarai, but Sarah shall her name be.
    [...She] shall bear thee a son indeed; and thou shalt call his name
    Isaac.}{Gen.}{XVII:5-15-19} There is more to this aspect of naming a person
  to which we will return in Book III. Here we are concerned only with the
  identity\noo{(whether of a thing or a person)} resulting (in the case of
  things and objects) from \co{dissociation} and reflected in the \co{acts} of
  naming or, generally, of using \co{signs}.}
%\pa

\label{pa:wordsPower}  
By this very token, by freezing, isolating and bestowing identity, words mean
also power. The primal power of God's over his people is expressed clearly by
(if not simply \co{equipollent} with) his power to name them. Solomon, knowing
the names of all the spirits, held them subject to his will.\ftnt{In a Hebrew
  myth written down around IX-th century, the revolt of Samael's is preceded by
  his defeat in the competition with Adam according to the rules set by God:
  \citef{I created animals, birds and reptiles. Go down, place them in a row
    and, if you are able to give them names which I would give them, Adam will
    revere your wisdom. But if you fail and he succeeds, you will have to revere
    his.}{Bereszit}{p.70}}\noo{
\citt{Solomon knew the names of all the
spirits, and having their names, he held them subject to his will.}{W.
James, Essays in Pragmatism, VI. What Pragmatism Means, p.145}}
A spirit, a thing named, that is \thi{frozen} and \co{dissociated} from its surroundings
becomes subordinate to the one who so \co{dissociated} it: gaining independent identity
it also becomes vulnerable.  Even though it must appear in
a wider context in order to be purposefully manipulated, its isolation is the
first step necessary for inclusion of {\em this} thing into its \co{complex}
context, and thus for manipulating {\em this} thing. This is quite a fundamental
aspect of the almost embarrassing triviality that in order to control $X$, $X$
must be there, one must be able to distinguish $X$ at all.

In the most specific sense, the power of words is the power of \co{reflection}.
To \thi{freeze} and set the limits, to \co{externalise}, means to {objectify}
and to {objectify} means to make \co{visible}. (\citet{To see means: to give
  preliminarily an object as an object. [...] seeing has the meaning of making
  available (of something object-like) in the distinctive sense of pure
  acquainting (with things).}{AugustinNeopl}{\para 14.b. \noo{[p.219]}}) The
structure of \co{visibility} -- \co{object}'s identity, independence from the
background and, above all, its limitation {\em within} the \hoa -- places
\co{object} within the horizon of our control. \co{Dissociating} contents from
their origin, externalising them as \co{objects} independent from the background
to which they belong, we gain power.


\pa\label{pa:wordsPowerC} The creative power of actual words reflects the processes of
distinguishing and recognising the identities. Eventually, and in most generous
sense, it is the power of dissociating and connecting, of setting (some of) the
\co{actual} limits. But the power of \co{reflection} is, in another sense, only
illusory. This power is only over that which enters the \hoa, over the
\co{actual signs} and not, in any case not always and not without much further
\la{ado}, over what these \co{signs} may possibly point to! \co{Reflection},
taken in itself, gives power over \co{signs} and only \co{signs}.\ftnt{Instead
  of control and power, we could speak here (and elsewhere when only
  \co{reflection} is concerned) about manipulation. Its Latin 
  etymology reflects the fact of being graspable, fitting into the hand
  (\la{manus}), and being underlied the authority of one's commands, like a
  small company, a handful of soldiers (\la{maniple}).}
% it is a mere power over the \co{actual signs}, not over the signified, the power
% over that which enters the \hoa\ {\em only in so far} as it enters it.
But the \co{distinctions} and the world of
\co{experience} are much more than the \co{actual objects} which can be grasped,
not to mention fully exhausted, within this horizon. And these \co{distinctions}
find also their expression in words.


\subsubnonr{Beyond actuality}

\pa\label{twoweeks}
\wo{I spent two weeks in Prague with my girlfriend.} What am I talking about,
what am I referring to by this \wo{two weeks}? A {\em concept} \thi{two weeks}?
Hardly, and if so my girlfriend wouldn't be pleased. What I mean by
this phrase is what the phrase is pointing to, namely, \co{this experience}.  I
am referring to {\em these particular} two weeks, to all the moments, events,
moods I experienced during these two weeks but, above all, to the whole
experience of these two weeks. Whoops! \wo{the experience of two weeks}? What is
that? Isn't experience something which always happens \herenow, within the
\hoa? I can experience the table in front of me, the window to
the left, the present situation -- but two weeks? What kind of thing is \thi{two
  weeks} that I can 
experience it? For, to be honest, I must tell you that I did experience it, not
only as a sum of single moments but as a one whole.

I see -- perceive -- a detail of a building. In itself it would hardly pass for
an experience anywhere outside the philosophical tradition, but since this, too,
can be a source of the unexpected, let it pass.  I watch \ger{Vltava} from \noo{the
  Charles Bridge} \ger{Karl\.{u}v most} enjoying a gentle breeze. I do it both --
simultaneously or interchangeably -- being aware and unaware, conscious and
unconscious of 
this experience. In a while the pleasure of the moment becomes so clearly
intensified that I am actually beginning to half-\co{reflect} over it, perhaps
recalling other similar moments, perhaps just staying in this one with full --
\co{reflected} -- appreciation. During the walk uphill to \ger{Hrad\u{c}any}, the
breeze and \ger{Vltava} got imperceptibly lost in the labyrinth of the narrow
streets, but nothing has broken the continuity of the experience. The same
moment from the bridge is now extending to the \noo{St.~Nicholai Church}
\ger{chr\'{a}m sv. Mikul\.{a}\u{s}e}, the buildings around \ger{Malostransk\'{e}
  n\'{a}m\u{e}sti}, the steepness of \ger{Z\'{a}meck\'{e} schody}.  It is {\em
  the same experience} furnished by a more variation in the material of the
world. When I meet my girlfriend at the portal of \ger{Katedr\'{a}la
  sv.~V\'{i}ta},\kilde{svat\'{e}ho} 
we have a brief recollection of a quarrel from this morning which changes the
mood. But neither of us is really up to a quarrel in such a nice weather and
place, and we start enjoying the surroundings together. It isn't any more exactly the
same experience from the bridge and to the cathedral. But it is
now the same {\em experience of being together} in Prague, furnished by yet more
variation in the material of the world, perceptions, moods and feelings.

\pa Just like the whole walk, the whole morning, the whole day
is \co{experienced} and can be \co{an experience}, so are the whole two weeks.
But one might say: I only {\em know} that I was there for two weeks but what I
experienced were only single moments. This certainly does not have to be so.
Surely, a lot of different things happened and I do remember some of them. I
have encountered various moods, ups and downs, different weather, places, people, etc.
But all these variations were underlied by a constant mood, the feeling of
congenial surroundings and company, which persisted through -- above or below --
all the moments of different small experiences. When I say \wo{I spent two weeks
in Prague with my girlfriend} I recollect my girlfriend, Prague and this mood.
Another two weeks in Prague will necessarily be different because, even if they
be accompanied by the same mood, it will be modified by the remembrance of the
first experience.

But, suppose that no such underlying mood was there, that I only experienced and
remember different days, different people, different places. I still have been in
Prague for two weeks and while I have been there I was experiencing not only separate
moments but also my stay. On the last day I had a definite feeling that the two
weeks have ended, that they, perhaps, weren't what I had expected them to be,
that I was disappointed by their character, or else, on the contrary, satisfied
in spite of the lack of some unifying impression of the whole.  The whole
\thi{two weeks} are experienced here as well, only, in a poorer, less meaningful
way. {Poorer and less meaningful} because now their \co{unity} gives place to a
mere \co{totality}, to the mere matter of a definite time span, that is, because
it is a {cut} from \co{experience} effected by an arbitrary criterion
utilising the determinations of objective time -- not by any unifying
\os. 

In either case, the phrase \wo{two weeks in Prague with my girlfriend}
refers to some \co{totality} (perhaps even \co{unity}) of \co{experience}; not
to any concept but to \co{a} concrete \co{experience}.  Obviously, this
experience is not fully contained in what is being said. But the phrase does not
abstract anything from it, it does not convey any \thi{conceptual} or
\thi{propositional content} 
distinct from and alien to the \co{experience}.  The phrase only refers to or
{\em points towards it}. It is an \co{actual} -- and abstract -- \co{sign} of
something which, in its 
\co{concreteness}, lies beyond \co{actuality}.

\pa\label{Prague} When I say \wo{two weeks in Prague with my girlfriend} I
recollect my girlfriend, Prague, and this mood.  It would be strange if the
phrase meant the same to my girlfriend and to you but, as words in general, it
carries enough meaning to establish some degree of common understanding among
all who hear it.  Now, what does it mean that I recollect Prague, what does the
word \wo{Prague} mean? Well, if I had never been to Prague, it would be just a
word for some place I have heard of, a point on the map, an abstract
\co{object}. But what \thi{place}, which place? What is a \thi{place}? If this
building is a place, and this square is a place, is also
this-building-and-this-square a place? and when I was there, saw and experienced
the city? Even more, if I was born there and it was the first city I ever saw.
It did not happen at any point, it simply took time to develop -- not the
concept, but -- the \co{experience} of my home-city. And what is it? What is a
city, what is \co{an experience} of a city?  Where does a city begin and where
does it end?  What can it mean \wo{to experience a city}? I walk around and see
buildings, streets, people. At what specific moment do I experience the city
Prague? At none but, at the same time, at all. Each moment is \co{an experience}
of an aspect, a part of the \thi{city experience}. But there is no one in which
I can say \wo{Now I am experiencing the whole Prague}, there is no single,
\co{actual experience} of Prague.
\noo{
The same distinctions as in~\refp{twoweeks} can be made. I can gradually develop
a feeling of the city's character which distinguishes it from other cities. And
certainly, this does not happen in a single moment.  \wo{Prague} will then be
the sign for this something (Prague) which arises the feeling of this character
underlying the totality of many small experiences.  If I do not obtain any
general impression, it will mean to me either some \co{objective} determination,
or else a lot of unorganised impressions, perhaps, the first ones which come to
my mind when somebody asks me \wo{What is Prague like?}.
}

\pa
One might ask, {if not only single moments, then why two weeks? Why not two
  years, twenty years? Why not the whole life?} Indeed, why not? The
experiential limit of \co{unity} is my whole life, and all particular
\co{experiences} are only  \co{actual} modifications, manifestations of this
fundamental \co{unity}.\ftnt{Quoting and referring extensively to
  \citeauthor*{SternPresence}, Husserl recognises the \co{unity} of an \co{act}
which extends beyond the ideality of a pure \thi{now}. \citef{That a mere
  succession of tones gives a melody is possible only because the succession of
  the psychic processes \thi{simply} unifies itself in one total form. In
  consciousness they follow after each another but they fall within one and the
  same total act.}{Zeit}{A:1.2.{\para{7}}} The \thi{now} becomes an
{\em extended} interval which imperceptibly emerges from the previous one, becomes
the next one and, eventually, dissolves in the horizon (of \co{actuality}?). In a
sense, we only extend this image of continuity and {unity} of the Husserlian
\thi{now} to the temporal \co{unity} of the whole \co{existence}. On the other
hand, we invert the perspective and do not ask about the constitution of unities
from the \co{actual} data, but only about the emergence of \co{actualities} from this 
prior \co{unity} of \co{existence}.}
Some might resist the idea that we {\em experience} \co{totalities} which go
far beyond any particular moment of time, beyond the \hoa.  Yet, it is quite
natural to speak not only about \wo{experiences gathered during my stay in
  Prague} but also about \wo{my experience of Prague}, not only about \wo{what
  different things I have seen there} but also about \wo{what Prague
  was like}.
It is so natural because, indeed, \co{experience} does not consist of a
\co{totality} of more or less minute \co{experiences}, is not a sum of some
\thi{objective} intervals marking separate \co{experiences}. 
% Already \co{recognitions} involved an element of \co{non-actuality}, albeit
% always involved completely within \hoa.
% The main purpose of our holistic approach with gradual differentiation is to
% emphasize the fact that at the bottom of our being is a temporal continuity, not
% a sequence of successive \co{actual} moments.
%
\co{An experience}, a \co{reflective dissociation} of some \co{totality} from the 
horizon of \co{experience} happens only on the basis of the continuity and
\co{unity} which precedes and \co{founds} the possibility of such a \co{dissociation}. 
\co{Experience} is a mode of being which is not restricted to the pure 
\co{actuality} of \herenow, but which develops in a temporal continuity exceeding 
any \co{actuality}. 
%
\thesisnonr{\label{th:extime}\co{Experience} exceeds the \hoa, and any particular
  experience may exceed this horizon. The \co{unity} of
  \co{experience} is not obtained from a \co{totality} of minute
  \co{actualities} but, on the contrary, \co{founds} such a \co{totality}.}

% and presents us with totalities extending far beyond the pure 
% \co{here-and-now}. It does not occur at any single moment but 
% spans some period of time which is a totality independent from (though usually 
% related to) all its single moments.}

\subsubnonr{Not concepts, not phenomena}

\pa The \co{experience} in this sense has little to do with the traditional,
least of all empiricists' or pragmaticists', notion of experience. What
corresponds to such a notion is \co{an experience} and a multiplicity thereof.
Our concept tries only to keep with the common usage of the term.  However, with
such an all-embracing idea, is there anything that is not \co{experience}? I
haven't been to Australia, and yet, Australia is something which definitely is
\co{distinguished} in my \co{experience}. Every particular, \co{distinguished}
thing (this table, Prague, Australia, anger, love) are elements of
\co{experience}.\noo{But not simply because they are \co{experienced}, but because
they are experienced in a \co{recognisable} form.} On the other hand, there are
\co{aspects} accompanying all \co{experience} which themselves can not be
\co{objects} of \co{any experience}. Yet every \co{aspect} of \co{an experience}
is itself \co{experienced}, even if it never happens to be an \co{object} of
\co{experience}. The sphere of the un-\co{recognised} contents, the \co{chaos}
of \pexp\ and \co{nothingness} are not any \co{experiences} but are,
nevertheless, \co{experienced}. To reduce their \co{experience} to merely
\co{actual experiences} is to completely misconstrue their nature, usually, to
deny them any reality.

\pa
\co{Experience} and what is \co{experienced}
comprises much more than phenomena.  When a phenomenologist analyzes a
phenomenon of, say \thi{life}, or \thi{his life} or \thi{world}, he does not
analyze anything which {\em actually appears} in his consciousness when he
thinks (anschaut) \thi{world}.  In the moment when I think \thi{world}, {\em
  nothing specific} appears for me, because what I know about, mean by,
experience of the world cannot be given within the horizon of any single
\co{act} of consciousness.  What he does is searching his \co{experiences}, is
following a chain of associations, looking for the aspects, properties which he
finds related to the \thi{world}.  In particular, he follows this chain beyond
whatever is present in his consciousness {\em in the moment} when he says
\wo{world}.  What is its intentional object supposed to be?  The best one can say
is that \wo{it is something -- everything?  -- out there, but we have no clue
  what}.  What is its essence supposedly resulting from the eidetic reduction?
And if you find any then how long did you spend looking for it, and how much
more -- or less -- would you find if you looked 2 more years?
Likewise, is there any phenomenon {\em of} \thi{life}?  The intentional object
of \thi{my life} is my life but it is a again
\wo{something which I do not know \co{precisely} what is}, and there is nothing to
indicate that the situation might ever change.  We do not have \co{any
  experience} {\em of} life, life is not anything one can experience at any
particular moment. But we \co{experience} life all the time, in a sense, to live is
to experience, and so just as we \co{experience} our \co{experiencing}, we
likewise \co{experience} life.
%
\noo{\co{Distinctions} and \co{recognitions} do not, in general,
possess any intrinsic essence.  (For \co{reflection} which does not go
into analytic model building, they are like accidental {cuts} from
\co{experience}.  A language we grow into and out of is a powerful, but by
far the only, teacher of distinguishing and \co{recognising}.)  Take
the \thi{world}.  What is it supposed to be?
}

\thi{Life}, \thi{beauty}, \thi{meaning}, \thi{God} and most other things of
significance are not reducible to phenomena, are not reducible to \co{actual}
contents of consciousness.  These, however, are aspects which
truly matter -- whenever \co{present} in \co{experiences}, and even more so whenever
absent from them.
No such things can be analyzed by looking at the \co{actual} contents of
consciousness alone.  If I start thinking about them I will almost for sure
arrive at different essences than you.  And this is so because their meaning,
transcending \co{actuality}, is a derivative of the form of \co{existence}, and
then of \co{experience}.
They are different from each other and they are \co{recognised} as such; saying
\wo{world} we do not mean \wo{my life}.  But this difference cannot be defined,
spelled out, cannot be expressed fully and adequately in \co{precise}, that is,
\co{actual} terms. \thi{World} and \thi{life} \co{transcend} any \co{actuality}
and even the posited \co{totality} of all \co{actualities}, and 
a definition must fail because it attempts to appropriate \co{experience}, to
\co{actualise} the 
essentially \co{non-actual}. \co{Experience} constitutes a \co{unity} not
reducible to any \co{totality} of \co{actualities} and \co{actual signs}. 
 And \co{signs} which forget that they are only \co{signs}
turn into mere words.
%
\thesis{\co{Experiences} appear for \co{reflection}, within the \hoa\ only 
as \co{signs}, e.g., as words which refer to them. These \co{signs} are the 
immediate, \co{actual} data of \co{reflection}.}
%
Everything transcending \co{actuality} can enter it only by
means of a \co{sign}. \wo{Two weeks in Prague} is a \co{sign}, and so is \wo{red},
\wo{square}, \wo{my life}, etc. But to be comprised under such an \co{actual sign}, the
corresponding {cut} from \co{experience} must happen in advance -- not
necessarily in the order of time, but at least in the order of \co{founding}.
These prior {cuts} constitute \co{unities} -- not \co{totalities} -- which
get differentiated into more \co{actual} contents and \co{experiences}. \thi{Two
  weeks in Prague} is not a \co{totality} of single moments but their
\co{unity}.  \co{Experiences} are interwoven into the continuous texture of
\co{experience}. Only \co{reflective dissociation} establishes them as independent
\co{objects} and then, their definite sharpness is just the \co{actuality} of
the \co{sign} through which they appear.
% \ftnt{Many of the \ger{Gestalt} school's
% experiments might serve as -- not proofs but merely -- illustrations, perhaps
% examples though mostly trivial, since related only to perception...}

The question about such individuals -- which transcend \co{actuality}, {cuts}
which traverse \co{experience} above the \hoa\ -- is much more fundamental than
the question about universals (which we will address in Book II). Although, like
universals, not
limited to any \co{actual} moment, they are the most individual and concrete
things: we were talking not about 
any \thi{two weeks in Prague} but about \thi{these two weeks}, not about some
\thi{world} in general, but about this very world we are living in, not about
\thi{life} but about \thi{my life}. In this, 
and only in this, consists a possible similarity to phenomena. But they differ
in that \thi{Prague} or \thi{these two weeks}, whenever made into
\co{objects} of 
\co{reflection}, appear at once at a \co{distance} from the \co{actuality} of
the phenomenon, announce at once the
inadequacy of whatever \co{signs} are used to describe them -- speaking
Husserlian, they make adequate intuition an impossibility. 
% \wo{Prague} means to me something very
% concrete, even emotional, but also external, which by no means can be
% explained away by a concept.
I may have no concept whatsoever of \thi{Prague} or \thi{these particular two
  weeks} when I am relating my \co{experience} of them.  And truly,
\co{experiences} and \co{distinctions} like these become associated with words
and phrases in such a free manner, that each time talking about them I may use
different formulations. For their character and \co{unity} is not constituted by
words or other  \co{actual signs} but, on the contrary, \co{founds} the possibility of
giving any coherent description.


\subsubnonr{Confrontation with transcendence}
%
\pa
\co{An experience} -- a \co{reflective} confrontation with \co{experience} -- is
the source of novelty and surprise. It always {\em comes to} \co{reflection} and
is never brought about by \co{reflection}.
% \co{Reflection} \co{dissociates}
% \co{experience} but does not \co{posit} anything in it.
Sure, I can make all kinds of plans and preparations in the attempts to provoke
some experience.  I may anticipate its character and help it occur.  I can 
decide to, book and go for a trip to Prague and spent two weeks there. But to the
extent \thi{Prague} and these \thi{two weeks there} are \co{experienced}, they
emerge as something independent, they offer themselves to me in their expected
and unexpected richness, in their concrete forms which I sought but did not
cause.  \co{An experience} is always given and never taken.  When it occurs it
does so from its own source, it brings me all that I, on my own, could not
produce -- which was the reason that I could only attempt to {\em provoke} it in the
first place.

% There is no \co{act} of positing anything in \co{an experience}, but even more,
% there is no \co{act} whatsoever -- only an event of \co{separation}.
\noo{
The utterance of the word \wo{Prague}, of the phrase \wo{these two weeks} are
single \co{acts}. But such single \co{acts} are always only elements of a
broader discourse or aspects of a context from which they arise as particular
\co{actualisations}.\noo{We do not refer here to any distinction of the
  \la{langue}-\la{parole} kind. \co{Actualisation} reminds more of sermo-vox}
Also, they refer to entities which exist beyond and independently from any such
\co{act}.
}

The aspect of \co{transcendence} is not, not even primarily, limited to
\co{externality} of \co{actual objects}.  Like every \co{actual sign} is
permeated by the \co{distanced 
  presence} of \co{non-actuality}, so an \co{act}, limited to the \hoa,
encounters not only an \co{external object} but also the context of
\co{experience} reaching beyond this horizon.
%
\noo{As \co{signs}, words are but \co{reflections}, yet we said that they create.
Words cast long shadows on things which, without appropriate \co{signs} of
words, wouldn't present themselves so \co{precisely} for \co{reflection}.  Word
is the \co{sign} of the \co{dissociation} of \co{an experience} from
\co{experience}. It witnesses to this fundamental event. Also, to an extent, it
gives it a permanence since it provides a sign which can be reused, also when
the character of the actual \co{experience} has been diminished or even
forgotten. Words aren't mere labels making retrieval of relevant information more
efficient although they can have this function too. As we said, every
\co{aspect} of \co{experience} is also \co{experienced}, even if it never can
become an \co{object} of \co{an experience}.
}
%
Words can refer not only to \co{experiences} but also to the
\co{experienced}, eventually, to all levels of \co{experience}.  As we will see
again and again, the deeper layer of life, the more constant it is but also the
more ephemeral, because the less fixed, are its \co{actual} manifestations.
Furnishing the \co{signs} for these deeper layers, words endow the ingraspable
and \co{non-actualisable} with the character of recognisable and repeatable
permanence. Although the \co{distance} separating such words from what they
express may be infinite, they nevertheless bring thus the most fundamental,
the least expressible \co{aspects} of life closer to the
\co{actuality} of \co{reflective} consciousness. Their \thi{creative} character
consists here in the fact that the form of manifestation of the
\co{transcendent} is almost totally dependent on the choice of the \co{actual}
expression, on the used \co{signs}. In the extreme cases, the \co{signs} which
became mere signs, mere words may even obscure its \co{presence}. 

\noo{
But words do not create in the sense that without them \co{experience} and
thinking would be impossible, that they actually {\em constitute} identity of
things.  One is tempted to explore such a hypothesis in the face of the world
where things lost their pretenses to possess any essence, where boundaries
became diffuse and unclear; in the face of the world which lost its supposed
ontological fundament of \co{objectivity}.  However, language alone is an even
poorer surrogate for certainty and truth than \co{objectivity}.  Words cast
shadows on things only because the supposed \thi{things-in-themselves} have no
definite essence, because things themselves are only {cuts} from the
background.  We can {cut} them in different ways and words assist us in
experimenting with different ways of doing it or else, as the case may be,
assist us in doing it in a right way. But this does not imply that words have
any clearer and more definite epistemological status, that words endow the
\co{objects} with identity which they otherwise lack.  Words are only as clear
as the \co{experiences} they signify, as the {cuts} they denote or make in
and below the texture of \co{experience}.  They are paradigmatic -- but not the only
-- \co{signs} of \co{experiences} which, in addition, can also make \co{present}
something exceeding \co{experience}.

\pa Now, \co{experience} is the source of all that is novel and unexpected.
When my \co{reflection} encounters new things and surprises, this does not
happen because it produced them in some transcendental act of schematisation,
but because they emerged from the, until then un-\co{reflected},
non-\co{dissociated} \co{experience}.\ftnt{One may well ask what mechanisms were
  involved in the pre-conscious structuring of this experience, and what made
  just this particular experience emerge into consciousness, but we leave such
  questions to the scientists and other champions of the irreflective.}  }
\co{Reflection} meets always only things which, in some sense, are already
familiar, which have already been there, \co{distinguished}, \co{recognised} and
\co{concrete}, that is, merged in the continuity of \co{experience}.
\co{Precise} \co{visibility} of a \co{reflected} \co{object}, clarity of a
\co{reflective} thought is achieved by \co{dissociating} \co{an experience} from
this continuity which, for \co{reflection}, remains vague and inaccessible,
unattractive and yet fascinating.  For \co{reflective} thought, \co{experience}
furnishing all its \co{objects}, phenomena and novelties is the horizon of
\co{transcendence}. It remains \thi{outside}, \co{reflection} can never
appropriate it but, at most, conform to it in the constant dialectic of
domination and subordination. Nevertheless, this \co{transcendence}, this
\co{presence} is the constant fact of \co{reflective experience} which surrounds
the variety of changing \co{actualities} with the \co{unity} which is as certain
as it is undefinable.  \citet{For the intellect, the unity is only a postulate,
  an act of faith.  For the spirit, the harmony is the experienced
  reality.}{IdealistView}{\label{ft:rada}}

\thesisnonr{\co{Experience} is a gift of \co{transcendence}. It is \co{an experience} 
only to the extent it confronts \co{reflection} with \co{transcendence}.}

\pa\label{pa:positing1}
Thus, after all and unlike what we said in 
\refpto{pa:wordsPowerA}{pa:wordsPowerC}, words do not give
power. \co{Reflection} \co{dissociates experience} but does not create, it
exercises its power only by means of \co{signs}. These  
\co{signs} are neither arbitrarily chosen nor voluntarily generated, they are
only the \co{actual} expressions of the \co{non-actuality} which, perhaps, can
never be reduced to the \co{actual} categories, can never be underlied
the \co{objective} determinations of \co{reflection}. 

%.. only to the extent they are adequate, that is, denote only actual items...

We do, however, encounter \co{reflection} \co{positing} an \co{object}, in
particular, in the attempts to appropriate \co{transcendence}, to bring
something \co{non-actual} to the level and categories of \co{actuality}.
The \thi{whole world} as the \co{totality} of all \co{objects} is, indeed,
something \co{posited} -- it has no counterpart in \co{any experience}. 
Speaking about \thi{two weeks in Prague}, I may lean toward closing
this experience entirely within the \hoa, I may attempt to \co{actualise} it, for
instance, by expressing its essence, appropriating it as a concept. But also 
such attempts are perfectly aware of their inadequacy. 

There is only a difference of degree between the two, both amount to
\co{positing} an \co{object}. \co{Positing} amounts to not only deliberately
choosing an \co{object} of \co{reflection} but also to actually constructing
it. There are two fundamental kinds of things which may require such a
constructive \co{positing}: ideal objects (in the sense of pure phantoms, 
constructed from \co{dissociated} bits and pieces of earlier \co{reflections}),
and aspects of \co{experience} which by their nature cannot be fully comprised
within \hoa. The latter, like the \thi{whole world} or the \thi{totality of
  time}, although corresponding to some \co{aspects} of \co{experience}, are not
experienceable in the form in which they are \co{posited} as \co{objects}.
Typically (\co{representation} of) such an object is \co{posited} as an ideal
limit, as a \co{totality} trying to \co{reflect} the suspected
\co{unity}. \wo{\co{Positing}} will be used equivocally about all three kinds of
\co{acts} (\co{dissociating} an \co{object}, constructing an abstraction and
reconstructing an \co{aspect} of \co{experience}) and the intended meaning will,
hopefully, emerge from the context. 

\noo{But an important difference occurs in the moment when the
\co{dissociated experience} loses its \co{aspect} of (vertical) transcendence
and appears purely \co{external}, when it ceases to be a part of \co{experience}
and appears as an \co{object} ... yes, merely \co{posited} by the \co{subject},
with the ballast of my will which starts getting the better of the \co{posited}
entity's own character.  We will reserve the word \wo{\co{positing}} for such,
more extreme cases of \co{reflection}, for instance, \co{attentive reflection}
which deliberately chooses its \co{object} or else \co{reflection} which
\co{posits} some ideal limit as a concept of a thing which it may suspect but is
not able to embrace (\thi{the world} as the \co{totality} of \co{actualities},
the \thi{infinite time} as the \co{totality} of successive \thi{nows}.).
}

\subsection{Some problems of reflection}

\noo{\begin{enumerate}\MyLPar
\item \thi{that...}
\item Concrete reflection\\
  -- choice of object, \para 63,64\\
  -- forming it\\
  Here begin problems: choosing and justification
\item Smth -- inappropriate:\\
  -- insufficient (too late)\\
  -- sick\\
  Dualism \para 66
\item Find original truth (inappropriate):\\
  -- Plotinus, \para 71\\
  -- scientism, objectivism, \para 69\\
  -- dualism \impl META\\
  SOL: nothing will ever be the same, \para 73 (Merleau-Ponty); don't reduce
  itself to reflection, know your being above it
\item Finitude (insufficient/confronting transcendence):\\
  -- insatiable \\
  -- ecstatic (moments, frenzy)\\
  SOL: relax, don't intensify, you are only an actual part of...
\item objectivistic illusion...  
\end{enumerate}  
}

\pa\label{pa:formalthat}
Abstractly, \co{reflection} is a simple observation \thi{that ...}  It
merely realises and \co{externalises}.  Every thought can be preceded by such an
\wo{I think that...}, but this leads only to a formal notion of reflection which
captures only part of its nature but completely misconstrues its potential and
significance. Such a \thi{that ...}, the explicitly formalised \co{attentive
  reflection}, is 
only repetition of \co{a reflective experience}. It repeats, in any case,
attempts to repeat, \co{an experience}. When I thus reflectively refer to some
situation or thought, I merely try to revoke it in order to bring forth some
side of it. But in order to do that, the situation, the thought must have
already been isolated in a prior \co{reflective experience}. This purely repetitive
character is what makes such an abstract \co{reflection} entirely contentless, a
mere \thi{that} which leaves all its content in the \thi{...}.
% \say \co{Self-awareness} is immediacy of both \co{aspects}: self- and -awareness
% (cf. \ref{sub:selfAware}, in particular, \refp{pa:selfaware}). In \thi{that...} the two are \co{dissociated} and can never meet
% again. The infinite regress of \thi{... that that that ...} horrified the
% thought but, as a matter of fact, it only tried to \co{reflect} the \co{unity}
% of \co{self-awareness}. Ideally, in the \co{posited} limit, after $\omega$ steps,
% this \thi{... that that that ...} should reach its fix point $r$ such that
% $\mbox{that}(r)=r$. [] Thus any 
% \co{totality} tries to reach again to the \co{unity} which \co{founds} its intuition...
%
% % In other words, the abstract \co{reflection} \thi{that ...} is merely an
% % expression of the abstract characteristic of \co{reflection} as a
% % \co{dissociation} of \co{a single experience} from its concrete context.
% Traditionally, one hoped to get at some universal truth
% (about reflection, or who knows what...) by isolating the purely formal aspect
% which, indeed, is present in every \co{act} of \co{reflection}. Unfortunately,
% the main result was that the whole event became an unbearably empty affair
% happening to some transcendental subjects acting in a frozen moment which only
% blindness could mistake for eternity.
The lack of content goes however hand in hand with some element of necessity. There
is something strict and structurally unavoidable in such an imagined
process of that-ing, even if nobody ever carries it more than two, at most three
steps (cf. footnote~\ref{ftnt:KnasterTarski}). This may attract attention of
formalisers but, concretely, has nothing to offer. 

Concretely, \co{reflection} is not a mere observation \thi{that ...} but an
observation of the actual \thi{...}  -- an act of isolating \co{a particular
  experience} but also of careful attention paid to its actual \co{object}. It
is intimately involved with this \co{object} and its problems are related
exclusively to that: the choices of its \co{objects} and the motivations for
these choices.  Such choices appear easily as arbitrary. In fact, among the
\co{dissociated} alternatives, all appear equally good (or bad) -- dissolution
into atoms, as independent as unrelated, is an \co{aspect} of \co{dissociation}.
Theoretically, one can 
try to explain the reasons for this rather than that choice but it is only the
final result of a series of concrete \co{reflections} which may, eventually,
reveal their motivations and reasons. To start with such a choice would be an
impossibility if \co{reflection} were not anchored in the deeper layers of
\co{existence} capable of lending it some pre-\co{reflective} guidance.

The lack of any proof of the universal validity of its choices is a
possible expression of the problems of \co{reflection}.  In a sense,
\co{reflection} does something inappropriate, it violates the order of things by
\co{dissociating} something from the \co{rest}, positing it as independent
entity and bringing it under \co{reflective} control. Sometimes and somehow this
seems simply to desecrate the innocence of \co{experience}, and the
\co{reflective subject} begins to thirst for the return to the \thi{original
  truth} (which can be almost anything: the original
state of nature, obviousness of senses, certainty of immediacy, frenzy of an
orgy, strength of a violent will, feelings, authenticity,...). Also, and in the midst of
the thirst, 
\co{reflection} is \co{aware} of its insufficiency. It comes always too late.
The confronting \co{transcendence} makes it impossible to forget that the
\co{reflective act} is only  embracing a mere \co{sign}
of something which forever evades the look, let alone the grasp. The \co{subject}
of a \co{reflective act} is as isolated as is its \co{object} -- both are not
only \co{dissociated} from each other, but primarily from the \co{rest} from
which they arose. When limited to the \co{precision} of \co{immediate visibility}, in
the constant \co{attention} paid to all the details of encountered objects and
situations, \co{reflection} can not eventually find any other form of higher
\co{transcendence} than narcissistic \co{self-reflection}. 
But \citet{I swear,
  gentlemen, that to be too conscious is an illness -- a real thorough-going
  illness.}{Underground}{I:2} Such remarks became very common with all the
psychologists of the Victorian times and, in particular, with those who -- like
Dostoevsky, Nietzsche or Kierkegaard -- opposed the primitive psychologism. They
express certainly a characteristic of that time, 
but also show a genuine possibility of \co{reflection}. Let us 
look at some expressions of this possibility. 

\noo{These problems originate in the \co{externalising} character of 
reflection and in the attempts to overcome it. They are
related to the two aspects of \co{externalisation}: \co{separation} of 
the object from \co{experience} and \co{separation} of the object from the 
reflecting person. Finally, there is a fundamental problem arising from the 
fact that anything can, almost anytime, be turned into an object of 
reflection.
}

\subsub{The original truth}% and the finitude of reflection}
\pa Much of our \co{experience} passes without any \co{attentive
reflection}.  \citet{And even in our conscious life we can point to many noble
activities, of mind and of hand alike, which at the time in no way
compel our consciousness. A reader will often be quite unconscious
when he is most intent: in a feat of courage there can be no sense
either of the brave action or of the fact that all that is done
conforms to the rules of courage. And so in cases beyond number.}
{Plotinus}{I:4.10}  
%
But a lack of \co{attentive reflection} is not the same as a total lack of
\co{reflection}, not to mention, of \co{(self-)awareness}. 
\co{Attentive reflection}, especially when carried over the edge of
its plausibility, thirsts for the immediacy of being in which it
could \thi{lose itself}. Trying thus to lose itself, it often would like to
lose \co{reflection}, or even consciousness, from which it differs only by degree.


\wo{How is the immediate belief in the independent existence 
of the world pertaining to the natural attitude possible?} The \thi{natural
attitude} is probably something to be  found outside the philosophical study
chamber. But the sensed inappropriateness of exaggerated \co{reflection} would
like to see in it some completely irreflective, almost animal 
being-in-the-world. The apotheosis of experience, like any other apotheosis,
can yield only a caricature.

But even assuming that such an irreflective, purely experiencing being existed,
does it make sense to ask about its \wo{beliefs}, \wo{certainties},
\wo{attitudes}? Such a being might act in its world and arrange it but here any
contact \co{reflection} may establish with it ends. The very fact that one seems
forced to use the words like \wo{belief}, \wo{attitude}, etc., proper for a
\co{reflective} being, witnesses to the dubious character of the irreflective
pole of the opposition.\noo{Sure, we do a lot of things without reflecting over
  them but should it therefore become one's goal to entirely get rid of
  reflection?} Unless one is willing to maintain merely metaphorical sense of
these words and act as a behaviorist. Eventually, this means to turn into a
student of purely \co{external} realities, to restrict all the attention
exclusively to the \co{objects}, \co{dissociated} not only from each other but,
primarily and most strongly, from the very \co{acts} of \co{reflection} in which
they are encountered. In short, unless one is willing to become a scientist.

%\tsep{scientism...}
\noo{
\pa Science, or rather scientism, with its apotheosis of \thi{objectivity}, like
all other mistakes and misunderstandings, should not take much space.  To be
sure, science is as 
respectable human activity as most others.  It is very well and fine that we
have science of, say, medicine or engineering; that if I am sick, somebody in
the world knows which pill I should take; that if I want to get to the island
over there, somebody knew how to build a bridge and did it. But as far as I am
concerned, all I know and relate to is that I am {\em sick}, that I {\em want}
to get over there. Does it concern philosopher how to build bridges and produce
pills?
}

Of course one does not, but one would nevertheless like to make a point for scientism.
Eventually, in the long run, science will uncover all the secrets of the world
and life and then it will turn out that our experience, our \co{subjective}
experience is only a particular instance of some general, \co{objective} laws.
Really? We leave such projects to those who are able to believe in them.  Let
us, however, for the moment assume that some scientific philosopher manages to
reconstruct the whole reality from his objectivistic assumptions. He managed to
eliminate all the \la{qualia} and impressions and demonstrated that ``we are all
zombies'', he managed to {\em prove} that freedom is a subjective illusion and
that, in fact, everything is really governed by inviolable laws of nature.
Besides causing some confusion in various intellectual circles, this would probably
give us powerful means of influence and control. Yet, would it really eliminate
all the aspects of our existence which were thus reduced to some inviolable
principles? Would I change anything in my way of viewing and reacting to people
and situations, in my preferences and values, in my hopes for life? Well, I
could, perhaps, if I didn't like a concert, take a pill and feel I liked it
after all. Still, I would take it {\em only because} I did not {\em like} the concert!

The projects of a total reduction, and reduction to scientifically determinable
\co{objects} in particular, have been around for long enough to suggest that
those who claim their all-embracing and universal possibility should demonstrate
their factual relevance and truth. There is first the great \thi{if} concerning
the very possibility of such a reduction. Even if this turns out to be possible
(does anybody believe this?), there remains the second, even greater \thi{if}:
would it actually give us the control over all the aspects of our existence. We
are not even able to control fully the processes of society which is, so it
seems, fully human creation.  Until these millennia old \thi{ifs} obtain positive
solutions, their champions can be safely left for themselves as the victims of
the \co{reflective} sickness to the {original and irreflective truth}, that is,
to the lack of \co{self-reflection}. Every explanation is a reduction and
reduction is {\em the} means of all science. However, when proposed as the
ideology of scientism, that is, when seen as a (never ending) attempt to
overcome the \co{dissociation} by ignoring one of its \co{aspects}, it
represents simply the outermost limit of irrelevance to our considerations.

%\tsep{Merleau-Ponty...}

\pa The \thi{original truth}, whether imagined as a pre-reflective state of
nature and innocence or else an \co{external object} of scientific
\co{reflection}, is not only a \co{reflective} construction. \co{Reflection}
comes always too late and it knows it.  \citet{Philosophy, following after the
  world, after nature, life, thinking, and finding all that as constituted
  earlier than itself, asks precisely this earlier being and asks itself about
  its relation to it. It is a return to itself and to all things, but not a
  return to immediacy, which withdraws to the extent philosophy tries to
  approach it and melt with it.}{VisInvis}{Inquiry and intuition;p.129}
%We can safely substitute \wo{\co{attentive reflection}} for \wo{philosophy}
\co{Reflection} can not forget itself, not to mention eliminating itself.  The
vaguest attempt to think in this direction shows that the attempt has already
been undertaken and leads invariably to the situation in which one cannot avoid
being \citet{firmly persuaded that a great deal of consciousness, every sort of
  consciousness, in fact, is a disease.}{Underground}{I:2} But despair is
only a \co{reflection} of the attempt to reduce the whole \co{experience} to the
categories of \co{objective visibility}, that is, of \co{reflection}. Such
attempts encounter \co{reflection} at every 
step and cannot pretend that it is not there; they must ask the question about
\co{reflection}'s (and that includes their own) relation to the \co{rest} of
\co{experience}.  \citet{When this question of second order was once asked, it
  can not be eradicated. From this moment on, nothing will be able to exist the
  way as if the question never occurred.}{VisInvis}{Inquiry and intuition;p.126}
But, let us repeat, this breeds despair only if one identifies one's 
being with this impossibility of coincidence, and starts 
 suffering the presence of \co{reflection}. And
then, as soon as one has constructed a \thi{solution}, a pre-reflective and
irreflective being explaining everything, \co{reflection} finds itself missing
and cannot rest 
satisfied with a result where it is not taken into account. And now, one suffers
the absence of \co{reflection}.

\noo{
\say Somewhere around here: Merleau-Ponty knows very well that reflection will
modify things; he discredits \thi{immediacy} as a philosophical project, and
yet, he seems to look for some original, pre-reflective experience, which would
not be any coincidence, yet would be more original\ldots

Even more: \citet{If it is true that philosophy, as soon as it 
considers itself reflection, that is, coincidence, determines also 
about what it finds, we have to once more start everything anew, to 
throw away all instruments invented by reflection and intuition, take 
place where they as yet do not differ from each other, in experiences 
which have not as yet been \thi{reworked}, which offer us 
simultaneously in disordered confusion \thi{subject} and 
\thi{object}, existence and essence, and which provide means for 
determining them anew.}{VisInvis}{Splot--Chiasm;p.135}
}


%\tsep{... finitude ...}
\pa\label{pa:equrel}
In a bit more sober way, one starts with the acceptance of the
\co{reflective} dualism, in a sense, one admits the finitude of
\co{reflection}. The danger, however, remains, namely, the danger of identifying
one's own being with \co{reflection}, in which case it seems that this very being
is thoroughly and fundamentally dualistic. And thus, although accepting the
\co{reflective} dualism, one does not accept it after all. This aspect has
innumerable forms, so we mention only a few examples. 

%\tsep{dualism \impl META}
%
%\pa
The \co{subject-object }dualism is involved into the
\co{traces} of their original \nexus\ which, however, in terms of the
\co{reflective} categories are at best expressed as some kind of relation(s)
between the \co{dissociated aspects}. The standard picture
%
\equ{\xymatrix{s
  \ar[rr]^r && o} \label{equ:rel}}
%
involves one immediately into the self-\co{reflective} regress. For $r$ is
itself a relation observed by the \co{subject} and so can be, and in the moment
of being addressed in fact is, a new \co{object}. This is nothing but the
contentless irrelevance of the \thi{I think that I think that I think~...}
% (which, following \refp{pa:formalthat}, is the formality of \thi{that that
%   that~...}).
%
The regress is an effect of the \co{dissociation} of the \co{act}
from its \co{object}, according to which \thi{that} of an \co{act} \thi{that ...}
is distinct from its \thi{...} and, consequently, can constitute a new,
higher-level object of next \thi{that \thi{that ...}}. As we observed in
\refp{pa:KnasterTarski}, the empty formality of this operation does not apply 
to \co{(self-)awareness}. 
% is seen in the fact that the original object \thi{...}  as if
% disappears completely -- \thi{that that that $X$} is exactly the same as
% \thi{that that that $Y$}, especially, when both are iterated to the
% $\omega$-limit. (But it is enough to try to capture intuitively the difference
% between two  such iterations carried beyond the 2-nd or 3-rd step -- they both
% become meaningless, at least to the common-sense.)
% In a complete opposition to this formality, the difference between consciousness
% of $X$ and consciousness of $Y$ is exactly the same as the difference between
% self-consciousness of $X$ (i.e., being conscious of one's own being conscious of
% $X$) and self-consciousness of $Y$.
The \equin\ of \co{awareness} and \co{self-awareness}, which \co{founds}
\co{reflection}, makes such a regress if not impossible so, certainly,
unnecessary for the phenomenon of self-consciousness. Every moment of
\co{awareness} (of an object $o$) is, by its very nature, also a moment of \co{immediate
  self-awareness} (of $r$, and hence also of $s$). But this \equin\ of
\co{awareness} and \co{self-awareness} is 
not easily recognised by \co{reflection} which emerges only from their
\co{dissociation}.

\pa\label{pa:equfm}
As a result of this \co{dissociation}, and as a medicine against the possibility
of infinite regress, there appear {meta}-considerations and
{meta}-modeling. In the most simple and \co{precise} formulations, 
like Tarski's hierarchy of meta-languages or Russell's theory of types, the
{meta}-levels are continued indefinitely. In the more philosophical
settings, one terminates the regress by postulating (or discovering) just one
(or only a few) level(s) above the basic one from
\refp{pa:equrel}.\refeq{equ:rel}, whether it is 
the level of intelligible forms above the material contents, or else of
transcendental subjectivity above the empirical ego. Of course, we do not want
to simply conflate all, often very different, variants of this basic idea. But
we do claim that various appearances of \thi{meta} (or \thi{trans}) originate
eventually in this 
basic element: the need of relating \co{dissociated} entities. Consider, as an
example, a simplified hylomorphism. The relation \refp{pa:equrel}.\refeq{equ:rel}, where $r$ is
taken epistemically as something like \thi{knowing} or \thi{understanding}, is
prevented from regress by isolating the intelligible \thi{form}, which reaches
directly the \co{subject}, from the merely perceptible \thi{matter} which
remains on the side of the \co{object}. The relation $r$ becomes thus more
refined, say, something like:
\equ{
\xymatrix{
s \ar@{-}[r]^{form} & | | \ar@{-}[r]^{matter} & o
}\label{equ:fm}   %%%WRONG paragraph number!!!
}
%We have, of course, simplified tremendously but the main point is that, u
Unlike in the simple-minded case \refp{pa:equrel}.\refeq{equ:rel}, attempting
now a self-application of the schema becomes, mildly put, problematic. The
problem which is addressed with the form-matter distinction begins with the
\co{dissociated subject} and \co{object}. The distinction itself, in a sense,
prevents one from asking about self-application and infinite regress. And it
does so by providing the {meta}-categories of \thi{form}, \thi{matter}, etc.
which, remaining fixed, solve the initial problem.\ftnt{\noo{The above suggests
    only a possible function of the form-matter distinction.}We view, of course,
  the distinction itself as a mere consequence of the \co{dissociation} of
  \co{subject} and \co{object} and would limit its validity (if any) to the
  sphere of \co{reflective dissociations}. \citeauthor*{DavidsonFormContent}
  argues against this distinction, the \thi{third dogma of empiricism},
  proceeding similarly from the holism which negates the validity of the
  distinction between the subjective and objective (elements of knowledge). We
  do not, however, claim the unreality of the subject-object distinction as
  such, but only its non-absoluteness, its relativity to the sphere of
  \co{reflection}.}

The most standard version of this distinction assumed the form of the duality of
essences, intelligible aspect belonging to the objects, and their reflection as
actual concepts in the mind. It made, however, very difficult for the two
elements to meet again (unless one sneaked in some form of pre-established harmony
between the two.)  The solution came with the return of the postulate of
unorganised and pure matter, which only mind could endow with any rational form.
The subject becomes now (with Kantian idealism) transcendental and responsible
for all the formal/rational aspects, leaving on the object side merely sensuous
matter (and the purely conceptual, negative rest of a disappearing point,
noumenon). It is the above schema \refeq{equ:fm} pushed to one extreme:
%
\equ{
\xymatrix{
s \ar@{-}[rr]^{form} && |{\ \scriptstyle{matter}}
}\label{equ:fmKant}  %%%WRONG paragraph number!!!
}
%
The forms (of perception, understanding, reason...) by definition require matter
and hence are not self-applicable. The {meta}-level of transcendental
subjectivity giving form to all appearances liberates the original relation $r$
to the \co{external object} from the problems of self-reference and infinite
regress.
\noo{Dually: according to the Stoics, matter had two
aspects, passive and active, and while the former corresponded to indeterminate
\gre{hyle}, the latter was a divine \gre{logos} which, immanent in matter, initiated its
rational unfolding (later \thi{seminal reasons}, or \thi{rational seeds}
(\la{logoi spermatikos}) which inhabited matter, lead its development.) 
  Such (or other) materialism (or some form of realism,
  objectivism, perhaps empiricism) would be represented as \xymatrix{ form\ |
    \ar@{-}[rr]^{matter} && o }
\label{ftnt:EmpiIdeal}  
 }
 
 In short, {meta}-modeling emerges naturally in the \co{reflective}
 attempts to avoid infinite regress, which is only another side of reaching
 toward the \co{dissociated, actual object}. However, the 
 very notion of a {meta}-level is constructed on the top of the 
 \co{dissociation} and, consequently, can never bring the \co{dissociated}
 elements together. The problem returns always with the increased force (even
 if with less impact) with respect to the {meta}-level.  For if
 understanding requires sensuous matter, how is it at all possible to understand
 the very relation of understanding?  What is the matter of and what are the
 categories used in the understanding of the role of the transcendental subject
 in the process of understanding/constitution? Thus, bringing together
 subject and object through a meta-relation which, to avoid infinite regress is not
  self-applicable, introduces the dualism of understanding and
 non-understandability of this understanding. 

%\tsep{Hegel, postmoderns}
\pa The thirst for rest, the rest in the otherness of an \co{object},
unsatisfied by \co{externality} of any single item, turns easily into
superficiality, \thi{bad infinity} which replaces \thi{better} by \co{more}. 
Perhaps, in order to be satisfied, \co{reflection} has to embrace everything
(whatever \wo{embracing} might mean).  But 
nobody can believe that finite \co{reflection} can grasp the whole
richness of the world and all its distinctions.
% (Unless, as it sometimes happens, one is willing to speculate about an absolute
% spirit or the like.) 
Without admiring Bataille's vocabulary and metaphors, one can nevertheless easily
accept the point that every system must leave some, perhaps even some most
significant aspects \thi{outside}. Collecting everything into a \co{reflective
  totality} is a simple impossibility.  The intellectual bias, which accepts
only \co{reflective precision}, together with the associated thirst for the
all-embracing universality, turn this intuition into despair, but a despair
which is much stronger 
than that which emerged from the thirst for the \thi{original truth}. Now it is
almost purely destructive: \co{reflection} cannot embrace everything --
consequently, get rid of \co{reflection}. This \co{reflective} 
hostility to \co{reflection}, and perhaps to rationality in general, is but the
intellectual form of \co{reflective} self-despair, the 
utmost consequence of the search for rational \co{precision} and systematic
\co{totality}, which to mere historicism seem to disappear in the past of modernity. 

In our language, this argument from finitude amounts to the first aspect of
\co{externalisation}.  Just like \co{distinctions} never reach \co{nothingness}
and \co{recognitions} never embrace \co{chaos}, so neither can \co{reflection}
ever stretch as far as (the whole) \co{experience}. \co{Reflection}
\co{dissociates} an \co{object} from its background; it is its constitutive
\co{aspect} -- not a mere accident. By its very nature, it is finite in this
sense: its \co{object} is \co{dissociated} from the \co{rest}. A \co{reflection}
embracing everything in one \co{act} is a self-contradictory notion.  And thus,
if its goal is to account for all the details of whole \co{experience},
\co{reflection} becomes an unbearable burden, which either has to be
\la{aufgehoben} or else to despair over its insufficiency. If one is not willing
to write mere amendments to past and introductions to further investigations,
one can be tempted to stop writing in an understandable way and start
\thi{writing otherwise}.

%\pa
%In a somehow post-modern  way, o
One can, indeed, start thinking that \co{reflection's} only goal is to eradicate
itself, is to cease thinking in systematic, logical, understandable, representational,
communicable -- in short, \co{reflective} -- terms.  Instead, renouncing itself,
reflection should open itself onto all that any system must leave \thi{outside},
onto all \thi{otherness} and even \thi{otherness as such}, without any
presumptuous attempts
to control and organise it. Instead of thinking in the old, reflective way, to
\thi{think otherwise}, to let the absolute \thi{otherness} embrace one in an ecstatic
fusion of multiplicity, that is, as everything seems to suggest, of cacophony.
This delirium is, as a matter of fact, only the final, disappointed stage of the
failed search for the
\thi{original truth}. Since the detailed richness of \co{experience} cannot be
\co{reflected} in a \co{totality} of a system, and since system anyhow is
alien to our 
life, stop thinking system and start \thi{thinking otherwise}.

\pa 
The sensed inadequacy of distinguishing \thi{form} from \thi{matter},
\thi{act} from its \thi{content}, turns into a fashionable habit of identifying
\thi{truth} with  
the \thi{expression of truth} which, in turn, introduces the ambiguity as to whether one 
denounces the former or the latter; whether one wants to say that \wo{there 
is no truth} or \wo{there is no proper expression of truth}. In the first 
case one gets a more existential despair of nihilism which, apparently, does not 
attempt to look for the lost innocence. 
In the latter, one would be more consistent staying quiet rather than
shouting. In any case, one does reject 
the gullible simplicity of the system in favor of \thi{supra-reflective} and 
\thi{extra rational} ecstasis. Yet, it is hard (and we would claim, impossible)
to assume the existential 
attitude of \thi{there is no truth} and those who seem to have assumed it, seem
also to have done it because they cannot find any \thi{proper expression of
truth}.
% But if we accept that some truth is there, and we do it not necessarily because
% we know it but because we feel, believe, sense that it is there in spite of our
% intellectual shortcomings, then its felt presence forces us to speak about it.

%\subpa
And indeed, lacking any proper expression, we hear many calling us to \thi{speak
  otherwise} and \thi{think otherwise}. But
isn't this call, this attempt to break the barriers and reach beyond, actually
very similar to the search for the original truth of lost innocence? The thirst
to \citet{get rid of one's personal ego and become embraced by the otherness
  which one believes to be one's essence}{Suicide}{IV:1 \noo{p.110}} is,
according to Durkheim, a characteristic sign of the \thi{altruistic 
  suicide}; a suicide committed with the best intentions for the others' sake and
  good, but still only a suicidal self-destruction. Although one may suffer from
  various forms of \co{alienation} and 
attempt to overcome them, such attempts may often turn into even more advanced forms
of estrangement, especially, when their goal is to  overcome {\em every} \co{distance}
  separating one from others and  to immerse \co{reflection} completely 
 into the \thi{otherness} of ... its world-object. 

\pa 
Now, ``one may believe that authentic time is an ecstasis; 
yet, one buys oneself a watch.'' One may believe that the depths of 
our being (or non-being, as nothing really is, or 
%%% this should work on Unix
%\setlength{\unitlength}{1cm}
%\begin{picture}(1,0.5) 
%\put(0,0){\line(5,1){1}}
%\put(0.1,0){being}\end{picture})
%
% and this works fine also on MAc
%\rule[0.1cm]{1cm}{.01cm}\hspace*{-.95cm}being) 
%
%but this is the macro from Valius:
\cross{being}) are permeated with uncontrolled and inexpressible openness; yet
one goes around {\em one's own} business and tries to express {\em one's own}
needs and thoughts in a comprehensible way. For even if the \co{foundation} of
communication rests in the \co{invisible} depths, the communication itself
happens \la{hic et nunc}, in the midst of plain \co{actuality} or ... not at
all. There may be some religious or 
mystical truths inexpressible in the plain language. But to speak such a language
one better have some such insights to communicate.
%(Pride, after all, is a cardinal sin, too.)

There is no such thing as \thi{thinking otherwise}, there are not different
kinds of thinking just like there are different kinds of formal logics.  To be
sure, there is non-thinking; there are also other modes of approaching reality,
perhaps, with their own logic. But whenever we encounter a human being with whom
we can communicate, we can also understand, even if only imperfectly, his
thinking -- and that {\em not in spite of} it being \thi{his} but {\em because}
it is \thi{thinking}. There is always a space for failures and
misunderstandings, and there is always, even primarily, a space for other forms
of communication. But anthropologists also do understand {\em thinking} of
strange tribes, just like a German can understand {\em thinking} of a
Chinese.\ftnt{Say, Eliade did understand something of the \thi{primitive}
  religious {\em thinking} and even made some of us understand something of it,
  that is, understand that it is not primitive but that it, too, is thinking. We
  need not, though of course we would like to multiply the quotations from
  L\'{e}vi-Strauss like that \citef{the savage mind is in my intention only a
    meeting place, a result of an attempt to understand taken up by \thi{me}
    putting myself in \thi{their} place and by \thi{them} put by me in \thi{my}
    place.}{CLSesprit}{\noo{p.156 in CLSrozmowyPl}} Referring to the study of
  South-American mythology in \citeauthor*{CLShoney}: \citef{it concerns
    understanding how the human mind works. [...] If the method is worth
    anything it will also allow one to exceed the South-American limits and
    reach a general experience.}{BellourCLS}{\noo{p.157 in CLSrozomowyPl}} }
Speaking otherwise, thinking otherwise, writing and acting otherwise may be,
indeed, \co{egotic} needs of adolescence trying to find and mark its place in
the world.  Among the adults, ecstatic intensity, like the intensity of madness,
may be seen only, and only at best, as an attempted medicine. Against what? As
it appears here, against its own nature which having become unbearable, tries to
become something \thi{other}. But there is no otherness without sameness and
without self-respect one is unable to respect anybody else.\noo{ (which, of
  course, means also that any genuine respect for another witnesses to respect
  one has for oneself).}


\subsub{The objectivistic attitude, the subjectivistic
  illusion.}\label{objectivisticillusion}

\pan A disease attacks and eventually affects only those whose constitution
exposes them to its effects, a virus deadly to humans may happen to be harmless
to monkeys or rats.  Various diseases of \co{reflection} have a sound basis in
human being, in the very emergence of \co{reflection} as such, and in its
relation to the deeper levels of \co{experience}.
The primary problem of, and then also for,
\co{reflection} is that it tends to conflate its own mode with the being
of the human person and thus, for instance, consider all the relations to
the world as if they were simple (though \co{reflectively} always inexplicable)
relations of the \co{subject}-\co{object} kind.


\pa The unlimited power of \co{reflection} consists in the universal {\em
  possibility} of turning anything whatsoever into an \co{object}, of using a
\co{sign}, often just a word, in order to bring something within the \co{horizon
  of actual} observation and inquiry. It is possible for the \thi{I think that
  ...} to be added to all my \co{recognitions}.\ftnt{It is only possible because
  this amounts to an \co{act} of \co{reflective dissociation} which constitutes
  \co{representation}. This \thi{I think} is thus not the \thi{primitive
    apperception} which \citef{must accompany all my representations, for
    otherwise something would be represented in me which could not be
    thought;}{CrPR}{I:2.I.1.2.2.Transcendental Deduction of the Pure
    Concepts... (edn. B)}. Kantian apperception corresponds rather to our
  \co{(self-)awareness} which \co{founds} the possibility of the \co{reflective}
  \thi{I think}.} It is possible to turn every \co{recognition} into an isolated
\co{representation}, to \co{posit} every \co{aspect} of \co{experience}, even 
  one which never is a thematic \co{object} of \co{any experience}, as the
  \co{actual} theme of \co{reflection}. 

This power, due to its universality, lies however at the bottom of
\co{objectivistic attitude} or, equivalently, \co{subjectivistic
  illusion}.\ftnt{Sure, one may use phrases like \wo{forgetfulness of Being}
  or \wo{metaphysics of actuality}.}  It consists in mistaking the universal
possibility for the universal reality: that \thi{everything can be made into an
  \co{object}}, that everything can, as a \co{sign}, appear within the \hoa, is
replaced by \thi{everything {\em is} an \co{actual object}}. The horizon of
\co{experience} is identified with the horizon of \co{reflection}, the entire
world becomes merely a \co{totality} of \co{objects}, its multiple levels and
dimensions are reduced to the \co{objectivity} appearing in \co{reflective
  dissociations}. 


\pa
The non-\co{actual} aspect of an \co{object} has itself an 
\co{objective} character. If I see a building, while actually I am only 
seeing its front-side, its back-side is also meant (\ger{gemeint}); and it might be my 
\co{object} too. What distinguishes here the given \co{sign} from its rest, the 
front-side from the back-side, is the mere fact that the former is 
actually within the \hoa, while the latter is here only possibly -- 
actually it is not, but it might be. 

%\subpa
But is it? Its possible \co{actuality} is, as a matter of fact, its 
present non-\co{actuality}. As Husserl shows, it is there, is an 
integral part of \co{the experience}, but it is there in a different 
way than the \co{actual} aspect. Turning the non-\co{actuality} of 
the building's back-side into a \thi{potentiality}, we have already 
assumed the \co{objectivistic attitude}, we have already falsified 
the way in which it is given and projected into it the assumed 
\co{objectivity}. 

%\subpa
It is, however, obvious that the back-side of the house, although not actually
given now, can be so given if I only go round the house. It is non-\co{actual}
now but it is not {\em essentially} \co{non-actual}. 
The situation becomes more dramatic when I try to \co{reflect} over the world,
God, love, life, even only the two weeks in Prague. It isn't common to call such
things \wo{objects} because, as 
a matter of fact, they are not. They make me acutely aware that what I am
positing and grasping is a mere \co{sign}. It is hard to imagine that the
\thi{missing parts} can ever be given in full \co{actuality}; to begin with, it
is completely unclear what they possibly might be. The \co{objectivistic
  attitude} will nevertheless stick to the conviction that they, too, are
\co{objects}; that if we only travel enough, we will see all that is to see of
the world, if we only analyse enough, we will embrace everything into our
\co{representation}, if we only think and reflect enough, we will eventually
discover all the hidden aspects of love, meaning, hate. And if they are not
themselves \co{objects}, then they are at least amenable to an \co{objective}
description, they are \co{subjective} categories (impressions, experiences, illusions)
which are reducible to the \co{objective} ones. 


Visible \co{objects}, \citet{reasons, seen from afar, appear to limit our view;
  but when they are reached, we begin to see beyond.}{Pensees}{IV:262} The
process of analysis and reduction is even by its proponents recognised as
potentially infinite.  All the declarations of infinity of the process of
analysing and gaining knowledge, all the mere \thi{prolegomena} and
\thi{introductions} one keeps writing in the hope that others will carry on the
  research, 
are expressions of this attitude attempting to view the whole world and
experience as a mere \co{totality} of \co{visible actualities}. 

%\pa
One might think that we are not looking for \thi{all possible aspects} but only
for the eventual \thi{essences}. This, however, is the same and no better.
\co{Objectivistic attitude} postulates the \thi{essences} as surrogates for the
missing \co{objectivity}. Unable to grasp higher things (nor, for that matter,
the source of \co{objects}' identity and independence) under the \co{actual}
spell of its \co{objectivistic} look, it tries to replace them with something
which would be eligible to such a look and treatment.  \thi{Essences} or
\thi{concepts} are means of doing that (and we will say more about them in Book
II:\ref{impressConcept}).  Intelligible object turns into essence, matter turns
into form, and in the midst of the triumphant \co{objectivity} there emerges
again the \co{subject} which, as a matter of fact, has never disappeared. Total
\thi{objectivism} and ultimate \thi{solipsism} are distinguishable only on the
basis of the \co{reflective} dualism from which they are projected. Each
sacrifices one \co{aspect} for the other and the border between the two is easy
to draw only in the most abstract terms when one does not ask about any
consequences. In fact, both live only by negating the other \co{aspect} and are
but absolutisation of the dualism from which they arise: both see only
\co{actualities}. We will therefore use the expressions
\wo{\co{objectivistic/subjectivistic attitude/illusion}} interchangeably.

 The search for \co{more} experiences and for \co{more} intense experiences
driving one around the world is, at bottom, the same as the search for \co{more}
knowledge of particular things, places, peoples, even of scientific results. As far as
they search for \co{more}, whether in the \co{objective} or \co{subjective
  aspect}, they assume that all that is is a \co{totality} of things 
from which one gets less unless one grasps for \co{more}.  Insatiability --
this intellectual (but also quite practical) equivalent of avarice -- is a
cardinal sin because it treats the world in a flattened, purely extensional
manner, where \co{more} becomes equivalent with {better}.  And, as already
Greeks taught, \gre{hubris} is always followed by
\gre{nemesis}.\ftnt{Intellectual insatiability finds also its expression in the
  more direct thirst for power. Platon's state of philosophers is an invention
  of \co{reflection} dreaming about its own power (in a way one might expect from 
  a failed and disappointed politician). Faustus, realising that the ultimate 
  power does not reside in the books of Aristotle or Galen, of physics or
  medicine, renounces them for magic, necromancy and Mephostophilis'
  promises -- only there he can still expect the \citef{world of profit and
    delight,\lin Of power, of honour, of omnipotence.}{FaustMarlowe}{I:51-52}
  Elderly professors seeking political, or at least academic, power and
  influence can be probably mentioned, too. Possible differences in the ethical
  evaluation of so unlike phenomena notwithstanding, one can discern in all
  of them an element of reflection and intelligence seeking universalisation of
  its own principle, seeking power.}

\pa The power of \co{subjectivistic illusion} has a firm grounding in the
\co{actuality} of \co{reflection}. Wherever I travel in the world, I do not
encounter new unexpected modes of vision -- whether in Paris or London, I
encounter new buildings, new people, new roads. Well, I may encounter new ways
of seeing the world and new attitudes, but they all are of the same character as
the ones I could, at least in principle, contemplate at home (a German can
understand a Chinese). And 
\co{reflection} will only encounter new \co{objects} even if it in fact searches
for something else. \citet{Life is, however, rich enough when one only can see;
  one need not travel to Paris and London -- and that does not help, when one
  can not see.}{Angest}{p.109.\noo{Livet er imidlertid rigt nok, naar man blot
    forstaaer at see; man beh{\o}ver ikke at reise til Paris og London, -- og
    det hj{\ae}lper ikke, naar man ikke kan se.} \citefi{Traveling on every
    path, you will not find the boundaries of soul by going -- so deep is its
    measure.}{Heraclitus}{DK 22B45} \citef{An ass which turns a millstone did a
    hundred miles walking. When it was loosed, it found that it was still at the
    same place.  There are men who make many journeys, but make no progress
    towards any destination. When evening came upon them, they saw neither city
    nor village, neither human artifact nor natural phenomenon, power nor angel.
    In vain have the wretches labored.}{Philip}{} Hysterical tourism, search
  for the \thi{exotic and undiscovered} corners of the world, are only common
  modern forms of the attempts to fill the emptiness of \co{objectivity} and
  boredom of \co{subjectivity} through \co{more} \thi{openness to otherness}.}  No matter
how long one thinks and \co{reflectively} analyses a phenomenon, one does not
encounter any {\em qualitatively new} modes of presentation. All the new aspects
and observations one encounters are of the same character -- they present one
with new \co{objects} (\co{subjective} feelings and mere \co{actual} impressions
are, at bottom, \co{external objects} and in any case never bring one out of the
circle of \co{reflective actuality}). They leave one, perhaps pleased, but deeply
unsatisfied. The more intensely \co{reflection} tries to grasp the \thi{essence},
the more it gets entangled into the \co{objectivistic attitude}. Analyses become
longer and longer, books thicker and thicker and the essence more and more
evanescent.


\subsub{Antinomies of actuality}\label{sub:antinomies}\label{antinomies}

\pa The \co{objectivistic attitude}, viewing every \co{non-actual unity} as a
mere \co{totality}, gives rise to innumerable antinomies or, at least,
illegitimate questions resulting from applying wrong categories in wrong
contexts. These are analogues of the limitations of pure reason expressed in
Kant's antinomies.\ftnt{As usual, the story is old and long. The relativisation of reason to the visible
  world became quite explicit with neo-Platonism and the confrontation of Judaic
  and Christian faith with the Greek rationalism in negative theology. 
  Maimonides, Albert the Great, Aquinas considered reason
  incapable not only of grasping God but also of answering the question about
  the creation of the world vs. its eternal existence. A bit
  earlier,\noo{Islamic theologian} Al-Ghazali denied reason's ability to deal
  with things outside the horizon of our experience.
  \kilde{Filoz.Sredni.p.141/147}} 
%As the relation/conflict between \la{ratio} and \la{auctoritas}, philosophy and
%theology, dialectics and antidialectics, it survivied through the Middle Ages
%Scholastics and mystcis were, too, well aware of them. See, for example,
%Nicolas Cusanus, De doctia ignorantia.}
In our case, these are {antinomies of \co{actuality}}, of applying the
categories of \co{actual representation} to things, \co{experiences} and
\co{aspects} of \co{experience} which can not be compressed to fit into \hoa.
The danger for such antinomies arises whenever \co{reflection} posits an
ideal \co{object}, in the sense explained in \refp{pa:positing1}.

\noo{(One might say, by
  positing a potential infinity as an actual one or, as we would say, by
  \co{positing} a \co{unity} as an \co{object} which is thought as a mere
  \co{totality}.)}
According to Kant, antinomies arise because one posits a possible series of
experiential distinctions as unconditionally complete. Applying his machinery,
he makes us \citet{select out those categories which necessarily lead to a
  series in the synthesis of the manifold,}{CrPR}{A415/B442} arriving at the
four cosmological ideas of absolute completeness with the respective antinomies
of: composition (limited vs.  limitless world in time and space), division
(infinite vs. finite divisibility of any substance), origination (determinism
vs. freedom in the universe) and dependence (existence vs. non existence of a
necessary being).

In our language, all these can be seen as examples of \co{positing} 
as \co{object} something which inherently isn't one, \co{positing} something
non-\co{actual}, or even 
\co{non-actual}, as \co{actually} given.\ftnt{It might seem that this is opposite
  to Kant's diagnosis, according to which antinomy \citef{arises from our
    applying to appearances that exist only in our representations, and
    therefore, so far as they form a series, not otherwise than in a successive
    regress, that idea of absolute totality which holds only as a condition of
    things in themselves.}{CrPR}{I:2nd Division.2.Book 2.2.7 [A506/B534]} But
  \co{positing} as \co{object} can be equally described as such an application
  of \co{unity} to a mere \co{totality}, in fact, the very \co{totality} itself
  is already such an application.} Moreover, what 
is so \co{posited} is, at the same time, itself thought as a collection 
or series of \co{objects}. Thus something which in \co{experience} 
arises before the \co{reflective objects}, is attempted thought 
in terms of the \co{objective} categories, a \co{non-actual unity} is attempted
modeled as a \co{totality}. The unavoidability of 
antinomies is just an effect of the universal possibility of turning 
anything into an \co{object} of \co{reflection}. Kant makes the 
absolute distinction between appearances and \thi{things in themselves} 
to make illegitimate the questions leading to the antinomies. We only say: there
are other modes of  
\co{experience} than the \co{objective}, \co{reflective} one. The 
\co{totalities} postulated as here \co{objects} often do have some counterparts 
in \co{experience}, but not in the \co{objective experience}. 

\pa
Kant insists that these four are all and only antinomies -- this 
fits nicely the table of categories. But antinomies, more or less similar to 
the above ones, are not limited to \co{positing} the ultimate \co{totalities},
to the paradoxes of the absolute limits and the associated
self-reference. They arise whenever we try to \co{posit} an inherently 
\co{non-actual} reality as an \co{object}. Its \co{non-actuality} 
becomes then a collection of other \co{objects} and the \co{posited object} 
itself a perplexing one-many. 

\co{Posit} any feeling as an \co{actual object}. Is it determined or is it free?
Both and neither (it is not completely without reason but any reason one might
find is not sufficient). Is it one thing or many?  Both and neither (it is this
feeling and not that, but it also comprises other feelings, more specific moods,
moments, perceptions).  Did it have a beginning or not? Both and neither (it must
have started some time because it did not last always, but it did not start at
any definite moment). All other kinds of unanswerable, that is, \co{objective}
questions are possible. Is it or is it not the same feeling as I had two weeks
ago? Which $x$ makes it different from that other feeling? Where does the one
end and the other begin? Countless antinomies can be produced, once it is
assumed that all that is are \co{objects}.

%\pa
The very antinomy of subject-object arises from the attempts to think the
underlying \co{unity} in terms of the \co{reflectively dissociated} poles. One first
\co{posits} a subject and an object as two completely \co{dissociated} entities
-- both, in fact, imagined as \co{objects} -- and then scratches one's head over
the question how they possibly might have anything to do with each
other.\noo{The more intricate systems of interacting components
(\co{objects}) one constructs, the less of any ultimate \co{objects} are left
\thi{out there}.} Beginning with the \co{dissociated} poles, one can only end up
reducing one to the other -- any unity respecting the genuine distinctness of
the two must appear as
\co{transcending} the \co{dissociation}, that is, as something \thi{mystical}  
beyond the admissible categories. 
Most generally, antinomies arise as a result of applying the categories,
that is, \co{distinctions} of lower levels to various higher \co{aspects} of
\co{experience} -- eventually, the categories of
\co{visibility} to the sphere of \co{invisibles}.\ftnt{We do not distinguish
  \thi{categories} from \thi{concepts} or, for that matter, \thi{patterns
    of understanding} from \thi{understanding particular things}. We will say a
  few words about the issue in Book II, but all such forms are just particular cases
  of (drawing) \co{distinctions}.} We look at a couple examples which will also
be of some relevance for later considerations. 

\subsubi{Matter vs. spirit}\label{pa:matter}
%
\noo{\begin{enumerate}\MyLPar
\item subject-object/mind-body/spirit-matter/active-passive: dissociation
\item Plato's dualism (Orphic, Pythagorean): analogical modeling (rather than
  reasoning): absolutisation of dissociation
\item Aristotle: substance predicated of matter? or not?\\
  individuates AND is individuated (by form)\\
  when all else is stripped off...: dissolves -- indistinct
\item this may as well be (or as badly) ``spirit'' (absolutising some activity)  
\end{enumerate} } 
%
%One of the most powerful and pervasive examples originates with the \co{dissociation}.
The \co{dissociation} of \co{subject} and \co{object} draws
its pervasive power from the obvious \co{experience} of the duality, perhaps even
opposition, of mind and body. But when pushed to the extremes of metaphysical
principles, it turns into an irreconcilable dualism of spirit vs. matter, and
of the associated \thi{attributes}, like active vs. passive, eternal vs. temporal,
higher vs. lower, etc., etc., etc...


\pa
\noo{The ambiguity of God vs. matter entered theology and then
philosophy via the Church Fathers and early Christian thinkers,
%\ftnt{E.g., \citeauthor*{AugustConf}, \citeauthor*{DivNames} [???check]},
who either elaborated Plato's dualism or inherited the opposition from
neo-Platonism.\ftnt{\citef{Everything which is simple in its substance, is either
    higher or lower than that which is complex.}{Proclus}{59} The apparent
  tautology reflects only the fact that unity and simplicity, characterising the
  ultimate One, seem equally discernible in the lowest elements of nature:
  \wo{the simplest is the last of beings, as well as the first one, for both
    come from the only original one; but while simplicity is in one case higher
    than all complexity, in the other it is lower.}}
}
%
Plato takes probably the first place, \co{dissociating} the world into the
sensible and the intelligible when following the Orphic and Pythagorean
tradition of 
opposing body to soul.\ftnt{E.g., Plato, in \btit{Cratylus} refers Orphic views:
  \woo{Some say that the body is a tomb of the soul, as being buried in it for
    the present life. And because the soul expresses (\gre{semainei}) by this
    body (\gre{soma}) whatever it may wish to express, so it is rightly called a
    tomb (\gre{sema}). The Orphics, in particular, seem to have given it this
    name, as they think the soul suffers punishment for its misdeeds.}{after
    Stromata III:3.16} Pythagorean Philolaus quoted by Clement: \citefi{The
    ancient theologians and seers testify that the soul is conjoined to the body
    to suffer certain punishments, and is, as it were, buried in this tomb.}{DK
    44B14 \citaft{Stromata}{ III:3.17}}{}
% \citef{For they [Plato, Pythagoreans, Marcionites] think the soul is divine and
%   has come down here to this world as 
%   a place of punishment.}{Stromata}{III:3.13} 
}
But Plato carries the \co{dissociation} of moral character into the metaphysical
opposition of the \thi{material} and the \thi{spiritual}.


\citet{Suppose a person to make all kinds
  of figures of gold and to be always transmuting one form into all
  the rest -- somebody points to one of them and asks what it is. By
  far the safest and truest answer is, That is gold; and not to call
  the triangle or any other figures which are formed in the gold
  ``these'', as though they had existence, since they are in process of
  change while he is making the assertion; but if the questioner be
  willing to take the safe and indefinite expression, ``such'', we
  should be satisfied. And the same argument applies to the universal
  nature which receives all bodies -- that must be always called the
  same; for, while receiving all things, she never departs at all from
  her own nature, and never in any way, or at any time, assumes a form
  like that of any of the things which enter into her; she is the
  natural recipient of all impressions, and is stirred and informed by
  them, and appears different from time to time by reason of them. But
  the forms which enter into and go out of her are the likenesses of
  real existences modeled after their patterns in wonderful and
  inexplicable manner.}{Timaeus}{18}

This is a perfect example of \thi{analogical reasoning}, or perhaps just
\thi{analogical modeling}. The trivial
\co{distinction} of \co{actuality} between the material from which a
thing is made and the thing itself, is applied to the \thi{universal
  nature}, which is \co{posited} as an indistinct substratum {\em
  receiving} possible forms; the \co{dissociation} of purpose and
achievement, plan and its execution, which is close to constitutive
for the daily acts and activities, is elevated to the principle of
the highest level.  Thus \thi{matter} becomes \citeti{formless, and free
  from the impress of any of these shapes which it is hereafter to
  receive from without. For if the matter were like any of the
  supervening forms, then whenever any opposite or entirely different
  nature was stamped upon its surface, it would take the impression
  badly, because it would intrude its own shape. Wherefore, that which
  is to receive all forms should have no form;}{Ibid.}{\noo{Timaeus 18}}
And due to its passivity and receptivity, it is in a bad need of
something else, of an external principle, \citeti{that of which the thing
  generated is a resemblance.}{Ibid.}{\noo{Timaeus 18}}

\noo{
\vspace*{0.5ex}\par\noindent\citet{For the present we have only to
  conceive of three natures: first, that which is in process of
  generation; secondly, that in which the generation takes place; and
  thirdly, that of which the thing generated is a resemblance. And we
  may liken the receiving principle to a mother, and the source or
  spring to a father, and the intermediate nature to a child; and may
  remark further, that if the model is to take every variety of form,
  then the matter in which the model is fashioned will not be duly
  prepared, unless it is formless, and free from the impress of any of
  these shapes which it is hereafter to receive from without. For if
  the matter were like any of the supervening forms, then whenever any
  opposite or entirely different nature was stamped upon its surface,
  it would take the impression badly, because it would intrude its own
  shape. Wherefore, that which is to receive all forms should have no
  form;}{Timaeus}{18}

\vspace*{0.5ex}\par\noindent\citt{Wherefore, the mother and receptacle
  of all created and visible and in any way sensible things, is not to
  be termed earth, or air, or fire, or water, or any of their
  compounds or any of the elements from which these are derived, but
  is an invisible and formless being which receives all things and in
  some mysterious way partakes of the intelligible, and is most
  incomprehensible.}{Ibid.}
} %end \noo

\pa\label{pa:matterAristotle} Students of Aristotle are well acquainted with the
difficulties to sort out what the Philosopher actually says in
\btit{Metaphysics}, \btit{Physics}, \btit{Categories} concerning matter. There
seems to be a difference between \thi{matter} and the \thi{ultimate substratum}
(\la{materia prima}), though it is not clearly articulated.\ftnt{Thus, for
  instance, \citef{The matter comes to be and ceases to be in one sense, while
    in another it does not. As that which contains the privation, it ceases to
    be in its own nature, for what ceases to be -- the privation -- is contained
    within it. But as potentiality it does not cease to be in its own nature,
    but is necessarily outside the sphere of becoming and ceasing to
    be.}{AristPhysics}{I:9\noo{[non-standard numbering]}} The quotations supporting
  either view could be multiplied. For us the relevant thing is not which one is
  correct but that it is impossible to agree on that. The very question whether
  Aristotle is committed to assume a characterless \la{materia prima} or not is
  impossible to settle (e.g., \citeauthor*{AristPrime} argues for, while
  \citeauthor*{AristNoPrime,AristSubsUniv} against his commitment to such
  a notion.)} Matter is supposed to individuate the forms but, on the other
hand, being completely formless it is itself in a dear need of being
individuated -- it \wo{desires the form}. Living beings are unquestionably
substances, but then, as composed of form and matter, some more basic substance
should be present as well. It is common to consider for instance
\btit{Metaphysics} VII:3\noo{Z} as denying substantiality to matter\ftnt{E.g.,
  \citeauthor*{SubstanceMatterX, SubstanceMatterY, SubstanceMatterZ}}, but one
can also present reasonable arguments for the opposite view according to which
Aristotle, at least in some sense, considers (the prime) matter to be
substance.\ftnt{E.g., \citeauthor{SubstanceMatter}.}  Distinguishing subjects of
change from subjects of predication (or logical subjects) does not help much
because matter seems the ultimate subject of change, and such subjects are also
subjects of predication.

%drawing the line between the two bothered Aristotle: matter vs. form 

\pa The history of \thi{matter} in the following philosophical tradition 
is a continuation of this \thi{analogical thinking} which establishes
a relative \co{distinction} as something ultimate. \thi{Matter} is
always an ideal \co{posited} by \co{objectivistic illusion} on analogy
with the \thi{stuff from which physical things are made}.
%starting with the existence of physical things, \co{actual, external
%  objects} \co{dissociated} from the \co{distinctions} whose limits
%they represent, and generalised beyond the limits of any \co{clarity}.
But when extrapolated beyond the limits of \co{actuality} as the primordial
  substance and first principle of the universe, the matter becomes
  \noo{Matter philosophers contend for is}
    \citet{an incomprehensible somewhat,
  which hath none of those particular qualities whereby the bodies falling under
  our senses are distinguished from one another}{BerkeleyPrinc}{\para 47} or,
for that matter, from anything else.

To be sure, \thi{spirit} as the similarly ultimate principle opposed to
\thi{matter}, is an equally empty result of the same absolutisation 
of relative \co{aspects}, of raising some properties of the \co{actually
  dissociated} \co{subject} to the level of the ultimate principle.  We can
admire Berkeley's arguments but not the attempts to reduce the opposition to one
of its terms. Granting primacy to \thi{spirit} over \thi{matter} is as good as
doing it the other way around. In either case what is left is only some
contentless and propertyless void, while one remains involved into
the dualism -- if not of the claimed elements, so in any case of the used
concepts, of \thi{spirit} opposed to and abolishing \thi{matter} or vice
versa.\ftnt{The remark applies also to somewhat ingenious 
construction offered by the Stoics who, denying any transcendent principles and
spiritual entities, distinguished two aspects -- or perhaps kinds -- of matter:
the active and the passive, the forming and the formed one. \citef{In themselves
  both are the same; it is the same being of which a part assumes the form of
  the world, while another retains its original form and in that shape appears
  as the moving cause or the Deity.}{Zeller}{\citaft{Powrot}} We could probably
find here quite a few analogies to our
presentation if only we were willing to ignore the ever present and dominating
opposition of the two aspects.}

%\

\pa\label{pa:ambigMatter} \thi{Matter}, when \co{posited} as anything more than
the physicality (\co{externality}) of particular \co{actual objects} perceived by the
senses (\wo{designate matter} as some Scholastics would say), simply dissolves
losing all its supposed \thi{intrinsic}, 
\thi{objective} qualities.  \citet{For my definition of matter is just this --
  the primary substratum of each thing, from which it comes to be without
  qualification, and which persists in the result.}{AristPhysics}{I:9} To save
such a vacuous residuum from 
total non-being, one has to take recourse to very special distinctions.
For instance, \citetib{we distinguish matter and privation, and hold that one of
  these, namely the matter, is not-being only in virtue of an attribute which it
  has, while the privation in its own nature is not-being; and that the matter
  is nearly, in a sense is, substance, while the privation in no sense
  is.}{AristPhysics}{\noo{I:9}} Perhaps privation is just \thi{stripping off} while
matter is what remains when \wo{all else is stripped off}, but \wo{being
  nearly}, \wo{being in a sense} substance, etc., (without actually being it)
%as observed in \refp{pa:matterAristotle})
suggest, if nothing more, then at least
that language reaches its limit in expressing the difference between the primary
substratum and the pure negativity.

A more empirically minded might ask:
is {matter} (composed of) atoms, elementary particles, quarks?  The recent
word in the chain is \wo{strings}.\ftnt{The word goes that strings as if
  vibrate, with different amplitudes -- doubtlessly, to an extreme delight of
  neo-Pythagoreans.}  But then one also knows that $E=mc^{2}$, that matter is
exchangeable with energy, is but a form of energy.  So what is energy\ldots?  We
won't say that it is the \citet{secret fire of the alchemists, or phlogiston, or
  the heat-force inherent in matter, like the \thi{primal warmth} of the Stoics,
  or the Heraclitean \thi{ever-living fire}, which borders on the primitive
  notion of an all-pervading vital force, a power of growth and magic healing
  that is generally called {\em mana}.}{JungArche}{I:68} No, we won't say that.  But what shall we say then\ldots?
%I do not feel competent to review the possible, even
%the most accepted, definitions.
We are not trying to ridicule the hard and thorough attempts of the physicists'
to mathematise physics. Their relevance to our philosophy is nil any way. But they,
too, end up dissolving \thi{matter} in something entirely un-matter-like. 

%\newpa

\pa \thi{Matter}
%(or, if one is looking for more definite analogies, energy -- no
%attempts to mimic physics or to attain physical plausibility intended)
is an image, a \co{symbol}.  Of what?  Ha!  It intends to stand for the
\co{external objectivity} raised to the level of the \co{absolute}. But then it
turns into the truly \co{absolute indistinctness}. On the one hand, it resides
in every \co{external object}, so one asks more and more specific questions --
atoms? quarks? strings? -- in search for the limit of the \co{distinctions}, for
the most \co{immediate} in the hierarchy of Being: the simple and
indivisible. On the other hand, as the universal substratum, the always formed
formless, it is, again, the limit of \co{distinctions}, namely, the ever
\co{indistinct}.  The two limits seem to coincide, for beyond the limit of
\co{distinctions} there 
remains only the ultimate \co{rest}, the \co{indistinct}. 

\noo{We won't deny that the chaotic multiplicity of \co{actual distinctions},
of most minute \co{objects} and sensations may, indeed, be termed
\wo{formless}. If one feels like that, one may even call it \wo{matter},
although this tends to create wrong associations.
But there is a difference between the \thi{formless} and the
\co{indistinct}, between the multiplicity of \co{distinctions} \co{posited}
as a \co{totality} and the lack thereof in the \co{original nothingness}. 
But when one  \co{posits}, beyond the \co{posited} totality of
\co{actual distinctions} some analogical substratum, some \thi{matter},
then ambiguities, and in fact, antinomies of oppositions are unavoidable.
}
%[p.106, The Psyche in Antiguity - cit!]  Sure, it appears only as a
%limit of downward movement, so it is hardly ever The Good Itself\ldots
%In more modern language, it is the ambiguity of \la{sublimatio}, the
%ascent following the original descent, vs.  \la{coagualtio}, the
%process of the one who \citt{ascends from earth to heaven and descends
%again to earth}{tPiA, p.104 [Emerald Tablet]}

If one can form any \co{concept} of \thi{matter} at all, it is simply
that of the \co{indistinct}.  As the formless, confused, \co{indistinct} it is,
indeed, the principle of individuation simply because it is the same as the
\co{origin}, the \co{one}.  It must not be confused with any particular
\co{distinctions}, whether quarks, energy, or \co{actual} material things.  As
Berkeley argues, ascribing any properties to some \co{posited} \thi{outside},
some independent \thi{matter}, brings it immediately into the relativity to the
one who performs such an ascribing.  All particular properties (of matter or
whatever) arise from the \co{origin} through the \co{invisible} process of
\co{creation} which, gradually, takes more and more familiar, human form.
%This is as far as I find it reasonable to go.

The image of something \citet{which remains when all else is stripped
  off}{AristMeta}{VII:3 [modified]} is as easy to \co{posit} as difficult to
maintain. In the language of substances and accidents it must, indeed, emerge as
  the ultimate substance. But substances have been earlier given the
  status of independent -- and, in particular, individual -- existents. Such
  \thi{something} -- an individual, independent existent above all temporal
  distinctions -- might perhaps be thought of not as \thi{matter} but rather as
  ... well, \thi{spirit}. 
The primordial \co{distinctions}, the
first acts of creation, do not introduce \thi{matter} as opposed to
\thi{spirit}, \thi{body} as opposed to \thi{mind} -- \co{birth
  separates self} from the \co{one} and the following \co{chaos} of
\co{distinctions} does not single out any of them as more basic, more
fundamental; it does not even oppose one to another.  So far, that is
all; there is as yet no structure, which the \co{distinction}
\thi{matter}-\thi{spirit}, not to mention \thi{body}-\thi{mind},
presupposes.  Before \co{subject} gets \co{dissociated} from the
\co{object}, before \thi{spirit} gets \co{dissociated} from
\thi{matter} and \thi{mind} from  \thi{body}, there is still only
the \nexus\ of \co{chaos}, where Being and Thinking are not two
different things, not even two different things which mysteriously happen to 
coincide, but just one, as yet undifferentiated \nexus. 

If one wanted to discern some \thi{materialism} here, it would amount simply to
saying that the 
stuff from which \thi{mind} is made is the same as the stuff from
which stars and galaxies are made, \thi{mind} and \thi{body} are made
from the same \co{one}.  But in the moment one thinks of the \co{one}
as \thi{matter} (whatever one might mean by that) which is distinct
from, perhaps even opposed to anything whatsoever, one has already
gone too far, for one has projected some \co{distinctions} onto the
\co{indistinct}.  If, on the other hand, one says that \thi{matter} is
the same as \co{one}, then one has said nothing about the \thi{matter} and
merely used a hardly appropriate name for the \co{one}.


\pa Perhaps, what we have described as the \wo{antinomy} of spirit-matter is but
a special case of Kantian fourth antinomy. If a Kantian
wishes, he probably might see it this way. But there seems to be a
significant difference between our use of the word \wo{antinomy} and
its Kantian version.  An application of lower \co{distinctions} to
higher spheres of Being need not, by necessity, result in a conflict
of reason as sharp as Kant illustrates it. \co{Dissociation} of spirit
and matter, natural and easy as it is at the level of \co{reflection}
and \co{experience} may be turned into an apparent contradiction when
transferred to the level of \co{chaos} or \co{nothingness}. But there
are other ways of coping with \co{dissociated} oppositions. The
simplest is just admitting the existence of incompatible metaphysical
entities, although a more common is to admit only one and refuse the
other. Whether such solutions are satisfying -- to one's reason
 -- will depend on how one wants to define reason. But answers, if any, can hardly be
 given in terms of \co{dissociated objectivities} which are exactly what is
 responsible for the antinomic character of the questions.
Calling them \wo{antinomies}
suggests that they need not be considered true problems but rather results of
a category mistake, of applying right \co{distinctions} in wrong context. But to
accept such a suggestion, \co{reflection} has to 
admit that its \co{objective} categories and plain \co{visibilities} do not
exhaust the field of meaningful answers, that it
%ADD
  \citet{is benefited by the examination of a subject on both sides, and
    its judgments are corrected by being limited. [...] For it is perfectly
    permissible to employ, in the presence of reason, the language of a firmly
    rooted faith, even after we have been obliged to renounce all pretensions to
    knowledge.}{CrPR}{II:1.2 [A745/B773]}
\vspace*{-1.5ex}

\subsubi{God vs. matter}\label{sub:Godmatter}\vspace*{-1ex}
\noo{
\begin{enumerate}\MyLPar
\item there is a moral dimension: continuous calm of reflective distancing
  vs. violence of sudden desires and hollowness of mere feelings...  
\item spirit-matter carried to the outermost principle: God-matter
\item ambiguity in Aristotle's (first mover vs. prime matter); neo-Platonism,
  etc...
\item both are but images of \co{indistinct}  
\end{enumerate}  
}
\pa
The opposition can be pushed even further into the transcendent sphere where no
longer \thi{spirit}, but the ultimate God or One stands on the other side, 
opposing matter. It is really only a continuation of the previous antinomy but
it makes the \thi{matter}-\thi{spirit} equivocity, which creeps in with a
recurrent insistence, painfully clear and deeply unpleasant. 

\wo{By $X$ I mean that which in itself is neither a particular thing nor of a
  certain quantity nor assigned to any other of the categories by which being is
  determined.\noo{For there is something of which each of these is predicated,
    whose being is different from that of each of the predicates (for the
    predicates other than substance are predicated of substance, while substance
    is predicated of matter).}[...] Therefore $X$ is of itself neither a
  particular thing nor of a particular quantity nor otherwise positively
  characterised; nor yet is it the negations of these, for negations also will
  belong to it only by accident.}  It would not be offending to negative
theologians if both $X$'s were replaced by \wo{God}. In the text \wo{matter}
stands for the first and \wo{ultimate substratum} for the second
one.\ftnt{\citeauthor*{AristMeta}{VII:3\noo{Z}}}


Plotinus on The One: 
\citet{The One, as transcending Intellect, transcends knowing: above
all need, it is above the need of the knowing which pertains solely to
the Secondary Nature. Knowing is a unitary thing, but defined: the
first is One, but undefined: a defined One would not be the
One-absolute: the absolute is prior to the definite. [...]
 Thus The One is in truth beyond all statement: any affirmation
is of a thing; but the all-transcending, resting above even the most
august divine Mind, possesses alone of all true being, and is not a
thing among things; we can give it no name because that would imply
predication}{Plotinus}{V:3.12-13} 
%
The image of matter is more familiar: \citetib{We utterly eliminate every kind
  of Form; and the object in which there is none whatever we call Matter: if we
  are to see Matter we must so completely abolish Form that we take
  shapelessness into our very selves.}{Plotinus}{I:1.10} It is hard not to see
the analogy, and so the fragment continues: \wo{In fact it is another
  Intellectual-Principle, not the true, this which ventures a vision so
  uncongenial.} Matter is \thi{not the true} first hypostasis but almost, 
it is not being but \citetib{By this Non-Being, of course, we are not to
  understand something that simply does not exist, but only something of an
  utterly different order from Authentic-Being.}{Plotinus}{I:1.3} It seems
that, if nothing more, then at least the language reaches its limit in
expressing the difference between the absolute One and the pure negativity of
matter (cf. \refp{pa:ambigMatter}). But why should language express
differences which cannot be thought and which, perhaps, simply do not obtain?

Just one more example.  The tension -- of analogous descriptions of preferably
opposite extremes -- was clearly observed by Eriugena: \citet{there are two, and
  two only, that cannot be defined, God and matter.  For God is without limit
  and without form since He is formed by none, being the Form of all things.
  Similarly matter is without form and without limit, for it needs to be formed
  and limited from elsewhere, while in itself it is not form but something that
  can receive form.}{Periphyseon}{I:499D;499-500A\kilde{p.167}} \citetib{And this
  similarity between the Cause of all things [...] and this unformed cause -- I
  mean matter [...] is understood in contrary sense.  For the supreme Cause of
  all things is without form and limit because of its eminence above all forms
  and limits.[...] Matter, on the other hand, is called formless by reason of
  its being deprived of all forms. For by it nothing is formed, but it receives
  different forms.}{Periphyseon}{II:167-169} This opposition notwithstanding,
there is even a further similarity.  \citetib{Matter itself, if one examines it
  carefully, is also built up from of incorporeal qualities.}{Periphyseon}{II:133} \citeti{Formless matter is the mutability of mutable things, receptive
  of all forms.}{after \citeauthor*{AugustConf}}{XII:6}\noo{After Augustine, Confessions
  [Peri.II:169]}\wo{[...] I think, 
  that if it can be understood at all, it is perceived only by the intellect.}
The \thi{intellect}, let us emphasize, not in the derogatory sense which it
often obtains in the modern expressions like \wo{mere intellectualism} or
\wo{intellectuals}, but in the sense of \gre{nous} or 
\gre{logos}, in the sense which it acquired from neo-Platonism, through the
Church Fathers to the Scholastics, as the highest faculty of the soul which
remains in the closest vicinity of Godhead and perceives the immediate works of
God, the first stages of creation, the primordial causes.\ftnt{One should always
be wary of the distinction between these two kinds which both may be called
\wo{intellect}: \citef{It is our separating habit that sets the one order before the
other: for there is a separating intellect, of another order than
the true, distinct from the intellect, inseparable and unseparating,
which is Being and the universe of things.}{Plotinus}{V:9.8}}

%\newp 

\pa\label{Godmatter} 
In short, \thi{matter} which \citet{is negatively defined as not being any one
  of the things that are}{Periphyseon}{II:141} is very hard to distinguish from
%  since it is from it that all the things that are created are believed to be
%  made.}{Peri.II:141}
\citet{the One which is beyond thought [and] surpasses the apprehension of
  thought, [...] the Universal Ground of existence while Itself existing not,
  for It is beyond all Being.}{DivNames}{I:1}\noo{\citeti{the Being of all
    things [which] is the Divinity that is beyond being.}{Pseudo-Dionysius,}{
    quoted in \citeauthor*{Periphyseon}{I:516B-C\kilde{p.205}}}} Conceptual
\co{distinctions} can be \co{posited} to maintain a kind of orthodoxy or good
conscience but both turn out to be just\ldots \co{nothing}. The {apophatic}
language of the divine, just like the negative descriptions of the ultimate
substratum, leave only the all-transcending, indefinite and \co{indistinct}.
Indeed, as the \thi{analogical modeling} of God never managed to go beyond the
image of a handyman busing himself with transforming raw materials into more or
less pleasing and useful artifacts, the two -- formal and material cause -- had
to be found also in the \co{indistinct}.  (Attempts to philosophise over the
divine Triunity might easily lead in the same direction of differentiating the
\co{absolute}.)  One had to distinguish the indistinguishable.\ftnt{The story is
  both long and has many turns.
One of the more interesting may be found, as indicated by the above quotations, in Eriugena's
  \btit{Periphyseon}. According to the different ways of 
  predicating being or non-being listed in the opening sections, God is
  nothingness -- but on account of excellence or infinity,
  \la{nihil per excellentiam} or \la{per infinitatem},  while
  matter is nothingness through privation, \la{nihil per privationem}.
  In the platonising school of the
  XII-th century Chartres, Clarembald of Arras had to stress the distinction
  between two senses of the \wo{indistinct} (or \wo{indifferent}): it can be
  understood \citef{in one way as possibility, in another as the unity of
    substance}{ClaremTrinitate}{I:\paras19-23}, i.e., either as the mere
  potentiality of formless matter, or as the ultimate simplicity of God which
  also unites things. Others had to distinguish, for instance, the
  \thi{negatively undetermined}, that is, incapable of being determined
  transcendence of God, and the \thi{privatively undetermined}, that is, the
  general concept of a (created) \thi{being} which is abstracted by the mind but
  is determinable and always exists only in a more definite form
  [\citeauthor*{Ghent}{ XXIV:q6. Henry considered nevertheless the two to be
    completely distinct concepts, confused only by the mind due to their
    similarity.}]; the \thi{simply simple}, the indivisibility which is not
  resolvable into essential elements, like \la{materia prima}, and \thi{not
    simply simple} or \thi{most highly simple} which, as a perfection of unity
  belonging to God, must admit also other perfections. [\citeauthor*{Bona}{
    I:d7.2.q1}, \citeauthor*{ScotOrdinatio}{ II:d3.1.q6}.]}

As we saw in the quoted passages, \thi{matter} is in all
respects like \thi{God} -- only with a huge negative sign making it actually the
opposite of \thi{God}.  The experience of the \co{actual} \co{dissociation} of
\co{subject} and \co{object} and extended to the opposition
\thi{spirit}-\thi{matter} finds the anthropomorphic, in the most negative sense,
expression in ascribing \thi{power}, \thi{activity}, \thi{spirituality} and
\thi{universality} or the like to \thi{God} and, on the other hand, mere
\thi{potency}, \thi{formlessness}, \thi{materiality}, confusion and on the top of
that -- at least until Duns Scotus pointed clearly out the involved
impossibilities\ftnt{\citeauthor*{ScotOrdinatio}{ II:d3.1.q5-6.} A bit earlier,
  \citeauthor*{GhentQ}{ V:8}, objected to viewing matter as the principle of
  individuation 
  and suggested a property of \la{suppositum}, i.e., individual, which however
  was not any positive reality as it were to become with Scotus. It was only a
  negative property and proximate cause of individuation, distinct from matter
  which still was the ultimate (or remote) cause of multiplicity.} -- the
\thi{individuating} potential to \la{materia prima} or, as the alchemists would
say, \la{materia confusa}.
% The consequences which have troubled the whole
% history of philosophy are all too well known to require enumeration.
\noo{Aristotle's' and other's misconceived theories of \thi{individuation}
  effected by imposing \thi{form} on \thi{matter}; the \co{dissociation} and
  opposition of sensual and spiritual life, of bodily pleasure and intellectual
  satisfaction; the needed but impossible rigidity of the \thi{intellectual
    world} which must contain ready-made all the \thi{forms}; dispersion of
  individual being into a \co{complex} of items from various lists of soul's
  assumed \thi{faculties}; etc., etc., etc..  -- The equivocity goes back to
  Plato and is clearly visible in the ontology of Proclus [E.Br\'{e}hier,...
  [Bog-Nicosc, J.Miernowski, p.22[ftnt.19]]].  [\he{Is it an aspect of `negative
    theology'}: Plotinus, Pseudo-Dionysius, Eriugena and then the whole
  tradition of Christian neo-Platonism originating from Areopagite with Eckhart,
  Cusanus, Bovelles, and all others to whom God appeared only (or at least
  primarily) negatively, by not-appearing. The theological aspects of
  Heidegger's fit of course in this tradition well enough.]}

\pa
This dualism projected into the \co{indistinct} carries moral dimension.
Certainly, the calmness of humble and dedicated contemplation can easily be
opposed to the abruptness of sudden passions, the certainty of deep convictions
to the unrest of hollow feelings. In an exaggerated and simplified form, the
goodness of the soul is opposed to the corrupting influences of the body and,
stretching this movement \thi{upwards} and \thi{downwards}, one ends with the
ultimate \thi{Good} on the one hand and the ultimate \thi{Evil} on the other.
As God becomes a mere limit of perfections, the \thi{most eminent},
\thi{more-than-...}, the active supra-Cause, an incomprehensible
\co{totality} of positive aspects and predicates, so there arises a deep need
for the corresponding negative principle, and \thi{matter} fills this need.
\citet{As necessarily as there is Something after the First, so necessarily
  there is a Last: this Last is Matter, the thing which has no residue of good
  in it: here is the necessity of Evil. [...] Matter becomes mistress of what is
  manifested through it: it corrupts and destroys the incomer}{Plotinus}{
  I:1.7,9 [The image of matter as a harlot is also invoked by Maimonides:
  \citef{How wonderfully wise is the simile of King Solomon, in which he
    compares matter to a faithless wife: for matter is never found without form,
    and is therefore always like such a wife who is never without a husband,
    never single; and yet, though being wedded, constantly seeks another man in
    the place of her husband: she entices and attracts him in every possible
    manner till he obtains from her what her husband has obtained. The same is
    the case with matter [...] the substance of dust and darkness, the source of
    all defect and loss.}{Perplex}{III:8}. One should only be very wary of
  concluding from such metaphors total depreciation of body and senses. For
  Plotinus, body is close to matter, but only the latter and not the former, is
  the cause of evil. For \citef{if body is the cause of Evil, then there is no
    escape; the cause of Evil is Matter.}{Plotinus}{I:1.8} Maimonides does
  preach restrain and warns against sensuous pleasures, but he also treats body
  with outermost respect, as witnessed if not by particular texts so by his
  life-long occupation with medicine.  (\citeauthor*{MaimonMatter}, proposes to
  view the indispensability of the material particulars and their epistemic role
  in establishing a contact with the active Intellect, as a positive
  counter-balance to rather harsh and ascetic treatment of the moral and
  soteriological aspect of body in the \btit{Guide}.)]}.
%
\noo{
\citet{For above mind there is only God and below matter (I mean
  body only), there is nothing -- not that nothing which is called so and
  thought to be so because of the transcendence of its nature, but that which is
  conceived and called so because of its lack of all nature.}{Periphyseon}{
  IV:824D/825ABC [p.199]}
%
matter -- not body -- is evil:
\citet{Now this [the required faint image of Being] might be the sensible
universe with all the impressions it engenders, or it might be
something of even later derivation, accidental to the realm of
sense, or again, it might be the source of the sense-world or
something of the same order entering into it to complete it.

Some conception of it would be reached by thinking of
measurelessness as opposed to measure, of the unbounded against bound,
the unshaped against a principle of shape, the ever-needy against
the self-sufficing: think of the ever-undefined, the never at rest,
the all-accepting but never sated, utter dearth; and make all this
character not mere accident in it but its equivalent for
essential-being, so that, whatsoever fragment of it be taken, that
part is all lawless void, while whatever participates in it and
resembles it becomes evil, though not of course to the point of being,
as itself is, Evil-Absolute.}{Plotinus}{I:1.3}

But Matter is not exactly body and sensible world:

\citet{Now this [the required faint image of Being] might be the sensible
universe with all the impressions it engenders, or it might be
something of even later derivation, accidental to the realm of
sense, or again, it might be the source of the sense-world or
something of the same order entering into it to complete it.}{Plotinus}{I:1.3}
}

Certainly, the Orphic-Platonic dualism, Christian ascetism, in the extreme
forms obsessed with \citeti{the flesh lusting against the spirit, and the spirit
  against the flesh}{after Gal.}{V:17}, the suppression of body and senses --
all that may have very particular and \co{actual} reasons representing a
response to real dangers. But the fact that the reasons are real does not, in
and by itself, justify the reaction.  We do not intend to resolve here the
opposition between \thi{Good} and \thi{Evil} but notice that its association
with the opposition between \thi{God} and \thi{matter} is of a very dubious
value. So far, we have not registered any opposition of the latter kind. 
\thi{God} and \thi{matter}, in so far as their ontological characteristics are
concerned, seem to be indistinguishable -- they both function as \co{symbols} of
\co{one} and the same.


Viewing the \co{indistinct} as the place of \co{birth} and the ultimate
\co{origin}, that is, identifying the negativity of \thi{matter} and \thi{God},
we are, perhaps, maintaining a heresy. Pantheism always threatens the back-rooms
of neo-Platonism and its associate -- negativity of the
absolute.\ftnt{Usually not because it is there, but because it can be read
  there. Theological reservations against aspects of Pseudo-Dionysius (in the
  times before the first serious doubts concerning his claimed identity as the
  convert of St.~Paul from Acts XVII:34, Dionysius the Areopagite, were raised
  by Laurentius Valla in the XV-th century.) concerned 
  the sense of creationism and possible pantheism. It seems (as far as we know
  from few remarks and the records of condemnation of now lost texts) that David
  of Dinant identified God, Demiurge (intellect) and matter \citef{by arguing
    logically on the \thi{God non-being} of Denis and Erigena. God is non-being,
    matter is non-being, therefore God is matter or matter is God.}{GilsonHCPh}{
    VI:1;p.242} The identification was called by Bett \citef{a reckless
    development of Erigena's doctrine}{adEriugena}{\citaft{adEriugena2}{}} and,
  indeed, it seems that condemnation (of \btit{Periphyseon} in 1210 and 1225)
  was based not so much on a thorough consideration of the text as on the {\em
    mere possibility} of extracting from it elements of pantheism. Statements in
  \btit{Periphyseon} II:546B/545C which might suggest pantheistic reading, are
  pronounced by the \la{Alumnus}, are discussed earlier in part I, and are
  refuted almost immediately by \la{Nutritor}. In general, Eriugena is rather
  clear on the double aspect of divinity. \citef{[...] God is both beyond all
    things and in all things [...] and while He is whole in all things He does
    not cease to be whole beyond all things, whole in the world, whole around
    the world [...]}{Periphyseon}{IV:759a-b\kilde{StanfordEncPhilos}} \citef{\ 
    `And so God is all things and all things are God.' Such a judgment will be
    regarded as monstrous [...]}{Periphy}{III;p.162} The favourite phrase
  concerning God as well as any substance, that \wo{we can only know {\em that}
    it is, but not what it is}, fits well precisely to the \co{indistinct} as
  such, whether transcendent God or inaccessible \thi{matter}. Pantheism was
  read also into the texts of the Chartres school, though the accusations can be
  met plausibly as was done, for instance, in \citeauthor*{GilsonHCPh}
  IV:3.2.~footnote 78. \noo{p622,ftnt.78, ftn.83}} We do not, however, propose any
pantheism. Neither do we {\em identify} \thi{God} with \thi{matter}: the latter
simply has no significant meaning and there is only \co{one indistinct} which is
not identified with anything. Thinking of it in any specific way is already a
mistaken projection and identification of God and matter is a resulting
antinomy.\noo{or an indication of genuine indistinctness transcending any (such)
  distinctions.}
For the time being we will take the risk of
offending some theological sensibilities and put the issue to rest. It will
return again in Book III, while more detailed remarks on pantheism will be made in
\ref{sub:pantheism}.
%
% -- namely, that, as the sphere beyond \co{distinctions}, there is no need to
% distinguish here \thi{God} and \thi{matter}, or for that matter anything else,
% since here there are no \co{distinctions} at all.


\sep


\pa Reflection is driven by a hunger, it searches. For the truth? For a totality? For 
God? For its own eradication? To begin with, 
it does not know. Goals remain hidden until they are reached.
% \citet{What seems at first a cup of sorrow is found in the end immortal
%   wine.}{Bhagavad}{XVIII:38} moved a bit forwards

%\subsubi{Antinomies and reflection}

\noo{
\pa\label{antinomies} In the most abstract formulation, antinomies arise by an
application of the principle of excluded middle to the levels where it
does not apply.\ftnt{Except for the fourth thesis, all the proofs
  of thesis and antithesis in Kantian antinomies are carried by
  \la{reductio ad absurdum}.}  We will say a few words about the
\thi{logics} of different levels later on, so here let us only state
that the absolute opposition, the contradiction between $x$ and
not-$x$, is the lowest, ultimately rigid and \co{precise} form of
\co{distinction}. It applies to the \co{objective} thinking based on
the \co{dissociation} of \co{experience} into independent
\co{objects}, where $x$ has been  \co{precisely cut} out and stands in its
isolated \co{immediacy} in an
insuperable opposition to everything else which it is not, to not-$x$.

But contradictions do not apply, do not exist at the higher levels.
\co{Experience}, no matter how orderly and controlled, borders on chaos and, in
fact, {\em is} \co{chaos}. The \co{chaotic} multiplicity, in turn, {\em is}
\co{one}.  In the language (though not the categories) of rigid \co{reflection}:
$x$ {\em is} not-$x$, because this \thi{is} indicates the anchoring of
everything in that from which it arises and which, in terms of the lower level
at which $x$ dwells, is {\em not} the same.  Similarly, \co{virtuality} of
\co{one} exists only through the \co{chaos} of many, and \co{chaos} is only
\co{virtual nexus} of \co{experience}.
There are well-known examples which one might want to model as contradictions,
although they are not.
I may like and dislike the same person at the same
moment, I may be attracted and repulsed by the same thing.
% USED in \refs{sub:contradict}
% \co{I} can simultaneously both love and not-love a person, the whole person.
% \citt{Odi et amo. Quare id faciam, fortasse requiris.// Nescio, sed fieri sentio
%   et excrucior.}{(I love and hate. How does it happen, you may ask.// I do not
%   know, but I feel, suffer and go mad.) Catullus, \btit{Poezje, Pie\'{s}ni} 85
%   [za Herbert, Labirynt nad morzem, p.196of197]}
%
Looking for \thi{real reasons}, \thi{deep causes} of such co-occurence of
\thi{contradictory} feelings in the differences which hide behind the surface,
is merely a work of \co{reflection} which can not stand contradictions. The
emotional blindness resulting from such a search should suffice to prevent one from
dwelling on it.
}

It might seem that \co{reflective} thinking is doomed for dwelling in
its antinomies. On the one hand, to leave \co{subjectivity}, to entirely forget
\co{objects} -- in order to traverse the \co{distance} separating the two and
achieve an ecstatic union -- is impossible. Drugs
pacify only for a moment. \co{Reflection} will always be aware
of this table, of that tree, of any \co{object} as distinct from itself.
% I may doze off, fall asleep or under a spell of a wonderful performance or a
% powerful feeling. (I may also be exposed to an accident, fall into coma, but we
% do not have to speak about pathological cases.)  But to believe that I may
One can not get dissolved in an ecstatic unity of the \co{indistinct} and still
be oneself. Such a dissolution, abolishing the \co{separated} terms, amounts to
impossibility not only of thinking and feeling, but of any form of relation
whatsoever. In short, it amounts to a new form -- perhaps universalised, perhaps
depersonalised, but still only a form of -- solipsism or, in more pathological
cases, of escapism. It helps little to pronounce \thi{the end of the subject},
\thi{the end of discursive thinking}, the end of whatever one feels does not
suffice, hoping that thus one will reach the \thi{otherness}. Otherness, like
any relation, presupposes \co{distance}. To \co{exist} means to be
\co{confronted} with the \co{non-actuality} 
of \co{experience}, with \co{chaos} and the \co{indistinct}; in the most \co{actual}
form, this \co{confrontation} is expressed as the \co{reflective distance} to
the \co{external object}. 

On the other hand, the projects of \co{reflective} reconstruction or conquest
are, as it seems, doomed to failure. Perfect \gre{mimesis} (whether in the
artistic form of ancient sculpture or academic painting, or else as the
scientific fantasies of doctor Frankenstein, AI, robotics or genome research)
appears, indeed, as one of the strongest driving impulses. It is, however, only
an expression of the \co{reflective thirst} for the coincidence with -- by the
re-creation of -- the original truth. This original, however, vastly
\co{transcends} the perspectives of \co{objectivistic illusion} and its
possibilities. Consequently,
neither any absolute conquest is to be expected. For such a conquest requires
reduction of whatever \co{transcends} the \co{actuality} of \co{reflective acts} to
the \co{signs} which, nevertheless, can be grasped within the \hoa. In short, it
requires reduction and, as the higher levels are not accessible in terms of
lower \co{distinctions}, the reduction can never happen to be complete.
Collecting the building-pieces and putting stone next to stone never finishes.
And \citeti{[t]he stone which the builders refused is become the head stone of the
  corner.}{Ps.}{CXVIII:22}
%{[t]he stone thrown away by the constructors becomes the spark of a new fire.}

The suggestions of dissolution, just like the search for \co{objectivity},
reflect the \co{thirst}, the search for intimacy with reality.  In either form,
such \co{reflective} projects aim at abolishing the \co{distance} between the
\co{actual} and \co{non-actual} which, eventually is the \co{distance}
constituting the very \co{reflective} being. Abolishing it, even if it were
possible, could not satisfy \co{reflection} leaving it alone in a solipsistic
universe. 


\pa Being a \co{subject} of \co{reflective} thinking, one is also much more than
that.  \co{Reflection}, one's \co{subjectivity}, is not doomed to suffer for the
reason of involvement in \co{actuality} which, after all, is its
constitutive feature. \co{Externalisation} is not the same as
\co{alienation} in the middle of an estranged world. It becomes so only under
the spell of the \co{objectivistic illusion} which absolutises the \co{actual
  dissociations}, in particular, of \co{subject} and \co{object}.

\co{Concrete reflection} is still \co{reflection}, it still operates with
\co{actual} \co{distinctions}. But there is a big difference, even if no sharp
border, between the two
modes of \co{reflection}. One retains the \co{signs} trying to hold on to its
\co{objects}, attempting to conquer time and stop the flow by the spells of its
\co{objectivistic illusion}. The other trades control for enjoyment and, merely
noticing, allows things merge back into their element.  It does not absolutise
the \co{actuality} of its \co{signs}, it does not create an \co{idol} from its way of
thinking, from the \co{externality} of its \co{objects} and the associated
\co{precision} of the most rigid \co{distinctions}.  Admitting its situation, it
admits only its own nature; instead of the impossible attempts to abolish the
\co{distance}, it simply acknowledges it. Only \co{distance} makes a relation,
and hence community, possible. And to keep the \co{distance}, one has to be
oneself, one has to respect oneself, also in the midst of one's openness,
receptivity and new encounters.  Although reflecting person is aware of
something more than the \co{actual} \co{object}, \co{reflection}'s ability does
not extend beyond it. It does not even extend beyond the \co{sign} under which
the \co{object}, or whatever else might be, appears.  The surrounding
\co{invisibility} can be made \co{present} through the \co{signs}, but never
enslaved.  Admission that its only power is over the \co{signs}, its \co{actual
  objects}, won't make \co{reflection} impotent.
% but, on the contrary, will allow it to use \co{signs} also for higher things
% than mere \co{objects}.
On the contrary, like all true humbleness, it makes stronger, that is, more
real. Goals remain hidden until they are reached -- \citet{What seems at
  first a cup of sorrow is found in the end immortal wine.}{Bhagavad}{XVIII:38}

\noo{\co{Reflection} becomes satisfied not when nothing remains \thi{outside} --
  most things always do -- but when what remains \thi{outside} does not threaten
  it any more.  The outcome need not, can not, be an all-encompassing
  \co{totality}. But it has to include the \co{reflection} itself.
  \co{Reflection} will not rest with a theory presenting the truth of
  pre-reflective being in which the \co{reflection} itself is only an
  inessential add-on, a methodological tool.  Once a person acquires a
  reflective attitude, it is hard, if at all possible, to dispose of it.
  Henceforth, \co{reflection} will accompany the most significant moments.
  Consequently, it cannot be satisfied with a result where it is not taken into
  account.  }

\noo{An analysis may approach the quality of common experience not by trying to
  dispense with its own, reflective activity but, on the contrary, only by
  taking reflection into account as part of experience.
  
  Such activities, however, belong to \co{reflective experience}, to the
  \co{experience} which is already \co{reflectively} differentiated into
  separate \co{experiences}.  \co{Attentive reflection} does nothing more than
  pulling out aspects of such \co{experiences} into even more narrowed focus of
  attention -- it merely \co{reflects} them.\ftnt{In a technical,
    mathematical sense, a mapping $r$ from a set $E$ to a set $T$ \thi{reflects}
    a property $P$ if, whenever the image $r(e)$ of a point $e\in E$ has the
    property $P:P(r(e))$, then so does $e:P(e)$.  In our case, $r$ can be
    thought not as a total function, but as a mere isolation of a subset of $E$.
    It thus forms a new level, but it \thi{reflects} all the properties of all
    the individual $e$'s which belong to it.  It does not, however,
    \thi{reflect} the properties of the whole subset.  (E.g., $E$ may be
    infinite, while $r$ extracts only finitely many elements from $E$.)}
%
  The important question is therefore not so much how a \co{reflected object}, a
  \co{representation} is related to its original (the question which hides in
  itself both the feeling of failure and alienation and the thirst for the pure
  -- and lost -- original), but only how the \co{reflective} separation, the
  attitude of \co{attentive reflection} can or should relate to the
  \co{experience} from which it arises.
  
  Hence, the aim of \co{reflection} is not to \co{re-construct} something which
  lies outside it. The aim is to \co{construct} its own world, to organise it
  around the \co{experiences} carrying particular weight and meaning for the
  \co{reflective} mode of being.  Sure, the \co{construction} has to take into
  account the rest of one's being -- for \co{reflection}, this means the
  \co{experience} from which it arises. But taking into account is not the same
  as \co{re-constructing}.
  
  Instead of getting rid of itself and \thi{thinking-otherwise}, reflection may
  do better by just ... staying sober, that is, concrete.  It won't rest
  satisfied in an ecstatic union with otherness because such a union excludes
  the very possibility of reflection. Yet, it may find some rest by finding its
  place -- to illuminate its place in relation to what is \co{inside} and what
  remains \co{outside}, to what is \co{visible} and to what is beyond its
  control.  }


\subsub{Two modes of \thi{givenness}}\label{sub:givens}

\pa
We have made occasional references to \thi{empiricism}. Let us explain briefly 
the intention behind such  
references. 
By \wo{empiricism} we mean, generally and rather indiscriminately,  
application of the principle that \citet{all things that exist are only
  particulars}{LockUnd}{IV:3.6}, that 
%
\begin{quote}
whatever can be distinguished {\em is} distinct; eventually, there are only
mutually distinct, but inherently indivisible and simple \thi{atoms}.
\end{quote}
%
Obviously, this comprises much more than what is traditionally called
\wo{empiricism}.\ftnt{\citef{Berkeley's nominalism, Hume's statement that
    whatever things we distinguish are as \thi{loose and separate} as if they
    had \thi{no manner of connection}. James Mill's denial that similars have
    anything \thi{really} in common, the resolution of the causal tie into
    habitual sequence, John Mill's account of both physical things and selves as
    composed of discontinuous possibilities, and the general pulverisation of
    Experience by association and the mind-dust theory, are examples of what I
    mean.}{PureExp}{I} But also the radical empiricism's statement that it
  \citefib{must neither admit into its constructions any element that is not
    directly experienced, nor exclude from them any element that is directly
    experienced}{PureExp}{\hspace*{-.6ex}} qualifies -- by narrowing the focus to the
  \co{actuality} of direct experience -- for inclusion under our heading of
  empiricism.
  
  Other examples abound, as do possible distinctions. In the \thi{Suspended Man}
  passage, Avicenna says something like \wo{If it is possible to conceive $x$
    without $y$, then $x$ and $y$ are {\em really distinct}; each has its own
    being independently from another.} We do not discuss whether \wo{conceiving
    $x$ without $y$} means conceiving consistently, admitting only logical or
  also real possibility, merely \thi{conceiving without} or, perhaps,
  \thi{conceiving $x$ without {\em conceiving} $y$}, etc. -- we only register
  the tendency.  Arguing against Scotus' formal distinction between an
  individual nature and its individuating difference, \la{haecceitas}, Ockham
  contends: \citef{If, therefore, some kind of distinction exists between this
    nature and this difference, it is necessary that they be really distinct
    things.}{OckSumLog}{I:c.xvi}} In fact, \wo{realism} might be almost equally
good label, in so far as it expresses the conviction about the real --
independent -- existence of particular things. Of course, only {\em in so far},
and not in the sense it obtained during the discussions about the universals. As
witnessed for instance by the example of Ockham, \wo{nominalism} may also fall
under this heading as it insists on the exclusive reality of dissociated
individuals, for which \wo{atomism} might be an alternative name. The tendency
culminates perhaps in \wo{phenomenalism} with its epistemologically motivated
attempts to find the irrefutable certainty in the ultimate immediacy of the
simple data.  We may occasionally use such other labels, but in every case it is
the above principle which underlies the more specific meanings of various names.


From our perspective it seems namely natural to distinguish this general
tendency from another one which, for the time being and in the lack of a better
term, we will call \wo{idealistic}. It recognises that \citet{what in empirical
  science are called {\em data}, being in a real sense {\em arbitrarily} chosen
  by the nature of the hypothesis already formed, could more honestly be called
  {\em capta}. By reverse analogy, the fact of [for instance] mathematical
  science, appearing at first to be arbitrarily chosen, and thus {\em capta},
  are not really arbitrary at all, but absolutely determined by the nature and
  coherence of our being.}{LawsForm}{Introduction;p.xxvii} Roughly, this
opposite thesis would be:
%
\begin{quote}
whatever can be unified {\em is} one; the highest unity is reflected 
in the laws, structures and organisation 
of the manifold of differentiated contents of experience.
\end{quote}
%
Both theses are stated in an exaggerated form and probably no philosophy has
ever been built exclusively around any single one of them.  Often, they might be
present within one and the same system of thought. Also, they can appear in more
ontological or more epistemological variants (the latter would go in the
direction of replacing the \wo{{\em is}} with \wo{is to be considered}).  We
certainly do not intend to play the rough terms of such a crude opposition
against each other -- the actual question for any representative of any one of
them is how far the application of the principle can be pushed and where it
must stop.  They only try to capture some general tendencies which can  
be discerned either in the totality of a particular system, or else in some of
its aspects.  The empirical tendency is to look for the ultimate atoms (whether
in nature or in experience) and build everything in a bottom-up manner from
these. (This comes close to our \co{reflection}.) The idealistic tendency is to
view experience as perhaps differentiated, yet in itself hardly organised matter
which receives structure only, or primarily, through the higher laws, as if in a
top-down process of confrontation of the unifying forces (whether of mind or
nature) with the lower manifold.

Materialism and idealism can often represent the (ontological) extremes of,
respectively, empirical and idealistic tendency.  Matter has to be seen as
discrete in order to provide an empiricist with the building blocks for higher
structures.  For an idealist, on the other hand, it may easily become an
undifferentiated \la{materia confusa} which obtains a structure only from the
constitutive laws of spiritual or transcendental character. (Recall figure
\refp{pa:equfm}.\refeq{equ:fmKant} -- its dual could represent materialism.)

\pa Both tendencies share one basic assumption, namely, that the lowest data of
experience are, in fact, differentiated.  This originates from the conviction
that at the bottom of everything are \co{immediate} sensations and that
experience is eventually what is or can be sensed, i.e., from the Aristotelian
principle that nothing obtains in the intellect unless it was prior given to the
senses.\noo{\la{Nihil est in intellectu quod prius non fuerit in sensu}. This
  assumption is itself an expression of complete bias of \co{reflection}}
Empiricism models this as some basic ideas, impressions, perceptions or other
atoms assumed {\em given} in \co{actual experiences}.  But to access and
manipulate it, one has to postulate more than mere sensations. In fact, it seems
that the ultimate atoms of sensations are needed only in order to specify mental
structures which enable one -- subject -- their arrangement and control.
Rationalism takes over roughly at this point, postulating some \thi{mental}
atoms, clear cognitions or precise intuitions which, at a closer look, also come
from experience, though not (exclusively) from sensations or perceptions.  On
the other hand, it will almost always speak about \thi{substances}, some
unitary, even if complex entities, which empiricism attempts to dissolve in, and
then reconstruct from, the flux of its atoms.  Thus in either case, the
\thi{givens}, ideally of purely sensuous but typically also of some mental
character, are there as the starting points -- the starting points for
philosophical \co{reflection} and, as it always turns out, for whatever this
\co{reflection} is trying to describe.  In particular, the \thi{givens} have
thoroughly \co{actual} character tending towards pure \co{immediacy}.

This assumption of \thi{givens} has deeper roots. It results from the posited
difference between two primary poles: higher and lower, unified and dispersed,
organised and chaotic, one and many. Or, in more familiar terms: subject and
object, mind and body, spirit and matter.  It is this initial dualism which
requires some differentiated or even structured \thi{givens} in order to allow
the two \co{dissociated} poles meet again.  As we have seen, when pushed to the
metaphysical limits, both dissolve in the homogeneity of indistinctness.

The problems with specifying the \thi{ultimate givens} make us attempt to speak
in a way relatively independent of what, possibly, might be considered to be
such \thi{ultimate atoms}.  This Book has thus described the process which,
starting with the \co{absolute indistinct}, leads to the very appearance of
\thi{givens} in the sphere of \co{actual experiences}.  The fact that
\co{reflection} and its \thi{given}, \co{dissociated} contents are only result
of this process, will not make us imply that they are in any sense \thi{unreal},
dispensable or arbitrary.  Appearing for \co{reflection} does not mean to be
constituted, not to mention, created {\em by} it. \thi{Givens} are given exactly
because they are discovered and not constructed. Only that this discovery, even
when true, is not \co{absolute}. It is relative to the discovering
\co{existence} as well as to the discovered contents.

\section{In a few long words...}
%It rests by changing. -- DK B84
\secQ{The waking have one common world, but the sleeping turn aside each into a world
of his own.}{Heraclitus, DK 22B89}
% \secQ{[The change of opposites is] a path up and down, and the world 
% is generated in accordance with it. For fire as it is condensed 
% becomes moist, and as it coheres becomes water; water as it 
% solidifies turns into earth -- this is the path downwards.}{From 
% Barnes, {\em Early Greek Philosophy}, p.107 [Heraclitus, p.33]}
This section was supposed to summarise the development in the current
Book. However, it has 
expanded beyond the limits of a reasonable summary. So, probably, 
it is a bit more than that...

\subsection{Separation}
\pa\label{pa:theonly} \co{Birth} is \co{separation} from the \co{origin}, from
the \co{indistinct nothingness}. It is the fundamental, in fact, the only
ontological event, the first hypostasis.  \co{Separation} is not alienation, it
does not establish a being which is now exiled and doomed to loneliness. On the
contrary, only \co{separation} makes it possible for a being not to be alone.
Only by being \co{separated} can a being be confronted with something else,
something it is not. \citet{To be united is divine and good; so whence this
  obsession\lin Among people, that only one and oneness should
  exist?}{Holderlin}{} The original \co{separation} establishes the ultimate
\co{transcendence confronting} the \co{separated} being, and \co{existence} is
this very \co{confrontation}.

\pa In fact, \co{separation} is a generic concept comprising the structural
similarities discernible in all subsequent hypostases -- the second one from the
\co{originally separated existence} to the \co{distinctions} of the limitless
\co{chaos}, the third one from \co{chaos} to the \co{recognisable} world of
\co{experience}, and finally, from \co{experience} to the {objectified} world
of \co{reflective experiences}.

At each level, \co{separation} happens against the background of the previous
level, and it happens through the emergence of more \co{actualised} \co{aspects}
distinguished from the \co{nexus} of the preceding level. The emerging
\co{actuality} is now confronted with the background from which it emerged and
which recedes into \co{non-actuality} and, on the other hand, with the
non-\co{actuality} of \thi{the rest} of the things \co{distinguished} at the
same level.  This event establishes a new level; the background withdraws giving
place to a new differentiation which will become the background for the next
stage.

\co{Separation} creates a \co{distance} which is manifested through \co{signs}
(except for the very first stages, where it is a mere \co{signification}).
\co{Sign} points primarily to the background from which it has been extracted,
and secondarily to the signified \co{distinction}, eventually, to the
\co{external} \co{object}.  A \co{sign} has thus always this twofold direction
of pointing towards something and also towards the background of this something.
In the Heideggerian language we could say that it reveals as much as it
\thi{hides}, it brings forth the \co{actual} and, by this very fact, it
\thi{veils} the \co{non-actual} background of this \co{actuality}.  \co{Sign}
relates not only to the signified but is also the means through which the
\co{separated} being \co{recognises} the limits of its \co{actuality} and thus
confronts the \co{non-actuality} surrounding it.


\pa \co{Transcendence} is what exceeds \co{actuality}, and \co{signs} are the
means by which the latter is confronted with the former.
% \co{Non-actuality} is what we call \wo{\co{transcendence}}.  By means of a
% \co{sign}, \co{actuality} is confronted with \co{transcendence}.
\co{Signs} are given tokens, terminal points of \co{traces} stretching,
eventually, from the \co{transcendent origin} to the midst of \herenow. Although
\co{signs} are purely \co{actual}, \co{dissociated} ..., their \co{traces} carry
nevertheless the \co{non-actual aspects} which thus remain \co{present} around
the \hoa. \co{Non-actuality} does not exclude \co{presence}, and so
\co{transcendence} does not mean any absolute isolation of the poles which one
might attempt to overcome on the way towards some coincidence. It means the
\co{distance} which, separating the poles, is a necessary condition of any
relation between them. As Merleau-Ponty says, both search for ultimate, exact
essences {\em and} attempts to achieve coincidence (of subject and object) are
failed, in fact, misunderstood ways of accounting either for experience or even
for the philosophical project. For \citet{every being appears in a distance,
  which is not an obstacle for acquaintance [\ger{Erkenntnis}], but on the
  contrary -- makes it possible. [\ldots] Independently whether one claims
  infinite distance or absolute proximity, negation or coincidence -- our
  reference to Being remains equally unrecognised.  [\ldots] One forgets that
  this frontal being before us, whether constituted by us or constituting itself
  in us as being constituted, is in principle secondary, is cut against a
  horizon which is not any nothing and which is not itself through any
  co-constitution.}{VisInvis}{Inquiry and intuition;p.133.\noo{ As is always
    the case with Merleau-Ponty, we would make serious reservations concerning
    his almost exclusive focus on body and sensuous experience.}}

Primarily, this \co{distance}, the constant and only implicit \thi{reference to
  Being}, is the \co{vertical}, qualitative \co{transcendence} of the
background.  Differentiation makes the background withdraw but not disappear; it
acquires a new character of differentiated, and ideal, \co{totality}.  Trying to
account for the background from a lower, eventually, \co{reflective} level, one
naturally projects into it the character of the \co{actual} objects.  This gives
rise to another form of \co{transcendence}, the \co{horizontal}, quantitative
\co{transcendence} of the signified -- a mere \co{more} of objects which are not
\herenow\ but \thth, \thi{outside actuality}.  At the level of \co{reflective
  experience}, this was identified as \co{objectivistic illusion}.

\pa \co{Separation} is not a simple, mechanic dissociation; creating
\co{distance} it \co{confronts}, and thus is also \co{self-awareness}.  To be
confronted means to encounter \co{transcendence}.  Encountering something
distinct is, analytically, inseparable from encountering oneself: confronting
the \co{transcendence} of \thi{...} means the \co{awareness} of \thi{...} being
distinct from oneself, {\em eo ipso}, the \co{awareness} of oneself being
distinct from \thi{...}.

This structure is present from the very beginning, from \co{birth}; 
% \co{Separation} {\em confronts} the poles it \co{separates}. This
% \co{confrontation} makes up the immediate \co{self-awareness};
%
the only aspect varying from one level to another is the degree of its
sharpness. The primordial ontological event, \co{separating} a being from the
\co{nothingness}, is the ultimate site of individuality, \la{haecceitas}.
\co{Distinction} confronts this being with the \co{chaos} of \co{distinctions}
-- the individuality acquires \co{(self-)awareness} of own finitude.
\co{Recognition} confronts \co{actuality} with \co{experience} --
\co{self-awareness} becomes a more confident feeling of predictability, of being
at home in the world and, at the same time, of not being the world. Finally,
\co{reflection} confronts one with its \co{object}, with \co{an experience} in
which \co{self-awareness} \co{founds} the possibility of \co{dissociated acts}
of \co{self-reflection} which, focusing on the internal, \co{subjective aspect},
reach towards the \co{external} \co{objects}.

This is the general structure of \co{separation} and, in fact, the totality
of the concept of \co{experience}. 
% 
% Of course, the concrete character of the emerging relation is determined by the
% background from which it was \co{separated}.
%
As we have seen, the successive stages bring about a gradual refinement of the
\co{distinguished} contents and of the character of \co{actuality}. The further
we proceed, the more definite become the objects and the more involved into
spatio-temporality. This involvement is only a side-effect of the fact that also
spatio-temporal \co{aspects} became \co{distinguished}.  As the
\co{distinguished} contents acquire sharper boundaries leaving more and more
layers of \co{non-actuality} behind, the \co{actuality} itself becomes more
clearly \co{dissociated} and, by the same token, confronted with
\co{externalised} \co{objects}.  At the level of \co{reflective experience},
this results in discovering the \co{objective} order of the world organised
along the complementary dimensions of the \co{objective} time and space.

The \co{confrontation} with \co{transcendence}, the uniqueness of the
event of \co{birth} and, consequently, of the whole \co{experience},
is what constitutes \co{unity} -- not \co{totality} -- of \co{existence}.  It is
individuality irreducible to any \thi{this} or \thi{that}, \thi{why}
and \thi{how}.  It is individuality established by the primordial
ontological event, by the first hypostasis; individuality which, in
the face of the unique world, stretches beyond its horizon, as well as
beyond the horizon of time and temporality.  What makes up this
singularity, \la{haecceitas} is \citet{neither matter nor form nor the
composite thing [...]  but it is the {\em ultimate reality of the
being}.}{ScotOrdinatio}{II:d3.q6 \noo{no.15 : Opus Oxoniense}} We will 
eventually return to this aspect, in II:2 and in III. 


\subsection{One -- not Many}\label{subsec:OneMany}\label{sub:OneMany}
%
\co{Chaos} is a pure manifold, where \wo{pureness} denotes the lack of any
internal relationships, the mere heterogenity of \co{distinctions}.  One might
claim that it is easier to imagine than the \co{nothingness} of the \co{one}.
Why do we start with the \co{one} then?  Why not start, as many would, from pure
manifold?  Why Parmenides and not Heraclitus, perhaps, Plato rather than
Aristotle, Spinoza rather than Descartes, monotheism and not polytheism,
%in fact, deism not pantheism, 
idealism and not realism?

\pa\label{pa:IndistinctOne} For the first, \co{positing} chaos of differences is
an act of exactly the same kind as \co{positing} the \co{one}.  It is neither
easier nor harder, it is neither closer to experience nor further from it.  It
is an \co{act} of \co{positing} -- in one case, of the \co{unity}, in the other
of the \co{totality} reaching beyond the \co{experience}.  If we accept
\co{indistinctness} of the \co{one} as the \co{origin} of \co{distinctions},
then there is hardly even a possibility left for claiming any primordial
plurality -- any plurality requires a \co{distinction}, and any \co{distinction}
involves plurality, \citet{for all distinct things are two or more, but all
  indistinct things are one.}{EckWisdom}{(Ws.~VII:27a) \citaft{Eckhart}{
    p.167\label{ftnt:IndistinctOne}} Indistinctness appears as the univocal
  concept of Being already with \citeauthor*{ScotOrdinatio} I:3.1.2.44,57;
  II:1.4-5.15 \citaft{FilSr}{ p.232}} \thi{Two \co{indistincts}} is an
impossibility, and we do not even need Ockham's razor to complete the
traditional proof of the uniqueness of the first beginning.

On the other hand, \citet{[a]ll multitude participates in a certain respect of
  The {One}.  For if it in no respects participates of The One, neither will the
  whole be {\em one} whole, nor each of the many of which the multitude
  consists; but there will also be a certain multitude arising from each of
  these, and this will be the case to infinity. [...] All multitude is posterior
  to The One.}{Proclus}{\para 1/\para 5 [my emph.]; also \para 69. \citef{But
    there must be a unity underlying the aggregate: a manifold is impossible
    without a unity for its source or ground, or at least, failing some unity,
    related or unrelated. This unity must be numbered as first before all and
    can be apprehended only as solitary and self-existent.}{Plotinus}{V:6.3,
    also VI:6.11.} In a phenomenological version: \citef{Every genuine
  irreducible <<sphere>> of being is an eidetic unity which is {\em given} as a
  <<background>> {\em before} positing reality [<<Realsetzung>>] of any entity
  which is possible within it and, consequently, it does not form a mere sum of all
  accidental facts.}{Sympatia}{C:II\kilde{p.354}}}

One can not \co{posit} pluralism without \co{positing} a pluralistic {\em
  universe}.  Examples might be multiplied \la{ad infinitum} so let us quote
only a few.  \co{Positing} a chaos of differences one is forced, sooner or
later, to perform the common trick -- start talking {\em as if} it were one.
\citet{{\em The collection of all} effects is itself an effect; hence it is an
  effect of a cause which in no way is an element of this
  collection...}{ScotPrimo}{3.13\label{ft:JDS} [my emph.; Aquinas, with his
  insistence on the difference between the events and mechanisms acting
  \thi{within the world} and acts of God addressing \thi{the totality of the
    world}, might be quoted extensively here, too.]} Or else \citet{Eternal
  Return is the being of {\em this world}, the only The Same which can be
  predicated about {\em this world}, excluding from it any prior
  sameness.}{DeleuzeDiff}{p.338. [my emph.]  This Eternal Return is, perhaps,
  not exactly the same but still very similar to our \co{chaos}.}
%
James' radical empiricism, with its generous pluralism, starts with something
like: \citet{Nothing shall be admitted as fact, except what can be experienced
  at some definite time by some experient;}{Radical}{VI;p.160} But very shortly
after we read: \citetib{though one part of our experience may lean upon another
  part to make it what it is in any one of several aspects in which it may be
  considered, {\em experience as a whole} is self-containing and leans on
  nothing.}{Radical}{VII:1;p.193 [my emph.]} At what \thi{definite time} is
this \thi{experience as a whole} experienced?  OK, let us admit that there may
be special \co{actual experiences}, as rare as they are important, which seem to
address the whole -- of our life, of our world, of the world.  But is \thi{all
  my life} experienced only at some \thi{definite times}?  Is it not also
experienced all the time, as if, underneath the particulars which furnish and
exhaust the contents of \co{actual experiences}?  If it is only a matter of
particular \co{actual experiences} then what makes just these ones so special,
so much more important than all the other \co{actual experiences}? It seems, it
must be their content, not mere \co{actuality}. It seems that they address
something special, very special, something which is in fact the aspect along
which and around which all the \co{actual experiences} are structured. \co{Any
  experience} is only a part, an aspect of one's life.  But a pluralist has
always tremendous problem with accounting for the fact that not all experiences
are of the same order, not all are pieces of equal value. He, too, is bound to
end up with some kind of structure, some \thi{totality of all causes}, some
\thi{Eternal Return as the only The Same}, or what not, which either is
\co{posited} implicitly and sneaks in through the back-door, or else leans
towards mere associationism (what else could it be in a universe of essentially
equivalent pieces?). Or he may turn a bit more rationalist and start discerning
some structure in the experience itself, but then he is already on the way out
of \co{chaos}\ldots

\citetib{Since the acquisition of conscious quality on the part of an experience
  depends upon a context coming to it, it follows that the {\em sum total} of all
  experiences, having no context, can not strictly be called conscious at all.
  It is a \thi{that}, an Absolute, a \thi{pure} experience on an enormous scale,
  undifferentiated and undifferentiable into thought and thing.}{Radical}{IV:4;p.134} James, though methodologically biased by his pragmatism and empiricism,
remains nevertheless always honest and acute in his descriptions.  It is not the
lack of intuition but the ghost of a \thi{pluralistic} universe of
\thi{experiences} which -- confronted with the emptiness, unconsciousness and
apparent pragmatical irrelevance of such a \thi{totality} -- deters James from
deeper consideration, allowing him to confuse it with the \thi{Absolute} and to
rest satisfied with a mere phenomenological description of religious
experiences.
  
  \noo{ \ftnt{\citeauthor*{Varieties}{} does not bring us out of the pragmatic
      methodology.  Like phenomenology of religion, it treats wonderfully about
      what is indicated in the title, but it is just that: actual experiences.
%\citf{The higher Presence, namely,need not be the absolute whole of
%things, it is quite sufficient for the life of religious experience to
%regard it as a part, if only it be the most ideal part.}{VI\&VII. The
%Sick Soul, footnote 4} The\wo{whole of things}, or a part of it, is
%as far as one gets. But unity is not the same as totality. 
      In the Conclusions, there is a modest attempt to recognise a sphere of
      religion as lying \co{above}, yet pragmatism \citf{makes it claim, as
        everything real must claim, some characteristic realm of fact as its
        very own.  What the more characteristically divine facts are, apart from
        the actual inflow of energy in the faith-state and the prayer-state, I
        know not.}{Conclusions;p.519} Intentions notwithstanding, there is
      hardly anything in the intellectual baggage of James' pragmatism which
      could admit religion as anything more than merely \thi{some characteristic
        realm of fact}.  We will return to these questions in the last Book;
      here I only wanted to indicate that common problem of turning a
      multiplicity into a totality, even unity, is present in pragmatism, too,
      even if an excellent work like `The Varieties\ldots' might make one think
      that it was not.}  }


\pa In short, considering the chaos of differences, the plurality of
distinctions as primordial, one still has to turn it into a \thi{one}, even if
one resists making it \co{one}.  Speaking about \thi{pure manifold} one is
already speaking about the \co{one}, one \co{posits} the totality of
\co{distinctions} as something one wants to speak about - \citet{for you cannot
  conceive the many without the one.}{Parmenides}{} But this is merely a
necessity of speaking, one might say.  Yes it is, but it is also what we are
doing -- speaking.

If one denies the \co{unity}, and claims merely the \thi{totality of all
  differences}, then what makes one so inclined to turn it into {\em a}
\thi{totality}?  Is it only because the listeners demand something like that? It
may be just the way we speak and use our language, but such \thi{therapeutic}
gestures have hardly any serious appeal.  Totality simply can not be thought
without a unity, the idea of unity is a priori condition for the idea of
totality. One might nevertheless insist that this concerns only the order of
ideas and thinking, but that \thi{out there}, \thi{in reality}, things are
actually other way around, scattered and independent from each other, without
any unity except of being placed in ... well, not {\em one} world, but just
scattered around. To such empirically oriented and grounded suggestions there is
one main question: what {\em multiplicity}? Why not {\em multiplicites}? 

\pa
We think that, no matter how paralogical and antinomous, the idea, the need to
comprehend {\em all} things in form of some \co{unity} (and not merely a
\co{totality}) is more than a mere illusion.
It is a reflection %of the immediate awareness
of the \co{unity} of \co{existence}, not only of reason and \co{actual}
apperception, but of the individual \co{existence} whose \co{unity} is
established by the primordial \co{separation} from the \co{origin}.  It is much
more than a mere application of \co{reflective} thinking in terms of
\co{dissociated objects} to the \co{totality} of everything.  Such an
application is merely a source of antinomies and impossible questions.  However,
the \co{one} is not a \thi{one}, is not an \co{object}, and \co{reflection}
\co{positing} it as such for the purpose of discourse must remember that.  It is
not an \co{object} whose identity has to be established and whose
differentiation needs a proof.  It is the \co{indistinct}, that beyond which no
\co{distinction} is possible, because everything \co{distinguished} enters by
this very token the world of \co{distinctions} leaving the \co{indistinct}
behind. As the limit of all \co{distinctions}, the indistinguishability-as-such,
it is the very essence of \co{unity} and identity.  As the \co{origin} of both
identity and differences, it comes before them and hence cannot be explained in
their terms -- either we start with it, or else we will never reach it.


\pa The fact that we can not think \co{chaos} without thinking it as one
\co{chaos}, might be an argument, but it would be {\em only} an argument, an
attempt to reduce something \co{invisible} to the plain, all too plain,
categories of \co{reflection}.\ftnt{This might, indeed, trouble many for what
  proof, or at least an argument, could we offer? None. If one wants proofs, one
  better study mathematics; while the only value of arguments is that they
  possibly may help to clarify what one means.}  It is not one which is
opposed to many and it does not contain multiplicity {\em within} itself. It is
not a Maximum, not any \la{plus quam \ldots}, \la{ens realissimum}, \la{omnitudo
  realitatis} -- as Heidegger would repeat after negative theologians and
Scholastics, it is Being, not {\em a} being.  It is a pure \co{virtuality}, like
a single cell is a virtuality of a living being, like (according to some
theories) a single $\gamma$-ray is a virtuality of the whole universe.  It
contains everything that follows -- not \co{actually}, however, not potentially
(which is but a form of \co{actuality}), not \thi{within itself}, but only
\co{virtually}, as the \co{nothingness} of a true beginning contains all that
follows, as the indispensable condition contains everything it makes possible.

An empirical pluralist, a lover of manifold, is afraid that \co{one} would take
from him the glorious variety of \co{actual} multiplicity.  The liveless
monotony of a de-concretised \thi{one}, just like that of an over-rationalised
universe of rules and laws is certainly something nobody wishes.  But \co{one},
being the \co{virtual origin} of manifold, does not negate it, does not oppose
it, does not abolish it -- it only remains \co{invisibly present above} it. In
fact, as a pure \co{virtuality}, it becomes \co{present} only through
\co{chaos}, only through differentiation. It only stays beyond and \co{above}
it.

\noo{ But is it therefore an ontological difference, does it mean that \co{one}
  has any ontological status at all? It means that if we want to speak, we have
  to speak {\em as if} \co{chaos} was \co{one}. We are here far beyond the
  horizon of \co{actual experiences}, for even if there are some special
  \co{experiences} which may be referred to the \co{one}, such \co{experiences}
  neither exhaust nor even fully reveal its meaning -- for it is the constant
  \co{presence}, the \co{unity} \co{founding} all manifold.  }


\subsection{To be is to be distinguished}\label{se:toBeDist}
\noo{
\begin{enumerate}\MyLPar
\item distinction: that = what; fact-of/pure distinction = what is distinguished\\
  -- aspects, only formally distinct\\
  -- empty concept/fact = content/what\\
  -- univocal = equivocal
\item to be = to be distinguished  
\end{enumerate}  
\begin{enumerate}\MyLPar
\item `not a real predicate': -- unreal? yet, positing as being =/= a s not-being
\item being =/= essence (Thomas, quod est (thing, substance) vs. quo est (form,
  making it what it is, `esse') + esse (the act making both be)
\item \citet{esse est aliquid fixum et quietum in ente}{CG}{I:20.4, Gilson p.368}\\
  -- highest act\\
  -- highest, univocal, fact\\
  -- empty concept!
\item being is only what actually is = essence (or even no essence, just being
  itself)\\
  -- being is only being something: Being = being(s)
\end{enumerate}  
}
\pa\label{pa:notforme}\label{pa:notHegel}
\co{Birth}, the original \co{separation} is {\em the only} ontological
event. But do not later \co{distinctions} have any ontological significance?
Certainly they do: to be is to be \co{distinguished}, that is, to make a
difference.\ftnt{We can claim here some support of etymology which constructs
  \wo{existence} (not only in our, but also in the general sense of
  being) from \la{ex} = \thi{out} and \la{stare} = \thi{stand}, i.e, as 
  \thi{standing outside}, \thi{being separated} or \thi{exiled}. This may but
  need not be made 
  different from the interpretation making existence into \thi{standing outside itself},
for \thi{being separated} is just \thi{having outside}, and \thi{standing
  outside} means also to \thi{to be (through) what one is not}. } This is the
ground on which the general association of \thi{being} 
and \thi{independence} rests. Independence, as being \co{distinguished}, is not
a property of something that is -- it is what makes it be. 
This association can be, and was, pressed to the ultimate limits
by claiming that only particulars \thi{are} -- particulars, that is, the most
definitely \co{distinguished} entities, eventually, completely \co{dissociated}
ideal substances, prone to  enscription within the ideal limit of the
horizon of \co{immediacy}, in a single point. But being is not only the event of
the utmost \co{actuality}; it begins at the very beginning. 

\co{Distinction} involves two \equi\ \co{aspects}: the fact of
\co{distinguishing} and the \co{distinguished} content.  Distinguishing
anything, we focus naturally on the distinguished content but along with it, we
also experience the mere fact \co{that} we distinguish, 
\co{pure distinction}. This contentless and universal fact is the
univocal emptiness of the (im)possible concept of \thi{being}. The latter, the
\co{distinguished} \thi{what}, is the content which can be further refined and
underlied, eventually, \co{actual} determinations. It \co{founds} \co{actual}
characterisations of various distinguished things and the conceptual differences
between them. These two \co{aspects}, present and discernible in everything that
is, determine the two main lines of considerations on \thi{being}.

\pa \citet{\thi{Being} is obviously not a real predicate; that is, it is not a
  concept of something which could be added to the concept of a
  thing.}{CrPR}{I:2nd Division.3.4, A598-599/B626-627} Indeed, logically,
\thi{being} does not seem to be any real predicate -- it is conceptually empty,
as is, logically, any predicate which can be applied indiscriminately to
everything. But, \thi{unreal} as it seems, it has the fundamental function, for
there is a big difference between \thi{positing} something as being and
\thi{positing} it as not-being.  \citetib{It is merely the positing of a thing,
  or of certain determinations, as existing in themselves.}{CrPR}{}
This \thi{mere positing}, conceptually negligible as it perhaps is, expresses
the most fundamental fact, \citeti{the first act, the first division,}{Eckhart}{\btit{Latin Sermons}
  Ga.III:16-22. [\citeauthor*{Eckhart}{ XXIX, \btit{God is
      one}\kilde{big p.226}}]} the primordial recognition of being. 

\citet{\ \thi{Being} is something fixed and restful in being(s).}{CG}{I:20.4,
  <<esse est aliquid fixum et quietum in ente>>; Gilson p.368} Being (\la{esse})
is, with Aquinas, something more than the mere thing which is (\la{quod est})
and its form which makes it what it is (\la{quo est}).  It is a pure divine act,
above the duality of essence-existence and form-matter, which endows a
\thi{what} with actual existence. (Although conceptual distinction had to be
maintained, its proximity to \la{actus purus}, as Alexander of Hales
characterised God, is unmistakable.) It is \citet{the most perfect of all
  things, for it is compared to all things as that by which they are made
  actual; for nothing has actuality except so far as it exists. Hence existence
  is that which actuates all things, even their forms. Therefore [...]  it is
  not compared to other things as the receiver is to the received; but rather as
  the received to the receiver. When therefore I speak of the existence of man,
  or horse, or anything else, existence is considered a formal principle, and as
  something received; and not as that which exists.}{SumTh}{I:q4.a1.ad3}

Duns Scotus pointed out that we can grasp that something is without grasping
whether it is a substance or an accident, or let's put it more generally,
without grasping what it is. The ultimate Being, conceptually as empty as rich
in the existential possibilities of beings, seems to rest at the bottom of all
metaphysics, not only according to Scotus.  \citet{For all beings participate in
  Being.  Therefore, if participation is removed from all beings, there remains
  simplest Being itself, which is the essence of all things.}{DDI}{I:17.51} It
can be, indeed, very difficult to imagine in what sense this eventual
abstraction manages to be the essence of all things, for \citetib{when I
  mentally remove all the things that participate Being, nothing seems to
  remain.}{DDI}{} There seems to remain nothing and yet, it is Being, and so
Heidegger still asks: \wo{\ger{Was ist das \thi{es} das gibt?}}\ftnt{\wo{\ger{Es
      gibt}} is the German \thi{There is}, which literally says \wo{It gives}.
  One can be led by this German phrase towards something \thi{\co{that} is} (and
  gives) easier than by its English equivalent.}  What remains is the \thi{first
  event}, the mere beginning, the \co{distinction} which breaks the silence of
\co{nothingness}. As an almost plain and \co{visible} illustration consider,
\woo{for instance, when somebody, approaching from a distance, causes in me a
  sense-perception with the help of which I can judge only that what I see is an
  existent. In this case it is clear that my first abstractive cognition
  (first, that is, in order of origination) is the cognition of existence, and
  of nothing less general; consequently it is not a specific concept nor a
  concept proper to a singular thing.}{OckQuod-I:q.13 \noo{p.29}} The example
illustrates well the intended meaning of \co{distinction} as \thi{the first},
perhaps \co{vague} and indefinite, and yet \co{clear} apprehension of
\thi{something being there}.\ftnt{\citeauthor*{OckQuod}{ I:q.13.\noo{p.29}} We
  certainly won't follow Ockham in his insistence on the merely
  \thi{abstractive} character of this \thi{cognition}, that is one \citef{by
    which it cannot be evidently known whether a contingent fact exists or does
    not exist}{OckOrd}{ Prologue:q1 \noo{p.23}}.  The example seems, on the
  contrary, to indicate that it is \thi{intuitive cognition}, i.e., one
  \citefib{that enables us to know whether the thing exists or not}{Ibid.}{}
  This latter \thi{cognition} seemed to require a precise grasp of the
  \thi{individual thing}, and our point is precisely the opposite -- the
  existence of a thing, its \co{that}, although actually not preceding its
  \thi{what}, is nevertheless logically prior and remains \co{above} all more
  precise determinations of its nature, character, or properties. \thi{The
    cognition of existence} in the example is, perhaps, very \thi{general} but
  only in so far as merely cognitive content is concerned. It is, on the other
  hand, the most specific experience of something particular being there.}  This
\thi{something} is as yet unclear, its particularity remains still veiled in the
barely discerned fact of existence. But its individuality is already fully
transparent.  And thus, the abstract generality of Being is, in fact, the most
specific individuality of particular beings.

\pa Quoting Ockham, however, we seem to be moving in a completely opposite
direction. Along this first line, following the intuition of the ultimate
\co{that}, Being emerges as a univocal and distinct principle lifted above and
transcending all particular beings. But its conceptual emptiness can cause some
worries, especially, for the epistemologically oriented (whether nominalism,
empiricism or linguisticism), for which it amounts simply to bare emptiness.
Indeed, how can we claim any significant difference between being and that
which is, between existence and essence? After all, there is only that which is,
\thi{to be} is necessarily \thi{to be something}: \citet{essence and existence
  are not two things.  On the contrary, the words \wo{thing} and \wo{to be}
  signify one and the same thing, but the one in the manner of a noun and the
  other in the manner of a verb [\ldots] there is no more reason to imagine that
  essence is indifferent in regard to being and non-being, than that it is
  indifferent in regard to being an essence and not being an essence.  For as an
  essence may exist and may not exist, so an essence may be an essence and may
  not be an essence.}{OckSumLog}{III:II.c.xxvii \noo{\citaft{OckAll}{ p.92-4}} }
This identification of \la{essentia} (and \la{esse}) with 
\la{existentia} is a general tendency of the empirically and also analytically
oriented philosophy (the difference being only that the former renounces Being
(if not also essences) on the grounds of atomistic ontology while the latter for
its irresolvable involvement into more specific conceptual and linguistic
contexts.)  \citet{There are as many kinds of existential statements, as there
  are kinds of the objects of discourse.}{MalcolmOnto}{III\noo{after Filozofia
    Religii (fragmenty filozofii analitycznej), p.112}} Why not follow such a
line of thought all the way and say: \wo{There are as many kinds of existential
  statements, as there are [] objects of discourse.} Now, the kinds seem to
disappear and we are left with: \wo{There are as many existential statements, as
  there are objects of discourse.} A tautology? Not really, for it reflects only
the tendency to dissolve Being into atomic existents, once existence ceases to
have any \co{transcendent} \co{aspects}, in particular, when Being is reduced to
\thi{being something}, i.e., existence is reduced to essence and, eventually, to
the mere empirical fact of distinctness of \co{actual} things.\ftnt{We are
  deliberately ignoring here the distinction between the postulated essences and
  their conceptual, that is, mental counterparts. This issue will be addressed
  in II:\ref{impressConcept}, in particular, \refpf{pa:essences}} An almost
equivalent formulation might be: \citet{[b]ecause singularity immediately befits
  that to which it belongs, therefore it cannot befit it through something else;
  therefore if something is singular, it is singular by itself.}{OckOrd}{
  I:d2.q6 \para 85 [translated by John Kilcullen]\label{ftnt:noPrincInd}}
%
Of course, assuming only particulars versus their kinds may be the
distinction between nominalism and (some form of) conceptualism which does not
interest us. In either case, the tendency is the same: as being means being a
particular individual, Being has no meaning except, perhaps, as a totally
equivocal abbreviation.  Yet, no matter how many {\em kinds} of existential statements
one manages to postulate or even identify, they will be all kinds of {\em
  existential} statements.

\pa Where does it lead us? For we do not want to follow the (im)possible
variations of these two tendencies, where concepts become mere words or else
\thi{internal} reflections of essences, \la{quiddities} disappear or else become
only visible reflections of exemplars, the eternal exemplars are divine ideas
which, perhaps, are thoughts co-eternal with God's being or else are only
results created by His thinking, etc., etc. The distinctions and ever new
intermediary stages which, like in the third-man argument, seem to be required
by the initial dualism in order to create an impression that the dissociated
poles nevertheless meet. The \co{dissociation} is, as usual, that of \co{object}
and \co{subject}, only here it appears under the aspect where the
\thi{objective} existence seems a bare \co{that}, while the \thi{subjective}
conception is the actual content, the \thi{what}.

The two, apparently contrary tendencies, are elaborations of these two
\co{aspects} of the unitary event of \co{distinction}. For \thi{to be} is to be
\co{distinguished}. As soon as you \co{distinguish} something it is; it {\em is}
even if it remains \co{vague} and \co{unclear} \thi{what} it is. On the other
hand, if something remains \co{undistinguished}, it is not even a
\thi{something}, it is not even an \thi{it} -- there remains \co{indistinct},
but it is not \thi{it} that remains \co{indistinct}.\ftnt{We will not
  reduce this to a mere \thi{formal concept} (an act of mind or a concept merely
  representing an object), as opposed to \thi{objective concept} (the
  represented objects), for this distinction arises only as a consequence of the
  \co{dissociation} of \thi{subjective object} (thought or formal concept) from
  its \thi{objective object} (or objective concept).\noo{(and leads to a
    repetition of the whole debate at another level of 
  abstraction).} To be \co{distinguished} is not to be a formal or other concept,
  a mere mental accident -- it is to {\em be}.}
%
\label{pa:univocity} \wo{To be} signifies nothing determinable because it
merely places whatever is \co{distinguished} in its \co{indistinct} 
\co{origin}. The copula lends its subjects the universal
privilege of participation in Being, of being \co{distinguished}. 
\co{Distinction} is only secondarily a \co{dissociation}
of $a$ from $b$; primarily, it is \co{distinction} of $x$ from the indistinct
background, and \co{traces} of this \co{aspect} mark all \co{actuality}. 
The universality of \ldots -- the concept?  the idea?  the intuition?  -- no, of
the \co{experience} of \thi{to be} is coextensive with the universality of
\co{distinguishing}, that is, with all life.  Brought to the level of language,
there is, of course, no need for a particular form, a particular verb.  As
Derrida, quoting Benveniste, observes discussing the transcategoriality of
\wo{to be}: \citet{the 
  strangeness is in the facts -- that the verb of existence, out of all verbs,
  has this privilege of being present in an utterance in which it does not
  appear.}{DerCopula}{The Remainder as Supplement... p.202. One can recall
  here the example of Semitic languages which dispense with the use of \thi{to
    be} as copula and express it in the nominative sentences (e.g., \wo{Pegasus
    winged horse.} for \wo{Pegasus is a winged horse.})}  It is there, and it
is everywhere, because every word and gesture brings in a \co{distinctions},
while without \co{distinctions} there would be no world and, consequently, no
words.

The celebrated equivocity of \wo{is}, of \wo{to be}, is only the equivocity of
\co{distinction} -- \co{distinguished} contents may have nothing in common, no
common genus, no links of similarity, except for being \co{distinguished}.  The
equivocity is the possibly unlimited differentiation of the distinguished
contents. In fact, \citet{the difference between the existence of chairs and the
  existence of numbers seems, on reflection, strikingly like the difference
  between numbers and chairs.  Since you have the latter to explain the former,
  you do not also need \thi{exist} to be
  polysemic.}{Fodor}{III;p.54}\noo{Recognising the primacy of this intuition,
  the primacy of being \co{distinguished}, we can renounce the atomistic
  ontology. At the same time, we can also follow Ockham in his radicalisation of
  Scotus' principle of individuation, that is, in its total removal (cf.
  quote~\ref{ftnt:noPrincInd}).  The difference is that while Ockham coupled
  removal of the principle with the postulated existence of \co{posited}
  particulars, we do not postulate anything --} There is what is
\co{distinguished} and, beyond that, only the \co{indistinct}.


\subsub{Relativity and objectivity}\label{sub:relativeWhat}

\pa\label{pa:dreamsAre}\label{pa:universality}
But wait, is there anything which
we do not \co{distinguish}? Does it mean that {\em everything} is? Yes, it does.
And dreams, and square circles, and Pegasus? OK, we would say \wo{Pegasus is a
  horse with wings} or something like that. A cheap, grammatical trick would be
to point to the \wo{is} in this sentence, but we do not rely fully on mere language
usage, let alone, grammar.\noo{Such grammatical tricks and linguistic
  observations are, of course, relative to the language or language group.
  Semitic languages do not use \thi{to be} as a copula: in Hebrew, for instance,
  one forms instead noun clauses or even noun sentences, like \wo{Pegasus-winged
    horse}.} Of course, that Pegasus {\em is}. It is in a very different way
than the horse out there, but still it is, it is \co{distinguished}, even much
more, it is \co{distinguished} as something.  That it, perhaps, does not have
material existence, that it is not a living being, that it is a concept or a
mythical figure are truths which do not in the slightest affect the fact that it
{\em is} -- we all know \thi{what} it is, so we should not be so concerned whether
it, in fact, is.
Even \citet{fictions are from God, because some of them are mental
  entities, some vocal, some written signs, and all of these are real beings and
  thus are from God, just as lies are from God, since they are real
  entities.}{OckQuod}{III:q.3\noo{p.130} (As is often the case, we do not have
  to subscribe to the details of the argument.)} There is nothing wrong with
\thi{being} of a thought -- a thought {\em is} as much as a horse, a table, or a
meaningful relationship.

\pa\label{pa:keepDistinguish}
But sure, one usually means something more specific with \wo{being}. What?

The dream I had yesterday {\em is}, the image, the phantom of perfection I am
chasting {\em is}, the illusions I nourish {\em are}, the feeling I have {\em
  is}.  It is impossible to get rid of this ascription of \thi{being} in spite
of the fact that we would say that all these things {\em are not}. They are not
because they are {\em only} my subjective feelings, imaginations, ideals... Yet, to be an
image, is not that also \thi{to be}?  That they are all \thi{subjective} does
not in the least deprive them of being because they, too, are
\co{distinguished}, even \co{distinguished} as these specific \thi{whats}.  They
are called \wo{subjective} because they are relative only to me -- though, as a
matter of fact, now not any longer only to me! One used to say they are only
\wo{for me} but this suggests that something might be \thi{for nobody}, \thi{in
  itself} which is exactly what we find completely implausible.  Relativity to
particular person or group of people is a further differentiation of the
\co{distinguished} beings, of things which {\em are}.

Surely, one may keep distinguishing different ways of being: something being
relative {\em exclusively} to my act or experience versus something relative
also to the experience of others, something relative exclusively to my thought
or also to my perception, something relative to human experience in general
versus something relative to the corresponding experience of ants, etc. But all
such are secondary \co{distinctions}, in particular, the supposed
\thi{realities} they postulate are invariably of a limited scope: the
\thi{physicalist reality} is threatened by the \thi{reality of subjective
  qualia}, the \thi{reality of perception} by the \thi{reality of feelings}, the
\thi{reality of my life} by the \thi{reality common to all}, the \thi{reality of
  public consensus} by the \thi{reality of personal convictions}.

One would like to arrive at something which is constant and fixed, one and the
same \thi{for everybody}.  But populism and consensus is no measure of
\thi{reality}, although it is certainly the measure of the \thi{reality about
  which there is a consensus}.  The richness of the possible \co{distinctions}
and relativisations is only the richness of our world.  Interesting and
sometimes even relevant as such relative distinctions may be, they are not very
useful to us. They lead invariably to positing some form of being as \thi{{\em
    the} being}, \thi{{\em the} real}, and delegating all others to 
\thi{unreality}.  For there is only a very small and usually imperceptible
transition from the question about the ultimate principle of everything to the
exclusion from the reality of everything which does not conform to the
\thi{discovered} principle. Yet a principle which would embrace absolutely
everything can hardly have any determinate content -- principles tend to (if not
must) exclude something.\noo{, if not for other reasons so simply because they
  are formulated in some specific, not to mention \co{precise}, language and
  concepts.}  Eventually, the search for the \thi{true reality}, for something
which is both determinate and not relative to our way of \co{existing}, leads
invariably towards the abyss of the inaccessible \thi{reality} of \thi{things in
  themselves}, and dissolution of everything we know and \co{experience} in the
merely phenomenal \thi{unreality}.

\noo{Parmenides seems to have been the only one among the Greeks
who resisted the temptation to disgrace some forms of Being as \thi{unreal} --
\citt{ungriechisch wie kein andrer in den zwei Jahrhunderten des tragischen
  Zeitalters [\ldots]}{Nietzsche, {\em Die Philosophie in tragischen Zeitalter
    der Griechen}, 381 [Parmenides, p.45]} His identification of Being with
non-Being, \citt{die volle Perversit\"{a}t des Denkens}{Nietzsche, {\em Die
    Philosophie in tragischen Zeitalter der Griechen}, 386 [Parmenides, p.46]}
is, in our language, the \equin\ of \co{One} and Many, the structural
co-presence of \co{aspects} which can be \co{distinguished} but not
\co{dissociated}.
}


\thi{Reality} (and it is very tempting to credit the Greek philosophers with its
invention, even though they had no word for it), when opposed to anything, in
particular to the 
mind as something mind-independent,
%scared by the possibility of mistake, called eventually \wo{subjectivity})
%fighting against the mythological and religious tradition (or whatever other
%demons were hunting them),
becomes but a metaphysical extension of the \co{actual} \co{dissociation}
of \co{subject} and \co{object}.
% \thi{Reality} arises as a supposed medicine against the insecurity and
% uncertainty of a \co{subject} which is offered only truth reduced to the
% \co{precise}, \co{reflectively} comprehensible \thi{objectivity}.
And then, immediately, follows the search for the infallible, \thi{real}
criteria of \thi{reality}.  The Greek \thi{fall} from the reality of myths to
the \thi{true reality} of \co{objectivism}, praised as much as it always has
been for laying down the foundations of science and rationality, was primarily
drawing the \co{distinction}, in fact, \co{dissociating} the \thi{real} from the
\thi{unreal}.  But there is nothing unreal. How could there be? It takes a lot
of disappointment to rise suspicion, and then a lot of suspicion to claim that
reality consists of two parts: \thi{real} and \thi{unreal}.

Everything {\em is}, or else there is no thing, there is nothing,
which is not, about which one could not -- in one or another, but always
meaningful way -- say that it is. \citeti{It is and it is not so, that it is
  not.}{Parmenides}{DK 28B2. Parmenides seems to have been exceptional among the Greeks
in resisting the temptation to disgrace some forms of Being as \thi{unreal} --
\citef{un-Greek as no other in the two centuries of the Tragic Age.}{onParm}{
  <<ungriechisch wie kein andrer in den zwei Jahrhunderten des tragischen 
  Zeitalters>>\kilde{381,Parmenides, p.45}}} Once you start pointing at something, you
\co{distinguish} it and hence -- it {\em is}.  \citet{Being itself is manifold
  within itself, and whatever else you may name has Being.}{Plotinus}{V:3.13}
Everything is and this seems like another side of the fact that it is so hard to
say something which could not possibly make any sense.  The equivocity of \wo{to
  be} is the equivocity of all the differences which we can \co{distinguish};
its univocity is the universality and univocity of the fact of \co{distinguishing}.


%\subsubi{relative, and yet objective (not Berkeley)}


\subsub{A note on Berkeley's idealism}\label{sub:Berkeley}

\pa
We won't let grammar mislead us, grammar pointing out that saying
\wo{to be distinguished}, we already use the word \wo{be}.  It is only
grammar, besides only English or German grammar.\ftnt{In
Scandinavian languages, for instance, the passive form does not
require the usage of \wo{to be}. In Norwegian \wo{{\AA} v{\ae}re betyr {\aa}
skilles} says, literally, \wo{To be means (to be) distinguished},
where the paranthesised (to be) simply does not occur and where the
apparently active \wo{skilles}, \wo{to distinguish itself}, has a
marvelous ambiguity involving equally the passive aspect of
\wo{(being) distinguished} (as in \wo{ting skilles ...} which is as 
much \wo{things are distinguished...} as \wo{things distinguish 
themselves...}).}
%
Thus, while to \co{exist} is to \co{distinguish}, \la{existere est
  distinguere}, so instead of Berkeley's \la{esse est percipi} (which arises from the
same intuition), we would say \la{esse est distingui}.
We should probably comment  briefly on the relation to Berkeleyan
idealism.  For are we not actually reducing ontology to epistemology, being to
being perceived? No, we are not.

First, there is a difference between our \co{distinction} and perception. The
latter is the category of \co{actuality}, and Berkeleyan idealism suffers from
phenomenalism and nominalism for this reason. Restricting being to being
perceived, he reduces being to pure \co{actuality} and everything (not only
so vehemently criticised abstract ideas) transcending this horizon becomes
... \thi{unreal}. 

Furthermore, the \co{actuality} involves one into the dualism of
\co{subject}-\co{object}, so naturally conflated with that of
\thi{mind}-\thi{matter}. Berkeley gets rid of the latter but the dualism remains
effective, both in argumentation and, as we believe, in his thinking.
\citet{Ideas imprinted on the sense are real things, or do really exist; this we
  do not deny, but we deny they can subsist without the minds which perceive
  them, or that they are resemblances of any archetypes existing without the
  mind;}{BerkeleyPrinc}{\para 90} One could perhaps invent a distinction
between \thi{real} and \thi{subsisting} allowing one to accept the former and
deny the latter. As it happens, Berkeley can accept the real, continuous
existence of things subsisting without our minds, only because they subsist in
God's mind. This is again the disturbance caused by the mere \co{actual}
perception. For us, a table left unperceived in a room keeps existing
undisturbed, because we all go around with the understanding that it is there --
it is \co{distinguished} also when it is not \co{actually} perceived by
anybody.

\pa This, of course, still involves the relativity to the \co{distinguishing
  existence}. But here we are touching upon the main difference -- existence of
things \thi{in the mind}.  \citetib{Thing or Being is the most general name of
  all: it comprehends under it two kinds entirely distinct and heterogenous, and
  which have nothing common but the name, viz. spirits and ideas. The former are
  active, indivisible substances; the latter are inert, fleeting, dependent
  beings, which subsist not by themselves, but are supported by, or exist in
  minds or spiritual substances.}{BerkeleyPrinc}{\para 89} Just like the
previous quote, we could accept this one almost without any changes, except for
a minor detail which turns out to be the crucial difference: beings which
\wo{are supported by, or exist in minds of spiritual substances}. The vocabulary
of \wo{being supported by} and \wo{substances} should be taken very seriously
here, for the point is that things exist as if they were mere accidents of
spiritual substances. This is much more than our claim of relativity to
\co{existence}. Things are \co{distinguished}
{\em by} \co{existence} (or mind, to keep it closer to Berkeley's vocabulary,
though not his concepts), but the \co{distinctions} are made {\em in} the
\co{indistinct}.  I do not draw \co{distinctions} {\em in} my mind while you
{\em in} yours, we all draw them in the \co{one} and the same \co{indistinct}.
We may draw them differently, but this is another point.

\noo{\begin{itemize}
\item   
\citet{we have been led into very dangerous errors, by supposing a twofold
  existence of the objects of sense -- the one intelligible in the mind, the
  other real and without the mind; whereby unthinking things are thought to have
  a natural subsistence of their own distinct from being perceived by
  spirits.}{BerkeleyPrinc}{\para 86} + relative distinctions
\item     
  Berkeley is determined by the dualism sense-soul, matter-spirit and
  impossibility of getting-out: \citet{I do not see how the testimony of sense
  can be alleged as a proof for the existence of anything which is not perceived
  by sense.}{BerkeleyPrinc}{\para 40}

  \citetib{[I]f there were external bodies, it is impossible we should ever come
  to know it; and if there were not, we might have the very same reasons to
  think there were that we have now.}{BerkeleyPrinc}{\para 20}
\end{itemize}
}

For things which exist only \thi{in the minds} to be \thi{objective}, some
universal and \thi{objective mind} (typically and, for a bishop, quite naturally
God's mind) is indeed needed.\ftnt{The \thi{mind} need
  not be God and can, of course, enter in various disguises.  \citef{That is,
    there is no thing which is in-itself in the sense of not being relative to
    the mind though things which are relative to the mind doubtless are, apart
    from that relation.}{PierceFourIncap}{ p.68} They are \thi{apart from that
    relation} only because we imagine others who appear in our place with the
  same capacities to \co{distinguish}.  This leads to Pierce's notion of reality
  (repeated by other prophets of consensus or ideal rationality), namely,
  \citefib{that which, sooner or later, information and reasoning would finally
    result in, and which is therefore independent of the vagaries of me and
    you}{PierceFourIncap}{p.69}, which further implies that \wo{this conception
    essentially involves the notion of {\sc community}, without definite limits,
    and capable of a definite increase of knowledge.}  It was hardly Pierce's
  intention to make reality relative to a consensus, that is, eventually to a
  \thi{mind}. But, apparently, it did not take Durkheim to replace God with
  society.  Starting with \thi{minds} opposed to \thi{reality} and then thinking
  the latter in terms of a \thi{totality of things}, as is typically the case
  with the scientific bias, makes the reality appear as an inaccessible
  epistemological terminus, an \thi{ideal limit} of cognitive or experimental
  development.  Pierce's \thi{reality} is the ghost of Kantian \thi{things in
    themselves}, with their independence and immovability, which tries to
  overcome the impossibility of becoming flesh in the \thi{collection of {\em
      distinct} things} and process of discovery. Eventual consensus seems to
  provide a solution to this very impossibility.  But do I need any consensus to
  {\em know} that the edge of the fiord at which I am standing is real?  Do you
  need any consensus to {\em know} that the beauty of the view is real?  The
  eventual reality is the \co{one} which \co{founds}, rather than is founded by,
  any possible community.  All \co{distinctions} have their reality \co{founded}
  in the \co{one}.  The distinctions between \thi{subjective} and
  \thi{objective}, \thi{inner} and \thi{outer}, or \thi{private} and
  \thi{common}, are distinctions {\em within} the real, not ones founding it.}
But this is not so for things which are only discovered by, and hence only
relative to, the mind.  Not only \thi{being relative} does not contradict
\thi{being {objective}} but belongs -- even, belongs essentially -- to the
latter. One can not specify anything claiming its objective existence without
having first \co{distinguished} it. And \co{distinctions} are relative.  What
concrete things appear (are \co{distinguished}) depends on the organism and we
certainly live in very different worlds than ants do. But the fact that a thing
appears for an ant and not for us does not make it less \thi{objective} or
\thi{real} -- at most, only less relevant for us.  This relativity does not
deprive the appearances of any \thi{reality} or \thi{objectivity}.
%
Existing \thi{in the minds} (whatever that might mean) involves dependence on
these minds. But relativity to an \co{existence} does not require (nor imply)
dependence on this \co{existence}. A daltonist and I can see the same object as
having different colours -- our perceptions of it are relative (to our minds,
including organs of perception). But they are not dependent on, in the sense of
being caused or otherwise determined by, our minds.\ftnt{We face here the
  possible ambiguity of the word 
  \wo{dependence} which can involve only necessary but also sufficient reasons.
  We tend to take \wo{relativity} as meaning the necessary reasons -- without
  \co{existence}, no \co{distinctions} -- while \wo{dependence} as causal
  dependence, meaning the sufficient reasons or efficient causes.}
\co{Existence} does not generate \co{distinctions}, it encounters them, meets
them in the \co{indistinct} -- this is the sense of \co{confrontation} 
in the sphere of the \co{distinguished}.
%
Consequently, what is being \co{distinguished} are not mere appearances but the
very things 
which {\em are}.  This does not, of course, mean that all \co{distinctions} are
equally adequate, or that no mistakes are possible, but truth and adequacy are
just categories of relating some (set of) \co{distinctions} to another, and we
will take it up at some later time (II:\ref{sub:truth}).  Things are
\co{distinguished} {\em from} 
the background which remains their \co{origin}.
\co{Distinctions} do not turn them into mere \thi{appearances}, in the sense of
something opposed to some \thi{reality}.
\co{Distinctions}, and eventually also \co{reflection}, \thi{create} things of
\co{experience} (or these things \thi{subsist} only \thi{supported by} the
\co{distinguishing} \thi{minds}) only in the sense that other people might
\co{experience} things differently, while other beings might \co{experience}
entirely different things.

But freedom of such a \thi{creation} does not mean
arbitrariness or voluntariness. The
\co{distinctions} made by ants reflect something of the world as much as the
\co{distinctions} made by us. They are all relative (to those who \co{distinguish})
but, at the same time, fully objective.
 For what is \co{distinguished} is always a reflection
of the \co{indistinct}, it comes {\em from} the \co{origin} and is made {\em
  into} it.  Each \co{distinction} is made {\em from} the \co{indistinct one},
and this \thi{from} for ever leaves the stamp of objectivity on whatever is
\co{distinguished}.
%The fact that appearance of something is dependent on being \co{distinguished}
%does not mean that this something is \thi{created} by so being
%\co{distinguished}.
Every \co{distinction} is a \co{distinction} of Being and hence a reflection of Being,
it reflects the possibility of being \co{distinguished}. 


\pa So, \citet{there can be no things with determinate natures unless there are
  true descriptions, and no true descriptions unless the intellect is already at
  work.}{ClarkPlotIntell}{\kilde{p.429}} Adjusting the Plotinian vocabulary:
every thing, everything that is, is relative to some (\co{distinguishing})
\co{existence}.  The problem with accepting such a claim is the same as with
imagining that something which is once \co{distinguished}, could remain
undistinguished.  Once a \co{distinction} has been made, it becomes
ineradicable, unerasable -- it can acquire entirely new sense, it can be
modified, it can be declared \thi{untrue} or irrelevant, it can sink into deeper
layers of \co{virtuality} -- but once made it can not be un-made. This is the
primal ground for the \co{experience} of objectivity (as well as for the not so
infrequent insistence with which we stick to once acquired opinions). But this
does not mean that we were not involved.  Say, this table or the chair you are
  sitting on, if you died now, it
still would be there, wouldn't it? What does one mean by this \wo{still}? One
means that if, when you died, somebody else came here and looked, he would
perceive this chair, too. Sure -- it is not relative exclusively to \co{your 
  existence}. But if only most primitive bacteria were left...? No! Even if
nobody were around, the chair would still be there. It would be there, perhaps,
in the sense of a potential \co{distinction} to be made by somebody capable of
it. But if we imagine (let us keep imagining for a while) that until the end of
the world the only living organisms were such that the presence or absence of
this chair could not possibly make any difference to them, what sense would it
make to say that it is there?  As Berkeley observed, such an insistence on its
being there, independently of you, me or anybody \co{distinguishing} it, harbours a
vicious circularity. First we \co{distinguish} a thing and then pretend that it
did not matter.  But insisting on {\em this} thing being there is exactly saying
that somebody might \co{distinguish} it again, in fact, is sticking to this very
\co{distinction} in the very moment one tries to ignore it.\ftnt{\citef{[T]here
    is nothing easier than to imagine trees, for instance, in a park, or books
    existing in a closet, and nobody by to perceive them. I answer, you may so,
    there is no difficulty in it; but what is all this, I beseech you, more than
    framing in your mind certain ideas which you call books and trees, and the
    same time omitting to frame the idea of any one that may perceive them? But
    do not you yourself perceive or think of them all the
    while?}{BerkeleyPrinc}{\para 23}}  When one maintains that, if humankind died
out, the same things would  {\em still be} in the world, one assumes (as
  witnessed by \wo{still}) a human
\co{existence} in that world -- departed by all humans -- to whom these things
still {\em are}, and are the same.%make difference.


\citet{For what precisely is meant by saying that the world existed before any
  human consciousness? An example of what is meant is that the earth originally
  issued from a primitive nebula from which the combination of conditions
  necessary to life was absent. But every one of these words, like every
  equation in physics, presupposes {\em our} pre-scientific experience of the
  world, and this reference to the world in which we {\em live} goes to make up
  the proposition's valid meaning. Nothing will ever bring home to my
  comprehension what a nebula that no one sees could possibly
  be.}{PontyPerc}{III:2\kilde{p.432}}
  
It is obvious that when I die, other people will continue living in the same
world, that is, among the same things as I did. The world, although intimately
\co{mine}, is not relative only to \co{myself} and most things in it are
\co{distinguished} similarly by all people. But the world without {\em any} 
\co{existence} to differentiate it, the \thi{objective world} of physics where
only things, \co{objects} or, perhaps, atoms or strings float around, in short,
the world \thi{in itself} is an image, or rather a phantom -- it is
just \co{indistinctness}, into which one has projected some selected \co{distinctions}.

\noo{
In short, \co{distinctions} are relative, but it does not make them merely subjective
or unreal. Relativity to a \co{subject} is an aspect of their \co{objectivity},
in the technical sense to be specified in Book II, but also in any at least
slightly \co{concrete}, that is, not completely \co{dissociated} sense of the
word -- the differences may only concern the scope of this relativity.
}

\subsub{One is}\label{se:OneIs}
\pa
Our uneasiness with such a generous notion of being comes from
the expectation that being should be independent, should be exactly that which
is {\em not} relative to our nor any other existence, that reality is \co{one}
and the same 
for all. This requires a bit more precision: who are all, or else, 
independent from whom? Certainly, not only from \co{my existence}, not only from
any particular \co{existence}. So, perhaps, from all \co{existences} of all
people who ever lived and will live. We start suspecting something uneasy, don't
we? For why not exclude also relativity to all other living organisms? And then,
not only to the actually living (future or past) \co{existences}, but also to the
mere forms of all possible \co{existences}. If we follow such chains we end with
the \thi{things in themselves} which, unknowable and inaccessible as
they are, are neverless \co{posited} by \co{reflection} as the ultimate and
truly \thi{real} objects. 

Among other requirements for a \thi{true Being} one listed unchangeability,
simplicity, self-identity, and all such were on various occasions ascribed to
the particulars which one considered to be \thi{truly existing}.  From what we
have said so far,\noo{in \ref{sub:relativeWhat},} no such things are to be
expected. Yet the 
expectations do have their source -- they are \co{traces} of the
\co{originally}, but always also \co{actually}, \co{present indistinct one}.
%
As Hegel observes following Berkeley, and Husserl following Hegel, even the
\co{distinction} between thought 
and its object is a distinction relative to thought.  But (and this is
reminiscent of the ontological proof) this \co{distinction}, although made
relatively to the distinguishing being, reflects the fundamental
\co{distinction} of this very being, its \co{confrontation}: it is relative but
not arbitrary, not to say merely \thi{subjective} -- it is not only \thi{within
  the mind} (what could {\em that} mean?) for it {signifies}, it means
something. It is an \co{actual sign} of the fundamental ontological
\co{distinction}, of the fact of \co{existential confrontation} with 
\co{nothingness}. That this is impossible to establish from purely
epistemological, that is \co{actual}, 
assumptions shows only that such assumptions are at best secondary. Indeed, the
\co{subject}-\co{object} \co{distinction} is but the most \co{actual} form, the
lowest \co{trace} of this \co{original} \co{separation} of one from \co{one}.
%
The \co{existence} is constituted by
\co{confrontation}; although all differentiation of the \co{indistinct} is
relative to \co{existence}, it is differentiation of and in the \co{indistinct},
not in anybody's mind. Thus, although we may live in different worlds, we always
\co{share} the ultimate \co{origin} of \co{existence}. It lies beyond any
\co{actual experience}, it lies \co{above} all particulars of our lives, but it
is the ultimate pole of \co{existential confrontation}. Most things we are
saying may remind about Berkeleyan, or sometimes transcendental idealism, but
the bottom line, that is, the starting point is quite different. Perhaps, we
might call it \wo{transcendental realism}, for the \co{origin} of all
differentiation itself {\em is}, and it is in the most \co{absolute}
sense.\ftnt{Of course, this expression must not be taken in the sense given it
  by Kant. It stands here for almost exact opposite of the transcendental
  realism which \citef{[a]fter wrongly supposing that objects of the senses, if
    they are to be external, must have an existence by themselves, and
    independently from the senses, [...] finds that, judged from this point of
    view, all our sensuous representations are inadequate to establish their
    reality.}{CrPR}{I:2nd Division.2.Book 2.1.1st paralogism [A369]}}


\pa\label{Oneis} But: if to be is to be \co{distinguished}, then \co{one}, as
\co{indistinct}, is not.  It certainly is not {\em a} being, is not a something.
\citet{Being must have some definition and therefore be limited; but the First
  cannot be thought of as having definition and limit, for thus it would not be
  the Source but the particular item indicated by the
  definition.}{Plotinus}{V:5.6} But, as a matter of fact, it {\em is} -- not by
being defined but by being \co{distinguished} as the \co{indistinct}, it is
\co{distinguished} from everything which, being differentiated, falls under the
categories of \co{distinctions}.  According to Eckhart it is even
\wo{\co{indistinct} from all things} and we can agree to that in the sense of it
being \la{esse omnium}, the being of all things, that which is \co{present}
\citeti{everywhere and everywhere entire.}{Eckhart}{\btit{Latin Sermons}
  Ga.III:16-22. [\citeauthor*{Eckhart}{ XXIX, \btit{God is
      one}\kilde{p.224big}}]} But this entire \co{presence} remains nevertheless
entirely \co{transcendent} because, as undifferentiated \co{origin}, it is
\co{above} all \co{distinctions}.  It is \co{nothing} because it is
\co{indistinct}, undifferentiated, but it is \co{distinguished} from all the
\co{distinctions}, and so it \citet{is not nothing, for this \co{nothing} has a
  name \wo{nothing}.}{CusaHidden}{\label{ftnt:Fridu}

  In the letter to
  Charlemagne, \citeauthor*{Fridu}{}, states this argument using his
  contemporary conceptual apparatus: If to the question \thi{Is nothing
    something or not?} one \wo{answers \thi{It seems to me to be nothing}, his
    very denial, as he supposes it, compels him to say that something is
    nothing, since he says \thi{{\em It} seems to me to be nothing} [...] But if
    it seems to be something, it cannot appear not to be in any way at all.}
  Then, \wo{if \thi{nothing} is a name at all, as the grammarians claim, it is a
    finite name. but every finite name signifies something. [...] Again,
    \thi{nothing} is a significative word. But every signifying is related to
    what it signifies. [...] Every signifying is a signifying of that which is.
But \thi{nothing} signifies something. Therefore...}


St.~Anselm resolved this linguistic pseudo-difficulty in
quite a Wittgensteinian way, namely, by 
pointing out that \wo{the word \thi{nothing} in no way differs in meaning from
  the expression \thi{not something}} which \citef{indicates that every thing,
  whatever expresses any reality, should be excluded from the mind [...] So it
  is not necessary that nothing be something just because its name in a certain
  way signifies something; rather, it is necessary that nothing be nothing,
  because its name signifies something in this way.}{AnselmFall}{XI} Anselm's
{nothing}, i.e., \wo{\thi{not something} signifies no thing or reality}, it
\citef{puts aside something, without positing anything in the
  understanding.}{AnselmFrag}{C.3} But having nothing in understanding is just
the right way of \thi{comprehending the incomprehensible}, is the whole
\co{concept} one might have of \co{nothingness}.  Things, that is,
\co{visibles}, and even less \co{actual} things and understandable thoughts, do
not exhaust our notion of reality, \co{nothing} may still be both \thi{no thing}
and real.  Quite a similar possibility \noo{duality} appears, for instance, in
the formulation like \citef{the unconditioned meaning viewed as an abyss of
  meaning.}{TilRel}{I:1.1.e}

%MOD
The distinction between the creator and the creation -- of central importance
for all Abrahamic religions, for the ancient Greeks, Hindus and Buddhists, and
one might be tempted to say, for all religion -- appears, for instance, in
\citeauthor*{Sokolowski}{} as \wo{{\em the} distinction}: everything except God
is created by God and so {\em is not} God. Although creator and \co{indistinct}
may seem to have nothing in common, the creation and the \co{distinguished}
certainly have a lot. In our case, the difference between the two does not
depend on any extraneous predicates but is almost analytical...  In any case,
\co{indistinct} may help to take care of the apparent problems with the names
like \wo{nothingness}, or the apparent paradoxes like that \wo{it is the
  \thi{unknowable}~}, that it is \citef{incomprehensibly understandable and unnameably
  nameable}{DDI}{I:5}, that \citef{in every 
  term's signification God is signified -- even though He is unsignifiable}{CusaSap}{
  II:\para 29}, that \citef{Its
  definition, in fact, could only be \thi{the indefinable}~}{Plotinus}{V:5.6}.
No paradox seems to result from \co{distinguishing} the \co{indistinct}
from all that is \co{distinguished}.}

% If not physically, so in any case, conceptually it may be taken as
% \thi{nothingness that is} -- as the \co{indistinct}, this does not (seem to)
% involve any contradiction.]}

\pa\label{pa:essenceBeing}\label{primacyOne} 
So, is \co{indistinct} only by being \co{distinguished} as such?
%Being \co{distinguished} is only the minimal requirement.
Although our point of departure is \co{birth} \co{founding} the
\co{confrontation}, that is, an indissoluble relation between the \co{existence}
and the \co{one}, the latter retains also primacy in spite of this apparent
dependence on the \co{existence}.  \citet{Not that God has any need of His
  derivatives: He ignores all that produced real, never necessary to Him, and
  remains identically what He was before He brought it into being.}{Plotinus}{
  V:5.12.  Eriugena expresses clearly the same thought that \co{immanence} and
  \co{transcendence} of God are not exclusive opposites but 
  complementary \co{aspects}: \citef{the Creative nature permits nothing
    outside itself because outside it nothing can be, yet everything which it has
    created and creates it contains within itself, but in such a way that it
    itself is other, because it is superessential, than what it creates within
    itself.}{Periphyseon}{III:675C}}
\begin{itemize}\MyLPar
\item The \co{indistinct} is \co{one} and the same for all -- \citet{there can
    be only one such Being: if there were another, the two [as \co{indistinct}]
    would resolve into one, for we are not dealing with two corporal
    entities.}{Plotinus}{V:4.1. Agreeing on the conclusion, we certainly do not
    need Plotinus' argument, and we prefer Eckhart's observation that \wo{all
      distinct things are two or more, but all indistinct things are one}
    (footnote~\ref{ftnt:IndistinctOne}). Having characterised God in one way or
    another, one often felt the need to \thi{demonstrate} that it (He?) must be
    only one. From this popular theme, we mention here only a few
    examples like \citeauthor*{Console}{ 766B-767A}; \citeauthor*{Monologion}{ 3};
    \citeauthor*{SumTh}{ I:q11.a3}.]}
%\thi{two  \co{indistincts}} would be almost analytical impossibility.
\item  
It is immutable -- no matter what \co{distinctions} are made, it remains
unchanged beyond and \co{above} them, as the eternal horizon. All
\co{distinctions} belong already to the world, and leave \co{indistinct} behind
-- unchanged, unaffected, untouched.\ftnt{Stoics distinguished the universe from 
  the whole. The differentiated and finite universe was surrounded by the
  infinite and immovable void. Only the two together constituted the whole.}
The \co{indistinct nothingness} does not 
diminish as a 
consequence of all \co{distinctions}, it does not shrink while science makes its
progress; nor as God does in Isaac Luria's process of \la{<<tzimtzum>>}
(\thi{contraction} or \thi{withdrawal}), making place
for the creation through the introvert act of self-limiting withdrawal; nor as
the perfect mixture and unity of the elements, the Whole,\noo{\gre{Sfajros}} has
to dissolve, according to 
Empedocles, into the conflict of the active Love and Hate\noo{\gre{Filii} and
  \gre{Neikos}} between the separated passive elements in order to 
make the emergence of phenomenal world possible.\noo{Powrot, p.63-64} Spatial
analogies may require shrinking or dissolution of the \co{indistinct} as the
\co{distinctions} are made in its texture, but these are only spatial
analogies. 
% The \co{one} remains equally \co{indisitinct} as it always is because its
% \co{indistinctness} is completely unaffected by all the \co{distinctions} of the
% world.
\item  
It is thus not only \co{indistinct} but indistinguishability-as-such, not
something which has not been \co{distinguished} \thi{as yet}, but something
which by its very nature never can nor will be differentiated. It is, we can say,
the ultimate limit of all \co{distinctions}, the limit beyond which no
\co{distinctions} are ever drawn. As it happens, \co{unity} is exactly a limit
of distinctions. As will be shown in Book II (especially, \ref{sub:idealImmed} and
\ref{sub:Identity}), in the sphere of relative \thi{whats} such a limit
establishes the identity of a thing; here it is the \co{absolute unity} of the
\co{one}. 
\item  
It is the \co{origin} of all \co{distinctions} -- not necessarily in
the sense of being the source emanating them in an eternal necessity or else
creating them by an act of free will, but in any case in the sense that all
\co{distinctions} are made into it and arise from it.
\item
  And finally, preceding (in the order of \co{founding}) all \co{distinctions},
  it is indeed not relative to them being made. It remains \co{above} them, as
  their horizon and source. But if no \co{distinctions} were made, then the only
  that would 
  be there would be the \co{indistinct}. It is thus, both as \co{distinguished}
  from the \co{totality} of all \co{distinctions} {\em and} as not relative to
  any \co{distinctions}, that is, as \co{absolute}. 
\end{itemize}
%\pa\label{pa:essenceBeing}
Notice that all the above (except the very last one) apparently positive
characteristics are only the characteristics which, we might say,
\co{indistinct} acquires in the \co{confrontation} with the \co{existence}.
\thi{In itself} it is just ... \co{nothing}, or \co{indistinct}: \citeti{until
  creatures came into existence, God was not \thi{God}, but was rather what he
  was.}{Eckhart}{\btit{German Sermons}, Matt.V:3. [\citeauthor*{EckSelected}{
    22}, \citeauthor*{LW}{ 52}]}

A possible mode of expression:  \la{quid sit} is \la{se esse},
its essence is its Being. % \wo{\co{one}'s essence is Being}. 
We could accept this Scholastic mode of expression but only because ... it does
not say anything. It is amazing how much ingenuity went on trying to either
derive something from this empty idea or, on the other hand, to
nevertheless apply more specific expressions to the \co{one} which, to begin
with, was proclaimed unnameable, \la{esse purum et simplex}.  \co{One}'s essence
is its Being and its Being is \co{indistinct}. This \co{indistinctness} captures
the primacy and independence of \co{one}; 
\co{indistinctness} is exactly that which is totally independent from any being,
for any being appears only as a consequence of \co{distinctions} which do not
affect the \co{indistinct}.  Being (of the) \co{indistinct} is thoroughly
independent from and unlimited by any being or beings. Absence of any relativity
makes this \co{absolute} Being.\ftnt{Indian counterpart to \citefi{I am that I
    am}{Ex.}{III:14} is uttered by Krishna: \citef{Know that with one single
    fraction of my Being I pervade and support the Universe, and know that {\em
      I am}.}{Bhagavad}{X:42}} \co{Absoluteness}, absence of any relativity,
means exactly that as far as it is 
concerned, it simply is, \la{esse purum et simplex}. All specific
characterisations emerge only 
in relation to something else, a \co{separate existence}. 

%\sep

\pa\label{nothingtoknow}
The \co{one} {\em is} and it is \co{above} \co{my}
\co{existence}, \co{above} any \co{existence}. It is \co{transcendent}, ultimate
reality which \co{founds} the reality of all specific things and
\co{distinctions}.  It is not something which merely \thi{appears}, not to
mention mere \thi{appearing for me} -- in fact, it does not appear at all: as
\co{indistinct} it can never appear. 
The constant \co{presence} of this
\co{transcendence} is what makes it for ever impossible for us to accept various
forms of mere immanentism, subjectivism, solipsism -- we know \co{that} is, and
we know \co{that} the more the less it \thi{appears} to our concepts and conceptual
constructions.

We know \co{that} with unmistakable certainty but
this does not imply any \thi{what} -- we know \co{that} God is, but
not {\em what} he is. We can say \co{that} there is, but \thi{what it is} is
already the question about relative, more specific \co{distinctions}.  The more
specific and \co{precise} \thi{whats} we find in search for \thi{objectivity},
the more they threaten with relativity -- in fact, the more \co{objective} they
appear, the more \co{subjective} they turn out to be.  \co{Subjectivity} is
exactly \co{actuality} and narrowing its scope in order to \co{externalise} the
appearing \co{objects} does not help the least to reach something \thi{truly
  objective}.

%\pa\label{nothingtoknow}
Insisting on \co{that} and opposing all \thi{whats} with respect to \co{one}, we
are not trying to actually distinguish these two \co{aspects} which, as already Duns
Scotus observed, can not be \co{dissociated}.\ftnt{E.g., \citef{For I never know
  anything to exist unless I first have some concept of that of which existence
  is affirmed.}{ScotOrdinatio}{I:d3.q1}\noo{p.16} Of course, we replace \wo{concept}
by \thi{what}, and the \equin\ of the two is just the \equin\ of \co{that}
and \thi{what} in every \co{distinction}.}
The inability to say \thi{what} is not due to our imperfect
knowledge and limitations -- it is simply because there is nothing to know about
its \thi{what in-itself}, because there is no \thi{whatness} beyond \co{that}, 
hidden from our view behind the eternal veil.
Even God in Himself is ignorant of God's essence.
This \thi{ignorance} is exactly the proper knowledge of \co{that}, of the fact
that \co{one} is none of the things of creation, that it involves no
\co{distinctions} which first can provide any material for (knowing) 
\thi{what}.\noo{The statements here can be found almost literally in John
  Scotus Eriugena, `Periphyseon' II 593C, 597C-D [64] But of course, analogous
  fragments abide. DDI is probably the most obvious work to mention here.}

\citet{He surpasses every intellect and all sensible and intelligible meanings
  Who is better known by not knowing, of Whom ignorance is the true
  knowledge.}{Periphyseon}{I 510B} \citet{But sacred ignorance teaches me that
  what seems nothing to the intellect is the incomprehensible maximum.}{DDI}{I
  17/51} 
\citet{We do not, it is true, grasp it by knowledge, but that does not mean that
we are utterly void of it; we hold it not so as to state it, but so as to be
able to speak about it. And we can and do state what it is not, while we are
silent as to what it is [...]}{Plotinus}{V:3.14}
Admirable as such quotations may be, they still indicate the
assumption that beyond, behind, above, there hides something which we can not
grasp, although one can not let thinking of it as something \thi{graspable}.
Treating the \co{one} as an epistemological limit, turns it into something
  relative and underlies the 
\co{objectivistic illusion}, according to which there is actually something to
be known, some \thi{essence of all things}, some \thi{maximum}, which isn't
known only because of the finitude of our mind or whatever limitations one wants
to postulate. Eventually, such a \thi{one} threatens with becoming a mere
  \co{totality}, a pantheistic \thi{substance}.
%One is in the best company of negative theologians,
%trying to say something about something, about which they have said
%that they cannot say anything.  

Inheriting the apophatic features constitutive for its predecessor, late
Hellenic henotheism, negative theology says that all
names are inadequate.\ftnt{The
  Christian origin of negative theology must be, of course, taken with serious
  reservations. Etienne Gilson, in a truly apologetic spirit, would like to
  detect Christian roots even of neo-Platonism itself, observing that Origen
  \citef{[s]tudied under Ammonius Saccas, in whose school he perhaps knew
    Plotinus.}{GilsonHCPh}{II:1.2.footnote 14} Observing such common influences,
  one should however mention also the persons like Numenius of Apamea who
  acting in the II-nd century contributed to, if not was chiefly responsible
  for, the transition from platonic idealism to neo-Platonic synthesis. As every
  irresolvable issue, we prefer to leave this one to the scholars and speak
  generally about negative theology, whether of Greek or Christian flavour.} Certainly, there is no need for names. But pretending 
that something hides, that \co{one} is more than the \co{indistinct}
background, is to project the assumed possibility
of \co{distinguishing}, if not any particular \co{distinctions}, into the \co{indistinct}.  It \co{transcends} our being in this
simple sense that this being is constituted by \co{birth} and \co{distinctions}.
However, it is not merely an epistemological limit beyond which no
\co{distinction} is possible. 
One can always draw more \co{distinctions} without in the slightest affecting
the \co{absoluteness} of the \co{one}, without approaching any limit.  It is
\co{that} which is never \co{distinguished}, no matter how many
\co{distinctions} we have made. It is the residual site which always and forever
remains \co{indistinct}.  It is \co{indistinct} and this is the whole and only
truth one can and need say about it. This
%is the whole understanding one can and need have about it,
is the only way of limiting it against that which
\thi{delimits} it -- the world of \co{distinctions}.\ftnt{\noo{This may 
sound like Parmenides, at least in the reading by \ldots }\citef{The ur-object,
  the absolute is not something yet not determined nor yet not determinable, but
something which by its very nature is devoid of any determinations as
such.}{MidMist}{\kilde{p.312}Irrationality of Meister Eckhart}
In short, \citef{[t]he human mind possesses an adequate knowledge of the eternal and
  infinite essence of God.}{SpinozaEthics}{II:47} Of course, there is hardly anything
  in common between our and Spinoza's understanding of \thi{the eternal and infinite
    essence} or its \thi{adequate knowledge.}}

\pa As there is no \thi{what} to be known there, as there are no
\co{distinctions}, the \co{one} seems to be an arbitrary invention. And indeed,
it is -- if
one needs proofs or arguments; any such involve distinctions which never reach
\co{one}.  Being \co{indistinct} means, in particular, that it never appears,
not to mention appearing {\em for} consciousness. Although it underlies all
phenomena, it is itself trans-phenomenal. We would probably have to disagree
with Heidegger that the \citet{specific element of phenomenological understanding
  is that it is capable of understanding something non-understandable, exactly
  when it radically leaves it in its
  non-understandability.}{IntroPhenomReligio}{\kilde{p.123}Supplement to \para 18-19}
 The eventual non-understandable is also trans-phenomenal and we don't
need phenomenology to understand
something non-phenomenal.\noo{There is no point picking on that rather verbal
  issue. We only wanted to emphasize, after others, the Christian and even
  mystical roots of Heidegger.}
%
In the moment we try to \thi{think} or \thi{intuit} (\ger{anschauen}) \co{one},
it becomes an 
empty word, a pure nothingness -- it refuses to appear. One could try claiming
that it is {\em just this withdrawal} which is the phenomenon of the \co{one}, that
\citet{exactly when the Letting-be in a particular way lets a being to which
it relates be, and hence unveils it, it veils being in general.}{vWW}{V. \orig{Gerade
indem das Seinlassen im einzelnen Verhalten je das Seiende sein l\"{a}{\ss}, zu
dem es sicht verh\"{a}lt, und es damit entbirgt, verbirgt es das Seiende im
Ganzen.} It is, indeed, like hearing that the \citef{unchangeable mysteries of heavenly
Truth [...] thou must not disclose to any of
the uninitiated, by whom I mean those who cling to the objects of human thought,
and imagine there is no super-essential reality beyond, and fancy that they know
by human understanding Him that has made Darkness His secret place.}{MysticalTh}{I}}
This, however, is not any phenomenon but only an \co{aspect} of every
phenomenon. 
% then one will never stop looking for the sense and meaning of this \ger{abwesend
%   Anwesenheit}, \ger{gewesend Abwesenheit}, \ger{Verbergung des Unverborgenen},
% \ger{Entbergung des Seienden}, or what else...
The meaning of the trans-phenomenal is itself trans-phenomenal.
%do not appear in any particular phenomenon either.

Except for this small (anti-)phenomenological proviso, we retain
Heidegger's  main statement about importance of understanding by leaving
not-understood, for indeed it
\wo{is better known by not knowing [when] ignorance is the {\em true}
  knowledge}. 
Its certainty is the same as the certainty of my \co{existence}, and this has
nothing to do with adequate or inadequate presentations in any acts of intuition
or apprehension. (This certainty itself might be, perhaps, analysed
phenomenologically.)  It is the most fundamental ontological notion, not any
supra-essential incomprehensibility but \co{nothingness} -- in the fully {\em
  positive} sense of this apparently negative word, as the undifferentiated,
\co{virtual origin}.  It is \co{nothing} but \wo{this \co{nothing} is not
  nothing, for it has the name \wo{nothing}.}
%{Nicolas of Cusa, {\em Dialogue on the Hidden God}}

\pa It is not any \thi{thing in itself} which epistemology has to invent
realising its self-reduction to \thi{objective} knowledge, which immediately
threatens with complete \thi{subjectivity}. It is not any unknowable,
inaccessible $x$. It is perfectly well known, if only we allow ourselves such a
(mis)use of the verb \wo{to know}: we know \co{that}, we know \co{that} is. This
irrevocable certainty, \equi\ with the certainty \co{that} I \co{exist}, has
only one counterpart -- the certainty \co{that} we will die.  These seem to be
the only \co{abolutely} certain things in life. The only things and, as a matter
of fact, one and the same thing. \co{That} which is, becoming \co{present}
through \co{birth}, is the \co{transcendence above} the \co{existence}. To
\co{exist}, to be \co{confronted} means thus also to know \co{that} I am finite.
(Sorry for misusing the word \wo{know} again. We can reformulate it: to
\co{exists} means to live the fact \co{that I am not the master}.) One can, in
principle, imagine a finite being which begins at some time but never ends. But
this is, at least here, but an empty, abstract image, \citeti{that which has
  become has also, necessarily, an end.}{Anaximander}{ DK 12A15
  \citaft{AristPhysics}{203b9}} Beginning is the end -- they are but temporally
differentiated
epitomes %epiphanies
of the ultimate \co{transcendence}.\ftnt{One might probably list some other
certainties of the similar kind, like the fact \co{that} I can not control {\em
  everything}, and the like. But these are only variations over the same theme
of the \co{absolute that}.}

\co{Separation} by \co{birth} \co{founds} thus the fundamental certainties of
life. These, 
deriving from the \co{confrontation} with \co{transcendence}, have all
\thi{negative} character. I know that I will die, but I do not know when, how. I
know that I can not control everything, but there is hardly anything particular
which I could not, at least in principle, bring under my control. And so on, and
so on.  The lack of \thi{positive} content in such certainties opens in fact the
horizon of \co{concrete} freedom -- it expresses only the ultimate \co{that}
\co{above} any \thi{what}, leaving \thi{whats} to the \co{actual} relativity of
\co{existence}.


\pa Delegating \thi{reality} out of the sphere of \thi{knowledge} to \thi{things
  in themselves} is as good as \thi{bracketing} it in order to save the
tranquility of undisturbed epistemological ruminations.  One might imagine us
doing essentially the same, by postulating eventual reality of some
\co{transcendent one}.  However, the intention, tendency and the conceptual
unfolding are exactly the opposite.  If various members of this epistemic family
(\thi{bracketings}, \thi{in-itselfisms}, and likewise scepticisms) did not
pretend that any certainty, which ultimately is the certainty of \co{that},
could be found again among some \thi{whats}, it was only because they could not
believe that it could be found at all. If we have only access to mere
appearances, what hope can we have for any certainty? For {\em if} it could be
found, then it was {\em only} here, among that to which we have access, our
appearances -- beyond the horizon of these \co{visible} \thi{whats} there
remains only the unknown and uncertain.  And since \thi{Being does not add
  anything to the concept} one could and should dispense with it.  We take the
opposite stand saying rather that no concept adds anything (of significance) to
Being; and the concept of Being (if we have any) is {\em only} a concept, trying
to indicate its meaning: the \co{indistinct} is not \thi{unknown} waiting for a
  successful conceptualisation -- it is known
perfectly well, as the \co{indistinct}, as the \co{absolute that}. It is not
uncertain but, on the contrary, the most certain of certainties which
\co{transcends} any relative \co{distinctions} and, in particular, any
  \co{actual distinctions} of reason.  Perhaps, this
certainty means only that, eventually, everything \co{visible} is only relative
and hence uncertain. Perhaps, but there is more to it and we will return to it
in Book III.
  
For the time being, we do not \thi{bracket} the \thi{reality}, we do not
\thi{bracket} the Being \co{above} \co{existence}. We only \thi{bracket}
everything that critiques wanted to save for the rational knowledge, everything
that \la{epoch\'{e}} was supposed to leave untouched -- all the
\co{distinctions} in their relativity to \co{existence}. We do not by this token
refuse them real being, we do not reduce them to mere \thi{appearances}
-- their being, \co{founded} in the \co{one}, is perfectly real, and their
\thi{whatness} may be perfectly \co{objective}. We only refuse that they are the
place of any \co{absolute} \thi{objectivity} -- except for their being, they are
relative to the \co{distinguishing existence}. And so, with few exceptions, we
\thi{bracket} their relevance to our considerations.  Thus, if we were to stick
to the Husserlian metaphor, we do not \thi{bracket} anything after all: we are
not after any \co{visible} explanations which would force us to \thi{bracket}
the inexplicable; we are not after \co{actuality} of adequate intuitions, after
\thi{seeing}, which would force us to \thi{bracket} the trans-phenomenal; we are
not after any specific \thi{essences} which would force us to \thi{bracket} the
significance of the fundamental experiences of \co{existence}.

\pa\label{pa:IndistinctNeedsNo} In short, things indeed are what they seem to be, they are as they appear,
because they appear exactly by being, that is, by being \co{distinguished}. But
\co{above} things there is the immovable \co{origin}, the eternal source which,
in the \co{visible} terms, amounts to the ever present possibility of meeting
something new, finding more.  The art of living is the art of drawing the
borders and everybody has to find the border between these two \co{aspects} so
that the \co{visible} is not threatened by the hidden. \co{Actual} knowledge
remains for ever partial and incomplete, because the \co{non-actual} sphere is
not reducible to the \co{precisely visible} categories. But incompleteness is
not any lack, is not any \thi{subjectivity} to be overcome; when treated this
way it is only a \co{sign} of the desire to reduce the irreducible.  The
\thi{truly objective}, in the sense of absolutely independent, ultimately
inaccessible, beyond and above any \co{subject} and \thi{subjectivity} is only
the \co{indistinct nothingness}.  But inaccessibility does not mean here that
it is completely beyond \co{experience}, that no \co{existence} can ever come in
 \thi{touch} with it.  It means only that the \co{distinctions}
constituting the contents of any \co{existence} never reach it, that all
\thi{whats} remain forever below the ultimate \co{that}.  \co{Existence} is in
fact defined by the \co{confrontation} with it and so we know it, for we know
\co{that} it is, we only can not know \thi{what} it is -- simply, because it
{\em only is}, undisturbed and unaffected by anything, as indifferent as
undifferentiated.
%, it {\em is} but it is not any \thi{what}.

%\tsep{...end ins...}


\subsubi{Asymmetry of being}\label{asymm}
\pa\label{asymBeing}
\co{One} is, but we are very far from saying that it is all that
is. The first \co{distinction} is that between the \co{indistinct} and the
differentiated, and 
everything that is \co{distinguished}, the whole \thi{sublunar world} {\em is},
too.  
We won't dwell on various kinds and hierarchies of beings, on species and genera,
but we want to register a dimension of possible distinction which will be of
relevance to us. 

The fundamental distinction concerns \thi{being} vs. \co{existence}.
\co{Existence} is \co{separated} directly from the \co{one}, not as its part,
but rather an image, in any case, as a \co{confrontation}. As such, it is not
relative to anything except the \co{one} from which it emerges. \thi{Being} of
all other beings means to be \co{distinguished}, i.e., it is relative to the one
who \co{distinguishes}. What may matter then is what something is
\co{distinguished} from.

$X$ being \co{distinguished} from $Y$ means that \thi{$X$ is $Y$} and this (one
might be tempted to say: relation) is asymmetric. In the most crude sense of a
copula, the fact that \thi{this pen is blue} says something about its being, but
this does not mean that \thi{blue is this pen}.
% This trivial case of applying a general predicate to a particular thing is,
% probably, the fundamental sense of the copula.
We might say that such a predication is not merely an assignment of some
accidents to some substance, but actually amounts to \co{distinguishing} this
pen from other blue things, and that this is what accounts for its \thi{being}
blue.  But at such a level of \co{actualities}, distinction has also often the
character of a \thi{distinction against}, as when saying that \wo{this pen is not
  that ball}, and we should not attach all too much meaning to the verbal
expressions which may easily lead to confusion of \thi{being} and \thi{not
  being}. It may, indeed, be a sign of pantheism to say that \la{omni
  determinatio est negatio}, for in such a case we seem to have only one, as if
horizontal dimension, where things may be only more or less, larger or smaller
parts of the whole, and where they are distinguished only mutually from each other.

\pa Important differences in `being' depend not on how big a part of some
imagined, though unimaginable totality, $X$ is, nor against which other
particular $Y$ it is distinguished, but on the background {\em from} which it is
\co{distinguished} or, more specifically, at what level of
\co{founding} it is \co{distinguished}. 

The most fundamental \co{distinction} is \co{birth}, \co{separation} from the
\co{one}, yielding units from the \co{unity}, \gre{henads} from the \gre{monad}.
Being \co{separated} from \co{one}, \co{existence} {\em is} \co{one}. 
But this does not mean that \co{one} is the \co{separated} being; in fact, it is
not, for it is just the \co{transcendence} which makes the \co{existence}
\thi{be} -- not by coinciding with it but by \co{confronting} it. (Cf. the
remarks on the primacy of the \co{indistinct} in \co{confrontation},
\refp{primacyOne}.)  \citet{I am not in them; they are in
  Me.}{Bhagavad}{VII:12} says Krishna about all the states of created beings.
Misusing the pantheistic mode of speaking, one might say that \co{one} is
\thi{more-than} \co{existence}, but we understand this \thi{more-than} rather as
the \co{confronting transcendence}.  All other ways of \thi{being}, down to the
most \co{actual} predicative copula, repeat this asymmetric pattern and are
\co{founded} in the primordial \co{separation}.\ftnt{\citef{[P]redicative
    \thi{is}, used in the context of theoretical explication, has its source in
    the original \thi{I am}, and not vice versa.}{IntroPhenomReligio}{
    II:3.\para 24} \noo{pp.24[p.85]} } 

\co{Chaos}, the primordial element of \co{existence}, as the first hypostasis, {\em
  is} \co{one}. But \co{one} is not \co{chaos}, it remains \co{above} and beyond
\co{chaos}, as the \co{transcendent unity} of differentiation. Speaking a bit
paradoxically, the \co{one} is the limit beyond which \co{chaos} ceases to be
chaos -- it is only by being its own limit that something at all \thi{is}.
\co{Experience}, arising from \co{chaos}, {\em is} \co{chaos}. This does not
mean that it is chaotic, only that \co{chaos} underlies it, is its \co{founding}
element. And again, being such a \co{founding} element, it remains beyond and
\co{above} \co{experience} -- \co{chaos} is not \co{experience}, but it may
appear as the horizon which limits the \co{experience}. Finally, \co{reflective
  experience} with its beings-at-hand is \co{experience} but not vice versa;
  \co{experience} is the 
limit of \co{reflection}, usually called \wo{its beginning}. In short, \citet{there
  is from the first principle to ultimate an outgoing in which unfailingly each
  principle retains its own seat while its off-shot takes another rank, a lower,
  though on the other hand every being is in identity with its prior as long as
  it holds that contact.}{Plotinus}{V:2.2}

% Seeking nothing, possessing nothing, lacking nothing, the One is perfect and, in
% our metaphor, has overflowed, and its exuberance has produced the new [... ]

%Immanence and transcendence
\pa\label{pa:immanentTranscendent}
The asymmetry of being corresponds to the fact that higher level,
\co{founding} the lower one and thus constituting its \thi{being}, is not
accessible to the categories of the lower level.  %\refp{pa:keepDistinguish};
The unity of the higher level is at best reflected only as some ideal
\co{totality} of the 
\co{distinctions} of the lower level, but such \co{totalities} never sum up to yield
the \co{unity} they only imperfectly reflect. 
%{Notice that we are not talking about the particulars of each level. They
%\thi{are} rather merely as distinguished from other things at the same
%level. But this \thi{being} is again \co{founded} in the unity of the current
%level, i.e., in the higher one...}
Put a bit differently, if \thi{$X$ is $Y$}, the asymmetry means that $Y$
\co{transcends} $X$, is \co{above} $X$. But at the same time, $X$ is \thi{in}
$Y$, participates in it, and thus $Y$ is thoroughly \co{present},
\co{immanent}. Thus although, for $X$, $Y$ appears remote and inaccessible, it is
in fact most intimate and close. Taken to the extreme, every \co{existence} is
\co{one} but \co{one}, seen from the perspective of the \co{actual existence},
is remote and \co{transcendent}.  \citetib{Nothing, however, is completely severed
  from its prior.}{Plotinus}{V:2.1}  \co{Confrontation} with it
constitutes the very being of \co{existence}, and thus \co{one} is most
intimately \co{present} around it.


\subsubi{Against pantheism}\label{sub:pantheism}
\pa
Everything is \co{one} before it becomes two.  All \co{distinctions} originate
from \co{one}, and this might make one think of \co{one} as the mere sum of
everything. But Being is asymmetric -- every thing is (from) the \co{one} but
\co{one} is not everything. %(\refpf{asymBeing}).
It is not any sum, not any \co{actual} totality -- it is the \co{virtual
  origin}.  \co{Totality} of \co{distinctions} is only that -- a \co{totality}. It does
not sum up to any \co{unity}, because it does not sum up at all.  \wo{Everything} is
but an expression we find for \co{nothingness} in the differentiated world of
feelings and concepts, eventually, of \co{reflection}.  But this
\wo{everything}, meaning indeed the totality of things, points to
\co{nothingness}, their unity which in no sense belongs to \co{this world}. It
is, in fact, thoroughly \co{transcendent}, for once the \co{distinctions}
  dispersed the \co{one}, to 
 \co{reflect} it eventually as the \co{totality} of \thi{everything},
\co{nothingness}
%remained and remains
withdrew and remains  \co{above}, unreachable through the categories of
\co{distinctions}.\ftnt{Thus, not even the label \wo{panentheism} is appropriate
  to the extent it involves any spatial associations of a universe somehow
  \thi{contained within} the God. Absolutisation of feeling, as it often happens
  in modern forms of panpsychism following panentheism (e.g., Gustav Fechner,
  Alfred North Whitehead), or of life as it happens in the organismic analogy of
  Charles Hartshorne, amount to a conflation of the \co{transcendent unity}
  of the world with the \co{totality} of its (feeling, living) contents -- the
  identification characteristic for pantheism. But to the extent panentheism
  points towards the intimate presence of God and his {\em both} aspects --
  transcendence and immanence -- the label might be acceptable.} \citet{The First remains intact even when other entities spring from
  it.}{Plotinus}{V:5.5}

\pa Although \co{absolutely transcendent} and never \co{actual}, \co{one} remains
\co{present} in everything by virtue of being the \co{origin}.
For as the lower hypostases bring us more and more within the sphere of
\co{actuality}, the higher hypostases do not disappear as stages of a
development to be left behind, but remain \co{present}.  In more \co{actual}
terms, we might say that they  remain \co{present} as the horizon surrounding,
and lending \co{unity} to the \co{totality} of the lower level. \co{Actual presence} of
the \co{one} is merely as the horizon of distinguishability, the horizon beyond
which no more \co{distinctions} are made and whence all new \co{distinctions} emerge.
%But \co{presence} has also another aspect. It is not only the
%\co{presence} of the \co{origin}, but also of the horizon.

No matter what and how many \co{distinctions} populate our world, this world is
surrounded by the sphere of the \co{indistinct}. This sphere is
not merely something which is not-yet-distinguished but something which is the
indistinguishability-as-such. New \co{distinctions} can emerge from it without
ever diminishing its character and scope, without ever violating
its \co{absolute transcendence}, its \co{original indistinctness}. The more
and more \co{precise distinctions}, entering gradually the sphere of
\co{visibility}, leave by this very token behind and \co{above} them the
\co{indistinct origin} which surrounds them as a horizon from which always more
new \co{distinctions} can emerge.

\pa \co{Birth}, \co{confronting existence} with its \co{origin}, preserves the
latter -- the \co{existence} is \co{separated} {\em from} the \co{origin}. But being
thus \co{separated} {\em from} the \co{origin} does not mean to be its part. 
Neither \co{existences} nor other
\co{distinctions} sum up to some \co{posited} \co{totality} of the
{absolute}; on the contrary, every \co{existence} has a full contact with the
\co{absolute}, \co{participating} in it.  \co{Participation} (in the current,
merely ontological sense) means simply the constant \co{presence} of \co{one} in
every \co{existence} -- analytically, since \co{existence} is just the
confrontation with the \co{one}.  Yet, as will be elaborated in
\ref{pa:OneOnlyVirt}, \co{one} is \co{present} only through \co{chaos} and
further hypostases. \co{Existential confrontation} happens only through these
lower levels, eventually, it finds always place in the \co{actual} situation.
The \co{presence} of \co{one} is an \co{aspect} of every \co{actual experience},
and thus it is always \co{experienced} but it is never the \co{actual object} of
\co{any experience}.

\co{Participation} has nothing to do with \thi{being a part}, as the Latin
etymology might suggest. Parts do not \co{participate} in their \co{totality},
they form or constitute it.  \co{Participation} corresponds rather to the Greek
\gre{metousia} which means \thi{to have being after}, as we say, \thi{to be
  \co{founded} by}.  True, to be a part is an aspect of being
\co{distinguished}, but only in so far as we imagine \co{objects} extracted from
the background. But a \co{distinguished} being is not a part of the
\co{indistinct} but only of the differentiated world -- by the very token of
being \co{distinguished}, it ceases to be a part of the \co{indistinct}.

\pa
According to pantheism, being a part is the primordial mode of participation.
Pantheism assumes that the background is somehow given as already
differentiated. It identifies, on the one hand, the \co{transcendent indistinct}
with the \co{visibly distinguished} and, on the other hand, this \co{visibly}
differentiated with the \co{totality} of basic entities. In this respect it is
similar to empiricism, and it steps beyond empiricism only by postulating the
\co{totality} of all differences (and ascribing them some divinity).

It takes severe \co{objectivistic illusion} to make these identifications. 
If we were to emphasize these distinctions, along with the distinction between
being and \co{existing}, we could say:
\begin{itemize}\MyLPar
\item[i.] to be is to be \co{distinguished} -- and as such to be a part of the
differentiated world, relative to \co{existence} but also, by the same token,
not to be a part of the \co{indistinct};
\item[ii.]
to \co{exist} is to be a
\co{distinguished} image of the \co{origin}, \la{imago Dei} -- not in the sense
of being a part of it, but of being \co{founded} by \co{participation} in it; to
some extent, to be like it, in that \co{nothingness} of \co{self} and
\co{nothingness} of the \co{one} are the same, but primarily  only in the sense of
being its \co{separated reflection}, an event of a direct \co{confrontation}
with it, \refpp{pa:imago}.
\end{itemize}

\pa\label{immanentTranscendent}
We observed in \refp{pa:immanentTranscendent} that
\co{one}, being most intimately \co{present} and \co{immanent} is also, unlike a
pantheistic substance, ultimately \co{transcendent}. 
\co{One} is \co{present} not {\em in} every thing but only {\em behind} every
thing; not because every thing is \thi{its part}, but because every thing points
to it, being surrounded by the \co{invisible rest} which eventually leads to its
\co{origin}. Likewise \co{existence} is not \thi{a part} of it but
\co{participates} in it, is its image in the sense of being eventually
constituted by the \co{confrontation} with it.  Remaining undifferentiated (and
indifferent) \co{above} not only the \co{totality} of all \co{distinctions} but
also the \co{existences} through which things come forth, \co{one} is ultimately
\co{transcendent}, inaccessible to any \co{actual} look. At the same time, it is
most intimately \co{present}, as the site of every \co{existential unity} and
the ultimate horizon from which everything originates and in which it
\co{participates}. \citet{All these things are the One and not the One: they are
  He because they come from Him; they are not He, because it is in abiding by
  Himself that he gives them.}{Plotinus}{V:2.2 [translated by A.~H.~Armstrong]}

\noo{
Although everything participates in the \co{indistinct origin}, the sharper
\co{distinctions} of \co{reflection} become, the easier it is to \co{posit}
things as independent, non-participating, individual entities, that is, the
easier it becomes to bereft them of Being by \co{dissociating} them.  Their
being is thus fragile, depends on the way we \co{distinguish} them.  The only
truly ontological event is the one taking place in the center of Being, that is,
of \co{nothingness} -- it is the creation of the world, the true beginning,
\co{birth} from the \co{virtual nothingness}.  All the rest of
ontology is indistinguishable from epistemology. We might say that \co{presence}
of \co{one} in \co{visible} things and contents is \thi{analogical}, by virtue
of the \co{existence} which makes these contents \co{visible} and which is the
constant witness to the \co{one}.\ftnt{John Scotus Eriugena could be quoted
  extensively here.  As Deirdre Carabine puts it \citef{The idea that God is
    manifest in creation is true, and the fact that God remains transcendently
    unmanifest is also true.  And yet, {\em neither are true when understood
      singly}: the `problem' is resolved by coupling both truths in a
    dialectical formulation that reveals the tension between, and the
    simultaneous truth, of both.}{adEriugena2}{ p.49} 
  Accusations not only of pantheism (the apparent reason
  for the condemnation of his writing), but also of deism, might be taken as
  indications of the correctness of some basic intuitions.}
}


\subsection{Virtuality}\label{sub:virt}

\subsub{Two kinds of causes} \pa\label{pa:foundingCausing} As Plato, or rather
his neo-Platonic version, had been sieved through the Aristotelian categories, the
distinct hypostases and stages of emanation became gradually replaced by various
kinds of \thi{causes}: the One became the Primordial Cause, the multiplicity of
\gre{henads} became a multiplicity of Primary Causes, then secondary causes,
etc. Of course, things were not so simple and to reflect the essentially
different order of \co{founding} one had to significantly adjust the four causes
of Aristotle.  John Duns Scotus for instance, divided the \la{ordo essentialis},
concerning the ontological order of being, into \la{ordo eminentiae} and
\la{ordo dependentiae}.\ftnt{\citeauthor*{ScotPrimo}{ I:6-I:9\noo{p.4+112}}. Only
  slightly different classification of causes can be found in
  \citeauthor*{ScotOrdinatio}{ I:d2.q1 [The existence of
    God]}\noo{p.40,45,49,58}.}
%Cusanus, DDI, e.g. II 9/142, and many other places
Unfortunately, the same words are used for \thi{causality} in both (as
well as other) orders and this may easily cause confusion. Nevertheless, the
\thi{causality} in the order of excellence or, as we might say, of
\co{founding}, is certainly of different kind than that in the order of
dependence.  With some good will, the later could be said to correspond to usual
causality while the former to the hierarchy of \co{virtuality}, of higher and
lower levels of Being.  A form of causality in \la{ordo eminentiae} might then
be interpreted as \co{actualisation} of \co{virtuality}.\ftnt{These are, of
  course, remote analogies, not any detailed relations.  \citef{[W]e may
    distinguish two sorts of causes, the one divine and the other
    necessary}{Timaeus}{ my: III.37} could be taken as the original form of this
  distinction. }
%

\pa\label{pa:contActVirt} \co{Actualisation} of \co{virtuality} is to be sharply
distinguished from the actualisation \fre{\'{a} la} Aristotle which only
materialises one among the given possibilities.

The \co{origin} does not \co{actually} contain all the hypostases -- they are
present only \co{virtually}.  This, as with Bergson, is the difference between
\thi{possibility} and \thi{virtuality}.  The former contains its realisations in
their actual form, so to speak ready-made, and realisation is a mere selection
or, in the case of general concepts, specialisation and, eventually,
individuation by a mysterious conjunction of immaterial form with matter.  But
the relation between higher and lower, the \co{founding} and the \co{founded},
is not that of the general to the particular, that of instantiation and
specialisation. It is the relation of \co{expression}, possibly of incarnation,
and in the ontological form addressed so far, that of \co{actualisation}.
\co{Virtuality} of \co{birth} is the very site of individuality which contains
only potentially its \co{actualisations} and \co{actualisation} is
differentiation.  In particular, what emerges as its result is entirely different
from, in no way similar to that from which it emerged.\ftnt{The example, so
  beloved by the hylomorphic dualists, of a \citef{perfect artisan [who] has a
    distinct knowledge of everything to be done before he does
    it}{ScotOrdinatio}{\noo{p.61}I:d2.q1.a2} hardly applies to any more genuine
  creative activity than a mere construction work following plans and drawings
  made usually by somebody else.  An artist does not have any precise,
  ready-made \thi{form} in mind which he so \thi{applies} to the \thi{matter}
  \thi{actualising} its \thi{possibility}.  Starting a work he will typically
  have a more or less vague intuition which lacks any precise form.  Indeed,
  \citef{how irritating is this introductory phase when one has to fetch from
    within the first shape of the work, so awkward, not yet enriched with all
    the tiny inspirations which the pen will only encounter later
    on.}{Diaries66}{1957:II\kilde{p.18}} The process of artistic work is exactly
  the process through which this vague intuition {\em for the first time} finds
  an actual form and expression; it is like birth and not like causation.
  \citef{A true artwork emerges <<from the artist>> in an arcane, mysterious and
    mystical way. Detached from him, it becomes an independent live, a
    personality...}{Kand}{VIII\kilde{p.132}} Whether this happens during the
  actual performance of the work of art, or else in advance, does not matter nor
  does it change the basic structure of \co{actualising} a \co{virtuality},
  rather than \thi{realising} a \thi{possibility}. \citef{[T]he artist himself
    goes to back, after all, to that wisdom in Nature which is embodied in
    himself; and this is not wisdom built up of theorems but one totality, not a
    wisdom consisting of manifold detail co-ordinated into a unity but rather a
    unity working out into detail.}{Plotinus}{V:8.5}
The insistence on plain \co{visibility} in search of explanations can not admit such mysteriously \co{virtual} elements. It pervades all attempts at reductions,
the most recent one being that of thought to its expression. \citef{What happens
  when we make an effort -- say in writing a letter -- to find the right expression
  for our thoughts? [...] Now if it were asked: <<Do you have thought before
  finding the expression?>> what would one have to reply? And what, to the
  question: <<What did the thought consist in, as it existed before its
  expression?>>}{WittPI}{ I:335} The latter question can be, indeed, hard to answer
\co{precisely}, but this need not imply negative answer to the former. The lack
of \co{actual} linguistic expression need not mean the lack of any
expression. The very {\em effort} to find {\em the right expression} is
itself an expression of some \co{presence}. 
  
  A \thi{possibility} is a category of
  \co{actuality}, it is something definite and \co{actual}, even if only
  potentially.
%% Cusanus againts `absolute possibility' = `matter',  
%% DDI II 8/132,136,137
 % 
  \co{Virtuality} should be distinguished not only from \thi{possibility}, but
  even from latency, which is a kind of middle notion. Latency does not contain
  a ready-made result of its future actualisation and in this it resembles
  \co{virtuality}. But it has only one, or very few, possible actualisation(s),
  so that these can be determined and predicted in advance. Thus, like
  \thi{possibilities}, latency moves within the determinations of
  \co{actuality}. \label{ftnt:artist} }
%
What Plotinus says about \thi{forms} applies to \co{distinctions} arising in the
process of \co{actualisation}: \citet{Form is only a trace of that which has no
  form: indeed, it is the latter which engenders form.}{Plotinus}{VI:7.33.
  Although Plotinian emanations follow some principle of
  similarity, one should not confuse that with any kind of
  causality. Commenting on
  V:9.2 \wo{What then is it which makes a body beautiful?}  a scholar remarks
  \citef{I take here \gre{to poiesan} as meaning \thi{the principle responsible
      for the existence of an instantiated character in its bearer}, which
    amounts to excluding both the meaning of efficient cause -- in so far as
    this involves deliberation and change -- and the one of creative cause -- in
    so far as this conveys the idea of a beginning out of nothing. None of these
    meanings fits well, with the Plotinus' account of the causality of the
    intelligible principles.}{SeparateForms}{II:footnote 29} Thus, both
  emanations and our \thi{emergence from \co{virtuality}} are quite distinct
  from causality.}

\pa\label{pa:nexus}
The \co{virtual origin} is a \nexus\ of \co{aspects} which cannot be
{dissociated} from each other without changing their
character. \co{Actualisation} amounts exactly to such a \co{dissociation},
giving rise to new elements and forms and, in most general case, to new levels
of being. We have illustrated this general process in Sections \ref{se:nothing}
through \ref{se:reflection}, and we have seen several examples of more specific
\nexuss\ giving, eventually, rise to various elements of
\co{actuality} (e.g., \co{one}--\co{signification}--\co{sign}--\co{sign as a sign};
\co{one}--\co{simultaneity}--\co{spatio-temporality}--space\&time;
\co{confrontation}--\co{awareness}--\co{consciousness}--\co{reflection}).

There are, though, no clear lines separating one level from another, just like
there are no definite limits separating a baby from a child, a child from an
adolescent.  All is a continuous process without sharp boundaries except
those used for the purpose of description.  Nevertheless, the distinctions
of nature, which we ascribe to different levels, are thoroughly real, just as is
the difference between an adult and a child.  They mark emergence of more
differentiated systems from the prior \nexuss\ of \co{aspects}, of more involved
and sharply distinguished elements, which were present only as a \co{virtual}
germ at the previous levels.  The nature of a new level can not be explained in
terms of the previous ones, it can not even be understood in such terms.  It
requires new concepts for expressing a more complex interplay of several
aspects.  In this sense, there is a qualitative \thi{spring} between levels.
Yet, viewed as stages of the process of differentiation and \co{actualisation}
of the \co{virtual} \co{origin}, they are but distinctions of degree.

\label{pa:preserves}%should be after next \pa, but then something is wrong
\pa Most importantly, {\em all} levels belong to every
\co{experience}, all preceding levels and forms of \co{transcendence} remain
\co{present}, though not \co{actual}.  The successive stages are not passed to be
left behind -- they accumulate. \citet{We have not been cut away; we are not
  separate; [...] we breath and hold our ground because the Supreme does not
  give and pass but gives on for ever, so long as it remains what it
  is.}{Plotinus}{VI:9.9. Likewise, albeit a bit metaphorically, the fruit
  contains in itself the seed from which it arises: \citefi{the fruit tree
    yielding fruit after his kind, whose seed is in itself [...] the fruit of a
    tree yielding seed}{Gen.}{I:11/I:29}} Results of earlier \co{separations}
are gathered underneath the later ones.  Beyond the actual horizon of each
relation, there remains the background -- not only as merely \thi{more objects}
of the same kind, but as something truly inaccessible to {\em this new} form of
relation, as an \citt{indefinite murmur of being}{E.~Levinas} under the newly
emerged, newly differentiated, visible surface of things.

This means that every \co{actuality} remains involved in the \nexus\ from which
it emerged and interwoven with the other \co{aspects}. Even if, at the
\co{actual} level, the \co{aspects} have been completely \co{dissociated}, the
\co{presence} of the whole \nexus\ is marked by a \co{trace} which, usually,
takes on the form of a relation between the \co{dissociated} and substantialised
entities (e.g., the \thi{relation} of \co{meaning} between the abstract
\co{sign} and the signified, which is the \co{trace} of \co{sign}; the relation
\co{after} which is the \co{trace} of the \co{unity dissociated} into
\co{subject} and \co{object}; the objective space and time \thi{relating}
distinct \thi{places} and \thi{nows} as the \co{traces} of
\co{spatio-temporality}). Something similar to a \co{trace} was called by James
a \wo{conjunctive relation}.\ftnt{\citeauthor*{PureExp}} The fact that $X$
appears (always) {\em with} $Y$ shows the relation of \thi{withness} which is as
real an element of experience as are $X$ and $Y$.  The fact that, e.g., knower
and known always go together, that continuous transition involves a very close,
inner connection between its phases, are to be taken as facts of experience
showing \wo{different degrees of intimacy}. The degree of intimacy might be
taken as the inverse of the degree of \co{dissociation}, though James would not
ascribe to it ontological priority we ascribe to \nexuss.  A more genuine
analogy may be found in Duns Scotus' \la{distinctio formalis}: the \thi{formal
  distinctions} are distinctions between \co{aspects} which, although formally
distinguishable and truly distinct, remain really inseparable.\ftnt{They are
  not, however, mere 
distinctions of reason alone, \la{distinctio rationis tantum}, which Scotus
considers to be {\em caused} by the reason as, say, the distinction between a
definition and the thing defined. Thus, for instance, soul was
formally distinguished into its faculties, but even God could not posit will as
something existing without soul and its other faculties. A predecessor of this
idea can be discerned in Plotinus' hierarchy of (the concepts of) numbers. On
the one hand, (the lowest) numbers stand for mere quantity: \wo{You take one
  thing with another [...] a dog and a man, or two men; or you take a group and
  affirm ten, a decad of men: in this case the number affirmed is not a Reality,
even as Reality goes in the sphere of sense, but is purely Quantity.} On the
other hand, there are true and essential numbers which precede and found their
quantitative applications: for \citef{the case is different when you consider
  one man in himself and affirm a certain number, duality, for example, in that
  he is at once living and reasoning.[...] this is another kind of number;
  number essential; even the duality so formed is no posterior; it does not
  signify a quantity apart from the thing but the quantity in the essence which
  holds the thing together.}{Plotinus}{VI:6.16}} It is equally crucial
to us to keep the \co{trace} of the \nexus\ from which all \co{aspects} emerge, as it
was for Scotists to stick to the unitary reality of formally distinguished
aspects.  The difference here may concern the fact that, in our case,
differentiation and even \co{dissociation} does take place and may have
significant consequences: only by differentiation can a \nexus\ give rise to
something new, staying however unchanged and the same above its differentiated
contents.

\Nexus\ is not a
term of explanation, for Cartesian explanations are its exact opposites; it does
not provide sufficient reasons nor efficient causes. It is the term of the
origin, indicating only that some things go together, not in a mere
\thi{togetherness}, but in a most intimate and original closeness -- even if we
can \co{dissociate} them and \co{posit} them as independent entities, they
remain inseparably bound together by their origin in the same \nexus. \Nexuss\ 
are like \la{logoi spermatikoi} of Stoics, the \thi{rational seeds} of active
matter which, after every turn of the Great Year when cosmos has again dissolved
in the elemental fire, \gre{ekpyrosis}, initiate its regeneration always
following their immanent principles of growth.\kilde{Koncepcja wiecznego
  powrotu...p.78: Diogenes Laertios VII:136-137 + p.72,77,81}
Commenting on Parmenides' poem, a
scholar remarks: \citet{Parmenides creates here the impression of the archaic
  argumentation in which, once the system and the convictions are given, the
  premises and conclusions tend to appear in the presentation as merely put next
to each other.}{ParmenidesIt}{\kilde{p.52}[\orig{In der Tat erweckt Parmenides
  hier den Eindruck des 
  archaischen Argumentierens, wo die Pr\"{a}missen und die Folgerungen, wenn das
  System und die \"{U}berzeugung einmal da sind, dazu neigen, in der Exposition
  blo{\ss} nebeneinandergestellt zu werden.}]}
%{G.~Calogero, {\em Studi sull'eleatismo} [Parmenides p.52]}
%
\Nexus\ 
is the central element of such an archaic understanding which does not attempt
to \co{dissociate} things and make them more \co{precise}, but rather to keep
them as \co{vague} as they originally appear in the barely differentiated but
as-yet-not-\co{dissociated} mixture of mutual dependencies, \citet{opposites that still are not opposed.}{Plotinus}{VI:6.3}  The ultimate
\nexus, \citetib{[t]he One is all things and not a single one of them: it is the
  principle of all things, not all things, but all things have that other kind
  of transcendent existence; for in a way they do occur in the One; or rather
  they are not there yet, but they will be.}{Plotinus}{V:2.1 [translation of
  A.~H.~Armstrong]}

%\newp 

%\subsub{stereograph project}

\pa\label{pa:stages} The series of drawings below %in Figure~\ref{fi:stages}
captures some of the
essential aspects of our development.  We should speak here in three
dimensions, with plane representing the \co{one} and sphere the
developing \co{existence} but, for simplicity, let us draw it in two
dimensions.\ftnt{As usual, one should be careful with not pushing 
such analogies too far where they necessarily break down.  However,
they can often be quite useful, if only taken with a bit of
salt.}
%\begin{figure}[ht]%\refstepcounter{FIG}
\[\xymatrix{
0) & \ar@{-}[rrrrrrrr]  & & && && && & \\ 
1) & \ar@{-}[rrrrrrrr]  & & &&\bullet & &&&&
}
%\vspace*{4ex}
\]
The line 0) represents the \co{indistinct one} and the point 
$\bullet$ in 
1) the \co{birth}. The \co{born} being begins to \thi{grow} which is 
represented by the gradually larger circles. The main analogy concerns 
some properties of this, so called stereographic projection, which 
become effective in the moment the point turns into a circle (for 
instance in 2). 
\[\begin{array}{l}
\xymatrix@R=0.3cm{
2)& && \ar@{.}[rrr] & & & & & \\  & & & &  & & \\ 
& && & \Drop{\ci{6}} &      & & &  \\ 
 & \ar@{-}[rrrrrrrrr]  & &
\ar@{-}[ur]+<5.5mm,6mm>^>>>>>>>{A'} 
\ar@{}[ur]+<6.3mm,4.7mm>^>{\displaystyle\bullet}
&& & 
  &&  \ar@{-}[ullll]+<5.5mm,6mm>_>>>>>>>{B'} &&& \\
& && {\scriptstyle{A}} &&& && {\scriptstyle{B}} &&
}
%\vspace*{5ex}
\\ 
\\ 
\xymatrix@R=0.35cm{
3) & & & & \\
& && \ar@{.}[rrrr] & & & & & \\ & & & & \ci{15} &  & & \\ 
& && &  &      & & &  \\ 
 & \ar@{-}[rrrrrrrrrrr]  & & 
\ar@{-}[uuur]+<10mm,5.5mm>^{A'}  
\ar@{}[uuur]+<12mm,4mm>^>{\displaystyle\bullet}
&& & && & && \ar@{-}[uuullllll]+<0mm,5.5mm>_>>>>>>>>>>>>{B'} && \\
 & && {\scriptstyle{A}} &&& &&&&& \scriptstyle{B} &&
}
\label{fi:stages}%% \caption{The stages}
%\end{figure}
\end{array} 
\]
\noindent
There is, namely, a one-to-one correspondence between {\em all} the points on
the infinite line and all the points on the circle. The mapping is obtained by,
starting with a point on the line, say $A$, drawing an imaginary line to the
pole of the circle marked with $\bullet$. The point where this line intersects
the circle, $A'$, is the image of $A$.
%It is obvious (though for the untrained person, perhaps, a bit surprising) that a
All the different points of the infinite line will thus be mapped to different
points on the finite circle and vice versa.  The point at which the circle
touches the line will be mapped to itself. The points close to it will be
relatively exactly mapped on the lowest part of the circle.  The further away
from the circle the points lie on the line, the \thi{denser} will they be mapped
to the points closer to the pole $\bullet$.

The pole $\bullet$, the \thi{origin} is, too, an image of something originating
from the line. Of what? Of its infinity.  Two lines are parallel if, being in
the same plane, they do not intersect. Put in a somehow more abstract language:
two lines (in the same plane) are parallel iff they intersect in
infinity. The further from the circle we move, the closer to the \thi{origin} the
images of the points will be; the line determining the image $B'$ of $B$, as $B$
moves toward infinity will be \thi{more and more parallel} to the bottom line.
The two lines: the bottom one and the one parallel to it and touching the circle
at its top pole, will intersect in infinity.  The \thi{origin}, reflecting the
so called \wo{point in infinity}, is thus the image of the infinity of the line
on the finite (but unlimited) figure of the circle.\ftnt{\citef{The soul is not
    a circle in the sense of the geometric figure but in that it at once
    contains the Primal Nature [as centre] and is contained by it [as
    circumference], that it owes its origin to such a centre and still more that
    the soul, uncontaminated, is a self-contained entity.}{Plotinus}{VI:9.8}}

2) represents (an early) stage of \co{experience} with \co{chaos} lying
somewhere in-between 1) and 2), one could say, immediately after 1), when the
circle is still very small.

The short dotted line corresponds to the level at which \co{actuality} is
constituted as \co{distinguished} from \co{non-actuality}.  3) represents the
stage of \co{reflective experience}. The circle became big enough to cross this
line which now marks two spheres: what lies below it (e.g., $A'$) represents the
\co{actual} and what lies above (like $B'$) the non-\co{actual}, and eventually
the \co{non-actual}, aspects of \co{an experience}.

What lies on the circle \thi{under} the dotted line represents the
\co{actuality} which we can also characterise as simultaneity. With respect to
2) this means that all the \co{distinctions}, all the images on the circle are
simultaneous.  Time has not begun to flow and all \co{distinctions} still
coexist in a manner similar to the \co{chaotic} co-presence.  At 3) the
simultaneity becomes limited to the \co{actual} contents, to the \hoa.
%horizon of \co{actual} perception.


\pa %%%WRONG placement of the paragrah number
Imaging further \thi{growth} of the circle, we would soon reach the 
stage where the \co{actual} part is almost negligibly small compared to the 
non-\co{actuality} above it.

We can also point out how the \co{origin} -- the pole, and the \co{chaos} 
of \co{distinctions} -- the 
\thi{dense} images of the remote points compressed close to the pole, withdraw during 
the \thi{growth} further and further away from the \co{actuality}: the 
\co{distance} between the two is marked by the growing number of \co{distinctions} 
which separate them, the points on the circle between the dotted line 
of \co{actuality} and the pole $\bullet$.

\label{pa:density} Finally, imagine the circle \thi{moving} along the bottom line. 
As the circle in 3) \thi{moves} to the right, the image $B'$ of the point
$B$ will \thi{slide down} the circle -- from its presence up there, close
to the \thi{origin}, and entering at some moment the \hoa, when also
the actual point $B$ on the line gets close enough.  (Eventually, if the
circle stops at the point $B$, the two would coincide.)  This could be
taken as a picture of the process of \co{actualisation} which
\thi{pulls} the vague, unclear image $B'$ out of the compressed density
close to the \co{origin} and isolates it in sharper and sharper form
as it becomes \co{actual}.\ftnt{\label{ftnt:density}Of course, technically, the
  image $B'$ 
is equally \wo{clear}, no matter how close to the pole it is.  By
\wo{vagueness} here we should understand the density of the images
which are closer to the pole (\citef{what is closer to the \co{one}, is lesser
  with respect to quantity}{Proclus}{ \para 179}), as opposed to their \thi{more
  adequate}  
representations, the closer they are to the point where the circle
touches the line.} 

The \co{actual experience} is thus a juxtaposition
of the \co{actual} contents of the line (close to, or under the
circle, like $A$ in 3) {\em and} the \co{traces} of these contents as they
enter the sphere of actuality on the circle \thi{from above} ($A'$ 
in 3). This goes equally well with the \ger{Gestalt}-like psychology 
of perception, with the \thi{filling in} of the unperceived aspects by 
the \thi{mind}, as well as with the deeper phenomena of \co{vague} anticipation
and foreknowledge, things and events which are, consciously or 
subconsciously, anticipated and which are as much reflection of the 
approaching events as of the psychic and intellectual structure of the 
person who happens to be receptive to this kind of experiences. The \co{traces}
are what connects the \co{actuality} with its non-\co{actual}, and ultimately
\co{non-actual} and  \co{invisible} roots. 

\pa\label{pa:analogy4}
Let us push this analogy just one step further. During a finite 
\thi{life time}, the circle will traverse only a finite portion of the 
line. Traversal corresponds to gathering the \co{actual experiences}. 
Thus we mark two extreme points on the line $L$ and $R$ -- the 
limits of the \co{actual experiences} the circle ever may have. 
%Figure~\ref{fig:limits}
The drawing 4) 
below illustrates the situation when the circle is on the $L$ 
extreme -- the image $L'$ is on the edge of the \hoa.  (The dotted line of 
\co{actuality} is adequately lowered indicating the \thi{growth} of the 
circle. $L'$ coincides now with the point of intersection of the circle 
and this line.) The points lying 
on the circle above it, that is between $L'$ and the pole, will never 
enter the \hoa\ (because the circle can move only to the right).

%\begin{figure}[hbt]\refstepcounter{FIG}
\[
\xymatrix@R=0.10cm{
4) &&&&  &&&& && &&&\\
& \ar@{--}[rr] &&& & & & & & & && \\ 
& &&&&  && && &&& \\ 
&  \ci{40} && &  && &&& &&& \\
  & && && & & && &&&  \\
&  && & \ar@{.}[lll]-<1mm,0mm> &&& && &&&\\
\ar@{-}[rrrrrrrrrrrr]  
  &|\ar@{}[uuu]^{L'} 
 \ar@{}[uuuur]+<2mm,7mm>^>{\displaystyle\bullet}
&& && & && & && | \ar@{}[uuuulllllllll]+<0mm,5.5mm>_>>>>>>>>>>>>{R'} && \\
& {\scriptstyle{L}} & &&  &&& &&&& \scriptstyle{R} &&
}
\]
\label{fig:limits}%\caption{Limits of \co{actual experiences}}
%\end{figure}
%
The point $R'$ is the current image of the other extreme $R$. The points 
on the circle lying (clockwise) between $L'$ and $R'$ are those which 
never will be images of something within the \hoa\ on the line -- 
their pre-images lie either to the left of $L$ or to the right of $R$.
Now, as the 
circle moves towards this other extreme point $R$, $R'$ will \thi{slide down} 
reaching, eventually the edge of the \hoa\ (symmetric to the current 
$L'$), while $L'$ will \thi{slide upwards} reaching the point opposite to 
the current $R'$. These two images, the 
current $R'$ and the analogous position for $L'$ obtained when the circle 
moves to the $R$ extreme, induce the sphere which is marked with the dashed 
line. This sphere represents (relatively to the circle, not to its 
position on the line nor to the 
line itself) the part of the circle 
which never corresponds to any \co{actual experience}. It is the most 
condensed collection of the images originating beyond the limits of ever
experienced \co{actualities} between $L$ and $R$. 

\pa %%%WRONG placement of the paragrah number
The \co{objectivistic illusion} ignores, if nothing else, at
least this part.  It bases its understanding exclusively on the fact
that anything between $L$ and $R$ can be given in \co{an actual
experience}.  This is then extrapolated beyond these limits.  Now,
there need not be anything wrong with such an extrapolation.  If this
(or some other) circle moved beyond $L$ or $R$, it would encounter new
\co{actualities}.  But the inexcusable mistake lies in ignoring the
ever present sphere of the essential \co{non-actuality} (above the
dashed line), if not also of the non-\co{actuality} which is closer to
the \co{actual experience}, but still not an \co{actual} part of it
(on the drawing, the points of the circle between the dotted line of
\co{actuality} and the dashed line going through $R'$).

As we travel by car, or even better by carriage, the speed with which various
objects pass by is inversely proportional to our distance from them: the close
ones pass by very quickly, those which are not so close much slower, and those
which are so far away as to be almost indiscernible seem also to become
practically motionless. New impressions and things emerge from beyond the
horizon but what remains constant and unchangeable through the whole journey is,
if nothing else, the simple fact of this inverse proportionality {\em
  as well as} the very presence of the horizon. To claim that beyond the horizon
there are things passing by as quickly as those closest to us is right only if
one has already placed oneself there. Such a placement, however, is a
displacement -- it falsifies the very character of the experience which is always
accompanied by the immovable horizon.

%Performing an operation of the kind of which Cusanus was so fond, w
The illusion pretends that there is no horizon. It attempts to grow the circle
to infinity in which case, in a truly Cusanus-like, unimaginable fashion, the
circle would become the line itself, coinciding with it at every point,
comprising everything within the \hoa. (Do not even ask what would happen with
the pole and \thi{all the rest} of the circle.)  We need not say that such an
operation not only does not help to understand the finitude of the circle -- it
also creates a confused mixture of this finitude and the infinity of the line,
completely obliterating their respective character and, consequently, their
mutual relations.





\noo{
Everything is a formal distinction:
\begin{itemize}
\item  
not the one created/caused/added by mind (relations/distinction of reason), but one
   relative to it 
\item
different meaning, but predicated about the same \co{one}
\end{itemize}
(Scotus: Names of God)

Everything is, eventually, predicated about the \co{indistinct}; not, of course,
in the logical sense which would be a simple contradiction, but in the sense
that everything which is, is \co{distinguished} from and in the \co{one}, is an
\co{aspect} of its indifferent \co{unity}. 

Plotinus: intellect
}

\subsub{What makes One differentiate?}  \citet{[I]f we follow the
  theologians who generate the world from night, or the natural philosophers who
  say that \thi{all things were together}, the [same] impossible result ensues.
  For how will there be movement, if there is no {\em actually existing cause}?
  Wood will surely not move itself -- the carpenter's art must act on it; nor
  will the menstrual blood nor the earth set themselves in motion, but the seeds
  must act on the earth and the semen on the menstrual blood.}{AristMeta}{XII:6
  [my emph.]}  The \thi{analogical modeling} -- a typical example of \co{objectivistic
  illusion} -- is transparent here.  If one posits \co{one} as some
\thi{one} different from \thi{another}, as a \thi{being} among \thi{beings} only
raised (in some strange way) above the differentiation -- the question, implying
an irresolvable antinomy, makes no sense.  But \co{one} is nothing of this sort;
the figure in the previous section should also illustrate the inadequacy of the
question. Nothing makes \co{one} differentiate, because \co{one} is the for ever
undifferentiated, the \co{indistinct}.  The answer to the question \citet{Why
  has the Primal not remained self-gathered so that there be none of this
  profusion of the manifold which we observe in existence and yet are compelled
  to trace to that absolute unity?}{Plotinus}{V:1.6} is thus simply: it {\em has
  remained} so self-gathered.  Yet, it is the \co{origin} of all the
\co{distinctions}, so one may still wonder.\ftnt{Plotinus' answer concerning how
  the {One} engenders the first hypostasis, the Intellectual-Principle (or
  intellect), culminates in the passage: \citef{Simply by the fact that in its
    self-quest it has vision: this very seeing is the
    Intellectual-Principle.}{Plotinus}{V:1.7} Unfortunately, scholars can not figure
  out and agree whether \wo{its} refers to the {One} or to the
  Intellectual-Principle which would make rather quite a bit of a difference. (A
  good account of interpretations and positions is given in
  \kilde{p.252}\citeauthor*{Rereading}.) We will continue with only loose
  Plotinian associations.}
%
Let us therefore reiterate a few points related to this question, which will
also clarify our debts to neo-Platonism and evolutionism.

\subsubi{Virtual co-presence}\label{pa:OneOnlyVirt}
\pa
Our starting point is not \co{one} \thi{in itself} -- it is \co{birth}, the
primordial \co{confrontation} of \co{existence} and \co{one}.
The precedence of levels concerns the relation of \co{founding}: indeed, an
ontological relation, but not in the sense of the \co{founding} element having
existence before and independent from the \co{founded} ones, but only in the
sense of not being relative to the \co{distinctions} of and between these lower, 
\co{founded} elements.

In so far as we can legitimately speak about the \co{one}, it is not the
\co{one} \thi{in itself} (nor \thi{for itself}), but only its \co{presence},
that is, its \co{transcendence confronting existence}.\ftnt{Only in this
  sense the double meaning of the Greek \gre{arche} applies to \co{one}: it is
  the \thi{origin} from which everything emerged {\em and} the \thi{principle}
  governing all, not in any specific sense but merely as the constant and always
  the same presence surrounding every actuality.} It is a pure \co{virtuality},
a background behind the \co{chaos} -- it has no \co{presence} except through
differentiation, staying always \co{above} it.  Although it is the first, it is
inseparable from the second; although it is \co{one}, it emerges only through the
\co{chaos} of many. The unchangeable, eternal Platonic Being is not the opposite of
temporality, becoming and impermanence; the two do not constitute disjoint and
completely dissociated ontological spheres -- they are only the two extremes
of the continuous line stretching from the \co{origin} to every, most minute
\co{immediacy}, the extremes between which \co{existence} unfolds.
We should never \co{dissociate} \co{virtual} elements (whether \co{aspects} of
one \nexus\ or levels of one \co{trace}) and consider
them \thi{in themselves}, as separate entities.  They have meaning only in
connection with each other, only when seen in the unity of the process in which
they are involved.
\citet{Here conspires with There and There with Here [...] And since the higher
  exists, there must be the lower as well. 
The Universe is a thing of variety, and how could there be an inferior
without a superior or a superior without an inferior? We cannot
complain about the lower in the higher; rather, we must be grateful to
the higher for giving something of itself to the lower.}{Plotinus}{III:3.6-7}

\noo{
  If we keep in mind this character of \co{virtuality}
then saying \wo{the {\em totality} of differences} or \wo{the differentiation of
  the \co{one}} is really only a matter of speaking.  Plotinus denoted the One
indiscriminately as Formless, Unmeasured or Infinite.  The possible differences
of accent and emphasis between these two phrases, especially between the static
character of the first and dynamic of the second, become relevant only if we
violate \co{virtuality} of the \nexus\ -- \co{dissociate} its elements as if
they could be given, not only thought, independently from each other.
}

Forgetting this \co{virtuality}, one quickly gets involved into antinomies of the
kind: if the \co{one} is really \thi{one}, then there is nothing which can
affect its differentiation; but if there is something which can do that, than it
cannot be the \co{one} itself, and so the \co{one} is not \thi{one}.  This is
precisely the form of antinomy we have seen in \ref{antinomies}. The question
assumes the differentiated world of things, concepts, principles, reasons and,
at the same time, \co{posits} the \co{one}, or rather \thi{one}.  And then it
tries to apply the \co{reflective} categories of the differentiation back to
their \co{origin}, as if \co{one} was \thi{one}, a \co{dissociated object}
which, analytically, means one among many.  \citet{You cannot take reality to
  pieces and then see how once more it can be combined to make
  reality.}{BradleyTruth}{p.38}
%But \co{one} is just and only the \co{origin} of all
%\co{distinctions}, the source from which all \co{distinctions} draw
%their reality.


 But, of course, the \co{one} is not \thi{one}.  \co{Birth} is creation, not
only of the \co{born existence} but also of the world, or of the \nexus\ of the
world. In these antinomous terms, the \co{one} might be taken as the state of
{\em this} being and its world before it was \co{born}. But we do not want to
multiply unnecessary antinomies. \co{One} is the \co{transcendent} pole of the
event of \co{birth}. It is the \co{existence} which differentiates.  And the
more it does so, the more relative the \co{distinctions} are to this
\co{existence} -- differentiation is the expression of the sensuous mechanism,
nervous system, needs, abilities, life style, \co{reflections}, etc.  of this
being.  The world created with the \co{birth} of an ant is different from the
world created with the \co{birth} of a human being.  Even worlds created with
\co{births} of different human beings are different.

The \co{birth}, the \co{separation} from the \co{one}, is the very
individuality, \la{haecceitas} of the \co{existence}.  The rest is the
differentiation of life -- if you want, development of the
embryo.

\pa
This probably does not sound like a satisfactory metaphysical principle. But we
are not trying to construct abstract metaphysics. We are interested in
philosophical anthropology for which the starting point can only be the \co{concreteness}
and uniqueness of human \co{existence}.
%Metaphysics only in so far as it emerges from and reflects . 
The ultimate \nexus\ of \co{birth} is reflected in the archetype of a seed or
egg, like that which, according to Aristophanes,\kilde{Birds,683ff.} {Nyx}
(Night) laid in {Erebus} (the Darkness of the 
Underworld) and from which, in due time, {Eros} (according to some versions,
the very first of gods) was born; or else like that
which, according to Basilides' gnosis, was deposited by God 
before generating a series of beings and eventually the visible universe, which
as \citeti{a germ, pregnant with hot and cold, separated itself off from the
  eternal, whereupon out of this germ a sphere of fire grew...}{Anaximander}{ DK
  12A10} 

If one definitely wanted to insist on a metaphysical \thi{principle} of
differentiation, then let us use a more modern image: Bergson's abstraction of
\ger{elan vital} -- 
the driving force of creative differentiation.
%I do not want to go that
%far because I am interested exclusively in the life of an individual. 
We could say that \co{one} is the name (as good as \co{nothingness} or
\co{Being}) given to the {initial virtuality} of life, and \citet{Whatever the
  nature of matter, it may be said that life will at once establish in it a
  primary discontinuity, expressing the duality of need and of that which must
  satisfy it.}{MattMem}{p.198} Life will at once establish \co{distinctions},
life {\em is} the force of distinguishing.  The first primitive would then be,
instead of the \co{one}, the \thi{force of differentiation}.\ftnt{This will be
  different from Heraclitean \gre{pyr aeizoon} if the latter is taken as a
  principle of mere strife between \co{distinctions}.  However, following
  Philo, one can easily attempt an interpretation according to which the
  \gre{arche} of fire is the very force of differentiation, the one which {\em
    is} many: \citefi{This world, which is the same for all, no one of gods or
    men has made. But it always was, is, and will be: an ever-living Fire
    (\gre{pyr aeizoon}), with measures of it kindling, and measures going out.
    -- Men do not know how what is at variance agrees with
    itself...}{Heraclitus}{ DK 22B30-B51} Then, the war of opposites is only
  \citefi{father of all, king of all}{DK 22B53}{}, i.e., precedes the world and is
  \wo{the same for all}. Following the lines of such an interpretation, would
  probably eliminate most significant differences.}
%
But this force has to start somewhere, so take also the \thi{pure virtuality}
(which, to begin with, should be the same as the force itself).  In principle,
we won't object.\ftnt{Bergson intended such an interpretation of Plotinus which,
  more recently, has been expounded, e.g., in \citeauthor*{OneGives}.}  The
\co{one} is not a dead, frozen \thi{one} which needs an external force to breath
life into it, to turn it into \thi{many}; it is not a \thi{potentiality} of
formless matter which needs additional forms and all kinds of causes or motions
to join them and turn into \co{actuality}.  The \co{one}, as \co{indistinct},
is \co{pure virtuality} of possible \co{distinctions}. In particular, although
it can be \co{posited} by \co{reflection} as a kind of \thi{object}, it is
\co{present} {\em only} through differentiation which is an inseparable part of
\co{existential confrontation} with \co{one}.  This differentiation establishes,
however, another level of Being, leaving \co{one} untouched and inaccessible.
The \co{one} \citeti{rests by changing.}{Heraclitus}{ DK 22B84} \co{One} taken as
a metaphysical (meta)principle would thus be the force of \thi{bearing}
\co{existences} -- with both connotations of giving birth and support. All the rest
of the process of differentiation is relative to 
such events of \co{confrontation}.
%But this does not mean that it does not make sense to speak about it\ldots
%causes: formal, material, final, etc
%motions (also joining): Atropos, Clotho, Lachesis (Cusa, DDI, II 10)


% \co{One} is the unity of the differentiation.  As such, it is a static principle
% which I find purposeful to idenitfy with the genealogical \co{origin}.  As far
% as an individual being, \co{existence}, is concerned, the only thing that
% matters here is that it is rooted in this unity, or as I say what amounts to the
% same, that it was \co{born} from the \co{one}.

\noo{We will not speak so much about the force itself
as about what it does, what effects it achieves, what stations it
passes through.  The continuity of the process, that is, the presence
of the force, will be there all the time but only in the background.
}


\subsubi{Birth as mystery vs. birth in time}

\pa Now, since \co{birth} is our starting point then, as the first moment of
creation, it must remain a mystery. And so it does. But this is a mystification!
We know perfectly well what birth is and how it finds place in the course of the
world, of the objective world. Why would one try to obscure such a common event?

Speaking (and thinking) \co{objectively}, you, like everybody else, are just an
accident of the world, of the \co{objective} world... Indeed, but this is as
true as it is uninteresting, or let's only say, existentially irrelevant.  You
are an element of the \co{objective} world but only in so far as you view
yourself as an \co{object}. This \thi{object} seems so unclear, indeterminate
and undefinable, that one hardly needs arguments against such a reduction. Our
exposition in this Book amounts also to the claim that it is just that: a
reduction. From such an \co{objectivistic} perspective one can then learn
nothing about oneself -- at best, only something about one's \co{objective}
aspects.  Looking all the time for the ultimate atoms which would explain
everything, it only finds ever new \thi{whats} and suspects, if not knows
perfectly well, the partiality and insufficiency of constructions starting from
them. The explanations never reach any bottom, remain always conditioned by
\thi{something more} which, although still hiding, is just around the corner
and once unveiled, 
will yield the ultimate sufficient reasons and efficient causes. Thus science
with its explanations (and every explanation is a reduction), risen to the
ideology of \thi{objectivism}, remains a perpetual project, striving after some
regulative idea which, if reached would in some inexplicable manner explain
everything but which also everybody knows is impossible. With respect to the
meaningful things of existential importance, \thi{objectivism} either ignores
them or keeps promising future answers, 
without ever having anything to offer.\ftnt{The projects and promises are as old
  as reductionism and all its forms could be quoted as examples -- starting
  perhaps with some philosophers of nature from Miletus, then Democritus' atomism,
  through most forms of empiricism, and then, after Descartes, all forms of
  scientism, with instances like Laplacean dismissal of the unnecessary hypothesis,
  later Skinner's behaviorism, early champions of AI and the very idea of
  Turing test, Wilson's biological reductionism, etc., etc., etc. All instances
  when science becomes scientism can and will serve as examples.}


\pa Let us emphasize: \co{objective} thinking has its irreplaceable and all
important role in the project of control, in the project of \co{reflectively}
arranging the \co{dissociated} pieces into agreeable and useful complexes. We
object only to the \co{objectivistic illusion} which claims \co{absolute}
validity of \co{objectivity}. For all constructions bottom-up must start from
some \thi{given} bottom, from the ever evading \thi{atoms}. What such
\thi{atoms} are and what counts as such \thi{atoms}, is always a result of
\co{experience} and \co{reflective dissociation} of \co{experience}. They have
to be discovered before they can be used and, moreover, everybody has to
discover them anew -- in the course of his \co{experience}.  \co{Objectivity}
itself must be discovered which means, must be encountered in the
\co{experience} of \co{objectivity}.  But no matter \thi{what} appears, at a
given moment, as the ultimate pieces of the \co{objective} world, it turns
sooner or later into relative elements of more general understanding or deeper
assumptions, that is, into something unpleasantly \co{subjective}. For what we
can do with the ultimate \co{that} is either take it for what it is or project
on it relative \thi{whats}. 

We claim truth of some theories and untruth of others, we discover mechanisms of
the world and history which were active long before our birth and will continue
long after our death.  They are all, hopefully, true but to be discovered,
\co{experienced} {\em as true} something more is needed. Namely, the basic idea
of truth, not of any conditions or ways of ascertaining it, but of its very
sense. The basic idea of \thi{being there}, not of something being there, not of
any \thi{what} but simply -- \co{that}. We can populate the world \thi{beyond}
our experience with people and events, only because we have such a \thi{beyond},
that is, only because we know \co{that}; only because \thi{reality},
\thi{objectivity}, call it as you like, is given in advance, as the very first
condition and fact of
\co{existence} and \co{experience}. %(The ontological proof has something to it.)
It does not follow from any \co{experience}, not to mention, from any
\co{experiences}. It is a mere \co{that} which does not arise from any
\thi{whats} which are discovered already in \co{experience} and are relative to
it.  So, in a good, Kantian fashion -- it is \la{a priori}.  Unprovability {\em
  and} certainty constitute together a good sign of \la{a priori} as does, more
generally, irreducibility to the merely \co{actual} categories.\ftnt{Of course,
  we do not share Kant's concept of \co{a priori} which, according to
  phenomenological understanding, is not given independently from
  \co{experience} (there is no 
{\em such} thing), but is \co{experienced}, if not given, in any
\co{experience}.\noo{For instance, \citef{the opposition of a priori and a
  posteriori [is concerned with] two kinds of experience: a pure and immediate
  one {\em and} one mediated and conditioned by positing of some natural
  organisation of the real subject of the acts.}{MaxForm}{p.71}}}

In short, the projects of elimination of \co{existence}, and of our
\co{existence} in particular, from the ultimate (absolute and not relative)
explanation of the totality of the world can arrive only at the \co{indistinct
  that}, that is, at no explanation at all. Every explanation is relative, and
hence also partial, which, however, does not deprive it of value and
objectivity. But absolutised \thi{objectivism} ends up as a projection of a
solipsistic \thi{subjectivity}, whether the latter takes the form of some
transcendental \thi{constitutions} or of atomic \thi{sensations} and
\thi{ideas}.


\pa This, however, is still only a mystification. \thi{Objectivity} is not
\co{experience} of \co{objectivity} and \thi{objective} time is not
\co{experience} of \co{objective} time. 
% One can -- and should -- attempt seeing and explaining things not only the way
% they appear for one, but also, as they appear for others, as they are
% independently from one being here or not. This is, as a matter of fact, exactly
% what we are trying to do.  For the universal \co{aspect} of any appearing is
% that it appears for somebody.  We are only objecting to the unwarranted
% absolutisation, to the process which is in every case the same: first one
% discovers objective world and time -- discovers them in one's own experience --
% and then tries to reduce this very experience to what has been thus discovered.
%
There is always the problem: the genesis of (the experience of) time involves
and presupposes the temporal genesis, the genesis {\em in} time. If the
discovered world and time are \thi{objective}, this means exactly that they are
present independently of this very discovery. This crux of the matter is
impossible to ignore, and we are far from doing that. We have explained in
\ref{sub:ObjConst} that our experience of time, like all our experience, amounts
to a discovery, not to a constitution, let alone \thi{subjective} experience and
projection.  \citet{Man learns the concept of the past by
  remembering.}{WittPI}{II:xiii} Saying no more is to remain in the grips of
empiricism.\noo{perhaps, now linguistic and no only logical} For, if there were
no past to learn, learning its concept would amount to an empty game. The
concept of the past as well as the objective past do arise through \co{actual
  experiences} and \co{reflection}.  But they do no more than, first,
\co{reflect} the \co{experience} and, eventually, \co{confrontation} with the
ultimate \thi{objectivity} and \co{transcendence} of the \co{one}.  This
accounts for the natural interweaving of one's \co{experience} of time with the
\co{objective} time, of one's \co{experience} of the world with the
\co{objective} world. The former does not \co{represent} the latter in some
\thi{internal duplication}, it only \co{distinguishes} it and thus enters it,
weaves itself into it.

Every \co{birth} is also an \co{objective} event, the mystery of the beginning
finds place in the time we understand as \co{objective}. But the fact
that the ontological event coincides with the \co{actual} one does not mean that
we should confuse its ontological character with its \co{objective} form.  The
fact that the order of ontological \co{founding} happens to coincide (in so far
as we are concerned) with the order of temporal succession\noo{, of the individual's
evolution,} does not mean that they are one and the same. The former has an
\co{absolute} beginning, while search for the beginning of the latter (with the
associated antinomies) continues since ... the beginning of time.

We certainly do not want to oppose the attempts to understand \co{objective}
processes of \co{actual} things and relations between them. We only claim that
such attempts will never even touch the fringe of \co{concreteness} which
surrounds every \co{actual experience} and every abstract conception. The value
of such attempts is as praiseworthy as their absolutisation is regrettable --
they are, after all, complete opposites of the \co{absolute}. An
\co{existence} arises from the ultimate objectivity of \co{nothingness} and it
arises in the process of differentiation. This process can be viewed from
\thi{inside} (as we are trying to do) or from \thi{outside}, in terms of
\thi{objective givens}. The latter view always has to
re-construct the primordial \co{unity} from its \thi{givens} which, however, themselves
are relative to the prior differentiation. Consequently, \thi{how} this process
proceeds in the \co{objective} time and \thi{what} are its \co{objective}
elements are questions involved already into relativity to the inquiring
\co{existence}, to the context of inquiry, to the level of objective knowledge,
to the historical and cultural situation.
If one finds in such a relativity reasons for  universal
scepticism or historical relativism, it is only because \thi{objectivism} itself
is only matter of faith. And we would say, of bad faith, because its lack of
ultimate reason and justification rests on the constant absolutisation of the
current results of relative \co{distinctions}, on claiming the status of
\co{absolute that} to the relative \thi{whats}, in short, on the
\co{objectivistic illusion}.  

\noo{No matter \thi{how} and with \thi{what} I populate the world beyond \co{my
  experience}, even beyond my \co{existence}, it will always remain surrounded
by the \co{indistinct nothingness}, the ultimate, \co{absolute}
\co{transcendence}.  In the same way, the \co{objective} world remains all the
time surrounded by unimaginable void, antinomies of beginning and infinity.  But
these are {\em only} \co{objective} terms, terms which at best are \co{signs},
imperfect \co{actual} images.  \co{Objective} world is but a \co{sign} of the
world \co{above} from which it emerges in every \co{existence}; the \co{posited}
totality of \co{objects} is but an image of the primordial and ultimate
\co{that}; \co{objective} time is but an image of the confrontation of
\co{actuality} with its \co{founding presence}, its beginning.  And every
beginning is a mystery, or else it is not a true beginning.
}

\subsubi{Creation, emanation, evolution}
We obviously owe quite a lot to neo-Platonism. At the same time, we seem to mix
it with other elements  (creationism, evolutionism) which are typically
considered as its contraries. Let us therefore comment these aspects. 


\ad{Creation vs. emanation}\label{se:createEmanate}
%\pa
\co{Birth} establishes \co{existence} as the \co{confrontation} with \co{one},
the (force of) 
differentiation of the \co{indistinct}. We called it \wo{\la{creatio
    ex nihilo}} but we also used the language of neo-Platonism, identifying this
creation with the gradual hypostases emanating from the \co{one}. This 
might easily appear as an unjustifiable conflation of the two ideas which exclude,
even contradict, each other. For either the world is created from nothing,
created by a free act of God's will, or else it emanates by necessity, and hence
in all eternity, from some archetypal principle, from the unity of the
First.\ftnt{Interpretation of Genesis as \la{creatio ex nihilo} was suggested to
  Christians by Galen, who criticised their capricious God creating in a
  completely arbitrary manner. For Moses \citef{it seems enough to say that God
    simply willed the arrangement of matter and it was presently arranged in due
    order; for he believes everything to be possible with God [...] We, however,
    do not hold this[...]}{Galen}{XI:14} \noo{[The Christians as Romans saw
    them, p.87]} Gnostic theologian Basilides seems to have been the first
  Christian thinker to articulate the rudiments of the doctrine (second quarter
  of the 2-nd century), followed shortly by more definite formulations of
  Theophilus, the bishop of Antioch \citaft{exNihilo}{}.\noo{[Gerhard May,
    \btit{Sch\"{o}pfung aus dem Nichts. Die Entstehung der Lehre von der Creatio
      ex Nihilo}, Arbeiten zur Kirchengeschichte, 48, pp.63-85, Berlin, 1978
    [Christians as Romas saw them, p.89]]} Christian thinkers of neo-Platonic
  orientation did not make much out of the claimed contradiction. The
  idea of emanations was forcefully opposed by Aquinas and then
  Ockham, both emphasizing the \la{ex nihilo} aspect as well as the fact that
  God creates every individual being directly, and not by any gradual
  individuation of some universal essence. We can easily agree with this
  critique when restricted to individual \co{existences}. With respect to
  particular things, it is a different matter which has been presented so far
  and will be addressed further in Book II.} 

Indeed, it is easy to construct incongruent, even contradictory, images.  On the
one hand, God -- no matter what one says, imagined as an external agent, sitting
there and waiting for the moment when his freedom makes Him say \wo{\la{Fiat!}}.
And, \fre{voil\`{a}}, here comes the world -- but in fact, just a new object (or
totality thereof) created by an agent, just like a house is \thi{created} by the
construction workers.  On the other hand, One -- no matter what one says,
imagined either as a dead object, lying there and waiting for being given life
from outside, or else as a uniform and undifferentiated {\em being} -- yes! an
object again -- which somehow externalises its hypostases, throws them out of
itself; moreover, since it contains the principle of the emanation within
itself, it can not do anything else, it emanates all hypostases with eternal
necessity.

Both pictures are equally childish, yet it is easier to state than to help. It
is these pictures which make one 
consider the two, creation and emanation, contradictory. But \co{one} is
\co{nothing} and \co{nothing} is \co{one}, \ref{se:toBeDist},~\refpf{Oneis}, and so
creation from \co{nothingness} is the same as emergence from the
\co{one}. \co{One} is not any 
agent with will and other faculties similar to ours -- speaking about His free
will is indeed funny antropomorphism. An individual \co{existence} is created --
\co{born} -- directly, as it is determined exclusively by the
\co{confrontation} with the \co{one}. This aspect of creation remains a mystery
as far as its reason and first stages are concerned. \wo{How?}, \wo{why?},
even \wo{what?} are inadequate questions, because when there are only \co{pure
  distinctions}, or perhaps even only the mere fact of \co{confrontation}, there
are no grounds for answering them. Every attempted answer will be only
 a reduction to some \thi{whats} assumed more primordial. We start with
\co{birth}, the \co{absolute} beginning 
and do not attempt to give an account of its reasons -- whether one calls them
\wo{God's free act} or \wo{One's generous goodness} makes no difference: both
are equally inadequate.  Finally, the \co{origin} does not remain \thi{outside} the
emerging world but in its midst, the \co{presence} of \co{one}, as the
\co{indistinctness} surrounding eventually every situation, penetrates the
whole creation. 
\wo{Emanation} emphasizes the \co{aspect} of \co{immanence}, while \wo{creation}
that of \co{transcendence} of the \co{origin} in relation to the differentiated
and, eventually, \co{visible world}.


\ad{Neo-Platonism vs. evolutionism}\label{se:neoplatonism} Evolutionism is
easily considered the successor of neo-Platonism, almost as if it were its
scientific improvement which deservedly replaced the hapless and unenlightened
ancestor. In fact, while the differences may appear contrary, there are also
apparent similarities which justify the sense of continuity between the two.
Our middle ground differs, being a middle ground, slightly from both. Let us
therefore indicate briefly the respective relations.

\noo{
\item not chorismos (reflection of Plotinus): the lower 
  \begin{enumerate}
    \item --: is not an `image' of the higher but a \co{sign}, perhaps, a
      \co{trace}; i.e.: 
    \item --: not as likeness [Ennead V:9.5, III:7.1], but as distinction and
      aspect \\
      this forces neo-Platonists (and gnostics, in spite of opposing them) to
      multiply the intermediary stages \la{ad infinitum}; \\
      but we have our continuity, so that indeed stages can always be added...
    \item --: not as inferior and degenerate, because not similar -- just a new level
    \item +: is dependent on the \nexus\ which it reflects, as its \co{trace}, as
      the \nexus\ persists \co{above}... [Ennead V:1.6]
    \item --: is not determined by higher, nor follows from it by necessity\\
        a particular \nexus\ \co{founds}, but does not generate all its lower
       \co{aspects};
  \end{enumerate}
}

There are many passages in neo-Platonic texts suggesting that emanations are
not mere reflections but genuine differentiations, that the emerging entities
are truly different from their origin.\ftnt{Plotinus uses the analogy of a seed:
%   With Plotinus, Intellect and Intellectual Principle remind often about \nexus\ 
%   of only formally distinct \co{aspects}: \citef{All are one there [in the
%     Intellectual Principle] and yet are distinct: similarly the mind holds many
%     branches and items of knowledge simultaneously, yet none of them merged into
%     any other, each acting its own part at call quite independently, every
%     conception coming out from the inner total and working singly.}{Plotinus}{
%     V:9.6}
  everything \citef{must unfold from some concentrated central principle as
    from a seed, and so advance to its term in the varied forms of sense. The
    prior in its being will remain unalterably in the native
    seat;}{Plotinus}{IV:8.6} and the element of differentiation is, of course,
    always there: 
  \citef{Every thing which participates of The One, is both one and not
    one.}{Proclus}{\para 2}} Yet, the accepted interpretation, as well as many
other fragments, suggest that \citet{emanations proceed through
  similarity.}{Proclus}{\para 166 [Likewise, in \para 18: \woo{Everything that
    by its existence benefits others, is in itself originally that which it
    bestows upon the recipients.}{Proclus \para 18}]} This aspect may be
particularly prominent in Proclus' dry conceptualism, but the principle of
\gre{chorismos} (reflection as likeness of the original) is quite central in
Plotinus.\ftnt{E.g., discussing time and eternity, Plotinus says: \citef{We
    begin with Eternity, since when the standing Exemplar is known, its
    representation in image -- which Time is understood to be -- will be clearly
    apprehended [...]}{Plotinus}{III:7.1} Generally, \citefib{in things of
    sense the Idea is but an image of the authentic, and every Idea thus
    derivative and exiled traces back to that original and is no more than an
    image of it.}{Plotinus}{V:9.5} There is certainly space for discussion
  whether apprehension of the image through the knowledge of the exemplar
  implies for Plotinus similarity in any trivial sense, but we leave this to the
  scholars. In \citeauthor*{Fielder}, likeness is listed as one of the four aspects of
  the Plotinian image (the other three being distinctness from, inferiority to and
    dependency on the original).}
  
  Our \thi{emanations} proceed by \co{distinction} and even \co{dissociation}.
  Even if \citet{[e]verything that is in another emerges {\em exclusively} from
    that other,}{Proclus}{\para 41 [my emph.]} then it is
  also the case that the emerging entity, the emerging \co{aspects} introduce
  entirely new elements not present originally.  The process of differentiation
  is a gradual \co{actualisation} of \co{virtual nexuses} -- its result is
  \co{founded} in the \nexus\ but not determined uniquely by it.  The result
  need not be -- in fact, never is -- in any way similar to its \thi{cause}, to
  the \nexus\ from which it emerged. The former is not an image of the latter
  but its \co{sign}, it does not have to resemble the latter but only point to
  it. The dependence on the origin is neither conceptual (similarity) nor causal
  but evolutionary, where everything \wo{unfolds from some concentrated central
    principle as from a seed.}
\noo{(as the dependence of a fruit on the seed from which it emerged, of Prague
  today on Prague of the XIII-th century, of the person I am when writing this
  on the particular zygote some time ago).}

Although the relation of similarity is not transitive, using it as the
principle of emanation poses the question about {\em how far} the succeeding
stages should resemble each other, exactly {\em how} similar the image should be
to its prior and, in particular, to its immediate predecessor. This principle
carries the primary responsibility for the 
neo-Platonic multiplications of the intermediary stages almost \la{ad
  infinitum}. Plotinus' disciple Amelius distinguishes three additional
hypostases of the Intellect; Iamblichus adds yet another One above the One of
Plotinus, some additional intellectual principles or demiurges, supra-terrestial
and other souls; Proclus, arranging the hypostases into triads, brings in some
order but the number of levels of beings hardly diminishes;
%Plotinus' critique of gnosticism notwithstanding, which concerned Christians!
the gnostic cosmogonies and ontologies inherit this disease and, even more than
the late neo-Platonic hierarchies, slip out of control enmeshing the student
into the intricacies of ever longer and longer series of spiritual beings which
follow each other according to the similarities and oppositions as fantastic as
they are unbearable.\noo{ Valentinus begins with \gre{Bythos} -- Abyss or Depth
  -- beyond all attributes, from which emanate \gre{Ennoia}, or \gre{Sige} --
  \thi{Thought}, or \thi{Silence} [yes! \wo{or}] -- and eventually, through
  innumerable stages: \he{???}, all \co{visible} beings. [in Creation of
  Consciousness, p.56,70] The texts from Nag-Hammadi But I have problems with
  recognising the intricacies of the complicated ontologies of spiritual beings
  which follow in such gnostic expositions, so I won't dwell on them.  } Our
levels, the steps in the process of \co{actualisation}, could also be multiplied
\la{ad infinitum}. The reason is simply the continuity of the process as opposed
to the discrete structure of its conceptual representation; or put differently:
the \co{unity} of \co{existence} as opposed to its \co{reflective}
account.\ftnt{Hopefully, the way we have structured these stages does not appear
  completely arbitrary. It will be further justified in the following Book.}
One might be tempted to excuse the principle of similarity for the
multiplication of the hypostases as being if not a perfect image, so at least a
vague reflection of this very continuity.

The kind of development, however, is different. The lower levels in our process
of differentiation are not, as in neo-Platonism, inferior to the higher ones. As
they are not mutually similar, they are not to be compared either.  The lower
levels constitute truly new dimensions of Being, just like new evolutionary
stages go beyond the previous ones. The \wo{higher} and \wo{lower} do not refer
to any valuations but simply to the precedence in the order of \co{founding}.
The \co{founding} itself is also different from -- and much weaker than -- the
relation of being generated, of reflecting, of necessarily emanating, or almost
whatever interpretation one might assign to the neo-Platonic relation between
the lower and the higher.  The lower levels are, indeed, dependent on the higher
ones but only in the sense of the latter being their necessary conditions.
A \co{dissociated} (or only \co{distinguished}) \co{aspect} depends on
the \nexus\ from which it is thus \co{distinguished} (just like a \co{sign} or a
\co{trace} depends on that {\em of which} it is a \co{sign}/\co{trace}) but the
\nexus\ itself is not its efficient or other cause.  There is no causal or
direct generative relation, the lower is not determined by the higher, nor
follows from it by any necessity.

%Being \co{founded} is different from (and much weaker than) being generated.

\noo{
  \item we, too, maintain direct relation of soul to the One, perhaps, even more
  direct than Plotinus, [Ennead V:1.3; but then also V:1.10\para 3-4\normalsize]
  
  ; \co{dissociation} and \co{actualisation}, on the other hand, are a bit more
  similar to neo-Platonic \thi{individuation} by descent -- not, however, into
  matter, though yeas, into space and time, but primarily, into \co{actuality}
}

\pa Another important difference concerns the understanding of
  individuality. The soul, according to Plotinus, is divine, by its power
  \wo{the manifold 
  and diverse heavenly system is a unit; through soul this universe is a God}.
Yet soul \citet{for all the worth we have shown to belong to it, is yet a
  secondary, an image of the Intellectual-Principle.}{Plotinus}{V:1.2-3} And
the soul treated here is not yet the individual soul of the individual human but
the comprehensive soul-principle from which individual souls emanate in further
stages.  With us individuation happens not by descent and joining the matter,
but is the first event of \co{birth}: \la{haecceitas} is not at the lowest
level of Being but in its center, preceding all \co{distinctions} of substances,
attributes, causes and effects. A unique individuality of \co{existence} is the
beginning and not only the final result of a gradual differentiation and
eventual enmatterment.

\pa Neo-Platonism has been taken as the abstract metaphysics of the objective
world, supposedly {\em explaining} the emergence of souls, people, particular
things in the process of objective generation.  Although such an interpretation
might perhaps be defended, we do not find it plausible to dwell on such literal
images.\ftnt{Such an objectivistic interpretation can be seen in a close
  association with magic and spiritualism which infected Plotinus' mysticism
  almost immediately. While Plotinus lived, \citef{he lifted his pupils with him.
    But with his death the fog began to close in again, and later neo-Platonism
    is in many respects a retrogression to the spineless syncretism from which
    he had tried to escape.}{GreekIr}{Appendix II:2\kilde{p.286}} Porphyry, Proclus,
  Iamblichus not only commented extensively on the theurgic ground work,
  Julianus' \btit{Chaldean Oracles}, but mixed religious devotion both with
  magic statuettes and oracular images (their power resulted from the natural
  sympathy linking image with original) but also with conjuring spirits and gods
  in mediumistic seances which would be hard to distinguish from the practices
  of modern spiritualists.  This tradition becomes reinforced in the
  neo-Platonism of the Renaissance which, joining it with the newly imported
  Cabala, tried to apply the system to magical purposes. Such applications
  seemed possible because the natural world was seen as literally dependent on
  (generated from and subordinated to) the celestial and then supra-celestial
  one. The whole hierarchy turned thus -- in the hands of Marsilio Ficino, Pico
  della Mirandolla, Cornelius Agrippa, John Dee and many others -- into a system
  of objectified, because usable, entities, whether angels which could be
  conjured, or letters and symbols which could be manipulated according to the
  numerological formulae. (\citeauthor*{YatesRen} provides a good, general
  overview.)}  In our case, the separation of concerns should be completely
unambiguous -- we are doing philosophical anthropology, not any abstract
metaphysics, not any objective theory of everything. In this respect we believe
to be in full agreement with neo-Platonism and only want to distance ourselves
from its objectivistic (mis)interpretations. Treated as such a \thi{theory of
  the objective world}, neo-Platonism has been replaced by evolutionism.
Although such a replacement witnesses to a misunderstanding, it offers also
conceptual tools of reinterpretation which, so it seems, we have been utilising.
Let us, therefore, comment now briefly on this aspect.

\noo{
\begin{enumerate}\MyLPar 
\item Neo-Platonism has been taken as the abstract metaphysics of the world,
  supposedly explaining the emergence of souls, people, particular
  things. Although such an interpretation might perhaps be defended, we do
  not think that any neo-Platonism has ever been meant it as such. In our case,
  the separation of concerns should be completely unambiguous -- we are doing
  philosophical anthropology, not any abstract metaphysics, not any universal
  theory of everything... Yet, just like evolutionism replaced neo-Platonism
  wrt. general development, so our version has strong evolutionary elements,
  which we will comment below...
\item (Book III): evil arises in paradise! it may invisibly arise from
  here-and-now, but it affects the higher, invisible regions, except the direct
  ontological dependence on One... 
% \item our \thi{mysticism} (if any) does not refer to any \co{actual experiences}
%   of union, ecstasis, etc. -- it is consumated in \co{non-actuality}, whole
%   life... [sensible-to-intellectual: awakening IV:4.5,8.1; or violent uplift
%   V:3.4; then also -to-One:...]; though scholars dispute (e.g., \wo{By dint of
%     the exercise of God's presence, divine union becomes
%     continuous. Contemplation of the world of Forms and the experience of the
%     love of the Good are no longer rare and extraordinary events. They give way
%     to a state of union which is in a sense substantial, as it sizes our being
%     in its entirety.}{Hadot, Plotinus on the Simplicity of Vision, 71 [after,
%     Bussanich in ACPQ, LXXI, p.358]})
%  
%   \citet{Neither temporality nor the feelings of the soul are central to this
%     state [...] The duration of the state is not significant to Plotinus, nor
%     are the phenomenological characteristics evinced in the experience. These
%     would be distractions to the soul's true task, for they lie, in Plotinus'
%     taxonomy, at the level of sense-perception, an outwards directed aspect of
%     the psyche. [...] It does not become something else, nor does it become
%     absorbed at some moment into the One. Its union with the One is something
%     that always obtained, but it had hithero failed to grasp this fact
%     adequately.}{PlotMystB}{III}
\end{enumerate}
}

Just like  ontogenesis repeats philogenesis so here, every \co{actuality} repeats
the levels of its development, repeats this development in the structure of
layers surrounding it in its very \co{actuality}.  In this way, the dynamic
element of evolutionism present in our development is fully compatible with the
static universe of neo-Platonic inspiration.  \thi{Evolution} (let us use
temporarily this designation for our \co{actualisation}) does not simply leave
the past stages behind, transforming the less advanced forms into the more
advanced ones. In the process of temporal development, the primordial -- and
higher -- remains completely unaffected, it is not replaced by, but remains
around and underneath the lower. Unlike in the evolutionary process, here the
source is preserved and remains untouched, \refp{pa:preserves}.

Just like the lower is not any debasement of the higher, so the \thi{later} is
not any improvement of the \thi{earlier}. The \thi{evolution} is indeed
differentiation but this seems the only ground for comparison of various levels.
More importantly, what completely distinguishes our \thi{evolution} from
evolution is the fact that the lower (\thi{later}) aspects may in fact influence
the higher (\thi{earlier}) ones. The higher does represent an \thi{overflow},
though not in the productive sense as often used by Plotinus, but only in the
sense of inexhaustibility by the lower, of being \co{transcendent} in relation
to and inaccessible by the lower. The inaccessibility concerns however only the
lower restricted and completely enclosed within its own categories. For
instance, using and focusing exclusively on the categories of \co{reflective
  dissociation}, one will never reach any genuine experience. But as one is more
than the \co{reflective actuality}, as one is the \co{unity} of the whole
\co{existence}, so the events of the lower and more \co{actual} order may
influence the higher levels. We leave these remarks for the time being, because
the mechanisms of such an influence will be discussed in more detail in Book II
(\ref{sub:Identity} and \ref{sub:lowHigh}) and in Book III.

The above two points -- that \thi{earlier} stages get accumulated and remain
\co{present}, and that the \thi{later} ones can influence them -- lead to the
final point of difference.  Unlike evolution -- and similarly to neo-Platonism
-- our process leaves open the possibility of a kind of return. \citeti{Whence
  things have their origin, Thence also their destruction
  happens.}{Anaximander}{ DK 12B1} The eventual return to \co{indistinctness} --
death -- will not concern us much, but much of Book II will be concerned with
the relations between the \co{actual reflection} and the higher levels which, in
some sense, might be called \wo{return to the origin}.

\pa In short, we have adjusted neo-Platonism and evolutionism by restricting
both to the development of a human individual. This is development of a
\co{unity}, of a unique \co{existence}, which does not appeal to any externally
differentiated environment but, to use such a language, is only the development
of this very environment.  There is no essential gap between \co{existence} and
the world because this world emerges only with the \co{existence}, because it is
the world of \co{distinctions} which this \co{existence} is able to
\co{recognise} (not only \thi{understand}, not only \ger{erkennen} in any narrow
sense of this word). The \co{objectivistic} assumption that every particular
\co{act} is an event involving some ready-made entities, \co{dissociated} more
or less \co{precisely} from each other, not only makes any meeting impossible
but also tries to begin with something which is only the end of the fundamental
process.  The evolutionary aspect is present in the creative differentiation of
the \co{indistinct} through the subsequent \nexuss\ and \co{dissociations}. But
it is neither any mechanic development in which a system passes through stages
with different contents but the same form, nor any \co{objectivistic} evolution
which forgets the previous stage as soon as it reaches the next one. The
\thi{past} stages are preserved as the deeper \co{aspects} underlying the
\thi{subsequent} ones, and this conservative aspect of the process distinguishes
it from evolutionism and brings closer to neo-Platonism.


