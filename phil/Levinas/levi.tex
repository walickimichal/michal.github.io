\documentstyle[a4wide]{article}

\newcommand{\qpa}[1]{[???\hspace*{1em} {\em #1}\hspace*{1em} ???]}
\newcommand{\lev}[1]{{\sf{#1}}}
%\newcommand{\prg}[1]{\paragraph{[#1]}}
\newcommand{\prg}[1]{\vspace*{1.5ex}\par\noindent{\lev{#1.}}\par }

\newcommand{\quo}[1]{``{\em #1\/}''}
\newcommand{\comment}[1]{\\[1ex] {\small{$\lceil$#1$\rfloor$}}\\[1ex]}

\newcommand{\MyLPar}{\parsep -.2ex plus.2ex minus.2ex\itemsep\parsep
   \vspace{-\topsep}\vspace{.5ex}}

\begin{document}

\title{Levinas for the Beginners \\ 
  {\normalsize{-- an Incompetent Introduction}} \vspace*{-1ex}\\ 
  {\normalsize{with a Few Comments and Critical Remarks}} }
\author{ %\small
{\em Micha{\l}\ Walicki} \\ {\small{michal@ii.uib.no}}}

\date{\hfill\small\today}  % August 8, 1996
\maketitle 
\vspace*{-5ex}\section{What and Why}
The language of modern philosophy seems to suffer from two sicknesses at once which, if combined
together, should neutralize each other. On the one hand, ther is a camp of people struck by the
disability (which they call unwillingness) to 
relate to any question which cannot be defined and then discussed with approximately 
mathematical precision. As a regretable consequence, they have a lot of interesting,
but no significant things to say. The latter are reserved for their private circles.
%
As if trying to bend the stick to the other extreme in order to straighten it, the
other camp
resists any call for even a minimal amount of clarity. Its inhabitants want to talk about significant 
issues and
recognize their depth. But so comes the fallacious, albeit in practice very powerful
syllogism: the deeper, the darker, and so the darker, the better.

As usual, the only things crossing the border, except for a few homeless, are accusations and
scornful remarks. ``You are talking rubbish! No one can uderstand this.'' -- ``Well, at least we do not neglect the
most fundamental aspects of human life.'' The homeless, as homeless, spend a few days here
and a few there, and do not seem to look for a permanent camping place. And so, the only 
remaining things are accusations.

Emmanuel Levinas, whose few basic concepts I want to review, belongs definitely to this 
latter club. A prominent representative of its virtues, he is also one of the worst examples
of its vices. Indeed, it is hard to understand why anybody should be
willing to obscure one's thoughts by means of one's language.
According to Talleyrand, \quo{language serves the purpose of
hiding the thoughts of a politician}.
A philosopher may  use
it for the same purpose in order to hide shallowness and limitations, or just to hide
that he has nothing to hide. But it does not seem that Levinas has so much to be ashamed of.
Still, one has a constant impression of being exposed to some
attempts at convincing one that these are really deep, really very deep things we are
talking about here, so the atmosphere of mystery and dark depth is highly appropriate -- 
a prophet should not speak in a plain manner. 
People often tend to speak and write more than they have to say, and being carried away by
a prophetic fascination certainly does not help for that.
One can only regret that a philosopher 
misunderstands himself for a prophet. On the top of this, Levinas seems to have inspired 
similar, and even worse obscurity in many of his followers, and so I see no reason to 
excuse him. 
Having drawn his inspiration from the Jewish tradition and the respective part of the 
Bible, he might have missed the advice from the New Testrament 
\quo{Let your language be: yes-yes, no-no; and whatever is more comes from the Evil}.

These few pages represent a purely linguistic experiment: is it possible to 
clarify a language which confuses and hides and, at best, merely indicates
some deep and interesting thoughts.
If what you read here will make the same impression of misuse of the
language, I will be more willing to excuse Levinas and admit that, perhaps, these thoughts 
are not expressible in a more adequate and understandable way. 
Perhaps, one is doomed to unclear speech when starting to talk about new
matters for the first time? 
New experiences and insights appear only gradually, emerging from the indistinct waters of
unclear intuititions. 
After all, here they concern not the concrete trivialities, but the abstract ones. 
The main intention of Levinas' is to tell a story of 
something fundamentaly different, something lying beyond the clarity of the concepts, embracing,
rather than grasped by the consciousness. Yet, this something is an essentail -- if not the 
most important -- aspect of our life. To deny it is to misunderstand oneself. 
(Listen you, guys from the other camp! But no names are mentioned, of course. 
After all, we are in Academia.)

Each subsequent section, as well as paragraph, is related to 
one special concept or a natural complex thereof. 
Sometimes I was tempted to use different names for Levinas' concepts. In either case,
the original names are introduced in \lev{Sans Serif font}. Places 
\comment{written in separate paragraphs in small style with the funny `begin' and `end' marks}
contain my own remarks and comments -- they may be (and probably are) related to some texts of 
Levinas of which I am not aware. Things in \quo{italics and quotation marks} are direct 
citations from Levinas. The last section~\ref{sec:lazy} contains an abstract 
which can, as well, be read first.

\section{\lev{The Same} and \lev{The Other}}
\vspace*{-2ex}
\prg{The Same}
In our conscious life we are confronted with various more or less well-defined objects:
things, situations, impressions, people. An object entering the horizon of consciousness
becomes, so to speak, accessible and transparent. It is pulled out of the horizon of
darkness and appears in the light of consciousness.

The objects of consciousness are {\em definite} -- 
they display some definite qualities which are relevant to our perception and 
apprehension of the context in which they appear. 
As far as we are conscious of them, they are 
{\em intentional objects} in the Husserlian sense of the word. The principle of 
intentionality implies first of all that the objects are not vague and indefinite. On
the contrary, even if the characterisitics of an object remain slightly dim, an 
intentional act posits an {\em object}, that is, grasps it as if it possessed some
(for the most of philosophy, mystical) site of identity, an \lev{ipseity} (sameness).
Consciousness -- the subject-object relation -- is characterized by this
definiteness, or sameness of its object. 

In so far as objects are accesed through intentional acts, their form and appearance, 
although originating in something external 
and transcendent, is determined by the attitutde and interests of
the subject. 
They are appropriated by the subject, they become ``my'' objects -- entering the horizon of
consciousness, they become immanent.\footnote{In spite of all the talk about their
transcendence which, in fact, affects only the mystical site of identity but not the
actual characteristics and properties, that is, all that is of relevance to the subject.}
In fact, consciousness, and even more so 
knowledge, are forms of control. \quo{Possessing, knowing, and grasping are synonyms of power.}
Making things definite and transparent is a way of
appropriating them; giving them a definite, final form is a way of 
removing the source of
potential uncertainity and surprise from them, a way of relegating the \lev{mystery} 
(Ooh! these words!) to the store of the irrelevant.

All these aspects are essentialy equivalent or, at least, co-extensional. Light implies 
visibility which also requires definiteness and makes appropriation possible. 
They are gathered under the name of \lev{The Same} or, as I would like
to call it, {\em The Visible}. 
This is the only world of (the traditional) philosophy of 
consciousness. Even the
most evanescent phenomena, like deep feelings, personality or Self, are thought of as
being somewhere, at the very bottom, at their deepest site, \lev{The Same} -- determined 
and identical to themselves.
%
\prg{The Other}
In spite of the continuous attempts at appropriating the world and predict its surprises, 
everybody knows the situations where the world overcomes one's powers. It is not 
necessarily even the talk about the ``limiting situations'' in the sense of Jaspers. 
It may be simply an encouter with a situation which I am not able to handle, with a person
whose intelligence surpasses my abilities to argue and resist. It may be
a situtation in which I do $X$, even if I wish, argue and feel that I should do $Y$. 
Above all, it may be
an encounter with a situation in which I have an irresistible feeling that I should do $X$, 
even if all my wishes and all my arguments would like me to do $Y$. 

``We are not masters of our being'' because the true master is \lev{The Other} or, as I 
would like to call it, {\em The Invisible}.
\lev{The Other} is the totally different, the alterity which it is in principle impossible
to appropriate. 
Consciousness constructs \lev{Totalities} -- objects, or complexes thereof -- with the
intentional sites of identity. {\lev{The Other}}, on the other hand, is the overpowering, the
essentially transcendent, and hence ungraspable \lev{Infinity}.
In the moment we try to draw it into the sphere of clear light and make it
Visible it ceases to be \lev{The Other}. Its visible image becomes crippled, while it itself
withdraws into even greater darkness. \quo{If one could possess, grasp, and know the other, it
would cease to be other.} Yet, \lev{The Other} is constantly in me, in my continuous
self-consciousness I know that I am not a mere subject controlling the visible world, but
that there is more there -- there is \lev{The Other in The Same}. I know it, and I am in
contact with it, but this contact is not of the same kind as counsciousness of an object.
%
\prg{The Other in The Same}
%
There are three crucial points concernig these two notions. 1)~\lev{The Other} is absolutely, 
unsurmountably different, it is an alterity which can never be appropriated and made
\lev{The Same}. 2)~Then, it follows, \lev{The Same} must and does remain \lev{The Same}, it cannot disappear
in a unity with \lev{The Other}, it cannot leave its subjectivity and become something else.
I am a self-consciouss individual, that is, aware of my being here-and-now and not 
there-and-then, aware of the external world, others, and my limitations. 
So it has been, and so 
will it always be. 3)~This awareness is an expression of the fact that
\lev{The Same} cannot even pretend that \lev{The Other} is not there. There is a tension and 
there always be, in spite of a long series of attempts to pretend that it can be overcome.

Yet, at the same time, \lev{The Other} in its alterity, is not just something external and remote.
In a sense, it is, or may be, very close and manifest itself in 
its proximity.\footnote{{\em ``Proximity''} is used to distinguish the 
closeness of \lev{The Other} from the {\em ``presence''} of \lev{The Same}, of the objects.}
\lev{The Other} cannot be appropriated by the subject without losing its ``otherness''. 
There is no fusion, no mystical
union, no similarity or community. All these are possible relations with \lev{The Same}. \lev{The Other}
can only be encountered as definite otherness. Thus, we renounce the idealisitic dreams of
subject's fusion with the object, of identity of knowledge and being, of unity of man with
God. Man has to live on the earth. But this life is not necessarily the fall of Heideggerian
{\em Dasein} into the world of beings, nor the Platonic return from the realm of ideas
to the world of appearances. It is simply {\em the} form of our being. Just as it 
cannot be dissolved in a fusion with \lev{The Other}, so neither does it preclude  
the possibility of experiencing its proximity.

A play of words illustrates this: \lev{Infinite} is more or less synonymous with \lev{The Other}, 
just like \lev{Totality} is \lev{The Same}. But the ``in'' in the ``In-finite'' is as much an
expression of the negation of the finite, as it is a preposition indicating 
``in the finite''.\footnote{Cute, and you can do the same with The In-visible.}
At a bit more technical level, the contact with \lev{The Other} amounts to suspending the 
principle of intentionality
which governs only consciousness. {\lev{The Other}} never enters the sphere of intentional 
acts, it is never up to the subject (whether transcendental or not) 
to determine its proximity or form. One is not even
conscious of {\lev{The Other}}, although, one is not unconsious of it either. If this sounds
implausible, we should repeat that one is not consious of it in the sense of
intentional acts. But one is not unconsious of it because  it is close to one and one
is aware of it; for instance, in the form of immediate self-consciousness, in the
vague, indefinite intuitions, in the experience of duties which one feels obliged to 
conform to, even though they do not originate in one's conscious will. 
One knows \lev{The Other} but not as one's object, as something one can grasp.
On the contrary, it is \lev{The Other} who grasps me, \quo{invades and persecutes} me, who 
sizes me and takes me into possession.

Man is a borderline between  \lev{The Other} and \lev{The Same}, between heaven and earth.
His relation
to \lev{The Same} is basically that of control and command, while to \lev{The Other} 
that of being under its command, of being summoned.
A wide variety of words and concepts is applied to indicate 
this being summoned:
\lev{trauma}, \lev{wakefulness of consciousness}, \lev{one being a hostage of \lev{The Other}}, 
\lev{responsibility}, \lev{obsession}, \lev{passivity -- more passive still than the passivity 
of things}.
%}
I'll try to say something more about this in section~\ref{sec:theOther} where a more
concrete and, as it seems, the originally motivating form of the contact with 
``otherness'', namely the relation with another person, is described.
At the moment, let it suffice to emphasize that the concept of {\lev{The Other}} seems to 
encompass all that is not withing subject's power, all unpredictability and overpowering
of the world and others. Consequently, there seem to be a lot of (intended or not?) 
equivocations: {\lev{The Other}} is once another person, once the future, once my self, 
then God, then again another person.
%
\comment{
\lev{The Other in The Same} accounts, among other things, for the distinction between Ego and Self, 
or between
{\em myself} and {\em my self} (i.e., my deeper or higher self). I myself am the subject 
who relates to 
the world of my objects, who with greater or smaller, but in principle always possible 
success, controls these objects; I am a being-in-the-world. In all this activity
I am always accompanied by my self, by a constant proximity of another, bigger, and evanescent
me. My self is mine, it is within me, but it is not myself -- 
just like \lev{The Other} is ``\lev{in The Same}'' but is not \lev{The Same}.
I am myself but not my self -- just like \lev{The Same} is \lev{The Same} but not \lev{The Other}.
 My self is ``greater'' and somehow 
``beyond me''. It is the self which is mine, which I have, but which I never am and 
which I never can grasp in full.

This self of mine is not a mere pure self-consciousness
as idealism would like it. Pure self-consciousness is thoroughly actual, it is always
here and now, in perpetual presence. Everything appearing within the horizon of
actuality is for this self-consciousness arbitrary and meaningless, unrelated to other 
moments of pure actuality. This
is the source of all the problems with reconstructing any continuity of the self from the
notion of self-consciousness alone.
\lev{The Other} -- here, my self -- is by definition beyond the consiousness' horizon of 
actuality and, I am tempted
to assume, Levinas would not like to construct its identity out of the pure moments. 
Besides, the whole question of identity of the self, which is \lev{The Other}, seems illegitimate
since identity is exactly the category of \lev{The Same}; it applies only to the objects 
with \lev{ipseity}, to the \lev{totalities} of consciousness but not to \lev{The Other}.
} \vspace*{-3ex}
%
\section{\lev{There is}}
Since Kierkegaard, through Heidegger, to Sartre, nothingness has been the central concept
of existentially oriented philosophy. One imagined that its expreience, bringing one face to
fact with the most personal phenomenon of death, could serve as a means of defence against 
the mechanical and impersonal thinking of the time. And so, the human, the existentially 
relevant became, if not synonymous, so in any case necessarily related to nothingness, the 
absurd, death, anxiety, etc.

Levinas offers a simple, yet natural, way out of the fascination by this concept, its
variants and derivatives. In fact, he simply postulates that the primary experience is not
that of nothingness but, on the contrary, of something being there. 
%
\prg{Nothingness}
Nothingness is perhaps a legitimate concept but \quo{to `realize' the concept of nothingness
is not to see nothingness but to die}! The talk about nothingness is, in a sense, as
irrelevant as the nothingness itself. One never encounters it in one's life because
to encounter it means to perish. Death is for Levinas, unlike for the existentialists,
always a \lev{death of The Other}. The death of another person is the death as I 
experience it -- 
I can never experience my death for in the moment it happens, I am already not there.

As to nothingness, one cannot even imagine it -- in any situation where
one tries to imagine something, perhaps even nothingness itself,  in any attempt at 
removing all the contents, at 
reaching to the bottom of all things, to the absolute limit, there remains a residual
feeling that something is there. 

\prg{There is}
This ``something'', however, is not anything definite or limited. 
It is not personal and has no ipseity, it is not this or that. Rather, it is 
a sheer feeling that simply \lev{There Is} something, a \quo{murmur of being}.
It is the unavoidable confrontation with the existence, the impossibility of withdrawing 
without disappearing, not the presence but -- on the contrary -- the \lev{impossibility of 
death}.

The peculiar character of \lev{There Is} consists in abolishing the subjectivity of the
subject. It is impersonal through and through -- not presenting one with anything particular, 
it does not leave the subject any space for movement or action. There is no object and, 
consequently, the subject ceases to be a subject. One only experiences (if we can still call it
an experience) the irresistible proximity of something else, something more, \lev{the horror
of existence}. 
%
\comment{I am tempted to see this as a situation of a person who by all means would like to
escape, avoid the world but is unable to decide to commit suicide. He is terrified by 
the prospect of tomorrow, horrified by the necessity to get up and live through yet another day.
The world threatens 
him, he has lost his self-confidence and power to take up his life, but he is too
weak, too attached to the world, to put a definite end to his suffering. 

Thus, as the \lev{the horror of existence}, the experience of \lev{There Is} has taken over the 
fundamental characteristics of the
existentialists' `death', `being-towards-death', `nausea'. The only difference is that the
experience is caused not by nothingness but by \lev{There Is}.
I think, this is a historical confusion, or else a tribute paid to the existentialists, rather 
than a genuine insight and intention of Levinas. \lev{There Is} is very much like \lev{The Other}, 
except for its entirely contentless abstraction. Nothing more follows from \lev{There Is}, 
while \lev{The Other} has thousand faces. Levinas' attempts to found ethics on the transcendence and
the proximity of \lev{The Other} should rather imply that even this contentless experience 
is not associated with dread and anguish. Instead, it may be a 
genuine experience of acceptance, of thankfulness and gratitude for being.}
%
There are some indications that he might think in this direction, namely, that the ultimate
experience (or the experience of the ultimate?) is not determined in itself but depends on
whether we comport ourselves with anguish and rejection or else with faith and trust.
It seems that nothingness, or \lev{There Is}, marks a break of subjectivity caused by the
invasion of \lev{The Other}. It is not \quo{a limit or negation of being} but is \quo{possible
as interval and interruption}, an interruption of \quo{consciousness' aptitude for sleep, 
suspension}. Consiousness, dwelling in the security of \lev{The Same}, is really ``sleeping'' -- 
only \lev{The Other} can awaken it, confronting it with something other than itself and summon it to
a genuine encounter. 
\comment{
Nothingness is an experience of a subject confronting \lev{The Other} but not recognizing it, of a subject 
engaged into the secure world of immanence, who suddenly becomes bereft of this world and sized by the 
thrill of Angst. Since this world has been reduced to the transparency and visibility, 
it has been stripped of all unpredictability and mystery. Yet the feeling of 
unpredictability, or else the knowledge of one's own finitude and limitation, is an 
unforgettable part of life. There is a dose of insecurity at the bottom of each confidence.
When this feeling of insecurity confronts the projects of subjective control
within the transparent world of consciousness, the subject does not find anything specific, 
anything objective -- The Invisible cannot be made Visible and it forever evades the 
subject's categories and determinations. Thus, nothingness is only a finite reflection 
of Infinity.

But this only means that nothingness appears as the last (or first) experience only as 
long as one stays attached to the subjective form of being and perception, as long as
one attempts to turn everything -- even \lev{infinity} -- into a visible \lev{totality}.
Renouncing uniqueness and self-sufficiency of The Visible,
recognizing that man is not a master of his being, immediately opens up the
sphere of The Invisible. The sheer \lev{There Is} is thus another possible reflection
of \lev{Infinity} in the finite, of The Invisible in The Visible which now has recognized 
the proximity of \lev{The Other}. Both nothingness and \lev{There Is} are contentless 
experiences, 
they do not provide the subject with anything Visible. But while the former provides it
with nothing, the latter shows that beyond the sphere of Visibility there is an 
\lev{infinite} world of \lev{The Other}.

See also the final comment in \lev{Freedom} on page~\pageref{freedom} (in the following 
section~\ref{sec:theOther}), where this line of thought is further substantiated.
}\vspace*{-4ex}
%
\section{The Other -- or the other person?}\label{sec:theOther}
The main motivations of Levinas are clear enough: ethics and moral
responsibility have been neglected in our world; the relations with other people are all
too often reduced to some formal and external contacts, etc., etc. For quite a long time
he has been exchanging letters with Buber, who paused speaking about You and 
Others only when Zionism made claims to his time. 

Let's now review briefly some forms in which \lev{The Other} announces its proximity. As a matter
of fact, for the most it will be the other announcing {\em his} (or hers) proximity.
I will stick to the form ``\lev{The Other}'' but, for the most, will use the personal pronoun ``he''
instead of ``it''. The latter simply does not make much sense, but you are encouraged to
figure out what the possible meaning might be after replacing ``he'' by it (i.e., ``it'').
\comment{
Levinas wants \lev{The Other} to be a broader notion than 
just `the other person'. Yet, the other person is like a shadow appearing each time
\lev{The Other} is mentioned. I made an attempt to put something more into it (to begin with by
calling it ``The Invisible''), but am not sure how much of his use could not be rephrased
(perhaps even clarified) by inserting ``person'' just after ``\lev{The Other}''.

He is reluctant to draw explicitly a clear distinction between the level of objectless 
experiences and attitudes and the level of concrete manifestations, between  the 
ontological and ontic levels (to speak Heideggerian), between the levels of higher and 
lower values (as Scheler would say), between forms and contents (as, very inadequatly,
one could put it in the more traditional language).
Nevertheless, the distinction is present, only that it gets submerged in the
distinction between \lev{The Other} and \lev{The Same}. This makes it difficult to recognize the ``stratification''
which apparently exists {\em within} \lev{The Other}, the differences between the more abstract and the more
concrete forms of contact with \lev{The Other}. Most of the following notions seem to belong to the higher, 
more abstract level. They remind of Heidegger's
``existentialia'' -- the ultimate dimensions of our being, preceding and determining the concrete
forms of expression and manifestation.
} \vspace{-3ex}
\prg{Obedience}
Just like subject commands its world of \lev{The Same}, so does \lev{The Other} command the subject.
This is not a categorical imperative nor any specific commandment --  only (the proximity of
\lev{The Other} felt by me as) the sheer need to obey, the summon to listen. 
No specific commitments follow from this for they are at a different level. 
Obedience without commitments -- perhaps, a
modesty willing to listen, a lack of pride willing to serve before even there is somebody
to listen to and to serve. 
\comment{
As such, it denotes an existential attitude rather then any concrete experience. 
Probably, it becomes a particular experience through concrete calls, through concrete
commitments. The insistence that it comes {\em before} the particular commitments seems to 
originate
from the fact that the ``before'' does not refer to any temporal or ontological order but 
rather to an axiological one. In Scheler it would say that ``commitements are 
attitudes/reactions to the values which {\em are founded by} a higher value to which we
react by obedience''. Or else, the forms of commitements may vary indefinitely without 
violating the obedience -- a potential infinity of various commitements may express the same
obedience.

At the same time, although it is not a concrete experience, it is still an experience. We may
feel and know, that is, experience our continuous obedience in the immediate self-consciousness,
we may be aware of this attitude independently from any particular commitements. 
This experience will then  be at a deeper level of self-consciousness underlying all
concrete experiences.
}\vspace*{-3ex}
\prg{Responsibility, Hostage}
I do not grasp \lev{The Other}, do not control or possess it. On the contrary, it is \lev{The Other}
who invades me and -- takes me a \lev{hostage}. Everything is built on this basic 
category of power: I am the master of \lev{The Same}, \lev{The Other} is the master of myself -- I am his
hostage.

This \lev{being a hostage} contains the intended meaning -- I am not a slave but I have been
captured, made a hostage, and now I have to buy myself out of this
captivity. Since being a hostage of \lev{The Other} is the primary relation, I can never do
it once and for all, but I have to do it over and over again, constantly.
\lev{Responsibility} is the way of paying the ransom, 
of obediently answering the summon of \lev{The Other}. I take him into myself and make myself
responsible for him. 
\comment{Again, this responsibility is not for anything specific, it is for \lev{The Other} as other. 
It is more like a ``responsible attitude'' preceding any concrete deeds and sins, 
like ``responsibility for everything other than myself''. Reality of the Original Sin seems to
work in the same direction: making us sinners, even before we could commit any sin, it also 
makes us responsible for everything including what we have not done. The difference is that 
\lev{responsibility} does not appeal to any prior sin -- it is {\em the primordial} relation, 
I am responsible before anything has happened.
% Does it not remind of the Original Sin?
At the same time, this responsibility ``for everything'' is like a feeling of friendship 
with and proximity of the whole universe. 
It embraces and establishes contact with everything lying beyond me. The vagueness of this notion
indicates that it should be taken  as meaning ``an open, ureserved acceptance of being, 
of \lev{The Other}'' rather than ``responsibility for something''.

Since \lev{The Other} is not what I can determine and control, it is 
responsibility {\em for something I have not done}. 
Before you shrug -- this does make a lot of sense if only we do not focus on 
the notions like free will and legal kind of responsibility which is just accountability.
What I do does not always follow from my wishes and decisions. One says sometimes ``It was not me, I
was not myself when I did it''. If it was not myself, then who? My self, I guess,
\lev{The Other} in me.
I may not be accountable for things done in unconscious state, but I am still responsible for them, 
no matter whether I wanted and tried to realize them or not. One is responsible for oneself as well as 
for one's self, for \lev{The Same} and for \lev{The Other}.

One often hears the excuses of the kind ``he had a difficult childhood, he grow
up in a criminal environment and, consequently, we should blame this social environment,
and not him''. Levinas' notion dismisses such ridiculous arguments -- nobody is
to blame but only I am responsible, for myself as well as for \lev{The Other}, for what I wanted
and intended as well as for what I did not.
}
Since \lev{The Other} is what, in Levinas, replaced the traditional notion
of transcendence, responsibility becomes thus the way in which transcendence infiltrates the
immanent, or, more precisely, the way in which subject comes into contact with the transcendent.
At the bottom of this lies the idea that the first philosophy is ethics -- it is the
true philosophy of transcendence and of its relation with immanence.

\prg{Asymmetry}
It should be obvious by now that in the relation I have to \lev{The Other} there is no symmetry. It is
not, as many would like it, a relation of reciprocity. 
The critique of Buber boils down to just this point -- with him the I-You relation is mutual, 
reciprocal and hence, Levinas claims, formal, it loses its existential concreteness.
But \lev{The Other} is
the master -- I am his hostage. The relation is not consummate through any fusion of the
two terms, through any similarity or community of interests or attributes, not even through
sympathy, no matter how many reasonable words Scheler might have spilt over it.

Relation to \lev{The Other} is always based on the recognition of his impassable alterity -- he 
is above me and never
becomes mine, he can always surprise and, above all, always puts upon me the obligation of
responsibility. In this context Levinas invents a new word -- \lev{substitution} -- which seems
to mean: I have to ``substitute myself for \lev{The Other}'', make his existence, his 
life, his happines {\em my} responsibility. 
At the same time, of course, I cannot expect him to substitute himself for me. As \lev{The Other},
he is above me, greater than me, the one who commands and summons. 
By a kind of definition, he is `the one for whom I can and should
substitute myself but by whom I cannot be substituted'. 
\quo{[...] in the recurrence to oneself there is a going to the hither side of oneself. [...]
Is not the signification of responsibility for another, which cannot be assumed by any freedom,
stated in this trope?}

\prg{Freedom}
This fact, that nobody can be substituted 
for me is Levinas' notion of the ``unique individuality of my person'' -- I am 
irreplacable,
nobody can do what I have to do. This goes as far as saying that \quo{to be free is to have
to do something that nobody else can do}. Thus freedom is based on bondage, on my being
hostage and, at the same time, is intimately related to my individuality. And this is the only
notion of freedom which, instead of separating me from the others, actually presupposes
a relation to the others.

Pure freedom, which is posited as the very first principle and determination of human beings,
which I can freely assume in my conscious or subconscious life, is an act of separation. 
Above all, it makes it impossible to grasp the meaning of responsibility,  of this 
genuine relation with \lev{The Other}. Two such freedoms
are like monads without a chance for a communication, necessarily enclosed within their 
respective worlds.
%
\comment{Finally, Levinas couples freedom not just with ethics, but with almost metaphysical
goodness. \quo{This antecedence of responsibility to freedom would signify the Goodness 
of the Good:
the necessity that the Good choose me first before I can be in a position to choose [...]
It is antecedence prior to all representable antecedence: immemorial. The Good is before 
being.}\label{freedom} 
It seems to go like this: `freedom is based on bondage, responsibility is what comes first and
makes freedom possible. But responsibility is to the Good.' Hmmm...?
Notwithstanding the fact that responsibility for (or to) something does not necessarily
mean that this something is good, let us buy this passage for the time being.
Levinas does not make the connection between this and the 
\lev{There Is} but I think it is what he intends. I would be also tempted to remove one `o' 
from the `Good' (but not from the `Goodness') or else, replace the latter by \lev{There Is} --
the expression `Goodness of the Good'' does not even deserve a comment.
The meaning of this quotation seems to point exactly in the direction that the experience
of \lev{There Is} is not dread and \lev{horror of existence} but, on the contrary, a meeting
with goodness which I accept as coming from \lev{The Other}. And notice that this is goodness 
of \lev{There Is}, not of any particular being; it is before any ``representable'' being, 
before any concrete acts, commitments. It is an attitude!
If I perceive what thus comes first, before any concrete being,
as good, if I accept it -- no matter what it may turn out to be {\em in concreto}, then it
is easier to understand that the genuine relation to \lev{The Other} is responsibility. For although
I can possibly be responsible for (or to) something bad, it is easier to assume responsibility 
for (or to) something good. (\lev{The Other}, and consequently, responsibility are so broad  notions that
whether it is `for' or `to' or something else is hard to tell. Probably it is 
all these things together.)

I would need to read something more of his ethics to determine whether this is a 
justified interpretation.
Perhaps it involves some undesirable emphasis on myself, that ``I am responsible 
{\em because} it is Good'', while Levinas would like to have it that ``I am 
responsible {\em and therefore} it is Good''.
} \vspace*{-2ex}
% 
\prg{Proximity, absence, distance}
\lev{The Other} is never present -- present is only \lev{The Same}, objects, contents of consiousness.
Thus, \lev{The Other} is \lev{absent}, but it is not an absence of this or that, the absence 
of my aunt who has left for America, the absence of my pen which I left at the office desk. 
It is the absence in the sense of `non-presence', the absence in the sence that 
\lev{The Other} is not there, nor here, in the way in which objects are.

To illustrate this point, Levinas comes up with a cute analysis of caress. It is not, as one
usually assumes, a purely sensual pleasure. Caress, loving
tenderness, is a sensuous act in which one tries to reach  the other. True, in caress one 
touches the other's body -- what else should one touch with one's hands? But the tenderness is 
addressed not so much to this body as to the other himself. 
It is a physical expression addressed to the one who is absent, that is, not present in the 
same way as his body is.

We say that ``the other is present'' -- Levinas would just say the it is his proximity
and not presence, we feel.
I guess that \lev{The Other} can also be totally absent (in the usual sense) -- this would simply mean
that we do not even feel his proximity. This is the attitude of the subject which does not 
want to recognize \lev{The Other} as the absolutely other.

The crucial fact is that the proximity of \lev{The Other} can never turn into presence,
he remains forever absent. And my relation with him cannot be genuine unless I recognize his
absence, his `non-presence', his unsurpassable alterity, in short, the \lev{distance}. 
Relation presupposes a distance -- when there is no
distance, the two terms disappear within each other. There is no proximity without a distance.

\prg{Time}
Alterity of \lev{The Other} is the impossibility to appropriate him, to enslave him. He is free 
in relation to me, but I am his hostage. He is always capable of surprise, of otherness.
But this is like the unpredictability of the future!

Future is essentially \lev{The Other}. Levinas is aware of this equivocation but, nevertheless,  
he insists on it. My relation with \lev{The Other} is essentially a relation with the future. This
goes fine for a couple who have just fallen in love with each other, but I am not so sure (have to check)
whether
it applies equally well to a 90 years' old one, after a life-long marriage. Yet one cannot 
deny that personal relationships degenerate when ``the routine takes over the place of 
spontaneity''.
A living relation with the other needs, besides some amount of secure sincerity, an element
of surprise, of \lev{The Other}. And this need not be any surprises which actually happen! It 
can be a mere possibility of something new -- or old! -- of something which comes to me not
from within but from outside, from a distance.

From this nice equivocation, one arrives at a general conclusion about time: 
\quo{The condition of 
times lies in the relationship between humans, or in history.}
No matter how unconvincing this may sound, how many implications have been turned into 
equivalences, the thought that a relation with the other is a
relation with the future deserves appreciation.

\newpage\section{For the Lazy Ones...}\label{sec:lazy}
Let me summarize briefly:
\begin{enumerate}\MyLPar
\item
Levinas is the first one who, in a serious sense of the word, overcomes the philosophy of
consciousness. This happens not so much by negating its validity or neglecting consciousness,  
but by merging it with the whole way of thinking in the categories of \lev{The Same} -- primarily, identity 
and unity. 
\item
\lev{The Other} comprises everything which eludes these categories, everything which a conscious
subject would call ``not mine''. This is truly a philosophical concept paying attention to the
fundamental similarity of things apparently as different as my self, another person, future, God, 
overpowering of the world, moral obligations.
\item
Subject, man, is situated {\em between} these two fields (not extremes!). He must occupy himself with 
\lev{The Same} but has also contact with \lev{The Other}.

The basic category of relation is that of power understood in a very broad sense. I am the master of
\lev{The Same}, I am able to control the world according to my concepts and definitions, the world of
immanence. But the master of my being is \lev{The Other}. \lev{The Other} embraces me, summons me, dictates me.
All that is not under my control will often limit my control, will determine the extent to which
I can act freely according to my conscious wishes and projects. And when I cannot act according to 
these, it is because \lev{The Other is in The Same}.
\item
Thus the two fields are antithetical, but this only means that they cannot be merged, the one cannot
{\em become} the other. However, they  are not external to each other: \lev{The Same} is {\em within}
\lev{The Other}, is surrounded by it. 
 \begin{enumerate}\MyLPar
 \item
\lev{The Other} cannot be grasped in the categories of \lev{The Same} because this would amount to it ceasing being
\lev{The Other}. \lev{The Other} is dissolubly other and must be approached on its own conditions, that is, 
the subject must renounce partly
its categories and listen to the peculiar summon in which \lev{The Other}, announcing its proximity, 
puts an obligation on the subject.
 \item
And so, no unity is possible -- unity itself is a category of \lev{The Same} and would amount to abolishing
the otherness of \lev{The Other}. The distance cannot be overcome and superseded by a secure presence,
 and the tension -- which is the true mark of the existence, of the confrontation with 
something more than myself -- can be ignored only for the price of inauthenticity.
 \item
In particular, there is no way out of subjectivity -- such a project, aiming at unity, is futile and 
still dominated by \lev{The Same}. The subject for ever remains a subject, here on earth. And only as a
finite subject can it look up and recognize heaven above himself.
 \end{enumerate}
\end{enumerate}

\end{document}