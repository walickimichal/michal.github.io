\documentstyle[a4wide,11pt]{article}

\newcommand{\qpa}[1]{[???\hspace*{1em} {\em #1}\hspace*{1em} ???]}
\newcommand{\lev}[1]{{\sf{#1}}}
%\newcommand{\prg}[1]{\paragraph{[#1]}}
\newcommand{\prg}[1]{\vspace*{1.5ex}\par\noindent{\lev{#1.}}\par }

\newcounter{THESIS}
\def\theTHESIS{\arabic{section}.\arabic{THESIS}}
\newcommand{\thesis}[1]{\refstepcounter{THESIS} \\[1ex]
% \vspace*{1ex}\par\noindent
  (\theTHESIS)\hspace*{1em}{\parbox[t]{13.5cm}{\sf #1}\\[2ex]}}
%     \vspace*{2ex}\par\noindent}
 %%     \begin{center}{(\theTHESIS)\hfill\sf 
 %%         #1}\hfill{$\bullet$}\end{center}}
%%%     \vspace*{1.5ex}\par\noindent{\em #1}\hfill{(\theTHESIS)}\par }

\newcommand{\comment}[1]{\\[1ex] {\small{$\lceil$#1$\rfloor$}}\\[1ex]}

\newcommand{\co}[1]{{\sl{#1\/}}}
\newcommand{\wo}[1]{``#1''}
\newcommand{\thi}[1]{`{\sl{#1\/}}'}
\newcommand{\phe}[1]{\thi{#1}} %{{\sf{#1}}}

\newcommand{\quo}[1]{``{\em #1\/}''}
\newcommand{\quot}[1]{\begin{quote} #1\end{quote}}
\newcommand{\ee}[1]{#1} %{{\sf{[#1]}}}

\newcommand{\ep}{\co{epoch\={e}}}
\newcommand{\Ep}{\co{Epoch\={e}}}

\newcommand{\MyLPar}{\parsep -.2ex plus.2ex minus.2ex\itemsep\parsep
   \vspace{-\topsep}\vspace{.5ex}}

\renewcommand{\abstract}[1]{{\small{#1}}}

\renewcommand{\baselinestretch}{1.5}

\begin{document}

\title{\co{Intensjonalitetsprinsippet}  -- metodens begrensninger \\
{\normalsize{(semesteroppgave -- filosofi grunnfag)}}}
\author{{\em Sunniva Vigeland} }

\date{}%\hfill\small\today}  % August 8, 1996
\maketitle 

\abstract{Jeg vil i oppgaven argumentere at fenomenologien, som 
enhver metode, har sine begrensinger. Studiet av intensjonale akter 
og bevissthetsobjekter gjennom fenomenologiske reduksjoner 
ekskluderer viktige erfaringsaspekt. Her menes ikke ekskludering i 
betydningen ignorering. Som universell metode fors{\o}ker den {\aa} 
presse erfaringer til {\aa} passe inn i en metodologisk korsett, hvor 
de kommer til syne p{\aa} en tildekket, mangelfull m{\aa}te.
Del~\ref{sec:over} gir et kort overblikk over noen aspekt ved den 
fenomenologiske metode -- med vektlegging av 
intensjonalitetsprinsippet -- som er relevante for argumentasjonen 
som f{\o}lger i del~\ref{sec:limits}.}

\section{Et kort overblikk}\label{sec:over}
Husserls prosjekt begynner med samme beklagelse som Kants:
\quo{I do not say that philosophy is an imperfect science; I say 
simply that it is not {\em yet} science at all, that as science it 
has not {\em yet} begun.} (\cite{RS}, min uthevelse) Med 
\ee{oppmerksomheten rettet mot} de ulike vitenskapers utvikling 
gjennom noen hundre {\aa}r, hevder ikke Husserl, som Descartes eller 
Kant, at vitenskapen skaffer oss den absolutte, endelige sannhet. Det 
viktige ved vitenskap er ikke s{\aa} mye resultatenes 
uimotsigelige sikkerhet som det at disse resultatene oppn{\aa}s 
og akkumuleres under veiledning av en metode. Metoden er det 
vesentlige trekk ved vitenskapen: vitenskapen trenger ikke en metode 
-- den {\em er} en metode; og fenomenologien skal v{\ae}re metoden 
for den vitenskapelige filosofi som \quo{has not {\em} yet 
begun}.\footnote{En fin balanse m{\aa} holdes her da den innledende 
intensjon var {\aa} oppn{\aa} tilgang til uimotsigelig kunnskap. Men 
siden et slikt kunnskapsideal ikke lenger finnes selv i 
vitenskapen, er man n{\o}dt til {\aa} oppgi det i den 
vitenskapsinspirerte filosofien. Det som er da igjen er metodologi.} 
Presisjonen og sikkerhet i matematikken har fasinert mange filosofer. 
Som Descartes og Kant, begynner ogs{\aa} Husserl sin s{\o}ken etter 
en metode med denne paradigmatiske vitenskap.

\subsection{Det matematiske ideal}
I matematikken er det ingen tvil \wo{hva en snakker om} -- objektene 
er gitt fullstendig, eller \co{adekvat}; deres definisjoner 
utt{\o}mmer det gitte innholdet, selv om deres konsekvenser kan 
forbli uklare. N{\aa}r en matematikker betrakter \phe{naturlige tall} 
eller \phe{triangler}, er han fullstendig vitende om hva de er -- 
selv om han ikke er vitende \ee{(i samme {\o}yeblikket, eller 
generelt)} om alle egenskapene de kan besitte.

Husserl observerer at sikkerheten som er oppn{\aa}elig i 
mattematikken skyldes den rent \co{intensjonale} karakter til 
objektene en behandler. Til tross for diskusjoner mellom platonister, 
intuisjonister og formalister, \ee{er matematikkens objekter} 
\wo{konstruert} heller enn \wo{oppdaget}; de er resultater av 
skarpsindige definisjoner og konstruksjoner, og ikke av en 
kofrontasjon med eksterne objekter og erfaringer.

Den fenomenologiske metode er ikke en direkte anvendelse av 
matematikk (eller matematisk metode) p{\aa} filosofi. Ikke desto 
mindre, m{\aa}let er {\aa} avdekke bevissthets strukturene og 
gitthetsm{\aa}tene til dens objekter som \ee{kunne m{\aa}les mot den 
matematiske tenkningens sikkerheten og presisjonen av dens objekter}.

 
\subsection{Intensjonalitet}
Startpunktet for Husserl er, som det var for Descartes, analysen av 
\co{cogito}. Men konklusjonene er sv{\ae}rt forskjellige. Husserl 
p{\aa}peker at det som er uimotsigelig gitt i \co{cogito} ikke er, 
som Descartes hevdet, bare \co{cogito} selv, det faktum et \phe{Jeg 
tenker}. Med samme sikkerhet og n{\o}dvendighet er der gitt ogs{\aa} 
denne \phe{tenkningens} objekt, \co{cogitatum}. Det kartesiske 
\co{cogito} avdekker, if{\o}lge Husserl, ikke en eksistens av et 
substansielt subjekt -- det avdekker bare seg selv, dvs. en akt, og 
det som er tilstedet i den.

Det som er tilstedet i en slik akt er ikke et separat subjekt og et 
eksternt objekt. Tilstede er bare akten -- en enhet\footnote{Eller, 
rettere sagt, \wo{samtidighet}.} av to poler: av \co{cogito} som er 
bevist av et gitt \co{cogitatum}. 
\quo{Og likevel er det klart at \co{cogitatio}-et i seg er cogitatio
av sitt \co{cogitatum}, og at dette \co{cogitatum} som s{\aa}dan, 
slik det er der, ikke kan skilles fra \co{cogitatio}-et.} \cite{RI},~s.216.
Disse to polene kan ikke sees 
separat. Det som kommer til syne for bevisstheten er ikke eksterne 
objekter, liggende \wo{utenfor \co{cogito}s boksen}, men 
\co{intensjonale objekter}, n{\o}dvendige \co{korrelater} til 
bevissthetsaktene.\footnote{Denne innsikt l{\o}ser bekymringene om 
\co{ting-i-seg-selv} som alltid unnviker bevisstheten. Den kommer 
ogs{\aa} utenom det feilaktige bildet av bevisstheten som en 
\wo{lukket boks} som pr{\o}ver {\aa} omfatte, vhja. interne 
representasjoner, eksterne ting.}
I hver bevissthetsakt, er det ikke bare bevisstheten selv, men 
ogs{\aa}, med samme n{\o}dvendighet, et objekt. 
Intensjonalitetsprinsippet formuleres ofte som:
\thesis{Bevissthet er alltid bevissthet om 
\co{noe}.\label{th:int}}\vspace*{-6ex}
%
\subsubsection{\co{Cogitatum}s transparens}
Siden \co{cogito} n{\o}dvendigvis involverer \co{cogitatum}, er det 
ingen barriere som separerer bevisst\-het fra det som er gitt for den: 
\quo{enhvert orgin{\ae}rt givende anskuelse skal v{\ae}re en rettskilde for 
erkjennelse, at alt som i intuisjonen frembyr seg for oss orgin{\ae}rt [...], 
det skal man ganske enkelt ta inn over seg som det det gir seg som, men ogs{\aa}
bare innenfor de grenser hvor det gir seg} \cite{Id1}.
Bevissthetsobjektene er ikke eksterne ting som \ee{bevisst\-heten 
m{\aa}, p{\aa} en eller anne m{\aa}te, skaffe seg adgang til}. Et 
objekt er et vesentlig aspekt ved bevissthet, et uunv{\ae}rlig 
korrelat ved enhver bevissthetsakt. Kanskje, er det ikke 
\wo{konstruert} p{\aa} n{\o}yaktig samme m{\aa}te som et matematisk 
objekt er. Men den er en uatskillelig del av den hele bevissthetsakt i 
hvilken bevissthet \wo{retter seg mot noe}, \wo{intederer sitt 
objekt}. Som et slikt, ikke eksternt men \co{intensjonalt objekt}, er 
det like fullt {\aa}penbart for \co{cogito} som \co{cogito} selv.

En rasjonell refleksjonsakt gj{\o}r \co{cogito} s{\aa}vel som 
\co{cogitatum} tilgjengelig for bevisstheten. Og \co{cogitatum} kan 
bli gitt med samme grad av tilstrekkelighet og fullstendighet som 
\co{cogito}, med samme grad av presisjon som matematiske objekter er 
gitt for matematikeren. Og siden dett er det generelle trekk ved 
bevisstheten, gjelder det ikke bare for persepsjons- eller 
matematiske objekter, men for ethvert objekt som kan komme til syne for 
bevisstheten.

\subsubsection{\co{Vesensintuisjon} og \co{eidetisk reduksjon}}
Men selvsagt, ikke alt vi er oss bevist fremtrer med en slik 
full klarhet. En god del ting viser seg bare utydelig, uten skarpe 
grenser og karakteristikker. Tross dette, fortsetter tesen, kan alle 
objektene til sist bli gitt i en fullstendig klar intuisjon. 
Dette fordi objekter, som \co{intensjonale objekter}, er objekter {\em for} 
bevisstheten og er \wo{konstituert som korrelater til intensjonale 
akter}.

Uten {\aa} g{\aa} for mye i detalj, la oss bare vektlegge 
hovedpoenget: hvert \co{intensjonalt objekt} har et \co{vesen} 
(\wo{essens}) som kan bli gitt i en \co{vesensintuisjon}. Veien til 
en slik intuisjon g{\aa}r via \co{eidetisk reduksjon}, eller 
\co{ideasjon}. I hovedtrekk: \co{ideasjon} inneb{\ae}rer det {\aa} 
variere m{\aa}ten objektet blir anskuet p{\aa}, {\aa} plassere 
objektet i nye kontekster, se det fra ulike sider. Det 
tilbakev{\ae}rende av en serie av slike variasjoner, det som er igjen 
etter at tilfeldige aspekt er blitt byttet ut og erstattet av andre (\quo{i vesensskuen gis
det avkall p{\aa} tilfeldigheten} \cite{RI},~s.223), 
{\aa}penbarer \co{essensen} -- objektet til 
\co{vesensintuisjon}.\footnote{\co{Essens} er et idealt objekt og 
\co{vesensintuisjon} er meget likt den \co{intellektuele intuisjon} 
s{\aa} bestemt benktet av Kant.} \quo{Ethvert individuelt noe har en \co{eidos} som kan
skues eller fattes generelt i seg},~\cite{RI},~s.231.

En m{\aa} ha klart for seg at \co{ideasjon} ikke er en prosess hvor 
man {\em avleder} essenser fra fremtoninger. \co{Essens} er ikke avledet 
fra framtoninger -- den kan bare {\em sees i} fremtoninger. Spesielt, 
\co{ideasjon} har ingenting med induksjon {\aa} gj{\o}re, den er 
ingen generalisering fra en rekke erfaringer -- den er bare en metode 
for {\aa} finne \co{essens} blandt fremtoninger.
N{\aa}r en \co{vesensintuisjon} finner sted, er det ikke lenger 
n{\o}dvendig med mer variasjon. P{\aa} en m{\aa}te avslutter 
\co{vesensintuisjonen} usikkerheten i forhold til hva den neste 
fremtoning kan bringe med seg. Den etablerer en innsikt som, fra 
n{\aa} av, hersker over den uforutsigbare rikdom i 
framtredelsesverdenen.
Fenomenologien beskriver m{\aa}tene {\aa} oppn{\aa} vesentlig 
kunnskap, og \co{essenser} gitt i \co{vesensintuisjon} er dens
prim{\ae}re studieobjekter.

\subsubsection{\co{Epoch\={e}} -- reduksjon av det irrelevante}
Fenomenologi er dermed studiet av \co{essenser} som framtrer i
fenomener som \co{intensjonale objekter}. Og Husserl mener bestemt at 
alt som er verd {\aa} studere h{\o}rer innefor dette domenet. Det er 
bare et gammelt sp{\o}rsm{\aa}l som ikke kan l{\o}ses av studiet av 
bevissthetsinnhold,nemlig, sp{\o}rsm{\aa}let om disse innhold 
ogs{\aa} eksisterer {\em uavhengig av} bevissthet. Men siden en slik 
\wo{uavhengig eksistens} ikke kan gies {\em i} bevisstheten, er 
sp{\o}rsm{\aa}let ansett som irrelevant. For {\aa} slutte {\aa} 
stille det, anbefaler Husserl den \co{transcendentale reduksjon}, 
\co{epoch\={e}}. Dens m{\aa}l er nettopp dette: \wo{sette verden i 
parantes} og, isteden for {\aa} bekymre seg for det ul{\o}selige 
(fordi ikke gitt som et fenomen) sp{\o}rsm{\aa}l, fokusere p{\aa} 
innhold som er tilgjengelig for bevisstheten. \quo{Man m{\aa} alts{\aa}
f{\o}rst fatte den rene bevissthets vesen, og s{\aa} se hva som ikke er
ren bevissthet, hva som i ontisk forstand er transcendent i forhold til bevisstheten.
Som f{\o}lge av dette er det mulig {\aa} gjennomf{\o}re reduksjonen, 
og derved f{\aa} den rene bevissthet for seg selv, som rest.}~\cite{RI},~s.252.

Reduksjonen rettferdiggj{\o}res ved den overbevisningen 
% at alt annet enn dette sp{\o}rsm{\aa}let kan studeres fenomenologisk, 
at utenom 
fenomener st{\aa}r ingenting annet igjen enn sp{\o}rsm{\aa}let om en 
slik \wo{ekstern eksistens}. F{\o}lgelig er det eneste som blir 
ber{\o}rt av \co{epoch\={e}} det som er \wo{utenom fenomener}, som 
ligger \wo{utenfor ren bevissthet}, dvs. den mulige eksistens av objekter 
uavhengig av bevissthet. \co{Epoch\={e}} etterlater alt innhold 
uforandret og tilgjengelig for bevisstheten og ekskluderer 
sp{\o}rsm{\aa}let om uavhengig, dvs. ikke-fenomenaktig, eksistens:
%
\thesis{\co{Epoch\={e}} reduserer bort alt det ikke-fenomenaktige, 
alt det som, fordi det ikke fremtrer for bevisstheten, faller utenfor 
intensjonalitetsprinsippet.\label{th:epint}}
%
Legg merke til at gjenstanders eksterne eksistens
er forskjellig fra det faktum at de {\em er gitt 
som} eksterne. Men de er gitt som eksterne {\em for} bevissthet og 
det faktum at de {\em er gitt som} eksterne betyr ikke at de faktisk 
{\em er} slike. Denne \wo{gitthet som eksternt eksisterende} er et 
viktig aspekt ved mange objekters gitthet og vil bli observert av den 
fenomenologiske analyse. Men at de faktisk {\em er} eksterne er ikke 
-- {\em kan ikke} v{\ae}re -- gitt i bevissthet, og h{\o}rer dermed 
ikke inn under et fenomenologisk studium.

\section{Intensjonalitetens begrensing}\label{sec:limits}
Metodologi sikter alltid mot {\aa} ekskludere det som ansees uviktig 
ellers ikke relevant. I vitenskapen ekskluderes slikt typisk gjennom 
implisitte antakelser og stilltiende forst{\aa}else i forhold til hva 
som er forsvarlig og hva som ikke er det. Metodologiske diskusjoner 
kretser alltid rundt sp{\o}rsm{\aa}l om hvilke fremgangsm{\aa}ter, 
observasjoner, til og med fakta, som skal tillates og hvilke som skal ansees 
ugyldige. \co{Epoch\={e}} er fenomenologiens m{\aa}te {\aa} 
utelate det som metodologien anseer for {\aa} v{\ae}re irrelevant. 
Denne delen vil prim{\ae}rt pr{\o}ve {\aa} begrunne p{\aa}standen at 
%
\thesis{Intensjonaliteten gjelder ikke for all former for 
gitthet.\label{th:nointGen}}
%
I (\ref{th:epint}) nevnes det \wo{ikke-fenomenaktige} og det som 
\wo{faller utenfor intensjonaltetsprinsippet} som om de var 
ekvivalente begreper. Fenomener og intensjonalitet er hos Husserl, om 
ikke ekvivalente, s{\aa} i det minste, \wo{samh{\o}rende}: ethvert 
intensjonalt objekt er tilgjengelig kun i fenomener og, p{\aa} 
den andre side, intensjonalitet er det fundamentale trekk ved 
bevissthet - fenomener fremtrer bare som fremtoninger av intensjonale 
objekter.
(\ref{th:nointGen}) indikerer at disse to aspektene ikke 
n{\o}dvendigvis h{\o}rer s{\aa} tett sammen. Da b{\o}r vi ogs{\aa} 
spesifisere om \co{epoch\={e}} reduserer bort det ikke-fenomenaktige 
eller det ikke-intensjonale innhold.
Tesen at
\thesis{\co{epoch\={e}} ekskluderer mer enn bare sp{\o}rsm{\aa}let om 
den eksterne eksistens\label{th:epex}}
kan framlegges under forutsetning av at vi, i (\ref{th:epint}), 
vektlegger det ikke-intensjonale, det som \wo{faller utenfor 
intensjonalitetsprinsippet}. Hvis man isteden vil framheve det 
ikke-fenomenaktige, vil vi ikke insistere p{\aa} riktigheten av 
(\ref{th:epex}). Uansett valget vi gj{\o}r her, forblir 
(\ref{th:nointGen}), som er hovedtesen v{\aa}r, uber{\o}rt
%
\subsection{Aktualitet}
Eksemplene p{\aa} fenomener som man finner hos Hussserl er for det 
meste av to slag: enten persepsjonsobjek eller matematiske entiteter. 
Det er relativt lett {\aa} f{\o}lge en analyse av hva som fremtrer i 
min bevissthet n{\aa}r jeg tenker p{\aa}, f.els., \phe{dette bordet 
foran meg}. Hvis jeg pr{\o}ver {\aa} se hva som fremtrer under 
analyse av \phe{bordet i seg selv} (eller \co{essensen} av fenomenet 
\phe{bord}) blir det vanskeligere men ogs{\aa} dette intensjonale 
objektet synes {\aa} v{\ae}re gitt med en tilstrekkelig grad av 
klarhet. Med litt innsats og oppfinsomhet kan hende, kommer jeg fram 
til noen \co{vesentlige} karakteriseringene av disse. Slikt kan jeg 
g{\aa} videre med analyser av \phe{rommet jeg sitter i}, 
\phe{byggningen jeg er i}, \phe{Bergen by}, \phe{min verden}, 
\phe{verden}, osv.

Er det virkelig ingen forskjell (metodologisk sett) mellom disse 
fenomener? Hva skal den antatte \co{essensen} av \phe{Bergen by}, eller 
v{\ae}rre, av \phe{min verden} v{\ae}re? Noe fremtrer i min 
bevissthet n{\aa}r jeg tenker disse, men hva? Jeg kan ha noen vage 
tanker om \phe{Bergen by}, men om \phe{min verden} kan jeg bare si at 
\wo{det er noe, men jeg aner ikke hva}. Det er \co{noe} -- som alt jeg 
er bevisst om -- og for {\aa} finne ut hva det er ville Husserl 
foresl{\aa} \co{eidetiske reduksjon}. Men hvor lenge skal jeg hold 
p{\aa} {\aa} variere og redusere? Resten av livet, antar jeg, fordi 
hva \co{min verden} var i g{\aa}r gjelder ikke i dag, og hva den blir 
til i morgen, kan bare morgendagen vise. Allikevel, min verden i 
g{\aa}r var like mye \phe{min verden} som min verden i dag er, og som 
min verden i morgen kommer til {\aa} v{\ae}re -- det er den samme 
\phe{min verden}, og den er sikkert forskjellig fra \phe{din verden}.

En viktig distinksjon som jeg har i tankene er mellom objektene 
(fenomenene) som naturlig passer inn i bevissthetens
\co{aktualitetshorisont} og de som ikke gj{\o}r det. Bevissthet er 
tvers gjennom {\em aktuell}, den er, muligens, \wo{den rene 
aktualitet}.\footnote{Selvsagt, med alle \co{protensjoner} og 
\co{retensjoner} men fremdelse bare en akutalitetsenhet sentrert rund 
et intensjonalt objekt.} Hver gang jeg er bevisst (om noe) hender det 
her-og-n{\aa}, innenfor en begrenset (hvis i det hele tatt utstrakt) 
aktualitetshorisont. Begreper, eller persepsjonsobjekter passer bra inni 
her, men \phe{min verden} gj{\o}r ikke det.

Ta et annet eksempel. Jeg m{\o}ter en vis man, dvs. fenomenet 
\phe{denne vise mann} fremtrer i min bevissthet. Det er et h{\o}yst 
individuelt fenomen og slik kan jeg reflektere over dette 
fenomenet, analysere dets innhold som er gitt meg her-og-n{\aa}. Men 
s{\aa} g{\aa}r jeg videre til et annet fenomen: \phe{visdom}, og ...? 
Hva finner jeg i bevisstheten min? Antakeligvis bare en mengde 
assosiasjoner som umiddelbart blir sett p{\aa} som \wo{psykologiske}. 
I vanlige termer, vil \co{ideasjon} bety \wo{{\aa} fors{\o}ke {\aa} 
bringe orden} inn i disse assosiasjonene. Men for {\aa} n{\aa} 
\co{essensen} av \phe{visdom} m{\aa} jeg virkelig v{\ae}re heldig -- 
vise mennesker har fors{\o}kt det i noen tusener av {\aa}r. (Riktig 
nok, s{\aa} kjente de ikke til fenomenologien, men jeg har fortsatt 
f{\o}lelsen av at jeg m{\aa} v{\ae}re sv{\ae}rt heldig. Og selv om 
jeg er det, s{\aa} hvor mange andre heldige mennesker er der som er 
uenig?)

At \co{intensjonale objekter} overskrider \co{aktualitetshorisonten} og 
fremtrer ulikt gjennom forskjellige fenomener var opplagt for Husserl. 
Men poenget her er at
%
\thesis{Det finnes fenomener som er vesentlig unnvikende, hvis 
\co{essens} ikke kan gies \co{adekvat} som et \co{intensjonalt 
objekt} innenfor aktualitethorisont.\label{th:noint}}  
% 
Slike objekter kan kalles \wo{transcendente} -- ikke i noen av 
Husserls betydninger men presist i betydningen av (\ref{th:noint}): ikke 
bare de selv, men ogs{\aa} deres antatte \co{essens} er \wo{for 
rike} til {\aa} bli \co{adekvat} gitt som \wo{intensjonale objekter} 
innefor rammen av en enkel akt. Spesielt, fremtrer de aldri som 
skarpe, \ee{veldefinerte} objekter med et klar 
\wo{identitetssentrum} -- deres \co{essens} er aldri gitt adekvat 
i \co{vesensintuisjon}. Tvert imot, fenomener som \phe{visdom}, 
\phe{verden}, \phe{kj{\ae}rlighet}, \phe{det onde}, har ingen 
\wo{ensartethet} (\wo{ipseity}) og blir aldri utt{\o}mt gjennom 
intuisjon eller tenkning. Ulik de \co{intensjonale objekter} gitt i 
\co{vesensintuisjon}, er disse alltid \wo{for rike} og \wo{diffuse}. 
De er mer som \wo{felter} -- noen overlappende, andre disjunkte -- av 
uutt{\o}mmelige muligheter av stadig nye manifestasjoner. 
F{\o}lgelig, den antatte \wo{bevissthets absolutte v{\ae}ren} har 
ingen mulighet til {\aa} gripe deres \co{essens} for, hvis de har 
noen, transcenderer den den \co{aktualitet} som bvevissthet er begrenset 
til. Istedenfor {\aa} bli fattet, tilkjennegies de gjennom {\aa} 
invadere bevissthet, gjennom {\aa} omslutte den og ta den i sin 
besittelse.

\subsection{Ikke-intensjonale fenomener}
Sant nok, det er ikke mye mer for en (ikke dogmatisk) filosof {\aa} 
analysere enn innholdet av hans bevissthet. Men det f{\o}lger ikke 
herav at alle slike innhold n{\o}dvendigvis er gitt som tydelige objekter.

Husserl analyserte fenomener som \phe{Lebenswelt} ved {\aa} anvende 
sin metode p{\aa} disse som p{\aa} alle andre fenomener. Men slikt 
gjorde ogs{\aa} Heidegger og Marcel, Scheler og Sartre. Ingen vil 
hevde at de har kommet fram til engang liknende konklusjoner. Husserl 
ville si at en genuin \co{vesensintuisjon} ikke kan motsi en annen, 
s{\aa} mest sannsynlig, noen av de endte ikke opp med en slik intuisjon.

{\AA} pr{\o}ve {\aa} bestemme hvem av de som gjorde og hvem som ikke 
gjorde det, ville antakelig bare f{\o}re til enda mer uenighet og 
meningsforskjeller. P{\aa} den annen side, ved {\aa} anta at de alle 
gjorde det, kan vi sitere en velkjent observasjon: i analysen av mer 
generelle fenomener f{\o}rer ikke fenomenologien i det hele tatt til 
enighet mellom forskjellige tenkere. \wo{Mer generelle fenomener} er 
ting som \phe{verden}, \phe{visdom}, \phe{mitt liv}, \phe{det onde}, 
osv., referert til i (\ref{th:noint}). 
Disse fremtrer ogs{\aa} for bevisstheten, men mangler denne 
bestemthet og \wo{ensartethet} som karakteriserer de \co{intensjonale 
objekter}. S{\aa} la meg kalle de \wo{ikke-intensjonale fenomen}.

En spesiell side ved disse, nemlig at de f{\o}rer til  
meningsforskjell heller enn et enhetlig syn, p{\aa}minner oss om en 
annen observasjon. En kan vanskelig uttrykke sin mening om et 
generelt emne uten {\aa}, p{\aa} samme tid, avdekke noe av seg selv. 
Ved {\aa} uttrykke mitt syn p{\aa} \phe{verdens karakter}, 
\phe{kj{\ae}rlighetens kvalitet}, \phe{ondskapens kilde}, framsetter 
jeg ikke bare noen fakta om objektene -- f{\o}rst og fremst 
\wo{uttrykker jeg {\em mitt} syn}, dvs., forteller hvem jeg 
er.%\footnote{Dette betyr ikke at det m{\aa} v{\ae}re subjektive...}

\subsection{Subjektivistisk illusjon}

Allikevel, som sagt tidligere, hevdet Husserl {\aa} ha utf{\o}rt 
fenomenologisk analyse av noen slike fenomener (som 
\phe{Lebenswelt}), og slikt hevder ogs{\aa} andre {\aa} ha gjort. 
(\ref{th:noint}) hevder at ikke alt i det konkrete bevissthetsliv 
fremtrer som tydelige, \co{intensjonale objekter}. Men det utelukker 
ikke muligheten for {\aa} omgj{\o}re et \co{ikke-intensjonal fenomen} 
til et slikt objekt og, dermed, gripe det innefor bevissthetens 
aktualitet. De \co{ikke-intensjonale fenomener} overskrider 
\co{aktualitetshorisonten} og fremtrer bare i \wo{diffuse}, uklare 
intuisjoner, i stadig nye former. Allikevel, kan de fratas deres 
uforutsigbarhet og \wo{annerledeshet}, og bli konstituert som 
\co{intensjonale objekter}, dvs., tvunget til {\aa} fremtre innenfor 
\co{aktualitetshorisonten}. 

\subsubsection{Begreper}
N{\aa}r jeg opplever kj{\ae}rlighet, kjenner jeg at dens karakter og 
makt overskrider ikke bare min forst{\aa}else men ogs{\aa} meg selv, min 
v{\ae}ren -- \wo{den er st{\o}rre en meg}. Men jeg kan, sikkert og 
visst, reflektere over den, bestemme noen sider ved den, for{\o}ke 
{\aa} karakterisere denne erfaringen, kort og godt, ta under 
overveielse et {\em kj{\ae}rlighets begrep}. Dette minner om 
\co{eidetiske reduksjon} med det forbehold at der siktes det  mot en 
\co{vesensintuisjon} og ikke mot et begrep. Virkningen synes, 
allikevel, {\aa} v{\ae}re mye den samme: den opprinnelige erfaringen 
forsvinner og, istedenfor, konfronteres en med et inbefattet, 
redusert tegn, et \co{intensjonalt objekt} -- et {\em begrep} -- som 
kan gripes i en enkel akt.

Begreper (eller, dersom det fremdeles insisteres, 
\co{vesensintuisjoner}) er middelet  gjennom hvilket
\wo{ufattelige}, \wo{for store}, \co{ikke-intensjonale fenomener} som 
overskrider \co{aktualitetshorisonten}, kan bli gjort, i det minste delvis, 
tilstede innenfor denne horisonten. Heideggers an\-a\-lyser av fenomenet 
\phe{verden} eller Sartres an\-a\-lyser av \phe{kj{\ae}rlighet} er ikke 
ennet enn an\-a\-lyser av {\em hva de forst{\aa}r  med} (assosierer, ser 
i) ordene \wo{verden}, \wo{kj{\ae}rlighet}. Resultatet er enten et 
begrepsmessig nettverk eller en stemning og en underliggende 
f{\o}lelse av undring (i det f{\o}rste tilfelle) eller avsmak (i det 
siste). Og f{\o}rst og fremst, f{\o}lelsen av at disse analysene 
peker mot noe vi kan erfare, men at uansett hvor mye som er sagt om 
det, vil det alltid v{\ae}re mer igjen {\aa} si.

\subsubsection{Reflektiv vs. umiddelbar bevissthet}
Det er viktig {\aa} huske at fenomenologisk analyse er en prosess av 
\co{reflektiv bevissthet}. Jeg setter meg ned og bestemmer meg for 
{\aa} analysere, reflektere over, erfaringen av noe. %ditt eller datt. 
Dette impliserer tydeligvis {\aa} henlede min 
oppmerksomhet mot det tilsiktede objektet, dvs. den konstituerer det 
\co{intensjonale objektet} -- for {\em refleksjonen}!! Og videre, 
refleksjonen kan fokusere p{\aa} objektet, p{\aa} min reaksjon p{\aa} det, 
p{\aa} dets gitthetsm{\aa}te. Alle trekk som Husserl 
tilskriver bevisstheten -- spesielt intensjonens rettethet -- 
passer uten restriksjoner til \co{reflektiv bevissthet}.
If{\o}lge v{\aa}r hovedtese (\ref{th:nointGen}) og dens presisering 
(\ref{th:noint}), kan ikke alle disse karakteristikkene overf{\o}res 
til \co{umiddelbar bevissthet} (eller til v{\aa} pre-reflektive 
v{\ae}ren) uten {\aa} forfalske den.

Den \wo{subjektivistiske illusjon} indikerer f{\o}lger av en slik 
overf{\o}ring: alt {\em kan gj{\o}res til} tydelige intensjonale 
objekter og studeres slikt men dette betyr ikke at alt {\em 
opprinnelig er gitt} som slike objekter. Den subjektivistiske 
illusjon best{\aa}r i {\aa} forveksle {\em muligheten} av {\aa} behandle alt 
som avgrensede objekter innenfor \co{aktualitetshorisont} med det at 
alt {\em faktisk er gitt} som slike objekter. Ved {\aa} betrakte 
\co{ikke-intensjonale fenomener} p{\aa} denne m{\aa}ten, forvirrer 
man deres natur og forminsker den betydning de har i v{\aa}re liv.

\section{Konklusjon}
Fenomenologien ble ikke den universelle metode for vitenskapelig 
filosofi som den var ment {\aa} bli. Isteden, ironisk nok (og for 
Husser ville det sikkert v{\ae}re en bitter ironi), ble den brukt 
hovedsakelig for {\aa} tillate eksistensialismens inntreden i domenet 
av respektabel filosofisk diskurs. 

Mye av fenomenologiens tiltrekkende kraft -- og mye av grunnen til at 
eksistensielt orienterte tenkere fant den s{\aa} tiltrekkende -- 
best{\aa}r i at den ikke ekskluderer noe fra feltet av potensielle 
unders{\o}kelser. Faktisk, alt kan (m{\aa}?) tiln{\ae}rmes via 
m{\aa}ten det fremtrer for bevisstheten. Spesielt, den avviser ikke 
individuelle erfaringer men, tvert imot, s{\o}rger for 
mulighet og middel for {\aa} studere nettopp slike konkrete 
fenomener. Dermed bringer den en filosof tilbake til en verden av 
\wo{konkret liv og erfaring}.

Men metodens forf{\o}rende kraft best{\aa}r i den myke overgangen 
mellom, ja faktisk en forveksling av, den \co{reflektive} og 
\co{umiddelbare} bevissthet. Den f{\o}rste er alltid rettet mot 
spesifikke (selv om ikke n{\o}dvendigvis klart gjenkjente) \co{intensjonale 
objekter} -- objekter av den aktuelle refleksjonen. Men, som jeg har 
pr{\o}vd {\aa} argumentere, slike objekter utt{\o}mmer ikke 
bevissthetens umiddelbare liv som alltid hjems{\o}kes av 
\co{ikke-intensjonale} fenomener.

Begrensningen ligger i at metoden er laget for {\aa} studere 
\co{essenser}, spesielt, at den foreskriver {\aa} betrakte alt som 
\co{intensjonale objekter} fremtredende innefor en
\co{aktualitetshorisont}. F{\o}lgelig m{\aa} de 
\co{ikke-intensjonale} fenomener, som unnviker en slik 
fremtredelsesm{\aa}te, presses inn i horisonten -- en prosedyre 
som n{\o}dvendigvis reduserer dem til noe de ikke er. For analysens 
form{\aa}l er dette, antakeligvis, uung{\aa}elig.  Det er ogs{\aa} 
greit -- men {\em bare s{\aa} lenge} en er klar over denne reduksjonen 
og gj{\o}r den eksplisitt. I motsatt fall, vil en bli stadig forvirret 
av det faktum at ens \co{vesensintuisjon} som 
\wo{avdekker} \co{essens} av \phe{verden}, \phe{d{\o}den}, 
\phe{kj{\ae}rligheten}, merkelig nok, savner den antatte \co{essens} 
og ikke er s{\aa} eksakt som den skulle v{\ae}re.

\begin{thebibliography}{SU}
\bibitem[1]{RS} E.~Husserl, ``Philosophie als strenge Wissenschaft'', 
{\em Logos}, I, 
1910-11, 289-341. %{\em Philosophy as Rigorous Science},
\bibitem[2]{Id1} E.~Husserl, {\em Ideen I}, s.52. 
   (etter \cite{RI})
\bibitem[3]{RI} R.~Ingarden, {\em Innf{\o}ring i Edmund Husserls fenomenologi}, 
   Johan Grundt Tanum Forlag, Oslo 1970.
\end{thebibliography}

\end{document}

