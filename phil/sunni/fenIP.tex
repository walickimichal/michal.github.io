\documentstyle[a4wide,11pt]{article}

\newcommand{\qpa}[1]{[???\hspace*{1em} {\em #1}\hspace*{1em} ???]}
\newcommand{\lev}[1]{{\sf{#1}}}
%\newcommand{\prg}[1]{\paragraph{[#1]}}
\newcommand{\prg}[1]{\vspace*{1.5ex}\par\noindent{\lev{#1.}}\par }

\newcounter{THESIS}
\def\theTHESIS{\arabic{section}.\arabic{THESIS}}
\newcommand{\thesis}[1]{\refstepcounter{THESIS} \\[1ex]
% \vspace*{1ex}\par\noindent
  (\theTHESIS)\hspace*{1em}{\parbox[t]{13.5cm}{\sf #1}\\[2ex]}}
%     \vspace*{2ex}\par\noindent}
 %%     \begin{center}{(\theTHESIS)\hfill\sf 
 %%         #1}\hfill{$\bullet$}\end{center}}
%%%     \vspace*{1.5ex}\par\noindent{\em #1}\hfill{(\theTHESIS)}\par }

\newcommand{\comment}[1]{\\[1ex] {\small{$\lceil$#1$\rfloor$}}\\[1ex]}

\newcommand{\co}[1]{{\sl{#1\/}}}
\newcommand{\wo}[1]{``#1''}
\newcommand{\thi}[1]{`{\sl{#1\/}}'}
\newcommand{\phe}[1]{\thi{#1}} %{{\sf{#1}}}

\newcommand{\quo}[1]{``{\em #1\/}''}
\newcommand{\quot}[1]{\begin{quote} #1\end{quote}}

\newcommand{\MyLPar}{\parsep -.2ex plus.2ex minus.2ex\itemsep\parsep
   \vspace{-\topsep}\vspace{.5ex}}

\renewcommand{\abstract}[1]{{\small{#1}}}

\renewcommand{\baselinestretch}{1.5}

\begin{document}

\title{The \co{Principle of Intentionality}  -- limitiations of the method}
\author{ }

\date{\hfill\small\today}  % August 8, 1996
\maketitle \vspace*{-3ex}
%\vspace*{-5ex}\section{What and Why}
\abstract{Phenomenology, like any method, 
has its limitations. The study of intentional acts and 
objects of consciousness under the guidance of phenomenological 
reductions excludes important aspects of experience. 
More precisely, it does not exclude them in the sense of ignoring
them. 
%is inappropriate for their adequate description. 
As a universal method, it attempts to squeeze them to fit into its 
methodological corset, thus making them appear in a disguised, inadequate form.
Section~\ref{sec:over} gives a brief overview 
of some aspects of the phenomenological method -- with the particular emphasis on the 
principle of intentionality -- which are relevant for the argument following in
section~\ref{sec:limits}.}

\section{A Brief Overview}\label{sec:over}

Husserl's project begins with the same complain as Kant's: 
\quo{I do not say that philosophy is an imperfect sicence; I say 
simply that it is not {\em yet} a science at all, that as science it 
has not {\em yet} begun} (\cite{RS}, my emphasis).
Having seen some hundreds years of the development of 
various sciences, Husserl, unlike Descartes or Kant, did not claim
that science 
provides the absolute, final truth. Its distinguished feature is not so much the 
indubitable certainity of its results, but rather the fact that these 
results are obtained and accumulated under the guidance of a \co{method}. 
In fact, method is the essential feature of science:
science does not need a method -- it {\em is} a method; and 
phenomenology is {\em the} methodology for scientific 
philosophy.\footnote{A very fine balance must be kept here because, after all, the 
initial intention is to gain access to {\em indubitable} knowledge. 
However, since the ideal of such a knowledge cannot, any more, be 
found 
even in science, one is forced to renounce it also in philosophy. 
What is then left, so it seems, is methodology.}
The rigor, precision and certainity of mathematics has fascinated many philosophers.
As Descartes and Kant did, so does Husserl begin his search for 
a methodology with this paradigmatic science. 

\subsection{The Mathematical Ideal}

In mathematics, there is no doubt as 
to \wo{what one is talking about} -- the objects in question are 
given 
fully, or \co{adequately}; their definitions exhaust the given 
content, even if their consequences may remain unclear. When a
mathematician considers \thi{natural numers} or \thi{triangles}, he is 
fully aware of 
what they are -- even if he is not aware (at the same moment, or in 
general) of all the properties they may possess. 

Husserl recognizes that the rigor and certainity
achievable in mathematics are due to the purely \co{intentional} 
character of 
the objects with which it deals. Discussions between platonists, 
intuitionists and formalists 
notwithstanding, the objects of mathematics are \wo{constructed} 
rather than \wo{discovered}, they are results of ingenious 
definitions 
and constructions, and not of a confrontation with some external 
objects and experiences.

The phenomenological method is not a straightforward application of
mathematics (or mathematical method) to philosophy. 
Nevertheless, the aim is to unveil the structures of consciousness
and modes of givenness of its objects which could match
the certainity and precision of mathematical thinking and objects.
% and the precision of mathematical objects.
 
\subsection{Intentionality}
% \subsubsection{The Principle of Intentionality}
The starting point for Husserl is, as it was for Descartes, the 
analysis of \co{cogito}. The conclusions, however, are very different.
Husserl observes that what is indubitably given in \co{cogito}
is not, as Descartes claimed, 
merely the \co{cogito} itself, the fact that \thi{I am thinking}. 
Its object, the \co{cogitatum}, is 
present with equal certainity and necessity, it is just the opposite 
pole of the totality of a 
single act. Cartesian \co{cogito} reveals, according to Husserl, not 
the existence of a substantial subject -- it reveals only itself, 
that is, an act and whatever is present in it.

What is present in this act is not a separate subject and an external 
object. Present is only the act -- a unity of two poles, of the \co{cogito} which is 
conscious of some given \co{cogitatum}. These two poles cannot be 
considered separately.
What appears for consciousness are not some 
external objects, lying outside the \wo{box of \co{cogito}}, but  
\co{intentional objects}, necessary correlates of the acts of 
consciousness.\footnote{This insight resolves the 
worries about \co{things-in-themselves} 
which always evade consciousness. It also dispenses with the mistaken 
image of consciousness as a \quo{closed box} which tries to encompass 
external things -- by means of some internal representations.}
In each act of consciousness, there is not only consciousness itself
but also, with equal necessity, an object. 
The principle of intentionality says:\vspace*{-4ex} 
\thesis{Consciousness is always 
consciousness of \co{something}.\label{th:int}}\vspace*{-6ex}
%
\subsubsection{Transparency of \co{cogitatum}}
Since \co{cogito} necessarily involves \co{cogitatum}, there is 
no barrier separating consciousness from what is given for it.
Objects of consciousness are not external things which, somehow, have to 
be \wo{admitted to the immanency of consciousness}. 
An object is an essential aspect
of consciousness, an indispensable \co{correlate}
of  any conscious act.
Perhaps, it is not \wo{constructed} in exactly the same 
way as a mathematical object is. 
But being an inherent part of the whole act, it is \wo{intended} by consciousness, 
it is not an external, but an \co{intentional object}. As such, it is as fully revealed 
as the \co{cogito} itself. 

The acts of rational reflection make accessible to consciousness the 
\co{cogito} as well as the \co{cogitatum}. And the latter can be 
given with the same degree of adequacy as the former, with the same 
degree of precision as the objects of mathematics are given to a 
mathematician. 
And since this is the general feature of 
consciousness, it applies not only to the objects of perception or of 
mathematics, but to any objects whatsoever of which we can be 
conscious.

\subsubsection{\co{Essential intuitions} and \co{eidetic reduction}}
But, of course, not everything we are aware of appears with such a 
full clarity; some objects emerge only indistinctly, without sharp 
limits and characteristics. Nevertheless, although not all objects are 
immediately given in a fully adequate intuition, the thesis goes, 
they always can be. This is so because objects, being  
\co{intentional objects}, are objects of consciousness and are 
\wo{constituted as correlates of intentional acts}.  

Without going 
into too many details, let us only emphasize the main point: each 
\co{intentional object} has an \co{essence} which can be given in
\co{essential intuition}. It can be found by means 
of \co{eidetic reduction}, or \co{ideation}.  Roughly, \co{ideation} amounts to 
varying the appearance of the object, to positing it in new contexts, 
attempting to see it from various sides. The residuum of a series of 
such variations, what remains after the accidental aspects have been 
changed and replaced by others, reveals the \co{essence} -- the object of 
\co{essential intuition}.\footnote{Obviously, \co{essences} are 
ideal objects, and the \co{essential 
intuition} is very much like the \co{intellectual intuition} so 
definitely denied by Kant.} 

%The following point concerning \co{ideation} should be emphasized. 
One has to keep clearly in mind that \co{ideation} is not a process of 
{\em deriving} \co{essence} from appearances.
\co{Essence} is {\em not derived} from the
appearances -- it can only be {\em seen in} the appearances. In 
particular, \co{ideation} has nothing to do with induction, it is not 
a generalization from a series of appearances -- it is only a method 
of finding the \co{essence} among appearances.
Once an essential insight (or intuition) occurs, there is 
no need for more variations. In a way, \co{essential intuition} puts an 
end to the insecurity as to what the next appearance may bring with 
it -- it establishes an insight which, from now on, rules over the
upredictable richness of the world of appearances.
Phenomenology describes the ways of 
achieving the knowledge of \co{essences}, and \co{essences} given in 
\co{essential intuition} are the primary objects of its 
study.
% ~\footnote{This goes a bit counter the general intention of the 
% phenomenological project. Inspired by the analogy of scientific 
% development, phenomenological investigations are expected to develop 
% (forever?) by gradual verification and accummulation of essential 
% insights. But why should it be so, if essential intuition presents us 
% with {\em the} essence?}

\subsubsection{\co{Epoch\={e}} -- reduction of the irrelevant}

Phenomenology is thus the study of essences of the phenomena which 
appear as intentional objects of consciousness.  And Husserl 
certainly means that everything worth studying falls within this 
domain. There is only one old question which cannot be settled by 
such a study of contents of consciousness, namely, the question 
whether these contents also {\em exist independently} from 
consciousness. But since such an \thi{independent existence} cannot be 
given {\em in} consciousness, the question is deemed irrelevant, and in 
order to stop asking it, Husserl recommands \co{phenomenological 
reduction}, \co{epoch\={e}}.
It amounts precisely to that: \wo{putting the world in 
parentheses} and, instead of worrying about irresolvable (because not 
given as a phenomenon) issue, 
concentrate on the contents accessible to consciousness.

Its justification lies in the conviction that 
everything else, but this question, can be studied phenomenologically, 
that except phenomena, there remains nothing but the question of 
external existence of \co{intentional objects}. 
Consequently, the only thing affected by \co{epoch\={e}} is the \wo{non-phenomenal}
-- what is \wo{outside} consciousness -- that is, the possible existence 
of the 
objects independent from consciousness.\footnote{This is the 
phenomenological way of avoiding the fruitless diputes about the 
existence of external world: the issue is 
simply declared irrelevant.} \co{Epoch\={e}}, leaving all the contents 
unchanged and accessible to consciousness, excludes the question about 
the independent,  non-phenomenal existence from the field of
 phenomenological inquiry. 
% Thus, we can formulate the phenomenological reduction in the following way:
%
\thesis{\co{Epoch\={e}} reduces away everything which is 
non-phenomenal, everything which, not appearing for consciousness,
falls outside the principle of intentionality.\label{th:epint}}
%
Notice that the question whether things {\em are} external is 
different from 
the fact that things {\em appear as being} external. However, 
they appear as external {\em for consciousness} and  the latter 
fact does not imply the former.
This \co{appearing as external} is an intrinsic aspect of 
givenness of many objects and, 
as such, will be observed by the phenomenological analysis. But that 
objects actually {\em are} external is not -- yeah, {\em can not} be 
-- given in the immanency of consciousness, and so is not a phenomenological issue.
% And so this question is 
% excluded from the field of phenomenological intersts.

\section{Limits of Intentionality}\label{sec:limits}
Methodology amounts always to excluding things considered incorrect 
or 
irrelevant. In sciences, this follows typically by some implicit 
assumptions and tacit understanding as to what is justifiable and what is not. 
In methodological discussions, 
however, one always argues about which procedures, observations, even 
facts, are to be allowed and which are to be considered illegitimate. 
Observing this restrictive function, Feyerabend arguing \quo{Against 
Method} suggests that in real scientific development 
\quo{anything goes} -- methodological justification, if it is not to restrict 
the researchers, may come {\em post factum}. This 
kind 
of anarchism may be too extreme for many, but the concession that 
methodology, {\em any} methodology, forbids some ways of thinking or 
arguing, would naturally lead to a dose of scepticism about its value. Does also
the phenomenological method suffer such a drawback? And if 
so, what does it exclude?
This section will attempt to illustrate the thesis that 
\thesis{Intentionality does not govern all forms of 
givenness.\label{th:nointGen}}
In (\ref{th:epint}) phenomenal contents and principle of intentionality were 
mentioned in one line. Indeed, the two are in Husserl, if not equivalent, then 
at least co-extensional: the access to 
any intentional object goes exclusively via phenomena and, on the other hand, 
intentionality is the constitutive feature of consciousness and so any phenomena 
appear only in the context of some intentional object.
However, distinguishing between the two as (\ref{th:nointGen}) suggests, one should 
specify whether \co{epoch\={e}} reduces the non-phenomenal or the non-intentional 
contents. 
%I do not want to enter the speculations about which aspect should be sacrified.
The thesis that
\thesis{\co{epoch\={e}} excludes more than the mere question 
about the external existence\label{th:epex}}
can be proposed under the assumption that, in (\ref{th:epint}), we put 
emphasis on the \co{non-intentional}, 
that which \wo{falls outside the principle of intentionality}. If, instead, we 
emphasize the \co{non-phenomenal}, then we should ignore the claim (\ref{th:epex}). 
In any case, the main thesis (\ref{th:nointGen}) is not affected by this choice.
%
\subsection{Actuality}
The examples of phenomena to be found in Husserl's writings are mostly 
of two kinds: either objects of perception or else mathematical 
entities. In fact, it is relatively easy to follow an analysis of 
what appears in my consciousness when I think, for 
instance, \phe{this table in front of me}. If I try to analyse 
\phe{table as such} (or the \co{essence} of the phenomenon \phe{table}) 
it becomes more problematic but still, this intentional object seems 
to be given with a sufficient degree of clarity. 
And so on, I can start analyzing the phenomena of \phe{the room I am 
sitting in}, \phe{the building I am in}, \phe{the city of Bergen}, 
\phe{my world}, \phe{the world}, etc.

Here, however, there seems to arise an important distinction. What is 
the essence of \phe{the city of Bergen}, or even worse, of 
\phe{my world}, supposed to be? Surely, something appears in my 
consciousness when I think these things, but what is it? I may still 
have some vague ideas about \phe{the city of Bergen}, but when it 
comes to \phe{my world} I can only say that it is \quo{something but I 
do not really have a clue what}. Well, it is \co{something} -- like 
anything of which I am conscious -- and in order to find what it is, 
Husserl would suggest \co{eidetic reduction}. But how long am I supposed 
to keep varying and reducing? For the rest of my life, I guess, 
because what my world was yeasterday does not apply today, and what 
it is going to be tomorrow, only tomorrow can tell. Yet \phe{my world} 
yeasterday was as much my world as \phe{my world} today is, and as 
\phe{my world} tomorrow is going to be -- it is the same \phe{my 
world}.

The distinction I have in mind is between the objects (phenomena) 
which naturally fit within \co{the horizon of actuality} 
of consciousness and those which do not. Consciousness is actual 
through and through, it is, perhaps, the \co{actuality pure and 
simle}.\footnote{With all the \co{protentions} and \co{retentions}, for 
sure, but still a unit of actuality which is centered around an
\co{intentional object}.}
Whenever I am conscious (of something) it happens here-and-now, 
within a narrow (if at all extended) \co{horizon of actuality}. 
Concepts, or objects of perception fit well into it, but \phe{my world} does not.

Take another example. I meet a wise man, that is, the phenomenon 
\phe{this wise man} appears in my consciousness. I can reflect over 
this appearance and analyse its contents, since it is a particular 
phenomenon appearing here-and-now. But so I pass to another phenomenon: 
\phe{wisdom}. And...? What do I find in my consciousness? Probably 
just a bunch of associations which are immediately deemed 
\wo{psychological}. %The suggested ideation would, i
In common terms, \co{ideation} would 
amount to \wo{trying to bring some order} into these 
associations. But to hit the \co{essence} of \phe{wisdom} I have to be 
really lucky -- wise people have been trying for thousands of years. 
(Ok, they did not know about phenomenology, but I still have a 
feeling that I have to be really lucky. And even if I am, then how 
many other lucky people are there who disagree?)
%
%% \thesis{There are phenomena which cannot be fully present as \co{intentional objects}
%% within the \co{horizon of actuality}.\label{th:nonact}}
%

That \co{intentional objects} exceed the \co{horizon of actuality} and appear
differently in various phenomena was obvious to Husserl. 
But the point here is that 
\thesis{There are phenomena which are 
intrinsically elusive, whose \co{essence} cannot be given adequately as 
an \co{intentional object}.\label{th:noint}}  
% 
These objects 
% whose supposed \co{essence} evades the actual intuition and
can be called \wo{transcendent} -- not in any 
sense of transcendence envisaged by Husserl, but precisely in the sense 
of (\ref{th:noint}): not only they themselves, but also their supposed 
\co{essences} are \wo{too rich} to be adequately given as \co{intentional 
objects} within the scope of a single act. In particular, they never appear 
as sharp, 
well-defined objects with a clear \co{site of identity} -- their 
essence is never given adequately in \co{essential intuition}. On the 
contrary, things like \phe{wisdom}, \phe{world}, \phe{love}, \phe{evil} have no 
\wo{sameness}\footnote{Called by Levinas \wo{ipseity}.}
and are never exhausted in intuition or thinking. Unlike the
\co{intentional objects} of \co{essential intuition} they are always \wo{too rich}
and \wo{diffuse}. 
They are more like \wo{fields} -- some overlapping, some disjoint -- 
of inexhaustible possibilities of ever new manifestations.
Consequently, the supposedly \wo{absolute being of consciousness} has 
no power to grasp their essence for, if they have any, it transcends 
the actuality to which consciousness is confined. Instead of being 
grasped, they rather announce their presence by invading 
consciousness, by embracing it and taking it into their possession.

% Intentionality excludes \wo{richness}.

\subsection{Non-intentional phenomena}

True, there isn't much more for a (non-dogmatic) philosopher to 
analyze than the contents of his consciousness. But it does not 
follow that all such contents are necessarily given as 
definite objects.

Husserl did analyse the phenomena like \phe{Lebenswelt} applying his method 
to them as to all other phenomena. 
But so did Heidegger and Marcel, so did Scheler and Sartre. Nobody 
will claim that they arrived at even remotely similar conclusions. 
Husserl would say that one genuine essential intuition cannot 
contradict another, so, most probably, some of them did not arrive at 
such an intuition.

Trying to decide who of them did and who did not, 
% apply the phenomenological method correctly, 
would probably lead only to more 
disagreement of opinions. Assuming, on the other hand, that they all 
did  (and why shouldn't we?), we can quote the well known observation:
in the analysis of \wo{more general phenomena}, phenomenology 
does not at all lead to an agreement between different 
thinkers.
The \wo{more general phenomena} are things like \phe{the world}, 
\phe{wisdom}, \phe{my life}, \phe{evil}, etc., referred to in (\ref{th:noint}). 
They, too, appear for 
consciousness but lack this definiteness and sameness which characterizes the 
\co{intentional objects}. So let me call them \wo{non-intentional phenomena}. 
% and refine (\ref{th:nonact})
%
% \thesis{There are phenomena which are 
% intrinsically elusive, whose \co{essence} cannot be given adequately as 
% an \co{intentional object}.\label{th:noint}}  

A peculiar thing about these \co{non-intentional phenomena}, namely that 
they lead to a variety of 
opinions rather than a unified, essential view, reminds of another 
observation. One can hardly express an opinion in some 
general matter without, at the same time, revealing something of 
oneself. Expressing my views about \thi{the character of the world}, \thi{the quality of 
love}, \thi{the source of evil}, I am not merely stating some facts about the 
objects -- first of all, I am telling who I am.\footnote{This does not 
necessarily mean that such views are purely subjective, but it is another story.}

\subsection{Subjectivistic illusion}

Yet, as said before, Husserl claimed to have performed 
phenomenological analysis of some such phenomena (e.g., \phe{Lebenswelt}), 
and so seem others 
to have done. (\ref{th:noint}) claims that not everything in the 
concrete life of consciousness appears as sharp, \co{intentional objetcs}. 
But it does not exclude the possibility of transforming a \co{non-intentional phenomenon}
into such an object and, subsequently, of grasping it within the actuality of 
consciousness.
The \co{non-intentional phenomena} exceed the \co{horizon of actuality} 
and appear only in \wo{diffuse}, ever new and unclear
intuitions. Nevertheless, they can be stripped of their unpredictability 
and otherness, and can be constituted as \co{intentional objects}, that 
is, forced to appear within the \co{horizon of actuality}. 

\subsubsection{Concepts}
When I experience love, I know that its concreteness exceeds not only the powers 
of my understanding but also of my being -- \wo{it is greater than 
me}. But, certainly, I can try to reflect over it, to determine some  
features of this experience, to characterize it, in short, to entertain 
a \thi{concept of love}. This is reminiscent of 
\co{eidetic reduction}, with the reservation that there the objective is not a 
concept but \co{essential intuition}. The effect, however, seems to 
be very much the same: the original experience disappears and, 
instead, one is confronted with a comprised, reduced sign, an 
\co{intentional object} -- {\em a concept} -- which can be grasped in a single act.
 
Concepts (or, if you still insist, \co{essential intuitions}) are 
means by which phenomena exceeding the \co{horizon of actuality} can be, at 
least partially, made present within it. 
Heidegger's analyses of the phenomenon of \phe{the world} or 
Sartre's analyses of \phe{love}
are nothing more than analyses of {\em what they understand by} (see, 
associate with) the words \wo{world}, \wo{love}. The result is either a 
conceptual nettwork of interrelations, or else a mood and an underlying 
feeling of hesitation (in the former case) and disgust (in the 
latter). And, above all, the feeling that these analyses point towards something
which we can experience, but that no matter how much has been 
said about it, there is always more to be said.

\subsubsection{Reflective vs. immediate consciousness}
It is important to keep in mind that phenomenological analysis is a 
process of \co{reflective consciousness}.\footnote{\wo{Reflective} 
consciousness is by no means the same as \wo{reflexive} one. It is 
highly unclear (at least to me) what the latter could be.} 
I sit down and decide to 
analyze, that is, reflect over, the experience of this or that. This 
decision obviously implies directing my attention at the intended 
object, i.e., it constitutes the intentional object -- {\em of 
reflection\/}! And certainly, reflection may focus on the object, 
on my reaction to it, on the mode of its givenness. All the 
characteristics ascribed by Husserl to consciousness -- in particular 
the directedness of intention -- seem to apply 
without restrictions to \co{reflective consciousness}. The theses of 
this section suggest that these characteristics can not be 
transferred to the \co{immediate consciousness}, or else to our 
pre-reflective being, without falsifying it.

The \wo{subjectivistic illusion} from the title of this subsection 
indicates the effect of such a transfer: 
everything {\em can be made} into a definite, intentional object and 
studied as such. But this 
does not mean that everything is {\em originally given} as such an object.
The subjectivistic illusion consists in mistaking the possibility of 
treating everyting as a definite object for the actuality of everything being 
given as such an object.
Treating the \co{non-intentional phenomena} in this way confuses their nature 
and diminishes the significance they have in our life.

\section{Conclusion}
Phenomenology did not become the universal method of 
scientifc philosophy. Instead, ironically enough (and for Husserl this 
would certainly be a very bitter irony), the main use it was put to was to 
admit existentialism into the domain of respectable philosophical 
discourse. 

Much of the attractive power of phenomenology -- and certainly, much 
of the reason that existentially oriented philosophers found it so 
attractive -- consists in that it 
does not exclude anything from the field of potential investigation. 
On the contrary, anything can (must?) be approached via the way in which it 
appears for the consciousness. In particular, it does not dismiss individual
appearances -- it provides the means for analyzing concrete,
individual phenomena, thus bringing a philosopher back to the world of \wo{concrete
experience}.

But the seductive power of the 
method consists in the smooth transition between, in fact, the equivocation 
of the \co{reflective} and \co{immediate} 
consciousness. The former is always directed at some specific (even if 
not clearly recognized) \co{intentional object} -- the object of actual 
reflection. But, as I tried to argue, such objects do not exhaust 
% are not always to be found in 
the immediacy of consciousness which is always haunted 
by the \co{non-intentional phenomena}.

The limitation of the method lies in that it is designed for the 
study of \co{essences}, in particular, that it prescribes to view everything
as \co{intentional objects} appearing within the \co{horizon of actuality}. 
Consequently, the \co{non-intentional phenomena} eluding such a mode of 
appearance, must be 
squeezed to fit within this horizon -- a procedure which necessarily
reduces them to something they are not. For the purpose of analysis this is, 
probably, unavoidable. It is also fine --
%This may be fine (probably, is even unavoidable)
but only as long as one is aware of this reduction and makes it explicit.
In the opposite case, one will be constantly perplexed by the fact that
 one's \co{essential intuitions}
and insights, \wo{revealing} the \co{essence} of \phe{the world}, 
\phe{death}, \phe{love}, etc., strangely enough, %all the time
miss the supposed \co{essence} and are not so exact and definitive as 
they are supposed to be.

\begin{thebibliography}{SU}
\bibitem[1]{RS} E.~Husserl, ``Philosophie als strenge Wissenschaft'', {\em Logos}, I, 
1910-11, 289-341. %{\em Philosophy as Rigorous Science},
\end{thebibliography}

\end{document}
