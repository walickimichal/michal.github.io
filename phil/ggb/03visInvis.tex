\footnotetext{\btit{German Sermons}, Ac. XII:11
  [\citeauthor*{LW} I, pp.48-57; \citeauthor*{Eckhart} 3, pp.244-246]}
%
\pa We live among things which we control and use for our purposes, among things
and institutions built to perform definite functions. But we live surrounded by
things which are \thi{greater than us}, things which, although sometimes 
 can be, to some extend, modeled successfully as \co{complex totalities}, 
cannot be embraced completely by a network of \co{concepts} and rules.  These
\thi{things}, 
slipping out of our grasp and control are not, however, outside our reach, they
are not inaccessible infinities \thi{in themselves}.  They too are
\co{distinctions} which have been made after we were \co{born} and are part of
our \co{experience}. They are announced by various \co{signs} and by all the
\co{traces} which permeate every \co{actuality}. They are \co{present}, albeit
never as entirely \co{actual}, \co{precise} and fully transparent \co{objects} of
\co{reflection}.

\citet{Know that the knowable things are of two kinds. Some can be described by
  means of definitions, while others can not be defined.}{Ibn}{III:\para 25\label{cit:Ibn}} In
the most mundane sense, the \co{invisibles} are what \wo{can not be defined},
what can not be encircled within the \hoa.  Speaking of love as the paramount
example of such an undefinable experience, Ibn`Arabi continues \wo{It is
  known by him in whom it lives and whose object it becomes, while the person is
  unable to understand its nature nor negate its reality.}$^{\ref{cit:Ibn}}$ Inability to
grasp \thi{what} does not exclude perfect awareness of \co{that}. In the
eventual limit, \co{invisibles} dissolve in the mere \co{that} of the
\co{origin}, leaving all \thi{whats} to the finitude of understanding.  Even if
consciousness notices some \co{invisibles}, it has completely different
character from the thematic, positional consciousness of a \co{visible} content.
\co{Invisibles} present themselves always as essentially \co{transcending} the
\co{actual consciousness}, as inexhaustible by it, as touching the
edge of eternity.

\pa
\citet{Now the created soul of man hath [...] two eyes. The one
  is the power of seeing into eternity, the other of seeing into time and the
  creatures, of perceiving how they differ from each other [...], of giving life
  and needful things to the body, and ordering and governing it for the
  best.}{TheolGerm}{VII. There are multiple examples of apparently similar
  duality, as e.g.: \citef{I am a member of two orders: the one purely
    spiritual, in which I rule by my will alone; the other sensuous, in which I
    operate by my deed.}{FichteVocation}{III;p.140} In our terms, these two orders
 are both within the sphere of \co{visibility}. Similarly, Hugh of St.~Victor
  distinguished the \thi{eye of the flesh}, by which we perceive external world
  and the \thi{eye of the reason}, by which we attain knowledge of
  ourselves. But with him both were, in turn, distinct 
  from the \thi{eye of contemplation}, which allows us to achieve knowledge of
  things \co{above} us. 
%   Moreover, being a borderline implies that neither of the orders is
%   \thi{purely} this or that. There is no \thi{purely spiritual} order, for there
%   is no \co{spirit} which is not \co{incarnated} in the flesh. This Book will
%   clarify such issues.
}
%
An individual is a borderline between \co{visible} and \co{invisible} -- this is
the existential situation. It is irrevocable and does not depend on anything
particular, on the level of his understanding, on the scope of his knowledge, on
his character or life experience.  Yet it is not, for this reason, formal, it is
not related to the concrete qualities of different lives as form is (supposed to
relate) to matter.  It is, in fact, \co{experienced} through and through,
encircling the horizon within which this particular life, as well as every other
life, unfolds. The \co{invisibles present} in every \co{actuality} are what give
it its full, lived \co{concreteness}.  All concrete \co{experiences} are woven
into the interplay of the \LL\ and \HH.  No matter to which level one directs
\co{reflection}, no matter what are the predominant feelings and inspirations of
one's life, this life's fundamental character is determined by the experience
and attitude toward the sphere of \HH\ and \LL, by the way in which one
experiences and moulds the borderline between them.

%\pa
%[In this sense it is absolute -- but is our relation to the Absolute the same
%as our relation to this situation, i.e., to the fact that we are between \HH\ and \LL?]
%
%Absolute -- always present, but never identical with anything (imm. 
%\& transc.)
%

This is the {\em existential} situation.  The \HH\ is not something which is
merely \wo{not known}.  Knowledge, especially in the philosophical tradition,
might have functioned as an image of this situation, where reason was placed
between the \thi{known} and \thi{unknown}.  But it is a poor analogy.
The distinction \thi{known}-\thi{unknown} places us at the level of
\co{actuality}, it carries the character of a contradiction, of an absolute
opposition and, moreover, of \co{objectivistic attitude} -- the \thi{unknown} is
either irrelevant emptiness or a determinate \thi{knowable}, even if for the
time being it remains unknown. 
%
For a knowledge oriented person, it seems important what he knows and, possibly,
what he does not know but would like to know. This positivistic element veils
completely (though not necessarily) the fact which is important: that no matter
how much and what one knows, it is always limited and surrounded by the
immovable horizon of the \thi{unknown}. This fact, trivial as it is, has the
fundamental existential import.

Being a borderline is a \co{concrete} expression of \co{existential
  confrontation}. It is not merely formal but thoroughly \co{concrete} because
man, transcending himself, lives only his own limits.  The \co{confrontation}
itself is never \co{actually experienced}, is never \thi{given} as an
\co{object} of adequate understanding; it manifests itself as the constant
\co{presence} of something \co{above}, witnessed to by the occasional \co{signs}
and by the constant \co{traces} connecting what is \co{actually visible} to its
\co{invisible} origins. Being a borderline is a \co{concrete} expression of
\co{existence} as \co{participation} -- perhaps not yet a \co{concrete
  participation}, which consists in {\em actually being} on the \co{trace}, but
at first only the simple \co{participation} of the {\em mere intimation} of the
\co{trace}, the {\em mere intimation} of being surrounded by \co{invisibles}
which remain \co{dissociated} from their \co{signs}, remote and not \co{present
  concretely}. The impossibility of grasping this \co{presence} can suggest that
one should turn away from it and concentrate on the plain \co{visibility}, thus
securing that one's being will not be devoured by the surrounding \co{chaos}.
But remoteness, the \co{distance}, is reflected as a constant \co{thirst} -- and
when pushed to the extreme \co{dissociation}, even as despair -- of the
\co{soul}, for \citet{the desire for the bliss, which she had lost, remained
  with her even after the
  Fall.}{Periphyseon}{IV:777C-D\kilde{88}\label{ftnt:afterFall}} But attracted
thus by the inexpressible promises and \co{vague} reminiscences, one can end up
in the abyss. Attempting to bring them into \co{this world}, to explicate their
intricacies, to grasp them in the language and concepts, one risks dissolving
all \co{actuality}, concepts and \co{signs} in the incomprehensible mesh which
constantly tries and never manages to capture the \co{vague clarity} of the
\co{experienced} intimations. Neither attitude -- that of the apparently safe
dismissal and sobriety, nor that of the gnostic fascination and ecstasy -- seems
satisfactory, not to say, complete. But being a borderline, we balance between
the two. Abstract as this balancing might seem, it is the fundamental, the most
\co{concrete} border we draw in our \co{existence}.
%\vspace*{-1ex}
%\section{The existential situation}
% \secQQQ{7}{Because the soul has the potentiality of knowing all things, it 
% never rests until it comes to the first image where all things are 
% one. There it rests, there in God.} 
% %In God no creature is nobler than another.}
% {Eckhart\ftnt{\btit{German Sermons}, Ac. XII:11 [\citeauthor*{LW} I, pp.48-57;
%   \citeauthor*{Eckhart} 3, pp.244-246]}}
%


\section{Thirst}%\vspace*{-1ex}

\pa\label{pa:thirst} Young people look hopefully into the future which is like
a huge promise of the whole world -- and thirst for its coming. Adults keep
putting the last 
brick on the construction they have been rising all the time, the last detail
needed to complete the perfect totality -- and having put it, find another one
which needs to be put, but only one, the last one, and then another\ldots Or else,
they throw dim looks far away, to the places of the next visit on the other end
of the earth, to the remote, half real places they can hardly imagine, like the
magical atmosphere of the childhood home.  Old people thirst only for the thirst
of their youth; or, perhaps, liberated from the past as much as from the future,
for the tranquil withdrawal from the
actuality.  The poor thirst for the easiness of affluence and the rich for the
resistance of the world and matter without which reality seems to 
dissolve in decadence. As Goethe says, \woo{man thirsts constantly for what he
  is not.}{p.21} We are never entirely satisfied with our achievements,
and we are never entirely satisfied with whatever we obtain from gratuitous
generosity. And if one rests satisfied, when one stops thirsting, it is said
that he has lost the taste for life.

\citeti{We are as if we were not.}{Heraclitus}{DK 22B49a
  \noo{\citaft{Vorsokr}{}\citaft{Powrot}{p.42}}} There is something which enters
our \co{experience} only as \co{thirst}; it is not longing for anything
specific, even if in most situations one will fill the objectless character of
this \co{thirst} with something graspable, will give it a name, and hence a
goal. But it is not longing for\ldots, not a thirst for\ldots  -- it is simply
\co{thirst}, without any object, without any goal. True goals remain hidden
until they are reached.

\pa \co{Thirst} does not show anything \co{visible}; if we were to talk about
its correlate, its noematic intention, it could only be \co{nothingness};
\co{nothing} appearing through the entirely negative noesis, the experience of
lack, perhaps even a loss. Yet, this seemingly negative noesis, this apparent
\co{nothingness}, has a thoroughly positive character. For \co{thirst} announces
something which -- by the very fact of being \co{thirsted} for -- represents
some good, and -- by being \co{thirsted} for indefinitely and undefinably,
incessantly and indelibly -- perhaps something \co{absolutely} good. It might
seem that what is so announced remains \thi{absent} and that the whole
announcement amounts to nothing but announcing its \thi{absence}.  The negative
experiential content, \thi{absence}, may indeed seem to be not merely an
analogical \la{modus significandi}, but strangely inadequate way of presenting
the intended, fully positive \la{res significata}.  \citeti{Not understanding,
  although they have heard, they are like deaf. The proverb bears witness to
  them: <<Present yet absent.>>}{Heraclitus}{DK 22B34} For all that appears
negative and inadequate only when we expect things, \co{objects},
\thi{substances}, \co{visible} images, in short, \co{actualities}. For only then
we consider \co{non-actuality} to be an \thi{absence}, a lack. This
\thi{absence}, however, is a thoroughly positive \co{presence}, and \co{thirst},
this sense of incompleteness, is the genuine bridge over the borderline
separating \co{this} and \co{another world}.  \citet{That our good is There is
  shown by the very love inborn with the soul; [...] the soul, other than God
  but sprung from Him, must needs love.}{Plotinus}{VI:9.9\noo{Of course, there
    is no need for any \co{actual} love, for any \co{acts}. This, as for
    instance Duns Scotus called it, \wo{natural love} is the ontological
    relation of perfectability that exists between the \co{actual existence} and
    that which perfects it.}} And thus \thi{remembering}, one keeps looking.
\citet{Sometimes I feel as if I were approaching happiness and I stand before a
  flower that blossoms through an old stone wall and I am unable to draw nearer.
  I am left with the feeling of always waiting for happiness, and everything is
  suddenly diminished by the melancholy of having once being able to achieve
  that joy.}{LouR}{Rilke's letter from 16.03.1914, p.66} This remembrance of
apparent loss, this \co{thirst}, \thi{natural love} and desire of
\thi{Something}, is the first form of \co{invisible presence} in \co{actual
  experience}.  Before elaborating this point, a few \co{impressions}.

\noo{
\pa There are no guarantees.  Not only the goals remain hidden, but
\co{reflection}, being a \co{concrete reflection} of this particular
person, has no unambiguous and \co{precise} \co{signs} it could
follow.  The \co{invisibles} are not prone to such \co{signs} and,
indeed, every person may need to travel a different way, must travel
his own way.  The only possible guidelines are as \co{vague} -- and
\co{clear} -- as the \co{invisibles} themselves.  \co{Invisibles} do
not \co{command} a determinate course of action, but man has to direct
his attention towards \co{invisible} to bring heaven and earth
together.
}%\vspace*{-1ex}

\subsubsection{The moods of silence}%\vspace*{-1ex}

\citet{Silence is a fence around
  wisdom.}{MaimoTraits}{\citaft{EvilFaces}{p.72}\noo{The Many Faces of Evil,
    p.72}\label{cit:fence}} Nowhere, nowhere and never happens more than in a
moment of silence -- for silence is the voice of God. \citet{I am the taste of
  living waters and the light of the sun and the moon [...], sound in silence
  [...]/ I am the silence of hidden mysteries; and I am the knowledge of those
  who know.}{Bhagavad}{VII:8/X:38} But used to the voices, we can't hear the
silence and searching for \co{visible signs}, we keep \co{thirsting} for what
\co{transcends} them.

\begin{verse}{\small{\em
Then Theotormon broke his silence, and he answered:

"Tell me what is the night or day to one o'erflowd with woe?\\
Tell me what is a thought? \& of what substance is it made?\\
Tell me what is a joy? \& in what gardens do joys grow?\\
And in what rivers swim the sorrows? and upon what mountains\\
Wave shadows of discontent? and in what houses dwell the wretched\\
Drunken with woe, forgotten, and shut up from cold despair?

Tell me where dwell the thoughts, forgotten till thou call them 
forth?\\
Tell me where dwell the joys of old! \& where the ancient loves?\\
And when will they renew again \& the night of oblivion past?\\
That I might traverse times \& spaces far remote and bring\\
Comforts into a present sorrow and a night of pain.\\
Where goest thou, O thought? to what remote land is thy flight?\\
If thou returnest to the present moment of affliction\\
Wilt thou bring comforts on thy wings and dews and honey and balm,\\
Or poison from the desert wilds, from the eyes of the 
envier?"\ftnt{\citeauthor*{BlakeAlbion}. 
In Blake's poetry, Theotormon represents thirst, 
desire which, when suppressed and restrained, turns into envy or greed.}
}}
\end{verse}

%\input{031MoodsOfSilence}

%lasting pain
\pa Have you ever felt the constancy of a pain, vague and indefinite
or, perhaps, clean-cut and if not with a known source, then at least
with a clearly recognisable target, pain which did not leave any space
for hope, whose intensity was spread over the soul or rooted in the
body so that no point was adequate to begin recovery in which you
could trust?  Have you? Pain which might have lasted for years so 
that, eventually, it became a companion, almost a friend, on whom you 
could rely, who you could be sure will visit you again, but whom
you never wished to meet directly, whom you always tried 
to avoid, pretending that you are not at home, whenever the doorbell 
rang.


\noo{
\citt{A woman passed by the window yesterday morning and when the 
last glimpse of her died out in my eyes, she had taken almost all of 
my tenderness away with her.

I would like to describe this in detail because I believe it is one of 
the few vital experiences possible to a [terminally] sick man or a 
prisoner [for life].

(May it not be at least conjectured that the sick and the imprisoned 
in our midst form the top-soil out of which grows all that is 
beautiful and desirable in the backyard of this ugly mill?)

I saw the faint outline of her long-distance-veiled face, but the face 
itself -- the thing one might live and dream with -- I did not catch a 
glimpse of. A long brown cloak which fell carelessly from her 
shoulders all but covered her tall, lingering form. It was the most 
desperate guess-work to imagine her breasts, her waistline, her tights 
and her feet.

Nowhere, nohow could I get even the echo of a hint of what her eyes 
might be like. 

As she moved into my vision she became everything I had of hope. 
While she remained in my eyes she was all I knew of beauty. I was 
outside the pale sweetness when she finally faded out of the landscape.

I once thought that a bird which passed this same window might be God
on an inspection tour.}{The last sentence could have been from St. 
Francis, couldn't it?  He would have stopped there.  But here, we have
one more sentence, which breaks the poetic spell and brings the 
intellectual back: \wo{But what God
ever brought as much into a human life as did this anonymous woman?} 
It is, after all, Nietzsche, {\em My Sister and I}, VI:1. The 
insertions in [\ldots] are mine.}
}

%emptiness
\pa
Have you ever met a dark moment of dark thoughts, in the middle of 
a restless night of despair? The emptiness of crowded streets, unreal 
cities, wastelands? Have you ever been at the outermost cliffs,
%of loneliness, 
far from Dover, not peaceful coast of sunny Californian 
Pacific, but remote and desolate, stony beaches of Faroe Islands, 
empty, like mathematical line curved in the frozen magma of an 
Icelandic mountain, in the steepness of a Norwegian fiord, stone under ice, blown 
with the wind, and waves, not singing, not enchanting, but dividing and
divided, those which stay, and those which return\ldots 
unceasingly, without purpose\ldots

\pa
%longing
Have you ever felt the emptiness of infinite longing, the emptiness which filled
the whole world, the whole life, the whole universe with the unbearable beauty
of its silence.  Have you?  The emptiness which does not negate the things of
this world but which presses its presence between them, which surrounds them
with full respect and recognition, and yet\ldots makes them all appear
insignificant, that is, disappear.  The emptiness which is not void, which is
not absence but, on the contrary, the fact, the feeling, the presence which
meets the longing at the horizon where the ineffable dawns, and thus, where it
languishes in calm.  The emptiness which is all and only correlate of this
longing, because you know that you are not having a yen for this or that, for a
better house, nor for a nicer company, for more success, nor for more happiness.
Such things, real and satisfying as they might be, would not suffice.  You are
not longing for anything, and yet you are longing\ldots This is not a longing
for the impossible, rather, an impossible longing, often, aroused by a minor
thing, an inspiringly resigned tune of Celtic quietude, Irish flute, by a stormy
breeze on an empty beach boulevard among the faded tables and withered benches,
by the light of the moon diminishing in the dark waters of an evening lake, by a
passing woman reminding you of the impossibility of Love.  But these are only
signs, impressions, psychological reminders.

\pa
%dreams
And all the good dreams, understood or not, dreams of foreign lands, of
remote islands, of shiny future, believed and unbelievable. Day-dreams of the
ultimate fulfillments, hardly admitted and only vaguely felt, with
unrecognisable contents though recurring moods, arising in a morning from the
ashes left by the nightmares of their failures.  All the good, beautiful dreams,
the more precise the less possible, never matched by reality and yet constant
and unshaken, impossible to retain, impossible to forget.

Dreamer, dreamer, what do you dream of?\noo{better with `for'!}
%Whereto do your dreams lead?

%restlessness
\pa Have you ever felt the restlessness
of soul which, although apparently should be happy and has no reasons
for dissatisfaction, does not find calm and rest in any of its
achievements, in any of the joys and pleasanties it has encountered? 
Have you?  Have you ever felt that everything is in perfect order, so
that it hardly could be better or neater, that you have everything
you wanted and yet, something is missing, that you have nothing to
complain about and yet \ldots

And if you felt it, haven't you then tried first to ignore it, and then either
to find something which \co{actually} was missing to fill the gap, or else to do
something which could occupy your mind and your hands, which could at least
serve some useful, even if tiny, purpose?

\pa Under all the attempts to think that the meaning is more specific, that
there was a goal, that all is about something more definite, there hides
unquenchable \co{thirst}.  Awaiting new things, important events, the most
significant solutions, we flirt with time, yearning for eternity. And the deeper
we yearn, the more intensely we flirt. It seems that a \citet{mere trifle
  consoles us, for a mere trifle distresses us.}{Pensees}{II:136} But any moment
devoid of the hope of eternity becomes a desperate expectation of the next
moment. One may reach the point where mere novelty, multiplication and
\wo{increase are the highest and only goal}, although even the preachers of
blind differentiation\noo{Dewey's or Rorty's kind} know that
this is usually called \wo{cancer}.  The \co{thirst} is not for this or that,
and so it can not be quenched by anything, least of all, by \co{more} of
anything.  It may turn into incessant and restless search, into constant
attempts to acquire \co{more} or experience something new, but \wo{avarice is
  serving the idols}. The \co{more} ends up in a stupefying perplexity, like the
oversensitivity of an autist leads to a shutdown. The \co{more}, the less\ldots
%avaritia
Any attempt to quiet \co{thirst} with this or that will only make its \co{presence} more
intense.  The multitude of distractions may help to survive a day, a week, a
year, but it only breeds more \co{thirst}. In fact, \co{thirst} becomes the
stronger, the weaker any feeling of its presence, the less \co{visible signs}
announcing it.
%Eriguena ?
All its \co{signs} tend to get hidden under \co{more} and \co{more}
goals, activities, experiences, novelties. But hiding the \co{signs} does not
help against \co{thirst} which now starts emanating from that which intended,
and initially even managed, to obscure it.  \citet{More!  More!  is the cry of a
  mistaken soul, less than All cannot satisfy Man.}{BlakeNoNatRel}{II}
%\vspace*{-2ex}
%\newpage
\subsection{The hermeneutics of thirst}\label{sub:spiral}
%\secQQQ{5}{To despair is to lose the eternal}{S.~Kierkegaard\footnotemark{}}
\subsub{Search}\label{sub:moloch}
%\footnotetext{\btit{The sickness unto death} I:C.B.b.$\alpha$.1\kilde{p.82}}  
%{Search -- Approaching [Moloch]}
The soul \wo{never rests until it comes to the first image\ldots} says
Eckhart\ftnt{Also Augustine's opening remark that \citef{our hearts are restless
    till they find rest in Thee}{AugustConf}{I:1} is a more psychologically
  appealing variant of the Biblical (and rather moralistically sounding)
  \citefi{For what shall it profit a man, if he shall gain the whole world, and
    lose his own soul?}{Mrk.}{VIII:36; Mt. XVIII:26} \citef{The spiritual sense,
  the instinct for the real, is not satisfied with anything less than the
  absolute and the eternal.}{IdealistView}{III:2}}, and this lack of rest may
easily become a search -- not any longer a search for this or that, but a search
for \ldots God? Yet \citet{there is truly no searching for God, for there is
  \co{nothing} where one could find him.}{IchDu}{III.\noo{es gibt in Wahrheit
    kein Gott-Suchen, weil es nichts gibt, wo man ihn finden
    k\"{o}nnte.}\noo{[my translation] p.81}} The search is but the recognition
of the lack, only the indefinite and \co{clear} \co{thirst}.  Young, adult or
old, successful or damned, nobody lives without this \co{thirst} and nobody can
live without it. But real is exactly that without which one can not live.  The
\co{thirst} announces the \co{presence} of the most real aspect of our whole
\co{experience}, but accepting it may be even more difficult than realizing it.

\pa %idols
\co{Thirst} is already a search, but it does not know for what.  Of course, we
\citet{all desire happiness with one will,}{DeTrinitate}{XIII:4.\para 7.
  [\wo{the will to obtain and retain blessedness is one in all}] } we are looking
for \la{vita beata}, for the \thi{highest good}, for \thi{paradise}\ldots But
what does all that mean? -- specifically, concretely, precisely?  Indeed,
\co{nothing} in particular.
\citet{You play and work and meditate.\lin But still your mind desires\lin That
  which is beyond everything,\lin Where all desires vanish.}{Ash}{XVI:2} 
So yearning for eternity we flirt with time, unable
to find \co{invisibles} we keep looking for this or that, and end up with mere
\co{signs}, for only what is \co{visible} can be searched for, found and
possessed.  We replace \co{thirst} with thirst for Being, then thirst for Being
with thirst for truth, thirst for truth with thirst for understanding, thirst
for understanding with thirst for recognition; we want to think that some form
of paradise on earth is possible, and end up constructing totalitarian
monstrosities; we recognise the ever present, the \thi{unavowed theologeme} and
end up mixing faith, messianicity with \thi{democracy to come} and other
socio-political fantasies, which criticize such earlier fantasies only by
turning in the opposite direction along {\em the same} line\ldots An \co{idol}
is a finite, relative thing made absolute, a \co{visible} thing used to suppress
the \co{thirst} -- as it often may seem, to quench it.  \citet{What idol
  actually attempts to erase is the remoteness, the distance separating us from
  divinity\ldots Filling this gap, the idol presses itself on us as divinity,
  confirms it and eventually degenerates.}{MarionChas}{[after
  \citeauthor*{Chasydzi}, p.106]} In search of paradise, we find \co{idols};
\co{thirsting} to the woods, we raise cities, and to convince ourselves that
this is enough we \co{idolise} them the more, the less calm they bring and the
stronger our suspicion of their insufficiency.

\pa %paradise searched for = pride
A \co{vague} sense of
some loss, the loss of something we do not know precisely what is, something
like~\ldots paradise, some happy state, a natural dwelling place 
% an \gre{ethos} in the sense reintroduced by Heidegger, to which we feel entitled
-- the sense of such a loss is a form of \co{thirst}, too.  But we are not
supposed to lose, and even if we do, we lose only what is ours.  So \citet{who
  can yet believe, though after loss,\lin That all these puissant legions, whose
  exile\lin Hath emptied heaven, shall fail to re-ascend,\lin Self-raised, and repossess
  their native seat?}{Milton}{I:631-634 [spoken by Satan]} Humans deserve~\ldots
well, what? It is not quite clear, but no matter what \wo{their native seat}
might be, an indication of any metaphysical \thi{deserving} or \thi{entitlement}
will not, eventually, stop before the highest unimaginable -- \thi{paradise},
\thi{happiness}, \thi{salvation}\ldots  Entitled to repossession of the lost seat,
\citet{[l]et us disdain things of earth, hold as little worth even the astral
  orders and, putting behind us all the things of this world, hasten to that
  court beyond the world, closest to the most exalted Godhead.  There, as the
  sacred mysteries tell us, the Seraphim, Cherubim and Thrones occupy the first
  places; but, unable to yield to them, and impatient of any second place, let
  us emulate their dignity and glory.  And, if we will it, we shall be inferior
  to them in nothing.}{Pico}{} Although one might emphasize the calls to
transcend the merely human conditions present in \btit{The Oration}, its tone is
that of inspired Cabala, or in more ordinary terms, of the unrestricted 
\thi{humanistic} optimism -- entitled self-sufficiency, that is,
\co{pride}. What it veils, or rather what it does not unveil, is that its
search for paradise, \wo{impatient of any second place}, must evoke numerology
or Cabalistic practices, magic or spiritualistic media, in order to convince
itself of the  
sufficiency of human efforts -- precisely because this sufficiency is not given
and has to rely on magical devices, precisely
because everything originates from the sense of loss, or only 
lack, which at a deeper level feels irreparable.

\pa\label{wellmeant} %good intentions - unintentional evil
Let us emphasize: search for paradise is totally well meant and involves only good
intentions; no pride, no offense is intended.
The whole world is full of good intentions, and the best of them are to
ensure paradise -- for oneself, for family, for others\ldots \citetin{I would like my
  love to embrace the whole mankind, to warm it and clean it from the dirt of 
  modern life [...] Often it seems to me that even mother does not love
  children as warmly as I do.}{CCCP}{p.31 \citaft{Tischner}{ p.45}}
%  1917-1991 [in Polish], Krak\'{o}w, p.31 [quoted after J.Tischner, Sp\'{o}r o
%  istnienie cz{\l}owieka, p.45]}
It may seem strange that such feelings might have underlied the activities of
\wo{the bloody Feliks} Dzier\.{z}y\'{n}ski, one of the main architects of Cheka
and the communistic terror in Russia. But sympathy and compassion for $X$ can
easily involve hatred, even cruelty, towards $Y$, if only the latter is
perceived as being guilty of the former's misery. And if this misery is ultimate
evil, so the guilt is inexcusable and deserves most cruel punishment.
\co{Idols}, \co{idols}\ldots

Any idealized society (where justice, equality and happiness rule over human
imperfections and sense of incompleteness) is an \co{idol}, any deep and genuine
dream of it a clear \co{sign} of \co{alienation}, and any attempts to construct
it are guaranteed to end up the way they always used to end.  Hell is paved with
good intentions and those who end up there are almost exclusively those who have
looked all too intensely for paradise. An infinitely thin line separates all too
good intentions from all too ambitious goals. \citeti{Those who seek gold dig
  much earth and find little.}{Heraclitus}{DK 22B22} Few, if any people ever
commit crimes in order to achieve evil. There is always some good which
motivates even the worst deeds. But exaggerated intensity in digging for some 
good witnesses rather to its opposite. The higher and the greater is the good claimed
to \thi{motivate and explain} a particular \co{act}, the deeper is \co{idolatry}
and, usually, the more terrifying the result. Good, like wisdom, can enlighten but
not explain. 
Particular \co{acts} are never \co{visibly} traceable, not to say necessary,
consequences of any higher good. It is only conflation of the highest good with
the \co{visible} form of an \co{idol} which may seem to dictate with necessity
any definite \co{acts}, as it turns the infinite \co{love} into activism,
religiosity into moralism, \co{commands} into directives, and \co{thirst} into
lack to be filled with \co{visible} goods.
%But visible efforts do not sum up; they can help in establishing invisible
%consequences but then they must be carried over under the signs of the invisible

An \co{idol} is a \co{sign} of a loss, of a broken chain,
\co{alienation} from the \co{origin}. \co{Thirst} \co{experienced} as a mere
thirst, a mere loss which 
grows into unacceptable pain and searches only for ever new \co{visible}
tranquilizers, is a \co{sign} of \co{invisible} rebellion which, making one
look for paradise, directs one to hell. The initial voice of this rebellion
calls to an active open war which, in more \co{visible} terms, means only
intensified activity; not as yet any evil will but a blind and restless, no
matter how apparently purposeful, search capable only of rising \co{idols}. No
pride nor offense is intended, at least none can be seen, but their
\co{invisible} seeds have already started germinating\ldots

% . \citt{What can be worse Than to dwell here, driven out from bliss,
%   condemned In this abhorred deep to utter woe; Where pain of inextinguishable
%   fire Must exercise us without hope of end [...]?}{Milton, Paradise lost,
%   II:85-89 [spoken by Moloch]}

\noo{
\kom
Search for paradise, re-posses it  Milton 168[631-36]; all know and want good
243[810ff], 244[845ff]; cf. Pico
\begin{itemize}
\item   
wspolczucie dla X \impl nienawisc do Y (szukanie winnego, wyjasnianie
[Dzirzynski/Tischner]) 
\item
Augustine -- 3rd tenation (?)
\item
dobrymi checiami jest pieklo wybrukowane
\end{itemize}
Thirst \impl ako. pain/lack \impl ako. ``evil''; experiencing it as ``evil'' is
already a sign of Satan's rebellion (who promised you will be happy?), but we do
not know it yet.

\kom
Moloch walczy dumnie, jeszcze nie widzi... 174-6

cretion of idols -- is ako. alienation/separation/dissociation

\kom
We choose, and approach without knowing (goals remain hidden until they are
reached).

... wants good, but effects evil ...
} %end \noo


% Shall the needy ones \citt{sit lingering here, Heaven's fugitives, and for their
%   dwelling place Accept this dark opprobious den of shame, The prison of his
%   tyranny who reigns, By our delay?}{Milton, Paradise lost, II:56-60 [spoken by
%   Moloch as the first in the council advising Satan on the course of action]}

% \citt{Belial, [...] A fairer person lost not heaven; he seemed For dignity
% composed, and high exploit}{II:109-111}

\subsub{The circle of despair}\label{sub:belial} % [Belial]
{Attachment} to \co{idols} only increases \co{thirst} -- the more we believe
that it has disappeared and the less we feel it, the more nagging its \co{presence}
becomes. Whether we actually feel it or not, whether we have any actual
\co{signs} of it or not, we all the time know this \co{presence} -- at best, we
can only keep it at the threshold of \co{actuality}, for some time\ldots
Psychoanalyst could perhaps say that we suppress it, but it is not a simple game
of conscious and unconscious. It is \co{present} and as such not suppressed. It
lends all its power to our \co{idols} -- the more intense and unbearable its
\co{presence}, the more absolute power has to be ascribed to the current
\co{idols}.

But less than All cannot satisfy man. We search for and find more things and
matters to consider, more 
goals to achieve, more intensity and engagement, more power in lesser gods --
but insatiability, or avarice, is still only serving the \co{idols}. Behind the
circle of \co{more}, behind the horizon of \co{visibility}, there lurks already
emptiness of despair. As always, it is not critical whether it enters the sphere
of \co{actuality} and consciousness, or not -- what matters is its very
\co{presence}. One need not know that one despairs to despair. But knowing that
one does may intensify the despair. 

\pa \co{Idolatry} is already a form of despair. But despair intensifies when one
loses the faith in one's \co{idols} which until
now have been helping against the \co{thirst}, and then realizes that all the
idols are only images, masks, lies offering false promises and hopes.
% loses faith into the possibility of faith.
It is a meeting with \co{nothingness} under the spell of \co{visibility}, and
hence only in its negative character, as emptiness, pure void. \co{Idols} had
seemed to be something or at least to hide somebody. Some face was
expected behind their masks, like the unreachable goals used to absolutise the
relative, or the ultimate and \co{visible} goals used to justify the
unjustifiable. But now, when all the masks have fallen, no face 
appears -- and bare emptiness stares into one's face. 
One tries to fill it with this or that, with some old or
new \co{idol}, with work, fascinations or orgies, eventually also to pretend
that, since \co{actually} nothing has happened so nothing has happened
really. But void of \co{nothingness} is not like an empty glass and can not be
filled with anything. One is bound to begin to live through, if not also
realize, the fact \wo{that there is no truth, that there is no absolute
  character of things, no \thi{thing in itself}} and \citet{that all faith, all
  accepting as truth is by necessity a falsehood: for there is no 
  such thing as the true world.}{NietNihil}{XII:9.35 and XII:9.41
  \citaft{Wokol}{}\noo{p.83} }
Admitting this ultimate indifference of the world, its aloofness which one still
resists to take as enmity, 
is often misunderstood as an act of intellectual honesty. But it is only an
act of existential despair.

\pa\label{noconsolation} One may try some more desperate acts. Ixion, having
fallen in love with Hera, dreams of possessing her and makes successful advances
(in some versions, Centaurs are the offspring resulting from their intercourse).
His boasting of having had slept with a goddess is, however, a result of an
illusion: he slept with a cloud which Zeus created in resemblance of Hera. Thus,
trying to reach a goddess, he catches the air and, as a punishment, he is bound
to a wheel on which he is whirled by winds for all eternity. Heaven is, according
to the Greeks, always a {\em gift} from gods. The vanity of any attempts to
reach it by one's own means recurs with figures like Bellerophon, Icarus,
Prometheus.

  
Such attempts, although deserving further punishment from gods, are themselves
\co{signs} of already being in despair. The time spent on
unsuccessful attempts to escape begins gradually to suggest: there is no escape!
One may try to look for reasons and explanations, that is, for excuses and the
guilty ones (\wo{Hell is the others!} cries Sartre behind the \btit{Closed
  doors}\ftnt{The US title of the play was \btit{No Exit}.}), one may confront one's own sinfulness (\citet{Angst
discovers freedom but this is the same as discovery of sin}{Angest}{V
\noo{p.212++} [modified]}), but all that does not change anything:
from emptiness, where there are no walls, 
there is no exit either. The impossibility is eternal.
% and when man \citet{is in this hell, nothing may console him; and he cannot
%   believe that he shall ever be released or comforted.}{TheolGerm}{XI}

%\pa
\label{properdespair}
Proper despair is to surrender to despair. As no exit is \co{visible}, one is
doomed 
for remaining inside forever. \wo{Inside}? But inside of what?  There is only
emptiness around. Yet the walls of emptiness create the most terrifying
\thi{inside} -- they isolate and \co{alienate} making man dwell \citeti{in
  desolate cities, and in houses which no man inhabiteth, which are ready to
  become heaps.}{Job}{XV:28}
\co{Alienation}, the apparent freedom of empty nothingness is exactly the
inescapable damnation -- in one: accusation, trial and conviction.  One remains
\thi{inside} the imaginations and hopes of \co{visibility}. Their
\co{experienced} and \co{clearly} known insufficiency to bring any consolation
testifies to some \thi{outside}. But there is only emptiness, so any
\thi{outside} is impossible. It is, it must be because one needs it so much, it
must be real because one cannot live without it -- and
yet it is impossible that it is.
%This is a possible way of saying \No. 

To surrender to despair is to say \No\ to the possibility of something being there,
\thi{outside}. Not only there is no \co{visible} exit, but there is no exit
whatsoever because there is nothing toward which one could exit.
% used in II; 1.4 - Level of Invisible:
% \citt{`In hell there is no redemption.' Of this state hath one said, `Let me
%   perish, let me die! I live without hope; from within and from without I am
%   condemned, let no one pray that I may be released'.}{Th. Germ. XI}
\citet{Also let a man mark, when he is in this hell, nothing may console him; and
  he cannot believe that he shall ever be released or comforted.}{TheolGerm}{XI. \citefi{He shall not depart out of darkness; the flame shall dry
    up his branches, and by the breath of his mouth shall he go away. Let
    not him that is deceived trust in vanity: for vanity shall be his
    recompense.}{Job}{XV:30-31\noo{He shall not believe, being vainly deceived
      by error, that he may be redeemed with any price}} In a much more
  profane language, \citef{there is, alas, the loneliness which is without any
    hope of compensations, the loneliness due to the individual's failure to
    reach some common understanding with the world. This is the bitterest
    loneliness of all, the loneliness which is eating away at the heart of my
    existence.}{SisterI}{IV:31}}
%[though his hell is self-accusation \simu knowledge of one's sins]
As there is no hope of exit, as all we confront is the eventual void, the
  \citet{final hope\lin Is flat despair: we must 
  exasperate\lin The almighty victor to spend all his rage,\lin And that must end us,
  that must be our cure --\lin To be no more; sad cure;}{Milton}{II:142-146,
  [spoken by Belial advising against the open war recommended by
  Moloch]\label{ftnt:Belial}}  
The circle of despair is self-strengthening as, accepting the impossibility of
  exit, one begins to despair over one's
own despairing.  Hell has no end in time, it is \thi{eternal}.
\citet{Let us think this idea in its most terrifying form:
  existence, as it is, without meaning or aim, but inevitably recurring, with no
  end in nothingness: eternal return. It is the most extreme form of nihilism:
  eternal nothingness (nonsense)!}{NietNihil}{p.77.\verify{} \thi{Eternity} of hell
  is always posited as the 
infinity of objective time (whether eternal return or just eternally lasting 
  suffering). Infinite time, this bad image of \co{eternity}, witnesses to the
  continuing \co{attachment} to \co{visibility}. \citef{The fear of future turns
  into the fear of death, and the fear of death into the fear of hell. It
  is always fear of the fate in time, of the lack of any end in time, that is,
  fear of the lack of exit from objectivisation, of infinite objectivisation.}{Bier}{IV:3\kilde{p.86}}}

Surrender to despair is, as the initial despair itself, an
\co{invisible} event.  Consciously one opposes it and tries to get out of
it, one may be terrified and frightened. 
But as one keeps trying to avoid it, one only sinks deeper into the despair
over one's own despairing. The desperate attempts to oppose it are the
\co{actual signs} of the surrender, of the \co{invisible} defeat, the \No\ said
silently in the depths.  

% Accepting despair one may pretend to cherish it. One may be afraid of
% something even worse. \citt{Is this, then worst -- Thus sitting, thus
%   consulting, thus in arms? What when we fled amain, pursued and struck With
%   heaven's afflicting thunder, and besought The deep to shelter us? This hell
%   then seemed A refuge from those wounds; or when we lay Chained on the burning
%   lake? That sure was worse.}{Milton, Paradise lost, II:163-169 [spoken by
%   Belial who counsels against Moloch's suggestions of open war]}

\noo{
\kom
Widzi rozpacz (swiadomosc), jeszcze jej nie rozumie ale juz jest bierny

\kom
Po kilku rundach widzi, ze tu zostanie, ze nie ma wyjscia. Rozpacz wlasciwa =
akceptacja rozpaczy. 

\kom
Belial boi sie jeszcze czegos gorszego (estetyzuje, Zapfe?) 176-8

Mainl\"{a}nder, p.49: pragnienie spokoju to tylko pragnienie smierci
} % end \tin

\subsub{Saying No}\label{sub:mammon} % [Mammon]

\pa\label{pa:death} The circle of despair is the circle of damnation from which
there is no exit. With one exception\ldots?  As Belial suggested
(footnote~{\small{\ref{ftnt:Belial}}}),  
%in \refp{properdespair},
one may attempt the cure of non-being. As a spiritual being, he cannot commit
suicide and non-being can only be a gift from God. For man it is a different
matter: death \citet{is the only god who must come whenever we only call
  him.}{Hebbel}{vol.IV/V:4311 \citaft{Wokol}{\kilde{str.32}}} In the circle of
despair, in the middle of nothingness which is the ultimate unfreedom, suicide
appears as the last possibility of retaining and proving one's freedom.
\citetib{Man can kill himself because he has such capacity; and this capacity
  without the right to its use would be a
  luxury.}{Hebbel}{vol.IV/V:2292.\noo{\citaft{Wokol}{\kilde{str.29}}} Perhaps
  one of the most extreme expressions of this direction of thought is
  metaphysics of annihilation in \citeauthor*{Mainl},\kilde{
    \citaft{Wokol}{p.71}, } according to which the \wo{thrill of annihilation}
  and \wo{will of death} is the ultimate truth of the thirst for peace, in fact,
  of any spiritual thirst.  In our days, Zapfe's \wo{Uninhabited planet is no
    tragedy} seems to repeat this aestheticism of annihilation, whose usual and
  only attempt at self-justification is appeal to nature -- \wo{what difference
    would it make to her were the race of men entirely to be extinguished upon
    earth, annihilated! she laughs at our pride when we persuade ourselves all
    would be over and done with were this misfortune to occur! Why, she would
    simply fail to notice it.}\noo{the stupid pride of man, who believes
    everything created for him, would be dashed indeed, after the total
    extinction of the human species, where it to be seen that nothing in Nature
    has changed, and that the stars' flight had not for that been retarded
    [...]}  This last quotation, perhaps a bit unfortunately, is from
  \citeauthor*{SadeBed}. Yet another effort, Frenchmen...
  \kilde{p.74\citaft{EvilFaces}{p.194}.}}  The argument is rather strange,
suggesting that everything which is possible should also be allowed. But we
sense the need to justify suicide by ascribing it the element of freedom in
addition to, or perhaps even instead of, the reactive character of ultimate
despair. (In Mainl\"{a}nder, the universal fact of the death of finite beings is
even called the \wo{will of death}, though in humans and living beings in
general this metaphysical, even divine, will of death is covered up by
the apparent \wo{will of life}.)

\pa\label{suic=no} The negative character of this freedom (if freedom it is) is
obvious -- it is a door of escape, \thi{liberation from\ldots}.  Over 50\% of
studied suicide attempts are classified as individuals trying to achieve
surcease. The rest are either trying to \thi{manipulate} the environment (to
have revenge on a rejecting lover, to ruin the life of another person, to have
the final word in an argument as in the so called \wo{altruistic suicide}) or
are combinations of both: surcease and manipulation. In every case,
self-inflicted death seems to be the last thing one is capable of achieving, is
the last possibility of a self-chosen act.\ftnt{Research indicates that most
  suicide attempts are not preceded by a clear and definite decision but that
  such people for the most are undecided about living or dying and as if gamble
  with death leaving it to others to save them. Such cases would nevertheless
  fall under our description of \wo{self-chosen act} or \wo{voluntary choice
    made by \co{myself}}.\noo{, which mean of course more than \co{reflectively
      conscious} choices of goals and definite course of actions.} Psychological
  differences between a person merely gambling with death and one determined for
  and efficiently carrying out a suicide do not concern us -- each has chosen
  suicide and each has chosen it \co{himself}.}
%
% Not excluding the possibility of suicide being a consequence of some disease
% (not to mention offering one's life without any suicidal motives, as in the
% cases of so called \wo{altruistic suicide}),

In either form, the voluntary choice of death is saying \No; it is accepting
that only emptiness surrounds the horizon of \co{visibility}; and since
\co{nothingness} is nothing, the only hope of transcending the unbearable
situation is to pass into non-being. One might, perhaps, discern an element of
heroism in attempting such a free \co{act} in the depth of spiritual
enslavement, in sacrificing one's life when confronted with a higher truth.  But
it is lamentable when this higher truth turns out to be nothing and the apparent
freedom is only escape.\ftnt{Admissibility of suicide is always an expression of
  absolutisation of the \co{visibility} beyond which nothing can be~\ldots seen.
  The dignified suicide of a samurai or a Roman official, as the last way of
  preserving one's honor, perhaps even of expressing one's respect for the one
  commanding the suicide, is probably the best example of reducing human
  \co{existence} to a tool of the social system. Suicide is also often defended
  by reference to the need of preserving one's dignity and self-respect in the
  face of unbearable suffering. The complexity of the issue eludes any simple
  judgments but we would, nevertheless, point out that taking one's life amounts
  to ingratitude for this ultimate \co{gift}. (\wo{It's {\em my life} and I can
    do with it what {\em I} want!} \co{I} certainly can. But if you hear a drug addict
  pronouncing such an opinion you do not think he is right. You think he is
  terribly wrong.) Retaining \co{visible signs} of (self-)respect by neglecting
  \co{spiritual thankfulness} is no less dubious deal just because it is
  common.}
% quoted later...:
% \citt{The light of the body is 
%   the eye: if therefore thine eye be single, thy whole body shall be full of
%   light.  But if thine eye be evil, thy whole body shall be full of darkness.
%   If therefore the light that is in thee be darkness, how great is that
%   darkness!}{Mt. VI:22-23; Lk.~XI:34}
\newp 

\pa But there is also another way of saying \No. Having surrendered to despair,
one now accepts it. I am in prison, and there is no way out.  \citet{Nay, cursed
  be thou; since against his thy will\lin Chose freely what it now so justly
  rues.\lin Me miserable! Which way shall I fly\lin Infinite wrath and infinite
  despair?\lin Which way I fly is hell; myself am hell.}{Milton}{IV:71-73} Since
I am imprisoned, so I will stay imprisoned -- there is nothing I can do about
{\em that}. In this way, staying spiritually imprisoned amounts to willing this
imprisonment. \citet{If it should now happen that God in heaven and all the
  angels were to offer to help him to be rid of this torment -- no, he does not
  want that, now it is too late.}{Despair}{I:C.B.b.$\beta$;p.103} There is an
element of sick will in staying imprisoned, even if this will seems to be not
\co{mine}, but somewhat imposed on \co{me} from above.  This willing is no
longer despairing over one's despair, nor is it any longer aesthetising this
despair.  It is now \citet{despairing of forgiveness, when someone because of
  the extent of his sins completely gives up hope in God's
  goodness.}{AbelardEthics}{\para 177} It is now accepting one's despair, trying
to turn it into something good. Just like voluntary passing into nothingness,
suicide, so also this acceptance seems to be self-chosen, even if not
self-willed. It is the despair of defiance, as Kierkegaard says, \citet{the
  despair of wanting in despair to be
  oneself,}{Despair}{I:C.B.b.$\beta$\kilde{p.98}} of insisting on \co{myself}
when \co{I} should completely give up \co{myself}, of not realizing that \co{I}
am only getting the more imprisoned the more \co{I} resist to surrender.

\pa This active choice of despair agrees on the impossibility of salvation -- it
is the final acceptance that there is no exit. And so, \citet{if heaven I can
  not bend, then hell I will arouse.}{Virgil}{VII:312 [modified
  translation]\noo{spoken by Juno (Jupiter's wife) who, in her bitter hatred of
    everything Trojan, is determined to destroy Aeneas for any price}}
The only thing one can do now is to
turn this evil of damnation, \citet{the torment of perpetual
  penalty,}{AbelardEthics}{\para 168} into good, pretend that evil is good.
\citet{Evil, be thou my good: by thee at least\lin Divided empire with heaven's king
  I hold.}{Milton}{IV:110-111\label{ftnt:evilBeGood}} \wo{I hold} because \co{I} still act on \co{my} own
initiative, from \co{my} own choice.  Yet, \co{I} never forget that exit was all
\co{I} wanted, and so this \thi{free} choice of \co{mine} is only a
renouncement, ultimate 
resignation.  It knows, like Milton's Satan and all his associates know, that it
wished and still wishes good, something good, which here means
exit. But this knowledge has hardly any \co{visible signs} and remains
hidden beyond the \hoa. The active choice of \No\ amounts to denying this
knowledge. But: \citeti{Woe to you that call evil good and good evil.}{Is.}{V:20}

\pa
Having lost paradise, having \wo{lost the eternal}, and now also the hope of
regaining it, Mammon advises to 
do only what can be done by one's own powers. Indeed \citet{great things of
  small,\lin Useful of hurtful, prosperous of adverse,\lin We can create, and in what
  place so e'er\lin Thrive under evil, and work ease out of pain\lin Through labour and
  endurance.}{Milton}{II:258-262}
In another context, this might sound almost convincing, but here Satan draws the
eventual consequence of 
this whole \co{invisible} development -- to turn in revenge against the
\co{visible} world, against God's last creation, \citetib{some new race called Man,
  about this time\lin To be created like to us [...] Thither let us bend all our
  thoughts, to learn [...] what their power\lin And where their weakness; how
  attempted best\lin By force or subtlety; though heaven be shut,\lin And heaven's high
  arbitrator sit secure\lin In his own strength, this place may lie exposed,\lin The
  utmost border of his kingdom, left\lin To their defense who hold it.}{Milton}
  {II: 348-362. (Blake's scheme of thirst which, 
  when suppressed, turns into greed and envy owes much to Milton.)}
One can doubt the psychological reality of such a direct and pronounced choice of
evil as one's good. But if ever made, it is only a \co{visible sign}, a
\co{symbol} of the \co{invisible} defeat, of the active \No. As this \No\ removes the
\co{invisible rest} from all \co{actuality}, it results in \co{visible}
consequences which will permeate the whole \co{existence} down to its most
\co{actual} element. 

It belongs to the nature of damnation that it universalizes itself.  Just like
one can not be happy in the evil world, one can not be damned in the middle of
saints and saved. \citet{To the unhappy, it is a comfort to have had companions
  in misfortune.}{FaustMarlowe}{V:42. (The Latin version -- \la{Solamen miseris
    socios habuisse doloris} -- was quoted by many authors, but its origin
  remains unknown.)} And when companions are hard to find, one starts producing
them by demonstrating the universality of misfortune.  \citet{[T]he more I
  see\lin 
  Pleasures about me, so much more I feel\lin Torment within
  me,}{Milton}{IX:119-121} says Satan who can nothing else but try \citetib{all
  pleasure to destroy,\lin Save what is in destroying; other joy To me is
  lost.}{Milton}{IX:477-479} \citet{Rebelling against all existence, it thinks
  it has acquired evidence against existence, against its goodness. The
  despairer thinks that he himself is this evidence.}{Despair}{I:C.B.b.$\beta$}
The damned, the active \No, comes thus out of its closed room without, however,
ever leaving it; it comes out as the \co{visible} activity trying to embrace
everything but driven by the substantialised lack, its emptiness.  Damnation
finds its \co{expression} in every single thing and situation, it permeates all
\co{actuality}, even when it itself remains non-\co{actualised}.  It is, after
all, the very impossibility of \co{actualisation} for it has removed everything
which possibly might be \co{actualised}; equating the ultimate \co{invisibility}
with emptiness, it has cut away the source of \co{actualising} meanings and,
eventually, turns even \co{actuality} into nothingness. Only absolute emptiness
can be absolutely insatiable. Insatiability, the impossibility of satisfaction,
is a substantialisation of \co{thirst}: like Thyestes devouring his own body,
like Tantalus ever hungry and never able to reach the water and fruit brushing
his lips.  Insatiability of damnation is to spread its despair over all and
everything, in search for a community beyond its \co{alienation}. \citet{Man
  communicates by means of despair when he no longer has any other
  community.}{Suicide}{III:6.\kilde{p.101} Durkheim speaks here only about the
  tendency to, what he calls, \wo{egoistic suicide}, arising from the
  dissolution of social structures and increased individualisation or, as we
  might perhaps say, alienation. \wo{The individual's appetite for life
    diminishes because the connections relating it to the society are weakened.}
  We would not identify our \co{alienation} with Durkheim's
  \thi{individualisation}, just like we would never identify \co{absolute} with
  \thi{society}. But putting absolutisation of the social aspect aside, we can
  easily recognise the accuracy and relevance of Durkheim's observations.
  \noo{(In fact, Durkheim's \thi{explanations} and \thi{causes} are of
    psychological character -- it is only in the establishment of the
    \thi{facts} that statistics is of any use to him.)}}

It is also eternal, for one can not possibly get out from the place where there
is no \co{visible} exit. The lack of exit means that \citet{no end is limited to
  damned souls,}{FaustMarlowe}{XIX:171. (\wo{limited} meaning here appointed, fixed
  definitely.)} means the eternity of damnation. (\thi{Bad eternity}, of
course, infinite temporal duration, for the damned remains thoroughly within
time.)  The only relief one can then find consists in the confirmation that
\wo{damnation is the truth}, that \wo{so is the world}, that \wo{hope is an
  illusion}.  Extreme pain soothes lesser pain; common degeneration in the world
around seems to attenuate my own degeneration; nothing seems to allay more the
meaninglessness of the private suffering and despair than the realization that
this is actually the universal truth of life. It would be futile to ask what
comes first -- suffering or the perception of its common (if not universal)
nature. Suffering is indistinguishable from its \co{experience}, while
\wo{universal}, \wo{common}, etc. are here only \co{actualised expressions} of
the \thi{objective} character of evil, of the fact that it overcomes \co{me}, is
\thi{greater than me}.  At this last circle of despair, the suppressed
\co{thirst} solidifies, one could say, {substantialises} the ultimate emptiness
as a universal truth. But since this \co{actual} truth is not any truth, one is
bound to keep searching for its confirmations.



\noo{
\kom
Wiezienie = ``pragnienie'' bycia wiezniem [Tischner 98,191]; Milton (used:
224[71]!), 175[85-9],(used: 176[142-3])

\kom
wiec nic nie bede robil {\em z tym}. 
Choose -- on my own; and do -- myself (Milton 152[105ff], 179[252]) whatever I
can still do (Mammon, imitate Heaven 179-80)

it is only now that Satan turns against Man, that is, everything becomes
more and more \co{visible} 

\kom
will stay here and cherish it/destroy for others: Evil be thou my good, 182ff +
(used? 224[71]!), w naturze potepienia lezy uniwersalizacja potepienia [Tischner
191] 

246[937-8], revenge 167[605ff], destroy for others 246-7[939-40]
} % end \noo

\sep

\pa What we have called \wo{despair} Kierkegaard would classify as only its
higher stages, perhaps, as \wo{the despair which is conscious of being
  despair}. Already our \co{thirst} would be classified by him as lower levels
of despair. Indeed, the stages of the gradual intensification of \co{thirst} to
despair referred to above correspond closely to the intensification of despair
described by Kierkegaard in \btit{The sickness unto death}, I:C.B.\btit{Despair viewed
  under the aspect of consciousness.} 
So, what's the difference? Is there any?

We certainly do not want to be so dramatic, for life is not. There may be
humility in \co{thirst} which would be hard to find in despair. But we also
sense a significant difference. What we said in \refp{pa:thirst} about the
commonality of \co{thirst}, might have been expressed by Kierkegaard, for
instance, as follows: \citet{An older woman who has supposedly left all illusion
  behind is often found to be fantastically deluded, as much as a young girl, of
  how happy she was then, how beautiful, etc. This \la{fuimus} [we have been],
  which we so often hear from older people, is just as great an illusion as the
  younger people's illusions of the future; they lie or invent, both of
  them.}{Despair}{I:C.B.b.$\alpha$.1 \kilde{p.89}} All that is probably true,
in a sense, but it is not right. If people \wo{invent or lie} and they do so
throughout the whole history of mankind, then the problem lies rather in the
diagnosis than in the diagnosed.  \co{Thirst}, as we shall see, is not something
which, like Kierkegaard's despair, one just has to dissolve in active
consciousness. It certainly is not, as is despair, a sin which cannot avoid
deepening 
\co{alienation}. And this is the crucial difference -- Kierkegaard insists on a
kind of consciousness: \citetib{what characterises despair is just this -- that
  it is ignorant of being despair.}{Despair}{I:C.B.a\kilde{p.75}} Yet,
\citetib{the more consciousness, the more intense despair,}{Despair}{I:C.B.a\kilde{p.72}}
%   this might probably expire, cos `consciousness' will eventually
%   mean `right understanding' and this, in turn `being saved'. But no, see
%   Socratic definition of sin II:2 [120ff]}
and so it seems we have a slight problem with the relation between consciousness
and spirituality. The two seem often simply identified, \citetib{inwards, at an
  even higher level of consciousness}{Despair}{II:B\kilde{p.141}}, as if inwards
was impossible without \co{actual} consciousness. 

Perhaps, it is only a minor technicality in need of a proper
interpretation, but it seems to harbour the fundamental conflict of
Kierkegaard's, of which the tension between the intensity of actual
consciousness and the passivity of the spiritless (bourgeois) is only an
epitome.  \citetib{But despair is exactly man's unconsciousness of being
  characterised as spirit.[...] Most people live without being properly
  conscious of being characterised as spirit -- and to this one can trace all
  the so-called security, contentment with life, etc., which is exactly
  despair.}{Despair}{I:B \kilde{p.55-56}} Such desperate classifications 
(smelling if not gnosis, then Lutheran pietism and sense of sinfulness)
we are unable to share. Being \thi{properly
  conscious} of anything is no goal of life, neither is seeing {\em desperate} 
involvement in every world involvement, especially in the secure and content
one. Sure, one can attempt a bit sharper interpretation, giving more
plausibility to this opposition. But we think it is unnecessary because what
counts in Kierkegaard's, as in any other philosophy, is the fundamental mood,
the fundamental truth which it elaborates in all possible forms.  The mood of
Kierkegaard is that of a dramatic tension, yeah, of a prophecy arising from the
opposition to the neglect and disrespect shown by the world, by \wo{the small
  market town}, to the higher, spiritual 
things.  There may certainly be a tension between the two
elements, but spiritualising consciousness makes it into an unbearable 
contradiction.  The result seems quite a bit unhealthy, especially, if we take
into account that \citetib{[h]ealth is in general to be able to resolve
  contradictions.}{Despair}{I:C.A.b.$\beta$ \kilde{p.70}} Kierkegaard is the unresolved
contradiction between the two poles: an individual, free spiritual existence and the
world sunk in impersonal spiritlessness. If you like, it is the contradiction
between the self, founded in the relation to God, and God himself, whom the self
can not reach through mere consciousness. Whichever form, we do not want to end in the
same point. If the world is an enjoyable place of comfort and (why not?
aesthetic) content, then the goal is not to leave it. And if the world is
impersonal, inauthentic, despairing, then the goal is not to leave it either.
One lives in \co{this world}, and the fact that its platitudes and
spiritlessness can sometimes feel discouraging does not mean that spirit lives
somewhere else. If it lives anywhere, it is only in the midst of \co{this
  world}, not perhaps {\em in} its \co{dissociated} things and spiritless
activities, but between them. Such a depersonalised world, such deindividualised
people as 
existentialists, following Kierkegaard, used to describe are \wo{inventions or
  lies}, for people appear so
only when viewed through the requirements of plain \co{visibility} and
transparency. The opposition of a reflected personality to the stupefying images
of mass-media or narrow-minded images of a market town does not extend to the
contradiction between spirit and spiritlessness, nor that between
faith and sin. The world one lives in is the world as one is able to see
it. Accusations against it turn out, eventually, to be self-accusations. 
%Besides the world is {\em as one sees it} -- general thought, idea...

Consciousness of despair can certainly intensify the despair, but despair can
also reach quite deep levels without active consciousness.  We have emphasized
that deepening of despair is, at the bottom, an \co{invisible} process, and
realizing it consciously is only an additional possibility -- certainly,
tremendously complicating, but not necessary.  It is unnecessary because one
always somewhat, in the depth of \co{irreflective self-awareness}, knows one's
\co{spiritual} condition. This knowledge, however, is not that of \co{actual}
consciousness which fully realizes what's going on, whence it comes and whereto
it goes. It is this very inability of \co{actually} seeing whereto the
experienced despair is going which, on the one hand, deepens the despair and, on
the other hand, suggests that the possibility of healing lies elsewhere and not
in the autonomy of consciousness.  We will consider this issue in more detail in
\ref{sub:beingKnow} but now we are turning to an elaboration of the hermeneutics
of \No.


\section{Spiritual choice of No}\label{sec:evil}
\pa\label{pa:evilsSuffer}
Despair is a form of \co{alienation}, of turning \co{nothingness} into nothing
or, what amounts to the same, being cut off from the \co{origin}. It is the
\thi{ontological} \co{separation} carried to the \thi{epistemic} extreme of
\co{dissociation}. In the \thi{epistemic} categories of \co{actuality}, there is
indeed no difference between the two\ldots But this \thi{epistemic} mistake affects
the whole \thi{ontology} -- for in the sphere of \co{spirit} there is no
distinction between being and knowing. The form of \co{confrontation} is the
\co{confrontation} itself, and what and how it is lived determines the
\thi{ontological} character of what it encounters. The ultimate emptiness,
apparently so abstract and irrelevant for \herenow, once it finds the site in
the depth of one's being will only spread further and further down, putting
gradually more and more regions under its spell. Whether it experiences itself
as evil, or only finds evil in its experience, it is a seed from which more evil
arises. For even if evil is at the bottom lack and negativity, it is lack which
propagates and grows, it is negativity which universalizes itself.

%Despair -- and, most generally, evil -- \co{alienates existence} from its
%\co{origin}, making the latter appear as empty nothingness, total lack. This
%total lack, this ultimate void of emptiness is \he{substantialisation} of
%\co{thirst}, effected by the \No, by the denial of the possibility of salvation.
\co{Alienation} is the {substantialisation} of \co{thirst} which, unable to
maintain the positive -- even if impossible, \co{object}less and hence
unimaginable -- character of its intention, absolutises the negative character
of the experience as absence, as a mere lack.  Despair may for quite a long time
remain in a suspension as to its character -- as suffering, and hence an evil
experience which, however, need not be an experience {\em of} evil. As long as
it remains so suspended, it is suffering but it does not become evil.  Equating
the \co{invisible} possibility with the impossibility of exit, it ceases to
suffer. But it ceases to suffer only because it ceases to feel and know, only
because it has now turned its suffering into the \thi{objective} impossibility
of liberation, into the ultimate \co{alienation}, that is, evil. Despair need
not be evil though, in the moment it begins to re-cognise evil behind its
suffering, it is on the way to become it itself. Every evil expresses this
\co{alienation} which, unable to stand the suffering, begins to \co{experience}
it as a substantial entity and {objectifies} it as \thi{evil} (cause, person,
accident, life, world). Thus, distancing itself from it, it becomes also
distanced from an \co{aspect} of its life and, at the bottom of it, from its
source. Evil, we might say, is suffering which became substantialised, and thus
apparently \co{externalised}, in an attempt to escape itself.

\wtsep{emergence}

\pa Unquenched \co{thirst} brings pain and pain, just like suffering, is a great
danger; it can infect the \co{soul}.\ftnt{\wo{Pain} can be here taken to refer
  generally to all kinds of what is often called \wo{natural evil} -- physical
  pain, sickness, natural catastrophes, etc. Hardly any of the following
  formulations would require change if we interpreted it this way. We do not
  draw rigid distinctions, but \wo{suffering} is meant in the more fundamental
  sense, as the personal -- whether physical, emotional, moral or spiritual,
  whether only felt or also lived and deeply experienced -- pain, which has
  become objectless and is no longer relative to its possible causes.}  \wo{A
  hit with a hammer into the head can damage the soul.} The one who is suffering
asks all the questions of Job's and, eventually and inevitably, asks ``Why?''.
Left without answer, one grants \co{oneself} the right to accuse the
surroundings, other people, the world -- for undeserved suffering, for neglect,
then for injustice, for immorality and, finally, for evil.  \wo{I did not
  deserve this! It is evil!} And who are we to know what we deserve and what
we do not?  Whatever the accusations, at their basis lies \No\ which, having
all the reasons (bad rather than good, but seemingly sufficient) to blame and
accuse, becomes hate. Hatred is not an irrational, unjustified feeling without
reasons. As anger, according Seneca, it hardly ever occurs without reasons and,
typically, it has plenty of reasons for its own justification, it \woo{does
  necessarily presuppose an injury, either done, or conceived, or
  feared.}{SenecaAbs\kilde{p.19}} It only \citet{proceeds to the resolution of a
  revenge, the mind assenting to
  it.}{SenecaAbs}{\citaft{EvilFaces}{\kilde{p.22}}} But hatred, again like
anger, \wo{passes the bounds of reason, and carries it away with it} or, as we
would say, it is an expression (one of the strongest ones) of \co{alienation},
of the negative separation from the hated \co{object}, person, the other, the
alien, which caused my pain.  Then, when the ``why?''  does not find any
satisfying answer, and one is unable to stop dwelling on one's pain, the granted
hatred may embrace the whole world.\ftnt{Self-pity is, in fact, also an
  expression of \co{alienation}.  Pitying itself, it dwells in \co{attachment}
  which is underlied by the image of \thi{the evil world}, the \thi{evil} which
  is out there, against which it is impotent.}
% Here could come a formulation about evil as the lack of answer to `why', as
% the  lack of justification 

Suffering which does not disappear grows.  Time heals only wounds which have
ceased to cause pain.  But even a negligible pain, if it lasts, turns out to be
a constant element of \co{my} life, which may affect deeper and deeper levels of
\co{my soul}.  Initially it may affect only \co{my actual} situation which
\co{I} am able to face with all the vitality of \co{my actual} strength. \co{I}
may gather my strength and say \wo{It hurts but I can stand it}. And sometimes
it works.  But if it does not\ldots \co{I} can not distinguish clearly \co{my
  soul} from \co{my world}, nor \co{my world} from \co{this world}. A suffering
\co{soul} sees only suffering which, eventually, can spread over the whole world.
When so it encounters something else, it is a mere appearance, an accident, a
transitory insignificance.

\pa We do not have to list examples of how pain experienced by children may
deform their personality.  But we should keep in mind that such a pain is not
necessarily a child abuse, a sadistic attitude of the father, or any other plain
form of violence.  A molestation, minor annoyances, tokens of indifference or
undeserved blame, when confronting a sensitive \co{soul} may be experienced as
deeply hurting and painful.  A mere negligence, or else, high demands and
expectations, when not compensated by the overall atmosphere of love, underlying
care and understanding may constitute painful experiences. A mere presence of a
child at a scene of a humiliation, revile, deception, ravishing, even of a
simple quarrel between adults may cause enough pain.\ftnt{The director Robert
  Wilson, in cooperation with an anthropologist, 
  conducted a film analysis of a situation mother-child: the child is crying,
the mother lifts it. Split into 300 frames sequence displayed unexpectedly
that, in the first phase, the mother reacts aggressively and that the child
responds with a complex of movements and sounds expressing fear; only then the
actions of the mother become caring and protective. The woman did not want to
believe that. Wilson's conclusion: there are subconscious <<frequencies of
contact>> below the globalising level of words, exchange of signals so minute
that they can be brought to the surface only by an analysis in slow motion.
\citaft{Wolicki}{\kilde{p.170 in Dialog; else: Bettelheim, Cudowne i
    Po\.{z}yteczne I, str.24-25}} Recall the difference between the
\co{reflectively} registered 
\thi{substantive parts} and the flowing \thi{transitive parts} in the time
experience, I:\refpp{fig:nows}. }


\pa Evil is primarily a reaction -- a reaction which seems justified, but which
is neither controlled nor even realized. The first form of love -- because it is
also the form of \co{thirst} -- is the need to be loved. Evil is born between
men from the lack of love but also, and primarily, from pain caused often
without any intention, as if by accident.  Yet pain, even suffering, need not be
evil. It becomes so when one rises a wall of defense against the \thi{evil
  world}, against all the forces which bring suffering, when suffering begins to
\co{alienate}.\ftnt{Although it is common to consider suffering an evil, we will
  not identify them without further ado. The Hebrew language of Old Testament,
  for instance, does not even possess an equivalent of \wo{suffering}. The
  primitive root \heb{\wo{ra'\'{a}}} means to be bad, displeasing {\em or sad},
  and the noun \heb{\wo{ro'\'{a}}} denotes evil or wickedness but is likewise
  used for sadness or sorrow (e.g., in Ecc.VII:3, \wo{{\em Sorrow} is better
    than laughter: for by the {\em sadness} of the countenance the heart is made
    better.})  There are multiple variations which, however, denote suffering
  only in special contexts and in derivative sense. E.g., \heb{\wo{tsame'~$\!$}}
  = to (suffer) thirst; \heb{\wo{ra'eb}} = to suffer famish; the passive
  form\noo{(\la{<<niphal>>})} \heb{\wo{`arek}} of the verb \heb{\wo{`arak}} = to
  prolong/be long, means being patient and also to suffer exposure to evil, and
  similar associations obtain with words like \heb{\wo{yanach}},
  \heb{\wo{nathan}} which, referring primarily to resting, putting down, leaving
  or letting go, are also used in passive forms for being inflicted. But there
  is no single word capturing the distinction between suffering and evil and the
  Greek word for suffering, \heb{\wo{pascho}}, occurs only infrequently in
  Septuagint. One could elaborate various speculations on the ways the lack of
  this distinction influenced the concept and understanding of evil and justice.
  We will, however, keep the distinction for the main reason suggested above and
  earlier at the end of \refp{pa:evilsSuffer}: suffering is the \co{experienced}
  quality, one would be tempted to say, an essential aspect of human life, while
  evil is a consequence or only a conclusion drawn from suffering, as possible
  as unnecessary.}

There are innumerable ways of causing pain and of suffering, and it would be
futile to attempt 
their classification. It would be futile first of all because, obvious as some
of such ways might be, we do not know them all
and we can not know them -- it is not a matter for any general classification, but
for the \co{concrete} attention paid to the \co{actual} situations. There are no
\co{objective}, \co{visible} criteria not only for what, in more subtle cases,
may constitute a {painful} experience, but, above all, not even for what
consequences inflicted {pain} may have.  An unsatisfied need of a crying baby
may turn out just an insignificant accident, but also a first suggestion of a
lack, on which later disappointments will grow; an aggressive sentence stated
carelessly to the spouse, may happen to be ignored by a child accidentally
present in the same room, or else it may hit the most sensitive core of its
understanding; a misunderstood joke may become a mortal offense.

There may be far from such details to evil, but like all \co{actual
  experiences}, they too penetrate gradually the soul's \co{virtual} depths.
Evil emerges at first as a mere consent to evil, perhaps only ignorance of it,
and it emerges from such \thi{misunderstandings} -- it is born between men, but
it is born {\em into} them. 



\subsection{Privative evil}\label{sub:privativum}
%{Malum privavium}
%(which would do good ever yet forever works evil)
% -- \ref{sub:moloch}
\pa Emptiness of eventual \co{alienation} is not anything one chooses for its
own sake. At first, it is perhaps only an irrelevant annoyance, a slight threat,
then it becomes terrifying and, eventually, turns into a horror, \la{horror vacui}.
Nothingness is not anything one chooses at all, for it lies far \co{above} the
sphere of possible \co{actual} choices. To begin with, \wo{Nothingness is a
  thirst for Something}, as Jacob B\"{o}hme says.  Not knowing what this
\thi{Something} might be, it is a confused circling, a search, a desperate --
because incessant and indelible -- search for \thi{Something}.  The \co{soul},
even the infected \co{soul}, does not want anything evil. But missing
\thi{Something}, missing \thi{Something} to believe in and to rest on, it may
get involved in evil as well as in good. This early stage of nihilism -- not
knowing what to believe\ftnt{And barely suspecting that nothing can be believed.
  It was called \wo{passive nihilism} in \citeauthor*{NihilB}, and in
  \citeauthor*{NihilC}.}
% [Zur \"{U}berwindung des europ\"{a}ischen Nihilismus (1946/47) in
% \citeauthor*{NihilA}], and so does W.~Kraus [, , Frankfurt 1985]}
-- may still have all the 
signs of innocence and undeserved suffering. But it may also, and more typically
it does, appear in a variety of forms determined by the still functioning
\co{idol} (all \co{idols} are also forms of \co{alienation}): as egocentrism, as
a self-satisfied activity (which negates all that does not serve its goal), as
amiable aesthetism of Dorian Grey (whose soul rots in the closed room), as an 
American-dream hero (who only occasionally must visit his psychoanalyst), as an
obedient functioning of a scrupulous clerk (accidentally, working in a
concentration camp).
%Biesy, Martwe dusze

Anonymous and impersonal evil grows on the passivity of such a nihilism which,
at first, still tries only to find some new \co{idols}.
%not knowing what to believe and, consequently, what to do, on the passive
XX-th century has taught us the lesson of the most impersonal workings of
evil; as Hannah Arendt described, it is the mere failure to reflect which accounts
for the impersonal banality of evil.
%Thinking and moral considerations: a lecture, in Social Research,
%vol.38,no.3,1971; {The Many Faces of Evil]
Satan does not any longer visit individuals the way Mephistopheles visited
Faust; he remains invisible and unheard, appears only now and then as a hardly
identifiable, even if remarkable, person. He acts, as Woland in Bulhakow's
\btit{Master and Margarita}, kind of \la{incognito}, impersonally, through other
executioners, and only the final results show to the public that devil must have
been involved.  He is no longer a psychologist but a sociologist, perhaps, a
politician.\ftnt{Another powerful image in modern literature was drawn in
  reference to the society suffering equally and from the analogous disease as
  the communist Moscow afflicted by Woland. Visiting Leverk\"{u}hn in
  \citeauthor*{Faustus}, XXV, devil enters as a petty individual of a very
  dubious appearance and social status, but even this personal, or rather
  impersonal trait changes several times during the conversation, suggesting the
  presence, as effectual as imperceptible, throughout all the layers of the
  society and culture.  In 1995 French theologian summarises: \citef{Satan's
    greatest success in modern times is replacement of the direct activity --
    arising the fear of the devil -- by an organic, imperceptible, and hence
    tranquilizing activity, which penetrates the social texture without noise,
    devoid of the signature of the prince of this world, run by his agents
    occupying appropriate, strategic
    positions.}{SatanRene}{p.118\kilde{Diabel,p.139}} In spite of the catholic
  insistence on the personal and concrete being of the devil, his workings
  appear quite impersonal: \citefi{We know well that this dark, destructive and
    disquieting being really exists and acts, preparing against us sophisticated
    traps meant to destroy the moral balance of humans.}{Paul VI, 1972, in
    \btit{Report on the State of Faith}}{} But this depersonalisation of devil,
  his dissolution in impersonal forces, is at least as old as the modern
  nihilism. Devil appearing to Ivan Karamazov is a completely average person, an
  anonymous member of middle or lower class. In the 1830-ties, Aloysius Bertrand
  can see devil penetrating every corner of the social system: \wo{He argues in
    the Parliament, leads a defense in the Court, plays on the Stock
    Market.}{\citaft{SatanMinois}{VI:2\kilde{p.122}}} Depersonalisation was then
  but another side of the romantic acknowledgment and justification of devil
  (Clavinhac, Byron, George Sand), followed by the apparent rendering him
  harmless in the truly naive (even if powerful) socio-positivistic spirit of
  Hugo's \btit{La fin de Satan} or Balzac's \btit{Melmoth r\'{e}concili\'{e}}.
  His powerful return in the beginning of the XX-th century tells probably
  something about the price of such optimism.}

\newp
\pa
Evil of this lowest, most impersonal, but for this reason also the most
global kind, is not anything \co{I} choose but something \co{I} participate in,
it is \thi{greater than me}. It is unintentional, unwilled, perhaps even well
meant -- a \wo{force which would do good ever yet forever works evil}. This fact, that
it happens as if in spite of \co{me}, through \co{me} but 
not by \co{me}, illustrates the aspect of \co{alienation} almost at the
psychological level. Persons affected by Woland become like machines: not
because they suffer from some form of a %merely psychological
depersonalisation disorder, not because they merely {\em feel} detached, not 
because they cease to think and lose the ability to choose, but because the
objective world has taken away 
their possibilities to act and choose -- they get involved into situations created
completely behind their back, which they can only continue acting without
a slightest possibility of exercising any influence on further development.


It is an old mantra that \wo{passivity increases chances for (becoming) evil}
that, in the words of Paul Wladimiri, \wo{omission is nothing else but an
  alliance with evil} and \wo{truth which is not defended will be defeated}.
Are these trivialities? Of course -- if we only assume that \thi{the world is
  evil}. The evil of many socio-political systems of the XX-th century left the
astonishment: \wo{How can it be possible?}
%\wo{People prepare such a  lot for other people
Humans seem to do all that, but it is inhuman; nobody wants that, and yet it
happens: \citet{all I know is that there is suffering and that there is none
  guilty.}{Karam}{Ivan to Alosha in II:V.Pro et contra:4.Rebellion [The Grand
  Inquisitor]}
Having first declared human beings innocent for the evils for which only
defective social organisation carries responsibility, 
it takes now nothing less than brightness and precision of the French-most 
intellectualism of a sociological provenience to draw the conclusion that 
\thi{subject must be dead}. Helpless and meaningless as only abstract
concepts can be, once formulated it may however count on a wide popularity
-- human \thi{subject} could not possibly effect cruelty, murder and torture on
such a broad scale. \wo{It was not me! I did not want it, and hence I did not do
  it! Nobody wanted it, hence nobody did it!}  Unfortunately, all that happened,
and so this petty human \thi{subject} is apparently so insignificant that it
must be declared non-existent.

We know well enough that even if \co{subject} has never been more than a dead
abstraction, so human person is as alive as it always has
been and that \co{I} am as responsible for my acts as \co{I} always have been. The
removal of human subject amounts, willy nilly, to postulating an analogous center of
subjectivity at the level of society (whatever that might mean). 
We can thus sense here the reappearing ambiguity of \thi{subjectivity}
vs. \thi{objectivity}, \co{myself} vs \thi{us},
%\co{myself} vs. \co{my self}
from II:\ref{sub:objsubj} (in particular, \refppf{civObjSubj}). But no matter
how much agency we manage to ascribe to the impersonal forces of cultural
formations and socio-political systems, transferring one's responsibility to
them only deepens \co{alienation}. 

\pa
The character of evil can be very impersonal and involvement into it can take
the form of participation which happens beyond, and even in spite of, whatever
\thi{subjective} choices and intentions \co{I} might have. But it still needs
some necessary conditions which here happen to be the participating individuals.
\co{I} participate in evil which is \thi{greater than me}; \co{I} do not have
control over its full strength and effects, \co{I} contribute only \co{my} small
part to the totality which happens to be beyond every single among the involved
individuals. Nobody controls it! And yet it happens\ldots Perhaps, we should revive
the notion of collective responsibility (as it was done after both World Wars).

%more!
%
In its passive, \co{privative} form, evil can appear everywhere, but it will grow
only in certain conditions of axiological passivity. Although it never lives
fully in any individual, it is always among us, if not \co{actually} then only
\co{virtually}, germinating. But \co{virtuality} is already fully real, and
when it starts blooming, it may bring forth fruits which are 
surprisingly and incomparably more sour than the seeds from which they have
grown. The shock of the XX-th century is not madmen like Stalin, Mao or Pol Pot,
but the legions of common people who carry out the most inhuman operations --
the more inhuman, the higher are the \co{idols} who bless their actions.
The
terrifying inhumanity of the genocide on the native Americans is not embodied by
the people like Ltn. Colonel George Custer, U.S. Cavalry soldiers from Wounded Knee
Creek, from Bloody Island or other places of Indian massacres.  
On the contrary, it emerges through the 
apparently positive developments (exchange of goods, expansion of the missions)
and, eventually, even underneath the genuine and honest attempts to repair the
damages, best wishes of people like John Collier (Roosevelt's chief of the Bureau
of Indian Affairs implementing \wo{New Deal} for the natives) -- the inhumanity of
the genocide whose mere scale excludes any personal guilt, 
%which not only overcomes any particular evil done to the offers,
but which continues {\em in spite of} the increasingly good intentions or
bad conscience of the guilty ones.

Even if we subtract, in all such cases, the expected amounts of private gain, of
personal will to power and money, of resentment and revenge, we are still left
with legions of~\ldots \thi{normal people} participating in global evil.
(Unfortunately, we are dealing with \thi{normal people} even if we do not
subtract anything.) They are legions because to participate in this evil is so
easy, almost natural: just obey the orders, or do your job, or sometimes, do not
pay too much attention. Elaborating on this theme of the banality of evil,
Oksenberg Rorty observes that 
corruption \citet{can begin with the perception of injury or threat; or with a
  vision of what seems a tantalizing good. A society can become so pervasively
  corrupt that its members can typically fail to recognise their
  viciousness.}{ARorty}{\kilde{p.283}}

\pa\label{pa:originalSin} It seems that we thus encounter something which might
remind of~\ldots the original sin, or at least its possible variant.  We could
certainly refer to the above paragraphs to counter the unreserved claims of
Pelagian flavour that human can freely choose goodness which is in his power and
that sin (as a cause or result -- or both -- of evil) is the matter of every
individual. It is in one's power to \co{actually} decide, or rather only admit,
that one wants something which is good, that one \wo{thirsts for Something}, but
one's power ends about there.  \wo{They want good but effect evil, for they know
  not what they do.}  As desire for good is fully compatible with it, this
original sin is not an \woo{innate sinful depravity of the heart}{Edwards, I:1.1
  \citaft{EvilFaces}{ p.126,131}}, it is not a moral category appealing to
personal consciousness and making it \co{visible} \citet{that the soul of man,
  as it is by nature, is in a corrupt, fallen and ruined state,}{Edwards}{I:1.1;
  I:1.3} that his \citet{whole nature is a seed of sin; hence it can be only
  hateful and abhorrent to God.}{Calvin}{\citaft{EvilFaces}{ p.121}} We
certainly do not want to get all too puritan, revivalist, Calvinist or Lutheran.
In a strange way, it is the sin which anchors individual in the community, the
sin which is as if committed only by the community, and therefore one in which
everybody participates. It is impersonal evil which spreads among individuals
affected by even the slightest degree of despair or nihilism -- it grows and
effects results which might not have been intended by anybody, for which no
particular person \co{actually} carries full responsibility.  Yet, everybody is
responsible for it, so we might say that original sin is one which no individual
commits, but for which every individual is responsible.\ftnt{\label{resp}Thus
  this sin, at least in some sense, gathers (while evil divides). Also, it is
  not something intended and actually willed (while \citef{[t]here can be no sin
    that is not voluntary}{AugustTrueRel}{XIV:27}), and so the followers of
  St.~Thomas would call it a sin only in an analogous sense, only a shadow of
  sin properly so-called.  Participation in sin implies responsibility for it,
  and this is the whole and only sense of its \thi{voluntary character} in the
  present context -- for one is responsible for everything one participates in,
  even though no \co{actual} willing is involved and no \co{actual}
  responsibility can be imputed.
  
  In the older and more traditional societies, the same idea seems to have been
  present as the fear of pollution with correlative craving for ritual
  purification. Pollution results from \thi{unclean} actions but can also become
  infectious or hereditary. (Thus, for instance, there seems to be no signs of
  such infectious or hereditary transmission of \gre{miasma} in Homer, but in
  the Archaic Age it became both and was accepted as such to the Classical Age.
  Plato would still debar from religious or civic activities those who had
  voluntary contact with even slightly polluted person, until they have been
  purified (\btit{Laws}, 881DE).)\kilde{GreekIr,p.36,ftnt.43} Such involuntary,
  uncontrollable and almost mechanic workings of pollution suggest equally
  mechanic purification which develops from simple forms performed by laymen in
  Homer to advanced rituals of \gre{catharsis} in the Archaic Age.  }

Just like the lowest level of despair is characterised by the lack of any
knowledge thereof, so the one (that is, everybody) affected by the original sin
hardly ever realizes it, never meets its efficacy in the \co{actuality} of one's
consciousness, and consequently hardly ever confronts any \co{actual} choice related
to it. One may still live the ethos of one's parents, family, nation, one may
still be apparently active in all possible ways, yet one is already exposed to
the \co{unclarity} of values and \co{concrete} decisions.  One is not evil -- on
the contrary! Many of Hitler's willing (and unwilling) executioners were decent
citizens; good family fathers were tools of the most inhuman
evil.\ftnt{Dostoevsky notes in his Notebook: \wo{In fact, we were the
    nihilists, we in the constant search for a superior idea.}}


\co{Privative evil} spreads by not being recognised or not being opposed
actively enough. It spreads as we sit and as we walk -- \wo{Never
  forget:\lin we walk 
  on hell,\lin gazing at flowers.}\ftnt{Japanese \la{haiku} poet Issa
  (1763-1827) \citaft{ZenPoetry}{ p.108}} Studying mere psychology and analysing mere individuals will 
hardly ever give its concept except, perhaps, as a demonic force overcoming
individuals with irresistible power and taking possession of their \co{souls}.
For it is anonymous, it only sneaks between humans through their
\thi{misunderstandings} -- on the local, personal, or else on the social scale.
These two, apparently opposite poles (of personal and social interactions),
share the same element of unintentional, non-voluntary, we could almost say,
natural emergence of evil.  Its
germs may appear without anybody noticing. We do not know all the conditions
under which \co{thirst} becomes lack.  In the traditional language, the
\thi{natural} predisposition to the emergence of evil, the appearance of evil
between humans in spite of their natural \co{thirst} for God, was called
\wo{original sin}.  Perhaps, it is not a necessary feature of human nature, but
it seems to have been always present in human history; it seems to be so common
-- and so powerful -- that \co{dissociating} the two might be a futile exercise
of intellectual optimism.

\noo{ \kom impersonal, participated-in, unwilled -- Satan is active HERE (and
  only here?); only those looking for paradise get to hell (Bellero.., Milton
  298[17-20])

  creating idols \simu ako. separation/alienation
  
  \kom Evil is (for) the most unintentional; happens through me but not by me
  \kom What went wrong? alienation -- why? where from?
  
  \kom All that was at the personal level -- but this is evil of the clerks,
  evil of the impersonal machine of a concentration camp... Nihilism of this
  kind prepares ground for the most terrifying evil on the global scale (Woland
  actually helps the Master out of this shit...)
  
  -- Tischner: smierc podmiotu, bo to cale zlo (XX wieku) ``to nie ja'', ``to
  nie czlowiek'', ...
  
  \kom Pelagianism does not hold (in general, though in many cases it does) for
  evil `kan ogsaa sveve over oss' }


\subsection{Negative evil}\label{sub:negativum}
% -- \ref{sub:belial}
\pa Just like the transition from despair to the despair over despairing can be
hard to observe, so the transition from not knowing what to believe to not
believing anything can be imperceptible.  This most negative stage of nihilism
marks also a more personal level, for the experience of the all-embracing
emptiness, erasing all \co{idols}, acts also as a painful
\thi{individualisation} principle.  In terms of evil, it is its loneliness,
reflected by an internalisation: evil walks alone, it becomes introvert,
self-centered; it acquires the character of privacy which is only another
\co{expression} of the progressing \co{alienation}.
%-- the more it hurts, the more intense the defense. 

It should not be necessary to repeat all the stories and analyse the
differences between various forms of meaninglessness and boredom, nausea,
insensitive irritability, strangeness, remoteness and foreignness, etc. They have
been thoroughly enough studied and described by writers from H\"{o}lderlin and
Hebbel, through the  Russians like Turgieniev, Gogol, Dostoevsky, to
Sartre, Camus, Beckett\ldots
% (Nietzsche and Kierkegaard marking as if the mid-period...)
Personal disintegration of the heroes of this tradition is the evil of
\co{alienation} reflecting the metaphysical emptiness surrounding them and the
world in which they live. As there is hardly any \co{distinction} between
\co{me} and \co{my world}, the  emptiness of the latter results in
the dissolution of the former. And this is no paradox that emptiness causes
dissolution, nor that the indissoluble person disintegrates -- for dissolution
is exactly the \co{alienation} of \co{actuality} from the \co{self} and it is
effected by the emptiness which sneaks in between the two, which
\co{dissociates} heaven from earth and, eventually, announces nothingness of the former. 
What can be observed in \co{actual} situations is alienation from
the world, from the surroundings and other people, but these are only 
consequences, only \co{visible signs}. 

\label{pa:unreality} 
The common theme underlying this process in all the above mentioned (and others)
variants is the sense of unreality: first the \co{invisible} becomes unreal (for
only what is \co{visible} is real), then the world around \co{me}, losing all
sense and meaning becomes unreal, too, and finally, even \co{I myself}, \co{my}
whole {life} become unreal. This sickness to unreality is but another face of
the despair over one's own despairing. % (section \ref{sub:belial}).
It, too, is self-strengthening, for once started it can only spread until it
embraces the whole world. And once completed it, too, offers no exit, for having
embraced everything, having turned everything into unreality, it has left
nothing \thi{real outside}.\ftnt{\citeauthor*{NihilZ} describes 3 stages of
  post-Nietzschean nihilism with the last stage being characterised by the whole
  reality becoming unreal (albeit, in his case, as a consequence of the fall of
  the \co{idol} of collectivism which defines the second stage).  Rilke:
  \citef{I really did build my own house and everything that was in it. But it
    was an external reality and I did not live and expand with it. [...] it does
    not give me the feeling of reality, that sense of equal worth, that I so
    sorely need: to be a real person among real things.}{LouR}{Letter from Rome,
    1904, p.45} In this close association, if not \equin, of the sense of
  unreality and \co{alienation} -- from oneself as much as from the world and
  others -- \citefib{when others feel themselves understood and totally
    accepted, I feel prematurely torn from some sort of hidden
    place.}{LouR}{Letter from Oberneuland, 1905, p.60} Gombrowicz's works,
  starting in the 30-ies,\kilde{(Ferdydurke, The Marriage),} far from being nihilistic,
  give nevertheless an excellent description of this aspect of unreality, where
  nothing is itself any more, where even \wo{the Fear itself is but a Fear
    caused by the lack of Fear}, and where the only \citef{wish of my soul is:
    that something would Happen.}{TransAtl}{p.88/114}}


\pa\label{whoisevil} It may be unclear what is being called \wo{evil} here. Evil
seems to happen to the affected person who is suffering all these calamities
rather than participating in them, not to mention contributing to them. Is such
a person evil? Hardly. He seems even less evil than the ones
passively accepting it -- he only suffers it. Yet this suffering has a malicious
element of acceptance, just like the second stage of despair was only despairing
over one's own situation. He is in a grasp of evil which is much stronger,
deeper and more penetrating than the passive, \co{privative evil}.  The
\co{alienation} has progressed further, has reached a higher, that
is, deeper level.

But he himself does not cause any \co{alienation}, he does not spread evil! Or
so, at least, one would like to see it, believing that one is responsible
only for one's free and voluntary choices. Does he really not spread evil by
going around (or, for that matter, closing himself \thi{inside}) and being so
deeply affected by it?
%
Evil brought against oneself is in no way better than evil brought against
somebody else.\ftnt{Killing oneself is in no way \thi{better} than killing
  another. Not because it is worse, but because there is no sense in such a
  comparison across different persons. Whatever effects \co{alienation} is evil,
  and whatever is evil is so irrespectively of whom it affects. Degree of evil
  may be, \co{vaguely}, associated with the degree to which it \co{alienates},
  but then, again, irrespectively of the affected person. This may be, perhaps,
  referred to as the \thi{objectivity} of evil, though it is in every situation
  relative to the affected person and the way in which this person is being
  affected.}  OK, perhaps, but {\em he} did {\em not} bring it over himself, it
happened {\em to} him! He needs help, not accusations! But -- everybody may need
help and nobody needs accusations. We are not accusing anybody. And we admit
that it may be a tantalizing thought that one serves evil in the midst of
opposing it, in spite of all one's hopes and expectations; that one is
responsible for it only because {\em it} has chosen its site in one's soul; that
one is guilty by a strange accident, which accuses one of evil in the middle of
the fight one leads against it. Unfortunately, every fight witnesses to the
\co{presence} of some adversary. And when the adversary is in \co{me}\ldots?

Deep suffering, hopelessness, and despair over hopelessness, are ways of being
affected by evil and, at the same time, of answering for being so affected. The
one suffering is not, of course, evil but he is exposed to a trial in which evil
can easily enter his soul. Prolonged suffering can lead either to
\co{externalisation} of evil as some devilish power responsible for exposing one 
to it, or to recognition of one's own responsibility -- not, perhaps, for any
voluntary acts and evils but for one's imperfections. Either choice deepens
\co{alienation}. \citet{When a man truly Perceiveth and considereth himself, who
  and what he is, and findeth himself utterly vile and wicked, and unworthy of
  all the comfort and kindness that he hath received from God, or from the
  creatures, he falleth into such a deep abasement and despising of himself,
  that he thinketh himself unworthy that the earth should bear him, and it
  seemeth to him reasonable that all creatures in heaven and earth should rise
  up against him and avenge their Creator on him, and should punish and torment
  him; and that he were unworthy even of that. And it seemeth to him that he
  shall be eternally lost and damned, and a footstool to all the devils in hell,
  and that this is right and just and all too little compared to his sins which
  he so often and in so many ways hath committed against God his
  Creator.}{TheolGerm}{XI. The exclusive emphasis of this aspect would be a bit
  one-sided (it is probably one of the most \thi{Lutheran} passages in this work
  discovered by Luther), but the description applies well to other forms of this
  stage. Stavrogin, for instance, says in his final letter: \citef{I know that I
    should kill myself, erase myself from the surface of the earth as a harmful
    insect.}{Devils}{III:8.Epilogue\kilde{p.667}} } The legalistic or pietistic
bias of such self-depreciation, according to which suffering bears only witness
to the responsibility for some guilt, according to which perhaps every suffering
calls for responsibility and guilt, is indeed evil consuming one's \co{soul}.

\pa Although only half-personal, and hardly chosen (not to mention voluntary
choice), this form of being partakes of evil. And to participate in evil is to
be affected, even consumed, by it (even if we won't say that it is to {\em be}
evil). This is a higher, more spiritual level of being a victim, perhaps, a
victim of plain violence. Having been exposed to an act of violence does not, by
itself, make one evil. But it poses before one a lot of choices which open and
strengthen the possibility of saying \No: the arising will to settle the
accounts, perhaps the conviction about the right to exercise unmitigated revenge,
perhaps to nourish hatred, etc..
%
Even more may be going on behind the scene, in the \co{invisible world}. The
\co{alienating} power of the sickness to unreality, not to mention exposure to
suffering, lies in strengthening the tendency to pollute with the appearance of
evil not only particular situations and events but the world -- \thi{the whole
  world}, the \co{quality} of \thi{my whole life}, of \thi{human life as such}
can begin appearing as evil or as originating in evil. And the step from seeing
evil around oneself to choosing it is even smaller than that from pitying the
fate of $X$ to hating $Y$ who caused it.
% I can still suffer evil around me without becoming so...

\noo{

\pa
The more we approach the spiritual dimension, the less possible become
distinctions like active vs. reactive, \co{subjective} vs. \co{objective},
\co{mine} vs. not-\co{mine}. 

positivism leads to it....
}

\noo{
\kom [Nihilism here: not believing anything = no thing]

\kom  half-personal, alienation (from One: meaningless, boredom, unreality; from
 others: absolute foreign/stranger; abstract transc. without imman; divides),

\kom  unbearable, impossibility of exit,

\kom eternal `repetition' -- self-strengthening!
}


\subsection{Active evil}
% (Evil be thou my good)
% -- \ref{sub:mammon}
\pa
As we have learnt from Nietzsche, there is a difference between not believing
anything and believing nothing.
%
One struck by the \co{negative evil} lives the fact that there is nothing, no God,
no sense, that questions about meaning are not only unanswerable but
ultimately empty\ldots But not finding anything, he may still resist the
conclusion 
that there is nothing. It is, indeed, possible to balance on the edge of this
apparent contradiction, but it is very difficult. In fact, the more intensity in the
attempts to retain the balance, the stronger the force dragging one towards the
conclusion which only confirms the actual situation -- there is nothing
\thi{outside}, only emptiness, void. 

As Dostoevsky observed, if there is no God, then everything is
allowed.\ftnt{\citef{[I]f you have no God what is the meaning of
    crime?}{Karam}{II:Bk.VI:Ch.3 [Father Zossima]} \citef{[S]ince there is
    anyway no God and no immortality, the new man may well become the man-god,
    even if he is the only one in the whole world, and promoted to his new
    position, he may lightheartedly overstep all the barriers of the old
    morality of the old slaveman, if necessary.}{Karam}{IV:Bk.XI:Ch.9 [Ivan's
    Nightmare]}}
%
The lived emptiness breeds nihilism all the way down -- nihilism, that is, the
lack of any criteria, the total \thi{freedom}, \thi{freedom from\ldots} or, what
amounts here to the same, meaninglessness (for since every
meaning carries with it a \thi{threat} of external authority, total \thi{freedom
  from\ldots} can appear only as arbitrariness.) 
And the more devastating consequences
it has in the lower, eventually even \co{visible} sphere, the greater the chance
for the conclusion that, actually, there is nothing and that one should draw
some consequences of this \thi{fact}.  It is impossible to exit for there is
only emptiness \thi{outside}; so \co{I} must obviously stay here, in the
middle of this emptiness, but \co{I} can, for that matter, \co{act} -- true,
towards things and situations which became immersed in emptiness, which lost all
their significance and importance, but which still offer all the \co{visible}
material for \co{action}. 

\pa
\wo{Naught} can mean both \thi{nothing} and \thi{evil} (as in \wo{naughty}).
Tradition associating evil with negativity utilised also the distinction between
mere lack, privation, and  negation. The former, {privation} could be read as
negation of something particular, of an individual thing, of this or that, and
then negativity would correspond to a total emptiness, negation of
Being as such. We have changed slightly the sense of \co{privation}, but retained
\co{negativity} more or less in this form.
In either case, it was difficult to see any positive activity in evil, since it
was merely an ontological lack, a pure non-being. It was a non-substantial
negation, incapable of any action emptiness.  \citet{For evil is the absence of
  the good [...]  But only good can be a cause [...]}
%because nothing can be a cause except inasmuch as it is a being, and every
%being, as such, is good.}
{SumTh}{I:q49.a1}

Eventually, we will perhaps follow this tradition but we should not, for this
reason, forget the active character of evil.  Emptiness is not necessarily
physical annihilation and destruction, but \co{alienation}, the spiritual
emptiness of heavens, \co{nothingness} which became void.  The \No\ which,
declaring this ultimate emptiness, sells its soul, expects some
\co{visible} cash in exchange.  Evil has a tendency -- not to
say, power -- to grow and spread. It does not help calling it \wo{negation},
\wo{lack} or \wo{emptiness}, because these may seem empty and inactive only when
taken abstractly. But evil acts {\em in} human soul, consumes and corrodes it.
Although this corrosion may be viewed as a gradual negation, in fact it is a
deterioration and increasing \co{alienation}. The fact that the final result is
negative emptiness, does not mean that the process which led to it was equally
empty and non-existent.  Evil becomes \co{active} when it
reaches the \sch\ of \No, when it declares that there is nothing \thi{outside},
and when having thus annihilated the sphere of \co{invisible}, it turns to what
is left -- \co{acting} in the \co{visible world} to compensate for, or else to
revenge, the \co{invisible} loss \co{above}.  This evil is not only capable of
action, but could be almost defined by it, as opposed to the two kinds we have
considered before.  \wo{Evil, be thou my good: by thee at least\lin Divided empire
  with heaven's king I hold}$^{\ref{ftnt:evilBeGood}}$ -- these are words of a resolute (even if defeated
and resigned) being determined for action.  This is resoluteness of the ultimate
resignation.
%{IV:110-111}
Accepting the defeat and impossibility of reconquering paradise, it turns away
from the emptiness and directs its activity to all, and only, \co{visible
  world}.  \la{Satan sum et nihil humanum a me alienum puto.}\ftnt{\citef{I am
    Satan, and deem nothing human alien to me.}{Karam}{IV:Bk.XI:Ch.9 [Ivan's
    Nightmare]}\label{ftnt:satanSum}} \co{Active evil}, the evil of \co{active} \No, apparently
leaves the passivity of negation (to which it was merely exposed) and decides to
\co{act}, to take its damned fate into its own hands, and turn it into whatever
it chooses, that is, into whatever it is able to. For \woo{to be weak is
  miserable}{Milton, I:157} while \citet{To reign is worth ambition, though in
  hell:\lin Better to reign in hell than serve in heaven.}{Milton}{I:157; 262-3}

\pa This decision is not, at least not primarily, any
\co{actual} decision. It is of the same order as the \co{spiritual choice}, an
\co{invisible} event which starts plaguing the \co{soul}. \co{Actual} decisions,
like all the psychology of evil, are only \co{reflections} of the \co{invisible}
\No\ which, in turn, is  \co{thirst} culminated in the resigned defeat.

Now, there is no question about the {\em psychological} possibility of being
motivated and attracted by some evil. Such a possibility may be inexplicable for
the psychology identifying 
the \thi{good} with the \thi{desired}, but it certainly obtains. \co{I} can be
attracted to some evil not only because \wo{there is a certain
  show of beauty in sin}, but also because \co{I} desire it for its evil's 
own sake. Augustine recollects: \citet{The malice
  of the act was base and I loved it -- that is to say I loved my own undoing, I
loved the evil in me -- not the thing for which I did the evil, simply the
evil.}{AugustConf}{II:4 [More recently, similar point is made for instance in
\citeauthor*{EvilOwn}.]} 
%But the essay is only what it announces to be: a bit of psychology. It
But does it mean that our \co{activities}, our whole lives, do not, after all,
go on \la{sub specie boni}?\noo{Discussing detailed \thi{goods} and exemplifying
attractive force of detailed \thi{evils} may merit descriptive correctness. But}
Behind every \co{actually} willed evil there hides a \co{non-actual} motivation,
as one used to say, a disposition of the \co{soul}. And just like \co{soul} may
consent to something it 
does not want,\ftnt{\citef{There are also people who entirely regret being drawn
    into consenting to lust or into an evil will, and are compelled by the
    flesh's weakness to want what they don't {\em want} to want. Therefore, I
    really do not see how this consent that we don't want is going to be called
    \wo{voluntary} [...]}{AbelardEthics}{\para 33-34}} so man may live \la{sub
  specie boni} and yet choose evil, 
even evil for evil's sake.  \citet{For he certainly desires to be blessed even by
  not living so that he may be blessed. And what is a lie if this desire be not?
  Wherefore it is not without meaning said that all sin is a lie. For no sin is
  committed save by that desire or will by which we desire that it be well with
  us, and shrink from it being ill with us. That, therefore, is a lie which we
  do in order that it may be well with us, but which makes us more miserable
  than we were.}{CityGod}{XIV:4} Preferring and choosing
evil for evil's sake is possible at the level of \co{actual} will, but it is a
result, a \co{reflection} of both the \co{original thirst} and its
misunderstanding, of the \co{invisible No}.  It is a sickness of the \co{soul}
and the whole issue lies 
here and not in the defections of the \co{actual} choices, of their \co{actual}
goals or motives -- for will's \citetib{defections are not to evil things, but are
  themselves evil.}{CityGod}{XII:8} The \citet{people
with sick souls crave and love the bad character traits and hate the good way.
They are careless about following it, and it is very difficult for them,
depending upon the extent of their illness.}{MaimoTraits}{\citaft{EvilFaces}{
  X}\kilde{p.71}} 
The illness, moreover, is very
peculiar for the affected person will seldom, if ever, have any intimations of
it, and the more serious it becomes, the less capable the person is to admit it.
%In short, it is not a matter of psychology. 

\pa History knows many examples, people like Nero, Gilles de
Rais,\ftnt{\citeauthor*{BatailleRais}} Billy the Kid, Marquis de Sade, whom
we would like to classify as pathological cases, assign them appropriate labels
and shut in a cabinet with horrible curiosities.  But they provide examples of
spiritual deterioration which, irrespectively of their actual causes and
context, reveal an inherent possibility of human \co{existence}.  And every single
example of a human being illustrates the potential of being human, the potential
which can find its expression also in other humans, of which, at least in
principle, every human being is capable.\ftnt{\citef{[A]ll physical, psychical,
    and vital forces and organs that are possessed by one individual are found
    also in the other individuals. [...] There is no difference between
    individuals of a species in the due course of Nature; the difference
    originates in the various dispositions of their
    substances.}{Perplex}{III:12. An alternative
    translation has \wo{matter} instead of \wo{substances}.}
  \noo{All natural, psychic, and animal faculties and all the parts that are
    found in one particular individual are also found, as far as essence is
    concerned, in another [...] There in no way exists a relation of superiority
    and inferiority between individuals conforming to the course of nature
    except that which follows necessarily from the differences in the
    disposition of the various kinds of matter [...] - Moses Maimonides, The
    Guide to the Perplexed [The Many Faces of Evil, p.69]}}
%
Intelligence of de Sade makes his texts express with particular clarity most
points we are making. In particular, he declares his choice of \No\ with
exceptionally self-conscious determination.  We will follow him for a moment,
but only in order to arrive again to the point that \co{actual} wanting of evil
is only a \co{reflection} of the inability to want anything else.

\citet{[I]t is not the object of libertine intentions which fire us, but the idea
  of evil, and [...] the greatest pleasure is derived from the most infamous
  source.}{SadeSodom}{The eight day\kilde{p.139}}  Or,
the same fascination with transgression, expressed in a slightly different way:
  \citetib{beauty belongs to the sphere of the simple, the ordinary, whilst
  ugliness is something extraordinary, and there is no question that every
  ardent imagination prefers in lubricity the extraordinary to the
  commonplace.}{SadeSodom}{Introduction\kilde{p.34}} Such an \thi{ardent
  imagination} finds its inspiration in the low and the ugly, in their
variation and manifold which, negating everything \co{above} and \thi{outside}
the horizon of their plain \co{visibility}, can only attempt to intensify its
narrow contents. \citetib{It is the filthy act that causes the greatest
  pleasure: and the filthier it be, the more voluptuously fuck is
  shed. [...] the more pleasure you seek in the depths of crime, the more
  frightful the crime must be.}{SadeSodom}{The seventh day-The eight
  day\kilde{p.128;140}} 
\noo{
\woo{What would pleasure be if it were not accompanied by crime?  It is not the
  object of debauchery that excites us, rather the idea of evil.}{SadeSodom} Or,
the same fascination with transgression, expressed in a different way:
\woo{Beauty is a simple thing; ugliness is the exceptional thing.  And fiery
  imaginations, no doubt, always prefer the extraordinary thing to the simple
  thing.}{SadeSodom} It finds its inspiration in the low and the ugly, in their
variation and manifold which, negating everything \co{above} and \thi{outside}
the horizon of their plain \co{visibility}, can only attempt to intensify its
narrow contents.  \citet{If it is the dirty element that gives pleasure to the
  act of lust, then the dirtier it is, the more pleasurable it is bound to
  be.}{Sade}{120 Days of Sodom}
}

As sadism became a label for something one might even be willing to call a
particular \wo{sickness}, one might also be less willing to consider it evil.
Such labels serve the general tendency of reliving the conscience and ensuring
everybody that it was not his fault. But if evil happens to be nobody's fault,
it only means that {\em everybody} is guilty. In case of \wo{sick} people, like
de Sade, there should be little doubt. Sickness is not necessarily evil nor is
it necessarily making one evil -- but as every pain and suffering it can do 
both. (Pain, sickness, deformity, as natural associates of evil are consistently
 symbolised by all hunchbacks, deformed sorcerers and
ugly witches in fairy tales.)  However, if one acts evil, it does not matter
much whether it is because of some experienced pain, sickness, unhappy
childhood, or whatever. As there are no sufficient reasons, no amount of
negative experience ever justifies evil. Usually, it functions only as a better
or worse excuse.

Actual evil is evil, whether the person causing it had happy or unhappy
childhood, whether he is healthy or sick, whether he suffered much or not. Evil
is the impossibility of justification, therefore it always looks
for excuses. Let us follow de Sade a little bit more. 

\pa \citet{Certain souls seem hard because they are capable of strong feelings,
  and they sometimes go to rather extreme lengths; their apparent unconcern and
  cruelty are but ways, known only to themselves, of feeling more strongly than
  others.}{Sade}{Last Will and Testament} Strength and intensity of feelings,
  feelings which one 
experiences and not manufactures oneself, work as a sufficient excuse for de
Sade, in fact, as the highest good itself.  Following such impulses is only to
follow the nature.  \citet{We are no guiltier in following the primitive
  impulses that govern us than is the Nile for her floods or the sea for her
  waves.}{Sade}{Aline and Valcour\label{cit:Nile}}

Felt intensity is always concentrated, narrowed to the ultimately \co{actual},
\co{immediate}. One searches for moments, moments of sensation which could fill
one with the stimulating experience. These, at least, seem to offer something
capable to overcome the emptiness, to leave it behind at the negative stage.
Stavrogin still complains \citet{Here I liked to live least. But even here I was
  unable to hate anything. [...] I may desire to make a good act and it causes
  me pleasure. But just in a moment I desire an evil one and feel equal
  pleasure. Both this and that feeling is as always too flat, and I never desire
  strongly.}{Devils}{III:8.Epilogue} De Sade has his answer: \citet{True
  felicity lies only in the senses, and virtue gratifies none of
  them.}{Sade}{Aline and Valcour} It is the intensity of a momentaneous
sensation which appears 
as the most gratifying, the most true element of experience.
%to which one dedicates the whole attention.
And the highest intensity can be found in pain.  \citet{[W]e are much more
  keenly affected by pain than by pleasure: reverberations which result in us
  when the sensation of pain is produced in others will essentially be of a more
  vigorous character, more incisive, will more energetically resound in us [...]
  hence pain must be preferred, for pain's telling effects cannot deceive, and
  its vibrations are more powerful.}{SadeBed}{Dialogue the Third. 
  (Masochism might be
  here considered only a variation on the same theme as sadism.
\thi{Morally}, perhaps, more acceptable than the latter, it expresses the same
  desperate yearning for irrefutable \co{immediacy} of \thi{truth}, pain or
  pleasure, the same deterioration of the \co{soul}. Masochism and sadism affect
  often the same person and vary only depending on the \thi{balance of power} with
  the actual partner.) \noo{Inflicting pain on
  oneself is as evil as inflicting it on somebody else.}}  \wo{Pain, be thou my
  good} is but another version of the motto we have extracted from Milton's
Satan.

\pa \citet{My manner of thinking, so you say, cannot be approved.  Do you
  suppose I care?  A poor fool indeed is he who adopts a manner of thinking for
  others!  My manner of thinking stems straight from my considered reflections;
  it holds with my existence, with the way I am made.  It is not in my power to
  alter it; and were it, I'd not do so.}{Sade}{A letter to his wife.} Strangely
enough, this might almost sound plausible~\ldots but not when spoken by {\em
  this} person! For although it might express a respectable strength, so in this
  case one can discern behind these words the deep
loneliness of an \co{alienated} individual.  \citet{All creatures are born isolated
  and have no need of one another.}{Sade}{Aline and Valcour.
  \label{ftnt:alienation}Variations on the theme of alienation, of \citefi{all
    seek[ing] their own}{Phil.}{II:21} are, of course, all too numerous to allow
  any review. Typically, they have a strongly sociological flavour -- the
  individual being alienated from the society, usually, due to dysfunctional
  social mechanisms. Ignoring the fact
  that even such an alienation is but an individual experience and thus,
  ultimately, an alienation from the \co{self}, and looking for reasons and
  excuses in the (often deplorable) social surroundings, they differ completely
  from our notion in that they hardly address an individual but a social process
  and hence, at most, a schematic exemplar of a member of a species or, as in
  the historically most successful variant, of a social class.
% Let us only notice one (without, of course, subscribing to the associated theory
% of alienation).
% Accusing the industrial towns for the most shameless immorality in the oppression
% of proletariat, Engels recognises there the same signs of evil: people
% \citef{crowded by one another as though they had nothing in common, nothing to
%   do with one another, and their only agreement is the tacit one, that each keep
%   to his own side of the pavement, so as not to delay the opposing streams of
%   the crowd, while it occurs to no man to honour another with so much as a
%   glance. The brutal indifference, the unfeeling isolation of each in his
%   private interest, [the] dissolution of mankind into monads, of which each one
%   has a separate essence, and a separate purpose, the world of atoms, is here
%   carried out to its utmost extreme.}{EngelsCond}{IV.The great towns}.
Sociological observations underlie also a more recent version which, however,
does not quite manage to hide its personal background.  We are
\citef{discontinuous beings, individuals who perish in isolation in the midst of
  an incomprehensible adventure, but [who] yearn for our lost
  continuity.}{BatailleErotism}{p.15}\noo{York, J.D., Flesh and Consciousness,
  JCRT 4.3,2003} The lost continuity can be regained -- so goes at least (some)
dialectical saying -- by a deification of otherness-as-such; deification which seems
particularly tempting for disappointed intellectuals and which can equally well
serve as a \thi{natural} explanation -- justification! -- of 
coprophilia, necrophilia and other personal deviations. \citef{As soon as the
  effort at rational 
  comprehension ends in contradiction, the practice of intellectual scatology
  [the science of excrement] requires the excretion of unassimilable elements,
  which is another way of stating vulgarly that a burst of laughter is the only
  imaginable and definitively terminal result -- and not the means -- of
  philosophical speculation. And then one must indicate that a reaction as
  insignificant as a burst of laughter derives from the extremely vague and
  distant character of the intellectual domain, and that it suffices to go from
  a speculation resting on abstract facts to a practice whose mechanism is not
  different, but which immediately reaches concrete heterogeneity, in order to
  arrive at ecstatic trances and orgasm. [...] To the extent that man no longer
  thinks of crushing his comrades under the yoke of morality, he acquires the
  capacity to link overtly not only his intellect and his virtue but his
  \fre{raison d'etre} to the violence and incongruity of his excretory organs,
  as well as to his ability to become excited and entranced by heterogeneous
  elements, commonly starting in debauchery.}{BatailleSade}{\para 12-13.}}

\co{Alienation} is a break in continuity -- first, continuity between
\co{actuality} and its \co{origin}, then continuity with others and the world
and, finally, continuity of time, of this moment with other moments.  Having
turned away from the \co{invisible}, there is only one possibility: to embrace
and conquer the \co{visible}. \citet{What is remote is no longer important, only
  yesterday; and tomorrow is more than eternity.}{LouR}{Rilke's letter from
  Oberneuland-bei-Bremen, July 25, 1903, p.44} But without the continuity with
the \co{origin}, \co{this world} shrinks and begins to disappear, becomes first
mere \co{actuality}, \co{more actuality}, even \co{more}, until it reaches the
limits of \co{immediacy}, and threatens with disappearance in emptiness.  A
moment devoid of the element of eternity becomes a desperate expectation of the
next moment.  Intensity is, we could say, a noetic counterpart of such a
noematum \thi{moment}.  Intensity tries to dissolve in this noematic correlate,
and failing -- tries again.  (Thirst for strong, intense feelings is a form of
nihilism, too, even if it affects mostly adolescent girls and disappointed
women. It is, however, also an expression of the very common reduction of the
values of human being to \co{subjectivity} -- feelings, personal
\co{experiences}, private choices\ldots)
The intensity searched for is also the impulsivity emanated, and as the moments
become more and more intense, they also fall apart, each becomes its own
universe of intensity collapsing inward, and giving rise to an impulse arising
from nowhere.  Acting from an impulse has often been associated not only with
unreasonable lack of control but with evil -- evil which surprises, is
unpredictable, emerges suddenly. The word \wo{impulse} carries the meaning of
\thi{application of sudden force} but also, even if only secondarily, of a
\thi{suggestion coming from an evil spirit}. For \citet{evil is
  unstable}{DivNames}{IV:23.\kilde{p.121} \citef{We shall be reminded that the
    Vicious Soul is unstable, swept along from every ill to every other, quickly
    stirred by appetites, headlong to anger, as hasty to compromises, yielding
    at once to obscure imaginations, as weak, in fact, as the weakest thing made
    by man or nature, blown about by every breeze, burned away by every
    heat.}{Plotinus}{I:8.11}} or, as Kierkegaard says, \wo{the demoniac is the
  sudden}. Demons wake up not when reason is asleep but when it has nothing
higher to strive for, when it loses itself in the insatiability of \co{more} and
\co{more} -- \co{precise}, momentaneous, intense, ecstatic\ldots


\pa \co{Active evil} is the most personal -- in the sense,
\thi{individualised} -- form of evil. This \thi{individualisation}, however, has
nothing to do with the uniqueness of individual \co{existence}, only with the
transgressive and \co{alienating} \co{actuality} of its \co{acts} and \co{activities}. It
\thi{individualises} by losing all individuality, it grants \co{actual}
moments, \co{visible} sand corns in exchange for the \co{invisible} reality and
\co{origin} of its sense. It \thi{individualises} by erasing the person of its
carrier -- a sinner becomes eventually nothing but a {substantialisation} of evil,
the place where its negativity unfolds its active \co{presence}. 
This \thi{individualisation} through \co{actual} choices, distractions and
intensity, apparently filling and overfilling every moment, is only an
\co{inverted} form of the underlying emptiness and, at the bottom, only a  
perverted form of the genuine \co{existential} possibility. 

Consciousness, as before, has little to do with any significant aspects.  One
hardly ever says \co{reflectively} and explicitly \wo{Evil, be thou my good}.
At most, one can \co{actually} repeat this after it has been said in the depth
of one's soul; after long time, when most of the consequences became
\co{visible} and one finds it impossible not to accept them. Here, as elsewhere,
the \co{actual subject}, the \co{reflective} consciousness becomes affected by
the \co{invisible}. It loses all meaning of its contents, and then also all the
contents, except, perhaps, for the most \co{visible} elements of natural
necessities. Eventually, consciousness, too, dissolves in the emptiness which
has eroded the whole being.  \co{Active evil} does not any longer try to
escape this unfortunate, undeserved, unjust fate -- living this inability it
only seeks to avenge itself and through revenge \co{alienates} itself only more
and more.



\subsection{Impersonally personal}
% As all schemata, also the three levels of evil give only an approximation; one
% can always refine typologies and multiply distinctions.  We wanted only to
% indicate what seems to be the central aspects of the \co{nexus} of evil.
Let us gather these levels of gradual growth of evil in some common points. 

\wtsep{original sin}

\pa\label{pa:originalSinII}
We said \wo{believing in something}, \wo{believing in nothing}. Such
formulations  did not concern, at least not necessarily, any
\co{actual}, reflectively pronounced beliefs. As usual, with respect to the
\co{invisible origins}, the \co{reflective}
consciousness does not matter much -- it only registers by \co{dissociating},
\co{reflects} something which is there already.
Evil is never willed for its own sake; even if extreme suffering perverts one's
soul to the point of accepting evil, it is still only a helpless reaction, or as
the case may be, hope that at least {\em this} will be some good.

It arrives unwilled, from unregistered and unrecognised meanings of one's acts
and words, from \thi{misunderstandings} which prove hurtful, from depersonalised
rigidity of humans turned clerks, from impersonal heights of socio-political
system, from God knows where\ldots It is born between men (sometimes between man
and \thi{natural evils}), but it is born {\em into} them and once born, it can
grow. An individual may attempt, 
and often even succeed, to check his evil predispositions, but he is never
guaranteed that evil emptiness won't ever affect him. On a larger, social scale,
emergence of evil is simply unavoidable, and the only thing we can try to do is
to moderate its strength and scope. The unintentionality of evil, underlying
these unpleasant observations, was termed the \wo{original
  sin}.\ftnt{\thi{Original sin} is not meant to {\em explain} the appearance of
  unintended evil (we do not explain anything) -- it is only a term of description.}

\wtsep{only human}

\pa The unintentional and impersonal (as one also might say, generational and
collective rather than individual) character of the original sin does not in any
way abolish human participation in it. \thi{Killing the subject}, dispensing
with it (in the name of impersonal forces of power, capital, social mechanisms or in
whatever name one chooses), does not make any amends, does not improve
understanding of evil -- it only apparently relieves the participants from the
sense of guilt and responsibility.

Evil is an event of human life and without humans there would be no evil. We
have ignored all kinds of \thi{natural evils}, usually distinguished from
\thi{moral evil}.  We are probably not justified in ignoring such an important
aspect. But it does not appear as any problem, nor even a relevant point, to our
considerations. Many things happen due to natural processes, and some of them
have more or less devastating consequences for some humans. We would not only
distinguish it as \thi{natural evil}, but would not call it \wo{evil} at all.
For evil calls for a justification (which is impossible), while a storm or an
earthquake do not -- \citet{catastrophes are innocent.}{HerbertAnts}{Cleomedes}
It is either the infantile idea of some omnipotent Being with good will and
\co{actual} intentions, or else some idea of ultimate {objectified} meaning,
eventual \gre{telos}, which might suggest one to look for any \thi{explanation}
of this platitude.  Natural disasters, like diseases, call for strength to put
up with them when nothing can be done, and for inventiveness in preventing them.
Calling them \wo{evil} is like getting offended on the world for not pleasing
us.

Evil is maintained only in humans, it requires a human, though this might appear
as a mere consequence of its understanding as \co{alienation} from the
\co{origin}.  As this is, in principle, possible for every \co{existence}, we
could probably suggest that it constitutes the \la{differentia specifica} of the
human species: \thi{to be capable of evil}.  There is a strong tendency to see,
eventually, only innocence in all the cruelty of the living nature; a predator,
an animal killer is not evil -- how could it be?! -- it is a survivor. It is
impossible, or in any case naive, to transfer any such observations to the world
of humans.

\wtsep{yet, impersonal, invading}

\pa\label{evilimpersonal}\label{evillonely} Although we do not place evil in
nature but only in humans, it does not mean that it becomes human. It remains
impersonal, because human being is only a tool of evil. It invades one, often without
much warning, without giving any account. The \co{subject}, \co{I myself} am
affected by something which has found place \co{above me}, in \co{my self}. And
because this force which found its site there is adversary -- which is
experienced as various degrees of \co{alienation} -- it is foreign.  Whether an
evil social or political system, whether Woland who never argues but only
commands, or else Mephistopheles who appears {in person} to discuss with Faust,
the force which is brought forth is not \co{mine}, does not belong to \co{me}
nor, for that matter, to \co{my self}.  It is impersonal, because its strength
does not flow from the \co{original} site of personality but, on the contrary,
prevents \co{me} from regaining this site. It \thi{individualises} by breaking
continuity, by \co{dissociating} heaven from earth and thus \co{alienating} and
isolating.  Loneliness, like foreignness, is an \co{actual} image of
\co{alienation} which follows evil even in the midst of the thickest crowd. This
loneliness is \co{alienation} from the \co{origin}, the loss of the personal
center, and thus the opposite of the unique individuality of \co{existence}.
Participation in evil, submission to this impersonal force, amounts indeed to
\thi{selling one's \co{soul}}.

If we were to personify such events, we would ascribe evil the {\em intention}
of becoming \co{visible} (its surrogate for \co{concreteness}), which it
achieves by invading a human soul and, through it, overcoming its own impersonal
abstractness. A person of active \No, but also a sinner who approaches the
deeper layers of emptiness, is a {substantialisation} of evil; a
{substantialisation} which proceeds gradually, as the emptiness embraces the
soul and finds its \co{visible} expressions, but which never becomes complete,
which never reaches the goal. For the goal would be, as it always is, to reach
the very center of Being, to achieve the ultimate justification by meeting
the \co{origin} -- this, however, is the exact opposite of evil.  Evil, although
originating in \co{this world}, begins without being noticed, and then
penetrates the soul to its deep, \co{invisible} roots. However, its ultimate,
\co{invisible} site is not in the center of Being, in the \co{origin}. For it is
exactly \co{alienation} from the \co{origin}, \co{alienation} of \co{myself}
from \co{self}, which is evil.
%So, it can only stay a step \co{below}, maintaining its destructive activity
Thus, \citet{evil always lessens good, yet it never wholly consumes it; and thus,
  while good ever remains, nothing can be wholly and perfectly bad. Therefore,
  the Philosopher says (Ethic. iv, 5) that "if the wholly evil could be, it
  would destroy itself"; because all good being destroyed (which it need be for
  something to be wholly evil), evil itself would be taken away since its
  subject is good.}{SumTh}{I:q49.a3}

\wtsep{impossibility of justification}

\pa\label{pa:noJustification} The impossibility of reaching the \co{absolute},
the lack of \co{concrete founding} in the \co{origin} is
\co{alienation}.\ftnt{We are not, of course, speaking about any \co{actual}
  grasp on the \co{absolute} which is missing. (An \co{aspect} of evil is to
  both deny the \co{absolute} and identify it with something \co{visible}.)
  \wo{Reaching the \co{absolute}} refers to the continuity of being,
  \co{existential openness} to the \co{other world founding} steadiness in
  \co{this one}. It has nothing to do with epistemic, nor other \co{actual},
  pretensions. We will return to this aspect.}  Allow us to call\noo{(for the
  reasons to be clarified later)} such a continuity of Being, the \co{concrete
  founding} in the \co{origin}, \wo{\co{justification}}. Then evil as
\co{alienation} is exactly the lack and impossibility of \co{justification}; it
is what can not be \co{justified}.\ftnt{\citeauthor*{Nabert}.\noo{after
    Tischner, Spor o istnienie czlowieka, p.36} \noo{The theological plausibility of
  this use of the name here may certainly be disputed and is not claimed.}}

This impossibility finds its expression in the exclusive directedness towards
the \co{visible world}, which becomes the sole source of motivation and
explanation -- the substitutes of \co{justification}.  Soul infected by evil
keeps trying to fill the expanding emptiness; it keeps \co{thirsting} for
\co{justification}, for \wo{the desire for the bliss, which she had lost,
  remained with her even after the Fall.}$^{\ref{ftnt:afterFall}}$
\co{Justification}, however, can only come from \co{above}, while evil -- seeing
\co{nothing} \co{above} and declaring it to be void -- must produce it itself.
All it is capable of producing are arguments supposedly explaining the
attitudes it develops and actions it performs; explaining, that is,
demonstrating that this is actually right, natural, or even necessary, thing to
do.  \wo{We are no guiltier in following the privative impulses that govern us
  than is the Nile for her floods or the sea for her waves.}
% used earlier {Marquis de Sade, Aline and Valcour}
Being the lack of \co{justification}, evil always tries to explain, that is,
excuse itself.  

A common form of such attempts at self-justification is to point to the actual
evils in the world.
%In a sense, this corresponds to what we are saying (, but the
%difference appears when one starts calling things \wo{evil} and, above all, when
%such things serve as justification of one's own evil tendencies.
This form may display rare ingenuity in the search for evil, so that eventually
nothing remains which would not appear affected by it.  But a
\citeti{good man out of the good treasure of the heart bringeth forth good
  things: and an evil man out of the evil treasure bringeth forth evil
  things.}{Mt. XII:35; Lk.VI:45}{} Using some evil as a justification of
anything is dubious, if not directly dangerous, for a step from there to
justification of more evil is invisibly small.  A rigid moralist defends the
world against all evil which lurks behind people's back, without them noticing
it.  As a matter of fact, he only tries to defend himself against his
progressing \co{alienation}. But finding evil in all
corners of the universe and 
human soul, this defense only strengthens the adversary; it multiplies evil
instead of diminishing it -- multiplies it at least in the soul which sees it
everywhere and forgets \citeti{that whatsoever thing from without entereth into
  the man, it cannot defile him; Because it entereth not into his heart, but
  into the belly, and goeth out into the draught, purging all meats?}{Mk.}{VII:18-19} A profound moralist, whether of a revolutionary or pietistic flavour, will
often turn out faultlessly cynical in his \co{actions}, which no longer aim at
the person but only at the evil hiding behind. And thus, \citet{[i]n morality,
  man treats himself not as an <<individuum>> but as <<dividuum>>.}{HATH}{II:57}

Evil used as a justification only increases the need for more evil and when it
does not find it, it produces it.  \citeti{That which cometh out of the man, that
  defileth the man.  For from within, out of the heart of men, proceed evil
  thoughts, adulteries, fornications, murders, Thefts, covetousness, wickedness,
  deceit, lasciviousness, an evil eye, blasphemy, pride, foolishness: All these
  evil things come from within, and defile the man.}{Mk.}{VII:21-23} Eventually,
\woo{people who are completely debased find pleasure exclusively in other
  people's unhappiness.}{Goethe, Blad tkwi tylko w czlowieku,p.31} The ways in
which evil spreads are innumerable, so we only notice its general tendency to
expansion, if not to self-strengthening, which begins with \thi{seeing evil
  around} and serves as the means of supposed self-justification.


% Recalling the earthquake in Lisboa, and other disasters meeting Panglos and
% Candide, were meant to ridicule Leibniz's idea of the best possible world. 

% As such impossibility, it looks for explanations/rationalisations (does the need
% to explain, prove, necessiate come only from here...?)

\wtsep{evil is active, it acts}
\pa
This ability to expand and pollute all \co{visible world} around, to
universalize itself in the impossible search for \co{justification}, 
characterises the activity of evil.
% As a possible objection to the thesis of negativity of evil, St.~Thomas considers
% the syllogism: \citt{what is not, acts not. But evil acts, for it corrupts good.
%   Therefore evil is a being and a nature.}{Aquinas, Summa Theologiae,
%   I:q48.a1.obj.4} We will return to the issue of being or non-being of evil in a
% later section. For the moment we only want to consider its activity which, we
% must admit, seems to us incontestable.
Now, the long tradition (going at least back to Origen and Plotinus,
even Plato) used to deny being and, consequently, also any activity to
evil.\verify{}
%
\noo{Check, Plotinus, II:9.13 [``Against the Gnostics'']
    Once more, we have no right to ask that all men shall be good,
or to rush into censure because such universal virtue is not possible:
this would be repeating the error of confusing our sphere with the
Supreme and treating evil as a nearly negligible failure in wisdom- as
good lessened and dwindling continuously, a continuous fading out; -- another
translation: we should treat evil as good lessened...}
%
St.~Thomas' argument should be both sufficiently representative and detailed: 
\citt{A thing is said to act in a threefold sense. In one way,
formally, as when we say that whiteness makes white; and in that sense evil
considered even as a privation is said to corrupt good, forasmuch as it is
itself a corruption or privation of good. In another sense a thing is said to
act effectively, as when a painter makes a wall white. Thirdly, it is said in
the sense of the final cause, as the end is said to effect by moving the
efficient cause. But in these two ways evil does not effect anything of itself,
that is, as a privation, but by virtue of the good annexed to it. For every
action comes from some form; and everything which is desired as an end, is a
perfection. And therefore, as Dionysius says (Div. Nom. iv): "Evil does not act,
nor is it desired, except by virtue of some good joined to it: while of itself
it is nothing definite, and beside the scope of our will and intention."}
{Aquinas, Summa Theologiae, I:q48.a1.r-o.4}

We have agreed to the last point, namely, that evil is willed only as a good, as
a misunderstood good. The second point seems to be left rather uncommented, but
it refers probably to the fact that efficient cause must be a being, and as such
it is good (we will return to this point shortly). Formal causality, mentioned
first, does not seem to worry St.~Thomas as any real activity. Probably rightly, for
there is not much \thi{real causality} in it. These Aristotelean causal schemata
do not appeal to us any more and, today, we would not consider a subsumption of
an instance under a general concept as any form of causality. But this is all
St.~Thomas is willing to grant evil: it acts (causes)\noo{Let us ignore, that is
  accept, the underlying identification of activity with causality.} some evil
in the way whiteness makes white.

If I murder a man, I may grant myself the consolation of being, at the bottom of
my heart, good and, moreover, of not willing anything evil by this act. But I
committed it and the \thi{privation of good} which occurred is not a mere
formality of classifying an instance under a general rule -- it was not death
which made the man dead, it was me; and it was not any abstract, formal
principle of evil which {\em committed} this evil act, it was me. So I, a being
ultimately \citet{good is the accidental cause of evil.}{SumTh}{I:q49.a3.Obj5}
But the very fact that I accidentally cause evil is itself due to some evil
development, is due to the fact that I am affected by evil -- not formally, not
as a mere principle of classification, but most really. What hides (or at least,
can be found) in the sterility of a formal cause, seems often to be the most
real process of \co{actualisation} of the \co{virtuality}.\ftnt{We have
  commented it in Book I, beginning of subsection~\ref{sub:virt},
  \refp{pa:foundingCausing}-\refp{pa:contActVirt}.  Otherwise, \thi{whiteness of
    a white thing} was dealt with under the discussion of \co{concepts} and
  supposed \thi{essences} in II:~\ref{impressConcept}. Its difference from
  \thi{evil of an evil act} or, for that matter, anything which proceeds in
  \co{existence} from \co{virtuality} towards \co{actuality} should have become
  clear from the following sections \ref{sec:levelC}-\ref{sec:levelD} of Book
  II.}  Evil in which I participate, which \co{alienates me}, is capable of
\co{actually} expanding this \co{alienation} -- both within \co{my soul} and,
through \co{my} evil \co{acts}, in the world.  Once \co{present}, it tends to
grow. And very often we do not even know that it is \co{present} and that it
started growing until it is~\ldots well, perhaps not necessarily too late, but
often very late indeed.\ftnt{\citefi{Whatever we nourish within ourselves, that
    grows: this is the eternal law of nature. There is within us an organ of
    dislike, of dissatisfaction, just like an organ of enmity, of suspicion. The
    more we nourish and exercise it, the greater it becomes, until it eventually
    turns into a terribly overgrown tumor which devours everything around,
    swallowing and annihilating all life-giving juices.}{Goethe}{\kilde{Blad
      tkwi...,p.41}}}

This is the main meaning of the claim that evil acts -- left for itself it may,
perhaps, eventually, effect its own self-destruction, but in the meantime it
will try to infect everything in its vicinity.  True, it always requires \co{me}
to be around, as a tool; all its activity must pass through \co{my actuality};
everything it possibly can do, it can do only with \co{my} hands. Yet, it acts,
that is, \co{actualises} itself, for after all, \co{I am not the master}.  It
unfolds like a hermeneutical spiral of self-elaboration: from the \co{virtual},
impersonal and \co{invisible} seeds, which mature and ripen unnoticed, to the
eventual consequences, deterioration, \co{visible} dissolution and emptiness.
The uprising of Satan, his fight against the good, finds place in heaven -- the
earth only observes the consequences.

This emphasis on the active element of evil may be a mere subtlety which does
not reflect any significant disagreement -- we follow closely the tradition
which sees evil primarily (though not merely) as negativity. Indeed, we do.  There
is, however, another aspect of this tradition which is harder to swallow.


\wtsep{But, there are evil things, acts, deeds....}

\pa \citet{[N]o Thing is contrary to God; no creature nor creature's work, nor
  anything that we can name or think is contrary to God or displeasing to Him,
  but only disobedience and the disobedient
  man.}{TheolGerm}{XVI.\label{ftnt:obedience} 
  \citef{God is the supreme existence, that is to say, supremely is [...]
  Consequently, to that nature which supremely is, and which created all else
  that exists, no nature is contrary save that which does not exist. For
  nonentity is contrary of that which is. And thus there is no being contrary to
  God, the Supreme Being [...]}{CityGod}{XII:2}}
\wo{Disobedient} or, as we would say, \co{alienated}. St.~Thomas does not stay
behind that in optimism.  \citet{Every being, as being, is good. For all being,
  as being, has actuality and is in some way perfect; since every act implies
  some sort of perfection; and perfection implies desirability and goodness, as
  is clear from a1. Hence it follows that every being as such is good.}
{SumTh}{I:q5:a3.}  According to Pseudo-Dionysius, \citet{evil hath no being, nor any
  inherence in things that have being. Evil is nowhere \la{qua} evil; and it
  arises not through any power but through weakness. Even the devils derive
  their existence from the Good, and their mere existence is
  good.}{DivNames}{IV:34 \noo{p.129}}

Common to all these variations on this neo-Platonic theme is the Aristotelian
opposition between act and potency which coincides with that between perfection
and imperfection and, eventually, between good and evil. Thus, everything which
\co{actually} is must, indeed, be good.  Devils, perhaps, in so far as they
\co{exist} are good, too. But what about a torture dungeon? It might have arisen
\wo{not through any power but through weakness}, though St.~Thomas would probably
still argue that there is some perfection, actual power, in the mere fact of its
actualisation.  But even in the tradition which sanctified Aquinas, one can
state that \citt{evil is not only a lack of good, but a living and spiritual
  being, though one who is deprived and depriving.}{Paul VI, in
  \citeauthor*{RatzingerRep} \citaft{SatanMinois}{ p.165}}  The unreserved ontological optimism does not
appear plausible. 
Any thing, say a house, may be bad as this particular thing, it may be a bad
house, but still, in so far as it {\em is}, its very existence, is good. But it
does not seem possible to argue such a case!  A concentration camp is evil and
it is not an evil which merely deprives some substantial good -- the very {\em
  fact} of its existence is evil, its very existence is evil. Evil not only
\wo{inheres in this thing which has being}, in fact, it constitutes its
\thi{being and essence}, for except for being evil, this thing is nothing,
without being evil, this thing would no longer be itself. It won't help to claim
that its buildings might have been used for other purposes; it won't help to
blame the formal, nor even the material causes for the evil which accidentally
inhabited this essentially good being -- it is evil through and through,
including all the involved engineering perfections; it is the more evil, the
more such perfections it involves!  The fact that it got \co{actualised} won't
help anybody to claim that somewhere, at the bottom, it must be good -- it can
only indicate that not only there is no equivalence between, but not even any
implication from \co{act} to perfection, from \co{actuality} to goodness.  The
mere actuality, the mere fact of the existence of Birkenau must be very
\wo{contrary to God and displeasing to Him.}\noo{Unless, of course, one is
  willing to follow much more intricate meanders of theodicy than we are able
  to.  Goodness and all the perfections actualised in the design and
  construction of Birkenau might, perhaps, appeal to the Scholastic pedantry,
  but no sane person would ever appreciate that. The very efficiency of a
  concentration camp makes it the more evil.}

\pa That \woo{every being, as such, is good}{Aquinas, Summa Theologiae,
  I:q48:a1}, that \citet{[t]hings solely good [...] can in some circumstances
  exist; things solely evil, never, for even those natures which are vitiated by
  an evil will, so far as they are vitiated, are evil, but in so far as they are
  natures are good,}{CityGod}{XII:3} that every \co{actualisation} as opposed to
a mere possibility is good -- this other side of the face which sees evil as a
mere negation, which \woo{by the name of evil [signifies] the absence of
  good}{Aquinas, Summa Theologiae, I:q48:a1} -- can not, possibly, stay
unmodified.~\ftnt{\label{evilAlways}And this not because something particular
  happened with Auschwitz, Stalin's collectivisation or Khmer Rouge regime which
  would require re-evaluation of anything. The only special thing about these examples
  is their recentness. If any comparisons were allowed, one would have to
  admit that the fate of native Americans witnesses to evil much more powerful,
  long lasting and, eventually, more successful and hence more terrifying than
  the relatively brief, even if horrifying, excesses of Nazism.  (The fact that
  it has been perpetuated by the countries which, at present, possess enough
  power even to adjust the official definitions of the terms like \wo{genocide}
  in order to exclude their own case, is at most of only political relevance.
  [\citeauthor*{Genocide}]) Many peoples disappeared not due to some processes
  which we might find excusable and understandable in historical terms, but as a
  consequence of intentional policies applying the most advanced technologies of
  the time for systematic extermination of other peoples. On the other hand, it
  does {\em not matter at all} if the intentions were mere suppression and
  subordination -- in many cases it was more than that. What counts is the evil
  of the final effect -- extermination -- which is approximately as old as the
  recorded history.  Describing his expedition against Damascus, the Assyrian
  king Shalmaneser II records various skirmishes: \wo{I desolated and destroyed,
    I burnt it: 1200 chariots, 1200 horsemen, 20.000 men of Biridri of Damascus;
    700 chariots, 700 horsemen, 10.000 men of Irhulini of Hamath; 2.000
    chariots, 10.000 men of Ahab of Israel [...]}  The expedition found place in
  854\bc, and similar boastings can be found on clay tablets and in chronicles
  ever since.  That numbers are probably exaggerated does not change the fact
  that the mood and intentions are not. If one wanted to maintain that
  extermination in battles is not simply extermination (which, unfortunately, it
  still is), so let us observe that people were also more systematic than that.
  Assyrians of the Second Empire (after Tiglath-pileser III, 745-727\bc) are the
  recorded inventors of mass deportations of peoples with the object of breaking
  down their national spirit, unity and independence. Thus ended the existence
  of Hittites, whose wealth and trade passed into the hands of the Assyrian
  colonists after the fall of the capitol Carchemish in 717\bc\ Assyrians
  themselves disappeared from the history after the fall and demolishion of
  their capitol Nineveh in 606\bc\ Romans were certainly not exterminators yet,
  was it merely their systematic warfare which left only residual rests after
  Celts who once populated most of Europe?  Systematic neighbours left no traces
  after the \thi{christianisation} of Jatwingians, Prussians (the original
  Baltic people, not the Teutonic Knights and German settlers who claimed their
  place and name), Slavic tribes like Polabians, Abotrites, Liutizians, etc.,
  etc., etc.}

Speaking abstractly, evil is a negation, but negation of what?  For certainly,
negation is not evil, that is, it is not always evil.
In our setting it is negation of the \co{origin}, negation which turns
\co{nothingness} into emptiness, which thus berefts the \co{existence} of its
fundamental character of \co{confrontation}, negation which
\co{alienates} \co{myself} from the \co{self}, and then, from others and all 
the world. In the derivative sense, everything which leads to and 
strengthens \co{alienation} is evil, too.

As the \co{confrontation} with the \co{origin} is \co{existence}, we may
consider it as such, as \co{existence} \la{simpliciter}, to be good.
% Origen, Commentary to The Gospel of John, II:7 &
% II:9.2: \la{certum namque est malum esse bono carere}: zlo to brak dobra
Now, as St.~Thomas says, \citet{by the name of evil is signified the absence of
  good.  And this is what is meant by saying that "evil is neither a being nor a
  good."  For since being, as such, is good, the absence of one implies the
  absence of the other.}{SumTh}{I:q48:a1} Besides the unclear use of
implications\ftnt{Every \wo{is} in this quotation is probably meant as and
  equivalence. Thus, first \wo{being = good}, and then not only \wo{evil is a
    non-being, is a lack of being}, but also \wo{any lack of being is evil} -- a
  disputable matter, to put it mildly.}, this is a bit too abstract for us.  We
say that \co{existence}, as such, is good, but we would not say that, for this
reason, its negation, namely non-\co{existence}, is evil. Not because it must be
wrong but only because we do not know what that possibly could mean. What is
\wo{non-existence}?  Is it the total lack of any \co{existence} whatsoever, the
total lack of life? We might, perhaps, feel that things are better with life
than without and agree that extermination of life would be evil. But if life had
never appeared? It is hard to judge contrafactuals, especially ones with
conditions excluding the very possibility of judgment. But it is also hard to
imagine what inherent evil would be in the total \co{indistinctness}.  So, would
non-existence be the non-existence of Pegasus? Of a person we could imagine to
exist? Of my grandfather (that is, the fact that he is not alive any longer)?
Perhaps (but only a huge perhaps!), it might be OK if some of these now
\co{existed}. But even then, this would not force us to see anything evil in
their \co{actual} non-existence. Unless, that is, one wanted to see natural
death itself as some evil, which would be to protest against the very character and
  nature of \co{existence}, that is, to restrain \co{existing} itself.

\pa\label{alienX} Evil is \co{alienation} -- it is that which inhibits
\co{existence} in \co{existing}, which breaks the continuity between
\co{actuality} and its \co{origin} and thus prevents their \co{confrontation}.
But this inhibiting, this preventing is not the same as the flat negation:
non-\co{existence}, lack of \co{existence}. \co{Alienation} amounts, so to
speak, to closing off \co{existence} within some horizon, putting an artificial
-- because \co{non-actual}! -- limit to its otherwise \co{open} unfolding
towards its \co{origin}. It is a special kind of negation, a special form of it,
and this specificity suggests to the language the word \wo{evil}, in addition to
the mere \wo{negation}.  An aspect of this specificity is that it is primarily
concerned with \co{existence} and not merely (that is, generally) with all
being.  Evil is the event of human life, it is only in humans that evil can find
a site for its unfolding, and it is primarily humans that it affects.
\citet{Indeed, our vices or sins, which are what are properly to be called
  evils, are unable to exist except in souls -- that is, in good
  creatures.}{AbelardPJC}{II:\para 401} One can hardly be evil towards dead things; one's
destructive tendencies towards them may, at most, indicate some evil processes
going on in one's \co{soul}, but these are only \co{signs}. One can be evil towards
living organisms (which fall under our generous definition of \co{existence}),
though here, too, the judgment will often see the \co{signs} of potentially
greater evil: what seems most appalling in the image of a person molesting or
even torturing an animal is not only the pain he causes but the question
\wo{What must be going on in him? What a rotten person he must be!}  Eventually,
evil affects only human being, \woo{yet it never consumes it.}{Aquinas repeated,
  ST, I:q49.a3} Things solely evil cannot exist in so far as evil is the accident of
human \co{existences}. These -- {\em and only these} -- are, in
so far as they \co{exist}, good.\noo{The neo-Platonic elements of ontological
  optimism in Augustine's theory of evil are mitigated by its moral character:
  good is related to God and, secondarily, to human being, while evil centers
  exclusively around human -- because voluntary and wilful -- defiance, around
  the misdirected love whereby the will fell away from the immutable to the
  mutable good.}
%Title of Bk.XII:Ch.B in The City of God

% {In On Evil, q2.art.1, Aquinas says the ontological optimism with respect to every
% being gets a similar flavor: \wo{the subject itself is called good inasmuch as
%   it is in potentiality to perfection, for example the soul to virtue, and the
%   substance of the eye to acuteness of sight.}{[The Many Faces of Evil, p.86]}}

In the derivative (and this does not mean {metaphorical} nor {weaker})
%only relative)
sense, things can be evil to the extent they serve evil.  But unlike
\co{existence}, things (and also \co{acts}) can be wholly and totally evil.
That \wo{evil always lessens good, yet it never wholly consumes it} applies to
the \co{existence} which is affected by evil but is never
totally underlied it. For evil is but the privation of
\co{existence}, the \co{alienation} from the \co{origin}, which indeed is never
complete. Things, on the other hand, or
generally anything \co{actualised}, when \co{dissociated} and locked 
within the \hoa, can become dead \co{signs}, epitomes of mere evil and nothing
more.  The problem with \thi{things} like concentration camps is that even a
slightest attempt to look for anything good in them is inappropriate, if not
directly detestable. They not only served evil purposes (while, perhaps, they
could have served others) -- their mere being is purely evil, as they epitomise
nothing but the strength and depth of evil which was consuming the humans who
invented and utilised them. They are thoroughly evil because they have no
\co{rest} \co{above} the pure evil of their purpose, and of the \co{precision}
in its execution -- they are \co{actualisation} of nothing but evil. \citt{[A]
  corrupt tree bringeth forth evil fruit.  A good tree cannot bring forth evil
  fruit, neither can a corrupt tree bring forth good fruit.}{Mt. VII:17-18} But
while a corrupt tree can, sometimes, recover, its evil fruit, once it has fallen
on the ground, can not.

\pa
 The short history
of mankind, according to Anatol France, is: \wo{They were born, they were suffering,
  they were dying.} But the fact that all kinds of \thi{evils} are a constant element
of human history, and every individual human life, that \citeti{ye have the poor
 always with you,}{Mt.}{XXVI:11} does not entail the
conclusion that \citet{it is good for any evil to exist, although nevertheless
  no evil is good.}{AbelardPJC}{II:\para 412} Such a conclusion is motivated by a 
series of postulates, the most important among which are that, for the first,
\citet{everything that becomes or is created 
  must of necessity be created by some cause, for without a cause nothing can be
  created,}{Timaeus}{28a} then that all such causes converge in one common cause
and, finally, that such a cause is itself good or for some other reason creates
only good things. 
% \citet{Everything begotten is begotten from some necessary cause. For nothing
%   happens for which a lawful cause and reason does not precede its
%   arising.}{Timaeus}{28a}
But we have replaced \thi{cause} with \co{virtuality} and \thi{causation} with
\co{actualisation} through \co{existence} -- the connections between the last
effects and their first origins are neither so plain nor so \co{visible} and
they may get corrupted at every stage. We can,
nevertheless, discern the powerful \co{existential} call in the claim that 
 existence of everything, even of evils, is good. It translates into a call to
 \co{openness}, \co{humble} acceptance of every particular as
 something, at its bottom, if not at its surface, good. In so far as this aspect
 is concerned, we are in full agreement. Also, such a \co{concretely founded}
% (section~\ref{sec:schYes}, in particular, \ref{sub:concrete})
 acceptance does in fact lend all the things the element of goodness, of
 \co{participation} in the \co{origin}. But such a goodness does not apply
 universally and unreservedly; it is not any fact of mere ontology which one can
 discern in the matter of the objective world if only one analyses things
 thoroughly enough -- it, just as its opposite, is an \co{existential} possibility.

% in footnote~\ref{evilAlways}
% \citet{Woe unto the world because of offenses! for it must needs be that 
% offenses come; but woe to that man by whom the offense cometh!}{Matt}{XVIII:7}


\noo{ Force?
\tsep{is it a force?}

\pa \wo{Every \co{existence}, as such, is good.} And it has all the rest of its
\co{concreteness}, all the hierarchy including the celestial one and down to the
most \co{actual} and physical details of its body. Evil -- phenomena leading to
\co{alienation} -- can appear at any level of this hierarchy, except the highest
one, the pole of trans-phenomenal and trans-temporal, the ultimate
\co{transcendence} of the \co{origin} -- for evil is everything which negates
its \co{presence}. % (not in word, but in fact). 
% Comment on possible `innocence' of such mere negation... Also, innocence of
% being affected by evil vs. acting evil, yet being responsible for both!
% ????????? 

But what is this evil, eventually?  Only things, \co{acts}, \co{activities} may
be evil -- \co{existence} as such, as \co{confrontation} with the \co{origin},
is good.  So evil appears only in \co{this world}. But as a \thi{force} it is
not \co{visible} -- it grows and matures \co{invisibly}; only its consequences
are observed. So is it \co{visible} or not?  Then, it requires a person, but it
is not personal.  Is it some \thi{metaphysical principle}, some kind of an
impersonal \thi{force}?  A fateful, devilish power, from which there is no
escape?  A \thi{substance}?


\pa The ontology of the flat opposition \thi{being} vs. \thi{non-being},
\thi{real} vs. \thi{unreal}, centering around the notion of \co{actual}
\thi{substance}, seems too narrow to accommodate the active reality of evil.
If evil be granted a \thi{substantial being}, it threatens the whole
metaphysics with Manichean
dualism, If, on the other hand, it is denied such a \thi{being}, it becomes a
\thi{non-being}, pure nothingness; and this suffers in the
face of the question: Is evil real? Does it really exist? For, of course, it
does -- everything which is, that is, which is \co{distinguished}, is real.
%Calling it a \wo{force} may alleviate the problem a bit but, eventually, this
%too tends towards substantialisation of evil
Consequently, one seems to be forced either to accept the ontological optimism
or else to dissociate good and evil from ontology and
to delegate their whole issue to another sphere
-- ethics, axiology, agatology\ldots
%Levinas claimed ethics to be the first philosophy, but it 


\pa The same word \wo{evil} can be used both as a noun and an adjective -- we
would be satisfied only with the latter.  We have objected to the unreserved
claim that \citet{evil can exist only in good as in its subject} {SumTh}{I:q49.a3 (q48.a3)}, because a thing, a being can be thoroughly
evil. But we have less to object against the fact that, eventually, \citet{good
  is the accidental cause of evil.}{SumTh}{I:q49.a3.Obj5}
\noo{CF. changing the gender of adjectival names frequently varies the sense: x is
good vs. x is a good - Abelard, Dialogue 2, (302), p.120.}

To begin with, we said that evil is misunderstood \co{thirst} -- or better:
misunderstanding of \co{thirst}. In this sense (and with all the necessary
translations) we do say that \co{existence}, \wo{good is the accidental cause of
  evil}.
%as only \co{existence} may misunderstand its \co{thirst}.
Using the language of \thi{substances} we could say that evil is a property,
namely, the property of effecting \co{alienation}. Because of this purely
functional definition, no final list of evils (nor goods) will ever be possible.
Even if some \co{acts} tend almost always to be evil (killing, lying,
humiliating, etc.), and some things are nothing but evil, so yet every
\co{existence} may bring in new forms of evil (as well as good) into the world,
and may become \co{alienated} by things which never \co{alienated} any previous
\co{existence}.  The imperceptible connections between things, the unintended
side-effects of words and \co{acts}, involuntary associations -- all may sow
\co{invisible} seeds of evil, due to the specificity of the context, particular
sensitivity of the involved persons.  Evil can be a thing, an \co{act}, an
\co{activity}, a \co{motive}, a character of a person, a disposition\ldots


Yet, this \thi{being a property} is not, in our setting, necessarily being a
property of something.  The distinction between \thi{substances} and
\thi{accidents} is relative -- a \thi{thing}, a \thi{substance} is only a limit
of \co{distinctions}. Evil is one of such \co{distinctions} -- as real and as
effective as any other could be. Therefore a particular thing can be thoroughly,
unreservedly evil.  With respect to \co{existence}, however, evil may appear
only as an accidental  property.  For evil is not the highest principle but
exactly that which breaks
%\co{existence} from establishing or maintaining
the continuity with
what could be called such a principle.  It is that which breaks the continuity
with the \co{origin} and thus inhibits \co{participation} in it. It is an
obstacle, an inhibition (\wo{Satan} derives from the Hebrew root \gre{stn} which
means adversary, just like the Greek \gre{diabolos}), but these
negative characterisations must not make us think of it as a mere negation, for
it actually spreads and is experienced as an active force.

As we have emphasized, it may spread in spite of the best wishes and intentions
of all involved persons, it may begin to consume one's \co{soul} long before
this becomes \co{visible} to one.  It is just a way in which things, other
people, surroundings, and eventually even the involved person himself, can act
on and -- voluntarily or not, consciously or not -- influence \co{existence}.
Some factors or their constellations have \co{alienating} effect on a person;
others do not.  Evil's primary activities are \co{invisible}, and only sometimes
the closest friendship or deep wisdom may \co{recognise} its initial 
workings.  Eventually, it is only through their effects that they become
\co{actually visible}. \citt{Wherefore by their fruits ye shall know them.}{Mt.
  VII:18-20}
% This, not any utilitarian felicity calculus
% of \co{visible} consequences, seems to us to lie in the metaphor: \citt{A good
%   tree cannot bring forth evil fruit, neither can a corrupt tree bring forth
%   good fruit.  Every tree that bringeth not forth good fruit is hewn down, and
%   cast into the fire.  Wherefore by their fruits ye shall know them.}{Mt.
%   VII:18-20}

% alienation is fragmentation
\pa\label{pa:evilNoPrincip} Not only evil is not any highest principle, any
highest truth of 
\co{existence}, but it does not itself have such a truth.\ftnt{We do not
  particularly like the word \wo{principle} because it tends too much in the
  direction of \co{actual} consciousness. Talking about \wo{highest principle}, we
  do not mean anything which need be consciously present to somebody. We mean
  the \thi{truth} of the situation, whether it is realized or not, whether it is
  consented to or not. There is nothing discursive about such \thi{truth}; it
  has very little (if anything at all) of epistemic content or relevance, which
  can be so easily associated with \wo{principle}.} What consumes it is
the search for filling the emptiness -- the emptiness of \co{alienation} which
enters the cracks between the \co{dissociated} parts of \thi{the world}, expands
these cracks, multiplies them, and eventually dissolves the world into \co{more}
and \co{more} isolated fragments, monads, unrelated bits and pieces, unreal cities,
wastelands.  The progressing process of dissolution, dissociation,
\co{alienation} -- this, one might say, is to suggest evil's highest principle!
OK, this is the ground for \co{distinguishing} it.

But this does not provide any principle, any truth in which evil, as such,
\co{participates}; it only announces its lack. Evil has no ultimate \thi{highest
  principle}, no master, for it is exactly the lack thereof; it does not
\co{participate}, for it is exactly negation of all \co{participation}.  \co{I}
can participate in evil because evil is, usually, \thi{greater than me}. But the
evil in which \co{I} thus participate is itself only the negation of anything to
participate in, is the lack of \co{participation} -- at its bottom, there hides
emptiness. And as there is nothing ultimate to gather around, so there is
nothing to gather and everything gets scattered.  \citt{He that is not with me
  is against me; and he that gathereth not with me scattereth abroad.}{Mt.
  XII:30, Lk. XI:23} The first indications of \co{alienation} may be uncertainty
about and strangeness towards \co{oneself}, fear of unclear character, that is,
anxiety of \co{oneself}. But its eventual effect is dissolution and
fragmentation of the \co{existence} and its whole world, \citet{it divides
  instead of joining.}{SenecaAbs}{ [said only about anger, but anger as \wo{a
    vice point-blank against nature}] \noo{The Many Faces of Evil, p.19}}
\citt{And they were scattered, because there is no shepherd: and they became
  meat to all the beasts of the field, when they were scattered.}{Ezek. XXXIV:5}
The \thi{principle} of evil is the sickness of \co{existence} which losing its
\co{foundation} in the \co{absolute origin}, loses itself and, eventually, even
its \co{visible} world -- \citt{smite the shepherd, and the sheep shall be
  scattered.}{Zech. XIII:7 [Mrk. XIV:27]}

%\citt{he that gathereth not with me scattereth abroad}{Mt. XII:30}, and 

\noo{
\kom
We have to learn it... This goes across the \co{visible}-\co{invisible} border...

\kom
It is real but not \co{absolute}, it is power -- it can kill and destroy -- but
not ultimately powerful -- there is something \co{above}.

From the point of view of the person affected by evil:
}



\noo{  % gone to {Existence as such is good...
\citt{It must be said that every evil in some way has a cause.
For evil is the absence of the good, which is natural and due to a thing. But
that anything fail from its natural and due disposition can come only from some
cause drawing it out of its proper disposition. For a heavy thing is not moved
upwards except by some impelling force; nor does an agent fail in its action
except from some impediment. But only good can be a cause; because nothing can
be a cause except inasmuch as it is a being, and every being, as such, is good.}
{Aquinas, Summa Theologiae, I:q49.a1}

We have, of course, replaced the language of \wo{causes}, especially the
primary, as we might say, vertical causes which became the hypostases, down to
the \co{actualisation}, of the \co{virtual nexus} of \co{existence}. The center
of \co{existence}, the \co{confrontation} with the \co{origin} is not only the
good in itself, but is also the true beginning from which proceed all the
\co{distinctions}. In this sense, we may agree on the last sentence: every
\co{existence}, as such, is good and nothing can be the beginning (rather than a
cause) except inasmuch as it is an \co{existence}. 

Evil is any form of \co{alienation} which breaks the continuity with the
\co{origin} -- it is the absence of good. The \wo{cause drawing a thing out of
  its proper disposition}, which St.~Thomas insists must be good, is in our language
\co{thirst} which gets misunderstood -- for it \co{thirsts} for the \co{absolute
  nothing}, but one tries to quench it with the relative, with \co{idols}. And
so, indeed, a conglomeration of good causes may sum up to an evil effect
} % end \noo gone to...

} % end \noo Force?


\subsection{Attachment}

\pa Despair and evil are basic forms of \co{alienation}, of broken continuity of
\co{traces} which no longer lead to the \co{origin} but stop short of
it. \co{Alienation} results from denying the originarity of the \co{origin}, 
from the fundamental %\co{spiritual} event, the
\co{spiritual choice} of \No\ in which {Psyche}, following the doubts sown by
the oracle and her wealthy sisters, not only prepares to kill but actually
succeeds in killing her heavenly husband, {Eros}, whom she has never seen.

The \co{choice} is \co{spiritual} because, for the first, it is not made by
\co{me} -- it is made \co{above me}, but also {\em for} \co{me}, so that \co{I}
carry all the consequences, as well as full responsibility, for it. There is,
indeed, nobody to blame, and looking for
excuses %(in the childhood, family, society, ...)
leads nowhere -- psychology may know about various sufferings of the soul,
but it knows nothing about damnation. For the second, the \co{spiritual}
character amounts to the \co{absolute} objectlessness of the \co{choice}. It is
not directed towards anything whatsoever, whether \co{visible} or \co{invisible}
-- it is lifted \co{above} all \co{distinctions} and directed towards
\co{nothing}.  \No\ turns this \co{nothingness} into 
mere emptiness, total void, lack. It says: \wo{there is no exit, because
  \thi{outside} there is only void}, \wo{there is nothing in \co{nothingness}},
or perhaps, \wo{\la{nihil ex nihilo}}.
%but with a much less epicureanism and much more dramaturgy than the Lucretian original. 
This refusal, the denial of the \co{foundation} in the \co{invisible origin} of
\co{nothingness} is a \co{nexus} of several denials.

\pa\label{pride} Denying the \co{invisibility} of the \co{origin} amounts to the
claim of self-sufficiency.  Things are \co{visible} and there is nothing which,
at least in principle, could not be appropriated, embraced by the \co{actual}
look, grasped by the \co{actual} power of our faculties. It is \co{I} who decide
and control, \co{my life} is entirely the matter of my choices. The exclusive
directedness towards \co{this world} denying \co{that I am not the master} and
attempting to reduce everything to the
\co{visible} and controllable can be called \wo{\co{pride}}.
%As the opposite of \co{humility}, we may call it \wo{\co{pride}}.

Freedom, in its negative form, is an aspect of \co{pride} in that \No\ turns
\co{nothingness} into emptiness and thus does not recognise anything which might
be \co{above}. It is freedom to arrange the \co{visible world} entirely as
\co{I} find it appropriate since, at the bottom, it is just the freedom from any
higher \co{commands} which might be understood as limitations of \co{my} free
will. The absolute autonomy, the absolute self-government of the \co{I}, the
absolute freedom from \ldots\ is possible only as a reflection of ultimate
emptiness.

% Unfortunately, \citt{one cannot will into void}{quoted by George Cotkin -- from
%   (?) `What the Will Effects?'}, and so eventually \citt{man has to go mad to
%   prove that he is free.}{After Dostoevsky}

\pa\label{ingratitude} If \co{I} am something, \co{I} am in particular the
source of \co{my} actions and achievements. And certainly, I am, but here there
is more to it -- I am the {\em only} source of all that, and if it is not
personally \co{me}, then it is \thi{we}, a multitude of \co{Is}. But a multitude
of \co{Is} is no \co{community} -- it is merely a \co{totality} which does
reduce to its components. So it is \co{I} who am the master and there is no
reason for any indefinite \co{thankfulness} which, as a matter of fact, would
actually offend my dignity. Since the \co{visible world} of \co{mine} is all
that is, there is nobody to be \co{thankful} to, and nothing to be \co{thankful}
for. On the contrary, there is a lot to be blamed, as whenever some evil makes
itself effective. It is always \co{unclear} what actually is to be blamed,
whenever one pronounces a \co{general idea} of the inherent evil, or at least,
malice of the world. As with most \co{general ideas}, it ends as a mere
statement of \thi{the fact} which only reflects a \co{quality} of one's
life. This statement is, too, an \co{aspect} of \No\ -- 
let us call it \wo{\co{ingratitude}}.

% (As all other \co{aspects} of the \co{spiritual choice}, this statement need not
% be pronounced explicitly -- it can be lived without ever becoming
% \co{reflectively conscious}, without ever being \co{actually} and \co{precisely}
% realized.)

% Finding numerous examples
% of wrongs and evils in \co{the world}, ridiculing, as Voltaire did, the
% powerful idea of Leibinz's of the best possible world by pointing to earthquake
% in Lisboa and other disasters meeting Panglos and Candide, and citing them as
% understandable reasons for lack of gratitude, is all too easy to be admired.
% But such an attitude is not the origin but a consequence of the fundamental \No,
% perhaps active, most often passive.  To have a word for this aspect, too, let me
% call it \wo{\co{ingratitude}}.

\pa\label{closed} \co{Nothingness} surrounding everything is a mere void, while
\co{this world} is here, that is, \thi{out there}, actually, in a very definite,
objective sense. What it is is not easy to say, and the most natural intuition
is that it is all that is \co{visible}, the \co{totality} of all things, facts,
people. These facts and things, having at most some causes but no \co{origin},
are \co{experienced} as given; the variety of 
\co{visible distinctions} is found with the unmistakable stamp of being there,
being ready-made.\ftnt{Heidegger would say \thi{ready-at-hand}, though he
  would all too definitely identify that with technical manipulation. Sartre might
  say that they are \thi{in themselves} -- 
  things turned into dead \co{objects}, even others enslaved by the restless
  freedom of  \thi{for-itself}. It is this extreme possibility we are intending
  here.}
This certainly offers an inexhaustible field of possible
inventiveness but, in the \co{spiritual} sense, it is a \co{closed} -- because
dead -- world. It
does not invite to unconditional acceptance of whatever one might meet but, on
the contrary, to separating and \co{dissociating} -- things from things, people
from people -- to searching for and selecting
only what is agreeable. Although such a world is potentially open, in the sense
unlimited and indefinitely flexible, the very definite givenness of its building
blocks, the Sartrean \thi{in-itself} of its \thi{hard facts}, marks it by a kind
of rigidity and stiffness -- 
let us call this \co{aspect} of \No\ \wo{\co{closedness}}.


\pa\label{pa:notno} \co{Pride}, \co{ingratitude}, \co{closedness} are but
\co{aspects}, only a few \co{aspects} of the \co{spiritual} \No.  As the
fundamental reduction of the ultimate \co{invisibility} to emptiness, it amounts
to {self-centeredness} or -- what is here a synonym -- {world-centredness}.
\co{Attachment} is \No\ to the ultimate \co{origin} said 
through the {\em exclusive} directedness towards \co{this visible world}.
%
% Negating the \co{invisible}, directs \co{me} 
% exclusively towards \co{visibility}. Thus, although it results from 
% an \co{act} at the same level as \yes, the resulting attitude is not 
% \co{spiritual} to the same extent as the attitude of 
% \co{love}.\ftnt{I do not want to imply it with the full force, but 
% one can in fact compare it with the Socratic idea that right 
% understanding brings right moral attitude. It does not follow here 
% only because the right understanding {\em is} the right moral 
% attitude. One only has to be careful with what one puts into 
% \thi{understanding}. In this context, it has little, if anything, to do with 
% mental operations, with mere insight, with any merely \co{visible} 
% categories. In this context, it simply is the right moral attitude, 
% the living founded in \co{holiness}, which is never something \thi{pure, 
% instinctive, unreflected}, but always involves understanding and 
% self-understanding.
% }
% %%% go to Morality \ldots

\No\ does not  necessarily signify hatred, nor evil, nor despair, 
%which one might think to be the opposite of \co{love}, 
though it eventually manifests through such forms.  \co{Attachment}, with all
its \co{aspects}, is a \co{spiritual} attitude, that is, addresses \co{nothing}
(even if it is not directed towards it) and does not imply any unique ways of
\co{actual} being and thinking.  It is not necessarily evil nor despair, it is
not necessarily egoism nor egotism, it is not even necessarily selfishness. It
may involve unselfish \co{acts} and attitudes, but the very fact of their being
unselfish reflects the underlying \co{attachment} to the categories of
\co{mineness}.  One may truly attempt to reach beyond \co{oneself}, to establish
and live according to some unselfish principles. One may -- and, indeed, one
often does -- make absolute claims. But this absoluteness inevitably degenerates
into a mere universality, a crude subsumption of all thinkable instances.
\co{Visible world} is what is \co{below me} and \co{I} am \co{my life}, \co{my
  life} is \co{my world} and \co{my world} is \co{myself},
II:\refppf{pa:worldislife1}.  The two can not be dissociated: attention to
\co{myself} happens already within the horizon of \co{visibility}, and
preoccupation with the \co{visible world}, in whatever form, involves
\co{myself}.  Narrowing the attention exclusively to \co{this world}, \co{I}
narrow it to \co{myself}. Whether \co{I} do it in a selfish or unselfish mode
may make a difference to adolescent psychology or sterile ethics, but
\co{spiritually} both -- and, first of all, the very opposition itself! --
amount to \co{attachment}.

The names of manifestations of \co{attachment} are \wo{plenty} and easy to
imagine. We therefore only sum them up in saying that \co{attachment} is 
the pattern of all \co{idolatry}, \co{absolutisation} of the \co{visible
  world} which forgets its \co{invisible origin}. 


\section{Spiritual choice of Yes}\label{sec:schYes}
%\input{033yes}
\pa
In a Sumerian myth (written down around 1750 \bc)  Inanna, the queen of
Heaven and Earth (or else, the goddess of love, fertility of nature and war),
\wo{from the Great Above opened her ear to the Great Below}, 
to the moaning call from her sister Ereshkigal, the goddess of the Underworld.
Descending to the Underworld, Inanna is on the way stripped naked of all her
clothes by the servant of Ereshkigal. After 3 days in the Underworld, she returns
helped by her dedicated servant and a cunning plan of the god of Wisdom and
Water, Enki, which seems to fool moaning Ereshkigal. (In some versions, to leave
the Underworld, \wo{she must provide someone in her place}, and the one is
Dumuzi, her husband, the Shepherd or the Lord of the Sheepfolds, ensuring
fertility and fecundity, who now has to leave the world for the half of every year.)

The theme of the descent and the challenge of facing nakedness, isolation,
helplessness, recurs frequently  in later Indo-European
mythology.  Looking for his way back to Ithaca, Ulysses had to descend to Hades,
so that the dead seer Teiresias could \citet{tell you about your voyage -- what
  stages you are to make, and how you are to sail the sea so as to reach your
  home.}{Odyssey}{$\!\!$, end of X} Orpheus had to visit the house of shadows to
regain his love Eurydice (who however dies again on the way back, because of
Orpheus' turning around against the prohibition of Hades).  Heracles was granted
immortality on the completion of the 12-th labour -- capturing Cerberus, the
guard dog of Hades (whom, on the god's command, he had to defeat with bare
hands).\ftnt{For our purposes, the Underworld can be considered synonymous with
  hell, though more detailed distinctions and
  comparisons are easily possible. In order to keep analogy, we won't count
  Elysium, the Isles of the Blest, as part of the 
  Underworld, while Tartarus, the place of ultimate punishment, should certainly
  be included.}
Paradigmatic (though written down only in the second century \add\ by Lucius Apuleius)
is the story in which  {Aphrodite}, in 
her attempts to annihilate {Psyche}, orders her to fetch some water from
{Styx} and then 
even to enter the Underworld and obtain a piece of beauty from
{Persephone}. Only successful completion of these tasks (with some help from
{Eros}) leads to the final recognition of {Psyche's} right to her divine
husband and the grant of immortality.

Among other variations, involving additional aspects but still centering around
the same theme of temporary isolation before renewal or rebirth, we could
mention the common motif of the child who, threatened by the envious ruler, is
led by the mother to a seclusion or remote country.\ftnt{E.g., Abraham according
  to the midr\u{a}sh \btit{Ma`ase Avraham}, in \citeauthor*{BHM}{I:25ff.};
  Jesus.} Sometimes, the future hero is abandoned in the mountains,\ftnt{Paris;
  Oedipus; Cyrus the Great (who, according to the legend in \citeauthor*{Herodotus}{
    I:108-113}, was not so abandoned only thanks to the disobedience of king's
  executioners.)} or else placed in a boat or chest which, put adrift, reaches
safely some shore far away from the civilised dwellings and where the hero is
helped and reared by modest people or even animals.\ftnt{Moses; \noo{according
    to \citeauthor*{MityHebr}{XXIV:3}} Romulus and Remus; in some versions
  Oedipus.}  Likewise, in the myth known already to the Sumerians and Hittites,
the deluge, sent by God as a punishment and for the purification, is survived in
the isolation on the ark only by a few God-chosen ones.\ftnt{Utnapishtim from
  \citeauthor*{Gilgamesh}, \thi{the first man} Manu from the Vedas,
  Biblical Noah, Greek Deucalion.}  A less dramatic variant is that of being
hanged -- as if suspended, in a thin air, in a state of isolation and helpless
awaiting for relief or enlightenment, as in purgatory. Jesus' death on the cross
was but the first stage before descent.  In Tarot, the Hanged Man is the card
signaling a state of solitude and submission to the divine will, suspension between
the forces of heaven and earth and sacrifice bringing mystical knowledge and
redemption.  Odin had hanged head down from the World Tree, Yggdrasil, for nine
days, pierced by his own spear, thereby acquiring sacred wisdom, learning nine
magical songs and eighteen magical runes.  Scholars not willing to see in this
Norse myth merely a garbled version of Christ's crucifixion, point out other
related motifs: in shamanism, climbing of a World Tree by the shaman in search
of mystic knowledge is a common religious pattern; sacrifices, human or
otherwise, were commonly hung in or from trees, often transfixed by
spears.

\pa
We certainly do not intend any review of mythology nor any elaborate
interpretations.  Hanging in the air may have vast structural differences from
surviving a deluge, while we only want to see in both the aspect of isolation
and complete immersion in the elemental power. 
We do not want to see the descend and rebirth as identical with, nor even as
related to the cycles of nature, the eternal return of the seasons. 
We see it purely \co{existentially} -- rebirth
is not a cyclic event of nature, but a unique possibility of \co{existence}.
%(The former/latter might be considered symbols of the latter/former.)

We want to see all the above as examples of the same pattern: the {\em
  necessity} of a lonely descent to the Underworld, of surviving the flood
(locked in a chest or ark), of temporary isolation in the air -- in order
to revive, to obtain the ultimate reward, enlightenment, salvation.
\citet{Christ's soul must needs descend into hell, before it ascended into
  heaven.  So must also the soul of man.}{TheolGerm}{XI} This often postulated
necessity, the assumption that the way to paradise must lead through hell,
causes us some trouble.
%We certainly do not want to deny the fundamental role of suffering in the
%\co{spiritual} development of a person. But a
Although suffering plays a fundamental role in the development of a person, it
does not follow that also hell, its extreme form, is a necessary step on the way
of purification.  This necessity seems to arise only when the ultimate reward
has the character of enlightenment, is somewhat associated with knowledge. The
knowledge-thirsty Odin hangs himself from Yggdrasil exclusively for the sake of
sacred wisdom; gnostics, equating salvation with insight, have to go through the
evil world only to renounce it. Al-Ghazali makes this relation to knowledge very
clear: \citet{For were it not for night, the value of day \lin would be {\em
    unknown}. Were it not for illness, the \lin healthy would not enjoy health.
  Were it not for \lin hell, the blessed in paradise would not {\em know} the
  \lin extent of their blessedness.}{Al-Ghazali} {56-60 \citaft{EvilFaces}{
    p.54; my emph.}} The following subsection addresses our troubles with
equating salvation and knowledge and clarifies the possible meaning of such
equivocation.

\noo{ %perhaps add ...
The reward is big: purification, eternal life, love....
But there are also unlucky ones: 
like Theseus and Pir... who got fastened
to the Chair of Forgetfulness (only Teiresias was allowed by Persephone to
retain his reason in Hades)
}

\subsection{Being and knowing}\label{sub:beingKnow}
\pa\label{pa:notKnow} It is not necessary to \co{actually} know in
order to be; it is not necessary to know that one is in hell to be there.  Past
visits in hell can intensify the \co{actual} realization that one is not there
any more; the realization which, perhaps in itself, can mean that one is in
heaven. But it is not a necessary precondition for being in paradise, for it is
equally unnecessary to \co{actually} know that one is in heaven in order to be
there.  Active search for heaven is the more suspicious, the more \co{visible} it is,
and the doubts about its genuine character coincide with the doubts about the
value of \co{actual} knowledge -- or, more generally, of the \co{visible signs} -- of
heaven.\ftnt{\citef{If any one saith, that a man, who is born again and
    justified, is bound of faith to believe that he is assuredly in the number
    of the predestinate; let him be anathema.}{Trent}{VI:XVI.On
    Justification.15}}

The insistence on the necessity of a passage through the heart of darkness in
order to reach the light, need perhaps not, in itself, be a sign of a gnostic
dualism. But it has similar origins in an intellectual bias towards the
\co{visibility} of \co{actual} manifestations and demonstrations.  It is not
necessary to know in order to be. But, certainly, \co{actually} knowing one pole
of a contradiction requires and implies knowing the other.  It is the knowing
experience of hell which, when contrasted with the experience of heaven, makes
the fact of its presence \co{visible}.\noo{The myths of descent, or Underworld
  in general, often have this aspect, too, though it is less dramatic and less
  visible.} Entering the Underworld amounts to losing reality. The involved
helplessness and nakedness indicate the aspect of purification but one finds
also the element of oblivion and forgetfulness.  Persephone takes away the
memories and understanding from the souls entering the Underworld -- Theresias
was one exception, but otherwise, the Greek dead (at least those who did not end
in Tartarus nor Elysium) are shadows living in a somnambulistic unreality, in
despair over the loss of and thirst for the real.  Only those who manage to
return from there retain their mental powers or, as we might say, the
strengthened consciousness of their presence in the living reality as opposed to
the shadowy 
oblivion of the Underworld. The return from the Underworld does not concern so
much the purification but, primarily, the manifestation of the status of the
chosen one.  Similarly, the flood does not serve the purpose of purifying the
survivors.  They survive it {\em only because} they had already been pure, or as
the myths have it, chosen by god -- the flood only clarifies the scores, makes
the predestined results visible.\ftnt{An element of arbitrariness in the
  selection of the survivor by the god appears in most flood myths: Utnapishtim
  is chosen by the goddess Ishtar (the Babylonian counterpart of the Sumerian
  Inanna), or sometimes Ea/Enki (Tablet XI) for no apparent, in any case no
  mentioned reason; when \abr{jhvh} decides to destroy the world with the flood,
  Noah simply \citefi{found grace in the eyes of the Lord}{Gen.}{VI:8};
  Deucalion was a Greek, so reasons and explanations get longer, though the
  bottom line remains unchanged -- he was warned about the flood by his father,
  Prometheus who, although not a god but only a Titan of second generation, was
  as such immortal. In the Hindu version it is \thi{the first man}, Manu, who is
  warned by a fish about the coming flood (in \btit{Mahabharata} the fish is
  identified with the god Brahma, while in \btit{Puranas} with incarnated
  Vishnu).}

Thus, we distinguish clearly the two aspects: on the one hand, the
\co{invisible} event of being selected, the God's decree which for the
\co{actual} understanding can easily appear as an arbitrary predestination
and, on the other hand, the \co{actual} knowledge, the \co{visible}
\co{manifestations} of this fact.

\pa\label{pa:beKnow}
Knowledge and being are closely related but neither is any simple function
of another. Some forms of knowledge are impossible
without some forms of being and knowing something may promote particular
way of being. As always, we will stay satisfied with few
necessary conditions without looking for the sufficient ones. 
%We gather a few threads from Book I concerning being and from
%Book II:\ref{sec:levels} concerning knowledge.

Most abstractly, knowledge is a relation while being \co{participation}. One
might immediately object that \co{participation}, too, is a relation but it is
not. (At least, we now want to make a distinction which earlier might have been
blurred, even by using the word \wo{relation} for \co{participation} as, for
instance, when speaking about being as an asymmetric \thi{relation}.)  A
relation, a \co{reflective} relation\ftnt{That is, not a reflexive relation, but
a relation as perceived by \co{reflection}.}  presupposes \co{distinct} entities
which it relates.  Relation to $x$ requires $x$ to be something else,
something distant, opposite, not-mine -- it requires a distance, and if
there is no distance, the relation will create it. This distance appears as the
distance separating \co{me} from $x$, but at the bottom it is the distance
separating the poles of the relation from their own being.  In short, relation, a
\co{reflective} relation, presupposes prior being of its poles (which is but
another way of saying that \co{reflective dissociation} presupposes prior, also
\co{non-actual}, \co{distinctions}.)

\co{Participation} is that which {\em constitutes} the being of these
\co{dissociated} (and associated) poles. Although
it involves the \co{separation} of the \co{participating} being, it is not a
relation, for it makes the poles not merely related, but intimately involved
into each other. To the extent various relations appear as \co{traces} of prior
\nexuss\ from which their \co{aspects} have been \co{dissociated}, we might even
say that \co{participation} is the limiting case, or rather the initial stage, when
the \co{distinctions} have not as yet resulted in
\co{dissociation}. \co{Participation} in $x$ requires $x$ to be \thi{greater
than 
me}: I do not \co{participate} in my \co{acts} -- I perform them; I do not
\co{participate} in 
my life -- I live it. Yet, this \thi{greater than me}, although \co{above me},
is not an opposite and distanced pole of a relation but, on the contrary,
something which embraces it, something very intimately mine,
eventually so much mine, as my own definition, as the ground of my very
being. We could say that \co{participation} is the relation
which is not \thi{added} to the given entities but which constitutes their very
being. But this mode of speaking tends to conflate the \co{horizontal} and
\co{vertical} dimension and, like the assumed spatial analogies of
spirituality, confuses rather than clarifies the latter. 

Knowledge is concerned with the appropriation of the alien element, it stretches
always beyond itself trying to reach what is \thi{out there}, remote and, in fact, 
inaccessible because by its very nature of \thi{givenness}, separated by the
distance.  It extends 
along the \co{horizontal} dimension of \co{transcendence}; having a fixed
\thi{subjective} pole, it now tries to extend its scope along the categories and
distinctions pertaining to its level: as a \co{subject} it reaches towards the
\co{object} and the objective; as an \co{ego} it thirsts for \co{more}; as
\co{me} it searches for what is not-mine, whether psychological insight,
subordination or understanding of others, personal love, alternative worlds\ldots

Being, on the other hand, is concerned at most with dissemination and radiation,
it does not search, it finds; it does not reach to grasp, it gives; it is quiet
and peaceful. It does not have to 
search because it already is, it does not have to climb the \co{vertical} steps
of transcendence, for these steps reflect only the perpetual anchoring of
\co{actuality} in its \co{founding origin}.  Being is the \co{presence} of the
\co{vertically} transcendent element, eventually, the \co{presence} of the
\co{origin}.\ftnt{In I:\refpp{pa:toBeDist} and then in \ref{se:toBeDist},
we said that \thi{to be is to be distinguished}. And this remains the most
generic notion. But being we are talking about now is a more specific being in a
more specific context, namely, the being of an \co{existence} in relation to 
knowing.}

Knowledge, as a relation, is always \co{founded} in being -- not of its object,
but of itself. Relation binds the \co{distinct} poles and knowledge asks only
about \thi{being} of its opposite pole. But it is the being of the whole
relation which \co{founds} it and its poles.  The \co{object} of my
understanding and the \co{concept} by which I understand it are opposite poles
of a relation, they are in no way the same. But they both originate from a
higher unity, from the \co{distinctions} made in the texture of \co{experience},
of \co{chaos}, eventually, of the \co{indistinct}. It is only \co{actuality}
which definitely \co{dissociates} the \co{subject} and the \co{external object}.
%
\noo{ If we were to speak about any opposition between being and knowing, we
  would not oppose \thi{knowing $x$} to \thi{$x$'s being}, or \thi{$x$'s being
    for us} to \thi{$x$'s being in itself}.  Instead, we would only speak (as we
  did in Book II) about two orthogonal dimensions of \co{transcendence}: the
  \co{horizontal transcendence} of \co{distinctions} of the same kind versus the
  \co{vertical transcendence} of the \co{foundation} in higher \co{virtuality}.
  Knowledge, although relation to a distanced pole, is \co{mine} -- just like
  the \co{concept}, so also its \co{object} are \co{distinguished} from \co{my
    experience}, from \co{chaos}, eventually from the \co{indistinct}.  The
  \co{object} is \co{external}, is not \co{myself}, and yet \co{I} never cognise
  and never get to know anything except myself. This \thi{sameness}, \thi{same
    kind} represent, too, the \co{horizontal} dimension of transcendence. On the
  other hand, \co{I} {\em am} not only \co{myself} but more than \co{myself},
  \co{I} am only because \co{I} have received my being and because \co{I} thus
  \co{participate} in something higher than \co{myself}.} Knowledge, as a
\co{reflective} enterprise, fixates the \co{actuality} of the \co{subject
  dissociated} from the \co{object} and keeps asking about their coincidence or,
at least, relation. In this way, it is indeed determined by the character of its
object, or rather, of that which it makes into its \co{object}. As contents are
fetched from different levels, knowledge must adjust its character to the
character of the respective contents. Thus, what we call \wo{knowledge} is much
more than what is usually so called.\noo{\ger{Erkentnise} would be only a little
  better, \he{Wissen}...perhaps...} We will now relate these abstract remarks to
various levels and recover the more common meaning as the \co{objective} and
\co{actual} form of the general notion.  The points \imm-\inv below summarise
the respective subsections
\ref{sec:levelA}-\ref{sec:levelD} from Book II. %:\ref{sec:levels}

\noo{ \say below interwave with \thi{kinds of reason}: If we let \wo{reason}
  denote the manipulation of \rss\ (probably rather the mere {\em fact}, in any
  case not any faculty), then we get variants of reason depending on the
  signification of the \co{signs}:

  \act utilitarian, pragmatic reason: concepts
  
  \mine practical reason, a bit in Kantian sense, but also much more mundane, in
  the sense of existential import: general ideas
  
  \inv contemplative reason (of Plotinus, mystics...), intellect: symbols as
  objects of contemplation; \thi{unity} intellect-intelligible, though not as
  the \co{actual signs}, but only as what they denote/reveal
}

\pa \imm At the lowest level, \co{actual} contact with an \co{object} is a form of
knowing.  Whether the \co{object} is given physically -- sensed, perceived or
felt -- or else only ideally -- \thi{thought}, remembered or imagined, in a
complete \co{externality} -- is not so significant here: it does not change the
fundamental importance of its proximity and \co{immediacy}, inscription within
the \hoa.  The constitutive feature of this, say \co{immediate} knowledge, is on
the one hand its total \co{dissociation} from its \co{object}: the \co{object}
is known (felt, seen, sensed, imagined) but remains \co{external}, that is, not
affected by the relation; and on the other hand, it is the emptiness of the
\co{actual concept} which here reduces to the pure \co{immediacy} of
\co{distinguishing} \thi{this}, the mere consciousness \co{that it is}. It is
the knowledge of \thi{this} not being \thi{that} without, perhaps, being able to
specify the difference; it is the knowledge with which I know --
\co{immediately} -- my body, without
knowing anything {\em about} it, with which I \co{immediately} know the pain
without as yet reflecting over its character and causes. 

\co{Subject}, i.e., the subject of this form of knowledge, apparently exhausts
its being in the relation to such an \co{external object}. But this is only
apparent, immanent description of the relation. \co{Subject} {\em is} not by
\co{acting} or re\co{acting} (cognising, perceiving, feeling, etc.) within this
horizon of \co{immediacy}, but only because it is immersed in the
\co{vertically} \co{transcendent} element, because it emerges from a higher level as
an \co{actualisation} of \co{ego}.
  
\act Essentially the same, though more developed kind of knowledge obtains at
this higher level of \co{actuality}, where \co{concepts} of elaborate
\co{reflection} yield understanding of \co{complexes} -- internalised through
this understanding, but appearing all the time as the residual
\co{externalities}.  This \co{dissociated} form of knowing, \gre{episteme}
(whether of \co{actual} or purely \co{immediate} kind) allows one to ruminate on
the general characteristics of knowledge and its acquisition, on the methodology
of science, on the most universal laws of reason, etc., etc., etc.  To ask any
epistemological questions, one has to assume that the crucial aspects of
knowledge can be treated independently from its object.  Such a possibility
obtains because one has already decided the scope of investigation limiting it
to the \co{objective} knowledge.\noo{ Whether it happened only with Galileo and
  Descartes or already with Aristotle and Plato does not concern us. Likewise,}
We will not dwell on the possible definitions of what, properly and precisely
speaking, might count as such a knowledge and distinguish it from believing,
imagining or assuming. We are not looking for any criteria of \co{actual} things and
relations which are never final because they are always involved into a wider
sphere capable of upsetting any relative definitions. We stop here with the
\co{objective} (\co{actual} or \co{immediate}) knowledge, at the mere
observation of its distinctive project to traverse the distance between the
\co{actuality} and its \co{more}, between the \co{dissociated} poles of the
\co{conceptual} models and the \co{external} realities.

Again, although the subject of such a knowledge, \co{ego} (or what often tries
to hide under the depersonalised entities like \thi{mind}, \thi{intelligence}),
spends its time on associating and dissociating, matching and modeling, its
{\em being} is never exhausted by such relations. \co{More} work and
thinking may generate only yet \co{more} work and thinking but it never reaches
any being. To get a sense of it, it has to notice the real person, \co{oneself}.

\mine %life is knowledge %sophia
As we move higher up in the hierarchy of being, the \co{dissociation} of the
\thi{objective} and \thi{subjective}, or -- let it be allowed to say -- of being
and knowing, becomes less and less \co{precise}.  Being an immoral
bastard is not at all affected by the fact that the person \co{actually} knows
it. Depending 
on what one knows and how, it may make the immorality more cynical and
repulsive, or else more amiable in its understanding of fallibility; in some
cases, it may even indicate a direction of possible change.  But by \wo{knowing
  that I am immoral bastard}, we refer here to the \co{objective} knowledge of
the fact \thi{that\ldots}  This knowledge actually includes knowing various
\thi{whys}, \thi{hows}, \thi{whats} but all these only signal the 
increased \co{reflection}, that is, \co{dissociation}. (The more systematic
analysis we attempt, the more confusion seems to result and, eventually, the
more all our self-knowledge seems to reduce to the mere \thi{that} from which we
started.)  It is no particular art to know \thi{that} one is a bastard, the big
art is to cease {\em being} one. Even stupidity usually knows itself to be
stupid -- it only can't help it. This gulf between knowledge and being is a gulf
between the \co{objective} knowledge and the horizon of \co{mineness}, the
distance between the \co{actual} and the \co{non-actual}.

However, as we have moved higher up in the hierarchy of being, the
\co{dissociation} became less and less \co{precise}, eventually, losing
completely its justification, if not entire sense.  \co{I} am \co{my life},
\co{my life} is \co{my world}, \co{general thoughts} are as \thi{subjective} as
\thi{objective}.  Looking for any \co{objective} knowledge at this level amounts
to reducing it to the level of \co{actuality}.  It may be quite true that I am,
indeed, an immoral bastard; it may be a fact, an \co{objective} fact. But no
such \co{objective} truths, nor any combination and sum thereof, ever capture the
truth of \co{my} being -- at best, they may express an aspect of it, approximate
it. The merging of the \thi{objective} and \thi{subjective} aspect is reflected
in a more intimate interleaving of being and knowing. Not only because here
knowing is essentially knowing \co{oneself} (possibly, another person), but
because this knowing, if knowing it still is, loses the \co{objective} character
and does not any longer know so \co{precisely} \thi{what} it knows.

The \co{general thoughts} expressed in literature or poetry, the thoughts of Vedas
or Bible, the wise advice of old men or good friends -- all these teach us
something, we learn (at least we can learn) from them and thus, perhaps,
increase our~\ldots knowledge. Living through new situations, confronting new (or
old) challenges, wining or losing, we learn something, but what is it that we so
learn? We learn how to live, OK, but this does not say much. We learn but we do
not quite know what we learn, we know more than a year ago, but we do not quite
know what we know. We gather \thi{life experience} -- we learn something about
the world as much as about ourselves, for all that amounts to refining and
clarifying the {\em relations} we have with the world and other people, and the
ways we handle them.  There is no need for making it explicit, it is knowledge
which lives in one's body and instincts, in one's instinctive reactions and habits,
in ways of responding to and initiating things,  of creating
and handling situations; only a tiny part of it becomes occasionally an
\co{object} of explicit \co{reflection} or verbal expression, and even that
happens only \la{post factum}.  It is knowledge of life, of one's life, and
one's life is only living this knowledge -- the \co{nexus}
which, once \co{dissociated}, will never return to itself and can at most strive
to reestablish some equality. To distinguish it from the
\co{objective} knowledge, let us call it the \wo{life knowledge} (or, perhaps,
\wo{\gre{sophia}}). 

It is nevertheless knowledge, for it spans the relation between \co{me} and what
is \co{mine} and, on the other end, \co{not-mine}. From this constant relation
there emerges also the residual point, the noumenal \co{self} as the center of
\co{my} being.

%\newpage
\inv %gnosis
% \co{confrontation} is a \co{nexus} of \co{separated existence} and the
% \co{indisitinct One}).
%
% Book I, \refp{pa:thinkBeingA} -- all is \co{one} before it becomes two; and section
% \ref{sub:ontoepi} -- \co{that is} is all to know about it, i.e.,...its essence
% is its being
Somewhere at the bottom, past the bottom of one's soul, and somehow, definitely
though \co{imprecisely}, \co{clearly} though \co{vaguely}, one always knows
\co{oneself}, one knows the basic mood and \co{quality} of \co{one's life}.  It
may be merely recognition of the same, recurring doubts, recognition of
something various moods of silence seem to intimate without unveiling.  But
beyond that, \co{above} all \co{visible signs}, one knows even more, one knows
also if life is a generous \co{gift} or something else: a strange accident, a
suffering of a constant trial without any goal or reason, only rarely interlaced
with brief pleasures; or, perhaps even an unbearable damnation, a doom of
eternal incarceration. This is no longer any \gre{episteme} or \gre{sophia}, any
form of knowledge which one could utilise and apply. This is not any \co{actual}
conclusion of a state, but the mere state; something that one can never
capture in the \co{actual} words but, nevertheless, something one always
expresses -- by simply being what one is and doing what one does. It is
\co{spiritual} knowledge which, at the risk of creating completely wrong
associations, let us call \wo{\co{gnosis}}.\ftnt{We are not intending here any
  \co{actual} knowledge, and hence any associations with any form of the
  traditional \la{gnosis} are out of place. In one respect, namely, in its
  complete lack of any dualism, it might be compared to the \thi{optimist
    gnosis} of neo-Platonic Renaissance which is opposed to the dualism of
  traditional \thi{pessimist gnosis} in \citeauthor*{YatesGnosis}.\noo{Still,
    our gnosis has nothing to do with any valuation or value -- it is
    inseparable from being what one knows but it is a form of {\em only}
    knowing, it still may be knowledge of good or evil which neither implies nor
    is capable of any choice.}} It is knowledge of \co{nothingness} and is
always a variation of one of the two main forms.

\ad{Knowing Yes}
%\item (Being = knowing (in \yes))
Eventually, at the \co{spiritual} level -- the level where there are not only no
\co{objects}, but no \co{distinctions} which could \co{actually signify} any
\co{invisible} meanings, the level raised not only \co{above} the earth but also
\co{above} heavens, like a wind which \citeti{bloweth where it listeth, and
  thou hearest the sound thereof, but canst not tell whence it cometh, and
  whither it goeth,}{John}{III:8\label{ftnt:wind}}  \citet{invisible [...] to mortal eyes, beyond
  thought and beyond change}{Bhagavad}{II:25} -- at this level being is indeed
knowing.  This knowledge, however, must not be confused with \co{actual}
knowing, \co{reflective} consciousness, which are only possible but never
necessary \co{manifestations} of the \co{spiritual} state. Being and knowing are
the same not because the \co{dissociated subject} and \co{object} mysteriously
coincided, but because the {\em knowing} and the {\em being of the one who
  knows} are here indistinguishable; because \co{nothingness} of the \co{self}
is \co{confronted} exclusively with \co{nothingness} of the \co{origin}, and
thus there is no longer any distinction between the \co{horizontal} and
\co{vertical} dimensions of \co{transcendence}; because the only possibility of
this level is \co{participation} in the \co{one}.  \citetib{He who knows I am
  beginningless, unborn, the Lord of all the worlds, this mortal is free from
  delusion, and from all evils he is free. [...] He who knows my glory and power, he
  has the oneness of unwavering harmony.}{Bhagavad}{X:3-7} This is not any
\thi{intellectualism}, any reduction of morality or salvation to knowledge, for
the knowledge concerned is \co{gnosis}, the knowledge of and through being, of
and through \co{participation} which is nothing else but a \co{concrete} form of
this very \co{participation}. The \co{actual} knowledge of such truths is only
abstract, it does not involve understanding and serves only at best as a
preliminary stage, where the \co{aspects} belonging inseparably together are
still \co{dissociated}.

%not actual know \subpa 
In II:\refp{pa:thinkBeingB}, we pointed to the primordial \co{spiritual} unity
of being and thinking. This unity was not the result of any extension of
\co{actual} knowledge, of any gnostic insights into the structure and details of
\la{pneuma} and \la{pleroma}, nor of any idealistic coincidence of \thi{subject}
and \thi{object}. On the contrary, it was the result of a total lack, of the
poor \co{nothingness} which, offering only \co{nothing}, does not provide any
ground for \co{distinctions}, not to mention \co{actual} knowledge.  In terms of
\co{actuality}, this knowledge (if knowledge it is) is a mere knowing \co{that},
but \co{that} on the opposite end of the \co{immediate consciousness}
\thi{that\ldots}, \co{that} which is not concerned with any \co{actuality}. It
only knows itself to be an \co{existence}, a \co{confrontation}: \co{that
  actuality} is immersed in the \co{non-actual} element, \co{that I am not the
  master}, \co{that is}.

In terms of \co{actuality} it is at best only a \co{vague} premonition,
\co{imprecise} sense of \thi{Something} and, sometimes, simply the straight
negation, as when one realizes that, knowing \co{that}, one does not
\co{actually} know anything specific. 
%(\wo{suppression}, as a psychoanalyst would say).
It does not imply \co{actual} consciousness and understanding, it is not
exhausted in the \co{actual signs}, it is not by any necessity manifested in any
particular, \co{actual} form.  Its \co{sign} is rather silence, as when
\citetib{the sage of silence, the Muni, closes the doors of his soul and,
  resting his inner gaze between the eyebrows, keeps peaceful and even the
  ebbing and flowing of breath; and with life and mind and reason in harmony,
  and with desire and fear and wrath gone, keeps silent his soul before his
  freedom, he in truth has attained final freedom.}{Bhagavad}{V:27-28} As a
relation to the \co{absolute}, it is an \co{absolute} relation, relation which
does not involve any \co{distinctions} except for the primordial
\co{separation}.
% and which only may -- but need not and seldom does -- express itself in
% particular \co{actual} states.
It is not an event of insight, nor is it an insight which leaves any permanent
certainty about some \co{object} -- it only leaves permanent certainty. It is a
lasting, \co{spiritual} state which permeates the whole \co{actuality} with its
\co{traces}, which sometimes even \co{actualises} in the revealing \co{signs},
but which, primarily, is known only in the sense of permeating the whole
\co{actuality} with its aura, of putting its \co{invisible} stamp on it, which
one is unable to grasp and only realizes its \co{presence}, its bare \co{that}.

% \citt{immediate self-knowledge: \thi{just like one senses one's health or pain},
%   in the direct \he{wsluchiwaniu sie} feeling into oneself.}{\'{S}ankara,
%   {Commentary to Gita, IX:2 [after Otto, Mistyka... p.46]}}
\pa
This \co{spiritual knowledge} is the being of \co{spirit}. It is knowledge
because it is relation (\co{separation} which \co{founds confrontation}), but it
is being because this relation is \co{absolute}, because \co{confrontation} is
simply \co{participation} in \co{one}, because there is no longer any
distinction between the \co{vertical} and \co{horizontal} dimension of
\co{transcendence}.  In terms of \co{actual} understanding, it is simply knowing
\co{that}: \co{that} \co{I} \co{exist} through \co{confrontation}, \co{that I am
  not the master}, perhaps, \co{that} \co{I} know \co{nothing}.
%and certainly nothing of the eventual significance. \co{Actually} {\em knowing}, b
Being \co{actually} 
certain of any such \co{that} points in the direction of \Yes. But if one asked:
\wo{How can anybody be certain of such a thing?} then one can only keep asking, 
for \thi{how} asks already for more (that is, less) than the \co{absolute
  nothingness}. One might only remember that there are many degrees of knowing
\co{that} and the more definite interpretation man attempts, the further
one is from actually knowing it.  For the \co{actual} claims to verifiable
certainty, it remains only an indefinite sense of gratitude and thankfulness 
which does not present, let alone fill, the soul with any thing nor image.

%\subpa %(Rather inverted)
\co{That} is never given, \co{that} is something one can never see or feel, for
any \co{actual} state or feeling is at most a \co{sign} which, unless already
{\em known}, is most naturally ignored and reduced to insignificance. Thus
it is something easily contradicted by all kinds of \co{visible} examples and
arguments.
% -- Voltairean grimaces at Panglos' disasters and 
% optimistic naivit\'{e} seem so obviously convincing that they will always appeal
% to adolescent \thi{rationality}.
Never being given, never being any \co{actual} \thi{what}, such a \co{that} is
ignorance rather than knowledge.  \citeti{You will so ask: what does God effect
  without `image' in the foundation and essence of the soul. I am not able to
  say that, because soul's faculties can perceive only through `images'. And
  because the images enter into her from outside, it remains hidden from her.
  And this is most salutary for her, because this ignorance tempts her with the
  mystery of something wonderful and makes her chase it. For she feels very well
  \co{that} it is, but does not know \thi{what} it
  is.}{Eckhart}{\citaft{Mistyka}{ A:II.4.b\kilde{p.36} my emph.}}  \co{That} is
beyond any \thi{what} and need no \thi{what} -- and this is all to be known
about \co{that}. Consequently, \co{that} can not be known without being it,
because as long as I am not \co{that}, I really do not know it, my \co{actual}
pseudo-knowledge of \co{that} never goes beyond the insecurity of a merely
possible hypothesis, a vulgar idea of faith. The apparent paradox would be that,
since \co{existence} is \co{confrontation}, every one knows \co{that}, too.
Indeed, but in \co{actual} terms this \co{spiritual knowledge} has rather the
form of ignorance, it takes only the form of a \thi{merely possible hypothesis},
the one which is as difficult to accept as it is to get rid of. It is expressed
by \co{thirst} and perpetual search for the ultimate. It is, in short, 
unknowing knowledge -- a mere relation to the \co{absolute} without any
reflection in the \co{actuality}, a being reduced to knowledge which does not
know what it is, which keeps searching for \co{more} without realizing how
little there is to be known. For in order to {\em know} it -- not merely imagine
it as a possibility, posit it as a thought experiment but \co{actually} know --
one has to renounce the project of \co{actually} knowing, that is, dominating
it, one has to {\em be} on the \yes-side.  Living \No\ it is impossible to know
\Yes, for living \No\ means precisely accepting only \co{visibility} and denying
any meaning to the ultimate \co{transcendence}. Living \Yes, on the other hand,
is nothing more than renouncing \No. But what is renouncing other than
dismissing what it renounces as unworthy or untrue, that is, knowing better?
Living \Yes\ may have almost any imaginable \co{actual} form and may involve any
kind of \co{actual} knowledge. For the knowledge it {\em is}, is not concerned
with anything \co{actual}, it is not knowledge of this or that, it is not
knowledge {\em of} anything. It is only the \co{vaguest} peace of the ultimate
\co{participation}, the peace which is known because it marks all the
\co{actuality} with its \co{invisible rest}, yet which itself never
\co{actualises}. If you try to point at it, to capture \thi{what} of this
knowledge, it evaporates, but if you let it be, it remains with the most
\co{clear presence} and \co{absolute} certainty. \citet{In what concerns divine
  things, belief is not fitting. Only certainty will do. Anything less than
  certainty is unworthy of God.}{WeilWaiting}{Forms of the Implicit Love of
  God:Implicit and explicit love;p.139. Such a \co{spiritual knowledge} amounts
  to (a variant of) what has been called \wo{ontologism}, for instance, of
  St.~Bonaventure and his followers. Indeed, (1) the \co{absolute} is the first
  not only in the order of Being but also in the order of knowing (\co{gnosis}
  is the constant knowledge of Godhead, if not of God.)  \noo{(the Aristotelian
    principle \la{Nihil est in intellectu quod prius non fuerit in sensu} is
    wrong) i,e.,} (2) This knowledge is intuitive, not abstractive -- it really
  coincides with the fact of \co{existing}. The last point, (3) that in the
  light of the idea of \co{absolute} we acquire all other ideas, can be taken or
  rejected, depending on the meaning of \wo{idea}. If it is an intellectual
  construction, proceeding from the \co{reflective dissociations}, then only its
  strife after unity reminds about this first knowledge. But if idea is taken as
  anything whatsoever which can be distinguished and conceptualised, then
  certainly it arises only from the \co{indistinct}.\noo{(like everything
    else)... [St.~Bonaventure seems to have been similarly ambiguous on the
    issue; cf.~Gilson.]} }

\newp 
\ad{Knowing No}
The \co{existential confrontation} is nothing else but this \co{participation},
and even the most active \No\ is only turning away from it, and thus is still
\co{confronting} it.  It might thus seem that \No\ is a possibility of the same
order as \Yes, that it is but another alternative. It is, however, an event of a
different order -- it does not reach the \co{absolute}.  It only denies:
refusing to be through \co{confrontation}, refusing to \co{participate} in the
\co{origin}, it seems to remove this \co{transcendent} pole but, as a matter of
fact, it only pushes it away at an inaccessible distance, reduces it to a mere
relation, exchanges being for {\em mere} knowledge.
  
\citet{To despair is to lose the
  eternal}{Despair}{I:C.B.b.$\alpha$.1\kilde{p.82}} says Kierkegaard.
Seemingly, one might know that one lost the eternal, but what {\em knowing} is
that? What do \co{I} {\em know} then, what is it that I so {\em know}?  Nothing,
except my \co{actual} state, feelings of despair, irrecoverable loss, sense of
damnation -- I know the \co{distance}, impassable, from here to eternity which
appears as the ultimate emptiness.  My knowledge may be quite correct, I may
know how it feels and even what I feel but is it~\ldots knowledge? In fact, I
have hardly any idea of what is going on, for it is impossible to lose the
eternal, simply and plainly impossible. We live in it, whether we feel it or
not, and the despair of the loss is the despair over one's own \co{actuality} --
not over the loss of the eternal but over the loss of the contact with it, of
its \co{concrete presence}.  Knowledge of the loss, of having lost the eternal,
eventually, knowledge of being in hell, will seldom call things with such words.
But it knows them because it lives them, it lives its \No, and no matter how
rosy all \co{actual} things really are, at the bottom it is scared by the
gnawing suspicion of their insufficiency, of a mistake, of the \co{distance}
which no longer relates but divides. The lost eternal is the lost \co{spirit}
or, as one used to say, the sold soul. But it is lost in spite of the fact that
it can not be lost; it is only the feeling of loss one despairs over. The
despair is real because it is deeply felt, but the loss over which it despairs
is completely untrue.\noo{\citeti{The light of the body is the eye: if therefore
    thine eye be single, thy whole body shall be full of light.  But if thine
    eye be evil, thy whole body shall be full of darkness.  If therefore the
    light that is in thee be darkness, how great is that
    darkness!}{Mt.}{VI:22-23}} Despair is indeed to say something like \wo{I
  have lost the eternal} and think that it can count for \thi{knowing}, that it
can mean anything more than a mere status report of one's \co{actual} state and
feelings.
  
%\subpa %(Inverted)
As with the \co{spiritual knowledge} of \yes, we are by no means implying that
one always has full consciousness of \No.  For the most, we \co{actually} do not
know it.  This was the reason for objecting in \refp{pa:notKnow} against the
\co{actual} knowledge as a necessary \co{sign} of the \co{spiritual} state.  The
\co{actual} knowledge will often be an \co{inversion} of the factual
state.\noo{\citet{[T]he more consciousness the more intense the
    despair.}{Despair}{72} says Kierkegaard, but he also admits that the first
  stage of despair is exactly the state of not realizing that one is in
  despair.} Yet, \co{actually} knowing \No\ can be also something like a
prevailing sense of \co{ingratitude}, disappointment, meaninglessness or
unreality, which in the \co{reflective} form turn into negation of some of the
earlier mentioned \co{that}'s: \co{I} control my \co{existence}, \co{I} am the
master, \co{I} know. In the extreme cases, it may be also \co{actual}
realization that I am damned, which opens the doors to deeper hell. As hell,
with its despair over mere \co{visibility}, is much closer to earth than the
\co{invisible} paradise, it is easier to imagine a kind of certainty -- coming
close to the \co{actual} knowledge -- of being in the former than in the latter.
Any prolonged suffering gives an intimation and, however inadequate, an image of
it.  One has observed that traditional representations of paradise -- whether in
painting, sculpture or literature -- are unbearably dull and monotonous as
compared to the fascinatingly complex and eventful representations of hell.
This could be classified merely as a result of the simple psychology of
mass-media and news reports (according to which devastating tragedy, being more
intriguing and fascinating, sells better than peaceful happiness). We would,
however, see here an expression of a deep difference: heaven generates few, if
any, univocal \co{visible signs}, it does not inspire our imagination with any
definite images as hell does. And there is good reason for it: the former is the
point of ultimate \co{invisibility} embracing {\em everything} \co{visible}, and
then the bare reflection of \co{nothingness}, while the latter is exactly its
negation, not only directedness towards \co{visibility}, but {\em exclusive}
directedness towards \co{visibility} and {\em only} \co{visibility}. The
distinction between directedness and exclusive directedness (toward \co{this
  world}) is impossible to define \co{precisely}, and so the gnostic tendency to
identify \co{this visible world} with the source of all and only evils (if not
with the hell itself) is and will be the constant theme in the history of
spirituality.

\ad{Not a mystical state -- its foundation} Since this might look like some
obscure mysticism, let us clarify one crucial point.  That an \co{existence} is
a \co{confrontation} with the \co{origin} can now be said equivalently as: an
\co{existence} {\em is} its \co{spiritual knowledge}, for this knowledge is
nothing else but living it, whether one knows it \co{actually} or not.  What we
are intending must not be confused with any particular form of \co{actuality},
whether \co{actual} insight or some mystical states.
%
% which, after many mystics, James characterises as
% having a \thi{noetic quality}, for \citet{although so similar to states of
%   feeling, mystical states seem to those who experience them to be also states
%   of knowledge.}{Varieties}{XVI\&XVII}
%
Such states are often described as experiences of union, if not identity, 
experiences in which the seer \citet{cannot then see or distinguish what he
  sees, nor does he have the impression of two entities; rather, it is as if he
  has become someone else, and no longer \co{himself}.}{Plotinus}{VI:9.10} But
such peculiar \co{experiences} are not needed. They are only
\co{actual signs} announcing a new mode of experiencing \co{open} to the
\co{absolute presence} beyond any \co{actual} contents. Like return from the
Underworld (which itself can be such \co{an experience}), they only give an
\co{actual} expression to something which can be lasting and effective also
without such intense \co{experiences}. 

\co{Nothingness} of the \co{self} is the \la{imago} of \co{nothingness} of the
\co{origin}.  \citeti{Just like Godhead is nameless and remote from any name, so
  also the soul is nameless. For it is the same as
  God.}{Eckhart}{\citaft{Mistyka}{ A:I.3.c \kilde{p.23}}} But \co{unity} is not
union -- in any case, not any understood flatly as some experienced coincidence
with the \co{absolute}. With Eckhart, too, God (\la{deus}) is not the highest
principle. His \la{Eynikeyt} is only soul's identity with it and not with
Godhead (\la{deitas}). But \citetib{God and Godhead are as different as heaven
  and earth. But heaven is still thousand miles higher. And so is Godhead above
  God. God becomes and passes by.}{Eckhart}{{A:I.3.e\kilde{p.25/26}}} We are not
considering any \co{actually} experienced coincidence. We are considering the primordial
\co{unity} of the two dimensions -- \co{vertical} and \co{horizontal} -- of
\co{existence}, of being as \co{participation} and knowing as \co{gnosis}, which
only when \co{dissociated} in terms of \co{objective} knowledge, give raise to a
search for \co{actual} experiences.  Once \co{dissociated}, it is impossible to
imagine the \co{unity} of the two, for imagining requires images while here
every image is inadequate.\ftnt{It becomes equally impossible to decide the
  order of dependence: whether being precedes knowledge or other way around.
  The first sentence of the following suggests the former, while the last one
  the latter: \citef{It is not possible for anyone to see anything of the things
    that really exist unless he becomes like them. This is not the
    way with man in the world: he sees the sun without being a sun; and he sees
    the heaven and the earth and all other things, but he is not these things.
    This is quite in keeping with the truth. But you saw something of that
    place, and you became those things. You saw the Spirit, you became spirit.
    You saw Christ, you became Christ. You saw the Father, you shall become
    Father. So in this place you see everything and do not see yourself, but in
    that place you do see yourself - and what you see you shall
    become.}{Philip}{}} This inadequacy, when experienced as the absence and
lack, gives rise to the search for an \thi{adequate} state, for a \thi{true}
experience. But \citet{[t]he mistake lies precisely in the search for a
  particular state,}{WeilWaiting}{Forms of the Implicit Love of God:Love of the
  order of the world;p.111} in the attempts to merely exchange one
\co{actuality} for another.

\co{Experiences} of \la{unio mystica} are conditioned by the \co{openness}
to the \co{gifts} of the \co{origin}, by the acceptance that \co{I am not the
  master}.  Such \co{experiences} are probably the most intense \co{signs} of
\co{self} and its direct \co{confrontation} with \co{one}.  In a sense, one has
to \thi{forget \co{oneself}} in order to become 
\co{self} and reach the \co{spiritual} purity. But \thi{forgetting oneself} does
not mean \thi{wiping oneself out}, only wiping out the \co{attachment} to
\co{mineness}, the preoccupation with what is \co{mine} and what is not. As
Gabriel Marcel would say: stopping the insistence on {\em having} (things, experiences)
marks the beginning of {\em being}. It is still \co{I} who {\em am},
\co{I} who {\em am} experiencing, but the \co{concreteness} of this {being} of
\co{mine} means only \co{humble} appreciation of the experience without
dissolution in it. If this sounds like a paradox, then probably it is one,
because it means suspension of the \co{dissociation} into \co{subject} and
\co{object} and termination of the \co{objectivistic illusion}, according to
which everything is an \co{object} and \thi{real} are only \co{actually visible},
only \co{precisely} discernible \co{objects} or feelings and not also some
\co{vaguely present rest}.\ftnt{Both, apparently contradictory aspects, of
  \thi{forgetting oneself} and remaining oneself, are contained
  in the short formulation: \citef{How, in fact, could God be all in all things
    if there remained in man something of man? His substance will no doubt
    continue to be, but under another form, another power and another
    glory.}{BernardLove}{X}\kilde{GilsonHCPh,p.165} According to St.~Bernard, the
  union amounts to harmony of wills, not to any confusion of substances.}


The \co{spiritual} dimension of \co{self} does not in any sense abolish
\co{myself}. The \co{unity} is not, in any case not necessarily, any felt and
\co{actually experienced} union, and it obtains whether one has such mystical
\co{experiences} or not. Having them can at most make the otherwise only
\co{vaguely} felt \co{presence} more \co{clear}.  \citet{Neither temporality nor
  the feelings of the soul are central to this state [...] The duration of the
  state is not significant to Plotinus, nor are the phenomenological
  characteristics evinced in the experience. These would be distractions to the
  soul's true task, for they lie, in Plotinus' taxonomy, at the level of
  sense-perception, an outwards directed aspect of the psyche. [...] It does not
  become something else, nor does it become absorbed at some moment into the
  One. Its union with the One is something that always obtained, but it had
  hitherto failed to grasp this fact adequately.}{PlotMystB}{III}

  \noo{added and inserted from II
    
    our \thi{mysticism} (if any) does not refer to any \co{actual experiences}
    of union, ecstasy, etc. -- it is consummated in \co{non-actuality}, whole
    life... [sensible-to-intellectual: awakening IV:4.5,8.1; or violent uplift
    V:3.4; then also -to-One:...]; though scholars dispute (e.g., \wo{By dint of
      the exercise of God's presence, divine union becomes continuous.
      Contemplation of the world of Forms and the experience of the love of the
      Good are no longer rare and extraordinary events. They give way to a state
      of union which is in a sense substantial, as it sizes our being in its
      entirety.}{Hadot, Plotinus on the Simplicity of Vision, 71 [after,
      Bussanich in ACPQ, LXXI; p.358]})
  
    {\bf My self does not abolish myself}
    
    \pa The tension which obtains all the time between \co{myself} and \co{my
      self} witnesses to the mistakes in understanding \la{unio mystica} in the
    sense of any coincidence or \co{actual} unity. It is \co{an experience} of
    the unity and as such \co{an actual experience}, it is also the proof of its
    own failure, witness to the difference between this \co{actuality} and
    \co{non-actuality}.
    
    \co{Invisibles} do not abolish \co{myself}, they only \co{transcend} the
    level of \co{mineness}.  To the extent that \co{invisible manifests} itself,
    it does so for \co{me}.  And then, even if it does not \co{manifest} itself
    as a \co{command} or \co{inspiration}, it still presents a challenge to
    \co{myself}, the challenge put by \co{my self} to \co{myself}.  Like with
    everything that visits the \hoa, it is \co{I} who am confronted with it,
    \co{I} have to try to understand something of it, of its relation to its
    \co{signs} and \co{my} relation to it.  And, unless \co{I} have received
    very particular gifts and favours, the only way for \co{me} is to work
    through the \co{signs} toward what they reveal.
    
    \co{Invisibles} \co{manifest} themselves to \co{me}, it is \co{I} who
    \co{re-cognise} or experience happiness, love, beauty.  This may lead
    \co{me} to believe that they have a \thi{subjective} character, that only
    \co{I} partake in them, or else that others do while \co{I} do not.  But
    there is a difference between observation, thinking or feeling and following
    a \co{command}.  One may observe other's happiness without experiencing it,
    without taking part in it, perhaps even envying it, that is, without
    experiencing this happiness as \co{incarnated}, without it being
    \co{revealed} to one.  Observing and envying are, however, acts of
    \co{myself} which belong to \co{this world}.  They are instances of the
    \co{separating} activity of \co{myself}, of the \co{dissociating} activity
    of \co{reflection}.
    
    \tsep{end inserted from II}
} % end \noo added and inserted
  
% The mystical knowledge, as a particular insight or state, is an
% \co{actualisation} of the \co{spiritual knowledge}, an \co{actual experience} of
% \Yes. But \co{spiritual knowledge} has a lasting permanence transcending any
% kind of \co{actual experiences}, feelings, illuminations, ecstatic states.

The \co{spiritual knowledge}, being a lasting state rather than a feeling, does
not require the intensity of such \co{actual experiences} and, being essentially
\co{non-actual}, it does not have to involve explicit consciousness. It is
knowledge, but not any \la{episteme}, one can suggest it, but not grasp it, one
can indicate \co{that}, but not describe any \thi{what} for all one
\co{actually} knows is \co{nothing} and its eventual expression is silence.
%
% \la{Gnosis} is knowledge, but knowledge inexpressible and incommunicable in a
% direct way, for all it \co{actually} knows is \co{nothing}.
%
\citeti{It is only when you hunt for it that you lose it\lin You
cannot take hold of it, but equally you cannot get rid of it\lin And while you
can do neither, it goes on its own way.\lin You remain silent and it speaks; you
speak, and it is dumb.}{Yung-chia Ta-shih}{\citaft{Huxley}{ I\kilde{p.9}}} Knowing thus
\co{nothing} is the same as being \co{nothing}, 
and as such completely different from \co{actually} knowing {\em of} or {\em
  about} \co{nothing}.

\pa One could recall here a long series of examples: Augustine's observations
that \wo{one knows God better through ignorance}, and that one does not have any
knowledge of Him \citet{except that which knows that it knows nothing about
  Him}{AugustOrder}{II:16.44, II:18.47 \citaft{FilSr}{\kilde{p.68}}}; Spinoza's
\la{amor Dei intellectualis} comes probably close to what we intend, at least in
the mood, if not in the exact content. Eckhart's \thi{unknowing knowing}, or
\la{docta ignorantia} of his dedicated reader Cusanus, are of the similar kind.
% Now, such a knowledge (a mystical insight, if you like), turns out to be only an
% \co{actual sign} of \la{gnosis}, of the \co{spiritual knowledge} which is the
% same as \co{spiritual} being.
In the XX-th century, Karl Rahner advanced forcefully the thesis of the positive
content of \thi{incomprehensibility} as the primary name for God.  Even
one thinking that the world is simply the \co{totality} of (logical) facts amenable to a
complete analysis may be tempted to say: \citet{What we cannot speak about we
  must pass over in silence,}{Tract}{7} thus repeating the credo of negative theology:
\citet{and the hidden Mysteries which lie beyond our view we have honoured by
  silence}{Celestial}{XV}, \citet{God is invisible and beyond expression by words
  [... We apprehend] not what He is, but what He is not.}{Stromata}{V:12, V:11}
%However, when it comes to the ultimate \co{origin}, t
There is, however, a big difference between \thi{honouring by silence} because
one does not understand what -- but expects that something -- is hiding behind
the veil of one's concepts, or else because the veil marks only the
\co{absolute} border between the differentiated and the \co{indistinct}.  The
adjective \thi{hidden} is therefore useful only in a metaphorical sense: the
\co{origin}, God -- or, following Eckhart's distinction,\noo{to which we will
  return in Section \ref{se:GodGodhead},} the Godhead -- is hidden because no
\co{visible} categories of understanding, requiring \co{distinctions}, are
applicable to the \co{indistinct one}. It is hidden but it is not hiding. For
\co{reflection} this difference is crucial because we thus limit the
understanding of the \co{invisible origin}, \la{Deus absconditus}, merely
against all differentiation, without suggesting any particular contents, not to
mention \co{objective} contents, which might be hiding there.

% One must only avoid the tendency to interpret it in any \co{actual} manner, as
% some inadequate knowledge, as some failed attempt to define the indefinable
% which, nevertheless, hides something definite. \co{Nothing} is not only all that
% {\em we can} know, but all that {\em there is} to know.
\noo{ \say Forms of unity? : \act actual experience (mystical experience,
  James...); \mine the lived unity (?) (not necessarily accompanied by any
  experiences) but still in separation; \gre{sophia}; \inv the simple unity of
  \co{Self} and \co{One} -- \co{nothingness}, \gre{gnosis}

-- more specific, differences ? Scholars move away from psychologism of actual
   \thi{mystical} experiences \la{a la} James, and consider \inv rather than
   \act in Plotinus: \citeauthor*{PlotMystA,PlotMystB} 
}


\sep

\pa Salvation is not enlightenment, yet knowledge taken in the deeper sense, the
\co{spiritual knowledge that}, is an integral part of the transition. The
transition amounts to this \co{spiritual knowledge} becoming, in one form or
another, \co{concretely present}. This knowledge amounts to recognition of
relativity of all \co{visible} things and here relativity means complete
irrelevance for the \co{absolute} or, let us pretend some circularity,
insufficiency for salvation. None of these things can help in finding an exit
and the \co{awareness} of this insufficiency, the renouncement of \co{idols}, is
what was traditionally called \wo{purity}. The most \co{concrete} way to this
\co{awareness} goes through suffering -- yet, suffering which does not break
one's person but liberates, that is, purifies one.  Return from the Underworld
is the extreme image of the most extreme form of it.  One who has suffered {\em
  knows} what suffering is and \thi{what} of suffering involves unmistakably
\co{that}. For in the face of suffering all \thi{whats}, all reasons,
explanations, justifications, correct as they may be, lose all relevance. They
may serve for future considerations how to avoid further sufferings but they
cannot help the one who is suffering now. Behind this simple and painful
observation that one is suffering, there hides an even deeper \co{that}.
%For who knows, \co{actually} and \co{objectively},  \thi{what} suffering is?
Suffering, depriving one of all \co{visible} hope, can break the \co{soul} which
henceforth becomes capable only of clinching desperately to the reminders 
of its \co{visible} world as they turn gradually into insignificant emptiness.
But one can also manage to go through it unbroken, 
which is possible only when one leaves all \thi{whats} where they belong and
finds the source of strength \co{transcending} all \co{visibility}, beyond the
doors opened miraculously in response to one's \wo{Release me\ldots} The
\co{self}-respect resulting from such a trial has nothing to do with any
feelings of complacency and satisfaction with \co{oneself}. It is a true respect
for \co{self} which, erasing all pretensions to one's ultimate power and
abilities, makes one admit \co{that I am not the master}. Such \co{that}, reached
in suffering, is a liberation of a purified soul. 
%  The more specific and \co{precise} such
% \thi{whats} become, the less understanding of suffering they convey. 
% Suffering can not be explained to one who did not suffer. Or let us put it
% better: {\em if} there were a person who did not suffer, one could not explain
% to him what suffering is.

\pa
James distinguished between the religious attitude of once-born and twice-born.
The first represents \citet{the healthy-minded temperament, the temperament
  which has a constitutional incapacity for prolonged
  suffering,}{Varieties}{VI\&VII} for whom \citetib{the world is a sort of
  rectilinear or one-storied affair, whose accounts are kept in one
  denomination, whose parts have just the values which naturally they appear to
  have, and of which a simple algebraic sum of pluses and minuses will give the
  total worth. Happiness and religious peace consist in living on the plus side
  of the account.}{Varieties}{VIII}
% As far as I can tell, such people are simply a psychological invention, and a bad
% one for that, since built on a purely external view.\ftnt{I have to admit, though, that
% examples of sectarianism or flat idealism quoted by James give
% enough sense of superficiality. We won't analyse his reasons for classifying them
% among religious attitudes.}
\citetib{Now in contrast with such healthy-minded views as these, if we treat
  them as a way of deliberately minimising evil, stands a radically opposite
  view, a way of maximising evil, if you please so to call it, based on the
  persuasion that the evil aspects of our life are of its very essence, and that
  the world's meaning most comes home to us when we lay them most to
  heart.}{Varieties}{VI\&VII}

Although the world is indeed full of tempting superficialities and charming
platitudes, \thi{a constitutional incapacity for prolonged suffering} is almost
a contradiction in terms. One could, perhaps, imagine some people whose goals
and thoughts were so simple and means so rich that they could obtain everything
they wished.  But it seems like an idealized image, for everybody lives his
\co{confrontation} and knows \co{that}.  Many a one will try to escape from
suffering, but suffering is exactly something from which there is no escape.
\citet{For there cannot be a \thi{suffering} at all except where something
  happens against one's will; no one \thi{suffers} when he accomplishes his will
  and when what happens delights him.}{AbelardEthics}{\para 18} If you can
escape from it, it may be a problem, a trouble, but not a suffering. A
\thi{constitutional incapacity for suffering} might characterise a superficial
character from a \abr{tv}-series or a suicide candidate, not a healthy-minded
temperament.  Suffering is one of the fundamental forms of meeting the ultimate
\co{transcendence}, of being sentenced to something one can not stand and, by
the same token, being called to \co{transcend oneself}. \thi{Healthy-mindedness}
can in this context refer at most to the ability to face suffering and the
strength which does not yield to its possibly devastating influence. We do not
make the distinction between those who can, or even need suffer, and those who
can not -- if it is a meaningful distinction, it is only psychological.\ftnt{One
  might probably look at James' personal fascination with Emerson, whom he
  mentions as an example of \thi{healthy-mindedness}, and the opposition between
  Emerson's idealized strength and James' own post-war, depressive years, as the
  background of this distinction.} We distinguish between those who suffer but
do not end in hell and those suffer and do.  Hell is the extreme form of
suffering, the place where the suffering, initially human, becomes, by being
answered with the active \No, almost inhuman, impersonal. \Yes\ does not require
a descent to such depths.  Yet, besides the fact that suffering is a common lot
and that some can find themselves at its bottom, the opposition to hell provides
the most clear illustration of the dynamics as well as the meaning of the \sch\ 
of \Yes. For just like \No\ ends up substantialising its suffering as
\thi{objective} evil, so \Yes\ learns that salvation is not the liberation from
suffering but from and that the former need not involve the latter.  We will
therefore follow this opposition in describing the choice of \Yes.


\subsection{Yes}\label{sub:yes}

%\tsep{inserted from II}

\pa\label{pa:manifest1} Our vocation is to listen, not to talk; to listen to the
silent \co{presence} which fills our life with all its contents.  And
\citeti{not knowing how to listen, neither can one speak.}{Heraclitus}{DK
  22B19} At the deepest level of our being, at the point where \co{one} becomes
many, the \co{invisibles} become \co{present} in the ways which we can hardly
feel, and never control.  Without any \co{reflective attention}, the
\co{invisibles} are \co{present} as the most constant \co{distinctions} -- not
moods, not feelings, not thoughts -- which do not have any \co{objective}
content and which do not pertain to any \co{object}; which therefore can be
predicated of everything, though we will tend to ascribe them to the most
general ideas, to life, world, existence.


The experience \co{that I am not the master} is what makes true listening
possible.  Since my control over all \co{visible} things does not exhaust my
life, there are, perhaps, other voices worth listening to. In the \co{reflective
  experience}, the \co{invisibles} can be \co{present} as {unreal} dreams,
{impossible} ideals, something we long for without any hope to obtain it -- not
because we are unable to hope, but because we are unable to imagine \thi{what}
we are hoping for.  Such dreams turn out to be much more \thi{real} than all the
\thi{reality} of \co{actual} objects and situations, not only because they are
more persistent but also because they persist with the calm and yet intense and
irrefutable force.  They do not go away as long as what they announce does not
find a \co{concrete}, incarnate \co{presence} in our life.  They are only
images, always false ones, but these images remind us of something which,
apparently forgotten, remains \co{present} \co{above} all our \co{acts} and
\co{activities}, \co{above} all \co{visible} and \co{invisible distinctions}.
\citet{[...]  and St.~Philip said: `Lord, show us the Father and it will suffice
  us', for the `Father' indicates birth and not likeness, and it denotes the One
  in which likeness is reduced to silence and all that desires to be is
  still.}{EckDivConsol}{II\kilde{p.73}}


Imagine a man whose whole life was, by any reasonable standards, a series of
failures and disasters, an unhappy, unrealized, misfortunate life.  Then, when
his last moment came, on his death bed he says: \wo{I had a good, gratifying
  life.} And it is not misunderstanding of the words, it is not any
self-deception trying to endow the last moment with the value which all earlier
moments missed.  Do you think it can possibly make any sense? 
-- like a moment of {revelation}, when the \co{invisible} sense of his life
becomes \co{manifest}, when he realizes that this life was worth living, that
the very fact of living is gratifying and deserves thankfulness.  A moment when
one hears \citeti{Verily I say unto thee, To day shalt thou be [...] in
  paradise.}{Lk.}{XXIII:43} In spite of the wretched life, in spite of constant
misfortune, \wo{to day shalt thou be in paradise}. A \thi{moment of
  truth}? But the same might have been true all the time, all his life, even if
the man never \co{recognised} any \co{sign} of it.
%His life was an \co{incarnation} of \co{invisible}, here \wo{on 
%earth as in heaven}. 
\noo{Such \co{signs} may, sometimes, make this fact more \co{visible}, but even
without them it is the basis of all \co{experience} and hence, the most basic
\co{experience}.}

%\tsep{end inserted from II}

\pa
We said that living \No\ one can not say \yes, just like living \yes\ amounts to
denying \No. But then, having once ended in the \No, perhaps, at the bottom of
hell, one can not say \Yes, so the story ends there. Perhaps, and often it does, but
it is interesting -- or, we should like to say, miraculous -- that it does not have to. 

Nihilism, despair, the deepest circles of hell are all consequences of
\co{attachment}, eventual consequences of the declared and exclusive dependence
on \co{visibility}.  And thus, there can be no cure against them, for the only
medicine one could possibly accept would have to be \co{visible}. Insisting on
objectivity or inter-subjectivity, on externality or demonstrations, one wants
only one thing: to be convinced {\em in advance}, that is, to avoid the
discomfort of trust, to accept only what one finds acceptable, to keep one's
life unchanged. But he \citeti{that loveth his life shall lose it; and he that
  hateth his life in this world shall keep it unto life eternal.}{John}{XII:25;
  Mt.X:39; Mk.VIII:35; Lk.IX:24, XVII:33}
%Losing one's life need not be the same as descending to hell, though certainly
%the latter implies the former.
The cure is only the change of fact into value, of the merely ontological
\co{foundation} into the \co{concrete} form of it:
\co{nothingness} is not emptiness but \co{origin} --
\citet{who shone\lin where nobody appeared to come}{StJohnPoems}{Dark Night} -- equally
dark, silent and \co{invisible} as emptiness and yet, its complete opposite\ldots
This, however, is a \co{spiritual choice}, which can not be enforced by logic,
arguments, sufficient reasons, efficient causes, anything \co{visible}. He who
would like to be 
convinced, to {\em see} why and how he should choose so and what it is, will
never see anything.  What could one see in total darkness?  There is no \co{visible} way out of \No; hell is the place
surrounded by void and hence with no exit, for one can not exit into void, 
\citet{[o]ne cannot will into void.}{Will}{\citaft{Cotkin}{}\label{ftnt:willVoid}}
%hell is the place with no exit.
What could one hear in total silence? \co{Nothing}, indeed -- the beat of one's
heart, the whisper of one's breath. This, however, is no longer void. Nowhere
happens more than in such a moment of silence, when nothing is heard because
only \co{nothingness} whispers and \co{opens} one's heart. \citet{[A]nd I saw
  nothing then,\lin no other light to mark\lin the way but fire pounding my
  heart.}{StJohnPoems}{Dark Night}


%trying to keep \co{this world}, they will lose \co{another} but, as we have
%seen, eventually even \co{this one};

\ad{It has been there already. }\label{pre:openness}%trust + openness
In \ref{sub:spiral} and \ref{sec:evil} we described the gradual sinking
into hell, as well as development of evil, as the results of misunderstood
\co{thirst} for paradise. The culminating \wo{Evil, be thou my good} conforms
fully to the old claim that each being seeks the highest good and, as Scheler
taught, that such a \thi{highest good} is, objectively and \la{a priori}, a
value above any other, even if nobody \co{actually} knows what \co{precisely} it
might be.

It has been common to see in such a highest good the return to some original
state. Descent to the Underworld happens in search of something or somebody lost
(Orpheus seeks Eurydice, Ulysses is on the way back to Ithaki, Innana wants to
help her sister Ereshkigal). The survivors of the flood are from the beginning
marked by God and the survival represents only the return among the righteous.
\citet{This excellence whose necessity is scarcely or not at all manifest to
  search, exists, if we could but find it out, before all searching and
  reasoning.}{Plotinus}{V:8.6} Then the theme is repeated again and again:
\citet{in human soul there is engrafted desire of true
  good}{Console}{III:2.4\noo{\citaft{FilSr}{, p.106}}} and the search for, and
then \co{recognition} of it, is possible because the soul \citet{did not forget
  itself completely}{Isagoge}{V:3.22 \citaft{FilSr}{\kilde{p.97}}\verify{}}; the
soul \co{thirsts} because \citet{the desire for the bliss, which she had lost,
  remained with her even after the Fall.}{Periphyseon}{IV 777C-D\kilde{88}} And
so, what she finds, is something that already has been there.  \citet{Thou
  wouldst not seek Me, if thou hadst not found Me.}{Pensees}{VII:553} In
II:\refp{anamnesis} we have objected to Plato's \gre{anamnesis} theory of
knowledge. But now we can give it the place it deserves -- not as a theory of
knowledge, but as yet another record of the insight into the character of the
true \sch\ as an event of repetition and return.\ftnt{The idea has innumerable
  appearances: \citefi{[I]t may be asked, is \la{Brahman} known or not known
    (previously to the enquiry into its nature)? If it is known we need not
    enter on an enquiry concerning it; if it is not known we can not enter on
    such an enquiry.}{\Samkara{}}{\citaft{Indian}{\kilde{p.511}}} The aspect of
  reminiscence/return can be found in the concept of \la{prolepsis}, which
  Clement of Alexandria borrowed from Epicurus in order to characterise faith as
  a preliminary stage of knowledge, an \thi{anticipation} of truths which are
  not, as yet, recognised fully by reason. We are very far from understanding
  faith as a mere approximation, an imperfect reason.  But we would emphasize
  the aspect of \la{prolepsis} which makes it thus function as the light leading
  reason in the right direction [\citeauthor*{Zacheta}{ I:16},
  \btit{Miscellanies} II:4].\noo{ \citeauthor*{Stromata}{II:4} \citaft{FilSr}{
      p.37}} It would be easy to see the intended meaning as a mere potentiality
  opposed to the actuality.  But \thi{anticipation} in this context means rather
  a foretaste -- which has not yet given up all the pretensions of \co{actual}
  reason -- of \co{virtual presence}, the \thi{desire for the lost bliss} which
  underlies the search, though does not give it any definite direction. It is
  engrafted in all souls \noo{[\citeauthor*{Stromata}{ V:133.9}]} so that
  \citef{All beings in every starting point have a certain relation to the
    father and creator of the universe.}{Zacheta}{V:133.7 \citaft{FilSr}{ p.38}}
  Augustine's \la{memoria} expresses the same pattern: at the bottom, it is
  concerned with the eternal truth -- when he searches his soul for God, he does
  not discover Him but {\em finds} Him\ldots in the depths of \la{memoria}. The
  most recent form is the hermeneutics of Being, defining human being by its
  character of questioning the issue of Being which questioning, however, is
  pre-reflectively present and precedes any explicit questions.}
% passage from Isaiah quoted by Clemens: 
% \citt{If ye will not believe, surely ye shall not be established.}{Is. VII:9}

The \co{choice} of \yes\ means to recognise \co{nothingness} as the \co{one},
the ultimate \co{invisibility} as the \co{origin}. It thus returns to its
source, and every return is a repetition. Here, it is the \co{spiritual}
repetition of the only ontological event -- \co{second birth}. As such, it will
also lead to another repetition, the \co{concrete} counterpart of the
ontological \co{founding}, but this will be addressed in \ref{sub:concrete}. For
the moment we are concerned only with the event itself.\ftnt{This double aspect
  -- of re- and -birth, of repetition and new \co{foundation} -- is expressed
  marvelously by the Greek expression \gre{gennethe anothen}. \gre{Anothen} is
  sometimes translated as \thi{anew}, \thi{again} (e.g., \wo{Except a man be born {\em
      again}, he cannot see the kingdom of God} or \citefi{Ye must be born {\em
      again}}{John}{ III:3/7}) But more often, it means \thi{from above} (\wo{He
    that cometh {\em from above} is above all}, \citefi{Thou couldest have no
    power against me, except it were given thee {\em from above}.}{John}{
    III:31/XIX:11} In Acts XXVI:25, it is rendered as: \wo{Which knew me
    {\em from the beginning}}.)  Likewise, \gre{anagennao}, appearing in 1 Peter
  I:23, \wo{Being born {\em again}, not of corruptible seed, but of
    incorruptible,} and meaning literally \thi{to produce again}, is also used in
  metaphorical sense for having one's mind and attitude changed. As is often the
  case, attempting to disambiguate such apparent ambiguities, is to confuse the
  issue rather than to clarify it. Both aspects are present and \co{second
    birth} is as much second as it is birth, as much resurrection
  from the dead as birth from the new \co{foundation above}.}

\pa As one sinks into the hell of despair, one gradually accepts despair for its
own sake, as the inescapable lot and damnation. And one is advised to continue,
for \citet{whilst a man is thus in hell, none may console him, neither God nor
  the creature, as it is written, `In hell there is no
  redemption.'}{TheolGerm}{XI, [Probably, reference to Ps. ILIX:8: \wo{For the
    redemption of their soul is precious, and it ceaseth for ever}.]}  When one
starts despairing, the only thing one can do is to
despair more. %\citet{???}{???}{Kierkegaard}
%
But this despair is not any emotional complain, ruefulness, nor any sense of
undeserved loss, which all reflect only the conviction of entitlement. Only
heroes wander in the Underworld, for it takes courage and determination to
say: \citetib{Let me perish, let me die! I live without hope; from within and from
  without I am condemned, let no one pray that I may be released.}{TheolGerm}{XI}
\citeti{Let the day perish wherein I was born, and the night in which
it was said, There is a man child conceived.}{Job}{III:3}
% {Which long for death, but it cometh not; and dig for it more than for hid treasures;
%  Which rejoice exceedingly, and are glad, when they can find the grave?}{Job III:21-22}

As long as it stops short of pronouncing the ultimate \No, such a resignation is
in fact an expression of deepest \co{trust} and \co{hope}.  For underneath all
despair and resignation, one has always an \co{invisible} reserve, a \co{rest}
unable to accept the situation to which one has already given one's consent, an
\co{invisible rest} which underneath all \wo{Let me perish, let me die!}  says,
in a silent, unheard voice: \wo{I am damned, I can't expect anything, so I have
  to perish. {\em But I do not want to!!!}}  This apparently childish and
irrational act -- not even of will, for it is a mere exclamation, and hardly any
factual \co{act} but one made at the bottom of one's soul -- comprises the essence of
the survival.  Intensifying despair has the meaning, for only then one can reach
the \co{invisible} seed of \co{hope}, expressed in such an event. This\noo{I
  have to, even though I do not want to} is the admission that \co{I am not the
  master}, that I would like to leave the place, but I can not do it on my own.
It is a desperate scream for help and, as such, already an expression of
\co{trust}. It is not any faith, it does not believe {\em in anything}, it does
not hope {\em for anything}, it only turns towards \co{nothing} and admits,
without saying: \wo{This is unbearable}.  This apparent surrender, this deepest
resignation in the face of \co{nothingness}, is a \co{sign} of \co{trust}. As
long as it does not say it explicitly, when it is no longer able to say it, then
it really says: \wo{Please, release me\ldots} For \citet{to {\em believe} in
  one's own undoing is impossible.}{Despair}{I:C.The forms of this sickness
  (despair)A.b.$\beta$\kilde{p.69}}

% \kom
% The choice is not between good and evil -- such a choice is impossible, for even
% \wo{Evil, be thou my good} seeks only good. (goes further, to \ref{kom:noEvil}?)

%\newp

\pa\label{pa:trust} One can reach such a surrender at various degrees of despair
and humiliation which everybody can imagine for himself. What matters to us here
is the fact, rather than the intensity, of the apparent paradox: the
impossibility of any \co{visible} release and, on the other hand, the
impossibility of accepting this lack, \citetib{that in human terms the undoing is
  certain and that still there is possibility.}{Despair}{\noo{I:C.The forms of this sickness (despair)A.b.$\beta$\kilde{p.70}}} \co{Trust}
does not appear between these two -- \co{trust} {\em is just the tension} of
this, as one would like to call it, paradox, is just the ability to live (with)
it.\ftnt{This is only an apparent paradox, because it
  appears so only \wo{in human terms}, from the perspective of \co{this world},
  of one who expects sufficient, preferably even \co{visibly} demonstrable
  reasons. But the transition is, in truth, very well possible. \woo{Health is
    in general to be able to resolve contradictions,}{Despair-I:C.The forms of
    this sickness (despair)A.b.$\beta$\kilde{p.70}} 
  so turning the lack of \co{visible} sufficient reasons into a
  \thi{contradiction} is then the exact opposite of health.
  
  The Greek \gre{pistis}, translated usually as \wo{faith} or \wo{belief}, can
  be, and often should be, rendered as \wo{fidelity}, \wo{assurance} or
  \wo{loyalty}. (This is even clearer in the primary verb, \gre{peitho}, from
  which \gre{pistis} is derived, and which means to persuade or be persuaded
  (e.g., Mt.XXVII:20,XXVIII:14, Acts XXI:14), to trust (Mt.XXVII:43, Lk.XI:22),
  to obey (Acts V:36-37).) Besides reliance, this openness for \thi{persuasion}
  (or conversion) is the primary \co{aspect} of \co{trust}.} It has the dual
aspect of faith and hope, both understood in the \co{spiritual} sense, that is,
without relation to anything \co{visible}.  It is only admitting the unbearable
character of the present state, is a mere reaction which, however, is directed
{\em against} it, without recognition of any chance to overcome it.  (Again, the
\co{reflective} consciousness presents one only with the desperate and
unbearable life, and knows little about the \co{trust} which might underlie it.)
This \thi{against} carries the character of faith, hides the impossible
possibility of overcoming that which, according to all \co{visible signs}, can
not be overcome. As such, it has also the seed of \co{hope} -- not any definite
hope as to how this impossibility could, perhaps, occur, but \co{hope} contained
already in the very exclamation -- disrespecting all the \co{visible} proofs to
the contrary and asking for help where no possible source thereof can be seen.
The help it is asking for concerns absolutely \co{nothing}, it is not help to do
this or obtain that, it is not help {\em to anything}, for the one who is crying
has not the slightest idea what he needs help to. In such a time, \wo{[t]he soul
  knows for certain only that it is hungry. The important thing is that it
  announces its hunger by crying. A child does not stop crying if we suggest to
  it that perhaps there is no bread. It goes on crying just the same.}
\wo{Release me\ldots} -- not pronounced loudly, not pronounced at all, but lived
underneath the despair -- that is all.  Release me whence? How? Whereto?
Who? %\co{I} do not know,
The danger is not lest one does not find any answers but lest one forgets that
one has ever asked.  \citet{The danger is not lest the soul should doubt whether
  there is any bread, but lest, by a lie, it should persuade itself that it is
  not hungry.}{WeilWaiting}{Forms of the Implicit Love of God:Friendship;p.138}

\ad{Necessary, but insufficient. }\label{pre:humility}%grace - humility
\Yes\ says only that \co{nothingness} is fullness,\noo{\citet{is within them all
    is the fullness. Beyond it, there is nothing within.}{Philip}{} This
  fullness means only} that emptiness is untrue, that beyond it there is~\ldots
\thi{Something}, and hence that there is an exit, even if \co{invisible}. The
silent cry \wo{Release me!} is the witness of that, for the most real is that
which you can not live without. At the same time, the cry is also an admission
that \co{I} do not have the power to exit on my own. After all, \co{I} do not
see any exit. The \sch\ says \yes\ at first only in the form of accepting
damnation and despair and yet, in spite of that, not accepting it, nourishing
somewhere in the depth the inadmissible \co{trust} that, after all, \co{I} won't
stay here forever. This paradox reveals only the insufficiency of any
\co{actual} choice (or, for that matter, of anything \co{I} can do) to effect
the transition. All \co{I} can do is to say \Yes, first in the deep silence of
\co{invisibility}, then perhaps in a louder and more conscious voice. But saying
so \yes\ is only saying that \co{I} am willing to accept the possibility of
release, that \co{I} indeed ask, seek and knock and hope that \citeti{every one
  that asketh receiveth; and he that seeketh findeth; and to him that knocketh
  it shall be opened.}{Mt.}{VII:8}\noo{The impossibility of believing in one's
  own undoing, especially when deprived of any possibility of exit, is the same
  as the fact that \citet{[f]ar from the destitute of a divine idea is
    man}{Stromata}{V:14}.} The cry, knocking or, as we can also say, the mere
consent \co{opening} one to the possibility, is all one can do. It is \equi\ 
with the infinite patience because, as one can not do anything, all one can do
is to wait; and as one has ceased waiting for anything \co{visible} to occur, one
must wait \co{eternally}, that is, \co{above} time. \citet{[I]t is necessary
  that one has the patience to begin so, that one in truth admits, that it is a
  feat of patience}{Taalmod}{IX.To gain one's soul in patience;p.155} and only
of patience, through which one gradually regains one's soul. This patient waiting
in \co{openness}, with or without clear consciousness thereof, calms down all
the storms.


\ad{Gift. }\label{pre:thankful}% + thankfulness
The \co{choice} of \yes\ is only a necessary but not the sufficient condition of
exit. As usual, we do not know the sufficient conditions, but here they have
been given a name -- \wo{grace}.\ftnt{It would be probably closer to the actual
  grace, which is withdrawn after the performance of the \co{act} for which it
  was granted, than to the sanctifying grace, which has the constant, habitual
  effect and makes one permanently holy, but we will leave such distinctions to
  the theologians.} Since we do not deal with the sufficient conditions,
\wo{\co{grace}} will mean for us simply the possibility of the apparently and
humanly impossible, the fact that, although there is no \co{visible} exit, some
people do return from hell.  \co{Trust} is thus \co{openness} to the possibility
of \co{grace} and both together can become effective only when \co{I} have said
\Yes, admitted \co{that I am not the master}, that \citet{I live yet do not live
  in me.}{StJohnPoems}{}

%As said about suicide in \refp{suic=no} -- it, too, tries to \co{transcend},
%only that one can not transcend into nothingness

\co{I am not the master} and so \co{grace} is a \co{gift}, a true -- that is,
undeserved (or, as a philosopher might prefer to say, unaccounted for) -- \co{gift}.
But it is not a gift from anybody, for \co{I} have not found any other,
particular master. Just like \co{hope} and \co{thrust} are directed towards and
ask into \co{absolute nothingness}, so \co{grace} comes only from there, it is a
\co{gift} of \co{nothingness}. It did not come from any \co{visible} place; it is
only all surrounding \co{nothingness} which is the source, the \co{origin} of
this \co{gift}.  One can receive gifts coming from no one, from \co{nothingness},
and one can likewise be thankful without being thankful to anybody.
\co{Spiritual thankfulness} does not concern anything in particular -- it
concerns \co{nothing}, that is, everything.  It is not even thankfulness {\em
  for} \co{grace}, for \co{thankfulness} is but an \co{aspect} in the \nexus\ of
\co{grace}.

\newp

\ad{Ex nihilo?}\label{pre:exnihilo} The \sch\ of \yes\ finds \co{invisible}
richness of the \co{origin} in the \co{indistinct nothingness} which for \No\ 
remains an irrelevant void. \yes\ creates something which was not there before:
from the deepest \co{thirst} for \thi{Something}, from the thoroughly logical
\co{actuality} of solipsism, perhaps, from the deepest despair of emptiness, it
emerges into the full \co{presence}, from the {knowledge} \co{that}, it emerges
into \co{participation}, into Being.  It thus seems to create \la{ex nihilo},
from the total emptiness.  It looks like an arbitrary decision, perhaps, a mere
projection, and those who like will always see it in this way. But it is only
the assumption of emptiness, the assumption that \co{nothingness} is indeed
void, which makes everything that follows \Yes\ into a mere projection. Such an
assumption wants, first of all, to {\em see} some definite reasons which would
oppose it, which would invalidate the sense of emptiness. But no such reasons
can be given and then everything that follows turns into a void equal to that
which is there from the start -- in human, \co{visible} terms, indeed, \la{nihil
  ex nihilo}, \citet{nothing can come out of what does not
  exist.}{AristPhysics}{187a.33-34} In short, this is a way of saying \No, and
we are not concerned with it any more.
 
 The \co{choice} of \yes\ does not create \la{ex nihilo} (which form of creation
 pertains to Godhead alone). It turns \co{alienation} into \co{concrete
   participation} and thus, creates good from evil.  It creates
 \co{participation} by {\em finding} the \co{origin}, non-emptiness, {\em
   finding} fullness where before it found only emptiness.  To see here only a
 projection is the same as to see nothing, as denying the meaning of the whole
 event, as \wo{dragging the revelation of the greater down to the level of one's
   littleness.}$^{{\rm{II}}:\ref{cit:little}}$  In fact, it seems possible, at least
 logically, to maintain the 
 doctrine of extreme, subjective idealism, of absolute solipsism, even of
 absolute immediacy (where the world is re-created every moment anew, though
 strangely enough, in the shape deluding us to believe that it is the same world
 as the one a moment ago.) The logical possibility of such doctrines illustrates
 only what such a possibility is worth as a criterion of anything.  For any
 interpretation in terms of the \co{actual} knowledge, this \thi{finding} is
 impossible and untrue. In such terms a relation which assumes the being of one of
 its poles never counts as sufficient for the being of the other; in such terms
 nothing, nothing \thi{objective} has been {\em found}, only something has been
 thought, \thi{subjectively} posited. Thinking which knows nothing but its
 \co{actual concepts} is a sad affair and it indeed can not get further than
 such a denial. It can not recognise that this \thi{thought} of fullness beyond
 emptiness is not an \co{objective} thought of a possible state of affairs or an
 \co{actual} agent; that thinking it as a merely possible hypothesis is to deny
 it and think its opposite; that this thought would not be itself without being
 already a \co{trustful} admittance, \gre{pistis}; that it is impossible to
 approximate it, for in its simplicity, it can only be thought as \yes\ or not
 thought at all; that it is knowledge which, {recognising} \co{presence},
 becomes \co{participation}, or simply, \co{participation} which {recognises}
 itself as such. But this knowledge, this \co{gnosis}, is not of the \co{actual}
 kind, it does not provide any \co{visible} justifications or reasons forcing
 one to accept it. If it did, the \co{choice} would not be a 
 free event. Only in the fact that it can be denied, in the complete lack of any
 reactive character, consists the absolute freedom of the \sch.\noo{ (and then of its
 consequences).}


\ad{Good from evil}\label{pre:goodfromevil}
% The lack of such reasons constitutes the aspect of free creation in the
% \co{choice}.
%
The silent scream \wo{Release me!} fills the emptiness with~\ldots \co{nothing}.
Yet, \co{nothingness} ceases to threaten with hollow darkness and void and,
instead, becomes the source, the (new) \co{origin}. \co{Spiritual} \co{aspects}
-- \co{humility}, \co{openness}, \co{thankfulness} -- turn the \co{indistinct
  nothingness} into a warm and living friend, both remote and close.  Of course,
it is still \co{nothingness}, there is nobody \thi{out there}, but the
\co{indistinctness} started to live, and its life is fully consummated in the
\co{spirit}, that is, \co{absolute} relation. It is like creating by mere
willing, but willing at this level means close to nothing -- in \co{actual}
terms, perhaps, only surrender without resignation, \co{trust} without hope,
\co{hope} without expectations\ldots  The \sch\ of \Yes, aiming at \co{nothing} and
presenting \co{nothing}, may indeed suggest inventing something more definite,
something more communicable which could be \co{posited} as the active agent
responsible for everything -- Nature, Fate, Zeus, God, {\sc jhvh}\ldots  \co{Reflection}
is bound to do that, and the \thi{objective images} nourish its natural, though
only subconscious, sense of \co{participation} as the sense of dependency.  But the
only active agent is the \co{spirit}, the tension between \co{nothingness} of
the \co{one} and \co{nothingness} of \co{self}, between God and God-image,
between God and \co{existence}. The rest is a more or less adequate manner of
speaking\ldots

We think that emptiness is when nobody speaks, so we look for some \co{signs},
wait for the sound of some words. But \woo{silence is a fence around
  wisdom}{MaimoTraits}$^{\ref{cit:fence}}$ and it is God's voice. Emptiness is
not when nobody speaks but when nobody listens, when we speak, scream into void.
Strangely, here being listened to is simply to dare to speak, to admit that
\co{I} can't will into void, that \co{I} can't live in the middle of emptiness.
The most real is that which you can not live without.\noo{, and the ontological
  argument conceptualises the meeting of the highest reality, of \la{ens
    realissimum, originarium, summum, entium}, of the \co{indistinct One}.}
This responsive listening arises from the prior \yes, from the will wanting
\thi{Something} to be, but \thi{Something} which is not this or that but, as a
matter of fact, everything\ldots \citeti{What you desire strongly, with all your
  will, you already have and this can not be taken away from you neither by God
  nor by any creature, if only your will is complete, wants it because of God
  and stands in front of Him. Let there be no \wo{I would like}. This would only
  be a future.  But \wo{I want -- now -- and hence it is}. Truly, with my will I
  can everything.}{Eckhart} {\citaft{Mistyka}{ B:A.I.1}\kilde{p.200} This might
  be misconstrued as a sheer voluntarism but will which is complete, which is
  \co{concretely founded}, is not exhausted by its \co{actual} intention. We
  will return to it below in \ref{sub:reflYes}.}


The \co{choice} of \yes, raised \co{above} all \co{visible} reasons, is a free
creation -- it is needed to create the situation of \co{participation} from
\co{alienation}.  The \co{choice} of \No\ does not have this \co{aspect} because
it is motivated by the \co{visible} misery and \co{alienation} -- it only
accepts and surrenders to it. The alternative of these two possibilities
represents the \co{absolute} freedom of \co{choice} between creation and
resignation, between \co{participation} and \co{alienation}: \citeti{if thou
  seek him, he will be found of thee; but if thou forsake him, he will cast thee
  off for ever.}{Chr.}{I:28.9\label{ftnt:seek}} The \co{choice} is what it
creates: \citeti{He who knows the \la{Brahman} as non-existing becomes himself
  non-existing. He who knows the \la{Brahman} as existing him we know himself as
  existing.}{\Samkara{}}{\citaft{Indian}{\kilde{p.516}\noo{For our purposes, we
      don't distinguish \la{Brahman} from \co{one}.}}}  \No\ creates
\co{alienation} by giving up the possibility of finding anything and eventual
\co{alienation} is nothing else but this \No. This is only a resignation, it
creates only \la{nihil ex nihilo}\noo{fit}, so we won't call it \wo{creation}.
It is \Yes\ that creates by transcending the emptiness; it creates \co{concrete
  participation} which, in turn, is nothing but accepting the \co{gift} of the
\co{origin}.  In short, both \Yes\ and \No\ are \nexuss\ -- of choosing,
receiving, being, knowing -- both, of course, \co{founding} the opposite
\co{concrete} modifications of all the \co{aspects}.

\ad{Without me God would not be God}\label{pa:GodneedsMe}
\Yes\ is a free creation but not creation \la{ex nihilo}, only
of good from evil, of 
\co{participation} from \co{alienation} -- or else of \co{concrete} God from abstract
Godhead.  \citeti{Without me God would not be God. 
  I am the cause of God being God.}{Eckhart}{\btit{German Sermons},
  Mt.V:3. [\citeauthor*{EckSelected}{ 22}, \citeauthor*{LW}{ 52}] \kilde{Otto,
  Mistyka...p.121}} Meister does 
not say \wo{I am the cause of God}, only that the \co{existence} is the cause of
\co{nothingness} being God. Indeed, without the \co{confronting existence}, the
\co{indistinct} would remain \co{indistinguished indistinct}.

\begin{verse}
    {\small{\em Was wirst du tun, Gott, wenn ich sterbe?  \\
Ich bin dein Krug (wenn ich zerscherbe?) \\
Ich bin dein Trank (wenn ich verderbe?) \\
Bin dein Gewand und dein Gewerbe,\\
mit mir verlierst du deinen Sinn.

Nach mir hast du kein Haus, darin\\
dich Worte, nah und warm, begr\"{u}ssen.\\
Es f\"{a}llt von deinen m\"{u}den F\"{u}ssen\\
die Samtsandale, die ich bin.

Dein grosser Mantel l\"{a}sst dich los.\\
Dein Blick, den ich mit meiner Wange\\
warm, wie mit einem Pf\"{u}hl, empfange,\\
wird kommen, wird mich suchen, lange --\\
und legt beim Sonnenuntergange\\
sich fremden Steinen in den Schooss.

Was wirst du tun, Gott? Ich bin bange.\ftnt{\citeauthor*{RilkeStunden}. Vom
    m\"{o}nchischen Leben } }}\end{verse}
%
Just like the alternative \Yes-\No\ offers no third possibility (\citeti{He that
  is not with me is against me;}{Mt.}{XII:30; Lk. XI:23}), so God is either
living or dead, and he lives {\em only} in the human \co{soul}.  \citeti{The
  soul is a heavenly housing of eternal Godhead. So that He completes His divine
  work only in it.}{Eckhart}{\orig{So wird die Seele eine himmlische Behausung
    der ewigen Gottheit.  So da{\ss} er seine g\"{o}ttlichen Werke nun in ihr
    vollbringt.} {According to a Mesopotamian myth from VI-th century \bc,
    Marduk, \wo{in order to prepare a habitation for gods in the thirst of their
      hearts\lin Created humankind.}\citaft{MityHebr}{I:2}} The theme dominates
  neo-Platonic anthropology with a clear expression in Eriugena, developing the
  quote from Maximus Confessor \wo{For they say that man and God are paradigms
    of each other.}\kilde{StanfordEncPhilos:Eriugena, and PL CXXII:1220a} A
  recent return of the theme -- perhaps, in its academic fashion, somehow
  disguised and politicised: \citef{God is supposed to be absolutely powerful in
    our tradition. [...] I'm trying to think of some unconditionality that would
    not be sovereign, that is, to deconstruct the theological heritage of the
    concept, the political concept, of sovereignty, without abandoning the
    unconditionality of gifts, of hospitality, and so on.  That means that some
    unconditionality might be associated not with power but with weakness, with
    powerlessness. [...] I'm trying to think of some divinity dissociated from
    power, if it is possible.}{DerDivine}{\citaft{Caputo}{}} } \citeti{The
  spirit of man is the candle of the Lord.}{Prov.}{XX:27}
%
The \co{concrete participation}, this living \co{presence} of God's in the
\co{soul} is not a fact, an \co{objective} truth -- it is only the possibility
of \yes.  Without Godhead's \co{nothingness}, there would be no \co{me} and no
world.  But without \co{me}, without the place where Godhead can become
\co{concrete} and where God can \co{incarnate}, there would be \ldots no God, or
else, God would have to remain \co{one}, a mere principle, perhaps, a
\co{reflective} abstraction, \thi{the first mover} or \thi{the ultimate cause}.
It is said about God \citeti{I love them that love me}{Prov.}{VIII:17} but, in
fact, God's life is {\em nothing else} but this \co{love}.  The
\citet{intellectual love of the mind toward God is the love with which God loves
  Himself.}{SpinozaEthics}{V:Prop.XXXVI. (We must distance ourselves from
  Spinoza's partition of God and summation of parts back into His totality
  again. The actual formulation says \wo{...is part of the infinite love with
    which God...})}  If \co{I} deny this \co{love}, if \co{I} do not live it,
then what can God do?  Man's eventual freedom is God's helplessness. His
\co{command} leaves \co{me} free, it always leaves the place for saying \No, and
if \co{I} say \No, if \co{I} die -- ``What will you do, God?  I am worried.''

\pa\label{pa:twoFaces} Abstractly, the \co{choice} is between nothing and
everything. \Yes\ recognises the \co{indistinct} as the \co{origin}, it
\co{distinguishes} it \co{above} all \co{visibility}, and so it {\em is}.
\wo{It} may be taken to refer to the \co{one}, but primarily it refers to the
\co{existence} saying \yes.  It acquires being which is no longer merely
ontologically \co{founded} in the \co{one}, but which is \co{concretely founded}
in it, which is \co{concrete participation}.  As such a transition from
\co{nothing} to \thi{Something}, \yes\ is a new creation of the world, or what
amounts to the same, the \co{second birth}.  \No\ sees only emptiness, does not
\co{distinguish} it \co{above distinctions}, and so it {\em is not}.  It is
ontologically \co{founded} in the same \co{origin}, but this \co{founding}
remains as abstract and irrelevant as a mere fact, as any simply unavoidable
truth; it does not find a \co{concrete} counterpart in one's life.

There is only one God, and everything is his \co{sign}. \thi{In itself} He is
\co{nothing}, that is, everything. But although the  
\co{presence} is obvious, it is not obvious that it is His \co{presence}, and so
he has two faces: \co{nothingness} can be all or nothing, he can be life or
death, generous giver or sower of despair, peaceful love or 
fearful vengeance. It is not entirely up to \co{me} to choose which face he
will show -- the \sch\ is not \co{my act}. We will have something more to say 
about things \co{I} can do, but first let us make a small digression.


\subsub{Anselm's argument}

\pa
\la{Aliquid, quo maius cogitari non potest}, 
\thi{being greater than which nothing can be thought}\ldots
The shocking content of Anselm's argument from \btit{Proslogion} consists in the
fact that existence is demonstrated from a mere concept, that being follows from
knowing. The unjustified -- and unjustifiable!? -- transition from \la{esse in
  intellectu} to \la{esse in re}, raised as one of the earliest objections 
already by Anselm's contemporary Gaunilon, has the same content as the
\thi{creation}, the \co{second birth} we were just speaking about. 
% even though Anselm treates this distinction carefully in second chapter of
% \btit{Proslogion}.} 

The list of other objections of various kinds could be rather long: that
\thi{being}/\thi{existing} is not a predicate which could be added to the
concept of anything, that \thi{being greater than} remains unspecified and
cannot be given meaning making the argument work, that \thi{being greater than
  which nothing can be thought} is not a legitimate concept and one should at
least show that it is not contradictory, that what is demonstrated is only
necessity of a being provided that it exists and not its existence,
that\ldots\ftnt{After the well-known criticisms by Gaunilon, Descartes, Leibniz,
  Kant, Hegel, the discussion still continues. E.g.,
  \citeauthor*{Gilson}{\noo{p.62; after \btit{Filozofia Religii},} p.118,
    footnote~35.};
%E.~Gilson, \btit{The spirit of medieval philosophy}, New York, 1940
  \citeauthor*{Findlay}; 
%J.~N.~Findlay, Can God's existence be disproved?, \btit{Mind}, no.~57,
%  pp.176-183, 1948;
  good review in \citeauthor*{MalcolmOnto}.
%  N.~Malcolm, Anselm's ontological arguments, \btit{The Philosophical Review},
%  vol.~LXIX, no.~1, 1960, pp.41-62;
  (Some of these actually do not oppose Anselm, but only his
  argument, and some not even that, though they discuss possible objections.)
  
  On the other hand, one should remember the tradition using the same
  \thi{definition} of God but with \thi{better} instead of \thi{greater}, as for
  instance, \citef{nothing can be thought of better than God, and surely He,
  than whom there is nothing better, must without doubt be good.}{Console}{
  III:10\kilde{.22-25}}. God \citef{is thought of as 
    something than which nothing is better or higher.}{AugustChrist}{I:15},
  \citef{What is God?  That than which nothing better can be
    thought}{BernardConsider}{V:7.15} \thi{Greatness} itself appears already in
  Seneca: \citef{What is God? The totality that you see and the totality that
    you do not see. His greatness belongs to him in such a way that nothing
    greater can be conceived}{SenecaNat}{I:Preface 13}.\kilde{from Eckhart, big,
    p.223; \citaft{FilSr}{ p.164}} Also, though with respect to the perfection
  of the universe, \citeauthor*{CiceroGods}{ II:7-8.}}

All the objections, with all the pretensions to formality, may be interesting
and nice, but they necessarily involve \co{actual concepts} and \co{distinctions}
which do not apply at the level addressed by the argument.  We leave the
pedantic analyses of the logical forms, merits and demerits of this beautiful
argument to those who deem such exercises worthwhile.
%\tsep{already there}

\pa The argument intends not so much to {\em prove} God's existence as to confirm
it, make it more transparent. Anselm
repeats Augustine's \la{credo ut intelligam} -- \citet{For I do not seek to
  understand so that I may believe; but I believe so that I may
  understand.}{Proslogion}{1. The subtitle given by Anselm to \btit{Proslogion}
  was \la{Fides quaerens intellectum} -- Faith in quest of understanding.}
Such a search is underlied by the sense of looking for something already
present, \refpf{pre:openness}.
%the soul \wo{did not forget itself completely} and \wo{the desire for the bliss,
%  which she had lost, remained with her even after the Fall},
%{John Scotus Eriugena, \btit{Periphyseon}, IV 777C-D [88]}
If we were to accept the name \wo{ontological}, given by Kant to the argument,
it would not be because it somewhat deduces being from a concept but, on the
contrary, because it finds something which already is there, which is presumed
on a different, and stronger, basis than the conceptual context of the proof
itself.\ftnt{In terms of mere conceptual inferences, we have perhaps only
  \citef{presupposed an existence as belonging to the realm of the possible, and
    have then, on that pretext, inferred its existence from its internal
    possibility -- which is nothing but a miserable tautology.}{CrPR}{I:2nd
    Division.3.4, A597/B625.} Perhaps, in terms of mere inferences. But
  all valid inferences are only tautologies, and all tautologies
  are equally miserable.}
%
The argument \co{reflects} the \co{presence} of the \co{origin}.  It does not
tell, in the manner of \la{a posteriori} \thi{proofs}, how finite understanding
could reach the infinite -- Anselm's dissatisfaction with his earlier proofs
from \btit{Monologion}, his search for an \la{a priori} argument, can be seen as
an expression of the fact that only such a structure reflects the underlying
postulate that this has already happened, that understanding already is involved
in the infinite.

\pa If nothing else, then at least the constant presence of the argument since
the XII-th century, shows that \thi{being greater than which nothing can be
  thought}, or perhaps only an idea(l) thereof, is highly troublesome for the
partial \la{ratio} with its pretensions to universality. The troublesome aspect
is that knowing, which coincides with being, is \co{gnosis} and not plain
\la{episteme}. \la{Episteme} can go no further than the \co{actual
  dissociations}: the \citet{real object is one thing, and the understanding
  itself, by which the object is grasped, is another,}{Gaunilon}{3} 
%a monk of Marmoutier, \btit{In behalf of the fool}, 3};
%. An answer to the argument of Anselm
%\btit{Liber pro insipiente} = \btit{In behalf of the fool}
hence: the argument would be a valid proof {\em if}
the idea of God in human mind {\em and} God's being were identical.  This identity,
however, is impossible for \co{actual} thinking -- for it, \wo{being} means only
\thi{real object}, \co{external objectivity}, completely \co{dissociated} from
the \thi{human mind}, and as such the opposite of a \thi{mere idea in the mind}.
This form of objection does not really consider the form of the argument at all
but merely points out the impossibility of proving that anything is: a proof
is a thought, while thing is a being -- the two, once \co{dissociated}, can not
be the same by their very nature, \thi{by definition}. 

The argument has nothing to do with any thing which can be thought in the mode
of such a \co{dissociation}. It does not apply to any \co{actual} things, and
\citet{if anyone should discover for me something existing either in reality or
  in the mind alone -- {\em except} \thi{that than which a greater cannot be
    thought} -- to which the logic of my argument would apply, then I shall find
  that Lost Island and give it, never more to be lost, to that
  person.}{ReplyG}{3} But God who, without me, would only be impersonal Godhead,
God about whom I should worry, in case I die, God who does not live somewhere
else, in a deistic \co{dissociation} from this world and human life but, on the
contrary, in its midst, who is hardly anything else than the \co{spiritual}
tension of this life, the \co{absolute} pole of the \co{existential
  confrontation} -- well, with such a God there is no difference between his
being and being \co{present}, between his being alive and being alive in me (or
in you), or -- if one  insists on the inadequate mode of expression
-- between his \thi{being in and for and by himself} and the \thi{idea of him in
  one's mind}.

Thinking it is not yet knowing it. For to know it\noo{ (and it is possible to
\co{actually} know it)} is nothing else but to recognise this knowledge as one's
being, to recognise the relation of \co{confrontation} as \co{participation}.
Anselm's argument appears as the recurring shibboleth of the ever recurring
suspicion: that knowledge becomes being which it already has been, that the
\co{actual} understanding, the understanding of the \co{visible}, follows in the
\co{invisible} \co{traces} of its \co{origin}, that the mere \co{totality} of
all things presupposes a \co{unity} from which it has arisen.  These, in turn,
are only conceptual figures of the deepest possible transformation of human
being, in so far as man can contribute to such a transformation, of the creation
of good from evil, transition from emptiness to fullness.\ftnt{It would be
  tempting to draw here yet further analogy -- with the Socratic contention that
  virtue is knowledge.
% or even with the general Platonic identification of Goodness with Being.
%As argued in \btit{Sophist}, d
  Distinct virtues form a unity (a \nexus) in prudence, the knowledge of what is
  truly good; and as knowledge can, according to Plato, concern only the
  unchanging, the object of prudence, the true goodness is \co{absolute}, better
  than which nothing could be thought. It would require a bit to overcome the
  conceptual differences, but at least some of the basic intuitions would
  probably turn out to be the same: the knowledge of the ultimate good is also
  its being. (For the mere \gre{episteme}, this identification is as offending
  as the one in Anselm's argument was to Gaunilo.) Even if we accept this
  analogy, we should nevertheless observe the important difference which would
  concern the opposite pole. The simple-minded negation might suggest that,
  since virtue=knowledge, so evil=ignorance. But flat negations and
  contradictions are of little use outside formal logic -- negation of
  \co{gnosis} is non-\co{gnosis}, and this can be ignorance as well as adequate
  \co{actual} understanding; negation of \co{participation} is
  non-\co{participation}, and that can be intentional evil as well as despair or
  a mere confusion. The resignation and surrender of the \co{spiritual} \No\ 
  seem to have no counterpart in the Platonic evil as a mere mistake or
  ignorance.  \wo{Evil, be thou my good} read with Plato could say, at most,
  something like: \wo{I choose evil because I believe it to be good}. In our
  reading it is rather: \wo{I choose evil because I am not able to choose good},
  where \thi{not being able} can have the flavour of \thi{not knowing better}
  but also of \thi{having not enough strength} and even of \thi{not willing}.}
%, which repeats in human terms the \co{creation} \la{ex nihilo}.

\newp \pa The disputes over the validity of the argument will hardly ever stop,
because the possible interpretations will always not only reflect the available
repository of concepts, but also mimic the assumptions, or rather the deep
\co{motivations}, one had in advance.  It is of little value to argue
\citet{[h]ow far the idea of a most perfect being, which a man may frame in his
  mind, does or does not prove the existence of a God [...]  For in the
  different make of men's tempers and application of their thoughts, some
  arguments prevail more on one, and some on another, for the confirmation of
  the same truth.}{LockUnd}{IV:10.7} Men may, of course, disagree not only with
respect to the validity of proofs and arguments concerning \thi{the same truth}
but also with respect to this truth itself.  In the most abstract and extreme
form, the poles of this disagreement \co{reflect} either \yes\ or \No, and there
are no \co{visible}, \co{objective} reasons allowing one to choose between them.
\citet{The ontological argument is a report of experience}{IdealistView}{V:11}
and reduces really to what one wants to understand by \wo{\la{aliquid, quo maius
    cogitari non potest}}.  If one takes it to be what is intended, to be God,
then one has already drawn the conclusion. For thinking God -- who is, and who
is Being: \la{id, quod est esse} -- without thinking him as being (i.e.,
\la{quod est}, not \la{esse}) is to think something else: either an empty
concept, a mere word, or a non-empty concept which, by its very non-emptiness,
can not be the concept of God.  Without any \co{existential} import, the mere
\co{objective} thinking can not possibly think God.  \citet{For in one sense a
  thing is thought when the word signifying it is thought; in another sense when
  the very object which the thing is is understood. In the first sense, then,
  God can be thought not to exist, but not at all in the second sense. No one,
  indeed, understanding what God is can think that God does not exist
  [...]}{Proslogion}{4} The argument invites, even provokes, one to realize that
\thi{merely thinking God} is not thinking him at all, that thinking God properly
is not an operation of mere \la{episteme}, a play of \co{concepts}, but requires
this thinking to have \co{existential} import or, in a more traditional phrase,
that one \wo{lifts one's mind to the contemplation of God}, \co{above} the
\co{actual dissociations} and arguments.
%Without this \co{existential} import, thinking will never be able to think God. 



\noo{    
\pa In \refp{removeLab} we quoted: \citet{But you saw something of that place,
  and you became those things. You saw the Spirit, you became spirit.  You saw
  Christ, you became Christ. You saw the Father, you shall become Father. So in
  this place you see everything and do not see yourself, but in that place you
  do see yourself - and what you see you shall become.}{Philip}{}
The multiplicity of \thi{things, spirit, Christ, Father,\ldots} might make one
think that there is indeed some multiplicity of $x_1,x_2,x_3,...$ and that one
becomes now $x_5$, now $x_{184}$, now yet another $x_i$.

We won't quote multiple texts of Aristotelean origin and Scholastic provenience
about becoming (equal to) the object of one's knowledge. Such seem to us an
uncritical applications of the \co{spiritual} indistinguishability of being and
knowing to the lower levels of \co{visibility}.
}

% used earlier: 
% The \co{object} of my understanding and the \co{concept} by which I understand
% it are opposite poles of a relation, they are in no way the same. But they both
% originate from a higher unity, from the \co{distinctions} made in the texture of
% \co{experience}, of \co{chaos}, eventually, of the \co{indistinct}. It is only
% the \co{actuality} which \co{dissociates} the \co{subject} and the \co{external
%   object}, me-here and object-there (= my-experience-out-there!).
% The \co{object} or the \co{complex} is \co{external}, is not in my head, and yet
% I never cognize and never get to \co{actually} know (erkennene) anything except
% myself. (But I am more more than myself
    
\noo{ REUSE - here?
One may long, hope and search all the life, without finding
the tranquility which has been there all the time. For looking for 
it, one never finds. This is the
tranquility of longing for nothing, the tranquility of hope for
nothing, the tranquility of acceptance of everything -- the ultimate, 
the silence of eternity pregnant with time and all \co{visibility}.
%a soul can achieve if it only grants it
%to itself in all humility of trust and thankfulness.

%hope
\pa This longing expresses an infinite resignation, an irrecoverable
loss without, however, yielding to it.  This loss is the ultimate
gift.  In the sudden silence of the night, in the darkness of the
waters, it finds a faint reflection of the last sun ray.  It resigns
without giving up, it admits defeat without surrendering.  For like
there is a great danger in every great victory, there is a great gain
in every defeat.  The longing is but a hope -- not hope for something
impossible, but an impossible hope for something which already is
here, a hope for impossible fulfillment which, mysteriously, or even
mystically, already has taken place.  To admit the irresistible
strength, in fact, the inaccessible and yet fully consummated reality
in this longing, is to fulfill it.  Hope is a state where there are no
expectations, hoping {\em for something} is to have already lost all
hope.  No visible sign will ever quench the soul's \co{thirst}.  Only
seeing the impossibility of fulfillment, only appreciating the
impossibility of longing, makes you realize that no fulfillment is
necessary, that any fulfillment is a lie, like any success is a
destruction of something.  Longing is but a sign of hope but this hope,
in turn, is nothing else but its own goal.


%acceptance
\pa And thus, what emerges beyond the infinity of longing, beyond the
impossibility of hope, is quietude.  Beyond the horizon!  Beyond the
horizon!  Yes, but with time, we learn that beyond the horizon there
is next horizon, and beyond that, yet another one.  We can never
transcend them all, which means, we can never transcend {\em the}
horizon.  This is not resignation, only the gain one may collect from
a defeat.  The presence of the horizon makes us realize that what is
within may be worth at least as much as what is beyond.  And that what
is beyond is not so different from what has already been within.  The
acceptance is not the same as resignation, for resignation merely
admits the defeat, while acceptance finds a precious treasure
underneath.  But you can not accept things within, you can not find
them worthy enough, unless you recognise the evanescent yet undeniable
presence of the horizon.  To accept things is to resign.  To negate
them, that is, to accept the horizon is, by an apparently paradoxical
calculus of the soul, the same as to accept the things within, to
consent to them and their simple being within and to appreciate them
as such.

%quite & silent
\pa And thus, what emerges in the infinite longing, in the
impossible hope is tranquility.  \ger{\"{U}ber Gebirge ist die
Ruhe.} Longing abides in this tranquility and hope abides in it, too.
} %%% end \noo{ to be REUSED

%end of \subsub{The ontological argument}

\subsub{Reflective Yes}\label{sub:reflYes}% (in abstracto)
\pa
As announced at the end of
\refp{pa:twoFaces}, we now want to say a few words about things which are in our
power and which may have some influence on the \sch. 
\Yes\ is not an \co{act}, it is an event happening \co{above} the horizon of
\co{mineness}. It is not an \co{act} because \co{I} do not perform it and because
it is not limited to the \hoa. It is an event in the sense that \co{I}
\co{participate} in it, it affects not only \co{me} but \co{my} very
being. \co{Participation} does not require 
even slightest \co{reflective} consciousness -- it is exactly something which
\co{transcends} the \co{actuality} of a mere \thi{being a part of} or \thi{taking a
part in}. \co{I participate} because \co{I am not the master}. If \co{I} were,
\co{participation} would be impossible, for one does not \co{participate} in
things and actions one controls -- one executes or masters them.
%; physical or mental presence has little to do with \co{participation}.

But, on the other hand, one can agree (or disagree) with all that has been said
above about the \co{choice}, one can \co{reflectively} consent or dissent,
\co{reflectively} say \Yes\ or \No. As \co{my self} can be confused with
\co{myself}, the transcendental ego with the empirical one, so here we have the
possibility of confusing the \sch\ with the \co{actual} one, the event with the
\co{act}. The latter will \co{recognise} some \co{actual signs} of various
\co{aspects} of the \sch.  A possible \co{reflective sign} of the \co{trust},
for instance, might be the \co{actual} faith that it is indeed possible to exit
hell from which there is no exit, that suffering is not an evil revenge but a trial
teaching one that evil can not possibly have the last word. But it might equally
be the absence of any such explicit 
faith which, however, still does not exclude the possibility of release.
% This is no paradox, and if it appears to be one, one need not worry. As long as
% one knows not what is being said, as long as one has not been in the vicinity,
% \citet{as long as one is not able to receive this truth immediately, one will
%   not understand this language.}{Eckhart}{\citaft{Mistyka}{
%     A:III.6}\kilde{p.47}}
This might, perhaps, look like a paradox to somebody who has not been in
vicinity of \co{nothingness}.  But who has not been in the vicinity?  \Yes\ does
not require a descend to hell.  Moreover, although its reality, \co{grace},
happens in the \co{invisible} depths, in the very \co{origin} of Being, we may
attempt a \co{reflective} description of some of its \co{aspects}.  Such
\co{actual} expressions will constitute elements of the \co{reflective} \co{act}
of the choice of \yes, of the \co{act}, or rather variety of \co{acts}, which it
is in our power to perform and which may influence the \co{invisible choice}.

The \co{choice} of \yes\ means to recognise \co{nothingness} as the
\co{one}, the ultimate \co{invisibility} as the \co{origin}.
In such a recognition, \co{I} admit several things. 


\noo{ To know only that it is \co{indistinct}, that is, \co{above} all
  \co{distinctions} but, as opposed to \No, that this \co{indistinctness} is the
  \co{origin} of all \co{distinctions} and not mere void and emptiness.
  
  And only because it knows it, or if you prefer, because it is it, it can also
  deny it, veil it by saying \No\ to the \co{confronting transcendence}.
  
  and \co{that} this element can be a damnation or blessing. \co{That} is all.
  
  \co{That} makes it also possible to still call knowledge, although it has only
  remote relation to what is usually called by this name.  It is not anything
  \thi{self-produced}, anything \thi{made by me} -- \co{that} is encountered as
  \co{transcending} the horizon of \co{mineness} }

%\ad{Humility. }
\pa\label{humility} Firstly, \co{reflecting} the insufficiency not
only of \co{my actual} but even the \co{spiritual choice}, \refp{pre:humility},
\co{I} admit that \co{I am not the master}. The \co{distinctions} among which
\co{I} live, and some of which \co{I} actually can control, originate beyond the
sphere of \co{visibility}. My understanding is limited to this latter sphere,
but it does not embrace it, and it can only \co{vaguely}, and hardly
intentionally, \co{reflect} the workings of \co{invisibles}. \co{I} understand
\co{myself} when \co{I} understand \co{my} limits, but the very fundamental
issue is first to recognise that such limits at all exist, that \co{I} end where
\co{this world} ends, and that beyond there is something which, from the
perspective of \co{this world} is but \co{nothingness}. \co{That I am not the
  master} is to say that there is something more than \co{I} and \co{this
  world}, that the \co{nothingness} beyond it is not emptiness, is not lack of
reality but, on the contrary, is the most real source of whatever is encountered
in \co{this world}.
%All this amounts to \co{humility}.
Eckhart asks \citeti{When does one stay humble? I answer: When you
  apprehend One separated from others. And when does one step beyond humility? I
  answer: When one apprehends everything in everything, then one steps beyond
  humility.}{Eckhart}{\citaft{Mistyka}{ A:IV.II.4.1.b)\kilde{p.58}}}

The limits which are not to be overstepped are not any \co{visible} lines
stretched around by some authoritarian ruler. They are limits of \co{visibility}
as such, the \co{invisible} limits which are impossible to overstep anyway, and
\co{humility} is simply an accepting recognition of this fact.
\citet{Humility is that whereby we refrain from the desire for empty glory, so
that we don't desire to seem more than we are.}{AbelardPJC}{II:\para 291}
\co{Humility} does not mean that \co{I} recognise any particular master who
is governing \co{me} or \co{this world}. \co{Humility} in this \co{spiritual}
form is not a submission to any definite power. Even more, it excludes such a
power or, to the extent it experiences it, it transcends it.  Encounter with any
awe inspiring, ineffably powerful \la{tremendum sacrum} may easily lead to
humiliation rather than to \co{humility}. Humility which is a reaction to
anything specific, which is caused by no matter how \co{vague}, but still a
particular cause or power, is perhaps an emotion, \citet{the sorrow produced by
contemplating our impotence or helplessness,}{SpinozaEthics}{III.On the origin
and nature of the emotions:Definitions of the emotions:26} but it is not a
true, \co{spiritual humility}. This latter is 
\co{humility} in the face of \co{nothingness}, one does not submit to anything,
yet one {\em submits} -- only \co{that} makes it truly \co{spiritual}
submission.  \citet{Let me be humble, that is,\kilde{one who} thirst for the
origin.}{HerbertPray}{}

%\ad{Thankfulness. }
\pa\label{thankfulnes} \co{Humility} faces the ultimate \co{gift} and amounts
simply to its acceptance.  In this acceptance, \co{I} admit that \co{the world}
in which \co{I} live is given to \co{me}, is the result of a process which might
have involved \co{my self} but certainly not \co{myself}. The world, just like
\co{grace}, is a generous --
because unmerited -- \co{gift}. %,\refp{pre:thankful}.
This does not mean that no single thing in \co{this world} is a result of \co{my
  activity}, only that, eventually, all such things are grounded in the
\co{transcendent} sphere of \co{invisibility}. \co{The world} is a \co{gift},
\co{my life} is a \co{gift} and everything that \co{I} ever encounter is a
\co{gift}. \co{I am} while there is no sufficient reason for \co{me} being here,
\co{I am} while \co{I} might not be.  Recognition of \co{the world} and \co{my
  life} as a \co{gift} of \co{transcendence} amounts to \co{thankfulness}.
\co{Thankfulness} is but an expression of \co{humility} which admits that \co{I}
have received and am grateful, unlike \citet{the miser [who] always fears
  presents}{Haavamaal}{48} because they threaten his sense of independence and
power.

It is essential for \co{spiritual thankfulness} that we recognise
the \co{gift} as arbitrary, as having no sufficient reason --
\co{creation} is a mystery.  The recognition of the \co{one} as the
generous source admits only that it is a necessary but not a
sufficient condition.  Any search for sufficient reasons, any attempt
to explain the necessity of this \co{gift} amounts to explaining it
away and to renouncing the attitude of \co{thankfulness}.  The
arbitrariness of the \co{gift} is what \co{founds} the \co{spiritual 
thankfulness}. Its \co{spiritual} character means just that it
 is not thankfulness for anything specific; it is
\co{thankfulness} for \co{nothing}, that is, for everything.  If only
one starts looking for reasons for being thankful, for any positive
things worth gratitude, one renounces the \co{spiritual} dimension of
\co{thankfulness}. (Which, of course, does not mean that one necessarily
opposes it.) 

%\ad{Openness. }
\pa\label{openness} The arbitrariness of this \co{gift} means that
it might not have been given or else that it might have been entirely
different. Instead of \thi{this} \co{I} might have gotten \thi{that}, instead
of being \thi{this} \co{I} might have been \thi{that}.  No matter what, in 
\co{precise}, \co{visible} terms, \co{I} have obtained, does not change the
nature of the \co{gift} and deserves the same \co{thankfulness}.  Anything that
\co{I am} or encounter is but an instance of the fundamental
generosity. This unreserved \co{thankfulness}, the unreserved
acceptance of the \co{gift}, amounts to \co{openness} in which a man,
\citet{[w]hatever befalls him,\lin He lives in 
  happiness.}{Ash}{XVII:7}

It \co{reflects} the \co{presence} which was suggested in \refp{pre:openness},
the \co{presence} which already permeates the whole world, which can be found in
every place, \wo{whether in a pigsty or in a palace}.  Consequently, it is not
\co{openness} to this or that, to anything specific; in particular, it is not
\thi{identification} with \co{my} things, \co{my} pleasures, \co{my} friends. It
is unreserved, unrestricted acceptance, \co{spiritual openness} to \co{nothing},
that is, to everything.

%\ad{Love}
\pa These three \co{aspects} -- \co{humility}, \co{thankfulness} and
\co{openness} -- we call jointly \wo{\co{love}} or, to avoid confusion, the
\wo{\co{spiritual love}}.\ftnt{The
Christian \wo{\la{agape}}, \wo{charity}, or even \wo{obedience}, in the sense used
by the Church Fathers and mystics, may be here equally good -- in fact,
synonymous!  -- words.}  \co{Love} does not \thi{consist of} \co{humility},
\co{thankfulness} and \co{openness}; it is the unified and \co{indistinct}
attitude and these are but \co{aspects} of the same \nexus\ of
\yes.\ftnt{According to Pseudo-Dionysius, the first, 
closest to Godhead sphere of the celestial hierarchy comprises three kinds of
Angles: The Seraphim -- the fiery, purifying, inspiring source, The Cherubim --
the illuminating light, and The Thrones -- the perfecting and receptive
openness.} Other \co{aspects} might be listed, but it should not be necessary to
exaggerate elaborations. For instance, \co{openness} amounts to \co{trust} as much as
to \co{hope}, while \co{humble thankfulness} to fidelity.
%To further such analyses , let us only summarise that 
\co{Love} is the \co{aspect} of \co{grace}, the first element \co{founded concretely}
in the \co{spiritual} \Yes. Or put
differently, \co{love} is \co{grace} in so far as it has \co{visible
manifestations}, eventually even as it becomes \co{actually reflected}. 

% is \co{actually reflected} -- expressed and \co{manifested}, but also,
% \co{reflected} over.
\noo{
In Book II we have made occasional references to love which, however, was
usually something different.  It was personal love, love at the level of
\co{mineness}.  The two are not to be confused.  \co{Love} as \la{agape},
\co{spiritual love} has no object, it is \co{love} of \co{nothing}, that is, of
everything.  It is \co{spiritual} because it is the attitude towards the
\co{origin} as \co{origin}, in its full, \co{indistinct invisibility}.  As such,
it is not restricted to any particular region of Being but penetrates the whole
of it and is \co{present} in every encounter with any particular being.
Personal love is, at best, only a restricted form of such \co{love}.
}

\subsubi{Works}

\pa\label{pa:sameSame}
% As the above references to the respective aspects of the \co{spiritual choice}
% \refp{pre:openness}-\refp{pre:thankful} indicate, it may be hard to distinguish
% the admissions of the \co{spiritual} \yes\ from those made in \co{reflective}
% way.
The difference between the \co{spiritual} and \co{reflective} \Yes\ concerns 
not so much the contents, as the 
place they occupy in the field of \co{existence}.
If a \nexus\ \co{founds} particular forms of understanding or \co{acts}, then
achieving such a form of understanding or performing such \co{acts} will contribute
to formation or strengthening of this very \nexus. 
\co{Grace} of the
\co{spiritual} \yes\ is the relation of Being in the heart of \co{existence}, it
embraces its whole being to its very depth, without leaving anything outside. At
the same time, it may remain almost indifferent with respect to \co{actual}
situations and moods -- indifferent, that is, unnoticed if one tries to capture
it by the \co{actual} look. The contents of the \co{reflective} \yes, on the
other hand, occupy \co{actuality} without, necessarily, witnessing to the
\co{presence} of their \co{spiritual} counterparts. \co{I} can think as long as
\co{I} wish about humility and thankfulness, without ever becoming \co{humble}
and \co{thankful}. \co{I} can even perform a lot of humble \co{acts} which,
however, do not make me \co{humble} (especially, if humility is my intention.)

A closer relation between the two levels obtains when the \co{reflective} choice
is made genuinely, that is, \co{actually} tries to \co{reflect} the \co{aspects}
like those listed above in its own attitude, when it does not deliberate
thankfulness, but tries to find it, does not ask about humility, but tries to
live it; when the \co{aspects} of \co{love} cease to be any goals of particular
\co{acts} and become constant \co{motives} and \co{inspirations} of one's whole
\co{activity} and life.  The \co{actual} choice is not even a necessary
condition for the \co{spiritual} event of \yes, but it is certainly helpful,
especially for \co{reflection} which has already been involved into the game of
\co{invisibles} and which is in a sore need of \co{clarifying} it.  Eventually,
the \co{spiritual choice} says only \yes\ to \co{nothingness}, and so does the
\co{reflective} one. But \citet{[i]t requires an eminent reflection, or rather a
  great faith, to sustain a reflection on nothing, which is to say infinite
  reflection.}{Despair}{I:B [56]}

\noo{There is, finally, yet another possibility, namely, that of the \co{reflective}
choice \co{actually} \co{reflecting} the \co{spiritual} \Yes, when the \co{act}
is \co{founded} in the prior \co{spiritual} \Yes.  We speak than of the
\co{actuality} and the \co{actual signs} being \co{concretely founded}.
The \co{invisibles} which are thus \co{present} in \co{concretely founded
actuality} are said to be \co{incarnated} -- \wo{\co{incarnation}} is the
expression for \co{concrete founding} viewed \thi{top down}. 
\co{Acts} and attitudes which help towards \sch\ of \yes\ are of the same kind
as those which are \co{concretely founded} by this \co{choice}.  We will now
present in some more details the \co{concrete founding} arising from
\yes. At the same time, it will provide examples of particulars which,
remaining within \co{my} control, may help \co{me} towards \yes.
}

Works and even particular \co{acts} -- the elements of \co{reflective} attitude
-- do contribute to the \co{invisible} sphere, and when performed in a right
attitude can contribute to \yes.  The general mechanism of such an influence is
the same as described in II:\ref{sub:lowHigh} -- although the \co{actual
  experiences}, feelings, thoughts, \co{acts} do not influence directly the
sphere of \co{invisibles}, yet they accumulate and pass gradually into the
\co{virtual} depths of the soul. It is not so that \woo{in every good work the
  just man sins,}{Luther, one of the theses condemned by Leo X} that
\citet{every work they attempt is accursed,}{Calvin}{III:19.2.4} for good works
accumulate and strengthen the goodness of the soul. But there are no obvious,
causal or otherwise, connections, no guarantees nor any \co{precise} rules
determining the \co{virtual} effects of the \co{actual} works. Also, all
\co{actual} elements are surrounded by the uncontrollable \co{rest} which, too,
adds up to the result. The descriptions remain forever partial\ldots
Fortunately, \citet{the gods have a care of anyone whose desire is to become
  just and to be like God, as far as man can attain to the divine likeness by
  the pursuit of virtue.}{PlatoRep}{613a:7.b.1\kilde{[Copleston I;p.218]}} \wo{As far
  as man can\ldots} because actually attaining to this likeness, the \sch\ of
\yes, is an \co{invisible} event. Its dependency on the \co{visible}
sphere\noo{(just like the \co{actualities} passing into \co{virtuality},
  II:\ref{sub:lowHigh})} may be claimed but never observed, may be concluded but
never proved. A \co{spiritual} event, when it comes, comes only and unmistakably
as a \co{gift}.

\pa 
%\ad{As above, so below}
\citet{The faith of man follows his nature. Man is made of faith: as his faith
  is so he is.}{Bhagavad}{XVII:3} As above, so below, and we have followed this
direction almost all the time.  And yet, \citetib{[n]ot by refraining from action
  does man attain freedom from action. Not by mere renunciation does he attain
  supreme perfection. For not even for a moment can a man be without
  action. [...]
%}{Bhagavad}{III:4-5}
  For there is no man on earth who can fully renounce living work, but he who
  renounces the reward of his work is in truth a man of
  renunciation.}{Bhagavad}{III:4-5/XVIII:11} The work done with thankful
acceptance of any, possibly even none reward, with humble renunciation of
one's pretensions to ownership and authorship, with exclusive attention to its
own standards -- such work marks the path on which \citetib{[n]o step is
  lost.}{Bhagavad}{II:40} Complete dedication means that the work, needed for my
life as it may be, is actually a sacrifice, an \co{expression} of
self-surrender.  \citetib{Offer all thy works to God, throw off selfish bonds,
  and do thy work. [...] This man of harmony surrenders the reward of his work
  and thus attains final peace: the man of disharmony, urged by desire, is
  attached to his reward and remains in bondage.}{Bhagavad}{V:10/12} This
bondage of \co{attachment} may, too, have the appearance of intense and
dedicated work. But such an \co{inverted} form announces not a peaceful
self-renunciation but self-annihilating and all-consuming insatiability. On the
other hand, \citetib{[t]he man who in his work finds silence, and who sees that
  silence is work, this man in truth sees the Light and in all his works finds
  peace. [...] In whatever work he does such a man in truth has peace: he
  expects nothing, he relies on nothing, and ever has fullness of joy. [...] He
  is glad with whatever God gives him, and he has risen beyond the two
  contraries here below; he is without jealousy, and in success or in failure he
  is one: his works bind him not.}{Bhagavad}{IV:18/20/22} Silence, after all, is
the voice of God, but to hear it, \co{above} noises, is the end and not the
beginning.

% , it means to be but not alone, not \co{alienated} but surrounded by the
% \co{invisible presence}.

\pa Pure work is {\em both} an \co{expression} of \yes\ {\em and} the means of
approaching it.  \citetib{Seekers of union, ever striving, see him dwelling in
  their own hearts; but those who are not pure and have not wisdom, though they
  strive, never see him.}{Bhagavad}{XV:11} Like many apparently vicious circles,
so this mutual dependence, underlying all the disputes about the primacy of
faith over works or works over faith, is but a \co{trace} of a \nexus\ -- here,
the \nexus\ of \yes. Nothing is first; the fact that we can decide and attempt
only what is \co{actually} in our power, does not make the works either superior
nor inferior to the faith. \citet{Faith without works is empty, works without
  faith are blind.}{CrPR}{I:2.Introduction.1, A51/B75} Both follow the same
course and neither is possible without the other. Works contribute to faith and
faith can not fail to manifest itself in works.

If only we do not try to reduce goodness to any utility, usefulness or other
\co{visible} categories, we can say: the works of a good man are good and only
such works are good.\ftnt{\wo{The ground upon which good character rests is the
    very same ground from which man's work derives its value, namely a mind
    wholly turned to God. Verily, if you were so minded, you might tread on a
    stone and it would be a more pious work than if you, simply for your own
    profit, where to receive the Body of the Lord and were wanting in spiritual
    detachment.}  Eckhart \citaft{Huxley}{XI\kilde{p.178}}} We would certainly
not speak about the necessity or indispensability of works; indeed,
\citeti{Though they dig into hell, thence shall mine hand take them; though they
  climb up to heaven, thence will I bring them down.}{Amos}{IX:2} Yet, we would
object to denying works any helpful function.\ftnt{The objections against such a
  possibility of \thi{influencing God's will} do not concern us.  It is clear
  that \co{indistinct} is not affected by anything happening within the world,
  among the \co{distinctions}. God's \thi{will} (if one wants to insist on such
  a language) remains unchangeable: \co{actuality} remains ontologically
  \co{founded} in the \co{one} and anybody willing to return is invited to and
  promised the possibility. But God's appearance changes and it changes exactly
  according to \co{my life}. It is \co{I} who live \yes\ or \No. Works help me
  to do the one or the other and they are {\em the only} way in which \co{I} can
  possibly help \co{myself}. But they are also {\em only} help: the eventual
  result is not in my power.} Thus:
\begin{enumerate}\MyLPar
\item Works are not indispensable but are helpful.
\item They are the only things which \co{I} can intend and, 
 to some extent, control, with respect to my
  \co{spiritual} destiny.  
\item They are thus the only \co{visible} means -- and hence the {\em only}
  means -- of striving for the \co{unity},
  of keeping heaven and earth together.
\end{enumerate}
Their \co{spiritual} relevance is determined not only by their content but
primarily by their \co{rest}. Those bringing \co{me} closer to \yes:
\begin{enumerate}\setcounter{enumi}{3}\MyLPar
\item are dedicated to God, are only a \co{visible} \co{expression} of the \co{spiritual}
  self-renunciation, 
\item are not \co{mine} and are not performed for any reward. 
\end{enumerate}  
In section \ref{sub:concrete} below we will consider 
\co{concrete founding} effected by \yes. In this connection, we will see several 
specific examples of attitudes \co{founded} by \yes\ which provide thus
%(as observed in \refp{pa:sameSame})
also examples of attitudes strengthening the
\co{invisible} currents leading to \yes. 

%%%%% end \subsubi{Works}
%%%%% - the rest (to \subsubi{Experiences or projections?}) is \noo{'ed

\noo{
\citet{`I am not doing any work', thinks the man who is in harmony, who sees the
truth. For in seeing or hearing, smelling or touching, in eating or walking, or
sleeping, or breathing, in talking or grasping or relaxing, and even in opening
or closing his eyes, he remembers: `It is the servants of my soul that are
working.'}{Bhagavad}{IV:8-9}

\citet{Set thy heart upon thy work, but never on its reward. Work not for
  reward; but never cease to do thy work.}{Bhagavad}{II:47}

\citet{In liberty from the bonds of attachment, do thou therefore the work to be
done.}{Bhagavad}{III:19}

\citet{A man attains perfection when his work is worship of God, from whom all
  things come and who is in all.}{Bhagavad}{XVIII:46}

\citet{Even as a burning fire burns all fuel into ashes, the fire of eternal
  wisdom burns into ashes all works.}{Bhagavad}{IV:37}

\citet{But work done without faith is \la{asat}, is nothing: sacrifice, gift, or
self-harmony done without faith are nothing, both in this world and in the world
to come.}{Bhagavad}{XVII:28}

\citet{Works of sacrifice, gift, and self-harmony should not be abandoned, but
  should indeed be performed; for these are works of purification. But even
  these works, Arjuna, should be done in the freedom of a pure offering, and
  without expectation of a reward. This is my final word.}{Bhagavad}{XVIII:5-6}
}


\noo{
\pa 
Thus, like every \co{actual experience}, an
\co{act} is a cut through and out of the sphere of \co{experience}, 
across all levels of my being, in particular, across the sphere of 
\co{visibility} as well as \co{invisibility}. It 
happens within
\begin{enumerate}\MyLPar
\item the \hoa  %-- the part of \LL\ made actual;
\end{enumerate}
but it also has                   
\begin{enumerate}\MyLPar \setcounter{enumi}{1}
\item the horizon of \co{presence} -- the \co{rest}, 
the part of the \HH\ sphere which manifests itself in the \co{act}.
\end{enumerate}
It thus also has the respective limits
\begin{enumerate}\MyLPar \setcounter{enumi}{2}
\item an \co{object} - the intentional focus, the limit of \co{actuality}; and
\item an \co{inspiration} -- the limit of \HH\ which is \co{present}.
\end{enumerate}
The \co{invisible} limit is, eventually, the \co{One} which is \co{present} 
only along with the cut through all the intermediary levels lying between 
it and the \co{actuality}. 

\pa The \co{presence} influenced by the \co{act} can never be its
intentional \co{object}.  To the extent \co{I} try to make it such, it
withdraws and changes its character.  In the most trivial cases, if
\co{I} think \wo{I have to learn swimming.  I have to learn
swimming\ldots} while trying to follow the instructions, \co{I} will
have very hard time.  The best \co{I} can do is to concentrate on
following the instructions.  If the intention of \co{my act} is
\thi{to be good}, \co{I} may happen to do a good thing, but the
\co{invisible aspect} of the \co{act} has then withdrawn beyond the
horizon of this intention and brought forth \co{presence} of something
more than what \co{I} intended.  (Probably, of what motivated \co{me}
in the first place to try \thi{to be good}.)  It is not that \co{I} am
not conscious of what, eventually, hides behind my \co{acts} -- it is
that \co{I} {\em can not possibly} make it consciously \co{actual}, in
any case not in the \co{act} itself nor in the context from which the
\co{act} emerges.
}
%%%%%%%%%%%%%%%
\noo{
\pa Acts of withdrawal, patience, humility (\?{humiliation}) will make
\HH\ present.

\pa
A single act in itself may mean many different things -- it becomes
signifying only in the context of its totality, i.e., its \?{inspiration}.
(cf.~\refp{anymisunderstood})

\pa We may misunderstand an act if we do not know the context of
activity, an activity if we do not know the life of the person, and
the life of a person if we do not know the absolute values which it
embodied.  


\pa Often, we have to know the horizon to understand the act.  Without
knowing a person, we will sometimes misunderstand or not understand
what he is doing.  But when we have known him for a while, we will say
``Of course, it is him, his way of doing this...''.
}

\noo{
\begin{tabular}{l@{\ \ \ \ }l}
\parbox{9.5cm}{\levels{Activity} 
\aretab{ll}{single act}{complex of acts (action/activities) & moods, thinking, control}
         {active passivity & attitudes, commitments}
         {passive activity & revelation} }
\end{tabular}
}

%%%%%%%%%%%%%%%%

\subsubi{Projections?}\label{sub:projections}

\pa One could perhaps ask the natural question which appeared briefly in
\refp{pre:exnihilo}. Does not \sch\ amount to a projection?  Do 
we not say that the \co{indistinct} and unknowable \co{one} has to be endowed
with the qualities of the source, goodness, power and what not?  The answer is
no, and if you see this, you may safely skip this section.

Indeed, there \citet{can be no greater incongruity than [for a disciple of
  Spencer] to proclaim with one breath that the substance of things is
  unknowable, and with the next that the thought of it should inspire us with
  awe, reverence, and a willingness to add our co-operative push in the
  direction toward which its manifestations seem to be
  drifting.}{Pragm}{I;p.19}\noo{ The \sch\ does not call for any awe,
  reverence, or any willingness to add this or that to this or that. It does not
  call for any attitude -- it helps create it. The \co{actual} attitude will in
  any case be determined also by \co{my} character, \co{my} intelligence,
  \co{my} physiology, by all \co{visible} and hardly \co{visible}, conscious and
  unconscious aspects of \co{my} being. In \co{precise} terms, the \co{choice}
  does nothing and the \co{love} merely adds a gentle touch, an \co{invisible
    rest}, which is as important, deep and decisive as it may seem
  insignificant.  } There might be an incongruity in suggesting that the
\thi{unknowable} should inspire one to anything. But we have neither anything
\thi{unknowable} nor call for any inspiration by something. The \co{indistinct}
is unknowable only if knowledge means the \co{actual} \la{episteme}, knowledge
of \thi{whats}. But we do have full knowledge of it, we know all that is to know
about it, if not in this narrow sense, then in the sense of \co{gnosis} -- it is
\co{indistinct}, \co{one above distinctions}. As to the inspiration then,
indeed, it is only to silence. But silence can be a calm voice of eternity or a
mute emptiness. It does not inspire: it leaves you completely free to make your
choice.  This choice, the \sch\ is a thoroughly real {choice} between the only
two alternatives offered -- not by the \thi{unknowable} but by the \co{absolute}
which, in its \co{indistinctness}, remains indeed indifferent. Only we are
affected, and we are affected by \co{confronting} the face of \co{one} which
corresponds to our \co{choice}.  What we have done with, or rather {\em out of}
the concept of \co{indistinct} in Book I, can be taken as a mere illustration of
the grounds which might incline one towards seeing it as the true \co{origin},
towards accepting it as truth, that is, towards saying \yes. There is, however,
no necessity, no sufficient reasons which might force one to make this, rather
than the opposite choice. If there were any, then it would not be any choice.

% But the \co{choice} itself is thoroughly real, it is a choice between two real
% alternatives.

\pa We should carefully distinguish the \co{choice} from mere psychological
effects.  According to James, \citet{to find religion is only one out of many
  ways of reaching unity [\ldots] In judging of the religious types of
  regeneration [\ldots] it is important to recognise that they are only one
  species of a genus that contains other types as well.  For example, the new
  birth may be away from religion into incredulity; or it may be from moral
  scrupulosity into freedom and license; or it may be produced by the irruption
  into the individual's life of some new stimulus or passion, such as love,
  ambition, cupidity, revenge, or patriotic devotion.  In all these instances we
  have precisely {\em the same psychological form} of event, -- a firmness,
  stability, and equilibrium succeeding a period of storm and stress and
  inconsistency.}{Varieties}{VIII\kilde{p.176}}

One can form hierarchies of genera and species as one finds appropriate but if
these have \thi{the same psychological form} (which here probably means
something like {psychologically indistinguishable}, even if contentually
different), then thank you very much for the psychological contribution --
\la{pneuma} is \co{above} \la{psyche} and here our ways part definitely with
psychology.  Indeed, having only \co{actual experiences} of a \thi{subjective}
psyche as the basis of distinctions, all such states may happen to end up in the
same sack.  Yet, even James does not include these later instances in his
treatment of the religious experience.  So, after all, they are distinguishable?
The sense of purpose, of direction and goal, of mission, or else of finding a
valuable sphere of \co{experience} may indeed, especially if taken as absolute,
give firmness and stability.  All \co{idols} can, and many minor matters can.
\co{Idols} are seldom entirely false -- they gain followers exactly because they
contain an element of truth.  But the \co{unity} \co{founded} by the \sch\ is
not derived from any sense of goal, direction or mission -- the goal is
\co{nothing}, the direction is \thi{anywhere}, and the mission is
\citet{\co{love}, and do what you wilt.}{AugustJohn}{VII:8}\noo{As much as we
  owe to pragmatism when applied to the \co{visible} phenomena, here our ways
  part definitely also with its ways.}  \noo{\citet{Love, and do what you wilt:
    whether thou hold thy peace, through love hold thy peace; whether thou cry
    out, through love cry out; whether thou correct, through love correct;
    whether thou spare, through love do thou spare: let the root of love be
    within, of this root can nothing spring but what is
    good.}{AugustJohn}{VII:8} }


\co{One} can \co{manifest} itself in innumerable ways which may be
psychologically as different as trembling and adoration, as fear and attraction.
These differences can often be found behind different \co{visible} characters of
various religions.

\co{An experience} of God's presence will almost inevitably have tremendous
influence on one's life, and the form of this influence may depend heavily on
the character of the experience.  But it is not its character, its content which
accounts for its influence -- it is the lack thereof, \co{expressed} as the
tremendous force, \la{majestas}.  Psychologically distinguishable content plays
its part but what is constitutive for such \co{an experience} is what this
content reveals -- the ultimate, \co{absolute} force which groans into one's
face without showing its own.  It is the {\em intensity} of such \co{an
  experience}, its irresistible power, which is its essential content, not the
form under which this content appears.  And this power is \co{objectless} and
contentless, it has no agent, it is the power of \co{nothingness}, but it {\em
  is}. There is, consequently, nothing to be projected, there is only something
to be \co{recognised} -- in the most mundane sense, \co{that I am not the
  master}.

\pa \Sch\ is not a matter of \co{any experience} just like religiosity is never
reducible to \co{any experiences} which, perhaps (though even this \thi{perhaps}
seems too much), may be psychologically indistinguishable from a sudden attack
of fear on a neurotic (or even a healthy) person, or from a sense of ecstatic
joy which recurrently visits an infantile or {senile} one. It happens in a
meeting with the \co{absolute nothingness} which involves {\em full and active
  participation} of the invaded.  \Sch\ can take place long after the actual
experience and also when no such experience ever found place.

A meeting with \co{absolute} \thi{objectivity} does not require any specific
context or experience, even if specific experiences play usually
(psychologically) some role of motivating factors.  Such experiences are
possible \co{actualisations}, as Otto says, \thi{schematisations} of the \la{a
  priori} ground of all \co{experience}.  To the extent the \co{presence} of
\la{numinosum} is recognised \co{beyond} their content, they themselves are
\la{a priori} -- irreducible to any \co{visible experiences} or categories,
concepts or feelings, but grounded in the ultimate \co{invisibility}.

%
The fact that \co{absolute} may (in fact, always does) invade only one
person and not another is such an argument for its \thi{subjectivity}
as it would be against the objectivity of Japan that some people were
there while others were not.  That it is unverifiable?  %Excuse me! 
What is?  %We have read Popper, have we not?
% \ftnt{Quite a clear statement of the basic idea of falsificationism can be
%   actually found in \citeauthor*{Fruth}, II:21.}
It is as unverifiable as the accusation of its being a \thi{subjective
  projection} is self-confirming.  For as long as one insists on such a
characterisation, one is simply unable to get any meaning whatsoever of its
nature and value.  But \citet{\la{a priori} recognitions are not the ones which
  everybody has but ones which everybody {\em may} have.}{Heilige}{XXII\kilde{p.195}
  (\wo{Recognition} translates here the word \wo{\ger{Erkenntnis}}; my emph.)}


\pa Simply: whatever meets one in any situation, comes from beyond the horizon
-- not only the \hoa\ but, eventually, the horizon of distinguishability.
Certainly, many things are expected and predictable but they, too, enter not on
one's command but on their own -- one can at best help them, never force them.
And they, too, eventually emerge from beyond the horizon of the distinguished
contents, from the \co{indistinct}. \co{Reflective} \yes\ is hardly much more
than that, than the admission \co{that you are not the master}, that
\co{indistinct} is and remains \co{indistinct}, albeit, remaining so is also the
source of \co{distinctions}. In particular, it is exactly the attitude of {\em
  not projecting anything} beyond the horizon of distinguishability, but merely
admitting its constant \co{presence}. It is, on the contrary, any other attitude
which amounts to a projection: either projection of emptiness, as in the case of
definite \No, or a projection of some \co{idol}, as for instance the \co{idol}
of \co{actuality} in the case of \co{objectivistic illusion}. \yes\ admits only
that, in the face of \co{nothingness}, any requirement or expectation of
something specific, of something \co{visible} is \co{idolatry} grown from
\co{thirst} which, in the face of \co{nothingness experienced} as emptiness, 
is amplified to angst. \citet{The soul or mind reaching 
  towards the formless finds itself incompetent to grasp where nothing bounds it
  or to take impression where the impinging reality is diffuse; in sheer dread
  of holding to nothingness, it slips away. The state is painful; often it seeks
  relief by retreating from all this vagueness to the region of sense, there to
  rest on solid ground, just as the sight distressed by the minute rests with
  the pleasure on the bold.}{Plotinus}{VI:9.3}

\Sch\ is not a choice {\em of} love, {\em of} morality, {\em of} charity, {\em
  of} unselfishness, or whatever.  It does not aim at any such nor any other
  thing.  It is the pure and 
bare \yes\ (or \No).  It does not choose any specific content which it might try
  (or has wished) 
to project \thi{outside} of its \thi{subjectivity}.  \yes\ chooses only silence,
the \co{confrontation} with \co{nothingness} (and \No\ exchanges this 
\co{confrontation} into words, concepts and, eventually, emptiness). \citetib{If the
  mind reels before something thus alien to all we know, 
  we must take our stand on the things of this realm and strive thence to see.
  But, in the looking, beware of throwing outward; this Principle does not lie
  away somewhere leaving the rest void; to those of power to reach, it is
  present; to the inapt, absent.}{Plotinus}{VI:9.7}
It is \thi{throwing outward} which amounts to projections, to either emptying
\co{nothingness} and reducing it to a void, or else to populating it with finite
\co{idols}.  As it happens, the choice of \yes\ has tremendous consequences, but
these are consequences, not projections.

\noo{ %not needed projections
  
  \pa Consequently, neither the genuine \co{analogues} are any kind of
  projections.  They are certainly not any psychological tricks played by
  subconsciousness on one \thi{who does not know what he is {\em really}
    speaking about}.  Such a critique \fre{\'{a} la} Feuerbach applies to the
  uncritical use of \co{analogues}, the use which \co{posits} some indefinite
  \co{object}, a projection of \co{reflective} understanding, and endows it with
  some attributes.  But when used in the genuine sense, they are
  \co{reflections} of the underlying \co{experience}, of the \co{incarnation} of
  \co{love} -- they are \co{expressions}.  Since this \co{experience}, once it
  comes, confronts \co{me} with the ultimate \co{transcendence} and its
  \co{gift}, there is no way of seeing it as a mere projection of human, all too
  \thi{subjective}, psyche, unless one is unable to renounce the absolutism and
  narrowness of the distinction \thi{subjective}-\thi{objective}.  Saying that
  \wo{God is but a projection of myself} is to conflate completely \co{myself},
  \co{my self} and \co{self}, and to confuse the \co{transcendence} of the
  latter met in the \co{experience} of the former.\ftnt{Instead of
    \wo{projection}, one might perhaps find another word for the unity of
    \co{spiritual aspects} as applied to soul or world.  Irenaeus, relates views
    of gnostic provenance, according to which the passions and \gre{enthymesis}
    (inborn idea) of Sophia are materialised and turned into substance.
    \citef{For her enthymesis having been taken away from her, along with its
      supervening passion, she herself certainly remained within the Pleroma;
      but her enthymesis, with its passion, was separated from her by Horos,
      fenced off, and expelled from that circle. This enthymesis was, no doubt,
      a spiritual substance, possessing some of the natural tendencies of an
      Aeon, but at the same time shapeless and without form, because it had
      received nothing. [\ldots] who would not expend all that he possessed, if
      only he might learn in return, that from her tears of the \gre{enthymesis}
      of the Aeon involved in passion, seas, and fountains, and rivers, and
      every liquid substance derived its origin; that light burst forth from her
      smile; and that from her perplexity and consternation the corporeal
      elements of the world had their formation?}{Irenaeus}{I:2.4/4.3\kilde{in
        Roberts and Donaldson, Ante-Nicene Fathers, vol.  I, [Creation of
        Consciousness, p.72]}} But who will want to see behind such a mere
    \thi{projection} a {\em genuine} process of \ger{Beseelung},
    \thi{ensoulment}?  If one thinks that projections are something discovered
    only in XXth century, one might consult again Gnostics and neo-Platonists,
    with their \gre{probol\'{e}}, which usually translated as \thi{emanation},
    may be equally well rendered as \thi{projection}.\kilde{Creation of
      Consciousness, p.76} }
  
  \pa As said in \refpp{GodheadGod}, the \co{analogues} are only reflections of
  the respective aspects of the underlying attitudes of \yes\ or \No.  But the
  word \wo{\co{analogue}} does not mean that these are some artificial
  constructions.  They are, in fact, the possible qualities of \co{experiences},
  and the tendencies, as well as one's attitude, can be \co{experienced}
  precisely through such \co{analogues}.
  
  \pa If one experiences world, life, things, the indefinite everything as
  fullness, as good, as powerfully meaningful, one has perhaps not said \yes,
  but one has a good chance to do so.  Such \co{experience}, or such qualities
  of some \co{experiences}, are ways of \co{presence} of \co{invisibles} even if
  the \co{act} of \sch\ has not taken place.  \co{An experience} \co{that it is}
  can be embraced by an aura of intense beauty and wonderful mystery, even if
  the \co{actual object} has no such qualities.  (Does any object {\em have}?)
  Such \co{experiences} are \co{gifts} of the \co{origin}, \co{gifts} of
  \co{presence} which need no active or volitional participation on our part.
  But also the \co{acts}, the \co{acts} of gratitude, benevolence, compassion,
  generosity, friendship, etc., which presuppose the \co{open} attitude at least
  toward the \co{actually} concerned object, situation or person, will result in
  a definite and \co{clear} \co{experience} of \co{participation} in something
  which is not limited to the boundaries of \co{mineness}.
  
  \pa In other situations, one may experience everything as basically void and
  meaningless, or, perhaps, simply as coldly indifferent.  \co{An experience}
  \co{that it is} may present one with a ridiculously meaningless \thi{this},
  dissociated from any context, an offense to any sense of purpose and meaning.
  It may be filled with disgust and nausea, so thoroughly studied by Sartre and
  other epigons of existentialism.  Such \co{experiences}, too, are \co{gifts}
  of \co{presence}, perhaps, reflecting the hidden currents of \co{invisible}
  \No, perhaps, signaling the stage of purgatory transition.  Taking such
  \co{experiences}, or such qualities of \co{experiences}, as the definite and
  last word about the world, one will look for reasons, explanations, excuses,
  eventually indeed \thi{projecting} them on the world and its essential nature.
  
  Experiences of this kind hurt to the bottom of one's soul which tries
  desperately to escape them.  Projections, trying to \thi{explain} and
  \thi{justify}, that is, bring release by re-establishing unity, by making me
  and the world friends again, paradoxically have the opposite effect -- they
  only \thi{objectify} the painful contents of the \co{experience}.  The
  \thi{objective} reasons may be innumerable -- it may be some \co{recognised}
  mischieves, some petty or important evils, eventually, the vicious, because
  indifferent, emptiness of the world in general.  If I would agree on calling
  them \wo{projections}, it is because they, genuine and \co{actually
    experienced} as they are, do not \co{express} the ultimate end and have to
  substitute for it \thi{the nature}, \thi{the world} or whatever they choose as
  that which \thi{actually and objectively is out there}.  Indeed, the
  \co{choice} of \No, \co{attached} to the distinctions of \co{mine} vs.
  \co{not-mine}, \co{subjective} vs.  \co{objective}, seems bound to the
  language of \wo{projections} (in addition to the common \thi{receptions},
  i.e., all forms of world's influence on \co{me}), because even in its luckier
  and more happy forms, it still can not avoid blurring these distinctions, it
  still can not distinguish \co{precisely} between world and \co{my world},
  between \co{my world} and \co{my life}.
  
  \pa Whether of the positive or negative character, such \co{experiences} of
  the indefinable \thi{totality}, the \co{experiences} which seem to announce
  the deepest essence of all things, accompany our life.  And they do it not
  only as occasional \co{actual experiences} but also as \co{qualities} of
  \co{our life} which happen to be also \co{qualities} of \co{this world}.  As
  long as they are not referred back to the \co{spiritual} \co{origin}, they are
  most \co{clearly experienced} only through the \hoa; first, as the indefinable
  \co{rest} of \co{acts} and situations and then, when we \co{reflect} over
  them, as qualities, as properties of something.
  
  \pa Conversely, the \sch\ will found \co{experiences} reflecting the attitude
  towards the \co{origin}, \yes\ will found \co{actual experiences} of fullness
  and goodness, while \No\ of meaningless, perhaps even evil, emptiness.
  
  As long as \co{I} remain centered on the sphere of \co{visibility}, \co{I} do
  not see that \co{experiences} of totality of the world and life are only
  \co{analogues} of \co{my} attitude because, as a matter of fact, \co{I} have
  not as yet assumed any attitude.  They are scattered pieces which come and go
  as they like.  \co{I} might have heard some \co{vague} \co{calls}, might have
  had some \co{vague intuitions}, might have had some particular
  \co{experiences}, but these have not been solidified.  As long as the \sch\ has
  not been made, such \co{experiences} will easily shift from one extreme to
  another.  \citeti{Those by the way side are they that hear; then cometh the
    devil, and taketh away the word out of their hearts, lest they should
    believe and be saved.  [\ldots] these have no root, which for a while
    believe, and in time of temptation fall away.}{Lk.}{VIII:12-13}
%
} % end \noo %not needed projections


\pa Both wheat and chaff, both \yes\ and \No\ tendencies \citeti{grow together
  until the harvest: and in the time of harvest I will say to the reapers,
  Gather ye together first the tares, and bind them in bundles to burn them: but
  gather the wheat into my barn.}{Mt.}{XIII:30; Lk.  III:17} Unlike any choice
based on some definite, particular \co{experiences}, on thoughts or feelings,
the \sch\ is \co{absolute}, not relative to any particular being or region of
Being.  Now, every choice suspends, at least to some extent, the relevance of
the subsequent feelings and thoughts; as an \co{act} of \co{my will} it simply
says: \wo{I choose this, no matter what might come}. But every choice related to
particular aspects of \co{experience}, every choice motivated by and based on
such particulars, will also continue being involved in them.  Eventually,
changes in their configuration can render sticking to the original choice the
matter of pure dogmatism, inflexibility, stubbornness. Every \co{actual} choice
is an \co{externalisation}, a projection of its \co{actual} decision into the
future. Every \co{act} and \co{action} is a projection -- as Heidegger would
say, a \wo{project} -- saying \wo{I want this thing to be {\em so}}.

The \co{reflective aspect} of the \sch\ is an \co{act} and thus, in a somewhat
similar way, it too \co{externalises} its content.  But this
\co{externalisation} does not result in any \co{dissociated object}, in any
particular goal, nor in any quality ascribed to something which, subsequently,
might turn out not to possess it.  It is the \co{act} of recognition of ultimate
\co{transcendence}, say, the ultimate \thi{objectivity}, which is not dependent
on the form or quality of any possible and \co{actual experiences}.  Through
this \co{act} \co{reflection} only admits this \co{presence} and, by the same
token, the insufficiency of its own modus.  It recovers the constant, underlying
all \co{experiences} \co{presence}, which it always knows, if only dimly,
through \co{self-awareness}, that is, \co{awareness} of the \co{transcendence}.
The truth of this \co{act}, its conformance to the \co{origin}, is thus lifted
\co{above} and lasting beyond and independently from \co{this world}. 
\citeti{Therefore whosoever heareth these sayings of mine, and doeth them, I
  will liken him unto a wise man, which built his house upon a rock: And the
  rain descended, and the floods came, and the winds blew, and beat upon that
  house; and it fell not: for it was founded upon a rock.}{Mt.}{VII:24-25/Lk.
  VI:47-48}

\pa The \sch\ of \yes\ does not require any feelings or impressions, does not
involve any specific thoughts or contents which might be projected.  It is a
response to the \co{command} of \co{nothingness} to become \co{self},
\co{reflection} of the \co{origin}.  It leaves all feelings and thoughts, all
\co{visible} \co{signs} and particulars \co{below}, centering around the
essentially \co{invisible}, the \co{origin} which now becomes also the axis of
one's world.  \citet{The person is like a wise fisherman who cast his net into
  the sea and drew it up from the sea full of little fish.  Among them the wise
  fisherman discovered a fine large fish.  He threw all the little fish back
  into the sea, and easily chose the large fish.\noo{Anyone here with two good
    ears had better listen!}}{Thomas}{8}

This \co{choice} is \co{absolute} also in the sense that it effects the final
division, the definite separation of \wo{wheat from chaff} which until now have
been mixed together. It takes the spell of \co{closedness} from \co{this world}
and \co{opens} it to the \co{inspiration} from \co{above}.  But this happens
only through the \co{absolute} renunciation of \co{this world} -- any
pretensions on its part to \co{absolute} validity and importance, to the
ultimate and all-determining \thi{objectivity}, are removed.  \noo{{Think not
  that I am come to send peace on earth: I came not to send peace, but a sword.
  For I am come to set a man at variance against his father, and the daughter
  against her mother, and the daughter in law against her mother in law.  And a
  man's foes shall be they of his own household.}{Mt.}{X:34-36}}

The \co{absolute} character of the \co{choice} amounts, in short, to the lack of
any particular contents which possibly could be projected.  Responding to the
\co{command} to renounce \co{oneself}, it is the \co{choice} of the attitude
which transforms the world: not in any of its temporal and \co{visible} aspects,
but in its \co{absolute foundation} -- it transforms the character of the whole
\co{existence}.  If somebody wants to call this \wo{projection}, it is of course
his choice, although such a choice amounts to much more than it believes to be
doing.


\subsection{Concrete founding}\label{sub:concrete}

\pa In Book I, we had only to do with ontological \co{founding:} there would be
no \co{experiences} without \co{experience}, there would be no \co{experience}
without the \co{chaos} of \co{distinctions}, and there would be no
\co{distinctions} without something to \co{distinguish}, or as it may be, to
\co{distinguish} from.
%
In Book II we saw, in the reverse order, its \thi{epistemological} counterpart:
from the \co{immediacy} of \co{reflection} to the \co{actually reflected
  experiences}, from \co{experiences} to the \co{experience} and to the
\co{awareness} of its \co{invisible} background, from \co{chaos} to the
underlying \co{non-actuality} of \co{invisibles} and, eventually, to the
\co{existential confrontation} with the \co{indistinct one}.\noo{In Book II we
  saw, in a reverse order, its epistemological counterpart: there would be no
  \co{reflection} without the \co{experience} to \co{reflect}, there would be no
  \co{experience} without the \co{awareness}, and there would be no \co{chaos}
  without the underlying \co{non-actuality} of \co{invisibles}, eventually,
  without the \co{existential confrontation} with the \co{indistinct One}.  }
The two hierarchies are, in fact, the same and differ only by the emphasis one
puts either on the element of \co{participation} or relation, being or knowing;
the difference of emphasis which is possible when viewing the same hierarchy
either, so to speak, \thi{bottom up}, from the assumed primacy of the
\co{dissociated} contents, or else \thi{top down}, in the order of
\co{founding}.

This \co{founding} is, as abstract ontology or epistemology in general, perhaps
curiously interesting but existentially, at best only helpful and, at worst,
irrelevant -- it happens and works according to its structure, no matter what we
do, and even if its understanding may reward \co{my} curiosity, it does not
really affect \co{me}.  We have several times observed that events at different
levels may happen relatively independently from those at other levels because,
in \co{concrete} terms, the ontological \co{founding} is not effective.  It
merely gives a general form and character to each level, which form and
character are conveyed to the character of the contents at the respective level.
But events of one level remain related to this level merely as \thi{matter} to
\thi{form}, and they remain relatively unrelated to the events of other levels.
Eventually, they lack anchoring in their \co{origin}, they lack the \co{concrete
  foundation}.  Thus, although the hierarchy does proceed from the \co{unity} of
the \co{one}, it is not \co{experienced} as such in \co{concrete} terms.  In
particular, \co{reflective acts} remain \co{dissociated} from their \co{origin}
which, in \co{actual} terms, means from each other.  The intimations of
\co{unity}, \co{clearly} known as they may be at the deeper layers of
\thi{knowing}, always slip out of the \co{precise} \co{reflective} grasp
and dissolve in \co{vagueness}.

If $X$ \co{founds} $Y$ then $Y$ \co{participates} in $X$: both these
\thi{relations} (which are one and the same) may also have, in addition to the
universal and ontological form, the \co{concrete} one.
% The \co{concrete participation} is not merely \co{founded} -- in the
% abstract, indifferent sense -- in the order of Being, but is \co{founded
%   concretely}.
\co{Concrete foundation} does not \co{found} \co{experience} in general, but a
particular way of \co{experiencing}, a \co{concrete experience} that the
particular things and \co{distinctions} of the lower levels originate from
those at the higher levels.  \co{Concreteness} is not, as the common confusion
and language usage suggest, \co{immediate precision}. The table in front of me,
the more \co{precisely} it is perceived and identified, becomes only the more
abstract, because, the more \co{dissociated}. \co{Concreteness} is the
experiential continuity between the contents of \co{actual experiences} and
their \co{foundation}, eventually, their \co{origin}.  \co{Concrete} is only
that which carries the \co{traces} of anchoring in the ultimate personal site,
and the lack of such \co{traces} amounts to abstractness, that is, indifference.
\noo{ Lower, smaller, more \co{precise} and \co{actual} things get embraced by
  the \co{unity} -- not the \co{totality} -- of the underlying \yes.  The result
  is a state -- not, however, of body or mind, not of consciousness or feelings,
  but a \co{spiritual} state of thorough \co{participation}. This state
  \co{manifests} itself, down to the lowest levels of \co{actuality} and
  \co{immediacy}, through a gradual separation of the elements which,
  \co{virtually}, still remain the same\ldots This manifestation, however, is
  not an \thi{objective announcement} encountered on the way to work. It is the
  same as discovering \co{traces} behind the most \co{actual signs}, and
  following these \co{traces} towards their \co{origin}.  }

Unlike the ontological \co{founding}, the \co{concrete} one is not something
that simply \thi{is that way}, that simply is granted by the hidden but
universal order of things which one must only discover and accept. Without \Yes,
without one's \co{love}, it actually is not\ldots I can find gaiety, joy, fun in
small things of \co{this world}, but unless this fun \co{participates
  concretely} in the higher mirth, and the mirth is surrounded by happiness of
the whole \co{soul} and by tranquility of the \co{spirit}, the fun can become
only an escape towards \co{more} fun. \citet{Fun I love but too much Fun
  is of all things the most loathsome. Mirth is better than Fun \& Happiness is
  better than Mirth -- I feel that a Man may be happy in This
  World.}{BlakeLet}{}

The \thi{happiness in This World}, however, as even the ascending levels in
Blake's description might suggest, is only a reflection of being \co{clearly}
anchored in the \co{other world}. Man is a borderline between what is \co{below}
and what is \co{above}, and \co{visible} is just another side of the
\co{invisible}.  Any attitude towards the one is, at the same time, an attitude
towards the other. The \co{spiritual} \yes\ to the \co{invisible nothingness},
accepting everything, \co{founds} also some \co{actual} attitude in the
\co{visible world}. Just like \citet{holiness is never the mere \la{numinosum},
  even at its highest level, but is something which is always in a perfect way
  permeated and saturated with rational, purposeful, personal and ethical
  elements,}{Heilige}{XV\kilde{p.131}} so \co{actuality} is not merely the site
of closed, \co{dissociated} \co{immanence}, but the eventual \co{sign} of the
\co{origin}, the meeting point of the \co{traces} of the \co{invisibles}, the
eventual place of \co{incarnation}.  When the \co{actuality} is \co{reflection}
of nothing less but the \co{origin}, when the \co{traces} reach all the way to
the \co{origin}, in short, when the \co{visible} and \co{invisible} spheres are
no longer \co{dissociated}, the just quoted words of Blake seem only to echo, in
the reversed order, those of Plotinus: \citet{The loveliness that is in the
  sense-realm is an index of the nobleness of the Intellectual sphere,
  displaying its power and its goodness alike: and all things are for ever
  linked.}{Plotinus}{IV:8:6.\noo{(We would, of course, substitute for
  \wo{sense-realm} and \wo{Intellectual sphere} our expressions. And then we
  would even stop distinguishing the lower and higher spheres, for the
  \thi{loveliness} is one and the same, \co{below} as \co{above}, even if it may
  have different forms of expression.)}}

The attitude towards the \co{visible world} which is an expression of the
\co{spiritual love} is \co{non-attachment}. This name, however, requires some
closer remarks.

\subsub{Non-attachment}\label{sub:nonattach}
\secQpar{3}{10.5}{Put not with God other gods, or thou wilt sit despised and 
forsaken.\\
  Thy Lord has decreed that ye shall not serve other than
  Him}{\footnotemark{}\vspace*{1ex}} 

\footnotetext{Koran. XVII:22-23}
\co{Thirsting} for eternity, we flirt with time, but the moods of silence are
never satisfied by anything \co{visible}.  The \co{thirst} is not for anything
particular, anything \co{visible}.  This does not mean that to quench it, one
has to deny all the \co{visible world}, that only death is the ultimate peace.
%, \refp{pa:death}, \refp{suic=no}.
This means only 
that \co{this world} itself is not enough, that it does not \thi{fill the soul},
that since it contains all and only answers, it never gives {\em the} answer
\ldots To quench the \co{thirst}? \citet{But how is this to be accomplished?

  Cut away everything.}{Plotinus}{V:3.17\label{ftnt:PlotQuot}}

%\subsubnonr{Renouncement}
\ad{Renouncement} Mystics and sages have always spoken about self-denial and denial of
\co{this world}.  In a sense, \co{grace}, living and lived \co{spiritual love},
the union of which mystics speak, lifted \co{above this world}, seems to be
exactly such a denial.
%\co{inspired} by the ultimate reality and importance of the \co{one},
%is exactly such a denial. 

However, the renunciation (not the denial) is only a possible part of
\co{spiritual} exercises leading to \yes\ and, as the means, should not be
confused with the goal.  The sages speak, at the same time, also about the need for
constant alertness, presence of mind, active attention to the \co{actual}
situation.  This constant vigilance may seem to contradict the supposed peace of
the union with God based on absolute self-denial.  There is, however, no
contradiction because, as a matter of fact, the \co{grace} is but the \co{second
  birth}, is \thi{re-birth} not only of soul but of flesh, is resurrection of
the body, that is, of \co{this world} as much as the \co{other one}.  The union
with God is also the union with the world.  The difference is that before,
\co{this world} was only ontologically grounded in the \co{other world} and thus
there was not a real, \co{concrete} unity of the two.  Resurrection is the
\co{spiritual} event which brings the two worlds together, which makes
\co{visible} not only a mere \co{actualisation}, but a true \co{manifestation}
of the \co{invisible}, making everything \wo{on earth, as it is in heaven}.
\citet{The strongest and deepest reality is where everything is included in the
  activity, the complete man without any reserve and the all-embracing God, the
  unitary self and the unlimited thou.}{IchDu}{III.\orig{Die st\"{a}rkeste und
    tiefste Wirklichkeit ist, wo alles ins Wirken eingeht, der ganze Mensch ohne
    R\"{u}ckhalt und der allumfassende Gott, das geeinte Ich und das
    schrankenlose Du.}\kilde{p.90}}

\pa The \thi{death to this world} means only that \co{visibility} loses its
absolute importance, that it is seen now in \co{non-attachment} \la{sub specie
  aeternitas}, with, as St.  Francois de Sales called it, {holy indifference}.
\co{I} remain \co{myself} as \co{I} have always been, but this \co{mineness} is
no longer the axis of the world.  Indeed, it is now seen and \co{experienced}
only as an accident of the \co{origin}, as only the \co{actual}, only one of its
possible \co{gifts}.  \citet{[A] man should so stand free, being quit of
  himself, that is, of his I, and Me, and Self, and Mine, and the like, that in
  all things, he should no more seek or regard himself, than if he did not
  exist, and should take as little account of himself as if he were not, and
  another had done all his works.  Likewise he should count all the creatures
  for nothing.}{TheolGerm}{XV}

This is the \co{reflective} attitude conditioning \co{spiritual love}.  But all
this \thi{counting for nothing} expresses only \co{non-attachment} to the
\co{visible} things, the acceptance that they should not make \co{me}
unconditionally dependent on them.  It is not denial of their existence, nor of
their possible relevance; it is only denial of their \co{absolute} power.
\citet{Fear not the flesh nor love it. If you fear it, it will gain mastery over
  you. If you love it, it will swallow and paralyze you.}{Philip}{ Describing
  the man who stepped beyond all unnecessary worries and distinctions,
  \citeauthor*{Ash} says: \wo{Because he is freed,\lin He neither craves nor
    disdains\lin The things of this world.} XVII:17} One lives among and
\co{acts} on things of \co{this world}, but one's life is not exhausted by such
\co{actions}, one tries to attain \co{visible} goals but one does not crave
them, one enjoys them but does not worship them.  And if one fails, if one does
not attain them, if one does not enjoy them, then \ldots it does not matter.
One's life is never exhausted by them, it always carries the \co{rest}, the
inexhaustible potential.  This \co{rest} contains \co{thankfulness} even for
one's failures.  For all these actions, attainments and enjoyments are
themselves only \co{visible} things of only relative value.  One can try again
or let it go -- one does not know what one will do, this will turn out in the
proper time and one may be vastly surprised. Everything is a \co{gift} and one
can not have anything which one is not prepared to lose.\ftnt{A closely related
  thought of Schopenhauer dismisses the possibility of any complains: justice is
  equally given to all, both happy and unhappy ones.}

%\subsubnonr{Idols}
\ad{Idols} \citeti{Do not strive to seek after the true, only cease to cherish
  opinions,}{Anonymous Zen master}{} after all, \citeti{[h]uman opinions are
  children's toys.}{Heraclitus}{DK 22B70. A related advice of Heraclitus is:
  \wo{Let us not conjecture randomly about the most important things.} DK 22B47}
\co{Idols} are not \co{visible} things, but \co{visible} things considered as
all important, which eventually means raised to the level of absolutes.
\co{Idolatry}, \thi{worshiping images} is exactly that -- to take as
\co{absolutely} important something that is not.\ftnt{Of course, we will not go
  as far as the iconoclasts\noo{ of the IX-th century} did in considering any
  representations as idolatry. The question, as always, is about the attitude
  towards things -- enjoyment of artistic expressions, whether religious or not,
  is very different from \co{idolatry}.}  \thi{Cherishing opinions} may be so
much, and may be nourished by so many mechanisms. (\thi{Being entitled}, often
\thi{entitled to one's own opinion}, and even \thi{entitled to be heard} are
quite common forms.) At the bottom it is to make an \co{idol} of \co{mineness},
is to think that something \co{visible} is worth cherishing an opinion about,
and that one is entitled to cherishing such an opinion.  Cherishing an
opinion, one cherishes \co{oneself}.

Again, all this does not mean that one can not mean anything about anything.
One not only can -- one is bound to.  One will have opinions about things, will
participate in arrangements of things, in research, in work, in all kinds of
activities of \co{this world}. Moreover, one will accept all these things as
one's part, as relative, yet \co{absolutely} real, though not as \co{absolute}
reality.  But in the moment one starts cherishing them, one becomes
\co{attached}, that is, starts worshiping \co{idols}.

\pa \co{Idols} are what \thi{possesses} man, \thi{being possessed}
consisting in making the relative into the \thi{absolute}.
% This is by its very nature unconscious which justifies the phrase \wo{being
% possessed}.
Even if one has all the good reasons for adhering unreservedly
to a given opinion, one's being possessed by it consists in the {\em
  unreservedness}, in the perhaps unintended but, therefore, the stronger and
more effective turning it into an \thi{absolute}.

Rationalism, defined as the acceptance of any \co{actual} statement or position
with the recognition of its limited validity (and in the best case, also of its
actual limits), is the opposite of being possessed.  In this respect, it
coincides with innocence which is just that -- being pure, that is, not being
possessed.  But every \thi{-ism} indicates a possibility of being possessed,
\thi{absolutisation} of some relative sphere or expression.  One's intense and
deeply convinced materialism or idealism, atheism or theism, liberalism or
dogmatism, Protestantism or Catholicism, intellectualism, existentialism or what
not, testifies against one's innocence.  One can become possessed even by
rationalism itself which, \co{unfounded} and \co{dissociated} in its \co{proud}
complacency, tends towards agnosticism, relativism, scepticism or just dry
rigidity.


%\subsubnonr{Obedience} %{renouncing idols = renouncing self}
\ad{Obedience} Giving up \co{idols} does not mean merely to replace the
\thi{object} of such a worship, to exchange the relative for the absolute, but
still retain one's attitude.  \thi{What} is being worshiped determines the
\co{visible} aspects of the attitude.  Worship of patriotism {\em is} different
from worship of communism, worship of scientism {\em is} different from worship
of money.  Yet, they are the same in so far as \co{idolatry} is concerned.  To
cease worshiping \co{idols} is to recognise their thoroughly relative character,
relative not only to each other and to particular circumstances, but also to the
\co{foundation} from which they emerge; eventually, it is simply to recognise
the \co{absolute} character of the \co{one} which is unconditionally \co{above}
the world of \co{distinctions}. Giving up \co{idols} is, at bottom, giving up
\co{oneself}, for reduction of \co{self} to \co{myself}, the sphere of
\co{visibility}, is \equi\ with \co{attachment}. And \co{attachment} is the
pattern of all \co{idolatry}. All \co{idols} are things which, being \co{below
  my self}, are not noble enough to possess \co{me}.

\citet{Behold, in such a man must all thought of Self, all self-seeking,
self-will, and what cometh thereof, be utterly lost and surrendered and
given over to God, except in so far as they are necessary to make up a
person.}{TheolGerm}{XLIII} The crucial \co{aspect} of \co{non-attachment} lies
not in any grandiose opening to the \co{above} and ascetic self-denial, but in
the small reservation \wo{except in so far\ldots}. One does not deny
\co{visibility}, one only renounces the image of its absoluteness, for \thi{in
  so far} as it is necessary \co{aspect} of life, its denial is not very
different from \co{pride}. 

\pa Another expression for this is \wo{\co{obedience}} which, however, does not
mean submission to any specific agent, even less to any specific commands. It is
\co{obedience} to \co{nothing}, that is, to everything.  \co{Obedience} is just
another way of saying \co{that I am not the master} and, on the other hand, that
I am \co{thankful}. These two -- not any servile submission, lack of autonomy,
sense of inferiority -- exhaust the sense of \co{spiritual obedience}. It is not
any conflict of the wills in which one must yield to the other, for only we,
limited human subjects, have any will.  If it is a conflict, then only of \co{my
  will} with \co{nothingness} which \citeti{is poor, naked and empty as though
  it were not; it has not, wills not, wants not, works not, gets
  not.}{Eckhart}{} \co{Obedience}, or as one often used to say, \wo{obedience to
  the Divine Will}, means only that one ceases to insist on one's will, that one
ceases to insist on \co{oneself}.  \wo{Do not strive to seek after the true,
  only cease to cherish opinions.}  \co{Ego} is the site of \co{idols} which
disturb more than one is ever able to realize, opinions which are one's own (and
true!), images which drive one's will in all possible, often pleasant,
directions. \co{Obedience} means only (only?) that one lets them go, one may
still use them, but one ceases worshiping them.
  
\co{I am not the master} and \co{I} am \co{obedient} -- eventually, this means
that \co{I} am \co{nothing}; not having any master and being \co{obedient} to
\co{nothing}, \co{nothing} becomes \co{my} whole treasure, \co{nothing} becomes
mine.  \co{Nothing} is mine, not only things but also the \sch\ is not \co{mine}
-- it happens \co{above me}; even \co{thirst} is not \co{mine} -- it was only
given to \co{me} as a \co{gift} of remembrance.
%(Th.Germ. X [end of 1st paragr.]) + Aquinas (choice is not mine either)
\co{Nothing} is all you have, is \co{your} only treasure, and \citeti{where
  your treasure is, there  will your heart be also.}{Mt.}{VI:21; Lk. XII:34}

\pa\label{pa:filling} We react against \thi{obedience} which berefts us of our
\thi{autonomy}, just like we think that emptiness is when nobody speaks.  But
emptiness is when nobody listens, and {\em the whole} \co{obedience} is to
listen to the silence, not to any specific orders. It is to accept that \co{I
  am not the master}, not finding another one nor even barely looking for
one. \co{Nothing} 
is the master and \co{obedience} is but \co{openness} to its \co{gifts}, free
\co{thankfulness} lifted \co{above} all particular gifts, as opposed to the free
rejection which loses all its autonomy to the degree it insists on it.

Eventually, \co{nothingness} listens -- it has heard the scream \wo{Release me!}
But then we do not have to speak any more. Silence is the ultimate
communication, \co{communion} which also teaches us that genuine response is
more than any \co{visible sign}, any \co{actual} word.  Insistence on the {\em
  right} to say one's meaning and to be heard is \co{attachment} to images,
\co{disobedience}. Silence is a moment of \co{eternity} in time -- nowhere and
never happens more than in such a moment. This \wo{mute country}$^{{\rm
    II}:\ref{cit:mute}}$ is our constant companion, and it scares only the man
bound by worship of images.
%Who understands that has read enough.
%We are created to listen, not to speak, and to listen to silence
%%%%%
%% Obedience is freedom, only to nothing...
%%%%%

%\subsubnonr{Spiritual unity}
\ad{Spiritual unity} \Yes\ says that, in the ultimate matters, the \co{visible}
impossibility does not count. In this sense it renounces \co{this world},
renounces its pretensions to \co{absolute} validity.  Renouncing all
\co{visibility} leaves only \co{nothingness}, the place where everything can
appear anew. But it is no longer a place divided into \co{this world} and
\co{another world}. \co{Incarnated spirit} is the \co{unity} of -- not only a
borderline between -- the \co{visible} and \co{invisible}.
% Renouncing all of \co{this world}, clarifies the other, \co{above} is only
% \co{nothingness}, and so 
In what does this \co{unity} consists, \co{concretely}? \co{Concretely}, in
\co{nothing} particular.  \citet{Damn the flesh that depends on the soul. Damn
  the soul that depends on the flesh.}{Thomas}{112. \citef{Hate and lust for
    things of nature have their roots in man's lower nature. Let him not fall
    under their power: they are the two enemies in his path.}{Bhagavad}{III:34}}
The \co{spiritual} unity of {soul} and {body}, of {higher} and {lower}, modifies
but does not change man's existential situation.  Man is a borderline between
what is \co{below} and what is \co{above} -- \co{visible} is just the other side
of \co{invisible}. The \co{unity} consists in being \co{concretely founded} in
the \co{origin} and being \co{open} to the \co{aspect} of \co{gift} in every
\co{actual} situation, in \co{experiencing} flesh as the \co{gift} of
\co{spirit} and \co{spirit} as the \co{concrete foundation} of flesh. In 
\co{unity}, this \wo{and} does not join two different \co{experiences} but
expresses their \equin.

\newp
\pa Consequently, the \co{unity} consists also in the fact that \co{spirit} is
directed towards \co{nothing}. This means simply directedness towards what is
\co{visible}, for \citeti{there is nothing better, than that a man should
  rejoice in his own works; for that is his portion: for who shall bring him to
  see what shall be after him?}{Eccl.}{III:22. Although we will not confuse such
  remarks with the Stoic endurance -- which is a matter of resignation and
  surrender to the world overgoing one's powers, not of \co{thankfulness} for
  its \co{gift} -- we may nevertheless notice affinity of expression: \citef{We
    must make the best use that we can of the things which are in our power, and
    use the rest according to their nature. What is their nature then? As God
    may please.}{Epictet}{I.1}} Or, we might add, to {\em see} what is
\co{above} him? This directedness towards the world is no longer {\em
  exclusive}, as it was with the \co{attachment} and \co{active} evil. On the
contrary, it reflects now only the recognition \co{that I am not the master} and
concentration on \co{visible} things which \co{I} can influence and even
control. 

For, \co{actually}, no direct, intentional attitude towards the \co{origin} is
needed, if at all possible.  Such an attitude is already an indication of a
mistake -- the \co{one} can not be made correlate of our intentions or
\co{acts}, unless it is reduced to some \co{objective} from.
% \co{analogues} can be \co{experienced} but they are not -- or in any case,
% should not be turned into -- independent entities.
Intentional \co{acts} find place only in the sphere of \co{visible} contents,
\co{distinctions} which are sharp enough to be turned into the \co{objects} of
\co{action} or \co{reflection}.  An \co{act} aiming at and consciously intending
\thi{goodness} is not good.  It need not be evil, nor wicked, nor malicious; it
may be thoroughly well-intended but it is not good. The intention of \thi{being
  good} pollutes every \co{act} unless it is withdrawn from the sphere of
\co{actuality}, unless it has become an \co{invisible} but \co{present rest}.
This applies to all higher things which one might possibly \co{posit} as one's
intentions, even goals.  An \co{act} whose main goal is {\em to be}
compassionate, is not compassionate, just like an \co{act} intending only to
prove and show one's freedom is not free.  A person focused on making always
\thi{right} decisions may, indeed, happen to make them \thi{right}. But he will
spend time in constant worry about doing just that. And since \thi{right} is
entirely \co{vague} category, one will never rest. A person focused on his
salvation may happen to do a lot of good things, but his focus will always
disturb him: \wo{Has it already happened or not yet?  {\em How} can I be
  certain?}

The higher, \co{invisible} things are not any intentions, they do not in any way
enter into \co{my} considerations. Salvation, being good, etc. emerges, as it
were, only as a side-effect of \co{acts} which themselves are occupied only with
their \co{actual} \co{objects} and \co{visible} relations. But having thus
withdrawn from the sphere of \co{visible} intentions, they do not disappear as
\co{attachment} to \co{visibility} would claim. They remain \co{present}
although -- and because -- they are no longer sought among \co{actual} givens,
because they are no longer conflated with their \co{visible signs}. 

% A person focused on morality may, indeed, happen to perform a lot of good
% \co{acts} and \co{actions}, but his focus will always ask in the background:
% \wo{Is it morally perfect?}, and if not, then \wo{Is it moral enough?}

% The reality of \sch, the \co{incarnated love}, is witnessed by its
% \co{manifestations} in our \co{experiences} of \co{visible world} and in our
% dealings with it.  \co{Non-attachment} comprises the \co{qualities} which are,
% so to say, on the edge of the sphere of \co{visibility}.  They reside in the
% \co{rest} of \co{acts}, as \co{inspirations}, \co{motivations}, \co{vague} hopes
% and \wo{desires to become just and be like God}...  But as everything else, they
% may be \co{posited} as correlates of \co{my} intentions, goals of \co{my acts},
% turned into kind of \co{objects}.  Such a reduction amounts simply to losing
% them.

\pa\label{pa:forget} \co{Spirit}, the objectless relation to \co{nothingness},
is purity and poorness. It does not aim at the spiritual, does not seek it and,
hence, does not pollute it with the \co{visible} images.  This is the only way
of its \co{concrete presence}.  \citeti{Blessed are the poor in
  spirit.}{Mt.}{V:3; Lk. VI:20} Walking the spiritual paths may be an
expression of a genuine spiritual \co{thirst} -- but this only means, the
absence of \co{spirit}.  \citeti{Hold up my goings in thy paths, that my
  footsteps slip not.}{Ps.}{XVII:5} Any strife, any search, whether spiritual or
not, expresses only some lack. The more spiritual such a search is, the more it circles
around the \co{vague} and indefinable \co{that} and the less it is satisfied
with any \thi{what}.  But spiritual search is not \co{spirit} because it cannot
avoid replacing \co{spirit} by some \co{visible}, if only \co{vaguely}
imaginable, intention.  \co{Spirit} either is, either is thoroughly,
\co{concretely} and \co{absolutely present}, or it is not at all, is only some
unidentifiable and ever missing \co{rest}, which one may desperately try to
replace by some \co{visible} substitutes or, as the case may be, by the
insatiability of \co{more} which never manages to quench the \co{thirst}.

\co{Spirit} is first of all \co{humility} towards the {spiritual}, towards
the \co{invisible origin}.
%used later\citt{Know what is in front of your face, and what is hidden from you
%will be disclosed to you.  For there is nothing hidden that will not
%be revealed.}{The Gospel of Thomas, 5}
By this very token, it directs its attention towards \co{this world}, because
\co{this world} is no longer \co{dissociated} from \co{another}, of which \co{I
  am not the master}.  If we view the ontological \co{founding} from Book I as
the descent of which mystics and philosophers of the neo-Platonic orientation
spoke, while its lived and understood \co{reflection} in the levels of Being
from Book II as the corresponding ascent, then the \co{incarnated love} marks
the final and definite return.\ftnt{This seems to be the natural way to
  interpret much of the mystical ascent though, of course, there are other
  possibilities.  Hermes Trismegistus is probably one of the clearest examples
  emphasizing this element of return -- not to some sphere \co{above} but to
  \co{this world}: \wo{as below so above} but also \wo{as above so below}.}  It
does not end in a momentaneous illumination, in an ecstatic contemplation, in
any \co{actual experience} of mystical union with a constant wish for its
repetition.  It does not live in \co{another world}, but it does not have to
descent into \co{this world} either -- there is only one world, and
\co{actuality} becomes the scene of constant, \co{concrete presence}.
%%% used earlier  \citt{there is nothing better, than that a
%%man should rejoice in his own works; for that is his portion.}{Eccl. 
%% III:22 }
The \co{origin} is no longer remote and separate -- \co{spiritual love} is
nothing else than the attitude towards the \co{visible world}. \co{This world}
is not {\em only} an imperfect \co{sign} of the \co{other world} -- it is {\em the only}
\co{sign}, the only form of \co{invisible presence}.

\co{Spirit}'s directedness towards \co{nothing} means that it \co{rests} with
the mere \co{that}. As there is nothing more to do about
\co{nothingness} than saying \co{that} it is, \co{spirit} is the \co{restful}
return to \co{this world}. To \co{rest} is to accept the \co{invisible rest} --
to give up all the attempts at making it \co{visible} -- and in this sense, to
\co{forget} it.
%\thesis{
\co{Spirit} is \co{forgetfulness} of the spiritual.\ftnt{You may notice that the
  verb \wo{\co{rest}} here (and the noun \wo{\co{rest}} as \thi{tranquility})
  is, obviously, something different from the noun \wo{\co{rest}} as
  \thi{reminder} which we have been using earlier. The homonymy, however,
  serves us perfectly because the equivocation is thoroughly intentional. To
  \co{rest} is to admit, to allow for, to accept the \co{rest}.}
  %}

%\subsubnonr{Forgetful remembrance}
\ad{Forgetful remembrance} Thus, \co{spirit} is the renouncement of \co{this
world} and \co{forgetfulness} of the spiritual. Indeed, \co{nothing} is left.
As soon as something more \co{precise} gets involved, a distinct thought, a
specific feeling, the \co{spirit} seems to evaporate, to lose \co{actuality}
giving place to the flesh -- perhaps, to \co{myself}, perhaps, to \co{my}
\co{ego} and {body}.  But \co{I} live only in the world of \co{distinctions}
and this withdrawal is \co{spirit}'s only true \co{presence} -- it
\co{incarnates} only when the attempts to \co{actualise} it
 have ceased.  Being \co{invisible}, it can never become \co{actual},
but it can be \co{present} around and \co{above actuality} which means, 
in the very midst of it.  If you look to the left, you won't find it, if you
look to the right, you won't find it, if you look forward or backward, in past
or in future, you won't find it.  Because, when you look for it, you have
already found it, you only have to stop looking. It \citet{\noo{[The kingdom]}will not
come by watching for it.  It will not be said, 'Look, here!'  or 'Look, there!'
Rather, the Father's kingdom is spread out upon the earth, and people don't see
it.}{Thomas}{113} But \thi{to stop looking for it} is as difficult
as it sounds easy.

There is a great difference, which may appear as a paradox, between
\co{forgetfulness} and forgetfulness or, perhaps, between \co{forgetfulness} and
denial. \co{Forgetfulness} of the spiritual is the deepest remembrance of
\co{nothingness} -- remembrance, however, not in the form of a constant,
\co{actual} remembering, of incessant focus on the desired, even if impossible,
\co{actuality} of the spiritual.\ftnt{This is the source of bad conscience, not
  yet in any moral sense, but in the spiritual sense which, transcending the
  notion of personal guilt, inflicts only the deeper sense of corruption; the
  bad conscience of Luther or Faust or Mann, so characteristic for the Germanic
  mind and almost unknown to the English ratio-empiricism or French enlightened
  aestheticism.} It is remembrance which, for the first, remembers only
\co{nothing}, only \co{that} it is, but does not worry constantly about
\thi{what} it is; and, for the second, remembrance which itself is not
\co{actual} but thoroughly \co{invisible}, which does not enter the sphere of
\co{actual} considerations and intentions and does not try to bring the
\co{invisible rest} into explicit \co{actuality} of \herenow.  It is
\co{forgetfulness} as far as the \co{actual} occupations of the \co{subject} are
concerned, for these deal only with \co{visible} things. But as far as \co{my}
being is concerned, it is the remembrance which \co{I} have become, the
\co{self} which is no longer overshadowed by \co{my self}, not to mention, by
\co{myself}. \co{Actual forgetfulness} is the \co{eternal} remembrance.

\pa By its very nakedness and \co{nothingness}, \co{spirit} grants
\co{actuality} all the validity it possesses as the only place of our \co{acts}
and works.  But to find this place, one has to lose it first.  \co{I} can not
have anything which \co{I} have not already lost.  Ibsen says, \citeti{Only the
  lost is eternally owned,}{\btit{Brand}}{IV:last scene.\kilde{Evig eies kun det
    tapte} \citefi{Whoever abandons things as they are accidental possesses them
    as they are pure being and eternal}{Eckhart}{\btit{German Sermons} Si.L:10;
    Ac.I:4. in \citeauthor*{Eckhart}{ 16b;29.\kilde{p.275,289}}} } but one might
sense here some literal and resentful meaning of loss.  Bitterness, \co{closing}
one's world in the ever narrower circle of disappointment, is a frequent
companion of \co{attachment} unable to live the \co{actual} loss.  But
\thi{having lost} precedes any \co{actual} loss and amounts rather to suspending
its validity without, however, negating it completely. \thi{Having lost} is more
like a joy over a minute thing which as if suspends the validity of the whole
world. Its focal point is the \co{actual} joyful thing but it does not close the
\co{experience} within this \herenow. It rather opens it up -- not for all the
things in the vicinity, not for all the \co{visible} things, but for the
\co{rest}, the indefinite yet \co{clear} joy, inflow of its rays.  Losing some
dear thing or person can likewise effect such an opening by pointing in the
direction \co{above} where some \co{rest} always survives any \co{actual} loss.
It is not so much the lost thing which becomes eternalised, it is only the
\co{eternal} element which announces its \co{presence} underneath the lost
things. \citeti{When the heart weeps for what it has lost, the spirit laughs for
  what it has found.}{Anonymous Sufi aphorism}{\citaft{Huxley}{p.106}} Unless
the loss is complete, before the whole world has been lost, every \co{actual}
loss is like a painful indication of the possible ultimate loss, a reminder of
the possibility of losing \co{this} whole \co{world}.\noo{If \co{eternity}
  enters through the loss of the \co{visible} so, once it arrives, it permeates
  all \co{actuality}...}  And if one has not lost \co{this} whole \co{world}, if
one stays \co{attached} to it, one is not able to fully and deeply enjoy any
particular thing, whose fragility keeps only reminding about the impermanence of
the world.

To have some particular thing is to have already lost it, to agree that it is
not \co{mine}, that \co{I} do not control it. Only then can \co{I} truly have
it.\noo{We have here obviously abandoned Marcel's distinction between
  \thi{being} and \thi{having}.}  Having already lost it is simply to admit its
fragility, which only makes the appreciation greater.  Expectation of its
possible loss may certainly cause some worry.  If, and when, one \co{actually}
loses it, this may naturally cause sorrow and pain.  \co{Spirit} does not
abolish such negative moods, thoughts, feelings. On the contrary, it \co{opens}
one for their thorough and deep experience. This happens because such worries
and sorrows are as real as they are relative, and although they affect one,
they do not affect the tranquil \co{unity} of the \co{spirit}.

We are not saying that \co{spiritual unity} is a tranquilizer, a placebo against
finite failures and \co{actual} dissatisfactions. \co{Spirit}, directed towards
\co{nothingness}, is fully directed towards such finite and \co{visible} things
and events, it does not supersede them. It only makes one worry for the things
of \co{this world} without worrying about the ultimate things, without looking
for the \co{absolute} in the \co{visible}, that is, without establishing
\co{idols}.
%
It makes one care for all finite things because, having \co{founded} one's being
in the only \co{absolute} of \co{nothingness}, it allows one to recognise their
\co{eternal foundation} and, at the same time, to accept their fragility. A
thing which one could not {\em possibly} lose (if it existed) would be, or in
any case would turn with time \ldots worthless. Eternal life, imagined vulgarly
as temporal infinity, would be, if not unbearable, then eventually boring. And
boredom would not come from the fact that there were no new things to encounter.
It would come exactly from the fact that there would be nothing else to
encounter than mere novelties.  Death is the complete return to
\co{indistinctness}. And it is the knowledge of this ultimate \co{nothingness},
of the fragility of all \co{visibility}, which makes life so valuable. However,
life occupied {\em exclusively} with the maintenance of itself, forgetting
\co{that}, i.e., that there is something more worthy than it, perhaps even
something for which it could be sacrificed, becomes a mere social, even a mere
biological phenomenon -- deindividualised, impersonal, eventually, contentless.
Although it is hardly possible to live fully such an idea, it is quite possible
to \co{actually} believe it.

\pa Eventually, only \co{visible} things of \co{this world} are given to us, so
that we can \citeti{have dominion over the fish of the sea, and over the fowl of
  the air, and over the cattle, and over all the earth, and over every creeping
  thing that creepeth upon the earth.}{Gen.}{I:26} But reducing Being, and what
then follows, one's own being, to such things only, \co{dissociating} them from
their \co{origin} (as is typically the case in the attempts to see,
\co{re-cognise} and admit the value which they do possess), turns them into dead
and empty \co{objects}. It makes us forget so that we do not remember.
\co{Forgetfulness}, too, directs one towards \co{visible} things but not as the
only and \co{absolute} form of Being.  \co{Forgetfulness} makes one remember
that all \co{visible} things are only \co{signs}, but also that they are {\em
  the only}, true and real, \co{signs} of the \co{invisible}.  \citet{Know what
  is in front of your face, and what is hidden from you will be disclosed to
  you.}{Thomas}{5.  {Bluntly put, \co{spirit} is the true life of the flesh, but
    it lives {\em only} in and through the flesh. As we learn from the long
    tradition, one can easily construct contradictions between the two, but
    easiness lends any credibility as seldom as contradictions witness to
    health. Say, a \thi{contradiction} between the defense of the lower,
    sensible world in \citeauthor*{Plotinus}{ II:9.}\btit{Against the Gnostics}
    and, on the other hand, passages like V:3.17, footnote~\ref{ftnt:PlotQuot}
    or, for instance: \citefib{The soul in its nature loves God and longs to be
      at one with Him in the noble love of a daughter for a noble father; but
      coming to human birth and lured by the courtships of this sphere, she
      takes up with another love, a mortal, leaves her father and falls. [...]
      This is the life of gods and of the godlike and blessed among men,
      liberation from the alien that besets us here, a life taking no pleasure
      in the things of earth [...]\noo{, the passing of solitary to
        solitary.}}{Plotinus}{VI:9.9} Such \thi{contradictions} disappear once
    we observe that what is wrong with the lower, sensible, material,
    \co{visible} world is not its being as such but our \co{attachment} to it.
    The calls to renunciation of \co{this world} do not try to negate its
    reality and even beauty, but only our \co{idolatrous} attitude towards it.}}

%\pa - perhaps, this should go elsewhere [love?]

%Trust, not faith, and not unfounded trust. Trust is unfounded when
% I choose to trust -- the very decision, the very consciousness of
% having decided makes me know the possibility of betrayal. This is,
% perhaps, the only form of trust we know from \co{this visible world}.
% Trust, which is commitment and unconditional reliance, is possible
% only in the face of the \co{absolute}.\ftnt{Again, let us remind that
% the Greek \gre{pistis}, translated usually as \thi{faith}, is often
% better rendered as \thi{trustfulness}...}

%\subsubnonr{Losing and winning}
\ad{Losing and winning} \label{pa:hardwork} \co{Non-attachment} is the consent to
having lost \co{this world}, it is \yes\ which, being directed to
\co{nothingness}, is unconditional. And just like \No, motivated by the
\co{attachment} to \co{this world}, turns it eventually into nothingness, so the
apparent renunciation of \co{this world} turns out to be  \yes\ to all the
\co{visible} things.
%
% \citt{For whosoever will save his life shall lose it: but whosoever will lose
%   his life for my sake, the same shall save it.  For what is a man advantaged,
%   if he gain the whole world, and lose himself, or be cast away?}{Mt. XVI:25-26;
%   Mk.  XVIII:35-36; Lk. IX:24-25}
%This is the ressurection of the flesh.
\co{Non-attachment} is the \co{concrete presence} in the midst of \co{this
  world}.  But it is not the goal to make one so concerned with \co{this world}
-- it is only the effect. To achieve it, one has to renounce it, for \citet{the
  Supreme for which the soul hungers though unable to tell why such a being
  should stir its longing-reason, however, urging that This at last is the
  Authentic Term because the Nature best and most to be loved may be found there
  only where there is no least touch of Form.}{Plotinus}{VI:7.33} Only giving up
all the forms, life acquires the ultimate and \co{concrete foundation} because,
in the end of the day, one truly possesses only what one has given away, whether
things or oneself. And thus, \citeti{whoever wants to save his life will lose
  it, but whoever loses his life for me will save it.}{Lk.}{IX:24, XVII:33; Mt.
  X:39; John XII:25}

We can care for things because we value them to the degree which we do not even
realize, we can do good works because our boss, our spouse, other people expect
that or because our hidden inhibitions prevent us from doing otherwise. All this
has nothing to do with the \co{spirit}, even if the \co{externally visible}
results may be exactly the same. For \co{external} results, as we know, do not
count for us as much, and do not give us as deep a satisfaction, as we often
would like to believe. They may be necessary but are never sufficient.  But we
can also care for things and do good works because \co{spirit} does not
distinguish that from the ultimate \co{humility} -- in fact, because it does not
leave us anything else to do.  Care for finite things, work carried out with
conscientiousness, respect and \co{humility}, do keep heaven and earth together.
Work -- hard, tiring, exhaustive work -- which has engaged fully body and mind,
makes \co{me forget}. \co{Forgetfulness} finds the \co{expression} as respect
for \thi{the order of things}. At the same time, it is the deepest form of
remembrance, \co{forgetfulness} of \co{spirit}.  Sloth is a cardinal sin because
there is no such thing as \thi{pure spirit}, disembodied, non-incarnated, not
involved into \co{actual} goals and works. There is only living, \co{concrete
  spirit}, which unfolds itself in the body, in \co{this world}.  Dedication and
thoroughness, hard work and conscientiousness are not, in any case, not
necessarily, signs of \co{attachment}. More often than not, they are signs of
\co{spirit}. And as all \co{acts} which are \co{expressions} of \co{spirit},
they also strengthen it or, as is often the case, prepare for it.
\citeti{Commit thy works unto the Lord, and thy thoughts shall be
  established.}{Prov.}{XVI:3}

To be sure, \co{concrete presence} of \co{spirit} is nothing common. Perhaps, it
is even very rare, though it seems that it is less rare than we want to admit or
are able to realize.  But the fact that no statistical investigation may ever
give a slightest indication of it means only that it is the most real, that is,
the most individual and personal possibility -- unrepeatable, not because of
varying \co{visible} conditions but because thoroughly \co{concrete}; and always
the same, because consummated in the same existential situation, in the face of
\co{one nothingness}.
%Statistics can never say anything about anything individual.


\subsub{Inversions}\label{pa:inversion}
Before giving some examples of \co{concrete founding}, let us observe some
general mechanism making occasionally \co{manifestations} of higher things
appear in a strangely inverted manner.

%Let's give a few more concrete examples of the \co{aspects} of
%\co{non-attachment}.
\pa\label{pa:allaspects} \co{Love} has unlimited number and forms of
\co{incarnation} which are always purely personal. \co{Love} is a \co{virtual}
\nexus\ which opens unlimited field of possible \co{manifestations}.
\citeti{Temperance is love surrendering itself wholly to Him who is its object;
  courage is love bearing all things gladly for the sake of Him who is its
  object; justice is love serving only Him who is its object, and therefore
  rightly ruling; prudence is love making wise distinctions between what hinders
  and what helps itself.}{St.~Augustine}{\citaft{Huxley}{ V\kilde{ p.92}}}
%
Furthermore, every \co{concrete manifestation} of \co{love}, although it may
seem to express only one or few of its \co{aspects}, is always a full expression
of all of them. For \nexus\ can not be divided and \co{present} only partially;
only its \co{aspects} can possibly exclude each other from the \co{actuality}
which they fill, leaving no place for others.  Say, modesty may seem a natural
example of \co{humility}, but it involves equally \co{thankfulness} and
\co{openness}. Modesty is not a servile admission of one's inferiority. It is a
humble gratefulness which does not argue about the qualities and conditions of
the \co{gift} -- one's own achievements and labor being, too, \citet{nothing
  more than the finding and collecting of God's gifts.}{Luther}{45;p.327
  \citaft{LutherTheol}{ II:10.1\kilde{p.109}}} And it is grateful for everything
it obtains, for a person who is now modest and now not, is simply not
modest but only behaves modestly in some situations.

\pa Now, all the \co{aspects} of \co{love} are predicated adequately about the
\co{spiritual} attitude, and only analogically about anything within the
\co{visible world}. Together with the \co{unity} of all the \co{aspects} in
every expression of \co{spiritual love}, this can give raise to apparent
\co{inversions}. Roughly, \co{inversion} is a \co{manifestation} through
something which appears as the opposite of the manifested. This happens
especially when judged by \No\ which does not recognise anything beyond the
\co{visible} categories of merely human, or even only \co{egotic}, level.
% It is only appearance, usually a misreading of the situation, which \co{inverts}
% its meaning.

\co{Inversions} originate in the most general schema of \co{nothingness} being
(the \co{origin} of) everything, which is also reflected in the fact that the
apparent renunciation of \co{myself} and \co{this world} in \co{non-attachment}
amounts truly to the genuine return to the world.  \co{Thirst} is a \co{sign} of
genuine \co{presence}, \co{spirit} is \co{forgetfulness} of the
spiritual.  %\refp{pa:forget}.
The apparent lack may be the true \co{manifestation}, albeit
in an \co{inverted} form.  On the other hand, the total absence of \co{spirit}
is, too, its total forgetfulness. On the surface the two extremes may be
indistinguishable, for what separates \co{forgetfulness} from forgetfulness,
\co{presence} from absence, is an \co{invisibly} thin line. With respect to
the \co{spirit}, such lines are the most crucial boundaries and only
\co{attachment} to \co{visibility} will view their expressions as paradoxes
which often permeate the language of \co{spirit}. 

\pa Modesty is to do everything one can. This is all, but we may also add: and
knowing that one can not do more.  Waiting resigned for a miraculous gift from
heaven has nothing to do with modesty; perhaps with laziness or sloth.  Modesty
sees even its own achievements as \co{gifts}, only ones which it can, to some
extent, influence.  An achievement is an \co{inverted} form of a \co{gift}.
Modesty works with full dedication, it employs all the abilities and potential
for achievement of its goals. It confronts the task and makes \co{me} disappear
in the process of this confrontation -- \co{I} am still there but, in a sense,
only for the sake of the task. Modesty is this disappearance of \co{myself}.
Only having done everything, only meeting the limits, one becomes modest. And
when one has done everything one could, one also knows it -- for knowing that
one can not do more {\em is the same as} having done everything one could. The
addition of \thi{knowing that...}  does not add anything; it only seduces us to
think of \thi{knowing} merely as explicit, \co{actual} and fully \co{reflective}
knowing.  (It even seduces us to think that what we said may be self-satisfied
and detached \wo{I am done with it ('cos I can't do anything else).} Modesty is
never done with anything, for it knows that no matter what it has done, more
could be done, only that it can not do that.)

A person trying actively to accomplish some task may spend a lot of time and
effort in this direction.  He may become a highly skilled expert with very high
professional standards.  From outside, and seen only in abstract terms, it may
look like he is only craving for reputation, recognition or just for
professional achievements of which he could be proud. Although often this may be
the case, it certainly does not have to be.
% \ftnt{I believe that Leonardo da Vinci, Albert Einstein and many other great
%   scientists and artists might provide examples but the issue would require more
%   personal acquaintance than I could claim.}
Modesty depends on one's capacities and standards one applies to
{\em oneself} -- if these are exceptionally high, others will rather see
ambition and pride.  But the person may -- though
only may -- be full of \co{openness} and modesty, and what is (typically, in the
impersonal sphere of gossip, rumours and newspapers) judged as
craving and striving may be but dedication, energy and \ldots true \co{humility}.

In short, what appears as arrogance may, in fact, be \co{thankfulness}; what
appears as preoccupation with one's little world may, in fact, be \co{openness};
what appears as pride may be \co{humility}.  Likewise with the opposite, we
never know \citet{under which tempting and affection-rising forms lie can, in
  spite of everything, penetrate to the deepest layers of [] spiritual
  honesty.}{ProustRuskin}{\noo{p.57}{\co{Inversion} is not a mere confusion. It
    is common that, for instance, \citef{vices shew themselves off as virtues,
      so that niggardliness would fain appear as frugality, extravagance as
      liberality, cruelty as righteous zeal, laxity as
      loving-kindness.}{GregoryEpist}{Book I:XXV. To John, Bishop of
      Constantinople, and the Other Patriarchs [Also, \btit{The Book of Pastoral
        Rule...} II:9]} \co{Inversions} can indeed give rise to
    misunderstandings, but it is their inherent feature that what is
    \co{manifested} appears \co{actually} as its opposite. They are more
    adequately described by St.~Augustine: \citef{A father beats a boy, and a
      boy-stealer caresses. If thou name the two things, blows and caresses, who
      would not choose the caresses, and decline the blows? If thou mark the
      persons, it is charity that beats, iniquity that caresses. See what we are
      insisting upon; that the deeds of men are only discerned by the root of
      charity. For many things may be done that have a good appearance, and yet
      proceed not from the root of charity. For thorns also have flowers: some
      actions truly seem rough, seem savage; howbeit they are done for
      discipline at the bidding of charity.}{AugustJohn}{VII:8}}\noo{{ought to
      understand how commonly vices pass themselves off as virtues. For often
      niggardliness palliates itself under the name of frugality, and on the
      other hand prodigality hides itself under the appellation of liberality.
      Often inordinate laxity is believed to be loving-kindness, and unbridled
      wrath is accounted the virtue of spiritual zeal. Often precipitate action
      is taken for the efficacy of promptness, and tardiness for the
      deliberation of seriousness.}{St.Gregory the Great}{The Book of Pastoral
      Rule of Saint Gregory the Great Roman Pontiff to John, Bishop of the City
      of Ravenna, Part II:IX} }} One does wisely suspending the judgment of an
accidental situation with respect to such deeper matters -- not because they do
not make any difference, but because \co{invisible distinctions} \co{manifest}
themselves only indirectly, and often in an \co{inverted} form, through the
\co{actual} ones.


\ad{Humility -- pride} \citet{But if there were one in hell who should get quit
  of his self-will and call nothing his own, he would come out of hell into
  heaven.}{TheolGerm}{LI} \Sch\ is recognising \co{myself} as \co{nothingness},
admitting not only that \co{I am not the master} but that, in fact, \co{I am
  nothing}.  If \co{I} think that I {\em am} anything -- no matter what, wise or
not-wise, good or bad, rich or poor -- I am still \co{attached} to \thi{images}.
\citeti{I am a son of X. These are my relatives. I am happy. I am unhappy, I am
  an idiot, I am a leader, I am pious, I have a relative, I was born, I died, I
  am old, I am a criminal.}{\Samkara}{\citaft{Mistyka}{ B:II.1.2\kilde{p.219}}}
If I use any names, not only for the \co{invisibles}, but also for \co{myself}, I
think that \co{I} am something.

% But \co{humility} of \yes\ is 
%
% \citt{We should be spirit for everything and everything should be for us 
% spirit in spirit. We should know/apprehend everything and divinise
% ourselves with everything. So we should be God from grace, as God is God from
% nature -- and we should release ourselves from all that, and leave it to God,
% and be as poor as if we then were not at all.}{Eckhart [after Otto,
% Mistyka...p.209]}
Saying, on the other hand, that \co{I am nothing} and, perhaps also, that
\co{nothingness} is the \co{origin}, can be construed as a proud detachment
attempting to rise itself above all such things of fundamental value to most
people. One can even attempt to construe \co{love} and \co{humility} bordering
on holiness as simple egoism, exclusive preoccupation with one's own self and
one's own salvation -- for \citet{that love occupies the highest place in the
  hierarchy of egoisms does not change the fact that it is
  egoistic.}{ProustRuskin}{\noo{p.70}}
%perhaps even as arrogant and self-centered pride.
However, calling holiness of love for \wo{egoism} is to deny it. Holiness does
appear as something higher and, perhaps, distant but, at the same time, it is
never absolute \thi{otherness} separated from us by an impassable distance. On
the contrary, it always embraces everything around itself as if telling to
everybody: that art thou, too. But not hearing this silent voice, one will think
the distance to be infinite, and see detachment instead of \co{presence}.  The
\co{humility} of \co{non-attachment} brings one, indeed, \co{above this world}
-- not, however, in any sense of despising \co{this world}, of contempt for
human weaknesses and vanity of all things, but only in the sense of not
accepting anything \co{visible} as \co{absolute}. \co{Humility} is \co{founded}
in the face of \co{nothingness} -- it is humble in the face of \co{visible}
things because they are its \co{gifts} and \co{nothingness} penetrates their
whole \co{actuality}. Thus, its \co{actual manifestations} can indeed be felt as
a proud challenge. They need not bear the appearance of obvious humility,
inferiority and self-depreciation of an ascetic or only Franciscan
flavor.\ftnt{For instance, it took St.~Bonaventure, \la{doctor subtilis} John
  Duns Scotus, \la{venerabilis inceptor} Ockham and few other Friars Minor to
  overcome the view, inherited after St.~Francis', which contemned knowledge for
  being a sign of pride.}  Just like the ambiguous modesty mentioned above,
\co{humility} may appear, in the eyes of the world, as its exact \co{inversion}:
instead of \co{non-attachment}, one can see detachment, instead of \co{humility}
-- pride raised \thi{above this world}.

%\pa
%Love, obedience, faith, charity, humility, thankfulness -- all these
%are synonyms! Do not try -- do not even try -- to separate them.

\ad{Passivity -- alertness}\label{pa:actpass} \co{Openness} \co{founds concrete
  presence}, in fact, omnipresence of God in all situations which,
\co{visible} and limited to the \hoa\ as they are, emerge from the \co{invisible
  origin}.
%The respect for things and people is but the
%\co{presence} of \co{openness}, an \co{expression} of God's \co{omnipresence}.
Every meeting, with a person, with a situation, with a problem, is a \co{gift};
sometimes, a challenge, sometimes, but a pleasant confirmation, sometimes a
plain disaster.  No matter what the specific character of this meeting, \co{I}
should be \co{thankful} for it because, at the bottom of it, the very fact of
being able to meet something deserves deepest \co{gratitude}, and because every
such meeting is also a meeting with the \co{origin}.

This \co{thankfulness}, however, does not mean that \co{I} am to fall flat and
thank God for bestowing on me yet another disastrous gift.  The \co{spiritual}
passivity is only to stop cherishing opinions, to stop \co{idolizing}
\co{oneself}. It is not to stop discriminating.  Being annoyed, being
displeased, being disgusted are impressions and feelings one need not get rid of
-- they are feelings of human saints as much as of human wretches.  To be
thankful for particulars is to stop absolutising them, to meet them with all the
respect they deserve as \co{signs} of the \co{origin}, and then to try to place
them on the right shelf in \thi{the order of things}.\noo{If one wonders what
  this phrase could mean, then I can not say -- I do not know! A person who
  pretends to know it -- in general -- is confused, perhaps, even dangerous. It
  is probably close to Homer's \citef{chain of gold}{Iliad}{e.g., VIII:18-27},
  or to \thi{the true value of things} in Lucilius' statement: \wo{Virtue is to
    be able to render the true value to the things among which we move and in
    which we live.}\citaft{SisterI}{ XII:17;p.226.}}
%
And if \co{I} have no clue where something belongs, then it can stay where it
is, at least for the time being.  Valuing things we also value our life and
\co{express} our \co{gratitude}.  The alertness and presence of mind is just the
steady preparedness to meet things with such an attitude.  It is \co{founded} in
the \co{transcendent openness}, but it concerns all the \co{immanent},
particular things. The \co{spiritual thankfulness} might seem to imply passive
acceptance of everything but, as a matter of fact, it is the opposite of
slothful passivity or mere aesthetism -- it \co{founds} active and vigilant
attitude to all \co{actual} situations.

\ad{Humility -- strength}\label{pa:humbleStrong}
Strength isn't much more than such an \co{open} alertness. It is not strength of
will, it is not strength of abilities but just that -- strength, preparedness to
meet everything with equal tranquility and \co{openness}, to face things and be
ready to handle them or, as the case may be, to be defeated by them.
One is strong when one has learnt that it is
impossible to lose, no matter what defeat one might suffer. 

This secure determination equals its meek \co{openness}. Strength has nothing to
do with hardness, with the defensive, self-protective shell one can, often with
ingenious inventiveness, rise as if in an anticipation of a threatening danger.
Hardness is but an extreme case of false security which spends years on
designing schemes and laws of things making everything fit neatly here and
there, on the right or on the wrong side, and which, eventually, realizes that
the whole scheme was but a construction; security which, in the most unexpected
moment, in the moment of the utmost surety and complacency, is suddenly
surprised, and that means defeated, to the bottom of its scheme.  The fear of
unexpected, natural as it might be, and which we might call insecurity, is
founded in false security, in \co{closedness} of \No, which tries to build
walls, houses, cities and yet, all the time, knows {\em that} there still may be
something it did not take into account, although it has no idea {\em what} it
might be.  Rising cities, it \co{thirsts} for the woods and fears fires\ldots

Hardness assumes that situations one has to be prepared for require
not only usual foresight but protection, are potentially dangerous and harmful. 
Strength sees the possible dangers, too, but its purpose is not to protect
itself. Greatness displays strength.  \citet{Great man [\ldots] is strong
  [\ldots] but he does not desire power. That which he desires is realization of
  his intention: realization of spirit. For this realization he needs, of
  course, power because power -- if we clean this notion of the dytyrambic
  pathos in which Nietzsche enveloped it -- means nothing else but simply the
  ability to realize that what one desires to realize.}{ProbMen}{p.55} Greatness
does not seek itself, but involves the ability to realize the \thi{objective}
intention.  Strength, too, will attempt to realize the intention, to maintain
\thi{the order of things}, but it will not be weakened, as greatness would be,
by its failure. Strength \co{founded} in \co{humility} may appear strange but is
always unmistakable. It may be associated with the abilities to posit and reach
particular goals, but its strangeness consists in that it does not depend on
such abilities.  Failing, it perseveres because what it attempts to achieve is
not merely a particular \co{actuality} but an expression of the \co{invisible
  command}, realization of spirit. This, on the one hand, admits various
expressions and, on the other hand, remains valid even when \co{actuality} seems
to ignore it. Strength is the infinite patience which is possible only because
it knows the \co{eternal presence}. (The ability to work, or wait, patiently for
a particular event, in a constant conflict between the expectation of the future
and the current situation, in short, living the tension between the
non-\co{actual} and \co{actual}, is a \co{visible} form of strength.)
Eventually, it is the strength of not expecting anything particular, of not
feeling that one is entitled to this or that, of having given up everything and,
therefore, having regained it. It is strength which does not have to search
because it has already found, which does not have to fight because it already
has everything.

\ad{Above the world -- in its midst} All the \co{inversions} can be seen as
variations of the apparent opposition \co{transcendence}-\co{immanence}.
\co{One} is fully both: remaining \co{indistinct} \co{above} the world, it is
the \co{rest} ever \co{present} between any \co{distinctions}. \co{Dissociating}
these two \co{aspects} is the common mistake which may be, partly, blamed on the
\co{inverted} form of \co{manifestations}.  \co{Dissociating} the
\co{transcendent aspect} of the \co{one} yields the abstract idea of some static
and immobile, incomprehensible ground.  \citeti{This ground is some homogeneous
  silence which remains immobile in itself. And yet from this immobility all
  things are set into motion and all things receive life, all which live
  suprasensually, silently in themselves.}{Eckhart}{\citaft{Mistyka}{
    B:I.1.5\kilde{p.197}}} The motion belongs apparently to the \co{visible
  world}, but the opposition immobile-mobile -- like that of
\co{transcendent}-\co{immanent}, one-many and most others -- is but a
construction: it may be required by the \co{actual} discourse, but it is harmful
when its terms get \co{dissociated}.

The \co{inversions} are no contradictions but only \co{reflective} expressions
of the \co{unity} of the respective \co{aspects}: \co{thirst} is the \co{sign}
of \co{presence}, \co{forgetfulness} is the way of remembering, strength is the
\co{sign} of \co{humility}, vigilant alertness of \co{thankfulness}, and rising
\co{above} the world, in the genuine sense, amounts to nothing but a full return
into its midst. \citeti{In famine he shall redeem thee from death: and in war
  from the power of the sword.}{Job}{V:20}

\noo{ \sep
%
  We are not after any necessity of contradictions, after any Lutheran stuff
  here. God may hide in the opposites of what one usually considers His
  \thi{true attributes}. But He may equally be \co{present} in all that is not
  such an opposite. If He is everywhere, there is no need to insist that He is
  {\em only} in what seems to contradict His nature. \co{Inversion} is a
  frequent possibility but still only a possibility.  Keeping this possibility
  in mind, let us review a few examples of \co{concrete founding}.  }


\subsub{Examples}\label{sub:examples}
The examples of \co{concrete presence} can be confronted with the possible
variations of the respective \co{experiences} which are not \co{concretely
  founded}. The main abstract difference between the two is that the former,
\co{originating} in the \co{nothingness}, span the whole hierarchy of Being,
while the latter are limited to the current level at which they unfold. They may
have all the amiable appearances of this level but they are unable to reach
beyond it, as if cutting the hierarchy at this point, and so remain \co{thirsting}
for \thi{Something}.


\subsubi{Love}
%\addcontentsline{toc}{subsection}{\ \ \ The founding of love}

\noo{
\found{Love}
  {sensuous : instantaneous pleasures (body)}
  {egotic (Ego) : of things/this world/humanity}
  {personal : of people/{\co myself}}
p{charity (spirit) : of \co{nothing}}
  {of {\em the person}/soul}
  {respect and enjoy: things}
  {pleasure}
  
[cf. Huxley, pp.83]
}

\pa \inv \co{Invisible, spiritual love} was described in section \ref{sub:yes},
in particular, \refp{pre:openness}-\refp{pa:GodneedsMe} and \ref{sub:reflYes}.
It is the \co{gift} of \co{grace} helped by the \co{reflective} attitude of the
whole person, which passes from \co{nothingness} towards the \co{visible world},
in the unity of \co{humility}, \co{thankfulness}, \co{openness}, as well as
other \co{aspects} which never exhaust its reality.
% By \co{analogy} (and only by \co{analogy}!) it founds the \co{experiences} of
% \co{omnipotence}, \co{goodness} and \co{omnipresence} -- of \co{nothingness},
% the \co{one}, the \co{origin}, Godhead.
They are {\em only} \co{aspects}, some \co{aspects}, of a one unified attitude.
All these \co{aspects} are not related to any particular region of Being but
\co{found} the unbroken continuity throughout \co{this} and \co{another world}.
The \co{concrete founding} of \co{love} amounts also to \co{concrete founding}
of the \co{unity} of \co{the world} which ceases to be split into \co{this} and
\co{another} one.\ftnt{In spite of many differences, the following hierarchy can
  be found to conform closely to that in \citeauthor*{Ibn}, in particular,
  II:13. Likewise, \citeauthor*{BernardLove}, \noo{after Damy XII wieku,
    p.310/311} lists similar stages, though with slightly different
  characteristics: sensuous appetite for carnal love, love of God for one's own,
  egoistic sake, love of God for His own sake, and the final union,
  \la{adhaesio}.}  This \co{unity} of the world and of oneself in the
\co{open confrontation} with the \co{origin} is the genuine sense of any
\co{actual experiences} of \la{unio mystica}. It is the same as \co{love} and
\citet{[i]t is therefore wrong to reproach the mystics, as has been done
  sometimes, because they use love's language. It is their by right. Others only
  borrow it.}{WeilWaiting}{Forms of the Implicit Love of God:Love of the order
  of the world;p.109}


\pa\label{pa:persloveB} \mine At the level of \co{mineness}, such a \co{love}
will find {expressions} as a living love with which the \co{soul} embraces the
world or, perhaps, its particular region.  The most obvious example is personal
love.\noo{ Whether it becomes love of the world, of all the people, of life, it
  still carries the \co{humility} which prevents it from focusing on
  \co{oneself} as well as from forgetting \co{oneself}, which \co{founding} it,
  makes it a true love and not a mere appearance of love.  } Love of another
person can have many degenerate forms but in its true form it is never a
separate focusing on this only person with the exclusion of everything and
everybody else. A true love of another person is impossible without the
\co{presence} of the underlying \co{love}. Love between two people is always
immersed into something bigger, something which only the lovers share and which,
in its \co{concrete} and yet \co{invisible presence}, often makes their love offensive
to the social law and customs, bringing the lovers out of \co{this world}, like
the magic of the fatal drink, and then the woods of Morois, to which Tristan and
Iseult flee from king Mark and his court.

\noo{ \subpa Love between two people unfolds always against the background of
  \co{presence} which has acquired a friendly, perhaps even ecstatic character.
  And yet, this very \co{presence} may have an oddly \co{inverted}
  \co{expression} when judged by the standards and customs of \co{this world}.
  \citeti{Think not that I am come to send peace on earth: I came not to send
    peace, but a sword.  For I am come to set a man at variance against his
    father, and the daughter against her mother, and the daughter in law against
    her mother in law.  And a man's foes shall be they of his own household.  He
    that loveth father or mother more than me is not worthy of me: and he that
    loveth son or daughter more than me is not worthy of me.}{Mt.}{X:34-37}
  Such an \co{inversion} does not mark a break, an immorality, but at most, an
  offense which is so perceived only by those who can not judge otherwise than
  according to the ethical -- and in cases of such conflicts, this always means,
  pharisean -- prescriptions.  }

Personal love is the highest form of relation with another, because it is the
ultimate form of \co{sharing} -- \co{sharing} the \co{origin}; it is thus not
really a relation but being, as one says, \wo{being together}.  As the meeting
with another person in the face of the common \co{foundation}, personal love is a
true \co{communion}, the \co{communion} of \co{sharing} the \co{origin}.  The
two lovers are meeting with something third, something \co{above} them both,
which lends its meaning and depth to their mutual relation, that is, their
being. And in all their \co{sharing} -- of life, that is of the world, of time,
of works and days, of joys and sorrows -- this \co{founding} element, this
indefinable \co{rest} remains \co{clearly present} as the \co{invisible} guarantor
of their \co{actual} love.  And thus, offering each other only \co{visible}
uncertainties of daily life, they raise from them a rock solid house.

The \co{concreteness} of such a personal love may involve fascination with this
or that feature, this or that characteristic of another person, but all such
features are but attractive accidents -- they may be needed for \co{me} to fall
in love, but they do not constitute the exclusive \co{foundation} of this love.
This love is directed toward the whole person, which means, toward the person as
\co{transcending} the particular features and particular ways of being and
behaving -- the person as the center and origin of all such particulars. \co{I}
do not divide the loved person into aspects and traits and decide to love her
because of $a,b,c$ and $d$. If \co{I} can tell why \co{I} love a person, then
\co{I} do not love. Sure, \co{I} can list a long series of agreeable and
wonderful features of this person, but if this list exhausts the reasons, then
this is a calculation rather than love.\ftnt{Analogous remarks concerning
  friendship figure in \citeauthor*{AristNico}{ VIII:3 [1156b]; IX:1 [1164a].}}

Love of a person is love of the whole person, the person seen as the site of
\co{incarnation} and this person's \citet{features, activities, abilities are
  included into love's object because they belong to this {\em particular}
  person.}{Sympatia}{B:III\kilde{p.256}} 
In this respect \citet{the loved one is impeccable
%stainless, spotless, immaculate, unimpeachable
in his vesture
%apparel
at the very beginning of being, because nothing lowers nor stains him in the
first moment of his revelation and being.}{Ibn}{V:88} \co{I} may, if not at once
then with time, see all the negative sides of the loved person, but to the
extent \co{I} love the person, these are but lower aspects, possible failures
which, as a matter of fact, can be charming too.

\pa %--\mine
The matter is quite different with love which is not \co{founded} in
\co{love}, but which stops at the level of \co{mineness}.

The lower we descend into \co{this world}, the more strength of will may be
needed to stay true to the \co{inspirations}.  But the strength of will is
needed only to the extent the \co{original commands} get clouded by the
lower aspects, as they become identified with some constancy or intensity of
feelings.
% Antigone says ``I know that I please where I am most bound to please.''  and
% this does not involve any conflict of her will.\ftnt{And I do not think that
%   this happens merely because, being Greek, she does not have the concept. The
%   whole tragedy is built around the choice which we may easily term \wo{a
%     conflict of commitment, or will}. }
The very attempts at nourishing and keeping the intensity of the beginnings are
already expressions of a loss, that is, expressions of \co{attachment},
\co{attachment} to the past. Whether \co{I} insist on \co{my} feelings, \co{my}
expectations, \co{my} goals it is all \co{attachment} to the \co{visibility} of
the past -- whether by attempts to preserve it or negate it -- which has
separated me from the \co{invisible} source of \co{love}.  Such an
\co{attachment} actually \thi{divides} the loved person, puts \thi{+} at
$a,b,c,d$ and \thi{--} at $f,g,h$, and when the calculus of \thi{+}s and
\thi{--}s yields a negative result, \co{I} become disappointed \ldots with the
person. The disappointments reflect only the fact that \co{my} love was not
directed towards the whole person -- it was cultivated and maintained not for
the sake of the loved person, but for \co{my} own sake.\ftnt{A simple example is
  love through which one merely seeks a compensation of some fundamental lack on
  one's own part. Not (necessarily) a lack of strength or intelligence or
  success, but a {\em fundamental} lack which one \co{vaguely} feels -- the
  emptiness at the bottom of one's \co{soul} which the other person would fill,
  the uncanny loneliness which the other person would cure, the undefinable
  dissatisfaction with \co{life} which the other person would calm, the
  \co{thirst}\ldots} Only preoccupation with \co{oneself} -- with its common
form of the sense of entitlement underlying all expectations -- can meet
disappointments; only nourishing \co{my} own image, can \co{I} imagine that the
world owes me anything. Disappointment is not a consequence of such an attitude
-- it is its inherent \co{aspect}. And when the \co{traces} of one's commitments
do not reach beyond the level of \co{mineness}, such disappointments can indeed
seem to sum up to the {\em whole} person who, because of $f,g$ and $h$, is no
longer worthy of \co{my} love.

There are no disappointments if one, instead of nourishing expectations,
nourishes \co{hope}. \co{Hope} is the lack of expectations, the acceptance of
\co{thirst}, unreserved \co{openness}, patience which does not await. \co{Love}
is full of \co{hope} not because it all the time awaits something new and
better, but because it does not -- it already has everything. It knows that all
the particulars need leniency, respectful openness and acceptance.  But such a
true patience and care for things and people are not ontological gifts of the
\co{origin}. They are \co{founded} only in the deepest \co{humility} and
\co{openness}. If they are not, the patience and respect will, sooner or later,
reach the end and then only laziness can prevent them from jumping to new
conclusions.


\pa \act Personal love, which at the level of \co{actuality} and \co{ego} may
also be expressed through infatuation, embraces things and situations lending
them the character of enchantment and agreeable vitality.  This may be a mere
feeling, a series of \co{impressions} which change and pass as soon as
infatuation goes away. As Eckhart says about the emotional and sensible love, it
\citeti{does not unify. True, it unites in act; but it does not unite in
  essence.}{Eckhart}{\citaft{Huxley}{ V\kilde{p.87}}} The \thi{unity in essence}
is not any emotion, is not a mere infatuation but a lasting love which immerses
the loved one, and the things \co{shared} with the loved one, in a peaceful
\co{presence}.  The traces of another's personality, expressed in \co{actual}
situations, transform them into a joy of \co{participation}. Even situations
which otherwise might be inattractive or repulsive, acquire this character
through the presence of the loved ones.\noo{It is better to lose with the loved
  one, than to find with the hated one.}

The \co{spiritual love} is a constant \co{inspiration} for the lower levels, an
\co{inspiration} to embrace, strengthen and invigorate -- whether the loved
person or the things towards which it turns at a given moment.  It is manifested
through care and respect for things, as well as for the particular behaviors,
feelings and reactions of the loved person.\ftnt{This is perhaps obvious, but
  let us emphasize that this care and respect are quite different from
  Heidegger's care -- \ger{Sorge}.  \ger{Sorge}, that is, \citef{the Being of
    Dasein ahead-of-itself-Being-already-in-(the-world) as Being-alongside
    (entities encountered within-the-world) [\ldots] is used in a purely
    ontologico-existential manner.  From this signification every tendency of
    Being which one might have in mind ontically, such as worry or carefreeness,
    is ruled out.}{SuZ}{p.192.}  This corresponds more closely to our horizon
  of \co{experience}, with its \co{spatio-temporal actuality} confronted with
  the \co{non-actual}, which precedes the constitution of time and separate
  \co{experiences}.  Care and respect we are talking about here are, to use
  Heidegger's terminology, precisely \thi{ontical tendencies} which Dasein may
  or may not realize.} This care and respect need not, of course, mean
unconditional acceptance of everything the loved one does.  But all particulars
which one finds blamable are placed at the level to which they belong, at the
level of \co{actual} failures, and in no way diminish the love.  As we know,
the blindness of love may not see, and if it sees will excuse, many things which
others find inexcusable.  Shall we say that this blindness is what we have
called an \co{inspiration} from \co{above}? Not necessarily because a mere
infatuation may have similar effect. But it does exemplify the general way of
\co{concrete founding}, that is, transformation of the events at the lower
levels by the \co{concrete} events at the higher ones.
%%
%%\subpa Sure, the lower we descend into \co{this world}, the more
%%\co{distinctions} and more possible choices we face.  The most
%%dramatic ones are those which involve a choice between lower and
%%higher.  Antigone {\em has to} burry her brother, 
%%%Polyneices, 
%%in spite
%%of Creon's interdiction and popular fear, shared even by her sister: ``I will
%%bury him: well for me to die in doing that.  I shall rest, a loved one
%%with him whom I have loved, sinless in my crime; for I owe a longer
%%allegiance to the dead than to the living: in that world I shall abide
%%for ever.  But if thou wilt, be guilty of dishonouring laws which the
%%gods have stablished in honour.''  But Sophocles does not leave any
%%doubt -- when the higher \co{command} is \co{expressed} with such a
%%\co{clarity}, the choice is obvious.

%\newp 

\pa %-\act
\co{Actual} love which is not \co{spiritually founded} will be directed at
things, typically, things which \co{I} want to possess, which is just an
expression of idealization of \co{my ego} as the highest value. It is hard to
recognise any true love in narcissistic self-idolatry, but even such extreme
forms of \co{egotism} may hide themselves behind the appearances of love.  Not
recognising anything higher than the \co{actuality} of \co{ego} and
\co{visibility} of its \co{objects}, one can still yearn for \co{love} and this
yearning can find occasional \co{expressions} of less \co{egotic} character. But
these will only be occasional \co{expressions}, constantly confused by the
tyranny of the \co{egotic} impulses.

Another person is no longer loved only for \co{my} sake, honestly though
confusedly, but for the sake of some particular thing. \wo{\co{I} love her
  smile}, \wo{\co{I} love her meekness}, \wo{\co{I} love her determinacy}, no
matter what particulars happen to arise the reaction, it is only a reaction
to the \co{actual} fascination which is cherished for the sake
of satisfaction it gives \co{me}.

\newp
\pa \imm At the lowest level of \co{immediacy}, love, like anything else, finds
only the most momentaneous \co{expressions}.  Sex can provide a good
example, as the infinite gap separating the purely carnal sex from the event
of making love to a loved person will be perfectly clear to anybody who has
experienced both extremes, and to many who haven't.  The sensuous pleasure is not
necessarily 
\co{spiritually founded} in any higher order of things, but it is tremendously
modified if such a \co{founding} has taken place.  On the one extreme, it may be
a mere moment of escape from the unbearable suffering, a moment of sudden
meeting with eternity in the midst of confusion and evil, like is experienced,
for instance, by the war time lovers of Remarque.  It may be an even more desperate
attempt to convince oneself that, after all, there are good things in life,
things which, in the brief moments of pleasure let one forget about the
otherwise empty and desperate life.  All such moments do provide the pleasure
they promise, but the pleasure turns out to be insufficient to calm the soul.
And then there remains only \co{more} pleasure, \co{more} intensity, \co{more}
-- \wo{the cry of a mistaken soul}.  On the other extreme, a sensuous pleasure of
a moment can be peacefully embraced by the context of mutual respect and
understanding, and, at the deepest level, of the ineffable \co{love} which, by a
lucky coincidence, found an \co{incarnation} in the other person, in this very
moment, like an unmerited \co{gift} which never ceases to surprise and
please.\noo{, even only by its mere presence.} 

\pa %--\imm
We will not try to reduce such considerations to the mere contextuality of
\co{actual experiences}.  The context of \co{an experience} is \co{more} of
other \co{experiences}, and a well experienced and well paid whore can create
such a context.  One can buy only moments, \co{actualities}, separate, scattered
parts.  But such parts, and contexts which are their \co{complexes}, especially
if they are merely bought, will never sum up to \co{an experience} which in a
single moment traverses the whole hierarchy of Being and reaches smoothly the
deepest intimacy of the \co{transcendent presence}.  One does not reach the
\co{origin} by taking, one after another, the same single step at a time, even
though taking one step at a time is all we can do.
% But the steps may be wrong while they must be right, and nobody can take them
% for me.
In all pleasant delusions, which \co{I} bought or arranged, which \co{I} find
purposeful and satisfying, \co{I} always also know -- if only \co{I} do not ask
too intensely -- that they are but momentaneous pleasures, delights of a
hedonist, real because \co{actual}, and insufficient, perhaps even empty and
unrewarding, because {\em only} \co{actual}.

\pa There is no such thing as a single experience which is not permeated by
the whole Being, and it would be as useless to focus on this fact as it is
impossible to ignore and forget it.  There is no such thing as a true moment of
love, unless this moment is immersed in the texture (not context) of body,
\co{ego}, \co{soul} and \co{spirit}, which all together agree on the
\co{humility} of \co{love} -- in an agreement which goes beyond the bottom of
one's soul and heart, where the \co{spiritual love incarnates}.  Only this whole
texture lends the \co{actual} moment its full meaning, only it makes up its
\la{quiddity}, makes it \thi{this \co{concrete experience}} rather than
\thi{that}. No \co{visible} rules can ever grasp this distinction with the
adequacy and \co{precision} the \co{actual} reason might desire. And yet,
everybody knows it, if only in some \co{vaguely} \co{gnostic} manner.
%, and most people can also \co{recognise} it. 
%%%% moved to morality...
% And no ethical rules or principles will ever grasp the moral value of an
% \co{act} which, to the extent it is moral, descends from the \co{invisible}
% heights and only as if by accident takes this particular form rather than that.

\noo{
\found{Levels of communion}
    {it: atomic / no relations}
    {he: common goals/useful persons: formal-institutional}
    {you: -- not me; isolation; loneliness}
{\co{participation}:communion with the world -- That art thou.}
    {Thou: -- above me; friendship, personal engagement}
    {helpfulness, empathy}
    {shared moments}
}

% Modifications of the Jungian archetypes of \la{anima}/\la{animus}
% happen to conform very exactly to our division. Ignoring the 
% intricacies of these archetypes and their possible projections, 
% I will use them as illustrations since they provide quite concrete 
% typography. We should only keep in mind that they represent only  
% some ideal, symbolic contents, which hardly give an exhaustive 
% account of the actual experience. 

%-- levels of anima [MHS, p.195] :
%
%\aretabb{rl}
%  {Eve: & insinctual, biological relations}
%  {Faust's Helen: & romantic, aesthetic, but still sexually dominated}
%  {Virign Mary: & raises love to spiritual devotion (`moral')}
%  {Sapientia: & wisdom transcending most holy and pure (`spiritual')}
%
%\noindent
%-- and of animus [MHS, p.206] :
%
%\aretabb{rl}
%  {physical strength: & muscle man}
%  {initiative and planned action: & highwayman, macho (politician)}
%  {power of word: & wise professor (`moral' guide)}
%  {incarnation of meaning: & religious leader (`spiritual')}
%
%Above \co{found} various ways \ldots though these have been included, 
%somewhere in Book III:
%
%  \parbox{12cm}{ \levels{Communication / with}
%\are{observation / things}
%    {control / animals, cultures}
%    {interaction (sharing with) / community, other persons}
%    {participation (in) / myself-God} }

%%%%%%%%%%%% 8.5, kl.20:50

\subsubi{The communion} 
%\addcontentsline{toc}{subsection}{\ \ \ The founding of communion}
\pa\label{pa:communion}
\co{Communion} is \co{sharing} and that which is \co{shared} determines the
character of the \co{communion}, in particular, whom the \co{communion} includes
and the way in which the others are \co{experienced}.  Nature and the physical
world is not ours, we \co{share} it with all the physical things and living
organisms, and this is a form of \co{communion}.  We \co{share} more with
animals than with dead things, and more with higher animals than with
lower ones.  We \co{share} quite a lot with all other people, but there are
always special people, friends, family, the loved ones with whom we \co{share}
much more than with the mass of anonymous individuals.

That which is \co{shared} is not to be confused with that which is merely
common.  Common -- in its full ambiguity of universal and ordinary -- is the
\co{objective} version to which \co{sharing} reduces when seen only from the
perspective of \co{actuality}. What is \thi{common} can not be \co{shared}, it
can only be multiplied, like a universal instantiated in many particulars, like
the sexual drive, common to most animals but never \co{shared} by any two.
Looking for most \thi{common} features and traits which, as one thinks, would
promote the most universal communication or the sense of community, leads only
to reducing everything to the least common denominator, to that which being the
most universal is also the most ordinary.  Such a search for universality is an
attempt to capture \co{more}, to overcome the \co{horisontal transcendence} --
it may be useful, but never \co{concrete}.

\co{Sharing}, on the other hand, refers always to the \co{vertical
  transcendence}, what is \co{shared} is eventually \co{above} \co{me}.
\co{Sharing} is not a relation between individuals, but being-together, that is,
\co{participating}-together in the \co{one} and the same. 


\pa\label{pa:originShares} \inv The ultimate \co{communion} is \co{sharing} the
\co{origin}, which does not get multiplied and distributed between many, but
which remains \co{one}, undivided and \co{indistinguished}.  That \co{I} {\em
  have} the \co{origin} does not mean that it is \co{mine}. It means only that
\co{I} have originated, that \co{I} was \co{born} -- the \co{origin} remains
\co{above me}, and thus can never be \co{mine}, it is \co{mine} as much as it is
\co{yours}. In the expressions like \wo{having the \co{origin}}, the word
\wo{having} does not express the possession but \ldots its opposite:
\co{participation}.
%To have the \co{origin} means to \co{participate} in it.

Having the \co{origin} is the same as \co{sharing} it, for everything has
originated from \co{nothingness}. It thus \co{founds} the highest form of
\co{communion}, the \co{communion} of \co{participation}, that is, of being.
The \co{origin} is \co{shared} -- with whom?  With nobody in particular, with
\co{nothing}, that is, with everybody.  This \co{communion} permeates the whole
world, includes all the people, all \co{visible} and \co{invisible} things,
everything which is.  It is the exact opposite of the {detachment} from \co{this
  world}, the deepest form of \co{communion} which announces \wo{That art thou},
which encounters everything \la{sub specie aeternitas}, as a manifestation of
the same \co{origin}.

Metempsychosis is perhaps one of the oldest expressions of the feeling of such a
\co{communion}. But it is only a conceptualisation, is the
\co{experienced unity} brought to the level of \co{visible signs}, perhaps, to
an attempted explanation. \citeti{For I
  was once already boy and girl,\lin Thicket and bird, and mute fish in the
  waves.}{Empedocles}{DK 31B117} Interpreting such pronunciations as declarations of
metempsychosis is a vast over-interpretation. They express genuine feelings, but
ones which manifest only something \co{transcending} any expression,
that is, something which may be \co{present} also without any expression. One
could hardly postulate migration of souls without \co{recognising}
that also animals and even things possess souls, that they too \co{participate}
in \co{one} and the same. Metempsychosis is but
an image of this \co{recognition} and hence, eventually, of the \co{experienced 
  unity}. And likewise, its images are the feelings of universal life and
ensoulment, \citeti{knowing that all things have their
  emanations,}{Empedocles}{DK 31B89}
the sense of deep kinship 
which permeates us and lets things grow into and out of us:
\begin{verse}
{\small{\em Durch alle Wesen reicht der eine Raum:\\
Weltinnenraum. Die V\"{o}gel fliegen still\\
durch uns hindurch. O, der ich wachsen will,\\
ich seh hinaus, und in mir w\"{a}chst der Baum.~\ftnt{\citeauthor*{RilkeEsWinkt}}
}}\end{verse}  
%
\co{Sharing} the \co{origin} has also a more primordial \co{aspect} of
\co{origin sharing} itself.  The {world} is a result of \co{dissociations}, of
\co{objectivisation} and \co{externalisation}, but it is also the world
\co{unified} in its nature of a \co{gift}.  This nature means exactly that it is
\co{shared} -- not {\em by} \co{me} with others, but {\em with} \co{me} as a
\co{gift} of \co{transcendence}. As a \co{gift}, the world is not \co{mine} at
all -- it is \co{shared} with \co{me} by its \co{origin}.  As such a \co{gift}
of the world, the \co{confrontation} with the \co{origin} is the primordial
\co{communion}.  This \co{communion} of the \co{origin sharing} itself, the
\co{existential} \co{confrontation} which is not opposition but \co{openness},
can be called the dialogical character of \co{existence}.  The \co{concrete}
God, the \co{incarnated} Godhead, is thus the one who \co{shares} the \co{gift},
Thou who communicate, a partner of a dialogue, albeit a dialogue which knows no
words, the silent dialogue which \co{existence} conducts with its \co{origin}.
\citet{Every concrete hour which, with its world content and destiny, is
  allotted to a person, is a noteworthy language.}{Zwie}{I:Verantwortung.
  \orig{Jede konkrete Stunde mit ihrem Welt- und Schicksalsgehalt, die der
    Person zugeteilt wird, ist dem Aufmerkenden Sprache.}\kilde{p.161}}

\noo{ \subpa Recognition of the the world as a \co{gift} requires the
  \co{non-attachment}, renunciation of \co{myself}, and thus it is not relative
  to \co{my} thinking, wishes, goals.  This \co{communion} is consummated
  \co{above}, it is \co{spiritual}, which means, unconditional.  But
  \co{spirituality} is first of all \co{humility} towards the \co{spiritual},
  towards the \co{invisible} things.  \citet{Know what is in front of your face,
    and what is hidden from you will be disclosed to you.  For there is nothing
    hidden that will not be revealed.}{Thomas}{5} By this very token, the
  \co{spiritual} attitude directs its attention almost exclusively towards
  \co{this world}.  \citeti{[T]here is nothing better, than that a man should
    rejoice in his own works; for that is his portion.}{Eccl.}{III:22 [We won't
    confuse these remarks with stoicism.  \citef{We must make the best use that
      we can of the things which are in our power, and use the rest according to
      their nature. What is their nature then? As God may please.}{Epictet}{,
      I.1} This says the same as what has been said above, but the Stoic
    endurance is a matter of resignation and surrender to the world which
    overgoes one's powers, not of \co{thankfulness} for its \co{gift}.]}  }


As was observed in II:\refp{pa:shareable}, the higher things are also those
which admit of more unconditional \co{sharing}, which are less diminished by
being \co{shared} among more. Since the \co{origin} is common to all of us, it
is most intimately and universally \co{shared} by all, we all \co{participate}
in one and the same source from which our lives and worlds originate.  This
\co{spiritual sharing} is not sharing of this or that, but is \co{sharing} of
the \co{origin}. To the extent we truly \co{share} any \co{actual} things or
moments, any \co{visible} entities which by and in themselves can only be
multiplied or divided but not \co{shared}, we do so only in the light of this
\co{invisible communion}.

\noo{ The \la{anima} of this level is, borrowing the Gnostic image, Sapientia,
  -- a spiritual wisdom transcending the \co{visible} world towards the most
  holy and pure. \la{Animus} is symbolised as an incarnation of spiritual
  meaning, a religious leader with spiritual strength and power transcending,
  though not contradicting, mere morality.  }


\pa\label{pa:recognizePerson}\mine
The center of personal being, the emergence from the
\co{confrontation} with the \co{origin}, has thus the character of a dialogue,
of a \co{confrontation} with another. This other is not, however, any foreign
otherness but the most \co{concrete} Thou, in the face of whom \co{I} become
\co{myself}. \citet{Man becomes \co{self} through Thou.}{IchDu}{I.\orig{Der
    Mensch wird am Du zum Ich.}\kilde{p.32}} This 
primordial dialogue, when expressed at the level of \co{mineness}, as a dialogue
with another person, amounts not to exchanging opinions, observations and views
of life, not even to agreeing on any such issues, but to the \co{recognition} of
the fact of \co{sharing} the same \co{origin}, of \co{participation} in the
same, higher sphere of Being. The genuine \co{community} is simply
\co{recognition} of the \co{communion}. \co{Recognition}, of course, is not
necessarily \co{re-cognition} and has basis in the \co{rest} of \oss, 
in the sense of unity only \co{vaguely} discernible in the background
of \co{actual} intentions. \citet{It seems to
me that both in the order of (atemporal) \co{founding} of functions and in the
order of genetic development, the feeling of unity <<is the fundament>> of
sympathy [\ger{Nachf\"{u}hlen}]. This statement concerns, of course, only the
emotional {\em functions} and not the emotional
states.}{Sympatia}{A:VI.a.\kilde{p.157} Scheler's \ger{Nachf\"{u}hlen} means, 
literally and inelegantly, re-feeling (or feeling-after) and is
intended as an emotional re-presentation of an object or another's
experience. It is, in turn the fundament of empathy, \ger{Mitf\"{u}hlen},
literally, co-feeling, feeling-with.}

\co{Thankfulness} for the \co{gift} of life and world amounts to trust and
fidelity, for acceptance of everything leaves simply nothing to mistrust and
betray. In the same way, the \co{community} of this \co{gift} is also the
\co{concrete foundation} of the mutual trust. At the bottom, it is only the
\co{recognition} that, eventually, the other is a person like myself,
\co{sharing} the same \co{origin}, \co{thirst} and all significant spheres of
\co{experience}.  \citetib{Acceptance of the sameness of reality conditions the
  spontaneous emergence of love to a human being, that is, love to a being only
  because he is <<human>>.}{Sympatia}{A:VI.c\kilde{p.160}} In fact, not only love, but
any truly personal relation -- of love or deep admiration, of dedication or
respectful enmity -- is based on such a recognition. In a bit different words,
it requires recognition of the {\em whole} person, that is, recognition that the
other reveals to \co{me} something that is thoroughly and intimately {mine}.
%, that it is impossible to say what \co{precisely} it is.
% \ftnt{\thi{Reveals to \co{me}} not necessarily by being \co{actually} the same,
%   but perhaps only by provoking some genuine aspects of myself to surface.}
The deepest truths of my \co{existence} are reflected in such
relations to the others, for the \co{community} of our differences is
\co{concretely founded} only in the sameness of the \co{origin}.  What is so
reflected, what is being \co{shared}, might seem to be both \co{mine} and
\co{yours} but, as a matter of fact, it is neither \co{mine} nor \co{yours}, for
it is \co{shared} only by being \co{above} us both.
%unfounded: relating only to an aspect of the person

\thi{Recognition and apprehension of the whole person} is exactly to see this
person as \co{sharing} the \co{origin}, as having an \co{invisible} pact with
God, just like the one \co{I} have.
%It is beyond any \co{visible} grounds and reasons. Thus
\citet{Human life touches the absolute through its \thi{dialogical character};
  [\ldots] man can become whole not through a relation to his own self but only
  through a relation to another self. This other self may be equally limited and
  conditioned as he is, but in being together one experiences that which is
  unlimited and unconditional.}{ProbMen}{II:1.5} Buberian emphasis on the
dialogical element is obviously concerned with the \co{experience} and the form
of the \co{experienced}, which we certainly can accept. But we are more
concerned with the \co{foundation} of such \co{experiences} which, as also Buber
maintains, is the unconditional, or \co{absolute}.  The dialogical
\co{experience} can be also \co{an experience} of the unconditional. But the
unconditional \co{presence} is not the \co{object} of such \co{an experience} --
it is the \co{rest} \co{experienced} only {\em along} the \co{actual} situation,
{\em through} the meeting with the other. It is the \co{participation} in the
\co{absolute} which \co{founds ontologically} such \co{experiences}, and only
the genuine \co{openness} of \Yes\ which \co{founds} their \co{concrete}
possibility.\ftnt{Dialogues with other persons do not exhaust such
  \co{experiences}. An unexpected \co{recognition} may often give rise to it, as
  when we suddenly get swept by the immense beauty of a landscape. An experience
  in which the fact of encounter comes strongly forth, involves often equally
  the sense of unity. These two \co{aspects} together constitute the dialogical
  character of \co{experience}. Mystical experiences provide many examples,
  while here is a bit more mundane case of a similar force: \citef{From a low
    hill in this broad savanna a magnificent prospect opened out to us. To the
    very brink of the horizon we saw gigantic herds of animals: gazelle,
    antelope, gnu, zebra, warthog, and so on. Grazing, heads nodding, the herds
    moved forward like slow rivers. There was scarcely any sound save the
    melancholy cry of a bird of prey. This was the stillness of the eternal
    beginning, the world as it had always been, in the state of non-being; for
    until then no one had been present to know that it was this world. I walked
    away from my companions until I had put them out of sight, and savored the
    feeling of being entirely alone.}{JungMDR}{[after \citeauthor*{CC},
    p.14-15]}}
%unfounded: not recognising the same origin

\pa This \co{founding} can be expressed by saying that the \co{community} is
established always via way of God.  \citet{If we both see that that which thou
  sayest is true, and if we both see that what I say is true, where, I ask, do
  we see it? Certainly not I in thee, nor thou in me, but both in the
  unchangeable truth itself which is above our minds.}{AugustConf}{XII:25} In
the deepest respects I understand you only because I already \co{participate} in
what you might want to communicate.  Indeed, \citet{in simple substances this
  influence of one monad over another is only {\em ideal}, and it can have its
  effect only through the intervention of God.}{Monad}{51. (With all possible
  reservations against the particular ways in which Leibniz imagined God and
  this \thi{intervention}.)\label{ft:commGod}} Likewise, a reader responds to
the author's appeal and joins him \citet{at the virtual center of the writing,
  even if neither one of them is aware of it.}{PontyIndir}{77
  \citaft{CathQuat}{}} Or as Buber says it: \citet{\co{Above} and \co{below} are
  tied to each other. His word, who attempts to speak to man, without speaking
  to God, does not fulfill itself\ldots}{Zwie}{I:Oben und unten.\orig{Oben und
    unten sind aneinander gebunden.  Wer mit den Menschen reden will, ohne mit
    Gott zu reden, dessen Wort vollendet sich nicht\ldots}\kilde{p.160} The
  sentence continues: \wo{but his word, who attempts to speak to God, without
    speaking to man, goes astray.} \orig{aber wer mit Gott reden will, ohne mit
    den Menschen zu reden, dessen Wort geht in die Irre.} \co{Concrete founding}
  of \co{communion} means that, indeed, existentially the \co{spiritual} \yes\ 
  and the true \co{communion} with others are indistinguishable, or better,
  co-extensional -- whenever there is the one, the other is too. But we are
  trying to be a bit more pedantic than Buber -- the dialogue with God, the
  \co{concrete founding} is still \co{founding}, and it \co{founds} a whole
  \nexus\ of \co{aspects} of which \co{communion} with others is only one.}
%unfounded: talking to others without (talking to) God

The other is not such an ultimate stranger as some preachers of \thi{otherness}
-- opposing with certain right the centuries of sameness, leveling off and
evening out -- would like to see him. If not at the sociological level (which
nourishes much of the \thi{otherness} talk), then certainly at the human level
the other is, for the most, the same. The other is the same as you -- only
another. One \co{birth}, like every \co{absolute} beginning, is \co{virtually}
the same as any other -- only numerically distinct. Sure, he is a different
person, with whom I even may be unable to communicate -- not to mention, to
agree. But communication and agreement have all too often been degenerated to a
petty accord of opinions or a sheer coincidence of wordings to deserve some
censuring. I can not understand the other by the sheer act of accepting his
otherness. {\em Accepting} otherness means nothing else but searching deeper
into oneself (which does not mean looking for more discoveries {\em about}
\co{oneself} but searching for the harmony {\em of} and {\em in} one's
\co{self}). I understand, or at least can understand the other, only because, at
the bottom of our souls\noo{(and this is much deeper than any psychological,
sociological or political constructions can ever reach)} we \co{share} the most
fundamental aspects of \co{existence}: eventually, we \co{share} the
\co{origin}, stand before the same God.



%\tsep{culture, tribe...}

\pa \co{Community} has, of course, also the inter-personal, cultural or
sociological dimension.\ftnt{In terms of the mere \co{objective} time and
  the sheer numbers this might seem to be much more of a \thi{community} than
  the personal relations which concern us. However, we do not study sociology,
  nor even the individual reflections of the social domain. Such a reflection is
  possible only because the individual is already a dialogical being, capable of
  genuine \co{sharing} independently from the form of community into which he is
  born.  Moreover, it is the personal relations which open most directly the
  sphere of \co{community} in the face of the same \co{origin}. The tribal,
  communal, social, traditional organisation will first of all veil and entangle
  the direct, \oo\ \co{signs} of \co{sharing} into the \co{objective} -- and hence
  both lasting but also less readable -- forms of \co{symbols} and other \co{reflective
signs}. To become a 
  member of a \co{community} means often to decipher the codes of its tradition
  in a way which allows one to live them satisfactorily, that is, with a
  personal conviction but also with a full respect, though not unreserved
  openness, to other traditions.}
The above, universally human \co{community} becomes relativised as \co{sharing}
concerns more specific contents. But it is all the way \co{founded} on
\co{sharing}.  It is not so that, for instance, problems or enemies create
\co{community}; at best, only {\em common} problems and {\em common} enemies
do. But in order to be common they must threaten some \co{shared} dimension,
that is, they only reveal, make \co{visible} the \co{community} which has
already existed.\ftnt{One might be tempted to see in various \thi{encounters
with \abr{ufo}s}, in much of the science-fiction frenzy and
in the search for extraterrestrial intelligence, the signs reflecting
a relatively high level of global consciousness which sees humankind as a unity
but, scared by its apparent emptiness, that is, lacking \co{foundation}, 
still needs \thi{another}, a \thi{common neighbour} if not enemy, as its 
\co{external} and \co{objective} guarantor.}\noo{Thus,
national or ethnic characters of various peoples are 
expressions of the \co{shared} background (historical, geographical, etc.). }

It is essential for a \co{community} to \co{share} something which
\co{transcends} mere \co{actuality}.  A group of mere common interests, an
\co{actual} group which shares only a hobby or a problem, is not a
\co{community}. Any group of only \co{actual} goals or interests is as relative
and transient as these goals and interests themselves -- it can be only a
surrogate of a \co{community}. A \co{community} transcends any \co{actual}
horizons, it can be a cultural tradition, a nation, a tribe, a family. I once
heard a native American saying to his children: \wo{White people have been here
for 500 years, {\em we} have been here for 15000 years.  They make choices based
on what seems cool and advantageous to them but this is not how {\em we} make our
choices.  {\em We} have got this land and we have to care for it for future
generations.  Our private wishes are not what counts most.} Belonging to a tribe
is to belong to the world which is far greater than \co{me}, is to be only a
member of a \co{community} which \co{transcends} the sphere of \co{my life}.  The
respect  shown for one's 
land and its tradition is an \co{expression} of \co{sharing} something which does
not belong to anybody, which is greater than \co{me} and you, than our ancestors
and successors.  Similarly, the respect for the ancestors, or generally for the
history (of the nation, tribe, family), is an expression of the constitutive
role of the \co{transcendence} -- not only the mythological beginnings, but the
whole past of the \co{community} lies beyond our \co{actual} grasp, and its
great moments express and witness to the unity of the \co{shared} ethos and
origin.


\noo{
\subpa For Westerners it may still sometimes be the consciousness of belonging
to the nation.  No matter what group is concerned (as long as it is not a merely
\co{actual} group), the recognition of the sound values of this group, which
override \co{my} private preferences, and healthy dedication to their
realization is a possible, \co{visible expression} of belonging to a
\co{community}.  As \co{I} am, at this level, distinct from you and everything
\co{not-mine}, \co{my community} will be distinct from \thi{others}, from
\thi{them}, but this distinction need not carry the negative meaning of an
opposition or even conflict.  For the first, it is not so much \co{my} group as
opposed to \co{not my} group -- it is {\em this} group as opposed to {\em that},
and it just so happens that \co{I} belong to this one.  And furthermore, the
distinction between the groups need not mean conflict, in any case, not a
fundamental enmity.  The groups just happen to live different traditions and
different values, occupy different regions of \co{this world}.  The \wo{healthy
  dedication to its values} means their recognition as {\em only} one -- as
opposed to {\em the only} one -- possible \co{manifestation} of the
\co{invisible foundation}. As far as this \co{foundation} is concerned, it
remains the same \co{above} all the \co{visible} differences. The capacity to
recognise also other's values as such a \co{manifestation} (and this does not
mean their adoption, but an acceptance and thorough respect, with or without
understanding the \co{actual} details) is the condition for being able to live
one's own values in a \wo{healthy} manner. Just like the incapacity to resolve
(which sometimes may simply mean: ignore) contradictions is a sign of a mind
falling into the narrowness of \co{actuality}, so any tendency to emphasize the
oppositions and conflicts between the apparently contrary cultural traditions
witnesses only to the insecurity of the tradition showing such
tendencies.\ftnt{We mean that it applies to all the successfully conquesting and
  expanding civilisations (whose initial, and the strongest, impulse springs
  from the opposition to a higher, or in any case, more confident and
  self-satisfied culture), as well as to all the unsuccessful ones.
  Unfortunately, we do not exactly know what \wo{successful} could mean in this
  context, so we won't elaborate.}
} %%%%%% end \noo{ Westerners

%\tsep{unfounded}

\pa  %--\mine
At the personal level, the \co{unfounded community} recognises the other as you but
not as Thou.  You may be an equal partner, perhaps a guide or a friend, with
whom I can establish a community through interaction and sharing values, views,
life-style, something more than mere goals and contents dictated only by the
\co{actual} situation. But as long as not \co{concretely founded}, this remains
an incomplete relation.
%, like those which tend to break up in the so-called \wo{mid-life crisis}.
The relation will carry the negative counterparts to those listed in
\refp{pa:recognizePerson}: I can recognise you as simply different person,
without recognising that we \co{share} the \co{origin}, I can reduce the
\co{shared} to something merely common, eventually, some \co{actual} contents
which tend towards \co{objective} and \co{egotic} characteristics.  Thus our
relation, bereft of higher \co{foundation}, becomes afraid of every conflict at
the level of \co{actual} opinions and decisions, becomes threatened by more and
more minute details, like ebbing love often does. As always, whatever is not
\co{concretely founded}, which in particular means, \co{founded} in something
higher, tends toward its lower version.


%\pa
\co{Unfounded communion} at the level of \co{mineness} is still sharing. It is
not, however, \co{sharing} something higher but sharing \co{myself}, whatever
\co{I} happen to understand by this at the moment. \co{I} can dedicate \co{my}
activities to a common good, to a beneficial work for the society, \co{I} can
become personally engaged -- but in all that \co{I} share only \co{myself} and
keep experiencing this sharing as sacrifice.  Enclosed in the circle of
\co{mineness}, \co{I} oppose selfishness to unselfishness, \co{my} private life
and interests to the good and interests of others. And thus the engagement into
the latter assumes a character of a sacrifice on \co{my} part. No matter how
possibly useful and socially valuable, such an attitude, called by Kierkegaard
``ethical'', finds itself in a constant conflict which it is unable to resolve
otherwise than by negating it or turning \co{my attachment} into the \co{idol}
of \co{my} generousness, \co{my} benevolence, \co{my} self-sacrifice.

But as \wo{one cannot will into void,}$^{\ref{ftnt:willVoid}}$ all \co{my}
dedication must be directed towards some positive contents and values.  A common
form of this is \thi{identification} with one's community, various cases of
communal or, perhaps, communistic consciousness, when the abstractly universal
good and interests of this community become one's highest values.  When one's
community -- tribe, nation, class, religion -- is the only source of truth then,
indeed, nationalism or tribal consciousness acquire unhealthy form of ethnocentrism.
Such an absolutisation of a relative is possible only because one does not
recognise its \co{foundation} in the higher sphere -- here it ends, there
is nothing \co{above}, and this is the last, absolute truth, its final
expression.

\pa\label{pa:notPerspectivism} 
% We see here the difference between this view of \co{community} and all forms of
% relativism and extreme perspectivism. (Book I:\refsp{sub:pantheism}.)
As a matter of fact, relativism is an example of such an \co{idolatry}, of the
inability to \co{recognise} the ultimate value \co{above} the multitude of
\co{visible} differences. It is different from \co{ego-} and
ethnocentrism in that it avoids absolutisation of one particular ethos. But,
staying at the same level, it is only its complementary and inverted 
side which does not reach any deeper.
% Let us begin to look at this form the 
% personal level -- the social one is only accumulation of the 

\citet{Every person coming to the world brings something new, something that has
  never existed before, something original and unrepeatable. [...] Every man is
  a new event in the world and is called to fulfill his uniqueness on
  earth.}{BuberChasyd}{p.22\noo{\citaft{Chasydzi}{\kilde{p.95}}}} So far, so
good, but here comes a related expression, which apparently only elaborates the
same idea.  \citet{Everything happens as if the multiplicity of persons -- does
  not the sense of the word \wo{personality} reside exactly therein? -- were the
  condition of the appearance of the full \thi{absolute truth}, as if every man
  through his unrepeatability guaranteed revelation of one unique aspect of the
  truth, and its other aspects would never be revealed if some persons were
  missing among the people. It suggests that the totality of truth is a sum of
  innumerable personal parts
  [...]}{LevinasChas}{\citaft{Chasydzi}{\kilde{p.96}\label{ftnt:personCommunion}}
  (The following is only an elaboration of this expression which, for Levinas'
  part, we would be willing to take as a bit unfortunate metaphor. Accusing him
  of pantheism and relativism would certainly be unjust, though shadows of both
  can be discerned, as usual, behind the eschatological predilections.)} We see
the reappearing tension between One and Many, which tries to disappear in the
close relationship between the perspectivism of the latter and the pantheistic
understanding of the former.  This last quote suggests that the \co{absolute} is
somehow the sum of the individual perspectives.  But such a sum is only a
\co{posited totality}, not anything \co{absolute}, it is the \co{vertical
  transcendence} of the \co{one} thought in terms of the \co{horizontal
  transcendence} of multiple \co{actual} manifestations. The unrepeatability of
every man is \co{founded} in the very \co{confrontation} with the \co{absolute};
it does not merely add a bit to its \co{actualisation}. But taken in the merely
\co{horizontal} dimension, it turns indeed into the ultimate relativism, where
the absolute aspect of every separate \co{existence} does not carry more weight
than that of the \co{actually} different contents, of one \co{actual} part of
the whole.  Like all the attempts to reduce the \co{absolute} to the \co{actual}
(whether the pantheistic \co{totality}, regulative ideas of reason, truth as the
inaccessible limit, etc.), it results in delegating the \co{absolute} not only
to the sphere of \co{transcendence} (where it indeed resides) but also {\em out
  of} the sphere of \co{immanence}, outside the horizon of anything an
individual might ever confront fully and \co{concretely}.

We follow only the first of the above quotes. The
uniqueness and unrepeatability of every \co{existence} not only does not exclude
the \co{confrontation} with the \co{absolute} but is \co{founded} exactly in it.
Every \co{existence} is a reflection of the same \co{absolute}, not a part of
it, but an \co{absolute} image, \la{imago Dei}. The images may differ without
ceasing to be the images of the same and without implying that only their sum total
catches the glimpse of what they reflect.  \citet{And just as the same town when
  seen from different sides will seem quite different, and as it were multiplied
  {\em perspectivally}, the same thing happens here: because of the infinite
  multitude of simple substances it is as if there were as many different
  universes; but they are all perspectives on the same one, according to the
  different {\em point of view} of each monad.}{Monad}{57. It is not clear to us
  if it bothered Leibniz whether this very statement is or is not only one
  possible perspective -- of the monad actually pronouncing it.}  The
differences of perspectives emerge with the \co{visible} contents where, indeed,
every one may have his own view and understanding. However, ignoring the lowest
and most trivial cases, these differences do
not result from any \co{subjective} choices. At the bottom of it,
nobody decides which perspective he will entertain -- one's perspective is an
integral \co{aspect} of one's \co{concrete existence} and evolves along with
it.\noo{This \thi{givenness} of the perspective is something of the absolute.} 
Above all, it concerns only \co{visibles}, for even our understanding of the
\co{invisibles} and \co{absolute} is only \co{actuality} of
\co{visibles}. Consequently, all these differences 
never sum up to give any whole, because the \co{absolute} is not any
\co{totality} of \co{visibles}. All the \co{actual} differences are equally
\co{founded} in \co{one} and the same, and all \co{existences} are equally
\co{confronted} with it. To reach this \co{confrontation} no addition is needed,
because the \co{absolute} is not any \co{totality} of differentiated contents,
but their prior \co{unity}.\ftnt{Thus, while the usual perspectivism, claiming
  that there are only different perspectives, must end in relativism, our
  \thi{perspectivism} admits not only different views, but also different levels
  of views, which all are only reflections of the \co{one}. \citef{Surely the
    diversity isn't in the thing gazed on, but in the {\em way} of gazing on it
    [...] While the same thing is understood, nevertheless it's not understood
    equally.}{AbelardPJC}{II:\para 325} Dissolution of particular \thi{substances}
  and \thi{essences} notwithstanding, we still (or rather exactly for this
  reason) have \co{one} and only \co{one} to \co{share} and understand. This
  \co{unity} is raised \co{above} all perspectivism.}

%In more intimate context, \co{I} may find some persons valuable and
%enjoy their values, life style, or even life, because these are also
%\co{mine} and so \co{I} am willing to \co{share} them.  A form of
%friendship or even personal love can be expressions of such
%communion.

\pa\label{pa:relativism} The problem is that the \co{unity} of this ultimate
\co{communion} is a mere \co{that}, while one would rather like to find it in
something more particular, in some \co{visible} \thi{whats}. But as no
particular \thi{whats} seem to offer the answer\ldots Relativism is but a
variation on this theme of egocentrism, is but its inversion absolutising the
absence of deeper values which would still be \co{visible}. It is, like
egocentrism, the inability to recognise anything higher than the \co{visible}
divisions, but combined with the inability to sign the doom which narrow-minded
egocentrism pronounces on all otherness. The inability to recognise something
higher is simply a sign of lacking self-respect, of the fact that one's own or
one's own community's values are not so convincing and deep as \co{that} one is
\co{thirsting} for.  Genuine respect in disagreement, recognition of other
values which \co{I} do not share, can be \co{founded} only in \co{sharing} of
their source. If such a source is not found while, at the same time, one feels
uneasy absolutising the \co{mine} with all its historical and social contingency
then, indeed, the only possibility is to state relativity of all \co{visible
  signs}. {\em This} relativity of relativism is not, however, the relativity of
the \co{actual sign} anchored in the \co{absolute origin} but, on the contrary,
the mere arbitrariness of this \co{sign} as opposed to that \co{sign}.
Relativism is grounded in the inability to find anything higher than the
multiplicity of \co{actual signs}.  The problem with relativism (a bit like with
negative theology) is not that it is too extreme but, on the contrary, that it
does not go far enough. Relativism is \co{idolatry} which seems and claims to
have escaped all \co{idols}. It does not put this value/nation/group/\ldots in
front of that, it does not \co{idolize} this, by putting it above that. It does
not. But it \co{idolizes} some level of \co{visible} \co{distinctions} as the
absolute one, \co{above} which no more \co{unity} can possibly obtain.

\newpa
\pa
Now, one can be proud of belonging to one's nation, one can be even willing
to sacrifice one's life for it, but if this nation is the deepest value which
one is capable to recognise then it will easily end in
nationalism of a dubious shade. However, one can be proud of that and, at
the same time, recognise the possibility of others' being proud of belonging to
their nations and even of some not bothering about such a thing at all.  The
conflict between one ethos and another may be of fundamental character but for
the most it is a conflict resting on the absolutisation of \co{visible
  expressions}, of the \co{signs} which merely announce, always only in one
particular form, the \co{invisible presence} which is truly \co{shared}.

One can live thoroughly the values of one's cultural or religious formation
and, at the same time, recognise equally validity of other values.  But
for this recognition to be genuine and honest, one must first find the
true \co{inspiration} in the values one is living, that is, to recognise
their \co{spiritual foundation}. This recognition of, at once, relativity and
\co{absolute foundation}, involves perhaps a bit more self-respect than one is
used to, but is far from impossible.  Now, just like all \co{visibility} is
\co{founded} in the \co{invisible}, so all \co{presence} of the \co{absolute} is
interwoven into the matter of \co{this world}. There is no other way of
\co{participation} than through some form of tradition, historical consciousness
and involvement into the \co{actual world}.  Only through \co{visibility} is the
\co{invisible} \co{present}. \co{My} task is to recognise the
\co{manifestations} of this \co{presence} in the world in which \co{I} live. The
multiplicity of religions is but an expression of the unavoidable
\co{incarnation} of the \co{invisible} in the \co{visible}. And to the extent
these are true religions, that is, to the extent they are based on the
recognition of \co{invisible origin above this world}, they all provide means
for finding a way to rebirth and salvation. The fact that somebody born in Tibet
does it on the way of Buddhism, while somebody born in Europe on the way of
Christianity, does not in any way diminish the ultimate \co{sharing}. This
\co{sharing}, however, does not mean uniform agreement. It is possible only by
living one's own, \co{concretely founded} ethos, for the differences between
cultural and religious formations, relative as they are, are thoroughly real.
Originating in the same, they reach the level of \co{actuality} where all our
\co{acts} take place. To the extent \co{I} live an ethos of a \co{community},
\co{I} can live only one such at a time, and whoever tries to live more than
one, ends up living none.  But there is no need to live more than one,
because each true ethos is a full expression of all levels of Being. If one
does not find it here, one won't find it there, and if one finds it there, one
will see that it has been here all the time.  It takes an analphabet to believe
that the \thi{truth} is written somewhere else. The inability to recognise the
deepest values embodied in one's culture underlies the ecstatic escape towards
\thi{otherness} and its \thi{truth}, and is only a stage of relativism 
%reflecting the inability to recognise \co{sharing} in one's own...
in which one ends up allowing all possibilities, that is, being unable to relate
to any. For where everything is allowed, nothing has any value.  Merely allowing
for other values or forms of worship, in a truly democratic and relativistic
way, is perhaps politically optimal but, at the same time, it amounts to
disrespecting all such values by delegating them to the sphere of everybody's
\thi{subjectivity}.  To respect them, is to recognise their genuine
\co{foundation}, but \citet{we must have [first] given all our attention, all
  our faith, all our love to a particular religion in order to [be able to]
  think of any other religion with the high degree of attention, faith, and love
  that is proper to it.}{WeilWaiting}{Forms of the Implicit Love of God:The love
  of religious practices;p.119} True religion is but a form of ultimate
\co{sharing}, and finding it once, one has no need of finding it once more
somewhere else. But if one did not find it, one keeps looking and, in the course
of deeper and deeper disappointments, one can eventually start believing that it
is nowhere to be found, and that one should better get rid of the ladder, once
one has climbed on it to the roof of this \thi{insight} in the relativity and
mere auxiliary role of all particular ladders.

\noo{ The Jungian \la{anima}/\la{animus} acquire at this level a moral
  character.  \la{Anima}, represented by Virgin Mary, raises love out of the
  sphere of mere instincts to a higher level of devotion. \la{Animus} takes
  often the form of a wise professor who commands the word and is capable of
  moral guidance.  }

\pa\act
The \co{actual} expressions of \co{founded communion} amount to \co{sharing} the
\co{actual experiences}, situations, problems. One would be tempted to mention here
also joys and sorrows, pains and satisfactions, but let us be a bit careful. A
multitude of \co{acts} can express \co{sharing} -- \co{acts} of cooperation and
exchange, of compassion and helpfulness, of criticism and appraisal\ldots
%Let compassion serve as an example. 
Such \co{acts} address the \co{actual} situation involving, perhaps, the other's
problem or achievement. This problem or achievement is not something which
belongs to the other and is privately his, just like {your} problem is not
merely {yours} -- they are just that: problems, achievements, sorrows, joys.
When {you} meet them, they are simply there and become {yours} by arising {your}
reaction.  They are \co{shared} to the extent {you} recognise their objective
character and do not focus on the fact of their belonging only to the other.
The strongest form of such \co{sharing} is actually living the same
\co{experience}, for instance, the loss of a beloved child \co{shared} by both
parents or more trivial examples of a team \co{sharing}, along with the same
goals, the successes and defeats in their realization.  But the same structure
underlies genuine \co{sharing experiences} which, phenomenologically, belong
initially to other person.  Meeting a smile, a joyful spark in the other's eyes,
a happy moment in his life, {you} do not \co{share} it by observing it,
concluding that it belongs to another, and then deciding to participate in it.
Neither do you \co{participate} in it by trying to evoke the feelings and
impressions the other might have. To the extent {you} \co{participate} in it, to
the extent {you} \co{share} it, the fact that it is another's and not {yours} is
thoroughly real, yet of negligible importance: a part of the world, the
\co{actual} situation is \co{shared} and it is neither his nor yours.  And
although \co{reflection} will tell you that there is a sharp \co{distinction},
you know that it is not telling the whole truth.

\co{Sharing} other's joys and happiness which, on the face of it, are entirely
irrelevant for us, expresses genuine \co{communion} for, as we know, it is not
always so easy to overcome the sense of envy which witnesses to \co{egotic}
limitations.  Let us, however, use as a much more common example compassion. It
is not any feeling which has to be aroused in order to reproduce another's pain.
\co{Actual} communion is not a mere empathy, a mere emotional identification
with that which is other's.  Reducing \co{actual} communion to empathy,
misconstrued as entering into another's feelings, is a \co{subjectivistic}
reduction, which not only misses completely the nature of the phenomenon but
also precludes the \co{subject} from leaving its solipsistic universe. In fact,
compassion need not be (though it often is) accompanied by any specific
feelings. Feelings, \co{moods} and \co{impressions} are only \co{signs} -- as
all \co{signs} -- of something \co{transcendent}, that is, not reduced to their
\co{subjective immediacy}. They reveal an aspect of the world and point to
something which can also be revealed in other ways. People with apparently cool
and unemotional personality are capable of perfectly compassionate attitudes and
\co{acts}, no less so than others.  \co{Sharing} expressed in an \co{act} of
compassion need not be accompanied by any specific feelings -- but it must be
\co{concrete}! It must spring from the depth of your person, not necessarily
from any \co{actual} and deep feelings but from the \co{recognition} of the need
of it, of the call from the \co{actual} situation to you.  The fact that one can
genuinely feel {\em with} another, pain or joy, is secondary to the fact that
one relates to the same sphere of the world from which his pain or joy arises.
One does not feel {\em his} pain, and one does not even try to {\em imagine} it.
One feels {\em one's} pain which participates in the same -- \co{shared} --
painful experience as does his pain. Whether this experience is given also
through other forms (for instance, one knows why he is in pain, or even why he
should be in pain even if he does not seem to be) or only through his painful
expression, is only of secondary importance. For compassion addresses {\em the
  same experience}, of which another's pain is only a \co{sign}, his reaction.

As it happens, \co{shared} pain diminishes. \citet{Pain is alleviated when
  friends share the sorrow.}{AristNico}{IX:11 [1171a]} Of course, one might say,
the pain of the one who suffered first, but not of the one who joins in. But no,
the pain of both or, let us put it this way, the \thi{total amount of pain}.
Pain and suffering is not any \co{invisible} truth which only increases by being
\co{shared}. On the contrary, particular pain -- just like money -- diminishes
when it is \co{shared} with another, it becomes divided between all who
\co{share} it. For the other, who comes with compassionate support, it
diminishes to the same extent as it diminishes for the one who was suffering.
For what \co{motivated} compassion was pain which he experienced, perhaps in a
very different way from the one \co{actually} suffering, but still entirely and
really. Moreover, genuine compassion comes with the voluntary acceptance of the
pain. This acceptance does not intend any compassion. It is a mere \co{act}, a
mere answer to the call which is \co{founded} in -- we may say, dedicated
exclusively to -- the ultimate \co{communion}, that is, \Yes\ for which pain is
not evil but trial.\ftnt{Avoiding pain is an obvious reaction and it could be
  used to motivate avoiding compassion which amounts to \co{sharing} it. The
  above says that, on the contrary, \co{sharing} does not amount to seeking pain
  but only to its diminishing. True compassion results always in help. Moreover,
  there is a difference between avoiding and escaping from, between not looking
  for pain and turning away in the face of it. In the latter case it is already
  too late and this is the case were compassion is called for. Escaping from
  \co{actually} encountered pain (of another's as much as of oneself) is like
  escaping from suffering which, as said in
  \ref{sec:evil}.\refp{pa:evilsSuffer}, is a source of \co{alienation}.}

\noo{And what is compassion? You will know when you see it.
  
  Feeling is here only a \co{sign} of the recognition of the \co{shared}
  reality.  \thi{Co-feeling and co-experiencing another's emotional state} is
  only (and only at best) a \co{sign} of \co{sharing} something more than just
  this state.  }

% Compassion is all embracing because it does not divide people into those who
% reached the adequate spiritual level and deserve understanding and
% those who did not and therefore have only themselves to thank for
% their problems.
\pa In a given situation, compassion is directed exclusively and completely
towards the person and yet, in a sense, it is completely \thi{impersonal}. This
\thi{impersonality}, however, is an expression of the deepest respect for the
person.  The situation where, for instance, I act compassionately towards the
suffering person but only because and in so far as he is a member of~\ldots my
family, my group, my nation, is not an example of compassion but of its
misunderstanding.  The suffering person is the \co{absolute} center of the
situation and compassion is an expression of the ultimate \co{communion} -- with
{\em this} person. \citet{It is not so that <<compassion -- as such -- is
  shameless>>, as Nietzsche says, but compassion without love towards the one
  whom we compassionate. [...] Therefore we notice also that every expression of
  compassion {\em without} love to the person is felt as a
  brutality.}{Sympatia}{A:XI\kilde{p.225-6}} \thi{Love to the person} is the
\co{concrete foundation} of \co{actual} compassion. Its ultimacy, or as we said
\wo{impersonality}, means only that if the suffering person happened to be
somebody else, the compassion would still be the same. In particular, this
expresses only the \co{concrete foundation}, not any universality of compassion,
of which the present person would be only \thi{an instance}.  Compassion is a
property of \co{acts}, not of life.  It is not like \co{love} which, underneath
every \co{act} and \co{activity}, extends to the whole world.  Suggesting such a
universality of compassion, as done for instance in \citeauthor*{Unamuno},
amounts also to suggesting that the whole world is in a soar need for it, that
the whole world is a scene of all embracing misery and that life has only tragic
sense.  Such an exaggerated compassion, a category of \co{actuality} applied to
the whole world, is but an exaggerated feeling which comes closer to patronising
in its lack of the basic \co{thankfulness}.  Compassion does not pity anybody
nor anything, for pity hides some lack of respect, we could perhaps say, pity is
compassion without respect. Compassion does not pity the tragic sense of
life, the unbearable and unavoidable involvement into the evil of the world, the
corruption of one's soul.  It arises only in \co{actual} situations which call
for it and, otherwise, knows that everything is a \co{gift}, though some of
these \co{gifts} may be harder to carry than others. Respect for the
suffering person resides in the acknowledgment of this truth and in the
communicated conviction that he, too, acknowledges it. 

\pa \co{Concrete founding} of \co{acts} is often expressed by an \co{inversion}.
Love, just like friendship, can point out what it perceives as mistakes or
failures on the other's part.  Recognising and apprehending the whole person, it
may (sometimes even should) lack the unconditional acceptance of everything the
other person does. Critique and disagreement is possible -- in full friendship,
or love, and appreciation of the person -- only because the \co{community} is
not reduced to the level of the \co{actual} situation and one's feelings,
thoughts, acts.  Only because it is \co{founded} in the recognition of the
personal value, in the deeper \co{community} of values and, eventually, of the
\co{origin}, it can judge the \co{act} without judging the person.  Thus, for
instance, help can take the form one would consider harmful, a good advice may
happen to be the opposite of what one wanted to hear and expected. If one
could anticipate it, one would not be in such a dear need of it, and so it would
not be {\em such a good} advice. When one hears it, one may not understand it,
one may not know {\em what} makes it a good advice; but then one also know {\em
  that} one should probably follow it.
In many situations %like those of particular hardship or confusion,
there will be much more friendship in saying right things which the other does not
want to hear, than in the flat acceptance of everything the other does.

\noo{ \thi{Other} is an \co{aspect} of the dialogical character of
  \co{existence}, of \co{communion}. We are considering the modifications of the
  former only to the extent they follow the modifications of the latter. The
  \co{actual community} is not limited to other persons but embraces most of
  \co{this world}.  The level, or the depth of \co{communion} depends on the
  possibility of \co{sharing}. It is only at the personal level, and only with
  an individual person, that the full \co{communion} is possible.  The less we
  can \co{share}, the lesser the \co{communion}.

  
  The intricacies of the possible biological classifications notwithstanding,
  all animals and living organisms belong, too, to the common \co{origin} and
  although stay \co{below} \co{me}, open the possibility of \co{sharing}.  This
  may be \co{expressed} as care and respect for their being, as a recognition of
  the value which is inherent in their being, a powerful example of which was
  given by St.~Francis of Assisi\ftnt{For example, in the \btit{Prayer for
      animals} he says: \woo{Give us the grace to see all animals as gifts from
      You and to treat them with respect for they are Your creation. We pray for
      all animals who are suffering as a result of our neglect.}{FrancAnimal}
    One might be tempted to see in (the reported) objection of Pythagoras' to
    somebody beating a dog -- \wo{I recognise my dead friend in him!}  --
    exactly the same attitude, not necessarily an expression of the faith in
    metempsychosis.}, and which Buddhists try to live almost literally. Now,
  this might sound almost like \citet{Arise and drink your bliss, for every
    thing that lives is holy!}{BlakeAlbion}{} But we should remember that just
  like holiness of a holy person is not his property but the state of
  \co{concrete participation}, so too this \thi{holiness} of living things is not
  their property, and even less any property of some ecstatic relation to them;
  it is an expression of \co{sharing} the \co{origin}.  }
%\tsep{unfounded... = absolutisation of subject}

%unfounded: allowing & accepting everything - infatuated woman, or confused parent
\pa The \co{unfounded} community of mere \co{actuality} will often do the
opposite and acclaim everything the other does, like confused parent or teacher
following guidelines of all too liberal pedagogy. One feels forced to accept
every \co{actual} wish and expression of the other because, without any deeper
\co{foundation}, everything one is able to relate to are \co{actual}
expressions, whether of genuine needs or of mere whims.  Thus, although one
still strives for a deeper community, one remains perplexed by the underlying
absolutisation of the \co{actual} feelings and \co{subjectivity}.

Without such a deeper \co{foundation}, \co{ego} remains a \co{dissociated} atom,
a pure \co{subjectivity}, reduced to the privacy of its \co{actual} feelings and
thoughts which -- not only etymologically -- amounts to
privation.\ftnt{\wo{\la{Privare}} means \thi{to deprive} and its passive
  participle, \wo{\la{privatus}}, serves also as adjective and noun.\noo{. = p.p.
    of privare (deprive); privationem}} \co{Ego} without \co{concrete
  foundation} is the archetype of \co{alienation}, and an image, a frequent
\co{actual} form of that is loneliness. Loneliness is another side of privacy, a
result of the other having been pressed outside the sphere of \co{my} privacy
and reduced to a mere aspect of \co{my} situation, perhaps, still an active
subject but not one to whom \co{I} have any personal relation -- the other
becomes a mere \thi{he}. It can be an anonymous adversary in a situation where,
although himself present, he functions only as a \thi{third person}, a mere
factor in the \co{actual} game.  This is the way we often relate to clerks or
salespersons in offices and shops, when the whole contact is reduced to an
impersonal relation dictated by the actual context. The communication is a mere
matter of routine exchange or else of gaining control over the external factors
which, accidentally, can also be embodied in the other person. Exactly the same
superficiality characterises communication at the cocktail-parties which have the
more jolly and merry surface, the more despair and hardship is trying to hide
under it. One often seeks a \thi{merry company} as a medicine against bad
\co{mood}. In such a company, however, one does not so much \co{share} the good
mood as is infected by it. Such a contagious, often heavy, atmosphere
characterises all forms of \co{actuality} in which the participants, wishing for
a community, almost force themselves to \co{actually} share it, overcoming the
felt \co{alienation} in the depersonalising power of a trance: a discotheque, a
politically agitating meeting, an orgy, a gambling hall, a sermon of a
televangelist.\ftnt{The classic of \citeauthor*{LeBon}, provides excellent
  descriptions of the involved mechanisms. }
\noo{
In a more pragmatic vein, one can recognise that goals and projects can be
shared between people, that one can meaningfully -- which here means, to one's
advantage -- engage in common activities and pay due attention to less
private issues.  What can be \co{shared} here are goals and objectives which can
be common to several people, perhaps, to whole parties and communities. A
typical example, besides political parties, would be mere interest or discussion
groups, where the participants might equally well remain anonymous if only they
could provide the same contribution to the common agenda.
%(E-mail lists and chat-rooms come all too easily to mind.)
}

Sartre's novels and plays provide extreme examples of such a community which is
a mere multiplicity of \co{alienated}, mutually \co{external egos}.  \btit{Being
  and Nothingness} gives a systematic description of the absolutised
subjectivity failing to establish any meaningful form of \co{community}, of the
constant attempts of the conscious for-itself to turn another into an
objectified and devitalized in-itself. Even love reduced for the poor man to a
mere master-slave dialectic between \co{dissociated subjectivities} trying to
subdue each other.  This, of course, led nowhere as benefiting from another's
submissive acts establishes perhaps dependence, but not any \co{communion}. And
even if the other's acts are voluntary and made of good will, they do not
necessarily open for \co{me} the door to \co{participation} in any
\co{communion}. For \co{concrete participation} requires a \co{choice} and is
thoroughly active. We own only what we give.\label{own-give} The strongest bonds
knit us not 
with people who did us good but with those to whom we did some good -- and the
bonds are the tighter, the more good we do to them.  Receiving gifts or
services, made of good will alone, is a much more difficult art than one
commonly imagines. Genuine freedom (\refpf{pa:invfreedom}) is required to
receive gifts without becoming inferior (for shamelessness can protect only
against the mere {\em sense} of inferiority).\ftnt{The custom of potlatch
  (\citeauthor*{Mauss}), amounting to almost destructive competition in
  surpassing the generosity of the received gift by the returned one, can be
  viewed as a social expression of the psychological dependence of the
  recipient.  On the other extreme, there are -- more advanced -- societies so
  thoroughly lacking \co{concrete community} that the fear of owing anything
  makes it almost unthinkable for an individual to receive any minor service, a
  cigarette or a lift, from a stranger without immediately attempting to pay
  one's debt.} When such a freedom is missing, obtaining more from others is
weighted only against giving more from oneself, with the resulting attempts to
either subdue others or to protect oneself against them -- in either case, the
\co{alienation} of increasing loneliness.

\pa
At the social scale, the lack of \co{concrete foundation} results in the total
anonymity which invades the threatened and \co{alienated ego}. As
\co{alienation} is the loss of \co{concreteness}, so community gets now reduced
to a mass of statistically anonymous individuals. The fear of anonymity
is but a \co{reflection} of the missing sense of \co{community}.\ftnt{We are
speaking about {\em both} the anonymity of a crowd of faceless units {\em
and} the anonymity of \co{my} being lost in such a crowd. The two are
\co{aspects} of the same anonymity.} The apparent 
medicine against it is \ldots success, public recognition and attention which
puts a photograph in place of the lost face. %and lost individuality.
This has also a deeper aspect. Dreams of exceptional achievements, of leaving
one's mark for the future development, of becoming
socially/politically/scientifically/\ldots respectable, as one's own monument in
one's home town -- all such \co{egotic} thrills, which often indeed form the
ground for outstanding achievements, are but expressions of the \co{thirst} for
\co{community} which got reduced to the purely \co{egotic} hope of establishing
an exceptional, even if only ephemeral case against the background of
statistical mediocrity. As if the most common sign of mediocrity were not
exactly the dream of not being mediocre, of being exceptional. To be exceptional
is \co{ego}'s only dream, its degenerate expression of the \co{thirst} for
becoming \co{self}.  Although such dreams express the need to confirm uniqueness
of one's \co{ego}, their second bottom is \co{thirst} for \co{community} -- for
the confirmation of uniqueness can for \co{ego} happen only by way of the
public, \co{ego} must become a \la{persona} (in the sense of Jung), as it
mistakes the common universality for its lacking \co{concreteness} and public
respectability for its lacking self-respect.


The anonymous crowd, which an individual confronts instead of a meaningful
\co{community}, is not a simple result of the increased numbers, of the mass
pseudo-culture which flattens and deindividualises the social sphere. It is
rather the other way around -- the anonymous mass confronts individual who
dwells exclusively at the level of his \co{ego} and, consequently, loses the
\co{concrete} sense of belonging to a \co{community}.\ftnt{We are not trying to
  negate the statistical prevalence of negative and alienating effects of
  various social diseases, like inhuman working forms, depersonalised public
  sphere or its spiritual emptiness. We would perhaps suggest that, to begin
  with, such forms are rather expressions of the \co{egotic} mentality raised to
  the social norm.  But most importantly, alien and anonymous crowd may confront
  individual under any circumstances, if only the individual reduces his life
  project to the level of \co{ego}. Also under unfavorable social conditions
  (like those just mentioned), it is eventually the individual himself who has
  to consent to the reduction of person to \co{ego}.} The \co{unfounded
  community} at the level of \co{actuality} is an anonymous crowd.

%%% moralism - unfounded egotic community...?

\noo{Jung's \la{anima} of this level is
compared to Faust's Helen, but she is indeed only an object of
feelings which, their possible aesthetic or even romantic aspect
notwithstanding, are dominated by sexual drive.  \la{Animus}
represents initiative and planned action and is often symbolised by a
highwayman, as gentle and cruel as he is smart, or else a brave,
strong macho, sometimes, with a flavour of a cunning politician.
}

\tpa{Communication.}
\co{Sharing} the \co{signs} and \co{actualities} amounts to communication. We
say \wo{\co{sharing}} because although all \co{signs} are 
\co{actual} and situations are  
common to more people, it does not mean that every \co{actual}
situation involves and every \co{sign} is a genuine communication. The
\co{unfounded} communication would be 
a mere transfer or exchange, and not \co{sharing}, of the \co{signs}. The
\co{immediacy} of \co{signs} makes them perfectly amenable to direct
exchange. But communication is not an event of exchanging \co{signs}, not even
of exchanging them according to some specified rules and protocols. It is not an
event of exchanging any \co{signs} but of comprehending
them, of recognising the \co{shared} reality through and beneath the \co{actual
signs}.

Communication is a possible \co{visible} \co{manifestation} of \co{sharing}; the
more we \co{share} with others, the easier and more complete is the possible
communication.  We do not communicate that well with bacteria or butterflies.
They have quite different structure of \co{experience}; their world has few, if
any, common points with ours.  There is probably close to no overlap between
ours and butterflies'.  We communicate a bit better with cats or dogs; we and
they perceive some of the same things as obstacles, we also find in them more
advanced expressions of \thi{feelings} than can be found in ants or butterflies.
Their \co{experiences} {cut} the background along the lines sufficiently
similar to ours and provide them with a lot of things which we too
\co{distinguish} and \co{recognise.} So, perhaps after all, \citet{if a lion
  could speak, we might understand him,}{WittPI}{\hspace*{-.2em}(modified)} though it
certainly would not be the same degree of communication as we can achieve with
any human being. And, of course, we communicate very differently with different
people.  \citet{Each word means something slightly different to each person,
  even among those who share the same cultural background.}{MHS}{\kilde{Jung;}p.28}
This, however, in no way makes communication impossible, in fact, it is what
makes it different from an exchange of information bits.  We can understand
words which for another mean something different because we \co{share} the
reality to which they refer. Eventually, communication is like pointing and
eventual answer to the question \wo{What do you mean?}, after a series of
clarifications and explanations, is simply \wo{{\em This} is what I mean, just look
  {\em here}.} It is an event of narrowing down the \co{shared} horizon
(of humanity, language, culture, personal experience) to the \co{actual} content and, by
the same token, of endowing this \co{actuality} with meaning -- the meaning of
belonging to the \co{concrete} and \co{shared} horizon.

Perfect unambiguity of expressions and ultimate \co{precision} of the language
is the domain of computer programming but whenever something is communicated, it
can also, at least in principle, be misunderstood. The possibility of
misunderstanding is a necessary condition of a successful communication, that is,
of conveying some meaning which goes beyond the \co{immediacy} of the \co{sign}.
It is only the lack of \co{shared} background which makes \co{precision} of all
\co{signs} ultimate necessity -- for where nothing is \co{shared} one can only
exchange \co{signs}. As Wittgenstein observed, even an ostentive definition
would be impossible without sharing enough to understand {\em what} the other is
actually pointing at.  \wo{Whoever has seen, knows what I am saying} was a
phrase used by a mystery-initiate when addressing others: perhaps, to avoid
divulging secretes but more probably because details would not help those who
have not seen.  We \citet{need not be surprised if only those ideas which least
  belong to us can be adequately expressed in words.}{BergTime}{II\kilde{p.136}}
And by \wo{adequately} one likes to mean unambiguously, \co{precisely},
excluding any possibility of misunderstanding.  According to Kierkegaard, there
is no direct communication, and although he too would limit this statement to
the deeper truths of the genuine faith, we would extend it to all communication.
For communication is conditioned by \co{sharing}: in its \co{presence}, many
different \co{signs} or words may be used, while in its absence no words will
result in communication.

This remains valid through all the levels. The most intimate communication is
\co{founded} in the most intimate \co{communion}.  It is only at the personal
level, and only with an individual person, that full \co{communion} is possible
and it leads to very specific ways of communicating, intentionally as well as
not, most intricate aspects.  But they are communicated not due to any univocal
\co{precision}, not due to universal adequacy of the used \co{signs} --
\co{signs} are here always inadequate.  Just like a \co{symbol} may seem an
almost arbitrary and accidental representation of the symbolised reality, so
here too, an apparently most insignificant word, a mere look, a sheer grimace or
gesture, a casual phrase, can \co{actually} carry the deepest meaning. It does
not, however, happen because the \co{signs} somehow carry this meaning in them,
but only because the possibility of this meaning is \co{shared} before it has
been pointed to. The art of communication does not consist in the ability to
interpret the unclear \co{signs} by narrowing their meaning to the most
\co{precise} content, but rather on the contrary, in the ability to use the
\co{actually precise signs} to grasp the \co{imprecise} (and often \co{clear})
meaning which \co{transcends} their \co{immediacy} and which they are trying to
unveil.\noo{This applies equally to any contents which, no matter how
  \co{actual} and \co{precise}, always \co{transcend} the \co{immediacy} of
  their \co{signs}.}


\pa Wittgenstein has often asked questions like: How can I be sure that
saying \wo{green}, I and you understand what is being said, in particular,
understand the same thing? The problem is of course with \thi{the same}. But
this problem arises only when one is committed to some form of psychologism, to
some private impressions and ideas which somehow live within one's
\co{subjectivity}, and which get, rather mysteriously, transfered between the
monads by words like \wo{green}.  But every event of a successful communication
is \co{founded} on \co{recognition} -- the \co{actual} and mutual
\co{re-cognition} follows only the \co{recognition} of \co{shared} background.

If by the meaning (of \wo{green}) we understand some intrinsic properties, some
impression or idea of greenness perceived or imagined {\em inside} one's head,
then indeed, it may be impossible to be sure.  But \thi{green} is only a
limit of \co{distinctions} made in \co{one}, made by you {\em and} me in {\em
the same} \co{one}. The limits may vary from person to person (just like Prague
may end at different points for different people, just like stripes of the
rainbow interpenetrate) but these variations retain a major
overlap. We agree on the use of the words because we \co{share} the common
reality and structure our \co{experiences} in similar ways. (As we said earlier,
I:\refp{pa:languageA}, learning a language itself contributes
significantly to, but does not determine, this structuring.) In the same way,
though to a much lesser degree, we agree on the use of various \co{signs} with
dogs.  Communication, \wo{this influence of one monad over another is only {\em
ideal}, and it can have its effect only through the intervention of
God.}$^{\ref{ft:commGod}}$ This \thi{intervention of God} is not, however, any mystical
interference of some magical power, but the fact of, at first, only ontological
\co{foundation} in the \co{one}, of having the same \co{origin}, and then also
of \co{concretely sharing} the same reality.\ftnt{Plotinus thus describes the
  souls in their intellectual dimension: \citef{all their act must fall into
    place by sheer force of their nature; there can be no question of commanding
  or of taking counsel; they will know, each, what is to be communicated from
  another, by present consciousness. Even in our own case here [below], eyes
  often know what is not spoken; and There all is pure, every being is, as it
  were, an eye, nothing is concealed or sophisticated, there is no need of
  speech, everything is seen and known.}{Plotinus}{IV:3.18}}

One might still wonder: we agree on the use of the words, but do we agree on
their meaning?  \citet{With most names, we've come to know which things they go
  together with from their use in speech, although we are unable to determine
  what the correct meaning or understanding of them is.}{AbelardPJC}{II:\para
  399. Let's only remind that in the medieval grammatical theory \thi{names}
  included not only nouns but also adjectives (the identification reflecting the
  corresponding phenomenon of the Latin language). It would be hard to imagine
  why, in the quoted sentence, one could not allow also verbs, adverbs,
  etc.\label{ftnt:latinNames}} A variant of empiricism, say \wo{linguistic
  empiricism}, would attempt to reduce the latter to the former, even to replace
\thi{meaning} by \thi{use}.  The more the mood of such a project seems different
from that of behaviourism, the more surprising is the similarity of the goals
and procedures.  \citet{Why is it not possible for me to doubt that I have never
  been on the moon? [...]  But if anyone were to doubt it, how would his doubt
  come out in practice?  And couldn't we peacefully leave him to doubt it, since
  it makes no difference at all?}{Certain}{117/120} We certainly could but, the
absence of any observable difference in social praxis notwithstanding, could one
reasonably claim that there would be no difference for the person having such
doubts? The \thi{difference for the person} need not have anything to do with
what the person \co{actually}, practically does.  One can deny any such
difference for the involved person only by denying the reality of the doubt, or
in a more extreme case, by claiming its impossibility, perhaps, on the basis of
the impossibility of the private doubt and, eventually, meanings. The linguistic
empiricism tries to dissolve the phantom of the extra-linguistic meaning in the
inter-subjectivity of the language usage or social praxis.  But \co{sharing} is
much more than merely obeying similar rules of social or linguistic praxis. If
we did not \co{share} anything, how could we even agree on the consistent use of
any rules? Just like \co{communion} requires a distance, so \co{sharing} some
reality requires this reality to be \co{present}, as if independently, with all
who are \co{sharing} it.\noo{If neither of us had anything, then getting
  together would not result in anything new either.} It is only because we all
\co{share} most of the world (practically or impractically, \co{actually} or
not), that \co{actual} communication -- transfer of and agreement on some
meaning -- is possible and may even result in new forms of
\co{experience}.\ftnt{Davidson's principle of charity, in all its variants, is
  an excellent expression of this dependence of communication on \co{sharing}. A
  vulgar analogy from computer communication can illustrate the point in quite
  precise terms.  The transfered bits have no significance unless they are sent
  and received by programs obeying some protocols which in advance determine the
  scope of possible communications, as if, the shared space -- of use and
  interpretation.  But even following agreed rules is not sufficient for
  communication beyond mere transfer of messages. It is impossible to establish
  so called \wo{common knowledge} without the assumption of a prior sharing of
  some information. The idea is: you and I want to reach an agreement on the
  issue \thi{$X$ or $Y$?} but so that each of us knows that we both know that we
  have reached it. We do not, however, share anything except some communication
  channels through which we can send messages (say, by post which, to simplify
  everything, is $100\%$ reliable, though it does not guarantee any time of
  delivery -- I do not know {\em when} you will receive nor that you have
  received my message, unless I obtain a confirmation from you). Suppose I
  prefer initially $X$ and send you the message $1:X$. Suppose you agree. I do
  not know that, so you should respond, confirming $2:X$. Now I know that you
  agree with me on $X$, but you do not know that I know that. So, I have to
  confirm the reception of the confirmation, sending $3:X$. Now, you know that I
  know, but I do not know that you know that I know. And so on. (The scenario
  corresponds to theorem 6.1.1 -- in \citeauthor*{ReasoningKnowledge} -- which
  precludes the possibility of achieving common knowledge of anything having
  been delivered. Stronger versions, e.g., theorem 4.5.4, apply to slightly
  modified situations where {\em no} common knowledge may arise, as a
  consequence of complete asynchrony of the components.) Misusing such arguments
  for our purposes, we could say that in the absence of initial sharing, it is
  not possible to establish it either, even in the presence of most reliable
  communication, understood as mere transfer of messages (whether mere bits,
  pictures, \thi{mental} meanings or deepest ideas). Technically, one reaches
  the fix-point of the appropriate functional (where everybody knows that
  everybody knows that everybody knows that\ldots) only as the infinite limit
  which, misinterpreted in practical terms, means that it is unreachable. In
  short, if we do not initially share knowledge, the mere exchange of messages
  will never lead to achieving common knowledge. But {common knowledge} of
  agreement arises trivially if we, for instance, share the same location and
  both point at the $X$ (observing each other's action) or, as is usually the
  case, when we speak directly to each other. In the above example, if we
  initially shared the knowledge that each message is in fact delivered after,
  say, 1 minute, then 1 minute after I sent $1:X$ we would achieve common
  knowledge of $X$.\noo{(These cases correspond to some form of synchronicity
    which, as the respective theorem shows, makes reaching common knowledge
    possible.)}}

\noo{ \subpa To some extent, all such discussions depend on the definitions or,
  what in practice amounts to the same, the intuitions about the meanings of the
  fundamental words, in this case, the word \wo{meaning}.  In order to stay in
  touch with the intuition that we sometimes say (unintentionally or even
  unconsciously) something different from what we mean, that we often mean
  something before we speak, or even realize what we had meant only after we
  have said something (different), let us say that the meaning of a linguistic
  expression is constituted by the \co{distinctions} it addresses. This is
  \co{vague}, but hopefully not much more than the intuition of the meaning
  itself. The addressed \co{distinctions} are, of course, addressed in different
  degrees and organised accordingly. There is the context which narrows down the
  possible ambiguities; then, there is the explicitly addressed \co{complex},
  and behind it a possible \nexus; eventually, the \co{pure distinction}, the
  mere fact of distinguishing.  The content, the possible positive
  determinations of the possible meanings of an isolated phrase taken \thi{in
    itself} have close to no limit. They will always depend on what is called
  \wo{context}, on the set of \co{actually} addressed, \co{actual} as well as
  \co{non-actual} \co{distinctions}. It may be true that there is some kernel,
  kept more or less constant through all the contexts of usage. But a fixed,
  constant and \co{precisely} defined meanings are as plausible as eternal
  \thi{essences} of things, and identifying the meaning with such an invariant
  kernel would be like looking for the \thi{essence} of a thing through eidetic
  variation.  We certainly might find something in such a process but calling it
  \wo{essence} or \wo{meaning} would amount to a normative definition. It is the
  idea of \co{precise}, unambiguous meanings (just like earlier of similar
  \thi{essences}) -- which eventually means, the idea of reduction -- which
  underlies the search and has always only disappointments to offer.  }

\citet{If you are not certain of any fact, you cannot be certain of the meaning
  of your words either.}{Certain}{114} Allow us to reformulate it as follows:
If you do not \co{recognise} any \co{distinctions}, the mere words and
  their exchange will not teach you that either. Every
expression draws some boundary, either a boundary which, to some degree, already
was there (as in descriptions or clarifications) or one which appears only with
this expression (as in speech acts).  The \co{meaning} arises between the \co{actual}
utterance/reading of the expression and the background which acquires a
determination. The meaning of a statement, or any word, is not
\co{subjective} because to be \co{meaning} it must \co{transcend} the mere
  \co{immediacy} of the \co{sign}.
It is not private either because, at least in principle, \co{I} am never the
only person able to recognise it. And this is the case because all boundaries
are drawn in the \co{shared} reality, eventually, in the \co{indistinct}
background of the \co{one}.

But \co{I} may have doubts which have no consequences for others (nor, in
practical matters, even for me), \co{I} may go around meaning something which
\co{I} never manage to communicate to others, \co{I} may spend half, even the
whole, life intending something which \co{I} never manage to express. If this
were impossible, communication would be impossible also, or else, communication
would not be a \co{reflection} of genuine \co{sharing}, would not be a
conveyance of meanings but a mere exchange of labels, a mere transfer of
\co{signs}.

\pa Just like the \co{concretely founded} communication rests on the eventual
\co{sharing} and conveys it underneath everything it \co{actually} communicates,
the \co{unfounded} communication is determined by its lack, that is, the
constant fear of failure. \co{Sharing} is reduced to universality and any
possible meaning to its \co{visible} expression. One searches for some
\co{actual} and common basis which, as we learn from many attempts, it is
impossible to circumscribe \co{concretely}, let alone specify \co{precisely}.

The \co{idol} of \thi{rational argumentation} can serve as a common
example.\noo{We object unreservedly against the absolutisation of rational
  argumentation, and only secondarily against identifying it with the basis of
  meaningful communication.} Trying to convince oneself and others that we are
all first of all rational beings (whatever that means), one postulates some
ideal goal of rational morality consisting in the unreserved acceptance of
rational arguments. One may even insist that it recognises the dignity of humans
paying all due respect to their value -- which happens to be the same as the
value of their rationality. Now, rationality can be a valuable aspect of one's
attitude, if only we take it in some rather vague sense, for instance, as
accepting every statement (even this one!) as having only limited validity, or
else as reacting in a manner proportional (\la{ratio}), \thi{in number and
  measure}, to the actual situation.\ftnt{In this generous sense, rationalism
  can be seen as a continuation of the Biblical emphasis on the importance of
  the number and measure according to which all things were created, as well as
  of the Homeric, and almost certainly older, tradition according to which
  impersonal destinies set the inviolable limits even for the gods, cf. footnote
  \ref{ftnt:Moira} in Book I.} But here we are objecting only to the \co{idol}
of rationality which reduces it to the logical precision of the arguments.

Let us ignore the fact that there is hardly anything, hardly any action or
attitude, which could not be supported by plausible arguments. The problematic
issue concerns rather {\em which} arguments one is willing to accept. The
infinite regress appears as in most cases when \co{reflective precision} tries
to make itself self-sufficient. But, certainly, in various contexts (of which
the academia may serve as the paramount example), openness to other's arguments
is a matter of professional ethics.  In life, one can also occasionally learn
something from listening to other's arguments.  But when raised to the level of
the fundamental principle it becomes a caricature of genuine communication.
Have you ever been convinced by an argument?  That is, convinced not in some
petty matter of the choice between well-defined alternatives, of coming up with
a solution to a particular problem or a scholarly discussion, but in a matter of
significance, in a matter which you recognise as having existential relevance.
If one believes in God, is it because of an argument?  If one does not, is it
because of an argument? Doubtful, very doubtful. In this last case, it may
rather be because one does not find any argument, and rests satisfied with one's
\thi{rationality}. Arguments in such matters come only as a \la{post factum}
support for the conclusions which were accepted before the arguments started. In
such matters, there are no forcing arguments, or rather, no sufficient reasons;
at best, there are only clarifications of meanings, accounts of experience.
When arguments are applied beyond the sphere of \co{precisely} defined,
\co{actual} problems, they either become an intellectual game or, when taken
seriously, boil down to one thing: \wo{Either you are stupid since you do not
  recognise the binding force of my argument, or else you are respectably
  rational and accept it.} Argumentation and persuasion, when taken to the
extreme and {absolutised} are much closer to brute force than they are willing
to admit.\noo{We won't mention those who see in liberal democracy \thi{the best
    political system} known from history since it appeals to arguments and not
  force.  How can one call the majority vote, where a vote of a professor counts
  as much as that of a farmer, argumentation? How can one compare shows of
  political sophistry and demagogy, or for that matter advertisement's
  stupidity, to argumentative discourse?\noo{e.g. Habermas, {\em Strukturwandel
      der \"{O}ffentlichkeit}, V-VI} Well, perhaps one can because they are not
  so~\ldots totally different?  Blindness can be hard to distinguish from
  confusion.  Did not Hegel, the incarnated spirit of reason, make similar
  claims about the highest possible perfection of the Prussian state in {\em
    his} time?}  Appeals to some ideal attitudes or limits, like the
communicative reason and rationality, the {undisturbed rational communication},
{tolerance for the opposing views}, etc., are unable to cover up the underlying
disrespect for the human being -- for the {\em whole} human being. The calls to
assuming a respectful attitude towards the opponents become necessary, because
it has to be {\em added on the top} of all the arguments, like a meek tablecloth
covering a dirty table.\ftnt{As always, we are addressing only the personal
  aspect, not any forensic or socio-political ones. Now, giving arguments, as
  seems to be done here, for the irrelevance of arguments in the matters of
  ultimate concern, can face the charge of involving a self-referential
  paradox.\noo{Indeed, giving such {\em arguments} would lead to a paradox.} But
  we are not looking for arguments, that is, we do not attempt to prove and
  convince. We are only giving an account of a view which, we believe, to the
  extent it is acceptable, is so by the force of its contents and not possible,
  better or worse, arguments. Which aspects of this account are taken as
  arguments and which as argued for is a matter of decision. And no matter which
  are chosen as arguments, they do not rest on any foundation which would be
  more solid than the presented views themselves. Unlike giving arguments, which
  try to reach the immovable answers to all the \thi{whys}, giving an account
  leaves such \thi{whys} in a suspension. It does not unveil any
  incontrovertible mechanism underlied by any logical necessity, but merely
  tries to say \thi{what}, to clarify some distinctions in a \nexus\ of
  interrelated \co{aspects}. }


\pa A different example can be that of groups establishing and requiring the use
of secret codes.  Establishing {private codes} of communication -- words,
gestures, expressions which carry the full meaning only to those initiated in
the {community} -- is characteristic for lovers and close friends. But
their privacy is different from secrecy.  In the extreme cases of secret
organisations such {codes} are established for the purpose of hiding the
secrets as well as for the confirmation of the identity of the community and its
members. Secret initiation rites, secret rituals follow the clandestine
operations and hidden purposes. In many situations, secrecy may be
understandable and even justified (as, for instance, in the cases of
organisations opposing aggression, political oppression and the like) or less so
(as in the case of criminal organisations or mere fear of openness). But in
either case, the secrecy of the codes signals the broken \co{community}, the
impossibility or unwillingness to \co{openly found} the communication across the
social, political or even personal borderlines.

Secret codes, being codes, insist on the strictness of the rules -- the
unambiguity of the greeting sign, the rigidity of the ceremony, the
impossibility of deviation from the predetermined sequence of acts or
formulae.\ftnt{Rituals can be, of course, truly \co{symbolic} expressions of
  \co{invisible} meanings, but the problem is that they are not always so, and
  we are now speaking about the cases when they, having lost the \co{commanding}
  power, are reduced to the mere observance of the rules.\noo{Mere \co{signs},
    their possible \co{precision} notwithstanding, are usually insufficient to
    distinguish clearly between the two kinds.}} The less or the shallower is
the meaning to communicate, the greater the need for rigid rules, for their
\co{precision} is the last thing which may give an impression of
inter-subjectivity, of sharing anything with the others -- provided they follow
the same rules!  Sick cases are extreme examples of such a reduction to the
level of \co{egotic actuality} where the ability to follow sequences of sterile
and \co{precise signs} seems the last residual of communication.  Clang
associations (\wo{real, seal, deal, heel}) or irrelevant, though possible,
associations (a person sending a new year's greetings and wishing another a
fruitful year, ends with the wishes of good apple-year and pear-year, and then
sauerkraut, and cabbage year\ldots); difficulties with abstract reasoning and
the resulting literal/specific interpretations (e.g., the proverb \wo{A bird in
  the hand is worth two in the bush.} gets commented: \wo{If you are able to
  catch a bird you might be able to sell him for money.}) -- all such symptoms
of schizophrenia are also expressions of the communication which gets disturbed
by the rigidity in following plain rules in the use of the involved
\co{signs}.\ftnt{Of course, other psycho-somatic disturbances are possible, but
  they do not concern us here.}  The disturbance is not the lack of a rule but
the lack of anything but the rule; it is not the lack of any meaning but the
fact that the whole meaning is only the mere fact of conforming to some rule,
that the content of the actually followed rule is the only discernible content
of the message.  Even more extreme example of a similar rule-rigidity is a
correct but impersonally stiff language which is spoken too perfectly and too
grammatically, as if by a person using a foreign language he learned in a
classroom; the lack of colloquiality and subtleties of emotional tone; the
adequate knowledge and application of the formal rules of the language along
with the complete lack of the idea of communication. These defects, extending
also to nonverbal communication, characterise some forms of autism.

\pa One thing is the study of languages, of their properties and structures;
quite another is the obsession with Language.\noo{(As always, the observable
  differences may be negligible.)} Proliferation of the disciplines and intense
investigations in linguistics, semiotics, logic, grammar, parsing, machine
translation, etc., etc., can certainly prove to be invaluable. But they gain
paramount relevance {\em for a philosopher} primarily when he has lost -- or
sees the loss of -- the sense of any \co{community}. Language can be interesting
in most circumstances. But it is becoming imperatively important when it ceases
to function properly, that is, when its \co{foundation}, the \co{community}
which makes communication possible, deteriorates, when the \co{distance} to the
\co{shared} reality becomes impassable because that which is \co{shared} gets
reduced to that which is common and the \co{clear} meanings which might be
communicated to the \co{precise} means of communication.  All that one is still
able to hope for is \co{actual} agreement, adequacy of the \co{signs}, consensus
concerning the rules.  Fiddling with the language one hopes to improve the
\co{actual} communication and thus, perhaps, to reestablish the \co{community}.
But the \co{more} intense and \co{precise} are the determinations of all the
\co{visible} \co{signs}, the more all the sense of \co{community} disappears in
the empty cracks between them.\ftnt{%We are not talking merely about the post-Fregean,
%  and why not post-Turing, logicism and linguisticism.
  We would not claim any causal relations, but observing merely some rough
  simultaneities is too tempting, even if also a bit daring.  (1)~Some parallels to
  such a coupling of linguisticism and the lack of \co{community} might be
  discerned in Europe in the VII-th and VIII-th century: on the one hand, the
  final stages of the disappearance of the Roman culture, the gradual
  dissolution of the Merovingian empire, conflicts between the majordomos for
  the succession after once powerful dynasty; and on the other hand, the
  conviction that the nature of things are recognisable in the etymology of
  their names underlying the whole of \citeauthor*{Etymologies}; Fridugisus'
  linguistic arguments for the existence of nothingness (cf.
  footnote~\refsp{ftnt:Fridu}); Alkuin's minuscule and struggle against
  barbarian, germanized Latin; dealing at the same time with the nature of
  things and the properties of their names in the encyclopedia of
  \citeauthor*{Hraban}. (Even if the two last ones belong to the Carolinian
  renaissance, they can be seen as continuators of the linguistic line.)
%  
  (2)~Around the beginning of the XII-th century, the new money economy and the
  Gregorian reforms began to yield the divisive and destabilizing consequences.
  The reforms attacking the simony among the clerics opened up, as was claimed,
  for the emergence and proliferation of the heretic movements, almost absent
  since the V-th century. (Their appearance around that time does not, of
  course, reduce to the effects of Gregorianism.) At the same time, the
  increasing interest in
  language and the study of its foundations develop into the Scholastic
  grammar with its theory of supposition and, eventually, into Ockham's
  nominalism vindicating language and logic to the level of ultimate
  truth-bearers, along with the atomistic ontology. 
%  
  (3)~The social disintegration of the XX-th century hardly needs any comments.
  It is paralleled by the fascination with the philosophy of language, also
  emergence of formal languages and logic, and the
  thread leading through logical positivism to analytical philosophy which
  ends\ldots where it ends.  Perhaps, the ecstatic opening to \thi{otherness},
  just like the personal spiritualism of the New Age (both with roots reaching
  at least to the end of the XIX-th century), could be seen as a reaction
  against the stiffened linguisticism and predilections for rigid formalities.
  The associated cacophony of language looks like an inverted culmination of a century
  long analytical attempts to heal its metaphysical sickness by\ldots capturing
  and formalising its essence. Can we dare to consider it as yet another analogy?
  Namely, as a similarity to the post XII-th century heresies (especially
  Wyclif's but also, at least to some degree, most contestations of the time)
  which postulated replacing the visible, stiff and degenerate Church and
  tradition by the invisible and living church, the true community founded
  directly and exclusively on the revelation of the absolute \thi{otherness} -- Bible?
}


\pa\imm At the lowest level of \co{immediate experiences}, \co{communion}
amounts to \co{sharing} the \co{actual} moments.  With whom? It may be
the loved person, or else people who happen to be present. But it need not be
anybody in particular, nobody may be \co{actually} present.  \co{Thankfulness}
is \co{sharing} through \co{participation}, and every moment, even if lived in
loneliness, is but a \co{gift} of the \co{transcendence}. Recognition of this
\co{gift} in a single moment is the same as \co{sharing} it with others --
whether \co{actually} present or not. It is \co{sharing} with others because the
\co{gift}, although given to me, is not \co{mine}, is not given {\em only} to
me.

% \co{Sharing} in general and \co{sharing} the \co{actuality} in particular
% require the \co{shared} reality and the other who also \co{participates} in it.
Although \co{immediacy} does not seem to leave the space for any \co{distance},
the \co{concrete} communion of the moment respects both the \co{distance} to the
\co{origin} of the \co{gift} and to the other with whom it is \co{shared}.
Other, in particular, requires a \co{distance}. This platitude seems to be
forgotten quite often, so let us repeat -- only \co{distance} makes otherness
possible. The ultimate otherness is constituted by the ultimate
\co{confrontation} (which is a form of \co{distance}). But so is another person
-- he emerges as another only through and from the \co{distance} which separates
us. Only distance makes relation possible, and only distance makes
being-together possible. The true \co{communion} can find an expression in the
\co{immediacy} of a pure \thi{now} only if the \co{distance} is honestly and
carefully maintained. The sense of unity in and of a moment is possible only
when the \co{aspect} of the \co{distance} is retained, when one remembers that it is
unity of distinct poles, when we join each other from the \co{distance} of
distinct individualities. In fact, keeping the \co{trace} of this \co{distance}
in an \co{actual} moment is itself enough to experience the unity -- at once, of
\co{sharing} this very moment {\em and} of \co{sharing} its \co{origin}.  This
\co{distance} and otherness, \co{present} in the \co{immediacy} of a moment
which apparently makes it impossible, is the \co{trace} of the \co{concretely
  founded communion}.

Expressions of this are as varied as moments one may encounter. 
Let us say, it can be for instance respect for things in the
\co{immediate} vicinity, within the horizon of our \co{acts}. 
%
% \pa also RESPECT for THINGS:
% \co{Openness}, the \co{recognition} of God's \co{omnipresence}, breeds respect
% and love for things.  The \co{openness} means thus also to embrace and
% invigorate, to let everything grow (as Aristotle would say, to let it achieve
% its ultimate Good, to actualise its essence) and to take care for things which
% are within the horizon of our \co{action}.
%
The care we take for things is not grounded in our infinite love for their
\co{absolute} value, but in our love for them as \co{concrete} \co{gifts} of the
\co{origin}; it is love in \co{analogical} sense, love which is but an
\co{expression} of \co{communion}.  As always, telling one from another, telling
such a love from \co{idolatry}, is not a matter of any rules and laws, but of
\co{concrete}, personal presence.  What is the ultimate good of this or that
thing?  Fortunately, there is no general answer, because if there were, our
lives would be pretty boring. Petting a cat or watering a plant can hardly be
anything evil, but it may be an expression of a quiet pleasure or respectful
care or, on the other hand, of a nagging doubt about one's likability or
usefulness. In the former 
case the moment is \co{shared} and in the latter stolen.
%, and stolen only to be passed by.

\pa\label{unfoundedImCom} %--\imm
Exclusive restriction to the level of \co{immediacy} is hardly possible. Lack of
\co{concrete foundation} leaves then hardly any possibility of \co{sharing}
anything.  Things and \co{objects} viewed from this level appear as arbitrary
events of pure \co{immediacy}.  Consequently, all kinds of relations between
them, as well as between them and the \co{subject}, are as if purely nominal,
unreal, abstract, indifferent. Appropriation and minute enjoyment can be
attempts to establish some \co{immediate} community. \thi{Use-and-throw}
attitudes, \thi{things are for \co{me} and \co{I} do what \co{I} want with
  them}, all forms of disrespectful arrogance \co{acting} from the impulse of
the moment are \co{immediate} expressions of the \co{unfounded community}, that
is, of the lack of \co{community}.

In terms of relations between people, this lack of \co{foundation} amounts to
extreme atomicity, to positing every individual as a totally independent
\thi{it}, \co{dissociated} from any context and influences from \thi{outside}.
The other who has thus become a mere \thi{it} can be encounter  just
like other things.  Everybody may have his private goals and life, but these are
not in any way shared which means, other's life in no way affects mine.
\woo{All creatures are born isolated and have no need of one
  another.}{SadeAl}$^{\ref{ftnt:alienation}}$ 
\noo{Although it is impossible to live such an attitude
completely, it is possible to maintain it mentally, and even to \co{experience}
the surroundings and people around in this way.} 

\noo{The \la{animus} of the lowest level manifests itself as a \thi{muscle man},
  a mere vitality of strong body.  \la{Anima} is more generously associated with
  Eve, though she too, represents merely the level of instinctual, biological
  relations.  }

\noo{ \subsubi{Justification} \pa \citeti{The just shall live forever.}{Wisdom
    of Solomon}{V:15} \co{Justification} is an \co{aspect} of \co{communion}.
  Yet, although neither can ever occur without the other, they have been
  \co{dissociated} in the debates of more social and even political rather than
  theological character.  We have spent quite some space discussing
  \co{community}, so now we comment only on its association with
  \co{justification}.
  
  As talking about righteousness in the eyes of God suggests not only a serious
  anthropopathism but even a legalistic bias, we prefer to talk simply about
  \co{justification}.  Its ultimate form is quieting of the \co{thirst}.
  \co{Thirst} is, as Augustine says, \la{esse uelle}, an ontological thirst for
  Being, that is, for \co{concrete} Being, and \co{justification} concerns
  exactly one's very being, the very \co{foundation} and center of one's being.
  The element of \co{transcendence}, or more specifically of \co{communion}, is
  expressed in the fact that nobody can \co{justify} himself. This is
  experienced even at the level of individual psychology: self-justification is
  almost contradiction in terms.
  
  We have given some examples of the attempts at self-justification when
  discussing evil's impossibility of obtaining it, \refpp{pa:noJustification}.
  Glorification of the intensity of a momentaneous pleasure -- which, as the
  {\em all and only} good, witnesses to extreme \co{alienation} -- tries to
  explain itself as simply following the laws of nature. Achievements and
  successes (mostly of professional kind) function often as the surrogates of
  justification in the eyes of others and, above all, of oneself. A very common
  pattern of self-justification is to explain one's own actions as necessary
  reactions, as mere responses to some encountered \thi{objective evil}.  The
  attempts at explanation and self-justification -- inadequate and impossible as
  they are easily seen to be -- witness to the \co{alienation} from which they
  arise. Recourse to \co{visibles} does not help for nothing \co{visible} can
  quench \co{thirst} and abolish the \co{alienation}.  \citet{He that loves
    silver shall not be satisfied with silver.}{MaimoTraits}{ [after p.69]} As a
  more extreme example, since more or less precluding even attempts of
  self-justification, we can mention envy.  It can be easily recognised as an
  expression of alienation from another whom one envies possession of this or
  that. But like the other examples, this too is only an \co{actual} expression
  of a deeper defeat, of a conviction (whether conscious or not) of the
  impossibility; first, perhaps only the impossibility of obtaining this or
  that, but then also the impossibility of filling the lack, of achieving the
  ideal state, the object of one's dreams, in short, of quenching the
  \co{thirst}. As the \co{actual object} of envy (and similarly, of jealousy)
  begins to fill the whole horizon, so the feeling itself penetrates deeper and
  deeper, enslaving gradually the wider scopes of personality.  But this
  \thi{reaching deeper and deeper} concerns only the consciousness of
  corruption; envy from the very beginning witnesses to the inability to
  \co{share}, to the reduction of \co{sharing} to mere possession, in short, to
  the \co{alienation}.  The impotence (of which the impotence to obtain the
  object of one's envy is only an expression) is founded on impossibility, but
  this impossibility amounts here to a search for the ultimate quietude in
  something \co{visible}, in possessing the object of one's envy, in
  \thi{possessing} the person one is jealous about. The order of founding might
  be summarised as: \inv impossibility; \mine impotence; \act enslavement
  (envy).  This deep and usually unrealized conviction of impossibility is the
  same as the lack of \co{justification}, the \co{alienation} from the
  \co{origin}.
  
  Envy is \co{aware} (if not directly conscious) of this impossibility and it
  will hardly ever try to justify itself. At most, it will point to the
  attractive character of its object (or the person, as in case of extreme
  jealousy), but even that with the recognition of the impossibility to justify
  the deluge of its \co{invisible} thirst by the properties of any particular
  thing. Its heart is being eaten up by the impossibility of obtaining the
  object but, as a matter of fact, the worst thing which may happen is that it
  obtains this object. The impotence to obtain this particular object is only an
  \co{actual} image of the bottomless lack which lives its own life and cannot
  be filled. Satisfaction of the \co{actual} desire makes only the underlying
  emptiness more intense.
  
  \pa Living an insatiable desire may demand a lot of strength, acting against
  one's deep intentions and conscience may be the strength itself. But
  \co{visible} (even though deeply hidden) strength is often an \co{inversion}
  of defeat, like determination is often a \co{sign} of despair. It is only the
  {\em inability} to act so, the inability to act against \co{vague
    commandments} -- coming from the depths, that is, from \co{above} -- which
  is the true strength.  It is only the \co{openness}, and in more specific
  sense the openness to others, which strengthens by \co{justifying}.
  
  \noo{\co{Justification}, as all \co{absolute} events, begins \co{above} the
    \co{visible} world and then finds only its more and more \co{visible}
    expressions.}
  
  \inv \co{Justification} is the undeserved \co{gift} and it arises from the
  \co{invisible} depths where one's soul touches the point of the \co{origin}.
  It is \Yes\ which, abolishing \co{alienation}, restores the \co{sharing
    communion}; \Yes\ which not only does not coincide with any \co{visible}
  works, but which one will often say without even knowing it.  \citeti{I am the
    Lord, [...] I girded thee, though thou hast not known me.}{Is.}{XLV:5 [We
    are not concerned here with the opposition -- as unfortunate as real --
    between faith and works which has gone out of proportions since the apparent
    disagreements in the letters of St.~Paul and St.~James.\noo{Jas.II:17} It is
    the \co{invisibility} of the event which is of primary importance here. In
    this respect, equal \co{visibility} pertains to \co{actual} works or to
    faith understood not as an \co{existential} vocation but as a mere assured
    belief:\noo{fiduciary faith} \citef{If any one saith, that man is truly
      absolved from his sins and justified, because that he assuredly believed
      himself absolved and justified; or, that no one is truly justified but he
      who believes himself justified; and that, by this faith alone, absolution
      and justification are effected; let him be anathema.}{Trent}{VI:XVI.Canon
      14}]\noo{What Council calls \wo{justice} is today called \wo{sanctifying
        grace}}}
%
  \noo{ \citeti{for it is the power of God unto salvation to every one that
      believeth; to the Jew first, and also to the Greek.  For therein is the
      righteousness of God revealed from faith to faith: as it is written, The
      just shall live by faith.}{Rom.}{I:16-17}
  
    \citeti{Therefore by the deeds of the law there shall no flesh be justified
      in his sight: [...]  For all have sinned, and come short of the glory of
      God; Being justified freely by his grace through the redemption that is in
      Christ Jesus: Whom God hath set forth to be a propitiation through faith
      in his blood, to declare his righteousness for the remission of sins that
      are past, through the forbearance of God; To declare, I say, at this time
      his righteousness: that he might be just, and the justifier of him which
      believeth in Jesus.  Where is boasting then? It is excluded. By what law?
      of works? Nay: but by the law of faith.  Therefore we conclude that a man
      is justified by faith without the deeds of the law.}{Rom.}{III:20/23-28}
    
    \citeti{But to him that worketh not, but believeth on him that justifieth
      the ungodly, his faith is counted for righteousness.}{Rom.}{IV:5} }
%
  Evil is impossibility of \co{justification} simply because it \co{alienates}
  form the \co{origin}, isolates one from the \co{communion} which is the same
  as the very \co{justification}.
  
  \mine A most personal \co{community}, a life-long relationship like marriage,
  or sometimes friendship, gives a particularly \co{clear} sense of
  \co{justification}. It is \co{communion} where the two indeed become \woo{one
    in flesh.}{Gen. II:24} \co{Sharing} the whole life is much more than any
  kind of association and group. When, after years, one starts reflecting the
  other, the constancy starts overshadowing particular problems and dangers and
  the \co{shared} space extends to even most funny details, the true
  \co{community} gives the sense of \co{concrete justification} which remains
  forever inaccessible to a person spending the life alone.
  
  Only slightly different sense of \co{justification} arises from a genuine
  membership in a cultural or national \co{community}. Observance of the rites
  and tradition, when \co{concretely founded}, not only does not diminish one's
  sense of freedom but, in fact, increases it. Although one may risk involvement
  into all too legalistic elements of the traditional institutions, the ability
  to reflect their commandments and intentions provides the psychological sense
  of constancy and \co{justification} comparable to the personal relations.
  
  \act At the yet more particular level, we can observe the tremendous force of
  others' presence; force, we might say, of justification. A group of friends,
  or even of merely common interests, justifies every member: another simply
  resolves my doubts. A simple \wo{yes} or \wo{no} from another person, a
  trivial gesture of agreement or disagreement, turns my private meanings into
  something almost \co{objective}. My opinion on some issue may be reasonably
  clear but still open to adjustments; once expressed and confirmed by another,
  it changes the status, it becomes a common good, a solid, {objectified}
  sediment of the \co{subjective} flux. The miracle of such an interaction
  consists in that, to begin with, none of us is certain, each of us has his own
  doubts; and even if the doubts are not cleared away, the mere acceptance by
  the other may diminish their force, in fact, make me forget them. The arising
  \thi{certainty} is, of course, only a psychological effect which, in itself,
  does not make the matter either more nor less certain -- it concerns only our
  {\em feeling of} certainty. (The adolescent dreams of perfection and ultimacy
  of recognition exemplify the insufficiency of such associations which may, at
  most, quench the merely \co{actual} feeling, but not the underlying
  \co{thirst}.)  Consequently, such psychological details can easily be misused
  or else lead to formation of rather unpleasant groups serving exclusively
  mutual strengthening of the unhealthy tendencies of their members.  At this
  level, the two \co{aspects} are, of course, \co{dissociated}. One can easily
  find ways of communications without increasing the sense of justification or
  vice versa. But this does not change the underlying observation that such
  psychological effects remain the expressions of the close relationship --
  which obtains even at the level of \co{actuality} -- between justification and
  communication with others.
  
  END \subsubnonr{Justification} }

\subsubi{Freedom}
%\subsub{Freedom}\label{sub:freedom}
%\addcontentsline{toc}{subsection}{\ \ \ The founding of freedom}

%%\parbox[t]{15cm}{\levels{Freedom}
%%\are{necessity \& arbitrariness: spontaneous appearance of contents -- 
%%aestetics, Fichte}
%%    {necessity \& control: manipulation thanks to understanding -- 
%%    rationalism}
%%    {possibility: listening, context, possible dependencies -- 
%%    existentialism}
%%    {faith, trust: everything is possible -- Shestov} }       

\pa\label{pa:invfreedom} \inv % \inv--
We have said in \refp{pride} that freedom is an \co{aspect} of \co{pride}.  This
certainly needs some qualifications.  We are dependent on various things, we
have to eat and sleep, etc. We are involved into causal relations of \co{this
  world} but \ldots it in no way contradicts our freedom. For freedom is not
freedom \thi{from} every possible form of dependence but only from enslavement.
Freedom which tries to establish itself as a total independence \thi{from}
everything, ends up in the blind street of other self-referential paradoxes by
realizing that it is sentenced to freedom -- having proved (to itself) its
independence, it cannot escape \thi{from} this very fact, it becomes doomed to
this fact.\ftnt{The destructive
  character of such an observation relies on the combination of both the
  negative freedom \thi{from} and the attempts to {\em prove} it as an
  unavoidable truth. But the two easily go together.}  The attempt to rise
above, to liberate \co{oneself} \thi{from} every possible dependence is exactly
the \co{aspect} of \co{pride} for which every dependence seems a form of slavery.
But finding only emptiness \co{above} the interplay of \co{visible}
dependencies, the only project that remains is to keep liberating \thi{from}
this, \thi{from} that, \thi{from}\ldots It is this negative freedom \thi{from},
the insatiable freedom of emptiness which is an \co{aspect} of \co{pride}.

Liberating oneself \thi{from} this and \thi{from} that has also the aspect of
paying back one's debts. Having borrowed or obtained something makes one feel
unfree and settling the accounts one liberates oneself \thi{from} that.  Man who
owes nothing to anybody stays cool and free, above the pettiness of daily debts,
he remains remote, unaffected and~\ldots \co{proud}.  But freedom is not at all to
pay back all the debts -- this is simply impossible, and \co{thankfulness}
amounts also to the recognition of one's infinite debt. The freer one is the
more one owes, and \co{nothingness} of the \co{self}, owning \co{nothing}, owes
everything. 

On the other hand, some like to remember the unhappy events and days of
childhood, complaining about the family and relatives who did not do their
due\ldots About the society which did not and does not function to promote personal
happiness\ldots About others, met then and now, who take away one's spare time,
money, possibilities of enjoyment, one's life.  So one offers one's time and
energy and imagines that others owe him something.  All such complains can seem
justified but their only work is: enslavement.  Nobody, and least of all the
past, owes anything to a free man.

\pa There is the abyss of freedom, the abyss of \co{nothingness} which attracts
a slave with the indeterminacy of its emptiness, like a false promise. The dread
of this terrifying attraction is the price of the freedom which, equated with
emptiness, proves illusory.  The true, \co{absolute} freedom is not only to
possess nothing but also to owe everything. It is equivalent to a surrender, to
renouncing \co{oneself}, that is, renouncing all claims one might believe to
have, all debts others might possibly owe. It is the freedom to accept the
undeserved \co{gift}, to recognise the \co{absolute} character of the
\co{command}, which in particular means, the possibility one has to deny it.
Having accepted it, the complete lack of \co{actual} contents leaves one
entirely free to realize it, to give it an adequate \co{actual} form. (What form
is adequate, however, is too \co{concrete} a question to be addressed in such
abstract categories.) \citeti{God forces no one, for love cannot compel, and
  God's service, therefore, is a thing of perfect freedom.}{Hans
  Denk}{\citaft{Huxley}{ p.93}}

This freedom to \co{express} the \co{invisible}, this \co{presence} of
\co{transcendence} in the midst of \co{immanence}, is thus not \co{my} freedom.
As all dimensions of the \co{absolute} it simply is, or is not at all, and
\co{I} can at most \co{participate} in it.  To be free is to \co{forget} one's
freedom. (Let us only remind that \co{forgetfulness} is not the same as denial,
a mere slipping out of memory or simple blindness, \refp{pa:forget}.)  \yes,
suspending the presumed \co{absoluteness} of \co{this world} and anchoring
\co{my} being \co{above} it, makes me completely free in relation to it.  This
freedom is precisely the content of \co{non-attachment}, of erasing the
dependence on the \co{idols}, the \co{visible} pretenders to \co{absoluteness}.
It \co{founds} the thorough \co{experience} of freedom which permeates \co{my}
whole being, and which is not contradicted by any problems, obstacles,
restrictions at the lower levels. In a sense, it liberates one from the
dependence on the \co{visible world} simply by abolishing the need to look for
the final proofs and ultimate confirmations of freedom there.

% In fact, all such obstacles may then be taken as mere challenges to exercise of
% freedom (although this phrase might wrongly suggest that there is any point in
% \thi{exercising freedom}).

This is yet another example of \co{inversion}: just like \yes\ to Godhead
turns out to be the deepest \yes\ to \co{this world}, so the \co{humility} and
submission to the higher \co{commands} is the fullest form of freedom.
Freedom is  not any \thi{faculty}, any separate, empty power which could be
filled with arbitrary contents, and which, by some universal law of human
nature, every soul either possesses or not. It is an \co{aspect} of the
universal possibility, which everyone may desire or detest, attempt to achieve
or neglect, pray for or forget. 

% It is inseparably bound with the \co{invisible}
% content of the \co{command}, submission to which defines it.

\noo{ Freedom at the level of \co{invisibles} is no longer a freedom from ...
  Although Berlin applied the distinction between the negative \thi{freedom from
    ...} and the positive \thi{freedom to ...}  with respect to \co{my} freedom,
  it really applies first when there is something which not only is not
  \co{mine}, but which is not even relative to the categories of \co{mineness}.
  \co{My} \thi{freedom to ...}, to do this or that, is but another side of the
  absence of restrictions, obstacles or prohibitions which would limit this
  freedom.  }

\noo{ This is freedom of mystical faith, which finds itself the more free, the
  more \co{clearly} it can hear and follow the voice of the higher \co{command}.
  It is freedom for which everything is possible or, in any case, nothing is
  impossible, the freedom of paradox and miracle, which Shestov so forcefully
  opposes to the rationalism of Athens.  This is the freedom after the final
  \ger{Spring} of Kierkegaard's, freedom of Abraham who does follow the
  incomprehensible and paradoxical \co{command} of sacrificing his only son
  although he might choose not to.  }


\pa \mine % \mine--
% no dreams, no entitlement
\co{I} cherish \co{my} dreams, \co{my} images -- of happiness, fulfillment,
completeness -- and stay \co{attached} to these \co{idols} in constant attempts
to find the \co{actual} medicine against the \co{thirst}. One will say, OK, but
these are \co{my} images and dreams, and as such they make \co{me} free from the
external constrains. There may be some psychological truth in that, but of
little value. For it only posits this image in order to liberate itself
\thi{from} that, it tries to perform the impossible leap and get rid of the
\co{externality} which is only an \co{aspect} of its very \co{subjectivity}.
This may be a common image of freedom at the level of \co{mineness}, of
\co{unfounded} freedom which sticking to some \co{visible idols} is at the same
time trying to escape \thi{from} any such dependency.

\co{My} freedom, the \co{unfounded} freedom of this level, focusing on
\co{mineness} insists on making one's own choices, on following one's own course
of actions, on being an authentic \ger{Dasein}, or else an independent
\ger{\"{U}bermensch}.  But \co{mineness}, which has divided the world into
\co{mine} and \co{not-mine}, is exactly the site of the negative freedom
\thi{from}, independence \thi{from\ldots}\ftnt{We are, of course, simplifying
  tremendously all the references here.  For instance, Nietzsche's
  \ger{\"{U}bermensch} is supposed to act from the pure positivity of his own,
  vital energy.  But it is still {\em his own} and the main emphasis lies on
  independence, on \thi{not being concerned with others} which, somehow, remains
  bothered by others being there. Every call to liberation and independence
  witnesses to enslavement.}  Insisting on the independence \thi{from} whatever
is not \co{mine}, what has not been freely chosen by \co{me}, it can only
encounter deeper and deeper \thi{certitude of abyss}.  As it often happens in
the face of ultimate emptiness, this negative freedom turns around to seek
solace in the things of \co{this world} and one can indeed \citet{ask the
  question if man, following the need of psychological and metaphysical bonds,
  does not prefer dread over freedom.}{MannCzasy}{\kilde{p.470}} For dread
remains only as long as one is staying at a distance and does not plunge into
the abyss. Thus dread, real as it is, is only a new \co{idol} worshiped for its
being so deeply \co{mine} and allowing \co{me} to remain what I was. Balancing
on this edge, between freedom \thi{from} this world and dread in the face of
emptiness surrounding it, is the epitome of loneliness and \co{alienation}.
\wo{You are and~\ldots nobody cares} is the eventual truth of this state and
this \thi{freedom}, so throughly described in existentialist literature that it
hardly needs more words.


\pa %mine+ \nexus:
\co{I} cherish the dreams and images -- of happiness, fulfillment, completeness
-- but all these dreams are like a mist, vague and unclear, in fact, entirely
contentless, sheer ghosts. What do you dream of when you dream of happiness? Do
you ever {\em dream} of happiness? A dream seems to require an image, so perhaps you
manage to substitute this or that, but then you also immediately start to
suspect that it does not exhaust the meaning of your dream.

The deepest dreams \co{manifest} \co{thirst}, they aim at the \nexus\ of \Yes,
and so can be falsified by any attempt to bring them to the level of all too
\co{precise}. Every image threatens with a reduction of meaning and if it is not
kept at the appropriate \co{distance}, required by its origin, it becomes a
pretension or entitlement. For we are all entitled to happiness, and since {\em
  this} is happiness (for \co{me}!), \co{I} should be entitled to {\em this}.
Then \co{I} get this and so \co{I} should be happy, but \co{I} am still not. Or,
perhaps, \co{I} am eventually happy but then \co{I} am not~\ldots free. And what is
happiness good for, if one is not free? But everybody is entitled to freedom,
right? (human right!) so now, what does it mean to be free?

Wrong questions breed wrong \co{distinctions}.  The mist of {happiness}, the
ghost of {fulfillment} for which we thirst, is not different from freedom. They
are but \co{aspects} of the same \nexus. Speaking more specifically, we can say
that freedom at the level of \co{mineness} is \co{non-attachment}, is freedom
from the slavery to \co{idols}, be they images, things or ideas, or else empty
words which despair endows with ever more \co{precise} meanings -- they are all
gathered under the \co{idol} of \co{mineness}. Its ultimate and typical
expression is confusion of freedom with \co{my} sense of being free, of freedom
with \co{my} feeling of freedom. Thus \co{attachment} trying to realize its
freedom only engrosses itself into unfreedom.  Freedom, \co{non-attachment}, is
the total non-entitlement, which is the same as the lack of fear, that is,
\co{openness} for every \co{gift}. And thus freedom is unbreakably bound with
meaningfulness which is exactly the \co{concrete presence}, \co{openness} to
\co{vertical transcendence} and \co{thankfulness} for its gifts. The lack of meaning
is also the lack of freedom, just like is the lack of respect.



\pa \act %\act--
Freedom as total independence \thi{from} causal relations of
\co{this world} is an invention of \co{attachment}, of \co{attachment} to
\co{this world} which tries to {detach} itself from it, which tries to
liberate itself by rising above it and \ldots still stays in \co{this world},
because above it, it finds only emptiness.  Causality, this much
overemphasized notion, and more significantly the physical existence, the
body, the physiology, in short, the whole sphere of most \co{actualised}
contents, does not in the slightest oppose the freedom because
they belong to different levels of \co{existence}. But \co{attachment},
reducing all that is to what is \co{visible}, can not but oppose the two; for
it, if there is any freedom, it must be found exclusively within the sphere of
\co{visibility}.

This involves \co{reflection} over freedom into the opposition to determinism.
The \co{objective} world of \co{complexes} is rational and understandable, which
eventually means, underlied necessary laws.  The celebrated problems of free
will emerge as a result of reducing human \co{existence} to the same level and
considering it only in terms of the \co{objective complex} and natural laws. By
the same token, the will gets degraded to the \co{egotic} ability of freely
selecting goals among the \co{objects}, \co{complexes} and constellations of the
world.  In this tradition even predestination acquires a form of determination
and one has to take recurse to various distinctions like, for instance, between
necessity of all events when seen from the point of divine knowledge vs. their
occurrence through free will when seen in their own nature by human
understanding (Boethius),\ftnt{This makes, in fact, deterministic necessity into
  a mere phantom since it becomes merely a property of divine
  knowledge without any influence on and relevance to human life.  Everything
  happens as if there were no necessity, for God foreknows simply the results of
  free choices which, in temporal terms, remain undetermined until they are
  actually made: \citef{if Providence sees an event in its present, that thing
    must be, though it has no necessity of its own nature. And God looks in His
    present upon those future things which come to pass through free will.
    Therefore if these things be looked at from the point of view of God's
    insight, they come to pass of necessity under the condition of divine
    knowledge; if, on the other hand, they are viewed by themselves, they do not
    lose the perfect freedom of their nature. Without doubt, then, all things
    that God foreknows do come to pass, but some of them proceed from free will;
    [...] if they are viewed by themselves, they are perfectly free from all
    ties of necessity.}{Console}{V\kilde{p.165-166}}} a related distinction of
Leibniz' between the absolute necessity (whose opposite would involve a 
contradiction) and the hypothetical one (which rests on God's foreknowledge and
free decision and corresponds, as a matter of fact, to the contingent events of
our world),\kilde{LeibnizDM \para 13, p.64ff}
%free and rational vs. chaotic and irrational choices,
the distinction borrowed from Stoics by Spinoza between free acceptance of the
laws vs. unfree because unproductive opposition to them.\ftnt{Here, one makes
  any indeterminacy and freedom a mere illusion. Stoics
used the following illustration: \wo{If a dog which wanted to follow the
cart were bound to it, then it would both go and be dragged after the cart,
doing voluntarily what it must do of necessity; but if it did not want to follow
the cart, it would still be forced to do it anyway. The same happens to
the humans.}{ \citaft{Powrot}{p.76, footnote~185.\kilde{Hipolit [my retranslation]}}}}
%
Necessity and determinism is the ultimately objectified image of our
limitations, is the fact that nobody possesses unlimited power enabling him to
\thi{do as he wishes}, pushed to the ideal limit. And like every ideal limit, it
loses contact with the \co{concreteness} of \co{existence} and becomes only the
more troubling and apparently more important, the less relevance it retains.  In
fact, necessary and inviolable laws would need no observance -- this is the
essence of their necessity. So confronted with them, the only remnants of free
choice would be between 1) rejection and denial, in a childish opposition to the
unavoidable, or 2) resolute acceptance and obedience. As rejection of the
unavoidable is irrational, true freedom must then amount to expedient use of the
laws, to the ability to manipulate and apply them.  It is freedom which
liberates \thi{from} the necessities of \co{this world} only when these can be
utilised for one's own purposes. However, the more necessity, the less content
and, eventually, such laws which might be believed to apply unreservedly, get
applicable only to more and more \co{precise} and objectified contents,
II:\refpf{pa:necessity}. In practice, this is the way in which money and power
give freedom of choice and action, freedom which, mixed with and constantly
opposed to the limitations imposed by the surroundings, keeps fighting for its
own sake and, at the same time, doubts its ultimate \co{foundation}.  For
overcoming various \co{actual} limitations, it suspects some fundamental one,
but not finding any \co{visible} necessity, it can avoid the ultimate emptiness
only by pushing the idea of necessity further and further away from itself.

% For, as nobody has ever managed to pinpoint any necessary laws for anything more
% than the contentless objects of mathematics, their relevance also for the
% consideration of freedom seem limited.

\pa%freewill
The problem of freedom and free will is usually posed in the manner of an
\co{objective} question about some \thi{matter at hand}: \wo{{\em Are} we or
  {\em are} we {\em not} free?} \wo{{\em Is} our will free or {\em is} it {\em
    not}?} Also such questions lead naturally to what one
would consider the \thi{real}: the physical world, causality, the natural laws
-- these give at least a context for speaking about that which \thi{is}, and
\thi{from} which one might be free.  But such questions have only wrong answers.
If one says no, all common-sense objects forcefully, if not scholarly.
But if one answers yes, gives reasons and arguments showing that, indeed, we
{\em are} free, then one gets immediately captured into the necessity of this
answer which makes one unfree with respect to the (now necessary) law of
freedom.  However, freedom is not any fact but an \co{existential} possibility,
it is not given, it is not something everybody has or, as the case may be, does
not have.

We have, perhaps, liberated ourselves from this mode of speaking, but still, do
we not hear, occasionally, talk about the \thi{problem of free
will}?  It can, indeed, be made into a tremendous problem.  Here \co{I} am, in
\co{this world}, determined by the laws of nature, and yet \co{I} can choose to
do this or that, \co{I} do have a definite feeling that \co{I} have free will.
But do \co{I}?  How is it possible in the world, of which \co{I} am a part,
which is just a clockwork.  OK, we do not believe in this clockwork any more.  A
stone which hypothetically thinks that it is endeavoring to further its motion
as much as it can, a compass needle which desires to move north and opines that
it turns itself independently of another cause, or that dog of Stoics which,
desiring to follow the cart, was also bound to it so that it followed it both
voluntarily and under compulsion\kilde{Powrot,p.76,ftnt.185} -- these images do
not exercise 
such a strong influence on our sense of relevance, because the underlying idea
of an inescapable mechanism is not so convincing any more.
 We do not believe in this clockwork any more but we haven't got
anything else to believe in instead. Although we, everybody with \co{his} will, are no
longer confronted by (an image of) a deterministic clockwork, the doubts
concerning freedom seem to persist. 
The world, perhaps, is a bit more indeterminate than a clockwork, but the
question still remains: {\em am} \co{I} free?

Reduced to the level of \co{actuality}, freedom becomes only freedom to choose
one course of \co{action} rather than another -- one asks if the choice is made
freely, if the will is free. 
Will is \co{mine}, it is something \co{I} have, not something \co{I} am; that
is, it is an \co{aspect} of \co{ego}, in fact, the fundamental \co{aspect} 
emphasized always by every \co{egotic} being or culture.  As
\co{ego}, so also will is involved into the interplay of all \co{visible
  distinctions}, and all possible relations between them. Trying to liberate
itself \thi{from} the \co{visible} laws of the \co{visible world}, it has only
emptiness of indeterminacy and arbitrary choice left.  But replacing natural,
perhaps even necessary laws by some stochastic processes, by indeterminate laws
of social interactions, by Heisenberg's principle, does not change the least
thing, no matter how much one would like to believe that indeterminacy of the
world is more pleasing to one's freedom than its necessity would be.
Meaninglessness is an aspect of unfreedom and increasing the indeterminacy of
things does rather the opposite than what one would like to believe.\noo{Did the
modern preachers of freedom ever managed to preach anything more than
meaninglessness?} It is doubtful if anybody reading Camus, Beckett or Sartre is able to
discern any sense of freedom, of genuine freedom.  Their free will chooses
perhaps freely, that is arbitrarily, but it also seems strangely unfree,
captured somewhere between \la{nausea} and Sisyphean heaviness.\noo{So, perhaps,
reading Nietzsche? The need, the intense need of \co{attachment} to overcome
something, is but the need to deny it, is but an expression of will strained by
disrespect.\ftnt{You will be able not to misinterpret this. Of course,
  \co{I} can act in \co{this world} for various purposes, \co{I} can try to
  achieve various goals and this does not, by itself, mean that \co{I} am not
  free. What makes me unfree is when something becomes unbearable so that \co{I}
  {\em have to} achieve some goals, when something {\em has to} be changed, when
  the arrangements of \co{visible things} are \co{experienced} as all and only
  ground determining my freedom.  But freedom has nothing to do with arranging
  things in a manner which would be comfortable and pleasing to \co{my} will.}-}
Identifying freedom with the freedom of %will in making 
choices amounts to a tremendous reduction of the idea to the momentaneous
quality of a single \co{act}. The shadow of arbitrariness, which always appears
in the background of such discussions, is only to be expected.  Do we really
want to reduce freedom to such \co{actual} choices? Any \co{actual} decision
concerns only more or less petty matters, and the more we insist on deciding
\co{ourselves}, the pettier we become. Is one's freedom the freedom to decide
whether one will have an ice-cream or a chocolate?  Is this freedom we want to
speak about? Is this freedom? The issue is not to decide oneself but to decide
rightly, and freedom of choice has close to nothing to do with that.
%What matters is the choice, not freedom.

Free choice of free will might seem to ensure that \co{my ego} manages to detach
itself from \co{this world}. Perhaps. But detachment is only a form
(\co{inversion}) of \co{attachment}, a despair capable of nothing more than
negation.\ftnt{This is the constant association of the theme of freedom and
  {liberation from this world} with the variants of detachment and abnegation,
  whether in the original Orphic and Platonic, then Manichean, Gnostic and
  Cathar type, or more modern kinds, of which Heidegger and later
  existentialists are the prime example. Shestov's aggressive opposition of
  Jerusalem's faith to the reason of Athens' would also fall into this
  category.}
In fact, \co{I} would probably feel much more unfree in a completely chaotic
world, facing the \thi{certitude of the abyss}, than in the world of Newton, or
even Laplace.  Such a world, fully determined by causality (or other law), is
only an image and freedom does not amount to overcoming causality or other
possible determinations of our \co{acts}. Various \co{acts} in various
situations may be performed under various coercing factors. In fact, \citet{[n]o
  one wills what he can will because he can, without some other cause
  [...]}{AnselmFall}{27} But this does not change the fact that most important
human \co{acts} have no discernible, \co{visible} causes -- for causality, as
Kant teaches, is a category of mere \co{actuality}. (This does not mean that
\co{acts} are indeterminate and arbitrary -- they may not be caused, but they
are almost always \co{motivated}.)  \co{Experience} of a free (that is,
\co{motivated}, and not arbitrary) choice is irrefutable, and so determinism
must ignore \co{experience} and appeal to some \thi{deeper} aspects, possible
theories, splashy images, hidden mechanisms, future investigations\ldots
Consequently, possible perhaps as it in principle might be, it remains since
millennia a mere postulate -- the postulate to figure out all the sufficient and
determining causes.


\pa %\act+
\co{Non-attachment} is freedom from \co{attachment} but it is not
freedom \thi{from} \co{this world}. On the contrary, it is precisely
freedom to live and act {\em in} \co{this world}.
\co{Actual} freedom is not liberation \thi{from} things and their order but
respect for them, that is \co{thankful} acceptance. 

Because one no longer values any of the \co{visible} things as \co{absolutes},
that is, does not expect them to quench the \co{thirst}, one can accept
them, whether they are one way or another.  \noo{ In the most general sense,
  they are the way they are not because \co{I} or somebody else have made them
  so, but because this is the way they are.  Except for the narrow horizon of
  things close-at-hand which \co{I} can control, the most of natural as well as
  human creatures have to be left in peace the way they are.  To try to bring
  them in conformance with \co{my} wishes and whims, OK, with \co{my} true,
  meaningful and deep goals, is not only a futile \co{attachment} -- it is an
  idea which may only lead to disappointment, even despair.  }
%
Accepting things is very different from surrendering to them. Acceptance means
here the same as respect. One does not require explanations, reasons and
arguments, which is precisely to say: one respects things.  They run their
course, they may have their logic and it may be highly rewarding to study their
ways and to inquire into (not require) their reasons. One's freedom is the freedom
to do this.  Arranging them according to one's wishes and likings are but
petty consequences which may be useful but which have nothing to do with 
freedom.  Freedom, true freedom, is freedom to respect \thi{the order of
  things}, for \wo{no Thing is contrary to God; no creature nor creature's work,
  nor anything that we can name or think of is contrary to God or displeasing to
  Him, but only disobedience and the disobedient man.}$^{\ref{ftnt:obedience}}$

Disobedience is but another word for \co{attachment}; \co{attachment} which
worships instead of respecting, and thus remains enslaved in the midst of its
search for liberation. Just like one \co{actuality} excludes another, so one
\co{idol} opposes and tries to avoid or defeat another.  Any attempt to escape
from this or that, and the eventual form of the attempt to liberate \co{oneself}
\thi{from} the whole order of things and \co{this world}, is an expression of
\co{attachment}, of an involvement which makes \co{this world} the only
reference frame, of the underlying feeling of enslavement which sees its only
alternative in negation, in \co{detachment}. \thi{Use-and-throw} attitudes,
\thi{things are for \co{me} and \co{I} do what \co{I} want with them}, all forms
of disrespectful arrogance are expressions of un-freedom (just as they were
expressions of the lacking \co{communion} in \refp{unfoundedImCom}).  Also, an
\co{inverted} attitude, the Stoically resigned \thi{acceptance of the world}, the
realization that one can not oppose \thi{the whole world} and that therefore
it is wiser not to fight against it but humbly accept whatever it brings
\co{me}, is an expression not of freedom and wisdom but of defeat and surrender.
It may look like respect but, typically, it will be a mere servility, a mere
observance of all rules, regulations, customs.  Although there is nothing wrong
with all that in itself, it often carries at its bottom the feeling of unfreedom
when it is a mere \co{act} within \co{this world}, a mere defeat in the face of
\co{visibility}, that is, when it is not \co{concretely founded} in the higher
freedom.


\pa \imm %\imm--
Freedom viewed from the level of \co{unfounded immediacy} is hardly anything
more than arbitrariness of appearances.  This seems to be the character freedom
acquires in some forms of idealism, for instance in Fichte's Ego, and it is
quite explicit in Sartrean \thi{for-itself}.  The ideal \co{immediacy} of an
equally ideal \co{subject} leaves no room for anything except spontaneous
production, positing, apperception, appearance of arbitrary contents.  The only
alternative, in which thinking such \co{immediacy} inevitably gets itself
involved, is between the contents being posited by the \co{subject} or else
being completely independent from it.  Even the laboriously reworked by the
categories of understanding contents of Kantian sensations are, eventually,
arbitrary elements confronting the subject occupied with its transcendental --
and atemporal only because momentaneous -- activities.  The sense of freedom,
whether on the side of the \co{subject} or \co{object}, is just the spontaneous
emergence of contents. It is but a reflection of the \co{reflective act} which,
\co{dissociating} its \co{object} from itself and from all the \co{rest},
\co{posits} it in the \thi{freedom} of arbitrariness.\ftnt{We would probably not
  impute Fichte, and certainly not Kant, such a concept of freedom. We only
  identify the presence -- and significance -- of such an element in their frameworks.}

Some \oo\ feelings which might be related to such an idealized perspective would
present me with the \co{actual} situation as absolutely indeterminate.  The
world seen as a chaotic collection of isolated entities, as a pure play of
chance and arbitrariness, the alien and alienated world to which one
nevertheless still feels some form of belonging and which one would like to see
in an attractive, positive manner, in short, a deep existential crisis, may lead
to such a perception of freedom.  One day man will go mad to prove that he is
free -- as Dostoevsky prophesied.\ftnt{\citef{If you say that all this, too, can
    be calculated and tabulated -- chaos and darkness and curses, so that the
    mere possibility of calculating it all beforehand would stop it all, and
    reason would reassert itself, then man would purposely go mad in order to be
    rid of reason and gain his point!}{Underground}{I:8}} Mad minuteness is only
a step from minute madness which collapses the whole world to \co{immediate}
proximity.  Arbitrary spontaneity of such an isolated moment is the last resort
of apparent freedom left to a slave who had to escape that far.

The arbitrariness of all the events and complete lack of control over them
provide the grounds for denying that they have any value, that bad is as good --
since equally arbitrary -- as good, that my only role is to confront and accept
whatever is encountered. Let this description not mislead us -- it might almost
apply to the attitude of \co{thankful} acceptance. The difference is that
arbitrariness levels and equates all things because they have lost all meaning
and become equally empty, while \co{thankfulness} accepts all things still
differentiating and even choosing between them. As most thinkers, not only of
the rational school, always maintained, freedom requires a rational element --
it is not an arbitrary choice (which is only the other side of meaninglessness)
but one \co{concretely founded} in the higher sphere of \co{motivations}.

\pa %\imm+
Freedom, \co{founded} freedom of any \co{act} is \co{rest}, is its anchoring in
{\em all} the higher levels of being.  An \co{act} of will is still only an
\co{act}, and the sense of its freedom amounts to the degree of its
\co{dissociation} \thi{from} the causal dependencies which, eventually, means
simply the degree of its \co{dissociation}.
%\thi{from all the rest}. 
But the constitutive quality of an \co{act} is its very limitation to the \hoa,
its \co{dissociation}.  Thus every \co{act} carries with it this sense of
freedom. This sense, however, has no direct implications for the freedom of the
\co{act} which is almost its exact \co{inversion}: the \co{concrete} anchoring
in the \co{transcendence} as opposed to a \co{dissociation} from it.  There is
no such thing as a \thi{free \co{act} in itself}, for a free \co{act} is simply
an \co{act} of a free person, an \co{act} \co{founded concretely} in the freedom
of \co{existence}.

In one of the most cruel situations of enslavement, when ten prisoners are
selected to be tortured to death for the absence of one person at the roll call,
a lucky, unchosen man steps forth and asks to exchange the places with one of the
selected men. Thus died St.~Maksymilian Kolbe in Auschwitz, while the man whose life
he saved survived the war.  No situation deprives man completely of the
possibility to choose, that is, to \co{act}. The situation in which one, say,
has to lie or be killed may not have been chosen voluntarily, yet the choice of
the alternatives remains. \citet{Therefore although he either lies or is killed
  unwillingly, it does not follow that he lies unwillingly or is killed
  unwillingly.}{AnselmFreeWill}{V. The argument recurs also, e.g., in
  \citeauthor*{AnselmConcord}{VI}. (This observation is expressed in (most)
  modal logics as non-distributivity of modalities over disjunction.)}  All too
elaborate comments on the choice of Kolbe's would be inappropriate but a few
words should be allowed. It may serve as an example of an \co{act} of ultimate
freedom and, by allowing also others to retain the faith in its possibility, of
liberation.  If we were to call it an \wo{act of being-towards-death}, we would
have to emphasize that it is not any \thi{directedness} towards death and
nothingness, cherished for their liberating ultimacy, but only preparedness for
death, a true sacrifice choosing something one does not want, and choosing it in
the name of something which transcends infinitely any \co{actual} aspect of the
situation.\ftnt{Schopenhauer's definition of a saint as one who does nothing he
  would like and everything he does not like, is certainly exaggerated, but need
  not be dismissed completely.} Freedom of such an \co{act} is anchored in the
knowledge of its extreme consequences and their full acceptance, i.e., the
continued ability to put up with them.  Choosing thus what one does not want to
choose, consenting to what one does not want, effects an \co{inversion}, for by
choosing death Kolbe really chose freedom.  There may be situations where the
only free choice is Hobbson's choice, the choice of (or the consent to) the only
alternative of death. In such situations, the inability to sacrifice one's life
may make this life not worth living. In fact, man seems the less willing to
sacrifice his life, the less worthiness his life contains.\ftnt{An example of an
  attitude opposite to the one just mentioned is well documented by Tadeusz
  Borowski in the stories and novels from his time as a kapo in concentration
  camp. Few years after the war, he committed suicide.  The concentration camp
  syndrome (corresponding to PTSD, post-traumatic stress disorder, in more
  recent psychiatric classifications) originates with the exposure to
  exceptional cruelty and is not related to victim's premorbid personality
  (e.g., \citetit{Eiti}).  The very high prevalence of PTSD in the concentration
  camp survivors, about 85\%, still leaves 15\% which were not so severely
  affected.  Perhaps, the difference could be referred to the difference between
  giving up any resistance or else attempting survival for {\em any} price, on
  the one hand (yes! both these are on the same side), versus being prepared to
  die, as many did, preserving some dignity.  \citef{Survival in the camps [...]
    depended foremost on luck: to be able to survive, one had to escape being
    killed by SS. [...] If he was not murdered, how well a person was able to
    survive depended on how well he managed to maintain if not some of his
    autonomy, at least some of his self-respect and meaning his relations to
    others had for him.}{Bettel}{p.108\kilde{cudowne i pozyteczne...p.36}}
  Grzesiuk, who survived 5 years in various extermination camps, notes that he
  \citef{who wanted to survive must not have been afraid of death, for everyone
    who wanted to live and was afraid of death -- was afraid of being exposed to
    beating and executed dutifully all the commandments, awaiting a miracle and
    the end of war or that they will release him and that he will get through.
    When he realized that he was getting close to the end, it was too late and
    for such a one there remained only crematorium.}{Grzesiuk}{Preface} From the
  \co{egotic} perspective of a free choice, the ones like Borowski chose as
  freely (or unfreely) as those like Kolbe or Grzesiuk. But we would insist that
  while the latter remained free, the former did not, and the distinction has
  nothing to do with the way they made the choice but only with {\em what} they
  chose.}

%, and so some \co{acts} (free) are impossible without freedom...

A less tragic (because leading eventually to the survival) but more dramatic
(extending over several years) example of the choice of preparedness to die is
illustrated by the following.  \citet{We carried soil in wheelbarrows, bringing
  it from some 
fifty meters all the time running at the very edge of the stone pit. The wall
was some tens of meters high and was here quite perpendicular. Running we
carried the soil and kapo was running along with us, beating us with the stick
in the shoulders, hands, heads, faces. [...] After some minutes I realized that
I won't manage a whole hour. He will kill me -- I thought. Good, but you too,
bastard, will get killed and I will have greater pleasure going to heaven in
such a company. Kapo was short. I decided to catch him when he gets behind me
and fall back. If everything goes fine, we both flutter down to the bottom of
the pit, and there it stops, the end.}{Grzesiuk}{III. Mauthausen;p.95-96}

The examples are intentionally so extreme because they clearly illustrate that 
freedom is not at all the matter of the \co{external} situation. Although some
situations, contexts, political systems will make free \co{acts} more difficult,
and the feeling of freedom almost impossible, it is nevertheless
possible even in the most extreme cases which one might want to classify as the worst
examples of unfreedom.  \co{My} freedom is not liberation \thi{from} the
\co{actual} dependencies but the way of handling such dependencies. Most
importantly, \co{actual} freedom of \co{acts} and \co{actions} is not any
intrinsic property of them, but the fact of being \co{concretely founded} in the
freedom of the person. Thus \co{founded} freedom is indistinguishable from the
meaningfulness of its being, which lends its meaning and \co{motivation} to the
\co{actual} situations and performed \co{acts} -- every \co{act} and, in
particular, the free \citet{act of meaning is related to the unconditioned
meaning viewed as an abyss of meaning.}{TilRel}{I:1.1.e}


\sep

\pa
\noo{There is, in short, no such thing as \thi{the problem of freedom in
itself}.}Freedom is not a roving of a vagabond damned on selecting among arbitrary
alternatives but, on the contrary, the ability to select -- or what amounts to
the same, accept -- the right alternative.  True freedom is only an \co{aspect}
of the \nexus\ of \yes, is being where it is best to be. For \citet{one who is
  as he ought to be, and as it is expedient for him to be, such that he is
  unable to lose this state, is freer than one who is such that he can lose it
  and be led into what is indecent and inexpedient for him.}{AnselmFreeWill}{I}
 
The question is not if one, by a universal decree of human nature, by a solid
and undeniable, natural or unnatural law, is free or not, if one's will chooses
the ice-cream because of one's upbringing, social dependencies and digestive
problems or else just because it chooses so, in a complete indeterminacy of
emptiness.  The question is if one, by the power of one's {spirit}, is able to
live the apparent paradox of submission to the contentless \co{command}, and
thus become worthy of receiving freedom from \co{above}. \co{Actual} freedom is
but a side-effect, an aspect of \co{love}, of submission to the \co{command} and
the resulting \co{non-attachment}.  If this sounds like a contradiction then we
are pleased -- as far as freedom is concerned, this will suffice.

\subsubi{Responsibility}
Let us close this list of examples with a seemingly lesser and more modest
issue: responsibility. 
We touched upon this in connection with the original sin in
\ref{sub:privativum}.\refp{pa:originalSin}, footnote~{\small{\ref{resp}}}. 
%p.\pageref{resp}. 
Genuine responsibility is the \co{actual} attitude of response to the higher
voice. When this \co{trace} is followed all the way to the \co{absolute origin},
responsibility coincides with what the theological tradition called
\wo{obedience to God's will} and what we have recognised as \co{humility}.

\pa\label{pa:respInv} \inv 
% Responsibility here is the same as \co{community} and \co{sharing}; 
% there is no question about any choosing and willing at this level.
% participation
We have said, \refp{pa:originalSin} and \refpf{pa:originalSinII}, that the
original sin is neither willed nor deliberately committed and thus,
technically, it is not a personal sin. Yet, as we share in the
penalty for the original sin without personally sharing in the sin 
itself, many\noo{, for instance Abelard,} maintained that one 
may have to endure punishments which one has not merited.
% We share in the original sin, if not in any more specific way, then at least in
% carrying the responsibility for it.
We might exaggeratedly say that one is responsible even for sinking into the mud
of despair -- not because one can do something specific and avoid this sinking
once it started, but because one is still responsible for
taking up the challenge of not accepting the \co{visible} impossibility of
relief, that is, of accepting its \co{invisible} possibility.

We can exaggerate even more than that.  As a form of sickness of \co{existence}
which loses its \co{concrete foundation} in the \co{absolute}, the original sin
(in our sense) is evil and for it, like for any other evil, we are responsible.
Responsibility, as a response to the \co{absolute}, is simple non-acceptance of
any evil, of \co{alienation}. In this sense, everyone is genuinely responsible
for {\em all evil and all sins} which are committed, not only for those which
one has committed oneself. \citet{Every man who sins, sins against all people and
  every man is to some degree guilty of another's sin.}{Devils}{At Tichon's
  II\kilde{p.699}}
It is unacceptable 
to all manner of thinking involved into any form of
subjectivistic reduction -- of sin and guilt to feeling of sin and guilt, of
freedom to unconstrained voluntary choice, and eventually also of
responsibility to such a choice.
% (where, by the way, \thi{freedom} can appear only as arbitrariness,
% since every meaning carries with it a \thi{threat} of external authority). 
But we are responsible not only for what we choose but also for what we are; not
only for the \co{subjective acts} of choice, but also for choices which we live,
even though they were made
before and \co{above} us; not only for what we could somewhat, voluntarily and
actively repair but also, even primarily, for all that we can not.  One resists
such a responsibility, firstly, because it seems to restrain one's sense of
freedom and, secondly, because it cannot possibly be put into any specific
action. With regard to freedom we have just seen that it amounts to
\co{openness} to the \co{communion} and not to fortifying oneself behind the
walls of private choices. And this apparently \thi{empty} and
\thi{unproductive} sense of responsibility \thi{for everything} (which is just
another side of the deeper sense of guilt, not intended nor actively caused and
yet committed), could be equally well characterised as sympathy and fellow
feeling for everything, as a mere compassion with all the victims, a mere
recognition of the evil which met them, and as repentance for evil as such -- 
as \co{love} of universal \co{communion}. Regarding
  \thi{unproductivity}, let us quote Scheler's response to the same accusation
against repentance: \citet{The jovial gentlemen say: Do not repent, but design
  good projects and do better in the future! But the jovial gentlemen do not say
 whence the strength for designing good projects and even more for their execution
 should be fetched, if no prior  liberation and self-empowerment of the person, 
 through repentance, against the determining power of the past takes
 place.}{MaxReue}{p.36.\orig{Die jovialen Herren gar sagen: Nicht 
  bereuen, sondern gute Vors\"{a}tze fassen und Zuk\"{u}nftiges besser machen!
  Aber dies sagen die jovialen Herren nicht, woher die Krafte zum Setzen der
  guten Vors\"{a}tze und noch mehr die Kraft zu ihrer Ausf\"{u}hrung kommen
  soll, wenn nicht die Befreiung und die neue Sichselbstbem\"{a}chtigung der
  Person durch die Reue gengen\"{u}ber der Deteminationskraft ihrer
  Vergangenheit vorher erfolgt ist.}} Strangely enough, strength to carry out
\co{actual} tasks is an \co{inversion} of \co{humility}, and increases
proportionally to the latter, \refp{pa:humbleStrong}. 
%I carry responsibility for your wrongful acts.

The universal scope of this responsibility, surpassing any particular
\co{actuality}, is a \co{sign} of continuity between the \co{actuality} and its
\co{invisible origin}.  This form of responsibility may be different from what
is commonly, let alone legally, understood under this name, but responsibility
it is nevertheless (if not in other way, then simply by \co{founding} any
\co{active}, \co{visible manifestations} of responsibility.)  Refusing it, one
breaks the continuity of Being, reduces \co{oneself} to mere \co{subjectivity}
and the \co{communion} to mere association. And every \co{spiritual} reduction
is a form of \co{alienation}.  \noo{eventually, of evil.}  \noo{ In a sense, it
  is responsibility with no consequences; like every \co{spiritual aspect}, it
  does not imply any specific feelings nor require any specific actions. But it
  involves one is a special form of being. This responsibility is more l. It
  coincides with the highest degree of \co{humility} in which, according to the
  rule of St.~Benedict followed here also by St.~Bernard, compassion bewails
  one's faults and blemishes which, in a sense, are {\em all} the faults and
  blemishes.  }

% Here it is the sense of responsibility even for things I have
% not done or committed, for all the sins of the whole world, the sense of
% compassion and regret for the sins which are committed -- not related to my
% active and voluntary acts, but to what happens

\pa \mine In a more specific sense, \co{I} am responsible for \co{my} whole life
and, as strangely as above, also 
for all the evil which affected me. For even if it is not accepted voluntarily,
it affects me only if I consent to it -- consent, perhaps, by not seeing
anything, perhaps, by seeing in it no evil, perhaps by giving up the resistance
to it -- in either case, even if not accepting voluntarily so still
accepting. More precisely, I 
am responsible not for the evil which affects me as such, but for the fact {\em
  that it affects me}, that it affects me {\em as evil}. 
%Eventually \co{my soul} is the only place where evil can 

Herein lies an important difference between various people not only reacting
differently but also developing differently under apparently quite the same
circumstances. Eventually, \co{I} -- and only \co{I} -- am fully responsible for
what \co{I} have become -- blaming the society, school situation, family
conflicts, and what not, may have some merit {\em only in so far} as the
\co{objective} improvements of the respective social contexts are concerned.
Being exposed to evil influences is not a sin but one sins very easily by an
irresponsible response to such an exposure, a response by which one damages
oneself or, what amounts to the same, finds inexcusable evil in the world which
one makes responsible for the evil which infects one's \co{soul}.
Irresponsibility, whether inability to \co{recognise} one's responsibility or
escape from it, is thus not only escape from suffering but even from things
which are seen as its source.  It is a \co{sign} of broken \co{communion}, an
\co{aspect} of enslavement by evil -- eventually, the \co{posited} evil -- of
the world. In the ultimate form of such an \co{ingratitude} men accuse God for
having created or allowed all these evils and \citet{lay blame upon [...] gods
  for what is after all nothing but their own folly.}{Odyssey}{I:32} Yet,
through such a distancing oneself from the evil and responsibility for it, one
only deepens \co{alienation} from oneself. For one thus forgets that God acts
{\em only} through one's soul, that \citeti{[a]ll works are performed by warmth,
  [and] if the fiery love of God grows cold in the soul, the soul will
  die.}{Eckhart}{\btit{German Sermons}, Lk.VIII:54. [\citeauthor*{EckSelected}{
    26}, \citeauthor*{LW}{ 85}]}

% It does not in the slightest help \co{me}, because it does not help
% to justify me...  
%\newpa 
\pa \act\hspace*{-.5em}-\imm Responsibility from \refp{pa:respInv} can be
likened to an impulse to help and repair all the suffering which, although
impossible to follow for trivially practical reasons, \co{founds} all particular
\co{acts} of genuine help.  It is responsibility of a response to the call of
conscience, a response to the \co{command} reminding one about the suffering --
one's own or others' -- hearing which and remaining indifferent would amount to
a consent.  Of course, we do not suggest that the \co{communion} expressed in
the sense of responsible compassion with all suffering should be brought to the
level of \co{actuality} in the same universal form.\ftnt{Such constancy of the
  sense of guilt and repentance without any accompanying shrinkage of
  personality or decrease of energy is probably as seldom as holiness. An
  average human being can stand only some amount of suffering (whether one's own
  or others') and when this limit is passed, the reaction is a withdrawal:
  \wo{I can't stand it any more. We are not created for {\em this}.} It is the
  natural sense of entitlement which gets contradicted and attempts to protect
  itself. The saintly person, on the other hand, does not live his \co{attached}
  entitlement but his \co{thankfulness} which does not pick on the offered
  \co{gifts}. This \co{openness} to everything, including suffering, is
  sensitive to all kinds of \co{ingratitude}, that is, guilt. And
  thus,  \citef{[it belongs] to the increase of humbleness and holiness in man that --
    as life testifies to the most holy -- the sense of guilt gets functionally
    {\em refined} accordingly to its [guilt's] objective decrease and that
    thence even lesser misconduct is heavily experienced.}{MaxReue}{p.48}
 \orig{[Es geh\"{o}rt] zum Wachstum der Demut und Heiligkeit in Menschen,
    da{\ss} -- wie das Leben aller Heiligen bezeugt -- das F\"{u}hlen der Schuld
    gerade mit ihrer objektiven Abnahme sich funktionell {\em verfeinert} und
    da{\ss} daher immer geringere Verfehlungen schon schwer empfunden werden.}}
In every \co{actual} situation challenging one's compassion, and hence also
responsibility, one has to weight the possibilities of \co{actually} following
the call against multitude of other factors.  Even if all such factors prevent
one from \co{actually} doing anything specific, the mere compassion with the
needy ones is also an expression of responsibility. (Solon asked \citet{what
  should be done to make people commit as few crimes as possible, answered that
  also those who did not suffer from the crime, should be as much moved by it
  as those who were its victims.}{DiogenesL}{I:2\kilde{p.40}})

Irrespectively of situation and other factors, one carries full \co{actual}
responsibility for {\em everything} one does and everything one leaves undone.
The question may only concern the degree of this responsibility and its
consequences.
%
This \co{actual} responsibility for one's \co{acts} and deeds is not something
one may assume or not assume, but something that follows from their
\co{foundation} in one's being and its \co{communion}.  Ultimately,
responsibility is simply the fact of, on the one hand, ontological
\co{foundation} and, on the other hand, of the influence of the lower levels on
the higher ones. The first \co{aspect} concerns the \co{actual} responsibility
for everything one has done and is doing, also for avoiding or removing the
consequences of the evil which affected one. Having done something blameworthy,
one may point to being temporarily unconscious, affected by drugs or childhood
trauma, but no such excuses provide a complete justification.  Eventually, one
is responsible for what one is, and {\em everything} one is doing is a
\co{manifestation} of that. In the most trivial situations, saying \wo{I am
  sorry}, one is not sorry for one's bad will and intended acts but exactly for
something one has done {\em without} intending it, for something which merely
happened {\em through} one. \wo{I am sorry} not for intending to collide with
another person but for the very fact that I did not notice him {\em and} run
into him.  This is the \co{actual} responsibility for my past.\ftnt{Agamemnon
  admits: \citef{it was not I that did it: Jove, and Fate, and Erinys that walks
    in darkness struck me mad when we were assembled on the day that I took from
    Achilles the meed that had been awarded to him.  What could I do? All things
    are in the hand of heaven, and Folly, eldest of Jove's daughters, shuts
    men's eyes to their destruction.}{Iliad}{XIX:86ff} But this workings of
  Folly (or as others translate it, infatuation, momentaneous loss of control)
  is not cited by Agamemnon as any excuse; higher forces have been at work, but
  they worked through him: \citefib{I was blind, and Jove robbed me of my
    reason; I will now make atonement, and will add much treasure by way of
    amends.}{Iliad}{XIX:137}\kilde{Dodds discusses more}} The other \co{aspect}
is directed towards the future, is responsibility for what one becomes. It
amounts to the fact that everything one does, every \co{actual} project and
\co{act}, may with time seem to disappear from the \co{actual} memories, but it
nevertheless contributes to the \co{virtual} seeds of one's \co{soul}. Although
the exact measure of this contribution and its eventual consequences are only
seldom possible to discern, its very fact is hardly disputable,
II:\ref{sub:lowHigh}.  In this sense, every evil done, increasing the
\co{alienation} of the \co{soul}, breeds its own punishment. The call to taking
up responsibility for this evil amounts to the call of conscience and, following
it, to repentance and voluntary atonement.

\pa \co{Actual} responsibility not \co{founded concretely} in the higher
\co{community} gets involved into interminable search for the criteria
separating things for which one should be (held) responsible from those for
which one should not.  Such criteria are certainly of forensic importance but do
not concern us. Lack of \co{foundation} demands often explicit assumption
of responsibility. It is like marking a new area as being \co{mine}, belonging
to \co{me}, falling under \co{my} responsibility. Such \co{acts} are often
required in various contexts of cooperation and subordination. But if this is
their only foundation, it witnesses to an \co{ego} which is sufficiently
\co{alienated} to believe that there are things -- sufficiently remote,
sufficiently \co{not-mine} -- for which one might not be responsible. Reducing
responsibility to such very specific contexts and situations goes hand in hand
with other forms of reduction. For responsibility is interwoven into a whole
\nexus\ of ontic presuppositions which have to be reduced when one attempts to
reduce its scope.\ftnt{\citeauthor*{Resp} conducts a systematic analysis.} For
instance, a temporal loss of consciousness can be used to exempt one from
responsibility for the act performed in such a state. This, of course,
presupposes a specific reduction of person according to which, for instance,
sleeping man is not a person. Well\ldots No! Of course, he is but\ldots Well.
Likewise, one will often use past history of a person to excuse his acts
implying, as it were, that they are merely consequences of bad influences of the
environment. It is then really hard to get out of the impasse because now, so it
seems, responsible person is only somebody not exposed to any negative
influences, as if the crucial \co{aspect} of moral responsibility did not concern
exactly the ways in which one is affected by and reacts to such influences.
Reduction of responsibility to only conscious and voluntarily intended acts is
just another form of reducing person to a \co{subject}.

\pa Everything we have said about responsibility concerns only \co{my}
responsibility, only what \co{I} am responsible for. No consequences follow for
imputing responsibility to others.  The problem with such attribution, like with
any demands of moral behaviour, is that nobody can be forced to recognise its
validity. One can not make somebody responsible for something -- one can at most
try to force him to take responsibility. This is an issue for the law
enforcement units and not for us. In contacts with people one will, of course,
assess the level to which they feel responsible for various things and act
accordingly, perhaps, by suggesting more responsibility than they are prepared
to admit. Rising children one will teach them taking responsibility -- directly,
by requiring them to \co{actually} take responsibility for various things, and
indirectly, just by being responsible the way one is. Our \thi{universal
  responsibility}, like everything else, is only for personal use.  It is the
way of avoiding \co{alienation}, of avoiding \co{positing objective} evil by
imputing responsibility for encountered suffering. \co{Objectivisation} of evil
and resulting tense responsibility is the domain of a rigid moralist who is
ready to censure every single act, of others' and sometimes of oneself, for its moral
shortcomings.  Strict and solemn seriousness of such an attitude finds the more
offenders and becomes the more self-justifying and irritably sensitive, the more
doubts about its ultimate justification germinate at its bottom, that is, the
less \co{concretely founded} is its perspective on human \co{existence} which
terminates at its moral dimension, if not at the level of single \co{acts}.

\noo{Seriousness which can not stand a slip or a joke is as bad as irony which
  can not stand any seriousness.  Each in its way, witnesses to an exaggeration:
  one, to the exaggeration of the inflexible formalism which distributes
  responsibilities as name-tags on a conference, and the other, to the
  exaggeration of the formlessness of the background which, indeed, underlies
  all \co{actual experiences} but does not constitute identities of their
  contents. } \noo{ which everything eventually, unless kept with the power of
  the authority -- whether tradition, custom, communion, mutual understanding,
  or genuine dedication...}


\noo{ The burden of the insistent responsibility is not at all related to the
  difficulties of acting in a prescribed and purposeful manner which happen to
  go counter one's inclinations. This burden is the result of a deeper
  \co{alienation} which misunderstands responsibility as a matter of
  distribution of specific duties, which makes one think that mastering a
  situation requires control over events which turn out to be beyond one's
  control.  \co{Dissociation} of \co{actualities} without \co{concrete
    foundation} makes taking up responsibility for one issue a painful offer and
  regret over the impossibility of taking it also for another, equally important
  and valuable. As usual, what one ends up doing in practice may, at least from
  \thi{outside}, seem indistinguishable, but the ways in which it is done differ
  infinitely.
  
  The more such \co{alienation}, the more despair and the more of the
  accompanying sense of hopeless responsibility. In the extreme cases, this may
  lead to rigidity of a moralist for whom every single act must be consciously
  verified for its moral content, which insistence ends in fact in verbal
  formalism and lists of laws forbidding and prescribing all the details of
  one's conduct.
  
  \say add: Seriousness, solemn dedication to the objective goals, erasure of
  oneself in the directedness of the task, is just {like the resolute dedication
    to the opposite pole of personal responsibility, and} like seriousness in
  general -- it }

\noo{ We are also responsible for all the identities and substances: in the
  inanimate world, they are relative to our existence. We observed that their
  being \thi{in themselves} can be dissolved, just like a child can dissolve
  every issue by asking naively penetrating questions. Drawing from this the
  consequence that no identities and substances are to be taken into account
  would be ... childish or irresponsible. Blaming facts for one's behaviour...
}

\sep%\vspace*{-3ex}

\pa All the above examples might suggest an ideal which seems as fantastic as
unrealistic. But ideals which do not and can not live are \co{posited} phantoms,
regulative ideas or ideological goals, \co{egotic} projections, in short,
\co{idols}.  Well, for the first, \citet{[a]ll things excellent are as difficult
  as they are rare.}{SpinozaEthics}{[the last sentence]} More significantly, all
\co{invisible aspects} of \co{concrete foundation} appear ridiculous when
reduced to the \co{actual} categories; the appearance which is only strengthened
by possible \co{inversions}.  All-embracing, \co{spiritual} responsibility, when
attempted at the level of \co{actual} feelings and expressions, will result not
in any factual responsibility and acts but rather in the hysteric lamentations
of elderly (and good) women over the evils and cruelty of the world.
\co{Actual} responsibility for everything, \co{actual} freedom of every single
choice and action, \co{actual} communion with every man one meets, \co{actual}
love towards every person, animal and thing -- all such reductions reflect only
the reduction of \co{self} to \co{ego} (II:\ref{sub:objsubj}.\refp{pa:SelfEgo}),
the fundamental misunderstanding which attempts to interpret the \co{spiritual}
in \co{visible} terms, to turn the quality and \co{rest} of \co{acts} into
facts, the wind into stones. But \wo{[t]he wind bloweth where it listeth, and
  thou hearest the sound thereof, but canst not tell whence it cometh, and
  whither it goeth.}$^{\ref{ftnt:wind}}$ \co{Concrete foundation},
\co{forgetfulness} of the spiritual, dispenses with any such reductions which is
just another side of dispensing with any attempts to re-produce \co{invisible}
as some \co{visible} image, to reach the infinite making only finitely many
finite steps. Leaving all \thi{whats} aside, it keeps \co{thirsting} without
seeking, it keeps the \co{clear} sight of the fact \co{that} its \co{actuality}
is indeed \co{founded} in the higher sphere, and that it carries responsibility
for the \co{concreteness} of this \co{foundation}, even if it is not entirely
its work.

Unlike unreachable ideals which can only be approximated, we are dealing here
with the most \co{concrete presence} -- it need not be approximated because it
can not be made \co{actual} and palpable. Trying to approximate it in one's mere
\co{acts} is already a sign of failure, of an image that it is reducible to mere
\co{acts} and thus, that it can be captured by them. But one can not defend it, nor
fight for it, nor try to reach it -- one can at most live it. Some
impossibilities are difficult only because they are too simple. But, as
Confucius says, life is simple and only man insists on making it complicated.
\co{Any experience} of love which is not fully \co{incarnated love} is distorted
and is \co{aware} of its imperfection.  Most \co{experiences} are of this kind
but \co{love} is not, for this reason, an unattainable ideal, a regulative idea,
an inaccessible goal. If only one person in the whole human history reached its
\co{experience}, this would be enough to maintain its universal possibility.
There are all reasons to suspect that there were more than only one and, in case
of doubt, one should look carefully around oneself, {expecting the unexpected.}
The imagined absence is often only the inability to \co{recognise} the
\co{presence}.  \co{Concrete love} is not a mere state of mind, a mere feeling,
an unclear image of something ideal and in its mere desirability completely
\thi{unreal} because not \co{actual}. It is a thoroughly \co{concrete
  manifestation} flowing from the center of personal being and embracing the
whole life, from \co{above} all general thoughts and \co{qualities}, through all
\co{actions} and goals, to the most \co{immediate} \co{acts}.


\pa
But one might still ask: where is the necessity? Where is the necessity of
\co{concrete foundation} of all these lower elements in the higher ones and,
eventually, in the  \co{spiritual love}? We
can easily imagine a man who is strong but not patient, who is strong and
patient but not humble, who finds great sense in life without having confronted
\co{nothingness}, who is alert and vigilant but not \co{open}, who acts morally and
responsibly not only without any higher \co{command} but exactly because he 
does not recognise any higher authority, etc., etc., etc.
Indeed, we can as easily imagine such a man as we can imagine Pegasus. For
imagination enables us exactly to put together, almost arbitrarily, various
features earlier \co{dissociated} from each other. The games which
\co{reflection} can play with its \co{dissociated signs} are almost
unlimited. And they affect the world, just like other \co{distinctions} do. The
question is, however, whether such abstractly drawn \co{distinctions} and
arranged \co{complexes} correspond
to others, whether they can be woven into the \co{unity} of \co{existence} and
its \co{experience} of the world -- and that \co{concretely}, not as mere
imaginations.

Necessity of \co{concrete foundation} does not concern any specific aspect; as
we have emphasized, each particular element can be found at a lower level without
\co{concrete foundation} in the higher ones.  The old question, whether virtues
can be possessed separately or only all together, posits wrong alternative -- in
practice, in nature, both cases occur, albeit the former much more frequently
than the latter.  Focusing on one such feature, the difference between the two
cases might, for an \thi{outside} observer, seem slight to the degree of
insignificance. But the lack of \co{actually} observable differences and
behaviours does not, by itself, witness to anything of significance. In fact,
the difference is infinite and, like every infinity, \co{actually}
unobservable. So, after all, the alternative is real but it concerns something
much deeper than the mere occurrence of this or that virtue. It concerns the whole
person. 

The \co{unfounded} virtues can be \co{dissociated} and appear in very different
constellations.  One will then often strive for strengthening some of the
virtues one does not possess and such exercises can easily take big part of
life. Such virtues can be, piecewise, acquired. The process may turn out very
valuable and lead to a deeper development. However, there is also always the
chance that aiming particularly at, say, perseverance, one will keep biting
one's teeth and grow only more stubborn or more bitter, as the posited
perseverance keeps sneaking out of one's \co{actual} grasp never matching the
intended image. Developing (right) habits is a lengthy and complicated process
which even in the case of children escapes clear-cut rules and all too precise
guidelines. It is particularly difficult, if not outright impossible, when one
aims at only one specific aspect. The \co{unity} of a person has also this
trivial consequence that, for instance, quitting smoking may result in increased
consumption of candies, while developing perseverance may result in a loss of,
say, the sense of spontaneity. Every change of no matter how small aspect
affects the whole person or, as one also puts it, must be integrated into this
whole. Such psycho-\co{egotic} manipulations have always side-effects which are
as understandable, having once emerged, as they were unpredictable in advance.

According to St.~Thomas, the theological virtues of faith, hope and charity are
infused directly by God, and only by Him.\noo{SumTh.II:I.q62.a} But also moral
and intellectual virtues which, under natural circumstances, can be acquired
separately by human efforts, can be infused by God, so that they are
\citet{caused in us by God without any action on our part, but not without our
  consent.}{SumTh}{II:I.q55.a4.r6} The acquired virtues of this kind function,
so to speak, each for itself and without reaching the personal center. Their
infused versions, on the other hand, are \co{concretely founded} in the
\co{unity}: first, the lower \co{unity} of the \co{invisible} sphere where, as
we saw in II:\ref{sub:invArch}, various \co{aspects} cannot be meaningfully
\co{dissociated} from each other and, eventually, in the \co{unity} of the
\co{existential} center, where \co{grace} becomes \citet{charity, which through
  an image in the mind exhibits what is absent as present to ourselves, through
  love unites what is divided, settles what is confused, associates things that
  are unequal, completes things that are imperfect! Rightly does the excellent
  preacher call it the bond of perfectness; since, though the other virtues
  indeed produce perfectness, yet still charity binds them together so that they
  can no longer be loosened from the heart of one who loves.}{GregoryEpist}{Book
  V:LIII. To Virgilius, Bishop.}  Thus, even if for an \thi{outside} observer,
two kinds of virtues can seem indistinguishable (since patience {\em is}
patience, temperance {\em is} temperance), the ones are as if added to the
person while the others flow from the person.  The \co{concretely founded} ones
are only \co{manifestations} of the \co{spiritual unity} of the person, the
\co{visible signs} of \Yes. Only this \co{transcendent foundation} gives them
all their force and adds the \co{invisible rest} -- continuity and \co{unity} --
making all the difference.


\noo{
Now besides the theological virtues, according to the doctrine of St. Thomas,
there are also moral and intellectual virtues of their very nature Divinely
infused, as prudence, justice, fortitude, and temperance. These infused virtues
differ from the acquired virtues
*       as to their effective principle, being
immediately caused by God, whilst the acquired virtues are caused by acts of a
created vital power;
*       by reason of their radical principle, for the
infused virtues flow from sanctifying grace as their source, whereas the
acquired virtues are not essentially connected with grace;
*       by reason of the
acts they elicit, those of the infused virtues being intrinsically supernatural,
those of the acquired not exceeding the capacity of human nature;
*       whilst
one mortal sin destroys the infused virtues, with the acquired virtues acts of
moral sin are not necessarily incompatible, as contrary acts are not directly
opposed to the corresponding contrary habit.
}

\noo{ \citeti{O how good is charity, which through love exhibits absent things
    in an image to one's self as though they were present, unites things
    divided, sets in order things confused, associates things unequal,
    consummates things imperfect!  How rightly the excellent preacher calls it
    the bond of perfectness, since the other virtues indeed produce perfectness,
    but yet charity so binds them that they cannot now be unloosed from the mind
    of hint that loves.}{St.~Gregory the Great}{vol. XIII: Bk.XIV: Epistle XVII
    to Felix, Bishop Messana [probably spurious]} }



\section{The analogues of God}
\secQpar{5}{7}{Let your communication be, Yea, yea; Nay, nay: for
               whatsoever is more than these cometh of evil.}{Mt. V:37}
\noo{\begin{enumerate}\MyLPar
\item IncomprehensiblewholeTe
\item Names -- only proper names/naming
\item But also adjectives/predicates...
  \begin{enumerate}\MyLPar
    \item 'Names'... perhaps\\
      not of essence, but of action, yet...
    \item All essence beyond the action is only incomprehensible/indistinct\\
      while action = confrontation
    \item only spiritual choice
      \begin{enumerate}
        \item not of in-itself, but of the way of finding it
        \item 'names' can be correct or incorrect -- they should help one to say
          \Yes
      \end{enumerate}
    \item Two faces of the One
  \end{enumerate}
\item creation of evil...
\end{enumerate}
}
\pa
Book I, in particular, I:\ref{sub:OneMany}, described the bareness of the notion
of the \co{one} as the ontological \co{origin}. \co{Concreteness} of
\co{spirituality} is still \co{founded} in this bare \co{nothingness}.  But
while in Book I we were concerned with the merely ontological meaning of the
\co{one}, now we want to emphasize that also with respect to the \co{spiritual
  presence} and \co{concrete foundation}, equipping Godhead with all kinds of
attributes, whether in essence, in fact or only in name, is inappropriate
whenever it may obscure the fact of his complete \co{invisibility}.

One has often emphasized the \thi{human need to speak about God}. There may be
such a need, and it may be very human, but this is exactly the question and not
the answer.  The need reflects the fundamental meaning of the divine in our
life, the meaning which awaits if not an explanation nor even an account, so at
least an expression. But \citeti{[b]e not rash with thy mouth, and let not thine
  heart be hasty to utter any thing before God: for God is in heaven, and thou
  upon earth: therefore let thy words be few.}{Eccl.}{V:2} Speaking about God
may be harmful when it \co{posits} too many attributes and dwells on His
inaccessible \thi{essence}.  \citet{We cannot approve of what those foolish
  persons do who are extravagant in praise, fluent and prolix in the prayers
  they compose, and in the hymns they make in the desire to approach the
  Creator.}{Perplex}{I:59} For where is the border between praise and appraisal,
and then between praise and self-complacency over praising the right God in the
right way? There may be a border and it may be marked by the extravagant
exaggeration. Yet from the start, speaking about \co{nothing} we use
distinctions and distinct words which, in the last resort, means that we can
only speak about ourselves, about our \co{confrontation}. \citet{Everything
  which falls under a name is originated, whether [we] will or not.}{Stromata}{
  V:13} And so, only \citeti{silence is praise to Thee.}{Ps.}{LXV:2. Maimonides'
  interpretation is the same as St.~Jerome's, cf.
  II:footnote~\ref{ftnt:silencePraise}.  In \abr{kjv}, however, the
  whole verse reads: \wo{O thou that hearest prayer, unto thee shall all flesh
    come}.}
Having made these reservations, let us nevertheless pay some tribute to the human
need to speak about God. This need reflects the fact that God is not
merely an inaccessible and plain nothingness but is \co{concretely} and
\co{absolutely present}, which \co{presence} has fundamental \co{existential}
significance. 

\subsection{Proper names}
\pa No name is adequate for God, just like no name is adequate for a
person.\ftnt{Cf.~footnote~\ref{ftnt:latinNames} concerning the
  homonymy of \wo{name} standing for both nouns and adjectives.} This is why
language has proper names, for proper name is the only adequate name for a
person; no other name, not to mention any predicate or more or less definite
description, will even approach this adequacy. So is it with God -- \wo{God} is
His only adequate name. Perhaps, it should be {\sc jhvh}, for \wo{God} may be
used about lesser gods and idols, too.\noo{But it is not important what name is
  being used or misused, but what is meant.} The fact that, saying \wo{God}, one
can mean various things does not make the name -- when applied properly -- less
adequate, just like knowing two persons with the same name \wo{Thomas} does not
make any of its applications wrong or inadequate.  They are both persons and
each has {\em a unique} name (instead of two persons with the same {\em name}
\wo{Thomas}, one could rather speak of persons with the same {\em names}
\wo{Thomas}.) It is only a trivial accident of secondary importance that their
unique names have the same linguistic appearance. (The use of patronymics, of
family names, middle names, etc. can be seen as an attempt to keep the
linguistic appearances in accordance with the uniqueness of the named persons.)

The uniqueness of a proper name follows from the fact that the word used as a
name is inseparable from the \co{acts} of naming. There is hardly any
syntactically identifiable class of words which are \thi{names}.  Sure,
\wo{Thomas} and \wo{Berit} and hundreds of other words are standard names.  But
they do so exclusively for the reason of being so used, that is, of being used
for {\em naming}. Name is inseparable from the fact and \co{act} of naming, it
is the sedimented epitome of the latter, while naming itself is a
trans-linguistic, in fact, trans-phenomenal event of recognising uniqueness of
this \co{concrete existence}. Unrepeatability, ultimately the uniqueness of
\co{existence}, is the \co{conceptual} content of every proper name.  The actual
name, as well as the \co{act} of naming, is only an expression of the deep event
of naming in which one recognises the unrepeatable character of a being,
usually, of some \co{existence}. Getting a child, the parents have already named
it, already long in advance, even if they still do not know what actual name it
will get. The linguistic expression of this event, the \co{actual} naming,
endows it with the immediately recognisable aspect of \co{non-actuality},
solidifies it into a lasting, usually even social, fact. One could say, proper
name \co{reflects} the \co{eternal aspect} of the named \co{existence} in the perpetual
consistency of all different \co{acts} of naming.

\pa
Corresponding to the absolute uniqueness, a name does not capture any
essence, it is, as often observed, conceptually empty.
%
% (It would be ridiculous to say that \wo{Thomas} has any extension -- all persons
% named \wo{Thomas}, or perhaps, that its intention is the set of all persons with
% the name \wo{Thomas} in all possible worlds. Rather, all living beings, for I
% could have a dog called \wo{Thomas}.)
%
This conceptual emptiness makes proper names the most \co{concrete} among the
linguistic \co{signs} -- they are understood {\em uniquely}, or not at all. And
this unique understanding involves only \co{that}, identification of the person
named, identification which happens within the \hoa, but which only reflects the
\co{unity} beyond any concepts.  This trans-conceptuality of names expresses the
highest respect and recognition of the named -- in general, we name humans. In a
similar way, we express some amount of recognition and respect naming other
living beings, pets, etc.. But if one started naming one's pencils or pieces of
furniture, we would react to the misunderstood endowment of such disposable
things with the metaphysical quality of trans-conceptual being.\noo{(But as
  usual -- where is the border\ldots?)}  Names like \wo{Sitting Bull} or
\wo{Crazy Horse}, or usual {nicknames}, do intend to express some essential
aspects, one might perhaps say, some \co{concept}. But they are proper names
only in virtue of the {\em uniqueness} of their reference.  Proper name does not
capture any essence but only indicates the site of its possibility, the unique
individuality.  It indicates the ultimate limit of \co{distinctions}, beyond
which not only no more \co{distinctions} are made but were no \co{distinctions}
could possibly be made.
%It indicates a \thi{substance} which in our account is only \co{existence}.


\subsection{Names}\label{names}
The decisive issue is, just as with \wo{Sitting Bull} and the like, {\em what}
one intends when using a name. One can endow \wo{Sitting Bull} with the derogatory
content by simply reducing it to the mere description, that is, ignoring the
fact that it is a proper name. Likewise, forgetting that \wo{God} is a proper
name, one tends to reduce its meaning. The traditional discussion is concerned
with the admissible, if any, ways of such a reduction; not with the proper names
but with the names understood as possible predicates of God.

% The following few paragraphs are essentially only a repetition of Section
% \ref{sub:OneMany}. 
\pa \citeti{All creatures have existed eternally in the divine essence, as in
  their exemplar.  So far as they conform to the divine idea, all beings were,
  before their creation, one thing with the essence of God.}{Henry of Suso}{}
\citet{ `That which is perfect' is a Being, who hath comprehended and included
  all things in Himself and His own Substance, and without whom, and beside
  whom, there is no true Substance, and in whom all things have their Substance.
  For He is the Substance of all things, and is in Himself unchangeable and
  immovable, and changeth and moveth all things else.\noo{But `that which is in
    part', or the Imperfect, is that which hath its source in, or springeth from
    the Perfect.}}{TheolGerm}{I} Much interpretation (and even
misinterpretation) would be needed to make such fragments acceptable.\ftnt{E.g.,
  to \wo{comprehend and include all things in Himself} must not be taken in the
  Platonic sense of the pre-existing, ready-made archetypes; one should
  carefully distinguish \wo{His own Substance} from all the other
  \wo{Substances} (\thi{Substance} is actually understood by the author
  \citef{not as a work fulfilled, but as\noo{[substance or]}
    well-spring}{TheolGerm}{XXXII} which we could interpret in terms of
  \co{virtuality}) etc., etc..}  The language of Platonic exemplars, combined
with the need to emphasise God's goodness and other positive qualities, have
made it almost impossible to think of Godhead otherwise than as a collection of
some definite, yet always mysterious, attributes or essences which in an equally
mysterious way are meshed into one.  On the other hand, it was the image of the
highest somewhat \thi{containing} everything lower, as a box contains sand or as
genus contains species, that forced one to double things with exemplary ideas
and, eventually, to make Godhead responsible for all the details of the
\co{visible world}.  But \co{virtuality} does not \thi{contain} all the
\co{distinctions} which flow from it, except as their \co{indistinct}
\co{origin}.  {Substantiality} of self-identical, independent entities has been
discussed earlier in, hopefully, sufficient detail.  Application of this
category to the \co{invisibles} leads unavoidably to antinomies.  But such an
application is by no means necessary, even though similar examples of modeling
\co{invisibility} of the \co{origin} on the Platonic ideas superimposed on the
Christian intuitions, could be multiplied \la{ad nauseam}.

\noo{Names
  
  \pa \citet{One of the greatest favours bestowed on the soul transiently in
    this life is to enable it to see distinctly and to feel so profoundly that
    it cannot comprehend God at all.  [...]  they who know Him most perfectly
    perceive most clearly that He is infinitely incomprehensible. [\ldots] those
    who have the less clear vision do not perceive so clearly as do these others
    how greatly He transcends their vision.}{StJohn}{?}
  
  Let's only remember that such an `incomprehensibility' is not the result of
  any lack in our faculties, but the only thing to be known.  This
  \co{transcending} is not any convergence towards an inaccessible limit of
  fullness, but simply the lack of anything \thi{comprehensible}. The \co{one}
  \co{transcends} everything because the lack of \co{distinctions} is simply
  beyond any \co{distinctions}.  There is \co{nothing} to be known about it and
  no names are necessary even if almost all are possible.  Whether \la{El}
  (\wo{Mighty}) or \la{El Shaddai} (\wo{Allmighty}), whether \la{Adonai}
  (\wo{Lord}) or \la{Logos} (\wo{Word}), whether \la{YHWH} (\wo{I am who I am})
  or \la{Elohim} (\wo{Mighty One}), whichever of the 99 names one chooses, one
  could always invent a new one, because God can be named by any name, and every
  name is appropriate if only pronounced with the \co{humble love}.  All the
  disputed and undisputed names and attributes are but the attributes arising
  from the attitude -- the \co{spiritual} attitude -- of the one who pronounces
  them.  The question is not what names are and what are not appropriate, but
  what attitude makes a name appropriate.\ftnt{I saw once a Professor of
    Philosophy and Logic performing a formal proof that a God about whom we
    predicate \thi{Goodness} and \thi{Allmight} could not exist.  The proof was
    correct and the Professor seemed quite satisfied with himself.  One was led
    to suspect that he actually meant it seriously.[moved...]}  Just like the
  language of teaching will typically differ from that of communication between
  peers, so the adequacy of a name will depend on the context.
% \subpa Indeed, if one wants to imagine the need to speak about God, 
% the \co{spiritual love} will direct one towards
% silence and, perhaps sometimes, lend one the words.  \citet{Just say
% whatever is given you at the time, for it is not you speaking, but the
% Holy Spirit.}{Mrk}{XIII:11} If such \co{love} is absent, any naming 
% will be wrong.
%
  The questions \wo{Shall we say that God is?}, \wo{Shall we say that He is
    not?}, \wo{That He is something}, \wo{That He is nothing?}, \wo{That He is
    not something?}, \wo{That He is not nothing?} may all be answered in
  positive because they are all questions about a mere way of speaking.  Their
  insistent presence in the history is but a result of a need -- but not so much
  the need to speak about God, as the need to speak, eventually, the need to
  reduce everything to the sphere of \co{visibility} to which one remains
  incurably \co{attached}. But the need to guide the \co{attached} and perplexed
  should not overshadow the truth which one might want to convey.

}%end \noo{Names


\subsub{Godhead vs. God}\label{se:GodGodhead}
The \co{indistinct} is ultimately \co{invisible} and \co{nothing} can be said
about it. \citeti{Thou art a
  God that hidest thyself.}{Is.}{XLIV:15} What then is it one is actually speaking
about, and what is it one is actually saying?  \woo{Why dost thou prate of
  God?  Whatever thou sayest of Him is untrue.}{\kilde{Eckhart}} For \citeti{the
  Godhead is nameless, and all naming is alien to
  Him.}{Eckhart}{\citaft{Huxley}{VII\kilde{p.125}}}

\pa\label{GodheadGod}
Any doctrine starting with \la{deus absconditus} who,
somehow, influences or at least remains \co{present} in the world, needs the
distinction which eventually amounts to setting the hidden and inaccessible on
one side and its action and manifest influence on the other.

\citeti{The Godhead gave all things up to God.  The Godhead is poor, naked and
  empty as though it were not; it has not, wills not, wants not, works not, gets
  not.  It is God who has the treasure and the bride in him, the Godhead is as
  void as though it were not.}{Eckhart}{\label{cit:getsNot}} The first division
  of nature, 
according to Eriugena, is the uncreated which creates, the God in his primordial
and incomprehensible reality; the second division is the created which creates,
God considered as the primordial and ideal cause(s) of all things. (Spinoza's
\la{natura naturans} versus \la{natura naturata} is only an ultimately
pantheistic expression of the same distinction.)  Pseudo-Dionysius distinguishes
the Undifferenced Godhead and its Differentiated
nature,\ftnt{\citeauthor*{DivNames}, II:3ff} which for him, as the obedient and
dedicated Christian writing after not only the First Council of Nicaea but also
of Chalcedon, could only mean co-substantiality of Father and Son, and in fact,
the Triunity of persons. Proclus says: \citet{It is thanks to beings with which
  they are conjoined that all gods receive names, and hence thanks to these
  beings knowledge of various subsistences of these gods is possible, though in
  themselves they are unknowable.}{Proclus}{\para 162} Gods themselves should
be, probably, thought as occupying even more specific, lower spheres than the
One itself and the ultimate \thi{One-beyond-the-One} which Iamblichus posited
above the Plotinian One. The distinction can be found in Philo from
Alexandria\ftnt{Already God's reason, the intellectual realm, is beyond
  language: \wo{that world which consists of ideas, it were impious in any
    degree to attempt to describe or even to imagine: but {\em how} it was
    created, we shall know if we take for our guide a certain image of the
    things which exist among us.}  [\citeauthor*{PhiloCreation}{IV}] Philo's
  opposition: the possibility of knowing \co{that} God is vs. the impossibility
  of knowing \thi{what} He is, is discussed e.g. in \citeauthor*{PhiloAD}.}  and
then appears clearly in Clement, according to whom \la{via abstractionis} leads
from the experience towards a spiritual unity above time, space and
apprehension, but which is still not the ultimate God, for \citet{God is one,
  and beyond the one and above the Monad itself.}{Paidagogos}{I:8.\kilde{Also
    I:71.1, FilSr, p.39} \btit{Miscellanies} {V:11} addresses specifically
  abstraction from creature leading to God who remains above every image.}
Anticipating the later neo-Platonic hierarchy of emanations, Numenius of Apamea
had to associate with the changeless and all-transcending God a Demiurge of dual
nature who, relating also to matter, could accomplish the creation of the world.
Plato refused to discuss certain things, yet the Demiurge from \btit{Timaeus}
seems to be an active symbol of the One, \wo{the father and maker of all this
  universe [who] is past finding out} and who \wo{committed to the younger gods
  the fashioning of [...]  mortal bodies}. In the sixth letter Plato asks a
friend to swear \citet{in the name of the God who is captain of all things
  present and to come, and of the Father of that captain and
  cause.}{PlatoLetters}{VI:323.D} \kilde{[after Coplestone, p.178]} The
distinction is present in Indian thought from the Vedic beginnings
(\gre{Prajapati}, Lord of Creatures vs. \gre{Purusa}, Soul of the Universe),
through the more personal relation of late Upanishads and
Mahabharata,\ftnt{E.g., Krishna says to Arjuna: \citef{I am the Father of this
    universe, and even the source of the Father}{Bhagavad}{IX:17}} with perhaps
the sharpest expression in the relation between the unmanifest, absolute
\la{Brahman} sustaining the universe and the manifest \la{Brahman} bearing
qualities and containing the universe.  In some less orthodox disguises, the
distinction may be discerned in Martion's Creator versus Saviour, the God of the
Jewish Bible versus the Christ of the apostolic gospel -- the roots of all later
gnosis with its opposition of the evil Demiurg and the good
God.\ftnt{\citeauthor*{Gnostics}, II:4\kilde{p.82}} Traditional Cabala
distinguishes between the hidden vortex of Godhead, \la{En-Soph}, and its
manifestations in the world through the intermediary emanations, \la{Sephiroth}.
According to Zohar, even the Tetragrammaton \abr{jhvh} does not denote the
hidden vortex but only the first and highest among the \noo{seven}
\la{Sephiroths}.\ftnt{\citeauthor*{YatesRen,Scholem}. The Cabalistic hierarchies
  of divine names and emanations meet and merge with the Christian counterparts
  in the Renaissance. Pico was familiar with the system of \la{Sephiroths} of
  Cabala when repeating the angelic hierarchies of Pseudo-Dionysius, and both
  traditions merge in Francesco Giorgi's \btit{De Harmonia
    Mundi}.\kilde{YatesRen,p.39}}\kilde{Gnostics:Sabbatai Cwi made it into
  something above the \la{Sephiroths}, a kind of divine self. IN: Spinoza,
  S.Nadler,p.286-7}
%
% Condemnation of some of Eckhart's theses reflects probably more the opinions of
% the Archbishop of Cologne, Henry of Virneburg, on the role of Beguines, than
% Eckhart's unortodoxy.  Pseudo-Dionysius was all deadly correct and orthodox as
% long as he was St.~Paul's convert and pupil, first bishop of Athens, skilfully
% confused with St.~Denis, until critics of the XVII-th century pointed out that
% he could not possibly have been (he probably wrote in the end of V-th century;
% already Abelard expressed doubts concerning his identity).  Since then he has
% been known as Pseudo-Dionysius and earlier vague doubts as to his orthodoxy
% turned into serious and explicit accusations.  Just a funny story\ldots}
%



%\subsub{Not of essence, but of actions...}

\pa As usual, we prefer to overlook differences, sometimes even vast ones, in
the conceptual framing and significance of the distinction and, instead, focus
on the fact of its recurrence.  According to Maimonides, we can assign
attributes to God's action, but not to His essence.  \woo{Every attribute that
  is found in the books of the deity [...is...] an attribute of His action and
  not an attribute of His essence}{Perplex} and \citet{these attributes too are
  not to be considered in reference to His essence, but in reference to the
  things that are created.}{Perplex}{I:53.\kilde{ACPQ 76(1)2002, p.10} Among the
  early Christians, the idea appears in \citeauthor*{Martyr}{ II:6}: \wo{But
    these words, Father, and God, and Creator, and Lord, and Master, are not
    names, but appellations derived from His good deeds and functions.}  A real
  (formal?) distinction between the unknowable essence of God and His actions,
  or \thi{energies}, making possible a real communion, belongs to the orthodoxy
  of the Eastern Church since the councils in Constantinople in 1341 and
  1351.\kilde{p.11}}
%The impossibility of saying anything about the Godhead, the God \thi{in himself}



Indeed, as all the \thi{essence} beyond the possible \co{distinctions} is
\co{indistinct}, the only name is \co{indistinct} (\co{nothing}, \co{one}).
\wo{Unknowable} is already analogical suggesting, on the one hand, that there
might be something to be \co{actually} known and, on the other hand, rising
partiality (incompleteness and imperfection) of our \co{reflection} to the level
of ontological impossibility of contact. All other names (or else, all names
suggesting more than \co{nothing}) can refer only to the character of the
\co{existential confrontation}: the names apply to God's actions only.  In so
far as God is \co{actually} known to us, that is, \co{recognisable} in
\co{actual experiences}, He \citet{\noo{Because God}is not known to us in His
  nature, but is made known to us from His operations or effects, [and] we name
  Him from these [...\noo{,as said in A1; hence this name "God" is a name of
    operation so far as relates to the source of its meaning. For this name is
    imposed from His universal providence over all things; since all who speak
    of God intend to name God as exercising providence over all; hence Dionysius
    says (Div. Nom. ii), "The Deity watches over all with perfect providence and
    goodness."}] But taken from this operation, this name ``God'' is imposed to
  signify the divine nature.}{SumTh}{I:q13.a8.ans.} This divine nature operates,
as we said in \ref{sub:yes}.\refp{pa:GodneedsMe}, only through human soul,
first, in the process of \co{ontological founding} and, eventually, through the
\co{existential confrontation} which says \Yes.  We would thus tend to identify
the actions of God with~\ldots the deepest \co{aspects} of
\co{experience}.\ftnt{This may not conform fully to the orthodoxy which, for
  instance as Maimonides, sees in God's attributes the guidelines: \wo{for the
    chief aim of man should be to make himself, as far as possible, similar to
    God}, and enumeration of God's attributes \citef{is the lesson that we
    should acquire similar attributes and act
    accordingly.}{Perplex}{I:54/III:53\label{ftnt:similar}} As is often the case, so also here,
  regulative ideas may be all that is left from a \co{concretely founding}
  \nexus\ when it gets translated into the language of \co{reflective}
  decisions, will and \co{actual actions}. Here, the results of such a
  translation are only \co{analogical reflections} of various \co{aspects} of
  \Yes, to which we will return in a moment.}  Considered \la{in abstracto},
these \co{aspects} amount to the deepest layers of \co{ontological} and
epistemic \co{foundation} as it was considered in Book I (and also Book II, in
particular, \ref{se:JungPlotinus}).  But \co{concretely}, it is the sphere of
the constant tension between \Yes\ and \No, of the constant \co{foundation} of
\co{actuality} on the one side of this \co{invisible distinction}\ldots


\noo{
\begin{enumerate}
    \item all essence beyond action is only indistinct...
    \item only spiritual choice
      \begin{enumerate}
        \item not of in-itself, but of the way of finding it
        \item 'names' can be correct or incorrect -- they should help one to say
          \Yes
      \end{enumerate}
      \end{enumerate}
\tsep{action = confrontation}
}

\subsection{Two faces of the one}
\subsecQ{God hath spoken once; twice have I heard this;}{ Ps. LXII:11}
\noo{God said one and I understood two}

\pa God, the \co{incarnated} Godhead, is only where a human being is. The site
of its dwelling is the \co{existential confrontation} where \co{nothingness}
begins to differentiate, where, as Eckhart says, \wo{Godhead [gives] all things
  up to God.}  God, the \co{concrete presence} of Godhead, is determined by the
\co{spiritual} dimension of man's being, by the \yes\ or \No\ of the \sch.  The
\co{one} as the ontological \co{origin} is but the \co{indistinct}
\co{nothingness}, remote, ineffable and indifferent.  The \sch\ affects this
\co{nothingness} in the most fundamental way -- \co{love} \co{experiences} it as
the \co{origin} and generous fullness, while the self-centered \co{attachment} as
a mere void, at best, as an indifferent principle of ultimate transcendence.
These are in fact \co{experienced aspects} of life, world, various situations
but \co{aspects} which cannot be ascribed to any particulars. In so far as they
are \co{founded} in the \sch\ itself, they emerge as \co{aspects} of \ldots
\co{nothingness}. But \co{nothingness} has no aspects! Here -- between the
\co{nothingness} of the \co{origin} and the \co{nothingness} of \co{self},
between the naked \co{self} of Godhead and its \la{imago}, the naked \co{self} of an
\co{existence}, in the tension of the \sch\ which is not \co{my} choice but
which gives the flavor to \co{my} whole \co{existence} -- here the Godhead gives
all things up to God.



\pa In I:\ref{pa:matter}-\ref{sub:Godmatter}, we have discussed the
traditional doubling of the \co{indistinct} as, on the one hand, the formless
matter under all beings and, on the other hand, the ineffable divinity above all
beings.
\noo{Neo-Platonism, and then negative theology, have always struggled with the
ambiguity of the twofold negativity: \co{nothingness}
which is \co{above} creature, which is \thi{more than~...}, which is
ontologically complete and perfect and appears as nothing only to our limited
understanding and, on the other hand, nothingness which truly is {\em not},
which is non-being, ontological emptiness. The former corresponds to the highest
good, spirit or God, while the latter is identified in various contexts with
matter, flesh, senses, vacuum, eventually evil.\ftnt{Cf. The equivocity goes back to
  Plato and is clearly visible in  the ontology of Proclos
  [E.Br\'{e}hier,...  [Bog-Nicosc, J.Miernowski, p.22[footnote~19]]]. 
  [\he{Is it an aspect of `negative theology'}: Plotinus, Pseudo-Dionysius, Eriugena and then the whole
  tradition of Christian neo-Platonism originating from Areopagite with Eckhart,
  Cusanus, Bovelles, and all others to whom God appeared only (or at least
  primarily) negatively, by not-appearing. The theological aspects of
  Heidegger's fit of course in this tradition well enough.]}
}
Ontological \co{foundation} of both is \co{one} and the same, the \co{indistinct
  nothingness}, which concerned us in Book I. But ontological \co{foundation}
lacks \co{concreteness} which is our current subject. The \co{indistinct
  nothingness} can be stated, to some extent described and~\ldots left for itself.
But leaving it for itself is only an illusion because its \co{presence} is
perpetually \co{reflected}, if in nothing else then in the \co{thirst}, perhaps
even in the search for a \co{foundation}, that is, for the \co{origin}. The
  \co{foundation} is \co{concretely 
  present} in life, but this \co{presence} may assume one of the opposite forms,
depending on the \sch: \wo{if thou seek him, he will be found of thee; but if
  thou forsake him, he will cast thee off for ever.}$^{\ref{ftnt:seek}}$
\citet{God responds differently to different human
  attitudes.}{TilUnbed}{I:4.Biblical religion and quest of being.4.1}
One might be tempted to understand here by the \wo{attitudes} some \co{actual}
  behaviors and acts. Although this is, to some extent, possible, it is
  primarily the \co{spiritual} attitude which makes God show different face. 
\co{One} and the same never appears, it is never  \co{object} of \co{any
  experience}, yet it is \co{experienced}. It is \co{experienced} in the
\co{rest} of all the \co{acts} we perform and all the \co{actual} situations we
encounter. The \sch\ gives this ontological omnipresence a \co{concrete} form of
the \co{analogues} which, when brought down to the level of \co{actuality}, may
  become \co{objects} of \co{actual experiences} appearing as direct \co{signs}
  of God. 

\No\ encounters emptiness -- apparently indifferent but eventually terrifying --
not as a demonic fear, perhaps as Kierkegaardian \ger{Angst}, but then also, as
underneath the emptiness one starts to suspect one's illusion, as the
specifically \la{numinotic}, awe- and often terror-inspiring \la{tremendum
  sacrum}, \la{ira deorum}. It is \la{tremendum} so often aroused by the God of
the Old Testament.
% , as when he says \citet{I will send my fear before thee, and
%   will destroy all the people to whom thou shalt come}{Ex}{XXIII:27}.
% tremendum (groza), niesamowite (ungeheuer, p.56/57), straszne ; angest..., orge
% (gniew), ira deorum (indian p.23)
\la{Tremendum} arises from \co{nothingness}, but by
its very appearance and strength it announces the ultimate power, \la{majestas}
hidden beneath it.\ftnt{One distinguishes, of course, clearly this \la{tremendum}
  with awe and trembling which it arises from any fear of things, people, demons
  or the world itself. Cf.~\citeauthor*{Heilige}, IV.a.}
\citet{As roaring torrents of waters rush forward into the ocean, so do these
heroes of our mortal world rush into thy flaming mouths. And so as moths swiftly
rushing enter a burning flame and die, so all these men rush to thy fire, rush
fast to their own destruction.}{Bhagavad}{XI:28-29}

There is, of course, a more subtle sense of the dread of God which does not
arise from the \co{actual} fear of damnation but from its mere possibility,
woven into the understanding \co{that I am not the master} and into the awe of
God's \co{invisible} power.  \citeti{Serve the Lord with fear, and rejoice with
trembling. [...]  the eye of the Lord is upon them that fear him, upon them that
hope in his mercy;}{Ps.}{II:11/XXXIII:18} Such a fear is but an \co{aspect} of
another form of \co{confrontation}, for which God's face emerges \noo{Thou art
the Imperishable, the highest End of knowledge, the support of this vast
universe. Thou, the everlasting ruler of the law of righteousness, the Spirit
who is and who was in the beginning. [...] And I see thy face} \citet{as a
sacred fire that gives light and life to the whole universe in the splendour of
a vast offering.}{Bhagavad}{XI:19. Krishna answers the reverent invocations of
Arjuna: \citefib{Thou hast seen the tremendous form of my greatness, but fear not,
and be not bewildered. Free from fear and with a glad heart see my friendly form
again.}{Bhagavad}{XI:49}} For \Yes, \co{nothingness} emerges with the ultimate
goodness, majestic and sacred \la{augustum}. The attracting force of the
{sacrum}, called by Otto \wo{\la{fascinans}}, may \co{found} more \co{actual
experiences} of exaltation or mystical joy (rather than of fulfillment and
completeness). But it is primarily the \co{non-actual} force of constant
\co{inspiration} which has delivered one from the threatening emptiness and
\co{alienation}.

So God may be revengeful and merciless or else merciful and generous. \citeti{I
  form the light, and create darkness: I make peace, and create evil: I the Lord
  do all these things.}{Is.}{XLV:7} But He remains the same God for that -- the
two are only faces of the \co{one}.  Thus, in spite of all our debts to
neo-Platonism, there is yet another crucial difference (not addressed earlier in
I:\refp{se:createEmanate}-\refp{se:neoplatonism}).  The two -- let us say, the
ultimate \thi{Good} and \thi{Evil} -- do not represent opposite ontological
poles; ontologically, they are the same \co{nothingness}. \thi{Evil} is not only 
the last, lowest level of emanations from the \thi{Good}, or perhaps even only a
lack of some aspect of a full emanation. In a much more Christian way,
\thi{Evil} has its site at the very beginning, it emerges from paradise with the
first human: %or, as we would say, in the very depth of \co{invisible}:
it is very {close to} Godhead.\ftnt{Of course, one can find neo-Platonic
  fragments pointing in this direction, like for instance: \citef{The evil that
    has overtaken them [the souls] has its source in self-will, in the entry
    into the sphere of process, and in the primal differentiation with the
    desire for self ownership.}{Plotinus}{V:1.1}} But this closeness to the
\co{origin} does not mean that \thi{Evil} follows from it. It only means that
human \co{existence} reaches all the way to the border of \co{nothingness} --
only therefore \thi{Evil} can corrupt it so deeply, even if never completely,
when, instead of \co{opening} up this ultimate border, it makes it impassable.
The two represent only the characteristics of the {\em meeting} with God, the
extreme and opposite possibilities of the \co{existential confrontation}: the
\No\ which sees only emptiness and void, and the \Yes\ which, too, sees
\co{nothing}, but \co{nothing} which is the \co{invisible
  origin}.\ftnt{Considering the Biblical personalism, Paul Tillich remarks
  \citef{There is no sense to ask if the holiness itself is personal or if its
    carriers are persons. [...] The question is: What do they become as the
    elements of a religious meeting?}{TilUnbed}{I:4.Biblical religion and quest
    of being.3.1} We extend this claim to all assumed \thi{attributes of God}.}
%(Bovelles: Bog jest niezbicie dowiedziony, gdy wyjdzie sie od nicosci [Miernowski,p.83])
% Two \co{existential} possibilities in \co{confrontation} with the \co{one}; two
% extreme and opposite possibilities -- yet, the line separating one from another
% is invisibly thin.
%
The negative emptiness is not the ontological opposite of the \co{one}, but only
an \thi{epistemic} mistake in one's \co{experience}. It is despair -- and, most
generally, evil -- which, \co{alienating existence} from its \co{origin}, makes
the latter appear as empty nothingness, total lack. This ultimate void of
emptiness is a substantialisation of \co{thirst}, the deep, if involuntary, \No\
which affects one's whole life, that is, the whole world.

\pa The need to speak about God may arise from the \co{reflective} attitude
which, inscribed within the \hoa, cannot escape the spell of \co{objective} way
of speaking which, inscribed within the circle of its actions and voluntary
choices, cannot avoid deciding for or against. 
Indeed, speaking about God may be helpful as an admonition that
\co{visibility} of \co{this world} does not exhaust the field of \co{existence};
as a suggestion of the quality of the \co{confrontation}; as a reminder that
what seems impossible may nevertheless happen, that the reality of \co{thirst}
overcomes the \co{actuality} of all facts.  God has \citeti{set before you life
  and death, blessing and cursing: therefore choose life, that both thou and thy
  seed may live.}{Deut.}{XXX:19} Speaking about God may help in such a
\co{choice} for, as Bacchylides quoted by Clement says, \citet{one becomes wise
  from another, both in past times and at present, for it is not very easy to
  find the portals of unutterable words. [...] We speak not
  as supplying His name; but [...] we use good names, in order that the mind may
  have these as points of support, so as not to err in other
  respects.}{Stromata}{V:11/12.\noo{(Observe how much more cautious this is from the
  suggestion of repeating God's attributes, footnote~\ref{ftnt:similar})}}


The simplistic contradiction -- \wo{God is} vs. \wo{There is no God} -- is
possible only after one has reduced the supposed \thi{being of God} to the
level of \co{actual experience}, to the level of \thi{being a thing}. It helps
little to claim that this is not the intention, for this is, in fact, the
result; and when the results are clear, the intentions matter very little. Assuming
that such an alternative is at all possible, one has already falsified the
meaning of the possible positive answer. 
%
A better question would be \wo{What does \wo{God} mean?}, or even \wo{What does
  God mean?}, though this, obviously, involves one in the matters on which even
the most prominent theologians (or, perhaps, especially they) can not agree.
Any \co{precise} answer can be accused of arbitrariness (but this should not
worry us for we view any specifics of this matter at most as \co{analogues}).
The advantage of this form of the question is that the previous alternative
(\wo{He is not} vs. \wo{He is}) becomes now \wo{He means nothing} vs. \wo{He
  means something}. The former will quickly declare Him to be non-existent but
it will also tend to involve some uneasy awareness of rejecting more than one
intended.
%
No matter the declared choice, one feels that the question has now much more
direct relevance, that it addresses not only the universal order of the
\co{objective} world but also one's \co{existence}. And it does it because it
also makes more clear the underlying element of choice which is not dictated
merely by \co{objective} \thi{being} or \thi{non-being} of something, but by the
way one meets whatever one meets.

At the very beginning, I:\refp{pa:imago}, we likened \co{existence} to an
\la{imago} of \co{nothingness}, not in the sense of a similitude but of a
reflection, like one player reflects the moves of the other. The asymmetry of
\co{confrontation} can be likened to the fact that one player has a winning
strategy, in fact, is bound to win. But as the game admits the win-win
situations, the other player can win, too. \Yes\ amounts to finding the winning
strategy and becoming not only an \la{imago} but also \la{similitudo Dei}.  The
\thi{names} of God are attempts to indicate the winning strategy.  Men call God
\wo{good} and not \wo{evil}, \wo{omnipotent} and not \wo{impotent}, and
\citet{in saying that God lives, they assuredly mean more than to say the He is
  the cause of our life, or that He differs from inanimate bodies.}{SumTh}{
  I:q13.a2} The \thi{names} attributed to God are expressions of \Yes, and as
such are not arbitrary predicates ascribed to \co{nothingness} which cannot be
ascribed any. They tell one how one could and should \co{experience} -- not God
but -- the \co{confrontation} with the divine.  For \citeti{the form of God is
  itself the joy with which it is recognised.}{Visvanatha}{\citaft{Huxley}{
    VII}\kilde{p.138}}
%(Though this may point too much towards conscious recognition...)
The \wo{should} does not express any moral obligation but an admonition. If one
does not \co{experience confrontation} in the \thi{prescribed} way, one is not
necessarily condemned but only reminded about its possible a desirable form,
about what is possibly wrong, even if any \thi{how and why it happened} remains
unclear. Good advice is not that which one can understand but that which can, if
properly followed, help one. 

\pa Just like \co{commands} leave us free to accept or ignore them, so God's
face reflects only the \sch\ made in the soul's depth. \Sch\ 
\co{founds concretely} the \co{quality} of the world and its \co{experience}
which become permeated by the aspects reflecting the underlying \yes\ or \No.
\Yes\ establishes a relation between the contents of all the levels,
\co{founding} thus also \co{concrete} ways of encounter and \co{experience}. The
\co{traces} -- down to the level of the most \co{actual} \co{reflections} -- of
the \sch, the \co{signs} of the \co{absolute}, are what we call the
\wo{\co{analogues}}.

The term \wo{\co{analogy}} is used in the way St.~Thomas would only partly
accept.
%In \ref{sub:projections} we have argued for the thorough reality of the
%\sch\ which was furthest removed from any projections.
\co{Love}, \co{humility}, etc. are not only genuine \co{aspects} of \co{spirit}
-- they are also adequately, absolutely and non-analogically predicated about
it.  This is possible because \co{spiritual} relation to \co{nothingness} is not
something absolutely foreign to human \co{experience} but, on the contrary, the
most intimately \co{present}, whether \co{concretely} or not, \co{aspect} of
this \co{experience}. The impossibility to specify \co{precisely} their meaning
does not, in any way, diminish \co{concreteness} of this meaning. On the other
hand, they not only belong merely \co{analogically} to the aspects of
\co{visible} experience, but are also predicated \co{analogically} about it.
For this experience, even if prior with respect to the \co{reflective}
knowledge, is actually \co{founded} in the \co{invisibles}. It is from there
that words like \wo{love}, \wo{humility}, \wo{goodness}, etc.  obtain their
genuine meaning, which is only \co{analogically} applied to the \co{visible}
\co{analogues} of the \co{spiritual love}. The fact that we cannot define
\co{precisely} what they mean, does not mean that we do not know that. Even if
\co{spiritual love} has not been our share, so \co{thirst} is the ever
\co{present} reminder of what it could mean.\noo{(Besides, as observed earlier,
  meaning of almost any term can not be defined \co{precisely} and finally,
  depending always on the level to which \co{distinctions} are drawn.)} Aquinas
would say: \citet{as regards what these names signify, they are applied
  primarily to God but as regards the imposition of the names, they are
  primarily applied to creatures which we know first -- hence they have a mode
  of signification which belongs to creatures.}{SumTh}{[after I:q13.a6.ans.]} This
looks almost the same, but in our case not even the mode of signification
belongs to creatures -- with respect to particulars we never know exactly what
\wo{love} or \wo{humility} mean, we are seldom entirely certain if the use of
such names is perfectly adequate, if the \co{actual sign}, the \co{act}, is true
with respect to the \co{invisible} truth which it should respect.  This
uncertainty, however, does not prove that we do not know what they mean, but at
most that we do not know it \co{precisely}. We are uneasy with applying them to
all too particular situations because, as a matter of fact, they do not quite
apply there. To some extent -- but only to {\em some} extent -- we can recognise
particular \co{acts} of love or humility, but they emerge as such only because
we recognise their \co{foundation} in \co{love} or \co{humility}: there is no
\thi{act of love} without love, and we do not learn what \wo{love} means by
collecting examples of any particular (kinds of) acts, by taking a course.
% some solipsism seems to be sneaking in here...???!?!??!?!!!!

Most probably, we learn it {\em before} we learn anything else. In any case, it
is not so that we know \wo{creatures first}, unless we let \wo{knowledge} mean
nothing more then the \co{reflective} \gre{episteme}, \refp{pa:beKnow}.  The
\co{analogues}, these \co{traces} of the deepest and most \co{concrete aspects}
of \co{experience}, are not any \co{actual signs}, are not any misnomers
misunderstood as properties of some \co{actual objects}. (Although, of course,
there is also such possibility.)  They reside primarily in the \co{rest} of any
\co{actuality} and are also known -- consciously, though only \co{gnostically}
-- to reside there.\noo{ As \co{reflection} always tends to posit an
  \co{object}, any \co{experienced non-actuality} tends for it to assume the
  form of a subsisting entity.  Within the \hoa, the \nexus\ (of \yes\ or \No)
  gets \co{dissociated} into two main aspects of the \co{external objectivity}
  and the internal, arbitrary \co{subjectivity}. The qualities of the
  \co{experience}, the \co{traces} of the \co{choice} acquire then, on the one
  hand, the character of the \co{objective} properties of some mysterious object
  or agent and, on the other hand but simultaneously and equiprimordially with
  the former, of a merely \co{subjective} experience, eventually, perhaps,
  ungrounded caprice. In the former case, the \co{analogues} acquire the
  character of \co{objectivistic} features, in the latter of \co{subjectivistic}
  projections.  Although the whole schema is wrong, it may also be legitimate to
  the extent that \co{reflection} stays aware of that and does not identify the
  \thi{objectified} \co{analogue} with its original source or, perhaps, does not
  ascribe its {objectivity} to any \co{object}.
%
The uncritical \co{analogues} arise from \co{positing} an \co{object} to which
they are attached, which easily leads to all kinds of antinomies.  In a genuine
sense, the \co{analogues} are but expressions of the most \co{concrete},
underlying \co{experience}. 
%
-- expressions of the radiating, all-embracing
\co{love} which is not \co{mine} but, on the contrary, which dawns on \co{me}
from \co{above} as a \co{transcendent gift}, a \co{gift} in which \co{I} only
\co{participate} and which affects the whole Being.
%
Various aspects of the \co{experience} following from \yes, respectively \No,
may be expressed in the \thi{objective}, or better, \thi{objectified} terms as
properties of something.  What this something is, does not matter much and,
indeed, must remain for \co{actual reflection} completely \co{vague} and
indeterminate.  Yet, the \yes, the \yes\ to the \co{invisible} Godhead, is also
a \yes\ to the world, while \No\ is the \No\ to Godhead which is grounded in
the wish to say yes to the world.
}
%
\co{Reflection} may, and in fact it does, take recourse to the \thi{analogical
  way of speaking}, in which the \co{aspects} of the \co{spiritual} dimension of
Being find \co{reflections} as sedimented and \co{dissociated} properties, as
the \co{analogues} predicated about some being, about something. What this
something is remains forever hidden in the \co{nothingness} of its conception,
and the best name man has got is \wo{God}.  But in fact these properties and
predicates are but expressions of the \co{experience} starting in the
\co{spiritual} center of Being, of the \co{aspects} which defy any
objectification and remain forever in the background, in the \co{rest}.  From
this \co{invisible} depth they affect all the \co{actual experiences} in the
most significant way.  If God is good, omnipotent, forgiving, loving, etc., one
may imagine that all the problems are solved or, as the case may be, posed. But
such \co{analogues} neither create nor answer any problems. They only express
some \co{aspects} of the \co{spiritual} dimension of \co{existence} which,
living \Yes, confronts all the same problems and particulars as any other
\co{existence} -- what is different is only the form, the \co{quality} of this
\co{confrontation}.

\subsub{Yes}
The \co{analogues} of each \co{choice} are only expressions of various
\co{aspects} of the respective \nexus. Without attempting any completeness, we
therefore list only a few standard examples.

\noo{ thankfulness -- was also \thi{wonder}, \thi{philosophical wonder}...
  humility -- fear of God \simu admiration\& respect \simu obedience forbidden
  \simu invisible: do not try to actualise }

\ad{Omnipotence}\label{adhumility} The \co{humility} of \co{love} means
\co{recognition} that \co{I am not the master}, \co{recognition} of the
\co{origin} as the ultimate power which is the power of the source.  Even if
nothing \co{actually} is the way it was at the beginning, when it emerged \la{in
  illo tempore} from the \co{virtuality} of the \co{origin}, so without this
source, nothing would be there.\noo{Without the \co{indistinct} ground, no
  \co{distinction} would be possible.} This simple indispensability -- not the
ability to determine every minute detail of \co{this world} -- is the
omnipotence of God.  Without Him there would be nothing, which is very
different from saying that everything is the way He has made it.
\citeti{Without Me, ye can do nothing.}{John}{XV:5} -- is very different from
\wo{You do everything the way I want}.
  
The indispensability of having Him as the first condition accounts also for the
misused label of \wo{necessary being}.  Once we have reduced necessity to
logical necessity, even to an appropriately interpreted unary sentential
operator $\Box$, there isn't much left\ldots But we have objected to the linguistic
reductions. 
%%%%%% ????? indispensable - is it better? 
God is certainly not necessary in the sense of being a particular agent whose
absence would entail a logical self-contradiction.\ftnt{One could, however,
  attempt the \thi{proof} of the form: if there were nothing to \co{distinguish}
  (nothing \co{indistinct}), there could be nothing \co{distinguished}.} He is
necessary in the above sense of being omnipotent, in the sense that without Him,
there would be (not only \co{nothing} but) absolutely nothing, total emptiness.
\citet{If \co{one} is not, then nothing is.}{Parmenides}{} \co{Reflection} can
be more comfortable with viewing Him as the necessary condition, not the
necessary cause, of everything that is.  The cosmological argument claims
well-foundedness of the order of causality, but we would recast it in terms of
the order of \co{founding}. It can be then viewed as an expression of the
intuition that this ordering is well-founded and has the \co{origin}, the
indispensable condition of all the rest.\ftnt{This was expressed, for instance,
  as the distinction between \thi{productive} and \thi{conserving} causes (Cf.
  I:\ref{sub:virt}.\refp{pa:foundingCausing}.)  Ockham pointed out that the
  order of \thi{productive} causes with Duns Scotus\noo{(cf. footnote
    \ref{ft:JDS}, p.\pageref{ft:JDS})} could actually be continued \la{ad
    infinitum} (in time), and to avoid this possibility he \thi{modified} the
  argument (for God's existence) by introducing \thi{conserving causes}. The
  modification can be disputed for Scotus, too, observed in his proof
  simultaneity of the \thi{essential causes}.  Such a cause need not
  \thi{produce} a thing but it maintains it in its being, and as such is
  co-present with the thing itself.  ({\em Actual} infinity of causes seemed
  absurd to all, and so the conclusion about the first cause followed.)  This
  form of causation, then, seems to correspond to our \co{founding}
  co-presence of all levels of hypostases, which nevertheless coincides
  with the generative order of \thi{productive} causes.\noo{But we do not aim at
    an interpretation of the Scholastics and won't bother you with the details.}
  (Ockham's \thi{proof} can be found in \citeauthor*{OckPhysicorum}{ q.82-86}.)
  \noo{p.122}} The \co{indistinct nothingness} is, indeed, the first cause, but
not in any sense of \co{actual} causality -- it is the first cause being the
first and necessary condition of everything. The necessity of the \co{one} is
here, almost analytically, the same as the indispensability of the
\co{foundation} in its \co{original indistinctness}.  If we take the label
\wo{cause} in this deeper sense of \co{foundation} then \co{one}, remaining
\co{indistinct above} and {\em before} all \co{distinctions}, is not only the
cause of everything but is also \thi{self-caused}: \co{indistinct} needs no
\co{distinctions}, I:\ref{pa:IndistinctNeedsNo}, \refp{pa:IndistinctNeedsNo}.
That God is the only uncaused, that is, self-caused being, used to be but
another way of formulating His necessity, though here we are approaching all too
much the level of \co{concepts} and \co{visible} determinations.  Eventually,
His necessity is the same as His omnipotence -- not the ability to do everything
but to \co{create}, to bring something, primarily \co{existence}, out of
\co{nothing}, without which event nothing else would appear. 
%and to be the only one who can do that...

\pa In a much less genuine sense (meaning, leading to, or even perhaps
displaying, a misunderstood {objectification}), one speaks \co{analogically}
about \thi{His will}.
%and \co{my} obedience or disobedience to it.
But the \wo{Godhead is poor, naked and empty as though it were not; it has not,
  wills not, wants not, works not, gets not.}$^{\ref{cit:getsNot}}$ Only
\co{existence} can will.  \thi{God's will} is but an expressions conveying what
\co{I} should will -- \co{nothing} or, to give it more positive appearance, to
say \yes. There is no more content in \thi{God's will} than the salvation
through \co{love}.  \citet{Sin is nothing else than that the creature willeth
  otherwise than God willeth, and contrary to Him.}{TheolGerm}{XXXVI} Indeed,
but this \thi{will of God} which, in a deep confusion one can attempt to look
for in \co{visible signs}, is nothing else than that \co{I} do not sin, that
\co{I} do not make \co{idols} of \co{this world} but, instead, find the
\co{absolute love}.  For \citetib{as long as a man is seeking his own good, he
  does not seek what is best for him, and will never find it.}{TheolGerm}{XXXIV}
\citet{We strive always for the forbidden, and we desire things
  denied.}{OvidAmor}{III:4.17} \co{My} obedience -- if one insists, obedience to
God's will -- is thus nothing more (nor less) than directedness towards \Yes. It
is \co{humility} which does not try to reach for the \co{invisible} -- forbidden
-- fruits. It may be rightly called the \wo{fear of God} and, although there
need not be any \co{actual} fear in such an attitude, it is again an expression
of God's omnipotence.  \noo{Trying to \co{actualise} the \co{invisibles} is, as
  far as we have presented it, attempting an impossibility. Trying to quench
  \co{thirst} one only creates \co{idols}.  In refraining from that one may see
  respect for what is higher as much as a fear of it. In either case, the
  omnipotent being dwells beyond the limits of \co{visibility}, in a place (if
  place it were) inaccessible. }

\noo{
\subpa Similarly, one speaks about \thi{God's works}.  But He
\wo{works not}. Even \citet{[g]race does not perform any works; it is
too subtle for that and is as far from performing any works as heaven
is from earth.}{EckGermSerm}{DW 38, W 29;p.118.} The \co{invisible incarnation} precedes all
\co{distinctions} of \co{actual} kind which would allow any works. 
And all such works happen only through the being which makes the
respective \co{distinctions}.  \co{I} do the works, {you} do the 
works, \co{he} does the works. Like \thi{God's will} so also \thi{His
works} are expressions, not very fortunate ones, of the
\co{experienced} \co{transcendence} which \co{inspires} \co{me} to do
what \co{I} do, and which hardly ever determines the particular contents of
these works.
}


\ad{Omnipresence}\label{adopenness} The description of omnipotence might
have left some taste of deism in which Godhead, giving only the initial impulse
to the creative differentiation, withdraws from the world leaving it to the
secondary causes, if not entirely to itself. In a sense, this is what we are
saying, for God is never any \co{actual} efficient cause of anything, He never
interferes directly into the course of \co{actual} events: He remains
\co{transcendent}, stays \co{above} all \co{visible} and \co{invisible} things.
But at the same time, He is always \co{present}, as the deepest \co{aspect} of
every situation, as the eventual limit where \co{actuality} dissolves in the
\co{invisible} horizon of its \co{foundation}. And as this horizon surrounds
every situation, it not only provides all its \co{actual} and \co{visible}
elements but can also bring the completely new and unexpected ones.  For the
most, it is like the presence of a quiet person who apparently does not
contribute to the situation, does not influence the events but, by the very
presence, makes the situation into something completely different than what it
would be without this person. And as it later turns out, this was the person who
invited all the others, opened his house and paid all the drinks. We have just
recalled the distinction between the creative, efficient causes
%effect about 
and the conserving ones that do not produce but sustain and perpetuate the
effects.\noo{I:\ref{pa:foundingCausing}} Again, replacing the former with
the latter, \wo{cause} with \co{foundation}, we recognise such a constant
sustaining \co{presence} of the \co{origin} as the \co{invisible} \hoa\ and,
eventually, of all \co{visibility} and differentiation.

\pa \co{Openness} is openness of the heart for all gifts of the
\co{origin}. In more \co{actual} terms, it can denote preparedness to meet
with the open heart everything and everybody but, for the moment, we are relating
it only to the understanding of the (role of the) \co{one}. \co{Openness} means that we
\co{recognise} it as omnipresent, that everything is encountered with the
fundamental, only implicit rather than explicit understanding of its being a
\co{gift} of the 
\co{origin}, of its being a hierophany and, hence, of the \co{origin} being
\co{present} behind it. 

In the most \co{actualised} form, \co{openness}, this recognition of
omnipresence, can find expression as the constant wonder and joy, as a calm
intoxication with the world which at every moment unveils new and fresh events
which, by their very freshness and newness, acquire an aspect of 
miracles. It is not the intensity of such a joy which makes it the
\co{analogue}, but its constancy -- 
as Eckhart puts it \citeti{Who is joyful all the time, he is joyful above the
  time, liberated from time.}{Eckhart}{\citaft{Mistyka}{ A:V.8.a\kilde{p.83}}}
% \citt{Kto raduje sie caly czas, ten raduje sie ponad czasem, wyzwolony z
%   czasu.}{Eckhart, after Otto, Mistyka Wschodu i Zachodu, p.83}
The same \co{analogue} may be discerned in the \thi{wonder} owing to which,
according to Aristotle, \citet{men both now begin and at first began to
  philosophize.}{AristMeta}{II\label{cit:wonder}} Psychological and emotional
differences notwithstanding, this famous \thi{philosophical wonder} expresses
the same \co{openness} to the world which is \co{experienced} as a \co{gift}, if
not miraculous, so in any case generous and wonderful, even in its most wicked
appearances. Just like man who is \co{thankfully open} thinks himself
undeservedly rewarded, so \woo{a man who is puzzled and wonders thinks himself
  ignorant}{AristMeta,2}$^{\ref{cit:wonder}}$. Being ignorant of everything and
wondering at everything, being like a child, is to recognise the generous
\co{presence} in every \co{actuality}, to \co{experience} omnipresence.

\pa\label{selfomniscient} \co{Self-awareness}, I:\ref{sub:selfAware}, is an
aspect of every \co{actual} encounter, which makes \co{me} always, even if only
implicitly, aware not only of the thing, the \co{object}, the situation \co{I}
am confronting, but also of the fact of this confrontation, of its anchoring in
the field of \co{experience} \co{transcending} the limits of \co{actuality}.
Although formally we can say that it is \co{my} \co{self-awareness}, yet it does
not \thi{belong} to \co{me}, it is not something \co{I} determine and control --
it accompanies \co{me} as \co{my} associate, as a witness, not as \co{my}
attribute; it is an \co{aspect} of every \co{experience}, never its \co{object}.
In \co{my} focusing on the \co{actual} contents of \co{experience}, it witnesses
to the \co{presence} of something that \co{transcends} it. Feeble and dependent
on \co{my} recognition, on \co{my} acceptance of its voice, in the context of
\co{love} it \co{founds} the \co{analogue} we might call \wo{God's
  omniscience}.
%
This is the omniscience of which also those not recognising it are warned:
\citeti{Can any hide himself in secret places that I shall not see him? saith
  the Lord. Do not I fill heaven and earth?}{Jer.}{XXIII:24} \citeti{There is no
  darkness, nor shadow of death, where the workers of iniquity may hide
  themselves.}{Job}{XXXIV:22} It may take the form of a voice of conscience
which discloses the \co{spiritual} context of \co{my action}, it may be a mere
\co{awareness} of the \co{presence} which \co{transcends} \co{actuality} and
also, in the most figurative sense, modified by the \co{reflective} attention
directed towards it, it may appear as the feeling of \thi{somebody looking at
  me}. The lower such a form, the more common it is. But the \co{presence} of
its higher forms, and particularly of conscience, is indeed dependent on the
\co{concrete foundation} in \yes. The better a person, the more conscience he
has and, in fact, the more guilt he can feel -- criminals seldom {\em feel}
unclean conscience. For conscience is yet another \co{aspect} of \co{openness},
which both \co{opens} us for others and \co{opens} the \co{communion} with
others to us.

\ad{Goodness}\label{adthankfulness} \co{Thankfulness} amounts to
\co{recognition} of the \co{origin} as good or, what amounts to the same,
as the source of goodness.  \co{Thankfulness} is but a \co{reflection} of
the acceptance of the \co{origin}, of the \co{recognition} of its goodness.
This goodness, if taken in itself, is empty and impossible to characterise.
It does not mean {\em anything else} except the attitude of \co{thankfulness}
and acceptance, nothing except the fact that \co{I} \co{recognise} the
value of everything which \co{I} encounter and am willing to accept it with the
underlying \co{love}.  It is not the attitude of \yes\ which is good, it is
not some God who is good. Goodness is the \co{experience} of
\co{thankfulness} rendered in terms of \co{actuality}, is an \co{analogue} of
the latter.  Nobody who does not know this \co{thankfulness} can ever
experience, let alone understand, the goodness of God.
% \ftnt{If you
%   feel like that, you are welcome to draw from that the consequences for the
%   apparent and celebrated \thi{problem} of how possibly a good and omnipotent
%   God might have created all the evil in this world. We will return to this
%   issue.}

Again, in a less genuine sense, one speaks about \thi{God's love}, \thi{God's
  benevolence}, etc.  Misleading as such expressions may be, they stand for the
purity of \co{thankfulness} which is its own reward.  It \citet{is not chosen in
  order to serve any end, or to get anything by it, but for the love of its
  nobleness, and because God loveth and esteemeth it so
  greatly.}{TheolGerm}{XXXVIII}\noo{justice is rectitude of will preserved for
  its own sake{Anselm, On Truth,p.169}} There is no being, no non-being, no
\cross{being} -- if one insists, no God -- sitting there and loving or esteeming
anything.  This love and esteem are first of all the value and {nobleness} such
a \co{love} and \co{thankfulness} have in themselves, \co{opening} one to the
\co{transcendent gift} which gives the ultimate value to one's life. Certainly,
the ineradicable possibility of the \co{gift} of \co{grace}, the fact that it
may be given irrespectively of the earlier circumstances (and sins),
irrespectively of how deep one has plunged into the hell and despair, so that
even \citeti{they that dwell in the land of the shadow of death, upon them hath
  the light shined}{Isa.}{IX:2} -- this can be taken as an expression of {God's
  love} and forgiveness in a quite anthropopathic, almost mundane sense.  This
should be a legitimate way of speaking -- if only one remembers that it is
\co{analogical}.  \noo{For the ineradicability of the possibility of receiving
  \co{grace}, true as it may be, requires already a choice on \co{my} part, an
  \co{actual} choice which tries to say \Yes.}

\pa A yet more specific \co{analogue} of \co{thankfulness} is the goodness of
the world, with perhaps the most powerful expression in the idea of the best
possible world.\ftnt{Although we would tend to see its \co{existential} origin
  in the mere fact {\em that} one encounters suffering and is able {\em not} to
  turn it into evil, so in the context of theodicy one has to look also for
  explanation and justification of this fact.  E.g., \citef{Now, the order of
    the universe requires, as was said above (q22.a2.ad2; q48.a2), that there
    should be some things that can, and do sometimes, fail.  And thus God, by
    causing in things the good of the order of the universe, consequently and as
    it were by accident, causes the corruptions of things
    [...]}{SumTh}{I:q49.a2} Along slightly different, but still sufficiently
  similar lines, the world is described by Al-Ghazali as that which is
  \citef{according to the necessarily right\lin order, in accord with what must
    be and as it must\lin be and in the measure in which it must be; and\lin
    there is not possibly anything whatever more\lin excellent, more perfect,
    and more complete than it.}{Al-Ghazali}{41-45 [The following lines contain
    the standard argument why it must be so -- otherwise it would contradict
    God's generosity or justice.]}  Origen states it, in a way similar to
  Maximus of Tyre\kilde{Powrot,p.149!}: \citef{Evils, then, if those be meant
    which are properly so called, were not created by God; but some, although
    few in comparison with the order of the whole world, have resulted from His
    principal works, as there follow from the chief works of the carpenter such
    things as spiral shavings and sawdust, or as architects might appear to be
    the cause of the rubbish which lies around their buildings in the form of
    the filth which drops from the stones and the plaster.}{aCelsus}{VI:55} The
  reservation made by Timaeus in the myth of creation shows the same idea:
  \citef{God desired that all things should be good and nothing bad, {\em so far
      as this was attainable}.}{Timaeus}{}
  
The idea involves two elements: 1.~the \thi{totality of the world} -- it is {\em
  its} goodness and perfection which is maximal, even if it does not appear so
to individuals within it (this totality may be, especially in Christian
contexts, replaced by the totality of an individual being -- the encountered
evils serve then the salvation of one's soul even if one is unable to discern
this meaning.); and 2.~some inviolable laws which even God must obey -- these
are responsible for the individual \thi{evils} but they can be only
unproductively opposed if they are not resolutely accepted.  One might be
tempted to discern the idea {\em whenever} these two elements are present.
Thus one might impute it to Spinoza as much as one observes it in Leibniz. In
this form the theme can be found also in the pantheism of Stoics according to
which the world must be perfect since it is only the formed and visible aspect
of the immanent and active principle, god-\gre{logos}-\gre{pneuma}, which is
perfect by definition.\kilde{Powrot,p.73ff.} Neither of these points appeals
to our anthropology but we can recognise its underlying
value, the \co{motivating} force of acceptance. It is, however, a bit like
saying, in a resigned and moralising tone, \wo{Do not argue, things won't get
  better any way}.  It hardly comes close to genuine \co{openness} which
recognises, equally in the moments of happiness and of suffering, not only
  something one has to put up with but goodness deserving \co{thankfulness}.}
%  
It is impossible to agree concerning what is the exact acceptable \thi{amount of
  evil in the world} and whether it is greater or smaller than the total
\thi{amount of goodness}. Likewise it is impossible to specify exactly the laws
which make it impossible to improve this world. But the impossibility lies not
in our inability to see or think, but only in the fact that, except for the
\co{visible} details which do not affect that much, there is nothing to improve
for nothing else is under our full control. One may become very indignant at
such a dictum, just like one may deny any necessary laws which cannot be
transgressed in the constructions of new brave worlds and then in
elimination of some evils. But the more indignant one becomes, the stronger the
indication of one's inability to cope with~\ldots the world, that is, with
oneself. For indignation, and its associate moralism, arise exactly when one is
no longer able to see only various, lesser or bigger evils, but begins to
recognise The Evil, the unjust, unfair, inequitable world which offends one's
human dignity, in short, when \co{visible} evils acquire monstrous dimensions of
social or even metaphysical principles.\ftnt{One might be tempted to include
  here most kinds of revolutionaries, many (often well meaning) social and other
  reformers, as well as petty personalities like Gavrilo Princip,
    %Nedjelko Cabrinovic, Trifko Grabez ... Black Hand and the like,
  who cannot or do not want to distinguish the personal problems and diseases
  from the socio-political ones.}  The evils in the world may be innumerable,
but the conclusion that the world is evil is clear \co{sign} of
\co{alienation}.  The idea of the best possible world is the most charming
\co{conceptual reflection} of the understanding that neither God nor even the
world owes us anything -- least of all any reasons and explanations.
    
The world is good in the trivial sense that salvation is always possible. In
this respect, the world needs no improvements. All the detailed improvements of
the world can be needed for making the society more comfortable or more just,
but that has nothing to do with God, for He is getting involved when, and only
when, personal salvation is at stake. The performances of televangelists praying
for most specific items are certainly close to the peek of vulgarity. But we
should observe that a god who figures as a mere postulate of the sheer faith
that all my good deeds and my good life will be eventually rewarded with equally
good items according to a principle of justice, not to say, of just payment --
such a god is, too, reduced to an honest clerk matching the list of my deeds
against the list of \co{visible} wishes and goods.  \noo{Gl\"{u}ckseligkeit ist
  der Zustand eines vern\"{u}nftigen Wesens in der Welt, dem es, im Ganzen
  seiner Existenz, {\em alles nach Wunsch und Willen geht}, und beruhet also auf
  der \"{U}bereinstimmung der Natur zu seinem ganzen Zwecke, imgleichen zum
  wesentlichen Bestimmungsgrunde seines Willens.}  \citet{Happiness is the
  condition of a rational being in the world with whom everything goes {\em
    according to his wish and will}; it rests, therefore, on the harmony of
  physical nature with his whole end and likewise with the essential determining
  principle of his will.}{KantPrakt}{II:2.2.v [K224]} The pietistic opposition
of duty and nature, with the associated opposition of morality and happiness
(which is merely \citetib{rational being's consciousness of the pleasantness of
  life uninterruptedly accompanying his whole existence}{KantPrakt}{I:1.1.\para
  3}) calls indeed for somebody who might guarantee at least some, and at best
ultimate harmony, who might dose \wo{happiness proportioned to [\ldots]
  morality}.\ftnt{As nature has been completely \co{dissociated} from the
  spirit, and desires from morality, such a being \noo{soll den Grund der
    \"{U}bereinstimmung der Natur nicht blo{\ss} mit einem Gesetze des Willens
    der vern\"{u}nftigen Wesen, sondern mit der Vorstellung dises Gesetzes, so
    fern diese es sich zum obsersten Bestimmungsgrunde des Willens setzen, also
    nicht blo{\ss} mit den Sitten der Form nach, sondern auch ihrer
    Sittlichkeit, als dem Bewegungsgrunde derselben, d.i.  mit ihrer moralischen
    Gesinnung enthalten.}  \citefib{must contain the principle of the harmony of
    nature, not merely with a law of the will of rational beings, but with the
    conception of this law, in so far as they make it the supreme determining
    principle of the will, and consequently not merely with the form of morals,
    but with their morality as their motive, that is, with their moral
    character.}{KantPrakt}{[K225]} One might probably look here for an attempt
  to go beyond pure ethical formalism, although the attempt is left to God. The
  hypothesis of God is needed, as usual, to make the \co{dissociated} parts fit
  again; here, to make us believe that moral life can possibly pay off. This is
  the cornerstone of pietistic dualism, and its abhorrence of the senses from
  which it is unable to liberate itself, so that \citef{the moment the fool
    gives up concentration\lin And his other spiritual practices,\lin He falls
    prey to fancies and desires.}{Ash}{XVIII:75} The reward one expects must be
  given not only in principle -- and not only
%  where it is its own and highest reward --
  in the \co{spiritual}, that is,
  \co{absolute} dimension of life which affects its whole -- but also in all
  particular wishes and projects which, for such a repressive consciousness, remain
  for ever the all and only reality. \noo{ This pietistic \co{dissociation}, closely
    related to its gnostic predecessors, is but a sick result of the suppression
    of its dependency and} As if \citef{virtue itself and the service of God
    were not happiness itself and the highest
    liberty.}{SpinozaEthics}{II:49\kilde{p.125}}}
%
Unfortunately, such a guarantor of justice is not much better than one who can
offer to a rational being only an irrational hope of a new car or a plastic
surgery. Goodness of God is offered as the promise of salvation, as the
possibility of saying \Yes\ -- to God, and hence also to life in whatever form
it meets one. It has nothing to do with {\em any} rewards, not to mention
granting one a happy life \wo{according to his wish and will}.
% good life among the goods which one craves because most believe them to be
% desirable or for whatever reason one imagines oneself to need them.
God owes us nothing -- not only no candies or feelings of pleasantness, but not
even any just rewards. Who are we to know what we deserve and what
would be a just reward proportionate to our moral deeds? \citeti{We know not
  what we do.}{Lk.}{XXIII:34}

\ad{Person} Like characterisation of any other \nexus, this one could continue
indefinitely. Let us only mention one more \co{aspect} which is constitutive for
the \co{experience} of God as \co{transcending} all the more particular
\co{experiences} of \co{analogical} kind. The \co{analogues} like the ones we
have described so far can enter \co{any experience} and, in fact, find the most
\co{actual} expression (as the \co{actually} {\em felt} wonder, thankfulness,
presence, look from above). In no such \co{actual experiences}, however, one
meets a person.

What makes one a person is the capacity to enter personal relations (as \nexus\ 
precedes its \co{dissociated aspects}, so here the adjective precedes the
substantive, cf. quotation~\ref{ftnt:personCommunion}.) Personal relation is one
which concerns the very center of Being, which is a true \co{communion},
\co{sharing} of the \co{origin}.  This \co{communion}, this most personal
relation is possible in its deepest form only between persons. Only one person
can reach the personal-center of another, only one being \co{open} to the
\co{origin} can meet another being in the same \co{openness}; only one person
can tell the name of another, for telling the name is exactly the sign of
recognising the unique value of the person, which is but another side of
recognising the \co{shared origin}.  A personal meeting is a meeting in the face
of the \co{origin}, is a meeting where nothing is left outside, that is
\co{closed}, for what is being \co{shared} is the \co{absolute} beginning,
\co{nothing}, that is, everything.
%Person is the uniqueness of human \co{existence}.

There is a habit (going back to Locke and Hobbes, if not all the way to
Aristotle) of insisting on more definite aspects which would constitute a person
-- like responsibility, self-consciousness, rationality, freedom. But the only
reason for it seems to be the forensic need for a more definite \co{concept},
allowing us to distinguish persons from non-persons (sic!).  Saying \wo{He is
  not a (mature) person for he lacks the basic sense of responsibility} or
\wo{He is not a (legally responsible) subject for he lacks elementary
  rationality} may sound quite reasonable, unlike, for instance, \wo{He is not a
  person for he is not rational}. The parenthetical adjectives press themselves
into the formulations, for being a person comes before and stays \co{above}
being anything else, no attribute nor its lack can ever account for a human not
being a person.  We have described in Section \ref{sub:concrete} the meaning of
\thi{becoming what one is}, %how the personal-center can acquire
of becoming through the \co{concrete
  foundation}, so to speak, a full person, but this is a story about
the {\em kind of person} one might become and not of being or not being a person.

Ontological \co{foundation} of human \co{existence} is equiprimordial with its
personal character.  This \co{foundation} accounts also for the uniqueness of
every \co{existence} but, with respect to the aspect of personality, the
decisive fact is \co{participation} in the \co{origin} which \co{shares} itself.
\co{Confrontation} is the primordial event of such a \co{sharing} in which the
\co{origin shares} itself with the \co{existence}, \refpp{pa:originShares}. It is
thus God who \citet{makes of us a complete person and, consequently, in a
  meeting with us is fully personal.}{TilUnbed}{I:4.Biblical religion and quest
  of being.3.2} This making a {\em complete} person amounts to augmenting the
merely ontological \co{participation} with the dialogue of \co{concrete
  foundation}.  But primarily God makes us also a person, because \co{existence}
is constituted by \co{participation} in the \co{origin}, emerges only in the
face of Him, in a \co{confrontation} with Him.\ftnt{This seems to be the source
  of the well known close
  connections between personalism and some form of (typically Christian or
  Judaic) theism.}  Strictly speaking,
\co{nothingness} of Godhead has the non-personal, or trans-personal character of
\co{self}. But this trans-personal character is at the bottom of the very being
a person.  It is itself non-personal, void of any \thi{essence}, the mere purity
of the \co{distinction} of \co{birth}, and yet it is the personal center, it
\co{founds} the fact of being a person, the \co{unity} of \co{existence}
stretching all the way to the most \co{actual reflections}. The center of
personal being is not itself personal, and it is only by \co{confronting}
something \co{transcending} one's personality, something trans-personal, that
one is a person.\noo{Just like every thing is only totality of what it is not.}

This \co{confrontation} is the context but also the eventual content of personal
relation.  In it God Himself emerges as a person; firstly, by the very
definition, by being involved into the personal relation, by being the
\co{absolute} pole of the \co{spiritual} tension. \co{Spirit} is a fully
\co{concrete} person or it is not at all. More \co{concretely}, God is a person
because He says your name, because \citeti{your names are written in
  heaven.}{Lk.}{X:20} God uses quite some part of the book of Genesis for
telling people what their names shall be: no more Abram, but Abraham, no longer
Sarai but Sarah, and Isaac, and no longer Jacob but Israel, etc.
\noo{\citeti{neither shall thy name any more be called Abram, but thy name shall
    be Abraham; [...] As for Sarai thy wife, thou shalt not call her name Sarai,
    but Sarah shall her name be. [...She] shall bear thee a son indeed; and thou
    shalt call his name Isaac:}{Gen.}{XVII:5-15-19}::USED in BOOK I}
%
All these names, given by God, represent the personal character of the relation:
in the act of naming, God establishes the person as a person.  From the very
beginning, He is not a mere technician constructing only the mechanism of the
world -- He addresses a person, long before any consciousness can \co{actually}
grasp the fact.  \citeti{I have surnamed thee, though thou hast not known
  me.}{Is.}{XLIV:4} And He keeps addressing persons responding to the personal
calls and prayers or, as we might also say, to the deepest need of a person: the
need of reality, the need of help, the need of \co{grace}. He is a person
because our only relation to Him can be personal, consummated in the depth where
the center of our being meets the center of Being. Personality of God is the
ultimate expression of His highest relevance for life, is the \co{analogue} of
His \co{presence} in the ultimate \co{communion}, \co{sharing} everything with everyone.

\noo{
\tsep{Arguments and proofs...}

\pa\label{proofs}
We could try to find an \co{analogical} element (only an element) of
\co{openness} in the teleological proof, in its directedness towards the future
good, but it might be asking too much from an all too shallow analogy.
\noo{  
  it is wishing in order to obtain, it is expression of
  \co{thirst}... -- it is goal of everything, what all \co{thirsts} for (as
  Scheler observed, the hight of good in the hierarchy of values is
  indisputable, even if what \co{actually} is good remains in most situation
  imprecise) ; well,
  as salvation, unity of 
  heaven and earth, and then, eventually, as \co{indistinctness}, death
  (Mainlander?)... 
}
Nevertheless, we have indicated that proofs of God's existence, correct or
incorrect as they may be, have much more in them than the merely logical grid of
concepts and constructions.

Proofs of God, God's existence, reflect also insight into His {\em meaning} for
\co{existence}.

So sure, none of that is any proof. And they are not so clearly separated from
each other, as simple logical analysis would do it with their formal
expressions.

But they represent some of the most fundamental
observations concerning human \co{existence}. 

As with the proof of Anselm's, we should be able to attach some meaning to other
proofs also, provided we adjust the interpretation of terms...
}


\subsub{No} It is the \co{spiritual} \Yes\ which calls forth, from the abyss of
Godhead, the generous and benevolent person of God.  \citet{How beautiful must
  this appear to him who understands it; how absurd to the
  ignorant!}{Perplex}{II:6} The \No, on the other hand, encounters only its own
negativity which means encountering, in the whirlpool of distractions and
\co{actualities}, emptiness.  \co{Attachment}, refusing to accept that \co{I am
  not the master}, does not lend its characteristics to possible descriptions of
God. The primary \co{analogue} of this attitude amounts to the simple conflation
of \co{nothingness} and void -- beyond the \co{visible world}, there is
emptiness. (And the antinomic character of this emptiness, permeating like ether
the whole world, does not change the fact that self-centered \co{attachment} can
not get rid of its ghost.) Yet, the {objectified} characteristics of the
respective aspects -- of \co{pride}, \co{ingratitude}, \co{closedness} --
carry a lot of strength, even if one is unable to say to what they possibly
could be ascribed -- to the world? to the life? to my life? to the proclaimed
void surrounding all that? A bit to this and a bit to that, eventually, to
everything for \No, even when it remains most consistent, agreeable and conform,
lacks the \co{unity} and finds its reflections in all scattered bits and pieces
-- \citeti{smite the shepherd, and the sheep shall be scattered.}{Zech.}{XIII:7
  [Mrk. XIV:27]}


\pa\label{adpride} \co{Pride} is not necessarily a personal pride, an individual
attitude of superiority over others.  \co{Pride} is merely an attitude which
does not \co{recognise} any higher power, any \co{origin} beyond
\co{visibility}.  The \co{analogues}, its {objectified} expressions, embrace so
many \co{idols} that one can hardly attempt any enumeration.  The
\co{objectivistic illusion} from I:\ref{objectivisticillusion}, the assumption
that everything consists of {things}, eventually, of \co{objects}, is an
important example.  It underlies all kinds of intellectual arrogance, naive or
sophisticated scientism, exclusive worship of causality and \thi{hard facts}.
As has been frequently observed, humanism is another field providing a host of
examples.  \citet{But how can anyone judge or love what he does not
  know?}{Pico}{} -- this is, perhaps, the most concise summary of the first
humanist manifesto. The real question, however, is what one understands by
\wo{knowing} and what is there worth such \wo{knowing}.  In a sense, this text
elucidates only some fundamental issues of our life -- it is humanistic.  But
the adjective rings wrongly because the human nature is not so plainly and
\co{visibly} human as many \thi{humanists} would like to see it.\ftnt{Although
  the tone of \btit{The Oration} is certainly of the kind which could motivate
  the later development of humanistic \co{pride}, one should remember that it
  does not call for the simple reduction of the world to the merely
  \co{visible}, human, all too human, dimension.  \citef{Let us disdain things
    of earth, hold as little worth even the astral orders and, putting behind us
    all the things of this world, hasten to that court beyond the world, closest
    to the most exalted Godhead.}{Pico}{\kilde{quoted earlier}} Yet, although it
  emphasizes that such a transcending is possible, it does contain the germs of
  the latter anthropocentrism in which much of the humanism would like to
  enclose humanity. The later in history we move, the more \co{pride} --
  parading often as humility not only recognising but also, if only unwillingly,
  accepting regrettable human weakness -- can be discerned in whatever
  calls itself \wo{humanism}. The line separating the attempts to {know} the
  full scope of human existence in order to live it most completely from the
  attempts to \thi{know} it in order to control it, is often hard to draw.
  \citefi{I am Satan, and deem nothing human alien to
    me}{footnote}{\ref{ftnt:satanSum}} is a compact expression of the
  multitude of attempts trying to liberate humanity from the \thi{inhuman}
  subordination to uncontrollable forces, attempts based on the claims of
  unlimited, because of lack of any \co{visible} limits, possibilities:
  \wo{Human species [...] will once more enter onto the road of radical
    transformation and will become an object, in its own hands, of the most
    complicated methods of artificial selection and psycho-physical training.}
  The authorship of this quotation speaks for itself -- \citeauthor*{Trotsky},
  {p.229 \citaft{Walicki}{ p.368}} In the process of shifting \thi{human} to
  \thi{all too human}, the category of \thi{humanism} has been appropriated by
  forces and movements which even in the personage like Mother Teresa seem
  unable to see anything beyond humanitarianism.\noo{Therefore, the above
    reservations against the use of the word.}}

%\pa\label{adclosedness}
The {objectified} \co{analogue} of \co{closedness} may be expressed by the
simple statement \wo{The world is as it is}.  The facts are there, true,
\thi{objective}, irresistible, and the only thing we can do -- at least, do
rationally -- is to conform to them, possibly, to manipulate them so as to
achieve our goals.  The apparent activity of such an attitude of smartness
towards the givens is underlied by the fundamental, \co{spiritual} passivity,
the all pervading resigned acceptance of givens as givens -- the world, after
all, is as it is.  It becomes stiff and rigid, not necessarily because it is how
science is forced to see it, but because it has been raised to the status of the
highest and only reality governed by the irresistible laws of \thi{hard facts}.
Impossible as such laws may be to specify \co{precisely}, they spook behind all
our failures and defeats which, obviously, must have some \thi{objective}
reasons.

%\pa\label{adingratitude}
The {objectified} \co{analogue} of \co{ingratitude} is the image of life and
world as, if not basically and essentially so to a large extent, bad,
mischievous, perhaps, even evil.  In the world we meet many things and
situations and most of them require an attitude of suspicion and scrutiny.  Such
a project can hardly fail; the field for Voltairean grimaces at Candide's
disasters and the naivit\'{e} of Panglos' optimism is inexhaustible -- the
grimaces are so obviously convincing, that they will always appeal to the
adolescent \thi{rationality} of the Enlightened flavor.  One will always find
many serious examples which can be used as strong reasons justifying ungrateful
attitude, indeed, ridiculing any idea of \co{gratefulness}.  And, in fact, in
many situations one better stay alert. But there is a great difference between
seeing a danger in the particulars of a situation, and seeing danger everywhere,
between being wary of a person who creates an impression of dishonesty and being
wary of all people, perhaps, even of all people in general.  The suspicious
alertness is the fundamental modus of ungratefulness, reflecting and originating
in the general idea of the world rendered in terms of harm and reward; the world
which, unless one prevents it, will do one some harm. \citeti{Most are at odds
  with that with which they most constantly associate -- the account which
  governs the universe -- and \ldots what they meet with every day seems foreign
  to them.}{Heraclitus}{ DK 22B72}

%\subsubi{Subjective projections [go to \ref{sub:projections}]}

\sep
%
\pa All the \co{aspects} of the \co{attached} \No\ express the opposition to the
idea that it results from any choice, whether \co{spiritual} or not.  There is
nothing there, and so there has never been anything inviting to, not to say,
forcing any fundamental \co{choice}. One simply follows the given order of
things and \co{visible} nature and is no more responsible for it \wo{than is the
  Nile for her floods or the sea for her waves.}$^{\ref{cit:Nile}}$ But lacking the
\co{concrete foundation}, \No\ is occupied with the constant search for
\co{visible} reasons serving as its explanations -- in the world, in life, in
one's life\ldots The lack of \co{justification} only increases \co{thirst} for
it.
% The lack of \co{visible} correlate is sometimes compensated by an \co{idol} and,
% most often, by the incessant shifting of focus between various elements from the
% above list.
But as no \co{visible} and objective foundation can be obtained, 
the issue of \sch\ appears as pure subjectivism, a mere
projection.  For indeed, to great \citet{absurdities men were forced by the
  great license given to the imagination, and by the fact that every existing
  material thing is necessarily imagined as a certain substance possessing
  several attributes; for nothing has ever been found that consists of one simple
  substance without any attribute.  Guided by such imaginations, men thought
  that God was also composed of many different elements, viz., of His essence
  and of the attributes superadded to His essence. Following up this comparison,
  some believed that God was corporeal, and that He possessed attributes;
  others, abandoning this theory, denied the corporeality, but retained the
  attributes.}{Perplex}{I:51} All this, according to Maimonides, leads to
polytheism for one is eventually forced to deify each separate attribute or
element of the divine essence. We would rest satisfied with the statement that
this amounts to anthropomorphic image of God which becomes, indeed, a projection
of our forms of understanding.  This, \citeti{[t]he divine essence is nothing else but
  the essence of man; or, better, it is the essence of man when freed from the
  limitations of the individual, that is to say, actual corporeal man,
  objectified and venerated as an independent Being distinct from man
  himself.}{Feuerbach}{\citaft{Cople}{ vol.VII:II.15\kilde{p.296}}}

It is easy to agree with the element of criticism against the naive, childish
image of God -- and here one agrees with Maimonides as much as with
Feuerbach.\ftnt{And with many others, one of the first being Xenophanes accusing
  poets of ascribing to gods all too human features: \citefi{They have narrated
    every possible wicked story of the gods: theft, adultery, and mutual
    deception.}{DK 21B12/11}{} The observation that \citefi{Aethiopians have
    gods with snub noses and black hair, Thracians have gods with grey eyes and
    red hair}{DK 21B16}{} summarises Feuerbach's reductions of theology to
  anthropology.\noo{Of course, the philosopher from Colophon did not stop}} But
the two part ways very quickly.  For while Maimonides, Xenophanes and most
others criticise a {\em misconception} of God, Feuerbach identifies naivet\'{e}
of such a misconception with the essence of religion.\ftnt{It might be very easy
  to pretend that statements like \citefi{God created everything through me,
    when I was in the ineffable foundation of God [...] If I were not, there
    would be no God. There is no need to understand
    it}{Eckhart}{\citaft{Mistyka}{A:IX\kilde{p.121}}} support such
  identification (\ref{sub:yes}.\refpf{pa:GodneedsMe}). The difference concerns
  only the presence of the ultimate \co{transcendence} behind such statements --
  Godhead, after all is there, even if He acts only through me. Only this,
  apparently minimal, \co{invisible} difference separates two incommensurable
  worlds: that of an open religiosity and that of a denial and
  reduction to \co{visibility}.}
%
Answering the accusation of a subjective projection one could mention several
things.  \noo{1)}Firstly, even with respect to the brief quote from Feuerbach
above, the ontological argument remains: whence do we get the idea of our
essence (or anything) \thi{freed from the limitations}?\ldots Ideal limits are
usual accomplices of unsuccessful reductions, of failed attempts to grasp
something higher. \noo{2)}More importantly, there is the category mistake,
characteristic for the \co{objectivistic illusion}: \co{subject}-\co{object}
distinction does not apply at all at the level of \co{invisibles}.\noo{
  (II:\ref{sub:objsubj}, and then \refpf{pa:objsubjA}).}  Nobody is able to
voluntarily create the object of one's faith.  \co{I} certainly does not do it.
So God, to the extent it arises from \thi{man}, arises from \co{his self}. But
then there is no need to project it as \wo{an independent Being} because \co{my
  self} is not \co{myself}, nor \co{self} is \co{my self}. Man is more than
\co{himself}, \wo{man infinitely transcends himself.}$^{{\rm
    II}:\ref{cit:manMan}}$ One cannot ascribe the \co{analogues} (taken as
expressions of the \co{spiritual} \Yes) any \co{subjectivity} because they reside not
only far \co{above} the \co{actuality} of the \co{subjective acts} but also
\co{above me} and \co{my} whole life. The favourite triviality one might bring
against the reality of such a \co{choice} would be the claim that everybody can
decide it for \co{himself}, and this is what one means by \wo{subjective}. But,
for the first, the fact that \co{I} can choose whether to kill my boss or not,
does not make the alternatives of the choice, nor its consequences, anything
subjective: whatever \co{I} choose, will affect the world in which we live (the
\co{objective} world, if one insists).  But one would still object: \co{I} can
chose, it is only {\em up to me} to decide -- there is nothing {objective},
nothing \thi{outside} which forces me to choose one rather than other. This may
be true about usual \co{acts}, and \co{acts} of choice in particular, but not
about \sch. \co{I} do not decide, \co{I} can at most choose to consent to or
reject its result. Its {objectivity} resides in its \co{transcendence} -- \co{I}
do not create the alternatives, nor do \co{I actually} choose between them.
\co{I} can only relate to them, for even if they do not stay \co{objectively}
\thi{outside}, they by far \co{transcend} the sphere of \co{visibility}.
\noo{3)} \Sch\ is not something \co{I} decide and make -- \co{I} only carry it
with \co{me}. Its reality is not consummated in any \co{actual act} (in such
\co{acts} \co{I} can at most \co{reflectively} consent to the \co{choice} or try
to direct \co{myself} towards one of the alternatives) but remains constantly
\co{above} \co{my} world. In this sense, as the fundamental \co{aspect} of
\co{existence}, as the ever \co{present} possibility, it is much more
{objective} than any \co{actual objectivity}.
%
\noo{Becoming conscious of it changes relatively little, and projects of
  liberating oneself from it are only signs of a wrong \co{choice}.}

\pa
But, truly, it is not so that \Yes\ {\em sees} something that \No\ could not see;
both {\em see} just \co{nothing}. The difference lies in the character of the
\co{confrontation}, in that while \No\ finds nothing where there is \co{nothing}
to see, \Yes\ finds the \co{origin}. This truth concerns \thi{the world} as much
as the one who recognises it, and hence it cannot be reduced to a mere \co{act}
of comprehension, to a mere recognition of a factual, \co{objective} truth.
There is, after all, \co{nothing} to see.  When one finds God it is not because
one has traveled far enough and reached the insight, some new hidden truth. It
is because one has recognised that the very place from which one has not moved
and the very things which have been around all the time witness to the divine
\co{presence} and disclose it to one who is \co{open} to their silent voice.
%
\citet{Some people make believe to find God as a light or savour; they may find
  a light or a savour but that is not to find God. According to one scripture,
  God shines in the dark where every now and then we may catch a glimpse of him.
  Where to us God shows least he is often most.}{EckLastWord}{} Those similes
equating God with sun and our powers with eyes unable to look straight into it
because of its overwhelming brightness, lead easily to imagining some
\thi{objective} brightness to be sought \thi{out there}.  But this so imagined
\thi{objective being there} reflects nothing more (nor less) than the constant
possibility of \Yes; \thi{objectivity} is a figure of speech signaling this
constant \co{presence} independent -- just as the \co{objects}' temporal
duration -- from our attention and recognition of it.  Beyond that the image of
\thi{objectivity} may be deluding for it is \Yes\ which brings forth the
brightness, not (any \co{experience} of) brightness which \co{founds} \Yes.

The ultimate sense of absolute {objectivity} of God emerges from \Yes\ and
amounts to its invariable possibility. One may know or believe it, or not, one
may go around in complete indifference or despair, one may deny its very
possibility. No matter what one does, the possibility always remains, perhaps
only infinitesimal and almost disappearing, usually completely \co{invisible},
but still a possibility.  All the things which we can manipulate and arrange
according to our wishes, are as pleasing as they are uncertain -- if they can be
arranged, they can also happen to get dis-arranged.  But the indifference to our
attention, this non-relativity to being perceived or not, to being arranged or
not, even to being intended or not, is what one considers the most fundamental
characteristic of {objectivity}. Only that its certainty does not emerge from
any analyses, attempting to make it \co{visible}, but from \co{open humility}.
The luminous face of God waits unmoved, unchanged, patient for being discovered,
though not for being seen. And when it is discovered, it is discovered as having
been there all the time, waiting unmoved, unchanged, the same as it was
discovered for millennia ago and as it will be in all the future. The ultimately
certain -- unchangeable, \co{shared} and objective -- is the ultimately
\co{invisible}: \co{eternity}, the \co{presence} of the \co{origin}.

Truth of a mathematical theorem does not depend on anybody personally
understanding it or not.\noo{ -- it is usually announced by the more able to the
  less able ones.} Once proved, it remains an open possibility for everybody
interested to figure out its meaning and recast its proof. The fact that most of
us never do, does not take any {objectivity} away from it. We may even need an
occasional reminder that the theorem does hold, in spite of our inattention.
Even if its significance gets modified by the accumulated results and the actual
context, it still appears as only an unfolding, an explication of the potential
contained already in the statement of the theorem.  Certainty of its truth,
encoded in the proof and accessible to everybody, is the same as it has been
since its discovery. Its truth is the same as it always has been and as it
always will be.  Is there anything more objective, more certain?  Is there any
more absolute sense of truth?  \Yes, too, is such an \co{eternally shared} and
ever \co{present} truth. But its \co{concrete} proof has to be carried out again
and again, anew by every individual \co{existence}. As long as it fails, the
theorem may seem incomprehensible or even false.



% Augustum (semnon) jako dobro numinotyczne, prowokujace fascinans (subiektywna
% reakcja); 
% There is no element of anthropopathism in \co{analogues} unless, of course, one
% turns them into mere expressions of our feelings...





