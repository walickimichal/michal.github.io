\ad{act vs. content}
Retentions and protentions can be (mis)understood in two complementary ways,
which duality opens a vast field for sophistry. On the one hand, they can be
interpreted as modification of actual contents, as noemata fading gradually into
the immediate past or emerging gradually from the immediate future. On the other
hand, they can be interpreted as characteristics of the acts of
apprehension/perception, which present the content in a (temporally) modified
fashion. As I turn my head left, the cup on the table disappears on the right,
but it remains present in the retention. {\em What}, actually, remains so
present? The content-interpretation would claim that it is some hyletic content
which fades away, gradually dissolving on the right border of my visual
field. The act-interpretation would claim that it is only the intensional
correlate, the cup itself, which is no longer given really (\ger{reell}) but
only retained as the immediate awareness of the \thi{cup having-just-been/still-being
  there}. This later interpretation seems phenomenologically more adequate and
even if Husserl's formulations might, occasionally, suggest the former, 
it seems that he had the latter one in mind.\noo{(\cite{GallSync} presents the
opposition in the sharp form...)}

But the distinction need not be an opposition, in particular, if we consider
that although noema are distinct from noesis, they are inseparably bound
together, although what is distinguished is distinct from the (f)act of
distinguishing, the two can not be \co{dissociated}. The \thi{cup just
  disappeared on the right} is, certainly, another content of the \co{immediate}
experience than the cup which was seen there a second ago. It is, indeed, the
same cup but now presented in a new modus. If only the \co{immediate} hyletic
data are taken as real, then such a cup is already turning into an ideality,
even if it still remains within the \co{horizon of actual experience}. On the
other hand, fading away of the cup from the perceptual horizon, involves equally
\co{recognition} of this fading-away and as such a new act. One can posit
contradictions but, using the phenomenological language  (pregnant with all the
paradoxes of reality reduced to \co{imemdiacy}), we would rather
understand ``fading away'' as simply such a gradual transition into a merely
intensional presence (though often it can be accompanied also by the actual
fading away of the perceptual contents), as sliding away beyond the limits of
the \co{horizon of immediacy}. Seen in this way, already the most recent
memories carry the element of \thi{ideality} which in experience accompanies
only the more remote ones.
%
As usual, it all depends on what one is willing to consider as \thi{an act}.
The ambiguity between an event of \co{recognition} and an \co{act} of
\co{attentive reflection} is here imminent