We have followed the order of ontological \co{founding}, of gradual
\co{actualisation} of the \co{original} \co{virtuality}, of gradual
differentiation of the \co{indistinct}, in short, of something emerging out of
\co{nothing}. We have thus arrived at the level of \co{reflective}
\co{experiences}, of the \co{dissociated objects} appearing within the familiar
circle of the \hoa.

But familiarity and secure surroundings breed quickly, if not boredom, so some
restlessness, a thirst for a challenge of the unexpected, perhaps even for a
meeting with the eldritch and uncanny. \co{Reflection}, occupied with its
\co{dissociated objects} and their relations, remains \co{confronted} by the
\co{non-actual} element of \co{experience} which keeps reminding it of, if not
threatening it with, the impossibility of a complete reduction and of bringing
everything under its control. In principle, one might imagine some
further differentiation of the \co{reflective} contents which might, perhaps,
offer new and more \co{precise} means of control. But this seems possible only,
so to speak, in a merely quantitative way. Having reached the level of
\co{dissociation} where \co{objects} are seen and manipulated in
complete independence from each other, where they appear as autonomous and
self-contained \thi{substances}, there is hardly any further possibility of
differentiating the \co{indistinct}. Certainly, one need not accept any actually
given limits of \co{distinctions} as ultimate. A thing can be further
differentiated into its properties and constituent parts, the parts can be
investigated with increased \co{precision} leading to more and more minute atoms
and particles which are \co{recognisable} only with the help of more and more
sophisticated equipment. But all such differentiation yields only new
\co{objects}, more of them and more particular ones, but still only
\co{objects}. It does not establish any new hypostasis, any new level of
\co{experiencing} with a character which would be qualitatively different from
that of \co{reflective dissociation} -- it only increases the \co{precision} of
\co{dissociated objects}.

Therefore, we will not differentiate any further but will now reverse the
direction and consider the ways in which the ontological hierarchy of hypostases
is \co{reflected} in \co{experience} or, what amounts to the same, how
\co{reflection} may organise the world.  We will thus proceed with the
categories of \co{reflection}, with some limits of \co{distinctions}, and move
in the direction opposite to the order of ontological \co{founding}. Although
\citeti{the way up and the way down is one and the same,}{Heraclitus}{DK 22B60}
it looks and feels very different when walked up and down. We will follow the
realistic (in the sense of I:\ref{sub:givens}) way bottom-up, but will also keep
in mind the constant \co{presence} of all higher and earlier hypostases in every
\co{actual experience}.

% the importance of the original constitution will constantly disturb the
% assumptions about the metaphysical status of any givens, things or essences.
% \co{Reflective signs} are always permeated by the \co{signs} of the higher
% levels -- in any case, by the \co{original signs} of the higher level.

% The levels of Being, the character and the structure of each subsequent
% hypostasis, are not only forms of \co{experience} but are part of
% \co{experience} as well.  Through all the concreteness and specificity of
% \co{actual} \co{experiences}, we are also confronted -- in the background,
% underneath the \co{actual} contents -- with the \co{unity} of \co{experience},
% with \co{chaos} and, eventually, with \co{nothingness}.  These, however, are
% never \co{objects} of any \co{actual experience} -- being its \co{ontological
%   foundation}, they only surround it and permeate it with their constant and
% \co{invisible presence}.


\noo{We will see that different kinds of \co{experiences} and their qualities
  are closely related to, in fact, founded by the temporal scope of the context
  in which they are placed.  The hypostases of Being, having resulted in
  establishing temporality, even the experience of the objective time, are to
  high extent reflected in the \thi{time it takes} to experience something.  }

%%%%%%%%%%% Levels
%%%%%%%%%%%%%%%%%%
\section{The existential levels}\label{sec:levels}
\secQQQ{8}{And he dreamed, and behold a ladder set up on earth, and the
top of it reached to heaven: and behold the angels of God ascending and
descending on it.}{Gen. XXVIII:12}
%
\noo{ !!!

{\small{
\begin{tabular}{r||r|l|l|l||r@{\ -\ }l|r@{\ -\ }l}
%{r||r|l|l||rl|rll}
 & \multicolumn{2}{c|}{\co{sign}} \\
level & reflective & original & subject & object &\multicolumn{2}{c|}{attitude} & 
 \multicolumn{2}{c}{value} \\ \hline\hline
imm. & observation & sensation & sense & it & take&escape & 
pleasant&un- \\
  & \multicolumn{2}{c|}{perception}  & subject  & object   &  \multicolumn{2}{c|}{}   & \\  \hline
med. & concept & mood & body & complex & obtain&avoid & 
correct&in- \\
     & \multicolumn{2}{c|}{impression} & Ego  &  & \multicolumn{2}{c|}{} & just&un- \\ \hline
mine & thought & emotion & soul & totality & care&? & right&wrong \\
personality     & \multicolumn{2}{c|}{idea} &      &    & respect&dis-  \\ \hline
revel. & intuition & inspiration & spirit & unity & commit&betray & holy&un- \\ 
communion      &\multicolumn{2}{c|}{command/symbol}    &        & & obey&dis- \\
\hline\hline
& thinking & feeling & & & \multicolumn{2}{c|}{} & good&evil
\end{tabular} }} \vspace*{2ex}

\noindent
{\small{
\begin{tabular}{r||c|r|l|c|c|c}
%{r||r|l|l||rl|rll}
founding & \.{z}ywio{\l} & \multicolumn{2}{c|}{\co{sign}} & constit. of  & objective & signs of \\
  event  &           &    unifying/ & different./    &          time/space &   time  & a thing \\ 
  & & original/ & reflective/ & & & in objective\\
   & & passive & active & & & time  \\ \hline\hline
%
 separation & unity & \multicolumn{2}{c|}{\co{symbolic}} & * &  whole &  whence \\
  birth           &         & inspiration & intuition     & & time-line & whereto \\  \hline
%
distinction & vague/ &\multicolumn{2}{c|}{\co{idea}} & co-presence &  various & thing's \\
            & general &  feeling &  gen.~thought & (spatiality) & actualities &     past-future \\ \hline
%
recognition & mutuality &\multicolumn{2}{c|}{\co{perception}/image} & actuality & immediate & immediate \\
            &         &  impression & thoughts/ &   (spatio- &context &             past/future \\ 
            &         &  /mood         &  concepts         &        -temporality)   &        &             \\ \hline
%
reflection & localised &\multicolumn{2}{c|}{\co{individual/sensation}} & \herenow & ideal now & actual this \\
           &  isolation           & reaction & act           &     space -- time &          point & protentions-\\
           &                 &               &               &  simult --            linear   &  & retentions
\end{tabular} }} \vspace*{1ex}

\noindent

!!! }


\pa\label{pa:analogy5} As the earlier hypostases gather underneath the
differentiated contents of \co{actual experiences} and permeate them with the
\co{invisible rest}, they are also \co{experienced} -- not as \co{actual
  objects} but as layers which surround any such \co{object}, as \co{aspects} of
any \co{actual experience}.  We can roughly distinguish the levels of
\co{any experience}: \imm \co{immediacy} is like the ideal limit of
\co{spatio-temporality}, impossible and unavoidable companion of
\co{reflection}; \act \co{actuality} is determined by the contents within its
\co{horizon}; \mine the level of \co{mineness} encircles the limits of \co{my}
world and \co{my} whole life, contributing the personal \co{aspect} to
every \co{actual experience}; \inv \co{invisibles} are the ever \co{present
  aspects} which never enter the \hoa. Augmenting figure 4) from
I:\refp{pa:analogy4}, we can mark schematically
the levels as indicated below:
% Figure~\ref{fig:levels} below. 
% \begin{figure}[hbt]\refstepcounter{FIG}
\[
\hspace*{1em}\xymatrix@R=0.10cm{
5) &&&&   &&&&  &&&& && &\\
&&&& & \ar@{--}[rr] && \ \ \ \ \ \ \ \ \drop{\raisebox{7mm}{\co{invisibility}}} 
\\ 
&&&&   && \\ 
&&&&   &  \raisebox{1mm}{\ci{40}} && \drop{\raisebox{6mm}{\co{mineness}}} \\
&&&&   \\
&&&&   & \ar@{--}[rr] && \ \ \ \ \ \drop{\raisebox{-7mm}{\co{actuality}}} \\
 \ar@{-}[rrrrrrrrrrrr]%+<3mm,0mm>  
&\drop{|}&&&   & 
     \ar@{}[uuuur]+<2mm,7mm>^>{\displaystyle\bullet}
     &\circ &&  &&&\drop{|} & && \\
   & {\scriptstyle{L}} &&&  &&\Drop{\mbox{\co{immediacy}}} && && & \scriptstyle{R} &
}
\]\label{fig:levels}
% \caption{The levels of experience}
% \end{figure}
\co{An actual experience} involves all the {levels} and does not consist of some
\thi{four parts}. (Imagining the circle moving along the line, only the actual content
but not the levels become affected by its actual position.) We can nevertheless, and
consciousness typically does, view it
from a limited perspective of a single {level}, that is, within a 
definite temporal scope which determines the context of reference and the
character of the relevant contents. Each level is a \nexus\ of various
\co{aspects} (in fact, \co{traces} of the \co{original separation}), among which
we will address the following: 
%The involved aspects are:
\begin{enumerate}\MyLPar
\item\label{as:sig} the character of the \co{signs},  of the
\co{actual} appearances, specific for the contents of a given level;
%and determined primarily by the \co{distance} 
\item\label{as:cor} the correlate of the experience -- its \thi{objective pole}, 
  the character of its contents; 
\item\label{as:sub} the character of the \thi{subjective pole}
\item[{}](These last two 
  points only at the lowest level approach the usual \co{dissociation} of
  \co{object} and \co{subject}; they could be better called \wo{origin} and
  \wo{reflection}, if we have not reserved these words for other purposes.)
\item\label{as:tra} the form of \co{transcendence} pertaining to the contents of
the given level;
there are two different aspects which, together, 
constitute the character of \co{transcendence}:
\begin{enumerate}%\MyLPar
\item the \co{horizontal transcendence} of the correlate, of the \thi{objective
pole} of \co{an experience} at a given level; as a variation of the \co{horizontal 
transcendence}, one will usually 
encounter the merely quantitative transcendence of other correlates with
respect to the \co{actual} one
\item the \co{vertical} or qualitative \co{transcendence} which, referring to the 
\co{non-actual} aspects of the \co{experience} at the current level,  
points towards the higher one. 
\end{enumerate}
\end{enumerate}
In \co{reflective} terms, the basic factor distinguishing the levels is the
temporal scope of the involved experiences: from the ideal timeless point, pure
\herenow\ of a single \co{object}, through the finite and limited scope of 
\co{objective complexes}, then the finite but unlimited time of one's whole life,
to the -- again timeless, but now living -- eternity, the immovable \co{presence} of the
\co{origin}. Using this as the basis of distinctions, let us nevertheless
remember that these differences are only indications of the whole \nexuss\ of
\co{aspects} which distinguish various levels.


\subsection{Immediacy}\label{sec:levelA}
%\subsub{Temporal scope}
\pa\label{pa:horizonImmediacy} Let us try to imagine a shortest possible time
span in which we can still experience, feel, sense -- discern -- something. The
question is not about how long this time would be, but about what would remain
as the possible experience.  It might be, perhaps, a single sensation, a
punctual, localised, feeling of pain, pricking, heat. It might be hearing a
noise, a single sound, seeing a simple thing. It might be, perhaps, a single
thought, an isolated image, appearing instantaneously in our imagination. It
might also be all such aspects together in one moment. The problem is, of
course, that although we have a sense of \thi{one moment}, nobody has a
slightest idea what it possibly might be. It seems to have some experiential
basis but also, when \co{reflected} upon, it dissolves as the ideal limit of
\co{immediacy}, the point of pure \thi{now}.  We would propose to call it \wo{a
  shortest unit of experienced time} or, perhaps, \wo{a shortest unit of time
  which can be referred to an experience}.  We are not asking, as Locke
did,\kilde{Essay concerning...II:14} if such a durational \thi{specious
  present}, such \la{minima sensibilia}, can or can not be further divided into
more minute and \thi{objective} \la{l'atome du temps} of Poincar\'{e},
\thi{quanta of time} of Whitehead or \thi{chronons} of some contemporary
physicists\kilde{Dictionary of the History of Ideas:Internet} -- its
\thi{objective} duration does not matter to us. We might think of
experiential \co{immediacy} as the lowest level of distinguishing at which a child
development turns and starts to consciously construct the world. It is
impossible to say when, exactly, a child leaves the state of relative passivity
and becomes active because, as far as we can see, it is only a matter of degree
and none of these aspects ever occurs without the other. But we notice as the
irritating (and sometimes charming) short attention span of babies and small
children becomes gradually longer and the child gets less and less determined by
the immediate presence of stimuli -- slowly, from the minute bits and pieces
(which also arise all the time along the way), it begins to rise complex
constructions which reach beyond the \co{actual} givens towards the
\co{non-actual} limits.\noo{This is the level of pure, ultimate \co{immediacy}.}
\co{Immediacy}, left behind as the constructions proceed, remains however at
their bottom as that which is im-mediate because it has no time to be
mediated. What is experienced \co{immediately} may vary, in particular, with the
level of attention we pay to things but it will never last two hours, it will be
always comprised in a tiny, not to say infinitesimal, instant of time, at the
limit of \co{actuality}.  We will devote a few paragraphs to its experiential
basis but most of the section will be concerned with the status of the
\co{reflectively posited}, ideal and infinitesimal limits.

\subsub{The signs}

%\titp{\Oss}
\ad{Original signs}\label{sub:originalSigns1}
The \co{sign}, the \os\ of such an \co{immediate experience} is not 
announcing anything, or better, it is announcing itself and only itself. 
Whether it is \thi{subjective} or \thi{objective}, 
whether it is sensed pain, heard noise, felt dread, perceived object, imagined 
thought, it is a \co{sign} which fully coincides with the signified. 
It has been \co{cut out} of the horizon of \co{experience} but 
\co{attentive reflection} has not yet had time to carry out its \co{representing} 
explication. When I get burned by a glowing spark thrown out from the 
fire, I do not experience a sensation {\em and} a spark, it is the 
spark which hurts me, I might cry out \wo{Ah, {\em it} burns!}
\thi{It} is equally the spark and the place of my body where it burns. 

We are not here after any phenomenological intentionality. On the 
contrary, within the temporal scope of 
\co{immediacy}, there is hardly any difference whether the \co{sign} 
has some \thi{objective} or else only \thi{subjective} correlate. The 
\co{sign} and what it possibly signifies
may be, perhaps, distinguished by subsequent \co{reflection}, 
but they coincide in the \co{immediate experience}. There is no 
intentionality and the experience has the form of a pure \thi{state} 
with a definite quality (pain, warmth, meekness, etc.)

\pa An \co{immediate sign} coincides also with the reaction, if any.  Pain, like
that caused by a burning spark, is {\em nothing else} than the \co{immediate}
withdrawal, or attempt to withdraw, reaction of avoidance or defense which,
typically, is taken care of by the autonomous part of the nervous system.
Similarly, a pleasant sensation is nothing else than the response of my body to
its attraction. (This may become \co{reflectively} realised first when
the pleasant stimulus withdraws and I have to attempt to approach it, but
originally, there is no distinction between the pleasant stimulus, its
attractive force and the reaction of \thi{approaching it}, \thi{preserving it}.)

Such reactions are {reflexes}, that is, elementary \co{reflections},
involving an extreme narrowing of the focus to this particular, isolated point.
Bergson, identifying sensations with perceptions of one's body, says it this
way: 
\citet{The psychical state, then, that I call ``my present'', must be both a
  perception of immediate past and a determination of immediate future. Now, the
  immediate past, in so far as it is perceived, is, as we shall see,
  sensation,[...] and the immediate future, in so far as it is being determined,
  is action or movement.}{MattMem}{p.138.\kilde{See also p.57} The immediate past
  and future, so aptly related here to receptivity and action, which Husserl
  turned into retentions and protentions, are results of analysis which has
  placed such an \co{immediate experience} in the idealized, objective time or,
  as Bergson would have it, of duration.  But we do not intend to develop a
  theory of sensation and perception and will stay with the mere quality of
  \thi{my present}, \co{immediate experience}.}
%
Re-action, this \co{act} or movement which follows within the \co{immediacy} of
sensation, is what we call \wo{reflex}. It involves both receptivity, the re-, in
so far as it is triggered and not mediated, and also activity, the -action, in
so far as it is actually doing something, performs some movement.

There is no subject of such an event. Sure, it is I who experience the pain, but
in its \co{immediacy} it is not even relative to {\em my} body, but merely to a
particular sense, particular organ, particular point of the body. Its minute
localisation refers it to the reacting organ which, so to speak, only happens to
be mine. The experience itself does not involve myself, only the affected place
of my body.  In fact, a more \co{reflective act} is needed to refer such \co{an
  experience} to a subject, to myself. As long as I do not perform it, I
\thi{drown} in the \co{immediacy} of \co{an experience} and {reflexes}, which only
\la{post factum} appear \thi{as} mine.

%\titp{\Rss}
\ad{Reflective signs}\label{refl:A} \co{Reflection} happens always within the
\hoa, but it may bring within this horizon \co{signs} of different things. The
question analogous to that asked in \refp{pa:horizonImmediacy} would be now:
what can \co{reflection} focus on in a \thi{shortest possible moment}, or
better, what kind of \co{reflection} would make its \co{object}, not only its
\co{sign}, closest to itself? It would be directed toward some \co{actual
  object}, present \herenow, not to \thi{Prague}, \thi{meaning of life},
\thi{love}. And then, it would not be a \co{reflection} which contemplates how
this \co{object} is, what it is like, etc. It would be a mere registration that
\thi{\ldots}, \co{that it is}.

One can sometimes experience the astonishing fact {\em that} this something in front
of me {\em is}, that it at all {\em is}. Existentialists made such experiences
into a nauseous feeling of unbearable, meaningless presence but it may also be quite
a detached and full of gratitude realisation \co{that it is} while it might not
be. The question \wo{Why is there something rather than nothing?} may be given
other connotations but it, too, expresses such a
\co{reflection}, the amazement at the mere {\em fact} of this something being at
all.  The experience simply reveals object's naked \co{immediacy},
the mere fact %(ignoring its being so-and-so, coming form here or there)
\co{that it is} -- fully \co{actual}, in front of me, \thi{out there}. 

This idea, underlying the whole \thi{metaphysics of actuality}, has here its
concrete, experienced form. \co{Reflection} \co{that it is} shares with the
\oss\ the full consummation within the pure, experientially unextended
\co{immediacy}.  But as \co{reflection} it is, of course, already doubled. The
\co{object} of such a \co{reflection} appears in total \co{dissociation}, the
strangeness of its being at all is, in fact, the strangeness of its being here
\thi{on its own}, of its being so strangely alone \co{before} -- more in the
spatial sense of \thi{in front of}, rather than of temporal precedence -- or
against the \co{reflective act}. Thus, the \rs\ does not any more coincide with
its \co{object}. But \co{reflection} does not have time to \co{reflect} over the
\co{distance} which remains merely \co{experienced}. It is for \co{reflection}
simultaneously its \co{act} and the appearing \co{object}, both coinciding in
the infinitesimal limit of \co{immediacy}.


% \co{Reflection}, in particular, \co{attentive reflection} must posit 
% an \co{object}. Whether it is a sensation, a perception, an object of 
% perception or a cause of sensation does not matter much here. What 
% matters is the \co{immediate} presence of this \co{object} for 
% \co{reflection} which merely experiences \co{that it is}. Acts of 
% such \co{reflection} build upon the \co{immediate experiences} and 
% their \oss, but only in that that they narrow the \hoa\ 
% to pure \co{immediacy}, to the \wo{shortest unit of experienced time}.


\subsub{Substances, objects, particulars}\label{sub:substances}\label{sub:idealImmed}
%
\pa An \co{object}, appearing in the \co{reflection that it is}, appears as a
substance to which belong \wo{both separability and <<thisness>>}, as the
\citet{ultimate substratum which is {\em no longer} predicated of anything else;
%that which is not predicated of anything,
  but of which all else is predicated}{AristMeta}{V:8; VII:3}, as \citet{a
  complete indivisible being.}{LeibnizArnauld}{to Arnauld 28.11/8.12 1686.  [In
  whole this section, we exclude living beings from our considerations.
  Likewise, here we have to exclude the specificity of Leibniz' notion of a
  substance which ends up with a monad (corresponding roughly to our
  \co{existence}). Here we are only concerned with the characteristics one was
  expecting substances to have.]}
%
A {thing}, a physical {thing} of daily experience is a paradigm of an
\co{object}. And the word \wo{object} is particularly significant for it brings
a {thing} out of the context of such \co{an experience} and places it in the
context of \co{attentive reflection that it is}.  What it achieves, or in any
case attempts, is reducing the object to a dimensionless residuum, emptying it
of any specific content and merely suggesting its \thi{being there}, pure
\thi{substantiality} stripped of all \thi{accidents}.  It reduces the
\co{actuality} of a {thing} to a mere indication of something definite (even if
indefinable) and \co{cut out} from the surroundings, an independent \thi{being
  there}. It reduces a thing to a point.

The intuitions of Leucippus and Democritus in this respect, motivated by the
materialistic reductionism, can hardly be overestimated. The Greek
\wo{\gre{atomon}} means indivisible, and the speculative theory of atomism is
the first tribute paid to the idealized notion of a \thi{substance}. It captures
the abstract idea of the least and eventual building blocks of the universe in a
much clearer, if abstract, language than the elements inherited by the Ionian
philosophers of the ultimate nature from the earlier religious
thought.\noo{Physis is all-embracing soul and divine, i.e., has close to nothing
  in common with our nature. \wo{was a soul-substance, a supersensible and yet
    material thing, which was embodied in this or that element, rather than
    identified with it (though on this point our information is, naturally, not
    always clear). We know that they regarded is as God and
    Soul...}\citeauthor*{Cornford}{73-80.}} But it is less satisfying to the
empirically oriented mind and what follows, naturally, is the search for such
points in experience.  Whether these are identified with sensations,
perceptions, clear and precise ideas, objective things, atoms, elementary
particles, quarks or strings are secondary distinctions. They all move within
the horizon of \co{immediacy} under the evanescent shadow of the ultimate
simple, \citet{that bounding point [which] indeed\lin Exists without all parts,
  a minimum\lin Of nature.}{Things}{I:5}

\pa\label{pa:subIndependent}\label{pa:subUnchange}\label{pa:DescartesKnow}
All the criteria for something being a \thi{substance} are gathered in, and
sometimes follow from, this ideal limit of a pure point. But primarily they just
reflect the 
characteristic features of a limit of \co{distinctions} solidified as a positive
entity. We mention only a few examples. 

%independent 
The isolation of the \co{object}, its definite \co{dissociation} from the
horizon of the surrounding \co{experience}, gives rise to the idea of its complete
independence and self-sufficiency, its existence \thi{on its own}. This
independence is immediately given in the 
\co{reflection that it is}. It can be observed when we notice how a
{subjective} feeling, say, a sensation, becomes independent in a similar way
when it is posited as an \co{object}. An \co{immediate} sensation of pain gives
the possible \co{reflection} a localised focus but it is still my sensation, it
is still a sensation involved in the context in which it arises. As I
\co{attentively dissociate} it from this context and \co{posit} it as an \co{object}
of \co{reflection}, consider it \thi{as it is in-itself}, in short,
\co{externalise} it, it loses its concreteness and appears gradually as a more
and more depersonalised \co{external} entity, whether an independent substance
or an atomic impression. 


%\pa\label{pa:subUnchange}
A related aspect of sedimenting \co{dissociated distinctions} as a positive
limit or, eventually, of positing an ideal point as a measure of 
reality, concerns temporality. On the one hand, point gains the place 
in the, by now objective, time as the ultimate \thi{now}.
An \co{object}, imagined if not intended, posited if not thought as a point
becomes itself a timeless event, it resides in the purified \thi{now} which, by
this reduction which is also abstraction, has become \co{dissociated} from time.  A
substance, abstracted in this way from time, appears (shall we say,
necessarily?) as a solidified, unchangeable being.  It appears so not by any
analogy, not because it is permanent or because we have extrapolated to it
observations of something relatively constant, but because it has been pulled
out of time, confined to an ideal, timeless point of a pure \thi{now}.

%\pa\label{pa:DescartesKnow}
Another aspect of this timeless permanence, of the
determinacy of isolated independence is the idea of a perfectly clear, unambiguous
presence (or absence) of the point, that is, of a perfectly clear and unambiguous
knowledge. Idea of such a knowledge is, originally, with Aristotle or
there about, merely another side of the idea of a substance: well-defined,
clearly cut out, independent and self-same entity. It would be pointless to
follow the historical development of this idea, but we can remark that
eventually, with the emergence of epistemology, with Berkeley or Descartes, it
gives rise to the idea of knowledge coinciding with its \co{object} -- no longer
in the way intellectual realm of Plotinus or Scholastics contained eternal
intellections being self-same with their objects, but just as in the
\co{immediate experience}, the \co{original sign} coincides with the signified.
The hunt for the infallible, certain knowledge is a companion of the hunt for
the ultimate atoms: both emerge from the idealization of \co{immediate
  experiences}, from the narrowing of the \hoa\ to the idealized, pure
\co{immediacy} of a point.

\wtsep{dissolution}

\pa\label{pa:pointUnknown}
Paradoxically, but also very naturally, a point \co{posited} as the residuum of
\co{objective precision} and infallible knowledge is also the limit of any possible
\co{distinctions}, that is, of comprehensibility. \thi{In itself} it is
unknowable, not however because it hides the ultimate content from us but, on
the contrary, because it does not hide anything which could be
\co{distinguished}, and hence known. Thus, \co{positing} the points of the
ultimate \thi{substances}, one \co{posits} by the same token the unknowable
\thi{things in themselves}. 

Those pointing out how Aristotle's thinking was determined by the structure of
the Greek language and common sense, might certainly observe that the attempts
to capture the idea of a substance end up with the attempts to define what, in
the daily language, counts as {things}.  And they end badly. That which \wo{is
  first in every sense -- in definition, in order of knowledge, in time}; that
which can exist independently from all else; \citet{that which is primarily,
  i.e. not in a qualified sense but without qualification}{AristMeta}{VII:1} --
all that threatens with disappearance in nothingness.  Already in the IV-th
century St.~Basil suggested: \citet{Do not let us seek for any nature devoid of
  qualities by the conditions of its existence, but let us know that all the
  phenomena with which we see it clothed regard the conditions of its existence
  and complete its essence. Try to take away by reason each of the qualities it
  possesses, and you will arrive at nothing. Take away black, cold, weight,
  density, the qualities which concern taste, in one word all these which we see
  in it, and the substance vanishes.}{Basil}{I:8}
Indeed, \citet{there is no body of which I can say for certain
  that it is a substance rather than an aggregation of several substances, or
  perhaps a phenomenon.}{LeibnizArnauld}{to Arnauld, draft, 28.11/8.12 1686} A
particular can be analysed as consisting of other, more minute
particulars as any \co{distinction} can be refined by more
\co{distinctions}. What's wrong with the famous heap of stones? 
What does it lack to be a respectable \thi{substance}? Any inherent principle of
organisation? A genuine unity? A substantial form? What does not lack all that?
If it turns out that this very heap of stones was set up on purpose as a
signpost, will it acquire some of these lacking aspects? And if not then,
perhaps, when it turns out that not only it was not just a heap functioning as a
signpost, but that its exact form and the number of stones had specific meaning?

\thesis{\label{th:noindividual} There are no such metaphysical entities as
\thi{substances}, \thi{particular things}.  A particular thing is a~\ldots
\co{cut} from \co{experience} {signified} within the \hoa.} 
%
A particular thing is a limit of \co{distinctions}, is a place at which the
possible process of further \co{distinguishing} is terminated, or in any case,
suspended.\ftnt{The \thi{ultimate difference} of Duns Scotus seems to carry the same
  intuition, even if richer associations. Although one might classify specific
  difference (constituting the being of a species) and even transcendental
  difference (contracting \thi{Being} to the ultimate genera or categories) as
  instances of \thi{ultimate difference}, eventually, it amounts to the
  \la{haecceitas}, the individuating being and, in fact, to the mere indication
  of \thi{a being}, \thi{a thing}.\kilde{ScotWolter,ftnt.3,p.166} Of course,
  within the mind-things dualism, it has never been clear whether such
  differences are external beings (as Scotus typically maintains) or not. This
  latter opinion is the ultimate consequence drawn from Scotus by Ockham who
  ends up with the dogmatic empiricism: dogmatic, in the sense of postulating
  the objective existence of individuals (and only individuals) which it never
  manages to specify, with all other differences and universals being mere
  abstractions of the mind. We suspect that this, like any other dualism, must
  eventually dissolve {\em both} opposite poles, in this case, also the 
  individuals which end up as mere substantial residua. But we often find in
  Ockham's reduction of several notions to a single being or concept, a simile
  of our way of viewing such differences as \co{aspects} of one \co{nexus}.
  Thus, for instance with respect to the current point, \citef{unity is not an
    accident really distinct from that which is one, and added to it in reality
    outside the mind.}{OckSumLog}{I:c.xliv (Though Ockham makes a reservation
    that he only explains -- and does not necessarily subscribe -- to this
    view.)}}  Particular thing as a limit of \co{distinctions} -- which is
grasped or, in any case, signified within the \hoa\ -- can be equated with this
thing being everything it is not, being the totality of \co{distinctions} which
are excluded and left outside this limit. In this sense, opposites create each
other, though such a \wo{creation} need not be limited only to opposites.

\wtsep{limits of distinctions}

Violence creates mildness, hardness creates softness, repulsion creates
attraction (as well as indifference), etc.. Likewise, though in much more
complicated ways, very advanced \co{objects} may enter the \hoa\ which, too, are
determined only as limits of \co{distinctions}.  The early church created Greek
rationalism, Jerusalem created Athens, not in all respects but exactly in the
respect of its opposition to Jerusalem; Luther and Protestantism created
Catholicism, etc..\ftnt{\label{ftnt:binary}In general, there may be a whole
  series of contraries and not just a binary opposition. Blue does not create
  red, but not-blue, not-yellow, not-... all together give red. All colors
  mutually condition each other, arise by mutual distinctions from each other.}
\wo{Creates}, well, of course all such particular things might have existed
before: mildness might have been the general mood of life {\em until} the first
act of violence interrupted it; the church was catholic and Trent only made many
of the points of its faith more precise against the Protestants whom it, in
turn, created as one entity from the dispersed groups fighting each other, as
the opponents of the Catholic church. \wo{Create}
means often in such contexts a degree of sorting out, perhaps even making
conscious and precise, that is, \co{actualising} some \co{distinctions} and
making them more \co{precisely visible}. But \co{actualising} and making
\co{visible} means that the \co{distinctions} have already been made; the same
kind of \co{recognising} a \co{distinction} takes place at the deeper levels
where, indeed, the \co{distinction} enters the world for the very first time.
And it does so by being \co{distinguished} from everything else.
For although \thi{everything else} is initially, as we have emphasised in Book
I, the background from which the \co{distinction} arises so, once the
\co{distinction} has been made, it functions in opposition to all other
\co{distinctions}. 

\pa Leibniz: \citet{Fictions of the mind, everywhere; and if we cannot discover
  what is truly a complete being, or a substance, we will have no stopping-point
  [...]}{LeibnizArnauld}{to Arnauld 30.04 1687.}  Indeed, there is no
stopping-point.  Yet, the lack of any underlying \thi{substances} does not mean
that \co{distinctions} and their limits are \thi{fictions of the mind}.  On the
contrary, they are discoveries -- \co{distinctions} of and from the
\co{indistinct} -- found and not created, even if found only by the
\co{distinguishing existence} and hence relative to it.  Relativity means
neither arbitrariness nor subjectivity.  Some of \thi{things} are more practical
to fix as \co{objects}, to stop their further differentiation; mostly, those
which are prone to be perceived within the \hoa.  In the \co{attentive
  reflection that it is}, it is \thi{me} who determines the termination of such
a process. But for the most, the limit of \co{distinguishing} is determined by
my body, by the sensory and perceptive system, by the intellectual criteria or
tradition.  The common character of these systems makes it natural that \thi{we}
mostly agree as to what counts as particular things.  Starting from such a basis
-- and consensus -- one may abstract universals, common natures, construct
essences, as well as introduce distinctions. But at the bottom of it anything --
a piece of chalk, a chair, a leg of the chair, a heap of stones, redness,
anxiety, Prague, independence, love -- may be \co{posited} within the \hoa, that
is, turned into an \co{object} and thus given the status of a particular. This
does not mean that particularity is a fiction of the mind -- only that
metaphysics of particular \thi{substances} is an enterprise of a very limited
validity founded on the equally narrow basis as the metaphysics of actuality.

\co{Distinguishing} stops somewhere, usually, for purely pragmatic reasons. It
is much easier to handle a {sofa}, a {coffee-table} and each of the four 
belonging {armchairs} as separate objects than to consider the whole as a
one and indivisible \thi{sofa group}. But, of course, the latter is possible,
too, as is the case whenever you must buy the whole group instead of only one
armchair which you actually like and need. Every \co{object} admits further
\co{distinctions}, the group contains sofa, and table, and\ldots, likewise a
chair has all the parts, not to mention a house, the church, the state. The
limit surrounding the \co{object} has nothing metaphysical or absolute about it,
it can be pushed further up or down, depending on the
circumstances.\ftnt{\citef{But what are the simple constituent parts of which
    reality is composed? -- What are the simple constituent parts of a chair? --
    The bits of wood of which it is made? Or the molecules, or the atoms? --
    \wo{Simple} means: not composite. And here the point is: in what sense
    \thi{composite}? It makes no sense at all to speak absolutely of the
    \thi{simple parts of a chair}.}{WittPI}{I:47} Wittgenstein's examples in the
  following paragraphs are quite illustrative.}  And thus one has never managed
to specify a single \thi{substantial form} of any \thi{substance}. For
\thi{substantial form} is not an inherent property possessed objectively by a 
\thi{substance}, but the mere fact that we have to and hence always do stop
\co{distinguishing} somewhere, that even if the process can always be continued,
it is always suspended at some point, though the points may vary.\ftnt{\wo{[For]
    each particular there exists at least one monadic universal which makes that
    particular just one, and not more than one, instance of a certain sort. Such
    a universal will be a \wo{particularizing} universal, making that particular
    {\em one} of a kind.} This does remind strongly about the \thi{individuating
    entity} of Scotus. But we do not intend to discuss the point any further and
  only notice that such questions were asked not only by Plato, Aristotle or
  Scholastics but are still topics of fervent discussions. The quotation is from
  \citeauthor*{ArmstrongUniv}{ 64} \citaft{AristSubsUniv}{}} What determines
this point may be very strange in any particular case and quite different in
different cases.

\pa One thirsts, of course, for the universal consensus and keeps looking, but the
closest approximation one manages to get is when one resorts to the common
\co{objects} of perception. And even there one finds the obvious differences and
impossibility of establishing any final and positive \thi{substantial form}.
The positive characteristics, supposedly the primary and elementary, are in fact
secondary results of \co{reflective dissociation}; they emerge at the end of the
original process of \co{distinguishing}. Of course, the
process establishes elaborate structure of \co{distinctions} so that, when we
begin to \co{distinguish} the chair from the table, we are already in the room
and not in the desert, when we begin to \co{distinguish} the concept of a group
from that of a monoid, we are well within the mathematical curriculum and not in
the forest, so that \co{reflection} does not have to explicitly negate the nature,
then a forest, then this forest, etc., in order to arrive at a monoid.

Every \co{distinction} is a boundary, creates two poles of the \thi{inside} and
\thi{outside} or, perhaps, just \thi{left} and \thi{right}. We would thus go
further than Lotze who says that a thing is what it does -- a thing is
everything that it is not. If this seems paradoxical, than observe that the
whole work is done here by the word \wo{everything}. A thing is but {\em the
  sum} of all that it excludes, the limit of \co{distinctions} from all that
{\em it} is {\em not}. This is the whole \thi{positive essence} of anything and
knowing one is the same as knowing the other. In this sense, everything indeed
reflects the whole universe, every word means something specific only in the
context of all other words, microcosmos of every particular reflects the whole
macrocosmos.  This is also what makes it possible to dissolve (or as one says
today, \wo{deconstruct}) any issue, any concept, any theoretical construction
and so called \wo{understanding} -- in short, any identity -- in the infinite
web of correlations, themes and exclusions by a systematic, that is, merciless
analysis.

\pa\label{pa:pragmaticPositive} But one might still object for, after
all, it is so obviously natural to think in positive terms. Standing in front of
a house nobody thinks an infinite series of not-\ldots No, but
\co{distinguishing} is much more than thinking, not to mention \co{reflective
  attention}. We certainly distinguish this house from what surrounds it, this
is what makes one see this house at all. Learning the concept of a group, nobody
thinks an infinite series of not-\ldots, no, one thinks perhaps a monoid and
adds a few axioms. But this only means that, making new \co{distinctions}, we
usually introduce them within some given context, whether the context of
\co{actual} situation, the context of discourse, the context of mathematical
definitions and concepts we have acquired, in short, within some \thi{positive}
determinations. But there are some things to observe here. The first is that all
the examples (mentioned here, and usually mentioned elsewhere) concern
\co{reflective distinguishing}, that is, \co{distinguishing} which starts with
something given. This \thi{something given} is the \thi{positive} background to
which some more \thi{positive} attributes are added, as one used to say, it is
genus to which one adds \la{differentia specifica} or, as the case may be, a
species to which one adds accidents of material nature to obtain this
particular. But where does this \thi{positive} background come from? It is {\em
  already} \co{distinguished} from other \thi{positive} backgrounds, the
\thi{positive} backgrounds mutually limit each other. Well, perhaps, but one may
still claim that there is something inherently \thi{positive} which accounts for
this mutual limitation. Our point is that all the \thi{positive} content arises
only as the limit of \co{distinguishing} it from others. The \thi{positive}
character of the givens is the simple matter of efficiency.

Having $n$ distinctions which, in general, divide the space independently from
each other, we obtain up to $2^n$ distinct sub-spaces, each one given by a
combination of positive or negative (\thi{left} or \thi{right}) value for each
of $n$ distinctions.\ftnt{Limiting, for the sake of simplicity, the attention to
  binary \co{distinctions}, cf. footnote~\ref{ftnt:binary}.  Admitting
  \co{distinctions} with up to $k$ contraries does not affect the argument: we
  only replace the basis $2$ with $k$.} (If we denote distinctions by
$d_1,d_2,\ldots,d_n$, and the \thi{left}, respectively \thi{right}, side of $d$
by $d^-$, respectively $d^+$, then a sub-space $s$ corresponds to a choice
$d_1^{s1}d_2^{s2}\ldots d_n^{sn}$ where each $si$ is either $+$ or $-$.) A new
distinction will give $2^{n+1}$ sub-spaces, i.e., every new distinction amounts
to the exponential increase in their total number. Thus if we were to identify
every particular thing by the set of {\em all} \co{distinctions} separating it
from all the rest of the world, we would be exposed to this exponential growth
which would quickly put a limit to our finite abilities.
% But assume that some $x$ of the $2^n$ sub-spaces are more relevant
% and more frequently used or encountered. Or even better,
But assume that we now want to make a distinction $d$ which is relevant only
relatively to one -- call it $s$ -- of all the $2^n$ sub-spaces. (E.g., we only
want to distinguish blue from yellow, that is, our current $s$ is the sub-space
of colors.) In principle, it would require a new level of $2^{n+1}$
possibilities from which only two are of interest: those {\em within} $s$ which
lie on the \thi{left} or on the \thi{right} of the new $d$. Having fixed $s$,
perhaps giving it a name, perhaps associating with it something more than that,
in short, turning it into a \thi{positive} entity, we may now refer to the new
possibilities as $sd^+$, respectively $sd^-$, instead of the whole sequences
$d_1^{s1}d_2^{s2}\ldots d_n^{sn}d^+$, respectively $d_1^{s1}d_2^{s2}\ldots 
d_n^{sn}d^-$.  Explaining to somebody what \wo{yellow} means, we do not start by
saying that it is not a body, nor the taste of lemon, nor the view from Mount
Everest, nor\ldots We start by saying that it is a color -- this limits
immediately the horizon of attention to the relevant sub-space.\ftnt{This, by
  the way, may be necessary even if we were to attempt an ostentive definition,
  as observed in \citeauthor*{WittPI}{ I:27-28}.}  Thus, stopping the
\co{distinctions} at some points and assigning to their limits at these points
\thi{positive} determination -- which simply forgets the chain of negations
which constitute it -- reduces the burden of explicitly handling further
{distinctions} which are to appear {\em within} these, not \thi{positive},
sub-spaces.  The \thi{positive} determinations allow one to forget the --
enormous and typically \co{reflectively} unknown -- number of \co{distinctions}
which, although present, would only disturb dealing with the \co{actual object}
confusing the context with a multitude of irrelevancies. Nevertheless, the
mutuality of \co{distinctions}, their full series behind the \thi{positive}
determinations, can always be invoked and sometimes even used to disturb the
discourse.

% This is the case, for instance, whenever the problems with agreeing on the
% meaning of some terms require constant refinements, new associations and
% distinctions. Other examples are provided by the situations where subject and
% predicate can be freely switched, both expressing the same intention:
% deliberating desire or desiring deliberation, square circle or circular square,
% violent love or loving violence.

\wtsep{summarising...}

\pa Thus, although we grant the positive determinations with their eventual
expression in the idea of a \thi{substance} all practical value, we view them
exclusively as such: pragmatic devices. The ontological status of
\thi{substances}, \thi{substantial forms} and the like is, like of most other
things, the same as their epistemological status, which in this case is: the
limits of \co{distinctions}.
%Rather than the negatively sounding
\la{Omni determinatio est negatio.}\ftnt{Let us only remember that living beings are
  excluded from the present considerations. We should also recall, from
  I:\ref{sub:pantheism}, our objections to pantheism which might seem weakened
  by the above phrase. We are now working with some established
  \co{distinctions}, within a differentiated world where, indeed, choosing some
  determinate contents amounts to excluding others. With respect to our earlier
  development in Book I, the more appropriate formulation would be \la{Omni
  negatio est determinatio}. The two can be accepted jointly as expressions of
  the dual aspect of \co{distinction}: by drawing a border, splitting the space
  in two, it both negates (whatever falls \thi{outside}, \thi{to the left}) and
  determines (whatever falls \thi{within}, \thi{to the right}).}
%positive form: \la{omni negatio est determinatio}. 

We bring thus, indeed, elements of negative theology into the trivial matters,
traditionally treated easily by the positive finitude of the Greek dislike for
even the slightest smell of \gre{apeiron}.  Certainly, a possible and most
natural issue for discussion and investigation is {\em where} the \co{actual}
limits are drawn and what \thi{positive} contents they determine. But as the
Aristotelians have spent on that few thousands years, we could hardly contribute
to the discussions in any way.  Above all, we consider their focus slightly
mistaken because they are almost always underlied by the \co{objectivistic}
assumption of studying the \thi{real}, the \thi{substances} and \thi{essences}.
We do not believe in metaphysical status of some absolute \thi{substances}, we
do not take the particular things as something more primordial than the \hoa.
Nor vice versa.  For just like a particular thing is no ultimate
\thi{substance}, so neither is the \hoa, in the present case, the
\co{immediacy}, \thi{the shortest unit of experienced time} any absolute unit --
it is just an \co{aspect} of \co{an experience} of a particular thing.  The two
mutually condition each other, form a \nexus, a whole system of correlated and
mutually dependent \co{aspects}: an \co{immediate actuality} is constituted by
\co{an experience} of a particular thing, and a particular thing would never
appear if it were not \co{dissociated} and narrowed down to the horizon of
\co{immediate actuality}.

%\subsubi{Idealizing immediacy...}
%
\noo{Although such a \co{reflection} is still an experience, there is a very
  short way from it to the well known ideal, unexperienced constructions.
  
  \say ??? \Rss\ are but \co{dissociated} \oss; though the former may also be
  invented by \co{reflection} for purposes of re-creating the unity of
  \oss\ldots ?
  
  \say Idealization is a twofold process. On the one hand, it is but
  \co{dissociation} carried to an extreme. Such a \co{dissociation} always calls
  forth other elements, something which might account for the interrelatedness,
  totality, eventually unity of the whole. This additional elements appear as
  \thi{ideal} entities \co{posited} beyond the sphere of \co{dissociated}
  contents. Often, they will correspond to the \oss\ which get lost in the
  process of \co{reflective dissociation}. Examples involve all kinds of
  \thi{limits}, whether in Kantian form of \thi{regulative ideas} which account
  for the \co{experiences} of unity, or else in pragmatists' equation of reality
  with the \thi{ultimate fixation of belief}. They are typically effects of
  scientific bias (with Kant, Pierce, Comte) which would like to bring
  philosophy \citt{to a condition like that of natural sciences}{Pierce,
    Selected\ldots p.184}
  
  They are indeed only \thi{ideal} entities which have no \thi{real} existence
  from the point of view of the \co{actuality}\ldots Eventually, they are
  expressions of \co{objectivistic illusion}, or its variants, which restrict
  the sphere of \thi{real} to some limited and relative sphere\ldots }

The eventual dissolution of any thing, as its supposedly accidental features and
properties are being removed, is now a standard objection against the very
notion of a \thi{substance}.  We share it and, as stated in
\refp{th:noindividual}, do not find \thi{out there} any solid and indissoluble
\thi{substances}. But we find them everywhere where our thoughts, perceptions,
feelings, \co{acts} and \co{activities} stop for even a shortest moment,
bringing out of the \co{chaotic} flow of \thi{transitive parts} of time the
\thi{substantive parts} which can be retained and carried over to the next
moment, I:\refp{pa:rest-flow}.  The idea of a substance has a solid foundation
in the elementary operation of \co{reflection}, in the observation \co{that it
  is}, in this \co{reflective positing} of a being as the limit of
\co{distinctions}. It is a constant \co{aspect} of \co{experience}, the result
of the real, actually performed, \co{acts} which comprise the horizon of
attention to the extreme limit of \co{immediacy}.

\pa The idealization of {things}, their reduction to \co{objective},
\thi{substantial} points affects an analogous reduction on the side of the
\co{subject}. Historically, a lot of time and \co{reflective} effort has been
needed to fetch the \co{subject} out of the cave where it was put along with the
\thi{subjectivity}, sophistry and the like offenders of objectivistic
seriousness.  The idea of a purely \co{immediate subject}, a \co{subject} which
no longer has anything to do with human existence but merely with the logic of
universal -- and as momentaneous as timeless -- constitution, may appear as the
extreme offense to this seriousness but, as a matter of fact, it only recognises
a \co{subject} which is a true accomplice of the \co{immediate} givenness of the
\co{object}, which resides within the equally narrow, whether temporal or spatial,
horizon.  It emerges with the epistemology of Cartesian \la{cogito} and reaches
its peak with the problematic of constitution within German idealism.  The
transcendental subject operates always in the ideal \co{immediacy} of an
unextended point, its \co{object} appears in the \co{immediacy} of a single
\co{act} which, carrying the burden of the constitution of the whole world,
becomes as complicated as it is instantaneous.  Contents appearing at a point
\co{dissociated} from its surroundings, from its temporal and spatial context,
appear as arbitrary or, whenever one wants to effect more positive connotations,
as spontaneous.  The mad spontaneity of an \co{immediate} \co{subject} is just
the other side of the arbitrariness of the \co{object} emerging -- no matter
through how intricate meanders of transcendental constitution -- \la{ex nihilo},
not even in the \co{actuality} of \herenow, but in the ideal limit of pure and
timeless \co{immediacy}.  Consciousness -- whether Descartes' cogito, Kant's
\thi{I think}, Fichte's Ego, Husserl's transcendental consciousness or Sartre's
for-itself -- is \co{actual} through and through, is an \co{immediate},
instantaneous -- and by the same token, or rather {\em only in this sense},
spontaneous -- \co{act} of constitution of an \co{object}.  Spontaneity of this
\co{act} of endowing with form (in-forming?) is, as we just said, only another
side of the arbitrariness of the appearing matter -- both \co{aspects},
\co{dissociated} from the surrounding background of \co{experience}, find no
other justification beyond the positive connotations of the word
\wo{spontaneous} which, in fact, could equally well be replaced by
\wo{whimsical}.

\subsub{Transcendence}

\pa \co{Reflection that it is} experiences the transcendence of its \co{object}
primarily as its \co{externality}.  We could almost say that it is nothing else
than such \co{an experience} of \co{externality}, that \co{externality} is the
universal content of every \co{reflection} which merely notices \co{that it is}.
This is even the case with the spark which burns me. Although the \co{sign},
\co{object} and {reflex} coincide temporarily, the very localisation, the very
narrowing of the horizon of the event and reaction, amounts to
\co{externalisation} which becomes apparent in the moment I direct my
\co{reflective attention} to it. It is not me, it only affects me; it does not
involve me, only a part of my body. \co{Externalisation} does not have any
inherent connection to extension or space, only to the narrowing of the temporal
horizon to \co{a} minute, \co{immediate experience}. It is the eventually
abstract minuteness which constitutes the sense of being somehow foreign, not
quite mine, \co{external}. The aspect of \co{spatiality} enters this relation as
simultaneity, in that the \co{external object} is experienced simultaneously and
\thi{as} simultaneous with the \co{subject} of this very experience.

This simultaneity, however, has all the ambiguity of different levels. It is
indeed co-extensional with the \hoa, but it also harbours the \co{reflective
  act} which, narrowing this \co{horizon}, arrives \co{after} its contents. This
dissonance between the \co{immediacy} of the given \co{object} and its temporal
sliding in the \co{reflective after} is exactly the experience of
\co{externality} with the germinal element of non-\co{actuality}, the most
elementary form of \co{horizontal transcendence}.

\noo{
  Nevertheless, as we saw, \co{externality} involves also an element of
  \co{non-actuality} which gives such \co{objects} the character of
  \co{transcendence}.  Of course, \co{objects} of \co{immediate experiences} can
  also be spatial, in which case their \co{externality} is the \co{immediate}
  presence of their \co{non-actual} \thi{parts}, like the back-side of a
  building, given in protention and retention.
}

\pa\label{pa:madSpontaneity}
%%%%%%%???????
We have also pointed out the character of double \co{dissociation} involved in
\co{externality}, I:\refp{pa:doubleSep}. On the one hand, \co{externality} is an
\co{aspect} of consciousness encountering its \co{object} as distinct from
itself, which amounts to the same as self-consciousness, I:\ref{sub:selfAware}.
But the sharp distinctness of the \co{object} from consciousness is but another
side of its \co{dissociation} from the background, from its \co{experiential}
origin.  An \co{object}, imprisoned within the horizon of \co{immediacy}, has
been \co{dissociated} from everything else. Its independence, its being entirely
\thi{on its own} is, on the one hand, the apparently constitutive feature of its
\thi{substantiality} and, on the other hand, its fundamental lack, its abstract
unreality.  The strangeness of an \co{object} appearing in the \co{reflection
  that it is}, the strangeness of its being at all is, in fact, the strangeness
of its being here \thi{on its own}, of its being so strangely alone.  The
\co{virtual signification} makes itself thus particularly strongly felt along
with the \co{external objects} -- their \co{externality} is the ultimate
\co{trace} of \co{signification}.

This unreality, which is both a part of the \co{immediate experience} as such
and of the \co{reflective} attitude towards it, is the \co{vertical aspect} of
\co{transcendence} involved here.  An \co{object dissociated} from its
background and surroundings, a pure \thi{substance}, appears as a spontaneous
(arbitrary) fact \co{that it is} -- as intriguing as it is meaningless.  Not
only has it no reason to be so and not otherwise -- it has no reason to be at
all. \wo{{\em Why} is there something rather\ldots?}  This may give rise to the
existentialistic {nausea}; but it may also be grasped with a grateful
fascination or detached thankfulness.  However, these later cases are possible
because they already involve more than the pure \co{immediacy} of the given,
because they already witness to the \co{vertical} dimension of
\co{transcendence}.\ftnt{Although one might be tempted to apply here Hegelian
  schemata, there is no need for such. The situation itself is here abstracted
  from \co{experience} and, if it appears meaningless and unsatisfactory, it is
  only for a special kind of \co{reflective} attitude. The \co{vertical}
  dimension of \co{transcendence} is here only another level of experience, is
  \co{the experience} in the context of which this \co{immediacy} found place.
  It is not something towards which the Spirit is driven by the force of
  dialectical negation -- at best (or, rather, at worst) it is something towards
  which philosophical reflection is driven by whatever force drives it.}


\noo{ Doubt... -- goes to level-2 of mediacy; so does reference to Wittgenstein
  on certainty (the burden rests on those asking such questions...)
  
  \twot{other objects}{\co{complexes}, meaning (as `purpose', relation);
    `absolute separation' = meaninglessness; \co{that it is so-and-so}} }

\noo{???  \pa The \co{reflection that it is} is an intense, concentrated
  experience -- it breaks the undisturbed continuity of natural experience,
  pulling out of it something unnecessary and purposeless but, at the same time,
  strange and fascinating -- the fact \co{that it is}.  But this possibility of
  \co{attentive reflection} is \co{founded} on the already \co{reflective
    dissociation} of the contents of \co{experience} -- the \co{reflection that
    it is} only confirms it and brings it to the focus. \co{Reflection}, and
  \co{attentive reflection} in particular, encounters some \thi{substantial
    forms}, some \thi{given atoms} -- they are simply a constitutive \co{aspect}
  of \co{reflective act}. \co{Reflective experience} is, to some extent, built
  on such \thi{atoms}, namely, to the extent the \co{experience} is
  \co{reflected}.  But, as we well know, \co{reflection} does not capture the
  whole \co{experience}. Primarily, it cannot account for many \co{unities}
  which are \co{experienced}, but which are not encountered \co{reflectively}.
  The tension between \co{experienced unity} and \co{reflectively dissociated
    totality} is constitutive for the whole \co{reflective experience}. We have
  seen its first manifestation in the tension between the limits of
  \co{distinctions} (where the apparently negative character referred all the
  time to the ultimate \co{unity} of the \co{indistinct} background), as opposed
  to the \thi{positive entities} of \co{reflective attention} (whose apparently
  positive content witnessed only to the \co{reflective dissociation}). We will
  encounter {traces} of the same tension throughout the other levels of
  gradually wider and wider temporal scope.

}


%%%%%%%%%%%%%%%%%%%%%%%% ACT


\subsection{Actuality}\label{se:mediacy}\label{sec:levelB}
Let us now consider much more mundane cases of what, in the more common sense of
the word, would be called \wo{experiences}; situations which are not reduced to
a single moment but which are organised within a relatively short, yet not 
\co{immediate} time span; \co{experiences} with a finite and
limited but no longer extensionless temporal scope.

\subsub{Complexes} \pa\label{th:complact} Entering a room for the first time, I
acquire first a vague, general impression of its character. It may be ugly,
cosy, warm, cold, dark, pleasant\ldots After a few moments in the room, my
experience of it changes in that I now become more attentive to its various
aspects and parts. Looking around I notice the arrangement of furniture, the
colors, fracture of the walls etc. At the same time the room itself recedes in
the background, I no longer perceive the room but this window, this corner, this
armchair.  If, however, suddenly asked ``How do you like this room?'' I can at
once posit the whole room as the totality to be addressed. I can now analyze its
aspects and point out their interplay, but the object of my talk is the room. It
is both a unity and a multiplicity. If I try to see all of the room I will
easily fail, if I try to embrace simultaneously all the details I have learned
about it in one act, I can, at best, summarise them in a general impression.
Very often, if not always, the very same impression I had on entering the
room. But I know that this totality is composed of much more than this
impression, this \os\ which is all of the totality of the room that I can
\co{actualise}.

A room exemplifies what one typically considers to be objects of experience.
\co{Actual experiences} involve not just isolated \co{objects} but their
\co{complexes}, also in the sense that a single \co{object} may appear as a
\co{complex}, either a totality of some parts, or a unity involved in an
interplay of relations with other \co{objects}.  No such \co{complexes} are more
fundamental than others.  Before a child sees that a chair can be moved away
from the table, the two can be experienced as one \co{complex}; one, because
neither is yet definitely \co{dissociated} from the other, and \co{complex}
because itself internally differentiated.  A picture hanging on the wall is not
part of the wall, nor something on it -- there is neither a picture nor a wall
but a totality of one \co{complex}. Once we have learned that chairs and tables
come separately and that different pictures may be hang on the same wall, we
live with the immediate consciousness of these \co{complexes} whenever we
encounter them. But this is the end rather than the beginning of the story.
\thesisnonr{A variety of aspects
% does not appear as one \co{complex} because it, 
% somehow \wo{objectively}, is one \co{object}, but it
becomes one \co{complex} when
it is \co{cut out of experience} as both differentiated beyond the \co{actual}
givens and yet posited as one \co{totality} within the \hoa\ through 
a unifying, \co{actual sign}.}
%\ (cf.~\ref{th:hoa}).}

In this sense, \co{complexes} are like \co{objects} which emerge not due to any
metaphysical \thi{substantial form}, but due to their particular relation to the
\hoa.  The constitutive feature of \co{complexes} is that, while they are
\co{recognised} as \co{totalities}, sometimes even as \co{objects}, they present
themselves, so to speak, incompletely; although given in \co{an actual
  experience}, they are not fully \co{actual}, they are always \co{signs} which
carry with them more \co{distinctions} slipping out of the \hoa.\ftnt{To refer
  to protentions and retentions again -- one might argue that everything is a
  \co{complex}, that every experience, every perception involves this kind of
  sliding away out of the \hoa. But such an argument would be based on the
  idealization postulating an idealized scope of the \hoa, namely, an
  extensionless point. By its very nature, it cannot contain anything and so
  everything must slide away. Thus, it would have to reject the \co{immediate
    experience}. We do not reject it, though we claim neither that it is
  temporally extensionless nor that it has any privileged epistemological
  function -- it allows for further differentiation and \co{dissociation} into
  ever more minute idealities.}

\pa The word \wo{thing}, which was used for a correlate of an \co{immediate}
experience, should not mislead one to think exclusively of a thing.  Its
constitutive feature is that, appearing within the \co{immediate} limit of the
\hoa, its experience coincides with its \co{sign}. Its self-identity, its
\thi{substantiality} is a derivative of that.  In practice, a {thing} may be
almost anything. When I enter a new room, my first \co{immediate} experience is
not of furniture, walls, pictures etc. but of \thi{???} which then turns out to
be~\ldots this room. (At least, as long as there are no unusual objects which,
attracting my attention, would prevent me from seeing the room.) I do not look
attentively into each corner, contemplate the ceiling, the floor, in order to
finally conclude ``Yeah, it is this room''. At first, it is \thi{???}, that is,
the {thing} of my experience and it is, of course, no thing in the usual sense.
It becomes more like a usual thing when I realise that it is actually the room,
though then one may at once start asking the questions about unity and
\thi{substantiality} of an \co{object} versus \co{totality} of a \co{complex}.

The {thinghood} or \co{objecthood} of something is constituted by the
possibility of grasping it fully (that is, without experiencing that anything
was left outside) in one \co{act}.  The \co{complexes} may be,
\thi{objectively}, the same things. But in addition to being experienced as
\co{objects}, involving merely unity, their experience involves also
multiplicity, \co{totality} of aspects or elements.  The experience of
\co{complexes} involves therefore not only their \co{objective} unity but also
the experience of their being complex, even if not of their full complexity.

\subsub{The signs}\label{impressConcept}

The involved \co{signs} refer not only to the immediate
\thi{givens} but also to the {aspects} which, at the moment of experience, are
not present within the \hoa\ -- and are experienced as such (like all the parts
of the room which I know are there, which I have seen and registered, but which
are not \thi{given} in the same way as those I am \co{actually} looking at). The
differences between various \co{signs} of \co{complexes} concern primarily
whether they focus on the aspect of unity or multiplicity involved in a
\co{complex}.

\ad{Original signs}
%
I wake up and feel strength, a lot of vital energy, a joyful 
vigor. Is it the sunny morning which is the cause of that? Was it the 
light supper yesterday evening? Is it\ldots? It does not really matter. 
Hopefully, it will last the day long but it may also 
easily disappear very shortly. No matter what its reasons might be, the 
\co{mood} does reveal to me something, if not anything specific then, 
in any case, \thi{how I am now}. And this both reflects and will be 
reflected in my perception of the situation, for I will act 
differently (even if doing the same things, say, while getting up and preparing
breakfast) than I would if I woke up and felt fragile, feeble and low.

I leave the house, drive to the city, park my car and enter a cafe.  I am
enjoying the perspective of a quite hour over a cup of coffee and a good book.
The coffee I get is not exactly the best, but it is not enough to spoil the
\co{mood}.  Unexpectedly, I see a friend approaching my table. He asks how long
I have been here, how I came here, where I parked my car. At this moment I
realise that I locked my keys in the car. Oh sh...t! The \co{mood} of the
expected quietude disappears suddenly and I am getting upset. What makes me so?
Not the keys locked in my car because, in themselves, they are too little to
ruin a nice hour. It is the whole \co{complex} of the situation, the anticipated
trouble, perhaps, the money I have to pay, the spoiled hour at the cafe. The
simple fact of locked keys is certainly the focal point of the whole situation
but my being upset unveils much more than this simple fact, this single
\co{impression} unveils the significance of the simple fact which has been
placed in a broader, \co{complex} context of related facts and consequences.

Calm voice of my friend, reassuring me that it is no big problem, we just call
this and that number, wait outside smoking a cigarette and they will come and
open the car, helps a lot. One could say, he only rationalises away my
\co{impression}.  Indeed, but how? By bringing into the situation aspects,
points of view, possible solutions and, not least, his calm attitude, which all
together modify the \co{complex} and, consequently, my \co{mood}.

\pa The \oss\ of \co{actuality} are all kinds of such \co{moods} and
\co{impressions}.\ftnt{One might say that, for instance, \co{moods} are lasting
  \co{impressions} or draw even more specific distinctions. We will not, however,
   attempt to differentiate here any further.} They are direct and original
in the sense that they can be experienced without the respective \co{complex}
being \co{actually} given.  In fact, they often appear before the respective
\co{complexes}.  I can get a feeling of fear without knowing exactly what is
frightening me.  I can be in a bad mood without knowing exactly why.  In this
respect, \co{moods} are typically not accompanied by the respective
\co{complexes} at all, they merely announce~\ldots well, my general \co{mood}.
% \co{Impressions} will more often have the associated \co{complex}
% given along with them though \co{moods}, as lasting \co{impressions}, may too
% have recognisable origin.\ftnt{Multiple examples can be found, for
% instance, in M. Scheler (e.g. ???)  or W. James who treats them under
% the heading of \thi{conjunctive relations} (e.g., Essays in Radical
% Pragmatism:III.The Thing and Its Relations).}

The border between the \co{impression} and the \co{complex} it announces may be
very vague. The difference between a \thi{violent passion} and a \thi{passionate
  violence}, between an \thi{intense curiosity} and a 
\thi{curious intensity}, between a \thi{distasteful food} and the
\thi{disagreeable taste} is as discernible as it is negligible. 
%
In the evening we are sitting with some friends around a table in a pub having
an enjoyable conversation about nothing. After some time the neighbour who was
sitting on his own joins in. There is some intense curiosity in his eyes and as
if slight irritability in the way they search through the whole place. But he
seems to be doing quite well in joining and, in fact, modifying the
conversation. After a few questions and answers he focuses on something
particular one of us said and follows it up with more and more detailed
questions. \wo{So what did you really mean by that?}~\ldots Hmmm. \wo{Was it this or
that?  But then, you see, you would have to say that\ldots} His interest seems
a bit uneasy, perhaps, impolite and too detailed but so far there is nothing
directly wrong with it. And nothing wrong happens later on, either. After
leaving the place, all of us have the same \co{impression} of the guy with a
somewhat inquisitive attitude, as if afraid of unveiling his own meanings;
interrogative, perhaps not quite a Porfiry Petrovich but still a bit like a
detective.  It is impossible to say at which point 
this \co{impression} started to make itself felt. Was it when he started to ask
the questions? Was it when he joined our conversation? Even earlier?  Likewise,
it is impossible to say to what precisely this \co{impression} refers. We could
mention a lot of small examples, things he said, ways he looked but it is not
the mere sum of such minute particulars. Saying \wo{the detective} means much
more the \co{impression} he created than any of his \thi{objectively given}
features.  Of course, there is far from here to any judgment of the person, but
the \co{impression} has already painted a whole, even if incomplete, picture.
Referring to him, we will now say \wo{the detective}.

\pa%multiplicity through unity of an impression
We can thus list three characteristic features of the \oss\ at the current
level.  These \co{signs} announce the \co{complexes} lending them their unity,
they comprise a \co{totality} of a \co{complex}, a situation or an object, into
a unity of one \co{sign}, the distinctive quality of the \co{mood} or
\co{impression}.  An \co{impression}, \citet{an emotion is always a simple
  predicate substituted by an operation of the mind for a highly complicated
  predicate.}{PierceFourIncap}{III;p.58\label{ftnt:simpleComplex}} This is the fundamental role of \oss\ 
of \co{complexes}, and also what distinguishes them from their \rss.

Another common feature of all the above examples is that the given \co{mood}
allows a certain variation of more minute impressions, of perceptions and
\co{immediate} sensations.  More generally, one can experience the same
\co{mood} in different situations. The \oss\ of \co{actuality} may be as if
incarnated in a variety of lesser forms. \citet{Every one knows how when a
  painful thing has to be undergone in the near future, the vague feeling that
  it is impending penetrates all our thought with uneasiness and subtly vitiates
  our mood even when it does not control our attention; it keeps us from being
  at rest, at home in the given present.}{Pragm}{I;p.13} A more pervasive
\co{mood}, like that induced by the pending expectation, may allow for
modifications of more minute \co{moods} and, in particular, for a large
variation in the sensations, perceptions and \co{immediate signs}.

%-- not directly reactive = unclear object/cause
Finally, we also notice the less reactive character of \co{impressions} as
compared to \co{immediate signs}.  \co{Impressions} do not constitute
\co{complexes} -- they are both \co{aspects} of \co{actuality} and the latter
may also appear without associated \co{impressions}. But any \co{impression}
indicates a \co{complex}, a \co{cut} from the \co{experience} which, through the
\co{impression}, receives a unified \co{sign}. This \co{complex} may be, in
fact, close to indistinguishable from the very \co{impression} announcing it.
Thus, \co{complexes} have a very wide meaning: a table, a room, a situation, two
weeks in Prague, inquisitiveness of a person, all are examples of
\co{complexes}, of things which are differentiated into aspects and various
sides, but which, nevertheless, appear as \co{totalities}, as focal points of
all the involved differences. The interplay of these differences is the
\co{mood} or, the \co{mood} is what gathers the interplay of {\em just these}
differences.


\ad{Reflective signs}\label{pa:rssA}
%
\co{Moods} and \co{impressions} give the \co{complex} a unified \co{sign}
through which the \co{complex} appears within the \hoa.  The corresponding
\co{reflection} will no longer stop at the mere observation \co{that it is} but will now
observe \co{that it is so-and-so}; it will go beyond the \co{attentive} positing
of something that merely is, and surround it with all kinds of accidents,
predicates, properties. \co{Reflection} differentiates the unity
of a \co{complex}, the unity of an \co{impression} into multiplicity of its
aspects and elements.  Eventually, \co{reflection} creates lists, lists of
properties, aspects, features, and then tries to reconstruct the totality of the
\co{complex} out of these scattered parts. We will call these \rss\ of
\co{complexes} \wo{\co{concepts}} and \wo{\co{thoughts}} -- \co{totalities} organised
around a unity which, however, for the \co{attentive reflection} remains
typically only a mere \co{sign}, an ideal limit.\ftnt{Again, we are not
  interested in any (im)possibly detailed analyses of thought processes and
  their dependence on the available and unavailable concepts. We allow ourselves
to conflate the thought and the concept, just as earlier we conflated the
\co{distinction} understood as the act of distinguishing and as the
distinguished content.}

These words have here a broad meaning. There is nothing particular in a 
\co{concept} except that its unity gathers a collection of \co{actually}
discernible, more or less \co{precise} characteristics. 
%
\noo{The mental elements involved in perception, especially, the elements which are
\thi{filled in} into the incompletely perceived object, will often be
\co{thoughts} in this sense.}
%
Discussing the detective or the locked keys, one will collect a whole series of
thoughts which make up the important aspects of the situations.  Eventually, the
whole \co{complex} may become a single \co{thought}, although this will merely
mean establishing a simple \co{sign}, like ``the detective'' which now, in a
truly artificial fashion, signifies the respective \co{complex}.  There may be a
difference between \thi{the detective} taken as such a \co{concept} or as an
\co{impression}, but then it concerns the psychology but not the content.  For
even the artificiality and \co{vagueness} of the \co{concept} \thi{the
  detective} has most serious fundament in the experienced content which
converges at the unity of the original \co{impression}.  In a sense, to be a
\co{concept}, a \co{sign} has to be \thi{primitive} -- it must draw a set of
\co{distinctions} as a unity, no matter how coherently or incoherently, how
logically or illogically they are related and how explicitly or implicitly this
unity appears.\ftnt{This unity, which lacks any {\em conceptual} unity, together
  with the fact that its \thi{content} is the respective experiential limit of
  \co{distinctions}, are the reasons why almost no \co{concept} is definable.
  For instance, \citef{\thi{doorknob} is primitive (unstructured); and, for that
    matter, so is too practically everything else.  [every other
    concept]}{Fodor}{VII;p.147} Yet, this primitive undefinability concerns only
  the unity, the line where the limit of \co{distinctions} is drawn; it does not
  mean that various concepts are completely independent from each other, that
  \wo{[s]atisfying the metaphysically necessary conditions for having one
    concept {\em never} requires satisfying the metaphysically necessary
    conditions for having any other concept.}\kilde{p.14} If one claims that
  \wo{a mind that has only one concept (say, \thi{doorknob}) is a metaphysical
    possibility}, then one obviously means by \wo{metaphysical possibility}
  merely the lack of logical impossibility. Such a claim is only yet another
  example of empirical atomism whose epistemic flavour quickly pollutes also
  ontology.  }

\pa Just as an \co{impression} \wo{is always a simple [\co{sign}] substituted by
  an operation of the mind for a highly complicated
  predicate}$^{\ref{ftnt:simpleComplex}}$, so a 
\co{concept} substitutes a \thi{complicated predicate} arrived at by analysis
for the simple \co{sign}.\ftnt{The Latin \la{conceptus} reflects well the
  tension between the unity of the origin/embryo (retained in the English
  \wo{conception}) and the multiplicity which it gathers and stores as in a
  container.}  Thus, 
psychologically, the two are incommensurate: at any \co{actual} moment, it is
either \co{concept} or an \co{impression} but never both. If, involved in a
situation, one tries to observe the arising \co{impressions}, they will get
polluted and falsified; \co{I} can not \co{reflectively} catch, not to mention,
control \co{myself} in the moment of getting an \co{impression}. Trying
something like that, dissolves all \co{actual impressions}, makes them withdraw
behind the imposed \co{reflective} contents. Likewise, proceeding with a
conceptual analysis, \co{I} can not pay attention to the \co{impressions} which,
possibly, accompany it in the background.  A \co{concept} is an
\co{externalised} intuition, an \co{impression} \co{externalised} as a list of
its most minutely discernible aspects. The two may appear in any order and the
transition between them is rather smooth -- the moment when a \co{thought}, or a
series of \co{thoughts} becomes an \co{impression} or vice versa is often
imperceptible.

Yet, although the two can not be \co{reflectively} posited and observed
simultaneously, they can always go together.  It is not only that, sometimes, one
  can talk oneself out of an \co{impression}, not only that a unified \co{impression}
already contains the possible results of decomposing it into a series of
\co{thoughts}. Every \rs, every \co{thought} 
{\em has} a \co{mood}, it creates an associated \co{impression} which reflects
the unity posited by the \co{thought}. (Of course, receptivity to and above all the
significance one attaches to such
\co{moods} may vary tremendously.) \citet{Every concept in our conscious
  mind, in short, has its own psychic associations. While such associations may
  vary in intensity (according to the relative importance of the concept to our
  whole personality, or according to the other ideas and even complexes to which
  it is associated in our unconscious), they are capable of changing the
  `normal' character of that concept.\noo{It may even become something quite
    different as it drifts below the level of consciousness.}}{MHSJung}{p.29}
%In fact, it need not drift anywhere
Every system of {thought} has a \co{mood}, every philosophy has, besides its
more or less intricate hierarchy and system of \co{concepts} and ideas, a
general \co{mood} which hangs like a cloud above and flavours its more specific
aspects.  And sure, just like my understanding of the situation with the locked
car keys influences my \co{impression} of it, so will one's understanding of a
philosophical system influence its \co{mood}, the shape and the density of the
cloud.\ftnt{This \co{mood} is like the \co{sign} of typical experience
  underlying the given philosophy (provided that it has character -- for as
  Nietzsche says: only \citef{[i]f one has character one also has one's typical
    experience which recurs again and again.}{BeyondGE}{70})
  Say, for example, the {mood} of Heidegger: gnostic thirst for the
  hidden truth and resentment over its absence in the lower world and among
  fellow men; the mood of Nietzsche: unrewarded intensity turning into violent
  despair, heroic scream lost in the darkness; the mood of Spinoza: he who is
  satisfied with oneself need not judge others, but deep satisfaction with
  oneself is not from this world; the mood of Wittgenstein was accurately
  described as empirical mysticism or, perhaps, mystification of empiricism; etc., etc.
  Finding such superficial and general characteristics insufficient does not
  change the fact that one,
  nevertheless, \co{recognises} their origins, one understands the \co{mood}.}

\newp 

\ad{Essences}\label{pa:essences}
As being, the fact \co{that it is}, has been delegated to the limit of an
extensionless and incomprehensible simplicity of a point,
\refp{pa:pointUnknown}, there must remain some positive content which can be
assigned to such contentless points, there must remain some \co{distinctions}
which could justify \co{distinguishing} one thing from another.  There is also
the problem with the exponential growth of the number of the
negative, exclusive \co{distinctions} which can be restricted to a smaller
number of inclusive, positive determinations, \refp{pa:pragmaticPositive}.
These two, only apparently quite unrelated, problems of coping with the ideal
points which reject to cease being multidimensional \co{complexes},  are
handled by \co{dissociating} being from its character, the existential from the
intelligible, the mere fact of being, \la{esse}, from that which makes a being
what it is, \la{quo est}, i.e., the forms or \thi{essences} of beings.\ftnt{Although
  the distinction has been present before, it comes clearly forth in the
  existential focus of Aquinas, who builds much of his philosophy on the
  refinement of the Boethian distinction between \la{quod est} and  \la{quo est} by
  adding to it the third term, \la{esse}, accounting for the observations like that
  \citef{I can understand what a man is or what a phoenix is and nevertheless
    not know whether either has existence in reality. Therefore, it is clear
    that existence is something other than the essence or \la{quiddity}, unless
    perhaps there is something whose \la{quiddity} is its very own existence,
    and this thing must be one and primary.}{BeingEssence}{IV} Let us notice
  only in passing how the arguments of such a form conform to the principle of
  realism-empiricism from I:\ref{sub:givens}, which postulates
  dissociated existence of everything distinguishable.}


\noo{ \citet{The essence of each thing is what it is said to be propter se. For
    being you is not being musical, since you are not by your very nature
    musical. What, then, you are by your very nature is your
    essence.}{AristMeta}{VII:4}
  
  \citetib{For the essence is precisely what something is; but when an attribute
    is asserted of a subject other than itself, the complex is not precisely
    what some 'this' is, e.g. white man is not precisely what some 'this' is,
    since thisness belongs only to substances.  [...]  there is an essence only
    of those things whose formula is a definition. But we have a definition not
    where we have a word and a formula identical in meaning (for in that case
    all formulae or sets of words would be definitions; for there will be some
    name for any set of words whatever, so that even the Iliad will be a
    definition), but where there is a formula of something primary; and primary
    things are those which do not imply the predication of one element in them
    of another element. Nothing, then, which is not a species of a genus will
    have an essence-only species will have it}{AristMeta}{VII:4}
  
  \citetib{essence will belong, just as 'what a thing is' does, primarily and in
    the simple sense to substance, and in a secondary way to the other
    categories also,-- not essence in the simple sense, but the essence of a
    quality or of a quantity. [...]  definition and essence in the primary and
    simple sense belong to substances.}{AristMeta}{VII:4}
  
  \citet{definition is the formula of the essence, and essence belongs to
    substances either alone or chiefly and primarily and in the unqualified
    sense.}{AristMeta}{VII:5}
  
  \citet{Each thing itself, then, and its essence are one and the
    same}{AristMeta}{VII:6} }

\citet{The essence of each thing is what it is said to be \gre{propter se} [...]
  the essence is precisely what something is.}{AristMeta}{VII:4} The essence
captures the~\ldots essential and it does it in a unique and \co{precise} way,
it captures the ultimate truth about what it means to be this thing:
\citetib{Each thing itself, then, and its essence are one and the
  same.}{AristMeta}{VII:6} We certainly won't spend time on studying the
modulations of the idea of essence and its detailed relations to substances,
forms, common natures, complete notions and other themes which surrounded the
idea in its long history. We notice only its role: \citetib{there is an essence
  only of those things whose formula is a definition.  [...] definition is the
  formula of the essence, and essence belongs to substances either alone or
  chiefly and primarily and in the unqualified sense.}{AristMeta}{VII:4;5}
Moreover, the \citet{attributes attaching essentially to their subjects attach
  {\em necessarily} to them.}{PostAnal}{I:6} \thi{Essences} are to make the
ideal of necessary truths and knowledge, of the \co{immediate} givenness,
possible.\noo{The epistemological optimism need not be a negative feature, but
  it may go too far. One may certainly interpret both \thi{essence} and
  \thi{definition} in a somewhat ontological fashion, but one can not forget
  that these are conceptual elements of the system. The identification of
  essence with definition achieves the goal of capturing substances, and hence
  all entities, in understanding. The \thi{essence} belongs to the order of
  understanding, it is that which can be understood from a being.  But it is
  equally that which makes this being be what it is, it constitutes its
  ontological place in the reality.} The goal, if not the achievement, of
\thi{essences} is to reduce a \co{complex} to something which can be adequately
given in the purely \co{actual}, ideally, \co{immediate} intuition.  An
\thi{essence}, whether with Aristotle, Descartes or Husserl, is a graspable unit
which can be made \co{actual} in a single \co{act} of intuition, comprehension,
perception, understanding, or whatever variation of a \co{reflective act} one
wanted to utilize.  The supposed \thi{essences} were to give metaphysical
rigidity and \co{precision} to the \co{distinctions} of \co{experience}, were to
turn \co{distinctions} into \co{precise}, rigid distinctions.  And
\co{precision} is but another word for \co{immediate} \thi{givenness}. An
\thi{essence} is the \co{reflective} hope of \co{immediacy}.\ftnt{\thi{Essences}
  correspond only to the intelligible \thi{substances} and, in fact and
  eventually, to the intelligibility of \thi{substances}. (One should probably
  keep in mind that the original Greek \gre{ousia} can be, and in various
  contexts is, translated as either substance or essence; Aristotelian
  \gre{hypokeimenon}, \thi{that which underlies}, is translated either as
  {substance} or as {subject}.) Their role, and ambiguity, lies in the narrowing
  of ontology to the horizon of epistemology, that is, of \co{actuality}, in the
  attempt to make the two coincide.}
\noo{ Providing thus a basis for the sphere of intelligibility, they are
  intimately related to sufficient reasons, efficient causes and all the other
  means of the attempted reduction to \co{actuality}, coverage of the possible
  differentiations by final, closed definitions. The reduction amounts to the
  faith that if everything consists of \thi{essences} and their interplay, of
  positively definite and closed entities interacting which each others, then
  there is little reason to suspect that these interactions are of entirely
  different character and can not be exhausted by an analogous modeling. On the
  other hand, when the \thi{interacting} entities do not have final definitions
  and, having lost their \thi{essence}, can always unveil a new and unexpected
  side, then also the conditions influencing their \thi{interactions} get
  loosened, and one can hardly guarantee the effects expected of the sufficient
  and efficient reasons.  }

\newp

\pa\label{pa:understand=limit} However, to understand something is not to grasp
its supposed \thi{essence} but to know its limits. It can amount to knowing {\em
  what} \co{precisely} separates this thing from others but usually it suffices
to know {\em where} the boundary between this thing and its surrounding
goes.\noo{An understanding, a \thi{rational} statement or attitude is one which
  is not \thi{carried away}, which knows the limits of its validity or, at
  least, knows that such limits exist.}  The difference is the same as between
the assumed positive contents and the negative limits of \co{distinctions}
discussed in section \ref{sub:substances}, in particular,
\refp{pa:pragmaticPositive}.  To understand something, one has to start with
this \thi{something} which is already given, \co{recognised}. (As Scholastics
would say, an \citet{act of judgment in reference to a \la{complexum}
  [proposition] presupposes an act of apprehending the same
  proposition.}{OckOrd}{Prologue, q.1 \kilde{p.19}}) The initial moment of
understanding is a \co{reflection} which isolates this \thi{something} as an
\co{actual object}.  Often, we may at first not even know what it is that we are
trying to understand -- something happened, but for the moment it remains a mere
\thi{something}, because we do not know what it was, whence it came nor what it
meant.  Only an \co{original sign} has appeared, which, in spite of, or rather
precisely {\em through} its \co{vagueness}, calls for a closer attention.  Then,
and only then, the questions like ``What is this thing?'', ``What makes it what
it is?'' can be asked. They ask for an explication of the involved \co{cut}, of
the boundaries separating something already \co{recognised} from the \co{rest}.
The answer involves further \co{reflection}, which now attempts to isolate these
boundaries from the totality of the \co{complex}.  Thus explication
\co{dissociates} further, it \co{dissociates} the boundaries from the
\co{complex} they bind, \thi{essence} from \thi{substance}, in search for the
eventual definition, the rigid limit of the involved \co{distinctions}.

\pa\label{pa:understand-context} But explication has no \thi{given} limit.
Although in a given context, say of making a table, it may be natural not to ask
questions about the atomic structure of the wood, or about consequences of the
act for the world peace or future generations, such or similar questions are
always possible and may bring in \co{distinctions} which were not at all
involved in the original situation. Every  \co{distinction} and limit thereof
which has not been drawn to the ideal limit of a point can be refined which
often means: considered in another context.
%
Trying to understand the \co{concept} of a group, I do start with its
definition, but the process involves necessarily relating it to other concepts,
seeing what happens if I drop some axioms, if I add some, analysing possible
examples and non-examples, relating it to a monoid, perhaps to a ring, etc.
\noo{ This is different from eidetic variation because, as a matter of fact, I
  am not looking for any \thi{essence}, any residuum of the thing. I have the
  thing given for me -- it is this \thi{something} I started with! In the
  present example, I have the definition and to understand it I have to figure
  out when it can be applied and how it differs from other definitions, that is,
  where its limits go.  } Understanding is relative to the context in which the
thing is considered and this context influences the boundaries of the thing. If
the only other concept I have is that of a monoid, my understanding of group may
be very poor. (Yet, it will be understanding! A group is distinguished from a
monoid.) If I am able to relate the group to a large variety of other concepts,
the understanding will be respectively deeper -- group will be distinguished
from more things/concepts with which it otherwise might be confused.

In principle, one should also take into account something which happens entirely
implicitly, namely, that some contexts are completely irrelevant. One won't try
to relate the mathematical groups to cows, last weekend's trip, one's mother.
Although implicit, this is equally essential aspect of understanding since it,
too, tells something about the limits of the thing. A great challenge of
teaching, for example, consists in being able to explicitly delimit the object
under consideration against the horizon of completely (or only partly) unrelated
issues which, however, intrude typically on the apprehension of the object. Poor
understanding will often violate exactly such implicit limits which for a more
advanced understanding are simply not worth mentioning. Many jokes work simply
by violating the assumptions of such implicit contexts and dissolving the stiff
form of habit and rule in the flow of the possible, even if unexpected,
associations. (Inflexibility of character, whether personal or national, which
feels threatened as soon as some rule is violated, is the same as the lack of
wit and humor.) Laughter brings such implicitly and often dogmatically assumed
elements to the front and in this consists its often powerful and destructive
function -- it abolishes the \thi{positive} and determinate character of the
\thi{given}. For, as we observed, the function of \thi{positive} determinations
assigned to various limits of \co{distinctions} is exactly to exclude the
possible which is irrelevant and disturbing.  But this working of laughter is
negative and threatening only to stiffened and \thi{frozen} forms, to dogmatic
prejudices, for which it is a painful reminder of their relativity, of the
\co{experienced} flow from which they had emerged.

\pa Rigidity of some limits of \co{distinctions} is only ideal and, at best,
secondary.  Do you understand what a \thi{wave} is?  Whenever at the seashore,
you can \co{recognise} waves or else see that there are none.  You know
\thi{what makes a wave}.  But, do you?  What is it?  How high must it be in
order to be a \thi{wave}?  And where does one \thi{wave} end and another begins?
It is impossible to say because there are no definite, rigid limits.  There are
only \co{vague cuts} which tell \wo{this is a wave and this is another}.  You
may be unable to tell exactly where one wave ends and another begins, but still
you are perfectly able to select some area, perhaps, around the peak, and say
that this marks a wave. We do not have an {\em explicit} understanding of a
\thi{wave} because we are unable to isolate \co{reflectively} its limits.  But
we know that it is not a whirlpool, nor a current, nor a building.  We are able
to delimit it against other things and these \co{distinctions}, whether
verbalized or not, make the \thi{wave} \co{concept} sufficiently
familiar.\ftnt{Leibniz would say that we have a \co{clear} but \thi{confused}
  understanding of a \thi{wave}.
\label{ftnt:clear}  
  {A clarification may be in place since the words will occur quite often.
    \co{Clear} is not opposite of \co{vague} but of \co{precise}.  \co{Clear}
    can be conjoined with \co{vague} in the way entirely analogical to that in
    which Leibniz associates \co{clear} and \thi{confused} knowledge.
%There are, of course, levels of \co{clarity} and Leibniz' example is quite
%appropriate.
\citef{When I recognise one thing among others without
being able to say what its differences or properties consist in, my
knowledge is \thi{confused}.  In this way we sometimes know
\thi{clearly}, without being in any doubt, whether a poem or a
painting is good or bad, because there is a certain \fre{je ne sais
quoi} which pleases or offends us.}{LeibnizDM}{24.} 
%
The insistence of this \noo{Corneillean}  \fre{je ne sais quoi} has
been obviously destined to eternal recurrence.
%
\citef{It is possible to \thi{understand} 
something, deeply, intimately, without \thi{grasping} it rationally, 
for instance, music.}{Heilige}{ XVIII}
% I am not here after the distinction between aesthetic feelings and rational
% understanding, though it might probably provide a simple example.
\wo{\co{Vague}} corresponds to Leibniz' \wo{confused}, that is, \wo{indistinct}.
The phrase \la{je ne sais quoi} (\thi{I do not know what}, \thi{a certain
  something}) expresses quite accurately the \co{rest}, the impossibility of
grasping this \thi{certain something} within the \hoa.  \co{Precision} amounts
to excluding, cutting off (\la{praescindere}) all that threatens with slipping
out of \co{immediate} control. This exclusion is what is \co{precise} about, for
instance, a \co{concept}.  It can be taken as an opposite of \co{clarity} -- it
increases with the narrowing of attention towards the limit of \co{immediate
  actuality}, which diminishes the latter.  Leibniz also associates
\thi{confused} with the large number of aspects which, so to speak, make it
impossible to comprehend them distinctly in one act.  \citef{Our confused
  feelings are the result of a variety of perceptions which is indeed infinite
  -- very like the confused murmur a person hears when approaching the
  sea-shore, which comes from the putting together of the reverberations of
  innumerable waves.}{LeibnizDM}{33} He does not use the word \wo{clear} here which
might strike one as inappropriate, but this is exactly the word we might use.
Thus, although not synonymous, \wo{\co{clear}} is close to co-extensional with
Leibniz' \wo{confused} or our \wo{\co{vague}}.}}
%
\thi{Wave} is a much more fundamental example than a \thi{group} or a
\thi{chair}, and understanding the former illustrates much better the process of
understanding which always originates in what some call \wo{tacit knowledge} and
which we call \wo{\co{recognitions} of \co{experience}}.  An \co{object} is a
limit of \co{distinctions}, of \co{cuts} from the background of \co{experience}
and thus \co{an experience} of something involves obviously its tacit
understanding.  Explication of this understanding is but a further
\co{reflection} which tries to isolate the limits of a thing.  Sometimes, such
an isolation may be carried very far and sometimes it can not.

\noo{The supposed \thi{essences} usually try to reduce a 
thing to such \thi{internal} characteristics, try to capture the 
thing -- a \thi{substance} -- in its supposed independence from the 
rest. Understanding the working of a car involves, indeed, the same 
kind of relations and delineations of various involved processes 
against alternative ones which would not make it work. But if we reduce a 
car to its working principles, then we have not really got very far, 
for how are these principles different from what makes a motorcycle, a 
triped, a bus work? 

If we relate explanation to thinking in terms of internal,
\thi{essential} characteristics of separate objects, then we could say
that explanation is not only distinct from understanding but is always
conditioned by the latter -- to analyze the \thi{inside} of a thing, I
have to first \co{distinguish} this thing.  Explanation is a reduction
to the internal \co{distinctions} considered as constitutive,
\thi{essential} for the object of interest.  And this distinction
between explanation and understanding would not apply merely to
natural sciences as opposed to humanistic ones but would go across
most, if not all, particular sciences.
}

%\subsubnonr{give up essences}
%and no eventual limits...
\pa
One might perhaps say that the \thi{essence} of a thing is the totality of what 
\co{distinguishes} it from other things. 
Perhaps to grasp the supposed \thi{essence} is the same as
to know the thing's limits.
Perhaps, though then we have changed a bit the traditional sense of the word
\wo{essence}. In particular, we must 
admit that no such final \thi{essences} obtain since, as said in 
\refp{pa:understand-context}, what distinguishes a thing from others 
depends on what others we take into account. And the field of references 
is inexhaustible.

If one had managed to display at least one convincing \thi{essence}, one might,
perhaps, also manage to arrive at some acceptable concept of a concept. No such
thing seems to be available but, fortunately, cognitive scientists took over the
quarrels. Since terminating the \co{distinctions} at {\em some} limit is
unavoidable, somebody will always think worthwhile to ask what the \thi{essence}
of, say, a chair, might be. Typically it has four legs, but it may have only
one, or none but rockers instead. One may try to define it functionally, as
``something to sit on'', but then anything I can sit on becomes a chair, for
instance a table. One need not deny sincere ingenuity of many attempts to
specify the most purposeful way of defining things. But one should not confuse
the normative, whether forensic or only administrative, character of such
endeavors with any ontological foundation, not to mention existential relevance.
%(It seems that they fell from favour with cognitive scientists, too.) 

\citet{The time has arrived to give up the myths of induction and
  \ger{Wesenschau}, which are carried over, as some point of honour, from
  generation to generation. For it is clear that even Husserl himself never
  gained \ger{Wesenschau} so that he would not have to re-consider and re-work
  it again, not to disqualify it but to force it to express something that it
  originally did not express at all [\ldots]}{VisInvis}{ Inquiry and intuition;p.122} The fact, which should have been disturbing, that nobody ever managed
to demonstrate unchangeable \thi{essence} of anything, except, perhaps, for
normative definitions, in particular in mathematics, reflects the secondary
character of \co{concepts} as compared to the \oss\ and, above all, the merely
auxiliary character of positive, \thi{essential} determinations.  It reflects
also the accidental character of most \co{totalities} -- no \co{complex} is more
fundamental than others and there is no necessity for a \co{complex} to be so
rather than otherwise -- primordially, it is but a \co{cut} from
\co{experience}. The boundaries between things are, more often than not,
\co{vague} enough to defy \co{precise} definition but also \co{clear} enough to
admit understanding.


\pa But, one might worry, have we not gone a bit too far?  Does not a group, in
its strict mathematical sense, have a well defined essence?  Well, this kind of
mathematical analogies have rather bad history in philosophy, so one better be
careful with them.  But OK, it is a well defined \co{concept} with unambiguous
definition.  What does it show?  That if we start with \co{concepts}, some
\thi{primitive} and precisely given ones (here, of a binary operation,
associativity, inverse, etc.), we can arrive at new ones.  This can 
certainly extend far beyond mathematics.  Have you never met an intellectual who
is not able to relate to the world otherwise than through the \co{concepts} and
definitions he has acquired?  His knowledge may be impressive but still this
omniscience is as pitiable as his apparent omnipotence.  The power of
conceptualisation, like of \co{reflection} in general, concerns only the
\co{actuality} of \co{signs}. Its domain is exclusively the incessant
\co{reflection} over things modeled as \co{objects}, as independent
\thi{substances} -- reduced to their \thi{intelligible essences}.  \citet{Since
  I have never been able to understand what the \thi{essence} of a concept is, I
  must be excused from discussing this point any longer.}{Fruth}{II:18} But this
does nothing to the \co{concepts} themselves, because we still can give partial, more or
less adequate, descriptions of \co{experienced} things, classes of things, kinds
of things -- the \co{cuts} in the \co{experience} are there as they have been
all the time, and so are the \co{signs} with their power to bring such \co{cuts}
into the \hoa.


\subsubi{Concepts vs. impressions}
\wtsep{understand -- limit}
\pa Understanding a \co{concept} is, psychologically, very different from
being in a \co{mood}, or having an \co{impression}.  But it is only a
psychological difference. \co{Moods}, \co{concepts}, \co{impressions} -- all are
modes of understanding, for to 
understand something is to know its limits.  We hardly ever know these limits
\co{precisely}, but that granted, understanding \thi{Prague} is to know, more or
less, where it ends and ceases to be Prague, or else, what makes Prague
different from Paris, just like feeling {irritation} is not to feel tranquility or
pleasure. More abstractly, it is to know, more or less, where \thi{irritation} 
ceases (or begins) to be one. Likewise, to understand the mathematical concept
of a group is to be able to recognise what is a group and what is not, is
to be able to distinguish it from a monoid and from a field. In either case, it
is \co{experience}, the texture of the wider context of \co{distinctions}
 which determines the adequacy, determinacy and, above all,
understanding of the \co{concept} or, generally, of the \co{sign}. I understand the 
\co{sign} $X$ when I know what  
it refers to, yes, but what it refers to is not (primarily) any positive
\thi{essence} but simply the limits of its application.  It is not a question
about properties of any subjects and accidents of any substances but, simply,
about what distinguishes the \co{complex} $X$ from others, where $X$ begins (and
ceases) to be itself. 

\wtsep{unity -- multiplicity}

As we observed in I:\refpf{pas:signsMean}, \co{signs} denote \co{distinctions}
and not only \co{objective} entities in the physical world.  \co{Moods} and
\co{impressions} introduce limits of \co{distinctions} within the \hoa, the
\co{complexes} they address have the same scope, in fact, viewed as
\co{externalised} contents, may be the same things as those addressed by
\co{concepts} and \co{thoughts}.  The difference concerns the character of these
\co{signs} -- \co{concepts} point towards the structural multiplicity and
\co{moods} towards the unity of the \co{complexes}. \co{Concepts} dissolve the
\co{non-actual} aspects into a multiplicity of \co{actual} determinations,
\co{impressions} gather the \co{totality} of a \co{complex} into a one, unified
\co{sign} which cannot be dissolved into components without changing the
\co{impression}.  This difference, often viewed as a tension between the two
kinds of \co{signs}, is merely a difference of emphasis and does not affect the
fact that both have essentially the same function of signifying \co{complexes}.


\wtsep{subjective -- objective}
\pa\label{pa:moodCommunicate}
\co{Moods} seem \thi{subjective}, only privately mine, while \co{concepts} are
claimed to be \thi{objective} or, at least, public and shared. As most claims
within this dualism, this one means close to nothing and sounds as plausible, as
it is misleading.

The \co{reflectively dissociated} elements, the minute distinctions which can be
pointed to and even captured by \co{actual} definitions are most easily
communicable. And \co{concepts} are built upon such \co{distinctions}; they are
designed not to be public or communicable, but to be {\em easily} communicable.
\thi{Subjective} \co{impressions}, on the other hand, capture the unity of a
situation in a way which is solidly anchored in the background of the person
experiencing it -- they \co{reflect} the situation as much as the person.  (A
person with an obsessive fear of revealing his privacy might experience our
detective as a persecutor, while one with a purely social interests as a
bothering snob.)  But they are not therefore private and incommunicable.  (After
all, we {\em all} got the \co{impression} of a detective.)  They announce the
(limits of) \co{distinctions} made in the \co{experience}, and this eventually
means, in the \co{one}.  As such they are accessible, at least in principle, to
everybody.\ftnt{\citef{Human emotions are to a large extent socially
    objectified, and not subjective. Only a part of emotional life is subjective
    and individual. Human thinking can be very subjective and often is so;
    thinking can be more individual than emotions, it depends less on the
    social objectification [...] Currently, one admits more and more often that
    there exist emotional apprehension [\ger{Erkenntnis}]. It was claimed by
    Pascal, Scheler, and was emphasized by Keyserling.}{Bier}{I:1\kilde{p.13}}}
With respect to feelings and \co{moods}, one speaks perhaps about \thi{empathy}
or \thi{sympathy}. Although many people have limited capacities in this respect,
this may be because feeling {\em exactly the same} as another person, is a
rather futile attempt to overcome one's sense of \co{alienated} subjectivism.
But if one is unable to \co{recognise} what another person is feeling, and one
is unable to do so {\em in principle}, then one should have one's psychology and
emotional constitution looked at.  All people have, at least in principle, the
capacity to face the same situation, to address the same (limits of)
\co{distinctions}. What matters is that one relates to (roughly) the same
\co{totality} of a situation, which involves also the reactions and feelings of
other persons, and \co{sharing} another's feelings is, at bottom, no different
from \co{sharing} his situation and problems. Whether one does it using one's
emotional intelligence rather than distanced reflection is of no significance as
long as the results are (roughly) of the same character. It is only by reducing
feelings and \co{moods} to their affective \co{aspect}, to the mere fact of
\thi{affecting me here and now} and, on the other hand, imagining \co{concepts}
as some eternal \thi{essences}, that distribution of the \co{dissociated} labels
\thi{subjective}, respectively, \thi{objective} may have some appeal. But
statements like \wo{It is sunny} or \wo{Two plus two is four} communicate
exactly as much as \wo{I am feeling depressed} or \wo{I had a bad day}. They
communicate the meaning which, when understood, is \co{shared} by the
participants of the communication. In fact, the communicated statements are
understood only because something is already \co{shared}, and the actual
utterances only \co{actualise} something of the \co{shared} horizon. The
meaning, of course, is different in each (pair) of the cases, but this does not
impair the possibility of communicating and \co{sharing} the \co{moods} just as
one can {share} the \co{objective} contents of observation statements.\ftnt{We
  will say more about \co{sharing} in
  III:\ref{sub:examples}.\refpf{pa:communion}; here we only wanted to signal the
  fact that assumed \thi{subjectivity} of feelings and \co{moods}, relating
  eventually, as everything else, to the \co{indistinct} \co{one}, is not so
  totally \thi{subjective} after all.}

Having different feelings in a given situation is just like having different
understanding of this situation. Incapacity of some persons to \co{share} the
feelings of others witnesses to the \thi{subjectivity} of feelings as much as
the incapacity of some persons to understand the \co{concepts} understood by
others witnesses to the \thi{subjectivity} of \co{concepts}.  There is also this
final characteristic of understanding in general, and \co{conceptual}
understanding in particular: it amounts to knowing the limits. As we said, not
in the sense of knowing \co{precisely} where the border between $X$ and $Y$ goes
(which is an illusory ideal), but only in the sense of \co{recognising} $X$ as
the focal point, the center excluding everything which is not $X$. The
\co{distinction} of $X$, drawing its limits, is not voluntarily determined by
one's wish or arbitrary caprice. Yet, it is relative to the \co{existence}
drawing this limit and here lies the \co{aspect} of freedom, or in any case of
some indeterminacy. Where to draw the limit? Does Earth really, in fact, move or
not? Of course, we know that it does, yet living on Earth we will never stop
seeing the sun {\em rising} above the resting horizon and {\em moving} across
the sky over the immovable surface of our planet.  \co{Concepts} change over
time; a \co{concept} as if evolves, centering perhaps all the time around the
same basic intuition, by changing its place in relation to other \co{concepts}
and even the borders separating it from others. There is a discernible line
which goes from the basic idea of weight of daily objects, through the fact
that air too has some weight, through the Newtonian mass, to the relativistic
mass. Yet the actual \co{concepts} are in all cases different, even if
\co{vaguely} related. One might ask if we then deal with the same \co{concept}
or different ones, but we leave quibbling to the quibblers. For the most,
\co{concepts} do not have any unique and \co{precise} boundaries and where such
boundaries are set may depend on the historical situation, on the social context
and even on the individual user.  \noo{Having dispensed with \thi{substances} as
  any \thi{things in themselves}, we should not have much trouble with
  recognising that neither \co{concepts} are any metaphysically rigid measures
  \thi{in themselves}.  } In this respect, if \co{concepts} differ from
feelings, it is because they change during the history to the degree which it
would be hard to discern in the emotional and spiritual complexion of the
humans.\ftnt{We can refer to \citeauthor*{EvolConcept} for a general theory of
  concepts as evolutionary entities. Yet one may feel that this is stretching
  the point too far. At least the mathematical concepts do not depend so on
  extra-mathematical circumstances. In a sense, we agree, and we will return to
  this sense in the last section of this Book. Consider, however, \citef{[t]he
    concept \thi{trisection of an angle with ruler and compass}, when people are
    trying to do it, and, on the other hand, when it has been proved that there
    is no such thing.}{WittPI}{ I:334} The fact that, say, the concept of a
  group has become what it is (a non-empty set $X$, closed under a total binary
  function $\_\cdot\_$, which is associative, $\forall x,y,z\in X:x\cdot (y\cdot
  z)=(x\cdot y)\cdot z$, has an identity element $1$ such that $\forall x:1\cdot
  x=x\cdot 1=x$, and inverse $\_^-$ such that $\forall x\in X:x\cdot
  x^-=x^-\cdot x=1$) was a matter of development. Just a few examples from a
  long list of variations attempted in the history.  Ruffini considered (around
  1799) only groups of permutations and used explicitly only the property of
  closure (the other properties hold for permutations \thi{automatically});
  Galois used (in 1831-32) the group properties extensively without ever
  defining the concept; Cayley introduced (in 1854) a collection of operators
  (on another set) which were closed under associative composition, had an
  identity and inverse. Intuitions point unmistakably in the same direction but
  did they all have the same concept, and if so, was it the same concept as we
  have today? (Permutation groups are, after all, only a special case of the
  general concept (even if the crucial one, since every finite group can be
  represented as such a one). Cayley's requirements are almost the modern
  axioms, but his definition applied to a collection of operators over some
  other set.) If one feels tempted to answer yes, \noo{since the abstract
    properties used by these three and many others were, in fact, the ones we
    list today among the axioms } one should better consider that, for instance,
  non-zero rationals with multiplication form a group so, perhaps, the concept
  was known to the ancient Greeks\ldots? The mathematical adventure has led to
  drawing just these, and not other, division lines (choosing just these, and
  not other, axioms.) This terminus became abstracted and posited as a new
  entity, a new object of possible study. {\em This entity} did not
  \co{actually} exist earlier in the mathematical world, it \thi{was there} only
  as a \co{virtual} possibility of being distinguished. And so \thi{were there}
  all the variations which ended up as special cases or failed approximations,
  as well as all those which still \thi{are there}.\label{ftnt:group}}

\newp 
%\ \vspace*{-2ex}
\noo{Degrees % (and individuals!)
  \pa \co{Concepts}, like \co{impressions} and almost everything else, admit
  degrees.  Psychologists report results about children exhibiting understanding
  of classifying \ccom{Check that ???}  concepts \ldots Does the ability (or
  just a tendency) to classify horse and cat separately from sparrow and stork
  witness to the possession of the \co{concept} \thi{bird} as opposed to, say,
  \thi{non-flying-animal} (\thi{mammal} might be too much to expect)?  Well, it
  certainly witnesses to the ability to distinguish the two.  Whether it is a
  \co{concept} or not is a matter of definition.  The children may, in fact,
  have no fixed \co{sign} which unifies different appearance of birds.  Strictly
  speaking, their \co{experience} is structured in the way which opens the
  possibility of forming such a \co{concept}.  This, we might say, happens when
  such a unifying \co{sign} emerges, either as the word \wo{bird} or, perhaps, a
  mere recurring \thi{\ldots}, where \thi{\ldots} is whatever the perceptive
  system uses to identify birds or whatever word, in the entirely {\em private}
  language, the child might use.  }

\wtsep{Gestalt vs. abstraction}

\pa\label{pa:lab} Emphasizing the importance of the \oss, we were not trying to
repeat the analyses of Gestalt psychology of perception which, although
certainly applicable in many cases, are not universal. Neither were we trying to
postulate that every \co{impression} is accompanied by the actual thought of its
\co{complex}. 
The two represent only different modes of presentation but, although addressing the same
\co{complexes}, need not go in any pairs. In particular, a \co{complex} may also
enter the \hoa\ after its aspects have been seen, analysed, contemplated --
experienced. In the epistemological tradition underlied by the image of ultimate
\thi{atoms}, this has in fact been the only possibility: abstraction (of
higher concepts from instances), induction (of general laws from special cases), 
deduction (of consequences from given axioms), construction (of complex
structures from simple atoms) are all variations on this theme, and we will not
add any more for rather obvious reasons. But the same possibility of a unifying
\co{sign} arising after various aspects of the corresponding \co{complex}
applies also in the case of \co{impressions}. 

Entering a room, one need not get immediately any general \co{impression}. One
can have none and get one first after being in this room for a while, after
having discovered different aspects and objects collected there which, together,
build up a unified \co{impression} of the whole.  Looking for a way in a foreign
city, one can be forced to stop at each cross and ask for directions, to consult
the city's plan, etc. Eventually, one finds the way.  The next day one may still
have difficulties but the intermissions won't be that frequent. After a few
times one knows the way \thi{by heart}, one has it as a one entity, given in a
single moment not with all its details but with the clarity of the single
\co{sign}: \wo{I know the way.} One has \thi{built a totality}. Sure, to begin
with one might have had a mere idea of this way from the hotel to the restaurant
but now one has \co{an experience} of it. Shall we call it a \wo{concept} of
this way? An \wo{impression}? An \wo{understanding}?  We leave quarrels about
that to those who find them worthwhile.

A good example of a \co{concept} emerging after the experiences of its parts
involves reasoning, for instance, understanding a mathematical definition.  To
begin with, one has to work one's way through the notation, then through the other
concepts applied, then their interrelations, finally, its implications and
relations to other definitions.  Then one may understand it but it is not the same
as \thi{getting it}.  To \thi{get it}, one has to grasp the whole in one \co{act}
-- of intuition, inner perception, call it as you like -- which gives perhaps
a wrong \co{impression} of certainty that, even if one does not know all possible
implications and applications of the definition, one knows {\em how} it possibly
can be used, that is, where it can not be used.  \thi{Getting it} is an
\co{impression} of several things falling on their place, of having understood,
which accompanies acquisition of a \co{concept}. It is not, of course, the
\co{impression} corresponding to (the same \co{complex} as) the \co{concept}
itself. The latter may be a more subtle matter. Developing understanding of some
field, one develops also~\ldots intuition. And that not only in the vague and
general but also in the quite technical sense of that mode which \citet{relates
  immediately to the object, and is single. [...]  In whatsoever mode, or by
  whatsoever means, our knowledge may relate to objects, it is at least quite
  clear that the only manner in which it immediately relates to them is by means
  of an intuition.}{CrPR}{A320/B377-A19/B33\kilde{Doct of Elements} [Without,
  of course, agreeing to the Kantian limitation of intuition to sensuous
  perception.]} This kind of intuition comes very close to what we mean by an
\co{impression}, a unified \co{sign} of a \co{complex}. And it arises when
\wo{[a]t the end of a certain time ordinary meditation produces what is called
  acquired contemplation, which consists in seeing at a simple glance the truths
  which could previously be discovered only through prolonged
  discourse.}\ftnt{Although the passage refers to the development of spiritual
  insight, it is obviously applicable. The quotation is from
  \citeauthor*{Alphonsus}{ Appendix I:7}.}  All \co{reflective} efforts are
guided by the search for such a unifying intuition -- but \wo{for the intellect,
  the unity remains only a postulate, an act of
  faith.}\ftnt{I:footnote~\ref{ft:rada}.}



\wtsep{precise -- vague}

\pa
\co{Concepts}, being events of \co{reflective experience}, and even 
of \co{attentive reflection}, like to have a \co{precise} 
foundation, a basis of some primitives from which more 
complicated \co{concepts} could be built.\noo{This seems to be one of a 
few characteristics shared by all theorising about concepts.}
Concepts are built bottom-up, either by construction or by abstraction, from
initial \thi{givens}, like \thi{substances} or rather their conceptual
counterparts, \thi{essences}.

% Let us observe a certain similarity of \co{distinction} and abstraction -- the
% latter, as understood in Aristotle's theory of forms, or later in Abelard's
% theory of universals.

For us, abstraction is a secondary function of \co{reflection} which, starting
with the \co{objective} limits of \co{distinctions}, extracts from them some
\thi{features}, general \thi{properties} which, indeed, are observed only in the
{objectified} particular things. As Abelard, repeating Aristotle, points out,
\citt{The understanding considers separately by abstraction, but does not
  consider as separated [...]}{P.Abelard, \btit{The Glosses on Porphyry}
  \citaft{Jones}{V;p.194}} We would 
say: \co{distinctions} are not \co{dissociations}. One might perhaps quote Hegel
saying that a 
\co{distinction} is an abstraction from the \co{indistinct}, but this would only
confuse the language. The difference is that \co{distinctions} are primary and
abstraction secondary: abstraction must start from somewhere, and it believes to
start from some basic, \thi{atomic givens}.  Only the primordial
\co{distinctions}, and the \co{reflective dissociations} in particular, provide
this very ground from which abstraction may possibly 
start. 

Any \co{distinctions} can be refined, and so no eventual substances are to be
found. As \co{concepts} mean only what they refer to, so even trying to
establish \thi{positive essences}, no such things as primitive, basic
\co{concepts} are to be expected.  \co{Distinctions} may introduce apparent
universals before one has accumulated enough \thi{substantialised} instances to
perform any abstraction. As phenomenologists say, induction is not necessary and
often it is sufficient to encounter one single instance of a phenomenon, to
perceive its essential (universal) aspects. Yet this apparent universal, like
everything else, may be eventually confronted with further differentiation. A
person may acquire the concept \thi{BC} without understanding what \thi{B} and
\thi{C} are. For the sake of example, let us take \thi{B} to be \thi{brown} and
\thi{C} to be \thi{cat}.  A child growing up with only \thi{BC}s, but with no
separate \thi{B}s or \thi{C}s, might learn to distinguish \thi{BC}s from
\thi{ZY}s, even have a word \wo{bc}, without ever getting the idea of
dissociating \thi{B} from \thi{C}.\noo{This reminds strongly about James'
  \thi{conjunctive relations}.} But meeting \thi{BD} (brown dogs?), can give
rise to separate concepts -- \co{dissociation} -- of \thi{B} and \thi{C}.  As
long as \thi{B} and \thi{C} appear only in \thi{BC} they may be, to some degree,
\co{recognised} but they are not (necessarily) \co{dissociated}.  What happened
with \thi{BC}, can now happen to \thi{B} and \thi{C}, which remain
\thi{primitive} only as long as further \co{distinctions} are not drawn.

We are not saying that, in practice, one must not built conceptual constructions
from some basis of determined primitives. This is the only way to raise any
conceptual constructions.  But such constructions proceed only upwards from
earlier \co{dissociated} pieces and hence will never reach any bottom which they
must, in fact, always presuppose. To put it trivially, \co{reflection} can never
capture \co{experience}.\noo{One is free to call it \wo{scepticism} but we would
  object. For the first, we aren't even half-way through, yet. And for the
  second, identifying knowledge, understanding, or epistemology, with the
  adventures of \co{reflection} is a bad habit.}  There are no metaphysically
ultimate \thi{atoms} from which \co{reflection} could once and for all
reconstruct the whole world as it is \thi{in itself}. But also, every
\co{reflective} project must needs start from some basis, from some
\thi{givens}. These two facts not only are not contradictory but do not imply
any scepticism either. It is exactly the tension between them which harbours all
the \co{concreteness} of which \co{reflection} is capable.

\ad{Universals}
%
We were trying to point out the analogies, rather than the differences, between
\co{moods} and \co{concepts}: they have the same temporal scope; addressing
\co{complexes}, they move in the tension between the unity of a \co{sign} and
the multiplicity of the aspects and properties of the \co{complex}; we have
emphasized the communicability of \co{moods}, if not an irresolvable association
so at least complementarity of \co{concepts} and \co{moods}, and the fact that
not only \co{impressions}, but also \co{concepts} come in various degrees of
\co{precision}. Eventually, the difference between the two turns out to be the
difference of tendency and degree: \co{concepts} \co{dissociate} striving for
mathematical \co{precision}, while \co{moods} keep the \co{complexes} in a
\co{vague} -- but \co{clear} -- unity.

This, however, is hardly the whole difference!  Isn't it so that \co{concepts}
are (composed of) universals \citet{and by the universal we mean that which is
  predicable of the individuals,}{AristMeta}{III:4} that which \citetib{is
  common, since that is called universal which is such as to belong to more than
  one thing.}{AristMeta}{VII:13} Moreover, it is that which can be {\em known}
about particular, it is \citet{implicit in the clearly known
  particular.}{PostAnal}{I:1} \co{Impressions}, on the other hand, are
particular and unique, and certainly not known with the \co{precision} with
which universals constituting essences can be known.

This isn't exactly so. This opposition reflects indubitable psychological
difference but also (perhaps by the same token) the source of ambiguities
inherent in Aristotelian and post-Aristotelian metaphysics: the division into
individuals and non-individuals which is supposed to apply \thi{objectively} not
only to living beings but also to material \co{distinctions}, things and their
properties. The \co{dissociation} into atomic, \thi{substantial} entities,
existing each completely on its own, makes any connection between them at best
secondary, and at worst unreal. These two positions with respect to universals
-- considering them either as secondary or as completely unreal --
seem the only possible once we assume the ontology of particular \thi{substances}.

\pa First, concerning the opposition to feelings and \co{moods}, there is what
we just said in \refp{pa:moodCommunicate} about the possibility of \co{sharing}
them.  Then, one can have a \co{concept} of some \co{impression}, say a
\co{concept} of irritation. Probably, it will be only the \co{sign}, the word
which comes to one's mind whenever experiencing irritation, but it may be more.
But even if one does not have such a \co{concept}, one can still experience the
same irritation on different occasions, one can \co{recognise} the feeling one
is having now as the same one had before. Yeah, this looks like a kind of
\thi{private language}.  The word \wo{language} here is bad, but \wo{private}
and \wo{\co{recognisable}} are definitely on the proper place.  The worrying
phrase is, of course, \wo{the same}, but then remember that we are not talking
about any \thi{essences} reappearing in new instantiations at completely
dissociated points of time. We are talking about \co{cuts} from the continuity
of \co{experience}, eventually, from the \co{indistinct}. A \thi{new instance}
of irritation does not appear from nothing to get re-cognised and connected to
some old and merely remembered instance. It enters the \hoa\ through the levels
of gradual \co{actualisation} from some \nexus\ which constitutes its
\thi{relatedness} to a whole field of other \co{experiences}.  This \nexus\ 
involves also more \co{virtual traces} and various \co{actual experiences} emerge as
\thi{instances} of the same because they arise from the same \nexus\ and pass
through the same \co{traces} on their way toward \co{actualisation}.
\thi{Identity} of one \thi{instance} of irritation and another is constituted
{\em before} they get \co{actually} \thi{instantiated}.  This is all we can
answer to the question \wo{{\em How} can I be certain that it is the same
  irritation as before?} -- the two are indistinguishable in their \co{virtual}
source and the only difference concerns the two \co{actual} manifestations,
appearance at two different moments. One should not confuse an \co{impression}
with an \co{actual} \thi{instance} of the \co{impression}, the whole \co{trace}
on which the \nexus\ of some \co{impression} resides with a particular situation
where the \co{impression} is experienced. Only the assumed ontology
of \co{dissociated objects} and epistemology of discrete time of unitary
\co{reflective acts} make one worry about the \thi{unreality} of identity and,
in fact, of any relation, between \co{experiences} whose only observable
difference consist in their appearance at different points of time.

True, conceptual universals can be arrived at by abstraction, which means, their
relation to the particulars they subsume will be much more \co{precise} and
\co{visible}. But to the extent they are not mere \co{reflective} constructions,
they will also \co{reflect} some \co{distinctions} which \co{cut}
the \co{experience} beyond any possible \co{actuality}.

\newpa 

\pa
Everything that is is a \co{distinction} or a limit thereof, a \co{cut} from
\co{experience}.  And every \co{cut} is unique: what it
\co{cuts} out is not \co{cut} out by any other \co{cut}.  A \co{cut} is
just that which it \co{cuts}.  Certainly, \citet{it is impossible to
abstract universals from the singular without previous knowledge of
the singular}{ScotAnima}{22:3 [This does go for 
many \co{concepts} formed by \co{reflection} in the process of 
abstraction -- but this does not make it the fundamental 
characteristic of universals.]}
but it is only the traditional reduction of
\thi{reality} to \co{actuality} which forces one to assume that universals 
arise only by abstraction, that to encounter \thi{chair} one has to first see
several chairs. 
%
\begin{itemize}\MyLPar
\ans Concepts are abstract and universal. Universal, that is, they may have 
many particular instances, be encountered in different experiences.
\que Recalling \refp{th:noindividual}, at this moment, I wouldn't be so 
comfortable with the 
word \wo{particular} but let go. Is \thi{Prague} a universal?
When I go around, I see this building and that corner, from the Old Town 
Square I do not see \ger{Vy\v{s}ehrad} and from below of \ger{Vy\v{s}ehrad} 
I do not see \ger{Hrad\v{c}any}.
\ans Yes, but still it is a particular thing, only ``too big'' to be seen 
all at once.
\que It is too big and at each of these places I see a 
part of \thi{Prague}, I have a different \co{actual experience} of Prague, something 
which, with enough of bad will, can be called ``a particular instance'' of 
\thi{Prague}.
\ans No, you see different parts of a big city, you do not see {\em the 
same} universal exemplified in various particular places.
\que OK, what I see of the \thi{horse} in this  horse is exactly the same 
as what I see of it in another horse. But what I see of \thi{this chair} 
now is {\em the same} as what I saw of it yesterday. And do not tell me 
that it does not count because \thi{this chair} is a particular.
\ans It counts because  you may have several 
distinct exemplifications of the same universal {\em simultaneously} and, 
furthermore, you may {\em always encounter more}. With this chair you cannot 
do the former -- nor the latter if I burn it.
\que What about \thi{all my grannie's chairs}? It seems, it looks more 
like a universal than like a particular. But they all are here -- she has 
never had more than these four. And, besides, she is dead, so you will never 
get new ones. If it is ``my grannie'' who annoys you here, then what 
about the dinosaurs? We have rather run out of the possible new
instances.
\ans Forget new instances; a universal is universal even if no instances exist.
\que Have you turned extreme realist in this matter? But tell me first what they are.
\ans
Take \wo{abstract} -- a universal is not an 
independent being, it is always only an aspect of a particular.
\que You have just said that it need no particular instances\ldots
So, after all, the \thi{chair} (or the \thi{dinosaur}) is not universal?
\ans It is, but it is what is common to many chairs, what can be predicated
equally of many particulars.
\que By \thi{common} you mean, probably, something like a stereotype, 
a paradigmatic instance, or just the \thi{essence} -- but we know that these
won't quite do. 
But who are these particulars? And isn't \thi{Prague} the same at 
this particular \ger{Vy\v{s}ehrad} and at this particular 
\ger{Hrad\v{c}any}?
\ans It is, but \thi{Prague} is not {\em predicated} about them.
\que Really? One does predicate \wo{praski} about this particular
\ger{Hrad}.\noo{and this particular \ger{Vy\v{s}ehrad}.}  But even if one
didn't.  So it would be just a matter of how we use the language? If I say
``This city is exactly like Prague while that one is not.'' will it do?
\ans Of course, not. First you use ``Prague'' as a name of a particular 
city; then you use it as a \ldots 
%
\que Predicate? A concept? Then, does it mean that I can 
pick any particular (reference) and turn it into a (predicable) concept?\noo{Why
  not? Just like any 
  predicate can be nominalised, i.e., turned into 
a referential concept, so any particular reference, like \thi{Prague}, can be
turned into a referential concept (often, by means of a definite description)
which can then be turned into a predicate, \thi{Prague-like}.}
And what is the difference, except for 
the purely grammatical one, between ``Prague'' used in the first and in the 
second way?
\ans Predicating \thi{Prague} of another city, you are really predicating some
universal which is implicit in Prague and which you also find in this other
city.
\que Perhaps, but then one should tell me what universal it is. As far as I am
concerned, Prague has a very specific character and atmosphere,
indistinguishable from its uniqueness, which can hardly be characterised 
better than ``Prague-like'' and which can be in various forms or degrees found
at some other places.
\item[] etc., etc., \ldots
\end{itemize}
%

\pa 
Being brought on a visit, at the age of two, to Prague I did not acquire any
\co{concept} \thi{Prague}. But I certainly \co{experienced} something. On later
visits I could gather all the variety of views and impressions of the city under
the common name \wo{Prague}.  Yes! \thi{Prague} is a \co{concept}; as good as
any other -- it is a \co{sign} unifying a variety of experiences and, if one
insists, \co{objective} facts, into a totality of one \co{complex}.
\thi{Conceptually} it is, perhaps, a very poor \co{concept}, but it still can be
predicated about other elements of experience.  With such a generous
understanding of a \co{concept}, we are not interested in possible distinctions
between various kinds of \co{concepts}.  Such \co{distinctions} will be but
distinctions concerning {\em what} I may have a \co{concept} {\em of}.
Certainly there is a huge difference between the \co{concept} \thi{Prague},
\thi{bird} \noo{my dog Philo} and \thi{prime number}.  But there is nothing
inherently {\em conceptual} about these differences -- it is but the difference
of content, the difference between Prague, birds, prime numbers. \co{Concepts}
are meanings, \co{distinctions} which can be \co{actualised} enough to be
grasped and understood in the unity of one \co{act}.  Now, as to the universals:


\thesisnonr{Universals are the \co{non-actual} things of
\co{experience}, the \co{cuts} through the \co{experience}
exceeding the \hoa.} 

As \co{non-actuals}, universals appear first of all as non-individuals.
Characterising the individuals as whatever can be grasped within the \hoa, their
opposite becomes anything that \co{transcends} it. The relation between the
\co{actual} and the \co{non-actual} is much more intimate than the bare
opposition, but it reminds of a relativised (to the \co{existence}) version of
the relation between particular substances and universals.  \citet{If they are
  universal, they will not be substances; for everything that is common
  indicates not a \thi{this} but a \thi{such}, but substance is a
  \thi{this}.}{AristMeta}{III:6} Universals are possible stations of
\co{distinctions} which, sedimented into \co{signs}, allow then further
\co{distinctions} providing, as one says, \thi{particular instances}.  But the
primary difference, {\em the} difference which counts, is only that between the
possibility of an \co{experienced} completeness, of being given within the \hoa\ 
versus the \co{experience} of its factual (or even essential) \co{non-actuality}.
With respect to the \hoa, the whole Prague, Europe, irritation, my life are as
much outside of it as the assumed universals. All that can appear within this
\co{horizon} are their respective \co{signs}.

The problems with the status of universals are encountered with respect to
individuals as the problems of their identity or their \thi{substantiality}. The
question: what makes different occurrences of \thi{blue} or \thi{chair}
instances {\em of the same} universal \thi{blue} or \thi{chair}, is not really
different from the question: what accounts for the fact that the chair I see now
is {\em the same as} the one I left here yesterday (or what is one in the
experiences of Prague.)  Once we have drawn a limit of \co{distinctions} around
Prague, it has become one city and we may have thousands of experiences, all
being experiences of Prague, all exemplifying this particular city in the same
way as one chair exemplifies \thi{chair} or, perhaps, another chair.  The one is
the \co{sign}, the word ``Prague'', ``chair''.  But these are not mere empty
words, they refer back to the \co{cuts from experience} which, stretching far
beyond the narrow scope of any particular \herenow, terminate at some limit
beyond the \hoa.  The \thi{oneness} has no accurate verbal definition, nobody
can tell \co{precisely} where Prague begins and where it ends, what it {\em
  really} is, etc.  Just like nobody can tell \co{precisely} what constitutes a
\thi{wave} nor what the \co{concept} \thi{chair} is (when something I can sit on
ceases to be a chair, acquires rockers or, perhaps, becomes a {stump}).

The impossibility of verbalizing such \thi{identities} does not witness to the
impossibility of attaining any knowledge of them but merely to the fact that
linguistic, and then \co{conceptual}, \co{distinctions} do not exhaust the field
of our knowing. One can distinguish a chair from a tree stump which should be
enough to say that one knows the difference between the two, even if one can not
spell it out \co{precisely} and unambiguously. The assumed \thi{essences} play
the crucial role in handling \co{concepts} and universals by reducing (or rather
attempting to reduce) the non-\co{actual}, and even \co{non-actual}, things to
purely \co{actual} aspects, just like the \thi{substances} justify the
objectivistic illusion in its restriction to mere \co{actuality}, in populating
the whole universe exclusively with individual beings which, moreover, tend to
dissolve as the vanishing points of ultimate self-identity.

\noo{
the most universal things
are elements (because each of them being one and simple is present
in a plurality of things, either in all or in as many as possible),
and that unity and the point are thought by some to be first
principles. Now, since the so-called genera are universal and
indivisible (for there is no definition of them), some say the
genera are elements, and more so than the differentia, because the
genus is more universal; for where the differentia is present, the
genus accompanies it, but where the genus is present, the
differentia is not always so. {V:3}
}

\pa The problems of the universals originate from the assumption that there
are some definite particulars -- some basic substances, independent, simple and
indecomposable and, moreover, that such particulars are the only genuine objects
of experience; in short, from the reduction of \co{experience} to
\co{actuality}. This assumption is inseparable from the less desirable one, 
namely, that experience consists of some isolated points, \thi{nows}, strangely
and always inexplicably succeeding each other.  But 
 \co{actual experiences} and things which can be experienced within the \hoa,
which can be made \co{actual objects} of attention, are interwoven into the
continuous texture of \co{experience}, are only its \co{actual reflections}.
Encountering \thi{a new instance of a universal} may be a new \co{actual
  experience}. But this is not \thi{added} to the rest of \co{experience}, as a
new item to a collection -- \noo{It may be so \thi{added} only to the
  \co{totality} of \co{experiences}, but this is just the way of thinking in
  terms of \co{actuality} alone, which we are opposing.} it is 
rather subtracted, for it has emerged as \co{an actual experience} {\em from}
\co{experience}.

What do we encounter the first time we see a cup? \thi{This cup} or \thi{a cup}?
I guess, {\em both} for there is no reason to draw any such distinction. Having
both means that we have neither, neither particular nor universal, just
\thi{cup} (or, perhaps, a special \thi{this}). Encountering the same cup for the
second time, it is not added to the earlier \thi{cup} -- it is subtracted from
it (and, sure, it is also subtracted from all the \co{rest}, but it emerges from
the horizon through the \co{trace} whose last point is this earlier \thi{cup}.)
This subtraction establishes simultaneously {\em both} the universal \thi{cup}
and its two particular instances.  As a matter of fact, we still have not
encountered \thi{this cup} or else, there is still no distinction between (the
universal) \thi{cup} and \thi{this cup}. (Such a distinction may arise when, for
instance, somebody brings in a new cup and puts it next to our \thi{cup}.)  In
the same manner, \ger{praski Hrad} and \ger{Vy\v{s}ehrad} become
\co{dissociated} from the same \co{trace} of Prague which, in turn, has been
detached from one's \co{experiences} of one's home-town and from the whole
\co{experience}.

The \co{reflective} thinking happens already within the assumed ontology of
\co{dissociated} and \co{externalised objects}, and equally \co{dissociated}
points, \thi{substantive parts} of time.  But when \co{my reflection} encounters
something completely new 
and unexpected, when it \thi{adds} a new and some old instances, performing the
abstraction of, say, \thi{Prague} or the common nature \thi{cup}, these have
already been \co{distinguished} as unities \co{before} and \co{above} this
\co{reflective act}. The \co{reflectively} new and unexpected has emerged from
some \co{trace} in the process which is performed in my \co{experience} (in my
body, my perceptual mechanism, my brain, or wherever one wants to look for it)
and of which \co{reflection} only finds the final results.  Every new cup is not
added to but is \co{distinguished} from the same \co{trace} as, for one reason
or another, a new entity. To emerge as distinct, the two must have first been
the same. When \co{I} notice a new cup and \thi{associate} it with an earlier
one, this is only a conscious or subconscious \co{reflection} of the process
which has already \co{dissociated} the two from the same origin.  Eventually,
everything emerges from the \co{one} and the same \co{indistinct}, everything is
one before it becomes two.  Of course, once such \co{dissociations} have been
established and sedimented, \thi{new instances} of Prague, cup etc. can be
encountered, that 
is, universals can start appearing as common features.  But it is the \co{unity}
of \co{experience} which \co{founds} the continuity in the \co{experience} of
the particulars as well as of the connections between various instances of
universals, in short, of the very possibility of \co{reflective}
\thi{association}. 
%, or getting back to our language, of the \co{non-actual} things.

Universals, and more generally \co{non-actual} things, witness thus to this
\co{unity} of \co{experience}, \co{founding} also the very experience of
particulars. Trying to get rid of the former, one ends up dissolving the latter
as well, because both happen not only to mutually condition each other but also
because both are underlied by the same principle of constitution: as the
sedimented limits of \co{distinctions}; and particulars are but the \co{actual}
limits of \co{traces} which \co{cut experience} far \co{above} the \hoa. 
%contra post-modernists


% The differences concern only the degree of consciousness, accompanying such
% experiences, that the \co{object} is actually a \co{complex}, that it is not
% fully \co{actual}.


The degree of similarity, or the character of what is common to many \co{actual
experiences}, distinguishing the non-\co{actuality} of \thi{Prague} from that of
\thi{chair} and from that of \thi{this chair} is only a difference of degree.
\co{Concepts} are the \co{actual signs} of (often \co{non-actual}) things whose
characteristics, whose repeating aspects are more prone to \co{actualisation}:
it is easier to indicate (but still only indicate) what possibly might be
included in -- perhaps even exhaust! -- the \co{concept} \thi{chair} than the
\co{concept} \thi{Prague}.  What connects one experience of Prague with another
may be different from what connects one instance of \thi{chair} with another.
But the fact that in the latter case this connection seems more organising and
is based on \co{actually} identifiable similarities rather than some vaguely
intuited emotional 
characteristics, or mere geography, is merely a quantitative difference of
degree. The basis is not a connection which \co{reflection} has to
establish between \co{dissociated actualities} but a \co{cut} through
  \co{experience}, a \co{cut} made in the \co{experience above} the \hoa.

\pa \co{Positing} particular, most \co{actual} entities as the primordial
\thi{substances}, is based on the prior \co{reflective dissociation}.
Abstracting activity of \co{concept} construction starts naturally from those
basic entities which are most \co{precisely} grasped by the singular \co{acts}
of \co{reflection}.  The conflict between nominalism and conceptualism (viewing
universals as conceptual abstractions) is played already on the ground of the
ontology of particulars. The question concerns only to what extent particulars
contain any reality beyond their particularity or \thi{substantiality}. As such,
both these views are opposed to the earlier realism which was willing to assign
the universals independent reality. Our universals remind indeed closely of the
independent forms which can exist -- be \co{distinguished} -- also before any
particular instances are identified. But we never try to \co{dissociate} the
different levels of \co{experience} -- in this sense, once the \hoa\ is
established, no \co{non-actuality} can exist without and independently from it.
A very limited space is left for conceptualism in that one can sometimes obtain
new universal characteristics by abstraction, but abstraction accounts neither
for their primary form nor for their eventual ontological status. Nominalism
must rest satisfied with the fact that universals, just like all \co{experienced
  distinctions}, are relative to the \co{distinguishing existence}.


% Realism = platonic, unity before units
% Nominalism = pure empiricism, only immediate
% conceptualism = closer to nominalism, still ontology of particulars; 

\noo{ \pa Do we end in nominalism? Not at all. The reality of \co{concepts} is
  much more than realism was able to assign to them.  The one is the \co{sign},
  the word ``Prague'', ``chair''. But these are not mere empty words, they refer
  back to \co{impressions}, \co{experience} and the \co{cuts from experience},
  which stretch far beyond the narrow scope of any particular \herenow.
  
  The status of a universal is the same as of a \co{sign} announcing presence of
  a \co{complex} which itself never can be fully \co{actual}.  And the
  ontological status, if we may misuse the phrase, of its denotation -- which is
  the same as its connotation -- is the same as that of a \co{complex}: a series
  of \co{experiences} (if we want to focus on \co{actual experiences}), each
  bringing forth different or similar aspects of a totality. The point is that
  this totality 1. has been \co{cut from the background} of the whole
  \co{experience} as a \co{unity} which then 2. is gathered in the
  \co{actuality} of one \co{sign}.  In short, it is an \co{experienced} thing, a
  \co{cut from experience}, which differs from the particular things only in the
  impossibility of squeezing it within the \hoa.
  
  Entering the \hoa\ as a \co{sign}, each such thing can be, in turn, made into
  an \co{object} and treated as if it were one. This is where the issue between
  realism and nominalism arises.
  
  It is also what is usually understood by conceptual analysis and abstraction
  starts only here, in the assumed givenness of \co{actuality}. But this is a
  story of further constructions from already constituted \co{actual} contents.
  
  Before I can begin to build \co{complexes} from their components, I had to
  acquire enough \co{distinctions}. This did not happen at once but is a
  continuous process. Once enough \co{distinctions} have been made, I may
  proceed to distinguish further and, at the same time, compose them into
  \co{complexes}.  Some of such \co{complexes} I call `Prague'', ``Europe'' and
  some others ``chair'', ``horse'', ``irritation''.  I may arrive at the
  supposed concept of a chair through a series of experiences, some analysis and
  abstraction.  But if I am to arrive at anything whatsoever, I had to have the
  {\em experience} of various chairs and I had to make a \co{cut} through the
  sphere of my whole \co{experience} which brought forth this \co{complex}. No
  matter what I arrive at, will always leave room for further possibility of
  \co{distinction}.  }

\ad{{Eidos}, {noema} and {anamnesis}}\label{anamnesis}
%
Turning the tension between the fleeting \co{actuality} of particulars and the
\co{presence} of universals into an opposition, then putting all kinds of
non-\co{actuality} and \co{non-actuality} -- concepts, universals, ideas,
values, souls -- into one sack of the \thi{spiritual} as opposed to the
\thi{material} and, finally, recognising the effective reality of the spiritual,
one will easily come up with some \thi{unchangeable essences} stored in an
\thi{ideal, intelligible world}. Roughness of this operation, also when seen in
the light of attempts to provide a rational counterpart of mythology, opens too
much space to be addressed in detail here. It infected the whole tradition with
the \thi{intelligible world of Ideas} which, somehow, emerged ready-made from
the {original One} or from God's mind.\ftnt{The doctrine which made Forms/Ideas
  cease to live their own life and become concepts of the divine mind \noo{seems
    to appear for the first time in the second century \add with} is present in
  Philo of Alexandria,\kilde{De opificio mundi,17;20} Atticus,\kilde{Fr.9,40}
  \citeauthor*{HandPlat}{ IX,XIV}.  \citeauthor*{DivineForms} presents the
  history of the transition.} Thus predetermined and fixed, it hangs over \co{this
  world} as a static double -- \thi{intelligible} only in empty postulates
because, as a matter of fact, completely incomprehensible and in constant need
of being someway connected to \co{this world}.

\gre{Anamnesis}, learning as recollection, arises easily from such a sublimation
of \co{concepts} and their assumed \thi{essences} to the level of the immortal
habitat of the soul.\noo{We leave it to the post-Heideggerians with
  philological inclinations to investigate the obvious linguistic parallel which
  links the theory of \gre{anamnesis} to the privative form of \wo{truth} in
  Greek, \wo{\gre{aletheia}} -- relating \gre{lethe} to the waters of
  forgetfulness, \wo{\gre{a-letheia}} reads as
  \thi{non-forgetfulness}.\label{nonforget}} But Socrates' \gre{maieusis}, as
exemplified in \btit{Meno}, need no such speculative foundation.  If
\co{reflection} finds only the \co{distinctions} in the \co{experience}, then
obviously what it finds has already been there -- it is a \co{repetition}.  The
only thing one has to do is to identify soul with the \co{reflective precision},
rational explicitness, and call its emergence from \co{experience} into the
\co{actuality} \wo{birth}, to postulate that the soul knew these ideas before
birth.

The questions which Socrates asks constitute the whole and essential element of
getting the right answers from Meno's slave. \wo{To ask the right question is
 half the solution of the problem.} The questions involve the relevant
\co{distinctions} of which the boy had been unaware, which he had not made
(or not connected) before.  The procedure might be pedagogically admirable but
it proves nothing of 
what Plato would like to put into it.  Being able to make the relevant
\co{distinctions} and identifications is different from having actually made
them -- making them is precisely that: making {\em new} \co{distinctions},
getting {\em new} insights.  That
the boy is able to follow the geometric argument (yes -- argument!)  of Socrates
shows simply that he is not entirely stupid, that he can \co{recognise} the
\co{distinctions} and their relations suggested by Socrates. Finally, the
possibility of a 
\co{distinction} being at all made is different from it actually having being
made and stored ready in some \thi{intelligible world}, \thi{divine mind}, or
whatever depot one manages to imagine.

\noo{Used as an argument for immortality of soul, it is only a smart piece of
sophistry.}

Soul's immortality remains as open an issue after the argument as it was before.
However, we will not dismiss the Idea so easily. With all due respect to the
spiritual inspirations -- we will return to \gre{anamnesis} in a more
appropriate context (\refp{pa:anamnesisB}, and then III:\refp{pre:openness}).
Here we can observe that the distinction between Idea (\gre{eidos}) and concept
(\gre{noema})\ftnt{Suggested by Socrates in response to the \thi{third-man
    regress argument}.  For \citef{may not the ideas, asked Socrates, be
    thoughts [concept,\gre{noema}] only, and have no proper existence except in
    our minds?}{Parmenides}{\kilde{132b4?}} The received interpretation of the
  passage is, however, that it eventually rejects the view since \wo{The thought
    must be of something} and \wo{something which is apprehended as one and the
    same in all, [will] be an idea}. No matter what Plato's eventual view might
  have been, one could find in his writings grounds for this distinction which
  were to acquire much importance.}  could be interpreted as our distinction
between the non-\co{actuals} arising in the primordial process of
differentiation and universals (as well as \co{concepts}) which arise
secondarily through all forms of \co{reflective} abstraction (generalisation,
induction, construction). A \co{concept} is then a \co{concept} {\em of} the
Idea, is a \co{reflective} attempt to repeat the (limit of) \co{experiential
  distinctions} as an \co{actual} construction.\noo{Thus, the \co{concept}
  \thi{blue} tries to capture the well-known blueness, the \co{concept}
  \thi{friendship} the known and possible \co{experiences} of friendship, etc.,
  but also the emerging \co{concept} \thi{group} the felt, \co{vaguely} intuited
  and more and more \co{precisely recognised} nexus of properties and
  significance (cf.~footnote~{\small{\ref{ftnt:group}}}).} Such repetitive character of
concepts and explicit understanding would square well with the postulates that
Ideas are perfect exemplars to be merely imitated by the \co{actual} things and
thinking. But the value of such repetitions becomes the more dubious the deeper
and more personal/spiritual issues they try to capture. \noo{But in the lower
  spheres of \co{objects} and \co{complexes} they are the source of control so
  clearly manifested in the technological power.} We hardly have any
\co{precise} Ideas of justice, goodness or beauty to be merely imitated. This
other property of Ideas -- most clearly seen in the used examples -- namely,
that of being lofty and \co{vague} generalities, comes closer to our
\co{virtuality}. In this sense, Idea would correspond to a \co{virtual nexus}
which is not so much imitated as simply \co{actualised} by the things and
\co{concepts}. Even if one might still say that such an \co{actual concept} is
{\em of} the respective Idea, the \wo{of} has a very different meaning than in
the previous case. In either case, Aristotle's critique of Plato, amounting to
replacing Ideas by mere \co{concepts}, was possible because the two were not
sufficiently distinguished.  Unfortunately, the replacement amounted to a simple
reduction.\noo{(even God is, with Aristotle, just the tram driver, if not the
  tram engine).} The tradition attempting to reconcile the two masters, has
tried to correct this unfortunate replacement maintaining both elements of the
distinction, albeit in various forms and often only as a distinction between two
kinds of concepts.\ftnt{Plotinus, for instance, views the concepts (our original
  universals) as a kind of \thi{impressions} received from the Intellect and
  thus distinct from those received from the senses (for which we would have to
  substitute all secondary abstractions). The soul comprehends \wo{the
    impressions, superior and inferior}: \citef{The reasoning-principle in the
    Soul acts upon the representations standing before it as the result of
    sense-perception; these it judges, combining, distinguishing: or it may also
    observe the impressions, so to speak, rising from the
    Intellectual-Principle, and has the same power of handling these; and
    reasoning will develop to wisdom where it recognises the new and late-coming
    impressions [those of sense] and adapts them, so to speak, to those it holds
    from long before -- the act which may be described as the soul's
    Reminiscence.}{Plotinus}{V:3.2} The history and examples of the issue are
  reviewed in \citeauthor*{ConceptPlat}.}  We do not contest the reality (nor
significance, nor creative potential, nor technological relevance) of this
\thi{Aristotelian} thread but only its primacy, not to mention, exclusiveness.


\subsub{Ego, body, action, control}

\wtsep{complex - ego}

\pa The kind of \co{complexes} one is able to relate to, their character and
degree of complexity is relative to one's skills -- skills to differentiate and,
at the same time, to \co{re-cognise} the interplay of various aspect within one
totality.  Unlike bare \co{objects}, \co{complexes} are not relative to one's
mere presence, but to one's shrewdness, intelligence, skill to see the
connections; also to one's capacity of compassion and sympathy, of predicting
other's way of thinking and acting, etc.

What is a \co{complex} for one person need not be so for another. Having more or
less the same organs of perception and similar
capacities for discrimination, we typically agree on the status of single
things and \co{objects}. But many of these things may not even exist
in the world of a bat, whose perception mechanisms will doom irrelevant, i.e., leave
unrecognised, many things we distinguish. On the other hand, a
dog's smell will differentiate things and situations which, for us, may seem
completely indistinct. Different humans can, similarly, have different
abilities of forming and connecting \co{complexes} which, to some extent, are
smoothened by functioning in a linguistic community where words establish much
of inter-subjective agreement on a host of \co{distinctions}.  This applies for
skills at all levels.  For a professor of algebra, the rings are quite different
things from groups; a student may, to begin with, have problems with seeing that
these are two different \co{complexes}; an illiterate may not even understand
that one is talking about anything meaningful.  In short, the complexity of the
world one lives and acts in, the complexity of \co{complexes} one relates to is
the reflection of the \co{complex} of one's skills and abilities which we call
\wo{\co{ego}}.
%(not to say, of the \co{Ego} complex) and vice versa.

\wtsep{ego is outside (transcendent)}
\pa
\co{Ego} is the aspect of a person which can be reduced to \co{actual}
expressions and described in \co{actual} terms. It reminds a bit of Jung's
\thi{persona} as opposed to \thi{person},\noo{\wo{social person} as opposed to the
\wo{intimate person} in Scheler, though with him it concerns self-perception in
which also the latter is capable of all kinds contacts with others, not only
egotic social contacts, while the former is ultimate site of \ger{Einsamkeit},
Formalismus p.549} in so far as \thi{persona} is a
\co{totality} of \co{externalised} properties, comparable to the similar
properties of others and observable within the \hoa. The inquisitive attitude of
the guy from the pub who became \thi{the detective} belongs to his
\co{ego}. (The \co{trace} may, of course, go deeper.) 
\co{Ego} is the first object encountered by a simple, teenager's form of
self-reflection. It does not address being or, if it does, it does so only
indirectly. It is primarily occupied with the facts \co{that I am so-and-so}.
\wo{I have too round face.}, \wo{I have too light hair.}, \wo{I won't wear this
  -- what would others think?} Such worries pass quickly into slightly more
fundamental ones, marking the crisis of adolescence, which, nevertheless, still
carry the \co{egotic} character: \wo{I am not as good as he is.} \wo{I am
  insensitive.} \wo{I am ugly.}

Such characteristics and self-characteristics, genuine and honest as they may
happen to be, consist of {objectified} attributes which get attached to their
noumenal subject as some \co{external} properties. \co{Ego}, we could say, is
oneself viewed from \thi{outside}, with
other's eyes. But even when others are not invoked, \co{ego} is still
\thi{outside} oneself: any predicate involves, at least implicitly, a comparison
and, moreover, it happens only to be some \co{actually} observable fact which
never manages to reach the intended \thi{innermost essence} of one's person.

The non-\co{actuality} of a \co{complex} is of the same kind as the
non-\co{actuality} of \co{ego}. It is not {\em essential} \co{non-actuality} but
only non-\co{actuality} of something which might, or even was, \co{actual} at
some other time.  \co{Ego} is a person filtered through the \co{actualising}
sieve of \co{reflection}.  \citet{The Ego never appears except on the occasion
  of a reflective act.}{TEgo}{p.53; Sartre has ``I'' and not ``Ego'' here.}  And
it appears always \thi{as} -- confused, smart, late, amiable\ldots \co{Ego}
signifies a \co{complex} of properties, features, facts each of which, taken
separately, may be perfectly \co{actual}.  \co{Ego} is such a \co{complex} of
features, skills, ways of behaviour, etc.  which one attributes to a person.  It
constitutes the subjective pole of \co{actual experiences} of \co{complexes}.
\citetib{The Ego is the unity of states and actions -- optionally, of
  qualities.}{TEgo}{p.61} \co{Ego} is the \thi{ideal} unity (\co{posited} and
never experienced) of skills, abilities, qualities and actions.\noo{The unity of
  \co{ego} is only \thi{ideal} because it is \co{founded} in the deeper layer of
  \co{myself}.}

\noo{Although identifying \co{oneself} with one's \co{Ego} is not a healthy
  attitude, the devaluation of \co{ego}, of \co{my actual} determinations and
  capacities, may easily result, for instance, in the inability to \co{act} and
  interact successfully with the \co{actual world}.}

\wtsep{body is actuality (for reflection -- always transcendent)}
%
\pa Besides skills and abilities, the fundamental \co{aspect} of \co{ego} is the
body. The \oss\ of \co{actuality} are often relative to the body. \wo{It is
  nice} does not refer the feeling to a particular organ (even if it emerges
there) but to the whole body which \wo{feels nice}. On the other hand, it is not
relative to the \co{I}.  Even if we might say \wo{I feel nice}, it is only a
feeling \co{I} {\em have}, not something \co{I} am. (Contrary to the case when
saying \wo{I am nice}.)

Thus body, although most intimately \co{mine}, appears for \co{reflective
  attention} as foreign, as \co{external} as all the contents of the \co{ego} --
\co{mine}, yet impersonal.  More generally, all the \co{signs} of \co{actuality}
are relative to such an impersonal feeling of vitality, of life -- it is not so
much {\em my} life, as life in general, even if it is actually my experience.
The \co{mood} of vigor and vital strength is the feeling of \thi{my {\em life}}
rather than of \thi{{\em my} life}. Vital feelings signal flow or ebb, increase
or decrease of life energy, and it is this life energy, only, so to speak,
accidentally seated and felt in one's body, which is their primary correlate.

\wtsep{body -- also, horizon of action} \noo{ \co{Actions} are not correlated
  with, are not expressions of an instantaneous \co{subject}. They are vital
  functions of an organism, of a whole body or else of our controlling and
  manipulative abilities. }

\pa The \hoa\ is marked by one's body and if we were to play the games with
words, we could say: body is the \co{sign} of the \hoa.  Body anchors one in the
\co{actuality} and, by this very token, encircles the horizon of one's
\co{action}.  The horizon of \co{action} can be taken as synonymous with the
\hoa.  \co{Action} is not merely an \co{act}, an \co{immediate} \co{reflex} or
other minute movement -- of body or mind -- consummated in a single moment.
\co{Action} unfolds in the entirety of the \hoa, it addresses all, or several of
the \co{actual} aspects. From the point of view of \co{attentive reflection} we
may always say that \co{action} consists of a series of \co{acts} but one can
equally well say the opposite: an \co{act} is an \co{aspect} of an \co{action}
(unless it is an entirely spontaneous, that is mad, outburst unrelated to
anything in its vicinity).

A single \co{object} is a correlate of an \co{act}. Similarly, a \co{sign} of a
\co{complex} is a correlate of a single \co{act}. However, a \co{complex}
itself isn't merely a correlate of a single \co{act}. It may be a correlate of
an \co{action}, one can manipulate it, act upon it for a specific purpose, one
can think and reason about it, assemble or disassemble it, in short, bring it
under one's control.  \co{Objects} are under one's control only to the degree to
which they are parts of \co{complexes}. \co{Object} itself, as a purely
\co{immediate} given of consciousness, and considered only as such, appears in a
somehow impulsive fashion, \refp{pa:madSpontaneity}.  It emerges for no apparent
reason, {\em ex nihilo}, and offers consciousness only the
\co{immediate} alternative: yes-no, take-avoid, accept-reject. Although object
serves as a paradigm of the controllable, taken in itself, it is not. It
becomes so only when seen in the broader context of \co{visibility}, when it is
seen as a \co{complex} or part of a \co{complex}.

Just like sensations, limited to the \co{immediacy}, involve the responsive
attitude choosing between the bare alternative of pleasant/unpleasant, yes/no,
so do \co{impressions} and \co{moods} involve immediately a response.  Their
unifying role is not only that they present us with a totality rather than its
parts, but that this presentation involves at the same time a reaction.  Fear
can be felt before its object is \co{recognised}, the source of satisfaction or
irritation may remain unknown even when the \co{impressions} are felt clearly
and intensely. Often, though not always, the reaction precedes \co{recognition}
of its reason.  To the extent the \co{complexes} are also given, \co{moods} and
\co{impressions} disclose them through the unified \co{signs} which, at the same
time, are one's reactions to them.  \citet{When I use the word \thi{feeling} in
  contrast to \thi{thinking}, I refer to a judgment of value -- for instance,
  agreeable or disagreeable, good or bad, and so on.}{MHS}{I;p.49\noo{Jung}}


\wtsep{control, more...}

\pa \co{Concepts}, unlike \co{impressions}, lack this reactive aspect.
Emphasizing the complexity of the \co{complexes}, they increase their
\co{dissociation}, their isolation and, consequently, indifference.  At
the same time, however, the resulting \co{externalisation} of distinct aspects
offers us the possibility of manipulation and control.\noo{We see here
  inadequacy of identifying the synthesising function with emotions and the
  analyzing one with thinking. There are people of quite an emotional
  intelligence; there are demagogues with quite a skill of grasping the feelings
  of the masses. The power of such people to control and manipulate situations
  is based on the analyzing skills, on the abilities to distinguish and relate
  various aspects of a \co{complex}. However, it is hardly appropriate, in
  general, to call such skills \wo{intellectual}.}  It is typical that what one
requires from a \co{concept}, and why one prizes it as something higher or
better than an \co{impression}, is the more \co{precise} differentiation of its
components and their relations.  \co{Precision} is the result of
\co{distinguishing} brought to the limits of \co{actuality}, into the closest
possible vicinity, into the \co{immediacy}. Unambiguous \co{precision} -- the
mark of instantaneous \co{immediacy} -- \co{founds} the possibility of control,
but is not yet synonymous with it.  The crudest form of control, brutal physical
force, conforms to this claim in that it can only be exercised on the
\co{actual} \co{objects} in the immediate reach.  \noo{Its eventual goal, in the
  whole of philosophical tradition, has been to remove the distinction between
  the \co{sign} and the signified.}  \noo{It is a bit odd that {things} of
  perception are actually the basic paradigms of such a closeness,
  indubidability and clarity. They too, because of being determined by the lack
  of the \co{sign}-signified distinction and because of the ``small enough''
  size, fit within the narrow scope of our \co{actuality}, can be moved and
  manipulated.}

%\pa
Control is most naturally associated with the intellect, its source being the
\co{dissociating} and \co{externalising} effects of \co{attentive reflection}. 
Control -- and this involves also some purposefulness -- requires individual
\co{objects} to be \co{dissociated} and available for rearrangement. But it
arises only with the multiplicity of \co{objects}, when they are included in a
\co{complex} assigning to each its place and function. This \co{dissociation} is
exactly the aspect which \co{moods} and \co{impressions} lack and thus, although
addressing the same \co{complexes}, they end up at the opposite end to
\co{thoughts} and reasonings.  Eventually, also the control of reason over
emotions amounts to approaching a \co{complex} through its complexity rather
than unity. But as one should know very well, such a control is seldom as easy
as control of the single \co{objects}. That is, control is the easier, the less
complicated, the more minutely determined \co{complex} it tries to control. A
single cup is in my complete power in that I can, at any given moment simply
smash it. It would not, however, be so easily controllable if I wanted to change
its painting or shape, which would require viewing it as a rather complicated
\co{complex}.  Likewise, lower emotional phenomena like sensual lust or even
pain, can be controlled by the will, that is, by \co{reflection}, to quite a high
degree. Often a mere change of the focus of attention can eliminate them.
\co{Moods} and emotions, on the other hand, arisen by and making present much
more complex situations, require much more analysis which only seldom brings
about the control which could satisfy a true Stoic. \noo{or an average psychologist.}


\subsub{Transcendence}
%Control - but all slips out

\pa A \co{complex}, emerging within the \hoa, appears as a single \co{object}
which, in addition, consists of parts and properties which themselves can be
seen as \co{objects}. As an \co{object}, it appears as \co{external} -- it is
not me.\ftnt{Everything we are saying here about \co{complexes} applies equally
  to the specific \co{complex} of \co{ego}. I am not my \co{ego} -- I have it.
  It is also \co{transcendent} in the way all \co{complexes} are, in the way
  everything superficial or \co{actually} formalised is, as something foreign
  (even if \co{mine}) and irrelevant (even if practically significant).}  But
its \co{transcendence} is not exhausted by the simple \co{externality}.  An
\co{object} can suddenly reveal a new side which, as long as one merely focused
on its mere fact of being there, remained hidden.  We are thus also
aware of the difference between the \co{actual sign} and its correlate; they do
not coincide. Perceiving a house as a \co{complex}, one sees its front but also
knows that it has a back-side. One knows it, that is, it is a part of the
experienced totality.  Yet, these two aspects are not present in the same way,
and one is aware of this difference: one knows that there is \co{more} to the house
than what one is \co{actually} seeing of it.  In the same way, even if one sees the
whole room at once, one knows that one does not see all its aspects equally well.  Some
are sharper, some dimmer, some are closer, some further away. And the same again
with a definition: one grasps the defined concept, has it in front of one's eyes in
its intuitive totality, but one knows that one does not \co{actualise} all the details,
that its complexity exceeds the \hoa, that it has a potential which can be used
or explained only through a lengthy labour. The \co{transcendence} of a
\co{complex} involves thus the \co{more} which is merely indicated by the
\co{actual sign}.  This \co{more} is not, in itself, anything qualitatively
different. It consists of \co{objects} and \co{complexes} of other, possible or
past, \co{experiences}.
%
\thesisnonr{\label{th:more}The \co{more} of a \co{complex} consists of  
other \co{complexes}.}

The difference between the \co{complex} and its \co{more} concerns only the way
of appearance -- the former is \co{actual} and the latter can be, but is not
now.  The \co{more} is not essential \co{non-actuality} but only something
which, accidentally, happens not to be \co{actual} now.  This quantitative
\co{more} constitutes the \co{horizontal aspect} of the \co{transcendence} of
\co{complexes}.  It may be \co{more} of the same \co{complex} which is not
\co{actual} at the moment, or it may be \co{more} \co{complexes} to which the
\co{actual} one refers along some among the infinity of possible relations: as
its cause or effect, as its predecessor or successor, as its part or its whole,
as its motivation or purpose.  No matter how far one follows this dimension of
\co{transcendence}, one never encounters anything but more \co{complexes} or
more complex \co{complexes}.  Eventually, one may reach the first or the second
antinomy of pure reason, where thinking in terms of \co{complexes} and their
relations must stop: there is nothing \co{more} left and to the extent we assume
that there is, it can no longer serve us since its breadth and complexity make
it inaccessible to the \co{actuality of reflection}, the finitude of reason.
%
In this way, but also {\em only} in this way, the \co{more} of a \co{complex}
tends towards its inexhaustibility, appears as the potentiality of an
\co{external object} to disclose ever new and possibly unlimited number of
aspects, sides, relations. This inexhaustibility is the ideal limit of \co{more}
and we will return to it in a moment. 


\pa The difference between the \co{actual sign} and the signified \co{complex},
the difference experienced now between the \co{actuality} and
non-\co{actuality}, marks also the experience of temporality.  The \co{more} of
a \co{complex}, the surplus hiding behind its given surface, is actually hiding
in its temporality. All \co{complexes} are temporal in the sense that they
emerge as unity of multiplicity, as \co{actual objects} which are not {\em
  fully} present, which have something \co{more}.\noo{Heidegger's temporality,
  based on the \co{projects} casting things and situations towards their purpose
  and destination seem to me a particular, if ingenious, case. A baby, seeing
  its mother approaching and leaving, must be involved in a temporal
  apprehension -- without having any projects or intentions. Stumbling
  accidentally onto a chair, which yields and moves a bit, must arise the
  \co{re-cognition} of this chair having, if nothing else, the potential of
  occupying different positions. True, it is a construction, and the temporality
  implied rather latent and not reflectively experienced. But it does leave a
  mark for the rest of the life; the mark of seeing the chair having the same
  potential of appearing through different \co{signs} as all other things will
  always have.}  They are temporally -- though only temporarily --
\thi{stretched} and it is not merely one of their features but an indispensable
\co{aspect} of their appearance as \co{complexes}.\noo{This might remind of
  transcendental constitution if one were to take the temporality as a necessary
  condition of such an experience. But it is not a condition -- it is a
  constitutive \co{aspect} of the experience.  It is hair-splitting but \ldots
  temporality is not any {\em a priori} form, independent from experience. It is
  a part of experience, it is the experience of the \co{actual} and the
  \co{non-actual} aspects of a \co{complex}.  Moreover, this temporality is {\em
    noting else} than the very \co{complexity} of \co{complexes}, i.e., an
  \co{aspect} of the same \nexus.}

The \co{horizontal transcendence} of a \co{complex} is thus not mere \co{externality} 
but \co{more}, which involves also temporality stretching  beyond
\co{here-and-now}. 


\wtsep{vertical...: the more the less}
\pa
 The \co{actuality} 
constitutes the \co{vertical aspect} of the \co{transcendence} with respect to 
the level of \co{immediacy}. \co{Complexes} give the single \co{objects} the
context in which the  arbitrariness of the mere \co{that it is} 
may find the first form of meaning: a purpose, a reason,  
a relation to other \co{objects}. 
Now, the \co{complexes}, the {objectified} 
relations of the level of \co{actuality}, reveal an analogous 
\thi{meaninglessness} indicating the \co{vertical dimension} of \co{transcendence}.

It seems to be a popular way of expressing dissatisfaction with the
\co{representational} form of \co{reflective} thinking and the mere
\co{objectivity} of \co{dissociated} \thi{substances} -- referring everything to
some \thi{context}.  There are innumerable variants which it would be impossible
to review here.  On the one hand, it has the charming appeal by introducing,
although often only by implicature, the \co{subject}, since context is hardly
something which can be determined in purely \co{objective} terms.  On the other
hand, since it is actually impossible to determine what context might possibly
mean and what might possibly constitute a legitimate context for anything, one
tends to extend it as far as \ldots everything.  All pointers to coherence,
totality, \thi{the whole}, are ways of \thi{contextualisation}, along with the
more mundane attempts to put everything in the \thi{historical context}, the
\thi{social context}, the \thi{inferential context}, the \thi{context of usage},
etc.  Admirable as many of such attempts may be, they suffer from the
inaccessibility of the eventual \co{more}. \thi{Context} tries to function as a
surrogate for the negative aspect of every \co{recognition} and
\thi{contextualisation}, starting with the \co{reflectively dissociated}
\thi{atoms}, meets almost immediately the combinatorial barriers,
\refp{pa:pragmaticPositive}.  If we have $n$ atoms and, in principle, any
combination thereof might be a context, we get $2^{n}$ contexts.  Take half of
them, one-tenth -- as $n$ increases, the number of contexts becomes very quickly
unmanageable.  Although context indeed points toward something endowing the
\co{object} or the \co{complex} with a purpose, if not with meaning, it is hard
to imagine how taking such (unmanageable number of) contexts into account is
supposed to help understanding a given \thi{atom} $x$. The only manageable, if
not also the only important, question concerns which context to choose (which is
nothing else but the question about the limits of $x$.)

\pa\label{pa:smartStupid} The \co{more objective} contexts to investigate, the
less understanding of the investigated issue; the \co{more} information
available, the more difficult to find any relevant, not to mention valuable,
information; the \co{more} 
ambitious professor of the more \co{imprecise} subject (psychology, sociology,
literature), the \co{more} attempts at 
mathematical \co{precision} in his research; the more persons with higher education,
the more stupid and less knowledgeable each one of them. This general law
 -- \thi{the smarter, the more
  stupid} or \thi{the \co{more}, the
  less}; the \co{more} it accumulates, the less it obtains -- 
%in a slightly parodic manner and only in its social dimension, 
\noo{\bf The smarter, the more stupid; the more serious, the more funny; the more,
  the less}
underlies the whole life and expansion of \co{objectivistic} insatiability. 
\wo{Look only at all the festivities of the intellect: these
  conceptions! These discoveries! Perspectives! Subtleties!  Publications!
  Congresses! Discussions! Institutes! Universities! And still: stupefying.} As
an example: \citet{Precision, richness, depth of the language in all
  expositions, not only the primary, but also secondary ones, or even those on
  the edge of mere journalism (like literary criticism) are worthy highest
  appreciation. But the overflow of richness exhausts the attention, and so the
  increased precision is accompanied by the increased distraction. The result:
  instead of increased communication, increased misunderstanding. [...] In the
  field of all discussions penetrating western \gre{episteme}, you will never
  hear a single voice which would start with \thi{I do not know exactly\ldots am not
    familiar with\ldots did not read through\ldots who could remember all
  that\ldots there
    is no time to read\ldots I know something, but not quite\ldots} Yet, exactly from
  that one ought to start! But who would dare?}{Diaries66}{1966:XIX} Instead,
proving trifles, we let them parade as the genuine truths until, eventually,
they start indeed to seem the same. And then more seriousness can only breed
more ridicule, more smartness only more stupidity, more achievements only more
despair and more truthfulness only more falsehood.
The smarter, the more stupid; the \co{more}, the less. 

\wtsep{over to ME (subj-obj)}

\pa \co{More} never sums up to any \co{unity} and, chasting its ideal limit of
inexhaustibility, keeps expanding into \co{more} and \co{more} comprehensive
\co{totality}. But, the \co{more} comprehensive -- the less comprehensible.  The
inversion reveals a lack. The \co{more} intensely one focuses on the \co{actual
  complexities}, the less sense and meaning one finds in them.  The more one's
understanding approaches the self-secure enlightenment of scientism, the
intellectual self-confidence or the safety of a bourgeois sterility, the greater
the chance that one may wake up as Gregor Samsa in Kafka's \btit{Metamorphosis}
-- in the known, safe surroundings, in the same orderly house, in one's own bed,
but transformed into a cockroach scared, or rather merely pacified, by the
inexplicable event of a meaningless loss of one's so far obvious and unproblematic
identity. At the limit of \co{more} one encounters~\ldots oneself, even if often
only as one's own caricature. 

The subject is encountered at the limit of the world and, perhaps, only
there. \citet{The subject does not belong to the world: rather it is a limit of
  the world.}{Tract}{5.632 [The statement seems to apply most adequately to
  \co{myself}, although Wittgenstein's \thi{subject} or \thi{philosophical self}
  \citefib{is not the human being, not the human body, or the human soul, with
    which psychology deals, but rather the metaphysical subject, the limit of
    the world -- not a part of it.}{Tract}{5.641} \co{I}, unlike \co{ego}, is
  its world by being its limit, the horizon within which life unfolds, as we
  will describe shortly.]}
%
\noo{  -- but \co{I}, being its world and human
  being, is not the ultimate \thi{metaphysical subject}, for \citef{man
    infinitely transcends man}{Pensees}{VII:434}, as we will discuss in the
  following Section, and then in Book III.}
%
But although never \co{actually} encountered, it is \co{present} in the world
from the very beginning.  The 
\co{objectivity} of a \co{complex} consists in its being given as an \co{object}
in \co{actual experience} and, moreover, in the fact that its \co{more} is just
more of the \co{actual}, \co{objective} aspects. However, to the extent that a
\co{complex} is not given \co{immediately} in its full complexity, it bears
always a mark of \thi{subjectivity}. This is no longer the \co{subjectivity} of
an ideal, purely \co{actual subject}, but a much more genuine \thi{subjectivity}
understood as that which brings non-\co{actuality}, and eventually also
\co{non-actuality}, into \co{experience}.  The status of \co{complexes} as
\co{external} and independent from us is much less evident than in the case of
simple \co{objects} encountered by \co{reflection that it is}. 

This aspect of \thi{subjectivity}, the presence of the \co{ego complex}, becomes
manifest, for instance, in the questions like: Can I be sure that it is one and
the same object? If I leave and come back, how can I know that it is the same?
It is a bit hard to know what \wo{to know} means in such questions, but we sense
that the lack of \co{immediate} certainty is taken as the possibility of error,
that is, \thi{subjectivity}. I -- my sight, my memory, my understanding, any of
my faculties, in short, my \co{ego} -- can be mistaken. Certainly, in some
cases, one can be mistaken. However, in most cases, one experiences this
identity with such an infallible certainty that, if one were to doubt it, one
could not be certain of anything. As Wittgenstein says: the burden lies here on
those asking such questions -- the burden of explaining what they would consider
as knowing.\noo{Of course, from our perspective such questions do not posit big
  problem. We are not constructing the world from a dispersed multiplicity of
  independent \co{actualities}. On the contrary, it is \co{actualities} which
  emerge from a unity of the background of \co{experience} which, in terms of
  objective time, is a lasting unity. A \co{complex} is \co{cut from experience}
  which stretches beyond the \hoa, and enters it only through particular
  \co{signs}. By this very token, it is precisely something which has been
  delineated as extending beyond this horizon. It is the \co{actual experience},
  the \co{immediacy} of an \co{object}, which is secondary, which is
  \co{founded} upon a prior \co{distinction} of a \co{complex}.}  We will not
deny the possibility of doubt but only observe that it arises exclusively from
the assumption that the only certainty can be obtained in the \co{immediacy} and
that everything which \co{transcends} its \co{objective} \thi{givens} threatens
with error, that is, \thi{subjectivity}. For if all that is real are
\co{immediate objects}, then even the relations between them, especially
the relations across time, turn out to be \thi{subjective}.

\pa \co{Ego} is the \co{aspect} of subjectivity within the \hoa. But as one asks
for \co{more and more}, as the \co{totality} of \co{complexes} extends
beyond the limits of one's possibilities and, gradually, begins to dissolve in
the flux of \co{experience}, \co{ego}'s status becomes more and more dubious.  The
ideal limit of \co{more}, the ideal \co{totality} of \co{complexes}, \co{this
  world} as a \co{totality} of things, is also the place where \co{ego} begins
to dissolve in the \co{experienced chaos}.  The obsessive rigidity of a
systematic organisation, when carried to the extremes, creates eventually mad
pandemonium, whether on personal or social scale.  The ever renewed and never
accomplished project of control encounters \co{more} in the most dramatic
fashion.  When absolutised and driven beyond its proper limits, it makes one
acutely aware of the uncontrollable \co{more} which lurks in the depth of
everything which one believes to control.  The disappointments of the projects
of total appropriation, the failed attempts to control the inexhaustible
\co{more}, throw us back onto ourselves -- for the limit of the \co{horizontal
  transcendence} is always but a shadow of its \co{vertical aspect}.

The subject, implicitly \co{present} all the time, is encountered by
\co{reflection} at the limit of {objectivity}.  But the subject encountered thus
at the limit of the world is encountered only as its own absence, as the lack of
something which should be there but which is not.  For it is neither the
\co{immediate subject} of a \co{reflective act} nor the \co{egotic complex} of
\co{dissociated} properties.  It is something personal, something which never
becomes \co{precisely visible} within the \co{totality} of \thi{objective}
determinations and \thi{subjective} \la{qualia}. Insatiability -- whether of
Hegelian conceptualisations, of positivistic optimism, of sociological
eschatologism, of progressive scientism or of those who, bored by the abundance
of all past novelties, await impatiently for the next novelty which will finally
cure their boredom -- all forms of insatiability breed, eventually,
dissatisfaction, the feeling of unfulfillment and incompleteness, of broken
promise, perhaps nihilism. From these ashes, when the impassable limit of
\co{more and more} has been left \co{below}, there emerges the more genuinely
subjective, the personal, the Nietzschean or Kierkegaardian.

\noo{
  
  \pa Reaching and feeling the limits, it is no surprise that one may be tempted
  to announce the end of philosophy, the end of reason, to renounce
  \co{conceptual} thinking for the sake of literature, poetry or mysticism or,
  as one likes to say, for \thi{thinking otherwise}. After all, they are
  concerned with life and what else was the original motivation? Such acts may
  witness to a deep personal choice and should not be disrespected. Yet, they
  may also reflect the despair over reason's inability to embrace everything
  which arises when everything is misunderstood for a mere \co{more}, a mere
  \co{complexity} of \co{objects} and problems which all have to be analyzed
  down to the minutest detail. Such an everything, an inexhaustible \co{more}
  unveils the finitude and end of our skills. But is it {\em the} end?
  
  It just so happens that thinking requires \co{representations}, it is the art
  of manipulating and arranging \co{signs} which \co{represent}.  And \co{signs}
  are just \co{signs} -- not things themselves.  What makes thinking \thi{good
    thinking} is not a narcissistic self-complacency with its own coherence,
  consistency, inter-textual contextuality and what not, but that it manages to
  stay in touch with things its \co{signs} signify.  It is, in fact, often as
  difficult as it sounds trivial.  One may try renouncing \co{signs} and
  \co{representations}, but then one does not start \thi{thinking otherwise} but
  simply stops thinking and starts doing something else.  Even if one
  \thi{thinks otherwise} that the original temporality, life, experience is an
  ecstatic unity, one still buys oneself a watch and goes around one's own
  business. If thinking about such daily trivialities is not satisfactory then,
  perhaps, one should start thinking about something deeper rather than trying
  to \thi{think otherwise}.
  
  \pa The personal, the \co{subjective} in the sense of the presence of
  \co{non-actual}, is also the sphere of meaning.  The questions about the
  meaning (which never can be reduced to the purpose) of particular things and,
  eventually, of their totality, about the sense of things and of doing things
  bring invariably the questions about \co{myself}, about the place of
  \co{myself} in the world, about other peoples' influence on and relation to
  me. These are the questions which reveal the vertical dimension of the
  transcendence of \co{actuality}.  They are not merely naive attempts to ask
  about the \thi{meaning of the world and life}. They are most natural questions
  which become the more pressing and relevant (even if not explicit), the more
  one directs one's attention to the \co{complexes} and their \co{objective}
  relations, the more one's \co{reflection} focuses on things within the world.
}

%%%%%%%%%%%%%%%%% MINE

\subsection{Mineness}\label{sec:levelC}
\pa\label{refl:C}
The confrontation with the chaos of \co{experience}, with the limitations of my
\co{ego}, of my skills and powers, suspends the importance of objective facts
and relations and engenders the \co{reflection that
  I am}.  It marks a breach in the continuity of being presenting \co{myself}
in as astonishing a light of \co{dissociated} independence as the realisation
\co{that it is} did earlier with a simple \co{object}.  \co{Reflection that it is}, the
mere observation \thi{that\ldots} of \co{object}'s \co{immediacy},
\co{dissociates} it and presents it as being \thi{on its own}.  \co{Reflection
  that I am}, the mere observation \thi{that\ldots} of my \co{presence}, does the
same with me.  In so far as I am considered in the \co{actuality} of this
\co{act}, that is, in so far as I am considered as its \co{object}, I appear as
isolated from the world, from any origin, I am alone, \thi{on my own}. By the
same token, it seems, I am free, absolutely, unreservedly; I emerge {\em ex
  nihilo} with the same ungrounded arbitrariness as an \co{object} appears for a
purely \co{actual subject} -- it is, but might not be, and there is no apparent
reason for its being rather than not being, or for its being so rather than
otherwise. On my own  -- as free
as arbitrary, as unconstrained as meaningless. 

This \co{dissociation} from the world, similar to any \co{object}'s, is
not accompanied by the similar \co{externality}.  Such \co{externality} belongs
still to the \co{ego} which is a collection of \co{externalised} contents. Here,
on the contrary, I know, in the immediate self-consciousness, that it is me
\co{myself} I encounter, an enigmatic site of continuous self-sameness.  The
enigma concerns precisely the sameness of \co{myself} who am reflecting in this
very moment and \co{myself} who was yesterday, years ago, and whom I am going to
be in all my future. And, surely, as long as one stays with the categories of
\co{actuality} alone, as long as one yields to the unreserved claims of
\co{objectivistic illusion}, this sameness poses an ever perplexing problem. In
particular, because one, being unable to deny the experience of sameness,
reverses the order of \co{founding} and tries, as Hume and others did, to refer it
back to, if not derive it from, the \co{visible} contents or, what amounts to
the same, to reduce \co{oneself} to the \co{visibility} of the \co{ego}.

\pa \co{Reflection} over \co{myself}, or \co{self-reflection}, is an \co{act}
which attempts to \co{actualise} its intended \co{object}, but for which this
object immediately slips away and remains a mere noumenal site.  And
\co{self-reflection}, in the immediate \co{self-consciousness}, is \co{aware} of
this insufficiency. In it one admits that one is something much more than what
can be made \co{actual} in any single \co{here-and-now}.  One \co{re-cognises}
\co{oneself} as oneself, that is, although one can never grasp oneself fully in
the \hoa, one knows that all the \co{signs} given in \co{self-reflection} point
to {\em oneself}, shallower or deeper appearances of mineness. One may discover
dark and unexpected sides, but they are all intimately {\em one's own} sides,
they are not, in any way, alien, \co{external}.  All the \co{signs} signify some
characteristics of \co{oneself}, they appear as {\em one's} characteristics, as
recurring themes of {\em one's} character, personality, being.

In \co{self-reflection} one \co{experiences} \co{oneself} as transcending the
\hoa, but merely as some noumenal site of identity.  One knows one's identity
which extends over time -- not because one managed to \co{re-construct} and
comprehend it but simply because one knows it. \co{Experience} of this
identity precedes any particular \co{act} of \co{reflection} and extends beyond
it. It is \co{experienced} in the course of one's whole life and does not depend
on one's reflecting over it or not. One knows it long before one
\co{reflectively} thinks about it.  If \co{experiences} can transcend the \hoa\ 
and be essentially \co{non-actual}, then so can the experiencing being. In this
form of \co{reflection}, one \co{represents} oneself as a being whose identity
is not, like \co{ego's}, constructed from the \co{actual} contents, but extends
over time and can only be made present for \co{reflection} by means of more or
less inadequate indications, allusions, \co{signs}.

%\ad{Temporality}
\noo{The \wo{other-worldliness} of \co{the experience} (not only of the
  \co{reflection}) \co{that I am} is based on the experience of another mode of
  temporality. The \co{I} does somewhat participate in the objective temporality
  of things of this world but, at the same time, it is outside it, it has its
  {\em own} time, its {\em own} past and future. In so far as this time is also
  considered an objective time, this is but a result of \co{reflection}
  isolating \co{myself} as its \co{object}. But \thi{my time} is also an
  expression of \co{myself} as a temporal being, of my \co{experienced} time,
  eventually, of the unity of my being which precedes its temporal
  determinations.  }

\pa Just like the temporal scope of \co{subject} (and \co{object}) is pure
\co{here-and-now}, while of \co{ego} (and \co{complexes}) some finite and
limited time, the \co{I} of the reflection \co{that I am} is finite but
unlimited -- it has no \co{experienced} beginning nor end. My birth is something
\co{I} may be told about but not something \co{I} experience; my death is a
perpetual not-yet.\noo{ and even when it approaches \co{I} do not experience
  death but dying.} \co{I} am stretched, or better, \co{I} am stretching
\co{myself} between these two limits, both real and yet ideal, since forever
inaccessible.

They are \co{actual} only when projected onto the objective time, as particular
points on its line.  The apprehension of the finitude of my being (here
of its temporal span) emerges thus as the result of imposing its ideal
end-points onto the infinite line of objective time.  The infinity of this line
is, of course, not experienced but only thought, and that only in terms of
\co{more}, that is, as potential infinity. 
% -- nobody thinks the actually infinite but one can easily say: no matter how far
% the time has gone, it will go still further.
But one can perform the opposite operation of mapping the infinite line onto, for
instance, finite but {\em open} interval, or on a finite circle (as in
figure~I:\refp{pa:stages}, p.\pageref{fi:stages}).  The ideal end-points (which do
not belong to an open interval, while on the circle become the same pole) become
then the image of the infinity -- the point in infinity. This point functions
thus as the beginning, and the end, of the supposedly infinite line. We could
say that this folding of the objective line onto finite circle represents
\thi{relativisation} of the objective time to the temporality of my being.


The question ``What was before the beginning of the world?'' is as unavoidable
from the perspective of objective time, that is, at the level of \co{actuality},
as it is unanswerable.  Potential infinity of the objective time is the backbone
of the proof of the antithesis of the first antinomy. \co{Objectivistic
  illusion}, being based on the unwarranted extrapolation, can not admit any
limits -- there is always \co{more} and time must be infinite. On the other
hand, the intuition of \co{origination}, of the ideal \co{unexperienced}
beginning and end is the source of the respective thesis.  The antinomy reflects
thus again the mixture of levels, where the \co{objectivistic} infinity is
confronted with the \co{experiential} finitude, the unimaginable beginning-end
with the intuited \co{origin}.  Temporality emerges only from the
\co{separation} of an \co{existence}, that is, it has an \co{origin}, though not
a beginning nor end. Beginning and end are only {objectified} images of the
\co{origin}.  This time which has \co{originated} somewhere but which has
neither beginning nor end, the finite yet unlimited time, is the temporal
context of the experience \co{that I am}.

\subsub{The signs}
The experience \co{that I am} is \co{an experience} suspending the unquestioned
validity of the objective world.  The \co{signs} of \co{mineness} are no longer
relative to specific organs nor even to the whole body; they are no longer
sensations nor vital feelings; they are no longer localised, are not narrowed to
the context of \co{actuality}. Laughter of a happy man is different from the
laughter of a desperate man, even if they both laugh at the same things.  The
\oss\ of \co{mineness} are kind of feelings; not, however, mere \co{moods}
related to the \co{actual} situation, but feelings modifying its perception,
\thi{soulful} feelings, relative to the \co{quality} of one's life or, if you
prefer, of one's \co{soul}.

%\newpa

\ad{Original signs}%\co{Emotions}
% fix a bit: are qualities of life the \oss, or only what the 
% \oss signify ?
The \oss\ at this level are what we will call \wo{\co{qualities of life}}, or
shortly \wo{\co{qualities}}.  Most vaguely, these are just feelings -- however,
not ones concerning a particular thing, situation or person, but the feelings
concerning some \co{vague unity}: of {\em my} life, of the world, of life.
Phenomenologically, one might perhaps say that \thi{my life} is their noematic
correlate. But it is rather so that they are given as qualities penetrating the
horizon of my life, not as its more or less accidental properties, but as the
qualities constituting the very character of this horizon.  They are not
\co{signs} of anything particular, of any things or situations as
\co{impressions} were; they are not relative to my \co{ego} but to \co{myself};
they are not specific, situated \co{moods} but \co{qualities} of the whole:
life, my experience, the world. We notice easily such a \co{quality} with a
child, often even before it starts talking. (The absence of verbal communication
seems to increase receptivity for other kinds of messages.) The whole future can
be seen -- not, of course, any details concerning the development, career and
the like, not any specific events of the future but \thi{the whole future}, the
\co{quality} of the person, \noo{ perhaps, some \vo{virtual} character traits}
the \co{quality} of his life.\ftnt{For instance, \citef{Plotinus foretold also
    the future of each of the children in the household: for instance, when
    questioned as to Polemon's character and destiny he said: \wo{He will be
      amorous and short-lived}: and so it proved.}{lifePlot}{11.}}  This strong
impression we often get from children becomes, with the adults, weakened by the
noise of all more specific features, habits and norms but, our claim goes, it
remains the same \co{quality}.

\pa A feeling of peace may arise in a particular situation and be experienced as
relating to this -- peaceful -- situation only.  But \co{qualities} like
\thi{peaceful}, even if uncovered by \co{moods} and \co{impressions}, are not
reserved for particular situations within the world.  The same \co{qualities}
may appear in a more intimate, deeper and hence more intense fashion as the
\co{qualities of life}. There is a fundamental difference between a feeling of
elation situated in a concrete situation and the joy of life which is only
\co{actualised} in a particular situation.  What distinguishes them is that the
latter lack any \co{objective} correlates, any \co{complexes} which might be
identified as their site, their proper origin.  \co{Complexes} are here only
sites where such feelings are \co{actualised}, not by which they are caused.  A
feeling of peace, in the sense of a \co{quality of life}, as a \co{sign} of
\co{mineness}, even if experienced in a particular context, is not limited by
it, is not experienced as exclusively a quality of this moment but, on the
contrary, as something which is merely \co{actualised} in it and which is a much
more solid quality of something deeper.  A feeling of joy, as a \co{quality of
  life}, is not the vital feeling of elation and vigour which may pass or change
into its opposite in the matter of hours or minutes. It is a calm feeling of my
life which, through all the variations of vital feelings and \co{moods}, through
all the variations of situated joys and sorrows, victories and defeats, unveils
a theme which underlies and as if surrounds all of them.  The \co{qualities of
  life} constitute as if a deeper layer of \co{moods} and \co{impressions}. They
are \co{experienced} underneath variations of different situations which, in
themselves, give rise to very different \co{impressions} and \co{moods}. Going
far away, to an exotic and unfamiliar corner of the world, is naturally
accompanied by an excitement and openness to the encounter with something -- as
the expectations go -- completely new. But after some time (perhaps a week,
perhaps a year), when the storm of novelties and initial \co{impressions} has
calmed down, there arises a specific \co{mood} of something familiar in the
midst of all this unfamiliarity.  One notices that, as a matter of fact, in
spite of all the differences and novelties, one has not traveled that far. Even
if the mood of the life \thi{out there} is very different from that at home, the
mood of my life seems to be only slightly affected by it and, at the bottom,
remained the same. Variation of lower \co{moods} and \co{impressions} uncovers
often deeper \co{qualities}. In general, they need not be so \co{visible}.  They
are constant themes which remain unthematic but are \co{experienced} as if in
the background of \co{actual} situations, as mere modifiers of the particular
\co{moods} and \co{impressions}. Laughter of a happy man is different from the
laughter of a desperate man. Being the \co{qualities} of {\em life}, they have no
particular object but embrace all objects.

%\pa
Having no particular objects, they may be \co{clear} and \co{recognisable}, but
are certainly \co{vague} and hardly definable.  This undefinable \co{vagueness},
this lack of any \co{objective} correlates gives them a calm character.
Even if the \co{actual} feelings expressing the \co{quality of life} are
restless and confused, the \co{quality} itself is 
not so. Because it is \co{experienced} as given -- not in the sense of \co{actual}
givenness of an \co{object}, but in the sense of givenness of something which is
greater than any particular \co{object} and situation, which is greater than
what can be controlled and influenced at the moment. Even if encountered in a
particular situation, it is not limited to this situation, it does not aim
at any \co{action} or expression -- it merely dawns on me. To the extent the
\co{quality} has a negative character, it is experienced with resignation or
wish that it be different. If it has a positive character, the same kind of
calmness becomes a thankful acceptance.  Any possible reactions, whether sudden
outbursts, protests or satisfaction, are at most \co{actual} expressions motivated by
such underlying feelings but not their proper \co{signs}. Calm reaction can express
certainty or resignation, laughter can express acceptance or contempt, almost
any \co{actual mood} and reaction can be associated with various deeper feelings
and, eventually, \co{qualities}. And it is only this deeper association, this
undefinable \co{rest}, which, on the one hand, modifies the \co{actual signs} so
that one's laughter is unmistakably distinct from another's and, on the other
hand, gives the \co{actual signs} their more profound meaning reaching beyond
the mere \co{horizon of actualities}.


\pa Most abstractly, the \co{qualities} might be divided into feelings of
spiritual\ftnt{The word \wo{soulful} would seem more appropriate here if it did
  not carry all too emotional connotations.} gratification, peace, on the one
hand, and sadness, mourning, on the other.  To most people the \co{quality of
  life} makes it worth living, but to some it does not. The immediate, \os\ of
the former is the simple fact that one does continue living, of the latter --
suicide.  But these are abstractions. Every person, every life has its unique
\co{quality} and it is only up to \co{reflection} to decide how far it wants to
abstract or distinguish, what it wants to consider as analogies and what as
differences.  Every man carries with him the \co{quality of his life} which can
be sensed and experienced (even if not contained in a precise concept) by
himself and by others. There are people whose life {\em is} light or shallow,
and whose life {\em is} sad or tragic. The \co{quality of life} of Ivan
Karamazov, sharply independent, confusedly intellectual is something very
different from the peacefully submissive, warmly open \co{quality of life} of
Alosha. Their thirst and search have quite different \co{qualities}, even if,
perhaps, both \co{thirst} eventually for the same.\noo{ Two persons can be
  equally stupid, comparably eloquent, may have similar laughter, etc., etc.,
  but all these similarities do not change the fact that one laughs in his own
  way which, in fact, is not the way of the other, that every single thing done
  by the one obtains different character than it would if done by the other.  }
Every person has, besides recognisable and repeatable features, besides some
character traits which can be shared with others, \co{above} all that which can
ever be captured by the \co{actual} look, something -- \fre{je ne sais quoi} --
which gives all these features a uniquely personal touch.  \noo{A good
  psychologist should not perhaps notice that, but every human being should.}
\noo{Empiricists would say that it is only a unique combination of the common
  traits and, certainly, the \co{quality} may be to some extent influenced by
  the course of life and choices one makes in it. But }

Unlike vital feelings, \co{qualities of life} are not \co{actual} expressions of
animated, alive energy, and not even of life in general, but of \co{my} life.
Life in general may have all kinds of alternative qualities; it may be tragic
{\em or} comic, meaningful {\em or} wasted, hard {\em or} easy, intense {\em or}
peaceful.  One's life, too, can have different qualities for which we need
different words, but to the extent these signify the \co{quality} of one life,
they are joined not by \thi{or} but by \thi{and}. One's life can be both
tragic {\em and} comic, but then this conjunction, as well as the particular
ways of its manifestations express, or rather only try to express, this 
particular and unique \co{quality}. The whole \co{concreteness} of the
\co{quality} lies here in this \thi{and}, in the peculiar way it
connects the descriptions which, when applied to the \co{actual objects}, would
appear contrary. 

\ad{Reflective signs} Verbal statements of such qualities involve already a
\co{reflective} attitude.  Very roughly and approximately, they are preceded by
\wo{I am\ldots} rather than \wo{I feel\ldots} \wo{I feel nice} suggests a
situated context in which this feeling arises. We do not say \wo{I am nice} in
the same sense.  \wo{I feel dissatisfaction}, too, refers implicitly to a
dissatisfying situation. \wo{I am dissatisfied} may be said in the same sense,
but it may also be said in the broader sense of \wo{My life is dissatisfying}.
The statements like \wo{Life is difficult/restless/meaningless/etc.}, or else
\wo{Life is peaceful, even though each day is a struggle.} are usually made in
response to particular circumstances, in a specific context of paternalistic
advice to a child or an intimate complain to a close friend. They thus
presuppose a particular frame of mind, a particular mood and, often, a
\co{reflective} attitude.  But these are only possible conditions for actually
{\em making} such statements. The statements themselves are incomplete and
imperfect expressions of the \co{qualities} -- the \co{reflective signs} of the
\co{experienced} \co{quality} of \co{my} life.

We are not concerned with the {\em truth} of such statements. They are never
true in the strict, objectivistic sense. And yet, they do witness to
\co{experiences} of a different order than the \co{experiences} of
\co{complexes}, they witness to feelings of different kind than \co{moods} and
\co{impressions}.  One can claim that such statements are merely generalisations
from a series of \co{experiences} and express nothing more than their common
features. Yet, if one wants to maintain this empiristic view, one should also
explain what the supposed subject of all these predicates is and what
constitutes the need for such useless judgments. In fact, the statements are
neither meaningless nor useless. Their meaning is grounded in \co{the
  experience} of one's life and even if they do not express \co{precise}
content, they can communicate the intended \co{qualities}, they can be
understood by a sympathy which need not \co{conceptual precision} but only
sufficient indications.


Mere words  are  never sufficient to communicate the
\co{quality of life}.  Typically, one has to tell a story, perhaps a long story,
and certainly a whole-hearted story, in order to communicate verbally the
\co{quality of one's life}.  But the \co{quality} itself, the {emotion} {\em is}
\co{experienced} by everybody, even if one usually lacks words for expressing it
unequivocally and borrows the words designating primarily \co{moods} and
\co{impressions}.  And it is precisely the very fact that such verbal
expressions are always given with the {\em accompanying awareness} of their
insufficiency and inadequacy, which suggests the \co{unity} of the underlying
experience. The experience of the words being insufficient for the expression of
the \co{quality of life} is also the experience of the \co{unity} of life which
simply \co{transcends} any \co{actual} expressions.

\noo{
Such a \co{reflection} can be provoked, for instance, by what one calls \wo{the
  transiency of the world}, the fact which displays its temporality.  It takes
time before one encounters such experiences. As long as the world of a child is
stable, there is nothing provoking its reflection to focus on this fact of
stability. The stability is \co{experienced}, i.e., a child living in a
peaceful, stable world does have the experience of this stability. But this does
not mean that the child experiences the world {\em as} stable, i.e., it does not
necessarily have a \co{reflective experience} of it.  A major change in this
world, a loss, moving to another place may be the first \co{signs} reflecting
the transiency of the world.  And such an experience reveals -- to my
\co{experience} and possibly reflection -- \co{that I am}.

Often, such \co{an experience} occurs in adolescence, when one for the 
first time, with unexpected clarity realises one's presence. Jung 
describes himself, at about the age of twelve, having such \co{an experience} 
realising that \citt{now I am {\em myself}! [\ldots] Previously 
[\ldots] everything had merely happened to me [\ldots] now {\em I} 
willed.}{Jung, {\em Memories, Dreams, Reflections}, pp.32}
}

\pa The \rss\ of the experience \co{that I am} are \co{general thoughts} setting
up a distinction, often an opposition, between \co{myself} and \co{the world}.
These are no longer mere \co{concepts} summarising the properties of this or
that and the relations of this to that, but unverifiable, \co{general thoughts}
about the character of the world, on the one hand, and the \co{quality} of life,
on the other.  The distinction \co{posited} by \co{reflection} between the world
and one's life is, in fact, only an artificial extension of the
\co{subject}-\co{object} dualism to the \co{posited totality}. The properties
ascribed to the world happen to reflect only the \co{quality} of {my life} and
vice versa.  Saying something like \citet{All the world's a stage,\lin And all
  the men and women merely players}{AsYouLikeIt}{II:7 (Jaques)} or \citet{This
  world is not for aye, nor 'tis not strange\lin That even our loves should with
  our fortunes change;}{Hamlet}{III:2 (Player King)} one seems to characterise
the world, but this is only a matter of grammar. One actually states something
about the life in the world, about the \co{quality} of life.  Certainly, one
does not state any true fact, even less a fact which could be verified. But one
does not make any false statement, either -- one expresses \co{general thoughts}
about the \co{quality of life}.

Discussing \co{actual} matters -- the current political situation in~\ldots, the
recent fashion in literary criticism, the choice of furniture, the events at
work~\ldots -- one can pretend to remain half-anonymous, to keep an
\thi{objective} distance to the matter at hand which, in fact, is depersonalised
and external. But the more general statements one makes, characterising the
totality of the world and the workings of human life and affairs, the less
possibility to retain the distance and the more one unveils \co{oneself}.
\label{pa:worldislife2}
Making a statement about the world or life in general, I am bound to unveil
\co{myself}. Saying \wo{The world is cruel and hostile.}  I make so much an
\thi{objective} statement about the course of the world as about my
\thi{subjective} experience of the world. The obvious, immediate conclusion
following such a statement could be something like \wo{One has to watch one
steps not to be trapped by the world's (other people's) hostile schemes\ldots} The
former involves and unveils (if not logically implies) the latter. And in the
same way, a suspicious \wo{I am always careful in all my dealings to avoid\ldots}
implies something about the understanding of the world.  The first, apparently
paradoxical, moment of the medieval \la{memento mori} is to direct the attention
to {\em one's} life. But this turn is to be done against the understanding of
the limited and passing value of the things of \co{this world}. The universal
doubt recommended by Descartes is so much an expression of a shrewd, suspicious
intellect, as of the understanding of the world which -- who knows? -- might be
under a spell of the \thi{Evil Spirit}. We do not know, but its mere possibility
makes the world untrustworthy.

\label{pa:worldislife3} \co{Qualities of life}, the \co{original signs} of
\co{mineness} do not necessarily involve the clear split into \co{the world} on
the one hand and \co{myself} on the other.  Although \co{reflection} will easily
distinguish \co{myself} from my world, the \co{general thoughts}, too, tell one
as much about the world as about the speaker.  The more \wo{general} issues one
addresses, the more of oneself one has to unveil.  \citet{He who says \thi{life
    is real, life is earnest}, however much he may speak of the fundamental
  mysteriousness of things, gives a distinct definition of this mysteriousness
  by ascribing to it the right to claim from us the particular mood called
  seriousness -- which means willingness to live with energy, though energy
  bring pain.  The same is true of him who says that all is vanity.  For
  indefinable as the predicate \thi{vanity} may be \la{in se}, it is clearly
  something that permits anaesthesia, mere escape from suffering, to be our rule
  of life.}{Pragm}{I;p.19}

Statements \wo{Life is\ldots}, \wo{The world is\ldots}  and \wo{My life is\ldots} do
not only appear \co{conceptually} equally empty (in spite of all the existential
content they carry), they are equipotent, they say the same about the same.
They can be supported by various arguments and examples, can be related to other
thoughts, elaborated \la{ad nauseam} over a glass of beer or whiskey (and another
glass, and another\ldots)  Eventually, they tell you nothing about the world or
life in general, but only about a {\em possible} \co{experience} of world and
life -- i.e., about the person making this statement and {\em his}
\co{experience} of {life}.
\citet{The world of the happy man is a different one from that of the unhappy
  man.}{Tract}{6.43}

%%%%%%%%%%%%%%%%%%%%%%%

\noo{\pa
If needed, we will use the word \wo{\co{ideas}} to refer jointly to the 
\rss\ and \oss\ of the level of \co{mineness}. 
This word may indicate something `one gets', like when one
says \wo{He has got this idea about life}, without having 
deliberately chosen it. \co{Ideas} are not
entirely under our conceptual control, they rather reflect the quality of
our experience; eventually, they cast the shadow on our \co{actual} 
\co{thoughts} and \co{moods}. 

Obviously, there is close to nothing these \co{ideas} have in common with those
of Plato.  There is, however, an analogy with Kantian use of the term, in that
\co{ideas} are \co{signs} of inexhaustible potential of life.  An \co{idea} is
never adequate, a \co{general thought} can always be mocked as a failed
generalisation. It merely suggests but never captures the whole it signifies.
That is, it never captures it in a precise, conceptual form.  But it can always
happen to communicate the intended \co{quality} to a person who is open for such
a communication, who has similar enough \co{experience} or, at least, is willing
to ignore imprecision of the \co{signs} in order to \co{recognise} the
\co{experience} they try to convey.
}

\subsub{{This world}}\label{sec:thisWorld}

\pa We must admit that the \co{totality} of all things of \co{this world} is
never given in \co{an actual experience}. But the conclusion that there is no
such unity as \co{this world} follows then only if one has reduced reality to
\co{actual experiences}.  If nothing more then, at least, our inability to give
up the mode of speaking involving such general judgments about the world or life
suggests that there may be some \co{experienced unity} which transcends the
distinctions made within it. We do not look for proofs and merely claim the
\co{presence} of such a \co{unity}. There are \co{totalities} which are
reducible to their constituents, which are only sums of their parts. \co{This
  world}, however, is not such a \co{totality}.\ftnt{In fact, we would not even
  know what the supposed parts are. The interested ones may follow the interminable
  discussions trying to decide whether these are things, facts, matters at hand,
  states of affairs\ldots We, of course, won't.}  If it can be \co{posited} as one
\co{totality}, it is only because there is already \co{experienced unity}, which
only calls for some \co{actual} expression or, in more obsessive case, an
explanation.  Like any unity which \co{transcends} some \co{complex} of
distinctions, this \co{unity} which is not reducible to the \co{totality} of
things is more primordial than their \co{posited totality}.\ftnt{In general,
  \co{unity founds totality} and such a \co{founding unity transcends} the
  respective \co{totality}. In particular, \co{the world} \co{transcends} the
  multiplicity of \thi{things within the world}.
%Heidegger's Being transcends multiplicity of beings. But \co{the world} 
%is not Heidegger's Being.
  \citef{That which in any multiplicity is unitary did not flow out of any of
    its elements, that which is unitary in [common to] all cannot come from one
    among all but remains characteristic property of that unitary
    one.}{Proclus}{\para21} It is like Cusanus' universe of \thi{all things}
  which, however \citef{are not \thi{many things}, since plurality does not
    precede each thing. For this reason, in the order of nature \thi{all things}
    have, without plurality, preceded each thing.}{DDI}{ II 5/117} Or in the
  words of Heidegger's \citef{Neither the ontical depiction of entities
    within-the-world nor the ontological Interpretation of their Being is such
    as to reach the phenomenon of the \thi{world}. In both of these ways of
    access to \thi{Objective Being}, the \thi{world} has already been
    \thi{presupposed}, and indeed in various ways.}{SuZ}{I:3.14 [H64]}} In some
strange way, phenomenologists managed to coin the phrase \wo{phenomenon of the
  world}. But \co{the world} is not a phenomenon; it does not appear in the unity
of a single \co{act} of consciousness. What appears in such acts is an empty
word, a \co{vague} idea of something close to void of any content, a mere
\co{sign} pointing towards some all-embracing container. True, the \co{sign} is
not entirely void and this is why one feels justified in analysing the assumed
\thi{phenomenon}. But such an analysis concerns only series of distinct,
\co{actual} phenomena.  Their \co{unity} has no phenomenal content.  The
divergent analyses of the supposed phenomenon among various phenomenologists
illustrate not only the failure of making \co{the world} into a correlate of
adequate intuition, but also the general fact of irreducibility of its
\co{transcendent unity} to the \co{actuality} of features, facts, things and
observations.


\pa Since we do not need adequate intuition nor exhaustive characterisations, we
should be allowed to say that \co{this world} is the givenness of the
\co{visible}, givenness not merely as an \co{immediate actuality} but as the
field of one's activities, passions, goals, in short, the field of one's life.
It does contain {things}, tools, situations, concepts but this is just an
analytical statement -- all these are just the \co{cuts from experience}
circumscribed within the \hoa.  \co{This world} is the horizon of
\co{visibility} within which everything \co{actual} appears.  The
Husserlian notion of a \thi{horizon} has much intuitive appeal and it should not
be taken as merely something surrounding \co{visible} things. It is something
from which all \co{visible} things emerge. We could say, it is the
\co{visibility} itself. If we were to use the language we have given up: it is
not an accidental \co{totality} (\co{totality} is, in fact, always somewhat
accidental) of \co{dissociated} \thi{substances} but, on the contrary, it is the
\thi{substance} of which all particular things are accidents.


Rephrasing this last paragraph: \co{this world} emerges as the third hypostasis
(I:\ref{se:begWord}), as \co{chaos} turned into \co{re-cognisable}
\co{experience}. \co{This world} precedes all the things which we later find
within it. And it does so not only ontologically but also epistemologically.
(For, as a matter of fact, the generative order of \co{founding} coincides with
the ontological order and, with respect to the most fundamental notions and
entities, also with the epistemological order: what emerges first, is also what
is most intimately \co{experienced}.)

\co{The world} considered from the present level bears a resemblance to
\thi{Lebenswelt} in that it manifests itself phenomenally only as the ideal
horizon of the contents appearing within it and, consequently, as the field of
our activity and life.  One has a strong tendency to see it only as a
\co{totality} of things rather than as a \co{unity} of its own because
\co{reflective experience} is unable to grasp this \co{unity} in the
\co{actuality} of a single \co{act}: \co{the world} is not an \co{object} of
\co{an experience}.  But \co{the world} manifests itself in \co{any experience},
not as its thematic \co{object}, but as the constant \co{aspect} which actually
connects all \co{actual experiences}, which is the same, constant field of their
unfolding.  In every experience, it is \ger{mitgegeben} but only as a kind of
noumenal unity, as an indication of the \co{presence} of the totality of things
of \co{experience}; not as a concrete \thi{horizon} of related things, but
rather as a mere background, the ground on which the \co{actual experience}
finds place.  In this sense, \co{the world} is indeed in each thing, is
\co{reflected} in every experience. But its \co{unity} precedes all these
\co{experiences}, it is not \co{founded} by them but by the respective
hypostasis which is beyond phenomenological grasp.\ftnt{\citef{But since the
    universe is in each [thing] in such a way that each is in it, in each thing
    the universe is in a contracted way that which this thing is contractedly,
    and in the universe each thing is in the universe, although the universe is
    in each thing in one way, and each thing is in universe in a different way.
    [...]  Indeed, in a stone all things are stone; in a vegetative soul all are
    vegetative soul; in life all are life; in the senses all are senses; in
    sight all are sight; in hearing all are hearing; in imagination all are
    imagination; in reason all are reason [...]}{DDI}{II:5/118}}


\subsub{{I}}\label{sec:I}

\pa\label{lev:I} %!!! used later
\co{Self-reflection}, \co{reflection} \co{that I am}, is an \co{act} which,
attempting to \co{actualise} its intention, is always aware of its
insufficiency.  In it \co{I} \co{re-cognise} that \co{I am} \herenow\ but, at
the same time, also that \co{I am} something more. \co{I re-cognise} myself {\em
  as} \co{myself}, that is, the \co{immediacy} of the \co{act} is only a
\co{sign} of something which \co{transcends} it and does it {\em essentially}.
\co{I re-cognise myself} as something which never can be reduced to the
\co{actuality} of an \co{act}. We could say, a constitutive feature of the
\co{act} of \co{self-reflection} is the \co{recognition} of its irreparable
insufficiency.

\co{I} \co{re-cognise} \co{myself} as \co{transcending} the \hoa, but merely as
some noumenal site of identity.  \co{I} know my identity which extends over time
-- not because \co{I} managed to \co{re-construct} and {comprehend} it but
simply because \co{I} know it. It is \co{experienced} before this particular
\co{act} of \co{reflection} and extends beyond it. It is \co{experienced} in the
course of my whole life and does not depend on my \co{reflecting} over it or not.
This known \co{unity} which never becomes an \co{object} of \co{an experience}
reflects the thesis~I:\refpp{th:extime}. If single \co{experience} can be
essentially \co{non-actual}, then so can be the experiencing being. In this form
of \co{reflection} \co{I} \co{re-cognise} myself as a being whose identity is
not, like \co{ego's}, constructed from the \co{actual} contents, but
\co{transcends} them, extends over time and can only be approximated by
\co{reflection} by means of more or less adequate indications, allusions,
\co{signs}.
%

\wtsep{2. unity vs. multiplicity}

\pa
\wo{Who am I?} is the question of adolescence emerging from the
\co{egotic} preoccupations and beginning to \co{recognise} the wider horizon of one's
life. But the question never finds an adequate answer. One would like to know: I
am (going to be) a carpenter, a family man, a dedicated father, a scientist, a
politician, a charmer,\ldots Or else: I am intelligent, I am pretty, I am weak, I am
too vulnerable,\dots One would like to get an answer in terms of \co{actual,
  reflective} categories but such an answer would amount to reducing \co{oneself} to
one's \co{ego}. The only answer is that \co{I am myself}, but to understand and
accept it one has first to live, transcend one's \co{egotism} -- for the mere
\co{reflective} thirst for plain \co{visibility} it is empty and disappointing.  

The question never finds an adequate answer because one has a multiplicity of
\co{egos}, none of which nor the \co{totality} of which exhaust \co{oneself}. 
The relation between \co{oneself} and one's \co{egos} can 
be compared to that between the residual correlate obtained in the process of
variation (like eidetic reduction, only varying the whole life) in which the
varied elements are \co{egos}. There are, for instance, persons with strong
skills for social adaptation, perhaps actor-like characters, who behaving
differently on different occasions, do not suffer from any identity crises; or
else, \thi{rich personalities} with a wide range of forms of expression
which may easily seem incongruent but which are underlied some higher form of
personal coherence and control. In fact, every person has a similar multiplicity of \co{egos}, or
\thi{persona}: one for work, one for home, another for friends at one's place
and another for friends at their place, one for children, another for a
party,\ldots Strength of a personality is much closer related to the wide span of
apparently incongruent \co{egos} the person possesses and controls, than to the
uniformity of one's \co{egos} across different contexts and situations.

%sick:

\pa\label{pa:egoDID} On the other hand, the multiplicity of \co{egos} is also
what makes personal disintegration possible.  The question which an ideal theory
of a simple, \thi{substantial}, \thi{atomic} \co{subjectivity} must answer in
negative is: \wo{Is the disintegration of personal identity, the loss of one's
consistency and continuity possible?}  In the \co{immediacy} of pure
\co{subjectivity} such a loss is impossible -- there is no {\em time} for it!
But identification of the subject, of the \co{unity} of human being with the
\co{immediate consciousness} is of little help because, as we well know, such a
disintegration is possible. It is possible because \co{I}, being stretched in
time, can lose the continuity in time -- \co{I} can be \co{dissociated} into a
multiplicity of \co{egos}.

Dissociative identity disorder, \abr{did}, shows that one can possess a
multiplicity of \co{egos}, each of which is sufficiently integrated to have a
relatively stable life of its own and recurrently to take full control of the
person's behavior.\ftnt{Milder forms of dissociative or organic amnesia, fugue
  states (flight from one's identity, frequent in war victims) can be mentioned
  here as well, even though some psychologists challenge the reasons for
  identifying -- if not the very reality of -- such disorders. \abr{did} used
  also to be called \wo{multiple personality disorder}. We do not see much of
  personality in the \co{complexes} affected by such disorders -- they seem to
  coincide with our \co{ego}.}  Differences between various \co{egos} of one
person may be astonishing -- amnesia of other \co{egos}, changed wishes,
attitudes, interests, hand writing, even different physiological indices like
heart rate, blood pressure, \abr{eeg}.\ftn{\citeauthor*{Lester}} A person
suffering from \abr{did} has, as a normal person, a multiplicity of possessions,
namely \co{egos}. The difference is that, while a healthy person possesses
\co{egos} keeping them under some degree of control, with a sick person it is
\co{egos} which gain uncanny autonomy and possess the person. The relation
between \co{I} and \co{egos} gets inverted, \co{I} is conflated to the level of
\co{egos} and, unable to organise them, suffers their multiplicity -- the higher
becomes a mere \co{totality} of the lower.

\pa\label{pa:iDID} It is easy to misuse such examples to suggest that \thi{in
  reality} there is no \co{I} and only a multiplicity of somethings, e.g.,
\co{egos}. But the fact that we can demolish a building into a heap of bricks
proves neither the unreality of the building nor that it is, \thi{in reality},
only a heap of bricks. Phenomena like \abr{did} represent disintegration and
not, for instance, simple multiplication. As a disintegration, it is possible
because subject (person, man) is not any ideal and extensionless point, but
because in its temporal duration it possesses \co{complex} aspects whose
configurations may change.  But there is more than the mere disintegration.  It
changes certainly the feeling of life and the sense of oneself, but it does not
change the fact that even such a person is \co{oneself} all the time.  Each of
the \co{egos} has only a \wo{{\em relatively} stable life of its own}.  While
psychologists focus on what constitutes the problem -- the dissociation of the
{\em sense of} identity -- one should not forget all the rest -- the \co{unity}
prevailing \co{above} the \co{actual} multiplicity.  For instance, usually there
is one dominant personality (who `knows' about others), and one can often change
at will from one to another, through a process similar to self-hypnosis
(\abr{did} patients are highly hypnotisable and susceptible to self-hypnosis).
It is the person \co{himself} who starts spawning new \co{egos} to handle some
unbearable emotional problems (child abuse is the recurring theme in the
etiology of \abr{did}, the first alter \co{ego} appears usually between the 4th
and 6th year of life); it is the person himself who still keeps some degree of
continuity;\ftnt{In the well-known Julie-Jenny-Jerrie case, Jerrie says: \wo{I
    wish Julie would stop smoking. I hate the taste of tobacco.} The split is
  there, but it is obviously maintained by somebody.  Jerrie knows about Julie
  while Jenny, the original \co{ego}, knows about both other.} it is the person
\co{himself} who addresses the therapist; and most importantly, it is the person
\co{himself} who is being treated and, as the case may be, cured. A successful
treatment of \abr{did} results in an integration of multiple \co{egos}: in
either merging or associating them in the \co{unity} of one person. It is not
the multiplicity of \co{egos} which got cured -- it is {\em the} person.


\wtsep{3. world = I; only abstractly dissociated}

\pa \co{Reflection dissociates} and \co{posits I} against -- as opposed to,
thrown into, confronted with -- a foreign world.  The \co{I} revealed to
\co{self-reflection} is not explicable in the way \co{ego} might be. It is not
\co{transcendent} in the way \co{ego} is, merged with the world and its
\co{visible}, even if \co{unclear} affairs. \co{I} is something other, something
beyond and above the world, something~\ldots noumenal.  It does not belong there
among things and \co{complexes}, it does not belong among others who, for the
moment, are just foreigners. The pure \co{I} of the \co{reflection that I am}
is, as Camus would repeat after so many others, a stranger.  The strangeness is,
however, only in the \co{externalised objects} and, likewise, in their
\co{posited totality}. {\em This} is indeed the kind of world in which \co{soul}
is a stranger, the world viewed by the abstract \co{reflection} as a mere
\co{totality} of things, of \co{dissociated} situations, eventually, of
irrational and meaningless, even if logically comprehensible, events.

\co{Concretely}, however, (and that means, without attempts at reduction to
the \co{visibility} of \co{actual} givens) \co{self-reflection} marks the
\co{experience} of one's life.  We have purposefully not distinguished between
the \co{qualities of life} and the feelings thereof. We would be tempted to
identify them.\ftnt{But identify -- not reduce the \co{qualities} to mere
feelings!  \co{Qualities} are not only \thi{subjective feelings}. Rather, both
are alternative, hardly distinguishable \co{aspects} of the same personal
\nexus.}\noo{\co{I} am not given {my life} as a substance, an \co{object}
which then can be attributed different \co{qualities}.} The feeling of {my life}
is the same as the feeling of its \co{quality}, and only \co{reflection} tends
to \co{dissociate} the two. I do not experience some feelings which then can be
identified as revealing some \co{qualities}. They are the same thing, and we
\co{dissociate} them only by \co{externalising} the \co{qualities} as some
\co{objects} distinct from the \co{signs}, the feelings through which they
appear.

{One's life} not only exceeds the \hoa, but also remains forever, {\em
essentially}, outside it, always richer, always inexhaustible.  Seen this way,
the richness of {one's life} is a counterpart of the noumenal emptiness of the
bare, self-identical \co{I}.  \noo{All \co{signs} of this level point to
\co{myself} and manifest shallower or deeper appearances of mineness.  I may
discover dark and unexpected sides of myself, but they are all intimately {\em
my} sides, they are not, in any way, alien.}  Experience of \co{myself} is
equiprimordial with the \co{experience} of {my life}.  One tends to think the
latter as an accident of the former but this is because the former has still
deeper roots to which we will return shortly.  {My life} is not something which
\co{I} of \co{self-reflection} has, it is something it is.
\co{I} do not live my life -- \co{I} am \co{my} life. 
%\pa
\noo{No matter what typography one attempts of human personalities, types, etc.,
one is always left with the concepts which shine with their emptiness, all their
possible honesty and adequacy notwithstanding. \co{Concreteness} of life slips
between the fat fingers of \co{conceptual} constructions.  Neither can any life
be contained in a book, be it a most detailed biography, even if much of it can
be captured by an adequate description and some essential traces expressed
successfully by a talented artist.  But to contain a life would be to strip it
of its \co{concrete} \co{quality}, to reduce it by repeating it.} 

\label{pa:worldislife1}
To use yet another word, we can say that the richness of {one's life} is one's
\co{soul}.  One can build {one's life} on an example of a person one respects,
develop one's soul inspired by another, learn something from another's life.
But one \co{soul}, the uniqueness of one life and its \co{quality} can not be
repeated.  There is nothing like \thi{{\em a} soul}.  Soul is always
\co{concrete}, it is always {this particular life}, this concrete {world}.
\co{Soul} is not alive, it is life.  And it is life because it is \co{the world}
and \co{the world} is life.  One's \co{soul}, \co{oneself} becomes emptied of
life to the extent it is \co{posited} by \co{reflection} as an independent
entity, an isolated being only potentially capable of an involvement in an alien
world.  \ger{Dasein} which \thi{falls within the world} is a gnostic abstraction
-- even if some feelings in the face of dissolving values and depersonalised
world could be described this way it is, eventually, a harmful \co{sign} of
\co{alienation}.  The \co{quality of life} is the \co{quality} of the \co{soul}
and is the \co{quality} of its world -- the three can be \co{dissociated} only
by \co{reflective} abstraction. \co{The world} understood as the continuity of
\co{experience}, as the \co{unity} preceding, and hence stretching beyond its
temporal differentiation into the \co{dissociated objects}, is the same as {my
  life}, is intimately \co{my} world.  \citet{The world and life are one. I am
  my world.}{Tract}{5.621, 5.63} \co{I}, being \co{the world} of \co{my} life,
precedes the things emerging within \co{this world} just like \co{the world}
does -- not only the emergence of \co{the world} but also the creation of man
\citet{is prior to those things which were created with it or in it or below
  it.}{Periphyseon}{IV:779ABCD\kilde{p.93} [We won't dwell on Eriugena's
  distinction of orders of precedence and the fact that \wo{prior} above refers
  only to knowledge and rank (corresponding to our \co{founding}) though not
  time.]}
%
\noo{This is more than to say that \thi{Dasein} is \thi{in-the-world} -- \thi{Dasein}
{\em is} its world.}

\wtsep{good -- only in a good world}

\pa The \equin\ of \co{I} and \co{my world} does not amount to any
\thi{subjectivism}.  \citet{Of course, you may confront me with: `But are you
  sure your story is really the true and right one?'  But what does it really
  matter what the {\em reality outside myself} is, as long as it has helped me
  to live, to feel that I am alive, to feel the very nature of the creature that
  I am.}{BaudSpleen}{Windows [my emph.] \citef{Why could the world {\em which
      is of any concern to us} -- not be a fiction?}{BeyondGE}{34}} In fact, it
matters quite a lot, unless one is willing to assume the attitude of a decadent
aestheticism which, by its very \co{self-reflection}, sets itself apart and, as
it would like to believe, above the world; the attitude which {\em opposes} the
two, which feels forced to claim that \citet{nothing is \thi{given} as real
  except our world of desires and passions, that we can rise or sink to no other
  \thi{reality} than the reality of our drives.}{BeyondGE}{36} The very fact of
opposing \co{oneself} and the \thi{reality outside} witnesses to a disturbance,
to a breach, a doubt not only about the \thi{reality outside} but also about
\co{oneself}.  In a sense, it is true that \citet{[t]here can be no progress
  (real, that is, moral) except in the individual and by the individual
  himself.}{BaudHeart}{} But such a progress is not the matter of one's
\thi{subjectivity}. It is the matter of \co{oneself} as much as of {one's
  world}. The two, being the same, change and progress together or not at all,
\citeti{and with what measure ye mete, it shall be measured to you
  again.}{Mt.}{VII:2}

Wanting to change {one's} life, not only this or that aspect of it but
\co{oneself}, one has to change {one's world}.  Good {soul} cannot live in an
evil world, for no matter to how much evil \co{the world} exposes the soul, the
good soul will still see -- in \co{the world} -- the reason to be good.  To be
good {\em in spite of the evil} in \co{the world} is to assume an attitude
which, at some point, does not reflect one's being. To be good in \co{the world}
which is evil is to be a rigid moralist, a pharisee or, in a more lofty variant,
a resigned Stoic who is only a tiny step from the apathy of a bored
intellectual, gnawed by the \thi{unreality} of his lofty \thi{ideas and ideals}.
To be good is not only being good but also finding goodness, finding the need
and reason for it not only in the self-goodness of one's inner life but in
\co{the world}. \woo{A good man out of the good treasure of the heart bringeth
  forth good things: and an evil man out of the evil treasure bringeth forth
  evil things.}{Mt.XII:35; Lk. VI:45} But \thi{the treasure of the heart} is
not any private \thi{subjectivity} of one's \thi{inner life}.  \citeti{For where
  your treasure is, there will your heart be also.}{Mt.}{XII:35/VI:21; Lk.
  VI:45/XII:34} The \thi{treasure} is something found, given to \co{me} from
\co{above} and not something \co{I} assume or decide to value.  This
\thi{treasure} is that which shapes \co{my} heart, \co{my soul}, while its lack
leaves the \co{soul} empty.

\pa Wanting to change {one's life}, not only this or that aspect of it but
\co{oneself}, one has to change the \co{world}.  One's life acquires a new
\co{quality} not because one has realised something important in a momentaneous
illumination but as a consequence of a new way of \co{experiencing
  the world}, of finding a new \thi{treasure}.  Certainly, there are
pathological cases against which a mere analysis, psychoanalysis or other
form of psychotherapy can help.  But their value is limited to the lower levels of
\co{egotic} disturbances.  The faith in their unlimited power can arise only
from the assumption of the genuine \co{dissociation} of a person from \co{his
  world}, according to which each can be \thi{treated} independently from the
other.\ftnt{Many examples of the resulting oppositions can be found in the
  society of Victorian \fre{fin de sicle}: positivistic scientism and
  utilitarianism are nervously opposed by the search for the freedom of the will
  and the calls to cultivate \thi{art for art's sake}; bourgeois norms and
  industrial routine is equally nervously opposed by the cult of intense
  experience, hashish, absinthe and Bohemian decadence; neurasthenic apathy and
  weary reflection call for passionate and heroic action; the progressing
  depersonalisation and society turning into masses, if not mob, are met with
  almost pietistic calls to personal concentration and authenticity.
  Baudelaire, Mallarm\'{e}, Wilde, Freud, Nietzsche, Spencer, Dostoevsky, James
  \ldots It is hard to avoid \noo{fend-keep off}the sense of artificiality -- if
  not of the view of the world and of the involved oppositions, then of the
  individual attitude attempted {\em in spite of} and against the world (or was
  it in spite of and against \co{oneself}?)  Much of the XX-th century's so
  called \wo{existentialism} sailed under the same banner.}  Healing a true
suffering of a \co{soul} amounts to healing \co{the world} of this \co{soul}
for, in deeper respects, one can hardly change understanding of the world
without actually \co{experiencing} a changed world.  Sometimes, a travel to a
remote place may be needed to regenerate the \co{soul}.  An emotional impairment
resulting from the lack of love and warmth can hardly be changed by a mere
realisation that this was its cause.  It may need not only the will to change it
but also the very \co{experience} of love and warmth.  This is what often makes
meeting new people worthwhile: they can make things which seemed impossible,
even non-existent, to appear obvious and natural.  They can show us a different
world which cures the lacks of the world in which we used to live.

% \wtsep{activity: changing the world = changing oneself}

\pa\label{pa:Zweck} \co{My} life, that is, \co{the world}, is the field of
\co{my} expression.  Its unlimited -- though finite -- temporal (and spatial)
scope is no longer a stage of single \co{acts} or manipulative \co{actions} but
of \co{activity} in the broadest sense of the word, \co{activity} which is not
merely a sum of \co{acts} and \co{actions}, which is not directed towards
achievement of some goals, but which expresses the \co{traces} of values, the
\co{motives} which shape the horizon for selecting possible goals.\ftnt{Using
  the distinction of Scheler's, a specific goal corresponds to \thi{Zweck}, a
  \co{motivation} to \thi{Ziel}, e.g., \citeauthor*{MaxForm}{I:I.3.\kilde{p.52}}
  \co{Motivation} is a horizon of values which, on the one hand, functions as
  the \co{foundation} for choosing particular goals and, on the other hand,
  becomes \co{actualised} through their realisation -- it comes both before and
  after the goals, it surrounds them.} {My life} is just that -- the way I spend
my time.  A common answer to the question \wo{Who is he?} would simply tell what
the person is doing for living.  A profession tells something precisely about
that: what one does with one's time, or even better, to what one dedicates one's
time.  A (deliberate) choice of profession may involve detailed arguments about
specific demands and forms of activity associated with it.  But it always
involves more, a hardly expressible feeling of the \co{quality}, of the
character of the profession, which should correspond to the feelings of one's
values and the sense of what is meaningful.  One wants one's \co{activities} to
reflect the (passive) feelings of \co{qualities} and values.  (Of course,
profession is hardly ever an exhaustive characterisation -- one may actually
hate one's work and devote much more and more intense time to other,
non-professional \co{activities}.)  The \co{activity} to which one devotes much
of one's time, expresses also the \co{quality} of one's life.  And it is no
longer talk about simple grasping-avoiding as in the case of \co{objects}, nor
arranging-preventing as in the case of \co{complexes}, but about dedication,
about accepting some values \co{transcending} the \co{actuality} and about
dedicating one's life to their expression.

\pa Not only changing one's life involves changing \co{the world} but also vice
versa.  If one wants to change \co{the world}, that is, not this or that thing
but \co{the world}, one has to bring to it a changed \co{quality} of one's life.
Achievements, deeds, goals, reforms do change the {objective world}. But it is
never certain if they also will change \co{the world}. In most cases, they do
not, and the more extreme changes of the world make only the remoteness of
salvation more clear. \citet{How can an event which, like war, eliminates
  discussion and opens every possibility by denying every norm, bring salvation
  to humanity?}{Opium}{I:Concerning the political optimism\kilde{p.97}} How can
it bring salvation even only to its perpetrators?  Revolution as a means of
abolishing alienation is one of the most tragic human inventions but it is only
an extreme case of frustration turned into destruction of which there are many
examples. With respect to the personal dimension of \co{existence}, social
activism suffers from the insufficiency, if not irrelevancy, 
which broods even over the satisfaction from the \co{actual} 
successes.  No doubt, improving social institutions may be a venerable
\co{activity}. But looking in it for the medicine against personal problems and
changed \co{quality} of one's life is, at best, a misunderstanding and, at
worst, an instinctive resentment, the more dangerous because unchecked in its
convictions about its beneficent intentions.\ftnt{\citeauthor*{MaxRess}{.}
  The just quoted book of R.~Aron contains the classical analyses of a
  particular form of this sickness.} It is true that living in a particular
world may promote some and not other ways of \co{experiencing} and only some,
but not other \co{qualities of life}.  But these are, at best, statistical
tendencies. They never have a predictable effect on a particular individual
whose \co{world} is much more than the {objective world} of tools and political,
economical and social organisation.

\noo{
Feeling that \co{I} do not know who I am, so nowadays popular 
confused \thi{looking for oneself}, is the same as the feeling 
of the \thi{unreality} of \co{the world}, both are just aspects of the 
same experience of alienation. 

accepting and giving. 

Give -- receive; care-expect; love-hate, (responsibility?)
}


\subsub{Transcendence}
% fix a bit: what is really the vertical aspect?
% horizontal -- the same level vs. vertical -- Aufhebung of this 
% level's categories ?
\pa At the current level, one finds the distinction between the personal and
impersonal, but the distinction between the \thi{objective} and the
\thi{subjective} loses almost completely its meaning.  It is \co{myself},
\co{my} life which is involved, in a sense, the most \thi{subjective}
\co{aspect} of experience. On the \thi{objective} side there is, perhaps, the
world, in the sense of \thi{everything but me}. But this phrase can signify here
only the correlate of \co{my} life, the field of its unfolding.  \co{The world},
{my life}, \co{experience} and \co{myself} are close to synonymous.
 
To be sure, we do live in the {objective world}, we do make plans, manage
worldly situations, use tools.  But all these \co{complexes} are correlates of
my \co{ego}, they are only \thi{parts} of the world and are \co{below} me.
Viewing \co{the world} as their \co{totality} is a simple-minded, and always
unsuccessful, reduction.  Things, \co{complexes}, particular situations and
singular \co{experiences} not only never exhaust the experience of \co{the
world} -- they do not even provide the ground for such \co{experience} which is
\co{founded} upon the \co{unity transcending} their \co{totality},
\co{transcending} the {objective world}.

\pa The tension of \co{horizontal transcendence} at this level does not arise
from the \co{more} of \co{complexes} but from the basic opposition between
\co{mine} and \co{not-mine}.  It is immediately present in the \co{reflection
  that I am} which performs a highly artificial operation of \co{dissociating
  myself} from something -- from \co{the world}, perhaps, from {my life},
apparently from that which is \co{not-mine} but
eventually, as a matter of fact, from \co{myself}.  This \co{dissociation} finds
an expression in the 
opposition between the \thi{inner} and the \thi{outer}, the \thi{inside} and the
\thi{outside}.  \thi{Inner} would usually refer to the \thi{inner life} but it
may be related to anything which is in some way \co{experienced} as \co{mine}:
\co{my} feelings, \co{my} things, \co{my} friends, \co{my} family.  The
\thi{outer} is then everything else, everything excluded from this \thi{inner}
circle, everything for which \co{I} do not feel a slightest degree of
responsibility, everything which perhaps influences \co{me} but is not influenced by
\co{me}.  The \co{transcendence} of that which is not \co{mine} may {confront me},
for instance, as \co{my} indifference opposed to care and responsibility I feel
for what is \co{mine}. But it can also emerge through \co{my} inability to
draw borders between the two, by \co{me} getting distracted, if not lost, among
the challenges and temptations confronting me in my meeting with \co{the world},
the inability \thi{to gather myself}.  When manifested in \co{actual signs}, it
involves the fundamental feeling of being only \co{myself} confronted with the
rest which, being \co{not-mine}, is foreign, disturbing, unwelcome, even
threatening and dangerous.
%, whether it is as Plato's soul, Camus' stranger, or quite 
%personal feeling of hopelessness or alienation.

\pa The \co{vertical aspect} of \co{transcendence} amounts here to a
confrontation with \co{chaos}. It is not, however, a mere chaos of
uncontrollable \co{complexity}, but a \co{chaos} in the face of which the categories
of \co{visibility} simply lose their meaning. It is \thi{something completely
  other}, something which can no longer be treated in the familiar ways
according to which I organise \co{my world}.  This \co{chaos} may, but need
not, mean disorder -- it is only lack of \co{objective} order, lack of any
sufficient reasons and explanations in the \co{objective} terms.

The \co{horizontal transcendence} of \co{mineness} involves a feeling of
alienation, of being confronted with empty and meaningless, perhaps dangerous
otherness of \co{not-mine}.  It is something foreign, but something from which I
can relatively easily retreat behind the walls of \co{my} home-castle.  The
\co{vertical aspect} involves likewise a confrontation of \co{mine} and
\co{not-mine}.  But unlike the former, it does not present \co{me} with a
definite, absolute \co{dissociation} of the two.  \co{Vertical transcendence}
makes the \co{not-mine} \co{present} in a way which is, strangely yet
\co{clearly}, intimately mine.  The \co{chaos} is \co{present} as a most genuine
element, as a most \co{immanent aspect} of \co{myself} and, by this very token,
can be either a source from which \co{my activity} draws strength and energy, or
else an abyss capable of devouring \co{me}.

Questions about the sense and meaning of one's life can be reduced to one's main
achievements, their importance for society or others, but typically they involve
one in a confrontation with the fact that this meaning, if any, arises somewhere
\co{above} and is not reducible to any plainly \co{visible} contents.
Kierkegaard's \la{Angst} is a well known example of such \co{an experience}.  It
confronts \co{me} with something totally other, something inexplicable and
irreducible to the familiar categories of \co{visibility}; something which
threatens \co{me} with its unfamiliarity and uncanniness. Such an encounter
amounts, in fact, to the question about \co{myself}; manifesting irreducible
otherness, it oppugns the familiar categories and assumed foundations of \co{my}
life, it poses the question about \co{myself}. But \la{Angst} is only an extreme
example.  Love which begins to penetrate my whole being without referring me to
any specific object, which makes the whole world dissolve in a continuity of
thankfulness and inspiration can be a good example, too.  Love, also personal
love, is not anything I arrange, even less anything I choose.  It is something I
meet, something coming to me from \co{above}, something \co{not-mine}. And yet,
it does not remain a foreign and merely accidental event which only happened to
\co{me} as is the case with \co{actual experiences}. Embracing my whole being,
it is thoroughly \co{mine}, it involves \co{myself} to the very depth of my
\co{soul}, to the point where \co{I} cease to be \co{myself}.

The \co{vertical aspect} of \co{transcendence} confronts one with otherness
\co{above}, with something over which one is not the master. It has these two
basic forms: of dissolution or enrichment, of destruction or creation.
\citet{And of madness there [are] two kinds; one produced by human infirmity,
  the other [is] a divine release of the soul from the yoke of custom and
  convention.}{Phaedrus}{265b.\kilde{p.82;???} It is easy to observe
  differences.  For instance, Plato's divine madness is \wo{subdivided into four
    kinds, prophetic, initiatory, poetic, erotic}, all related to some
  \co{actual} consequences (and reappearing later, with
  Ficino\kilde{Commentarium in Convivium Platonis, VII:14;letter to Pellergino
    Agli, in Opera} and Pico, as the four divine \la{furors}). But we need not
  agree on all particulars to retain the main idea.} Only at the limits of
{oneself} can one become \co{oneself}, but it is also where one can lose
\co{oneself}. There is an element of madness in every creative genius, for
creation -- the secondary creation, that of which humans are capable -- amounts
to organising the disorganised, to bringing \co{experience} out of \co{chaos}.
Any creative \co{activity} involves a deep personal engagement which arises only
from a confrontation with \co{chaos}, where \co{I} meets \co{non-I} and the
border between oneself and otherness, between \co{mine} and \co{not-mine} is
impossible to draw.  Such a confrontation may result in a dedicated love, in a
resolute patience, in new, beautiful works of art or science or in \ldots
madness.  A creative genius organises \co{chaos} emerging, like Jonas, after
three days from the whale's belly.  A madman is, too, swallowed by the
overpowering force of \co{chaos} but, unable to wrest himself from it, remains
there or else, if he returns, returns empty-handed.

\subsection{Invisibles}\label{sec:levelD}

\noo{ Level 4 is ... temporal scope \wo{Beyond time} is OK. \wo{Infinite time}
  is used sometimes, but it is not an equally legitimate phrase.  }

\pa \co{Self-reflection} discloses my \co{separation}, but it does so under the
mark of \co{reflective} \co{dissociation} -- it posits \co{myself} as an
independent entity which is therefore experienced as alienated. It centers
around the category of \co{mineness} with its basic mode of \co{my will}.  It is
the level at which I can still choose and control and where all my decisions,
actions and activities are referred back to \co{myself} as their protagonist.
Even when I feel that, as a matter of fact, I am not in control, I still persist
in the attempts to realise \co{my will}.  And as long as I persist in this
focusing on \co{mineness}, \co{my} goals, \co{my} wishes, \co{my will}, I also
keep experiencing \co{separation} as alienation. Even if I recognise the world
as \co{my world}, I do not appropriate it, it does not become fully \co{mine}.
For \co{my} world should conform to \co{my} projects, while the world does not.

\noo{Success -- always in \co{this world} -- may make it hard to recognise, and
  even harder to appreciate, the presence of yet higher dimension of life.
  Arrogance of self-confidence tells the story of somebody living only in the
  world he manages to control ...}

In \co{self-reflection} \co{I} \co{re-cognise} \co{myself} as transcending the
\co{horizon of actuality}, but merely as some noumenal site of mere
self-identity. \co{I} know intimately that it is \co{myself} \co{I} am
reflecting over, yet this identity remains ideal, unexperienced.  \co{Myself}
discovered by the \co{reflection that I am} is the result of a rather artificial
abstraction in which \co{reflection} dissociated \co{myself} from \co{the
  world}, even from \co{my life}, that is, from myself.  As much as \co{I} know
that \co{I} am \co{myself}, I also know that the reflecting \co{I} is not fully
\co{myself}; the \co{I} grasped by the \co{reflective act} does not coincide
with \co{myself} living \co{my life}; \co{I} am \co{myself} and yet \co{I} am
losing myself, \co{I} am close to \co{myself}, but also remote -- the same and
different. \co{Alienation} is more than the estrangement from \co{the world}; it
is first of all the estrangement from myself, the loss of contact with \co{my
  self}.

\pa There are situations where my self-identity, \co{posited} noumenally in
\co{self-reflection} as an incomprehensibly substantial property, becomes
dubious and as if suspended, if not totally absent.

On the pathological end, many cases of schizophrenia provide examples of a split
at the level of \co{myself}.\ftnt{We use the term in the usually imprecise sense
  which may cover a large variety of symptoms and their complexes, which should,
  perhaps, be classified as distinct disorders.} The etymology of the term
coined by Eugene Bleuler -- \la{schizo} = split, \la{phreno} = mind -- does not
intend multiple personalities but one split personality: the emotive and
cognitive functions are not only disturbed but also {\em dissociated} from each
other; hallucinations or delusions of grandeur or persecution {\em invade}
consciousness of a paranoid schizophrenic; {\em attacks} of silly and incoherent
laughter, grimace, unmotivated giggle are symptoms of disorganised
schizophrenia.  One hears the complains that patient's intestines are congealed,
that his brain has been removed or that some device has been implanted into it,
that a slightest movement will provoke an enormous catastrophe (catatonic
schizophrenia), etc.
%Abnormal Psychology, p.416ff
It is the  patient who realises that some alien force threatens {\em his} integrity
and takes over the control -- of {\em his} mind, {\em his} behavior, {\em his}
surroundings.  

Within a more \thi{normal} range, there are also experiences when {\em I am not
  in control of myself}, when \co{I} am seized by an impulse, an urge to act in
a way which, to all \co{my} consciousness and knowledge, is not \co{my} way of
acting, which does not originate in \co{my} will.  \citet{The primitive
  phenomenon of {\em obsession} has not vanished; it is the same as ever. It is
  only interpreted in a different and more obnoxious way.}{MHSJung}{p.32} An
impulsive act, a \thi{murder in affect} may be followed by an outcry \wo{It was
  not me, it was something strange in me!}

Such impulses and acts, although \co{mine}, emerge {\em as if}
from \thi{outside}, as if they were coming from some higher or deeper layers
which are not under my control, which, although originating \thi{in me}, are not
\co{mine} at all.  \citetib{A man likes to believe that he is the master of his
  soul. But as long as he is unable to control his moods and emotions, or to be
  conscious of the myriad secret ways in which unconscious factors insinuate
  themselves into his arrangements and decisions, he is certainly not his own
  master.}{MHSJung}{p.72 [We will later explain the difference between
  \thi{unconscious} and \co{invisibles}. For the time being, one may
  think in terms of the \thi{unconscious}.]}  \citet{We have intimations and
  intuitions from unknown sources. Fears, moods, plans, and hopes come to us
  with no visible causation. These concrete experiences are at the bottom of our
  feeling that we know ourselves very little; at the bottom, too, of the painful
  conjecture that we might have surprises in store for ourselves.}{JungArche}{IV:299}
%
%\citt{`Where there is a will, there is a way' is the superstition 
%of modern man.}{Jung, MHS, p.71}

Experiences of this kind, when \citet{one becomes two,}{BeyondGE}{From High
  Mountains (the concluding aftersong)} make \co{present} something
which \thi{lives in me but is not me}, which is \thi{inside myself} and yet is
not \co{myself}, which exercising often irresistible power over \co{myself},
stays \thi{outside} \co{myself}. Then \citet{the greater figure, which one always
  was but which remained invisible, appears to the lesser personality with the
  force of a revelation. He who is truly and hopelessly little will always drag
  the revelation of the greater down to the level of his littleness, and will
  never understand that the day of judgment for his littleness has dawned. But
  the man who is inwardly great will know that the long expected friend of his
  soul, the immortal one, has now really come, \thi{to lead captivity captive}
  [\ldots]}{JungArche}{III:217\label{cit:little}}

\pa The \co{reflection} necessary for overcoming \co{alienation} (once more,
\co{alienation} from \co{the world} is only a reflection of the \co{alienation}
from \co{my self}) is to realise \co{that I am not the master} -- not only of
\co{the world}, but neither of \co{myself}, of my very being.  My decision to
achieve a goal may be opposed by external factors or my own inability or
laziness.  This is trivial at the level of \co{objects} (which are \co{external}
and given rather than chosen) and of \co{complexes} (where there is always
\co{more} which \co{I} cannot conquer).  But \co{I am not the master} also in
the more profound sense of not being the master even of \co{my} being, of not
possessing even \co{myself}.  My will to be good may never get realised,
sometimes due to my obvious weakness or impatience, sometimes due to unclear and
hardly \co{visible} obstacles; where failure can be blamed on my own incapacity
as well as on bad luck.  \co{I} do not decide to fall in love with a given
person or not.  \co{I} may do even if, as far as \co{I} can see or as the course
of life shows, the person is not the one \co{I} would like to love.  My hope for
happiness may never find fulfillment -- not only because \co{I} constantly find
features of \co{myself} precluding it, but simply because \co{I} am unhappy.
There are sufferings of which \co{I} can be acutely aware and which \co{I} can
firmly defy but which, nevertheless, last for years leaving hardly any hope that
they may ever terminate.  There are states in which one wants to say: \citet{Let
  me perish, let me die!  I live without hope; from within and from without I am
  condemned, let no one pray that I may be released!}{TheolGerm}{XI} There are
no \co{signs}, no \co{visible} possibilities of redemption.  Despair is to yield
to this impossibility, is to accept it, but it is not something \co{I} choose
voluntarily: it happens to me, and \co{I} only can not resist it.

And yet, one day, \co{I} may find that all that was ceased to be, that as a
matter of fact, \co{I} am happy but \co{I} know neither when \co{I} became so or
how that happened.  \co{I} do not even know what it means -- \co{I} only know
that \co{I} am.  \co{I} may find one day that the insecurity or angst which have
been lurking in the depths of my soul disappeared and their place took
tranquility and peace.  But \co{I} know neither when nor how that happened --
only that \co{I} had wished, that \co{I} had prayed for that to happen and that
it did.  \citetib{[T]his hell and this heaven come about a man in such sort,
  that he knoweth not whence they come; and whether they come to him, or depart
  from him, he can of himself do nothing towards it.}{TheolGerm XI.}{}

% \citt{The wind bloweth where it listeth, and thou hearest the sound thereof, but
%   canst not tell whence it cometh, and whither it goeth}{John 3:8}.

\subsub{The signs}\label{sub:invisibleSigns}

\ad{Original signs}\label{pa:restA} \wtsep{1. acts - aura, rest} A \thi{murder
  in affect} need not be followed by the outcry \wo{It was not me, it was
  something strange in me!} It can be just committed and simply through that, by
being committed, witness to the \co{presence} of a power greater than
\co{oneself}. But this is rather an extreme witness.  \co{Acts} and \co{actions}
are involved in the texture of \co{the world} and, possibly, in the broader
context of \co{activities}. In this respect, they may be studied as
\co{objective}, purposeful ways of achieving various goals or as expressions of
various needs.  But this does not exhaust their significance.  \co{Acts} and, in
particular, the ways in which they are carried out, are not determined
exclusively by \co{my will} nor their \co{objective} context. Every \co{act}, in
addition to its \co{objective} context, in addition to its \co{visible} content,
involves an undefinable \co{rest}, a side which does not pertain to its
\co{objective} determinations. For instance, \citet{the value \thi{good} [...]
  is present as if \thi{behind} the acts of will, and this in the essential way;
  it cannot therefore be intended {\em in} these acts.}
{MaxForm}{I:1.2;p.48.\orig{Der Wert \thi{gut} [...]  befindet sich gleichsam
    \thi{auf dem R\"{u}cken} [Willens]aktes, und zwar wesensnotwendig; er kann
    daher nie {\em in} diesem Akte intendiert sein.} \thi{Der Wert gut} can be
  an example of \co{invisible}.  Let's only remind that for Scheler it is first
  of all the value pertaining to a person manifested, as the quotation
  indicates, only in the \co{rest} of the acts.}  An \co{act} directed by \co{my
  will}, aimed at a specific objective, has an involuntary aura around itself
which indicates something else, often different, than its intention has put into
it. Trivialising to the extreme (or, perhaps, complicating more than necessary),
one distinguishes the \thi{what} from the \thi{how}, as the case may be, the
\thi{matter} from the \thi{form}, the \thi{conscious} from the
\thi{unconscious}. With Heidegger the \thi{how} of the acts is not only more
important than the \co{objective} \thi{what}, but is the genuine,
non-{objectifiable} site of their ontological significance.\ftnt{Our \thi{how}
  tends more in the direction of explicit understanding, towards the
  \co{objectivity} of Heideggerian \thi{what}. Thus it seems we may be using
  this pair in exactly opposite way. In either case, both our \thi{how} and
  \thi{what} belong to the sphere of \co{visibility}, although the latter
  touches the ultimate \co{that}.}  It seems that nothing is ever the same,
nothing is ever fully itself, that the search for the ultimate \thi{in itself}
always encounters some overlooked \co{rest}, which germinates underneath the
established foundations so that \citeti{the stone which the builders refused is
  become the head stone of the corner.}{Ps.}{XIX:118.22} No amount of
intentional deliberation is able to remove this \thi{unintended} \co{rest} from
an \co{act}, to reduce an \co{act} to its \co{visible}, \co{actual} \thi{whats},
\thi{whys}, \thi{hows}.  \citet{The fool practices concentration\lin And control
  of the mind.\lin But the master is like a man asleep. [...] Because the fool
  wants to become God,\lin He never finds him. [...] Because the fool looks for
  peace,\lin He never finds it.}{Ash}{XVIII:33,37,39}

It would be wrong to say that \wo{I am what I do}, but what \co{I} do does
\co{manifest} what \co{I} am.  \co{Acts} are not any \co{dissociated}, isolated
and mutually independent events, they are involved in the context of
\co{actions} and \co{activities} and, eventually, \co{reveal} \thi{who I am}.
Their \co{rest} (if one insists, their \thi{how}), which for \co{reflection}
dissolves into \co{nothing}, is a \co{sign}, too -- the terminus of the
particular \co{trace} which, stretching through the current situation and
\co{moods}, feelings and \co{qualities}, anchors the \co{actuality} in its
ultimate \co{origin}.

\wtsep{2. not intuitions; constant; not limited region}

\pa Being in love finds its expressions in various \co{acts} and the ways of
performing them. In so far as \co{I} \co{act} from love, these \co{acts} are
\co{signs} of its \co{presence}. Sometimes, for some people, it may be necessary
to know that \co{I} am in love in order to apprehend the character of the
\co{rest} of my \co{acts}. But, typically, it is the other way around and the
\co{acts} eventually unveil this \co{rest}, their \thi{what} and \thi{how}
indicate, if not prove: {this guy is in love}.  On the other hand, \co{I}
\co{myself} may not know that \co{I} am, and the \co{signs} may be much more
\co{visible} to others than to me. As far as \co{I} am concerned, they are
\co{original} and not \co{reflected}. \co{I act} this love, \co{I} am lead and
forced to act in accordance with some \co{vague} intuition, according to some
\co{invisible command} which \co{I} do not grasp, perhaps, even do not know, in
any case, do not control. Love is not something \co{I} decide to experience but
something which \co{I} experience (or not); love, perhaps, toward a person
\co{I} would never expect \co{myself} to fall in love with, love which, perhaps,
\co{I} did not even want.  It dawns on me and then haunts me.

Just like a single moment of the \thi{murder in affect} reveals the underlying
conflicts of the person going, perhaps, to the very bottom of his being, so a
moment of loving intimacy may reveal and express the true love.  The expressed
conflicts or love \co{transcend} by far the \hoa, they are capable of infinite
manifestations and last far beyond their \co{totality}. And yet, the moment can
express them completely and adequately. Love finds its expression in every
moment and each such moment {incarnates} the whole \thi{essence} of this love.
Love is always more than its {incarnations}, \thi{overflows} any \co{actual}
expressions and, at the same time, is fully \co{present} in every true
\co{sign}.

\pa\label{pa:inspInt} We should make one reservation. We have used the word
\wo{intuition} which, in the usual sense (not in the sense of grasping the unity
of a \co{complex} in one \co{act}, as in \refp{pa:lab}), can seem appropriate
for the \oss\ of \co{invisibles}.  Strong intuitions have inspiring effect,
precisely by virtue of being on the one hand \co{vague} and, on the other hand,
definite, unconditional, \co{clear}.  It is this duality of \co{vagueness} and
some kind of definiteness of intuition which makes it so hard to ignore it and
let it go. It nags one and, having no \co{precise, objective} content, can not
be ignored until one follows it and finds out what it is intimating.  Intuitions
are usually only first announcements of something which, in due course, may be
unveiled and seen.  As Jung says, intuition is {\em perception} via the
unconscious. One has intuition {\em about} or {\em of something}.  With time and
effort, it will give place to specific explanations and \co{actual} reasons
which reveal their place in the \co{complex} from which they arouse -- it turns
out to be a \co{sign} of something \co{visible} which has only been hiding below
the threshold of consciousness.  Thus intuition is, in general, relative to a
particular region of Being.

The \co{signs} of \co{invisibles}, on the other hand, do not lead to any such
\co{actualisation}. The first \co{signs} of love may be \co{vague} and
\co{actually} imperceptible, but even when they become clarified, love is not
reducible to any \co{actual} insight nor, for that matter to any \co{acts},
\co{actions} or a \co{totality} thereof.  It does not reside in \co{acts} or
\co{activities} but, primarily, in their \co{rest}.  Intuition may be an
inspiration to follow its thread and \thi{figure out}.  Love, too, inspires one
before it finds an expression in \co{acts}. But its \co{inspiration} does not
end when \co{I} realise that \co{I} am in love. On the contrary, the
\co{inspiration} continues, increases and affects my whole being with an
atmosphere of strength and unlimited potential, with the sense of possibility to
perform not this particular \co{act} or that, but any \co{act} whatsoever.

Moreover, love is not limited to any particular domain of Being, it is not
restricted to any particular object or person.  For, although focused perhaps on
a particular person, it is love only to the extent it affects, transforms and
impresses the whole being with a \co{quality} and \co{command}: \citet{love, and
  do what you wilt}{AugustJohn}{VII:8} -- not a command to do this or that, but
a \co{vaguest} (not limited to {\em any} particular domain of Being), and yet
\co{clear} (intense and definite) \co{command} to do whatever you want {\em in}
love.\ftnt{Recall that \co{clear} is not opposed to \co{vague} but to
  \co{precise}, footnote~\ref{ftnt:clear}.}  To what \co{precisely} it
\co{inspires} remains undetermined and open, it will be determined by all kinds
of details -- the \co{command} is not a moral imperative to do this rather than
that, nor to do things in a prescribed manner. It is only a \co{command} to
listen to it, to remember its \co{inspiring} voice in all \co{actual}
situations. The \co{inspiration} is \co{clear}: \citt{Love is infallible; it has
  no errors, for all errors are the want of love.}{William Law} But also, since
it does not \co{command} any specific \co{acts}, it is \co{vague}: it
\citeti{does not perform any works; it is too subtle for that and is as far from
  performing any works as heaven is from earth.}{Eckhart, \btit{German
    Sermons}}{Luke I:26,28. The subject of the quotation is grace.
  [\citeauthor*{W}{ 29}; \citeauthor*{LW}{ 38}; \citeauthor*{EckSelected}{
    2\kilde{p.118}}]}


\wtsep{inspiration/command}

\pa The \oss\ of \co{invisibles} are not any feelings or concepts, are not any
insights or states of ecstasy, but \co{absolute} \co{commands}.  The character
of a \co{command} consists in the complete lack of reactive character.  These
\co{signs} neither are reactions to anything nor cause any specific reaction.
Paradoxically as it may look, this is exactly the \co{command} -- it challenges
but does not cause, it calls but does not force.\ftnt{We can say about these
  \co{signs} exactly what Bergson says about the mystics.  \citef{They have no
    need to exhort us.  They only have to exist, for their existence is a
    call.}{BergTwo}{}}  \co{Immediate signs}, like sensations, exemplify the
extreme opposite of a \co{command} in that there the \co{sign}, the signified
and the reaction coincide.  There is no \co{distance} between the \co{sign} and
reaction to it, no distance which could leave doubt and possibility of reacting
otherwise, no \co{distance} between the \co{actuality} and \co{non-actuality}
allowing the \co{actual sign} to challenge, to \co{inspire} a movement towards
the \co{non-actual}.  The \co{commanding} or \co{inspiring} character of a
\co{sign} is precisely this \co{distance} separating the \co{virtuality} of the
\co{invisible} from its possible \co{actualisations}.\ftnt{We might probably go
  as far as saying that any distance separating a \co{sign} from the signified
  is a command.  In any case, any experience of such a distance is also an
  experience of a command. In the most trivial case, it is merely the command to
  interpret the sign, to understand what it signifies, what it means. Here it
  has almost reactive character. We do not contemplate the commanding character
  of a road sign or a signpost -- we immediately understand it.  But the longer
  the \co{distance} separating the two (and it is something entirely different
  from the arbitrariness of an artificial sign), the more insistent the nagging
  to relate them.}

The \co{command} consists also in that it does not create any particular state
like, for instance, feelings do. There is no emotional or mental state
corresponding to love or holiness. A \co{command} acts also when it is not heard
and it can make itself heard at any time, in any situation, in any \co{mood}.
Likewise, it can be followed at any time, in any situation, irrespectively of the
variations of particular \co{moods} and \co{thoughts}.  \citt{Take ye heed,
  watch and pray: for ye know not when the time is. For the Son of Man is as a
  man taking a far journey, who left his house, and gave authority to his
  servants, and to every man his work, and commanded the porter to watch.  Watch
  ye therefore: for ye know not when the master of the house cometh, at evening,
  or at midnight, or at the cockcrowing, or in the morning: Lest coming suddenly
  he find you sleeping.  And what I say unto you I say unto all, Watch.}{Mk.
  XIII, 33-37}
                  
\wtsep{absolute} 

\pa The \co{absoluteness} of \co{commands} consists in that they are
not relative to any particular domain of Being.  Love which is limited
to a particular domain of \co{the world} may still be love but is not
the \co{absolute} love, because such a limitation to a particular
region of \co{the world} is the same as a limitation to a particular
region of \co{my life}. Both and any of these limitations make it
relative to this region. 

Saying that the \co{absolute commands} are not relative to any particular region
of Being is the same as saying that they concern the whole Being.  They
penetrate to and flow from the very depth of one's person, the point in infinity,
which is but the reflection of the infinity of Being. They do not concern any
being in particular and, by this very token, concern every particular being. But
they do so not by enumeration of all beings but by being seated in the very
center of personal being and thus are spread over all particular beings the person
may encounter.  Such \co{signs} either are given \co{absolutely}, with unconditional
validity, permeating one's whole being, or are not given at all.  A person can
not be \thi{partially holy}, just like one can be \thi{partially satisfied}. One
can not \thi{love a little but not entirely}, for such a thing is not love but
something else.

\wtsep{freedom}

\pa Instead of presenting some recognisable content, instead of providing one
with the imperative to do this rather than that, a \co{command} merely says
\wo{you shall love}. \citet{For commandments from the Lord should not be
  expected in matters that have an obvious usefulness.}{AbelardPJC}{I:\para 123}
Lacking any precise, \co{actual} content, the \co{commands} do not give any
reasons either, they do not provide any explanations or justifications.  They do
not try to convince but merely {manifest} and~\ldots leave one free.  Also
therein consists their \co{absoluteness}.\noo{One finds many expressions like
  \citf{God forces no one, for love cannot compel, and God's service, therefore,
    is a thing of perfect freedom.}{Hans Denk} The intuitions, I think, are the
  same.}

The freedom in confrontation with a \co{command} lies in the indeterminacy of
\co{actual} content.  But it may be, and often is, announced with an
irresistible force.  \Oss\ of \co{invisibles} may enter one's life in the most
rare moments of revelation, moments when \co{invisible} enters the \hoa\ with
imperative intensity.  They say that a dying person may experience his whole
life compressed into a single moment.  But one need not be dying.  There are
rare moments which reveal to us something fundamental, \co{inspirations} which
may turn out to determine our whole future life, or else, which show us the
meaning of our past life; moments, whether in dreams or in wake life, when the
content and meaning of the whole life seems to be compressed into a single
\co{sign}.  Such moments have a character of foundation, they insert into our
time \co{an experience} of meaning and value which exceed all \co{reflective}
understanding. These are the moments establishing \la{axis mundi}, founding the
cosmos out of chaos.  Although, in our \co{experience}, we might have lived
quite an orderly life before, confrontation with such moments has, then too, the
character of founding something which either gives a completely new direction,
or else lends extra strength of explicit \co{presence} to something which has
been only vaguely and implicitly intuited before.


%%
%%\pa
%%The highest \co{signs} reveal holiness -- of a person, of the \co{world}, of the
%%whole being. And again, to the extent they are \co{original}, one does not have
%%to reflect over being holy. A saint need not know of, and in any case, need not
%%be concerned with his holiness. Certainly, he may experience some
%%\co{reflective signs} of his being so but the point is that these \co{signs}
%%are not necessary for his awareness of holiness. It is sufficient that he
%%follows the \co{command} of adoration, humility and thankfulness.
%%
%%\pa\label{pa:weak}
%%In general, the \oss\ may be very subtle and weak intuitions
%%which do not force their truth on \co{me}. They cling vaguely in the
%%background of my thoughts and feelings, as if weak reminders of remote and
%%forgotten responsibilities. ...
%%
%%\pa
%%Scheler

%%%%%%%%%%%

\ad{Reflective signs}\label{pa:rsos} \co{Inspirations} do not reveal any
content, do not present anything which might be grasped in \co{actual}
consciousness. But they may be grasped by \co{reflection} precisely \co{as
  signs}. They appear empty since no \co{precise} content can be substituted for
them. And yet, \co{as signs}, they are not empty.\noo{\co{Inspiration} does not
  appear as a \co{precise concept} or thought -- it appears as a \co{vague},
  often hardly discernible, and yet clear and definite \co{intuition}.\ftnt{Let
    us only remind that \co{vagueness} is related only to indeterminacy of
    \co{actual} contents, which is something very different from \co{unclarity}.
    The opposite of \co{vague} is not \co{clear} but \co{precise}.}  But unlike
  an \co{idea} which never can be fully present, which is only an ideal limit of
  \co{complex} thought, what is \co{inspiring} admits full \co{incarnation} --
  it can be found \thi{in flesh}, in \co{this world}. A meeting with a holy
  person, an experience of love may give a full taste of the \co{invisibles}, of
  holiness, of love.  The similarity to an \co{idea} (and it is only a dry
  similarity, not any meaningful analogy) consists in that no \co{incarnation}
  is {\em the final one}, it will always admit other, different
  \co{incarnations}.  Thus, although we have met it \wo{in person}, \wo{face to
    face}, we still do not know, can not \co{actually} say, what it is, what its
  \thi{essence} might be.  Infinity of possible variations will never disclose
  any residual \thi{essence} but only an empty concept.  Indeed,
  \co{incarnations} are as unrepeatable as human persons -- each \co{birth} is
  an \co{incarnation}, it has no other \thi{essence} except \co{creation} of a
  new field of \co{manifestation} of the \co{invisible}.}\wtsep{1: symbols,
  vague, overflowing} \Rss\ of such essential \co{non-actuality} which never can
be reduced to an \co{actual} phenomenon are \co{symbols}.  A \co{symbol} does
not signify in the proper sense -- it merely \co{manifests}; although it does
bring forth something \co{vaguely distinguished}, its \co{inspiration} derives
from the \co{virtual signification}, from its pointing beyond the
\co{distinctions} towards their \co{origin}. The \co{symbolic} contents can
never be sharply \co{dissociated} from each other, for one immediately and
imperceptibly flows into another.  What in the Jungian analyses of unconscious
is called \wo{contamination} is such a \citet{moonlit landscape. All the
  contents are blurred and merge into one another, and one never knows exactly
  what or where anything is, or where one begins and ends.}{MHSFranz}{p.183} The
inseparability of \co{distinctions} is the main feature distinguishing the
\co{non-actuality} from \co{actuality}, and it only gradually increases as we
approach the ultimate \co{origin}.  We can certainly speak about \thi{symbols of
  God}, \thi{symbols of self}, \thi{symbols of transformation}, etc., but to the
extent these are \co{experienced symbols}, they do not emerge as so definitely
separated as they may appear when turned into \co{reflective} thoughts.  Genuine
\co{symbols} \citet{cannot be exhaustively interpreted, either as signs or as
  allegories. They are genuine symbols precisely because they are ambiguous,
  full of half-glimpsed meanings, and in the last resort inexhaustible. [\ldots]
  The discriminating intellect naturally keeps on trying to establish their
  singleness of meaning and thus misses the essential point: for what we can
  above all establish as the one thing consistent with their nature is their
  {\em manifold meaning}, their almost limitless wealth of reference, which
  makes any unilateral formulation impossible.}{JungArche}{I:80} Being
\co{signs}, that is, to the extent they appear, \co{symbols} are embraced within
the \hoa, but what they \co{manifest} is neither any specific content nor any
definite referent; it is immediately \co{recognised} as {\em essentially}
\co{transcending} this horizon.

\wtsep{2: collective, above one}

\pa \co{Symbols reflect} the \oss, the most individual \co{experience} of sacred
numinosity, the \co{experience} which confronts \co{me} not only with
\co{myself} but with \co{my self}.  \citet{When we attempt to understand
  symbols, we are not only confronted with the symbol itself, but we are brought
  up against the wholeness of the symbol-producing individual.}{MHSJung}{p.81}
This individual is not, of course, \co{oneself} nor one's \co{ego}, but
something greater than the individual himself. \co{Symbols} emerge through us
but they are not created by us, they are better thought of as \citetib{natural
  and spontaneous products. No genius has ever sat down with a pen or a brush in
  his hand and said: <<Now I am going to invent a symbol.>>}{MHSJung}{p.41}

\co{Transcending} thus {one's} personal sphere, \co{symbols} have a powerful
collective aspect.  As Jung's extensive investigations suggest, humans tend to
express the \co{experience} of \co{invisibles} (which, for the moment, we can
identify with his archetypes) by analogous, \co{symbolic} forms and ideas.
Whether \co{manifested} in dreams, in myths, in religious conceptions, or even
in philosophical \co{concepts}, the \co{invisible} sphere revealed by
\co{symbols} seems to be the deepest layer of human being, the collective (to
use Jung's term) aspect of the psyche, relatively independent from the personally
\thi{subjective} context and cultural tradition.  The deepest, the most personal
is exactly that which, being universally \co{participated} does not become a
commonality -- the \co{absolute}, unrepeatable \co{concreteness} of
\co{incarnation} of \co{invisibles}.  To the extent this becomes expressed and
embraced by a collective culture, it can happen only through \co{symbols}.


A \thi{holy stone}, a \thi{holy tree}, a \thi{holy brook} are \co{signs}
announcing the \co{presence} of sacrum. As Eliade aptly illustrates, they are
not worshiped \thi{in-themselves}, they are not \thi{the holiness itself}. They
are worshiped only because sacrum has marked its \co{presence} {\em at} these
places, because it has \co{manifested} itself {\em through} them. They are
hierophanies, the \co{signs} of sacrum which infinitely \co{transcends} them and
yet is \co{concretely present} in them.  They may serve as simplest examples of
\co{symbols}, the \co{visible, external objects} which \co{inspire} -- awe,
fear, wonder, reverence -- and \co{command}, not any specific \co{acts}, but
veneration and rituals which take their particular form from elsewhere, from
the myths, from the tradition, from the religious culture. 

%\newpa
\wtsep{3: personal - if not, become empty}

\pa \co{Transcending} {one's} personal sphere, \co{symbols} are nevertheless the
most personal things. To be a \co{symbol}, the \co{actual sign} must be
accompanied by the \co{original command}. \co{Symbols} are only
\co{externalised} and {objectified} \co{reflections} of the \oss.  Here lie of
course unlimited possibilities of discrepancies and conflicts between the
individual \thi{feelings} of the high, deep and reverent and the publicly
recognised \co{symbols} and accepted forms of their reverence.  The
\co{distance} separating the \co{actuality} of the \co{sign} from its meaning
is, in the case of \co{symbols}, virtually infinite.  The relation of
\co{signification}, once the \co{symbol} gets \co{dissociated} from the \oss,
seems completely arbitrary. Almost anything can become a \co{symbol} and there
is nothing easier than to ask: Why this tree? Why a tree?  Why the cross? Why
this and not that? Why anything at all? -- and then conclude that there is no
reasonable answer. \co{Symbols} become \thi{mere symbols} for all too intense
\co{reflection} which notices that mere signs are \co{actually dissociated} from
any real \co{presence} which they should announce 
with some forcing necessity. But even then \co{symbols} can act as
reminders of this \co{presence}, whose \oss\ have been forgotten underneath the
\co{visible} expressions.

\pa
Establishing \co{symbols} is one of the fundamental needs and {activities},
as they are the only \rss\ connecting the \co{actual} consciousness with the
sphere of \co{invisible presence}.\ftnt{It can be seen in the seriousness with
  which children, in their games, {\em are} mothers, fathers, policemen or
  arrange doll houses and build models -- without slightest disturbance by the
  fact, of which they are perfectly aware, that these are only games, toys.
  Likewise, the pictures from Lascaux are hardly mere traces of boredom or
  depictions of nothing but daily trivialities. Extensive studies of Eliade and
  others suggest that the maintenance of the symbolic, and yet concrete,
  proximity of sacrum to the sphere of profanum is one of the founding aspects
  of the earliest cultures.}  But \co{signs} become \co{symbols} only when they
\co{actually manifest} the \co{invisible}, that is, only when they are met and
experienced along with the respective \oss.  What constitutes a \co{symbol} is
the double aspect of the \co{invisible} flowing in through the \co{visible}, of
the \co{inspiration} arising through the \co{actual sign}. The \co{inspirations}
are not any emotions but, without picking on such details of expression, we
could say that the archetypal \co{inspirations} \citet{are, at the same time,
  both images and emotions. One can speak of an archetype only when these two
  aspects are simultaneous. When there is merely the image, then there is simply
  a word-picture of little consequence.  But by being charged with emotion,
%[feeling = value judgment for Jung], 
the image gains numinosity (or psychic energy); it becomes dynamic, and
consequences of some kind must follow from it.}{MHSJung}{p.87}

\co{Reflection} devoted exclusively to the petty and all-important matters of
its \co{actuality} is simply unable to meet a \co{symbolic} expression, even if
it meets its \co{visible sign}. The meaning of a \co{symbol} has close to
nothing in common with the meanings discernible at the level of \co{actuality},
the meanings of \co{precise} words, \co{concepts} or particular
\co{impressions}. And in the moment the \co{invisible presence} is declared
unreal, a \co{symbol} degenerates to an empty \co{sign}.  In the moment a
\co{symbol} starts signifying something \co{visible}, it becomes an allegory,
eventually, a mere \co{sign}. 

Put differently, a \co{command} is meaningful only in so far as it is not
\co{dissociated} from its \co{origin}.  It has unconditional validity only for
the one who hears it.  The living relation to the \co{actual} person is its true
nature.  Stripping \co{symbolic origin} of its intrinsically \co{invisible}
character (that is, \co{externalising} it in an {objectified} form,
\co{dissociated} from the reality of its \co{manifestation} through the living
person) leaves only arbitrariness of an artificial \co{sign} and the
incomprehensible \thi{so it is}. Such \co{symbols} may preserve some element of
the mystical character, but they lose their \co{commanding} force.  They may
then function as mere messages, \co{signs} pointing to \co{another world} in an
indifferent, anonymous way.  This is what happens to \co{symbols}, whether in
literature, painting or mythology, when they have been \co{dissociated} from
their \co{invisible} meaning.  They appear as arbitrary.  Empty \co{symbols} are
the \co{original commands} turned by tradition, culture, repetition or personal
estrangement into {\em mere indications}, pointers towards nothing specifically
discernible and therefore devoid of any concrete meaning.  Their originally
\co{vague} meaning and their lack of any identifiable referent turn into lack of
meaning and emptiness of denotation.  At best, they only try -- deficiently and
unsuccessfully - to indicate something vague, unknown, which \citetib{is never
  precisely defined or fully explained.  [And one can not] hope to define or
  explain it.}{MHSJung}{p.4} But \co{symbol} never explains {\em what} it is
saying -- it only says it.  It is a pure expression, totally open to
misinterpretation, which in particular means, to being ignored. At the same
time, it is entirely \co{clear} to the one who happens to grasp it, because to
grasp it means to already know what it expresses -- the \co{symbol} is only a
means of \co{actualising} this \thi{knowledge}, making it conscious.  Hence
\citetib{[t]o the scientific mind, such phenomena as symbolic ideas are a
  nuisance because they cannot be formulated in a way that is satisfactory to
  intellect and logic.}{MHSJung}{p.80}


\subsub{The invisibles} There are things which do not belong to \co{this world}
in the way tools, commodities, situations, daily objects, relations, feelings
and thoughts do, but which are from \co{another world}; world which does not
obey our dictates but which is the source of gifts and calamities surpassing our
powers.  They are from \co{another world} but this \thi{otherness} is not
absolutely foreign, alien -- \co{another world} is still the \co{world}.
Although \co{transcending} the sphere of phenomena, of all \co{actual
  experiences}, they \co{manifest} their \co{presence} in such \co{experiences},
they too enter the horizon of one's \co{experience}.  But even when encountered
in a single moment, in a single \co{act} of \co{actual} consciousness, one
always knows that what is so encountered is only a \co{sign} of something that
is \thi{greater}, something that only \co{manifests} itself without exposing
itself.

\pa %\ad{Symbols -- only incarnated}
\co{Symbols} and \co{commands} do not reveal anything definite.  One might be
tempted to say that there is therefore no distinction between the \co{signs} and
what they signify, that the only things made present are the very \co{signs}.
But such a description, which might be given at the level of \co{immediacy}, is
not adequate here.  For \co{signs} announce here not so much anything particular
but only the \co{distance} to whatever they may be announcing. They
\co{manifest} something \co{invisible}, something more than not only themselves
but than any \co{actuality} of \co{an experience}.  A mere \co{sign} \citetib{is
  always less than the concept it represents, while a symbol always stands for
  something more than its obvious and immediate meaning.}{MHSJung}{p.41} The
\co{commanding} character of a \co{symbol} does not determine any immediate
reaction, on the contrary, it only \co{inspires} to look for possible ways of
\co{actualising} the \co{manifested} \co{invisibles}.

However, this \co{invisible} something becomes \co{present} only through its
\co{manifestations}, it cannot be meaningfully abstracted from them. When
\co{posited} as a \co{dissociated} entity, it loses its fundamental feature,
\co{concreteness}, and becomes a \thi{mere word}, an empty concept. Its
\co{concreteness} is this impossibility to \co{dissociate} it from its
\co{manifestations}, to abstract from them any common \thi{essence}, without
losing the thing itself. It is \co{concrete} because, to the extent it is, it is
\co{manifested}, and to the extent it is not \co{manifested}, it is an empty
concept.

To be \co{invisible} is to be \co{essentially non-actual}, is to be a
\co{distinction} which can never be fully embraced -- as a \co{concept} or
\co{an experience} -- within the \hoa.  \citeti{The gentle flame of eye did
  chance to get\lin Only a little of the earthen part.}{Empedocles}{ DK 31B85}
\co{Invisibles} are not any \co{objective} entities and yet they are
\co{experienced} because they are \co{present} in our \co{experience}.  They are
\co{distinctions} without anything distinguished, appearances without objects,
phenomena without the noematic correlates, powers without any identifiable
center.  Being essentially \co{non-actual}, the \co{invisibles} can
\co{manifest} themselves for \co{reflection} only through \co{symbols} --
\citet{our only approach to divine things is through symbols.}{DDI}{I:11.32}
They \citet{spring from a deep source that is not made by consciousness and is
  not under its control. In the mythology of earlier times, these forces were
  called \la{mana}, or spirits, demons, and gods. They are as active today as
  they ever were.}{MHSJung}{p.71}


\pa %\ad{Objective}
\co{Invisibles} are \co{absolute}, that is, they are not relative to any
particular region of Being. Their \co{presence} precedes any \co{recognisable}
\co{distinctions}, and hence embraces the whole person, before one can \co{act}
and protest. They cast their shadow (or rather their light), as irrevocable as it is
ingraspable, as intense as it is \co{indistinct}, on all particular beings and
\co{actual objects}.

An aspect of this \co{absoluteness} is independence from any lower feelings and
thoughts. They have no well-defined expression and can be \co{present} in any
kind of particulars.  Thus, they allow variations at the lower levels which do
not affect their \co{presence}. Love remains love independently from the
feelings, \co{moods} and \co{sensations} one might experience in all particular
situations. In fact, their experiences will be affected by the love which lies
 \co{above} them.

%\ad{Indeterminate}
Irreducibility to and inexhaustibility by \co{actual} determinations means
furthermore the complete indeterminacy of the form, shape, character and context of
possible \co{manifestations}.  There is no unique, well defined range
of phenomena which exhaust the possible \co{manifestations} of
sainthood, love, ontological \co{thirst}, damnation. There are no \co{objective}
criteria telling one from another.  This is just another side of the
\co{concreteness} of \co{invisibles}, of their inseparability from
\co{manifestations}.  For though we lack any \co{actual} determinations,
though we lack concepts of love, sainthood, hatred, we do not lack
experience; we can \co{recognise} them
when meeting them in experience.

\pa\label{pa:anamnesisB}
Each person is an unrepeatable, that is, original source of variations, always
new variations over the same theme of \co{existential confrontation}, which
begins (just after {\em the} beginning) with the \co{invisibles}.  Nobody can
teach anybody what love means and 
how to love except, possibly to some extent, by the very example, that is, by
offering the \co{experience} of love.  Nobody can teach anybody what
it means to be a mother or father, for even the best (or worst) examples, may
eventually result in one being quite the opposite when 
playing the role oneself.  Even the destitute children raised without one
or both of the parents, know what motherhood or fatherhood means, if not in
other ways then simply by living their lack and thirsting for them.

This is much more adequate sphere for the application of Plato's
\gre{anamnesis} simile\noo{The word \wo{theory} would be vastly
inappropriate here.} then the field of \co{concepts}, \thi{essences} and generalities.
 \co{Invisibles} pertain to everybody's \co{experience} and
their structural similarity is far stronger than apparent differences
in contents, in \co{objective} particulars.  Encountering
love, hatred, mystical experience, spiritual strength, we suddenly
\ldots \thi{remember}.  We never know for sure, at least not at once: is {\em
this} love or 
not, is {\em this} sainthood or not. But the very doubt whether this
is {\em it},
\noo{(the reference to the respective \co{signs}, words, images),}
witnesses to the fact of \co{recognition}.  Even if we never
experienced it before, we know (\co{vaguely} and \co{imprecisely})
what we are meeting now for the first time.  The doubt is almost
unavoidable because it only reflects the complete lack of any
universal and \co{objective} characteristics, the thoroughly personal
dimension of such \co{experiences} and their irreducibility to any \co{actual signs}.
\noo{which call for a new, individual \co{manifestation}.}

\pa %\ad{flowing into each other}
\co{Invisibles}, as the first and deepest \co{distinctions}, form the sphere
surrounding the ultimately \co{invisible origin}. They are like \citet{the
  intellection that remains within its place of origin; it has that source as
  substratum but becomes a sort of addition to it in that it is an activity of
  that source perfecting the potentiality there, not by producing anything but
  as being a completing power to the principle in which it
  inheres.}{Plotinus}{VI:7.40}

In terms of the figure from~\refp{fig:levels}, p.\pageref{fig:levels}, the
\co{invisibles} are the most dense nuclei on the circle closest to the origin
$\bullet$, reflecting the part of the line to the left of L and right of R which
never enters the \co{actual experience}. They have no \co{objective}, nor even
{objectifiable} correlates, nothing \co{actual} can ever fully \co{represent}
them, no \co{actual sign} can ever coincide with them.  The \co{invisible}
contents may vary (say, depending on where, on the line, the circle is), but the
universal fact of primary importance to us is the very \co{presence} of this
sphere in our being and \co{experience}. The structural relation of this sphere
to the others will remain constant for a given circle and identical for all
beings \thi{of this kind}.

Still in terms of this figure, the \thi{closeness} to the \co{origin} means also
the inseparable connections, dense beyond the possibility of \co{dissociate
re-cognitions}.\ftnt{Cf. comments in Book I on figure in \refp{pa:stages},
  especially, \refp{pa:density} and the footnote~\ref{ftnt:density},
  p.\pageref{pa:density}.} In terms of \co{actual reflection} we like to
consider the problem of freedom, then of truth, then of meaning, then of love,
each for itself. But we very quickly realise that to the degree we succeed in
such a \co{dissociation}, the treatment and the results become so much more
sterile. Each of \co{invisibles} \citet{contains all within itself, and at the
  same time sees all in every other, so that everywhere there is all, and all is
  all and each all [...] In our real all is part rising from part and nothing
  can be more than partial; but There each being is an eternal product of a
  whole and is at once a whole and an individual manifesting as part but, to the
  keen vision There, known for the whole it is.}{Plotinus}{V:8.4}

Finally, interpreting the interval between L and R as the range of
\co{experiences} in the circle's life span, the changes -- as the circle moves
around -- in the sphere of \co{invisible} contents will be extremely slow
compared to the changes in the \co{actual experiences}.  \citet{Whereas we think
  in periods of years, the unconscious thinks and lives in terms of
  millennia.}{JungArche}{VI:499} The \co{invisibles} are the most constant
\co{aspects} of \co{experience}: the movements of the circle involve major
changes in the \co{actual} contents (closest to the point where the circle
touches the line), while the higher, \co{invisible} constellations remain
virtually unaffected by the changes of the circle's position.  Unlike changing
\co{actual} news, \co{invisibles} are always \co{present}, even if not
\co{manifest} in \co{actual signs}. This is the meaning of \co{presence} which
is very different from \co{actualisation}.  The latter involves explicit
presence, \co{actuality} of the \co{sign} or \co{object}; it is a matter of the
specificity of the moment which is dominated by a particular \co{sign} of a
\co{visible} or \co{invisible} content.  \co{Presence}, on the other hand, does
not require any explicit givenness; it denotes constant proximity of
\co{invisibles}, felt or not, as if in the background of, and hence independent
from, the \co{actuality} of our attentive observation. \co{Manifestations} are
\co{aspects} of these \co{actual experiences} in which \co{presence} comes forth
and becomes strongly \co{experienced}, even if it does not become the \co{actual
  object} of these \co{experiences}.

\pa If we only keep in mind the difference between \co{experiences} and
\co{experience} (which, perhaps, comes closer to experiencing), we would say that
\co{invisibles} offer the ground of all \co{experience} and, as such, are
themselves \co{experienced}:
\begin{itemize}\MyLPar
\item   
as the contentless indeterminacy, respecting one's freedom, the \co{invisibles}
offer \co{experience} of \co{nothingness};
\item   
as the overflowing surplus and inexhaustible potential for ever new
\co{manifestations} -- \co{experience} of the \co{origin}, the source of meaningfulness; 
\item   
as the \co{transcendence} unaffected by \co{my}
choices and actions -- \co{experience} of eternity; 
\item   
as the \co{inspiration} and \co{command} -- 
\co{experience} of \co{absolute} power\ldots
%\item   \ldots 
\end{itemize}
\noo{ check again Rudolf Otto}


\subsubi{Invisible or simply
  unconscious?}\label{sub:notunconscious}\label{se:JungPlotinus}
%
Most characteristics of archetypes (with few exceptions to be discussed below)
given by Jung can be applied to the \co{invisibles} as well.  They are not any
\co{visible} contents capable of being grasped within the \hoa, they are not any
specific \co{representations}, nor any mythological images or motifs. At best,
they stand for \citet{a tendency to form such representations of a motif --
  representations that can vary a great deal in detail without losing their
  pattern.}{MHSJung}{p.58} \citetib{They grow up from the dark depths of the mind
  like a lotus and form a most important part of the subliminal
  psyche.}{MHSJung}{p.25} We should, however, clarify one main difference which
concerns the status of the unconscious contents vs. \co{invisibles}.

\pa There are many known examples of scientists \thi{receiving}
solutions to their problems from unconscious.  Often these come from
dreams, like Kekule's dream of a snake biting its tail or von
Neumann's dreams of the actual proofs of his theorems.  Gauss tells about
a theorem which he found \citeti{not by painstaking research, but by the
Grace of God, so to speak.  The riddle solved itself as lightning
strikes, and I myself could not tell or show the connection between
what I knew before, what I last used to experiment with, and what
produced the final result.}{C.~F.~Gauss \citaft{MHSFranzB}{ p.383/385}}{} Intense
engagement in some well-defined problem will often stimulate the mind
to carrying further work, apparently at the same level of 
\co{precision}, although not involving \co{active reflection}.

Fascinating as such events may be, they are not exactly what we are aiming at
here. What emerges in such cases are \co{actual} contents expressed
\co{precisely} in the categories of conscious thinking.  True, they emerge from
the unconscious, but it is only the process which is unconscious -- the initial
input as well as the results are thoroughly \co{precise} contents of
\co{reflective} thinking.  A slightly different aspect may be adumbrated in the
apparently quite analogous experiences of artists.  Klee: \citeti{My hand is
  entirely the instrument of a more distant sphere. Nor is it my head that
  functions in my work; it is something else \ldots}{P.~Klee \citaft{MHSJaffe}{
    p.308}}{} Pollock: \citeti{When I am in my painting I am not aware of what I
  am doing. It is only after a sort of `get acquainted' period that I see what I
  have been about.}{J.~Polloc \citaft{MHSJaffe}{ p.308}}{(By the way, these two
  quotations illustrate also, in addition to the common aspect which concerns us
  here, the enormous difference between the intellectual poetry of Klee's and
  the uncontrolled expressionism of Pollock's paintings.) See also footnote
  I:\ref{ftnt:artist}, p.\pageref{ftnt:artist}.}  Although the process is
equally unconscious, the initial input is probably of a different order than in
the case of the scientists.  Often, there may be no discernible input
whatsoever, not even a hunch, but a mere impulse \thi{now I should/can paint}.
Although much conscious work may precede and be involved in the process of
artistic creation, the consciousness is here concerned with contents of a
different order than those of scientific consciousness.  What is received by an
artist is not a ready-made solution to an \co{actual} problem, but a
\thi{guidance}, as if by a \thi{directing force}, during the process ending with
the \co{actual} expression which so, perhaps after a \thi{get acquainted}
period, is seen as a \thi{match}, as a satisfying \co{actualisation} not of any
preconceived idea, but of the initial, \co{vague} %and hardly identifiable
intuition.\noo{The corresponding \thi{intuitive guidance} can be, sometimes,
  observed in great scientists -- not, however, so obviously at the level of
  single results as in terms of the whole life\ldots (Einstein catching
  light-beam)}

\pa
This should indicate the fundamental difference:
the difference not so much between conscious as opposed to 
unconscious, as between \co{actual} as opposed to \co{non-actual}. 
 The two distinctions are 
orthogonal, they cut the horizon of experience along pretty 
independent lines. 
One can be 
\begin{itemize}\MyLPar
\item[1.a] \co{reflectively} conscious of the tree one is looking at or 
\item[1.b] merely \co{aware} of it (which, probably, would be 
counted as being unconscious of it, since the fact that one does not 
stumble into trees, although one does not pay any \co{reflective} 
attention to them, is credited to unconsciousness or subconsciousness). 
\end{itemize}
But one can also be
\begin{itemize}\MyLPar
\item[2.a] \co{reflectively} 
aware of the indefinable thirst of one's soul, of a vague dissatisfaction 
with \ldots \fre{je ne sais quoi}, or
\item[2.b] entirely unconscious (only \co{aware}?) of it. 
\end{itemize}
Our distinction \co{actual} vs. \co{non-actual} is that between 1.  and 2.,
while the distinction conscious-unconscious is, in each case, that between a.
and b.\ftnt{We gloss over more detailed differences like, for instance, that
  with Jung consciousness involves necessarily opposites, while with us only
  sufficiently \co{precise distinctions}, of which opposites are extreme cases.
  Also, since our consciousness spans everything from \co{awareness} to
  \co{reflection}, we have obviously the degrees of consciousness.  The extreme
  of \co{awareness} will often be the same as psychoanalysis' subconsciousness.
  Perhaps the most significant is that Jung's consciousness is the {\em
    totality} of contents related to his \thi{ego}, which seems to be simply
  constituted as the subjective pole of this totality.  With us, \co{reflection}
  is always only an \co{actual act}, and the \thi{conscious ego} is nothing but
  the \co{actual subject} of such an \co{act}. The \co{totality} of such
  \co{acts} \co{transcends reflection} and pertains to \co{oneself} but in no
  way constitutes it.}


\noo{Jung says: \citet{By consciousness I understand the relation of psychic
    contents to the {\em ego}, in so far as this relation is perceived as such
    by the ego. Relations to the ego that are not perceived as such are
    unconscious. Consciousness is the function or activity which maintains the
    relation of psychic contents to the ego. Consciousness is not identical with
    the {\em psyche} because the psyche represents the totality of all psychic
    contents, and these are not necessarily all directly connected with the ego,
    i.e., related to it in such a way that they take on the quality of
    consciousness.}{JungTypes}{\para 700 \citaft{CC}{ II;p.36}}
  
  Consequently, unconsciousness is a container of all kinds of contents:
  sensations, perceptions and all kinds of physiological process not coming to
  the threshold of consciousness, but also feelings and yearnings, concepts and
  understandings, the transcendent God-image and the Self in so far as these are
  not addressed explicitly by the \thi{ego}.  }


\pa\label{pa:Freud}
Certainly, there is a big difference between being conscious and 
unconscious of something. But what matters much more is that {\em of 
which} we are conscious (or unconscious), and what we make of the contents of
our consciousness. 

Freud made unconsciousness pretty much the same as \co{reflection}, only \ldots
unconscious.  Its contents where repressed {\em conscious} contents. Only for
this reason one might postulate (as done constantly by Freud)\noo{, for
  instance, in \citeauthor*{FreudN})} that \thi{id} should be replaced by
\thi{I}, that \thi{I} should keep bringing under its control more and more
aspects of the unconscious \thi{id}, as if the ultimate (even though impossible)
goal were to eradicate the latter making all its contents \co{visible}.\noo{In
  the standard American terminology \thi{it} is called \wo{id} and \thi{I}
  \wo{ego}, but we follow closely the critique of such Freud translations
  presented in \citeauthor*{BettelFreud}, which accuses them not only for
  literal inadequacy and linguistic sterility but also for actual falsification
  of the spirit of Freudian humanism.} The main complication in this extension
of consciousness was a complex of mechanisms working to \thi{hide} the
unconscious (though always principally \co{visible}) contents. Thus, for
instance, for the dream interpretation, one had to invent a \thi{censor}, a
function of the psyche which twisted and confused all the \co{precise} contents
of unconscious in order to hide them from consciousness.  But the \citet{form
  that dream takes is natural to the unconscious because the material from which
  they are produced is retained in the subliminal state in precisely this
  fashion.}{MHSJung}{p.53} It is natural in so far as the \co{non-actual}
contents are not expressible directly in the \co{precise, reflective} form.

Even if to some degree unconsciousness indeed hides only repressed \co{visible}
contents, there is much more which remains essentially \co{invisible}. Jung's
departure from the Freudian psychoanalysis of merely \co{visible} but repressed
contents, and his study of the \thi{collective unconscious}, that is, of the
transpersonal and not merely private and subjective dimension of the
\co{experience}, is an admirable spiritual achievement of the XX-th century.
The \co{invisible} contents which he finds through dream analysis carry this
character of \gre{anamnesis}, of something which, although appearing for the
first time for consciousness, does not originate in it and yet can be
\co{recognised}.  \citetib{[\ldots] I have found again and again in my
  professional work that the images and ideas that dreams contain cannot
  possibly be explained solely in terms of memory. They express new thoughts
  that have never yet reached the threshold of consciousness.}{MHSJung}{p.26}

The assumption of psychoanalysis (at least, in its folklore) is that
there is nothing which, at least in principle, could not become
conscious. The unconscious is most intimately \co{present} and we are
\co{aware} of it, although there may be a long way from this \co{awareness}
to the full \co{visibility} in \co{reflection}. The important thing, in so far as
such a \thi{making conscious} is concerned, is that contents entering
\co{reflection} still retain fundamental mutual differences. Becoming
(\co{attentively}) conscious of the tree \thi{I did not see} is very different
from becoming conscious of the vague dissatisfaction I have felt but did not
realise. \co{Reflection}, whether of a tree or of dissatisfaction, is fully
\co{aware} of such differences, even if they do not become its \co{objects}.
They are \co{recognised}, so to speak,
in the background of the \co{reflective acts}, in \co{self-awareness}.

The \co{invisibles} are essentially \co{non-actual} and not essentially
unconscious. Yet, the consciousness of \co{invisibles} is of a very different
kind from the usual consciousness of \thi{this or that}.  The difference is
established by the \co{distance} separating the \co{actual sign} from the
content it signifies.  In case of an \co{external object} like a tree, the
\co{distance} is negligible.  In case of Prague, it becomes more apparent, even
if one sticks to thinking of Prague merely as a \co{complex} of \co{actual
  objects}.  In case of the \co{invisible unity} of the world, of the \co{vague}
intuition and \co{imprecise} feeling of dawning love or despair, and then of the
\co{clear} -- and still equally \co{vague} -- consciousness of love or despair,
the \co{distance} is obvious and given in the immediate \co{awareness} that what
one is \co{actually} conscious of does not capture that which one is
\co{experiencing}, \thi{the thing itself}.  Paradoxically as it may seem, the
longer this \co{distance}, the more \co{concrete} the content, that is, the
deeper it reaches into the texture of the personal being. With \co{invisibles},
the virtual infinity of this \co{distance} is an \co{aspect} of the \co{absolute
  concreteness} of the \co{experience} overflowing the \co{actuality} of
conscious \co{signs}.

\subsubi{Invisibles, archetypes and intellect}\label{sub:invArch}

\pa With the reservations just made, the archetypes carry the characteristics of
\co{invisibles}. They form, as Jung calls it, the \thi{collective unconscious},
the unconscious structures which, being \co{shared} by all humans, are not
relative to the individual, personal experience, even if they become
\co{manifest} only in such \co{experience}.  They emerge from \co{above}
(\wo{inborn} or \wo{inherited}, as the word may go), from the sphere which is
not underlied one's will and control -- they are personal but not
\co{subjective}.  They are expressed in symbols and rituals of all religions and
mythologies. At the same time, they are the most \co{concretely experienced}
events of psychic life, in which they find thoroughly individual expression.
Being the most universal in their \co{vague} and indefinite form, they also
\co{found} the most \co{concrete}, the most individual. When adequately
\co{recognised}, they bind the individual to the collective history and
tradition, which provide objectified, symbolic expressions of the deepest
\co{aspects} of personal \co{experience}.

\noo{
  \wtsep{1. Closest to the origin}

\citet{There is also the intellection inbound with Being- Being's very author-
  and this could not remain confined to the source since there it could produce
  nothing; it is a power to production; it produces therefore of its own motion
  and its act is Real-Being and there it has its dwelling. In this mode the
  intellection is identical with Being; even in its self-intellection no
  distinction is made save the logical distinction of thinker and thought with,
  as we have often observed, the implication of plurality.}{Plotinus}{VI:7.40}
}

\wtsep{1. Meshed with each other}

\pa Remaining closest to the \co{origin}, they can not be differentiated and
\co{precisely dissociated} from each other.  They can, at best, be recognised
beyond the symbolic expressions, always endowed with a perpetual overflow of
meanings which never reach any final form. The impossibility of \co{precise}
division and description is due to the surplus of meaning flowing into an
archetype once we attempt to isolate it from the surrounding archetypes.  They
simply can not be meaningfully \co{dissociated} from each other, even if some
patterns seem to be discernible.  \citet{It is a well-nigh hopeless undertaking
  to tear a single archetype out of the living tissue of the psyche; but despite
  their interwovenness they do form units of meaning that can be apprehended
  intuitively.}{JungArche}{IV:302} The archetypes represent the \co{invisible},
{\em essentially} \co{non-actual} contents, which emerge before, and hence
remain forever \co{above}, the \co{actual} consciousness. They are inaccessible
to the \co{dissociated} categories of \co{reflective} thinking for the
\citet{diversity within the Authentic depends not upon spatial separation but
  sheerly upon differentiation; all Being, despite this plurality, is a unity
  still; "Being neighbours Being"; all holds together.}{Plotinus}{VI:4.4. The
  interpreters can certainly inquire into the \co{precise} meaning of the
  difference between \thi{spatial separation} and \thi{sheer differentiation}.
  But we can take it to represent the difference between \co{dissociation} and
  initial, \co{vague distinctions}. The impossibility of drawing \co{reflective
    distinctions} does not mean the lack of any \co{distinctions}, and the
  sphere of \co{invisibles}, although differentiated, can also be unitary and
  indivisible. \co{Invisibles} are \co{distinct} but can not be
  \co{dissociated}, \citefib{the Intellectual-Principle is the authentic
    existences and contains them all -- not as in a place but as possessing
    itself and being one thing with this its content. All are one there and yet
    are distinct \lin The Intellectual-Principle entire is the total of the
    Ideas, and each of them is the [entire] Intellectual-Principle in a special
    form. [...] The recipient holds the Idea in division, here man, there sun,
    while in the giver all remains in unity.}{Plotinus}{V:9.6\lin V:9.8} There
  \citefib{every being is lucid to every other, in breadth and depth; light runs
    through light. And each of them contains all within itself, and at the same
    time sees all in every other, so that everywhere there is all, and all is
    all and each all.}{Plotinus}{V:8.4} In the language of Eriugena, the
  primordial causes are one, although their \co{manifestations} vary, and so one
  speaks about them in plural.
%\citf Rightly also are the primordial causes called incomposite. For
%they are simple and entirely lacking any composition.
  \citef{For there is in them the inexpressible unity and the indivisible and
    incomposite harmony which go beyond every combination of parts whatever
    [...]  before they entered into the plurality of the {\em spiritual}
    essences no created intellect could know of them what they were [...] [They]
    are always invisible and dark.}{Periphyseon}{II:550BC;550D/551A;551B/C} They
  are comprehended by the intellect, through which they pass to reason before
  being diversified by the senses: \citefib{everything which the intellect by
    its gnostic view of the primordial causes impresses upon its art, that is,
    its reason, it distributes through the sense [...]  All essences are one in
    the reason; in sense they are divided into different
    essences.}{Periphyseon}{II:577ABCD/578A} Scholastics used to ask analogous
  questions about the differentiation of angels which, although distinct, were
  lacking material body, and thus seemed to lack also any principle of
  individuation.} This \co{original unity} is no longer \co{one}, but involves
already some differentiation: too weak to be grasped \co{reflectively}, but
thoroughly real and effective. The \co{invisibles}, \citet{the objects of
  intellection \noo{the ideal content of the Divine Mind} -- identical in virtue
  of the self-concentration of the principle which is their common ground --
  must still be distinct each from another; this distinction constitutes
  Difference}{Plotinus}{V:1.4} or, perhaps, \fre{Diff\'{e}rance}.


\wtsep{2. being=thinking}

\pa\label{pa:objsubjA} Jung operates with the understandable distinction of the
level of \co{mineness} between man, his psyche and the objective, external
world.  The archetypes are the deepest structuring elements in the psyche, yet,
they are \citet{sheer objectivity, as wide as the world and open to all the
  world.  [In the collective unconscious] I am the object of every subject, in
  complete reversal of my ordinary consciousness, where I am always the subject
  that has an object.  There I am utterly one with the world, so much part of it
  that I forget all too easily who I am.  \thi{Lost in oneself} is a good way of
  describing this state.  But this self is the world, if only consciousness
  could see it.}{JungArche}{I:46} Consciousness of archetypes, even though it
remains in the \co{actuality} of an \co{act}, involves suspension of the
\co{reflective dissociation} into \co{subject-object}. More importantly,
however, at the level of \co{invisibles} such a \co{dissociation} simply does
not obtain. Sainthood has no \co{object}, just like genuine \co{love} does not
have any, they are not opposed to any \thi{outside} but contain the whole world
\thi{within}. They do not \co{act} on any \co{external objects}, for \co{actual
  objects} are not goals of their \co{actions}, but only occasions for
\co{manifestation}. This constant \co{presence} of the \co{rest}, of \co{clear}
if undefinable \co{inspiration}, and the character of \co{expression} rather
than of directedness towards any external goals, gives a sound, if not complete,
account of the unitary and \co{self}-oriented character of the intellect, this
first hypostasis which \citet{apprehends itself and is object of its own
  activity.}{Proclus}{\para20.  For \citef{a principle whose wisdom is not
    borrowed must derive from itself any intellection it may make; and anything
    it may possess within itself it can hold only from itself: it follows that,
    intellective by its own resource and upon its own content, it is itself the
    very things on which its intellection acts. [...] An Intellectual-Principle
    and an Intellective Essence, no concept distinguishable from the
    Intellectual-Principle, each actually being that
    Principle.}{Plotinus}{V:9.5/V:9.8}}

In epistemic terms: \citet{[t]he intellect's thinking is not true because it
  conforms to or corresponds to the ideas; it is true because it {\em is} the
  ideas, which are its thoughts.}{EmilssonIntellect}{2;p.29}
\label{pa:thinkBeingB} As we saw already at the level of \co{mineness},
\refpf{pa:worldislife1}, the distinction between \co{the world} and \co{my
  world}, and then between \co{my world} and \co{myself} or \co{my life} is
close to impossible to draw in a meaningful, not to mention \co{precise},
manner.  But here we encounter the true Parmenidean identity, where
\citeti{being and knowing are identical because if a thing does not exist no one
  knows it, but whatever has most being is most
  known.}{Eckhart}{\citaft{CC}{II;p.35} The argument certainly leaves something
  to be wished for but, not focusing on the arguments, we will not let this
  prevent us from accepting the conclusion.} Of course, \thi{knowing},
\thi{thoughts}, \thi{thinking} and \thi{being} must not be taken here in the
\co{reflective} sense involving the \co{actuality} of an \co{act} and
\co{dissociation} of its \co{subject} and \co{object}. In the \co{spiritual}
sphere there is as yet no such distinction. The differentiation of
\co{invisibles} is the condition \co{founding} the very possibility of
\co{experience}, and of \co{dissociated experiences}. Without these primordial
\co{distinctions}, no \co{actual objects} could ever appear. Consequently, in
the \co{spiritual} sphere, at the edge of \co{nothingness}, \wo{being} and
\wo{knowing} are synonymous -- not because they happened mysteriously to
coincide, but because they have not as yet been \co{distinguished}, because
addressing \co{nothingness} there is not, as yet, enough material to
\co{distinguish} the two.  The \co{spirit} remains, since the beginning,
\wo{\co{above} the waters} and its \co{unity} is not affected by all the
\co{actual distinctions} and affairs of \co{this world}.


\wtsep{3. known directly = not reducible to representation/actuality}

\pa The crucial feature of archetypes is that they are inexpressible in
\co{precise} \co{concepts}, in the categories of \co{reflection} and
\co{actuality}, in short, in images. They can be \co{symbolised}, but the
characteristic of the \co{symbolic} relation is the complete lack of the element
of reduction, which threatens both every imaging and every imagining.

\citet{This universe, characteristically participant in images, shows how the
  image differs from the authentic beings: against the variability of the one
  order, there stands the unchanging quality of the other, self-situate, not
  needing space because having no magnitude, holding an existent intellective
  and self-sufficing. The body-kind seeks its endurance in another kind; the
  Intellectual-Principle, sustaining by its marvelous Being, the things which
  of themselves must fall, does not itself need to look for a staying
  ground.}{Plotinus}{V:9.5} We certainly do not subscribe to the language of
\wo{authentic beings} and faint images among which we must live down here.  Yet,
in some respects, the \co{symbolic} relation of the \co{actual} to the
\co{invisible} can be compared to that of an imperfect image to its original.
The \co{invisibles} are irreducible to their \co{actual signs}, just as the
original is not reducible to any of its images.  But this irreducibility of the
\co{invisibles} does not mean their total inaccessibility. They are accessible
simply by being lived.  \citetib{Thus we may not look for the Intellectual
  objects\noo{[the Ideas]} outside of the Intellectual-Principle, treating them
  as impressions of reality upon it: we cannot strip it of truth and so make its
  objects unknowable and non-existent and in the end annul the
  Intellectual-Principle itself. [...]  Only thus\noo{[by this inherence of the
    Ideas]} is it dispensed from demonstration and from acts of faith in the
  truth of its knowledge: it is its entire self, self-perspicuous: it knows a
  prior by recognising its own source; it knows a sequent to that prior by its
  self-identity; of the reality of this sequent, of the fact that it is present
  and has authentic existence, no outer entity can bring it surer
  conviction.}{Plotinus}{V:5.2} Dispensation from demonstration and from acts of
faith amounts to their inadequacy which is but the other side of the direct
contact with them. \co{Invisibles} are the most directly accessible because they
are the most \co{concrete}, the deepest and most intimately \co{present aspects}
of \co{experience}.  It is the directness of this contact which is \co{actually}
inaccessible, which can not be repeated in the purely \co{actual} categories.
The \co{invisibles} may be apprehended by means of \co{symbols}, but such a
\co{reflective} apprehension is not the same as grasping an \co{object}.
In it, the \co{invisibles} are as if \thi{projected from within} (which
\thi{projection} has nothing to do with the psychoanalytic sense of unconscious
externalisation.)  \co{Symbolic recognition} may happen with full \co{actuality}
of consciousness, with full knowledge about the \co{origin} of the \co{symbols}.
But this is then a \co{recognition} of both the insurmountable \co{distance}
between the symbolised content and its \co{actual} expressions, and of the
impossibility of accessing or \thi{objectifying} it by any other \co{actual}
means than \co{symbolic expressions}.  The \co{symbolism} of \co{self} need not
be an unconscious \thi{projection} -- it may be a conscious \co{expression}. But
that {\em of which} we are thus conscious is primarily the \co{distance}. What
\co{precisely} is thus being \co{expressed} remains \co{vague}; it is
essentially \co{invisible} and the scope of \co{reflective} consciousness can
never embrace the entire \co{concreteness} of this relation.  The relation need
not be, and hardly ever is, consummated under the attentive look of \co{actual}
consciousness.  In fact, the more insistence on bringing it to consciousness,
the stronger indication of some, in the extreme cases pathological, obsession
with the opposition, of the attempts to annul the \co{distance} and make the
\co{invisible visible}.

\noo{But just like a happy man needs not constant \co{reflective} consciousness
  of his happiness, so \co{reflective} consciousness is hardly a necessary
  aspect of worthy and gratifying \co{experience}.   \pa In the sphere of
\co{invisibles} there is no \co{subject} and no \co{object}.  At most, there is
the distinction between \co{self} and \co{my self}, but even this is only
relative to a particular stage of development.  As long as \co{I} retain any
\co{distinctions}, the \co{distinctions} like these will be unavoidable and seem
insurmountable.  Yet, Jung insists that \wo{psyche is {\em real}}, not merely
\thi{subjective}.  \citt{All modern people feel alone in the world of the psyche
  [the qualification \wo{of the psyche} may be safely removed] because they
  assume that there is nothing there that they have not made up.  [\ldots] Then
  one is all alone in one's psyche, exactly like the Creator before the
  creation.  But through a certain training [\ldots] something suddenly happens
  which one has not created, something objective, and then one is no longer
  alone.  [\ldots] This experience of the objective fact is all-important,
  because it denotes the presence of something which is not I, yet is still
  psychical.  Such an experience can reach a climax where it becomes an
  experience of God.}{CC;p.80-81 [Jung, The Vision Seminars]}}

What, if anything, is being {projected} along the \co{symbolic} relation is the
very distinction \co{subjective}-\co{objective}, \thi{psychical}-\thi{physical};
it is projected into the sphere where no such distinctions are possible or
meaningful.  There is a level, albeit hardly a level of \co{experience}, where
\co{my self} is \co{self}, \la{Atman} is \la{Brahman}, and where this {\em really is so}.
Sure, any experience is \co{my} experience, and thus it involves \co{my actual
  subjectivity}.  As long as we are looking for \co{actual experiences} as the
only measure of convincing us about anything, we can at best encounter only
vague analogues, momentaneous \co{experiences} of \thi{oneness}, mystical union,
\la{coincidentia oppositorum}, which can only, and only at best, leave a mark, a
\co{vague trace}, as they disappear from the \hoa. These are only pale, even if
intense, \co{actual} reflections of something which remains {\em essentially} --
and hence forever -- \co{invisible}.  \citeti{It is not possible to draw
  near\noo{[to god]} even with the eyes, or to take hold of [it/him] with our
  hands, which in truth is the best highway of persuasion into the mind of
  man.}{Empedocles}{ 31B133} \noo{ \citaft{FirstPhil}{ 344;p.201}}


\noo{ \say Plotinus: ideas=intellect-divine-mind; are \thi{true}, for if truth
  is conformance to the addressed transcendence, here only \co{one} transcends,
  and \co{one} does not contradict anything. The first distinctions have yet no
  measure to assess their \thi{truth}, they found the possibility of truth
  
  \citet{is that if we are to allow that these objects of Intellection are in
    the strict sense outside the Intellectual-Principle, which, therefore, must
    see them as external, then inevitably it cannot possess the truth of
    them.}{Plotinus}{V:5.1}
  
  \citet{\noo{Thus we may not look for the Intellectual objects [the Ideas]
      outside of the Intellectual-Principle, treating them as impressions of
      reality upon it: we cannot strip it of truth and so make its objects
      unknowable and non-existent and in the end annul the
      Intellectual-Principle itself. USED} We must provide for knowledge and for
    truth; we must secure reality; being must become knowable essentially and
    not merely in that knowledge of quality which could give us a mere image or
    vestige of the reality in lieu of possession, intimate association,
    absorption.  The only way to this is to leave nothing out side of the
    veritable Intellectual-Principle which thus has knowledge in the true
    knowing [that of identification with the object], cannot forget, need not go
    wandering in search. At once truth is there, this is the seat of the
    authentic Existents, it becomes living and intellective: these are the
    essentials of that most lofty Principle; and, failing them, where is its
    worth, its grandeur?  \noo{ Only thus [by this inherence of the Ideas] is it
      dispensed from demonstration and from acts of faith in the truth of its
      knowledge: it is its entire self, self-perspicuous: it knows a prior by
      recognising its own source; it knows a sequent to that prior by its
      self-identity; of the reality of this sequent, of the fact that it is
      present and has authentic existence, no outer entity can bring it surer
      conviction. USED}}{Plotinus}{V:5.2} }

\noo{ \pa The Jungian expression of this contradiction is to say that the
  \thi{objective} contents representing the archetypes are \thi{projections} of
  the psyche.  This, in particular means, that the contents must be unconscious,
  because only unconscious contents can be \thi{projected}, according to the
  psychoanalytical theory.  The question is thus:

  \begin{quote}
\begin{enumerate}
\item The aim of the individuation process is integration of the unconscious, in
  particular, archetypal contents into the wholeness of personality.
\item The essential part of this process is conscious recognition of the
  efficacy of the archetypes in one's life.
\item The conscious experience of the archetypes is possible only through their
  symbolic \thi{projections}.
  \end{enumerate}
  
  Thus, what happens to them, how are they related, if at all, to the
  consciousness after the integration has taken place -- for an archetype, like
  anything else made conscious is no longer liable to being \thi{projected}?
  \end{quote}
  I am not suggesting that this is some unresolved contradiction in the
  excellent Jungian and post-Jungian practice and descriptive elaboration of the
  gathered empirical material. The question concerns only conceptual
  clarification.
  
  Jungian answer would probably be that consciousness ceases to \thi{project}
  archetypes into the \thi{objective} world (an activity typical for the
  primitive, but also more advanced, though still not entirely self-conscious
  mind of more modern mind) but, instead, becomes able to live them concretely.
  The genuine myth Jung proposes for the modern man is precisely to live
  archetypes in a concrete colloquy, a constant dialogue with one's unconscious
  which in its deepest sphere happens to be the same as the world, the past and
  the future, the ancestors and the future generations.
  
  \pa I do not know the primitive mind well enough and I do not think we should
  ascribe all too much unconsciousness to many thinkers of late Hellenism, to
  medieval mystics and others who seem to display signs of quite a close contact
  with the \co{invisible} sphere of Being.  Certainly, some did \thi{project}
  trying to re-establish a disturbed continuity by externalising the internal,
  unconscious processes, contents and assumptions.  \co{Objectivistic illusion},
  too, is an example of such a \thi{projection}, which believes \co{objects}
  inhabit all spheres of Being in the same way they inhabit the \co{actual
    experience}, thus completely forgetting oneself.  But this is seldom an
  unconscious process -- one will easily and quite reasonably argue that the
  world indeed consists of things, that everywhere we go we will find new
  \co{objects}.  I call it a \thi{projection} because I consider it an
  \co{illusion}, but the person may be perfectly clear that it is only a
  plausible assumption -- worth \thi{projecting} -- even though others may not
  share it.
} % end of Jungian projections...

\pa The juxtaposition of Plotinian remarks on the intellect and the Jungian
reflections on the archetypes is, hopefully, self-explaining.\ftnt{A few remarks on
  the similarities can be found in \citeauthor*{JungPlotinus}.}  One should
certainly remember that Jung and Plotinus diverge drastically when it comes to
the description of the contents of the respective spheres.  While the collective
unconscious contains only the archetypes of the most primordial elements of
human experience, Plotinian intellect suffers overpopulation similar to that of
Plato's ideal world. It contains \citet{qualities, accordant with Nature, and
  quantities; number and mass; origins and conditions; all actions and
  experiences not against nature; movement and repose, both the universals and
  the particulars: but There time is replaced by eternity and space by its
  intellectual equivalent, mutual inclusiveness.}{Plotinus}{V:9.10} And this is
only the beginning, because all items listed so far are only forms of sensible
things.

These differences of contents notwithstanding, the similarity of the general
characteristics is hardly disputable,\noo{ The emphasized impossibility of
  detailed and precise description of any particular contents makes us stop
  satisfied with the mere identification of this sphere of Being; we will return
  to its more specific treatment in Book III.}
%
and it is hardly accidental.  Modern sensibility is certainly closer to the
language of Jung than to that of neo-Platonic emanations. But, in general, the
differences concern only the language and concepts. It is easier to recognise
something fundamental in the dimensions of personal \co{existence} than in the
conceptual constructions of \thi{eternal essences}. But what is so recognised
belongs to \thi{collective unconsciousness}, that is, transpersonal
intellect. Every human being lives this eternal sphere anew and individually,
everybody is himself only through \co{participation} in the \co{invisible
  communion} of \co{shared origin}. 
  \citetib{Every soul, authentically a soul, has some form of rightness and moral
    wisdom; in the souls within ourselves there is true knowing: and these
    attributes are no images or copies from the Supreme, as in the sense-world,
    but actually are those very originals in a mode peculiar to this
    sphere.}{Plotinus}{V:9.13}\noo{For those Beings are not set apart in some
    defined place; wherever there is a soul that has risen from body, there too
    these are: the world of sense is one -- where, the Intellectual Kosmos is
    everywhere. Whatever the freed soul attains to here, that it is There.}
  
  The common characteristics yield, as Heidegger would say, \la{existentialia}:
  the \citet{explicata to which the analytic of Dasein gives rise, obtained by
    considering Dasein's existence-structure}{SuZ}{I:1.1, H44}.  Even if
  particular contents seem to vary (though, being \co{invisible}, one can not
  expect them to ever yield to a unique and univocal description), the very
  presence and general character of the sphere of \co{invisibles} provides such
  an explicatum.



\subsubi{Form vs. matter}\label{sub:formMatter}

\pa We seem to encounter the distinction between \thi{form} and \thi{matter},
perhaps, between the universal \thi{essence} or human nature and its particular
instances in given individuals. The \co{invisibles} could remind of the constant
and common characteristics which enter the life only through the \co{actual
  experiences}. To the limited extent the association may be legitimate but we
certainly won't accept it in the traditional form.  On the one hand, we have
dispensed with any \thi{matter} which could provide the individualising material
for ideal \thi{forms}. The main difference concerns then the fact that
\co{invisibles}, unlike the Platonic \thi{forms}, are not any other-worldly
entities, existing independently beyond the world of \co{concrete experience}.
They are fully \co{experienced}, and their \co{transcendence} means only that
they neither are \co{objects} of \co{actual experiences} nor are reducible to
such \co{experiences}.  But neither are \co{invisibles} simply abstracted from
the \co{actual experiences} as their common features, natures or concepts.
\co{Actual} instances are usually too few. One cannot be damned twice in one's
life time, just like one cannot commit suicide twice. There is no such thing as
multiple \thi{instances} of love from which one could abstract any meaningful
\co{concept}. There is not even \co{any experience} {\em of} love (even if
grammar and habit allow us to speak this way) -- there is only \co{experiencing}
love.

\pa Human \co{existence} is indeed a repetition but it is a repetition of the
unrepeatable, it is a repetition of the necessity to live one's life, accepting
this most unique and personal \co{gift}, and to live it from its unique source.
\co{Concrete} life does not amount to \thi{filling in} the abstract \thi{form}
with actual \thi{matter}, not to mention, with actual sensations. If we were to
use these distinctions, we would say that it amounts to actually finding this
very form, to forming it by drawing the borders -- new or old -- completely
anew. This drawing of the borders is not any \thi{matching} of particular
contents to pre-existing \thi{forms}; it affects equally the \co{visible} and
the \co{invisible} sphere, involving the most intimate reciprocity between them.

We can learn (from others, from the books) to understand many things, some
\co{distinctions} and borders between hate, love, friendship,
indifference\ldots, between hospitality, generosity, magnanimity, benevolence,
largesse, lavishness, wastefulness, squander\ldots But to live, it does
not suffice to \co{actually} know, we must also draw these \co{distinctions}
ourselves. To draw such \co{distinctions} amounts to \co{recognition} and
classification of particular \co{actualities} in their terms, as friendly or
unfriendly, as an expression of love or not, as an \co{act} of hostility or a
mere misunderstanding. To live is to \co{recognise actual} situations as
\co{signs}. Was his smile, his remark, an expression of understanding, of
sympathy, of irony, of superiority? Answers to such questions (only seldom
stated explicitly) are not arbitrary because they arise as the
results of \co{recognising} the \co{actual} events as \co{signs} terminating the
respective \co{traces}, which originate in and lead back to the differentiated
but hardly distinguishable sphere of \co{invisibles}.

So far, one might probably still see here only \thi{filling in} the abstract
\thi{forms} with particular contents. However, the interesting part only begins
here because there is no \thi{given} and pre-defined way of connecting the two
spheres. The way of classifying the \co{actualities} affects also the
\co{invisible distinctions} -- not by making them \co{concrete}, because they
always are so, but by drawing them at some \co{actual} limits, that is, by
drawing them at all. \co{Invisibles} live only through their \co{manifestations}
and can be \co{dissociated} from them only by abstracting \co{reflection}. I may
have a \co{vague} understanding of what friendship means and then, confronted
with an act of minor opposition or egoism, I conclude: no, if he could do {\em
  that}, he can not possibly be my friend, he can not possibly be a friend.
There is, fortunately, no recipe-book for drawing such conclusions, and this is
an aspect of repeating the unrepeatable. We do not live among shadowy images but
in the middle of the highest realities. Saying \wo{friendship} everybody will
understand (or misunderstand) something, even if we disagree whether this
particular conclusion, in this particular case was justified. We do not know
where the borders go but we must draw them.  Drawing the borders in \co{actual}
situations we as if define, again and again, what friendship -- as distinct from
all that it is not -- is.

We do not know, that is, we do not know exactly and \co{precisely} what
friendship is. Yet, without knowing it at all, could we have friends?  After some
time, the friend who did {\em that} and whom I declared not-my-friend, turns out
to be the most worthy person whose act followed from the most genuine friendship
or, perhaps, from some particularly restraining circumstances or passing
problems. Even more, I may not only learn about some earlier unknown
circumstances influencing his act but may admit that the act does not actually
contradict friendship after all, that its significance is (and always has
been) much less than what I mistakenly imagined. In either case, he
turns out to be, perhaps even to have been all the time, my true friend, and
friendship acquires a new \thi{essence}, the border separating friendship from
all the rest becomes re-adjusted.

In this tension between the non-arbitrariness of \co{invisibles} and the
constant need to find their \co{actual signs} lies the whole sphere of
\co{concrete} freedom. It is not freedom to decide and declare, but to find and
\co{recognise}; for instance, to \co{recognise} friendship and generosity where one
could see and earlier found only enmity and danger. Such \co{recognitions}
amount to a genuine, if secondary, creation, but this will be considered in Book
III. 


\subsub{Transcendence}
\wtsep{1. essentially}
The \co{transcendence} of \co{invisibles} consists in
their being {\em essentially} \co{non-actual}, {\em essentially} outside the
\hoa\ and, by the same token, \co{above me}.  They are what \co{separates}
\co{me} from the \co{origin}, the sphere through which \co{I} am \co{confronted}
with the ultimate pole of \co{invisibility}, the \co{one}.

\wtsep{2. irreducible}
\pa
\co{Transcendence} does not contradict \co{presence}; essential
\co{non-actuality} does not mean any \co{dissociation}, any insurmountable
foreignness. It only means inexhaustibility by any \co{objective} determinations,
the impossibility of a reduction to \co{visibility}, of being narrowed to the
\hoa\ and embraced fully by \co{actual} consciousness.

\co{Invisibles} are something which I not only cannot grasp but not even imagine
as entities.  Unlike \co{complexes}, they are not constructed from some simpler,
\co{actual} elements by the power of my understanding.  Unlike \co{general
  thoughts} and \co{qualities of life}, which are always relative to one's life,
they are not relative to anybody. Although they are \co{manifested} in one's
\co{experience}, they enter it as essentially \co{transcending} its horizon,
they enter it but do not appear enclosed {\em within} it.  Sure, they can
\co{manifest} to \co{me} something about \co{my life}, but they do it in the
\co{absolute} fashion, as something essential and indisputable, they do it from
\co{above}.

\wtsep{3. empty signs}

Irreducibility means also that \co{invisibles} cannot be
delineated by means of any \co{actual} \co{thoughts} and \co{impressions}. 
\co{General thoughts}, in spite of their \co{vagueness} and empty generality, do have
content which they partially reflect and ascribe to
\co{the world} or one's life.  \co{Invisibles} have no such content
which could be meaningfully, even if only partially, \co{reflected}. 
They are not \thi{properties} of \co{the world}, objective qualities 
encountered \thi{out there} by \co{reflective} thinking. They are 
\co{commands} to the individual being, \co{commands} which either are 
heard -- and that means, heard personally and \co{absolutely} -- or are not
encountered at all. 

Their \co{signs}, seen from the perspective of \co{actuality}, are the most
empty pointers, apparently arbitrary and unrelated to what they point
to.\ftnt{This, by the way, is the character of all kinds of rituals, hymns, Song
  of the Songs, love poetry, and the whole vast mystical literature with its
  invocations, prayers and poems -- they might be understood as inadequate {\em
    expressions} of experiences and attitudes, but not as {\em descriptions} of
  whatever they try to praise.}  They are appearances without objects or, as
Heidegger might have put it, they appear only (as) disappearing, they become
\co{present} without becoming \co{actual} -- \co{distinctions} which immediately
melt into one another and dissolve in the ever \co{present} \co{rest} of
\co{invisibility} surrounding all \co{actuality}.  Their whole and only possible
objective, \co{actual} determination is to~\ldots \co{manifest} -- to {signify}
by pointing towards an inexhaustible source, ever indeterminate and forever
distinct from all \co{distinctions}.

\wtsep{4. non-relative}
\pa 
This complete lack of \co{objective} correlate is an expression of
\co{absoluteness}, of \co{transcending} all regions of Being.\noo{ and, eventually,
all possible \co{distinctions}.} Belonging to the personal center, to the very
\co{self}, \co{invisibles} lie beyond beings but only in the sense of not
pertaining to any particular among them -- they pertain to all of them, to the
whole sphere of \co{actual}, non-\co{actual} and \co{non-actual}
\co{distinctions}.  A person is not holy \thi{over something} or \thi{in
  relation to something} -- he {\em is} holy, nowhere in particular, that is, in
his whole being and beyond it. A person is not damned for a particular \co{act};
a particular \co{act} can only reveal the depth of damnation which has
penetrated the person, that is, the whole world, to the outermost limits. One is
not damned temporally, but forever.  \co{Invisibles} penetrate the whole Being
and lend their force and character to every encounter with beings, to every
\co{distinction}. Thus they \co{transcend} the limitations of any particular
situation, hiding the \co{virtual} potential of ever new \co{manifestations}, 
like the promise of eternal repetition.

\wtsep{5. atemporal - over-temporal}

\pa 
Lying thus \co{above this world}, \co{invisibles} do not fall under its temporal
dimension -- only their \co{manifestations} do.  A bitter, tragic or trivial end
of a love story does not mean the end of love.  It is only the end of this
\co{manifestation} of love, of this \co{experience}.  Psychological difficulties
notwithstanding, one may be equally able to cherish love, to long
for its \co{manifestations}, to \co{recognise} and appreciate it when one meets
it again. Psychological difficulties mean only that one tends to lose this
ability, not that one cannot retain it. 

Independence from time can be seen in all kinds of founding events in which a
single \co{manifestation}, a single \thi{moment of truth}, expressed and
remembered in some \co{symbolic} form, \co{inspires} all future life (of a
group, a community, a person).  From the archaic ways of establishing the center
of the new settlement -- whether the placement of the totem, of the altar, of
the temple tent -- as the \la{axis mundi} along which gods intervene into the
affairs of people; through the legendary foundation events, like that of Rome at
the site where divine help had saved Romulus and Remus; to the laicized custom
of commencing a construction by placing the foundation stone\noo{; and likewise,
  an intense moment of silence and quietude, full not of expectations but of the
  \co{presence} of love which reveals its eternity to the two people, both
  knowing that, although the moment will pass and even dissolve in all kinds of
  disappointments which life can bring, so it has announced something
  unforgettable, something greater than the intensity of feelings and even their
  life} -- \co{symbolic} expressions of the \co{presence} of the higher element
accompany the events of foundation. This \co{symbolic} reference, by
establishing continuity with the origin, anchors the \co{actual}, temporal event
in the eternal element.  For \co{manifestations} reveal truth which is not
affected by the actual course of \co{this world}.  Even if, at some later time,
it loses its \co{actuality} and passes into oblivion, it still has left its mark
which cannot be denied, it revealed something which remains \co{above} time,
even if its \co{manifestations} and \co{actual} consequences may diminish or
disappear.\noo{The re-discovery, by people like Eliade and Jung, of the
  relevance and \thi{intelligibility} for the modern mind of the religious
  symbolism of epochs which are not only past but also seemed for centuries
  dead, is a good example.}

However, although \co{transcending} thus time, the \co{invisibles} are not
timeless in the way of \co{objects}' which appear as if in a \thi{frozen time},
on an abstract scene devoid of change and development.  They are \co{eternal}
and time does not contradict \co{eternity} but only, as Plato said, is its
moving image.\ftnt{\btit{Timaeus}, VII. Strictly speaking, \co{eternity}
  pertains only to the \co{absolute}, the ultimate sphere of \co{invisibility},
  the \co{confrontation} of the \co{nothingness} of the \co{one} and the
  \co{nothingness} of \co{self}; the \co{absolute} contentless {\em fact} of
  \co{confrontation}. But higher \co{invisibles} always reveal the \co{aspect}
  of \co{eternity} which has nothing to do with the \thi{infinite temporal
    duration}, but only with the \co{absolute} validity which \co{transcends}
  time as well as space or any other aspects of \co{this world}.}
\co{Invisibles} -- \co{manifested} through all \co{actuality}, at the horizon
beyond which it dissolves into \co{nothingness} -- penetrate also time. They
unveil in the sphere of \co{visibles} the order which remains \co{above} it, but
which also embraces and enriches everything \co{below}. Every such
\co{manifestation} reveals something \co{absolute}, something which is not
relative to any particular person nor any particular region of Being but which,
flowing from its \co{origin}, penetrates the whole of it. Every
\co{manifestation} of \co{invisibles} reveals their deepest \co{immanence},
their involvement in time, their life.

\wtsep{6. above me - not subject-object}

\pa 
\co{Transcending this world}, the \co{invisibles} \co{transcend} the sphere of
\co{mineness}. They are not only neither \co{subjective} nor \co{objective}, but
also neither \co{mine} nor \co{not-mine}.  In their true \co{manifestations},
not involving any \co{object}, they do not involve any \co{subject} either, or
rather, they erase the \co{subject}, \ger{<<aufheben>>} it.
Of course, they \co{manifest} themselves through \co{me}, through \co{you}. But
\co{you} and \co{I} are not indispensable in such \co{manifestations}, we 
are merely their possible sites. What has been \co{manifested} is not
changed if it happens to be \co{actually} experienced by somebody
else.

One can wish to attain holiness, peace, love and that is about everything one
can do about it.  \co{Invisibles} are not possible goals of any activity, they
are not meaningful intentions of one's \co{will} or \co{acts}. Intending
goodness one turns into a moralist, intending saintliness one turns into a
hypocrite. For intending is relative to \co{myself} and eventually, all
transcendental theories of intentionality notwithstanding, to \co{my will}.
\co{Invisibles} can not be acted but can only act, in the background, as
implicit \co{inspirations}.  \co{Participation} in them requires that \co{I} do
not consider \co{myself} as the \la{axis mundi}, or else, that \co{I} address
\co{myself} only forgetting \co{myself}.

For an experience of beauty or love, it is not essential that \co{I} am their
\co{subject}.  In a sense, it is enough that they are, that they find place.
The \co{subjectivity} of \co{actual} consciousness is merely the one who happens
to realise their \co{manifestation}. Creation of a beautiful work of art is a
very different experience from its appreciation. But the beholder is given the
same \co{gift} of beauty which was given to the artist, the latter was only the
one who \co{actually} happened to bring it to the \co{expression}.
\co{Manifestations} concern everything and everybody, they give joy to anybody
who is able to \co{recognise} them as a generous \co{gift}.  \citeti{If you love
  a thousand marks which are in your rather than someone else's possession, than
  this is not right. [...] If you love your father and mother and yourself more
  than you do someone else, then this too is not right.  And if you prefer
  blessedness in yourself to blessedness in another, that is not right
  either.}{Eckhart, \btit{German Sermons}}{2 Tim. IV:2,5 [\citeauthor*{W}{ 18};
  \citeauthor*{LW}{ 30}; \citeauthor*{EckSelected}{ 4\kilde{p.125}}]} Every
\co{manifestation} of holiness, of love is accessible to everybody -- it does
not have to \co{manifest} itself through \co{me} if \co{I} am to find the deep
peace and satisfaction in it, to experience its quality.  It suffices that they
are -- in fact, even if they do not \co{manifest} themselves.

\wtsep{7. unlimited = shared, not diminished}

\pa\label{pa:shareable}
Either there is \co{an experience}, a \co{manifestation} of the \co{invisible}
which also makes me see \co{my} subordination to what is \co{above me}, or there
is no such experience.  This is the character of \co{absoluteness} which either
reveals itself completely or not at all. When revealed, it knows no limits, in
particular, no limits between persons who can \co{participate} in it, or the
forms of such a \co{participation}.  It is an inexhaustible potential for ever
new \co{manifestations}, a surplus, an untiring force which, fully realised in
one situation, never ceases to look for new forms of \co{actualisation}, which
accessible to one person in one form, does not cease to be accessible to all
others in an unlimited number of other forms.
Love without any \co{manifestation} is hardly love, but
in any \co{manifestation} and, not least, in any failure, love remains the
potential for new \co{manifestations}.  Therefore it never coincides with its
\co{sign} because, fully \co{manifested} as it is, it also immediately overflows
the \co{actuality} of this \co{manifestation} towards the new ones.  In short, an
\co{invisible} is a \co{virtuality}, potentiality of ever new
\co{actualisations}. This marriage of \co{immanence} and \co{transcendence}
underlies the crucial feature of \co{invisibles} -- they can be genuinely
\co{shared}. 

Just like \co{indistinct} remains unaffected by all the \co{distinctions}, just
like multiplicity of \co{selves} is the primordial \co{communion} of the
\co{one}, univocal event of \co{birth}, so \co{invisibles} can be \co{shared}
without diminishing.  \co{Actual} goods, \co{objects}, \co{complexes} cannot be
so \co{shared}, because sharing them requires some kind of division between all
parts which involves diminishing them. (Money is a paradigmatic example but this
applies to the whole sphere of \co{visibility}.)  The fact that more people
\co{participate} in \co{invisible} does not, in any way, diminish its quality,
intensity and truth. Love can be shared without any restrictions, even if its
particular expressions and concrete \co{acts} need to be limited to the
\co{actual} context. But an \co{act} of love, in addition to being directed and
circumscribed within the \hoa, has the \co{rest} which is not addressed to any
particular region of Being. If more people witness to it, it does not lose any
of the love it \co{manifests}; on the contrary, it radiates and allows everybody
to \co{participate} in it.  \citet{All that is begetting in gods, emanates
  according to the infinity of divine power multiplying itself and traversing all
  beings, and its inexhaustibility manifests itself in particular in emanations
  of secondary beings.}{Proclus}{\para 152} An
\co{invisible} seed has no quantity -- like the five loaves and two fish which
are enough to feed five thousand people, so an \co{invisible} grain, of the size
of a mustard seed, is sufficient for any number of people.

\noo{ \co{Invisibles} are \co{true} -- not because object and subject coincide,
  not because they form some coherent system, but because these categories are
  entirely inapplicable, because \co{invisibles} do not ``know'' about \co{me},
  do not care about \co{me}.  \citt{Eckhart: I wouldn't ask God to console
    me...}{???}
  
  \co{Invisible} can be called \wo{\co{truth}} at least in the sense that \co{I}
  find it \co{above me}, as something independent not only from whether it is
  \co{me} or somebody else who experiences it, but also from whether it
  \co{manifests} itself at all, and whether \co{I} accept it or not.  An
  \co{empty symbol} is still a \co{symbol}, \co{I} encounter it with a
  \co{vague} feeling that it does, or did, mean something, something which
  \co{I} perhaps only want to reject, but still something which in its
  \co{vagueness} inspires if not awe then a slight insecurity, in that it
  totally escapes \co{my} understanding.
  
  But it is not true in the sense of a necessary truth which must be accepted,
  which, threatened by the possibility of not being recognised, invents all
  kinds of shrewd or stupid arguments which serve to convince the unbelievers.
  Unlike the \co{actual} truths, scientific or merely reasonable, truths about
  objects and relations, it does not force its \co{actuality} on \co{me}.  It is
  a \co{truth} which allows me to misinterpret it, even to deny it, that is,
  which respects \co{me}, \co{my} freedom.  \co{I} can go on living as if it did
  not exist, burdened alone -- all alone -- with the consequences of my
  rejection.  }

\subsub{Self}\label{sec:Self}

\noo{ \pa Myths are not, in any case not only and not necessarily, mere
  \thi{projections} of the psyche which has not yet reached sufficient level of
  consciousness.  Such a view keeps us still locked in the \thi{subjectivity} of
  the psyche as opposed to the \thi{objectivity} of the outer world.  They are
  genuine \co{expressions} of the unity involved into \co{distinctions} -- of
  \thi{body} and \thi{mind}, \thi{matter} and \thi{spirit}, \thi{good} and
  \thi{evil}, or whatever other distinctions \co{reflecting} life finds useful
  in its dealings with the \co{actual} problems and \co{invisibles}.
  
  Even if, to some extent, light can be taken, as Jung does, as the \co{symbol}
  of the emergence of consciousness, its opposition to the darkness of
  unconsciousness rests on the purely psychological, that is, personally
  accidental, opposition conscious-unconscious -- what is conscious and what
  unconscious is, after all, a pretty private matter of each individual.  What
  is \co{visible} and what \co{invisible}, on the other hand, is a much more
  fundamental matter of Being, and Jung's greatest achievement is the
  identification of the sphere of the \thi{collective} archetypes which quite
  unnecessarily has been coupled with \thi{unconscious}.  There is more to the
  symbolism of light and darkness than mere psychology of a confused mind
  trapped in the ethos of \co{visibility}.  The \co{invisible} which \co{founds}
  the emergence of \co{actual reflection} is darkness only as long as one denies
  its \co{presence}.  \citeti{The light shines in the darkness, but the darkness
    has not understood it.}{John I:5 [\citf{Light is thrown on all the lie of
      your being, We can move only in its shadow.}{Salomon ben Jahuda Gabirol,
      {\em The Royal Crown}:Niflaim \citaft{Heilige}{ V;p.41}}]} In
  fact, it is the \thi{light} of consciousness which may function as darkness or
  the darkening element.  \citeti{The eye is the lamp of the body.  If your eyes
    are good, your whole body will be full of light.  But if your eyes are bad,
    your whole body will be full of darkness.  If then the light within you is
    darkness, how great is that darkness!}{Mat.}{VI:22-23, Lk.  XI:34} To a
  mature and \co{open reflection} the \co{invisible} is the very source of light
  and ultimate \co{clarity}.  \co{Reflection} can live meaningfully only in this
  light, for the mere light of conscious oppositions, the mere strife for
  \co{more} consciousness, can make things \co{visible} only by
  \co{dissociating} them, perhaps, by putting them together again in
  \thi{understanding}, but eventually just making them meaningless.

Just like the important thing is not to realise that one is an immoral
bastard but to cease being one, not to realise that one is mistaken
and confused but to find the right way (for which purposes becoming
conscious may be at best a means), the clue is not to expand
consciousness but to make right use of it.
}

\pa Extensive empirical studies leading to the identification of some
archetypical patterns, did not provide sufficient grounds for Jung to conclude
the presence \citet{in the unconscious [of] an order equivalent to that of the
  ego.  It certainly does not look as if we were likely to discover an
  unconscious ego-personality. [...]  Personality need not imply consciousness.
  It can just as easily be dormant or dreaming.}{JungArche}{VI:503/508} Indeed,
\citetib{consciousness succumbs all too easily to unconscious influences and
  these are often truer and wiser than our conscious
  thinking.}{JungArche}{VI:504} This unconscious center, or rather, since no
center can be discerned there, the hidden source of personality is what Jung
calls \wo{self}.  Its phenomenology in the symbols of \thi{wholeness}, in
particular mandalas, occupy a significant part of Jung's investigations which it
is certainly not the place to review here.  We would denote all this concrete
material as \co{my self}.  As opposed to it, \co{self} is entirely contentless
and, as long as \co{my self manifests} itself, the \co{self} remains hidden
beyond and \co{above} these manifestations.  \co{Self} is the ultimate, {\em
  essentially} \co{invisible} source which, \co{manifesting} itself only as \co{my
  self}, remains always \co{above} it without slightest traces of similarity to
the \thi{ego-personality} -- the archetype of archetypes, the contentless limit
of all relative \co{distinctions} facing only the \co{absolute} \co{indistinct}.

As Ricour says, \co{self} can be apprehended only through \thi{text}. If we
forget the hermeneutic bias towards the \thi{text} and interpret it liberally as
\co{symbolic} expressions, also experienced \co{commands}, then the intuition
seems to be the same: the \co{signs} of \co{self}, by their very nature,
indicate \co{distance} to their \co{origin}, they \co{manifest} without
revealing.\noo{any substantial \thi{essence}.}  \co{Self} \co{manifests} itself
but is itself never reducible to such \co{actual manifestations}.  Its
\co{signs} reveal only, so to speak, its \thi{consequences}, \co{commands},
\co{inspirations} -- not any definable \thi{properties} of the \co{self} itself.
Its \co{manifestation} \citet{is a kind of understanding and perception of our
  self, in which we must be very careful lest, wishing to perceive more, we do
  not stray away from our Self.}{Plotinus}{V:8, 11, 23-24.} Plotinus refers
here to the understanding in the moments of ecstatic union but it applies more
generally.  Of course, the experiences of mystical union are, too, examples of
the \co{signs} of \co{invisible self}. But no matter what form its
\co{manifestations} assume, the attempts to \thi{perceive more}, to
\co{actually} see it, will never yield a satisfying result.


\pa\label{ontSelf} The \co{self}, the \co{trace} of \co{birth}, is the source
from which all \co{my} personal aspects emerge and which \co{founds} the
ontological unity of a person.  \citet{The Self can be defined as an inner
  guiding factor that is different from the conscious personality [\ldots], the
  regulating center that brings about a constant extension and maturing of the
  personality.}{MHSFranz}{p.163} The \thi{inner guiding factor}, the
\thi{regulating center} and the like never obtain any more specific content; any
more \co{precise} description betrays immediately its inadequacy.  Just like the
coherence and relative consistency of my \co{ego} and my \co{acts} are grounded
in the unity of \co{my life}, so the unity of \co{myself}, of \co{my world} and
\co{my life}, is grounded in the \co{invisible} and \co{indistinct origin}, in
the \co{self confronting one}. \co{Self} is the point in infinity, the
\co{nothingness} of a point reflecting the \co{nothingness} of the \co{one}, it
\citeti{has neither a past nor a future, and it is not something to which
  anything can be added, for it cannot become larger or smaller.}  {Eckhart,
  \btit{German Sermons}}{ Mt. V:3 [\citeauthor*{W}{ 87}; \citeauthor*{LW}{
    52}; \citeauthor*{EckSelected}{ 22\kilde{p.204}}]} The \co{self}, the center
of primordial \co{founding} remains \co{above} all specific \co{distinctions}.
\citet{For man does not subsist in these circumstances in which he now appears
  to be, but in so far as he exists he is contained within the hidden causes of
  nature after which he was first created and to which he is destined to
  return.}{Periphyseon}{II:533C/B;532D.\kilde{p.23} The definition: \citefib{Man
    is a certain intellectual concept formed eternally in the Divine
    Mind}{Periphyseon}{IV:768B} might require some interpretation, so let us
  only add \citefib{And I am afraid of those who define him [man] [...]
    according to those things which are seen by the intellect to relate to him,
    saying \thi{man is a rational mortal animal capable of sense and discipline}
    [...] But the concept of man in the Mind of God is none of these; for there
    it is simple, and cannot be called by this or that name, for it stands above
    all definition and all groupings of parts, for it can only be predicated of
    it that it is, not what it is.}{Periphyseon}{IV:768C\kilde{p.65}}}

This \co{invisible} point in infinity, the contentless fact of
\co{confrontation} facing the bare \co{nothingness}, is in fact the origin of
the idea of \thi{substance}. We have opposed all talk about metaphysical
substances with respect to the \co{visible} or material things. But Aristotle
and his followers always included living beings among the primary substances,
and here our characteristics of \co{self} may comply with those of a substance.
\co{Self} is independent in the sense of being completely non-relative; simple
and indivisible in the sense of being \co{above} all \co{visible distinctions};
timeless and unchangeable in the sense of facing only the \co{absolute}.
\citet{The \co{self} is always the same,\lin Already fulfilled,\lin Without flaw
  or choice or striving.\lin Close at hand,\lin But boundless.}{Ash}{XVIII:5}
Such descriptions can also be applied to the \co{one} which would then appear as
the ultimate substance, but we will return to some differences below in
\ref{sub:SelfOne}.


\subsubi{Self vs. My Self}

\pa\label{pa:dissociateUp} We have noticed the difference between \co{self} and
\co{my self}.  \co{Self} is initially experienced as merely \thi{inborn
  possibility}; \co{I} can recognise the sphere of \co{invisibles} centered
around the \co{self} and \co{manifested} through \co{symbols} and \co{commands}
-- addressed to \co{me} but coming from \co{above} any sphere of \co{my}
personality.  At the same time, \co{I} also meet empty \co{symbols}, in texts,
art, other people's relations, which \co{I} recognise as only possible
\co{manifestations} of \co{invisibles}, as ones which do not appeal to \co{me};
\co{symbols} which relate some \co{invisible} story but a story which is not
\co{mine}, which does not exercise the same \co{commanding} power as the
\co{symbols} encountered in \co{my} personal experience.

One might want to extrapolate the obvious difference between such
\co{experiences} to a genuine opposition between their origin(s).  One may, and
naturally one easily does, maintain the distinction \co{mine} vs.
\co{not-mine}, constitutive for the level of \co{mineness}, also with respect to
the \co{invisible} things of the \co{other world}.  Approaching the \co{self},
\co{I} can experience it as \co{mine}, as {\em exclusively} \co{mine}.  But such
form of \co{an experience} results from propagating \thi{upwards} the
\co{reflective dissociations}. It is grounded in the \co{attachment} to the
relativity of \co{actual} consciousness which insists on the categories of
\co{mineness} and \co{myself} \co{dissociated} from \co{not-mine}.  Talking
about \co{my self}, one tends to assume such a distinction at the level of the
\co{invisibles} as if \co{my invisible self} was opposed to a multiplicity of
others.  This, however, is to confuse \co{myself} and \co{my self}.  It is to
apply the categories of \co{actual dissociations} and oppositions pertaining to
\co{this visible world}, to the \co{invisible} world which does not offer
grounds for such distinctions.  \co{Self} is \co{separated} only from the
\co{one} -- this is its true and only counterpart.

\noo{The symbols of protective spirit or one's soul, like \la{Ba-soul} (Egypt),
  \gre{daimon} (Greeks), \thi{genius} (Romans), bush-soul (more primitive),
  Great Man (Naskapi) }

Everything else, every other \co{distinction} is \co{below} it and thus can be,
at least in principle, incorporated into it. The \thi{\co{my}} in \co{my self}
is only \co{my} experience of the \co{self}.  \co{My self} is \co{self}
experienced as \co{mine}, that is, in so far as the \co{commands} and
\co{symbols} are received with all their obliging force by \co{myself}, or else,
in so far as they actually, even if not consciously, exercise their directing
force on \co{my} being, also while this being is involved in the opposition to
\co{not-mine}.

\thesisnonr{\co{My self} is \co{my} experience of the \co{self}.
\label{th:mySelfmyexpirience}}

\co{My self} is \co{self} seen through the veil of \co{mineness}, even of
\co{ego} and pure \co{subjectivity}, in short, it is \co{self} in so far as it
is seen.  The \co{commands} address \co{myself}, and thus they turn \co{self}
into \co{my self}, but they do not originate in \co{myself}, they emerge from
the ultimate \co{origin} and mark the ever \co{present} \co{trace} of \co{birth}
-- \co{self}.  This \co{trace}, \la{haecceitas}, has no principle of
individuation beyond the fact of \co{birth}.  In terms of our figure in
I:\refp{pa:stages}, p.\pageref{fi:stages}, \co{self} is the point of \co{birth},
the pole \co{above} all \co{distinctions} which, as the point in infinity, is
only reflection of the \co{one} -- the spark of the soul, as Eckhart would say
or, perhaps, the \thi{seminal reason}, \la{logos spermatikos} which (borrowed
from the Stoics) is, according to Justin Martyr, \citet{implanted in every race
  of men [as if] part only of the Word [...]\noo{who is in every
    man.}}{Martyr}{8\noo{/10}\kilde{Powrot,p.171}} \co{Self} is \co{present}
beyond and irrespectively of any \co{experiences}, in particular, any
\co{experiences} of \co{my self}, not to mention, of \co{myself}.

%wanted to say: there are no experienceS of Self, only experience of Self -- 
%underneath all experiences

\pa\label{pa:multipleSelf} As the \co{trace} of \co{birth}, the contentless
point in infinity, one \co{self} is indistinguishable from others.  One point in
infinity is essentially the same as another -- they are only numerically
distinct.  This contentless difference of \co{selves} reflects the fundamental
character of \co{birth} as the purely ontological event -- it does not involve,
as yet, any particular \co{distinctions} but only this \co{pure} one.  Different
\co{selves} are only \co{traces} of different \co{births}, of \co{separations}
which established distinct poles of the same \co{confrontation} with the
\co{one}.  Thus one \co{self} is not opposed to other \co{selves}. All are
\co{traces} of -- all \co{share} in -- exactly the same primordial event of
\co{birth} and \co{confrontation} with \co{one}. The numerical distinctions and
multiplicity of \co{selves} is thus a thorough \co{community} of \co{sharing}
the same primordial event.

\co{Self} is the simple fact of \co{existence}, and hence it can not be
dissociated from \co{concrete existence}, not to mention positing it as any
self-subsistent entity. As such, one \co{self} is essentially the same as any
other -- they are only numerically distinct. Yet the difference, if one insists,
between one \co{self} and another is just the difference between one
\co{existence} and another, is the difference between one way (of \co{existing})
and another, which is eventually the difference between one person and another,
between \co{me}, \co{you}, \co{him}. These differences do not in any way
contradict the genuine \co{community}, but this will be discussed more closely
in Book III:\refpf{pa:communion}.

\subsubi{The \thi{sense of self}}

\pa\label{asymSelf} \co{I} am \co{my self}, \co{my self} is \co{self},
\la{atman} is \la{Atman}, but as we said in I:\ref{asymm}, \thi{being} is
asymmetric and none of these can be reversed. \co{Self} is not \co{my self} and
\co{my self} is not \co{myself}. In particular, {\em being} \co{self} is not
dependent on any feeling or \thi{sense of self}.  The \co{experience} of the
\co{self} is not \co{an experience} of any given identity {\em with} the
\co{self}.  \co{Self} is \co{above} \co{me}, \thi{greater than} \co{me}, it is
never \thi{given} in any \co{actual experience} and hence is not reducible to
any personal sense of being \co{oneself}. But at the same time, it is also the
source and the ultimate site of \co{my} \co{unity}.

We may want to ascribe to schizophrenics double personality. And this may be the
case, although it means that we have formed the notion of personality allowing
for such a multiplicity in {\em one} person. It may, however, seem that a schizophrenic
suffers exactly because he retains the \thi{sense of self}, because he notices
terrifying elements invading {\em his} being, because he becomes afraid about
\co{himself} and finds a temporal calm in alluding -- perhaps in an escapist way
-- to \co{his self}.  People suffering from the Korsakoff syndrome seem to have
lost the hold over personal memories and the continuity of their being seems
reduced to only the most immediate, last minute's past. However, they also
preserve some childhood and adolescent memories which indicate that the
reduction is not that total. But even if it were, even if the \thi{sense of
  self} and continuity disappeared completely, we still would be dealing with
{\em the same} person. If such a person is our friend or loved one, we will try
to help {\em him} -- he is no longer \co{himself} as he used to be, perhaps, he
no longer has the \thi{sense of self}, that is, of \co{himself}, but he is still
the same person, the same \co{self}.\ftnt{Certainly, forensic considerations may
  call for a more specific notion of a person, or rather of a legal subject, but
  we are not dealing with such issues.}  And when we find out that nothing can
be done, we grieve over {\em this} person, over our loss of {\em him} and over
{\em his} loss of \co{himself}.

Saying about somebody \wo{He is another person}, we always know that he is the
same; he only behaves, acts, speaks in a way which is not his usual way.
Perhaps, he has even changed completely, he acquired a new personality, due to
some mystical experiences, intense work on himself, some personal tragedy.  But
he {\em is the same} person, even if completely different.  The same applies to
a person who has completely lost his memory, who does not any longer \thi{know
  who he is}, to one with a severe dissociative disorder,\noo{ (like dissociative
amnesia, or identity disorder, called also \wo{multiple personality}),} to an
unconscious person kept alive under a drip.  If this person is my loved one, I
will care and treat him with all consideration and patience which I owe him --
because he {\em is the same person}.  When we respect the decisions written in
the last will of a deceased person, we do it from respect for {\em this} very
person -- this person remains himself, is still identical to himself, even when
dead.  Hmmm\ldots

Thus, not only no two persons are identical, the same person can be vastly
different from oneself.  Just like it is not any \co{externally} perceived,
observable distinguishability which constitutes identity of a person, so it is
not either any \thi{inner sense of self}.  If we feel insecure, we may need some
criteria to {\em convince ourselves} that the person is the one he says he is,
that he is not a spy, that my friend who just went out is the same who is now
coming back, that my wife today is the same person as yesterday, even, that I am
today the same person as I was yesterday.  If we feel
insecure\ldots or, perhaps, if we suffer from the Capgras syndrome.\ftnt{Capgras
  syndrome makes the affected person believe that some 
close friend or relative has been replaced by a deceiver or an \la{alter
  ego}. This conviction persists even though one can still recognise all the
usual signs -- the face, the body, the behavior, etc. -- of the other.} But what we
thus convince ourselves {\em about} is something different than the mere
conformance to any universal criteria.  What is it?  Where does the idea of {\em
  it} come from, if criteria are only to confirm {\em it}?  The \co{unity} of a
person lies beyond any tests and the variations in the possible criteria are as
numerous as the number of different people.

\pa\label{noselfsense} The asymmetry of \thi{being} in general,
 and of being \co{self} in particular, %\refp{asymSelf},
expresses irreducibility of higher levels to the lower ones. But also, the
higher level, the more \co{virtuality}.  \co{Self}, the \co{trace} of \co{birth}, is
the ultimate ontological site.  But as we have described in Book I, it is a
\co{virtual} site which exists only through the levels of \co{my self},
\co{myself}, \co{ego} and \co{acts} of the \co{actual subject}.  That is, there
is no such thing as \co{self} \thi{in itself}, a metaphysical subject without
\co{concrete} realisation in the world; all levels are co-existent and
co-extensional. We are not four separate souls, but one, for it is not \citet{a
  diversity of parts -- if we have to assert that it has parts -- which is
  distinguished in the soul, but a variety of functions and
  movements.}{Periphyseon}{IV:787B\kilde{[p.111]}
  The levels of human being according to Eriugena (or many
  later Scholastics following Aristotle), namely, Intellect, Reason, Interior
  Sense and Body, can be  easily recognised in our levels. \citef{Of course,
    everything which arrives at reason from the secret recesses of 
nature approaches through the action of intellect. Then that same art, as though
in a second descent, goes from reason to memory and gradually reveals itself
more clearly in \la{phantasiae} as though in certain forms. By a third descent
it is next diffused to the corporeal senses where, by sensible symbols, it
divulges its power through genera, species, and all its divisions, subdivisions,
and distributions.}{Periphy}{III;p.172} The divisions of nature from the
title of Eriugena's work, referred typically only to the four rather abstract species
(\citefib{that which creates and is not created, that which is created and also
  creates, that which is created and does not create, and the fourth which
  neither creates nor is created}{Periphyseon}{I:1}), find here most concrete
expression.} 

A loss at a lower level need not mean a similar loss at the higher level. One's
incoherent or inconsistent \co{acts} can witness to some disturbances in one's
\co{ego} but they, unless all too frequent and grave, need not contradict the
latter's integrity. Similarly, suffering an inflation of \co{ego} can as easily
lead to problems in one's social interactions as to a realisation of deeper
aspects of \co{oneself}. One can lose \co{oneself}, which we would typically
equate with the loss of the \thi{sense of \co{oneself}}, an existential crises
or a personality disorder. But this does not mean that one has lost \co{self},
that one ceased to be \co{self}, because this is impossible.

Identification of \co{self} with the \thi{sense of self} is a sad reductionism,
a psychologism of extreme \co{subjectivity}.  The \thi{sense of self} (which, to
avoid all too detailed distinctions and lengthy exposition, we do not
distinguish from the \thi{sense} of continuity or of unity) is not something
which establishes \co{self}.  On the contrary, it is possible to have such a
\thi{sense} only because there is something {\em of which} this is a sense.
Eventually, it is the sense of the \co{absolute} validity and uniqueness of the
fundamental event -- \co{confrontation} with the ultimate \co{transcendence} of
the \co{one} -- which establishes \la{haecceitas} and whose \co{traces found}
all the lower modifications of selfhood like personality, \co{ego},
\co{subjectivity}.

\subsubi{A note on scattered consciousness and self}

\pa Personal \co{unity} is constituted at \co{birth} as the fundamental
ontological fact.  No complete \co{visible} account of it can ever be given.
What we have seen in this Book is the stratification of personal being into
levels which, relatively, may be taken as various levels of personal unity.
Accordingly, \co{I} can experience being more or less {myself}.  Accepting
\co{self} \co{above} any particular \co{experiences} of \co{my self}, {I} reach
the most ultimate and definite \co{unity}.  If {I} stay \co{attached} to
\co{myself}, this \co{invisible unity} slips all the time from \co{my} attempts
to {\em see} it and appears as a merely noumenal identity -- identity
irreducible to and unaccountable for in the \co{visible} terms of \co{mineness}
and of the \thi{sense of self} but which, nevertheless, remains all the time
unquestionable.  Engaged exclusively in \co{my ego}, {I} become a confused
collection of traits, features, functions and inclinations.  And finally, trying
to account for \co{myself} in terms of \co{immediacy}, {I} become a pure
\co{subject}, entirely depersonalised \co{act} of \co{immediate} reactions, as
spontaneous as indifferent because \co{external}.

\pa The more we narrow the temporal scope of attention and the more
\co{objective} we try to be in inquiring into the nature of the \thi{subject},
the less we find of any subject.  Humean series of impressions provide
an obvious example, and so does Locke's person who, eventually, seems to become
merely a \wo{forensic term}.

Yet, \citet{we perceive it so plainly and so certainly, that it neither needs
  nor is capable of any proof.}{LockUnd}{IV:9.3} The awareness of personal unity
makes it hard to accept the attempts to dissolve \co{self} in a flux of
\co{actualities}. \woo{Guess what: everybody was wrong: there never was such a
  thing as the self [...]}{Rorty, ???} Certainly, there never was such a {\em
  thing}.  But one still says, for instance, that \wo{self is the center of
  narrative strategies} or of \woo{narrative gravity.}{Dennett: The Self as}
What center? Why does one find it appropriate to use this inappropriate word?
Or, perhaps, \wo{self is nothing more than a nominal handle we stick on the
  thread of continuity that seems to wind through our lifetime}. No matter how
much one would like to dissolve self in nominalistic arbitrariness of
post-modern or neo-pragmatic cacophony, one keeps trying to justify the use of
the \thi{handles} and to describe on {\em what} they \thi{get stuck} with the
\thi{seeming continuity}. Words do create things because they solidify limits of
\co{distinctions}. But this does not mean that the \co{distinctions} are
arbitrary, merely \thi{subjective} or \thi{unreal}.  Denying the \thi{reality}
of the \co{self} because it evades every determination and every \co{actual}
description, may be (in certain circles) motivated by sociological observations
and preoccupation with cultural abstractions and generalisations, but it carries
always the germ of empirical arguments, not to mention the intellectual
ambiguity of (self-)alienation and (self-)satisfaction.

\pa The \thi{sense of continuity}, not to mention any real continuity itself, is
as perplexing for today's empiricists as it was for Hume. His famous argument
shows, indeed, that \co{self} can not be accounted for in terms of scattered
\co{actual} \thi{perceptions and ideas}; no such events reveal \co{self}, even
if some might \co{manifest} its \co{presence} -- the \co{presence}, however,
which for ever \co{transcends} the \hoa.  As \wo{there is no impression constant
  and invariable}, and so none which could give rise to the idea of \co{self},
there are two possibilities: either stick to the method which tells us that only
impressions and perceptions matter, or look for \co{self} somewhere else.
According to the former, people \citet{are nothing but a bundle or collection of
  different perceptions}{Hume}{I:4.6} -- OK, but why do one need to use the
words like \wo{bundle} or \wo{collection}, when no unity is there? What or where
is the boundary separating one such collection from another? For something makes
it a \thi{collection}, this \thi{collection} as opposed to another one. We
prefer to see in the inability to {\em see} any \co{self} not any proof of its
non-existence but simply the limitation of the \co{actual} ability to see -- it
does not imply that \co{self} does not exist but only that if it does then it is
\co{invisible}.

The attempts to bring \co{sign} and signified into an ultimate identity, to
conflate action with reaction, lead nowadays naturally to the \co{complex} of
the brain composed of independent though intricately networked neurons and to
the binary minuteness of their firings.  As the theory of consciousness,
this amounts to postulating a \thi{multiplicity of Is}, one \thi{I} responsible
for every minute reaction and bunch of such \thi{Is}, each working in its own
direction, \thi{competing} with each other for creating an overall, unified,
conscious experience and, in the confused cases of self-deception, of qualia.
Consciousness thus explained seems to say \wo{We react, therefore we are}.  This
account of Dennett's goes back to James' theory according to which consciousness
is not any entity but a function.  It has never been entirely clear {\em of
  what} this is a function, but since consciousness is not any entity in our
account either, we should perhaps grant it some attention.

\pa Such a view certainly squares well with the cases of dissociative personality
disorders, some forms of schizophrenia, and the like.  Indeed, mental sickness
can manifest itself as a dissolution of the \thi{sense of personal unity} and a
fall to the level of \co{dissociated} impressions and sensations.  But does the
fact that self can be dissolved mean that it does not (or did not) exist?  It is
like saying that, since this building {\em could} be destroyed, it is not real. But
\co{self} is beyond and \co{above} any \thi{sense of self} -- \co{I}
may lose \co{myself}, \co{my} \thi{sense of self}, but this does not dissolve
\co{self}. For \co{self} is not a thing which can be lost, not a place which can
be left -- it is the very fact of \co{my} existence.

Consciousness is a pure \co{actuality} which culminates in \co{reflective}
splitting of all contents into \co{dissociated} entities.  Sufficient
\co{dissociation} leads almost inevitably to \co{positing} a \co{totality} of
all the bits and pieces which became completely unrelated.  Also \co{subject} (or
\co{object}) is not any metaphysical substance of its own -- it is just an
\co{aspect} (function?) of \co{actual experience}.  But, and this is the crucial
\thi{but}, even for a declared empiricist this can be only the beginning of the
story -- a deep sense of unity, at least in most normal and healthy persons, is
so obvious that one need to try to account for this, too.

\pa Simplifying only slightly, it should not make much difference which account
of this kind we consider.\noo{Since Hume's, or Dennett's, hobbies with labeling
  things \wo{real} and \wo{unreal} do not seem to deserve much attention, I will
  take a more serious person.} James says: \citet{Experience, I believe, has no
  such inner duplicity [subject-object]; and the separation of it into
  consciousness and content comes, not by way of subtraction, but by way of
  addition -- the addition, to a concrete piece of it, other sets of experiences
  [\ldots].}{Radical}{I:2 (The \wo{subtraction} refers to the conviction that
  \thi{pure experience} happens to a dual element of consciousness, so that one
  can be obtained only by \thi{subtracting} the other.)} So far, so good; our
\co{experience} does not happen within a sharp duality of
\co{subject}-\co{object} either.  But then, consciousness is merely one
experience looking at another, it is a relation between separate experiences:
\citetib{a given undivided portion of experience, taken in one context of
  associates, play[s] the part of a knower, of a state of mind, of
  \thi{consciousness}; while in a different context the same undivided bit of
  experience plays the part of a thing known, of an objective
  \thi{content}.}{Radical}{I:2} The crucial phrase here is \wo{the undivided
  portion of experience}, the beloved \thi{given}, \thi{atom}.  What constitutes
a part of experience as such an \thi{undivided portion} is not an appropriate
question -- this is just the way things are, this is how we experience: one
\thi{undivided portion} after another.  \citetib{The instant field of the
  present is at all times what I call the \thi{pure} experience.  It is only
  virtually or potentially either object or subject as yet.  For the time being,
  it is plain, unqualified actuality, or existence, a simple {\em
    that}.}{Radical}{I:2} The \co{actuality}, now explicit as \thi{the instant
  field of the present}, creeps irresistibly into the language, whether of an
empiricist or phenomenologist.  It is ignored, however, because it functions all
the time as the silent assumption.  Consciousness, or what seems to be the same
here, subject, emerges as a relation between such \thi{presents}.

But \co{actuality}, and hence all \co{actual experiences}, are something which
happen at the end, not at the beginning. \co{Actuality} emerges from a series of
hypostases as a final result. This final result is the beginning only for all
\co{reflection} and \co{actual} considerations.  Consciousness, or its germ
\co{awareness}, is an \co{aspect} of the \nexus\ of \co{actual} \thi{undivided
  bit of experience}, not something which is built on top of such
\co{actualities}.  There is no \co{actuality}, no \thi{undivided portion of
  experience} without being in which, by which, or for which it has been
distinguished as such a portion. This distinguishing, this split between the
\co{actual} and non-\co{actual}, is this being's consciousness,
I:\refp{pa:objects}.  \noo{ The definite \co{dissociation} of \co{subject} and
  \co{object} does not happen at the level of \thi{pure experience}. But this
  phrase with us does not refer to \co{an actual experience}, which being
  \co{cut} out of the horizon has already introduced \co{actual subject}, but
  rather to the field of \co{experience}, differentiated and \co{recognised} but
  as yet not \co{reflectively dissociated}.  }

\pa A significant consequence of the just quoted view is that consciousness
becomes necessarily and only a retrospective act directed towards an earlier
\thi{pure experience} \citetib{and the doubling of it in retrospection into a
  state of mind and a reality intended thereby, is just one of the
  acts.}{Radical}{I:2} But now we are obviously speaking about our
\co{reflection} which, indeed, always comes \co{after}.

The usual confusion of \co{awareness} and \co{reflection}, under the common
heading of consciousness, makes its work here forcing consciousness to be always
only consciousness of some past.  This, as we saw, is characteristic for
\co{reflection} -- the \co{after} is but the form of the distance separating
\co{subject} from \co{object}.  But we would not, for this reason, call
\co{experience} \wo{unconscious}.  Even if it does not involve \co{reflective
  dissociation} of \co{subject} and \co{object}, it does involve \co{awareness}
which is its very life; it is only because I am constantly \co{aware} of what is
going on that, at some moment, I can stop and \co{reflect} over it.
\co{Awareness}, however, is for the most a subconscious and non-intentional
process, which for James might easily mean: unexperienced.\ftnt{This, of course,
  is a simplification since James was well aware of the importance of
  subconsciousness. But it is not clear how this affects his notion of
  experience and, in particular, of the pure present.}

\noo{quote Fichte\ldots: You are pricking ME!}

Furthermore, we can ask what makes me, i.e., {\em this} \co{actual} experience,
look at an experience from yesterday as equally \co{my} experience?  Sure, there
is some sense of continuity which, too, is an aspect of {\em this} experience.
I, {\em this} experience, seems thus to posit an \thi{I} as a subject also of
this past experience.  Projection, I guess.  By analogy?  But I was not there
then, {\em that} experience had no subject, which is only projected back from
the \co{actuality} of {\em this} experience.

\pa One can certainly try to connect further and further, but we won't.  What we
possibly might get in this way is an \co{ego}, a collection of \co{actualities},
mostly conscious ones, which may be considered as \co{mine} but are never
\co{myself}.  The \co{unity} escapes, as always, the empirical
\co{dissociations} and can be only attempted reconstructed as a mere
\co{totality}.  The intimate relations between the pieces (aptly observed by
James as conjunctive relations) are still only relations between separate
pieces.  The very \thi{feeling, or experience of continuity} remains a fact but
a fact without any reason or, to put it as a pragmaticist would like, without
any goal or function.  What is a possible pragmatic goal of projecting \thi{I}
on past experiences?  Forensic aspects hardly get involved at this level of
considerations.  Consciousness is thus a function (of what?)  which itself has
no purposeful function, which merely adds to the \co{actual} experiences, and
always only past ones, a redundant feeling of having a subject which, by the
way, comes always only \la{post factum} to have a retrospective look.

\noo{Sure, one may imagine countless ways of explaining that, after all, it does
  have some function. Such explanations will be typically based again on the
  confusion of \co{awareness} with \co{reflection}. Let's only remark that the
  most prominent among such functions would probably be to establish (or account
  for?)  the very \co{totality}, the sense of all experiences being connected
  and mutually related.  I still do not know what the possible purpose of such a
  function would be, but let it go.  This relatedness, perhaps unity, however,
  was given already in the \thi{sense of continuity}.  So, perhaps,
  consciousness is nothing but this very \thi{sense}?  Perhaps.  The question
  then would be whether such a conscious experience reveals something beyond its
  \co{actual} content or not, whether this unity is only an aspect of this and
  similar experiences or else whether it obtains also when consciousness does
  not accompany the experience. You know my answer, so we may stop here.  }

The empirical bias, from Democritus to Hume, James and later, can hardly admit
any unity beyond \co{actual experiences}.  From the \co{objectivistic}
perspective -- at the level of \co{actuality} -- there is no such thing as the 
unity of a person; there is at most the unity of an \co{act}.  The empiricist's
creed -- whatever can be \co{distinguished} must be \co{dissociated}, because it
is independent -- with the accompanying {ontology} of exclusive \thi{reality} of
indivisible \thi{atoms} can not, if carried consequently, accept almost anything
as \thi{real}.  The multiplicity of Is, minute, preferably determinable
entities, is a natural, indeed, the only possible, counterpart of this
\thi{atomicity}.  It seems to enter the stage for no apparent reason, for if
everything is obtained from dissociated experiences, what is the need for any
\thi{I}?  Well, it does not enter the stage -- it sneaks in.  Dennett, to take
another example, seems to distribute the labels \wo{real} and \wo{unreal}
according to the following principle: to be \thi{real}, an $x$ must be
determinate, and determinate means that it is decidable whether something is $x$
or not.\ftnt{The precise bivalence of the non-contradiction principle (often
  quoted where excluded middle does the real job) as the ultimate criterion of
  \thi{reality} is but another side of this view. Since self is underdetermined
  (or even may seem to possess contradictory characteristics), just like are the
  characters fabricated in the novels, both are equally fictional -- at least,
  according to \citeauthor*{DennSelf,DennConsc}, and other writings of this
  author.} \wo{It is determinate, decidable} is a misleading depersonalisation
of the event -- {\em we and only we}, you and I, are the ones who must be able
to determine and decide.\noo{\co{Actuality} is the horizon of \co{reflection},
  which means, \co{objectivity} is inseparable from \co{subjectivity}.}
Attempts to reduce everything to the \co{objective actuality} of the givens, end
up in most intimate, if only confused, associations with \co{subject},
consciousness, \thi{mind}, or whatever is the current sign of the sphere
\co{transcending} the obviousness of the \co{immediate}.  The empiricist is
there all the time, experiencing, determining, deciding, and nothing helps
getting rid of {\em himself}.  Usually, he reaches his limit when the
\thi{atoms} begin to slip out of his view, when the well, from the bottom of
which he hopes to dig out the \thi{atoms}, begins to seem \ldots bottomless.
But this is only another side of the fact that his \thi{atoms}, his
\thi{reality} are but a function of this view; not necessarily of any voluntary
decisions but of his sensuous, perceptive, reflective, differentiating
mechanisms which furnish the \co{distinctions} necessary for arriving at any
\thi{atoms} in the first place.  Here, past the bottom of the well, in a
round-about way and with all possible reservations, we can join him.

\pa The intention was not to dismiss the attempts to establish consciousness as
a conspiracy of cells on the empirical basis.  Such attempts, as any others of
the sort, may be worth pursuing. On the one hand, such a reductionism is the
form of all science\noo{(in the strict sense)} and, on the other hand, they may
suggest, though never demonstrate, which aspects are not amenable to
\co{objectivistic} reduction. Even if they ever reached the goal, it would be of
a merely scientific relevance and would hardly affect the way in which one
comports oneself, consciously or not, to one's experience.  Even if they managed
to construct an imitation of \co{ego} from bits and pieces, this would hardly be
even the first step in giving an account of the \co{I} and its relation to the
world.  For consciousness (whether \co{awareness} or \co{reflection}) is only an
accident of \co{myself}, is only \co{actuality} of \co{myself}, always
interwoven into the texture of \co{my} whole being and stretching all the way to
the ultimacy of the \co{self}, the \co{invisibility} of the
\co{origin}.\ftnt{Attempts to (re)construct consciousness provide an example of
  the ever present project of \gre{mimesis}, the driving force of
  \co{reflection} (whether in art, science or technology) trying to regain the
  object it has \co{dissociated} from others and from itself. In early days of
  AI it has reached the ambitions of recreating human intelligence 
  by means of formal manipulation of symbolic representations.
  The inadequacy of the project became quickly apparent as information required
  for controlling even simplest tasks, like walking the stairs or avoiding
  obstacles, proved impossibly complex. The later projects of the nature-based
  AI resolve (some of) such difficulties by renouncing explicit models and
  designing systems of purely reactive components, where the sheer organisation
  -- without any explicit modeling -- allows robots move in real environment
  successfully avoiding obstacles (e.g., neural networks, subsumption
  architectures in robotics, \citeauthor*{SubsumeA,SubsumeB,SubsumeC}.) Of
  course, renouncing representation one loses the possibility of treating the
  factors which are not immediately given.
% except those which are hard-wired in the structure.
  But one might see in this development -- admitting that purposeful behaviour
  does not necessarily require explicit model and representation -- an
  expression of the fact (if not the insight) that there is no such thing as
  \thi{pure intelligence}, \thi{consciousness in-itself}; that whatever thinking
  is, is inseparable from the being (nature) which thinks; even though one has
  not yet said that to create human intelligence, consciousness, etc.  one has
  to create a human being. Leaving such projects to doctors Frankenstein, we
  could perhaps observe that there is a very old, efficient and even pleasant
  (though hardly \co{reflective}) way of \thi{creating copies} of human beings.}

\subsubi{Source of unity}
%
\pa The question how the unity of \co{myself} arises from the multiplicity of
\thi{perceptions}, \thi{impressions}, or \thi{ideas} is as irresolvable on the
basis of \co{reflection}, as is the question about this unity arising from an
interplay of various areas of brain, from a conspiracy of cells, patterns of
firing neurons, etc. It is irresolvable because it arises from the antinomic
attempt to reduce the experienced -- and yes!  known -- \co{invisible unity} to
the \co{objective} categories. The best it can do is to search for some
principle of consistency, reason for cooperation, \co{visible} explanation
of\ldots the \thi{sense of self}. This sense, like any other sense, may have
various explanations which never touch the issue.  For personal \co{unity}
precedes \co{reflection} and \co{founds} the latter's unity: {I} am the same
being not only in any particular moment of \co{self-reflective attention}, but
also through all the day, whole two weeks in Prague, all my life.  This
\co{unity} is not an identity constructed from smaller, more manageable pieces.
It is ontologically \co{founded} by \co{birth} -- the primordial event of
\co{separation} -- which precedes any other \co{distinctions}.  And having
happened \thi{up there}, \la{in illo tempore}, against the background of
\co{nothingness}, it remains ingraspable and inexplicable by means of later
\co{distinctions}.  It remains \co{above me}, which in particular means,
\co{above} all \co{my} feelings, perceptions, concepts, senses.\ftnt{One tends
  to interpret the (even less than one hour old) neonates' responses of
  imitation of facial gestures as expressions of a genuine presence of a
  separate -- even if only proprioceptive -- self, lived as distinct from the
  environment where the gestures to be imitated occur (e.g.,
  \citeauthor*{Meltz77,Meltz83,Meltz89}, \citeauthor*{Gall96}). We are not
  interested in precise timing of events; \co{birth}, which is the constitution
  of \co{self}, may be the moment of conception. But the fact that even
  psychological research draws the empirically discernible border of the
  emergence of (a residual \nexus\ of) self to the few minutes after birth,
  accords with our view much better than the earlier estimates (e.g., of
  \citeauthor*{Piaget62}, followed by \citeauthor*{Ponty64}) which denied the
  possibility of such an imitation before some late months of the first year.}

\pa\label{pa:usedBuber} \co{Reflection} is never identical to its \co{object}.
It is identical to, or better, it {\em is} the \co{act} of \co{reflecting}, and
in the immediate \co{self-awareness} \co{I} \co{recognise} this identity.  In a
more concrete way, \co{I} can direct my \co{reflection} onto \co{myself} -- then
\co{I} become immediately aware that what \co{I} am reflecting over is \co{me},
the one who is reflecting.  However, this identity remains for
\co{self-reflection} noumenal because its true site lies beyond it: it stretches
across the \co{actualities} which \co{reflection dissociates} in the very moment
it is asking about their unity.  The aspects of \co{myself} I can bring forth in
\co{self-reflection}, \co{general thoughts} and \co{qualities of my life},
appear as accidents of some ultimate \co{self}.  Whether the \co{self} remains
only an ideal, unexperienced and, as many an empiricist would say, \thi{unreal}
construction or, on the contrary, a \co{concrete} and most intimate
\co{presence}, depends on \co{myself}, on the way \co{I} decide to guide \co{my
  reflection} and which \co{signs} \co{I} decide to accept. The two possible
responses to the \co{experienced} irreducibility of \co{self}, and even of
\co{myself}, to the \co{objective} determinations are: impossible to
re-construct hence unreal; or irreducible because more genuine.

%\subpa\label{pa:usedBuber}
When I \co{reflect} over the \co{self} \co{re-cognising} it as the
\co{foundation} of \co{my} \co{unity}, this \co{unity} is no longer relative to
my \co{subjectivity}, it is no longer an \co{objective} identity consummated
within the \hoa.  Cartesian \la{Cogito}, just like Kantian unity of apperception
and all the similar attempts to obtain personal identity from the mere \co{acts}
or the structure of \co{actual} consciousness, never managed to arrive at
anything except that from which they started: the \co{immediate} self-identity
of the \co{self-reflective act}.  In such an \co{act}, \co{I} \co{re-cognise}
that it is {\em my} \co{self}, but \co{I} also know that it is more than
\co{myself}.  And \co{I} definitely know that it is more than what is
\co{actualised} in this mere \co{act} of \co{self-reflection}.  I am the same
and different from myself.  The \co{unity}, not the identity, becomes
\co{manifest} as the \co{origin} of \co{myself}, as the deepest source of my
being -- \co{I} am only its \co{visible} effect and the unity of my \co{act} is
only the final hypostasis of the \co{founding} process originating from the
unitary and unique \co{self}.

The \co{signs} of personal \co{unity} may be hardly \co{visible}, its
\co{experience} may have hardly any \co{actual} reasons, in fact, one may be
completely unable to re-construct this \co{unity} from the bits and pieces of
\co{actual} observations. But this need not mean that there is no such thing,
only that, if there is, one should not look for it among the \co{objects} of
\co{experiences}, let alone at the level of \co{immediate} and decidably
determinate, that is ideal and idealized, \co{objective precision}.

\noo{Hopefully we have at least made plausible the \co{original unity} of our
  being which precedes, rather than is constituted by, consciousness.  We are
  not doing any science here and it is only \co{existence} and, eventually, its
  spiritual dimension, which concerns us. From this perspective it may be better
  to keep multiple Is under control because, as a matter of fact, they can get
  loose like a \wo{herd of pigs}.  \noo{We are not objecting to any of the
    empirical theories of brain, etc. We are only objecting to the ever failed,
    empirical attempts to reduce I and, eventually, the unity of a person, to
    any observable trivialities and palpable vulgarities. As long as they are
    speaking about neurons etc., they are OK...}  }

\subsubi{Descriptive vs. normative self}

\pa There is a long spiritual tradition according to which all who merely live
their lives without any spiritual concentration and effort do not attain any
genuine unity but are only collections of separate and independent \thi{Is},
drives and desires, bits of consciousness.

Visions (true, of God, but in His terrifying rather than benevolent aspect)
often involve almost demonic manifold of strange, incoherent creatures which
appear and act in a dreadful autonomy.  \citeti{And every one had four faces,
  and every one had four wings.  [\ldots] As for their rings, they were so high
  that they were dreadful; and their rings were full of eyes round about them
  four.}{Ezek.}{I:6,18} The eye, Jung observes, \citet{is a symbol as well as an
  allegory of consciousness.}{JungMC}{II:47} Quite common motif is not only that
of the Eye of God, but also of the multiplicity of eyes, for instance, as
fishes' eyes which \citetib{are tiny soul-sparks from which the shining figure
  of the filius [divine child] is put together.}{JungMC}{II:46} Also,
\citeti{The eyes of the Lord are in every place, beholding the evil and the
  good.}{Prov.}{XV:3}

It would be too optimistic to draw any definitive conclusions from such
examples, but they may be viewed as witnesses to the old acquaintance with the
possibility, and even actuality, of single aspects of the whole possessing, or
acquiring, autonomy in their functioning.  It may be easily recognised in daily
experience.  At one moment, I think or feel this, at another that, and for the
most I do not even remember what I did a moment ago.  I make a promise today,
and in two weeks I forget what I promised, the situation has changed so that I
act as if I never promised anything.  I go around {\em thinking} that I am one
and the same person but, on a closer \co{reflection}, there is nothing
\co{actual} which could account for, let alone justify, this unity. It appears
dissolved in the multiplicity of transient moods, feelings, thoughts, reactions.

\noo{This may go back to sufism or buddhism, but we can also mention a suspect
  person of Gierdayev, whose teachings have been popularised by P.~D.~Ouspensky,
  e.g., {\em In Search of the Miraculous}, {\em The Psychology of Man's Possible
    Evolution}, as one among many emphasizing the actual multiplicity of
  \thi{Is} and the need for active effort in order to achieve \co{concrete
    unity} of being.}

What matters is not that such descriptions can be refined to a theory which
could even appear plausible. What matters is if it is a complete and satisfying
description.  There is no question about sick, or merely disastrous phenomena.
As a bushman can \thi{lose his soul}, so \co{I} can lose contact with \co{my
  self}, \co{I} can lose \co{myself}, dissolve into a multiplicity of \co{egos},
and suffer the associated personal and social problems. (Or, perhaps, it would
only be suffering of some of these \co{egos}?)  The multiple \co{egos} can
retain the character of selves, and even degenerate into autonomy of various
lower functions.  But even if \co{objectively} the relative autonomy of various
lower functions is a fact, it would not help me (that is, them) the least even
to lit a cigarette when I am (they are) feeling like having one (which one? what
one?).  The actual multiplicity of \co{egos} and their relative autonomy is a
fact which only calls for the stronger active effort in order to maintain
\co{concrete unity} of being.  It would be hard to deny relative autonomy of
various lower functions, not to mention cellular or even molecular processes in
the body.  But if we forget the little word \wo{relative}, if such an autonomy
is a \co{dissociation} of elements unrelated into any unifying whole, it becomes
like~\ldots demons which threaten, as they always did, with the abruptness of
their uncoordinated movements, as when they \citeti{went into the pigs, and the
  whole herd rushed down the steep bank into the lake and died in the
  water.}{Mat.}{VIII:32. It is tempting to quote several fragments of
  Empedocles: \citefi{Limbs wandered alone./Creatures with rolling gait and
    innumerable hands./Many creatures were created with a face and breast on
    both sides; offspring of cattle with fronts of men, and again there arose
    offspring of men with heads of cattle; and [creatures made of elements]
    mixed in part from men, in part from female sex, furnished with hairy
    limbs.}{DK 31B58/60/61}{} These might, perhaps, be only images of creatures
  arising from the mixture of elements. \wo{But in Wrath they are all different
    in form and separate, while in Love they come together and long for one
    another.}  [DK 31B21]}

\newp

\pa\label{pa:strengt} We said that \co{self} is the ontological \co{foundation}
of the \co{unity} of the person, and then
%in \refp{noselfsense},
that it is independent from any sense \co{I myself} might have of it.  Yet, a
strong personality will have a strong \thi{sense of self} and it is certainly
desirable to have such a sense. This sense, the \thi{inborn possibility} of the
\co{experience} of such a \co{unity}, of continuous and lasting \co{presence},
is something we can recognise as the normative aspect of the \co{self}. This
normative aspect is but a reflection of the ontological \co{founding}, it is the 
call \wo{Become yourself!}, the call to \co{concretely} realise the ontological
anchoring of \co{myself} in \co{self}, to live \co{concretely}, also in
\co{visible} terms, the \co{unity} \co{founded} in the \co{self}.  For it is from
\co{self}, from \citet{this central nucleus (as far as we know today), [that]
  the whole building up of ego consciousness is directed, the ego apparently
  being a duplicate or structural counterpart of the original
  center.}{MHSFranz}{p.169} \citetib{But this larger, more total aspect of the
  psyche appears first as merely an inborn possibility.  [\ldots] How far it
  develops depends on whether or not the ego is willing to listen to the
  messages of the Self.}{MHSFranz}{p.163 [\thi{Ego} in Jungian psychoanalysis is
  a mixture of what we call \wo{\co{ego}} and \wo{\co{I}} or \wo{\co{myself}}.]}
The \thi{inborn possibility}, the \co{self}, is felt and experienced only as
\co{my self} and that only to the extent that \co{inspirations} and
\co{symbols}, which communicate something to \co{me}, are received as
\co{manifestations} of something which is in constant need of
\co{actualisation}.  There is a long \co{distance} separating the \thi{inborn
  possibility} from the \co{actual} challenge, and the \co{actual} challenge
from truly and \co{concretely} becoming \co{one self}.

As far as the experience and \co{reflection} focused exclusively on the
\co{actual} contents are concerned, the \co{self}, \co{my unity} remains merely
an ideal noumen.  \citetib{The actual process of individuation -- the conscious
  coming-to-terms with one's own inner center (psychic nucleus) or Self --
  generally begins with a wounding of the personality and the suffering that
  accompanies it.  This initial shock amounts to a `call', although it is not
  often recognised as such.}{MHSFranz}{p.169} Recognising it as a call means to
recognise a \co{symbol} as a \co{command}, a \co{command} to
\co{myself}.\noo{Jungian individuation, the open relation to \co{one self}, may
  be phrased in Buber's words: \citet{Between you and it there is mutual giving:
    you say Thou to it and give yourself to it, it says Thou to you and gives
    itself to you.  You cannot make yourself understood with others concerning
    it, you are alone with it.}{IchDu}{???}  }
%
The challenge may seem abstract in its \co{vagueness}, but it is the most
\co{concrete command} of becoming \co{one self}, of seeking a \co{concrete
  foundation} of the \co{unity} of a person, as distinguished from the merely
ontological \co{foundation}.  Unlike the latter, the former is not something
simply given by nature -- rather, it is a possibility which nature only opens
before man. Strength is the ability to live the tension between the
\co{non-actual} and \co{actual} not as a conflict but as a \co{foundation}, and
this amounts to \co{recognition} of the direction involved in the higher
\co{command}, to \co{actualisation} of its \co{vague} imperative as a
\co{concrete} value. The \thi{sense of strength} is only a reflection of the
success in such a following of the \co{command}, its final \co{visible}
consequence. It has a moral, rather than ontological or epistemological flavour,
and is a goal rather than a fact. Yet, all \co{objective}, reductionistic
attempts are still only {\em trying to account} for the unity.\noo{ (which would
hardly be called an \wo{illusion} if it were not \co{experienced}).} Certainly,
\co{I} am often inconsistent, \co{I} have opposite tendencies and drives, \co{I}
often act incoherently, etc. But it is all the time \co{I} who act, \co{I} who
have these tendencies, \co{I} who realise their conflicting force. \co{I} am the
one who is called, not {\em a} brain. All \co{my}
spiritual deficiencies notwithstanding, \co{I} am \co{myself}, \co{I} do have
\co{experience} of \co{myself} and this is not merely a pure \co{actual
  experience}, but \co{experience} which \co{I} carry with me through \ldots the
whole life.  It is much more \thi{real} than the firing neurons and, from the
point of view of \co{experience}, much more important, even if not as curious as
brain's possible workings.  The principle of uncertainty does not at all affect
my \co{experience} of {\em this} being \co{precisely} at this point at
\co{precisely} this time, and other laws of quantum physics, true and relevant
as they may be, are equally indifferent for my way of handling physical objects.
\citet{We feel that even when all possible scientific questions have been
  answered, the problems of life remain completely untouched.}{Tract}{6.52} Any
reductionistic account of person, if it were ever obtained, would have no
influence on our \co{experience} of personal being, moral responsibility, etc.,
except, possibly, of inflicting a temporary confusion in some intellectual
circles. We leave this normative aspect of \co{self} for the time being; it will
return in Book III. 

\noo{Symbols of Self: \la{Ba-soul} (Egypt), my:\gre{daimon} (Greeks),
  my:\thi{genius} (Romans), my:protective spirit/angel, bush-soul (more
  primitive), Self:Great Man (Naskapi) [MHS, p.162-3] }

\noo{ {\small{ \pa \co{Reflection} contemplating the experience of \co{my self},
      or perhaps even trying to contemplate \co{self}, can hardly avoid
      \co{positing} the distinction, which is only a \thi{projection} of the
      distinction between the \co{actual subject} and the world.  But when the
      relation is lived \co{concretely} the distinction \co{my self}-\co{self}
      does not obtain, even if \co{reflection} remains an \co{actual subject} of
      such \co{an experience} which does not coincide with anything
      \co{external}.  This \co{concreteness}, and it is \co{concreteness} of
      being \co{self}, is however unattainable as long as \co{I} insist on being
      \co{myself}.  Being selfish has nothing to do with being \co{self}, not
      even with being \co{my self}.
      
      \pa \co{Self-reflection} can bring \co{me} closer to the \co{self} only if
      \co{I} renounce any questions about its temporal permanence, about
      \co{actual} reasons and grounds for its identity. For such questions
      involve already all the \co{distinctions} of \co{the visible world}.
      ``Here there is no whence'' nor when.  \co{Self} is involved in \co{my}
      temporal existence, but only through \co{incarnation} and, possibly,
      \co{actual manifestations}. Itself it is only \co{virtuality}, the source
      of the potential for ever new \co{manifestations}, if you like, the will
      to life or, perhaps and with all possible reservations, the will to power.
      
      \pa \co{Self}, as the place of \co{incarnation}, the \co{origin} of the
      personal being, is {\em essentially} \co{above me}, \co{manifestations}
      never reduce it to the \co{actual} \co{visibility} but only reveal within
      it something lying {\em essentially} \co{beyond}. In this way, \co{I} at
      first experience an absolute dependency on this \co{origin}, I can only
      accept its gifts or try to ignore them.  This absolute dependence is, as a
      matter of fact, a dependence on \co{nothing} -- the \co{invisibles} do not
      \co{manifest} any determinate contents, \co{commands} do not tell me
      exactly \wo{Do A and not B}. And even if they do\ftnt{In rare cases, the
        intensity of a \co{manifestation} may so thoroughly penetrate my whole
        being, that it affects all levels of \co{actualisation} and the
        \co{command} is expressed also through a more specific \co{sign}, as a
        direct picture of an \co{impression}, \co{thought}, even perception.}
      they still do it with a \thi{margin of tolerance} -- \co{I} can always say
      \wo{No}. If \co{I} do so, they will still be present influencing \co{my
        life}, but not determining it completely.
      
      \pa Thus, although absolute, the dependence does not involve any
      necessity, there is no irrefutable constraints in its \co{signs}.  A
      \co{command} \co{manifests} something which leaves it entirely up to
      \co{me} to decide what \co{I} shall do about it, even if \co{I} at all
      will accept its presence. And if \co{I} accept it, it is still up to me
      how \co{I} will live this presence. In short, absolute dependence on
      \co{nothing}, on \co{another world} is the same as entire freedom in
      \co{this world}, entire freedom of \co{action} and \co{activity}.  This
      freedom, this entirety of freedom in \co{this world} is possible only
      because \co{this world} is limited by \co{another world}, because the
      necessities of \co{actual} determinations are only relative to \co{this
        world} and do not even touch the sphere of \co{invisibles}.
      
      \pa The \co{virtual} character of the \co{invisibles} opens thus the field
      of freedom of their \co{actualisation}, the filed of \co{my life}. In this
      way, and in this sense, they are the source of meaningfulness.  The
      meaningfulness consists in participation, it is a property which we
      ascribe to \co{visible} things, acts, actions, life, in so far as they
      appear as originating in something deeper.
%
%This \thi{appearing as originating} are concrete feelings which
%\co{manifest} the presence of \co{invisibles}.
%
      The experience of meaningfulness lies in seeing \co{my life}, \co{my
        activities} and \co{acts} as flowing out from the \co{origin}, from the
      \co{self}, which is greater than \co{me}, which is \co{above me}, which is
      \co{my self} but not merely \co{myself}.
      
      \pa This should not be confused with all kinds of ways of attaching
      meaning to one's life by replacing the meaningfulness by purposefulness.
      Purposes and goals, \co{acts} and \co{actions} are in themselves, that is,
      to the extent they are \co{dissociated} from their \co{origin}, devoid of
      meaning.  Devoting one's life to useful social work will never make this
      life meaningful but just that -- useful.  But devoting one's life to such
      a work under the \co{invisible} \co{command} is a different matter -- it
      is already from the start involved in the sphere of meaningfulness, even if
      it turns out to be useless.  \wo{Works are required for justification,
        that is, for holding heaven and earth together, but they are not
        meaningful without the gift of grace.  They are required for salvation
        but do not guarantee it.}  }}
} %end noo

\subsubi{My self, self and one}\label{sub:SelfOne}
%
\pa The \co{transcendence} of \co{invisibles} does not any longer offer grounds
for distinguishing its \co{horizontal} and \co{vertical} dimensions.  It can
acquire horizontal character when, for instance, viewed only through the
\co{reflective, symbolic signs}.  To the extent \co{symbols} are {empty}, they
do leave a space for progress, they do call for being \thi{completed}.  But true
\co{symbols}, the ones which are experienced along with the \co{commands} and
\co{inspirations} of the \oss, reveal only pure and simple \co{transcendence} of
essential \co{non-actuality}.  \co{My self} is the field of \co{invisibles} and
\co{self} is, so to speak, its peak. \co{One} is beyond even that.  We comment
now on these three stages of ultimate \co{transcendence}, but first observe
their constant \co{presence}.

\co{Self}, essentially and ultimately \co{non-actual}, is \co{present} only
through \co{my self}.  Being only \co{virtual origin}, it does not \thi{exist}
without \co{my self} nor other, lower levels of \co{myself}; to the extent it
appears, it does so only through some more or less discernible, though hardly
ever \co{precise} contents.  Moreover, as \la{haecceitas}, the 
\co{trace} of \co{birth}, \co{self} is the constantly \co{present} \co{aspect}
of my all being, whether it is itself experienced or not.  This \co{presence}
(one would almost like to say, the \wo{\co{immanent presence}}, though this
could appear a \la{contradictio in adiecto}) does not contradict the
\co{non-actual transcendence}, as the \co{reflection} capable only of flat
oppositions and contradictions would like it. \co{Immanence} is \co{actuality}
and, strictly speaking, one can \co{posit} an insurmountable contradiction. But
\co{actuality} is always surrounded by the \co{rest}, the \co{invisible} horizon
or, perhaps, the horizon of \co{invisibility} which, \co{non-actual} and
\co{non-actualisable}, permeates with its \co{presence} the whole \co{actuality}.
This \co{presence} of \co{self} (or the whole world, or \co{my world}, even of
God) is not \co{actuality}-in-person, exhaustive and full presence of the given.
Yet, permeating and affecting the \co{immanence}, it can be ascribed if not the
\co{immanence} as such, then at least \co{presence} through and within the
\co{immanence}.\ftnt{\citef{The principal causes, then, both proceed into {\em
      the things of which} they are causes and at the same time do not depart
    from their Principle [...]}{Periphyseon}{II:551D/552A} Just like \co{self}
  exists only through \co{my self}, without being the latter, so \citefib{our
    reason for saying that the primordial causes are co-eternal with God is that
    they always subsist in God without any beginning in time, and our reason for
    saying that they are not in all respects co-eternal with God is that they
    receive the beginning of their being not from themselves but from their
    Creator.}{Periphyseon}{II:561D/562A} The arguments for and against Eriugena's
  pantheism and for or against his theism could probably be directed for or
  against our exposition.  The fact that he makes it close to impossible to
  distinguish between the alternatives seems to us the fundamental strength --
  and not any weakness -- of his exposition.}

\pa \co{My self} is \co{self} but being is asymmetric and so \co{self} is not
\co{my self}. \co{My self} is \co{self} to the degree the latter emerges as
\co{mine}.  At first, the contents of experiences of \co{my self} tend to be
identified with \co{self}. But no matter how little or much contents \co{I}
manage to discern in such experiences, the \co{self} seems never exhausted by
them. \co{I} can never grasp it.  True, contents can \co{manifest} some truth,
they can suggest some kind of direction but even although not \co{mine} in this
sense, they are directed to \co{me} and are what makes \co{self} \co{my self}.
No \co{self}, however, appears.  \co{Self} can never be content of any
experience and saying that \co{my self} is \co{my} experience of \co{self}, the
\wo{of} must not be taken as indicating the content, not to mention the
\co{object} of the experience.  \co{Self} is an \co{invisible aspect} of every
experience.  In the particular experiences of
\co{my self} it is the \co{aspect} which also becomes \co{manifest}, but its
\co{manifestation} is then not related to their content. It is \co{present} only
as their force: irresistible, binding, \co{absolute}. Experiences themselves can 
be entirely \co{vague} because the \co{objectivity} of their content is close to
nil.  It is only the shaking intensity, the tremendous power, the immediately
recognisable significance of such experiences -- of revelations, founding
events, {archetypal dreams} -- which, clothed in the more definite contents of
\co{my self}, signal the \co{presence} of the \co{transcendent self} beyond any
discernible contents.

This power and intensity, irrespectively of the content, is what makes such
experiences the {calls}, the \co{commands}.  They make \co{self} appear as a
challenge beyond \co{my self}; the challenge which tradition formulates as
\wo{Become yourself}, but which we would have to at least parse differently as
\wo{Become your self}, and preferably rephrase as \wo{Become self.}  The
adolescent questions \wo{Who am I?}, \wo{Who am I \co{actually}, really?} are in
a sense {expressions} of the call, although the call which does not necessarily
come as a particular experience at any definite time. They expect answers,
expect to find something which one could {\em see}, some \co{actual} \thi{what}
which distinguishes \co{me} from others -- and thus dissolve in \co{egotic}
divagations. Such questions do not help answering the call, although they may be
a stage towards it. For what is \co{commanded} is to forget \co{oneself}, to
stop \co{dissociating} \co{my self} from \co{self} without, of course, denying
the distinction between the two; to realise the being \co{I} have always been,
but which is not and never will be \co{me}.\noo{As the Aristotelians and
  Thomists would say, to \thi{actualise the universal form \thi{man}} (although
  the language hardly fits ours).}  \co{Self} is the \thi{inborn possibility} of
\co{my self}. But as long as \co{I} insist on \co{mineness} of \co{self}, \co{I}
still ascribe irrefutable and uncontested validity to the categories of
\co{visibility}, \co{I} still try to adjust the higher to the lower.


\pa Now, we have been speaking about \co{commands} and \co{inspirations} in the
plural form. But the differences between various \co{commands} concern only
their content. This can, to some extent, mean \thi{what} they \co{command}, but
primarily it means only the circumstances under which they occur and the way in
which they are received and interpreted.  Certainly, \co{distinctionless}
\ger{Angst} is a different experience from, equally \co{distinctionless},
mystical union; the sudden feeling that \thi{\co{I} am not living \co{my} life}
is very different in content from the religious experience of God's \co{presence}.
Yet, as \co{commands} they do not command anything particular, and to the extent
they do, it is close to impossible to say what. One rather experiences only
\co{that} something has happened, something ultimately significant has visited
the sphere of \co{actuality}, but {\em what} was it? And although it can be easy
to accept \co{that} I should become (my) \co{self}, it is hardly possible to say
{\em what} that possibly might mean.

All \co{commands}, irrespectively of the differences in content are, eventually,
this one \co{command}: \wo{Become self}. Because all the \co{traces} -- of
\co{my} experiences, of \co{my life}, of \co{my self} -- lead to and gather in
the center of \co{self} which is their \co{unity}. Following these \co{traces}
means to \co{participate} in their \co{origin}.

The archetypes (like the primordial causes or intelligibles of the intellect)
form a kind of \thi{field}, various 
points of which can be \thi{activated}, but which remains essentially
unitary. The archetypes are so intimately interconnected that
\thi{activation} of one will almost necessarily lead to
\thi{activation} of another. They are distinct, but are hardly
distinguishable, and if so, then only in \co{reflection} or partial
experiences, but not in their operation. One archetype leads
inevitably to another and, eventually, to the most primordial
\thi{archetype of all archetypes}, \thi{archetype of wholeness} --
\co{self}.\ftnt{A simple, concise account can be found in
  \citeauthor*{Synchronicity}{ Lecture III}.}
Irrespectively of the differences in the experiences of archetypes, of
primordial causes, of \co{invisibles} of \co{my self}, what emerges behind them
is the ultimate site of \co{invisibility}, the \co{non-actualisable}, ever
\co{transcendent self}.  In rare lives the two could almost be pronounced one,
there \la{Atman} is \la{Brahman}, the Son is the Father. In such lives the
\co{distinctions} of \co{myself} and \co{my self} cease to veil the \co{self}.

\pa
But unveiling is not \co{actualising}, and \co{self} remains forever \co{above}
any distinctions with the exception of the first: \co{separation} from the \co{one}.
For \co{one transcends self} just as \co{self transcends my self}. 
%Let's only consider one more distinction. Does \co{one transcend self} or not?

\citet{It is only through the psyche that we can establish that God acts 
upon us, but we are unable to distinguish whether these actions 
emanate from God or from the unconscious. We can not tell whether God 
and the unconscious are two different entities. [\ldots] there is in 
the unconscious an archetype of wholeness [\ldots] Strictly speaking, 
the God-image does not coincide with the unconscious as such, but 
with [\ldots this] special content of it, namely the archetype of the 
Self.}{JungJob}{\citaft{CC}{p.67}}

\co{Self} is by no means the same as unconscious or psyche, but this quotation
goes through without much further ado.  We can neither see nor tell whether
\co{self} and Godhead are distinct and if so, what distinguishes the two.  The
\co{self}, \thi{the archetype of wholeness} is, in fact, the archetype of all
archetypes, the \co{invisible origin}, so often and naturally \co{symbolised} as
a mere point, shall we say, the point in infinity?\noo{It would be hardly
  possible to make a complete overview: Jung's examples of a point, Swedenborg,
  Plato's One=point, Plotinus' One, Leibniz's `naked monad', Wittgenstein's
  `disappearing subject'\ldots} It is the point of \co{origin} which lies
\co{above} all possible \co{distinctions}.  The only point present from the very
beginning, the \co{trace} of \co{birth}.  There is no way to say whether
something comes from \co{self} or from \co{one} because everything, originating
from the latter, comes only through the former.

But although we can be unable to distinguish whether some actions emanate from
God or from the unconscious, we are far from identifying the two.
Individual psyche, the level of \co{mineness} with all \co{visible}
feelings, thoughts and experiences can be conflated with God only by a
psychological reductionism. It is only the \co{self}, hidden beyond and
\co{above} all \co{visibles} and \co{invisibles}, that comes closer to \co{one};
not, however, so close that it can be identified with it.

Our scheme will claim the difference between \co{self} and \co{one}; the
difference established by \co{birth} which constitutes \co{self} as the
God-image: the \co{nothingness} of the point which is the whole and only
reflection of the infinity of the line, of the \co{nothingness} of \co{one}; and
then, differentiation of life and thought which, distinguishing the
\co{indistinct}, conducts a constant dialogue with the \co{one}. Everything is
but a reflection of \co{one} and thus \co{one} is always \co{present}
(I:\ref{sub:pantheism}), it is \co{present} only through \co{self}, and
\co{self} is \co{one} because it is \co{separated} only from \co{one}. But
\co{one} is not \co{self}.  The \co{separation}, \co{birth} is exactly what
establishes the \co{confrontation} of \co{self} with the ultimate
\co{transcendence}, and what precludes their coincidence -- precludes, that is,
until death, which is the only return of coincidence, \co{indistinctness}.
\co{Birth} does not establish a being which then, somehow, becomes
\co{confronted} with \co{transcendence}. The \co{separation} of \co{birth} is
{\em nothing more} than such a \co{confrontation}.  There is no substance, no
essence, nothing more to this fundamental aspect of being \co{self}, than being
alive, that is, being \co{confronted} with the \co{transcendence},
\co{distinguishing} the \co{indistinct}.  If the essence of \co{one} is
\co{that} it is, the essence of \co{self} is \co{confrontation} with \co{that}.
This \co{separation}, and the eternal primacy of \co{one}, is what we can
understand by \co{one's} ultimate \co{transcendence}, I:\refpf{primacyOne}.

\noo{ There is nothing more to \co{existence} than to be \co{born},
  \co{confronted} with the \co{one}. And thus we can easily say \co{that} it is
  but never \thi{what} it is -- not because there is some hidden \thi{what} but
  simply because {\em all} it is is its own \co{that}.\ftnt{This looks like we
    have attributed to \co{existence} what the Scholastic tradition attributed
    only to God, namely, that \la{quid sit} is \la{se esse}, its essence is its
    existence (cf. \refpp{pa:essenceBeing}).  Perhaps, though, it happens simply
    because we are not looking for any specific universal \thi{form} of man
    which could be, in analogy with the concepts of things and objects,
    dissociated from his \co{existence} and \co{posited} as a simple,
    all-exhausting definition.  \co{Birth} is the ultimate principle of
    individuation and no \thi{matter}, no \thi{accidents} are needed to endow
    this \thi{form} with \la{haecceitas}, to make it a \co{concrete} individual.
    \noo{This does not make man a \thi{necessary being}, but neither did this
      assumed coincidence of \la{quid sit} and \la{se esse} made God into such a
      being. For it only means that if we want to {\em think} man (or any living
      being; not merely \thi{life in general}), we have to {\em think} him as
      \co{existing}. This platitude is not the whole point, but a big part of
      it.
      
      But! But! -- this does not distinguish man from other living beings!  No,
      it does not. {\em This} does not distinguish man from living beings
      because {\em this} is exactly what we understand by \co{existence}. There
      are plenty of other, lower level distinctions.  }  About the similarity to
    or difference from God we will perhaps say something later on. For the
    moment, let us only note perfect similarity of expression, even of
    \thi{contradictory} attribution, with respect to man and God, motivated by
    the fact that man is God-image: \citef{For the human mind [...] knows that
      it is, but does not know what it is.  [...] For just as God is
      comprehensible in the sense that it can be deduced from His creation that
      He is, and incomprehensible because it cannot be comprehended by any
      intellect whether human or angelic nor even by Himself what He is [...] so
      to the human mind it is given to know one thing only, that it [human mind]
      is -- but as to what it is no sort of notion is permitted
      it.}{Periphyseon}{IV:771ABC} }
} % end noo

\subsubi{Ambiguities of objective-subjective}\label{sub:objsubj}

\pa\label{pa:objsubj} \co{Self} as the \co{trace} of a particular \co{birth},
and even more \co{my self} as \co{my} {experience} of \co{self}, seems most
\thi{subjective}: it can never be grasped in the \co{actual}, \co{objective}
categories, it can never be reduced to whatever \co{actual signs} announce about
it.  But at the same time, being essentially \co{non-actual}, it is most
{objective} -- not because it is \co{external} (which it is not) but because it
is essentially \co{invisible}, lies beyond the \hoa\ and, hence, beyond the
horizon of any {subjective} control -- \citet{man infinitely transcends
  man.}{Pensees}{VII:434\label{cit:manMan}} \co{Self} can be listened to but not
commanded.  It is the place of \co{incarnation}, the most intimate, deepest
source of every \co{manifestation} of \co{invisibles} and the \co{origin} of
every appearance.  Thus, perhaps, the deep \wo{subjectivity is truth}, not
because there \co{subject} and \co{object} finally coincide but because there,
in the \co{virtual nexus}, lies the origin of this, and every other,
\co{distinction}.

\pa This ambiguity lies deeply in the usage of the word \wo{objectivity}.  We
tend to think of it as physically \co{external} {things}.  This crude
materialism (followed necessarily by some form of psychologism, as it must
reduce all concepts, even mathematics, to \thi{subjectivity}) is expanded a bit
by thinking of objectivity as that which exists independently from me, what is
\thi{outside}.  Here comes the crucial ambiguity, for what is independent from
or \thi{outside} me depends on what is understood by the word \wo{me}.  One will
typically understand by it \co{myself}, the horizon of things, feelings,
experiences which are classified according to the category of \co{mineness} as
opposed to what is \co{not-mine}.  This category comes very close to that of
will, that is, \co{my will} and, eventually, of control and mastery.  Things are
\co{mine} to the extend \co{I} have a sense of mastery over them or, in the most
vulgar sense, to the extend \co{I} possess them.

But then what is \thi{outside}, what is \thi{objective} is that which I do not
master. This characteristic not only turns everything I can control into
something \thi{subjective} but also threatens the absolute determination of
\thi{objectivity} which becomes relative to \co{my} mastery. This, however, need not
be so if there were things which I never could control, which in principle would
remain outside the horizon of my mastery. And so we keep digging deeper and
deeper, keep looking further and further, and fascinated by the new,
\thi{objective} discoveries, soon realise that they are as relative to us as
were the old ones which we left behind. Hunting for a shadow, we find pieces of
rocks and, dissatisfied with our constructions of stone, keep looking, chasting
the wind\ldots

Mastery and control do not seem to be the definite features distinguishing
{subjective} from {objective}.  Certainly, the glass in front of me, which I can
manipulate as I wish, is there \co{objectively}.  Why?  Because it is
\co{external}, it \thi{exists independently from me}, I can leave, disappear,
die -- the glass will still be there (unless something destroys it).  But this
\thi{independent existence} is but a new variation on the theme of mastery and
non-mastery, in any case, of relativity.  As a privative of \wo{dependence}, it
simply means that I do not have full control over this existence, that it keeps
something away from me.\ftnt{Latin \wo{\la{dependeo}} means to hang on, from or
  down, with variations such as to \thi{be derived from} or \thi{be governed
    by}.} What this something is, nobody could ever tell\ldots And so we return
to the relative character of \thi{objectivity}, relative to the \thi{subject}
from which it must be independent.

\pa The ambiguity can be formulated as follows.  The pure \co{subjectivity} is
the pure \co{immediacy}, the coincidence of the \co{sign} and signified, of
action and reaction. Everything which slips out of this ideally narrow circle by
carrying an element of non-\co{actuality}, acquires a degree of {objectivity},
though not necessarily \co{objectivity} as mere \co{externality} confined to the
\hoa. The problem is that non-\co{actuality}, where something \thi{objective}
and not relative to mere \herenow\ was expected, brings in something into the
\co{actually} given, something which seems uncanny subjective. Whatever I bring
into the \co{actual} situation is of a \thi{subjective} character; another
person might bring in other preconceptions and distinctions. As Bergson used to
observe, the apparent indeterminacy of one's actions and reactions derives from
one's memory, from this \co{non-actual} element which one's \thi{subjectivity}
contributes to the \co{actual} situation.  Thus \thi{objectivity}, having made
an unsuccessful leap towards the most universal and \co{non-actual} laws, tends
back towards more and more \co{actually} given. These two extremes -- whether as
\thi{form} and \thi{matter}, or as \thi{universal laws} and \thi{actual data},
as \thi{theories} and \thi{observations} -- delimit the understanding of
\thi{objectivity}.  Unfortunately, both extremes can be equally well
characterised as \thi{subjective}: the \co{non-actuality} which is accumulated
only in the \thi{mind} of a \thi{subject}, and the pure \co{immediacy} which is
not only \thi{theory dependent} but also relative to \herenow, that is, to the presence
of a \thi{subject}.

Instead of analyzing the possible and impossible variations of this ambiguity,
we only give a few examples which relate directly to the deflation of \co{self}
or, what amounts to the same, inflation of \thi{subject}.

\ad{Inside vs. outside} Insisting on {\em my} in \co{my self} amounts to a
projection of the \co{visible} categories and introduces an antinomic
dialectics of pseudo-opposites, of \thi{within} and \thi{without}, \thi{inside}
and \thi{outside}, \thi{subjective} and \thi{objective}.  A prophet speaking
from \thi{within}, can be accused of projecting extreme \thi{subjectivity} onto
the \thi{objective} world.  Yet, he does not speak {\em his} words but only
words which were revealed to him, which were \co{manifested} \thi{inside}.
\citet{They do not think so much as thoughts come to them.\lin The
  \gre{\d{r}\d{s}is} [seers] are not so much the authors of the truths recorded
  in the \btit{Vedas} as the seers who were able to discern the eternal truths
  [...]}{IdealistView}{V:2\lin III:3\kilde{p.147,70}} Prophets relate only what
has been revealed -- \thi{inside}, but {\em from} \thi{outside}; they only
\citet{repeat words they have heard in secret amidst the
  silence.}{WeilWaiting}{Letter IV.Spiritual Autobiography;p.35\noo{p.79
    \citeti{For it is not [they] that speak, but the Spirit of [the] Father
      which speaketh in [them].}{Mt.}X:20 } Absence of any \co{visible}
  sources and reasons of the visions and spoken words, combined with their
  eventual truth, might serve as at least partial justification for viewing
  prophecy as \citef{an emanation sent forth by the Divine Being through the
    medium of the Active Intellect, in the first instance to man's rational
    faculty, and then to his imaginative faculty; \noo{it is}[...] the highest
    degree and greatest perfection man can attain}{Perplex}{II:36}. But of
  course, one can also wait for a more mundane explanation which, however, is
  still, as it always has been, to be waited for.}\noo{[\citef{when ye received
    the word of God which ye heard of us, ye received it not as the word of men,
    but as it is in truth, the word of God}{IThes}{II:13}]} It is not our
objective to design psychological theories of possible factors conditioning such
experiences and classifying them as \thi{subjective illusions}.\ftnt{True,
  sometimes, they may be illusions.  The whole art is to learn to distinguish
  those which are from those which are not, and we are far from pretending that
  we have any clear-cut criteria for doing that.  \citef{Divining in advance
    whether our dark partner symbolises a shortcoming that we should overcome or
    a meaningful bit of life that we should accept -- this is one of the most
    difficult problems that we encounter on the way to
    individuation.}{MHSFranz}{p.184} } They may be right and may be wrong but
never touch the quality and meaning of the experience.  And in the experience,
on the one hand, it is one's \co{self}, something very intimate \thi{within},
which orders one to speak and act in this rather than that way, which is the
source of the experience and its \co{commanding} character.  On the other hand,
one's \co{self} is not \co{oneself}, it does not obey one's \co{will}, and so
the origin lies \thi{outside}.  It is both \thi{inside} and \thi{outside}.
% Comparing the presence
% of divine gentleness to light bursting forth in our \co{self}, Plotinus remarks
\citet{We wonder whence it came: from the outside, or from the inside?  Once it
  disappears, we say, ``It was inside -- and yet, no, it wasn't inside.''  We
  must not try to learn whence it comes, for here there is no
  \thi{whence}.}{Plotinus}{V:5.8}

The \thi{inside}-\thi{outside}, like other spatial analogies, can to some extent
apply to the level of \co{egotic} drives and their \co{objects}, perhaps also to
the level of \co{mineness} confronted with the vast \co{world} which is
\co{not-mine}, but never to the level of \co{invisibles}. Put other way, any
appearance of (not to mention dependence on) such categories signals a discourse
of a lower level, even if it 
attempts and pretends to address a higher one.

\noo{
\tsep{not spirit, not matter}
%%%%%% materialism
\pa This analogical thinking, which asks questions like \wo{What is
  \thi{outside} the mind?}, brings one eventually to apply the spatial analogies
where they possibly may be applied, and to ask instead \wo{What is \thi{outside}
  the body?}. The solipsism of \thi{brain in a vat} is a materialistic
counterpart of the traditional solipsism of idealistic flavour. It is, however,
much less convincing than the latter used to be. It terminates the Cartesian
search for the site of the soul somewhere in the brain and declares, instead,
the brain to be all that is experiencing. One can, however, hardly resist
doubting whether life without body and autonomous nervous system would feel
exactly as the life feels now. In our language, a brain is just an \co{actual}
organ, functionally perhaps more indispensable than many others but in its
metaphysical status not much different from them. Just like spiritual life can
never be reduced to mere \co{actual representations}, so body can not be reduced to any
of its organs.

The \thi{body}-\thi{mind} dualism reflects to some extent the dualism of the
\thi{objective}-\thi{subjective}.  But \co{self}, just like \co{one}, is not
spiritual as opposed to material.  It is both, or rather, as a mere \co{trace}
of \co{separation}, a mere point in the middle of \co{indistinct nothingness},
it is neither.  Jung puts it this way: \citet{\label{hmm}The deeper layers of
  the psyche lose their individual uniqueness as they retreat further and
  approach the autonomous functional systems, they become increasingly
  collective until they are universalized and extinguished in the body's
  materiality, i.e., in chemical substances.  The body's carbon is simply
  carbon.  Hence <<at bottom>> the psyche is simply <<world>>.}{MHSJaffe}{p.310
  [quoting Jung]} But, alas!, this looks like a plain materialism. It concludes
that the psyche is \thi{world}, but what it actually says is that \thi{mind} is
\thi{body}, the \thi{matter} of organic and chemical processes.

Indeed, \thi{mind} exists only in \thi{body}. In fact, \thi{mind} {\em is} the
\thi{body}, the \thi{body} which distinguishes. We shouldn't even say that it is
a totality of distinctions gathered in the \thi{body} -- it {\em is} the
\thi{body}.  As Spinoza might have said, \thi{mind} and \thi{body} are but two
\co{aspects} of a \thi{substance}, in fact, only different perspectives from
which a living organism can be seen.  The \thi{mind} of an ant is the \thi{body}
of an ant making the \co{distinctions} of which it is capable. You can not put
this \thi{mind} into a \thi{body} of an eagle. The \thi{mind} of a person who
lost his leg is different from his \thi{mind} before the loss.  The difference
may be negligible and indiscernible, yet in many cases it will give rise to
observable phenomena (as witnessed by all kinds of centers helping victims of
accidents and catastrophes to cope with psychical and physical damages.)

This may look like a bad materialism but it is not.
As we discussed in I:\ref{pa:matter}, not only the distinction
\thi{body}-\thi{mind} but also \thi{matter}-\thi{spirit} is already a
\co{distinction}. It happens far \co{below} the \co{separation} of \co{self}
from \co{one}, in fact, fully within the \hoa\ from which it is expanded beyond
the limits of its validity.  \co{Self}, just like \co{one}, is \co{spiritual} in
the more specific sense of being related exclusively to the \co{indistinct
  nothingness}. But this is not \co{spirituality} which is in any sense opposed
to matter, because it is not opposed to anything whatsoever. It is
\co{confronted} with, that is, it exists only in virtue of its involvement into
\co{nothingness}, and \wo{spiritual} may be as good (or as bad) a name for that
as \wo{matter}. Outside my body there are, of course, other bodies, just like
outside my mental horizon there are thoughts and ideas of others which I am not
capable to embrace or understand. \co{I} am able to draw the \co{distinctions}
between these variants of \thi{insides} and \thi{outsides}. But \thi{outside}
\co{self} there is only \co{nothingness}, and none of them is more nor less
\thi{objective} or \thi{subjective} than the other.
}

\ad{Objective values vs. subjective preferences}\label{pa:allsubjective}
The ambiguity between \co{objectivity} of \co{external objects} and more vague,
but also more common, {objectivity} of independence, illustrated by the
following example, is quite a common ground of confusion and misunderstandings.

Max Scheler claimed the existence of an {objective} hierarchy of values which,
of course, appeared most repulsive to all forms of empiricism identifying values
with \co{subjective} inclinations and mere \co{acts} of preferences.  \wo{Of
  course that things do not have any value in-themselves, there are no values
  independent from our (or other) being for which objective things may or may
  not have some value.} The conflict, if there is any, rests exclusively on the
understanding of the words \wo{objective}, \wo{subjective}, eventually, of
\wo{I}.

For the empirically minded opponents of the {objective values} the
\thi{subjective I} is an atom, an unstructured (even if, possibly, complicated)
entity which determines the pole of subjectivity.  Everything which is in one or
another (but always unclearly specified) way ascribed to this \co{ego} as
somehow originating in it, is subjective.  Sensations, perceptions, impressions
and feelings are as subjective as thoughts, ideas, intuitions, revelations --
they are all \co{mine}.  Going only a bit further, the \thi{objective} is
something which not only is completely independent from and not relative to
\co{me}, but which is so also in relation to all humans, perhaps even to all
other beings.  Strangely enough, as the examples of \thi{objective} beings, one
manages to mention only pieces of chalk, chairs, tables.  The rejection of
{objectivity} of values rests on the conflation of one's whole, structured being
with an ideal noumenal unity of a \co{complex ego}, and claiming that everything
relative in whatsoever way to this being is equally \thi{subjective}.
Equivalently (for this is only another expression of the same conflation), the
rejection rests on pointing at tables and chairs, and making the trivial
observation that values are not things of the same kind.

Values are not \co{objects} and yet they are found residing objectively
\co{above} every particular \co{existence}.
%
The following analogy may help clarifying the issue.  We consider thoughts to be
our possessions, \co{my} thoughts to be {\em mine}, in the sense that others do
not have access to them unless \co{I}, their originator and possessor, reveal
them. Their subjectivity consists also in that once formulated and conscious,
they can be to some extend manipulated and controlled by \co{me}. However, it is
only control of a limited degree because thoughts (unlike phantasies) have
content and presentation which is not completely underlied \co{my will}.
Moreover, except for some very special cases, \co{I} do not generate \co{my}
thoughts.  \co{I} encounter them, often when \co{I} think about completely other
things, often when \co{I} do not think at all (changing the focus of attention
after a prolonged concentration on a problem provokes often spontaneous
emergence of most creative solutions). One will delegate the responsibility for
the appearance of thoughts to the \thi{brain}, which is supposed to work beyond
the threshold of one's consciousness. But such a delegation of responsibility
can at best serve as a partial (pseudo-)explanation in terms of the currently
accepted imagery -- it does not in any way contribute to rendering
comprehensible the meaning of such a \thi{foreign element} in the midst of one's
most intimate privacy. Even if \co{I} possess \co{my} thoughts, \co{I} am not
their originator or, even if I am, it is not \co{myself} in the same sense of
the {subjective}, \co{actually} conscious \co{ego}.  \citet{Subjectivism is
  mistakenly connected with \la{apriorism} only when \la{apriori} is taken not
  only as (exclusively) primary <<law>> of acts, but also as the law of acts of
  an <<I>> or a <<subject>>.}{MaxForm}{II:A.\kilde{p.94}\orig{Subjektivismus ist
    mit dem Apriorismus aber auch dann irrig verkettet, wenn das Apriori nich
    nur als (ausschlie{\ss}lishes) prim\"{a}res \thi{Gesetz} von Akten, sondern
    au{\ss}erdem noch als das Gesetz von Akten eines \thi{Ich} oder eines
    \thi{Subjektes} gedeutet wird.} The point is much subtler than we are making
  it here. Scheler objects to any kind of <<subject>>, whether empirical or
  transcendental, as some posited content while, as he maintains, \la{a priori}
  can concern only acts as such: their structure, intentions, values they
  express. But our only concern is the fact that the \co{actual} intentions and
  \co{acts} carry \co{aspects} which are neither determined nor controlled by
  \co{ego}.  Whether these are \la{a priori} or not (and where one at all wishes
  to draw the line of this opposition) is another issue. }
%
The {objectivity} of the hierarchy of values can be understood plausibly along
the same lines as the fact that \co{I} do not create values, \co{I} do not
determine that \thi{good} is better than \thi{bad}, that \thi{holy} is higher
than \thi{pleasant} -- \co{I} encounter this as an {objective} fact.\ftnt{If
  somebody does not, and claims that, for instance, \thi{pleasant} is a higher
  value than \thi{holy} then he can be easily accused of some blindness to
  values or illusion in their perception.  But even if we do not do that, and if
  his statements are not based on merely arbitrary choices, then he, too,
  encounters \thi{pleasant} as higher than \thi{holy}, he {\em finds them being
    so} and does not merely decide them to be so.}  The objectivity of values
reflects the stratification of \co{existence} in which the \co{active subject}
encounters not only \co{external} things but also ones which, affecting his
motives, goals and \co{acts}, do not originate in his activity and are not
reducible to its results.

\ad{Self vs. ego}\label{pa:SelfEgo} The ambiguity just sketched becomes very
prominent in the\noo{quarrels about the meaning and} refutations of the ultimate
truth of the dictum \wo{Man is the measure of all things.}  Most trivially one
would interpret it as stating that \co{subjective} goals, wishes and whims,
feelings and perceptions measure the true being of all things and refute it as a
sophism -- at best immediately, and at worst, using half of a lengthy
dialogue.\noo{the same wind is blowing, and yet one of us may be cold and the
  other not, or one may be slightly and the other very cold [...]  not every man
  is the measure of all things -- a wise man only is the measure: Theaethetus} A
bit more psychologically, one might observe that indeed one's moods and general
situation will influence the way things are experienced and handled. One would
still try to maintain some opposition to the hard and \co{precisely objective}
truths about things, though this becomes a bit more problematic.  Finally, one
can take \wo{man} to mean not the \co{actual} or psychological subject but the
\co{self} of \co{existential confrontation}. \wo{Measure} loses then most of its
normative and moral character and becomes a mere reflection of the relativity of
beings to \co{existential distinctions}.
  
Historically, most forms of (the attempts at or claims to) \co{actually}
controlling the higher, spiritual dimension -- whether ancient forms of magic,
theurgy, this \citet{power higher than all human wisdom, embracing the blessings
  of divination, the purifying powers of initiation, and in a word all the
  operations of divine possession,}{ProclusTP}{\citaft{GreekIr}{ p.291}}
ambitions of Pico and then pretensions of a Renaissance magus personified
eventually as doctor Faustus, esoteric practices of the Mdm. Blavatsky circle,
Rudolf Steiner's guidelines for systematic development of genuine spirituality
or, finally, the so called \wo{New Age religiosity} (which is not newer
phenomenon than modernity and romanticism) -- all such forms result from the
conflation of \co{self}, which truly is the source, with the \co{actual ego} for
which this deeply \co{spiritual} truth turns into a trivially flat falsehood.
It leads to the flattening of \co{existence}, to considering it only along the
\co{horizontal} dimension of \co{visibility}, for everything else and, in
particular, \citet{the nature [<<Wesen>>] which man opposes to himself in
  religion and theology as something different from himself, is really only his
  own nature.}{Feuerbach}{III\kilde{p.31}}\noo{such statements can be endowed
  with deeper meaning if only one understands \wo{being} [<<Wesen>>] in a more
  genuine sense, they actually intend exactly the reduction to which we are
  objecting.} Such anthropomorphism, suggesting that the \co{spiritual selfhood}
is really nothing else than the \co{actual subject} with its volitions, choices
and preferences, can hardly avoid the spiritualistic pretensions to the
\co{actual} control of all significant dimensions of human existence. 

The Hindu thinkers avoided the confusion of the \la{atman} with the empirical self,
the \la{jivatman}, just like German idealists avoided the confusion of the
transcendental with the empirical ego.  Kierkegaard accused of this confusion the
Romanticist followers of Fichte: \citet{This Fichtean principle, that
  subjectivity, that Self has constitutive validity and is the only omnipotent,
  has been taken up by Schlegel and Tieck, and applied to the world. This caused
  double difficulty. For the first, one confused the empirical and finite Self
  with the eternal Self; for the second, one confused the metaphysical reality
  with the historical one.}{Irony}{II:4.Irony after Fichte} The same confusion
affected then Kierkegaard's religious individual confronting God; the individual
who, in the hands of French existentialists, became a lonely {ego} laden with
the impossibility to constitute any genuine meanings from \thi{within} its
{alienated subjectivity}.\ftnt{The accusation of this confusion must not be
  confused with the claim of the ontological independence of the two egos. Such a
  claim might, occasionally, seem present, for instance, in the opposition
  postulated by Husserl between the pure and empirical ego. Ingarden clarifies:
  \citef{The subject of the stream of consciousness is not a separate entity in
    relation to the soul [empirical ego] which unfolds in the experiences of
    this stream, but is inherent in it as the axis of its construction\lin Stream of
    experience, subject, soul, respectively human person, are only certain
    <<aspects>> of one, internally compact, conscious being: a
    monad.}{Swiat}{vol.II:XVII.V.79\lin 83} One would be tempted to recall again
  formal distinction of Duns Scotus, according to which, these two (as well as
  other) levels of \co{existence} are both real and yet not really separable.}
The prevalence of this confusion suggests the simple reflection: removing the
\co{absolute} which constitutes the ultimate dimension of \co{existence}, one is
removing also the ground for this very distinction, for now \co{existence} can
only be understood as \co{ego}, as unfolding on one level only, in the sphere of
\co{visibility}.  \noo{Heidegger: confusion of ontological and ontic dimension.}

\ad{Civilisation vs. nature}\label{civObjSubj} A similar ambiguity concerning
the meaning of subjectivity can be
discerned in the understanding of civilisation as opposed to nature. One says
that civilisation, history, society are creations of mankind and, imperceptibly,
tends to find in it a reason to pride.  Since they are created by \thi{us}, they
carry this character of \thi{subjectivity}, of being relative to \thi{us}, being
\thi{our} products, \thi{our} achievements. Nature, on the other hand, remains
\thi{out there}, retaining the character of \thi{given objectivity}. Such
oppositions may perhaps have some relevance in sociological, historical or
ecological studies. But from the point of view of individual \co{existence},
this collective \thi{subject} is but a \co{posited} ideality, like a noumenal self
endowed with features and functions of the \co{ego}. In fact, it
has nothing to do with any \co{subject}, not to mention \co{self}.  In \co{my
  experience} the role of history, civilisation, and \thi{us} who created them,
is very close, if not entirely analogous, to that played by nature -- they
are encountered, they are \thi{objective}, \thi{outside} \co{myself}.
\co{Positing} such a collective \thi{subject} leads quickly to 
the \citet{idolatry of history [which] is born of this unavowed nostalgia for a
  future which would justify the unjustifiable.}{Opium}{II:The control of
  history;p.192} The fact that historical creations are \thi{ours} does not
necessarily make them more human than nature. Inhumanity of some such creations
is no better, and often worse, than inhumanity of the most disastrous natural
calamities.\noo{It seems that the capacity for evil might be as good
  \gre{differentia specifica} of humans as rationality -- unlike humans, animals
  and natural \citef{catastrophes are innocent.}{HerbertAnts}{ Cleomedes}. In
  BOOK III}

This is not the place for yet another critique of historicism. We only want to
observe that the attitudes, reactions and activities required or only made
possible by society can be very similar to those posed by nature. It helps very
little that a political system was created by \thi{us} if to change it requires
sacrifice, courage, dedication and analyses surpassing those required in
confrontation with the non-human world. The human world, the world created
by \thi{us}, is not underlied our control any more than nature is.  \citet{Man is
  no longer able to control the world, which emerged thanks to him; this world
  overwhelms man, liberates itself from him, appears to him as independent and
  man no longer possess the word-spell with which he could {placate} Golem which
  he created.}{ProbMen}{II:1.1} That it is \thi{we} who create history and
society is as true as it is illusory.  \citet{Whether primitive or not, mankind
  always stands on the brink of actions it performs itself but does not control.
  The whole world wants peace and the whole world prepares for war, to take but
  one example. Mankind is powerless against mankind, and the gods, as ever, show
  it the ways of fate. Today we call the gods \wo{factors}, which comes from
  \la{facere}, \thi{to make}. The makers stand behind the wings of the
  world-theatre.}{JungArche}{I:49} We \thi{create} our society and history just
like ants create the ant-heap and its history. Sure, without ants, no ant-heaps.
But we conflate the necessary conditions with the sufficient ones only because
craving a final explanation and control, we are willing to settle for a mere
pretense of either. For an individual ant the heap is as much nature as it is a
society, it is encountered with the same character of givenness as the nature
around, even if this ant contributed far over the average to building the heap
(that is, to strengthening of the society or to preservation of nature?)  The
collective \thi{subjects} are of entirely different nature than the individual
ones, to the degree that talking about subjects seems misleading.\noo{ even if
  one can produce long lists of vague analogies.  If nothing else, then at least
  the time scale brings these two kinds of \thi{subjects} completely out of
  proportion with each other.  The continuity of will in a collective
  \thi{subject} is, like everything else, determined by the continuity of will
  of the involved individuals. A project started today may never reach its
  completion because in a few years the individuals responsible for it may
  change or, perhaps, remain the same but only change their minds. The
  \thi{control} \thi{we} have over our society and history is so feeble that it
  may be justly attributed to nothing else than good luck.  And where is good
  luck there is bad luck also.}
%
The illusion\noo{ so intensely analysed by Foucault,} that \thi{we} create
history and society in a way which is somewhat analogous to the way an
individual unfolds his life, comes from confusing an ideal \thi{us} with an
individual \thi{subject}. (The classical example being Feuerbach and then Marx,
not to mention more abominable examples, 
following Hegel in conflating man's essence with the universal human nature
common to all individuals, which is then posited as the society, the only
subject of history with its \thi{collective will} and \thi{collective reason}.)
The illusion that \thi{we} not only create but also 
\co{actually} control our history and society comes then from considering such a
\thi{subject} as not living on the border between \co{visible} and
\co{invisible} but as being in full control of itself. (At least, in so far as
it manages to escape the power of blind forces of the market and, carefully
calculating and planning its production and history, reaches to the communistic
freedom.)\noo{ For an individual, there are as many possibilities to influence,
  from his \thi{subjective} intentions, the social organisation as there are
  \thi{objective} aspects of this organisation far beyond his control. We wanted
  only to observe another example of the ambiguity: }\noo{ The illusion arises
  from the ambiguities of the {objective} and \co{objective}, of the higher and
  the \co{external}, eventually, of the subjective and the objective resulting
  from the inflation of \co{ego}, which such a generously \thi{humanistic}
  attitude endows with power of moulding its own (that is, \thi{our} own) world.
} The futurological phantasies and socio-historical optimism are but another
face of humanism turned positivism, of an inflation of \co{ego} which has been
endowed with the power of voluntarily moulding its own (that is, \thi{our} own)
world.
% We leave analysing any further distinctions of this kind to sociologists,
% economists and political scientists.

\noo{\pa Analyses of unconscious contents are certainly often needed in
  psychotherapy and may be of invaluable importance.  But the distinction
  between \co{my self} and \co{self} will, in the next chapter, lead us beyond
  the sphere of such analyses, for the healthy balance between \co{visible}
  contents and \co{acts} of consciousness and their \co{invisible origin} is
  eventually based on the \co{spirituality} of \co{self} to which all
  \co{distinctions}, whether \co{visible} or not, are subordinated.
  
  \pa{Shall we also add here, or somewhere around here -- the entering into
    \co{actuality} through {\em both} circle and plane (expectations, vague
    anticipations, etc.)  with exceptions (``Det Demoniske er det plutselige'';
    unexpected brick falling from the roof, a feeling aroused by other feelings
    rather than in relation to the situation (though this is a bit of
    anticipations)); and then, all modern theories of `perception' conditioned
    by `expectations', etc.  which are but variations on Plato's reminiscence
    theory, with \gre{a-letheia} as a privative form }
}  


\section{Above and below}
%Relations lower-higher (in abstracto)}
In Book I we saw the relation of \co{ontological founding} of lower levels by
the higher ones, the stages of hypostases, the gradual \co{actualisation} of the
\co{original} \co{virtuality}.  This Book has, so far, described the
\co{reflection} over and of these levels in the \co{reflective experience} which
begins and proceeds upwards with the categories of \co{reflective dissociation},
but which is grounded in the ontological hierarchy.  Our being is structured
into different levels according to the temporal scope and forms of
\co{transcendence}, and this structure reflects the ontological hierarchy of
regions of Being, the generative order of hypostases.  The hierarchy is
experienced through different kinds of \co{signs} and their juxtapositions in in
any \co{concrete experience}, as well as by the \co{rest} which surrounds every
\co{actuality} with the \co{traces} of its origin.

% The \co{distinction} of different levels may seem to be a bit
% arbitrary and, as always, I admit that there is no necessity of
% viewing things this way.  But that they are not necessary is no big
% problem because, in fact, no \co{distinctions} are necessary. 
% (\co{Any experience} involves all the levels and from the next section
% on, we will begin addressing this unity.)

The \co{distinctions} between the levels, grounded in \co{experience} as they
might be, are \co{distinctions} of \co{reflection}, that is, are prone to giving
rise to \co{reflective dissociation}.  \co{Reflection} tends to \co{re-cognise}
the \co{distinctions}, if not the hierarchy, of levels and it often builds its
models around particular level.  Various \co{complexes}, \co{concepts} and ideas
acquire vastly different character when viewed by \co{reflection} from a limited
perspective of a single level.\noo{The claim would be that most philosophical
  ideas, or rather most ways of modeling in philosophical thought, could be
  understood as originating in a perspective relative to a particular level.}
%%%%%%
However, \co{any experience} involves all the levels of Being -- the \co{one} as
its ultimate, \co{transcendent} and \co{invisible} yet always \co{present} source, the
unity of \co{life} and \co{the world} as the horizon of the \co{visible
  distinctions}, the 
\co{complexes} of \co{actual} situations and involved \co{recognitions} and,
possibly, the \co{immediate} \co{objects}.  In phenomenological terms, every
\co{immediate} perception is involved in the texture of \co{actual} \co{mood} or
\co{impression}, surrounded by the feeling of \co{my life} and, ultimately (that
is also, trans-phenomenally), by the \co{invisible} \co{inspiration}.

This section summarises schematically the distinctions between various levels
and the ways in which they are related to each other.  This should clarify not
only the differences offering \co{reflection} the possibility of considering
\co{experience} from different perspectives, but also the unity of \co{an
  experience} which always involves all the levels. Thus, this section addresses
in a preliminary fashion the proper \co{unity} of \co{existence} which will be
treated in Book III.

\subsection{The hierarchy of levels}


\noo{Each level has the associated kind of \co{distinctions}, their degree of
  \co{precision} and scope of limits, which characterise the ontology specific
  to this level level. The word \wo{being} or \wo{existence} will acquire some
  of this specificity.  Indeed, the levels of \thi{to be}, starting with the
  \co{existential confrontation}, proceed through the personal \co{I} living in
  the world towards more and more \co{objective} and eventually \co{external
    objects}, mere beings-at-hand. This apparent equivocity of \wo{being} is,
  however, only a symptom, is only a \co{sign} of various ways and levels of
  being, of various ways in which \co{One} is in \co{confrontation} with
  \co{existence}. }

\pa
First of all, the notion of time itself
% , which in high degree conditions the differentiation into levels,
acquires quite different determinations depending on the level of
\co{reflection}.
In terms of \co{actuality}, it is the primary factor distinguishing the levels.
The further down we move in the hierarchy, the more we approach
ideal \co{immediacy}.  In terms of objective time, it means simply
shorter time.  But objective time offers only an analogy of limited value.  It
would require to say that the level of \co{invisibles} corresponds to infinity
of time, which is a rather poor picture. The further down we move, the more we
approach the actual experience of \co{objective} time, while the further up, the
closer we are to the \co{origin} of time.  The level of \co{invisibles} lies
\co{above} time in the sense that its \co{distinctions} are not prone to
{objectification} and hence are not involved into the temporality which is
but an \co{aspect} of \co{objectivity}.  It is the level involving only essentially
\co{non-actual} \co{distinctions}, that is, \co{distinctions} whose value and
structure, if not the content, is not affected by the temporality of
\co{experience}. As lying 
\co{above} time, they may be termed eternal, while the lower levels, with their
inherent temporality are only \citet{moving image of eternity.}{Timaeus}{VII}. 

Trying to understand the world in terms of pure \co{immediacy} alone, positing
the pure \herenow\ as the only reality, would imply that there is no time but
only a point-like pure \thi{now}.\noo{Since it is hard to deny that we do
  have an experience of time and duration, one may be then led to claim (as some
  theories do) that the world is re-created at each (time-)point anew.  Who and
  why would bother to invest so much work into making us feel that things last,
  is entirely unclear to me, as are possible (if any) advantages of such a view.
  The fact that such a theory might be developed without contradictions should
  illustrate what the criterion of \thi{being contradiction-free} is worth.}
Perceiving the world in \co{objective} terms of \co{complexes} and their
relations yields the objective time, succession of ideally \co{dissociated} time-points.
Perceiving it in 
terms of \co{mineness} or, more generally, in terms of a unity of a living being,
yields the \thi{time of life} and, in fact, the \thi{lived time}. Bergson's
\fre{dur\'{e}e} is an excellent and most thoroughly worked out example of this
perspective. Less inspiring examples are provided by some phenomenologically
grounded existentialistic theories of time as unity of past, now and future
centered around the ecstatic actuality of lived experience.
Finally, \co{reflection} focused exclusively on the level of \co{invisibles} leads to
denying the reality of time. Unlike in the first case, however, it does not posit
the exclusive reality of pure \co{immediacy} but, instead, some form of
supra-temporal eternity, of which \herenow\ is but a manifestation. 

%Variants of this will often be present in mystical thought.


We can summarise this \co{trace} of temporality schematically as follows:

\levsTab %{9.5}
{The modifications of the temporal aspect}
{& \co{experienced} & {objectified time}}
{\thi{shortest experienced time} & {ideal now}}
{time of things & {finite and limited}}
{time of \co{my life} & {finite and unlimited} }
{\co{above} time & {infinite} }
%{experienced} {abstract}


\pa
Modifications of temporality are reflected in the character of 
the \thi{objective} contents of the \co{experiences} of the respective 
level.

\levsTab %{9.5}
{The modifications of the \thi{objective} aspect}
 {& in \co{experience} & \la{in abstracto}/objectified}
{the \co{sign} itself & \co{object}}
{situation & \co{complex}}
{\co{the world} & universe}
{\co{origin} & the \co{one}}
%
This \thi{objective} \co{trace} is thought in dissociation from the temporal
dimension.  It has a spatial character of simultaneity or \co{actual}
co-presence. Even when we think \thi{the world} with its history and
development, we still tend to \co{posit} it as an \co{actual} totality.
%
\noo{ still thinks \thi{the world} as a permanent entity which lasts through all
  the changes it has undergone.  The unimaginable, ideal limit, \thi{the
    universe}, \thi{the whole} corresponds then to the \thi{objective}
  permanence.}
%
Just as \thi{objectivity} carries this spatial character, so the temporal
dimension dominates the modifications of the \co{experiences} of the
\thi{subjective} aspect.

\levsTab%{9.5}
{The modifications of the \thi{subjective} aspect}
 {& in \co{experience} & \la{in abstracto}/objectified}
{organ & ideal subject}
{body & \co{ego}}
{\co{my life} & \co{myself}}
{\co{my self} & \co{self}, sacrum}
%
Of course, the \co{dissociation} of \co{spatiality} from \co{temporality} is
only an ideal abstraction.  Yet, space, in its \thi{frozen} simultaneity,
functions as the limit of temporality, as permanence.  But even though we tend
to refer the \co{vague} intuition of eternity to the permanence as some ideal
limit of atemporal \thi{objectivity}, it is only in the \co{actual} passage of
time that we \co{concretely experience} permanence.  The sense of permanence is
only another side of the sense of change and passage of time and it is naturally
acquired and recognised in the \co{experience} of getting old. \la{Mundus
  senescit}, \wo{The world has aged}, is not \co{any experience} of childhood.
%\subpa
\label{pa:oldpeople}
That the world, \co{this world} ages -- and aging, withers -- is only a
reflection of the fact that so does its \co{experience}.  This frequent, and
often only apparent, withering of the intensity of the \co{experience}, which
one tends to ascribe to old people, is but a modification of the level from
which old people \co{experience} their life.  Old age simply experiences time
(and the world) differently than the young one.  The approaching death, the
knowledge that most things one was to experience have already happened, make one
think of one's life as a whole, if not quite completed, than in any case as not
consisting merely of petty \co{actualities}.  The significant time span is no
longer a mere moment, a brief actuality but years, perhaps, decades.  The
details wither but they give place to a new quality of \co{experience}, which
with time becomes itself \co{an experience} and which younger people will simply
find irrelevant, if they find it at all.\ftnt{We are not concerned with the
  tasteless caricatures trying to the last moment pretend that they are
  experiencing the world with youthful intensity. \co{Attachment} and despair
  will be addressed in Book III.}
%Through the memories of all \co{experiences} 

Above all, all the changes of the surrounding world notwithstanding, one sees
that nothing has really changed, that something fundamental, even if
ingraspable, remained the same since the earliest days of one's childhood.  One
may recognise the same curiosity or indifference, the same feelings of wonder or
disenchantment, the same \co{vague} motivations and fascinations.  The
difference is only that one has already seen all that before, that one knows
that some things need time and thus single moments, enchanting and gratifying as
they may be, are no longer of the highest importance.  Shall we also mention the
quite so frequent turn towards various forms of religiosity?  It is not,
however, as one often wants to interpret it, a mere fear of facing the death, or
else a mere disillusionment and dissatisfaction with one's life (possible as
these are).  On the contrary, it is rather the sign of reaching the level of
\co{experience} at which one realises the limits of one's life and world,
perhaps, their insufficiency, if not vanity.
%It is great gift if one
%gets this feeling earlier \ldots
The \thi{objectified} permanence gives place to the sense of \co{eternity}, not
any infinity of time, but a lived \co{eternity}, the constant \co{presence} of
\co{invisibles}, perhaps even of the ultimate \co{nothingness}, which has been
strangely known all one's life.  It is neither necessary to get old to meet such
\co{experiences}, nor does getting old necessarily imply that one will meet
them.  But you have seen such a transition in old people more than once and more
often than in the young ones.
 
%%% signs... bottom-up consciousness
 \pa\label{pa:levsSigns}
The tension between the {objective} and {subjective} aspect 
enters the \hoa\ through various \co{signs}. Their character varies depending on
the level which they address, to which they point, from which their contents
arise. 

\levsTab%{9.5}
{The signs:}
{& \oss & \rss}
{sensations, instantaneous images
     & \co{that it is}: substances, \co{objects} }
{\co{moods}, \co{impressions}, vital feelings 
     & \co{that it is so-and-so}: \co{concepts} }
{\co{qualities of life}, \thi{soulful} feelings 
     & \co{that I am}: \co{general thoughts} }
{\co{commands}, \co{inspirations} 
     & \co{that I am not the master}: \co{symbols} }
%{\oss}{\rss}
%   
At the lowest level the \co{signs} coincide with the \co{immediate object} and
give rise to a merely reactive response relative to a particular organ, a point of
the body, a nervous nexus.\noo{This suits naturally and well all empiricist
reductionists, and we can leave it for them to limit their thinking exclusively
to this level.} The higher we move in the hierarchy, not only the response
becomes less reactive but also the more \co{clear} becomes the \co{distance}
separating the \co{sign} from the signified or, the \co{actual}
from the non-\co{actual} and, eventually, from the \co{non-actual}.
\co{Experience} of this \co{distance} is but a reflection of the form of
\co{transcendence} which is an aspect equally constitutive for each level as
temporality.
     
\pa \co{Transcendence} is the \co{presence} of \co{non-actuality}.
% and each level is characterised by a specific tension between the \co{actual}
% and \co{non-actual}, that is, by the specific form of transcendence. At
At each level below the \co{invisibles} it has two \co{aspects}: the
\co{vertical}, relating directly to the higher level, and \co{horizontal} which
is a reflection of the \co{vertical aspect} in terms and categories of the
given level.  For instance, the \co{more} at the level of \co{actuality} is but
a quantitative \thi{more} of \co{complexes} of which \co{reflection}
\thi{knows}, even if only implicitly or potentially.  But this \co{more} is only
a reflection 
of the horizon of \co{the world} and \co{my life}, which are as if projected
into the context which tries to understand everything as \co{complexes}.  The
\co{more}, rendered as the flux of \co{experience}, throws \co{reflection} back
onto itself and establishes meaning of \co{complexes} in relation to one's
  life.  Similarly, the \co{horizontal aspect} at the level of \co{mineness}
comprises that which is \co{not-mine}. This, however, is still a determination
in terms of \co{mineness} and it merely reflects the \co{vertical
  transcendence}, the \co{presence} of \co{invisibles}.\ftnt{The \co{vertical
    aspects} of \co{transcendence} could be described as what Paul Tillich
  called \wo{forms of meaning}. If one includes in this term \citef{\imm all
    particularities of individual meanings and \act of all separate connections
    of meaning and even \mine the universal connection of meaning, then in
    relation to the universal connection of meaning \inv the unconditioned
    meaning may be designated as the import of meaning.}{TilRel}{I:1.1.a [my
    numbering]} And the \citefib{import of meaning is the ground of reality
    presupposed in all forms of meaning, upon whose constant presence the
    ultimate meaningfulness, the significance, and the essentiality of every act
    of meaning rest.}{TilRel}{I:Introduction.c.ii}.}
%\subpa
Schematically:

% \hspace*{1em}
% \levels{The horizontal and vertical aspect of transcendence}

\hspace*{1em}\xymatrix@R=0.5cm@C=1.5cm{
\inv\ \mbox{\co{invisible}}:\ \ \ \  & \mbox{\co{origin}}  && \mbox{\co{the\ One}}\\
%
\ \ \mine\ \mbox{\co{mineness}}:\ \ \ \ \   & \mbox{\co{my\ life}}  \ar[rr] &&
\mbox{\co{not-mine}} \ar[ull]^<<<<<<<<<<<<<{\co{chaos}}|{\co{I\ am\ not\ the\
    master}} \\
\act\ \mbox{\co{actuality}}:\ \ & \mbox{\co{complex}}  \ar[rr] && 
     \mbox{\co{more}} \ar[ull]|{\co{experience},\ meaning,\ relevance}\\
%
\imm\ \mbox{\co{immediacy}}: & \mbox{\co{object}} \ar[rr] &&
     \mbox{\co{externality}} \ar[ull]^<<<<<<<<<<<{\co{an\ experience}\ \ \ \ }|{ \ context,\ purpose}
} \vspace*{1ex}

\noindent
The \co{horizontal aspect} has always a negative character: it is that which is
not~\ldots \herenow, \co{actual}, \co{mine}.  Using it as the criterion of progress
might easily lead into a Hegelian kind of dialectics.  But this negative,
\co{horizontal aspect} reflects only the \co{vertical aspect} which is known and
\co{experienced} not as a mere lack, limit and negation but positively. It
penetrates the quality of the \co{experiences} at the given level, placing them
on the \co{traces} which reach to the deeper, that is higher, sources.  It is
this higher level, the level less dominated by the \co{actuality}, which is the
source of the mode of \co{transcendence experienced} at the lower level.  The
\co{horizontal aspect} reflects the \co{vertical aspect} in the multiplicity
differentiated according to the categories of the lower level. It refers thus
always to the higher, \co{founding} element -- but only indirectly. Its negative
-- or quantitative -- character signifies, in terms of \co{reflective}
development, encountering a limit -- a limit beyond which the categories applied
so far seem to lose their meaning, beyond which there is nothing left except,
perhaps, a routine repetition stiffening the soul; a limit showing worthlessness
of the things and categories which so far have been experienced as the ultimate.
Encounter with the \co{vertical} aspect, on the other hand, is \co{an
  experience} of something fundamentally new, something which, at first, appears
only as a \co{vague} promise, but which with time discloses a deeper meaning, a
new way of seeing also earlier \co{experiences}.  This encounter is a true
\gre{anamnesis}, a \co{re-cognition} of something which has been known for a
long time, but only dimly and indistinctly, as a \co{vague} intuition, an
indefinable \co{rest}.

\noo{ \co{Horizontal} aspect of \co{transcendence} is a reflection of its
  \co{vertical aspect} in the multiplicity differentiated according to the
  categories of the lower level. It refers thus always to the \co{founding}
  element as, for instance, in Levinas': \wo{objectivity presupposing the
    other}, \wo{discourse presupposing the other}, point to the other, the
  transcendence \la{par excellence} for Levinas, as the element \co{founding} the
  more specific, lower level relations. Similar schema underlies the claim that
  \wo{social relations presupposing \co{justice}}, [p.76], etc.

  
  \ccom{ -- this isn't Plotinus' quote -- The two aspects of \co{transcendence}
    can be illustrated by the following remark from Plotinus: \citt{The soul
      [\ldots] can be turned about and led back to the world above and the
      supreme existent [\ldots] by a twofold discipline: by showing it the low
      value of the things it esteems at present, and by informing -- reminding
      [\gre{anamnesis}] -- it of its nature and worth.}{Plotinus [not really],
      p.103 -- The Psyche in Antiquity}
    
    \citet{And Intellection in us is twofold: since the Soul is intellective,
      and Intellection is the highest phase of life, we have Intellection both
      by the characteristic Act of our Soul and by the Act of the
      Intellectual-Principle upon us -- for this Intellectual-Principle is part
      of us no less than the Soul, and towards it we are ever
      rising.}{Plotinus}{I:1.13} }
} % end \noo{

\noo{ against Heidegger: \citet{Still there is a twofold flaw: the first lies in
    the motive of the Soul's descent [its audacity, its Tolma], and the second
    in the evil it does when actually here: the first is punished by what the
    soul has suffered by its descent: for the faults committed here, the lesser
    penalty is to enter into body after body- and soon to return- by judgment
    according to desert, the word judgment indicating a divine ordinance; but
    any outrageous form of ill-doing incurs a proportionately greater punishment
    administered under the surveillance of chastising
    daimons.}{Plotinus}{IV:8.5} }

\pa\label{pa:SchelerA} As we move higher in the hierarchy of levels, we leave
\co{actual} determinations.  We lose \co{objective} categories, the
\co{distinctions} become less rigid and \co{precise} and do not present us with
any definite \co{objects} -- they become \co{vague}.  But by the same token,
they become more intimate, they penetrate deeper into our being.  The intensity
of \co{signs} increases and so does the depth of satisfaction found in
\co{experiences} -- they become \co{clearer}.\ftnt{Big parts of this subsection,
  in particular, this and the following
  paragraphs,\noo{\refp{pa:SchelerA}-\refp{pa:SchelerZ},} are due to Max Scheler,
  in particular, \citetit{MaxForm}, \citetit{MaxAnthro}.  The main difference
  consists in that Scheler is concerned almost exclusively with the hierarchy of
  ethical values.  Yet, except for the aspect of temporality and transcendence,
  the characterisation of the differences between various levels, as well as the
  levels themselves, are very similar.  Another difference, appearing in
  \ref{sub:mit}, the entire \refp{pa:notSchelerA}, concerns the relation between
  different levels which here, unlike with Scheler, is not just that of
  \co{founding}.}  A momentaneous elation, a simple joy over some particular
event which passes in an instant, can be as natural and authentic as superficial.
It does not last, it does not reach the depth of the person, it is contextual and
localised.  A peace of soul, a joy of life, humility of sainthood are
\co{experienced} without any such localisation, independently even from the
context where they may happen to become \co{manifest}. Entering the \hoa, their
\co{signs} give a deep, spiritual satisfaction which is as \co{clear} as it is
undefinable.
% \citet{Fun I love but too much Fun
%   is of all things the most loathsome.  Mirth is better than Fun \& Happiness is
%   better than Mirth -- I feel that a Man may be happy in This World.}{BlakeLet}{}
%   Letter to Rev.  Dr.  Trusler, 23.08.1799 [used in III.non-attach]}

\co{Vagueness} means also that the \co{clarity} of the \co{signs} is merely
indicative, alluding, as if inviting, rather than forcefully demanding, imposing
its \co{objective} presence. It is calm and spreads calmness.  As Plotinus says
it: \citet{The Good is gentle and friendly and tender, and we have it present
  when we but will.}{Plotinus}{V:5.12\noo{33-34; The Good is in Plotinus the
    \co{absolute origin}, the \co{one}.}}  \co{Vagueness} of the \co{signs} and
\co{experience} of the higher levels is just another side of the lack of
definite \co{objective} correlates.  And this means that they are less relative,
they are less restricted to particular regions of Being, are able to embrace
more varied \co{experiences}.  Eventually, the \co{invisibles} are \co{absolute}
-- free from any limits of \co{objectivity}, not relative to any particular
region of Being.

\label{pa:constantVaried} The lack of \co{objective} 
determinations, in turn, means that the higher levels of \co{experience} are
more constant, remaining unchanged underneath the variations of lower levels.
A particular \co{mood} can allow a variety of sensations which do not change the
\co{mood}.  A \co{quality of life} will remain the same irrespectively of the
variation of particular \co{moods} and \co{impressions}.  Conformance to the
\co{absolute command}\noo{, sainthood or love,} will remain unaffected by any
variation in particular \co{moods}, \co{thoughts} and may \co{incarnate} into
very different \co{qualities of life}. And even a failure in
conformance leaves the validity of the \co{command} unaffected.


\pa The higher we move in the hierarchy of levels, the further are the 
\co{signs} removed from the signified.  At the level of \co{immediacy}
the two coincide, while \co{invisibles} never coincide with the
\co{signs} which \co{manifest} them, remaining essentially
\co{transcendent}, that is, being \co{experienced} {\em always and only} as
\co{non-actual} pure \co{commands}.\ftnt{A phenomenological analogy
could be to say that the \co{signs} at the higher levels have more 
intentional character, they are more \co{clearly} directed 
\thi{toward something} than are the \co{signs} at the lower levels. 
Yet, this \thi{something} becomes at the same time less and less 
identifiable and escapes completely phenomenological reductions.}
%

\label{nonreactive}
This is an aspect of non-reactivity of higher \co{experiences}.  Sensations are
pure reactions, the \co{sign}, coinciding with the presence of the signified, is
simply the elicited reaction.  \co{Moods} are already affected by the
\co{presence} of non-\co{actuality} which defers the possible reaction and
renders it partially indeterminate. Yet, although what an \co{impression}
announces can still hide behind it, the \co{impression} itself is \thi{given},
it is reactive.  The \co{qualities of life} are passive but are not reactions to
any specific situations. They announce something which can be only accepted,
though what the acceptance means in practice is far removed from the
\co{quality} itself. \co{Commands} and \co{inspirations} have no reactive
element at all, being completely independent from any particular beings and
regions of Being.

The non-reactive character of \co{signs} is proportional to (can be
\wo{measured} by) the extent to which they are influenced by our attention and
will.\noo{ The lower \co{signs} are least affected by it -- the presence of pain
  is not affected by paying attention to it or not.  In fact, the reactive
  character of sensation means that, in a sense, one cannot influence it by
  turning attention away, because reaction is exactly the amount of attention
  paid.  The presence of highest \co{signs}, on the other hand, demands
  respectful attention, attentive presence of mind. We should probably use
  another word, for instance, \co{openness}.
  
  \co{Openness} should not be confused with the possibility of control by the
  \co{acts} of will. In fact, they are inversely proportional.  }
\label{pa:SchelerZ}
The significance of the lowest \co{signs} can be to high extend determined by
the will. One can lessen the feeling of pain, virtually removing its relevance,
by an effort of will to overcome it, for instance, by turning away one's
attention and concentrating on something else. And, of course, one can easily
produce painful \co{experiences}, just like one can arrange circumstances so as
to produce pleasant effects for sight, touch or taste. To some extent, one can
also arrange the circumstances so that they will produce agreeable or repulsive
\co{impressions}; one can have some knowledge about kinds of circumstances which
result in particular \co{moods}.  But the higher we move in the hierarchy of
levels, the less power one has over the presence or absence of the respective
\co{signs}, not to mention their signified correlates. One can try to lead one's
life so as to give it a specific \co{quality}, but this \co{quality} is never
entirely under our control. One can crave happiness without ever achieving it,
while the regrettable \co{qualities} of one's life can be impossible for one to
change. (One can, of course, always do something, but the eventual effects of
one's \co{acts} and \co{actions} are not determined by one's intentions.) With
respect to the deepest aspects of being like holiness, despair, \co{love}, one's
will has nothing to say. They are \co{gifts} which one can neither refuse nor
provoke, one can neither cause their \co{presence} nor make them disappear. At
most, one can try to ignore them which is simply pretending that they are not
there -- apparently without any immediate consequences but, in the long run,
affecting one in the deepest way. On the other hand, their \co{manifestations},
which do not depend on one's will either, require a kind of attention, say,
\co{openness}, which is not any focusing of the will, but merely a humble
cooperation with the hidden, primordial causes. We should emphasize here the
difference between the \co{presence} of \co{invisibles}, and their
\co{manifestations} in \co{actual signs}. Their \co{presence}, the ontological
fact, is independent of our attention and cooperation. But the experienced form
of this \co{presence}, the character of their \co{manifestations}, is
conditioned by \co{openness} and spiritual directness.\ftnt{The spiritual and
  mystical writings abound in variations over the theme of inner concentration,
  focused attention, presence of mind in the face of spiritual life.
  ``Recueillement'' is a technical term of spirituality denoting the action or
  fact of concentrating one's thought on spiritual life in detachment from
  worldly preoccupations. In so far as it refers to \co{acts} or facts, it can
  be concerned only with \co{manifestations}. Much of \thi{spirituality},
  especially its degenerate and more hysteric versions, have never managed to
  reach beyond the realm of \co{actual signs}. Book III considers the
  \co{spiritual} dimension of \co{existence} in more detail.}




\subsection{As above, so below} %\co{co-presence}}
%\plan{The modes of manifestation}

\noo{
\noindent
{\small{
\begin{tabular}[h]{|llrl|} \hline
\HH\ is &  & in \LL & \\ \hline\hline
1. &  invisible &  -- & cannot be reduced to the \LL's categories \\
   &  diffuse &  -- & unclear, since greater scope of transcendence : \\
   &  unprovable  &    & has its own categories and \\
   & indefinable   &    &  does not operate at the \LL's detail/scale, and \\
   &            &   & hence is not ``needed'' to fix \LL's problems \\ \hline
\ref{lev:manif}. &  revealed  &  -- & incarnation, hierophany  \\
   &                    &  & in a single moment which then dominates life or\\
   &   expressible  & -- & over time, through work, commitment \\
   &  experienced & -- & if not as revelation, then as ``moods'', as meaningful/less \\ \hline
3. &  inexhaustible &  --& does not determine a unique form of appearance\\
   &                 &   & always allows more forms of expression \\
   &  unpredictable & -- & uncontrollable, can always surprise \\ \hline
4. &  reflected &  -- & disallows some forms of expression \\
   &             &    & if they occur, it is not the same \HH\ any more \\
   &   inviolable &  -- & within the narrower, the broader cannot
                                      be changed \\
   &               &    & it takes time (to change one's being) \\
   &  restricts  & -- & doing \LL$_1$ violating \HH$_1$ may lead to impossibility \\
   &         &    & of experiencing/expressing \LL$_2$ reflecting \HH$_2$ \\ \hline
\end{tabular} 
%


\qpa{Under 4. consider: The negative values at a higher level do {\em not} 
restrict the lower level(s) in any way. The positive ones {\em do}, 
but without \co{concrete founding}, not entirely and not really\ldots}
} }

}
%
We have divided \co{experience} into levels and talked as if our life was
composed of them, being but a \ldots \co{complex}, a \co{totality}? But every
life is an unrepeatable \co{unity} which is not constituted by various parts and
elements. All such parts and elements are but manifestations, but \co{actual}
expressions of deeper aspects and, eventually, of the \co{unity}. Every
\co{immanent} unity is \co{founded} in the \co{transcendence}, every
\co{actuality} draws its vital juices from the \co{non-actual} roots. The
\co{concrete unity} of \co{existence} will be discussed in Book III. Now, we
will review only the more formal aspects of this \co{unity}, namely, the 
co-presence of all levels and, in particular,  the \co{presence} of
\co{invisibles} in \co{actual experiences}. 

\subsub{Presence and co-presence\noo{Mitgegebenheit}}\label{sub:mit}% and manifestation
%%%% ??????
\noo{
\ad{\co{actuality} $\sim$ \co{simultaneity}}
\co{actuality} $\sim$ \co{simultaneity}, i. e., number of different 
\co{distinctions} which can be there simultaneously: this number 
decreases along the hypostases (\co{actuality} \wo{diminishes}): 
\begin{enumerate}
        \item \co{nothingness}=no sense
        \item \co{chaos}=all, 
        \item \co{experience},\hoa=some (?7-9?), 
        \item\co{reflection}=1
\end{enumerate}
}


\pa\label{rest} %[perhaps to \refp{acts}]
It is common to distinguish various aspects of an act like, for instance, the
intentional correlate, the pragmatic aspect, the ethical import. The
\co{immediate} correlate of an \co{act} is its \co{object}.  But being involved
into the context of some \co{action}, the \co{act} has always also some goal, it
has a pragmatic aspect. Whether the goal is immediate or remote does not change
its character as the \co{actual} objective intended by the \co{act}. We have
then distinguished between the \co{objective} goals and their \co{motivations}
which, encircling the horizon of possible \co{actions}, are themselves
counterparts of \co{activities}, expressing the lived and, preferably, also the
declared values, \refpp{pa:Zweck}. This level is much broader and comprises much
more than merely ethical issues but, being the level of \co{mineness}, and hence
of the relation to \co{not-mine}, it certainly embraces also the ethical
element. Finally, every \co{act} is surrounded by the \co{rest}, \co{expressing}
the \co{present} but \co{invisible aspects} which do not become thematically
\co{actualised}, \ref{sub:invisibleSigns}, in particular, \refp{pa:restA}.
Thus, every \co{immediacy} of an \co{act} expresses all the levels, carrying
their \co{unity} in its structure.
%Let us consider some more concrete examples.

\pa
%As said in \refpp{pa:constantVaried}, t
The higher levels of 
\co{experience} remain more constant and allow a large variation at 
the lower levels. The increased constancy (in the upward movement) is an
experiential counterpart of the co-presence of all levels. 

If I am in a good \co{mood}, I can accept a lot of small, insignificant
annoyances which do not bring me out of this \co{mood}. In fact, the \co{mood} I
am in will influence the way I handle these small situations, it will remain
\co{present} in all these \co{actual} situations.  Similarly, a person may 
remain generally dissatisfied with his life through all his particular
experiences; no positive event seems to be able to change this general
\co{quality} of his life.  And again, this \co{quality} makes itself felt and
efficient, \co{present}, in various ways in all concrete situations.  Perhaps,
by finding negative aspects in any, even most positive experiences, perhaps, by
awaiting always the inevitable end of such experiences, that is, awaiting always
for a bitter and unwelcome continuation.  It would be too strong to say that
this \co{quality} determines the character of all concrete experiences.  But it
casts its shadow over them, it moulds them in a specific way so that they seem
to conform to the general scheme of things which pollutes all
\co{actual} experiences.

But one should be careful with the criterion of constancy when applied to the
highest level.  For instance, the \co{quality of life} of St.~Francis, his
amiability and goodness seem to have accompanied him from the very childhood all
his life, while with respect to his sainthood, the dream on the way to the
Fourth Crusade marks a break and begins a new chapter. St.~Paul, before and
after the vision on the way to Damascus, was the same person and many
\co{qualities} (zeal, dedication, \fre{je ne sais quoi}) where present in his
life before as much as after the conversion.  The constancy of \co{invisibles}
is different from the possible constancy (and transience) of their
\co{manifestations}.  It is not relative to one's life but consists in
transcending the temporal dimension and the categories of \co{mineness}.  There
is nothing like \thi{{\em my} sainthood} or \thi{{\em your} sainthood}, and
sainthood remains sainthood whether it is \co{manifested} or not, or whether it
is \co{manifested} in one person or in another.  And yet, the \co{invisibles}
are \co{present} in \co{any experience} as its deepest aspects, the
\co{invisible} personal traits, the source which does not create the specific
details of \co{actual} situation but merely lends it an aura and puts a personal
signature underneath. \co{Manifestations} are only particularly intense and
\co{visible signs} of this \co{presence}. The constancy of \co{invisibles} is,
in fact, independent from any personal and \co{actual manifestations} -- it is
their \co{eternal} validity.

% The higher need not lower ?
\noo{The previous paragraph was concerned not so much with \co{an actual experience} 
as with \co{experiences}, like \thi{two weeks in Prague}, which span 
over longer periods of time. To the extent such a longer period of 
time becomes \co{an experience}, it will involve variations of  
more minute \co{experiences} which belong to the lower levels. 
But we can also see the co-presence within \co{an actual experience}. 
}

\pa\label{pa:notSchelerA}\label{notfounding}
 Now, higher levels do not create the specific contents
of the lower ones nor vice versa.  Each level, determined by its specific
tension between \co{actuality} and \co{non-actuality}, has its own
characteristic contents and ways of their presentation.  The \co{invisibles} do
not determine one's life.  The \co{quality} of one's life does not determine the
\co{actual} situations one gets involved in.  The \co{actual} \co{moods} and
\co{impressions} do not determine any particular sensations.  In fact, the full
range of lower phenomena can be \co{experienced} along with any configuration of
the higher aspects.  In short, the higher aspects of \co{experience} do not
\co{found} the contents of the lower ones, in the technical sense of phenomenological
founding (i.e., as necessary conditions).

% founding? - relatively yes; higher changes character of lower
But the higher aspects influence crucially the lower ones, they sink in and
penetrate whatever  
qualities may emerge at the lower levels. An annoyance is an annoyance 
but it changes its character when encountered in a good or in a bad 
\co{mood}. A joy or sadness of a pessimist is different from the 
respective feelings of an optimist. The drive and energy of life of a 
saint are different from the similar \co{qualities} in the life of 
a person nourished by negation, hatred or bitterness. 
%
A joyful feeling of a person who is generally dissatisfied with life will still
be joyful.  Yet, this joy will be limited to the level of \co{actuality}.  It
will be, so to speak, \thi{blocked} if it \thi{tries} to penetrate deeper into
the personal being; \thi{blocked} by a remainder of its transiency, by painful
memories, or simply \thi{blocked} by the general dissatisfaction with life.  In
other words, it won't be able to spread over the totality of the personal being,
but will remain localised.  You might have heard the difference between a short,
nervous, almost involuntary laughter which seems to be disturbed by the
immediate bad conscience, as if there was no {\em real} reason to laugh and one
did it only because one could not resist it, and, on the other hand, a cordial,
warm, full-blooded laughter which seems to flow from the very bottom of the
heart, in which the laughing face is but an expression of the soul which
embraces the whole world with its hearty laughter.  In the former case, the
\co{actual} level is not in conformance with the higher level of one's being and
the inability for a hearty laughter modifies the \co{actual} one.  It is still
laughter, over the same funny thing, but it testifies to another personal
involvement than the latter.

This penetration of lower levels by the higher ones in \co{an
  actual experience} reminds a bit of the \thi{founding} relation.  But it is
not founding because the \co{actuality} of a given lower aspect, of laughter or
joy, is not conditioned by the \co{presence} of any particular higher aspect --
the former is only {\em modified} by the latter.  This modification, this
\co{rest} and aura which the higher aspects extend to the lower, witness to the
unity of \co{an experience} involving the aspects of all levels.  This is the
form of \co{presence} of higher, eventually \co{invisible}, aspects in all
\co{experience}.
%evil look vs. ? good look, other examples... ?

\newpa 

\pa An \co{immediate experience}, an \co{act} of \co{reflection} is obviously
involved in the context of the \co{actual} situation.  As phenomenologists show,
especially with respect to perceptions, the \co{actual} contents are surrounded
by other, as they say, \ger{mitgegeben} (co-present) aspects which do not fall within the
focus of consciousness but which, nevertheless, are \co{present}.  Focusing
my sight on the entry door of a house, I still see, albeit only subconsciously,
only in the corner of my eye, the windows immediately to the left and right of
the door.  Furthermore, although I see only the front of the house, its sides
are also included, \ger{mitgegeben}, in the \co{actual} phenomenon.  The
question now is where to stop such inclusions.  I know that behind the house
there is a park.  It does not seem to be given, but is it \ger{mitgegeben} too,
or not?  And in the park, there is a lake, behind which there is\ldots It seems
implausible to assume that all this is \ger{mitgegeben}, for then all things
ever experienced, unlimited if not infinite number of them, would belong to
every phenomenon. There is the \hoa, which seems to circumscribe the scope of
\ger{Mitgegebenheit}. In I:\ref{sub:time}, \refpf{pa:twoTimesA}, we had the
problem with the phenomenology of time, which did not account for the continuity
across the limit of \co{actuality} towards the remote past. Likewise, here we
encounter similar phenomenological break in which what is \ger{mitgegeben}
dissolves in the emptiness surrounding the \hoa. And as in the case of time, so
also here, there is a \co{distinction} but there is no border because we have to
do with a continuity of \co{experience}.

\ger{Mitgegebenheit} of the \co{objective} (or, if one prefers, noematic) character
has its limit which is the limit of the \hoa. Somewhere, at the end of the front
wall, behind the house, behind the lake, behind the park, \co{objects} and
\co{complexes} cease to be \ger{mitgegeben}, there are no more \co{objects} and
\co{complexes} which gradually disappear behind the limits of the \co{actual}
phenomenon. Of course, it is enough to redirect one's attention to bring in 
other, new or expected \co{objects} and connect them to the ones \co{actual} at
the moment, but we are now considering an abstract, isolated \co{actuality} of
\co{an experience}, so let us not stroll, not move sight around.
What is \ger{mitgegeben} behind this line are no more \co{objects} but~\ldots
\co{moods}, \co{impressions}, feelings, intuitions, \co{qualities}.
\ger{Mitgegebenheit} becomes eventually \co{presence} of \co{invisibles}.
\co{Moods}, feelings, \co{qualities}, etc. are the concrete forms under which
the potential infinity (of things, of \co{experiences}, of \ger{Lebenswelt}) is
\co{present} in every experience.  If we were to use the objectivistic 
way of speaking, we might say: they do not bring in any \co{objects} but only
unified \co{signs} which comprise the overwhelming number of possible
\co{distinctions} within the limits of the \hoa; although they do not make any
more \co{objects} \co{actual},
%\ftnt{Some psychologists claim that human perception is
%limited to at most 7-9 objects at a time.}, 
they make them \co{present}, by providing \co{actual signs} which are nothing
else than a comprised multiplicity of \co{objects}.  But the objectivistic way
of speaking loses its adequacy as behind the line of \co{moods} and
\co{qualities}, behind the line where even the \co{actual} feelings become
blurred and indistinct, there are still \co{invisibles}, the \co{inspirations}
which oversee the whole \co{actual} situation.
We can illustrate it on our figure from~\refp{fig:levels}, p.\pageref{fig:levels}.
The non-\co{actual} points on the circle (above the line of \co{actuality}) are
all reflections of the \thi{objective} points from the line. They belong to the
\co{actual experience} but only in the form of the \co{signs} on the circle. And
the closer to the sphere of \co{invisibles} and the pole of the circle, the
denser condensation of these images, with the pole reflecting the infinity of
the line. This density of the images, the density of the increasing multitude of
remote points of the line comprised into decreasing segments of the circle,
would correspond now not only to the inseparability of \co{invisibles} but also
to the unobjectifiable content of the \co{signs} of the higher levels \co{present}
in the \co{actual experience}. 


\pa The \co{concrete, invisible presence} seems to underlie the intensity with which
Merleau-Ponty searches for it in the structure of \co{actual
  experiences}.  Although we might disagree with the focus on
sensibility and the hunt for the \thi{pure, raw, pre-reflective experience}, there
is, nevertheless, much we can borrow from his descriptions.  \citet{Visible
  actuality is not {\em in} time and space nor, of course, {\em beyond} them,
  for there is nothing in front of it, after it nor around it, which might
  compete with its visibility.  And yet it is not alone, it is not everything.
  Precisely: it occludes a further view, that is, both time and space extend
  around it and are {\em behind} it, hiding, in depth.  In this way visible can
  fill or embrace me only because I who see it, do not see it from the depth of
  nothingness, but from its own interior, and seeing it I am also seen.  Weight,
  density, content of every colour, every sound, every tangible tissue, of
  actuality and of the very world come from that that the one who receives them
  feels as if he emerged from them through a kind of spiral or splitting
movement, being originary homogeneous %\he{jednorodny} !not homogenous
with them, from that he is a self-directed sensibility which, in turn, is in his
eyes as if doubling and unfolding of his bodily tissue.  Space and time of
things are splinters %\he{odpryski}
of his own, of his spatiality and temporality, and not any multiplicity of units
separated synchronically and diachronically; it is a relief combined from that
which is simultaneous and that which follows after itself, spatial and
temporal mash %\he{miazga}
in which, on the way of differentiation, units emerge.  Things are now not in
themselves, on their place and time, here, there, now, sometime; they exist only
on the border of this spatial and temporal radiation emanating mysteriously from
my sensibility.  Their content is not the content of a pure object observed by
the mind from a distance; I experience it from inside, as I am among the things,
and they communicate through me as a feeling thing.  Actuality, visibility -- as
the veil of memories for the psychoanalysts -- has for me an absolute valor only
because of its hidden, unlimited content of past, future, and that which is
beyond it and which it announces but also hides.}{VisInvis}{Inquiry and
intuition;p.120-121} 

\pa
%\co{Actuality} = pure immanence; \co{presence} = imman. + transc. \ldots
Just like we have equated \co{transcendence} with the non-\co{actual},
\co{non-actual} and eventually the \co{invisible}, we can equate \co{immanence}
with pure \co{actuality}. Both terms are only abstractions which can hardly be
dissociated from each other.  \co{Immanence}, \co{actuality} and
\co{visibility}, arises at the limit of the process of differentiation, as the
final stage of encircling the \thi{hidden, unlimited content} within the \hoa.
Although it allows \co{reflection} to oppose it to the \co{transcendence}, the
latter does not disappear in the process of \co{actualisation}, it penetrates
the \co{actuality} with all the levels lying above it.  \co{Presence} is an
expression of this insoluble involvement of \co{immanence} into
\co{transcendence}, of the fact that \citeti{immortals are mortal, mortals are
  immortal: each lives the death of the other, and dies their
  life.}{Heraclitus}{DK 22B62}

The eventual \co{transcendence}, the \co{origin} is never accessible to the
categories of \co{immanence}, is \thi{neither this nor that}, is never
\co{visible} and yet, is always most deeply \co{present}. \co{Self} is, after
all, its \co{trace}, the \co{trace} of \co{birth}, and the eventual terminus of
this \co{trace} is the \co{actual subject}. All levels are \co{present} in every
\co{actual experience}, and the \co{invisibility} of the \co{origin} is what
remains the same also across all temporal \co{experiences}.  Using the
categories of \co{reflective} oppositions, the \co{actual} is the opposite, and
hence incommensurable, with the \co{non-actual}.  But, in fact, the most
\co{transcendent}, the ultimately \co{invisible} is that which penetrates all
\co{visibility}, which is most intimately, even if not \co{visibly},
\co{present}, and which therefore is the most \co{immanent}.  The \co{rest} of
the \co{origin} remains \co{present} throughout the life, in every \co{actual}
situation, but only as the \co{rest} -- it cannot be grasped, but it gives taste
to everything within the \hoa.

\pa
The father said: {\wo{Place this salt in water, and come to me 
tomorrow morning.} 
%
The son did as he was told. 
%
Next morning the father 
said, 
\wo{Bring me the salt which you put in the water.}
%
The son looked 
for it, but could not find it; for the salt, of course, had 
dissolved. 
%
The father said, 
\wo{Taste some water from the surface of the 
vessel. How is it?}
-- 
\wo{Salty.}

\wo{Taste some from the middle. How is it?}

\wo{Salty.}

\wo{Taste some from the bottom. How is it?}

\wo{Salty.}
}\ftnt{Chandogya Upanishad \citaft{Huxley}{ p.4}}

%\noindent


\subsub{The ontological and the epistemic}

\pa Another word for the \co{unity} of \co{experience}, the \co{unity} which
either reaches all the way to the \co{origin} or which, stopping at some
relative level as its mere \co{totality}, seems completely absent -- another
word for this \co{unity} is \wo{\co{concreteness}}. It has little to do with the
\co{precision} and determinacy of the \co{objects}, although such associations
will often be involved in the common usage of this word. A pure, isolated
sensation, a \co{precise object} or concept, is not \co{concrete}, in fact, it
is very abstract. It lacks all the layers of \co{non-actuality} which would
anchor it in \co{concrete} \co{experience}, that is, in \co{experience} of a
person. It is exactly the \co{experiences} where things get {objectified} and
\co{externalised}, where they lose connection and relevance for you,
\co{experiences} of, say, impersonal nature, which are least \co{concrete}.
Typically, they will involve very \co{precisely} discriminated \co{objects} but
the price for this \co{precision} is the loss of \co{concrete clarity}.
\co{Clarity}, not \co{precision}, is the characteristic feature of
\co{concreteness}. And it is grounded in the personal character of
\co{experience}, that is, in the character which \co{experience} gains from the
\co{presence} of the \co{non-actual} aspects, eventually, from the \co{presence}
of the \co{invisible}, most personal \co{aspects}.

We, or generally, living beings, are those who bring \co{non-actuality} into the
world -- \co{non-actuality}, that is, subjectivity.\ftnt{\wo{Subjectivity} must
  be taken here in the most broad and generous sense of this term. It is, in
  fact, opposite to our \co{subjectivity} which is pure \co{immediacy}
  \co{dissociated} from the equally \co{immediate objectivity}. Here it is
  simply that which is opposed to the \co{actuality} of the \thi{givens}, the
  overhead brought into the situation by the participants and not any
  \co{objective} facts. Cf.~\ref{sub:objsubj}.}  One \co{actual experience} may
differ significantly from another one by its \co{objective} content, its
particular meaning, etc.  But every such \co{experience}, every meeting with a
particular, \co{actual} being, is at the same time a thoroughly \co{concrete}
\co{confrontation} with Being in its full and undisguised \co{presence}.  This
Being can seem veiled by the \co{actuality}, since it never appears, is never
\co{visible} for \co{reflection}, but, as a matter of fact, this
\co{non-actuality}, this \thi{veil} of \co{invisibility} is its true and only
form of \co{presence}, if you like, the only mode of our access to it. \citet{If
  coincidence is lost, then it is not an accident; if Being remains hidden, then
  this is precisely its very feature and no unveiling will allow us understand
  it.}{VisInvis}{p.128}

\pa
The \co{presence} of all higher levels in every \co{actual experience} reflects
the order of ontological \co{founding}. It can be seen as the inverse of the
epistemic founding which begins with the \co{actual dissociations} of
\co{reflection} and strives after the known, because \co{experienced}, but
apparently abstract and unthinkable, \co{unity}.

Even if \co{presence} coincides in objective time with the \co{actuality} of
\co{an experience}, the two are very different. Primarily, \co{present} are
\co{invisibles}, while only \co{objects} and \co{visible} contents can be
\co{actual}. The difference concerns both the mode of presentation and the
contents. One might be tempted to say: \co{actuality} is the {objectified
  expression} of \co{presence} and \co{presence} is the \co{non-objectifiable}
\co{actuality}.  Or else, \co{presence} is the witness whose testimony is not
reducible to the epistemological, that is, \co{reflective} categories of
\co{actuality} and \co{objectivity}.
%not really: presence = actuality of invisible; 
%  while actuality = presence of visible

Referring \co{presence} of \co{invisibles} to the ontological \co{foundation},
we should keep in mind that \co{actuality} is only the ultimate narrowing of
reality which extends far beyond the \hoa.  The \co{virtual invisibles},
necessarily involved into the \co{actuality} as they are, are thoroughly real.
It is not so that only the \co{actual signs} and expressions lend them any
reality. In fact, the \co{actuality} of \co{objects} is only the \co{sign} of
the \co{founding} reality and as such (as well as by its correlation with
\co{reflection}) can be considered to have more epistemic, rather than
ontological, character.  \co{Presence} is \co{concrete} reality and
\co{actuality} only its abstract, \co{precise} modification.  \co{Virtuality} is
already a fully consummated reality of \co{invisible} -- not merely its latency,
not a mere potentiality for \co{actualisation}.  It is effective and experienced
even if it is \co{present} only in its deepest, most \co{virtual} form, even if
it never reaches the level of \co{actualisation} in a \co{manifesting}
\co{sign}. One can live a deeply joyful and satisfying life without any direct,
plain thoughts of it, without even any frequent \co{actual experiences} of joy
and satisfaction. Yet, even then one will know that one's life is satisfying,
although one may be unable to explain \co{precisely} why and how. Not all
reality must become \co{actual}; most important 
\co{aspects} of it never do.
  
\noo{
Yet, these qualities may be \co{present} in most situations -- not as
\co{objective} parts of the situations, not as their thematic contents, but as
their dim qualities, as an indefinable \co{rest} of personal presence.}

\noo{Although the \co{present} \co{invisibles} need not become \co{manifested}, there
is no \co{manifestation} without \co{presence}.}

%A \co{revelation} presupposes \co{incarnation}, that is,
%\co{presence}.
%
%Incarnation should be reserved for spiritual ... ?


\pa Critique of \co{actuality} has always proceeded from the standpoint of
\co{non-actuality}. This can be taken in the direction of {subjectivity} (as
endowing the \co{actual} givens with meaning if not also with being), in the
direction of \co{absoluteness} (as transcending the relativity of \herenow), in
the direction of \co{vague imprecision} (as transcending the \co{precision} of
\co{immediacy}).  We thus see how a series of examples conform to the same,
albeit very rough and general, pattern.  Kierkegaard's critique of Hegel from
the point of \thi{extreme subjectivity} and existential relevance, Bergson's
critique of space and objectified time of science from the point of lived
duration, Heidegger's critique of the epistemically oriented \thi{metaphysics of
  beings} from the point of \thi{Being} and fundamental ontology, Derrida's
critique of\noo{ -- or shall we say, crusade against --} the \thi{metaphysics of
  actuality} from the point of inter-permeability of opposites and
undefinability of concepts, the general dissatisfaction with the conceptual
analyses from the point of concrete existence, even mystics' reservations
against philosophical theology from the point of \thi{ineffable unity}, are all
essentially the same critique of epistemic focus on \co{actuality} from the
point of ontological foundation which escapes \thi{objectivity} and merges with
the personal origins. The list could be prolonged indefinitely, so let us only
add the antique, Platonic and neo-Platonic, and later Christian opposition,
between the world of senses and the world of spirit. The recurring praise of the
renunciation of the former for the latter is based on the conflation of
\thi{senses} with the \co{immediate actuality}, with the founded and merely
transitory (which is at once taken as \thi{unreal}) and, on the other hand, of
\thi{spirit} with the supra-personal, yet always somehow personal,
\co{non-actuality} of the \co{founding} (and therefore \thi{true and real})
\co{origin}.\noo{As always, searching for concreteness one can easily yield to
  the temptation to identify \co{vague} and deep elements of \co{another world}
  with the plain \co{visibles} of \co{this world}. The crusade against senses
  carries strong elements of misunderstood spirituality.}

%%%% Here some stronger transition to UNITY = visible vs. invisvisible
%%%% should be made

\noo{ LONG \pa The \co{presence}, the \co{presence} of \co{invisibles} announces
  something which not only is not \co{objective} but which is not
  {objectifiable}.  It can never be reduced to the categories of \co{actuality}
  and \co{visibility}.  It seems dim, diffuse, it is \co{vague} and
  \co{imprecise}.  It can not be pointed to, proved, for these are categories of
  \co{visibility} which do not apply.  The undefinability of this \co{presence}
  reflects also the fact that \co{invisibles} do not influence our life in
  \co{precise}, definite ways.  \citet{Grace does not perform any works; it is
    too subtle for that and is as far from performing any works as heaven is
    from earth.}{EckGermSerm}{ DW 38, W 29 \noo{p.118}} It takes a truly German
  turn of the phrase to deny, for this reason, its reality: since it is not
  \co{visibly} active, since it is not \ger{wirksam}, it is not \ger{wirklich}.
  
  \pa This \co{presence} is experienced and also \co{expressible} -- the life is
  an \co{expression} of this \co{presence}.  A most peaceful, ordinary life,
  occupied almost exclusively with prosaic, daily matters, is an \co{expression}
  of the thorough \co{presence} of \co{invisibles}, which penetrate the
  \co{actuality} of apparent trivialities.
%
  And, at the same time, it is \co{present} as something inexhaustible, as a
  constant overflow which never reaches its final expression.  Drawing eventual
  consequences amounts, often, just to meeting the bitter end, but the
  \co{presence} does not signify any end, does not imply any consequences, does
  not mean any conclusion.  It is a constant opening which, however, is not
  arbitrary and indeterminate even though it is indefinable.  It is an opening
  of a field where every particular \co{expression} is just an instance, a
  token, an indication which is neither determined nor exhausts the potential.
  
  The \co{invisible} can announce its \co{presence} in the most unpredictable
  and unexpected way, in an \co{act} or \co{action} which surprises \co{me}, as
  \co{I} would never expect \co{myself} to be capable of such a deed. Relying on
  \co{one's Self} is just to allow \co{oneself} to be thus surprised and, at the
  same time, have full trust that such surprises are positive and valuable
  \co{expressions} of something \co{I} cannot fully see.  Whether one lets it
  flow through or denies it, whether one rests assured of its power or else
  seeks desperately after its \co{visible} signs, and eventually just after
  \co{visible signs}, is another matter -- the matter which in no way can change
  the fact of this \co{presence}.

  
  presence vs. actuality
  
  ontolo. vs. epist.
  
  unity vs. ideal
  
  concrete vs. precise

  
  \subsub{The ontological founding}\label{sub:copresence}
%%%% co-presence of all levels in an experience
%%%% this was 2.5.2 in ch. II

%%% add also reference to horizontal transc. = reflection of vertical
  \pa In the most abstract sense, the co-presence of all levels may be seen as
  the fact that gradual narrowing of the focus of attention does not
  \thi{remove} the surrounding context.  I direct my \co{reflection} towards the
  \co{actual object} but this \co{object} is obviously pulled out of the context
  of other, simultaneous \co{objects}.  As psychologists maintain, we can keep
  simultaneously some 6-9 \co{objects} in our attention without blurring the
  \co{distinctions} between them.  (Such simultaneity might also be taken as the
  limit of the \hoa.)  All these \co{objects}, however, are not given here
  \thi{out of the void}.  They, too, are surrounded by something, although this
  something does not have the same character of \co{precise objectivity}.  It is
  a constellation of \co{moods}, feelings, of subconscious perceptions and
  things sensed though not focused on, in the background.  In a more abstract
  formulation, the \co{actual objects} are surrounded by the \co{chaos} of a
  multitude of \co{distinctions} which, however, are not \thi{given}.
  Eventually, even this \co{chaos} is underlied, or surrounded, by the
  all-embracing \co{nothingness}, the ultimate background from which all
  \co{distinctions} emerge or, as may be perhaps claimed about the \co{actual
    experience}, into which all dissolve and disappear.

%\tit{\co{Presence} = ontological founding}
  
  \pa All more \co{precise} and \co{actual} aspects (though not necessarily all
  contents) of \co{an experience} are \co{founded} in the \co{non-actual},
  eventually \co{invisible} aspects of our being. This relation of ontological
  \co{founding} is experienced as \co{presence}.
  
  The \co{presence} of \co{invisibles} is real -- effective -- irrespectively of
  our \co{attentive reflection}, irrespectively of \co{visible signs}.
  \co{Reflection} and attention paid to it can only modify the character of this
  \co{presence}, but not annihilate it.
  
  One may never experience moments of \co{revelation}; one may almost never have
  feelings of meaningfulness or meaninglessness, feelings of the \co{quality of
    one's life}, and even when such feelings appear vaguely at the horizon, one
  may easily ignore them. One may never feel the need to talk about such things
  and, indeed, the stronger the need to analyse and dissect them, the stronger
  the indication that something goes wrong. One may ignore completely the
  \thi{dreams}, \thi{ideals}
% like those from \refp{pa:manifest1}, 
  and yet, the more intense attempts to ignore them, are but manifestations of
  the more intensely felt \co{presence}.
%the more violently will they emerge 
%from the shadow once they reach the surface.

The same thing, the same \co{act} performed by two different persons 
may be entirely different. A simple ``Hello, how are you today?'' may 
have infinite variations. One will easily distinguish between the 
\thi{what} of an \co{act} or a particular situation, and the \thi{how} 
of its \thi{aura}, its \thi{character}, its particular \thi{form}. And as 
much as one can say about the former, as little one can grasp of and put 
the finger on the latter. The \co{rest} which escapes \co{objective} 
determination may seem negligible -- \co{objectively} it {\em is} 
negligible. Even if it is not ignored, it may witness to a mere 
\co{mood}, mere \co{complex} of the context. But eventually, this \co{rest} 
carries with it the depth of \co{invisible presence}, the source of the 
personal being. Our \co{reflection} is defenseless against this 
\co{presence} which always escapes the scrutiny of \co{attentive} look. 

%\subpa

\subsub{Signs of presence}
%{The \co{presence} of \co{invisible} -- manifestations}
%

\pa
The co-presence of all the levels is, in the sense, what constitutes 
the \co{unity} of our being. This \co{unity} is eventually \co{founded} only in the 
ultimate \co{transcendence}, in the \co{invisible origin}. The 
\co{invisibles} originating from there permeate the whole 
\co{experience}. \co{Recognising} the \co{signs} at different levels 
as the unity \co{founded} deeper than the \co{actual sign} is like 
following the vertical dimension of \co{transcendence} which, eventually, 
leads to the \co{invisible origin}.
We are now asking about manifestations not of 
\co{invisibles} in general, but of the ultimate, of the \co{origin}, 
perhaps even of the \co{origin} as such.

%and our existential attitude is determined by the \ldots 
%place we take between the two \ldots It gives unity \ldots Man is a 
%borderline \ldots 

These \co{manifestations} will be particular modifications of the \co{signs} at
each level, not general \co{moods} or \co{qualities}, but some special among
them. Sometimes it may be only the fact {\em that} such \co{signs} at all
appear, {\em that} I can get a hunch of the unity revealed by a \co{mood}, of
\thi{what the life is in general}, etc.
%\ldots It is the specific 
%questions which may be asked (at each level) and which express the 
%vertical transcendence
\noo{
\levels{Levels of Manifestation -- of the \co{One}!}\label{lev:manif}
\are{\refp{pa:manifest4}: single moment (visible) : 
hierophany/\co{command} (it is here and now) or \\
     \hspace*{11em}a moment dominating the rest of the life; !\co{symbol}}
    {\refp{pa:manifest3}: limited time (experienced) : moods, feeling of 
    meaningful/lessness; !dreams}
    {\refp{pa:manifest2}: whole life (expressed) : what/how we live; work, 
    commitments, dedication; !ideals}
    {\refp{pa:manifest1}: (incarnate) : who I am; faith as participation}
}

\pa\label{pa:manifest4} \imm
In section \refspf{sub:invisibleSigns} we saw some \co{signs} through 
which \co{invisibles} may enter the \co{horizon of actual experience}. 
These were \co{commands} and \co{inspirations} which \co{revealed} 
the \co{invisibles} in the \thi{moments of truth}, or else 
were \co{present} in the \co{rest} of any \co{acts}. These were the 
\co{signs}, that is, events consummated within the \hoa, the extreme 
forms of \co{manifestation} of \co{invisibles} which I call 
\wo{\co{revelations}}. These are rare events and the \co{presence} of 
\co{invisibles} is not limited to such \co{experiences}.


\pa\label{pa:manifest3} \act  
The \co{signs} of \co{actuality}, \co{moods}, \co{impressions} or 
even \co{concepts}, reveal unity beyond the \co{actual} 
determinations. Their very presence points, \co{vaguely} and 
indefinitely but still beyond the \hoa. 

The silly questions like \wo{How does all this hang together?}, 
\wo{What keeps all the pieces in one universe?}, \wo{Is the universe 
really one?} ask for a principle or a unity beyond all the bits and 
pieces. 
%In a bit less limited temporal context, one
%can be overcome by the particular moods and feelings which seem to
%reveal the deepest \co{qualities of one's life}; the feelings of
%meaninglessness or fullness, of constant failure or strength, of
%general dissatisfaction or joy.  
%One tries sometimes to express such
%feelings, saying \wo{My life is \ldots}, \wo{The world is \ldots},
%either to oneself or to an accidental person met in the evening during
%a few days visit on the other end of the earth.  
No matter what the (im)possible answer might be, it is, of
course, inadequate, inaccurate, incomplete.  But it is also
immediately recognised as such -- the distance separating it
from what it tries to capture is given, too.  The impassability of
this distance is just an indication of the essentially \co{non-actual}
character of what the \co{vague} intuition behind such questions 
indicates.

There are also, some particular \co{moods} and \co{experiences} which
seem to address something much deeper than the \co{actual} moment in
which they appear.\ftnt{Moods in Heidegger's \wo{Sein und Zeit}
might be read as having this role.  \la{Nausea} of Sartre, the empty
irrelevance of the world of Camus' stranger, the whole range of
existentialist literature has been occupied with particular feelings,
all expressing a particular way of experiencing 
and \citef{futile searching for the
  meaning of life at the bottom of empty soul.}{HerbertAnts}{Narcissus}  The popularity of this school
originated certainly in the fact that it managed somehow to link
philosophy with life.  But what kind of life!}
%
The variety of religious experiences, described in such a detail by 
James, find place within the \hoa, and it is their improbable 
intensity and irresistible power which bears witness to their 
\co{origin} far beyond the \co{actual} situation. 
Underneath the insufficiency of the
intentional, thematic content of such \co{experiences}, there lurks an
indefinable feeling of an ineffable \co{rest}.  The indeterminate
\co{presence}, the underlying \co{quality} is interwoven into the
temporality of \co{moods} and \co{impressions}.

\pa\label{pa:manifest2} \mine 
In a yet broader context, the \co{qualities of life},
ineffable as they are, find at least twofold \co{expression}.
On the one hand, one may try to say something like \wo{The life is 
\ldots}, \wo{The world is \ldots}. The eternal questions, the 
impossible answers. Like the expressions of \co{actual 
moods}, they are inevitably incomplete and inadequate, but this 
inadequacy is given and known, too, thus witnessing to the intuition of 
the inaccessibility of whatever they may try to \co{express} and 
pointing again beyond, but now beyond the horizon of \co{my life}.

More genuinely, the \co{qualities} find their \co{expression} in the
way we conduct our life, in the projects and goals to which we
dedicate ourselves, or else, in the existential situation in which we find 
ourselves.  A person's life
is primarily an \co{expression}, an \co{expression} of something which
one may, sometimes, try to capture in general and more or less vague
descriptions, in a biography or a novel, in an eulogy of another's
life, in an epitaph at a friend's funeral. 
(Of all the literary genres, I suppose, the need for biographies or 
autobiographies will never abate.) 
One can start to list the
achievements and fascinations, the projects and activities which
constituted the main stations of one's life, but any such listing may,
at best, indicate vaguely the intended \co{quality}, leaving always
the \co{rest} to the understanding of every person.

A word, a sentence, a book is never enough -- a simple, \co{actual}
\co{sign}, or a collection thereof, is never sufficient to communicate
the way in which this particular life \co{expressed} the subtle
complexity of \co{incarnate} \co{invisibles}, its unity.  On the other
hand, if we know the person, then a short remark, a simple action, a
vague gesture can suffice to indicate the relevant \co{quality}.  Such
a \co{sign}, however, is readable only to those who already know what
it means, that is, who know the person and his life.

\subpa After a long and active life, having (or having not) achieved
the status and independence, looking back at the failures and
successes and concluding that, eventually, everything has turned out
to the best (or the worst), one may still find the disquiet and
dissatisfaction which one recognises as one of the first feelings
which motivated one's incessant search and activity throughout one's
whole life.  There are no reasons for that, there are no \co{visible}
reasons to feel dissatisfied.  And yet \ldots the dissatisfaction is
more real than it ever was before.  What is missing?  \co{Nothing},
that is, everything.  The whole life seems to have been dedicated to
inessential activities, to achievements of irrelevant goals.  Whether
this feeling downs on one or not, what it signifies might have been
\co{present} all the time, the whole life has been but an
\co{expression} doubled by this \co{vague} intuition.  It has,
perhaps, never been \co{actually} focused upon, never \co{actually}
felt, and yet it has always been \co{present}.

} % end \noo{ LONG


\subsub{Traces}
Distinguishing thus ontological \co{presence} from its \co{manifestations}
through the \co{actual signs}, we want to emphasise its independence from such
\co{manifestations}. But these two\noo{ -- ontological \co{presence} and its
\co{manifestations} --} express only the extreme poles of the \co{unity} which is 
maintained throughout all the levels. 
% 
% There is yet another way in which \co{non-actuality} is \co{present} in the
% \co{actual experiences}.
%
\co{Actualisation} amounts to \co{dissociation} of higher \co{nexuses} into more
\co{actual} elements.  This \co{dissociation} proceeds only gradually and every
\co{actual distinction} carries the character of the \co{nexus} from which it
arose, as the \co{actual sign} of a continuous \co{trace}. Such \co{traces}
connect not only the adjacent levels but traverse their whole hierarchy.\noo{
  reaching from the \co{immediacy} all the way to the ultimate \co{origin}.} We
have, for instance, seen in Book I how the \co{immediate objectivity} arises as
a \co{trace} of \co{actuality}, \co{signification} and, eventually, the event of
\co{confrontation}, how the objective time and space arise from the \nexus\ of
\co{spatio-temporality} which leads back to (that is, emerges from) the
\co{chaotic} simultaneity of \co{distinctions} following the \co{original
  separation}, etc. The characteristic feature of a \co{trace} is that it cannot
be fully grasped within any single level, least of all, within the level of
\co{reflective dissociation}.  It is \co{clearly} understood, it seems perfectly
\co{visible}, yet what is so \co{visible} is only its \co{actual signs}, its
end-points. Any account of it leads beyond the comprehensibility of the
\co{visible} elements.  We give now two more elaborate examples of \co{traces}:
identity and truth.

\subsubi{Identity}\label{sub:Identity}
We have seen the variations of the identity notion  through the
preceding sections: in the idealization of \co{immediacy} and \co{actuality} to
the residual, self-identical \thi{substances}, in constitution of things as
limits of \co{distinctions}, in the \co{posited} \co{totalities}
  {of the world} and \co{myself}, in the \co{unity} of \co{self}. 
% We will now gather the main aspects related to the notions of identity and, by
% the same token, give an example of a thread which, going through all the levels, 
% illustrates their co-presence in every experience. 
These variations reflect only the different elements encountered on the
\co{trace} of identity, as if different points marked on a continuous line
passing through all the levels. 
  
Leaving and then returning to a room, I \co{re-cognise} the cup on the table as
the same which was there a while ago.  The cup \co{here-and-now}\ points to the
one \co{there-and-then}.  True, it points in a very specific way making the
identification of the two immediate, but it does point nevertheless, that is, it
is now a \co{sign as a sign}, a \co{sign} whose non-identity with -- the
potential distinction from -- the signified is given along with the very
identity of the two. The difference is the difference of the \co{actuality}
\herenow\ and the \co{actuality} \thth. Identity seems to connect the two,
stretching across the time interval which separates them.  \co{An experience} of
identity arises as an instance of a repetition which, in turn, seems to
presuppose memory. So we will start with a few general remarks on that. Then,
still in the preliminary fashion, we will comment briefly on the unnecessarily
exaggerated role of language in establishing identities. After this
introduction, we will discuss in more detail various forms of identity.


\subsubnonr{Memory}
We forget many things.  But what does it mean?  Do they simply disappear,
as if never happened?  Certainly not.  What we usually mean by memory is related
to particular facts and \co{actual} events which are stored in the \co{precise}
form ready to be fetched with a satisfactory exactitude of detail.  This ability
varies greatly for it happens often that such \co{precise} things get dissolved
in subconsciousness and have to be fetched back, as it may happen in
psychoanalysis.  But probably not even Freudians would assume the possibility of
a total recall of everything that ever happened. Some things just get lost, not
in the subconsciousness from which they might be restored in unchanged form, but
in a complete \ldots \co{virtuality}.  They are not kept \thi{the way they were
  experienced} but get as if \thi{compressed}, mingled with other contents
losing their rigidity and \co{precision} -- losing their identity.

\pa\label{pa:memovirt} According to Piaget, memory is a function very similar to
intelligence and \citet{the development of memory with age is the history of
  gradual organisations closely dependent on the structuring activities of
  intelligence.}{PiagetMem}{p.381 \citaft{PiagetPhil}{ p.75}} Can we know
something without remembering it? Can we remember something without knowing
it?\noo{(That is, knowing what we remember; knowing that we remember $x$
  involves rather obviously knowing also $x$.)} Such Wittgensteinian questions
do not, perhaps, await any answers, but we may try. Knowing and remembering
involve both the ability to re-produce or re-cognise. Knowing Pythagorean
theorem requires, in particular, that I remember it, am able to state it on
demand. But it involves more than mere remembrance. What more? This is unclear,
but it might be that I should also be able to use it in various situations, to
recognise the situations where it does {\em not} apply, perhaps, also to justify
or prove it.  On the other hand, when I remember the theorem, is it possible
that I do not know, do not understand it? In principle, this seems possible, I
can be able to re-state the theorem without, however, being able to discern
adequate meaning in it, I just remember the formulation. If I am to apply it, or
to decide if it is applicable, I can be forced to work through it again, try to
re-call, or figure out anew, its meaning, the ways of its application, etc.
It this sense, memory is only some minimal precondition of knowing or, as
the case may be, is only some residual rest which remains from richer knowledge
as its elements gradually disappear -- from memory.

\pa What is this rest? What happened to the elements which disappeared? Even if
I forgot most of them, it is much easier to bring them forth in my understanding
than it is when learning the theorem for the first time. They did not disappear
completely, they only as if waned away, but are still -- somewhere, somewhat --
around. Consider now what happens when we are thinking. I can work intensely
with the theorem, setting its various elements explicitly \thi{before my eyes},
trying to connect them, deduce consequences. This is the most active,
\co{attentive} thinking. When, trying several times, I get stuck unable to reach
the desired solution, the best thing to do can be to forget the whole problem
for a while. To literally {\em forget} it, erase it from the horizon of
conscious attention. It happens almost typically that the solution, or a new
creative suggestion, will just appear, as if by its own force, after some time
(cf.~\ref{sub:notunconscious}). \thi{Thinking} is obviously going on in the
background while one is not thinking actively and deliberately. What is
typically called \wo{a thought} is, more often than not, only an \co{actual}
result of such a hidden process, a \co{precisely visible} formulation which
reminds more of a momentary revelation than of careful construction. Of course,
careful constructions are helpful in obtaining such revelations, but for the
most they concern only verification, adjustment and adequate formulation of
\co{vague} contents which are \thi{given} to and not constructed by us (adequacy
being exactly the hardly definable proximity of the eventual formulation and the
barely \co{visible} image).\ftnt{Thus, just like \thi{nows} mark only particular
  peaks of intensity in the flow of time, according to
  I:\ref{sub:time}.\refp{pa:rest-flow}, so \thi{thoughts} are like \co{visible}
  \thi{substantive parts} marking the resting-places between the \thi{transitive
    parts} of the continuous -- and not necessarily conscious -- stream of
  \thi{thinking}.} 
It should not be all too daring to propose that such a subconscious
\thi{thinking} works not only with other materials which one could, if one
wanted to, bring to \co{actual} attention but also with contents similar to
those which started to wane away but did not quite disappear from the memory. In
fact, the creative solution one obtains in this way involves often exactly such
elements which were not available to immediate introspection. We might say,
\thi{thinking} reaches here into deeper layers of memory than does the active,
\co{attentively} controlled thinking.

\citet{The image [\ger{Vorstellung}] of pain is not a picture [\ger{Bild}] and
  {\em this} image is not replaceable in the language-game by anything we should
  call a picture. [...] An image is not a picture, but a picture may correspond
  to it.}{WittPI}{I:300-301.\noo{\thi{Image} returns several times, e.g.,
  I:330-331;386-397;653}}
% % For instance, \citet{Describe the aroma of coffee. -- Why can't it be
% done?}{WittPI}{I:610} For you do know this aroma, you can recognise it when
% sensing it. Without going into detailed interpretation, we would like to
%   suggest that w
What do you remember remembering, say, Eiffel Tower? Wittgenstein would ask: try
to describe it!  Try to describe what you are seeing (with your closed eyes)
when you are trying to recollect Eiffel Tower. You end up describing what you
would draw if you were asked to, you end up describing a picture. But you do not
{\em see} this picture. With your eyes closed, trying to recall Eiffel Tower,
you are trying to \co{actualise} it as a picture. But what are you trying to
\co{actualise}? What is \thi{there} to be \co{actualised}?  While \thi{picture}
is what can be given as an \co{immediate object}, what can be re-produced and
re-presented, so \thi{image} corresponds to a more \co{virtual} element which
simply does not have any unique \co{representation}, it only has many different
\co{actualisations}. Recalling Eiffel Tower you can draw it in various ways, you
can describe it with various words and pictures.  \citetib{<<The image must be
  more like its object than any picture. [...] it is essential to the image that
  it is the image of {\em this} and nothing else.>> Thus one might come to
  regard the image as a super-likeness.}{WittPI}{I:389. Let us also point out
  the close connection between \thi{image} and \thi{aspect}: \citefib{The
    concept of an aspect is akin to the concept of an image. In other words: the
    concept \thi{I am now seeing it as\ldots} is akin to \thi{I am now having
      {\em this} image}.}{WittPI}{II:xi\kilde{181}} Different as these concepts
  may be in Wittgenstein, both are closely related to our \nexus\ and
  \co{aspect}.  Discerning, eventually, \thi{words of thought} as only a germ in
  our mind, Wittgenstein concludes: \citefib{If God had looked into our minds he
    would not have been able to see there whom we were speaking
    of.}{WittPI}{II:xi\kilde{p.185}}\noo{Would be unable to \co{actually} see,
    for the deeper contents are \co{invisible}.} What is there \thi{in our
    minds} seems more like \co{virtual nexuses} than like any \co{visible,
    precise} pictures.}  Suggesting that image is \co{virtuality} of a picture
(of many pictures) we have stretched its meaning a bit further.\ftnt{Primarily,
  in that Wittgenstein seems to see it only in the context of (sentences describing)
  \co{actual} imagining, so that \citefib{[s]eeing an aspect and imagining are
    subject to the will.}{WittPI}{II:xi\kilde{p.182}}
  Even if it (always) were so, it still would not imply that also {\em forming}
  the \thi{aspects} and \thi{images} is any voluntary activity.} For remembering Eiffel
Tower is not so very different from understanding/knowing pain: we have some
(yes, a bit mysterious) \thi{image} which can only be \co{actualised} in various
pictures.  To recognise pain of a burning moth or wriggling fly, you have to see
them as \co{actual} pictures of a \co{vague} image: pain; just like to recognise
a particular drawing you have to see it as a possible \co{actualisation} of the
image of Eiffel Tower, and like to use the theorem in the \co{actual} situation
you have to find the \thi{way it applies}, you have to subsume the \co{actual}
situation under the generality of the theorem.\noo{All three cases are, of
  course, distinct and other aspects of each could be brought forth to
  illustrate it. We are concerned only with this aspect which they share.} Even
if the last case is simply a subsumption of a particular instance under a
general rule, it can be seen as related to the other cases where a particular
picture illustrates, or \co{actualises} an image.\ftnt{The {\em relation} of
  instantiation is no less mysterious than that of exemplification or
  \co{actualisation}. \citef{This schematism [...] is an art concealed in the
    depths of the human soul, whose real modes of activity nature is hardly
    likely ever to allow us discover.}{CrPR}{The schematism of the Pure Concepts
    of Understanding;A141/B181}} This image is no longer any \co{actual
  representation} but a more \co{vague} \nexus\ which no longer has a
\co{dissociated} existence of merely \co{actual object}, but is rooted in the
deeper layers of \co{virtuality}. The difficulty with remembering is to
reproduce an \co{actual} picture from this \co{virtual} \nexus, is to recall
anew various aspects of the understood theorem.  \citetib{The difficulty is not
  that I doubt whether I really imagined [Eiffel Tower, pain or] anything red.
  But it is {\em this}: that we should be able, just like that, to point out or
  describe the colour that we have imagined, that the projection of the image
  into reality presents no difficulty at all.}{WittPI}{I:386}


%Objective sign - remembered -> permamnence/objectivity/long time-span
\pa Understanding amounts to integration of particular elements into appropriate
contexts; integration, that is, ability to fetch them in appropriate situations
for particular purposes. Likewise, memory includes (images of the) remembered
things into deeper, more \co{virtual} layers of our being, from which they can
be fetched as \co{actual} pictures. The image-like character of a remembered
thing reflects the degree of its involvement into the \co{nexus} of other
elements, that is, the degree of its dissolution in the more \co{virtual nexus}.
An effect of this involvement, of an element becoming an integral part of the
whole, is, rather naturally, that the element's identity becomes less
recognisable -- no longer a \co{precise} picture, \co{actual} statement, but a
\co{vague} image.\ftnt{We can recall the increasing \thi{density} of the images
  close to the pole, as the projected points of the line lie further and further
  away from the circle -- I:\ref{sub:virt}.\refp{pa:stages}.}
%
Consequently, memory is not a mere recording machine which can, possibly,
influence the present (the view which Piaget attributes to Freud and Bergson).
It involves a successive and constant re-organisation by a process of
\citet{active and selective structuring.}{PiagetMem}{p.378.\noo{
    \citaft{PiagetPhil}{ p.83}} By the way, this fundamental point is
  consistently ignored by all pedagogy which, assuming artificial
  \co{dissociation} of memory and thinking, tries in a single-minded fashion to
  motivate every single step of its procedure and to develop \thi{understanding}
  before, or even instead of \thi{memory}, by releasing pupils from the boring
  memorization drill. Memorization develops deeper, because more \co{virtual},
  structures of organising \co{experience} than mere smartness (if one prefers,
  intelligence) developed by puzzle-solving and tested in IQ questionaries.}
All our adult \citetib{memories, no matter how trivial, isolable, or
  individualised, involve a host of spatial, temporal, causal, and other
  relations, and a whole hierarchy of planes of
  reality.}{PiagetMem}{p.131\noo{\citaft{PiagetPhil}p.92, ftnt.4}} Redressing
Freud's analysis of the Wolf-man dream in terms of Piaget's theory, Casey
arrives at the following schema:
\[
  \bigg(\ldots\Big(\big( (M_{0} \rightleftharpoons M_{1}) 
     \rightleftharpoons M_{2}\big)
     \rightleftharpoons M_{3}\Big) \rightleftharpoons \ldots \bigg)
     \rightleftharpoons M_{r}
\]
$M_{0}$ is the original event and the following $M_{i}$'s the successive
memories (or other influencing experiences) of it until the present recollection
$M_{r}$.  The arrows $\rightleftharpoons$ at each stage represent the
interaction and mutual influence of the involved elements.  This may be a more
realistic picture of the workings of memory, confirmed also by various
experiments of Piaget's.\ftnt{\citeauthor*{Casey}. This picture is to some
  extent consistent with the memory built as a retentional continuum in which
  the current retention of the previous phase retains also the retention of this
  previous phase, which itself contains retention of its predecessor, and so on.
  The crucial difference (besides the fact that we are not concerned exclusively
  with consciousness) is that here it is not a mere accumulation of the past
  phases, but that each stage may influence both its successor {\em and} its
  predecessor.  Also this observation goes back at least to the phenomenologists
  around Husserl. Scheler, for instance, remarks: \citef{[E]very experience of
    our past remains unready with respect to value and undetermined with respect
    to meaning, as long as not all of its inherent effectualities have been
    released. Only in the totality of the whole life, when we have died, the
    experience will become an unchangeable fact with ready meaning, like the
    past natural events are from the very start.}{MaxReue}{ p.34}\orig{... so
    ist auch jedes Erlebnis unserer Vergangenheit noch wert{\em unfertig} und
    sinn{\em unbestimmt}, solange es nicht {\em alle} seine ihm m\"{o}glishen
    Wirksamkeiten geleistet hat. Erst im Ganzen des Lebenszusammenhanges
    gesehen, erst wenn wir gestorben sind [...], wird so ein Erlebnis zu jener
    sinnfertigen, \thi{unver\"{a}nderlichen} Tatsache, wie es die in der Zeit
    zur\"{u}ckliegenden Naturereignisse von Hause aus sind.} } You can not
remove your past -- you can only change it. 

\pa The crucial point is that an analogous picture can be applied also to events
and things which we do {\em not} remember.  They get surrounded by other events
and experiences, conscious or not, remembered or not, and gradually lose their
identity retreating further and further, and merging into the sphere of
\co{virtuality}.  \citet{As the time-object withdraws into the past, it shrinks
  and therewith becomes dim.}{Zeit}{ A:1.2.{\para{9}.}\orig{Indem das zeitliche
    Objekt in die Verrgangenheit r\"{u}ckt, zieht es sich zusammen und wird
    dabei zugleich dunkel.}}  At the present moment, $M_{r}$ is confronted with
the whole past which, indeed, is not given as a collection of bits and pieces
glued together, but as a \co{virtual} unity of \ldots the past.  It involves
things we remember, and can \co{actualise}, as well as those we do not, and
which we can not.  Some might have been lost forever for \co{actual}
recollection, yet they remain present, albeit transformed beyond possible
recognition, dissolved and de-identified.  The picture is not exactly like the
one above but rather something like:
\[
  (\ldots\Big(\bigg( \Bigg(\bullet 
     \rightleftharpoons 
     \begin{array}{c}{\scriptscriptstyle{\vdots}}\\[-.5ex]
                     {\scriptscriptstyle{M_{1}}}\\ [-.5ex]
                     {\scriptscriptstyle{{\vdots}}}\end{array} 
     \Bigg) 
     \rightleftharpoons 
     \begin{array}{c}{\scriptstyle{M_{2}'}}\\[-.5ex]
                     {\scriptstyle{M_{2}}}\\ [-.5ex]
                     {\scriptstyle{{M_{2}''}}}\end{array} 
     \displaystyle\bigg)
     \rightleftharpoons M_{3}\Big) \rightleftharpoons \ldots )
     \rightleftharpoons M_{r}
\]
Well known examples -- of integration of the \co{actual} facts and
observations into the totality of our \co{experience} -- concern 
learning almost anything, in particular, some skills like riding a
bicycle. The scattered pieces of advice from the instructor, the failed attempts
to master one particular movement at a time, the intense consciousness trying
to organise all the bits in proper sequences and alliances of movements -- all 
that continues until one \thi{gets it}. And although we tend to focus on
the exact moment when we \thi{get it} for the first time, it is not the moment which
is important but the fact that all the labourious details, all the minute
successes and failures recede into the background of almost unconscious
automatism. The emerging consistency of the totality is a qualitative change in
relation to the tiny details which led to this emergence. The moment of
\thi{getting it} represents the formation of a new \co{virtual unit}, which
\thi{falls in place}, gets integrated with the \co{totality} of other elements.
The examples are not, of course, limited to acquiring motoric skills
-- learning to solve differential equations has exactly the same structure of
painful details receding gradually into the background of the acquired
skill.

In case it were said that here one is still able to voluntarily \co{actualise} the
acquired skill, and so it does not really illustrate the transition into
\co{virtuality}, let us consider some other examples. 

As we described in the opening sections of Book I, the lack of memories from the
earliest days and hours of our existence is not due to the lack of memory but of
anything specific to remember.  In the beginning we do not collect memories of
any \co{actual} things or events, but only some \co{virtual} \co{traces} which only later
get differentiated into more \co{precise} forms. These primal \co{traces}, too, may
be called \wo{memories}, albeit only in our generous sense of the word.

A still more illustrative example may be that of imperfect memories, memories
which lost not only some of the original details but {\em all} of them.  Proust
describes also the extreme cases when the \co{actual} element triggers the
search for its past counterpart which, however, fails; e.g.: \citet{[...] I
  sensed the smell of the cherries on the table and nothing else. [...] I could
  not, however, choose anything from the confused, known and forgotten
  impressions; eventually, after a short while, I ceased seeing anything and my
  memory for ever immersed itself in sleep.}{ProustSB}{Introduction\noo{p.120}}
One might say: the smell triggers a recollection which either became completely
unrecognisable, or at least is so in the current moment; (an event of) memory
without anything remembered.\ftnt{A similar and common case: \wo{I know this
    person, I am sure I know him but \ldots who is he? Where did I meet him?}
  \co{Recognition} of the remembered \thi{image} precedes here the \co{actual}, 
  conscious remembering. The person emerges from the surrounding (\co{virtual}
  background) already marked with the sign of his identity (the image is like a
  super-likeness of its object) -- even though consciousness still needs to
  decipher the tokens of this identity, to fetch the detailed pictures.}
  
In many situations, what remains are not any specific details but only
\co{vague} feelings of the atmosphere, of the character of the situation, of the
general impression which was \co{actual} then or, perhaps, which is so only now
and in some way gets referred, \thi{connected} to the original experience.
Particular things may, as with Proust, play a role but only auxiliary one, of a
trigger. Memories, according to Proust, do not {\em live} in things, they are
only imprisoned there. Memories, \citetib{every hour of our life, once it has
  passed into the past, incarnates into some material object and remains hidden
  there, imprisoned until we meet it on our way.}{ProustSB}{\noo{Introduction}} This
\thi{imprisonment} should be taken as a mere metaphor of the potential to
trigger a recollection: the \co{actual} things are needed only to {\em awake}
the memories from their sleep, that is, to awake the mind from its sleep in mere
\co{actuality} devoid the enlivening presence of memories.  Particular things,
and their remembrance, are only expressions of the true life of memories;
\citeti{voluntary memory, the memory of intelligence and eyes reproduces the
  past only as an imperfect picture, which resembles the original as much as the
  pictures of bad painters resemble spring.}{M. Proust in a letter to
  Ren\'{e} Blum.}{} \citet{Compared to this past which is an intimate part of
  ourselves, the truths of intelligence seem little
  real.}{ProustSB}{Introduction\noo{p.121}} 

Such \thi{emotional memory}, which is an intimate part of ourselves, is more
frequent than we commonly admit. For in \co{actual} terms what counts is the
\thi{voluntary memory}, are the \co{precise} details which we are able to
recount and recollect in the \co{actual} context, not any \thi{subjective} 
 feelings.  Yet, much of the childhood memories consist often of exactly
such \co{moods} and \co{impressions}.  Reading a book for the second time after
10 years, only some details will re-emerge from memory as you encounter them
again.  Many of them you simply do not remember.  Yet, you will quickly
re-cognise the general impression the book made on you, you will recognise the
image by means of a few pictures.  Only some accidental
\co{actual} element is needed: to hear the sound of a dropped tea-spoon to
recall Combray and the childhood home, to stumble over the uneven pavement in
front of the palace of the Guermantes to recall the walk in Venice.
% \citt{sound of \ldots}{Proust -- cookie dipped in tea reminds about Odette,
%   vol.I (W strone Swanna), part 1 (Combray)}, for the original text, the whole
% Combray coming back --
The recollections need not come back in all details but only with the details
sufficient to establish the connection between the two points in time.  Beyond
that, what is recollected are the significant \co{signs} (pictures) of the more
\co{virtual} elements, of the atmosphere and mood without which the memories
would remain \co{dissociated} and lifeless chips.

%\sep

\pa \thi{Voluntary memory}, the memory of intelligence and \co{reflection}, is
only one layer of memory, just like the \co{objective, dissociated} \thi{givens}
constitute only the lowest layer of any actual situation. These lowest layers of
their respective threads, \thi{voluntary memory} and \co{objects}, mark 
both the same level, are both \co{aspects} of the same \nexus\ of
\co{reflection} and \co{objectivity}.

\citet{A dog believes his master is at the door. But can he also believe his
  master will come the day after tomorrow? -- And {\em what} can he not do here?
  -- How do I do it? -- How am I supposed to answer this?\lin Can only those
  hope who can talk? Only those who have mastered the use of a language. That is
  to say, the phenomena of hope are modes of this complicated form of
  life.}{WittPI}{II:i} Recall two kinds of time (consciousness) from
I:\ref{sub:objectiveTime}.\refpf{pa:twoTimesA}: the phenomenal time of
\co{actual now}, with all its retentions and protentions vs. the objective time
of \thi{inauthentic consciousness of time, of remote past}. Expecting somebody's
arrival the day after tomorrow (or next year) presupposes consciousness of
objective time, simply because having at all the idea of \thi{next year}
requires such consciousness. We could say: if dog does not expect his master to
arrive next week, this happens for the same reasons for which he does not
consider how the ball he is playing with feels in the hand of his master nor,
for that matter, where {\em this} very ball was made.\ftnt{Promising is likewise
  a good example of this crossing point, of \co{actuality} lying at the point
  where the \co{foundation} in the deeper \co{unity} across time meets the
  \thi{frozen} objectivity. It is will, patience and perseverance
  which are capable of stretching the influences of \co{actuality} to remote
  future, and all these presuppose objective time. Opposing the autonomous man
  with independent will to the mere moralistic follower of custom, Nietzsche
  points out how the \co{unity} in the objective time is internalised and
  existentially grounded in the former and only \co{externally} accepted by the
  latter: \citef{the sovereign individual who resembles nothing except himself
    and who again is freed from the morality of custom [... is] the man
    possessed of a personal, independent, and long-lasting will and who is {\em
      competent to make promises.}}{GM}{II:3} The {\em competence} to make
  promises, unlike the mere customary ability to make them, is an expression of
  the lived \co{unity}, of the lived understanding that I am one person,
  immersed in but also independent from the flow of objective time. Evading
  promises one had made amounts to estranging oneself from one's past which,
  eventually, means \co{alienating} oneself from oneself.} We do not imagine dogs
to relate to the possible difference between {\em this} very ball and another
though indistinguishable one, that is, to have consciousness of objectivity. But
a dog can await and expect its master's arrival, and long for him the more, the
longer is his absence. For dogs, too, live in the temporality with its past and
future. They live in the same time as we do and are \co{aware} of the same time
-- only this \co{awareness} does not reach the crispness of \co{objective
  dissociations}.

\co{Reflective signs as signs}, constituting the \co{foundation} of language,
enter likewise into the \nexus\ of \co{objectivity}. Serving as important tools
of \thi{freezing} some (limits of) \co{distinctions} (I:\ref{sub:actnonPower}),
they serve likewise as tools of \thi{voluntary memory} or, as we also could say,
\co{objective} memory. We may have vivid \thi{emotional} recollections of some
particularly significant events from our remote past. But for the most, what
happened to us five years ago is not remembered \thi{in flesh} but merely as
abstract descriptions. I can say: \wo{Five years ago I was in Prague, I walked
  past \ger{Malostransk\'{e} n\'{a}m\u{e}sti} almost every day, I ate dinner
  several times at this place,} etc., but all these events are recalled as mere
facts, as merely \co{objective} facts which would feel and could be described
almost exactly the same way if I were relating events from a movie or sketching
an imaginary story I planned to write. Of course, I am relating my own past and
it is still some \co{virtual} image which underlies these recollecting
descriptions. So, in principle, one might manage that also without objective
time? But the role of objective time, and words, is quite crucial. Objective
time allows us to refer to such a remote event which has been \thi{emotionally}
forgotten, just like single \co{objects} or situations trigger, according to
Proust, vivid \thi{emotional} memories. One asks: \wo{What did you do in the
  summer for five years ago?}  Without objective time such a question would not
make any sense.\noo{and without words it could not be asked.} Events in our life
do not carry any inherent time stamp on them. One can remember meeting somebody
and have no idea if it was two, three or five years ago. One can remember two
distinct events and be unable to say which happened before which.
\co{Experiences} become mutually related and organised along the line of
\co{objective} time. Without it they would only interlock in a \co{virtual}
mesh, losing their identities and hence disappearing for future recollections.
Even if we could, in some unclear and unspecified sense, remember our whole life
while living at some pre-\co{reflective} and pre-\co{objective} level, we would
have no means to fetch these \thi{memories} and \co{actualise} them. And what is
a memory which can not be recalled? The events would simply keep dissolving in
the \co{virtuality} of our past, forming us, as experiences form also the
character of a dog. But the whole process would remain un\co{reflected} and
hidden in the same \co{virtuality} which \co{founds} it.

\pa Just like \co{actuality} of \co{reflection} is \co{founded} in the hierarchy
of higher \co{aspects}, so also the \co{reflective} memory
%, recalling something to reappear
has its deeper presuppositions.  The first is that it is needed at all. And it
is because life and world are not a whole given in the unity of one \co{act} but
are split into diversity of separate \co{actualities}. The need for
\thi{voluntary} memory, just like for ideal entities, arises with the
\co{dissociating} activity of \co{reflection} and is the more intense the more
\co{precisely dissociated} become the contents of our attention. (It is not
unusual that extraordinary intelligence is accompanied by the excellent memory,
even if this excellence is often limited exclusively to the memory of the things
occupying the intelligence.) 

The second presupposition is that it is what actually takes place, that I \co{actually}
remember {\em the same}.  This \thi{repetition as recurrence} requires the
possibility of \co{re-cognising} identity of the same across time. As we
suggested discussing time in I:\refpf{pa:objectiveTime}, and as we will
elaborate below, this is possible because new things and \co{experiences} are
not \thi{added} to any given collection but, like everything else, emerge as
results of differentiation from the \co{indistinct origin}. More specifically,
such \thi{repetitions} express the \co{recognitions} which are not necessarily
limited to pure \hoa. They arise from \co{virtual} \nexuss\ whose \co{unity}
precedes \co{dissociation} of \co{actualities}. Memory, as \co{reflective
  re-cognition}, is \co{an experience} of a \co{recognition} transcending the
\hoa.

And thus we arrive at the most fundamental, even if entirely trivial,
assumption: memory can take place only in a being whose unity stretches across
time, whose unity is not an \co{immediate} self-identity limited to pure
\co{immediacy} but \co{transcends} its horizon.  \citet{All beings confessedly
  continue the same, during the whole time of their existence. [...]  All these
  successive actions, enjoyments, and sufferings, are actions, enjoyments and
  sufferings, of the same living being. And they are so, prior to all
  consideration of its remembering and forgetting [...]}{Butler}{Appendix I}
Memory does not establish identity nor is it constitutive for personal identity
-- at most, it can help establishing the sense, the feeling of it. Losing
memory, one does not necessarily lose \co{oneself}, and even without remembering
anything of one's past one can still know {\em that} one had a past. 
%
It is not, for this reason, a mere ontic accident or an
epistemological device.  It is one of the fundamental aspects disclosing in  
\co{actual experiences} the \co{foundation} of this \co{actuality}
in the continuity through time and, in the last instance, in the \co{unity} of
\co{existence}. 



\subsubnonr{Language}

\pa\label{pa:language} Language provides the common means for drawing boundaries
between things, whose usefulness and practicality is not based exclusively on
their \co{precision} but often, on the contrary, on their roughness. If I cut
off the branches, what is left is a \thi{trunk} and no longer a \thi{tree}; if I
cut the whole \thi{trunk}, what is left is only a \thi{stump}.  But, of course,
we have no \co{precise} idea how low the stump must be to be a \thi{stump} and
not a \thi{trunk}, nor how many branches must be cut off for a \thi{tree} to
become a \thi{trunk}.  Growing up into a language which has only one word for,
say, both \wo{pain} and \wo{suffering}, one would tend to consider the two as
identical and, in any case, \co{reflective} establishment of the distinction
would probably take much longer time and might even appear as a deep discovery.
If eighteen or so Hebrew words for different shades of \thi{purity} get
translated by the same Greek word \wo{\gre{katharos}}, then the meaning of the
Old Testament must undergo some, hopefully only slight, changes.
%almost necessarily acquire a slightly different form.
But claiming that without the linguistic means, one is entirely incapable of
\co{experiencing} the difference, would be like claiming that the lack of names
for many colours and their shades makes also experiencing the actual differences
impossible. The identities and distinctions sedimented in the language express
only roughly the average relevance, as well as the historical development --
they are passed as pragmatic guidelines, but they {\em never determine} the
range of the possible \co{experiences} of identity and difference.\noo{They
  affect primarily the mediocre, though they certainly may -- may!  -- influence
  the exceptional, too.}  The relativity of identities (and possibly also of
elaborate systems thereof) can be discerned by observing some differences
between languages.

\pa Slavic languages provide almost unlimited means for modifying the nouns by
means of suffixes -- not only to form chains of diminutives, but also to
indicate features and impressions the things make on one, as if stretching and
comprising the stem, which in many other languages would require unbearable series of
adjectives.\ftnt{In Polish, for instance, \wo{ptak} is a bird, \wo{ptaszek} a
  small bird, \wo{ptaszyna} 
  an even smaller one, \wo{ptasior} is a rather large and ugly, possibly
  dangerous bird, \wo{ptaszydlo} is an extremely repulsive \wo{ptasior}, etc.} In
fact, most nouns can be turned, or dissolved, by such means into adjectives. The
ontology, the collection of identities, seems to dissolve in a landscape of
grades, variations and qualities without any definite and final
\thi{substances}.  In Germanic languages one utilises much more the opposite
operation to that just mentioned of forming adjectives from nouns, namely, one
which results in a noun formed from an adjective (the suffix -ness in English,
-heit in German).  A language like German, where the etymological connections
between words are still very tight, but variations less flexible, where the
formation of compound words (in particular, nouns, practically absent in Slavic
languages) seems to reflect the structure of entities, will suggest ontology of
hierarchies and systematic relations between basic entities, and by the same
token, emphasize the division between the natural and rational, the given and
the constructed element.  A language like English, an enormous collection of
words from multilingual sources, which provide great flexibility, but even when
expressing very close and related phenomena, remain etymologically unrelated,
will suggest ontology of minute, mutually independent elements. The
identities, as well as the character of the identity itself, are instinctively
established by means of the dissociated atoms, unrelated words.

As a more specific example let us consider only the word for \thi{reality} and its
folk-etymology. We are not asking for any genuine etymology. It is the
superficial, pseudo-etymological or even merely phonetic associations rather
than the scientific etymology which may -- may!  -- influence a child, long
before it possibly might start studying linguistics.

English \wo{reality} gives hardly any immediate associations.  Sure, one can
think of some Latin origins from \wo{res} but these are too advanced
considerations for us. It is but another word, as unrelated to \wo{house} as to
\wo{thing} or \wo{activity}.  The conceptual correlate, something like
the definition of \thi{reality} in a Merriam-Webster dictionary: \wo{something that is
  neither derivative nor dependent but exists necessarily}, seems as empty and
abstract as the lack of any deeper linguistic relatedness which it reflects.


German \wo{Wirklichkeit} is an entirely different matter. It is bound to be
associated, unconsciously and often consciously, with \wo{Wirkung}, \wo{wirken},
etc. \wo{Wirklichkeit} is something that acts, works, is efficacious, it is a
power rather than a state.  \wo{The world of the real is a world in which this
  acts on that, changes it and again experiences reactions itself and is changed
  by them. [...] What value could there be for us in the eternally unchangeable
  which could neither undergo effects nor have effects on us?  Something
  entirely and in every respect inactive would be unreal and non-existent for
  us.} Well~\ldots spoken by a true philosopher of
language.\ftnt{\citeauthor*{FregeTruth} {p.103/104.} Emphasizing relativity of
  every \co{distinction} to the \co{distinguishing existence}, and claiming that
  to be is to be \co{distinguished}, we come very close to the idea of the last
  quoted sentence. Every \co{distinction} makes a difference. However, the
  difference need not imply any activity and even less any effects of physical
  kind, which are exactly the aspects captured by \wo{Wirklichkeit}. The same
  associations obtain in the Scandinavian languages.}

Polish \wo{rzeczywisto\'{s}\'{c}} brings immediately associations with
\wo{rzeczy} -- \wo{things}.  The suffix \wo{-isto\'{s}\'{c}} has no inherent
meaning (it is used to form many other words), but it may easily lead a layman
to something like \wo{istno\'{s}\'{c}}/\wo{istnienie} -- \wo{being},
\wo{existence}.  Thus, \thi{reality} seems to be the state or order of things
(all things?), something given rather then acting, and acting only in the way
\thi{being} acts -- by simply being.\ftnt{Russian 
  \wo{deystvitelnost}, although including the aspect of effective activity,
  \wo{deystvenny}, gives similar associations to passive
  happening, something which is because it happens but which is simply the way
  it is.\label{ftnt:deyst}}

% \citet{Etymology is a wonderful way of exploring the unconscious; it is the
%   unconscious of language.}{PsycheAntiq}{I:5.~Parmenides and Anaxagoras}

\pa 
We are not linguists and all this is, of course, very rough. But could we not dare
to look here, even in such a superficial comparison and rough differences, for
at least some\noo{ -- never, final and all-determining --} reasons for the
general differences between the philosophical schools which dominate the spheres
of different language groups, just as it was argued that, for instance,
Aristotle's ontology, 
if not whole metaphysics, was firmly grounded in the Greek grammar and etymology?
It seems that we could but we will leave such investigations aside. 
%to the historians of philosophy
Each language has its \co{mood}, even its own \co{quality}, which only to some
extent can be traced to the simple rules of grammar or etymological
associations.  But in spite of all such differences (reflected also in the
indeterminacy of translation), different human languages have approximately the
same differentiating and unifying power.  Sure, some words, expressing some
particular \co{distinctions}, may be emphasized in one language and missing in
another, and almost always more \co{vague distinctions} will be drawn across
different, though hardly disjoint, semantic fields.  But, in general, the
\co{distinctions} expressed in one language can be reflected (even if sometimes
only clumsily) in another -- after all, and in spite of indeterminacy, a
translation is possible. Above all, one person can always, even if only in
principle, communicate to another, all such linguistic differences
notwithstanding.\ftnt{And so there is nothing which prevents, for instance, the
  active, apparently Germanic \ger{Wirklichkeit}, to appear in the Latin of 50
  \bc: \citef{whate'er exists, as of itself,\lin Must either act or suffer
    action on it.}{Things}{I:4\kilde{p.16}}} So we will leave language to the
linguists and the fascination with its mystical (or, as one rather should say
today, post-analytical) influences on the mind to those who have nothing better
to influence their minds, and start discussing identity.  We will steer away
from language for although it is a means of drawing and establishing
\co{distinctions}, it is \co{founded} in the ability to \co{distinguish} which
is both prior to and much wider than the sphere of the \co{actual signs}, not to
mention the mere words.

% \say
% But, language does not establish identity. Language itself is only a reflection
% of the underlying distinctions, most fundamental of which are not made by any
% subject, not even by me, but emerge before any actual consciousness can take
% hold of them...

% Identity, as we will see, arises from distinctions as the point at which
% distinguishing terminates. But this does not make it into anything mental, in
% particular, it does not mean that language creates things, that ideology creates
% ontology. The world is created only for an \co{existence}, but it is not created
% by it. 

\sep

%\subsub{Immediate self-identity: $x=x$}\label{immId}
\ad{\imm Immediate self-identity: $x=x$}\label{immId} The original
\co{repetition} is a \co{reflective} \thi{doubling} of the same, extraction of
something from the background and \co{positing} it as an independent, because
\co{dissociated} \co{object}, I:\refpp{pa:refrepet}.  The doubling brings forth
the self-identity which is just the expression of the fact that doubling did not
change the original phenomenon. The thing remains itself, it only gets doubled
in the perspective of the \co{reflective representation}. \citet{<<A thing is
  identical with itself.>> -- There is no finer example of a useless
  proposition, which yet is connected with a certain play of imagination. It is
  as if in imagination we put a thing into its own shape and saw that it fitted.
  \lin We might also say: <<Every thing fits into itself.>>}{WittPI}{I:216}
Every \co{object}, \co{dissociated} in the \co{reflective} \thi{doubling},
remains itself, remains the same as the \thi{doubled} object of
\co{experience}.\noo{ (or so, at least, one would like to have it.) }

Self-identity is the obviousness of the \co{immediacy}, of the fact that, within
a timeless \thi{now}, nothing can change and everything is itself. Everything,
that is, which can be grasped in the \co{immediate} limit of a single \co{act}, and
that is close to nothing. But self-identity (to be completely distinguished from
any identity of the self), the empty formula $x=x$, becomes the paradigm, the
governing norm of all further considerations of identity. The problems haunting
these considerations, as aptly illustrated by Wittgenstein's analogy of
\thi{fitting into its itself}, are the problems of determining the criteria of
identity. These, in turn, reflect only metaphysical arbitrariness of what, among
all temporal objects, could count as the ultimate \thi{substances}.
The problems with the criteria of identity are the problems of classifying several 
\co{actual} appearances as the same, so to speak, of fitting them into this
\co{immediate} limit of self-identity where they would coincide with each other. 

\ad{\act Actual identity: $a=b$}\label{actId}
%Same as transcendent!!!
% - hence impossible to account for in lower level's categories
The original \co{repetition} is the \co{reflective} \thi{doubling} of the same.
In most abstract terms, this is what also \thi{repetition as recurrence} is --
the same thing seen from two, or more, different perspectives. But now, these
perspectives are perspectives in a more intuitive sense of the word, they are
\thi{snapshots}, different \co{actualities}. Equality arises as a relation of
sameness across distinct \co{actualities}, where a point $a$ in (\co{actuality})
$A$ turns out to be the same as $b$ in (\co{actuality}) $B$.  As Frege observed,
the difference between $a=a$ and $a=b$ concerns the form of presentation. Just
like the statement $a=a$ does not say anything, the statement $a=b$ says quite a
lot -- their \ger{Erkenntniswert} is very different. The former states merely
the \co{immediate} self-identity, it \citet{is valid a priori and, following
  Kant, is called analytic.}{FregeSinn}{\label{ftnt:aaab}} The latter, on the
other hand, says that $a$ and $b$ are two different perspectives, two different
snapshots of something.  This something, this ever transcendent $x$, arises
precisely as the equality of $a$ (i.e., $x$ viewed as, or in the context of,
$A$) and of $b$ (i.e., $x$ viewed as, or in the context of, $B$).\ftnt{Frege
  says that $a$ and $b$ are simply different signs and that identity is an
  epistemic relation between signs, which obtains when both have the same
  denotation, \ger{Bedeutung}. This is probably what it becomes, eventually, in
  the \co{actual world} with its ready-made \co{objects}. Our point here will
  be, primarily, that this relation (or fact?) is \co{founded} and emerges as a
  residual \co{trace} in the process which first \co{dissociates} various
  \co{actualities}, and then must re-establish the connections between them. The
  ambiguity can be discerned in the naming. \wo{Identity} seems to refer to the
  absolute fact, while \wo{equality} to a relation between $a$ and $b$. We will
  use both words at the present level.}

Leaving and then returning to a room, I \co{re-cognise} the cup on the
table as the same which was there a while ago.  The cup \co{here-and-now}\ 
points to the one \co{there-and-then}.  True, it points in a very specific way
making the identification of the two immediate, but it does point nevertheless.
What does it mean? That it is \co{a sign as a sign}, a \co{sign} whose
non-identity with the signified is now given along with it? OK, but primarily,
this \herenow, the \co{actuality} of this cup, has been \co{dissociated} from
the \co{actuality} of that cup, the two have lost any connection.  This
non-identity is the difference of \co{dissociated actualities}, \herenow\ as
opposed to \thth; it is posed by \co{reflection} which, having
scattered various \co{actualities} across the time, dwells exclusively among the
\co{dissociated experiences}.

On a closer look, that is, on some \co{self-reflection}, the very identity of I
who am \co{reflecting} becomes only a mysterious quality for which
\co{reflection} can not account in terms of \co{actuality}. How can I know,
\co{actually} know and be sure, that
I \co{reflecting} in this very moment am the same as I a while ago? There is no
logical impossibility in assuming that the two are different and that everything
is re-created anew in every instant, with the amazing precision creating merely
an illusion that it is the same.
It is obviously false or, if we do not like the word, it is not a fruitful or
even meaningful way of speaking (showing, by the way, the value of such a
criterion as \thi{logical possibility}). At another level of \co{experience} I
obviously know, i.e., \co{experience} the cup \co{here-and-now} and the one
\co{there-and-then} as the same thing, and I know myself to be the same person
today as I was yesterday.  But where is the proof, where is the unshakeable
certainty of the two being the same?

There is none.  For \co{reflection}, burdened with the \co{dissociation} of
all \co{actual experiences}, this identity is problematic, to say the least.
Given two things separated into different \co{actualities}, their identity can
not arise in (yet another) \co{actuality} otherwise than as some mysterious $x$
which binds them across and in spite of the gap in time.  Such an $x$ is never
\co{actually} experienced.  Since Plato, one has been more than willing to say
\wo{Alas!  An ideal entity.  The cup here and the one before are just instances
  of the same.  And what is this \thi{the same}?\ldots  A universal.}\ftnt{One
  would be cautious to distinguish a trivial repetition of the same (thing) from
  instances of a universal but the meaning and the pattern are exactly the same.
  As argued in \ref{impressConcept}, universals
  are but special case of \co{non-actualities}; a single repetition, an
  appearance of the same only twice, presents already all the problems of
  indeterminately long series of repetitions, of ideal entities or universals.}
An $x$, an ideal cup existing beyond the \hoa\ and binding together the cup
\herenow\ with the cup \thth.  In order to account for the
identity across different \co{actualities} -- from the perspective of
\co{actuality} -- \co{reflection} ends up with postulating ideal entities.
Identity is a \co{trace} of that \co{dissociation}, and so is the noumenal $x$
which keeps forever receding beyond any horizon of
\co{distinctions}.\ftnt{At first, one may try saying things like that sameness
  means identity, which expresses the epistemological optimism that if $a$ now
  is the same as $b$ was then, so this sameness across the two
  \co{actualities} witnesses to the identity of the appearing thing. But quickly the
  natural consequence becomes relativisation of identity 
  to the operations of the mind. Identity is not a \thi{real} relation, existing
  somehow \thi{in} the object, but arises only as a consequence of different
  perspectives under which mind views the object. Identity \citef{can not be
    anything but a relation of 
    reason, because it is not between things distinct except by reason
    only}{ScotMet}{IX:1-2 \citaft{ScotPotAct}{(In the text, the quoted phrase
      concerns potency, but identity is given as an obvious example of the same
      kind of relation of mere reason. It is a relation which is not, like a
      formal distinction, real though mind-dependent but one which is actually
      caused by the mind in the object.)}} A more modern variant of the
  same idea figures, for instance, in \citeauthor*{FregeSinn}, referred to in
  the beginning of this paragraph.\noo{, ftnt.~\ref{ftnt:aaab}ff}}

%\ \vspace*{-2ex}
\pa What is so mysterious about the two being the same?  It is that one assumes
from the start that the true givenness, true presentation happens only within
the \hoa.  Whatever exceeds this horizon becomes suspicious, prone to deceiving
us, uncertain.  The paradigm of certainty is the \co{immediacy}, presence within
\hoa\ -- this horizon, eventually the evanescent point of \co{immediacy}, is
implicitly taken as the only point of contact with reality.  Taking thus units
of \co{reflective experiences} as the atoms of reality, one is, indeed, in dear
need to invent ideal entities to keep the scattered pieces of \co{actualities}
together.  Ideal entities are \co{reflective} tools useful for organising
\co{reflective experiences} -- but the problem of their (ontological or other)
status is based on the implicit conviction that the real is only the
\co{actual} and that everything else requires a justification in terms of
\co{actuality} and preferably of \co{immediacy}.

\noo{The origin of repetition, and hence of universals, is the \co{reflective
  repetition}, and then \thi{repetition as recurrence} which is \co{founded} in,
and thus witnesses to, the dimension of \co{experience} transcending the \hoa.
}
If we, instead of starting from \co{actual experiences}, start from the \co{unity} of
\co{experience}, eventually of our being, which only subsequently gets
diversified and split into \co{actualities}, then repetition is only
\co{an experience} of the same from different perspectives, from different
points of \co{actuality}. \co{Actual re-cognition} of $X$ as the same as another
\co{actual} (but not actual now) $Y$ is a \co{recognition}, that is, \co{an
  experience} of $XY$ which has manifested itself in two distinct
\co{actualities}.  It is not one, \co{actual} I perceiving $X$ who somewhat has
to establish a relation to another I who perceived $Y$. The moment of perceiving
$X$ is but an \co{actuality} emerging from the background of \co{experience}
where $X$ is but the \co{actual} aspect of $XY$.  Its definite {separation} is
only the result of \co{reflective dissociation}.

Does it mean that every $X$ which is \co{experienced} as (a repetition of)
$Y$, is actually the same as $Y$? Can't I be mistaken in taking $X$ for $Y$?
Well, no and yes.

I can't be mistaken because any such \co{experience} of repetition has a reason,
there is always something which -- in my \co{experience} -- founds the
\co{re-cognition} of $X$ as the same as $Y$. It is a \co{cut} through
\co{experience} which, \co{distinguishing} $XY$, precedes the dissociation of
\co{actual} $X$ from the \co{actual} (though not now) $Y$. I may \co{actually}
not know what it is, I may even, in principle, never be able to account at the
level of \co{reflection} for this $XY$, and yet the very fact of repetition
could not take place without such an $XY$. After all everything
originates from \co{one}. 

But from the objectivistic point of view, that is, assuming the primordiality of
\co{actually dissociated experiences}, I can certainly be mistaken.  There may be
thousands of reasons and further \co{distinctions} which, if taken into
account, could force me to consider $X$ and $Y$ as different
entities. In the extreme forms of empiricism, the mere fact of appearing in two 
different \co{actualities} can be taken as the justification of genuine
difference. 

\pa At some point, the \co{distinctions} terminate and this is what founds the
identity of the \co{actual} thing.  If I take at first moment the person
entering the room to be my friend Yngve, then this is the current limit of
\co{distinctions}.  In the next moment I may realise that the two are, in fact,
different, that it is not Yngve but Xavier, but to see this, I have to bring in
more \co{distinctions} -- I have to see his face more \co{precisely}, see some
of his movements which are totally un-Yngve-like, etc.  Yet, even then, the
\co{experience} of the same, the first impression which confused the two (that
is, which did not distinguish XavierYngve), remains valid, it has revealed
something which it was actually possible not to \co{distinguish}.\ftnt{As usual,
  we are not talking about the \co{attentively reflective} activity which starts
  with the ready \co{dissociated objects} and inquires only into the relations
  between or possible abstractions from them. Such an activity, in a pure form,
  is itself an abstraction which never obtains without being involved in more
  fundamental processes which we are describing. It is, by the way, a good
  example of the \co{dissociative} result of \co{reflection}. In the moment I
  realise that it was Xavier, Xavier becomes {\em the} \co{object} of
  \co{reflection} and thus also {\em the} \co{objective} aspect of the prior
  XavierYngve situation. It becomes opposed to the \co{subjective} impression of
  this having {\em only seemed} to be Yngve. Although XavierYngve remains and
  merely withdraws behind the curtain, it seems to disappear completely as it
  gives place to the two \co{aspects} of \co{reflective dissociation}: the
  \co{objective} fact of Xavier entering the room and the merely \co{subjective}
  impression of him being Yngve.}
%We thus arrive at the following claim: 

%\newpa

\thesisnonr{Identity is the actual\noo{(ly given)} limit of \co{distinctions}. 
}\label{th:identity}

It is the point beyond which no more \co{distinctions} are made, or rather, a
boundary within which no more \co{distinctions} are made.\ftnt{Almost identical
  formulation reads: \citef{For in general those things that do not admit of
    division are called one in so far as they do not admit of
    it;}{AristMeta}{V:6} But the following examples suggest a different, weaker
  meaning: \wo{e.g., if two things are indistinguishable \la{qua} man, they are
    one kind of man; if \la{qua} animal, one kind of animal; if \la{qua}
    magnitude, one kind of magnitude.} Two things can hardly be
  indistinguishable \la{qua} man, or \la{qua} animal, but distinguishable in
  other respect. The contention that two men can be, say, equally good and hence
  both be one kind of man, namely, a good man, has re-surfaced recently as
  relative identity and we will return to it in
  \refp{pa:idwithresp}.\label{ftnt:idwithresp}}

At which moment does the Theseus' ship cease to be itself and becomes a
  new ship? At none, because it has never been \thi{itself}, it has never been
any metaphysical (or ontological) \thi{substance} with intrinsic, self-identical
\thi{essence}.  It was the limit of \co{distinctions} which it was purposeful to
terminate at this point, at the point at which we said \wo{This is a ship}, or
perhaps even \wo{This is {\em this} ship}.  Replacing the planks, we begin
introducing further \co{distinctions} which suspend
the validity of the previous final boundary of \co{distinguishing}.  Its supposed
\thi{essence} was nothing but such a boundary.  But assuming that it is something
positive, something which constitutes thing's identical \thi{being in itself}, we can
not avoid being perplexed by this ingenious puzzle.


\pa An \co{aspect} of the crystallization of \co{object}'s independent
subsistence 
and identity, that is, of objective time, is \co{an experience} of change, of
one and the same \co{object} having earlier been so-and-so, but now being
otherwise.  This might seem to contradict the claim that identity is but a
limit of \co{distinctions} since change implies additional \co{distinctions}
which should differentiate the object before and after the change.  It does not,
however, contradict the claim but only shows the relative arbitrariness of what
counts as identity. For all practical purposes a river is a river, one and the
same river. And it remains one also after the observation that \citeti{[o]n
  those who enter the same rivers, ever different waters flow}{Heraclitus}{ DK
  22B12} pushes the limit of \co{distinctions} beyond that used for the ordinary
purposes. In principle, every issue can be dissolved (one might be tempted to say
today: deconstructed) into interminable series of aspects, views, perspectives
and possibilities, every situation can be discussed \la{ad infinitum} without
ever reaching any bottom, every object can be divided and gradually dissolved
into finer and finer distinctions. But this is so only in principle, that is, in
abstract terms of \co{actual dissociations}. In practice, that is, in the only
context of relevance, it is not quite so. 
%But it is so not because there are no real issues, situations or objects -- t
The fact that a house can be deconstructed does not prove its unreality; and
even if later generation will build different houses in very different ways from
ours, does not prove that we are building wrong and unreal houses. Even if
\co{actual} truths change over time, at each particular time they are given in
quite a stable fashion, in fact, sufficiently stable for people of similar
interests and levels of intelligence to be constrained in their formulations in
approximately the same ways. Objects, issues and situations arise relatively to those
who participate in them and, in particular, relatively to their ability (which
is not under voluntary control) to terminate the \co{distinctions} at some
specific points. But relativity means neither subjectivity nor lack of any
objective counterpart.

Identity is established as the limit of \co{distinctions}; as some
\co{actual}, not absolute limit -- beyond these, more \co{distinctions} may be
possible and actually take place.  In most cases, these further
\co{distinctions} will be considered mere \thi{accidents} of the identical
\thi{substance}, but in the extreme cases they may lead to the puzzles like that
of Theseus' ship.  I cut a branch from a tree; then I cut another; then yet
another; when I am finished with all the branches, I begin to cut, piece by
piece, the trunk, from the top to the bottom.  I end up with a heap of wood, but
at what point does the tree cease to exist?  This is arbitrary, that is,
dependent on the \co{distinctions} dictated by the circumstances.  If I left a
part of the trunk standing, even if dead, one might say that {\em this} is the
same tree which stood here yesterday -- only dead and without branches.  And
since what is relevant here might be the fact of something standing at this
particular place, we do not focus on the change which occurred, the
\co{distinction} that the trunk does not have all the branches the tree had, is
ignored.  If, however, the tree was the favorite one on which children used to
climb and play, then {\em this} tree has actually {\em ceased to exist} -- it is
no more, and what is left is not the same.

Does it mean that, to begin with, there were two trees and one survived cutting
off the branches while the other did not. Certainly, there were different trees
for people who considered it only as something standing there and for the kids
climbing it. But this question of Chrysippus concerns the \thi{objective} state
of affairs: are there, \thi{out there}, two trees or not? And if there are two
which one survives?  Since we do not subscribe to any \thi{objective substances}
existing \thi{out there} no matter what or who is around, we should be allowed
to claim that it still depends on who is looking at the matter.  If somebody
must reach some absolute compromise, let him say that there is a sense in which
there is only one tree which survived the process, namely, the tree identified
as the limit of \co{distinctions} remaining at the end. He should not, however,
try to convince the kids about it. Another problem with it is that one can
remove something more from this rest and ask the question again.  One won't find
any \thi{essence} but will end up postulating a residual point of self-identity.
In our view, it is not only uninteresting but directly inadmissible.  Any answer
to the question about Theseus' ship -- {\em at what point} does it cease to be
the old and becomes a new? -- has an aura of arbitrariness which has always
threatened \thi{substantialism}. For it represents an antinomy arising from the
insistence on the yes-no answer to a question which does not have one, an
antinomy of applying the \co{immediate} category of self-identity at the level
of \co{actuality}, of attempting to view a temporal object \thi{as if} it were
timeless, \co{precisely} delineated \thi{being in itself}, self-identical
\thi{substance}.

Before refining this issue and the thesis from~\refp{th:identity}, let us
comment on its relation to 
some alternatives which might seem possible.


\subsubnonr{Indiscernibility and relativity}
\wtsep{id of indisc}

\pa \label{ft:identidiscern} Identity of a thing is always only relative to the
drawn \co{distinctions} -- consciously or not, intentionally or not,
\co{actually} or not\ldots It is thus relative to the involved \co{existences}.
Does this mean, as also the claim in \refp{th:identity} might suggest, that we
are simply stating identity of indiscernibles? Not exactly, but the difference
may be rather subtle.

First, the principle itself is an ingenious and more \co{precise} variant of the
dictum: \citt{plurality must not be posited without necessity.}{E.g.
  \citeauthor*{OckQuod}{ V:q.1;p.97}, \btit{Ordinatio sive Scriptum in librum
    primum Sententiarum} I:Prologue 1.3 \citaft{FilSr}{ p.242} Earlier variants
of the razor figure, for instance, in \citeauthor*{Parmenides}{};
  \citeauthor*{AristPhysics}{ I:187b.10}; \citeauthor*{ScotPrimo}{}} Its
objective is to attune our understanding of identities to their metaphysical
realm, to bring epistemological distinctions into agreement with the ontological
\thi{facts}.  Viewed (perhaps with some degree of bad will) as such a project,
the principle is quite different from our view according to which identity is
not recovered but constituted as the limit of \co{distinctions}.  Identity is
not any metaphysical, supra-human quality of things.  It is a purely pragmatic
(albeit not voluntary) notion of fixing the limits of \co{distinctions} at
relevant points.  What makes it relevant is not given by any laws, but by the
context.  Identity of indiscernibles is indeed a good rough expression -- do not
\co{distinguish} what need not be \co{distinguished}.  But since we do not have
any \thi{substances}, any \thi{essential} nor \thi{accidental} properties, a
process which has been thus terminated at some point, at some identity, can
always, at least in principle, be carried further, for it never reaches any
\thi{metaphysical identity}.\ftnt{These remarks do not apply to \co{existence}
  which, being a limit of \co{distinguishability}, is also the site of ultimate
  \co{unity}.}  Returning home every day I do not wonder if the sofa standing
there now is {\em really the same} as the one from yesterday, and if I do I
should visit some specialist. It is the same because there is not the slightest
reason, in fact, not any possibility to distinguish the two. It is only the
assumption of some \thi{substances in themselves} which could make one wonder
how to prove that nobody in the meantime entered my flat and exchanged my sofa
for another but \thi{identical} one.\ftnt{The two are indiscernible not only
  because I can not discern any difference although I might suspect some to be
  there. Sitting there and thinking: \thi{The two are identical, I cannot
    possibly discern them {\em but} they still might be distinct!} -- such a
  suspicion is already a discrimination. Hopefully, it may be later refuted
  (with the help of a specialist?) and I can again start seeing the \thi{two} as
  one and the same. But the very suspicion is already a \co{distinction}
  \co{dissociating} the \co{actuality} of the sofa today from its \co{actuality}
  yesterday.}

\pa\label{pa:distLang} A more extreme exposition might be read into the
principle which also might seem to make it identical to the above claim from
\refp{th:identity}.  Its variant often plays a crucial role when one tries to
attune knowledge to \thi{facts} by actually {\em getting rid of} all
transcendence and pretending that one is in the possession of a complete logical
language, fitting perfectly the external world.  It underlies equally the
attempts to reduce all truth to mere language (``Whatever we cannot speak about,
we should keep silent about''), or even all reality (``Reality is the names we
give to it'').\ftnt{Plausibility of such statements depends almost solely on the
  intended meaning of \wo{we}, and then of \wo{I}.  Recall the ambiguities of
  objectivity-subjectivity from section \ref{sub:objsubj}, in particular,
  \refp{civObjSubj}.} Most abstractly, it amounts to reducing identity to the
\co{actually} discernible criteria of identity.

% This may seem to come closer to our view but, as a matter of fact, is even more
% remote from it.

The problem with this, like with most other principles, is to determine the ways
and limits of its application. The crucial question concerns what is allowed as
grounds for 
discernibility.  What counts as the properties to be considered when deciding
indiscernibility?  One will use \co{actual} observations but, of course, not
exactly, because no two observations, made at different points of time, are
exactly the same.  If we include even time, then we are left with pure
\co{immediacies}, as \co{dissociated} from each other as the monads themselves.
If we allow difference in time, then the question arises: how do we
determine that the two appearances have the same -- that is, identical -- value
of all the relevant properties?  In particular, among all properties there is
the property of \thi{being equal to $a$} and so such a definition is
circular.\ftnt{Recently, \citeauthor*{Impred},\kilde{p.8} argued that such a
  definition is only impredicative and not viciously circular, but we leave it
  to the concept analysts to decide if such fine distinctions do us any
  service.} So Leibniz wanted to limit the principle to \thi{substances} and
comparisons to the values of their properties, others started to distinguish
intrinsic and extrinsic, pure and impure properties, the linguistic bias would
like to include everything -- and only! -- that can be expressed in (a
particular) language, etc., etc.

In a sense, but only in a sense, our philosophy of \co{distinctions} and
\co{distinguishing} goes along with the principle: whatever is not
\co{distinguished}, remains \co{one} and the same.  The difference lies in that
we take \co{distinction} as the primitive notion. It comprises much more than
any \co{actual} differences, whether of linguistic, mental, physical or whatever
character.  \co{Distinctions} need not have any mental character: most
\co{distinctions}, even though relative to \co{me}, or rather to a form of
\co{existence}, are not made by \co{me} (nor you), they are just encountered and
their limits are \co{actual objects}.\noo{Thus, we can claim complete realism with
respect to the language, mind, or similar ideal entities -- their ideality does
not create any \co{actual} identities.} The relative, yet \co{transcendent}
character of many \co{distinctions} sets us apart from any understanding of the
principle which usually appeals to empiricism by reducing the discernibility to
the \co{actually} observable \co{distinctions}.


Moreover, there is no designated set of \co{distinctions} which, necessarily and
sufficiently, determine identity. In one context color, material, origin\ldots
may count, in another none of those may be relevant.  Two identical, but
numerically distinct, ships can be legitimately considered the same by a captain
who needs only one for the travel and views both equally fit for it.  (For him,
the situation is entirely the same as it would be with only one ship available.)
Claiming, on the other hand, that every possible distinction is relevant makes
the two immediately distinct. Our point concerning the relativity is that there
is no metaphysical set of criteria which could be used in every context for
determining identity or discernibility.  There is no meta-level for the
\co{distinctions} which could provide such criteria -- there are only sub-levels
of more and more specific classes of \co{distinctions}.  Eventually, \co{actual}
things are but limits of \co{distinctions}. As there is no meta-level for the
determination of such limits, this actually suggests the fundamental and
primitive, that is preceding any criteria, character of identity.

There are two aspects of identity which tend to determine two opposite camps
(perhaps, of Fregeans and Quineans, respectively): either identity is a
primitive notion, irreducible to others, or else it is in fact reducible to some
criteria. The latter comes in many variations, namely, variations over identity
of indiscernibles. We agree with it in so far as identity is a limit of
\co{distinctions}. In this sense, it is relative not only to a distinguishing
\co{existence} but also to the distinctions taken into account. A thing is,
after all, just the other side of the totality of all it excludes, it is, so to
speak, the \thi{complement of its outside}, \ref{sub:substances}. However, the
border between the \thi{inside} and \thi{outside}, the mere {\em fact} of it
being drawn at all, is not determined by any universal criteria. It is a {\em
  limit} of \co{distinctions} and as such a primitive event (usually, not an
\co{act}), for drawing a limit is not reducible to merely drawing the
\co{distinctions}. In this sense, identity is not merely a conceptual device for
simplifying thought, as Avenarius or Mach would have it. It is an \la{a priori}
condition of thought, a reflection of the limitedness of \co{actuality}, which
can not be reduced not only to any \co{actual} criteria, but not even to any
other notions. It is not an accident of \co{reflective} thought but its very
presupposition, an inseparable accomplice of \co{actual distinctions}.


\noo{ \pa I \co{re-cognise} this cup on the table as the same I left there just
  a moment ago.  Why?  There are no relevant \co{distinctions} to be made
  between the two. I may indeed wonder if the two are \thi{really} the same, but
  if want to wonder about that, I will never stop wondering -- the world does
  not consist of \thi{substances} but of \co{distinctions} and their limits. Say
  that there is a tree on the corner just outside my house.  One day, when I
  wake up, I see the same tree, only that it is 2 meters to the left where the
  tree was yesterday.  Is it the same or not?  What difference does it make?  I
  would say none, and so it is the same tree -- the change which occurred is,
  neither \thi{essential} nor \thi{accidental}, but simply irrelevant.  Of
  course, here is more, since we would say that it is the same tree, even if it
  were removed, put on a truck and driven out of town.  Here we could
  \co{distinguish} the two, but we do not.  Well, a tree is an \co{object} which
  so nicely fits into the \hoa, that we like to fix its identity.  To determine
  that the two are really one and the same tree, we would try to establish an
  unbroken continuity through time, but we do not have need (nor time) to do
  such exercises, and it is much more practical just to fix the
  \co{distinctions} at some point and not to \co{distinguish} things which need
  not be \co{distinguished}.
 }

\wtsep{relative id}

\pa
What certainly does not apply is the opposite principle -- indiscernibility
of identicals. Two different trees, the one with all branches and the one (the
same!)  with two branches cut off {\em may} be considered the same, and so {\em
  may} be Theseus' ship before and after renovation.  {\em We} fix identity at
some point and having fixed it, grave reasons may be needed to change it.  In
most common cases, this fixation is merely a result of unconscious process, even
of bodily functions, of perception, sensation or, as it may be, of
pre-\co{reflective} \co{distinctions}, perhaps, drawn in the early stages of
\co{experience}.

\noo{ Identity is not any fundamental, metaphysical principle but, on the
  contrary, a thoroughly pragmatic one -- not in the sense that it is what I,
  you or we want it to be; not in the sense that its possible purposefulness may
  override any kind of already given facts and reasons; but only in the sense
  that its fixation is entirely relative to our, or other beings',
  \co{experience}.  Consequently, having only a relative and pragmatic goal, the
  principle of identity of indiscernibles is not any absolute, universal
  principle, either.
  
  Summarising: the principle is not any epistemological tool which can be
  \co{actually} used in all cases, simply because the \co{distinctions} which
  may be relevant need not be of \co{actually} observable character. On the
  other hand, the observable differences may be irrelevant because the actually
  established identity chooses to ignore them.  }


\label{pa:idwithresp} There is a yet more specific variant of the principle
which also tries to negate indiscernibility of identicals, or rather, limit the
scope of its legitimacy. 
One says sometimes that identity is not an absolute notion, but something which
is relative -- $X$ and $Y$ may be identical only \thi{with respect to} some
$Z$.\ftnt{Recalling first footnote~\ref{ftnt:idwithresp}, here we can observe
  that already Hobbes presents this as a solution to the problems of 
  identity.  \citef{But we must first consider by what name anything is called,
    when we inquire concerning the {\em identity} of it. For it is one thing to
    ask concerning Socrates, whether he be the same man, and another to ask
    whether he be the same body; for his body, when he is old, cannot be the
    same it was when he was an infant, by reason of difference in magnitude; yet
    nevertheless he may be the same man.}{HobbesCor}{II:11.7.} Notice, that
  \wo{name} here denotes only the aspect \thi{with respect to} which identity is
  established -- it is not the \co{sign} which merely \co{represents} the
  identity.

This may have different emphasis but reminds also of the \thi{indifferent-realism} at
which William of Champeaux arrived under the critique from Abelard. Peter and
Paul are said to be \thi{indifferently} men or possess humanity \la{secundum
  indifferentiam} in that they are equally rational, mortal, etc. But they are
still two distinct men. The question how they can be two different men if,
\la{qua} men, they are (indifferently) the same was hard enough to drive master
William out of this position.} Unlike in our case, here 
$Z$ is not any \co{existence}\noo{(or limit of \co{distinctions} drawn by an
\co{existence})} but some other element of \co{objective} and already
differentiated reality.  
The old and the new ships are identical with respect to their function, their
owner, their legal status, etc., but not with respect to the material from which
they are made.  Sounds reasonable.  \thi{With respect to\ldots} is but a way of
pointing to some dimension along which no further \co{distinctions} have been
made.  However, if we asked about what exactly constitutes the identity along
this particular dimension, we would probably run into the similar, if not
entirely the same problems as the traditional notion.

This eventual reliance on the absolute identity (observed also in
\citeauthor{McGinn}), is reflected in the formulations of relative identity
which tend to involve absolutely distinct or identical objects. Thus young and
old Oscar are claimed to be the same dog but distinct -- absolutely distinct --
things, or as one says, logical objects.  The status of logical objects remains
a bit unclear, though they seem to be very similar to the limits of
\co{distinctions} -- viewed not as \co{actual} such limits, but as the further
indistinguishable residual points.\ftnt{\wo{Indistinguishable} is used here not for
  mutual indiscernibility of $a$ and $b$ but for the absence of further
  differentiation of a given limit $a$.}  \co{Distinctions} which can influence
identity need not have any mental or linguistic character.  The logical objects,
on the other hand, remind about mentally conditioned limits of distinctions
which are relative to the expressive power of the applied language.  But one is
anxious to emphasize that it is not what they are. On the contrary, they are
rather like the \thi{substances} with their independent, absolutely distinct,
being. They are not only further undistinguished but indistinguishable. The need
for such reservations arises from identifying (in often technical ways)
distinguishability with the expressive power of a language. Hence to escape
relativity to the language, one must endow the ideal entities with their own
being. Thus a dog Oscar and the same dog, only thought without one hair, are the
same dogs but distinct -- absolutely distinct -- logical objects.\ftnt{The
  absolute identity appears in a similar disguise in the attempt to handle the
  Hobbes' variant of the Theseus' ship problem. Here not only the old ship $O$
  gets replaced all the planks (becoming, perhaps a new ship, $N_1$) but also,
  at the same time, an identical copy $N_2$ of the old ship $O$ is built from
  the planks removed from $O$.  Both $N_1$ and $N_2$ seem to have equal claims
  to be the same as the original $O$ but this, admitting transitivity of
  equality, would make also $N_1=N_2$. So one tries to avoid identification
  $N_1=O$ by postulating identity of $X$ and $Y$ with respect to $Z$ if both $X$
  and $Y$ differ from $Z$ by at most one single part (or 5, or 20 parts).
  Assuming that one knows what a part is (a part of a part of $X$ is probably a
  part of $X$, but is an atom of a part of $X$ also a part of $X$?  --
  merologists may argue), what is the $Z$ in this equation? Trying to break the
  transitivity of identity across gradual replacement of the planks, $Z$ appears
  as an ideal measure -- an ideal, perhaps the original ship $O$ retained in the
  memory -- against which other instances are measured. It is thus only the
  absolute identity of $Z$\noo{(which seems to acquire some characteristics of
    Platonic idea)} and a sufficient proximity, similarity to it which make up
  the whole. If this sounds plausible, let us imagine two (physically and
  completely) distinct things, $T1\not=T2$, each composed of two parts
  $T1=P1_{T1}+P2_{T1}$ and $T2=P1_{T2}+P2_{T2}$, such that for both
  $i\in\{1,2\}:Pi_{T1}$ can be replaced with $Pi_{T2}$. Using $P1_{T2}+P2_{T1}$
  as the comparator $Z$ we then obtain

  \(T1=P1_{T1}+P2_{T1}=P1_{T2}+P2_{T1}=P1_{T2}+P2_{T2}=T2\)\\
which does not look particularly plausible. One can keep adjusting the
formulations and multiply the special side-conditions, but we will stop here.}

%\newp

The logical objects seem indispensable for the \thi{relative identity theory} to
be itself. Without them it would be hard to distinguish it from the identity of
indiscernibles, for then identity of two objects could amount only to the
objects being the same $Z$ for every possible $Z$.  But as long as such
\thi{absolutely identical logical objects} and the identity along the designated
dimension remain, the theory can, perhaps, account for much of the common-sense
usage of the {\em term and statements} of identity, but it does not seem to
offer any significant account of identity as such -- it only presses the
question one step further.  It helps little to summon {\em the proper} form of
an identity statement as only \thi{$X$ is the same $F$ as $Y$}, when the
statements which we, in fact, make of the form \thi{$X$ is the same as $Y$} make
perfect sense (e.g., in mathematics, about people, about \thi{Morning Star} and
\thi{Evening Star}.  Besides, admitting logical objects, there seems to be no
difference between the statement \wo{$X$ is the same logical object as $Y$},
\wo{$X$ is absolutely the same as $Y$} and simply \wo{$X$ is the same as $Y$}.)
Although in many cases of the statements of identity of things we might agree
with one or another variant of the relative identity statements, we take the
qualification \wo{one or another} quite seriously. For any set of criteria one
can produce a counter-example forcing an adjustment of any specific theory in
one or another way.

Identity (at least identity of \co{actual} things) is a completely relative
notion, it is not any metaphysical absolute but an accident of an \co{existence}
which
terminates \co{distinctions} at this, rather than that boundary. %\ftnt{The
%   ships $O,N_1$ and $N_2$ can be legitimately considered the same by a captain
%   who needs only one for the travel and views all three equally fit for it.
%   (Notice that for him, the situation is entirely the same as it would be with
%   only one ship present). Claiming that every possible distinction is relevant
%   makes the three immediately distinct.}
The only candidate to some metaphysical pretensions might be {\em the fact} of
terminating distinctions in such a way, the primitive and irreducible (even
though relative) character of such limits.\noo{  In any case, no 
universally valid criteria of identity are to be expected.} 

\pa A final word on the relativity of identity. According to
\refp{th:identity}, what is identical for me need not be for you, what I confuse
as one and the same thing, you may be able to differentiate as two.

Emphasizing relativity of identity to \co{existence}, we should keep in mind
that this is very different from saying that these \co{existences} have some
determining or constitutive power over identities. Identities, like
\co{distinctions}, are for the most {\em found} and not created (produced,
posited, generated, projected) by the \co{existence}. Relativity need not have
anything to do with \thi{subjectivity}, for being found it retains the element
of being \co{founded}, which disappears only in the most \co{dissociated
  experiences}.

Furthermore, saying \wo{relative to \co{existence}}, we often (and in the
current context almost constantly) mean relativity to a {\em form of}
\co{existence}.  Nobody decides what he wants to be identical and, moreover,
what appears so will typically appear so to most (if not all) \co{existences} of
the same form -- some things are equally identical to many. We will all agree on
the identity of this table here, or that tree over there. Just as there are
pragmatic, and this involves also natural, reasons for drawing some
\co{distinctions}, there are similar reasons for terminating them. The sensuous
apparatus of humans will, under normal circumstances, deem some things identical
almost irrespectively of who \co{actually} is involved in the situation. But as
the concerned contents become more \co{vague}, less prone to the narrow look of
\co{actuality}, the differences can become more significant. Is the feeling of
joy I have now, the same as the one I had yesterday? Is the love I experience
the same as that experienced by the one I love? Is the city in which I live the
same city in which my neighbour lives? Wait! Here it is obviously the same city.
Well, yes, but what constitutes its identity? Where does it end and where does
it begin? I always counted this particular suburb as a part of the city, while
for my neighbour it was already outside. As with the \co{distinctions} in
general, the city does not have any \thi{essence}, any sharp boundary, although
it seems to have some kernel, something which makes it {\em this} city. All
people may agree on the presence of such a kernel without, however, agreeing on
the precise boundaries -- they all may terminate the \co{distinctions} at
slightly different points. \citet{There are no sharp divisions of 
  reality.}{IdealistView}{VI:4} Thus, although talking
about the same city, they will \co{experience} and associate with it slightly
different \thi{essences}.

\pa This pragmatic relativity of identity is about as far as the possible
analogy with the \thi{relative identity theory}, as well as Locke and empiricism
in general, goes. For we have no atoms, no logical objects, no basic \thi{ideas},
\thi{perceptions}, \thi{impressions}, nor \thi{substances}. We do not share the
dream to differentiate everything which can possibly be differentiated and
then, having thus obtained the \thi{absolute atoms}, to reconstruct the reality from
them.

% Although empiricist must, for practical reasons, stop the first part of the
% project, he always thinks that it could be continued and attempts to carry out
% the second part.

Ireno Funes, the hero of a short story by Borges about perfect memory and
insomnia, would be a dream-hero of empiricism, nominalism and, so it seems, of
the relative identity.  \citet{Not only was it difficult for him to see that the
  generic symbol `dog' took in all the dissimilar individuals of all shapes and
  sizes, it irritated him that the `dog' of three-fourteen in the afternoon,
  seen in profile, should be indicated by the same noun as the dog of
  three-fifteen, seen frontally.  [\ldots] He was the solitary, lucid spectator
  of a multiform, momentaneous, and almost unbearably precise world.}{Funes}{}
The perfect \co{precision} of minute distinctions does not disclose any eventual
atoms but, on the contrary, dissolves all identities.  The search for such
atoms, for the most minute, ultimately self-identical elements can always be
carried on further.  It stops, from the point of view of metaphysics of
principles and sufficient reasons, at a completely arbitrary point; it stops at
some point only because for one reason or another, typically unconscious,
sometimes confused, but usually a good reason of avoiding \thi{unbearable
  precision}, we stop to \co{distinguish}. \citet{My {\em life} consists in my
  being content to accept many things.}{Certain}{344. A more
  existential expression of the search for the ultimate foundation and despair over
  its eternal invisibility is decadent boredom trying to entertain itself with
  merely aesthetic variations. Count des Esseintes in \citeauthor*{aRebours}, is a
  good example, trying to fill the bottomless emptiness with more and more
  subtle distinctions which only deepen the sense of emptiness. The distinctions
  of advertisement industry and ever new fashions, artificial needs craving only
  for more novelties, reflect on the social scale the despair over the lack of
  substance, mistakenly identified with the \thi{objective substances}.}
Just like the ability to handle a wide variety of distinctions tells 
us something about one's intelligence, so the points at which one stops
distinctions and rests satisfied tell us something about what kind of person one
is. \citet{What people accept as a justification -- shews how they think and
  live.}{WittPI}{I:325.\noo{ A good philosopher says only what he {\em can} say --
  and keeps silent. Others keep analysing, distinguishing, refining, repeating
  and \ldots filling wastebaskets.}}

\sep
%
Returning to the thesis from \refp{th:identity}, identity presents a mysterious
problem for \co{reflection} but only as far as time is concerned. One may wonder
\wo{How do I know that this cup today and the one yesterday are the same?}, not
\wo{How do I know that this cup here is the same?} nor \wo{How do I know that
  this cup here and the one over there are (not) the same?}
Since time and space are but \equi\ \co{aspects} arising from
\co{spatio-temporality}, a brief comment might be in place.

\pa If you see a building so high that you can not simultaneously see both its top and
its bottom, you do not wonder. Perhaps you should? If you can never see it in its
totality (say, it is surrounded by other buildings which make it impossible to
see it whole from a distance), if you can never perceive it in a unity of one
\co{act}, isn't the problem the same as with the same cup today and yesterday?
Recalling one of the multiple senses of unity listed by Aristotle in
\btit{Metaphysics}, V:6, one will immediately point to the possibility of a
continuous perception of the whole building from top to bottom; the continuity
which does not obtain with the cup yesterday and today. 
Let us first refine the claim \refp{th:identity}: 
%
\thesisnonr{Identity is
a limit of \co{distinctions} \co{represented} by a \co{sign}, that is,
contrived to the \co{actuality} of a single
\co{act}.}\label{th:idsign}

This \co{sign} will be, typically, an abstract \co{sign}, not merely
an aspect of the thing, but, for instance, a word which can be made
fully \co{actual} even though the thing itself can not.  It is not the
\co{sign} which establishes identity -- it only \co{represents} it. 
It \co{represents} the \co{actual} limit of \co{distinctions} within
the \hoa. This horizon sets the limit on the possible \co{experiences} of
identity -- identity is always \co{an experience} consummated fully
within this horizon where the \co{non-actual} aspects of \thi{the
identical} appear through the \co{sign}.

%\pa The continuity of relevance to identity is always continuity in time -- t
The continuous perception of the building, although traversing
a region of space, is important because it does not create any
\thi{gaps} in time.  For the \co{actual reflection} this continuity
amounts to the unbroken presence of the \co{sign} and (aspects of) its
perceptual correlate.  You see the different stories of the building but
the \co{sign}, \thi{this building}, is kept continuously (even if not
\co{reflectively}) as your sight moves 
along the walls.  Space, in any case its region which is accessible
to the observation, is exactly the field where such a perceptual
continuity is possible.
%In a manner slightly more
%general than before (and not entirely precise), we could say that it
%is the field of simultaneity
In the moment a break occurs, when you enter a shop and go out again or turn
around, the break of continuity is still a temporal break -- just as when
you close your eyes, the building disappears from your sight {\em for some
  time}. After such a break you still see the same building -- there is simply
not the slightest reason, in fact, not the slightest possibility of imagining
the two to be different. 

The continuity of space (not the \co{objective}, but experiential space) has the
same character of endowing the continuous region with identity; any \thi{break}
represents a \co{distinction} separating one thing from another. We could say,
just as the perceptual field is the scope of simultaneity, so the possibility of
continuous perception is circumscribed as the ideal limit of simultaneity of
\co{distinct} things in space. The identity of things seen only in their
spatial, simultaneous dimension is unproblematic -- it is the self-identity of
\co{immediately} given \co{distinctions} and their limits. The \thi{gaps} in
space are the \co{distinctions} we \co{recognise} as separating different things. The
\thi{gaps} in time have another import -- they reflect only the
\co{dissociation} of \co{experience} into \co{actualities}.  As long as we can
maintain the continuity across time, the identity of a given object might remain
as unproblematic as in its spatial \co{immediacy}.  The problem for
\co{reflection} is that such a continuity does not, in general, obtain for
things which it would like to consider the same. The continuity between the cup
yesterday and today is broken. And \la{vice versa}, whenever such an ontic
continuity is broken, it involves time.  Trivially: time is the dimension along
which things may cease to be the same. And a bit less trivially: identity is
{\em the means} \co{reflection} employs to keep them the same across
\co{dissociated actualities}. The question whether my sofa today is really the
same as the one yesterday is asking about the reasons I might find to conclude
they are not. \wo{The reasons I might have} means simply the \co{distinctions}
it might be possible to draw between the two. There are none (none of relevance,
at least), and the sofa
\herenow\ remains connected to (the same as) the sofa which has receded into the
past \co{actuality} of yesterday -- this \thi{gap} in time has been covered up.

\noo{But one will object, because the identity of the sofa has nothing to do
  with the reasons I might have to consider it one or two. Of course, but as
  long as \co{I} cannot see any reasons, as long as \co{I} cannot distinguish
  the two, \co{I} am bound to consider it the same.}

\pa
Thus, identity is a truly \co{transcendent} relation when viewed exclusively
from the perspective of mere \co{actuality} -- it represents a noumenal $x$
which lies beyond every \co{actual} appearance. Some such
$x$'s can arise as results of 
\co{reflective} construction. But most common and natural ones precede it and
are solidified as limits of \co{distinctions} prior to conscious, let alone 
attentive efforts.\noo{And yet, every \co{actual} appearance is \co{experienced}
  as but an aspect, a side, a property of this $x$. For $x$ has been solidified
  as a limit of some \co{distinctions}, if you allow me, we have created $x$ by
  \co{distinguishing} it from all other things and by fixing its
  \co{distinctions}.}  In either case, the \co{experience dissociated} into
separate \co{actualities}, temporality split into a mere succession of
\thi{nows}, call for an account of the \co{experienced} continuity.
%$x$ provides such an account.

\thesisnonr{Identity is a \co{reflection} -- a \co{representation} --
of the \co{experienced} continuity; it is the \co{trace} which, \thi{filling the
gaps} of \co{objective} time, {makes up} for its broken continuity which is no
longer \co{actually experienced} through the \co{dissociated} nows.}
\label{th:idcont}

\co{Actual} identity, the \thi{repetition as recurrence}, the equality of $a$
and $b$, is an \co{aspect} of conscious \co{experience} which lives time through
\co{actualities}. But it becomes {\em a problem} for \co{reflection} which,
taking its \co{concepts} from the assumed obviousness of \co{immediacy}, tries
to account by their means for the \co{unity} which \co{transcends} every
\co{actuality}. The identity of \co{objects}, things and ideas is, indeed, a
very fragile \co{aspect} of their \co{experience}.  Just like analysis can
dissolve every issue, it can likewise dissolve every identity, it can make
\co{reflection} disregard it as \thi{unreal} because not given \co{objectively}.
And from the threat of \thi{unreality} there emerge ghosts of \thi{ideal}
entities -- ideal $x$ of which $a$ and $b$ are only different appearances and
whose recurring appearances could be, in principle, repeated \la{ad infinitum}.
We have earlier seen universals which could be viewed from exactly the
same angle: accounting for the recurring repetitions of the same $x$. Ideal
limits, like regulative ideas, \co{posited} in the unthinkable infinity of time,
try functioning as the norms or else the ultimate witnesses to the absolute
truth which will be disclosed only in the eventual fullness of times. Of course,
idealities deserve to be dissolved, but the crucial difference is whether an
ideal is merely \co{posited}, or whether it is also \co{reflecting} some deeper
reality.  A total dissolution of everything which stretches beyond the limits of
ever narrower \co{immediacy} is only the ultimate expression of the
\co{objectivistic illusion}, of the thirst for the \thi{givens}, and the despair
over their absence. A dissolved self cannot avoid dissolving the encountered
things though this disease, like many others, bears the seeds of a possible
revival.

% The more intense the \co{reflective dissociation}, the more problematic, and
% also the more important identity becomes.

\noo{
\pa \co{Externality} of an \co{immediate object} is the paradigmatic structure
of identity, and most  examples and notions of identity
are based on the \thi{substantiality} of \co{objects}.  Although
\co{reflection} tends to ascribe the \co{object} noumenal, simple
self-identity, it is by no means a simple entity.  It is a limit of
\co{reflective dissociations} brought to an end. 
Since these \co{distinctions} can always be carried further, there is
no wonder that one has always had tremendous difficulties with
displaying assumed \thi{basic objects} or else \thi{essences} of
things.  \co{Object}'s identity appears noumenal because it is founded
in the pre-\co{reflective} \co{experience} on the basis of earlier
\co{distinctions} and \co{recognitions} which effected just this, and
not another, \co{cut} from the background.  Its assumed
independence, the effect of \co{representation}, \co{double
separation} from the background is an \co{aspect} of its identity. 
%Something is an \co{object} only to
%the extent it has been so \co{externalised}. The limit of 
%\co{distinctions}, so \co{externalised}, 
}

\ad{\mine Totality of visibles: $a_1+a_2+a_3+\ldots = x$}\label{minId}
Consideration of equality, of identity at the level of \co{actuality}, took some
space because 
it is where it belongs. Equality binds \co{dissociated actualities} together.
It appears always (unless reduced to the empty $x=x$ of a given \co{object}, of
an \co{immediate reflection that it is}) as a \co{transcendent} relation,
stretching beyond the \co{actualities} scattered across different points of time
and bringing them together.  But it {\em can} do it, and it is {\em needed} to
do it, only because our being is not exhausted by the \co{dissociated
  actualities} but has itself \co{unity} which \co{transcends} every
\co{actuality}.

Identity may be an \co{object} of \co{an experience}, as happens every time one
realises that $a$ is {\em the same as} $b$. This, inadvertently, requires both
$a$ and $b$ to be themselves \co{actual}, or in any case thought as such. Thus,
every statement of identity requires, or as the case may be reduces, its object
to be at the level of \co{actuality}.  There are however cases when such a reduction
is hard, if at all possible to imagine.  What about the infatuation which I felt
a week ago and I am still feeling? Is it the same or not? It is the same, it
concerns the same person, it has some continuity.  But a week ago it had a
slightly different flavour, I did not then see this person to have something of
vanity in herself, while I see it now. But it is still the same because\ldots 
Because I \co{re-cognise} the feeling and give it a name? 
The tendency to consider it the same is stronger than that, but only as long as it
retains enough of the similarity. With time it may simply -- and continuously! --
change, perhaps even into 
disgust, pity, repulsion. Then one
may find as many reasons to call it the same as reasons to call it different. It
is the same infatuation turned into repulsion, the same fascination turned into
boredom. Or was it, perhaps, from the start repulsion disguised as infatuation,
boredom disguised as fascination? Whether the same or different, it
 does not matter because what matters in this case is what one is feeling
and thinking, perhaps, also what one felt and thought before, but not if this
thought or feeling now is equal to some other.

\pa
The limiting case of \co{non-actualities} which, nevertheless, seem to require
an answer to the question about their identity is that of a person and the
world. The former was treated in \ref{sec:I} and
\ref{sec:Self}, so here we 
only comment briefly on the latter. But everything said about the \co{totality}
of the world applies equally to personal identity viewed from the level of
\co{mineness}.

% These two aspects -- that identity is \co{founded} in the unity of a being
% \co{transcending actuality} and, at the same time, it reduces its objects to the
% level of \co{actuality} -- ... brings in the questions about personal identity
% and the unity of the world. Seen only from this level they have essentially the
% same form. We will elaborate on personal unity in the following section, so here
% we concentrate on the world only...

%\subsub{Unity and totality}
\co{Complexes} are \co{reflectively dissociated} into, or \co{reflectively}
built from, their \thi{constituent parts}.  Most thinking is spent on construction
of such \co{complexes} where, as Pierce says, \citet{unity is nothing but {\em
    consistency}.}{PierceFourIncap}{p.71. We give the word
\wo{unity} a different meaning, but for the moment let us keep it this way.}
Sure, a \co{complex} may be a mere gathering of unrelated parts, like a 
\thi{composite substance}, a heap of stones, but it is usually some sense of
aim and purpose, or just consistency, which makes us consider this rather than
that as a \co{complex}, as a well-connected \co{complex} of parts, a table, a
car, a heap of stones\ldots We certainly won't attempt to improve on Pierce's
consistency trying to determine in what it possibly might consist.  We won't
because the possibilities may be innumerable.

Although \co{reflection} would like to arrange everything as a \co{complex}, a
\co{complex} of \co{complexes}, etc., in some cases the principle of
organisation can be completely unclear. One takes then often refuge to a
\co{posited} limit -- applying some principle to the \co{actual} elements beyond
the limit of \co{actual} possibility, one claims to obtain some ideal limit.
Quoting the limit construction one may easily happen to be very proud from the
analogy to the converging series in mathematical analysis. Unfortunately, the
image of countability and convergence is presupposed by all such limit
\thi{constructions} in philosophy. But this means that constructions do not
generate anything but only utilise the assumption of a prior unity.  All such
limits, regulative ideas, posited ideals are but attempts to regain the
primordial unity in which they are \co{founded} and without which they could not
even be imagined.\ftnt{This does not mean that various ideal limits so
  \co{posited} are always concretely \co{founded} in the respective unities.
  There is only one \co{unity}, and the \co{totality} founded in it has two
  respective aspects: of \co{myself} and of the world. But any particular limit
  is always posited under some principle which is merely applied to the already
  assumed totality.  The idea of \co{totality} is founded in the \co{unity}, the
  principle so applied -- furnishing the assumed consistency to the
  \co{totality} -- may be almost anything one can imagine. } They yield
\co{totalities} which, at best, are only reflected by the respective limit
constructions.  A \co{totality} is but a gathering of its parts, is {\em one}
only because the parts have been \co{posited} as a collection and are considered
as one. \co{Totalities} are not perfect unities but only \citet{illuminations of
  combinations}{Proclus}{\para64} or, to put it directly, just heaps of stones.
This is in fact as much as \co{actual} thinking is capable of making out of the
world, or personal identity -- the sum total, the \co{totality} of its
\co{visible} elements. Whether it should be called empiricism or pantheism
depends only on how much divine character is ascribed to the \co{totality} of
the world.

\noo{\pa The concept (if any) of the \thi{objective}, that is, \thi{external world}
has quite a schizophrenic character. On the one hand, it must be truly
\thi{external}, completely independent from us, in order to be solidly
\thi{objective}. On the other hand, it is to provide the truth and corrective
for our life and thinking and so, we need an access to it. No wonder that one
has to invent \thi{things in themselves}, transcendental subjects and other
monsters to make the ends meet.  The more sober ones stay aloof from the whole
problem.  Like most philosophical problems, it has been solved -- not because
anybody offered a satisfying solution, but because everybody got bored and
tired.  \wo{Sure, the world is there -- let's talk about something more
  interesting.} Sure, it is. But what is it that \thi{is there}, what is
\thi{this world}?}

\pa Saying \wo{external world} one usually identifies it with \co{externality}
of \co{objects}. How do we think, what do we mean by \wo{the external world}?
Easy, look at this table in front of you -- obviously, it is \thi{there}, it is
\co{objective}, \co{external}. The \thi{external world} is just the \co{totality} of
such external things. This is how much we are able to make out of the assumed
\thi{concept} of externality.

But, do you remember the \co{objectivistic illusion}
(I:\ref{objectivisticillusion})?  What \co{totality}?  There is no \co{totality}
of {\em all} things, there are just \co{external objects}, plenty of them, but
that's it.  And certainly, the \co{totality} of {\em all} things is
inaccessible.  It is never experienced, never given, nothing like that ever
confronts us in {\em any} \co{actual experience}; there is no \co{totality} of
things; things, \co{objects} do not sum up to anything, least of all to any
world.  At best, this world appears as their ideal, that is, impossible and
inaccessible limit.

The meaning of asymmetry of Being (I:\ref{asymm}) is that lower things never sum
up to give something higher -- higher level is inaccessible to the categories of
the lower ones. \co{Above} the \co{totality} of \co{objects}, there is the
\co{unity}, we might say, of the world, or perhaps, of the horizon of
\co{experience}, which precedes \co{experiences} of individual \co{objects}; the
\co{unity} of which the \co{posited totality} is but an imperfect \co{sign}.
Bracketing this \co{unity}, this \co{aspect} of all \co{experiences}, and then
trying to account for the \thi{external world} or, perhaps, for its
\thi{externality}, one ends up accounting for the \co{externality} of an
\co{object}, while trying to account for \thi{the world}, one ends up accounting
for nothing but only \co{positing} the ideal \co{totality} of things.

%But what is an \co{external object} other than an \co{object} of \co{an
%  experience}?
  

\co{Externality} of \co{objects} is, unlike the \thi{externality of the world},
\co{an experience}.  A thing, an \co{object} is not just \co{external} -- it is
{\em experienced} as such.  Things are independent, \co{external} and, at the
same time, \co{experienced}.  To be independent and \co{external}, they do not
have to be inaccessible -- they are \co{objective} in virtue of the very
relativity to our experiences.

But \thi{the world}, the \thi{external world}? By its very nature, it is
impossible to have \co{an experience} of it -- as a \co{totality}, it always
extends beyond any \co{actual experience}. So, perhaps, every experience is only
a partial experience of the world? Perhaps\ldots But it would seem that a
partial experience of $X$ is not an experience of $X$ but of its part, and if
one insists on it being a part {\em of} $X$, then $X$ must come from elsewhere.
Every \co{actual experience} is \co{an experience} {\em of} its \co{object} or
situation, but not {\em of} the world. And yet, the world, the \thi{objective
  world} stays in the background, the \thi{unified world} haunts every
\co{actuality} and is \co{experienced} underneath every particular
experience\ldots

% But then why do we need to speak about such {\em a} world, about such {\em a}
% totality?


\ad{Invisible unity: $\bullet$}\label{pa:perspectives} As said in
\ref{sec:thisWorld}, the \thi{wholeness}, the \co{unity} of the world is not
constituted in terms of \co{actualities}. We feel compelled to accept it as a
\co{totality} (and this always means, {\em one} \co{totality}) because, lack of
any unifying principle notwithstanding, its \co{unity} is \co{experienced} as
another pole of the \co{unity} of \co{myself}.  We would not get the idea of an
absolute \co{totality} of the whole world (nor of \co{ourselves}), if no
\co{unity} were \co{experienced} prior to it.  \citet{Every whole composed of
  parts participates in a unity preceding these parts.}{Proclus}{\para69
  \citef{Once there is any manifold, there must be a precedent
    unity.}{Plotinus}{V:6.3} We should, however, be careful with embracing all
  variants of neo-Platonic units too enthusiastically. The \thi{wholes} and
  \thi{unities} have much more conceptual flavor, and are not distinguished from
  the \co{immediate} identity. In fact, the contentless self-identity of the
  most immediate object is taken, here as elsewhere, as an epitome of unity.
  Even the justification of this very proposition ends with the possibility of
  the preceding units being just \thi{indecomposable atoms}, which must lie at the bottom
  of any division.}  Every whole participates in the \co{unity} which
\co{founds} its \thi{wholeness}.

% \co{Totality} of the world is \co{founded} in the \co{unity} of \co{existence},
% eventually, in the uniqueness of \co{self} and the \co{unity} of \co{one}.
%in \co{Self} which \co{transcends} every particular determination

But we have not seen any \co{unity}. So far, it might rather seem like there is
nothing \thi{really} identical, nothing possessing \co{absolute unity} which is
not merely relative to our ability to make, and suspend, \co{distinctions}.  Is
all that remains only \citet{a permanently tentative look?}{PontyPerc}{346-7}
Indeed, every \co{visible} determination of anything \co{invisible} is by its
very nature only tentative and approximate. So, is life a story in a search --
and that implicitly means, constantly failed search -- of a
narrator?\ftnt{\citeauthor*{RicSearch}} But \citet{thou wouldst not seek, if
  thou hadst not found.}{Pensees}{VII:553 [modified]}.

We definitely distinguish the question about the identity of the dog Oscar
before and after the loss of one of its hairs or, for that matter, about the
identity of my friend Paul before and after the accident which made him lame
and, on the other hand, the ship of Theseus being the same before and after
exchange of all the planks.  The problem of the possibility of knowing
particulars amounts indeed to the identity of indiscernibles. I can point at
this cup but all I know about it, except for its being \co{actually} given here,
is the same as what I know about the indiscernible cup standing in the cupboard.
Which of the two is here is entirely uninteresting and healthy \co{reflection}
will never (in normal situations) worry about such identities. The situation is
entirely different with human beings. I not only can point at Paul sitting next
to me; I actually know him, and knowing him is something infinitely more than
knowing his \thi{human essence}, than knowing him as being a man
\la{simpliciter}. Knowing Paul is as different from knowing Peter, as Paul is
different from Peter, and they are infinitely different (even though they are
twins). On the other hand, knowing one cup is exactly the same as knowing
another, though indiscernible, one.  We know individuals (that is, \co{existing}
individuals, not merely particulars) because they are not only limits of
\co{distinctions} which always can be refined further, but because they never
can be. To know a person transcends by far not only simple \co{conceptual}
constructions but also deepest psychological insights, for it amounts to a
\co{recognition}, beyond the character traits and psychological features, of the
unrepeatable uniqueness of this person -- not as an abstract property but as the
most \co{concrete} truth of fact.  In short, we distinguish between the
\co{unity} of \co{existence} and a mere identity of \co{actual} things.
\co{Totality} is an intermediate notion which arises whenever the desired
\thi{principle of unity}, establishing the whole of \co{actualities}, can not be
seen.

\co{Self} is the \co{confrontation} with the \co{one}, and its \co{unity} is but
the uniqueness of this \co{confrontation}.\ftnt{\wo{Unique} and \wo{unity}
  originate in the same Latin \wo{unus} -- one.} Each \co{birth} creates a new,
unique individual. At first it is only a \co{virtual} kernel from which this
individual will eventually develop in all \co{actual} manifestations.  It does
not contain any \thi{complete notion}, it does not contain all the future and
past \thi{contingencies} which this individual may encounter in life, and which
might be needed if the unity of a substance were constituted by its concept or
properties. Irrespectively of the conceptual indistinguishability from any other
\co{birth}, it is an event of the \co{absolute} beginning, emergence from
\co{nothingness}. As such, it establishes an ineradicable, numerical \co{unity},
\la{haecceitas} of this individual, which is his \co{origin}, the point beyond
thought and \co{experience}, where he touches \co{nothingness}.\ftnt{Thus we
  must finally admit the misuse of this Scotist term. According to Duns Scotus,
  every individual thing has the individuating entity which, so to speak,
  follows after and in addition to the being of its essence, \la{esse essentia},
  endowing it with the actual and individual existence, \la{esse existentia}. We
  do not worry so much about the identities of \co{actual} things, and we do not
  see so much difference between human \co{existence} and its \thi{essence} --
  the unique individuality constituted by the \co{confrontation} with the
  \co{one} can be equally identified with both, and \la{haecceitas} refers to
  this triple identification.} No \co{visible} criteria account for this
\co{unity}.  \citet{If we take wholly away all Consciousness of our Actions and
  Sensations, especially Pleasure and Pain, and the concernment that accompanies
  it, it will be hard to know wherein to place personal
  Identity.}{LockUnd}{II:I.11.\noo{One can easily find similar quotations in
    Hume, other empiricists, even James.}} It is not easy to imagine what
\thi{taking wholly away} might mean, but allowing that, it would be exactly the
place where to look for \thi{personal Identity}. \citet{[I]f we had nothing
  distinct in our perceptions, nothing heightened, or of a stronger flavour, so
  to speak, we would be in a permanent stupor. And this is the condition of the
  completely naked monad.}{Monad}{\para24}

\pa \co{Unity} is the \co{absolute} identity -- \co{absolute}, that is, 
trans-phenomenal, relative to \co{nothing}.  It is not a consistency of a
\co{totality}, neither is it the identity of \co{dissociated actualities}, but
their ultimate \co{foundation} -- the 
\co{virtuality} of the origin.
%
% from which any given variety emerges.\noo{Perhaps, we should follow Aristotle
%   and say, it is \gre{entelecheia}.[Psyche in Antiquity, I, p.69] But, as life
%   principle, it was introduced by Hans Driesch. For Aristotle, it was kind of
%   continuing-completeness, being-at-the-end, in analogy with \gre{endelecheia}:
%   continuity/persistence.}
%
It does not have \thi{parts} but \co{aspects}.  It is not an intrinsically
relative totality where the \co{actual} elements are put side-by-side and
connected by another element, a \co{visible} relation; it is a \nexus, a
\co{virtual} center, in which they have, as yet, not received their \co{actual}
determinations and in which they are inseparably woven together.  It is not
identity for the \co{aspects} are \co{virtually} distinct, although these
\co{distinctions} have not, as yet, reached the crisp delineations which they
will eventually obtain in the \co{actual experience} and
\co{reflection}.\ftnt{That the \co{virtuality} of the origin has not {\em as yet}
  been \co{precisely} separated indicates not only the temporal dimension -- the
  higher levels, the earlier hypostases remain \co{present} as the lower ones
  emerge.  The unities of various aspects, their \co{virtual} \nexuss\ are
  \co{experienced} thoroughly and intimately, even if \co{reflection} focuses
  its attention only on their \co{dissociated} \co{aspects}. Thus \co{unity} is
  a constant element of \co{experience}, the constant \co{rest} which looms in the
  background and for ever refuses to be drawn down to the categories of
  \co{actuality}.  }

According to Cusanus, \citet{there is only one essence of all, which is
  participated differently.  It is participated differently because [\ldots] two
  things cannot be perfectly alike and, consequently, participate one essence
  precisely and equally.}{DDI}{I:VII.48-49\noo{We, on the other hand, view all
    \co{existences} as participating (in any case, as being ontologically
    \co{founded}) precisely and equally in \co{one}.}} Hmmm\ldots What is the
ground and what the effect here?  Here is a clue: \citetib{The universe, as most
  perfect, has preceded all things in the order of nature, as it were, so that
  it could be each thing in each thing.}{DDI}{II:5.117. It should not be too
  difficult to draw close analogies to the unity of the Leibniz's system, his
  pre-established harmony, reflection of the whole universe in every monad, and
  the (resulting) principle of identity of indiscernibles.}
%
The universal, \thi{only one essence} is hardly anything obtained by identifying
indiscernibles. In \co{actual} terms it is obtained by identifying discernibles,
the incomprehensible \la{coincidentia oppositorum}. We do not need such a
\la{coincidentia}, because \co{actuality} is not our beginning.  The \co{one} is
that which precedes all possible discrimination; if you like, something always
assumed, never deduced. Its \co{unity} is not reducible to \co{actual}
observations, for it \co{founds} the unity of each level.  The \co{one} is the
\co{unity} of the \co{chaos}, \co{chaos} is the unity of \co{experience} and
\co{experience} is the unity of \co{experiences}.  Or more specifically,
\co{virtual signification} is the unity of \co{recognition} which, in turn,
\co{founds} the unity of \co{aspects} involved in \co{representation}; or
\co{simultaneity} is the unity of \co{spatio-temporality} and of \co{awareness}
which \co{found}, respectively, the unity of space, of time and of
\co{reflective consciousness}.  But these \co{founded} differences do not change
their \co{founding unity}.  Ultimately, every human \co{existence}
\thi{participates equally}, in the ontological order, the \thi{only \co{one}
  essence of all}. The difference is the difference of indiscernibles, is the
difference between one \co{absolute} beginning and others which all, beyond the
numerical difference, are indiscernible \co{unities}.


\pa The ultimate \co{unity} is the \co{one}: the \co{indistinct} is the ultimate
limit of all \co{distinctions} and of the very possibility of
\co{distinguishing}, remaining forever beyond their horizon. Repetition and 
\thi{recurrence} are not mere accidents; they provide an abstract,
that is \co{actual} and often \co{precise}, characterisation of all
our being -- \co{experiencing} the {repetitions} of the \co{one},
\co{recognising} \co{one} through Many, \co{experiencing} it
always in new ways, in new \co{actualities}, from new perspectives\ldots  The
\co{one} is not an \co{object} of any \co{actual experience}, all such
\co{experiences}, all \co{distinctions} are but manifestations of the
\co{one}, always under different forms, in different \co{actual}
clothes.  The \co{one} is \co{experienced} but only through, or under,
these variations -- it is \co{one} and the same, and yet,
\co{actually} always different, as one \co{actual experience} is
different from another.  The \co{one} is \co{experienced} only through
all the \co{distinctions} and thus it is the ultimate violation of
indiscernibility of identicals -- in \co{actual experiences} it is
thoroughly different, it is never given, and hence never given as the same, and
yet it is always itself, always identical.


Our being -- and our understanding, in particular -- is stretched between
these two limiting poles: on the one hand, the idealistic intuition of
everything being but a manifestation of \co{one} and the same and, on the other
hand, the fact which from the \co{reflective
  experience}'s standpoint is incontestable: that everything is a separate entity
related only, if at all, ideally to something else. It might be tempting to
construct a contradiction but the two do not contradict each other -- they only
express the extreme and complementary \co{aspects} of \co{existence} which
\co{experiences one} only through many.  The great challenge of \co{reflection}
starting from the \co{dissociated experiences} is ideal abstraction -- to
\co{re-construct} (parts of) this variety as manifestations of the same: star
movements and falling apples as gravitational force; matter and work as energy;
addition and multiplication as monoidal structure; God of the New Testament and
God of the Old Testament as the same God, perhaps, even as the God of Muslims
and the non-god of Buddhists; the yesterday's pleasure and today's conflict as
aspects of the same loving relationship.

%For ideal abstraction it is never enough to merely claim the identity
%-- being a {\em reflective} enterprise, it has to make it plausible
%for \co{reflection}, how the \co{distinctions} of \co{reflective
%experience} can plausibly be seen as manifestations of the same.

\citet{What a shock of {\em recognition} it was (as it actually
happened to me) while studying with wonder the plate of Corot
engraving -- to see it suddenly as a delicious episode from
`Parsifal'.}{ValeryCorot}{p.143}
Insights of the identity of
differences are among the greatest intellectual pleasures.  One 
knows experiences  which after a long and tedious work
bring two apparently unrelated ends of the reasoning chain into a
unity of a single realisation.
%Such experiences have also a different modus.
We have a problem in front of us which \thi{does not fit},
which we can not grasp because %having a uniform formulation
we feel that there is something which evades and prevents us from solving it.
And then comes an \thi{Aha!} -- a crucial \co{distinction} (and connection is,
too, a \co{distinction}) which solves the problem, which makes everything fall
on its place.  It may be too much of subtleties, but we would say that it is not
the evading \co{distinction} which brings the pleasure when it reaches the level
of consciousness.  This \co{distinction} has been there all the time, albeit,
merely in the \co{vague}, \co{virtual} form of the cognitive dissonance, of the
uneasy feeling that something is missing.  The pleasure arises from the fact
that this \co{distinction}, now clearly \co{re-cognised}, \thi{makes everything
  fall on its place} establishing a new unity.

The power of thought lies precisely in this: to \co{re-cognise} {repetitions},
to \co{re-cognise} the same in a variety of differences.  Only that, as far as
\co{reflection} is concerned, it is seldom enough to claim the underlying
identity -- it has to be \co{reflectively represented}, it has to become
\co{visible} for \co{reflection} if it is to be \co{actually} given and
convincing.  Appreciating all such thoughts and smaller or greater experiences,
one must always remember the great danger of any claims to unity misconstrued as
uniformity, of claims which simply try to erase the differences in the
postulated but never \co{experienced} pseudo-unity.  For better or worse, in the
sphere of \co{actual experiences}, \co{unity} is reflected mostly as only
relative identities.  Rilke's verse:
%
\begin{verse}
{\small{\em Wer rechnet userer Ertrag? Wer trennt\\
uns von den alten, den vergangenen Jahren?\\
Was haben wir seit Anbeginn erfahren, \\
als dass sich eins im anderen erkennt?~\ftnt{\citeauthor*{RilkeEsWinkt}}
}}\end{verse}
%
notices the fact, but this fact has to be rendered \co{concrete} in
every particular case: the reflection of one in another, the gathering
of \co{actual distinctions} into a unity, has to be established for
each situation or problem and, often, it has to be established again
and again to remind us constantly that \wo{[i]f many participate in one, they
  are unified in their relation to the one} even though \citet{they are
  different from each other to the degree, in which they are many.}{Proclus}{\para66}
The highest \co{unity} does not dissolve the differentiation of the lower
levels, does not mingle distinct elements in a flat uniformity. On the contrary,
it lets them remain differentiated, even opposed to 
each other, for all such lower oppositions and conflicts do not contradict the
genuine \co{unity}. 

\pa Eventually, the \co{absolute unity} is the \co{unity} of the \co{one}.  The
undifferentiated, \co{indistinct one} is one and the same, is the \co{unity}
underlying -- that is, in the order of ontological \co{founding}, preceding --
all the \co{distinctions}.  This \co{unity} finds its \co{concrete} place, its
\la{imago}, in the beings \co{separated} directly from, and hence
\co{confronted} directly with the \co{one}.  \co{Self}, the \co{virtual} center
of every \co{existence}, \co{founded} in the \co{one}, \co{founds} the
\co{unity} of the whole \co{existence}. The lower form of this \la{imago} is the
fact of \co{myself} being a repetition, a repetition of the unique event of
\co{birth}, and then \co{my self}, of the \thi{inborn possibility}. \co{My}
uniqueness is not constituted by anything more particular, 
but is given to \co{me}. Every \co{existence},
to the extent it is an \co{existence}, is unique and as such but a repetition of
any other \co{existence}: to the extent they \woo{participate in \co{One}, they
  are one in the relation to the \co{One}.}{Proclus \para 66 - another
  translation of Ibid.}  I am unique and you are unique and in this uniqueness
we are both the same.  There is no paradox here -- only the primitive character
of the \co{unity} of \co{existence} (if one likes, of the numerical difference of
indiscernibles) which is \co{founded} in the ultimate \co{invisibility} of
\co{birth} from the \co{origin}.  At the level of \co{actuality}, this is again
reflected in the repetition as recurrence of the same, in the temporal identity
of \co{actual} things, as well as in \co{me} seen as the merely \co{actual ego}
being the same now as \co{I} was yesterday.  Finally, in the \co{immediacy} of a
single \co{act}, the \co{unity} finds its \co{reflection} as the self-identity
of the \co{immediately} given \co{object} and as the unproblematic self-identity
of the \co{subject}, of the \co{immediate act} of \la{cogito}. In all cases,
these lower self-identities are only borrowed, are only \co{founded}
manifestations of the primordial \co{unity} which \co{transcends} them all and, 
hence, remains unaffected by their possible disturbances. 


\noo{\sep
  
  Merological essentialism: each part of the body belongs to the soul and
  contributes to its identity (we are here still within the actuality dominated
  thinking, although entities we are considering are known to be non-actual)...
  
  called by somebody the \wo{simple view} (but here only of personal identity)

  distinct from the 
  environment (the imitated gestures). E.g., \citeauthor*{Meltz77,Meltz83,Meltz89},
  \citeauthor*{Gall96}. We are not interested in precise timing of events but
  the fact that even psychological research draws the empirically
  discernible border of
  the emergence of (a residual \nexus\ of) self to the few minutes after birth,
  accords with our view much better than the earlier estimates (e.g., of
  \citeauthor*{Piaget62}, followed by \citeauthor*{Ponty64}) which denied the possibility
  of such an imitation before 8, and even 12 month.
  
  \subsubnonr{Summarising identity...}
  Just like the problems of identity arise from the antinomic attempts to apply
  the substantial self-identity of \co{immediacy} to \co{actually} distinct
  appearances, so the problem continues as the attempt to form an identity
  embracing the whole world. The identity does not obtain, however, and the
  closest \co{actual} thinking comes to such a unity is the totality of
  \co{actualities}, the epitome of pantheism. Then such a totality is tried as
  the principle of self which, however, never seems more than \co{myself}. No
  unity is obtained but a mere collection, and there are those who seem
  satisfied with that.
  
  Lower categories seldom yield any insight when carried up to higher levels.
  
  Although the \co{conceptual} formation follows the bottom-up, the motivation
  is really coming from above. The question about identity of distinct
  appearances would not arose without these appearances being already somehow
  unified; the question about the totality/unity of everything could not be
  asked, if the totality were not somehow \co{experienced}. The conceptual
  attempts simply to clarify these questions within the \hoa.
  
  But since their motives come from above, one has to reach beyond to account
  for them.
  
  \pa It is the person - grounded in the one - which founds unity; but the unity
  concerns the whole being. This, however, does not change anything for the
  virtuality of the personal origin cannot be separated from the whole personal
  being (just like one is only as and through chaos; just like all levels are
  present in every most immediate actuality).
  
  But this unity can be relativised to any of the lower levels psychological
  explanations can not do here anything more then elsewhere, namely, reduce
  everything to mere subjectivism, perhaps even solipsism, reduce identity to
  the mere sense of identity. This does apply at the level of mineness where the
  sense of being the same person, myself, is often of primary importance,
  although
  
  At the level of ego, identity is often attempted reduced to the sameness of
  the body, animal, and sometimes even of brain. Now, it is true that I can not
  exist without my body
  
  So it looks again like Kantian unity of apperception founding the unity of the
  object. To an extent it is so... But, eventually, it it the unity of One, and
  the direct confrontation with it, which founds the unity of Self

} %end \noo{\sep


\subsubi{Truth}\label{sub:truth}
As another example of a \co{trace} which, however, will turn out to be closely
related to the previous one, we consider the notion of truth. 

\noo{ \levs{10}{Truth (transcendence)} {corresp. -- statement (external)}
  {coherence -- theory (not here-now)} {consensus -- my life (not me): (ethos,
    society)} {revealed -- (above me)} }

\subsubnonr{What can be true} \pa\label{pa:whatcanbetrue} \wo{Please, close the
  door!} Can this be true?  No, of course, it is a command, not a statement
expressing a proposition. But then, suppose you close the door and he says:
\wo{No, I did not mean it, I was only joking}. The command was not true
or\ldots? The command was not meant as a command or even better, the sentence,
the \co{sign} which usually means a command in this particular situation did not
mean it. It was a joke, not a true command. One would probably wish that we get
rid of this \wo{true}, but we will not. The word \wo{true} used about a command,
a work of art, a feeling or almost anything is \citet{put in front of another
  word in order to show that this word is to be understood in its proper,
  unadulterated sense.}{FregeTruth}{p.86} Thus a true command is a command, and
not a joke.  But if a \co{sign} expressing a command can be a joke, is it not
fully natural and legitimate to speak also about \wo{true command} and
distinguish it from, well, \wo{untrue} ones? This is not much different from
distinguishing a command from a non-command, but the crucial issue is that we
have a \co{sign}, a sentence, which although usually is one may also be the
another. Just like Tarski formed originally the biconditionals for propositions
%
\equ{The sentence \wo{X} is true if and only if X. \label{equ:bicond}}
%
one can form a biconditional for almost every word and a thing $x$
%\begin{equation}
\equ{\wo{$x$} is a true \wo{P} if and only if $x$ is P.\label{eq:Wx}}
%\end{equation}
The meaning of \wo{true friend} might thus seem no different from the meaning of
\wo{friend} and this is probably what one would make out of it, if one were to
construct any formal theory.\ftnt{The schema \refeq{eq:Wx} is, more or less,
  the same as the basic case of the inductive definition of satisfaction for,
  e.g., first-order logic.}
But we are not after reducing anything to anything, and this schema is not fully
satisfying. \wo{True friend} can easily mean the opposite of what \wo{friend}
happens to mean in a given situation, and then it can mean at least as many
different things as \wo{friend} can. We do not believe in any genuine, primary,
\thi{true} meaning of a word, of which others would only be derivative or
adultered versions. The genuine and inherent \co{vagueness} of \co{signs} was
discussed in I:\ref{sub:represent}.\refpf{pa:signsMean}.  One
recognises, of course, the difference of accent and emphasis. As
prosententialism, or variants of deflationism might claim, \wo{true} functions
at best as a means of emphasis or indirect reference. However, the emphasis can 
amount to new \co{distinctions}: the increased need to emphasize that
things are {\em truly} what they are, that \wo{$x$ is a {\em true} friend}, that
\wo{$y$ is {\em truly} useful}, etc.  signals certain linguistic degeneration,
one might say, deflation of the meaning of the words where, like in the
Orwellian world, \wo{friend} no longer means what it truly means.\noo{If
  \wo{true friend} meant the same as \wo{friend}, then one would better explain
  why we sometimes use the one and sometimes the other expression. One should
  then also have some problems with the meaning of \wo{false friend} but,
  obviously, we do not have any.}
 
On the other hand, even if \wo{true friend} tends to mean the same as
\wo{friend}, so \wo{false friend} does not simply mean a non-friend.  A friend
is false when, contrary to all appearances, contrary to all \co{signs}
indicating and usually meaning friendship, he turns out not to be one.

\noo{ Continuing in the vein of schema (\ref{eq:Wx}), one may observe that:
  \equ{The word \wo{W} means W.\label{eq:Wmean}} This looks, if not
  irresponsibly empty, so at least quite Wittgensteinian.  \citet{The limit of
    language is shown by its being impossible to describe the fact which
    corresponds to (is the translation of) a sentence, without simply repeating
    the sentence.}{WittCultValue}{??? [The schema appears in
    \citeauthor*{DummettM}, p.108 [Truth, p.343, footnote~37]]} Even if this
  does not go for some words (which often can be explained), the point is the
  same as above: the function \wo{the meaning of\ldots} (no longer a predicate)
  can be used in exactly the same way, namely for disquotation, as the predicate
  \thi{\ldots is true} -- explaining the {\em meaning of} \wo{W}, one explains
  neither more nor less than W.\ftnt{With the minor, if occasionally
    significant, difference that the word is taken in \la{suppositio formalis}
    and not \la{materialis}.}  Most (modern) dealings with language happen at
  the meta-level (higher-order) of \wo{\ldots} and one is deluded to believe
  that such objects have any clearer and more manageable structure than the
  repulsively mental \thi{\ldots}, or even more repulsively metaphysical
  \ldots~. Unfortunately, except for the observations concerning the properties
  of the mere signs and (im)possible distinctions in their uses, all the old
  problems remain because all real questions involve immediately the
  disquotation, that is, the return from \thi{what is \wo{truth}?}  to \thi{what
    is true?}, from the meta-levels of the quotation marks to the object-level.
  At the object-level, however, the relation to the extra-linguistic world is of
  the primary importance. Even if we could not define it, make it \co{precise},
  embrace it fully and exhaustively, we may still try to understand it. And if
  we can not even understand it (through only some of its necessary conditions),
  we need not claim that it does not exist. Such a conclusion is a pure
  immanentism, as the case may be, reduction of the world to the language we use
  to speak about it. But then, the mere use of the word \wo{truth} would prove
  its reality.  }

\pa\label{pa:truthAll} \co{Signs} are \co{actual} points terminating the
\co{traces} of \co{transcendence}, \co{actual} expressions of the
non-\co{actual} and, eventually, \co{invisible distinctions}.  \co{Signs} which
we usually do not consider to carry truth-value (because they do not express
propositions? or even worse, because they are not declarative sentences?) can be
considered true in the generous sense of drawing the \co{distinctions} in
accordance with the \co{rest} of relevant \co{distinctions}, whether \co{actual}
or only \co{present}.  \wo{Please, close the door!} may be true -- with respect
to the actual wishes of
the one who pronounces it. Similarly, a question %, \wo{Is it raining outside?},
normally suggests that the person asking it does not know, and can
be true with respect to this. Rhetorical question is exactly an untrue question,
a non-question, which only appears as a question. In short, there is hardly any
linguistic \co{sign} which could not be endowed with the element of the
truth-value (which does not mean, that it must be its primary element). But
truth is not restricted to linguistic \co{signs}. 
\citet{If you were in a place where you knew that there were both healthy and
  poisonous herbs, though you did not know how to distinguish between them, but
there was someone else there whom you did not doubt knew how to distinguish
them, and when you asked him he told you which were the healthy and which
poisonous, and he told you that some were healthy yet he himself ate others,
which would you believe, his word or his deed?}{AnselmTruth}{9} His deed tells
the truth which his words tried to hide. 

This could be easily misinterpreted as follows: what we consider as possibly
true need not have the appearance of a proposition, but must be expressible as a
proposition. If a question can be true, it is because it expresses also some
proposition. \citet{It is easy to think of a language in which there is not a
  form for questions, or commands, but questions and commands are expressed in
  the form of statements, e.g., in forms corresponding to our \thi{I should like
    to know if\ldots} and \thi{My wish is that\ldots}}{WFM}{Appendix
  III:1\hee{???}}  So a command could be a joke or else a true command but then
there is a 
corresponding proposition which is false, respectively true. If one insisted
obsessively on this point, we might even let it pass but the problem is with the
status of propositions, not to mention their actual content.
% There are \co{acts} and attitudes which may be \co{recognised} as true without
% us being able to specify which proposition would express their content.
There is a much deeper sense -- than a mere rewriting of its meaning as
a declarative sentence -- in which a command, and every \co{act} or
\co{action}, can be true or untrue.  It can be true in the sense of being a good
command, a 
command which agrees with the human nature, a command the following of which
will help one to realise one's true goal, a command which commands to do what
should be done, in short, by agreeing not only with the \co{actually} given
facts and observations but also with the deeper \co{distinctions} drawn across
the field of whole life.  One can certainly deny the very existence of such things, but
such a denial will be rather only an expression of the uncertainty (or the lack
of consensus) as to which particular commands are good and how they can possibly
be distinguished from the bad (false?) ones.  On the other hand, one might object that
this is an illegitimate stretching of the meaning of the word \wo{true}. But such an
accusation starts from the assumption that only propositions, if not merely
declarative sentences, can be true (or false) and this is exactly the assumption
which we do not share. In a deeper sense (to which we will
return later) the truth-bearer is~\ldots \co{actuality}. Every \co{actual sign},
every appearance (whether a friendly attitude, a command, a statement) makes a
difference and hence carries an element of truth -- namely, truth with respect
to a broader horizon of \co{distinctions}. 

\noo{For the present, we limit this generosity to the \co{reflective} (and
  mostly linguistic) \co{signs}.  \noo{, i.e., those which are \co{dissociated
      as signs}.}  }

\subsubnonr{Truth and falsehood}

\pa \co{Meaning}, we said in I:\refp{pa:signTrace}, is a \co{trace} of the
primordial \nexus\ of \co{signification}. It arises as a consequence of the
\co{dissociation} of \co{sign} and signified giving rise to the abstract
\co{signs} as opposed to \co{distinctions} they may signify. Emergence of
\co{actuality} from the horizon of \co{experience} increases the \co{distance}
-- the \co{distance} which separates one from the \co{origin} but also, in the
most mundane form, the \co{distance} between the \co{actual signs} and the
\co{actual} as well as \co{present} \co{distinctions}. \co{Sign} functions only
in a context of usage but what makes it into a \co{sign} is the fact that it
carries its \co{meaning} -- the potential for drawing (particular)
\co{distinctions} -- with it. This peculiarity of abstract \co{signs}, of
\co{representations} as \co{reflective} doubles, dissolves the reactive
character of \oss\ and turns \co{representation} into an \co{act}, something not
merely conditioned by the \co{external} stimulus but involving also an element
of spontaneity.\noo{(Although conditioned by the higher levels, viewed from the
  pure \co{immediacy} of the \co{act}, this spontaneity appears entirely
  \co{subjective} and undetermined.)}


Truth is now an adventure of meaning, of the \co{representations dissociated}
from their origins and capable of actively effecting \co{distinctions}.
Original \co{signs} are, in a sense, always true, for they are not dissociated
from their \co{meaning}. \co{Dissociated, reflective signs}, on the other hand,
carry their own meanings, meanings which need not comply with the horizon of
\co{distinctions} to which they are addressed.  Falsity (in any form, as a
mistake, misunderstanding, lie) is possible only as a result of
\co{representation}, of the \co{dissociation} of \co{sign} (with its
\co{meaning}) and the context of the addressed \co{distinctions}. Most
abstractly, falsity is a break, a mismatch between the \co{actual} and the
\co{present} -- truth emerges as the desirable norm opposed to that.
\co{Representations} can also address the \co{actual distinctions} and thus
arises also the possibility of \co{representation} to disagree with the given
\co{actuality}. Truth vs. falsity is a distinction within the sphere of
meaning.\ftnt{We do not oppose the claim of Quine/Davidson's about the field
  linguist's need to figure out beliefs and meanings of native speakers
  simultaneously.\noo{after Rorty in TRUTH, p.333, ftnt.18} From such a
  perspective divorcing truth from belief and belief from meaning may be
  difficult. But we are not doing any linguistics, nor even any philosophy of
  language. We only claim that in the generative order of \co{founding}, meaning
  precedes truth. (Notice the following formulation: \citef{Suppose I think I
    see a mouse disappear behind a chair. Clearly this belief could be mistaken.
    But would this belief be wrong if I did not {\em truly believe} a mouse was
    a small, four-footed mammal, or a chair an object made for
    sitting?}{DavidsonEpistExt}{p.195} The emphasized phrase should probably
  read \wo{mean}, though this would only support our point and not the argument
  intending to show that any false belief requires a web of true ones. The
  conclusion of the argument may be acceptable but it should not require
  substitution of \thi{true beliefs} about mice for the meaning of the word
  \wo{mouse}.) A child's world is full of \co{signs} (occurring only as the
  \co{distinctions} which are their meanings) before some \co{dissociated signs}
  turn out to be lies, or mistakes.\noo{(Truth (and falsity) are, of course,
    much earlier phenomena than \co{attentively reflective} lies.)} Mistakes
  occur against the background of accepted truths but also, and primordially,
  against the background of meanings which, if they are \thi{true}, are so only
  in a very special sense of the original \co{signs} which, not being
  \co{dissociated} from their meaning, can not possibly be false.}  Lying is,
from the point of view of the development of consciousness, a pretty advanced
stage -- animals are not very cunning liars, even though they can sometimes be
deceitful.  Crying babies do not lie, crying youngsters and adults often do.
\citet{A child has much to learn before it can pretend. (A dog cannot be a
  hypocrite, but neither can he be sincere.)}{WittPI}{II:xi\kilde{p.194}} The
ability to lie marks the transition past the level of the immediate truth of the
\oss\ to their final, \co{reflective dissociation}, and the ability to lie in a
convincing and sophisticated way is \equi\ with an advanced system of
\co{representations}.

\pa Truth restores the unity disturbed by falsehood, and so the question about
the former presupposes clarification of the latter.  Truth as an explicit,
\co{actual} phenomenon -- as a desirable norm -- enters the stage only as
distinguished from, in fact, as a consequence of the possible falsehood. It is
the possibility of an \co{actual sign} to mislead, to lie, to mean something
which turns out to be different from what is the case, that is, to draw some
\co{distinctions} in a way which does not conform to the \co{rest} of
\co{distinctions} which are \co{present} (relevant) -- in short, it is the
possibility of mistake and failure -- which constitutes truth as the desired
norm. Thus, although truth as a fact, as the primordial \co{unity} is the
background against which falsehood emerges as its violation, it is falsehood
which founds the truth as the \co{actual} norm.\ftnt{First failures of innocence
  have the mood of disappointment with an unfaithful friend. One is disappointed
  -- not by one's incapacity and failure (such \co{egotic} worries come later),
  but by the world which deceived one.\noo{According to Lurianic Cabala,
  \heb{<<shevirat ha-kelim>>}, \thi{breaking of the vessels}, is the second stage
  of the world development, in which the divine light enclosed in the finite
  vessels disperses as the vessels break under the strain, resulting in
  disharmony and evil, which will now await for the final stage of
  \heb{<<tiqqun>>}, \thi{restoration}.} } \noo{ Using (\ref{eq:truth}) as the
  starting point, one might be tempted to say that S is false with respect to D
  simply iff it is not true, i.e., iff $\mean S\not\subseteq D$.  But as we
  emphasized, the pseudo-formal formulations must be taken with a lot of salt.
  This relation would hold, for instance, when $\mean S\supset D$, and one would
  rather not want that.\ftnt{So, if D are all the facts and observations for
    which Newton's laws are true, then relativity theory, comprising more than
    that, would be false with respect to this D. One may play futile games with
    other possibilities, e.g., $\mean S \not= D$, $\mean S \cap D = \emptyset$,
    etc.} We do not define falsehood on the basis of truth but rather the
  opposite.  }
%
It is the \co{actuality} of falsehood, rather than of truth, which
gives rise to the \co{distinction} between the two. It is falsehood which is the
crucial \co{distinction} -- the \co{distinction} between the \co{meaning} of an
abstract \co{sign} and the \co{distinctions} of the addressed background, the 
\co{distinction} whose possibility 
% does not obtain at the level of \co{signs} (coinciding with their meanings) and
% which is
appears first as a consequence of the \co{reflective dissociation}. 
%taking place with the \co{reflection} (Book I, \refpp{tripleDissoc}).  .

\ad{Excluded Middle?}
All \co{actuality} is a \co{sign}, and \co{signs} are
truth-bearers.\noo{, \refp{pa:truthAll}.} In a sense, one could therefore say
that, since every 
\co{actuality} is surrounded by the horizon of \co{presence} and hence can agree or
disagree with it, every \co{sign} is true or false and the scope of the
\la{tertium non datur} is unlimited. 
But we would prefer to view truth as an explicit norm which
arises only against the possible falsehood. As long as such possibility does not
arise, there is no need for truth or, as one could also say, truth is implicitly
granted. The fundamental claim is that truth is secondary in relation to
meaning, that \co{sign} is a \co{sign} in so far as it means something, as it
makes a difference by drawing a \co{distinction}, but it need not be true or
false for that.\ftnt{\citef{What a picture represents it represents
    independently of its truth or falsity, by means of its pictorial
    form.}{Tract}{2.22}\noo{It should not be necessary to emphasize, that our
  \co{signs} have little in common with Wittgenstein's \thi{pictures}.}}  We
traced an aspect of truth even in commands and questions but typically the
meaning of such \co{signs} overshadows the truth-aspect completely.\noo{ (simply
  because the \co{signs} are, as a matter of fact, true).}
Better examples are
given by the traditional paradoxes. 
%
\equu{(L)}{This sentence is false.}
%
The impossibility of assigning any truth-value to (L) has been declared a
paradox. But this appears so only in the context of bivalent logic which insists
on all statements having one of the only two truth-values.\noo{Even then one
  introduces further distinctions to handle, for instance, open formulae which,
  after all, are perfectly legal syntactic entities of the kind one might expect
  to have a determined truth-value. But it may happen that neither such a
  formula, $F(x)$, nor its (purely syntactically formed) negation, $\neg F(x)$,
  has the value \thi{true} (or both have the value \thi{false}) which, if
  nothing else, signals a special behaviour of negation with respect to
  satisfaction, and consequently truth, of such formulae. We are not saying that
  (L) is, formally, of a similar kind; only that just like truth-value of open
  formulae is, even if formally unproblematic, so a bit undetermined (in the
  sense that negation does not flip it), so may be the truth-value in natural
  language of the statements like (L).} The additional identification of meaning
with truth-conditions forces then one to declare the statement meaningless. But
one can not meaningfully claim that it is meaningless, since one arrives at this
conclusion by analysing its truth-value as a function of~\ldots its
meaning.\ftnt{If (L) is true, than {\em what it says} holds, but it {\em says
    that}\ldots One might object that it is only the meaning of the components
  of (L) which is known, while the whole statement fails to have one. But this
  looks like splitting the hair. We do not combine arbitrarily the meanings of
  \wo{false}, \wo{statement}, \wo{is}, \wo{this} but do it exactly according to
  the rule specified by this statement. It is only this compound meaning of the
  whole statement which causes the trouble with assigning to it any definite
  truth-value.} 

We do not see any paradox here nor, for that matter, in Tarski's general
formulation of such phenomena, the undefinability theorem, stating that a
semantically closed language can not obey the rules of classical logic.\ftnt{To
  avoid technicalities, let us think of a semantically closed language simply
  as one containing its own truth predicate and the self-referential
  capacities. \thi{Classical logic} means here primarily two
  truth-values.} The proof shows that such a language contains sentences which,
like (L), do not have any well defined truth-value.  Why should it cause any
worries? Why should it be so that every statement must have exactly one of two
possible truth values? We know, in fact, that it is not so -- besides (L) and
other paradoxes, there are other dubious cases like \wo{the sea-battle
  tomorrow}, \wo{the current king of France who is bald}, etc. Various ways may
be designed to endow such statements with a truth-value, but they tend to
over-interpret the intuitive meaning.  All the theorem says is that the
truth-value of (L) can not be determined in the world of boolean functions
working (according to the classical rules) on the two standard
truth-values.\ftnt{Some newer and more inventive theories of truth no longer
  insist on the bivalence and, instead, try to accommodate such phenomena, as is
  done, for instance, in the revision theory of truth, where truth of a
  statement is the fix-point of the appropriate unfolding of its meaning and
  where some statements may simply not have such a fix-point. E.g., 1.~if (L) is
  true, then it is false; 2.~if (L) is false, then it is true; 3.~if (L) is true,
  then it is false; etc. -- the series does not converge to any definite
  assignment of a unique truth-value to (L). \citeauthor*{GuptaRevisionA};
  \citeauthor*{RevisionAB}; \citeauthor*{GuptaRevisionB}.}

\noo{The D element takes care of this in that it relates truth of a statement to
  the context of its application. \wo{My mother's name is Jenny} is neither true
  nor false with respect to the universe of mathematics -- it is irrelevant.
  Similarly, \wo{$\displaystyle{\sum_{x=1}^{x=n}x \not= \frac{n^2 - n}{2}}$} is
  neither true nor false with respect to my family relations, but simply
  irrelevant. This does not mean that any S may be turned true or false by
  playing arbitrarily with the context D. As said in \refp{pa:notArbitraryD},
  this context is more or less uniquely given by the meaning of the statement
  (although some adjustments, narrowings or widenings may be needed in any
  particular case).  }

\pa\label{pa:fixedTruth} It is only the assumption that the world is a given and
fixed \co{totality} of things or facts \thi{in themselves} which leads to the
conviction that every sign (at least, proposition) is either true or false, and
that with the absolute -- that is unchangeable -- finality. The lack of such a
finality does not mean relativism in the sense of subjective arbitrariness; it
only means that any particular truth can happen to be extended/adjusted/modfied,
and that the \co{absolute} truth does not belong to such particulars but only to
the \co{absolute}.  \noo{ But, on the other hand, it makes most things true: the
  armchair there is green, and this is as true as any such a statement can be;
  Truth is a relation and hence is relative to {\em both} its arguments; but
  such relativity does not mean subjectivity nor arbitrariness; only
  non-finality, non-fixity, non-absoluteness...  }

The absence of precise bivalence and the limitations of the principle of
excluded middle go even further: truth, and hence also falsehood, is a matter of
degree.  Saying that \wo{$x$ is blue} about an $x$ which is pink is trivially
false.  Is saying that \wo{$x$ is blue} about an $x$ which is dark blue true?
Yes, if darkness does not matter. And no if it does. But when it is not true, is
it false? Perhaps, perhaps only to a degree.  A half-truth is often a falsehood
but, then, it is also a half-{\em truth}.  Proliferation of various theories of
fuzzy sets and fuzzy truth, of vague and probabilistic variants of the notions
treated traditionally as rigid distinctions, witnesses to the changing
understanding of these notions.  Truth, admitting of degrees, is not simply
opposed to falsehood which, too, is a matter of degrees. They are opposite but
the opposition is not a pure, univocal bivalence. Just like most \co{acts} are
neither good nor evil, so most \co{actual signs} are neither true nor false.
There is a large grey zone between the two extremes, and \co{signs} falling in
this zone may often be declared true {\em and} false or, as the case or wish may
be, as neither.

\subsubnonr{Truth} 
    
\pa\label{pa:notArbitraryD} Truth is a possible property of \co{actuality} and,
consequently, also of the \co{actual signs} and their linguistic expressions. It is a
relation between two (sets of) \co{distinctions}: the truth-bearing \co{actual
  signs} and the truth-giving (\co{actual} or not) \co{distinctions}. Denoting
the former by S and the latter by D, \wo{S is true} means 
\wo{S is true with respect to D}.  Frege's objection (actually, to the
correspondence theory, which here can be extended to D), that truth cannot
consist in a relation for this \citet{is contradicted by the use of the word
  \wo{true}, which is not a relation-word and contains no reference to anything
  else to which something must correspond}{FregeTruth}{} is a funny example of
the strange assumption that language, and in fact already its common usage,
contains all and only truth.\ftnt{There are few traces of the relative motion in
  ordinary language use, so that claiming that trees along the alley actually
  move as I am walking would be not only unnatural but {\em actually
    contradicted} by the {\em usage} of the word \wo{move}.}  D is hardly ever mentioned
explicitly because it is determined by the \co{shared} background of
communication, by the context of discourse.
%
We mentioned the possible disagreement between the \co{represented} and given
\co{actuality}. The \co{distance} between the two can be seen as the difference
between S and D, when not only the former but also the latter involves
(primarily) only \co{actual distinctions}.  Saying \wo{It is sunny}, we do not
specify \wo{at present, at the place where we are talking, with respect to the
  actually observed weather conditions, etc.}  -- all such indexicals are
implicitly given. Saying \wo{Life is a disappointment}, we do not specify that
we do not mean \thi{at present, at the place where we are talking, etc.}  In
practice, D is usually fully transparent, given implicitly by the {meaning} of S
and the situation.\ftnt{This {meaning} involves what Austin calls the
  \citef{demonstrative conventions}{Austin}{} correlating the words with the
  actual, historic situation in the world. But likewise, it involves also
  indications that some words are not to be so correlated. All such relations
  and correlations are aspects of \co{meaning} in our generous sense of all the
  drawn \co{distinctions}.} Thus, not only context disambiguates the meaning but
also vice versa, meaning narrows the context of interpretation -- the two are
\co{aspects} of one \nexus.

The \co{meaning} of S is some set of \co{distinctions} which itself constitutes
a part of the world and hence this \co{meaning} is by its very nature woven into
the texture of the world, into the \co{rest} of the \co{present distinctions}.
The statement \wo{It is sunny} draws some \co{distinctions} in the actual
situation and these are related to other \co{distinctions}, for instance, to
those which we can drawn by looking around or by (not) feeling raindrops on the
head.  The meaning of \wo{Life is a disappointment} draws some \co{distinctions}
in the actual situation: perhaps, it is a general statement about life, perhaps,
only an expression of the depressive period in one's life, perhaps, only a
sarcastic comment on the train of somebody else's complains. The \co{vagueness}
of the statement amounts to indeterminacy of meaning which, in extreme cases,
can make search for its truth futile. But even without any \co{clearer}
indications, one will recognise in such a statement an expression of some
\co{quality} which may seem more or less in agreement with one's own
understanding of life. The \co{distinctions} (D) implied by the meaning of the
statement are completely different from the previous case, but the truth of both
has the same general form of agreement between the respective sets of
\co{distinctions}.

% \pa
% Truth-conditionalists (as also later truth-verificationists) tried to reduce
% meaning to truth under the assumption that truth is either something simpler or
% more primitive than meaning. We agree that truth is something much simpler than
% meaning, but it is not more primitive. It is simpler because it assumes more and
% hence its intrinsic nature (what distinguishes it from the relevant opposite:
% falsehood) is simple. By the same token, it is less primitive because what it
% assumes is exactly the more primitive \nexus\ of meaning.
\pa\label{pa:truth}
The question about the nature of truth reduces almost to the question about
meaning; it \citet{depends on just two things: what the words as spoken mean,
  and how the world is arranged.}{DavidsonCoherence}{p.139} The crucial issue
concerns, of course, this last phrase. In our case, \thi{the way the world is
  arranged} corresponds, in every particular situation, to some more
\co{distinctions}, or else to some \co{distinctions} made by other means than
the (linguistic) \co{signs} whose truth is under the question. Very
schematically, we can express the required relation between S and D as follows:
%
\equ{S is true with respect to D iff $\mean S \subseteq \mbox{D}$\label{eq:truth}}
%
i.e., a \co{sign} (a collection thereof, an \co{actuality}) S is true with
respect to D iff it means/makes only \co{distinctions} made in D.\ftnt{As a
  curio: this can be reformulated so that every truth comes out as an identity,
  for instance,\noo{using Boole's $\nu$D for `a subset of D', and writing $\mean
    S=\nu$D, or in a more modern form,} using the equivalent formulation of the
  subset relation: $\mean S\cup D = D$.  Of course, we use this pseudo-formalism
  merely as a symbolic abbreviation. It may be helpful, but $\subseteq$ may
  equally well be replaced by some other form of \thi{fitting} or
  \thi{conformity}.  We do not intend any formally \co{precise} theory of
  anything.}

\pa
The relativity to D certainly will not lead us to any relativism or
scepticism. But it is not only apparent. Truth of a \co{sign} depends on the
object it addresses; truths about relative beings are, by this very fact,
relative, while \co{absolute} truth can only concern the \co{absolute}.
% In practice, D is almost never specified explicitly, let alone voluntarily, but
% is determined by the meaning of S. It may vary changing the truth-value of a
% given statement.\ftnt{The meaning of S can, of course, depend on the actual
%   situation, context of conversation and what not. Varying D reflects this
%   possibility.}
The stick lying at the bottom of a river is more like a snake, bending, swinging
and swaying, drawn half-way out of the water it is bent, and drawn completely
out it is straight.  Descartes concludes that our sense-perceptions delude us
(at least in the first two cases) and can not be trusted. It remains unclear why
they delude us in the two former cases but not in the last one, and the
particularly suspicious minds keep playing the games of total illusion, brains
in the vats, and the like. It rather seems that our senses are equally
truthful in all three cases: in the first two what we see is true with respect
to the visual distinctions (to put it blatantly: what we see is what we see, and
there is no falsity about it); in the third case, what we see is true also with
respect to other \co{distinctions}, say, those made by the sense of touch and,
for that matter, our knowledge that the stick remains unaffected by the way any
particular person might see it.\ftnt{The point expressed traditionally by saying
  that truth is not the matter of perception but of judgment, e.g., \citef{truth
    or falsity seems to me to be in opinion rather than in the senses. For if
    the inner sense is deceived, the exterior does not lie to it.  [...] This is
    the case when someone similar to someone else is thought to be him, or when
    hearing something other than human voice we think it to be a human voice.
    But it is the interior sense that does this. [...] So it happens that
    interior sense imputes its mistake to the exterior sense.}{AnselmTruth}{6}
  \citef{It is therefore correct to say that the senses do not err -- not
    because they always judge rightly but because they do not judge at
    all.}{CrPR}{I:2nd Division.Introduction.1;A293/B350} Judgment was supposed
  to relate to the ultimate sphere against which all particular cases were to be
  judged, while we have replaced this sphere with D.}

The relativity to D is the common feature of truth theories which differ
primarily with respect to what they consider to be the relevant D.\ftnt{D
  appears, in one form or another, in all explicit models: for instance, in
  formal model theory, as the actual structure (perhaps, with the assignments to
  the free variables) in which the truth of the formula S is evaluated; in
  situation semantics of Barwise and Perry, as the situation addressed by the
  utterance, in Kripke semantics for modal logic as the actual world, among all
  possible ones, etc., etc.  But we do not study formal theories.}  The schema
\refeq{eq:truth} contains the abstracted elements involved in most discussions
of truth and thus allows us also to see some of the differences between various
approaches. For instance:

\begin{itemize}
\item
Truth-conditionalism uses the same formula but to define $\mean\_$ and not
truth. Instead of assuming given $\mean\_$, it takes as fixed D (the world,
totality of facts, or the like) and the understanding of the left hand side. The
truth-condition, corresponding to the subset inclusion, amounts then to a
definition of the meaning function $\mean\_$.  
\item  
Correspondence theories take D (and hence also the target of
$\mean\_$) to be the external world. (How this is to be understood is another
matter.) 
\item  
Deflationists take D to be the whole universe (i.e., make any inclusion in D
trivial, which amounts to removing D), and let $\mean\_$ simply remove the
quotation marks around S.
\item  
Coherence theories also fix D but take it to be some set of accepted beliefs.
\item  
Pragmatists would like D to be some ideal, eventual conditions to be judged as
desirable (or, perhaps, just that which will remain \thi{there in the end}
(let us not ask where \thi{there} is and when \thi{the end} might be)), with
$\mean\_$ assigning to its argument the outcome of actions done in accordance
with it. 
\end{itemize}
It is impossible to dissociate the discussion of truth from ontology.  The
differences above concern the understanding of the reference frame D. With the
latter theories (coherentism, pragmatism) it is not, perhaps, ontology in the
usual sense, yet it acquires the fundamental meaning for the theory {\em because}
it functions as the measure of truth, as the element of a transcendent character,
if not as the transcendence itself.  A critique of some theory of truth amounts
usually to a critique of this frame, of the presupposed (or implied)
ontology. In the following, we will distinguish the kinds of truth
depending on D, on the addressed level of \co{distinctions}. But first one final remark.

%\ad{Not whole truth, various truth}
\pa Truth is a \co{trace} of \co{transcendence} surrounding the \co{immanence}
of the \co{actual signs} and the reference frame D can vary according to the
possible variations in the scope of \co{transcendence}. Something can be
considered true with respect to: the actual situation, some given set of
observations and experiments, the life experience, the (un)imaginable
\co{totality} of all \co{distinctions}.  Consequently, truth admits of degrees
and a statement true about the factual appearances can be false with respect to
other dimensions. But although D may vary as much as S, it is the same schema
\refeq{eq:truth} from \refp{pa:truth} which governs the relation at every
particular level.  Only having fixed D, one can possibly speak about {\em the}
truth. One could then imagine the equality S=D as the ultimate truth (about D),
while every partial truth about D as its proper subset. We could also have
disjoint S1 and S2, both true with respect to D -- completely different truths
about the same.  Also, triviality is possible only when D is fixed or, as we can
also say, when some things are obvious. Triviality is something which makes no
difference, not an S with empty meaning (every triviality means something)
% (such that $\meant S{$\emptyset$}$).\noo{One should always be wary of misuse of
%   mathematical concepts and symbols....} It has a meaning, only that this
but an S with meaning which in no way contributes to the actual D -- it only
repeats some already obvious \co{distinctions}.\noo{As one has observed an
  invention of a plethora of discourses -- like \thi{moral discourse},
  \thi{aesthetic discourse}, \thi{rational discourse}, etc. -- we should make
  clear that unable to discern much sense behind them, we mean by \wo{discourse}
  simply any particular situation (or as the case may be, a lasting
  relationship) in which some concrete people communicate by any means
  whatsoever. It may be hard to imagine a situation involving at least two
  persons which is not a discourse: \co{sharing} the situation is already a
  communication, involving a lot of common knowledge.}

In a particular situation, it is often much more important and difficult to
agree on which aspects to consider relevant, than on the content of each aspect,
it is more difficult to agree on D than on S. This is only a local reflection of
the more general fact, namely, that even the most stubborn realists have not
managed to come up with a convincing model of the world, of the ultimate D which could
(should and would) serve as the constant measure of truth,
\refp{pa:fixedTruth}. The world we live in is the world of \co{distinctions}
which we are able and forced to recognise. %This harbours some 
Now, the more \co{distinctions} D, the more (possible) truths but also, by the
same token, the more falsehoods.\ftnt{If such observations were of relevance,
  one could point that, according to \refp{pa:truth}.\refeq{eq:truth}, the
  number of truths grows, their triviality notwithstanding, exponentially with
  the number of \co{distinctions}.} \woo{One is certain of one's knowledge only
  when one knows little; doubts increase with knowledge.}{Goethe\kilde{Blad
    tkwi...,p.56}} There is probably some limit beyond which more
differentiation, needed as it may seem in the search for further knowledge,
breeds only more confusion and idle dispute, where \citeti{every solution to a
  problem is a new problem,}{Goethe}{\kilde{p.55}} where every new step,
multiplying the truths, or rather only probabilities and possibilities of
truths, removes us from the truth. Saying too much (thirst for \co{precision}
shares at least that much with its frequent companion -- the fear of triviality)
is more often than not a violation of the truth. The true art is not only to
know what can be said but, primarily, to know what can not and therefore should
not even be asked for. \citet{And it is no easie matter, being in the midst of
  the cariere of a discourse, to stop cunningly, to make sudden period, and to
  cut it off. And there is nothing whereby the cleane strenght of a horse is
  more knowne, than to make a readie and cleane stop.}{Lyers}{} As La
Rochefoucauld remarked, the problem of perspicacity is not that it does not
reach the end but that it goes beyond it.

\noo{Adding more and more precise, specific, specialised distinctions does indeed
differentiate the matter into more and more precise, specific, specialised forms
but, instead of approaching the matter at hand, it rather removes one from it.}


As D varies, and sceptics, relativists and post-modernists misuse the fact, one
might want to imagine its being fixed once and for all, as some ideal
\co{totality} of all facts, the ultimate reference frame where truth of all
possible truth-bearers is to be evaluated. \wo{The world} would be a common name
for such a thing, which would also dispense with the degrees of truth and
the unpleasant suggestions of relativity. Or so, at least, it seems. We have said
what such ideal \co{totalities} are good for, but we will return to this, as
well as to the question about the \co{absolute} truth, discussing the levels of
truth below.
%\refpf{pa:levelAtruth}. 


\subsubnonr{The levels of truth}
%
Truth-bearer is \co{actuality} or, in a bit more specific sense, the \co{actual
  signs}.  Truth is a \co{trace} of \co{transcendence} reflecting the anchoring
of these in the wider reality. What is taken as this \thi{wider reality}
determines the understanding, if not a detailed theory, of truth.  The level, or
depth, of a truth reflects only the level addressed by the \co{signs}, which
corresponds to D in the schema \refp{pa:truth}.\refeq{eq:truth}.


\ad{\imm The immediate truth}\label{pa:levelAtruth}\label{pa:immedSubject}
% At the lowest level of \co{immediacy}, the \co{sign} simply coincides with the
% signified. 
Sensation, if not also perception, can always be taken as true with respect to
itself, and this is the way of immanent truth taken by sensualism or
phenomenalism, whether of empirical or idealistic flavour. Every sensation is
true with respect to the fact of its occurrence.  In fact, every \co{sign} is
true if taken only with respect to the trivial \co{distinction} of its mere
being given: it makes a difference whether \wo{qukkda} is given here or not. It
is, however, such a trivial and irrelevant difference that one will hardly ever
take it into account. We are, after all, not interested in mere \co{immediacy}
but in what it means.

The rationalistic variant of this level of truth takes into account not only the
trivial coincidence of the \co{immediate signs} and their meanings but the
possible \co{distance} between meanings of abstract \co{signs} and the addressed
reality. It appeals to \co{immediate} ideas which appear as self-evident, that
is, unconditionally true.\ftnt{Truth becoming mere certainty is a constant
  aspect of this reduction, which starts perhaps with Descartes' postulate
  \citef{never to accept anything for true which I did not clearly know to be
    such [...] and to comprise nothing more in my judgment than what was
    presented to my mind so clearly and distinctly as to exclude all ground of
    doubt.}{DesMethod}{II\kilde{p.22}} Later, it accompanies verificationism and
  coherentism alike. \citef{All you have any dealings with are your doubts and
    beliefs [...]  Now that which you do not at all doubt, you must and do
    regard as infallible, absolute truth.}{WhatisPragm}{p.188-9}\noo{Doubt,
    belief indicate, if not prove, the focus on \co{reflective} attitude.}}\noo{
  In a more elaborate form, one can bring in the abstract \co{signs} and
  caricature the idea as claiming that \wo{blue} is true when related to a
  sensation of blueness. The verbal \co{distinction} relates to the visual
  \co{distinctions}.  It is not the abstract \co{sign} which coincides with its
  meaning, but the meaning of this \co{sign} coincides fully with the addressed
  \co{distinction}.  (In terms of the schema \refp{pa:truth}.\refeq{eq:truth},
  we have S $\not=\mean S$, but $\meant SD$.)} We will not repeat here the
remarks on idealized \co{immediacy} from \ref{sub:idealImmed} (in particular,
\refpf{pa:DescartesKnow}), but only notice again that, all the differences
notwithstanding, phenomenalism and rationalism join the ranks in so far as the
infallibility of the truth they are searching for is found in the
\co{immediacy}, whether of sensations or self-evidence.
%of a \co{sign} coinciding with its \co{meaning}.
In either case, the reduction
to \co{immediacy} means reduction to pure \co{subjectivity}. This conclusion was not
necessarily drawn by the proponents of the respective ideas, but as the
understanding of truth it amounts to solipsism:  the whole truth is the pure
\co{immediacy}, that is, pure \co{subjectivity}.
%-- with its sensations, impressions, ideas -- is.

\pa But truth is a \co{trace} of the ultimate  \co{unity} and hence it carries within
itself the 
sense of \co{transcendence} which is impossible to dissolve in any
subjectivistic reductions. Truth can hardly be dissociated from ontology.
%In a strange way (or perhaps not so strange, after all)
Cartesian ultimate certainty, \la{Cogito ergo sum}, places itself in the
\co{immediate subjectivity}: the point of self-evidence which, however, unless
augmented by God with some additional insights, would leave one in the perfect
self-satisfaction (or shall we rather say, despair?) of a solipsist. Exactly the
same threat appears to Philonous who, having reduced the existence (and truth)
to being \co{immediately} perceived, has to invoke higher consciousness to
ensure any {\em real} existence and truth.\ftnt{\citeauthor*{BerkeleyDial}{,
    II}}

To avoid solipsism one has to give up the limitation to the irrefutability of
  the \co{immediate signs} and admit some 
\co{distance}. As we have seen, the \co{objective aspect} of this level, 
its ontology (if we may misuse the term), is a disappearing point, an
\co{immediate, external object}. Supported by all the common-sense one can
invoke, truth becomes thus correspondence of the \co{immediate, subjective
  sign} and the equally \co{immediate} but \co{external object}. This
conception
has been dominating since Aristotle's famous:
\citet{To say of what is that it is not, or of what is not that it is, is false,
while to say of what is that it is, or of what is not that it is not, is
true.}{AristMeta}{IV:7\kilde{$\Gamma$:7.27}} We won't analyse it in all possible
aspects, but want to show its association with the principle declared by much of
the tradition to be the fundamental \thi{law of thought}, if not also of
\thi{being}. 


% \tpa{The logical aspect. }\ftnt{We will indicate how some aspects of understanding truth along
%   various levels can provide motivation for analogous distinctions within
%   formal logic.
%   The most crucial aspect seems to be the meaning
%   one attaches to negation, and we will consider some extra-mathematical
%   features which find their expression in various properties of -- or at
%   least the role played by -- negation. Of course, we do not address any points
%   of strictly mathematical or formally-logical nature.}
\pa\label{pa:logicA} \citetib{[I]t is impossible for anything at the same time
  to be and not to be.}{AristMeta}{IV:4} Indeed, the \co{objects}, the
self-identical, indivisible \thi{substances}, solidified complete notions whose
reality is consummated in the \co{immediate} limit of \co{actuality} make the
principle of contradiction most natural and obvious.  The condition \thi{at the
  same time} is decisive and it is this condition which narrows the horizon of
attention to pure \co{immediacy}, the ideality of extensionless \thi{now}.
Within so narrow a scope of the \hoa, an \co{object} can not both be and not-be,
and as it is assumed to possess only equally \co{immediate} properties, it can
not both be-so and not-be-so.

Let's also notice that, although the principle is stated few lines earlier in a
more epistemological terms, \wo{it is impossible for the same man at the same
  time to believe the same thing to be and not to be}, it is immediately turned
into an ontological principle.  In search for infallible and \co{visible}
certainty, not only understanding and \co{reflective} knowledge, but also being
get reduced to the pure \co{immediacy}.

%\pa
\label{pa:immedNexus}
But what is this, apparently so obvious, \thi{at the same time}? An $x$ can not
be simultaneously black and not black. OK, but the whole issue is this
\thi{simultaneity}. It is not even some imagined \thi{least possible passage of
  time}, but a residual and timeless point of immediacy. If I see an object
having and not having a property (or being and not being), I \thi{know} that
some passage of time must have been involved. (An alternative way of dissolving
the contradiction would be to postulate different objects.) We do not deal here
with a fundamental principle -- of non-contradiction -- but with a \nexus\ of
which this principle is only an \co{aspect}. Simultaneity, taken for granted in
the above formulation, is needed to obtain the principle but, equally well, the
principle itself can serve as the basis for \thi{defining} simultaneity -- as
the limit of differentiation, as the limit excluding any possible contradiction.
Such a temporal limit coincides with the imagined space point; each is only the
\co{posited} residuum of the respective dimension. This spatio-temporal residuum
is, in turn, an image of the absolutely self-identical \co{object}, the ideal
substance. Thus to this ideal limit of \co{reflection} addressed in
\ref{sub:idealImmed} -- the \nexus\ of \co{immediacy}, space-time point,
substantiality -- we can now add also the principle of non-contradiction.\noo{
  (and soon will also add necessity).}

The associated law of excluded middle expresses then the completeness of the
notion, of a \thi{substance} which is \co{precise} and definite. In the narrow
scope of \co{immediacy} it must either be or not-be, and for any equally
\co{immediate} property, it must either be-so or not-be-so.  This also expresses
the absolute character of negation, as the \co{reflection that it is} isolates
absolutely its \co{object} excluding everything else from the \co{actuality} of
\co{reflection}.  Indeed, with respect to the \co{immediacy} of an \co{act}, an
\co{object} has only two possibilities of either being here or not being here.
This narrowing of being to the ideal point constitutes the bivalent
\co{precision} of \thi{yes} vs. \thi{no}, \thi{being} vs. \thi{not-being}, and
underlies the bivalence of truth and falsehood posited as an equally
\co{precise} opposition.

\noo{Spinoza's \la{omnis determinatio est negatio}, even if it was meant to
  point in the direction of unlimited substance, describes excellently the
  origin of negation which results from the isolation of \co{objects} into the
  horizon of \co{reflective immediacy}.  }
%\newpa
  
\ad{\act The actual truth}\label{pa:truthB}
%\act $P_{1}\lor P_{2}\lor P_{3}\lor\ldots$ \\
The above correspondence theory, arising from the \co{subject}-\co{object}
\co{dissociation}, applies equally at the level of \co{actuality}. An \co{actual
  complex} underlies the idea of the \thi{fact} or \thi{state of affairs}, as
these arise in variants of the correspondence theory. This is the scope within
which all indisputable trivialities of the kind \wo{Snow is white}, \wo{The cat
  is on the mat}, etc.  have their \la{locus} -- both expression and
confirmation. Consequently, it is from this sphere that correspondence theorists
fetch all their \thi{obvious} examples.

Although emerging directly from the assumed ontology of \co{objects}, the truth
becomes a bit more multifacet than the contradiction of \thi{yes} and \thi{no}.
Negation does not any longer have the absolute and determinate character.  It is
no longer the mere binary \thi{yes} or \thi{no}, the bivalent contradiction, but
a mulitfacet \co{complex} of contraries.  A thing can be blue or green or red
or\ldots Various predicates do not stand to each other in relation of
contradiction, yet they may exclude each other.  For a property $P$, the
absolute negation not-$P$ becomes less informative -- it can be now taken only
as an abbreviation for an extensive, perhaps even infinite, alternative of all
its contrary properties.\ftnt{The logic still retains the laws of
  non-contradiction and of excluded middle, but its extra-logical character
  changes by de-emphasizing the role of negation and bringing forth the positive
  alternative of contraries.  As one example (of a multitude of logical systems
  which formally capture this transition) one can mention logics of finite
  observations -- pointfree topology, theory of locales -- which develop this
  idea by allowing one to define more advanced, lattice-like structures of
  disjunctions/conjunctions of positive concepts where negation, if present at
  all, is only of secondary importance.\noo{e.g., S.~Vickers, ``Topology via
    logic''.}\label{ftnt:locales}}

\pa The \co{horizontal transcendence} of this level is \co{more} of
\co{complexes}, and the search for truth keeps multiplying the analysed facts,
phenomena and truths about them.  Truth of a theory is the issue of this level
which, as has been observed, is in general  irreducible to the truth of
its empirically observable consequences. The question of the degree of truth, or
comparison of two theories, becomes of interest and the truth of a theory
seems to be of a different kind than the truth of a single statement.  Various
approaches, even if severely distinct in conceptual constructions, seem
nevertheless to be always variations over the pragmatic concept of truth. For we
have seen better -- more true? -- theories replacing the worse ones.  As Newton
superseded Aristotle, as Einstein superseded Newton, so progress seems to
belong to the notion of (such a) truth. But like every declaration of progress
-- either it is only a reflection of a prior ideal goal or else creates
immediately the question about such a goal.  Unless one feels satisfied with a
merely sociological account of the fact of {\em acceptance} of one theory
instead of another, one should somewhat account for the primacy
of some theories over others.
%  -- [CORRESPONDENCE/absolutised:OBJECTIVISM (obj.illusion)...]
The changeable nature of most formulations and theories accepted as true,
combined with the image of absolute and unchangeable truth, lead to the truth as
the  ideal goal, truth as a regulative idea.\ftnt{This is, in fact, inherent in
  the very questions \wo{How?} and \wo{Why?} asking for an explanation. They are
possible only on the prior assumption that things are {\em not} what they
appear, that every encounter has some hidden essence (whether cause, goal or
structure). This (we could say, transcendental assumption) turns the discovered
solution of one problem into a new problem because it, too, is only an
appearance in need of a new explanation. The postulated infinity of the process
of explanation is only a reflection of turning this  \co{dissociation} of
things from their meaning into the image of absolute, eventually-to-be-reached truth.}

  
Pierce: \citet{experience shows that the calm and
  careful consideration of the same distinctly conceived premises (including
  prejudices) will insure the pronouncement of the same judgment by all
  men.}{PierceFourIncap}{} It is very, very uncertain if experience really shows
anything of this kind. But even if it did, of
what concern could it be to me or to you, to any concrete human being?
  Everything \citet{which will be thought to 
  exist in the final opinion is real, and nothing else.}{PierceBerkeley}{p.82}
How long shall we wait to see what is real?  The essential questions concerning
{when} such a common pronouncement can be obtained and how could we possibly
know that it has been obtained (and won't be challenged any more) are not only
  open but also impossible to close. (All forms of eschatology are nothing but
  calls to the infinite waiting for the historical end of times.) Most
  unfortunately, however, such totalitarian futurologies 
forget almost everything. I am sitting now in the sunshine eating a tasty
ice-cream. It is extremely real, as real as the heat of the sun, but I doubt
anybody will ever, ultimately, on the day of settling the accounts, care to
devote any thoughts to this fact. Whether anybody would or not, its reality
\la{hic et nunc} is completely independent from what such a committee of minds
might eventually decide.

\pa The \co{horizontal transcendence} of \co{actuality} is \co{more} of
\co{objects} and \co{actualities}, and dissatisfaction with the correspondence
theories refers to this aspect. First come the trivial questions. Where is the
\co{precise} distinction between \co{sign} and signified, between meaning and
what is meant? Nobody ever gave a rigid, \co{precise} definition of it, and some feel
justified to conclude that there is therefore no distinction.

There is also a mightier argument with which Berkeley might seem to have
discredited even the mere possibility of maintaining correspondence.  For
  \citet{so long men thought [...\noo{that real things subsisted without the
      mind, and}] that their knowledge was only so far forth real as it was
    conformable to real things, it follows they {\em could not be certain} they
    had any real knowledge at all.}{BerkeleyPrinc}{\para 86 [my emph.]}
  \citet{How given that we \thi{cannot get outside our beliefs and our language
      so as to find some test other than coherence} we nevertheless can have
    knowledge of, and talk about, an objective public world which is not of our
    making?}{DavidsonCoherence}{p.141}\noo{We are not concerned
    here with Davidson at all, according to whom coherence is only a criterion
    ensuring correspondence.}
%
  Since we indeed \citet{\noo{we}cannot compare belief with non-belief to see if
    they match,}{RortyPDT}{p.334} then let us better reduce truth to some form
  of its corroboration which, at least in principle, should be possible
  \thi{inside} our beliefs.  Thus, one seems forced to replace truth with
  the corroboration of truth.\ftnt{Deflationism being the most recent and
  extreme form of such a reduction.}  A \thi{fact} gets reduced to the
  criteria of its verification, to its pragmatic utility, to coherence with
  accepted truths, to a consensus within a community, or what not. The more
  radical representatives claim that the criteria or characteristics they arrive
  at actually {\em are} the concept of truth, while the more sober ones admit
  that establishing acceptable criteria of truth need not necessarily require or
  imply the grasp of the concept of truth itself.


%\pa%Wittgenstein?? & coherentism
The general scheme of the critique is similar, technical and more particular
differences notwithstanding: what is a complex theory (not a single statement
but a {\em set} thereof; perhaps just a highly structured description) supposed
to correspond to? Also all the \thi{entities} like negations or disjunctions
seem to reside only \thi{in the mind}; they seem to have no simply 
\co{visible} correlates in the \thi{reality outside}, the
correspondence to which would constitute their truth.\ftnt{Objections of the same kind, but of more specific forms, concern the
  ontological assumptions frequently associated with (a variant of)
  correspondence theory. One would say that true statement corresponds to a
  fact, or a state of affairs, or something that actually obtains in reality.
  The problem is to clarify the understanding of such entities -- what exactly
  is a fact? (We won't follow the lengthy debates
                                %, like that between Austin and Strawson,
  whether \thi{facts} are something \thi{in the world} at all. It often seems
  that attempts to distance oneself from philosophy end  in bad
  philosophy rather than outside it.) If \thi{the cat on the mat} is a fact, then is it
  also the fact that \thi{there is no cat on the mat}?  What kind of (objective!
  external!) fact is that? There seems to be no satisfying answer from the
  proponents of the ontology of facts, but we would not take it as a successful
  critique of the correspondence theory.  If \thi{mental} vs.  \thi{outside
    reality} has any significant meaning, and any meaning for the theory of
  truth, then what with the statements like \wo{I love her.}? One may attempt
  behavioristic reductions but most would probably agree that if it is true then
  the fact, if any, to which it corresponds is pretty \thi{mental}. Certainly,
  for the one who is making the statement, it is in no way external.}

But such a critique (motivating idealistic coherentism or, most generally,
immanentism) seems to rest on some all too strong, if not directly wrong,
assumption. The assumption amounts to absolutisation of the dualism: the
\thi{objective public world} is \thi{out there} and we are trying to reach it
from some \thi{closed inside} -- not just a duality, presence of two distinct
aspects, but an absolute dualism. The world and the mind
(or language) are posited as totally incommensurable entities which, by the
initial and all underlying assumption {\em are not} and {\em can not} come into
contact with each other. As this makes the conflict indissoluble, the only thing
which remains is to get rid of one of the elements. Mind (or language) must
stay, as it is doing all the consideration, so the only element which can go
away is the extra-mental, or extra-linguistic reality. This form of idealism has been
the background for the coherentism from its beginnings with Bradley and
Joachim.\ftnt{If not with Berkeley. Statements like, e.g., \citef{It is a hard
    thing to suppose that right deductions from true principles should ever end
    in consequences which cannot be maintained or made
    consistent.}{BerkeleyPrinc}{Introduction \para 3} may not announce a
  full-fledged coherence theory of truth, but they signal the presence of its
  essential elements.}

Of course, you cannot both lock the box and have it open. But who has said that the
box is locked, that language does not refer to anything extra-linguistic, that
mind is able to relate only to itself? This, as we said, is motivated
exclusively by the immanentism, the wish to account for everything in the
\co{visible} categories and, in the present context of truth, to reduce it to
the \co{visible} criteria of its verification.  
The argument is simply this: saying that truth is a
relation of language to the world, we remain \thi{within} the
language. But do we really? If I hear that
\equl{\wo{There is a funny guy on  the square.}\label{eq:rel}
}
I convince myself about the truth of this statement by looking
out the window -- not very linguistic act and certainly not one keeping me
\thi{inside} the \co{actuality} of my solipsistic \thi{mind}.
We have -- we {\em do have} -- the relation between the linguistic expressions and
the world (or between S and D), something like:
\equl{\ \xymatrix{\txt{\wo{There is a funny guy on  the square.}}
    \ar@{<->}[rr]^>>>>>>>>>>r && \txt{{the funny guy on the square}}}\label{eq:relF}
}
%
Now, applying \refeq{equ:bicond} from \refp{pa:whatcanbetrue} and saying that
\equl{\wo{To
  check if \wo{There is a funny guy on the square} is true I have to look at the
  square.}\label{eq:relrel}
}
%
is indeed a linguistic expression which, in a sense, has turned \thi{is true}
into a higher-order predicate, has \thi{internalised} the $r$ from
\refeq{eq:relF}. As Tarski saw, natural language contains its own meta-language
and is semantically closed but this is only an impediment in rendering such a
language in purely formal terms of a mathematical theory. If one does not aim at
such reductions, this fact can be seen exactly as the source of the power which
natural language draws from the contact with its \thi{outside}, from the fact
that, unlike any formal language (or for that matter any language when raised to
the level of the only and exclusive object of study), it is not an isolated and
\co{dissociated} entity existing exclusively according to its own definition and
norms but, on the contrary, that it is only an \co{aspect} of \co{existence}
involved into its world.  Just like the universal possibility of {objectifying}
anything leads to the \co{objectivistic illusion}, so the universal possibility
of such an \thi{internalisation}, the mere fact of the relation \refeq{eq:relF}
being expressed linguistically as \refeq{eq:relrel}, can be misused for reducing
everything to a mere linguistic expression.  But just like the \co{objectivistic
  illusion} amounts to forgetting almost the whole sphere of \co{existential}
meaningfulness, so this reduction is possible only if one forgets that
\refeq{eq:relrel} is only a \co{reflection} of \refeq{eq:relF}.  All such
linguistic (or mental) reductions forget exactly this first fact, the original
relation \refeq{eq:relF}, and dwell exclusively at its \co{reflection}
\refeq{eq:relrel}, the mere \co{signs} over which they do have some power. But
truth is not the same as its expression or criteria and \citet{the thing
  [statement] which is true can be know before its truth is
  known.}{ScotOrdinatio}{I:d3.q4\noo{p.99}}

%As pointed earlier, \refeq{eq:Wmean} p.\pageref{eq:Wmean},
\citet{The limit of language is shown by its being impossible to describe the
  fact which corresponds to (is the translation of) a sentence, without simply
  repeating the sentence.}{WittCultValue}{\hee{???.} Or else, extending the
  schema \refp{pa:whatcanbetrue}.\refeq{eq:Wx}: the meaning of \wo{W} is W, as
  observed, e.g., in \citeauthor*{DummettM}, p.108\kilde{[Truth, p.343,
    footnote~37]}} Wrong assumptions lead often to right conclusions, just like
right observations can be used to justify wrong conclusions.  This very
observation seems to underlie two opposite views.  On the one hand, language can
not be explained by anything extra linguistic, because every such explanation
remains within the language. Limit, if there is any, is only an internal limit
of the language itself, and it provides the only constraint on its meaning and,
eventually, truth.  Certainly \wo{reality}, \wo{extra-linguistic reality}, might
mean something slightly different if \wo{world}, \wo{life}, \wo{language},
\wo{person} and many other words meant something slightly different from what
they do; and the necessity of repeating the sentence may be equated with the
nonexistence of perfect synonyms or, more generally, of different sentences with
perfectly same meaning.\ftnt{Which is not the case so that Wittgenstein himself
  discusses counter-examples, e.g., \citeauthor*{WittPI}{ II:20}.}  But then
there is not much sense in trying to explicate the phrase \wo{the relation of
  language to extra-linguistic reality} nor, for that matter, any other phrase.
Each means only what it means and is irreducible to any other.
%This may be welcome but hardly to the philosophical champions of linguisticism --
Language is for speaking, not for
speaking about, and the only task left is to catalogue various ways of
using various phrases. 

On the other hand, although this relation \refeq{eq:relF} can be named and
referred to {\em in} the language, although its meaning is understood relatively
to the understanding of the words it involves and other words -- it is not by
this token linguistic.  There may be little sense in trying to {\em prove} the
limitation of the language using the language, but such a limitation can be {\em
  shown}, or perhaps only pointed to. Even if it can not be \co{precisely}
specified within the language, its \co{presence} can still be \co{clearly}
indicated. This is one of the fundamental functions of language -- allowing us
to talk about something we cannot say, to point towards something we cannot
define \co{precisely}. This openness is both the strength and the flexibility of
natural language.  There may be as many relations $r$ in \refeq{eq:relF} as
there are linguistic expressions, for each draws the \co{distinctions} in
another way, and words do nothing more (nor less) than that. But even such an
unlimited variation does not change the basic fact which it, in fact, assumes:
there is a limitation of language and this limitation -- the openness onto
something more than mere language -- is the whole and only power of
language.\ftnt{Like every genuine strength, it is not owned by the one to whom
  it is given and can sometimes even appear as weakness.} The limitation of the
language is that one can hardly explain the meaning of any word or phrase to one
who does not understand it, that one cannot explain what \wo{blue}, \wo{taste}
or \wo{love} mean using the language alone; the limitation of the language is
that it cannot exist alone: \citet{to imagine a language means to imagine a form
  of life.}{WittPI}{I:19} This limitation is what underlies the meaning of
language; language lives only through the relation which words and sentences
carry by being (used as) what they are: \co{dissociated signs}, epiphanies
capable of forgetting their\noo{divine} origin but incapable of functioning
without it.
%THUS: use SHOWS the meaning, but not IS the meaning
%Wittgenstein??



\pa The issues of truth and of meaning, distinct as they are, are very
intimately related and the \thi{coherentism} of the one will easily spill over
the other. Perhaps the most dramatic variant of coherentism says: \wo{Nothing
  beyond the text.}\noo{Derrida, Of Grammatology,p.158} Text is absolutised as
{\em the only} carrier and container of meaning and, consequently, if it is
still a legitimate idea at all, of truth.\noo{ Well, not even that -- {\em reading}
of the text is the only such carrier.  True, unread text would hardly have any
meaning.  It is pity to have to spend time on trivialities but this is,
unfortunately, what such claims result in.}  Let us imagine two reading
scenarios.  1) I have a book, $B$, which I know was written by an author (whom I
may know or not, does not matter).  I read it and find a lot of interesting
stuff, meaning, even truth, in it.  2) In the other scenario, I have exactly the
same book $B$ only I know that it was written by an \thi{intelligent} machine.
% \ftnt{No impossible assumptions about enough
%   monkeys typing for enough long time and producing Romeo and Juliet are made
%   here. We are just assuming it for the sake of the illustration. For
%   \thi{textualism}, however, such assumptions might even seem welcome.}
Will the \thi{truth}, the \thi{meaning} of the book be the same in both
readings?

The champions of \thi{pure textuality} would like us to either abstract away the
difference or pretend that no difference obtains.  However, even if in 2) one
were able to understand exactly the same from $B$ as in 1) (except, perhaps, how
the heck did a machine manage to write {\em this}?), one would still not be able
to attach to it equal value nor, for that matter, equal meaning as in 1).  It
does not matter if one knows who exactly the author was, how much one possibly
knows about him and his time, although these things {certainly} {\em may}
influence the reading and understanding of the text.  But what matters is that
one knows {\em that} the text was written by a human.  This, {\em in and by
  itself}, makes the text a possible revelation of truth which can be trivial or
deep, simple or involved but, in any case, possibly relevant and touching
reader's humanity.  Reading is a form of communication with other humans, even
if the author is not present and if we want to call it fusion of the horizons of
meaning,\noo{(mine and text's),} a process of personal interpretation of an
impersonal cultural artifact, or whatever.  The fact that the text can involve
much more than the author ever intended to put there is as uninteresting as the
fact that an accidental slip of a tongue can mean something much more than the
one to whom it happened is conscious of -- we are still dealing with humans and
this is the universal horizon, the eventual reference frame which encircles
every text.

And finally, if one got the same book $B$ without knowing who, or what, wrote
it, its reading would be different from both 1) and 2).  Yet, reading it and
gradually discovering (yes!  in the text only!)  some understandable, human
meaning, one would create the image of the possible author in terms of 1), in
terms of a human who possibly might have written such a text.  At least, one
would think, bright guy, he says a lot of interesting things, while in the
moments where things get a bit dubious one would, as usual, try to ignore them
but, perhaps eventually, use them to construct a more refined understanding of
the \ldots {\em message}; perhaps, by augmenting what one understood with the
dubious parts, perhaps, by ascribing them to the idiosyncrasy of the author.
The point is only to repeat, after hermeneutics, that we do treat a text as a
possible message worth deciphering, but it is such a message only because it was
sent by another human. If it were sent by an ant or an\noo{ extraterrestrial or
  extracelestial} extracelestial intelligence, we would be deciphering different
-- slightly or perhaps vastly, different -- thing.

In short, many texts of accidental, merely informative or purely entertaining
value might be written by anybody, even machine.  But in the moment one assumes,
expects, or discovers, some deeper meaning in the text, the text itself becomes
a medium of human communication -- this is, eventually, the universal context of
every text.\ftnt{The {\em individuality of the particular person} who happened
  to write the text may, but need not, be of significance. \citef{[A] book is
    created by another I than that which appears in our habits, in company, in
    weaknesses. If we want to understand this I, we must try to recreate it in
    our own depth [...]  Man who lives in the same body with a great genius has
    little to do with the latter, but he is the only person known to his
    acquaintances; consequently, it is absurd to evaluate a great poet, as
    Sainte-Beuve does, according to what kind of person he is or what his
    friends tell about him.}{ProustSB}{The method of
    Sainte-Beuve's\noo{p.128/133:earlier in \btit{Sainte-Beuve and Baudelaire}}}
  Unlike those who, dissatisfied with the petty \co{ego} of personal
  folk-psychology, claim subject's total superficiality, here dispensing with
  the former serves only the purpose of emphasizing text's origin in the deeper
  layers of \co{self}, of genuine human subject(ivity).}

%\newp

\pa \wo{Nothing beyond the text} does not limit the attention to any single text
but to the \thi{textuality as such}; a single text, a {book} is but a piece of
\thi{interminable text}, where other texts are the natural reference frame for
any given one.  It is supposed to mean that there is no ground for assuming
\citet{anything beyond the signs, anything whose sameness and existence would
  not be conditioned by the process of naming.}{Text}{p.106 \citaft{RortyIM}{
    p.77}} This sounds almost OK -- disquietingly so\ldots Let us ignore the
tendency (of ones who seem to do nothing but reading and writing) to call
everything which involves signs \wo{text}.  Still, a sign which is not a sign
{\em of something else} is not a sign!  It is, perhaps, a thing, a \thi{that},
but what makes it a sign?  Only the \co{distance} between it and its meaning.  A
more drastic change of language seems to be needed, but what is the point in
changing the meaning of the word \wo{sign} to such an extent that it ceases to
denote \thi{sign}?  \wo{\la{Simulacrum}}, a copy without any original, tries to
take care of that, of reducing the \co{sign} to its purely \co{actual} dimension
and dispensing with anything signified, anything lying beyond its abstract
\co{actuality}.\ftnt{We take notice of the remarks that \wo{text} does not mean
  text but something hard to specify, some \co{totality} of \co{distinctions},
  the \co{totality} of meanings or, perhaps, meaning-carriers, a web of beliefs,
  a kind of network of interconnected -- or rather a magma of free floating and
  hardly identifiable -- units; eventually, a pantheistic substance overriding
  identity of its parts. The arguments of idealistic flavour, which point to the
  impossibility of a contact between two dissociated substances, yield one
  substance which, as also in our case, is the \co{one} underlied
  differentiation. The absolutisation of \thi{text}, underlying the claims of
  nothing lying beyond it, amounts to applying a similar argument which,
  however, can be legitimately applied only to the ultimate \thi{substance}.}

Most things exist and function completely independently from the process of
naming but, indeed, they emerge only in the process of \co{distinguishing}. The
difference might seem to concern only the wording, but it is quite fundamental,
as fundamental as that between \co{actual signs} and the general
\co{distinctions} or else, as that between \co{actuality} and \co{presence}. 
Certainly, \co{distinctions} arise only gradually and at every stage form a
whole system -- there is no such thing
as a single \co{distinction} existing \thi{in itself} independently from others.
% ; transitions between levels concern primarily the degree of
% \co{precision} of the involved \co{distinctions}.
Similarly, what is meant by \wo{love} is related to, perhaps even dependent on,
what is meant by \wo{sympathy}, \wo{affection}, \wo{hate}, etc.  So, in
principle, we can agree that meanings of all words are woven into a total web of
the language.  But this inter-dependence is only a reflection of the
\co{distinctions} drawn in the matter of \co{experience}, between things,
feelings, understandings.  The plasticity of the {\em web} of language at best
reflects only the plasticity of the \co{experience}, like the \co{actual signs}
\co{reflect} only \co{distinctions} drawn also beyond the \hoa.  But it does not
mean that every single, \co{actual} element of this web has a meaning only by
its relation to the rest and is conditioned exclusively by such a relation.  On
the contrary, unlike a scientific theory (which, according to Quine, is a vast
theoretical super-structure only at its outer boundary touching the world of
experience and experiments), natural language meets \co{experience} at {\em
  every} point; the fact that meaning of a particular statement can depend on
meanings of other words and statements does not in any way cut this particular
statement off from its relation to the extra-linguistic aspect of
\co{experience}, from the extra-linguistic \co{distinctions} which it draws and
addresses.

\thi{Text}, any system of \co{signs}, or language is indeed, as Merleau-Ponty
says, \citet{life itself, our and thing's life.  [But] not that {\em language}
  conquers and appropriates them, for what would be left for saying, if there
  existed only things said?  It is a mistake of semantic philosophies to close
  language as if it spoke only about itself, because it lives [by] silence;
  everything we throw to others grew up in this great, mute country which will
  never leave us.}{VisInvis}{Inquiry and intuition;p.131\label{cit:mute}}
Expanding a bit this 
statement, we would add that the \thi{mute country} can itself be organised by
\co{signs} which themselves need not have linguistic character, but even these
are \co{signs} only by the force of their meaning (eventually,
\co{non-actuality}) which they bring into the horizon of our attention.
\co{Sign} is the paradigm of \co{actuality}, and \co{dissociated sign as a sign}
the paradigm of \co{immediacy}.  The insistence on pure textuality and absence
of anything beyond the \co{signs}, in its attempts to escape from simplified
correspondence to \co{actualities}, falls straight back into the midst of
\thi{metaphysics of actuality} which, de-mitologised by the removal of the last
element of \co{transcendence}, becomes pure \co{immediacy} of a mere {sign}.
The \la{simulacrum} of \co{transcendence}, the \thi{interminable text}, appears
as the ultimate form of \co{subjectivistic illusion} which sees nothing beyond
the \hoa\ but other \co{actualities}, now not even \co{objects} but only
\co{signs}.  Perhaps, it is only an accident, perhaps, an intellectual sickness:
searching for \thi{the other}, it finds only itself; in the midst of its search
for the ultimate \co{transcendence}, it \co{thirsts} for ecstatic unity, and so
keeps removing the \co{transcendence}, narrowing the attention to what, being
\co{immediately} given, deludes \co{reflection} with the promises of \co{unity}.
And so, \citet{in that outward moving there is frustration or compulsion; a
  thing most exists not when it takes multiplicity or extension but when it
  holds to its own being, that is when its movement is
  inward.}{Plotinus}{VI:6.1}~\ftnt{The simplistic relation to formal logic
  suggested in \refp{pa:truthB}, footnote~\ref{ftnt:locales}, can be given more
  substance by observing the trend according to which logic is not about
  formulae but about proofs (which must not be confused with proof-theory!
  --Roughly, proof-theory studies derivations of valid formulae, while the
  genuine concern with proofs addresses only their internal mechanisms and
  structures). The work, and general philosophy, of Jean-Yves Girard can serve
  as an excellent example of the attempt to carry out the programme quite
  ambitious and admirable when judged by the strictly scientific standards: get
  rid of the traditional syntax-semantics split, explain the meaning of the
  (syntactic) entities by the role they play in proofs, internalise
  meta-considerations by requiring and utilising reflection principle (in the
  technical sense of converting higher-order statements into equivalent
  first-order ones). All these goals of pure logic can be seen as a reflection
  of the more general, philosophical immanentism discussed above. With respect
  to the last point, in \citeauthor*{LocusSolum}, we even find the quote from
  Wittgenstein: \citef{I may play Chess according to certain rules. But I may
    also invent a game where I play with the rules themselves.  The pieces of
    the game are then the rules of chess and the rules of the game are, say, the
    rules of logic. In this case, I have {\em yet another game}, not {\em a
      metagame}.}{WittRemarks}{}\noo{\btit{Locus Solum: from the rules of logic
      to the logic of rules} has a nice dictionary which, to some extent, can be
    understandable even for a philosopher.} As such projects and observations
  might be, and to some degree rightly, associated with our views, let us only
  emphasize the crucial difference: \co{one} is \co{above} all the webs and
  games which also are their own metagames. The moment in which this \co{transcedent
    origin} of all \co{distinctions} is ignored marks the beginning of
  immanentism.\label{ftnt:logicB}}

If we are willing to see in the \thi{text} a \co{totality} of meanings, a
plastic \co{totality} of fluid elements with always unsharp boundaries, we might
see in it an analogy to our \co{chaos}. But beyond \co{chaos}, there is
\co{nothingness}. There is something beyond the \thi{text}, something that may
seem infinitesimal and insignificantly small, but which, exactly by lying
beyond all differentiations, weights more than their whole web. It is a moment of
silence, of silence which is tranquil and not intense, grateful and not awaiting
anything. A moment of silence which neither grasps nor reaches for anything but
makes all the things and meanings light, weightless, which suspends all meanings
and thus, almost paradoxically, endows their totality with the deepest meaning
-- the touch of eternity \co{above} in which all meanings and differences
disappear, this \thi{transcendental signified}\noo{Grammatology,p.49} which
halts the regress of potentially infinite re-interpretations of \co{signs} by
\co{signs}, of one \co{actuality} by another. It is only from such moments, from
such an ultimately mute country beyond, that the \thi{intertextual} web arises.
Just like everybody enters this web from there, so in the moment of genuine
silence one also leaves this web and touches eternity \co{above} it. Although such
moments may be very rare, they do occur and, easy to ignore as they are, they
testify to the tremendous difference between the interminable search for the
\co{totality} of \co{visible signs}, which can rest only in the ideal limits or
eschatological phantasies, and the \co{unity} of truth \co{present} in the life
one is living.

\ad{\mine Truth of totality} The \co{objectivistic illusion} underlying
simplistic versions of correspondence theories, is likewise manifest as
\co{subjectivistic illusion} in coherentism: first, in the fact that the whole
became a mere \thi{text}, a collection of mere \co{actualities}, \co{signs}; and
then, in the usual multiplication of facts and truths (here, \co{signs}) as one
tries to embrace the whole; the whole which refuses to appear as anything more
than a \co{totality}.  Coherence (sometimes reduced to mere logical consistency)
is a desired property of such a \co{totality}, being an indication of the
possibility that one has \co{actually} grasped the whole.
 %, the \thi{significant whole}, as Joachim said,
The main problem we have with pictures like that of Pierce's truth is that the
whole is simply a \co{totality}; perhaps, well organised and structured but
still only a \co{totality}. Moreover, it extends beyond any reasonable limits of
inquiry -- this is why the truth about it must be postponed until indefinite
future, for it will take infinity of time to collect and study the infinite
\co{totality} of facts, problems and phenomena and to arrive at a consensus
concerning it. As most regulative ideas and ideal limits, it betrays the attempt
to capture something higher in terms of a \co{totality} of lower elements.
\co{Totality} is a project of \co{mineness} and the corresponding mode of
\co{horizontal transcendence}, the \co{not-mine}, has the particularly
significant modification: the others. Focusing on this aspect of
\co{transcendence} leads to replacing the more or less conceptual coherence with
the social consensus.

The eventual context of all considerations is the context of one's life. This,
however, is a never ended \co{totality}, a \co{totality} which seems to possess
no internal coherence except that \citet{which it happened to obtain at the last
  turn of hermeneutical circle.}{RortyPP}{\citaft{RortyIM}{ p.72}} Any possible
\co{transcendence} announced by \co{signs} is attempted removed, so that even
the system of \co{signs}, the ever escaping \la{simulacrum} of the
\thi{interminable text}, gives place to the arbitrariness of \co{mineness}.
Indeed, where is the possible distinction between mere opinion and knowledge,
between private idiosyncrasy and general validity?  Nobody ever gave a
\co{precise}, rigid definition of it.  So \ldots there is no distinction, at
least, none possible to maintain.

But nobody wants to be swallowed up by the arbitrariness of
\thi{subjectivity}.\noo{Derridean \thi{postcards} attempting to illustrate the
  impossibility, or at least inadequacy, of a general language and theories by
  insisting on private, purely idiosyncratic ways of constructing a
  \thi{narrative} provide an example of such a turn which tries also to maintain
  the absence of anything beyond the privacy of associations and its
  self-sufficiency, needing no involvement into the questions about possible
  truth.  This \thi{privacy} and opposition to generality (if not explicitly
  conceptual, then in any case, acted and expressed) squares rather badly with
  the claim \wo{nothing beyond the text} for now it seems rather that there is
  \wo{nothing beyond the reader}.  But whoops, no, no!  There is no reader, and
  if there is any, he is only an \thi{interminable text} himself.  Or, perhaps,
  it is something completely different, only one prefers to avoid playing {\em
    this} \thi{language game}. Sure, why not, it is nice to get a postcard, even
  if it does not say anything.}
%
Beyond \co{me} there are others, the whole community, tradition.  The
sociological aspect appears (albeit in as yet limited and not aggressive form)
already with Pierce since, for some reason, the eventual usefulness of an
opinion cannot be judged by its own standards, it requires a consensus.
\citet{And the catholic consent which constitutes the truth is by no means to be
  limited to men in this earthly life or to the human race, but extends to the
  whole communion of minds to which we belong, including some probably whose
  senses are very different from ours, so that in that consent no predication of
  a sensible quality can enter, except as an admission that so certain sorts of
  senses are affected.}{PierceBerkeley}{p.83} Indeed, one may try to pay the due
respect to the \co{transcendent} character of truth, to overcome
\co{subjectivism} or even solipsism by appeals to a community, by blurring the
distinction between the \thi{objective} and the \thi{inter-subjective}. This way
of \thi{taking care of} the \co{transcendence} of \co{not-mine}, characterising
the sociological invasion of philosophy, amounts really to its removal and
leaves, eventually, only a community of writing and speaking, a community of
\thi{narratives}.  Then there seems to be only one goal: let's interact, talk
and \wo{keep conversation going}.  Strangely enough, one drags in all kind of
extra-conversational, extra-textual things like ethics, society, culture,
solidarity\ldots So, after all, does not conversation suffice?  It seems it does
not.  For pragmatical mind, conversation is only means of action, and a good
conversation brings in powerful narratives, effective metaphors.  But appeals to
descriptions and metaphors which simply work, which are effective, which serve
the purpose seem rather empty if keeping the conversation going is {\em the}
goal.  (Sure, it is not {\em the} goal because there is no such thing as {\em
  the} goal.  One is only pressed by \thi{wrong language game} to say what the
point possibly might be.)  But even the strictest codes of a court etiquette,
enabling one to spend time conversing without saying anything, had always left
an opening for actually saying something.  Reducing everything to a game of
\la{simulacra} one would still like to escape the resulting arbitrariness, that
is, to retain the possibility of saying something.  And so one needs something
-- tolerance, solidarity\ldots -- which sounds sufficiently convincing
\thi{in itself}.  But is it convincing?  Sure, because \thi{decent fellows like
  us} do accept such values.  But what if I am not one of the conversational
club, if I am not a decent guy and do not accept such values?  I guess, I am not
part of the game, I am not admitted to the conversation.  But then, since one
talks about \thi{solidarity} and \thi{tolerance}, of constant extension of the
horizon of \thi{we}, so please, extend it, perhaps even to the guys who bomb
one's home town.  This, after all, is only an expression of a yet another,
powerful narrative metaphor.

\pa There is certainly an element of truth in it. Consensus, just like
coherence, need not exclude truth and it is often important to achieve an
agreement with others which means, to accept the same truths, with respect to
the same horizon of \co{distinctions}.  One can also think of more extreme cases
where all the \co{distinctions} of relevance are {\em only} those agreed upon by
the community -- then the criterion of truth will become conformance to them.
This could remind a bit too much of a petty-bourgeois conformism but this, or
the general smell of ethnocentrism around the consensus theories, can always be
dissipated by positing some more ideal communities like the one just seen in
Pierce, or a bit more limited community of rational discourse, or other
inventions of the kind. 

The element of truth hiding behind the consensus theories concerns the
\co{vague} intuition of its transpersonal \co{aspect}. This \co{aspect} of
truth, however, is not transpersonal because it resides only in some communal
consensus but, on the contrary, because, being \co{present} in every human, it
\co{transcends} {\em his} \co{reflection} and the \co{subjectivity} of his mere
\co{representations}. It is the \co{aspect} of \co{non-actual foundation}, the
Intellect, which surrounds every \co{actual} person with the sphere where
\citet{the object known [is] identical with the knowing agent, the
  Intellectual-Principle, therefore, identical with the Intellectual
  Realm,}{Plotinus}{V:3.5\label{cit:actIsObject}}
\ref{sub:notunconscious}-\ref{sub:invArch} and \ref{sub:objsubj}.  But replacing
\thi{truth} by \thi{inter-subjectivity}, consensus theory reduces possible truth
to that about which all might reach an \co{actual} consensus, and that is close
to nothing.  Failing to escape \co{my} subjectivity in the sphere of social
subjectivity, failing to rest satisfied with the inaccessible ideal limits, one
invents eventually the ideal of consensus allowing everybody run his own
business, the ideal of plurality and non-conformism.
% The \thi{pluralistic universe} where \thi{anything goes} may have valuable
% aspects, but in this context it may be read as \wo{anything is true -- just make
%   it work}.
This is only a slight variation on the old theme of pragmatic ethics in a
\thi{pluralistic universe} taken over by the recent masters of manifold and
variety.  Having lost, in the plurality of disparate opinions and
\thi{discourses}, any means of saying that something is more important, more
valuable, higher, deeper or more true than something else, and thus of saying
what possibly might be a meaningful goal for which things should \thi{work}, the
only possibility left is to embrace everything.  One even imagines systems
(political, cultural) allowing for non-conformism. (Positive, as the intentions
behind such, as behind most other utopias, can be, the result reflects only the
origin: the absolutisation of atomism on the social scale, only claimed to have
positive value.) However, just like a sign which does not signify is not a
\co{sign}, so a system open for everything is not any system but \ldots a lack
thereof. Also, non-conformism which is allowed as an option, as if calculated
into the system, is not any longer non-conformism.  A system allowing
non-conformism, by this very token, abolishes its possibility.  One could,
perhaps, keep conversation going.  But everyone should know the situations where
this becomes a mere gesture of unbearable politeness, because what the other is
saying, no matter how intelligent and well-argued, is simply nonsense.
Universal pluralism and tolerance would have one main consequence -- total
de-individualisation and indolence, uniformity, the exact opposite of the
intended variety\ldots Discourse in such a setting would, too, cease to have
meaning, because where everything is allowed, where all goals are equally good,
where everything is acceptable, there is no need to argue (not to say fight) for
anything. When everybody is entitled to be heard, eventually, nobody bothers to
listen. Sure, these are only idealizations,\thi{regulative ideas} since no such
thing can ever happen.  Yet, there are some who speak as if they wanted them to
happen\ldots

\pa Certainly, admitting {\em every possible} variety of narrative metaphors is
a bit too generous. But saying what should be excluded, by the very fact of
excluding something some members of the all-embracing community might defend,
points always beyond the limits of possible consensus. However, having only
narratives without any inherent and mutual values, such exclusion appears
arbitrary.  No matter how explicitly one admits ethnocentric assumptions, one
remains ethnocentric which, eventually, means arbitrarily self-satisfied.  And
if anything at all might possibly be meant by \wo{truth} it can not stand all
too much arbitrariness.  The convenient words one tries to invoke --
\wo{solidarity}, \wo{universality}, \wo{communicability}, \wo{tolerance} --
point towards something which one believes would cure this arbitrariness,
something \ldots fundamentally human.  Ethics seem to replace truth; not in the
Kantian way, though, as complementing the project of knowledge of appearances,
but simply {\em replacing} it.  Ethics, moreover, devoid of any truth, that is
an arbitrary \thi{ought} to mere worship of manifold and plurality, growth and
diversification, increase \ldots The popular medical term for that is
\wo{cancer}.  Truth is only what works, and so is good.  Powerful metaphors,
powerful narratives \ldots But what makes a metaphor, a narration {\em
  powerful}?  What is it that makes some language \citet{strike also the next
  generation as inevitable}{RortyCIS}{I:2;p.29}?  Communism was once a powerful
narrative, many next generations were struck by it as inevitable. And its
prolific, powerful consequences\ldots Yet its power was not very different from
-- in fact, was founded in -- its falsehood.  Is it so that people somewhat
intuit that this and not that narrative can be made effective, used to change
things?  Change to the better, perhaps?  These are, of course, \thi{wrong
  questions}, which assume that there is some \thi{what} behind the actual event
of a narrative simply happening to be powerful. It just happens, and so does truth.
So let it happen in most prolific ways, with no interference or persecution of
its possible happenings.  (Why such a proliferation is to be any better than
unification is not clear -- after all, consensus seems closer to the latter than
to the former but, we might guess, one wants to be politically correct.)\noo{
  claiming, without clarifying, both.}

\pa It might seem that we are confusing two distinct issues: consensus theories
of truth and the politically motivated views trying to admit free unfolding and
multiplicity of views. We join the two because they both arise from the same
perspective putting the social aspect in front of any other.  The problem of
non-conforming groups and individuals is equally annoying to the consensus
theories as to those preaching unrestricted proliferation of narratives: the
latter have to exclude deviant individuals and dangerous narratives, while the
former, in addition, also admit for those who announce some truths before the
times are ripe for a general consensus about them.  Somehow, they must take
non-conformism into account and, trying to do so, step beyond the purity of
their basic claim. Freedom of unfolding, as well as consensus, must be qualified
by some additional element -- of respecting the limits, of being reasonable, of
not coercing others. And indeed, even a mere fear of being persecuted can be
equally good means of coercion as actual persecution.  But if we need to exclude
persecution to upgrade a mere consensus, or a pluralistic non-conformism, or the
power of narratives, or whatever comes next, to the level of truth, why not
dispense with all these concepts and upgrades and only say that it is enough not
to persecute?  \wo{Truth is non-persecution.}  Well, it does not sound bad,
does it?  Why not?  Because it again touches something which one wants to
recognise as humanly correct, not to say true. Even if truth is something that,
in principle, everybody could recognise, it does not mean that it is something
everybody actually recognises. And it is likewise with falsehood. The servility
of praising something that \thi{strikes also the next {generation} as
  inevitable} is as astonishing as the depersonalisation and
de-individualisation implied by the unrestricted pluralism.  It is an invitation
(the less intended, the worse) to following the mob psychology, like that which
forced the victims of Marxism (in all its variants) to follow the
\thi{development} proclaimed inevitable for all future generations.

%   -- [not-me: CONSENSUS/absolutised: SOCILOGISM or
%   PANTHEISM : the absolute is totality]


The consensus theories, this absolutised sociologism, reflect the mentality of a
stock market and agents of public relations, or else the wishes of the
ministries of propaganda.  No doubt, there are sociological dimensions and
situations in which mob's convictions are sufficient for action or, as is
usually the case, mass hysteria.  Thirst for \co{visible} criteria ends
typically with conflating the criteria with the things they were only
supposed to be criteria of. But have we not heard enough lies which, repeated
sufficiently many times, refused to become truths.  \thi{Inter-subjectivity}, no
matter how pluralistic, how total or totalitarian, is still a {\em
  subjectivity}, something which can not constitute truth but only, and only at
best, discover it.\ftnt{If we were to draw free associations to some aspects of
  formal logic, we would focus here on the attempts to incorporate
  \thi{otherness} by not negating its value. Formally, it can be seen as a
  further limitation of negation's omnipotence by renunciation of the law of
  excluded middle.  Three-valued or, generally, many-valued logics are examples
  of that. (We are not discussing here technical intricacies.  The fact that,
  for instance, three-valued logic can be formally represented by introducing
  the meta-predicate of validity (or truth), within two-valued logic certainly
  does not concern us here.) As often happened in the history of mathematical
  logic, the beginnings go back to philosophy, indeed, to Aristotle's'
  syllogisms and future contingents.  A third logical value, introduced by
  {\L}ukasiewicz, denoted the \thi{unknown} truth-value of statements
  about future contingents.
  
  The diminution of the absolute role of negation reflects the recognition of
  some form of boundedness of our truth-gaining capacities, perhaps even genuine
  incompleteness of \co{our} knowledge.  In spite of all Leibnizean
  constructions of complete notions of substances which contain all future
  events which will happen to them, {\em we} do not know if \wo{there will be a
    sea-battle tomorrow}.  Already Arnauld pointed out to Leibniz that we should
  better concern ourselves with the ways {\em we} can know things, rather than
  with the ways God might know them.  The logics with non-contradiction
  principle and law of excluded middle, if taken as more then mere logics,
  reflect rather God's than our view of the universe.  Logics renouncing the
  excluded middle are motivated by incompleteness of {\em our} knowledge, by the
  simple fact that \co{I} may not know whether $P$ or not-$P$ is (going to
  become) true. (The (in)completeness of knowledge being modeled is not to be
  confused with the (in)completeness of the logical system itself.) Besides
  many-valued logics, one can mention here intuitionistic logic, built precisely
  around the idea of knowledge which remains incomplete and limited to the facts
  which can be positively constructed, as well as, fuzzy logics which try to
  model vagueness of concepts.  Linear logic of Girard is a completely different
  example where incompleteness of the modeled system corresponds rather to the
  boundedness of resources consumed during the reasoning process.\label{ftnt:logicC}}

\newp 

\ad{\inv The absolute truth} 
%One
\citet{Truth in its essential nature
  is that systematic coherence which is the character of a significant whole. A
  \thi{significant whole} is an organised individual experience, self-fulfilling
  and self-fulfilled.}{JoachimTruth}{\para 26 \citaft{Truth}{ p.50}}
Translating it into our language, the \thi{significant whole} is the
\co{existence} and the \thi{systematic coherence} its \co{unity}. But {\em this}
  \co{unity} is not consistency; it is the original fact of
  \co{existence founding} the possibility of any \co{actually} consistent whole
and unity. 

Recalling again our figure (I:\refp{pa:stages}, p.\pageref{fi:stages}), as we
move up the circle (above its horizontal diameter), there are not \co{more
  distinctions}, in fact, there are less, or if there are more, they become
dense beyond the limit of recognition. It is the sphere where the clue is not to
include \co{more} heterogeneous \thi{facts} into a unifying theory but rather to
understand that \thi{less is more}, that the same \co{eternal} things penetrate
all variety of \co{actualities}. The truth addressing the level of
\co{invisibles} is rather contained in simple words of wisdom than in complexity
of smart arguments. For \citet{many-branched and endless are the thoughts of
  the man who lacks determination [while] the follower of this path has one
  thought, and this is the End of his determination.}{Bhagavad}{II:41}

\noo{Indeed, \citet{that which is universal and
  eternal can manifest itself only through the
  particular.}{ProustIncomprehens}{\noo{p.31}} Such
}

%Contradictions
Truth, like every \co{trace}, reaches eventually to the \co{unity} which is its
\co{absolute aspect}, the unchangeable \co{one}, common to all who know it and
to those who do not.  But \citeti{[t]hough the truth is common, the many live as
  if they had a wisdom of their own.}{Heraclitus}{ DK 22B2 [Instead of
  \wo{truth}, the fragment has \wo{logos}.]}  Manifestations of \co{unity} need
not and do not conform to the \co{visible} rules of agreement between the
particulars which, usually, only unknowingly and involuntarily happen to
manifest the same eternal truth. The \co{absolute} truth does not embrace all
the lower, \co{visible} ones -- it abolishes them.  The \co{distance} separating
the \co{actual signs} from their meaning is here so remote, that no \co{visible}
rules can any longer govern the expressions of truth. And only the exclusive 
reliance on the \co{visible} rules and criteria can confuse their lack with
arbitrariness and the absence of truth.

\noo{The \co{unity} of \co{one} lies \co{above} the multitude of
  \co{distinctions} and even though \woo{He said one [...] I heard two.}{Ps.}}

The \co{unity} of one being certainly admits contradictions.  Viewing this
\co{unity} in its temporal aspect, there is nothing contradictory in being, at
one time $P$ and, at another time, not-$P$.  But much more can be said.  \co{I}
can, at the same time, both like and dislike a person.  And it is not of much
use to say that then \co{I} like and dislike distinct properties of the person,
because persons are not \co{complexes} of properties but individual beings.
Sure, \co{I} can like the person for \thi{being x} and dislike for \thi{being
  y}.  But \co{I} can also, simultaneously, both love and not-love the person,
the whole person.  \citeti{I hate her and I love her.  Why I do so I don't
  know.\lin It's just the way I feel, that's all, and it's tearing me in
  two.}{Catullus}{ \btit{Odi et Amo}, 60}{\noo{Odi et amo. Quare id faciam,
    fortasse requiris.\lin Nescio, sed fieri sentio et excrucior.}\noo{I love
    and hate.  How does it happen, you may ask.\lin I do not know, but I feel,
    suffer and go mad. -- Catullus, \btit{Poezje, Pie\'{s}ni} 85 [za Herbert,
    Labirynt nad morzem, p.196of197]}} \co{I} can have a \co{vague} feeling
about something which, when attempting to specify it, results in saying that it
is both pleasing and displeasing. This can be blamed merely on the
inadequacy of the language to express the actual feelings. But blaming it for
such an \thi{inadequacy}, one has already assumed that the feelings must be
prone to a \co{precise} description in terms of \co{immediacy}, that is,
non-contradiction.  Higher things seldom are prone to such descriptions and,
indeed, their descriptions can often be most adequate by using contradictory
predicates.  Coming to terms with such higher aspects of one's life, it is
necessary to realise that simple yes-no questions can have no answers.  One
sitting there and trying desperately to figure out \wo{Do I love her or not? Do
  I or do I not?\ldots} is probably still in his adolescence, trying to capture
the accumulated tension of \co{vagueness} and \co{clarity} in the categories of
recently developed \co{precision} and \co{reflective visibility}.
%
\noo{Much of the adolescent \thi{pain and suffering} is related exactly to the
  appearing need to disambiguate a series of issues which, so far, have lived
  quietly in their genuine element without any precise borders or
  determinations.}
%
When, eventually, some \co{action} must be undertaken, one may have to bring
everything down to the \co{actual} choice between yes and no: \wo{Shall I invite
  her for a dinner or not?} But it is a question about \co{actual} course of
action where, indeed, contradictions can be intolerable.\ftnt{The logical aspect
  of this level of truth takes the final step in decreasing the role of negation
  by renouncing also the law of non-contradiction, admitting states (or
  theories) with both $P$ and not-$P$. ({It is very doubtful or, in any case,
    quite disputable whether such things are still logics and if they are good
    for anything at all. But we are not concerned with the value, nor even
    comparison, of logical systems which we are observing here from an external
    and un-mathematical point.})  There may be various \co{actual} reasons for
  that.  One, exemplified in non-monotonic logics, is the wish to model
  development of states of belief. At some point, one may simply not know which
  of the two holds. This can be now taken not so strictly as in intuitionistic
  logic (which simply denies the law of excluded middle) but as saying that both
  are possible and one may be allowed to draw conclusion from the one and the
  other.  This is a weak form of para-consistent logic where one would simply
  allow an agent to \thi{know}, \thi{believe}, \thi{be in a state} including
  both $P$ and not-$P$.  Admitting such states in a stronger sense may be
  motivated by examples of, typically, scientific theories, which involve or
  entail contradictions, but which are not meaningless. This may be also a way
  of approaching some paradoxes, like liar paradox, which seem to be
  \thi{dialetheias}, \thi{true contradictions}, since they can be understood as
  being simultaneously both true and false. In any case, the main consequence of
  admitting (apparent or real) contradictions is that one has to prevent the
  possibility of obtaining from them all possible propositions (as would be the
  case in classical logic), which would again equate such states with something
  meaningless.
  
  Most generally, such logics relativise the meaning of negation. In the
  non-adjunctive system of Ja\'{s}kowski's discourse logic, negation applies to
  the statements of the same interlocutor but not to those of others; in
  non-truth-functional system of da Costa, the value of not-$P$ is completely
  independent from the value of $P$; in relevant logic, pioneered by Anderson
  and Belnap, negation becomes a purely intentional operation, so that $P$ and
  not-$P$ are interpreted in different structures. This can be related to the
  recognition that concepts are not precise but vague and hence negation does
  not posit an absolute alternative but remains itself, so to say,
  underdetermined.  Going in this direction, para-consistence might approach
  fuzzy or many-valued logics which, indeed, can be used for modeling
  para-consistency as proposed, e.g., by Asenjo.\label{ftnt:logicD} Nicholas of
  Cusa could be probably seen as a pioneer in this field, although his
  motivations were of purely spiritual nature and his attempts aimed only to
  communicate the true \equin\ of opposites in the \co{origin}. His
  \la{coincidentia oppositorum} concerns \citef{the Uniting Beginning, [where]
    we see opposites prior to duality, i.e., before they are two
    contradictories. [It is] as if we were to see the smallest of contraries
    coincide (e.g., minimal heat and minimal cold; minimal slowness and minimal
    fastness, etc.) [...] Hence, just as an angle that is minmally acute and
    minimally obtuse is a simple right angle, in which the smallest of contrary
    angles coincide, before acute angle and obtuse angle are two angles, so too
    is the situation regarding the Uniting Beginning, in which the smallest of
    contraries altogether coincide}{Beryl}{\para 41;p.810} or, perhaps, have not
  been \co{dissociated} yet.}

\pa The eventual \co{transcendence} is \co{nothingness}, the lack or the empty
set of \co{distinctions}, so the only \co{signs} true with respect to it are
those which mean \co{nothing}, $\mean S=\emptyset$, \wo{nothingness}, \wo{one}
or just~\ldots silence.  Let us notice in passing that a \co{sign} meaning
\co{nothingness} is very different from a \co{sign} meaning nothing, or from the
lack of \co{sign}: silence meaning \co{nothingness}, \ger{Stille und Ewigkeit},
is different from silence which does not mean anything or, as the case may be,
is only tensely expecting its own termination waiting for somebody to say
something.\ftnt{Thus, on the one hand, \citefi{Silence alone is Thy
    praise.}{Ps.}{LXV:2; St.~Jerome's translation} But not all silence is
  praise, for there is also silence of emptiness, of lack turned into
  disappointment: \citefi{Unless the Lord had been my help, my soul had almost
    dwelt in silence.}{Ps.}{XCIV:17\kilde{Ps.94}}\label{ftnt:silencePraise}}
Also (and now we see again the inadequacy of simplistic pseudo-mathematisations
like $\subseteq$ in \refp{pa:notArbitraryD}.\refeq{eq:truth}), as intentions
belong to meanings, there is a fundamental difference between a \co{sign} which
both intends and means \co{nothingness}, and one which, intending something,
means nothing, between a \la{koan} and an overlooked contradiction.
 
The \co{absolute} truth is \co{that}: \co{that} there is, or else, \co{that}
there is truth. And what is \co{that}? \co{Nothing}, \co{that} which makes it
impossible for any person or any community to {\em arbitrarily} decide what is
true. At the same time, \co{that} makes it possible to oppose any particular
\thi{truth} which \co{actuality} might attempt to \co{posit} as the truth, as
the \co{absolute} truth.  Falsehood with respect to the \co{nothingness} is not
its plain negation, something; it is not anything specifically distinguished as
such, because no particular thing, no \co{distinction} opposes the
\co{absoluteness} of \co{indistinct}.  \co{Indistinct} remains untouched
\co{above} the whole world, \co{above} all \co{distinctions} and falsehood with
respect to \co{that} amounts to projecting some \co{distinctions} into the
\co{indistinct}.  \co{Absolute} falsehood is something particular {\em only
  when} predicated about the \co{absolute} or, what amounts to the same,
absoluteness predicated about anything relative -- in short, an \co{idol}.

\pa \co{Idolatry} divides because, \co{positing} something relative as
\co{absolute}, it \co{alienates} from \co{that} which is \co{one}; trying to
make it \co{visible}, it not only veils but also falsifies it.  Relativity means
limited scope and validity, and \co{positing} it as \co{absolute} sets one
against all that contradicts this absoluteness, namely, all that falls outside
its scope.  And so, he that tries to unify under some \co{visible} slogans, ends
up dividing, \citeti{he that gathereth not [in truth] scattereth
  abroad.}{Mt.}{XII:30/Lk.XI:23} The scattered pieces, the pieces left outside
the scope of the \co{actual} unity remain, however, as the seeds of constant
restlessness, disquieting reminders of the refusal and exclusion.  And
\citeti{[t]he stone which the builders refused is become the head stone of the
  corner.}{Ps.}{CXVIII:22} The new building, to last, must not be just a
rearrangement of the old pieces including, in addition, a few pieces previously
excluded. It is raised only by accepting relativity of the relative and ceasing
to search for the \co{absolute} among the \co{visible} pieces.
      
Truth is the way of -- and, as a \co{trace} of \co{transcendence}, a norm, a
call to -- \co{unity}, keeping heaven and earth together. It is the
\co{transcendent unity} and as such a contra-distinction to the falsehood of the
\co{actuality} taken as \co{absolute}. It is the element of \co{transcending} --
beyond \co{actuality} -- which establishes it primarily as the {\em call} to
\co{unity}, as the {\em norm} of preserving the \co{unity}.  Following this norm
unites, but not necessarily in any trivial sense of a conceptual agreement -- it
unites primarily in a complete disagreement, \co{above} and as if in spite of
any \co{actual} conflicts.\ftnt{\citefi{What opposes unites, and the finest
    attunement stems from things bearing in opposite directions, and all things
    come about by strife. -- Graspings: things whole and not whole, what is
    drawn together and what is drawn asunder, the harmonious and the discordant.
    The one is made up of all things, and all things issue from the
    one.}{Heraclitus}{DK 22B8-B10.} We are, of course, very far from any
  pantheistic interpretations of the slightly unlucky phrase opening the last
  sentence.} It is not a trivial acceptance, which often means just
absolutisation, of all the differences. It is rather an admittance that these
differences are only of relative value, are manifestations of something which,
remaining \co{invisible}, unites.  The image of truth as the absolute either-or,
not admitting any degrees and compelling everybody to unconditional acceptance,
as both \co{absolute} and \co{visible}, is a fallen angel reminding of the lost
paradise. \citet{The urge to possess absolutely {\em only certainties} is a
  residual religious drive, and nothing more.}{WS}{16\noo{after Lou Salome's
    Nietzsche, p.82}} It is the \co{trace} of the highest aspect of
\co{absolute} truth which, however, concerns only the \co{absolute}.  Among the
relativity of \co{actual} facts, truth can hardly  conform to such standards.
%As everything \co{spiritual}, it is fully or it is not at all.)
Any demands for it to \co{actually} and \co{visibly} unify and gather are only
ways of reducing its meaning and power.  For \co{absolute} truth is not relative
to any particular aspects of our \co{actual} world and life, but exercises its
power \co{above}, as the ultimate norm, ever reminding us about the only
relative significance of whatever we manage to capture under our \co{actual}
look and grasp. Reducing it to any \co{visible} norms and criteria falsifies its
character of being exactly the \co{absolute} norm which remains valid when all
\co{actual} criteria have failed or been violated.

\pa Every \co{actuality} is a \co{sign} which may be true or not with respect to
the \co{absolute}.  In the deepest sense life, viewed as a constant
\co{confrontation} of \co{actuality} with \co{transcendence}, can be true.  To
live in truth is to live in conformance to the \co{origin}, in the \co{unity
  above visible} dispersion; to use Luria's inventive imagery, to gather the
dispersed pieces; in the constant process of restoration, \heb{<<tiqqun>>}, to
keep repairing the divine vessels broken in the earlier stage of dispersion,
\heb{<<shevirat ha-kelim>>}.\ftnt{The same intuitions seem to underlie the Orphic
  myth of Zagreus-Dionysus' rebirth -- return to the original unity -- from the
  pieces scattered in the souls of all people.}  Successful
%gathering
\heb{tiqqun} concerns the whole hierarchy of Being, \co{unifies} all its levels
around the highest truth; not in any coherent theory but in the full recognition
of the relative differences, even incommensurability, of lower
\co{distinctions}, whose possible consistency and compatibility never sums up to
the ultimate \co{unity}. The \co{signs} can be true in different degree,
depending on the level they address. An \co{idol} is hardly ever a complete lie,
it can be promoted to the special status precisely because it harbours some
truth; but the status of the \co{absolute} is undeserved because it is only {\em
  relative} truth. I saw once a professor (of philosophy!) conducting a proof in
first-order logic for the statement that God who is both omnipotent and good
cannot possibly exist.  Formally, the proof was correct and His Professorship
seemed very pleased with himself -- he almost seemed to believe that he actually
proved anything of interest. But correctness, and truth, with respect to the
level of mathematical immediacy need not reflect any truth with respect to
deeper perception and understanding of the world. One can be both right and
wrong at the same time.\ftnt{In a particular case, it can be hard to say if
  rigidity is an expression of pride, of insecurity, or of both, but it is easy
  to notice that one can be both intelligent and stupid. Such observations might
  be taken only as a preliminary to the observation \thi{the smarter, the more
    stupid} from \refp{pa:smartStupid}.}  Truth of most statements terminates at
the level which they address, and in this very fact there hides an additional
aspect of truth, of knowing and respecting the limits. But there are also other
modes of speaking. The similes, invoked in wisdom literature and various
Biblical stories, are true at several levels -- not because they can be
interpreted in various ways but because their plain meaning extends to and
merges with the senses at deeper and deeper levels. This is the rare \co{unity}
of wisdom which is able to embrace the truth of whole human being in the
\co{actuality} of one image, in a few simple \co{signs}.


%\pa
In the deepest sense, a life can be true, a life which is lived in conformance
to the \co{origin}. This does not in any way assume any \thi{essence of human
  nature}, but it does suggest that there are some fundamental aspects of the
\co{existential} situation which, deserving respect and recognition, can be
ignored and forgotten. (We will address them in Book III.)  Truth in the
strictest, metaphysical sense of an access to the unchangeable reality applies
only to the deepest sphere of life.  Simply because this sphere of ultimately
\co{invisible foundation}, and its relations to the lower sphere of
\co{visibility}, is the only constant aspect of human experience and history.
\citet{Our fundamental ways of thinking about things are discoveries of
  exceedingly remote ancestors, which have been able to preserve themselves
  throughout the experience of all subsequent times.}{Pragmatism}{p.83
  \citaft{Cotkin}{ p.165}} But the fact that they \thi{have been able to
  preserve themselves} is not an accident of the historical development prone to
pragmatic verification. They were able to preserve themselves only because they
reflect the deepest aspects of human situation.  The lower aspects and,
eventually, concepts, ideas and theories are certainly prone to steady
re-evaluation.  But this in no way affects the truth concerning the ultimate
reality of our being. This truth needs no arguments and demonstrations, it is
\co{above} all truths, unaffected by their passage and indefeasible by their
pretensions, waiting in its eternal silence until the skirmishes which future
times fight against the past fall silent too. \citet{We have incapacity of
  proof, insurmountable by all dogmatism.  We have an idea of truth, invincible
  to all scepticism.}{Pensees}{VI:395} This truth is what it always has been and
every human being can only attempt to live it \co{concretely} or, as the case
may be, fail to do so.  To dissolve the \co{absolute} in the relativity of
\co{visible} truths is to falsify it, for \citet{it is absurd to make the fact
  that the things of this earth are observed to change and never to remain in
  the same state, the basis of our judgment about the truth.}{AristMeta}{XI:6.
  We will not, of course, follow this observation with the conclusion that,
  since only heavenly bodies are \wo{always in the same state and suffer no
    change}, they have anything more to do with the \co{absolute} truth, for
  this is already reducing the \co{absolute} to the \co{visible}, if only
  remotely so.}  An unmistakable sign of such an absurd is ascription of
\co{absoluteness} to some relative truth.  This is the {\em only} \co{absolute}
untruth the history of human kind has ever seen.

\sep

\pa\label{pa:immedArbitrary} Let us summarise briefly.
The particular ways in which particular judgments or theories are checked 
vary tremendously leaving hardly any universal criteria to stick to.  This,
however, does not mean that I can judge as I please, and all {theories of truth}
try, in one way or another, to account for the absence of such arbitrariness.
Arbitrariness is the pure \co{immediacy}, the pure subjectivity of mere \la{hic
  et nunc} which, \co{dissociated} from all the \co{rest}, seems to offer its
contents in a wild spontaneity. Arbitrariness is pure
{immanentism}, \co{dissociation} of \co{immediacy} from its context,
surroundings and, eventually, origin.  To account for non-arbitrariness of
truth one does, willingly or not, point to some form of \co{transcendence}:
\thi{correspondence} to \co{externality}, \thi{coherence} to the \co{more} of
the context, \thi{consensus} to the \co{non-mineness} of other humans or culture,
and the last one (which does not have any established name nor any strictly
philosophical tradition) to the revelation of \co{invisibles} and silence of the
\co{origin}.

The tradition correcting the views which bring truth all too closely to
subjectivity observes that denying any sphere of
\co{transcendence}, one denies also the meaning of the word \wo{truth}. Such a
denial indeed solves all the problems in one stroke: there is nothing to talk
about. There may be some dose of positive intentions behind such claims. But the
word itself refuses irresponsibly to die and, moreover, it refuses even to be
reduced to any other word. Perhaps, we are playing our \thi{language game} a bit
wrongly? A bit irresponsibly? A bit too ecstatically? A bit too {immanently}?

Yet, this tradition (of correspondence theories, or realism) encounters the
problem of dualistic ontology which it can neither ignore nor solve.  To some
extent, we follow the opposing tradition (of immanentism, or idealism), namely,
to the extent that every \thi{what} is relative to (our) \co{distinguishing}. In
the world of \co{distinctions}, there is no strict dualism: the {meanings} of
our \co{signs} and the \co{distinctions} with respect to which their truth is
constituted are essentially of the same kind: they are both \co{distinctions} in
the same \co{indistinct}. Almost as Frege demanded of a correspondence that it
\citet{can only be perfect if the corresponding things coincide and are,
  therefore, not distinct things at all.}{FregeTruth}{p.86. Or, as already
  Plotinus observed, albeit with a much more profound reference to human
  reality, only with respect to the intellectual realm: \wo{the object known
    must be identical with the knowing act [or agent], the
    Intellectual-Principle, therefore, identical with the Intellectual Realm.
    And in fact, if this identity does not exist, neither does
    truth.}$^{\ref{cit:actIsObject}}$}

There is, however, a difference between the two in that the \co{actual
  distinctions} of mere \co{signs} can  merge into the \co{non-actual}
%or \co{external}
ones. Our \thi{immanentism} of relativity goes along with the fundamental
importance of \co{transcendence}, of \co{non-actuality} serving as the measure
and corrective of the \co{immanent signs}. We disagree completely with any
reductionistic attempts, with any form of verificationism\ftnt{Later
  Wittgenstein's practice seems to have strong pragmatic roots, and
  verificationism is a recurring theme. \citef{Rather, we must first determine
    the role of deciding for or against a proposition. [...] Really \thi{The
      proposition is either true or false} only means that it must be possible
    to decide for or against it.}{Certain}{\para 198, \para 200} } or
utilitarianism\ftnt{We are tempted to call so the pragmatic (mis)understanding
  of truth. It has both aspects of verificationism \citef{Truth \la{ante rem}
    means only {\em verifiability} [...]}{JamesTruth}{p.61} and utilitarianism,
  according to which truth \citef{is distinguished from falsehood simply by
    this, that if acted on it should, on full consideration, carry us to the
    point we aim at and not astray}{PierceFix}{5}
% \citef{To \thi{agree} in the widest sense with a reality {\em can only mean to
%     be guided either straight up to it or into its surroundings, or to be put in
%     such working touch with it as to handle either it or something connected
%     with it better than if we disagreed}.}{JamesTruth}{[p.59]}
Unfortunately, as already Russell pointed out, it is not only highly problematic
to specify the \thi{point we aim at}; the criterion of eventual \thi{betterness}
or usefulness is often useless if one were to use it for determining truth -- it
is much easier to ascertain that \wo{snow is white} than to figure out what
might be the use of such a truth.}
%
reducing truth to some \co{visible}, preferably observable if not measurable elements of
experience.  If one wants to insist that \wo{only what serves life is true},
then one must also add that \wo{only truth serves life}, if \wo{truth is what it
  is expedient to believe in} then also \citet{what is {\em really} true it is
  good to believe and evil to reject.}{PierceBerkeley}{[my emph., p.87]}
Constructing truth as some ideal limit terminating inquiry,
as a regulative idea extending gradually its actual scope, arises from the same
immanentism, albeit in an attempt to take care of the transcendent aspect as well,
adding it, as if at the ideal terminus of interminable sequence. By this addition it really inverts the immanentism and makes the
reduction especially well visible: for now no actual, 
visible and temporal sign can be (considered) true.

We get closer to the correspondence theories observing that if truth gets reduced to any
criterion then we have really dispensed with the very idea of truth which is
exactly the last norm remaining when all other criteria have been violated.
% This is an aspect of the absoluteness of truth -- relativity to the regions of
% the world, to the levels of drawing the distinctions and to the degree of
% conformance notwithstanding.
In particular, along with all criteriology, there disappears also what for it
appears as a big problem: the 
truths which might remain forever unknown.\ftnt{Certainly, our ontology of
  relativity to \co{existence}, makes also all truths, except the \co{one},
  disappear when no \co{existence} is left. But it does not imply that all the
  truths a particular \co{existence} discovers during its life time disappear when this
  \co{existence} dies.}  Taking $D$ (in our \thi{definition} 
\refp{pa:notArbitraryD}.\refeq{eq:truth}) to be \thi{the number 
  of brontosauruses that ever lived}, then we may meaningfully say that the
sentence \wo{The number of brontosauruses that ever lived is precisely 75,278}
is true (or false) with respect to  $D$, although we will never know which one it is.

We follow the correspondence theory in that truth expresses an agreement:
between the \co{immanence} of a \co{sign} and the \co{transcendence} where its
\co{meaning} resides. But this agreement is not between two incommensurable
elements. Truth is a \co{trace} of \co{transcendence}, of the fact that,
eventually, \co{I am not the master}. But this ultimate \co{confrontation} is
lived as the \co{unity} of \co{actual} and \co{non-actual}, of the
\co{distinctions} which reside on both sides of the boundary of \co{actuality}.
We do not live in two incommensurable worlds of mental and external affairs --
we live in one and the same world emerging through the
\co{distinctions} relative to our \co{existence}, which we draw and recognise
{\em in} the \co{indistinct}. 
% : the \co{external} things of the
% world are (limits of) such \co{distinctions} and so are our understandings of
% them.  Completely different \co{distinctions} will confirm the truth of
% \wo{There is a cat on the mat} and of \wo{I love her}, but in both cases these
% are \co{distinctions transcending} the \co{immediacy} of the \co{signs} which claim
% their presence.

% \ftnt{Slightly different \co{distinctions} would need to be present in
% the \co{actual} situation to make \wo{There is no cat on the mat}, \wo{There is
%   a cat or a dog on the mat} true.
% % But this is still trivially something we can
% % see, this is still something addressing the perceptually drawn \co{distinctions}
% There is no trouble with the logical constants like \wo{not} or \wo{or} for they, too,
% amount to drawing the \co{distinctions} in the way peculiar to their meaning.}

\pa 
Besides trying to balance the merits of these two views\noo{(which are contrary only
when \co{dissociated})} we have, of course, emphasized that truth depends on the
level addressed by the \co{signs}.  The relevant \co{aspect} of
\co{transcendence} is, except for the highest level, the
\co{horizontal transcendence} corresponding to the level of things addressed by
the statements (theories, views) which one wants to judge with respect to their
truth or untruth.\ftnt{\wo{It is hot} is confronted merely with the immediacy of
  the sensation; \wo{There is a cat on the mat} will be confronted with
  the \co{actual} facts (as will be \wo{There is no cat on the mat}!); an
  elaborate theory will be confronted with observations of its predictable
  consequences, with the requirements of internal consistency and, perhaps, of
  conformance or commensurability with other accepted theories, etc.; the
  general ideas, like \wo{Man is what he leaves behind him}, will be confronted
  with other, similarly general ideas, with the personality of one who
  pronounces them, with the pronouncements of others and, eventually, with the
  personal intuitions and convictions.} And so, even if we grant some
plausibility at the level at which the respective theories operate, they hardly
retain it with respect to the lower, or higher levels. The kind of theory of
truth one is able to propose depends primarily on the kind of things addressed,
because things of different levels are involved in different forms of
\co{transcendence}.

Thus truth borrows its specific character from the things addressed but it is
all the time a \co{trace} of the \co{original confrontation}.  \citet{Thus
  veritable truth is not accordance with an external; it is self-accordance; it
  affirms and is nothing other than itself and is nothing other; it is at once
  existence and self-affirmation.}{Plotinus}{V:5.2} From this \co{original
  foundation} it borrows the expectations of addressing something -- and hence
itself being -- unchangeable, one and shared by all.  But as long as it is
concerned with relative beings, the truth itself can only be relative; only what
concerns the \co{absolute} can be \co{absolutely} true.
%
Relativity of truth means only relativity of all possible \thi{whats}, of the
addressed 
\co{distinctions}. But given the addressed subject-matter or situation, the truth
with respect to this relative context can be definite and \thi{absolute} -- the
\co{actual signs} can conform fully to the addressed \co{distinctions}, the
relation between S and D, as indicated in \refp{pa:truth}.\refeq{eq:truth}, is
perhaps the 
matter of degree but not of any relativity. \co{Above} all relative \thi{whats}, 
the \co{absolute} truth is only the ultimate \co{that}, the reminder \co{that}
there is (truth) which can be expressed and manifested but never captured and
exhausted by any \co{actuality}.  Truths of the lower levels need not conform to
such absolute standards. The lower need not mimic the higher, need not
attempt any similarity, for this is impossible. But it must not forget the
higher either, for then correct observations can turn into false ideas, true
propositions can turn into false theories, correct arguments into wrong causes
and series of right decisions into disastrous mystifications. Truth, let us say,
is a \co{concrete participation} in the \co{traces} of \co{transcendence}, and
as there are different forms and levels of \co{transcendence}, so truth of a
lower level may turn out to be a falsehood of a higher one.  Truths which stop
short of the \co{absolute} remain relative, and this is a common lot. But if
they forget it, if they forget the \co{absolute}, they start imperceptibly to
claim absoluteness for themselves and thus turn into falsehoods.  \citeti{Your eye
  is the lamp of your body.  When your eyes are good, your whole body also is
  full of light.  But when they are bad, your body also is full of darkness.
  See to it, then, that the light within you is not darkness.}{Lk.}{XI:34-35/Mt.
  VI:22-23}

%As the Arabic proverb says: \wo{Trust God but keep your camel tight.}

% When applied to the sphere of
% relativity, it easily becomes an \co{idol} (of objectively univocal precisionl,
% of subjective certainty, of both), that is,  untruth.


%%%% ???? fix this ending !!!!

%%%%%%%%%%%%%  END of 028truth 
%%%%%%%%%%%%% only \noo{....

\noo{ The problems with truth arise as a consequence of the dissolution of the
  agreed reference frame given by D. It is insecurity about the meaning which
  creates insecurity about truth. One can always bring in new aspects and claim
  their relevance, one can always differentiate further the meaning -- sceptics
  have almost unlimited field of maneuver.
  
  It is trivialities of the kind \thi{Snow is white} which serve as the most
  stable and irresistible examples of truths because they relate to the most
  obvious and common reference frame of perception.\ftnt{It may be telling to
    observe that crucial examples used by Russell in his (in our opinion, mostly
    adequate) critique of other theories of truth (like pragmatism in
    \citeauthor*{RussellPrag}, coherentism in \citeauthor*{RussellTruth}) are
    precisely of this kind.}

  
  \ad{Coher/consensus} Reduction to verification leads naturally to some form of
  coherentism. Truth refuses to be reduced completely to merely immanent
  dimension as this would signify also its reduction to subjectivity, while the
  intuition that truth is the \co{trace} of \co{transcendence} in our
  \co{actuality} is hard to ignore. Coherence with a wider scope of \thi{truths}
  takes over the role of transcendence in relation to every single \thi{truth}.
  Whether it is coherence within the verification community (as Pierce, James or
  newer French sociologism), or else within the set of mutually
  supporting/verifying truths (Pierce, Quine, Wittgenstein (?)) is only a ...
  
  Consensus gives a mystical power to a community. It may have such a power but
  not with respect to truth, only with respect to falsehood. Truth by consensus
  is dependent on common acceptance, giving consent to it. We can imagine some
  legal ways of consenting (passing laws, etc.), but these do not extend to the
  domain of truth (at most, they may falsify truth). This is so, because
  community does not consent in the same way as individual does... Nobody will
  claim that we may have a case where the community consents to No while all
  individuals to Yes (or vice versa). In fact, here communal consensus consists
  exclusively of the consensus given by every single (involved, or for other
  reasons eligible) individual

  
  \ad{Deflationary}\label{pa:deflation} Deflationary\ftnt{We do not
    differentiate between the possible variants of deflationism (minimalism,
    redundancy theory, reductivism, etc.), as they seem to share the features
    which are relevant to ur discussion.} no theory, jut reduce truth to meaning
  (with no meaning theory, deflationary is empty, and says only that all `truth'
  is needed for are indirect references -- this, however, is no theory of
  `truth').  Now, indeed, \wo{I smell the scent of violets} may seem to have
  exactly \citet{the same content as the sentence \wo{It is true that I smell
      the scent of violets.} So it seems, then, that nothing is added to the
    thought by my ascribing to it the property of truth.}{FregeTruth}{} But
  \thi{complete thoughts}, the assumed though never precisely identified
  \thi{propositions}, are by far all the distinctions which can be made in the
  texture of the indistinct and, then, of the world.
  
  The problem with this \wo{theory} is that it attacks a false enemy. In the
  sentence \wo{~\wo{Snow is white} is true.} the \thi{is true} appears as a
  second (or higher) order predicate applicable to the sentences. In a sense, it
  is exactly what happens when hearing that snow is white we are asking if it is
  true. But the whole point is the meaning of this last phrase, is the question
about what it means for a %(let's say) first-order
sentence to be true.
%
% Notice that here quotation marks can be put only around the whole thing:
% \wo{..for first-order sentence to be true} and not as above, \wo{... for
%   \wo{first-order sentence} to be true.} We know (or, to be pedantic in the
% truly pseudo-scientific manner, let's say), we all assume that (all, or at least
% some)
%
Most expressions of the language relate to something extra-linguistic, no matter
what it is, let us talk about as \wo{the world}. We thus have the starting
point, where X is some \co{sign}, probably a sentence \equ{X -----?------ the
  world \label{eq:defA}} The self-referential capabilities of the language allow
us to express -- or rather to refer to -- this relation within the language.
Thus we can say things like \equ{The sentence \wo{X} has the relation \wo{?} to
  \wo{the world}\label{eq:defB}} All deflationists are able to claim is that in
most cases (when, as already Tarski observed, the \wo{Sentence X} is given
directly) the meaning of \wo{?} in (\ref{eq:defB}) is the same as the meaning of
? in (\ref{eq:defA}). They kill a paper tiger of a higher-order,
intra-linguistic definitionalism. The real question -- namely what is ? in
(\ref{eq:defA}) -- remains untouched.  Deflationism wants to claim that, since
the higher-order statement (\ref{eq:defB}) means the same as the lower-order one
($X$ in (\ref{eq:defA})), the whole issue of the relation ? disappears.
Unfortunately, our primary interest is not in which propositions \wo{X} are true
-- we are interested in true propositions.  As Ramsey observes, \citet{the
  problem is [...]  as to the nature of judgment or assertion [... I]f we have
  analysed judgment we have solved the problem of truth; for taking the mental
  factor in a judgment (which is often itself called a judgment), the truth or
  falsity of this depends only on what proposition it is that is
  judged.}{RamseyFacts}{p.106} Indeed, the equivalence of \wo{The sentence
  \wo{X} is true} and \wo{X} implies that the question about truth should be
answered by elucidating the question about the meaning (which is, however,
different from the question about asserting a proposition). But this still
leaves the whole problem (of the relation ?) open!  We can here agree with the
observation that \citet{once it is allowed that the role of \thi{true} is to
  mark a particular kind of achievement, or failing, on the part of a
  proposition, contrasting with its being warranted or not, there will have to
  be decent sense in the question, what does such an achievement, or failing,
  amount to?}{CrispinTruth}{p.215} Formally, it may be that \citet{the truth
  predicate is the device of disquotation}{QuineTruth}{p.13}, but this is only
an expression of the semantic ascent, that even when we move to expressions of a
higher-order form, we are still interested primarily in the objective reference
to the world.

Dubious affair of: we can't say what it is -- so, we do not know what it is --
so, let's be pragmatic, we do not need it -- so, let's remove it

} %end \noo Consensus...

\noo{ Logic in the broad sense is a part of rhetoric, is a part of the art of
  speaking. It is a very limited part, related only to the most \co{precise}
  aspects of rhetoric. As such it is anything but absolute.  At best, it may be
  helpful in organising one's way of speaking so as to make it most
  understandable. Contradictions and paradoxes are not figures of thought (or
  non-thought) but figures of speech.  Their misuse is a sin just as their use
  is an art.  }
% not contradiction principle + REFERENCES
%http://plato.stanford.edu/archives/fall1997/entries/logic-paraconsistent/


\subsection{As below, so above}\label{sub:lowHigh}
%\tsep{bottom-up processes}
The impression might have accumulated that relations between the levels concern
only \co{founding} of the lower by the higher ones. This is, indeed, the
fundamental relation but not the only one. As we have noticed contrasting
\co{invisibles} and \thi{forms} in \ref{sub:formMatter}, and then discussing
memory in \refpf{pa:memovirt}, there is also a flow bottom-up, in which
\co{actuality} influences the \co{invisible distinctions} and contribute to
formation of more \co{virtual nexuses}.

\noo{The \thi{bowed serial position curve} occurs when people are asked to
  remember a list of items (e.g., numbers), given {\em one} at a time. Most
  people tend to remember well the items from the beginning and from the end,
  but not from the middle, even though at the moment of the actual stimulus
  (each item) the subject does not know when the list is going to end.  Surely,
  these two `events' constitute a \co{cut} as they are more significant than the
  items in, and in particular in the middle of the list. They are more
  significant because, among other things, the subjects {\em are told} what's
  going to happen, that they will be given a list whose end will occur without
  any warning. This foreknowledge certainly plays a role in identifying -- and
  hence also remembering -- the end of the list, once it occurs.
% Fetching the most recent items from the memory may be ascribed to their presence
% in retention, from which the middle items have been expunged.
  
  eventually, one will remember only that there was a list which had beginning
  and end\ldots But is this last (that it had beginning and end) something one
  {\em remembers}? One may forget {\em how} it started and ended, but hardly
  {\em that} it did\ldots This may seems backward-projection }

\pa\label{acts} \co{Acts} are limited to the \hoa\ in that the unity of a single
\co{act} is consummated within this horizon, with the \equi\ \co{aspects} of the
\co{actual object} and \co{actual subject}.  But, of course, we do not \co{act}
in a completely spontaneous, that is, \co{dissociated} and meaningless way.
Every \co{act} has a \co{rest}, is anchored in a wider context.  The unity of a
complex of \co{acts}, of an \co{action} or even \co{activity} is constituted not
by their \co{objects} (they may vary and change) but by their \ldots well,
objectives, \co{motivations}, eventually, \co{inspirations}.  An \co{object} of
an \co{act} follows (is chosen by) the purpose of the \co{action}; the purpose
of the \co{action} follows the \co{motive} of the \co{activity}; and
\co{motives}, often life-long \co{motivations} and traits, are in turn
\co{expressions} of the \co{inspirations}.
%Provided, that everything is fine.

Thus, like every \co{actual experience}, an \co{act} is a cut through and out of
the sphere of \co{experience}, across all levels of one's being, in particular,
across the sphere of \co{visibility} as well as \co{invisibility}.  The
\co{invisible} limit is, eventually, the \co{one} which is \co{present} only
along with the cuts through all the intermediate levels.  The \co{presence}
influenced by the \co{act} can never be its intentional \co{object}.  To the
extent \co{I} try to make it such, it withdraws and changes its character.  In
the most trivial cases, if \co{I} think \wo{I have to learn swimming.  I have to
  learn swimming\ldots} while trying to follow the instructions, \co{I} will
have very hard time.  The best \co{I} can do is to concentrate on following the
instructions.  If the intention of \co{my act} is \thi{to be good}, \co{I} may
happen to do a fine thing, but the \co{invisible aspect} of the \co{act} has
then withdrawn beyond the horizon of this intention and brought forth something
more than what was intended.  (Probably, what motivated \co{me} in the first
place {\em to try} to be good.)  It is not that \co{I} am not conscious of what,
eventually, hides behind my \co{acts} -- it is only that \co{I} {\em can not
  possibly} make it \co{reflectively actual}, in any case not in the \co{act}
itself.  This \co{rest} of every \co{act} is the witness, on the one hand, of
its anchoring in the wider sphere of my being and, on the other, of \co{act}'s
possibility to influence it.

\noo{What happens is entirely analogous to what we have seen with memories
  retreating into the sphere of \co{virtuality}.  The contents of \co{acts} (and
  this means, the full contents, both the \co{objective}, \co{visible} and the
  hidden \co{invisible} motivation and background) are engraved in the memory
  and, gradually, influence the structure of the \co{invisible} sphere.  It is
  however crucial that, although each \co{act} is associated with the sphere of
  \co{invisibles}, such associations are thoroughly personal.  Statistically,
  one may and one does attempt some classifications, but it is never certain if
  the \co{acts} commonly known to produce desirable effects on the person will
  do so for a given individual.  \co{Vague} generalities are all we may expect
  from impersonal descriptions and general theories.  }

\pa A trivial example can be formation of more and more advanced \co{concepts}.
One will usually start with a very rough and simplified understanding of
something, but by prolonged study one's \co{concepts} will be refined.  One
tends to confirm it only by observing an increased ability to solve \co{actual}
problems, perhaps, to relate various aspects and provide more sophisticated
\co{actual} descriptions.  But it is not so that more knowledge gets simply
accumulated as separate pieces in a big sack.  With the possible exception of
photographic memory, a person forced to memorise more will also form more or
less explicit structures for arranging the increasing amount of facts.  More
{\em or less} explicit!  The less explicit ones may not be accessible to
observation or introspection but, the claim is, they are formed nevertheless.
Arranging mathematical or strictly formalised knowledge in such a way, one will
typically identify more central results and theorems which allow one to derive
secondary results.  But this is only a plain \co{visible} analogy.  Arranging
less formalised knowledge and memories, one will also develop structures which
researches on memory try to unveil.  For our part it is only the most rough and
general of such structures which is interesting -- the structure of
\thi{compressing} a manifold of \co{actualities} into a more unified \nexus,
perhaps, gathering them under the unifying \co{sign} of a \co{concept}.

We are not talking here only about memory in the common sense of the word, for
the \thi{compressed} \co{actualities} can, in fact, slip out of memory.  But
they have contributed their part to the formation of something which we might
identify with a \co{virtual} center.  We have indicated a generic example of
such a process when discussing memory and possible transition of \co{actual}
events to \co{virtuality}, \refpf{pa:memovirt}.  In the case of \co{conceptual}
constructions, this can manifest itself as the acquired ability to \thi{intuit}
a large number of related problems in {one} \co{act} (called often an \wo{act of
  intuition}, but what matters to us is only that it is {\em one} \co{act}).
Prolonged and dedicated study of an area leads to a {\em development of
  intuition}, to the state where a person is able to grasp, in one \co{act}, a
variety of aspects and to know that and how these aspects are related and might
be \co{actualised}. \citet{The successful practice of intuition requires
  previous study and assimilation of a multitude of facts and laws. We may take
  it that great intuitions arise out of matrix of
  rationality.}{IdealistView}{V:1.\kilde{p.139} Although intuition as described
  above concerns only unity of \co{visible complexes}, the mechanism of the
  great and deep intuitions, which are the primary concern of Radhakrishnan,
  seems to be of the same kind.}  The same happens when learning some skills,
like swimming.  At first, every movement has to be consciously attended and
actively controlled.  It is only through repetition and exercise that I, or
rather my body, \thi{gets it}, that all the movements, their sequences and
mutual relations converge into an intrinsic and organised whole.

%the same on \co{concepts} (built botom-up) and other
%\co{actual distinctions} which \thi{infect} feelings and higher signs;
%acts \ldots Shall we also include prayer, acts of compassion, etc. for 
%building up the attitude of \yes?

%\tsep{\co{Actual} influence ?}
\pa Now, such processes happen not only when we try to learn something but also
when we do not.  The level to which the involved \co{actualities} can be brought
may vary, but the idea is the same -- they get organised, or disorganised, they
get stored for an easy access in long term memory, or they get \thi{forgotten}.
As they \thi{move upwards}, they enter various (and hardly recognisable)
complexes and, eventually, disappear in \co{virtuality}.  
But although they may thus disappear in their \co{actual} form, they are
retained in the \co{invisible} centers to the formation of which they have
contributed.

In such a process, \co{invisible} centers can be formed, from which it may be
impossible to extract the \thi{original parts}.  This can be exemplified, for
instance, by the subconscious formation of complexes (in the Jungian sense of
the word) or other subconscious processes which analysis may only try to bring
forth again. Saying that \wo{time heals all wounds} we refer to quite the same
process of \thi{covering up} or, perhaps, \thi{suppressing} some
\co{experiences} by a long series of \co{actualities}. The original experience
and its memory seldom will disappear, but they can gradually dissolve, lose the
possibly violent or damaging potential, in an aura of works and days which force
man to focus on other things. 


Education and upbringing of children offer innumerable examples.  \citt{Sow an
  action, and you reap a habit; sow a habit, and you reap a character; sow a
  character, and you reap a destiny.}{James \citaft{Cotkin}{p.69}} True, but
which action will lead to what habit, which habit to what character, etc., are
things which we can only approximate in general terms.  \citet{When a man dwells
  on the pleasures of sense, attraction for them arises in him. From attraction
  arises desire, the lust of possession, and this leads to passion, to anger.
  From passion comes confusion of mind, then loss of remembrance, the forgetting
  of duty\ldots}{Bhagavad}{II:62-62} Plausibility of such observations depends
always on more specific and personal aspects and traits which determine
\co{concretely} what is and what is not \wo{dwelling}, \wo{lust}, etc.
Apocryphal stories describing childhood of a saint or a hero are exactly
reflections of such a rough and general understanding of the ways lower events
and actual experiences accumulate into higher traits of character and
personality, even if they usually reflect the only form under which such
\co{invisible} processes are \co{actually visible}, namely, as a
\thi{predetermined destiny}, as being \thi{marked by the gods}.
  
Dependency on the \co{concrete} personal traits is reflected by the fact that
each step upwards requires, of course, time.  A single event or \co{act} has
seldom deep consequences, and repetition of prescribed \co{acts} may even have
consequences quite opposite to the intended ones.  For between any two controlled
\co{acts}, many others happen, and even if the person does not \co{act}, he
still \co{experiences}.  Very few are lucky enough to have a wise tutor who is
able to give a constant, personal advice.  For the most, we learn and acquire
our habits and character through roughly accidental interaction with parents, family and
immediate surroundings.  Which, for the most, means, we do not acquire much
character and even less destiny.  Again statistically, the less advice, control
and guidance in upbringing, the less strength and character, that is, unity of
the personal being.  For loose and free confrontation with the indefinite, perhaps
only freely chosen, objectives, as many a pedagogue would say nowadays, the
\wo{promotion of independence and individual creativity} by avoiding obstacles
and high demands, breed perhaps individualism but hardly individuality.
%de-individualised individuals.

Tedious work, perhaps boring work is, more often than not, a blessing which
teaches a young person more than superficial overstimulation.  But sure,
whatever happens will have an influence on the habits, character, perhaps even
destiny.  We always get what we sow, but we hardly ever know what we are sowing.
\wo{We live forward, but we understand backward} says Kierkegaard and deep wisdom,
which is also knowledge of the future, is a rare exception.  It requires, above
all, the acquaintance not only with the seeds one uses to sow but also the soil
into which they are sown.  For \co{acts} and \co{actual experiences} are always
immersed into the \co{invisible} sphere of \co{concrete} personality and 
statistics may be helpful but will never suffice.

\noo{
\wo{She does not seem to love me any more. I have tried this and I 
have tried that. Nothing helps. I have no idea what I can
do\ldots Help!} Of course, I have plenty ideas of what might be done,
only that none of them seems to be meaningful, seems to serve the
purpose.  Because all \co{visible} purposes lost their connection to
the motives which have been their underlying theme. Perhaps, the love 
just died away and there is nothing to do. But perhaps \ldots
}

Just like a wise teacher knows how to proceed and what to teach, knows it and
follows the course even if none of his pupils understands \thi{why} and
\thi{what for}, so an adult may sometimes be in a need of a good advice.  A good
advice may be something I just did not realise but recognise once it is given.
But more profoundly, a good advice tells me to do something which I do not
understand, something I am {\em not able} to understand.  It tells me to act in
a way which I do not know whereto leads.  Surely, one has to have a lot trust in
the person to follow such an advice.  But if it was a wise advice, I will learn
once I arrive, because acting in the recommended way (and if heavens so wish)
will eventually lead to a new \co{unity}, if not to the place I had imagined and
wished, so in any case to a new resting place. Wisdom is the capacity to give a
good advice which, like a teacher's knowledge and instruction, transcends the
horizon of the one who is in need of it. Transcending this horizon means here
exactly that it knows the effects of accumulated activity over time, that it is
not restricted to the mere \co{actuality}.

Ontological \co{founding} of lower levels by the higher ones does not concern
any specific contents but only the general structure and character of each
level. The influence of the lower levels on the higher ones, on the other hand,
concerns specific contents, like bringing up a particular (kind of) person,
developing particular (kinds of) skills. We will consider such a more specific
relation between the contents of various levels in the following Book. But
before that, let us venture on a small excursion\ldots


\noo{
  \tsep{Acts ... old}

{\small{
\begin{tabular}{l|l|l|l|l|l||l|l|l}
& sign vs.  & appears & reactive & attitude 
\\
& correlate & in & Gef\"{u}hl & \\ \hline
1 &  coincide & sinnlich & sensation & take- \\
&   & Gef\"{u}hl & (percept.) & -avoid \\ \hline
2 &  collection & vitale-, & moods & thinking \\
&  relations  & LebensG. & & control \\ \hline
3 & expression  & seelische & my life is.. & commit, \\
&  & (geist. 502) & meaningf/less  & I choose  \\ \hline
4 &  revelation  & geistige & everything &serve,\\
 &  symbol & Gef\"{u}hl & world is .. & I have to 
\end{tabular}
}}

\levs{9.5}{Reaction / distance \co{sign}-signified}
{sensations / coincidence}
{\co{moods}, thoughts, plans / related, \thi{finite and limited}}
{approval, respect, commitment and negations / \thi{finite but unlimited}}
{faith, adoration, awe, submission and negations / \thi{infinite}}

}


\section{The origin of mathematics}\label{sec:om}
This section is a digression because we are interested in a unified picture of
\co{existence}, not in philosophy of any particular region of Being, let
alone, of (any particular) science.  The current Book does not present
epistemology as any \thi{theory of knowledge} guaranteeing any certainties and
resolving all doubts or -- what appearing more modest is even more presumptuous
-- offering a method for resolving doubts which might possibly arise. Our
epistemology (if epistemology it is) addresses only the general ways of meeting
\co{transcendence} and its \co{actual reflections}. The search for truth and the
ineradicable conviction that it not only means something but also is better than
falsehood, is only one special form of this fundamental \co{thirst} \co{founded}
in the \co{awareness} of the insufficiency of \co{visibility}; curiosity
or fascination, confusion or boredom, bafflement and even despair, are
others and all can occur in various combinations with each other.  Our
epistemology (if epistemology it is) presents only some 
\co{reflections} which might help maintaining the continuity
between the \co{actual} contents and their \co{transcendent} origins, the
tension without which \co{dissociated actuality} turns into dull 
emptiness devoid of the sense of meaning and reality.  Scientific activity can
be an expression of such a confusion or curiosity, but questions about the \co{actual}
scientific contents, the \co{actual} results of \co{objectivistic}
reductions, do not have much to do here.\noo{Usually (though not too often),
  only the most prominent researchers in some field reach the experience of the
  field's fundamental limitations.  Ignoring the outsiders, it usually takes a
  most prominent scientist to express the genuine concern about the fundamental
  meaning of science and its (in)capacity to diffuse this meaning.}

Yet, this digression has its reason.  Mathematics has always held a particular
place among the sciences.  Indeed, to such an extent that most other sciences
try desperately to approach mathematical standards (sometimes for better,
usually, for worse).  Good reasons for the prominence of mathematics can be
discerned at the level of abstraction at which we are moving.  One shouldn't
probably go as far as to say that the beauty and purity of mathematics have, in
themselves, existential import. But they are reflections of the spiritual dimension
of \co{existence} in the degree unmatched by any other science. The \la{a priori}
character of mathematical objects and constructions makes one suspect, if not
\co{clearly recognise}, the ultimately
\co{transcendent} origin of mathematical truths.

Discussing truth, %in \ref{sub:truth},
we related occasionally its levels to various forms of mathematical logic
(\refp{pa:logicA}, footnotes~{\small{\ref{ftnt:locales}, \ref{ftnt:logicB},
\ref{ftnt:logicC}, \ref{ftnt:logicD}}}.) We
would have to agree that such a {classification} is a bit arbitrary.  Formality
of logical systems allows one to mix various
notions, like negation, disjunction, as well as semantical constructions, and
apply them across all the levels.  And this is so because mathematical systems
do not describe things, feelings, actual perceptions, but uniform mathematical
objects. They may be motivated by extra-mathematical considerations but once
formulated they become part of mathematics. One mathematical theory can
postulate complex mathematical objects vastly different from those
postulated by another such theory.
But to the extent they are both {\em mathematical} they address, eventually, basic
mathematical objects and therefore, can be related to each other.

The question concerns not the detailed choices but the fundamental issue: what
is the relation (if any) between a logical, or generally, mathematical system
and extra-mathematical reality?  What are the ultimate objects of mathematics,
and what, if any, is their relation to experience?

% We will address these questions relying again on the general conviction that
% unity of \co{actual} differences is seldom, if ever, to be found in the
% \co{actual} state of affairs, in the affairs of \co{actuality}, but rather in
% their emergence from a \co{virtuality} of an origin.


\pa
In every science one finds the hard seed of pre-scientific reality and 
beautiful flowers of scientific imagination. The former, the origin, 
is rooted in our intuition and \co{experience}. As the virtual origin it 
neither contains all possible details of future results nor 
determines the ways in which science can develop. It only precedes any 
such development, lies beyond and before it, and lends its basic notions 
some intuitive content which can be appealing even to the uninitiated 
laymen.%\ftnt{Derrida Introduction to OG: origin ``sends'' its inspiration... to
       %which one tries to return in the zigzag...}

Origin is not a foundation. In fact, laying down a foundation marks already a
definitive break with the origin. It amounts to internalising the original
intuitions in terms of a language which from now on will develop according to
its own standards. We do not want to review the arguments between formalists,
Platonists, intuitionists, etc. We do not even want to see the differences
between classical and non-classical mathematics, between geometry, arithmetics,
algebra, topology, etc. Such distinctions involve one into mathematical
arguments. The question about origin is, on the contrary, the question about
what makes all these branches into branches of one and the same mathematics,
what makes the results of Pythagoras, Fibonacci, Viete, Riemann, Cantor equally
mathematical.

Quine's statement that a (mathematical) theory commits one to the ontology
determined by the range of bound variables, is certainly very clear and
convincing.  Indeed, the entities a theory describes are those which can witness
to the truth of the existential statements -- \wo{there exists an $x$ such
  that\ldots} Any particular, not only mathematical, theory has to specify such
entities.  But our point is very different. As we will argue, any mathematical
theory addresses, eventually, only one kind of entities.  As remarked above in
connection with various logics -- differences of modeled phenomena and
motivations notwithstanding, they are all {\em mathematical} logics.  Quine's
ontology -- the range of bound variables -- is still entirely
\co{objectivistic}. The theory represents epistemological apparatus which deals
with particular entities, that is, a particular ontology. One is concerned
exclusively with the objects explicitly treated by the theory, and these are
objects defined already {\em within} a mathematical world. As such, one does not
at all address the issue of origin but at most of foundation and, in fact, a
much more specific issue of differences between the local ontologies of various
theories or formalisms.  Asking about the origin of mathematics we will not be
concerned with such issues at all; we are not asking what objects can possibly
be constructed mathematically but, on the contrary, what primary objects give
rise to the mathematical constructions. The origin of such objects will
be found at the very first stages of differentiation, in the sphere where
ontology has not as yet got dissociated from epistemology.


\subsection{What is a point?} %-- pure distinction} 
\pa\label{pa:pointA} \citet{A point is that which has no part.}{Elements}{I,
  Def.1} It is the residual unity \citet{beyond which there cannot be anything
  less.}{DDI}{I:5/13} Intuition of a point is the same as the intuition of a
\thi{substance}, of a purely \co{actual object}, which is the mere site of its
self-identity, that is, a point.  It is like the least something which still is,
the least something from which nothing can be removed without removing the thing
(that is, the point) itself. Evanescent site of pure \co{immediacy}\ldots

Now, it might seem that to come from \co{actually} given \co{objects} to
mathematical points there is a need for abstraction, since an \co{actual object}
is always a particular thing with all its properties, while point is only the
residual site with no properties whatsoever. It might seem that a point results
from a process of abstraction in which \citet{we obtain from each object a more
  and more bloodless phantom.  Finally we thus obtain from each object a
  something wholly deprived of content; But the something obtained from one
  object is different from the something obtained from another object -- though
  it is not easy to say how\ldots}{FregeHusserl}{} It might indeed seem so, but
only when we assume that \co{objects} are the only original givens and that
their givenness is a primitive, simple \co{immediacy}. Then, indeed, anything
lacking in some \co{actual} content seems to arise from the \co{actual} givens
only by abstraction.

\pa But \co{objects} are not the original givens. On the contrary, \co{objects}
are abstractions from the \co{concreteness} of \co{experience}, results of an
interplay of \co{distinctions} within the \hoa.  Consequently, the process of
\co{founding} does not proceed from \co{objects} towards their \wo{bloodless
  phantoms}, for these phantoms are there, are given along with the \co{objects}
themselves.  An appearance of an \co{object} is equivalent with the narrowing of
the \hoa\ to \co{immediacy} which \co{dissociates} some \co{distinctions} from
their background.  The apparent independence of \co{objects}, not only from the
\co{subject} but also from each other, the fact that \wo{something obtained from
  one object is different from the something obtained from another object,} is
the result of this isolation.

What precedes, in the order of \co{founding}, appearance not only of
\co{objects} but of anything whatsoever is \co{distinction}.  And as in \co{any
  experience}, its whole structure, that is, its whole \co{foundation} is also
\co{experienced} in the \co{immediate self-consciousness}: \co{distinguishing}
particular contents we also \co{experience} (even though not thematically) the
very fact of \co{distinguishing}.

Furthermore, even if there are no rigid \co{distinctions}, that is, no sharp
boundaries \co{distinguishing} \co{precisely} and univocally one content from
another, the fact of their \co{distinctness} is given sharply and
\co{precisely}.  Just as one can be uncertain where one stripe of a rainbow ends
and another begins, so one is certain that they are different stripes, that each
has, if not a sharp boundary, then in any case some kernel which is distinct
from the kernel of another stripe.  The mere, yet quite fundamental, fact of
\co{distinguishing}, with the immediate awareness of definite distinctness, is 
\co{pure distinction}.  It has the same character as the pure \co{precision} of
givenness of an \co{object}, its mere \thi{being there}, in the \co{immediacy}
of \co{reflection that it is}.
%
\thesisnonr{Intuition of a point is the same as the intuition of a \co{pure
    distinction}.}~\ftnt{\thi{Intuition} here should not be taken in a thetic
  sense, as an \thi{intuition of\ldots}, positing some \co{object} \thi{\ldots}.
  It is an aspect of \co{an immediate experience}, a \co{non-reflective},
  non-positional \co{self-consciousness} of the structure of \co{the actual
    experience}.  Such an intuition, a \thi{point-awareness} is the same as the
  \thi{\co{pure distinction}-awareness}.
  
  This identification can seem to go counter our differing images of the
  two: point is a mere dot $\cdot$, while distinction a line $|$
  splitting the space in two. But these are only pictures. We could bring them
  closer, for instance, if we imagined distinction as a circle $\circ$ (still
  splitting the space in two, cf.~I:\refp{th:cut}). Since no distinction is
  rigid, the exact 
  circumference of the circle is blurred. But this does not make the fact of it
  being made less clear -- the fact which we could imagine as the point at the
  center of the circle. So understood, \co{pure distinction} corresponds to the
  neo-Platonic monad responsible (albeit always in a very unclear way) for the
  generation of actual numbers. \citef{The cause is in it [monad], and they
    [numbers] are causally in it because it subsists as the beginning of all
    numbers. [...] intelligible numbers are so poured out from the monad that
    in some way they become clear in the mind; next flowing out from mind to
    reason [...]}{Periphy}{III;p.172} \citef{In the case of numbers, the unit
    remains intact while something else produces, and thus number arises in
    dependence on the unit: [...] there is, primarily or secondarily, some form
    or idea from the monad in each of the successive numbers -- the latter still
    participating, though unequally, in the unit [...]}{Plotinus}{V:5.5}
  Distinction between monad generating the numbers and The One is maintained but
  never entirely clear.  (Eriugena calls even Creator a Monad: \citefib{Monad
    which is sole Cause and Creator of all things visible and invisible [vs.]
    created monad in which all numbers always subsist causally, uniformly, and
    according to their reasons, and from which they emerge in many
    forms.}{Periphy}{III;p.172-3}) Very close conceptual associations between
  the two are handled by more or less intricate hierarchies of (degrees of)
  units, like The One, monad, henad, henads, etc., into which we will not
  inquire.  }


Consequently, no abstraction is needed to arrive at a point, \citet{we have a
  direct awareness of mathematical form as an archetypal structure.}{LawsForm}{
  Introduction; p.xxiv\label{cit:numberArchetyp}} It is a structure present in any apprehension of an
\co{object}, an \co{aspect} of \co{any actual experience} and as such is itself
\co{experienced}.

\pa True, abstraction may be needed to posit a point as an \co{actual object},
to \co{reflectively} isolate this \co{aspect} of \co{an experience}.  But this does
not change the fact that it is an \co{aspect} of every \co{experience} and emerges from
there only as a result of \co{reflective} isolation -- it is not a mere
construction, an empty, or conceptual abstraction.  Taking a point in this way,
as an \co{object} of \co{reflection},\noo{Even if advanced \co{reflection}
  involves manipulation of \co{signs}, usually artificial \co{signs}, and
  construction of complicated models, primarily, it is only abstraction from
  \co{experience}, it merely reflects some of its aspects without preserving
  others.  In mathematics, \thi{reflection} is a property of a mapping between
  structures.  A mapping $f:A\rightarrow B$ \thi{reflects} a property $P$ if
  from the fact that $B$ has $P$, $P(B)$, we can conclude that also $P(A)$.  In
  general, our \co{reflection} does not have this property, since its function
  is to {objectify} and \co{externalise} aspects of \co{experience} which are
  not necessarily \co{objective}.  But to the extent \co{reflection} merely
  isolates an \co{actual} aspect of \co{experience}, as for instance in
  perception, it does have this property.  \thi{Reflection} is dual to
  \thi{preservation}: $f$ \thi{preserves} a property $P$ if from the fact that
  $P(A)$, we can conclude that also $P(B)$.  Of course, our \co{reflection} does
  not enjoy this property.}
%
we can specialise the above thesis:
%
\thesisnonr{A point is a \co{representation} of the fact of 
\co{distinguishing}, of \co{pure distinction}.
}

\co{Pure distinctness} can be characterised as the fact that points are mutually
indistinguishable yet distinct, they are distinctions without content,
differences without reasons: like the \co{posited} ultimate \thi{substances} or,
more \co{concretely}, as the absolute beginnings, identical in so far as the
mere fact of beginning is concerned, yet distinct by virtue of the absoluteness
of true beginning. 
\co{Pure distinctions} are the fundamental 
objects of mathematics, the objects which are not results of any 
mathematical foundation but which mathematics inherits from 
\co{experience} -- 
%
\thesis{Mathematics is the science of \co{pure distinctions}.}


\subsection{Numbers -- multiplicity of distinctions}
\pa
Introducing us to the notions of number and counting the teacher 
started to put apples -- one after another -- on the table. ``We have 
one apple. What happens if I put another apple? Well, now we have two 
apples. And if I put yet another one? Well,\ldots'' Did not your 
teacher do a similar thing? 

What should happen if he run out of apples? What should happen if he suddenly
pulled out a pear and put it on the table after a series of apples?  Can you
imagine the confusion?  An apple, yet another apple, more apples, a vague patter
begins to emerge and, suddenly, a pear!!  Not that the kids would for ever lose
the chance to acquire the concept of number but how much extra work for the
teacher!  How would he proceed to explain now that the fact that a pear is not
an apple does not matter at all?  How to explain that a pear is simply yet
another object -- a fruit, perhaps -- distinct from all previous ones?  An apple
is so much an apple that the sixth apple put on the table is the same as the
fifth one -- except that it is the sixth.  A pear after the fifth apple would
not be the sixth -- it is too different from the apples.  It would be the first
pear rather than the sixth fruit.  The difference of content would intrude on
the explanation of the \co{pure distinction} of number.

We do usually count apples separately from pears.  And if we count both we say
we are counting fruits.  Thus Frege says that \citet{number is the extension of
  a concept}{FregeNum}{{\para 68}. Literally: \wo{the number which applies to
    the concept $F$ is the extension of the concept <<equinumerous with the
    concept $F$>>.}} because as soon as we count quite different objects
together we seem to subsume them under some common, more general concept.
\wo{In fact, we do not ask `How many are Caesar and Pompey and London and
  Edinburgh?'~}\ftnt{The idea is, of course, old and renowned. \citef{For reason
    counts different things together with things of the same kind, so that
    clearly persons are counted with persons, qualities with qualities, and so
    forth with other things.}{ClaremTrinitate}{I:\para 46}\kilde{p.31} Ockham
  refers to those who similarly, though with a much stronger empirical bias,
  \citef{[c]oncerning discrete quantity [...] maintain that number is nothing
    but the actual numbered things
    themselves.}{OckSumLog}{I:c.xliv}\kilde{p.138}}
%
In fact, we do not -- but we could!  And counting cities is no different from
counting cities and persons, counting fruits is no different from counting
fruits and houses and the nasty persons one met last week.  Do we then subsume
them under a more general concept? What concept? Insisting on the positive
answer, we would eventually have to say: the concept of a \thi{mere something},
a point, a \co{pure distinction}.

Number does not express a \thi{property} of any concept, but rather the
unlimited ability to ignore any properties, any conceptual differences of
content. It precedes all concepts.\ftnt{We could hardly disagree more with any
  empirical reductions of mathematics like those suggested in the previous
  footnote. On the contrary, \citef{[w]ise men, indeed, do not say that the
    numbers of animals, shrubs, grasses and other bodies or things are related
    to the knowledge of the arithmetical art; but they assign to arithmetic only
    the intellectual, invisible, incorporeal numbers established in knowledge
    alone and not placed substantially in any other
    subject.}{Periphy}{III;p.163} One of those wise men (probably unknown to
  Eriugena) says: \citef{Where lies the need of [say] decad to a thing which, by
    totaling to that power, is decad already?\lin If the Beings preceded the
    numbers and this were discerned upon them at the stirring, to such and such
    a total, of the numbering principle, then the actual number of the Beings
    would be a chance.}{Plotinus}{VI:6.5\lin 10} \citef{[N]umber is proved to be
    the beginning of those things that are attained by reason -- proved to such
    an extent that if number is removed, then reason shows that none of those
    things would remain. [...] Furthermore, it is not the case that any thing
    can be prior to number. For all things other than number attest,
    necessarily, that there was already number. For all things deriving from
    most simple oneness are composed in their own manner. But a composite cannot
    be understood in the absence of number...}{CusaCon}{I:2.\para
    7-8}\label{ftnt:numberFirst}} Eventually, we count somethings, indeed,
points or, to use a synonym, \co{objects}.  The word ``object'' expresses
exactly this pure self-identity, empty identity of a noumenal $x$ which is
itself simply by not being anything else.  Frege's ``bloodless phantom'' is an
\co{object} -- a hardly imaginable site of the ultimate identity of the thing he
started with.\ftnt{This fundamental importance of distinction is well
  illustrated in the logicist's attempts to define numbers. E.g.,
  \citeauthor*{CarnapLog}, follows the procedure suggested by Frege: for number
  2, one begins by stating that at least two objects fall under a concept $f:
  2_m(f) \Leftrightarrow \exists x\ \exists y : x\not = y\land f(x)\land f(y)$.
  Then, number 2 is said to apply to a concept $f$ iff: $2_m(f) \land \neg
  3_m(f)$.  Identity (or rather its negation, distinctness) needed in the first
  part is the undefinable primitive relation of the logical language.}

If one wanted to object that apples on the table are not meant as an analogy of
\co{pure distinction} because they have different positions, appear on the table
at different times, and so on, that is, because they fall within the extension
of a concept where other differences are needed to distinguish between the
objects, then we would only repeat the question: why do the teachers not count
fruits but only apples?  The objection does not change their procedure which is:
make the difference as small as physically possible, make the objects so similar
that removing this last amount of difference would erase the distinction itself.
If one feels a need for it, one might define the empirical analogue of a
\co{pure distinction} as such a smallest possible difference (whatever that
might mean). \thi{Numerical difference} is the notion corresponding exactly to
\co{pure distinction} as distinct from \thi{difference of content}.

\pa\label{pa:numbers} But we still have some road to travel before we arrive at
numbers.  For the present, we only have the \thi{numerical difference}. Now, no
\co{distinction} occurs alone, there is nothing like \thi{the first
  distinction}, only a transition from the undifferentiated \co{one} to the
gradually increasing manifold of \co{distinctions}. The \hoa, which is like a
\thi{snapshot} of \co{experience}, contains always a multiplicity of
\co{distinctions}. Viewing these as \co{pure distinctions}, that is, focusing
only on this \co{aspect} of \co{an actual experience} which determines the mere
distinctness of \co{actual} contents, yields the intuition of a proper
multiplicity, that is, multiplicity of \co{pure distinctions}. Each \co{actual
  experience} is also \co{an experience} of such a multiplicity. This is well
reflected in the most primitive, unary notation for numbers, which merely marks
the \co{distinctions}: I, II, III,\ldots or even better $\bullet$,
$\bullet\bullet$, $\bullet\bullet\bullet$,\ldots

%\pa\label{pa:ordering}
Multiplicity as mere distinctness of the \co{actual} contents, as the
  \co{immediate self-cons\-cious\-ness} of \co{pure distinctions}, is the experiential
origin of a set. It is not yet a number which brings us already closer to a
possible foundation.  What makes a number into a number is not any mystical
quality but its relations to other numbers\ftnt{\citeauthor*{Benacer}}
and an elaboration of such relations is already a matter of mathematical
\co{reflection}.  Let us only sketch the most elementary beginning which follows
from the origin, from the \co{experience} of \co{pure distinctions} and their
multiplicites.

A primitive shepherd who not only cannot count but does not even have a
slightest idea of a number, had probably proceeded something like that.  To
check if all his many sheep return in the evening from the pasture, he let them
out in the morning one by one, marking each leaving sheep as a cut on a stick.
In the evening, he let them in one by one, marking each entering sheep on the
same stick, with another mark next to, or across, one of the marks made in the
morning.  If every morning mark is matched by one evening mark, everything is
OK.  If, however, some morning marks remain unmatched by any evening mark, some
sheep are missing.

The shepherd performs the most natural, in fact, the only possible operation one
can perform on two \co{actual} multiplicities -- he relates them by associating
points in one with those in the other.  He does it in a particular way serving
his particular purpose: he matches each evening mark with only one, but always
distinct, morning mark -- he establishes an injective relation (indeed, a
function) from evening marks to morning marks.  If this happens to be also
surjective (every morning mark gets matched by one evening mark), then the
conclusion is that the multiplicities of sheep in the morning and in the evening
are equal -- there is the same number of sheep.  If the function is not
surjective (some morning marks remain unmatched), the number of returning sheep
is less than the number of sheep which left in the morning.  This is the
well-known set-theoretical definition of ordering of cardinal numbers.\ftnt{A
  set $A$ has cardinality less than or equal to the cardinality of a set $B$, if
  and only if there exists an injective function from $A$ to $B$.  They have the
  same cardinality if there exists a function which is not only injective but
  also surjective (bijective). \citeauthor*{FregeNum}, \para 63, quotes Hume:
  \wo{If two numbers are so combined that the one always has a unit which
    corresponds to each unit of the other, then we claim they are equal.}}

Such an operation is performed not so much on the actual objects (sheep, marks),
as on their collections viewed as mere multiplicities of \co{pure distinctions}.
Indeed, to pose the problem in the first place, to have the possibility of even
asking the question about {\em all} sheep returning, the shepherd had to
\co{recognise} that the relevant aspect is the {\em multiplicity} of {\em
  distinct} sheep.  But any multiplicity is proper -- even if one uses some
particular, \co{objective} tokens, it is always multiplicity of \co{pure
  distinctions}.

The set-theoretical construction of cardinal numbers (as representatives of
classes of equinumerous sets) is already more than their \co{reflective
  experience}. The number 2 does not emerge exclusively as an abstraction from
different collections containing exactly 2 elements.  What would be the basis
for such a generalisation?  It would have to be the notion of \thi{the same
  number of elements} in different collections, as set-theory says, of a
bijective correspondence.  But such a correspondence presupposes that one has
already abstracted away all differences of content, that all such differences
{\em already are ignored}.  The shepherd could not form the idea of representing
the sheep by the marks on his stick, if he did not already have the notion of
the proper multiplicity of sheep.  The marks on the stick {\em represent}
something -- this something are not sheep but their multiplicity.\ftnt{This
  should suggest our attitude towards accounts like \btit{The origin of
    geometry} in \citeauthor*{Crisis} [Appendix 1; also II:9], which is not
  phenomenology of mathematics but of socio-historical emergence of geometry.
  One could be tempted to apply a kind of ontological argument (as those quoted
  in footnote~\ref{ftnt:numberFirst}) against such and similar approaches which
  all repeat, in one form or another, the idea from \citeauthor*{Herodotus},
  II:109, that the art of geometry had its origin in the challenge presented by
  the Nile to the Egyptians, and only later became an abstract science. But
  although \co{origin}, and the \co{original foundation} of mathematics in
  particular, exists only through the \co{actual} manifestations (and, one might
  want to add, its empirical history), it is in no way dependent on, let alone
  reducible to, such manifestations. If \co{pure distinctions} were not given
  originally (and originarily) in intuition, if relations of \co{pure
    distinctions} were not available to us \la{a priori},\noo{ (that is,
    independently from contents of any particular \co{experiences} though not
    independently from \co{experience}), } we would never be able to form an
  idea, to encounter a phenomenon of, say, a circle. \citef{No [sensuous] image
    could ever be adequate to our conception of a [circle] in
    general.}{CrPR}{The schematism of the Pure Concepts of
    Understanding;A141/B180} A circle, an ideal circle is determined and given
  {\em only} by a center, a point, and a radius, that is, {\em equinumerous
    multiplicites} separating each point of the circumference from the center.
  It could never arise as an abstraction from experiences, or as a repeatable
  correlate of acts, in short, as a perfected \ger{Limesgestalt} of imperfect
  circles, no matter how often encountered in nature.  (What would determine the
  direction of such a generalisation, of such a \wo{conceivable perfecting
    \thi{again and again}\ldots}? Husserlian \thi{repeatability} may be,
  perhaps, taken as a characterisation of ideality, but it is only a
  characterisation founded upon this ideality and not other way around.)  We
  could, perhaps, by accident come across and use flat objects which rotate and
  roll, but we would never invent a wheel.  Because wheel is not a
  generalisation of round objects.  It is a circle, an ideal circle (even if in
  practice it is not) which could not be even thought without the relations of
  number and equality of multiplicites. }
%%% chekc again `repeatability'

\newp 

\pa It shouldn't be necessary to go any further, since we already have the basis
for a number system: multiplicites of \co{pure distinctions} (various
\thi{numbers}) and the basics of an ordering relation between them.  The rest is
left for the creative imagination of the mathematicians. As the famous saying of
Kronecker's goes: \wo{God created the natural numbers, the rest is the work of
  man.}\ftnt{Although we can accept this remark, we would never draw from it
  Kronecker's conclusion, anticipating the early XX-th century's focus on
  finitary methods (whether of logical axiomatizations, constructivism or
  computability theory), that only finitely and explicitly constructible objects
  are legitimate mathematical entities.} In fact, a number system consisting of two
numbers only: 1, \thi{many}, contains already all essential features.
\citet{The Tououpinambos [native Americans I have spoken with] had no names for
  numbers above 5; any number beyond that they made out by showing their
  fingers, and the fingers of others who were present.}{LockUnd}{II:16.6.  It is
  interesting to observe that Locke speaks about \thi{names} of numbers.  The
  Indians obviously had understanding of numbers larger than 5 for which they
  only did not have names. Cf. also \citeauthor*{SavageMind},
  \citeauthor*{Hartner}.}  Anthropologists report the existence of tribes whose
whole number system contains only four numbers: 1, 2, 3, \thi{many}.  The
mathematics one can do with such a system is rather poor but it is {\em
  mathematics}, it is a number system.  Although it remains still in a virtual
form close to the experiential origin, it involves already the essential
intuitions on which also more advanced number systems are based.


\subsection{Infinity} %...  -- Chaos} 

\pa Thus point is a \co{reflection} of \co{pure distinction}, the \co{pure
  immediacy}, while number, initially as mere multiplicity, is the corresponding
\co{representation} of simultaneity in terms of \co{pure distinctions}.  Comparison
of multiplicites, not to mention the total ordering relation, are more advanced
constructions which bring us already close to a possible foundation. Just like a
point is a purely \co{actual}, \co{immediate experience}, this intuition of a
number, of multiplicity, is consummated fully within the \hoa. Even if sheep
enter the farm over some period of time, no time is involved in the fact of
having a given multiplicity of them.

Now, just like \co{distinctions} do not come alone, so the numbers do not appear
separately. There is no \co{recognition} of a single number without {\em all}
other numbers being given around it. Just like \co{distinctions} emerge in the
midst of \co{chaos}, so numbers emerge in the midst of infinity.

Just like \co{transcendence} is an \co{aspect} accompanying every
\co{actuality}, so infinity is an \equi\ \co{aspect} of multiplicity itself. It
is not some late and advanced addition to the simple intuition of finite number.
For instance, it is not only a consequence of, say, positional number notation,
where generation of ever greater numbers is a matter of a mechanical principle.
In the system with four numbers: 1, 2, 3, \thi{many}, the last one does play the
role of infinitely large number, it comprises everything which is \thi{more than
  3}. In the Roman number system, instead of \wo{three} one had \wo{thousand};
names for numbers greater than thousand were compound expressions of which the
highest component was \wo{thousand}. Roman notation for numbers made it hard, or
rather simply impossible, to write arbitrarily large numbers.\ftnt{So did the
  Greek notation, from which the Roman one developed. One can certainly see here
  the possible impediment in developing more advanced number theory or algebra,
  which the Greeks learnt mostly from the Babylonians whose positional notation,
  as well as number theory, was vastly superior. Even the primitive concept of a
  number which, with the Greeks, included only positive integers and rationals
  -- while with the Babylonians it included irrational numbers (if treated only
  by means of linear approximations) and, at least from the Seleucid period,
  also zero -- might be referred to the notational insufficiencies. But these
  are already considerations of the foundations, not of the origin.} But it
would not be plausible to conclude from this that Romans did not have the idea
of an infinity of numbers, although the \co{precision} of this idea might leave
much to be desired.  The problem was observed already by some Romans, as
exemplified in \citeauthor*{Arithmetica}, where the ambitions of arithmetic are
expressedly limited to low numbers, preferably below 9000. But even such a
limitation concerned only the correctness of calculation and not the universe of
numbers.  The question \wo{Is there the biggest number?}  is almost as natural
as \wo{Is there any limit to the distinctions we make?} or, perhaps, \wo{Is the
  world infinite?}.

\wtsep{potential infty}

\pa It may seem that the basic intuition of infinity comes in the form of
potentiality, with perhaps the most obvious experiential counterpart being
\co{more} of \co{complexes}. It is easy to imagine that there is \co{more} than
what one, at any time, can see here and now.  \citet{[Y]et there be those who
  imagine they have positive ideas of infinite duration and space.  It would, I
  think, be enough to destroy any such positive idea of infinite, to ask him
  that has it,-- whether he could add to it or no; which would easily show the
  mistake of such a positive idea.}{LockUnd}{II:17.13. We have to ignore here
  that Locke is speaking about infinity of time and space -- the same argument
  was used against any idea of actual infinity.}  Although there is no
limit, yet one \co{actuality} can always become next one; one can always add 1
to whatever is there already. We never arrive at anything, but as usual in such
cases, we obtain (or rather \co{posit}) a shadowy regulative idea, the
possibility of indefinite progression, which in this case amounts to
unboundedness, i.e., potential infinity.

\wtsep{act. infty}

Various forms of rationalism used to be less reductionistic than empiricism
and showed in general more liking for actual infinity.  The following might be
almost a direct answer to Locke: \citet{if an infinite line be measured out in
  foot lengths, it will consist of an infinite number of such parts; it would
  equally consist of an infinite number of parts, if each part measured only an
  inch: therefore, one infinity would be twelve times as great as the other.  --
  [...]  all these absurdities (if absurdities they be, which I am not now
  discussing), from which it is sought to extract the conclusion that extended
  substance is finite, do not at all follow from the notion of an infinite
  quantity, but merely from the notion that an infinite quantity is measurable,
  and composed of finite parts; therefore, the only fair conclusion to be drawn
  is that infinite quantity is not measurable, and cannot be composed of finite
  parts.  [...]}{SpinozaEthics}{Note to Prop.XV}
%
\thi{Measurement} seems to have to do with composition of finite (and discrete)
parts, in fact with successive progression. The conclusion is then that such a
progression does not lead to any infinity for infinity turns out to be
incompatible with \thi{measure}. It is present \la{a priori} or it is never
reached in any way. 


\wtsep{pot. founded on act. infty}

\pa Now, we do not intend to review the history of the conflict of actual vs.
potential infinity, because most of the involved arguments can be easily
dismissed once we have the precise concepts of infinity and
cardinality.\ftnt{Dedekind's definition (a set is infinite iff it is equinumerous
  with its proper subset) makes Locke's argument above obsolete, or rather simply
  wrong; while Cantor's
  calculus of cardinalities shows that the intuitions about number of elements
  in finite collections often do not generalise to infinite sets (already for
  the least infinite cardinal $\aleph_0$, we have $\aleph_0+\aleph_0=\aleph_0$;
  so, for instance, $12\cdot\aleph_0=\aleph_0$). One can admire Spinoza's foresight
  that such apparent absurdities, perhaps, are not absurdities.
  He nevertheless uses them as such to dismiss the idea of measuring the
  infinite. As often happens, an acceptable argument can serve to support a
  wrong conclusion.}
Indeed, philosophers seem to be less occupied with infinity since
mathematicians got the control over the concept. We know that infinity is
irreducible to progress (actual infinity irreducible to potential one), and
those who do not like it may simply refuse to deal with infinity at all but not
claim any reduction.\ftnt{One can obtain some sub-branches of mathematics, but
 these are only sub-branches. Intuitionism is 
a strong example, but likewise Hilbert's program of finitariness, 
and then also the theory of computability are
  expressions of this potent idea of the early XX-th century's \ger{Zeitgeist} of
  finitude and discretisation.}  Technically, this irreducibility is reflected
by the need for axioms of infinity -- in set theory, the axiom \wo{There exists
  an infinite set}, but also in Euclid, the axiom \wo{Any line can be prolonged
  indefinitely.}
In the case of continuum, any use of progression is known as, at best, a way
of approximating the actual results. 

But we do not intend any review. The crucial point is that although the concepts
and understanding of infinity have reached a very sophisticated stage, they have
been discussed for millennia -- perhaps, in a confused manner, but on the purely
intuitive basis.  No matter the concepts, one has always taken recourse to
something like infinity -- perhaps only an idea, perhaps not, but
something\ldots Even the mere unboundedness of indefinite progression is already
the idea of actual infinity in disguise -- it is infinity reduced by the
epistemological scepticism to \co{actuality}.\noo{FTNT. The distinction between
  infinite and unlimited in mathematics is, of course, a mathematical
  distinction which may but need not be referred back to the pre-mathematical
  intuitions.}  The fact that from a finite set of observations we nevertheless
make the spring to the potentially infinite indicates, in a manner of the
ontological proof, that the infinite is there already. All the emphasis one has
to put on \wo{potentially} (while, so to speak, the unfortunate word
\wo{infinite} sneaks in through the back-door), like all too insistent a need to
deny something, suggests the \co{presence} which only one's bias tends 
to label \wo{unreal}.

Potential infinity is only a conceptual reduction of actual infinity to the
epistemic \hoa. But infinity, the actual infinity itself, is 
\co{founded} in the \co{chaos} \co{above} \co{experience} and in the eventual
\co{transcendence} of \co{nothingness}.
%
This \co{experience}, or rather, this \co{aspect} of
\co{any experience}, the \co{chaos} viewed as \co{chaos} of \co{pure
distinctions}, is what \co{founds} the immediate intuition of the
\thi{largest possible} multiplicity, of the \thi{totality of everything},
\citet{maximum beyond which there can be nothing 
greater.}{DDI}{I:2.5}

\wtsep{apeiron}

\pa The experiential \co{foundation} knows \nexuss\ but not necessarily all the
distinctions which are so dear to later \co{reflection}.  The Greek
\gre{apeiron} can be and is translated {\em either} as \thi{infinite} {\em or}
as \thi{indefinite} {\em or} as \thi{unlimited}, and we mean {\em all} these
when speaking about \co{foundation} of the idea of
infinity.\ftnt{\citefi{[Earth's] part beneath goes down to
    infinity.}{Xenophanes}{ DK 21B28}\kilde{Xenophantes, fr. 28; Strabo I
    [Presocratic Philosophers [Eivind's], no.3, p.10]} \citefi{The unlimited is
    the original material of existing things [... It] is immortal and
    indestructible.}{Anaximander}{ DK 12B1/3}\kilde{[Anaximander] said that the
    principle and element of existing things was the indefinite or
    infinite:Theophrastus' account of Anaximander; in Simplicius {\em Phys.}
    XXIV:13 [Presocratic Philosophers, no. 101, p.107]} Etc., etc. One might be
  tempted to admit also the translation as \thi{unfinished} suggesting, as
  Greeks would certainly like, incompleteness and some unreadiness of the
  infinite.} And, of course, meaning {\em all}, we must mean {\em none}, for all
these distinctions are later than their common origin.
%
\noo{The order of \co{founding} may be roughly as follows: \co{indistinct}
  becomes indefinite which is in turn reflected as unlimited. Unlimited still
  lacks the distinction of infinity, it is that which transcends, goes beyond
  but not into any infinity -- perhaps, into a big void, but archaic outside
  (let's say, \gre{apeiron}) is neither infinite nor unbounded, neither
  potential nor actual, but just that which terminates any questioning,
  \co{indistinct} source.  \thi{Unlimited} (perhaps also \thi{endless}) may and
  often still does mean unlimited without meaning infinite. But it is very close
  to that which surpasses all possible limits. This is the final
  \co{actualisation} which brings in the infinity -- the horizontal projection
  of the unlimited, an \co{actual} expression of the \co{transcendence} which
  has become a mere totality of \co{actualities}.  Potential infinity is the
  event of further \co{reflection} which insists on the primacy, perhaps even
  the only \thi{reality}, of the \hoa\ and \co{immanent} contents.}%
%
Infinity, in particular discrete, countable infinity is the final
\co{actualisation}, an \co{actual} expression of the \co{transcendence} which
has become a mere totality of \co{actualities}. It is the horizontal projection
of the unlimited/ubounded/infinite which are so many ways of \co{representing}
the \co{indistinct}, \thi{everything}.  Potential infinity arises as a still further
\co{reflection} which insists on the primacy, perhaps even the only
\thi{reality}, of the limited \co{actuality}. 

\wtsep{math}

A \co{reflection} of the original \gre{apeiron}, as the \co{chaos} which
underlies every experience, will be present in one way or another in every
original mathematical intuition.  Its mathematical counterpart will vary
depending on the level of sophistication of the mathematical apparatus. It may
be \thi{3}, or \thi{more than 3}, \thi{many} or \thi{infinity}.  The most recent
version seems to be the \thi{totality of all mathematical objects}. Having tamed
infinities, Cantor retained the intuition that the universe of {\em all} such
objects can not possibly be a mathematical object, which was a premonition of
future problems.  The paradoxes of the \thi{set of all sets} can now be rendered
mathematically manageable (e.g., by restricting the axiom of comprehension, or
as in various axiomatisations of classes, or by whatever other means), but the
trick is always to exclude the \thi{totality of everything} from the
consideration -- whatever we cannot speak about, does not cause trouble.  But
whether represented within or expelled from the formal system, the totality keeps
always pointing to the same intuition of the eventual \co{transcendence}, of the
\co{indistinct} limit of \co{distinctions}, and reminding, along the lines of
the first antinomy, that the limit of the world does not belong to the world.

\wtsep{geom}


\pa The ineradicable presence of infinity can be better seen on the example of
geometry.  We started with the intuition of a point which was equated with the
(intuition, respectively \co{representation} of) \co{pure distinction}.  But
points do not appear alone.  Even if point's counterpart is residual
\co{objectivity}, the \thi{mere being} in the \co{immediacy} of the
\co{reflection that it is}, such a \co{reflection} is also immediately aware of
the \co{transcendent} horizon surrounding its \co{actuality} -- positing a
point, it posits an \co{actual} multiplicity of points.  Sure, we can
\co{reflectively dissociate} an \co{act} of imagining a point, from an \co{act}
imagining a multiplicity of points. But this is only \co{reflective
  dissociation}. A point appears always \thi{surrounded} by a background, even
if this be only a black, undifferentiated something -- shall we say, space?  --
against which the point is imagined.\ftnt{Here, space may be understood not only
  as \co{spatiality} -- simultaneity of \co{distinctions}
  (I:\ref{sub:spatialityBeforeTemporality}), but rather as the background from
  which the \co{distinctions} emerge.  It does remind about Kantian space as the
  \la{a priori} form of intuition.}  They emerge only against this
undifferentiated background, and here \wo{undifferentiated} means continuous.
Points represent thus a discretisation of continuity, of a continuity which
transcends the points of \co{distinctions}, that is, of an actually infinite
continuity. But while the primordial infinity of this continuous background is
actual infinity, so the multiplicity of points gives rise to potential infinity:
no matter how many points are (imagined, posited, thought to be) there, there is
always a possibility of \thi{extracting} more points from the undifferentiated
background.  Once we have a point, we have not only a multiplicity, but an
infinity of points.  For an \co{actual} point (whether imagined or drawn) is
only a sign of a \thi{point which already was there}, it merely marks the focus
of our attention.  Imagining a space, say, a plane, and \thi{putting} a point on
(or rather, \thi{extracting} a point from) it, the important thing is not {\em
  where} we \thi{put} it but that it can be \thi{put} {\em anywhere}. This is,
in one, actual infinity and continuity (of course, not in the technical sense),
the pure heterogeneity of \co{chaos} in the immediate neighborhood of
\co{indistinct one}.

Geometry, which with its points and planes gives the primordial intuition of
continuity, is also the first stage where the duality of discrete-continuous
arises.  The duality forms, as Brouwer put it, \thi{two-oneness}.  After the
\co{distinctions} have occurred it becomes perplexing to decide whether
continuum consists of parts or not, whether things are infinitely divisible or
not, whether infinite series can sum up to finite magnitudes and whether
Achilles will ever catch up with the tortoise -- whether \co{one} is a
\co{chaotic totality} of many or else whether \co{chaotic} many is really \co{one}.


\wtsep{}

\pa\label{pa:arithgeom} The differences between geometry (which starts with
infinity of \co{pure distinctions}, discrete points on a continuous background
and, shortly after, with the axiom of actual infinity) and arithmetics (which
starts with multiplicites of \co{pure distinctions}, for which potential
infinity is a theorem, and which only after long labour arrives at the continuum
of real numbers), interesting as they might be, are not essential for us, because
they involve us already into a consideration of foundations, if not of
mathematics itself.  Like the distinctions of actual vs. potential infinity,
infinity vs. unboundedness, infinite time vs. infinite space, etc., it only
witnesses to the multiplicity of possible ways of reflecting the origin,
possible ways of \co{actually} relating \co{pure distinctions} to each other and
to the \co{indistinct} background from which they emerge.

\noo{ %perhaps, expand...
\pa It seems easier to grasp potential rather than actual infinity, for even if
\thi{everything} happens to be actually infinite, its inaccessibility turns it
immediately into a merely epistemic potential.  Although easier to grasp, the
idea of potential infinity is more elaborate and sophisticated than that of the
actual infinity, for the former involves the actual finitude (of
\co{reflection}, of epistemic subject) which is not involved in the latter.

It seems easier to grasp a discrete structure, a few finite numbers
than the continuum, the calculus of infinities

All these \wo{easier to grasp} express the same reductive character -- of
ontology to epistemology, or as we say, of \co{invisible} to \co{visible
actuality}. 

As usual, \co{reflection} (and here we might say also: mathematical
reflection) starts at the lowest level of \co{actuality} -- \wo{the way up is
the way down}, or as Aristotle would say, the ontologically last is the
epistemologically first. 

Mathematical reflection never leaves this sphere

Beginning of \co{reflection} is only the end of \co{foundation}, and to
 forget the \co{origin} is to reduce.
(Reduction amounts to more elaborate ideas and
 explanations -- it increases complexity of the whole, even if it manages to
 reduce some basic notions.) 
} %end \noo{ perhaps expand



\subsection{A note on foundations}\label{sub:foundations} 
What makes mathematics is not its mere origin and the mere \co{pure distinctions},
but a structure and relations built on the top of these basic intuitions.
The ordering or, at first, only the two-term relations \thi{less than},
\thi{equal to}, \thi{more than}, arise from a particular way of relating the
multiplicites of various \co{actualities}.  Structures proper to mathematics are
founded on \co{actual reflections} of such relations.  These
\co{representations} can proceed in different directions and lead to different
foundations, not to mention different branches of mathematics.  It is not our
objective to review the historical schools of foundations but we will give a few
short examples and remarks illustrating how the origin from \co{pure distinctions} 
is reflected when forming various foundations.

\pa \wo{[A] universe comes into being when a space is severed or taken apart.
  The skin of a living organism cuts off an outside from an inside. So does the
  circumference of a circle in a plane. [...] The act [of original severance] is
  itself already remembered, even if unconsciously, as our first attempt to
  distinguish different things in a world where, in the first place, the
  boundaries can be drawn anywhere we please.} This quotation from the
introductory Note on The Mathematical Approach in \citeauthor*{LawsForm}, should
be self-explanatory at the present point. Starting with the space, \esp, in
which a distinction is (to be) made, \setlength{\unitlength}{.1cm}\dist, and postulating two laws, that i)
drawing the same distinction twice makes no more distinction than drawing it
only once, \dist\dist=\dist, and that ii) crossing a boundary of a distinction
and then crossing it back amounts to no distinction, \twice=\esp, and applying
these equations as rewrite rules to various combinations of distinctions, the
basic laws of arithmetics, algebra and propositional logic are derived which,
although do not develop the full mathematics, make the possibility of such a
development at least plausible. The texts on foundation of mathematics and, of
course, \citet{mathematical texts generally begin the story somewhere in the
  middle, leaving the reader to pick up the thread as best he can. Here the
  story is traced from the beginning.}{LawsForm}{p.xxix} One can, of course,
discuss the laws and the details of the development, but the presentation is the
most accurate expression of the idea of actually founding mathematics on
\co{pure distinction} alone. The reader is referred to this book which, if it
can appear a bit esoteric and idiosyncratic, so only because it has been
undeservedly and unjustly ignored.

\pa\label{pa:categories} A much more successful story, at least from the point
of view of scientific development and fashion, can be told about category
theory.\ftnt{A very simple introduction, accessible even to a person with only
  basic knowledge of set-theory, is \citeauthor*{Lawvere}. The origins go back
  to 1940-ties, and \citeauthor*{MacLane} is the standard reference.}  Its
initial motivations, as well as the subsequent focus and power, lie in the
ability to capture structural aspects at a high level of abstraction (often
referred to, by other mathematicians, as \wo{general abstract nonsense}).
Category theory assumes, as it were, given multiplicities and studies their
relations. In fact, it does not even assume multiplicites but just arbitrary
objects whose properties are determined exclusively by their mutual relations
(morphisms between the objects, required to satisfy only a few simple
postulates). It is only morphisms, and not any internal structure of the
objects, which account for all the differences between the objects.\ftnt{As a 
  simple example, consider the standard Cartesian product
  $A_1\times A_2$ of sets $A_1$ and $A_2$ -- it is the set of all ordered pairs
  $\langle a_1,a_2\rangle$ where $a_1\in A_1$ and $a_2\in A_2$. In set theory,
  we are given an explicit construction of a particular object. In category
  theory, one defines a product $A_1\otimes A_2$ of two objects, as {\em any}
  object with two morphisms, \wo{projections}, $\pi_i:A_1\otimes A_2\rightarrow
  A_i$ satisfying the universal property that for every object $B$ with two
  morphisms $\phi_i:B\rightarrow A_i$ there exists a {\em unique} morphism
  $u:B\rightarrow A_1\otimes A_2$ such that $u;\pi_i=\phi_i$.  One verifies
  that, for instance for the category of sets and functions, the Cartesian
  product $A_1\times A_2$, with $\pi_i$ being the $i$-th projection, satisfies
  the categorical definition. But so does also, for instance, the isomorphic
  object $A_2\times A_1$ (now with $\pi_1:A_2\times A_1\rightarrow A_1$ being
  the second projection, and $\pi_2$ the first one.) In the category of
  propositions with the logical consequence relations as morphisms, the same
  definition of categorical product, now applied to two propositions $A\otimes
  B$, turns out to be the conjunction $A\land B$ (or anything logically
  equivalent, e.g., $B\land A$).} In this way,
if we allow the interpretation of morphisms as \thi{observations} (or just
\thi{source of distinctions}), category theory exemplifies the observational
approach (which, as discussed in \refppf{ft:identidiscern} in connection with
identity of indiscernibles, is a variant of distinguishability).  Indeed,
objects obtained by all categorical constructions are determined only up to
\thi{indistinguishability}, that is, up to isomorphism.\ftnt{Various strict
  versions are studied but they represent only special cases. Identity still
  plays the important role but only when applied to morphisms, that is,
  \thi{observations}.}  The theory by far exceeds in mathematical generality and
sophistication intuitionism which degenerated the notion of \thi{observability},
or intuition, to finite constructiblity.\noo{Paradoxically, at least if we were
  to take the name \wo{intuitionism} seriously, it not only yields some~\ldots
  well, \wo{counter-intuitive} results but also finds most applications as far
  from intuition as the theory of computation.}  (Incidentally, the ghost of
\thi{category of all categories} haunts the theory just as the \thi{set of all
  sets} haunts early set theory. As the definition of a category starts with two
{\em collections} -- of objects and morphisms -- the foundational problems seem
to lead back to those familiar from the set theory.)


% \pa\label{pa:categories}
% In ``Conceptual Mathematics'' W.~Lawvere begins in a way more
% similar to \refp{pa:numbers} in that he assumes given
% multiplicites.
% Given two multiplicites, $A$ and $B$, the most elementary thing one
% can do with them is to form some relation.  This relation may be
% arbitrary, e.g., relating any point of $A$ with a designated point of
% $B$, relating points \thi{at random}, etc..  But there are easily
% recognised ways of relating which soon turn out to give most structure
% to the emerging relations.  If it is possible to find an injective
% function from $A$ to $B$, it tells that $A$ has at most as many points
% as $B$ -- it has a smaller number or, perhaps in a more abstract
% sense, is a subobject of $B$.  If it is possible to find a surjective
% function from $A$ to $B$, then it amounts to partitioning $A$ into
% disjunct subsets, each determined by the unique point of $B$ to which
% all its elements are mapped.  In other words, it amounts to
% classifying points of $A$ according to the \co{disitinctions} (points)
% of $B$.

% Thus, categorical foundation starts rather with \co{actual}
% multiplicites and relations (morphisms) between them.\ftnt{Even if
% the formal example throughout the whole book is the category of sets,
% the categorical foundation represents a significant departure from the
% tradition of set theory in that one does not any longer attempt to
% construct the universe of sets but takes it (as any other category)
% for given and instead develops all concepts in terms of relations
% between objects.}
% %
% This is still good and fine, because the question remains:
% multiplicites {\em of what}?  In category theory they will often be
% multiplicites of arbitrary, possibly highly structured, objects.  But
% the \co{pure distinctions} remain.  For the first, the objects will
% almost always be mathematical objects, that is, objects described
% (even if not exhaustively characterised) by means of familiar
% mathematical constructions.  It is only their characterisation which
% is transfered from the internal structure to the properties
% expressible exclusively in terms of the relations.  These relations
% (of which surjective, injective, etc.  functions are but trivial
% examples, which have their categorical counterparts) still have to be
% reducible to univocal mathematical definitions -- identity plays the
% paramount role precisely with respect to relations rather than
% objects.\ftnt{But we should observe that the categorical \thi{way
% of thinking} is not necessarily limited to mathematics.  The examples
% used by Lawvere are excellently chosen from the daily experience.  A
% sociologist may study phenomena exclusively in terms of their external
% relations, just as behaviourist does.  Of course, the kinds of
% relations, and the kinds of \co{distinctions} between phenomena will
% be then of entirely different order than in the mathematics of
% category theory.

% In this \thi{observational} approach (cf.  \refpp{ft:identidiscern})
% to mathematics, \thi{observations} mean simply relations between
% objects and are fully internalised.  Identity plays the important role
% when applied to the morphisms, but the constructions which yield
% objects are defined only up to \thi{indistinguishability}, that is, up
% to isomorphism.  The theory by far exceeds in mathematical generality
% and sophistication intuitionism which degenerated the notion of
% \thi{observability}, or intuition, to finite constructiblity. 
% (Paradoxically, at least if we were to take the name \wo{intuitionism}
% seriously, it not only yields some ...  well, \wo{counter-intuitive}
% results but also finds most applications as far from intuition as the
% theory of computation.)  There is a vast variety of more specific
% examples of the \thi{observational} approaches, including
% observational logics, pointfree topology, theory of locales, various
% properties of extensionality, i.e., of identity of indiscernibles.}
% %
% (Incidentally, the ghost of \thi{category of all categories} haunts 
% the theory just as the \thi{set of all sets} haunts early set theory. 
% The foundational problems seem to reduce to those familiar from set 
% theory.\ftnt{The standard reference is Saunders Mac Lane, 
% {\em Categories for the Working Mathematician}, and one better be 
% mathematician if one intends to read it.})

\pa The best known and most thoroughly studied foundation of mathematics is set
theory. Its fundamental primitive concept of a set connects it to the origin.
For Frege a set seemed to be an extension of a concept, but this is a heavily
logicist position influenced, as it seems, by the search for empirical
foundations. To begin with, it was much simpler: \citet{a set is a many which
  can be thought as a one.}{CantorGes}{p.204. Or in an earlier formulation:
  \citef{In refusing to allow the manifold to remain manifold, the mind makes
    the truth clearer; it draws a separate many into one, either supplying unity
    not present or keen to perceive the unity brought about by the ordering of
    the parts.}{Plotinus}{VI:6.13} Introducing, as Cantor did, actually infinite
  sets amounts to following this intuition of the possibility of \thi{being
    thought as one} all the way through, and has the obvious relation to the
  (neo-)Platonic creed of unity preceding multiplicity, invoked in
  \citeauthor*{CantorAggr}, e.g., \citeauthor*{Plotinus}{ V:6.3;VI:6.11},
  \citeauthor*{Proclus}{ \para 69} (cf. I:\ref{sub:OneMany}).  \noo{[The same
    intuition of a unity of a multiplicity, though degenerated and limited to
    finite sets only, appears some 500 years earlier: \citef{Now multiplicity,
      if it is finite, is not entirely outside the confines of unity, since
      insofar as it is finite it is one.}{PicoUnity}{VI}]}}
Notice how the word \wo{many} suggests the irrelevance of any actual contents
for the considered multiplicity, which is to be given in the simultaneous
\co{immediacy}, is to \wo{be thought as a one}. Cantor attempted 
alternative formulations -- \citet{by a set we are to understand any collection
  into a whole of definite and separate objects of our intuition or of our
  thought}{CantorCont}{\kilde{p.84} [also in \citeauthor*{CantorGes}, p.282]}
-- and one can certainly recognise here the importance of having \thi{sharp and
  separate} objects as members.\noo{ (which might be quoted as a justification
  for, say von~Neumann-Bernays-G\"{o}del axiomatisation of classes, forbidding
  them to be members of other classes due to their \thi{unsharp} character
  caused by unmanageable size)} But the basic intuition remains unchanged,
namely, the intuition of a multiplicity, of a collection of \co{precisely}
distinguished somethings, in general \co{objects} or mathematical objects but,
eventually, only of \co{pure distinctions} which are \co{posited} simultaneously
as an \co{actual} unity.\ftnt{In Fraenkel's set theory with \ger{Urelemente},
  different \ger{Urelemente} were distinct but indistinguishable (any
  permutation of \ger{Urelemente} could be extended to an automorphism of the
  whole universe). The formulation \wo{distinct but indistinguishable} might
  cause some worry, but to us seems a perfectly reasonable expression of the
  idea of \co{pure distinction}, \co{distinction} without any content or prior reason.}

Unlike category theory which studies properties of objects only to the extent
they are reflected in the {\em relations between} the objects, much of the
foundational effort in set theory went on actually constructing the universe of
sets. Even in such a construction, which does not presuppose any given
multiplicites, we can find \co{pure distinction} as the fundamental building
block. The construction starts with nothing, emptiness, that is, with the empty
set.  But is not there a difference between nothingness and a set which contains
nothing? The former is, perhaps: \esp, while the latter: $\{\esp\}$ (written
usually $\emptyset$). This, one could say, is only the matter of notation, of
the need to indicate emptiness. But it is much more.  There is a difference
between nothingness and nothingness captured, between emptiness and emptiness
confined, emptiness of a particular container, between nothingness and a set
containing nothing. The pair of parentheses $\{\ \}$ applied at this very
beginning reflects the \co{pure distinction}, the fact of difference which has
been extracted from nothingness; we could almost say, an \co{act} of
\co{actually} addressing nothingness as distinct from the unaddressed
nothingness itself. This \co{actuality} brings at once also the intuition of
multiplicity, of a simultaneous givenness of \co{pure
  distinctions}.  For once the pair -- the \co{act} -- $\{\ \}$ is there, it can
be applied to 
everything (even to nothing) and thus the rest follows.  The only set we can
obtain at the next stage from $\emptyset$ is $\{\emptyset\}$ -- the set
containing one element, the empty set $\emptyset$.  Of course, $\emptyset
\not=\{\emptyset\}$ -- the set $\emptyset$ has no elements, while
$\{\emptyset\}$ has one.  We can then continue adding the parentheses, obtaining
new, mutually distinct sets $\emptyset \not= \{\emptyset\} \not=
\{\{\emptyset\}\}\ldots$ This looks boringly similar
to unary numbers and, moreover, produces different sets only in so far that they
all contain different elements -- but they all (except $\emptyset$) contain
exactly one element.  This does not open up for internalising mathematics, in
particular arithmetics, {\em within} the set theory, so one has to show more
ingenuity (as is done in von Neumann's construction of ordinals).\noo{Having
  constructed the first two sets, one can make a new set with two elements --
  $\{\emptyset,\{\emptyset\}\}$.  The only thing needed (and here explicitly
  constructed using $\{\ \}$) is that the elements of such a two-element set are
  mutually distinct.  Having that, that is, three distinct sets/elements, we can
  construct a set with three elements by putting $\{\ \}$ around, $\{\emptyset,
  \{\emptyset\}, \{\emptyset,\{\emptyset\}\}\}$, etc.  In short, having
  constructed a set $\alpha_{i}$, we can in a generic way construct \thi{the
    next} set $\alpha_{i+1}$, distinct from all previous and with one more
  element than $\alpha_{i}$, by letting
  $\alpha_{i+1}=\alpha_{i}\cup\{\alpha_{i}\}$.  We do not have to review here
  the details and consequences of this elegant construction of ordinals proposed
  by von Neumann nor various axiomatisations of set theory.  The essential
  aspect is already there -- the rest, possible axioms and postulated
  constructions, in particular, also the above construction of ordinals, is a
  matter of foundation.} But the main point has already been made -- enough
\co{distinctions} are available and they are obtained from the original $\{\ \}$
which represents both \co{pure distinctions} {\em and} their
simultaneity. The rest -- possible axiomatisations, postulated constructions,
resolution of appearing paradoxes -- is a matter of more detailed foundation.
\noo{ The set formation using $\{\ \}$ reflects, as just observed, also the
  other aspect besides that of \co{pure distinction}, namely, that of pure
  multiplicity, simultaneity. A set of three elements amounts to mutual
  \co{distinctions} between the elements which are given simultaneously, in the
  unity of one \co{act}. A set represents an \co{actuality}, and it is only
  understandable that it has been for so long quite difficult to admit infinite
  sets as legitimate objects of mathematical discourse.  }

%\newpa 
%\tsep{identity:consider shortening/moving to \ref{sub:Identity}}

\pa One final remark before leaving the subject of foundation. Sameness is
complement of distinctness so, instead of saying that mathematics is the science
of \co{pure distinctions} we might, perhaps, say that it is the science of
equality (or even identity).  It might be an exaggeration to claim that identity
is the only form of mathematical theorems, but it is certainly the basic form of
mathematical statements.

Equality arises as a special case of relation between multiplicities, namely,
when we find a function which is bijective. 
Equipotence of $A$ and $B$ is the first moment when equality enters the stage of
explicit representation. But implicitly it has been there earlier. The very fact
of relating some point $a$ of $A$ with a point $b$ of $B$ means, in a sense,
identification. If $r$ relates $a$ with $b$, especially, if the relation $r$ is
functional, it amounts to saying that the image of $a$ under $r$ is, i.e., is
equal to $b$, $r(a)=b$. It depends, of course, on what $r$ is. Our shepherd did
not identify sheep with marks on the stick. But establishing a(n injective)
function amounts to identifying the points of the source with their images.
Equality emerges as a relation -- that is, it presupposes and is based on
\co{distinction}. Nobody would ever imagine saying that $a=a$ if the threat
of $a\not=a$ was not there.\noo{(It was the fear, the intuition, the suspicion of
the non-identity of the underlying \co{non-actuality} which drove one to
desperately emphasize the \co{actual}, empty law of $x=x$.)} This possibility
is, from the point of view of pure multiplicites and their
relations, the primordial reality: \co{immediate} things, viewed {\em only} from
the point of their \co{immediacy}, are different before they become the same.

\noo{MOVED:
  Identity arises as a relation across multiplicites, where a point $a$ in $A$
turns out to be the same as $b$ in $B$.  As Frege observed, the difference
between $a=a$ and $a=b$ concerns the form of presentation. Just like the
statement $a=a$ does not say anything, the statement $a=b$ says quite a lot --
their \ger{Erkenntniswert} is very different. The former \citet{is valid a priori
  and, following Kant, is called analytic}{FregeSinn}{}. The latter, on the
other hand, says that $A$ and $B$ are two different perspectives, two different
snapshots of something.  This something, this ever transcendent $x$, arises
precisely as the identity of $a$ (i.e., $x$ viewed as, or in the context of, $A$)
and of $b$ (i.e., $x$ viewed as, or in the context of, $B$).\ftnt{Frege says that
  $a$ and $b$ are simply different signs and that identity is an epistemic
  relation between signs, which obtains when both have the same denotation,
  \ger{Bedeutung}. This is probably what it becomes, eventually, in the
  \co{actual world} with its ready-made \co{objects}. Our point here is only
  that this relation is \co{founded} and emerges as a residual \co{trace} in the
  process which first 
  \co{dissociates} various \co{actualities}, and then must re-establish the
  connections between them. The present remarks give only the observations from
  Section~\ref{sub:Identity}, in particular \refpf{actId}, a more specific form.}
}

Of course, this \thi{becoming the same} has only epistemic aspect because
proving that $a=b$ one only discovers the fact, an $x$, hiding behind
the actual representations $a$ and $b$, and which has always made $a=x=b$.  As we
observed in \ref{sub:Identity}, especially \refpf{actId}, equality across
\co{dissociated actualities} is a transcendent fact, a \co{trace} of earlier
unity.  This fact cannot be accounted for within the mere \co{actuality} and
identity remains, on the one hand, a hardly questionable (ontological) intuition
and, on the other hand, an (epistemic) ideality which \thi{has to be
  constructed}.  Likewise in mathematics,\noo{ which is the science of pure
  \co{immediacy},} this relation remains forever as fundamental as undefinable.
On the one hand, equality is not axiomatisable -- any set of axioms valid for
the identity relation will also be valid for other relations (congruences, i.e.,
indistinguishabilities).\ftnt{This applies to first-order logic. In a sense, but
  only in {\em a} sense, second-order logic allows one to define identity, i.e.,
  force a relational symbol to be interpreted as such. The reservation concerns
  the need for additional semantic assumptions, in particular, that one works
  only with the standard model (all subsets of the domain) and not the general
  models (admitting various choices of the collection of subsets). Then, the
  definition of identity of individuals amounts to requiring them to be members
  of exactly the same sets, in particular, the same sets with only one element.
  Even if technically possible, it seems to leave too many holes (e.g., sameness
  of all, also one-element sets is presupposed) for a philosopher to agree that
  identity has thus been defined.}As Frege says \wo{Since every definition is an
  identity, identity itself can not be defined.}  But, axiomatisable or not, one
works with equality and knows its meaning. Equality is a semantic notion: it has
to be introduced into the mathematical foundation as a primitive, as if
\thi{from outside}.\noo{, and can, at best, be reflected by the machinery of
  multiplicites and their relations. This semantic notion allows one to extend
  the scope of other notions, to transcend their validity and relevance beyond
  the initial \co{immediacy}. Eventually, it relates the units of knowledge and
  \co{distinctions} of \co{representation}, reflecting the possibilities of
  viewing {the same} as different without forgetting that one is viewing the
  same\ldots} This is yet another reflection of the \co{purity} of the addressed
\co{distinctions} which are given, always and only, in the sphere of timeless
\co{immediacy}. In this sphere, everything is \co{dissociated} to the extreme,
everything is but an \co{immediate} point \co{purely} and absolutely
\co{distinct} from all the others. Equality of two such \co{distinctions} enters
the sphere as a transcendent event, connecting the \co{immediacy} of $a$ with
the \co{immediacy} of $b$ which connection, ideally, should be equally
\co{immediate}. But, strangely, $b$, appearing as distinct from $a$, must reside
in some other place, and their equality is what connects these two places. Once
the equality is established, the two become one, \co{immediate} $x$ ($=a=b$
which becomes likewise the mere, \co{immediate} self-identity $x=x$). 
But the very event of this \thi{becoming one} happens elsewhere, in the sphere
transcending their \co{immediacy}. Whether one calls this sphere \wo{the mind
of the working mathematician}, \wo{the mathematical activity} or else \wo{the
eternal world of ideas}, we leave to everybody's discretion. Our point is only that it
transcends the sphere \co{immediacy} where the equated objects of mathematics
have their \la{locus}. 

\noo{ Intuition of this fact might have been one of the motivations leading
  Leibniz to postulate \thi{identity of indiscernibles}.
  \ftnt{\label{ft:identidiscern}This principle plays crucial role whenever
    one tries to get rid of all transcendence and pretend that one is in
    possession of a complete logical language. Observational logics, theory of
    locales, various properties of extensionality based on this principle may
    have sound grounding in the formal setting where they are used.  But the same
    phenomenon can be seen, in a more general context, in the attempts to reduce
    all reality to language, in statements like ``Whatever can be said, has been
    said'', ``Whatever we cannot speak about, we should keep silent about'',
    ``Reality is the names we give to it''.
    
    In an apparently different way, the principle appears, for instance, with
    Cusanus, who says \citf{two things cannot be perfectly alike and,
      consequently, participate one essence precisely and equally}{DDI, I 7/49}
    Since, according to him, there is only one, infinite essence, the only
    differences between things result from different degrees of participation in
    this essence.  Then \citf{[t]he universe, as most perfect, has preceded all
      things in the order of nature, as it were, so that it could be each thing
      in each thing.}{DDI, II 5/117} The reader can easily imagine the close
    analogies to the unity of the Leibniz's system, his pre-established harmony,
    reflection of the whole universe in every monad, and the (resulting?)
    principle of identity of indiscernibles.  A form of the principle arises
    naturally not only from the attempts to reduce ontology to epistemology, but
    also when considering the original unity as the most fundamental truth --
    the unity which, as argued in \refsp{sub:Identity}, is not reducible to
    \co{actual} observations.}

  In a sense, but only in an analogical sense, the philosophy of
  \co{distinctions} and \co{distinguishing} goes along with this principle:
  whatever is not \co{distinguished} remains the same, remains the \co{One}.
  But in our hierarchy of levels, what is the same at one level, may look quite
  different at another. Identity is only, and best, a \co{reflection} of unity.
  Moreover, unlike the relations between \co{actualities} where identity arises
  only as a secondary concept, in the ontological order of hypostases, as well
  as in the epistemological order of founding, undifferentiated unity precedes
  the \co{distinctions}.  }
\noo{ \pa It allows one to subsume two different \thi{perspectives}, two ways of
  expression, as expressions of one and the same.  Identity as if pulls
  differences into the \hoa\ where they can appear as coinciding, that is, as a
  one \co{object}, even if the \co{object} itself appears only as an $x$, and
  the only things we can say about it are different \co{actual} determinations.
  Mathematical identity is a coincidence of different \co{actualities} within
  \co{immediacy}.  It establishes something, an $x$, as that which accounts for
  $a=b$, as an \co{immediacy} across the long chains of \co{actual} reasoning
  steps.
  
  Even if mathematical reasoning were a chain of tautologies\ftnt{Which it
    isn't because there is no such thing as a tautology.  It is not only the
    meaning of terms which might make a statement analytic.  It is also, even
    primarily, the assumption that there is such a thing as an analytic
    statement, which influences the way one understands the involved terms.  The
    very postulate of $x=x$ being a tautology, involves already an assumption
    about the entities one considers.  Certainly, $1=1$ and $2=2$ but to say
    $x=x$, or more precisely, $\forall x:x=x$, is to say something not about all
    possible $x$ but rather about $\forall x$, about what $x$ may range over.
    As Quine points out, quantification commits one to an ontology, namely the
    entities the quantified variables may range over.  The postulate $\forall
    x:x=x$, that everything is identical to itself, says something about what
    one understands by \wo{everything}.  It marks a narrowing of attention to
    \co{objects} (or form of understanding) for which this does hold, to the
    domain where it is a tautology.  Counting is good, I can count doors in my
    house and women I was in love with.  But can I count the feelings which
    confront me when I meet a new love?  Saying \wo{Life is identical to life}
    or \wo{Love is identical to love} is, if not empty, then simply untrue.
    Identity as the \co{precise} category of \co{immediacy}, does not apply at
    these levels of being.  There are no two lives, no two loves which would be
    identical in this rigid sense, and trying to posit them as such we have
    already assumed the \co{objectivistic attitude}, we have already reduced
    them to the level of \co{actuality}, to the totality of \co{actual}
    observations -- in short, we have asked a wrong question.  },
%
  it would in no way exclude the possibility of resulting in an insight, because
  such an insight would be exactly an \co{actualisation}, would reveal the
  \co{actual} coincidence of the extreme terms of the reasoning chain, which to
  begin with were (thought) different.  An \thi{aha!}  experience is such an
  insight which occurs at the end of a chain of apparently distinct
  \co{actualities} revealing that they are \co{actually} the same, that they can
  be legitimately seen as one and the same thing. This identity is like a
  self-sameness of a \thi{substance}, of an \co{object}, and it is what makes
  the thing, the $x$, accessible within the pure \hoa\ -- $x=x$ is the ultimate
  expression of focusing exclusively on such a form of accessibility.  Identity
  is thus, by its very nature, transcending the \hoa. The transcendence of $x$
  is just another name for the fact of $x$'s identity. One tends to think of it
  as the inexhaustibility in a potentially unlimited series of \co{actual}
  presentations, but it is enough to encounter the same $x$ only twice, once as
  $a$ and then as $b$, to encounter its full transcendence.  Every identity is
  transcendence. Both are but two aspects of this same fact; eventually, it is
  the fact of everything being but a manifestation of the same and all
  transcending \co{One}.
  
  \pa Thus mathematics, with mathematical identity in particular, expresses the
  tendency to consider things from the point of \co{immediacy}.  Mathematical
  identity reflects also the fact of its object being transcendent. Unlike the
  sentence \wo{The Morning Star is the Morning Star}, which says nothing, the
  sentence \wo{The Morning Star is the Evening Star} may express a true
  discovery: that two things considered distinct where only two different
  perspectives on the same.  As said in I:\refp{pa:perspectives}, our
  being is the \co{experience} of the \co{One} through the Many, in ever new
  \co{actualities}, from ever new sides.  Identity is essentially unobservable,
  since within what we take to be the horizon of observation, within the \hoa,
  it may only be empty $x=x$.  But it is \co{experienced}, it is
  \co{experienced} through and through, because \co{experience} is not limited
  to \co{actual experiences} within this horizon.  Mathematical identity is a
  reflection of this fact in terms of \co{pure distinctions} and categories of
  \co{immediacy}.  }


\subsection{Summarising\ldots}
\co{Pure distinction} is the most \co{immediate}, because entirely contentless,
event amenable to a grasp by a single \co{act} as the univocal
distinction of \thi{yes}-\thi{no}, \thi{being}-\thi{not-being}. But it is also
the primary event of the ontological \co{founding} which,
therefore, accompanies all other events. This primacy reflects the involvement
into the ultimate \co{transcendence}, the \la{a priori} of this event which, never
occurring alone, makes possible and accompanies all
\co{distinctions}. Mathematical objects have thus this double aspect: of the
ultimate \co{precision} and contentless univocity and, on the other hand, of the
\co{representations} of the primary, though only formal, event of mere 
\co{distinguishing}. 
% a priori - but no specific form
\ad{A priori} The given account can remind of Kantian \la{a priori} forms
providing conditions of possibility of experience.  \co{Distinction} is an event
of \co{any experience} and, with it, \co{pure distinction} its \la{a priori}
condition.  This, however, is only an analogy of form, in that \co{pure
  distinctions} play similar role to \la{a priori} forms which are not
thematical contents of \co{experience} but necessary aspects underlying \co{any
  experience}. But, unlike Kantian forms, they are themselves \co{experienced}
in the immediacy of \co{self-awareness}, in every experience.  Beyond that, they
do not provide any more specific form of \co{experience}, in particular, the
temporal and spatial dimension are only related, but much lower \co{aspects} of
\co{actual experience}.  The main difference, if we were to speak about \la{a
  priori} conditions, would concern the fact that they are not independent from
\co{experience} but, on the contrary, are \co{present} in every particular
experience and are themselves experienced.\noo{This difference boils down to the
  difference between our generous concept of \co{experience} including also
  what, without being an \co{object} of \co{any experience}, can still be
  experienced, and the experience understood as merely (the totality of)
  \co{actual} presentations of \co{objects}.}

Even if mathematical concepts have developed, evolved and proliferated, there is
something which makes Phytagorean and modern mathematics {\em equally}
mathematics.  This primal ground, reflecting its origin, has proved immutable
unlike in any other science.  Learning physics we never hear about the Ionic
philosophy (misconstrued, as is typically done, as the philosophy of mere
nature), or of Aristotelian principles.  But learning mathematics we still go
through the theorems of Thales, Phytagoras, Euclid which were also much earlier
known to the Babylonians or Egyptians. And also when we go back to Egyptian
engineering, Chaldean astrology or Babylonian accounting, we find sound {\em
  mathematical}, not pre-mathematical calculations.  As the contributions to the
mathematical knowledge they are as valid, relevant and {\em mathematical} as the
theorems of Gauss, Banach or Skolem.  Thus, unlike other sciences which have
either gone through the processes of essential changes before reaching their
modern form or else appeared only very recently, the character of fundamental
mathematical objects has remained unchanged since the very beginning. Even the
most primitive and underdeveloped mathematics is {\em equally} mathematics, is
as much mathematics as are its most advanced forms.  It may be less developed,
have different foundations, pursue only a limited range of questions, but it
cannot dissociate itself from its origin without ceasing to be mathematics.
Mathematics of other intelligent beings might be very different from ours. But
to the extent it is mathematics, it would rest on the same, \la{a priori} origin
and, as such, could not contain theorems contradicting the theorems of our
mathematics.

\newpa
% no abstraction
\ad{Abstraction} All other sciences emerge as a consequence of extracting from
the whole human experience some restricted domain -- of specific objects or
problems.  The notions of such a domain may then undergo a gradual abstraction
which eventually yields quite abstract entities with which most advanced
sciences are occupied.  The abstract character of a science is always the end
result, never the beginning.  But this schema obviously does not work for
mathematics.  Its original objects have not changed since its beginning.  And if
we try to elucidate the basic notions of point, number and the like by a
reference to the process of abstraction we would have to explain what made our
remote ancestors so intensely interested in just this line of extreme
abstraction and made them ignore more or less all others. Why did Babylonians,
Egyptians, Greeks carry out this line of abstraction to its very extreme while
in all other areas stopped at a very elementary level? Perhaps, simply because
nobody had to abstract himself toward the notion of a multiplicity by
disregarding more and more properties of actual objects.  If the {\em
  experience} of \co{pure distinction} lies both in the background of our being
{\em and} at the origin of mathematics, then there is no need to make our
ancestors so mystically different from us, because there is no need for
abstraction at all.

\citet{[Number] may well be the most primitive element of order in the human
  mind [\ldots] Hence it is not such an audacious conclusion after all if we
  define number psychologically as an {\em archetype of order} which has become
  conscious. [\ldots] It is generally believed that numbers were {\em invented}
  or thought out by man, [\ldots but] it is equally possible that numbers were
  {\em found} or discovered. In that case they are not only concepts but
  something more -- autonomous entities which somehow contain more than just
  quantities. [\ldots] then on account of their mythological nature they belong
  to the realm of `godlike' human and animal figures and are just as archetypal
  as they [\ldots]}{SynchronicityJung}{The Structure and Dynamics of the Psyche,
  \para 870ff.\hee{???}} In short, \wo{we have a direct awareness of
  mathematical form as an archetypal structure.}$^{\ref{cit:numberArchetyp}}$

Abstraction lies only in positing the original intuition of \co{pure
  distinction} as an \co{object} of independent study, in turning this intuition
into an explicit \co{representation}, if you like, in turning from the origin
towards a foundation.  Thematic study of mathematics may be difficult and
abstract. But it does not mean that its fundamental, original \co{object} is an
abstraction which has nothing to do with \co{experience}.

%syntheitc 
\ad{Synthetic and universal} 
Mathematics is not only \la{a priori} but
also \thi{synthetic} -- it applies to \co{experience}, in fact, to
\co{any experience}, simply because it addresses elements present in \co{any
  experience}. 
\co{Distinction}, \co{chaos} and \co{actuality} 
are constant \co{aspects}  of all our experience, knowledge and
activity.  All \co{experience} is \co{self-aware} and so with any
\co{distinction} there is associated the awareness of the fact of
distinctness, that is, \co{pure distinction}. Similarly, with the 
\co{actuality} of \co{an experience} there is given multiplicity, or 
multiplicites of \co{pure distinctions} and with \co{chaos} -- their 
infinity. These intuitions, even if not \co{represented} explicitly in 
mathematical or other concepts, accompany all our \co{experience}.

But this universal applicability amounts also to a reduction. Mathematics is
applicable to \co{an experience} only to the extent we view it through the
glasses of \co{pure distinctions}.  Mathematics applied to engineering, to
sociology, even to psychology is always the same mathematics and it says equally
much (or little) about each area -- it says only that much as can be expressed
in terms of \co{pure distinctions}.  Counting houses is no different from
counting sheep, nor from counting sheep and apples and friends, because counting
is always only counting of multiplicites, of points, of \co{pure distinctions}.
We can apply mathematics to any experience only to the extent we are willing to
disregard all possible differences of content and consider only differences of
number. The \thi{synthetic} character of mathematical enterprise is really the
same as its \la{a priori} character -- the fact that experience is an experience
only to the extent it is differentiated.

Hence mathematics is \thi{synthetic} and truly universal: not because it can say
something about the content of \co{any experience} but because it does not say
anything about such a content -- only that each content must be distinguished.
As usual, the price for generality is the loss of \co{concreteness}.\noo{This
  fact, that mathematics addresses \co{pure distinctions} and thus resides, so
  to speak, in the contentless limit of \co{immediacy}, makes its objects
  closely comparable to those of idealized \co{immediacy} from
  \ref{sub:idealImmed}, and highly dissimilar to those of the general
  experiential \co{immediacy} from the beginning of \ref{sec:levelA}. In section
  \ref{sub:truth}, \refp{pa:immedSubject} and \refp{pa:immedArbitrary}, the
  \co{dissociated immediacy} as the source of apparent spontaneity was mentioned
  as the basis of both {subjectivity} and arbitrariness. But such
  characterisations fit very badly mathematical contents. There is, however, no
  problem. Firstly, because these characterisations apply only to the
  experiential \co{immediacy} and not to its idealized limit. In this idealized
  limit, the \co{immediacy} of a point acquires entirely \co{external}, and in
  this sense \co{objective} character, retaining however the fact of
  \co{immediate} givenness which is responsible for the \thi{mental},
  not-experiential aspect of mathematical objects. The arbitrariness of
  experiential \co{immediacy} concerns always the appearing content; \co{pure
    distinction} is exactly the lack thereof and the experienced arbitrariness
  of minuteness turns in the idealized limit into its opposite, necessity of the
  timeless, which we will address in a moment.}

\noo{
%timeless
  \ad{Timeless} \co{Pure distinction}, although present in every
  \co{experience}, is itself liable to be grasped in the most \co{immediate}
  \co{act}.  \co{Vagueness} of most \co{distinctions} is then replaced by the
  sharpest, most \co{precise} rigidity of the fact of distinctness.  The whole
  mathematics, arising around the \co{pure distinction}, is constructed in terms
  of \co{immediacy}.  Its world is timeless in the same way as the \co{object}
  of \co{reflection that it is} -- reduced to \co{immediacy}, dissociated from
  the temporal dimension of existence.

%\subpa
  \noo{Relations between multiplicites do not have anything to do with time.
    They are just relations, that is, mutual correspondences of co-\co{actual}
    collections of points.  If one were to think of these in terms of processes,
    then such processes are essentially reversible.  (E.g., relation $r\subseteq
    A\times B$ is the same as its inverse relation $r^{-}\subseteq B\times A$;
    the existence of an injective function $f:A\rightarrow B$ is the same as the
    existence of a surjective function $f^{-}:B\rightarrow A$, etc.)  }
%\subpa
  Every mathematical relation and, in particular, mathematical identity,
  although of a transcendent, supra-mathematical origin, acquires in mathematics
  equally timeless character.  It is identity of two \co{immediate} givens, of
  two points (or objects constructed from such points) residing in their
  respective, \co{dissociated immediacies}.  Transcendent as it is, it is viewed
  always and only from the point of these \co{immediacies}. The $x$ making
  $a=x=b$ is itself a mathematical object which, once identified, becomes prone
  to the same manipulation as $a,b$ which gave rise to it. The relation of
  equality is \wo{frozen} and timeless as are the terms of this relation.  It is
  the opposite of the lived, continuous \co{unity} of a being which \co{founds}
  all the differences of its possible \co{actualisations}.  }


\ad{Necessary vs. universal}\label{pa:necessity}
It might seem that universality accounts also for necessity, that, as
Kant meant, ``the two are inseparable''.  But they are not only
separable but very different.  
\noo{Let me first remark that I am not concerned with the ontological status of
  necessary judgments.  I do not want to argue for nor against determinism, I
  only want to observe some differences in the meanings of the two notions.  }

Universality will say \wo{something is always valid}, necessity
\wo{something can not be otherwise}.  The former is quite a natural
concept.  If it is empirical, then it is exactly what makes it
natural.  To some extent everybody makes generalisations and arrives
at some universal formulations.  Now, one may say ``all ashtrays in
this room are green'' but we should not confuse the syntactic form
(the mere presence of the universal quantifier) with universality. 
Universality involves generality and is concerned with the totality of
\co{the world}.  That we always distinguish, that so it is, is a
universal statement.  But such ``so it is'' is not sufficient for
necessity because necessity is concerned not only with the actual
world but with all possible worlds.  It cannot merely say what is always the
case in \co{the world}, it also has to exclude its opposite from all possible worlds. 
Only by confusing the universal quantification over the objects within
the world with the universal quantification over the possible worlds, 
can one confuse universality with necessity.

Since universality is concerned with the actual world and necessity
with all possibilities, the former does not imply the latter. 

Necessity is thought {\em de re} -- it is a property of objects, relations and
states of affairs.  Saying ``this statement is necessary'' we mean ``what it
claims holds with necessity''.  It is the behaviour of objects or some state of
affairs which is characterised as necessary. As the paradigmatic example one has
always posited the causal relation which holds necessarily between $x$ and $y$
if an occurrence of $x$ is a {\em sufficient} reason for the occurrence of $y$.
After Hume's criticism it seemed impossible to maintain this idea of necessity
which was first relegated to the categories of pure reason and, eventually, to
the sphere of purely linguistic phenomena.\ftnt{This whole development,
  reflecting the atomistic ontology and leading to nominalism, is present
  already in Ockham. Following the assumption of exclusive reality of
  dissociated particulars, he argues for purely mental character of causality
  (as of any other universal relation), \citeauthor*{OckQuod}{II:9,IV:1,VI:12},
  and arrives at the impossibility of demonstrating any causal relations,
  \citeauthor*{OckSent}{II:4-5.i} \citaft{GilsonHCPh}{ p.497, footnote~27}. }
In this tradition, it is the analyticity of judgments which is supposed to
account for all possible necessity -- of judgments, of course. If such judgments
existed they would be necessary by being void of all real content, by being true
for purely linguistic reasons of mere meaning of the involved terms.  We could
agree that necessity implies removal of the actual content but not that it is a
purely linguistic phenomenon.  It is related to our understanding but not as if
this required language and opposed \co{experience}.  It is related to all the
\co{actual objects} but only to the extent these are reduced to the ideal
\co{immediacy}, namely, to the contentless \co{pure distinctions}.

\pa Universality involves not only \wo{for all $x$} but also a kind of
generality, totality of \co{the world}. Necessity, concerned with all possible
worlds, would thus imply universality. But this is only a superficial, formal
implication. Necessity does not require any generality.  ``In the experiment
which started at the Ridiculous Labs, CA, USA, on the 26th February, at
14:03':52'':18''', the generated positron had to turn left, the electron had to
make a U-turn and, colliding, they had to annihilate.''  Without making any
claims to the physical plausibility of this statement -- it says that something
was necessary.  It says that no matter what, given the above conditions things
could not have happened otherwise.  But one could hardly call it a universal
statement.  Replacing \wo{the generated positron} with \wo{any positron which
  might have been generated at this point} would only change the syntax giving
at most a resemblance of generality.

In fact, it is only by designating more and more specific conditions, by
isolating a situation or an \co{object} and excluding the possibility of
interference from the unpredictable surroundings that we arrive at the laws
which we consider necessary. If the result above is claimed to hold with
necessity {\em only} because there is a general law saying that any positron and
any electron will necessarily annihilate under given conditions, then it is just
another level of the same -- isolating and narrowing conditions to specify
sufficient reasons for some effect.  The \wo{any} may give an impression of
generality but it is only an impression. This apparent generality merely hides
the highly particular definitions of electron, positron, their specific
properties, in addition to the \thi{given conditions}, to the \thi{other things
  being equal} which underlies every claim to necessity.

The way to necessity goes via increased \co{precision} and specialisation, i.e.,
in the opposite direction than the way to universality.  The more content, the
less necessity.\noo{The statement $P\lor \neg P$, when taken
as a formula in propositional logic, is necessarily true. The statement
\wo{Paul will visit us or not} seems to be also necessary, provided that we know
who Paul is, who \wo{we} are and what \wo{visiting us} amounts to. The statement
\wo{It is raining or it is no}, especially when \wo{or} is taken as exclusive
disjunction, has lost almost all of the assumed necessity for \wo{raining} is
too vague and ambiguous a word to admit such a bivalence.} The richer the perception of a situation, the more
possibilities it unveils, the less tractable and the more difficult to control
it becomes.  And hence the attempts to design a grand theory of everything, to
subsume the whole world under the rule of necessary laws impoverish the world.
Certainly, some parts of the world can be reduced to simple entities which are
prone to the descriptions in terms of the necessary. (Such descriptions seem
always to conjure the possibility of control.) 
But the dangerous impoverishment occurs when the drive is uninhibited, when it
is the drive to defeat everything escaping control.  Only disappearance of
content makes perfect necessity possible.


\pa The obvious attempt to obtain necessity seems thus to look for judgments
with no content. Tautologies and contradictions were suggested but then one
should, perhaps, include also meaningless statements having no content. Besides,
%as we observed in \refp{pa:immedNexus},
even the non-contradiction principle is not necessary unless one assumes
appropriate reduction of the domain of discourse. This reduction goes in the
direction of \co{immediacy} and ends with mathematics. The alternative (to the
analytical necessity of empty statements) is to remove all content from the
considered objects, leaving only the ultimate minimum of \co{precise}
alternatives: \thi{yes} or \thi{no}, a \co{pure distinction}. Necessity amounts
to removing possibilities and the limit of this process is when only one
possibility remains.\ftnt{Formalisations do not concern us. The fact that a
  necessary fact is typically represented as obtaining in all possible worlds,
  does not change its status as the {\em only} possibility. In this respect,
  {\L}ukasiewicz logic of necessity, built on his three-valued logic and given
  the simple truth-value semantics, represented this point in a clearer way than
  the, technically perhaps more satisfying, modal logics with possible-worlds'
  semantics.} But to be able to exclude possibilities with full obviousness and
\co{precision}, these must be first \co{precisely} given.  Necessity of
mathematical results is only another side of their ultimate \co{precision} and
is based exclusively on the character of the fundamental objects -- the
ultimately reduced, most \co{immediate}, entirely contentless \co{pure
  distinctions}, devoid of any interfering context, in the pure isolation of
\thi{all other things being equal}. The source of this necessity is the pure
bivalence, the ultimate \la{tertium non datur}, the absolute character of
negation which, viewed within pure \co{immediacy}, allows two and only two
alternatives, \thi{being} or \thi{not-being}, \thi{yes} or
\thi{no}.\ftnt{$7+5=12$ is, as Kant argues, synthetic ($12$ is not defined as,
  nor included in, $7+5$) and necessary judgment.\kilde{CPrR,Intro:V[B];p.52}
  What makes it necessarily true is not any mere linguistic convention, although
  it is the meaning of the involved terms and operations. Its necessity lies in
  the fact that the only other possibility, $7+5\not=12$, not only does not
  obtain but also, given the (univocal and \co{precise}!) meanings of the
  involved terms and operations cannot possibly obtain.
  %
  Of course, in more advanced constructions, several possibilities can obtain.
  But even then one has the complete overview and a proof amounts to excluding
  all possibilities except one. A typical mistake in a proof is overlooking one
  possibility, one case which does obtain and yields another final result.
  
  Emphasizing the fundamental role of bivalence in mathematics we do not, of
  course, imply that, for instance, either continuum hypothesis or its negation
  must follow from ZF(C) or that, in general, given a mathematical context every
  possible question must have a unique answer. Such undetermined questions
  concern advanced mathematical constructions and not the original objects with
  which we are dealing here and which comprise only natural numbers and basic
  geometrical intuitions. But also such advanced and undetermined questions
  generate always inquiry into the conditions determining {\em uniquely} each
  of the alternative answers.}

Bivalent logic with non-contradiction principle, as suggested in
\refp{pa:logicA}, is associated with the level of \co{immediacy} and now we
encounter also necessity as yet another \co{aspect} of this \nexus. It springs
from the idea that things could not possibly be different, the idea residing in
the residual point of \thi{now}, where there is only what there is, ultimately
\co{dissociated} from the surroundings and thus as unavoidable and necessary as
it is arbitrary and spontaneous. The site of necessity is \co{immediacy} and
attempts to extend it beyond this narrow horizon fail rather miserably, as Hume
has shown.\ftnt{Let us remember that his analyses do not affect mathematical
  results. What \citef{mathematics concludes, in regard to such things as
    numbers, proportions and figures is indubitably true and cannot be
    otherwise}{Salisbury}{II:C.13} To the possible story-tellers, who would try
  to claim that not only natural science but even mathematics is only yet
  another way of telling a story, one could say that although mistakes in proofs
  are possible and we may be uncertain whether a proof stretching over tens of
  pages or one supported by unverified computer calculations is correct, the
  mathematical result itself is always stated as necessary. Fallibility of the
  \co{actual} criteria of certainty does not contradict the ultimate necessity
  of the correct results.} In particular, they must first reduce the objects of
interest to the residual points, and such a reduction is seldom satisfactory.
Furthermore, most imaginable alternatives, i.e., most possible worlds are
completely irrelevant, existentially uninteresting, and this is what makes
necessity an \thi{unnatural}, almost inhuman property.  Implausibility of
claiming that something could not be otherwise is also the reason for the almost
instinctive rejection -- at least by the common sense -- of all sophisticated
arguments produced in favour of determinism.

Mathematical statements are not empty tautologies.\noo{Perhaps, they are
  tautologies, if these be defined as necessarily true statements.}  Perhaps,
one could develop an apparatus making $1\not = 2$ an empty statement.  For our
part it seems hard to imagine how one could even start doing this without the
prior \co{distinction} between one $\bullet$ and another $\bullet$, without the
\co{experience} (not \co{an experience}!) of $\not =$.  Mathematical
propositions tell us the story of the objects they describe.  Their necessity
follows not from their emptiness but from the emptiness of these object.  It
does not hide in any formal properties of the proof techniques or particular
axiomatisations.  All such techniques have equally necessary character because
they all have to conform to the standards of \co{immediate} univocity set up by
their objects.  The fact that a mathematical theorem is either true or false
mimics only the contentless duality of the \co{pure distinction}.\noo{(And no
  many-valued logic can change anything in this respect.  Such a logic is based
  on multiplicity rather than duality of truth values and, more importantly,
  (meta)theorems about such logic are usual mathematical theorems which either
  hold or not.) } It is the ultimate poverty of mathematical objects which
accounts for the necessity of mathematical truths.


\pa There are degrees of approximation to pure \co{immediacy}, that
is, degrees of abstraction from the \co{concrete} content, and hence degrees of
necessity -- objects may be more or less abstract, depending on how close they
come to the level of \co{immediacy}.  Consider the increase of the \thi{real}
content in passing from mathematics to physics, then to biology, from biology to
history, from sociology to literature.  This increase is clearly accompanied by
the decreasing degree of necessary determinations or, if you allow, by the
increase of freedom.  It is no coincidence that the scientific and philosophical
attempts to establish a system of necessary laws end up with abstract
statements.  But the statements are abstract not because they are empty
tautologies -- they are abstract because they had to dispense with most of the
\co{concrete} content of the described objects.  The search for the infallible
laws leads sciences to construe their objects in a more and more simple and
elementary fashion because control depends on the formulation of sufficient
reasons and this sufficiency -- necessity -- requires \co{precision}, that is,
approximation to \co{immediacy}, reduction of the \co{concrete} content.  It is
always as tempting to postulate necessity -- in form of sufficient reasons,
binding explanations, inviolable laws of nature or reason\ldots -- as it is hard to
justify such postulates. For to justify them one has to reduce everything to
some form of mathematics.  Unfortunately, one attempts such reductions not only
in physics or quantum mechanics but in most sciences or, at least, what
tries to call itself \wo{science}.  Not only natural sciences but also economy,
sociology, even psychology display the symptoms of the mathematical disease.
The mathematical point, the vanishing (or rather the barely appearing)
indication of something-being-there, the shadow of the perfect atom is the constant
ideal of the knights of necessity.\noo{ fighting the windmills of \co{vagueness}. }

\noo{ Necessity, understood as {\em sufficient} condition,\ftnt{A
    misunderstanding is possible here here. Necessity is related, if not
    identical, to the sufficiency of some conditions, efficiency of some causes:
    if $x$ occurs, then $y$ {\em must} occur, too. This is the positive schema
    of necessity.  It is thus opposed to what is called \wo{necessary
      conditions} which we use throughout.  But \wo{necessary condition} means
    only that it is required and indispensable: if $y$ is to occur, $x$ must
    occur (too, or first). Even if both \thi{musts} were analysed as exactly the
    same (which easily happens if everything is reduced to a mere propositional
    implication), the relation between $x$ and $y$ in both cases (or, for that
    matter, between $x$ and $y$ in the first and $y$ and $x$ in the second) is
    entirely different and it is the relation which occupies us.  E.g., \wo{$x$
      is sufficient for $y$} and \wo{$y$ is necessary for $x$} are two
    equivalent readings of the classical implication $x\rightarrow y$.  But we
    do not imagine that the propositional implication \thi{$\rightarrow$} has
    captured all the important aspects of anything, and such a duality between
    sufficient and necessary conditions does not obtain, at least, not in our
    case. Our \thi{necessary reasons} involve, in so far as they are
    \thi{reasons}, some form of real precedence, either in the order of time or
    of \co{founding}; while in so far as they are \thi{necessary} such a
    precedence is indispensable. (One must not, however, \co{dissociate} the
    two -- \wo{necessary reason} is a pleonasm, for reason without any form of
    necessity is no reason.) The only form of necessity we thus get is of
    negative kind: if $x$ does not happen/obtain, neither does $y$. Necessity
    discussed in this section, on the other hand, is that of the
    \thi{sufficient} branch.}  is an ideal limit which obtains only in the
  totally univocal, because contentless, bivalence of pure \co{immediacy}, in
  the realm of \co{pure distinctions}, that is, in mathematics.
  
  That it is so natural to \thi{apply} the idea of necessity in other contexts,
  even without much knowledge of mathematics, is of course due to the synthetic
  character of mathematics, due to the fact that everything carries a \co{trace}
  of \co{pure distinctions}. But such an application requires reduction.  }
  
\sep

\pa \citet{[O]ur only approach to divine things is through symbols [and] we can
  {\em appropriately} use mathematical signs because of their incorruptible
  certitude.}{DDI}{I:11.\para 32} We can use them appropriately -- with
sufficient infallibility -- because their signification is limited to the
\co{immediacy} of \co{pure distinctions}, where \co{sign} and signified
coincide. But this appropriateness is exhausted by that.  According to Hugo
Steinhaus, \wo{mathematics is the science of objects which do not
  exist}.\ftnt{Bovelles described creature as \citef{beings which are
    not.}{Bovelles}{ p.75,97 \citaft{BogNic}{ p.91\kilde{ftnt.41}}} Misusing the
  analogy of expression, mathematics applies then to all creature; to the most
  fundamental, albeit very limited aspect of it.} Indeed, the limit of
\co{immediacy} is the point where objects cease to exist, cease to be objects
and become mere points. Applying mathematics to anything demands that we look at
the thing as a mere pure difference, a mere point of distinction. In spite of
attempts to reduce various sciences to a mathematical dimension, we do not
really think that it is entirely meaningful to transfer the necessity and
certitude of mathematics to other domains of \co{experience}.  In fact,
attempting such a reduction, we immediately realise that it is just that: a
reduction.  Mathematics captures and elaborates the fundamental aspect of
\co{experience}, the fact of \co{distinguishing}. This may serve as a source of
powerful analogies and useful similes.
%
The \thi{emptiness} of mathematical objects will always remain on the border of
mysticism and resonate deeply underneath the possible suspicions about
tautological emptiness of mathematical results. The emptiness of objective
content, the \co{purity} of \co{distinctions}, lifts it \co{above} all
\co{experience} and makes it almost as empty as its closest neighbour,
\co{nothingness} itself.
%
But the beginning does not contain the end, the \co{original virtuality} does
not determine \co{actuality}.  It is not so that \citet{[b]y number, a way is
  had, to the searching out and understanding of every thyng, hable to be
  knowen.}{PicoConcl}{Mathematical Conclusions:11 (\la{Conclusiones de
    mathematicis secundum opinionem propriam, numero LXXXV}), as quoted in
  \citeauthor*{DeePref}.} Trying to \thi{Pythagorise and philosophise by
  mathematics} alone ends, if not in the labyrinths of numerology, then at a
philosophical desert, as great as it is empty.  With respect to \co{concrete
  experience}, the mathematical images, built atop contentless \co{objects},
will always remain only, and only at best, analogies and similes.


