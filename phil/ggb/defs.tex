\usepackage[T1]{fontenc} % only for nice small <<, >>

%% Latex says that `footnote' is already defined
%% but this does work - starting the footnotes numbering
%% anew at each page
%%-- Do I need it?
%\newcounter{footnote}[page]

%%%% spacing
\newcommand{\newpa}{\newpage \vspace*{-3ex}} % at the end of \pa and new at new page
\newcommand{\newp}{

  \ \vspace*{-4ex} }  % to force pagebreak just before a new \pa after footnote 


%%%%%%%%%%%%%%%%%%%%%%%%%%
\newcommand{\ins}[1]{{\bf [#1]}} %modified after last printout, 22.01.04++

%%%%%% cross through the word argument
\newlength{\wordwidth}
\newcommand{\cross}[1]{\mbox{#1\settowidth\wordwidth{#1}\llap
{\rule[0.5ex]{\wordwidth}{0.04em}}}}

%overline - 
\newlength{\letterheight}
\newcommand{\ovr}[1]{\mbox{#1\settowidth\wordwidth{#1}\settoheight\letterheight{\large{#1}}\llap 
{\rule[\letterheight]{\wordwidth}{0.04em}}}}
%{\rule[1.1ex]{\wordwidth}{0.04em}}}}


%%%% temporary
\newcommand{\twot}[2]{\pan -- summarising transcendence --
  \begin{itemize}\MyLPar\item {\bf Horizontal} -- #1 
\item {\bf Vertical} -- #2 \end{itemize}}

% only auxiliaires (also for quotes from Thomas)
\newcommand{\titp}[1]{\vspace*{0.7ex}\par\noindent{\bf #1}}
\newcommand{\titu}[2]{\vspace*{0.7ex}\par\noindent{\small{{\bf -#1} #2}}}
\newcommand{\tsep}[1]{\vspace*{1ex}\par\noindent\hrulefill{\ {\bf #1}\ 
}\hrulefill}
% remove \tsep or \titp
\newcommand{\wtsep}[1]{}

%%%%%%%%%%%%%%%%%%%%%%%%%%%%%%%%%%%%%%%%%%%%%%%%
%%%%%%%%%%%%%%% Concepts, words, things, \ldots

\newcommand{\co}[1]{{\sl{#1}\/}} % concept
\newcommand{\cco}[1]{#1} % otherwise concept, but here - intro - not yet...
\newcommand{\wo}[1]{``#1''} % word
\newcommand{\thi}[1]{`#1'} % a thing, phenomen
\newcommand{\abr}[1]{{\sc #1}} % abbreviation
\newcommand{\bc}{{\sc bc.}}
\newcommand{\add}{{\sc ad.}}
\newcommand{\afterPhD}[1]{#1} %not in the submitted PhD!
\newcommand{\PSet}{{\cal P}}

% Laws of Form
\newcommand{\esp}{\ \ \ \ }
% \newcommand{\dist}{\hor\ver\hspace*{0.3em}}
% \newcommand{\hor}{\raisebox{1.5ex}{\line(1,0){8}}}
% \newcommand{\ver}{\raisebox{1.5ex}{\line(0,-1){8}}}
% \newcommand{\twice}{\dist\hspace*{-1em}\raisebox{2ex}{\line(1,0){11}}\raisebox{2ex}{\line(0,-1){11}}\hspace*{0.3em}}
\newcommand{\dist}{\hor\ver\hspace*{0.3em}}
\newcommand{\hor}{\raisebox{1.5ex}{\line(1,0){4}}}
\newcommand{\ver}{\raisebox{1.5ex}{\line(0,-1){4}}}
\newcommand{\twice}{\dist\hspace*{-1.5em}\raisebox{2ex}{\line(1,0){5}}\raisebox{2ex}{\line(0,-1){5}}\hspace*{0.3em}}


%%%%%% hidding
\newcommand{\woo}[2]{``#1''} % wording but from citation which appears elsewhere
                             % -- the second argument is the source which shall
                             % not be printed

\newcommand{\noo}[1]{}    % probably to be completely removed
\newcommand{\kilde}[1]{}  % just for record
%\newcommand{\orig}[1]{ [\guillemotleft #1\guillemotright]} % quote in the original
                                                        % language;
\newcommand{\orig}[1]{} % quote in the original
%%%%%%

\newcommand{\quo}[1]{``#1''} % quote
\newcommand{\imp}[1]{{\bf\em #1}} % important, I guess

% Dubious things
\newcommand{\?}[1]{{\bf ?}{\em #1\/}{\bf ?}}  % questionable word, passage
\newcommand{\he}[1]{{\bf ?[#1]?}} % check the word
\newcommand{\hee}[1]{} % check the word -- but do not mark in the printout
%\newcommand{\verify}[1]{\marginpar{{\bf !?!?}}} % verify
\newcommand{\verify}[1]{}

% Languages 
\newcommand{\la}[1]{{\em #1\/}} % Latin ...
\newcommand{\ger}[1]{{\em #1\/}} % German ...
\newcommand{\gre}[1]{{\em #1\/}} % Greek ...
\newcommand{\fre}[1]{{\em #1\/}} % French ...
\newcommand{\heb}[1]{{\em #1\/}} % Hebrew ...

% Dialogs
\newcommand{\que}{\item[{\bf ?}]}
\newcommand{\ans}{\item[{\bf !}]}


%%%%%%%%%%%%%%%%%%%%%%%%%%%%%%%%%%%%%%%%%%%%%%%%
%%%%%%%%%%% abbreviated Expressions

%%% equal + strong
% Having equivalent signification and reach; 
% expressing the same thing, but differently
% (for ME: expressing the same level)
%%% but also, possibly (though not for ME):
% having same meaning though differently expressed.
\newcommand{\equi}{\co{equipollent}}
\newcommand{\Equi}{\co{Equipollent}}
\newcommand{\equin}{\co{equipollence}}
\newcommand{\Equin}{\co{Equipollence}}

%%% Nexus - the `complex' of equipollent aspects
\newcommand{\nexus}{\co{nexus}}
\newcommand{\nexuss}{\co{nexuses}}
\newcommand{\Nexus}{\co{Nexus}}
\newcommand{\Nexuss}{\co{Nexuses}}


\newcommand{\G}{\co{Yes}} % good - Yes
\newcommand{\B}{\co{No}} % bad - No
\newcommand{\yes}{\G}
\newcommand{\Yes}{\yes}
\newcommand{\No}{\B}
\newcommand{\HH}{\co{invisible}} % higher - Invisible
\newcommand{\LL}{\co{visible}} % lower - Visible

\newcommand{\oo}{\co{original}} % 
\newcommand{\os}{\co{original sign}} % 
\newcommand{\oss}{\co{original signs}} % 
\newcommand{\Os}{\co{Original sign}} % 
\newcommand{\Oss}{\co{Original signs}} % 
\newcommand{\rr}{\co{reflective}} % 
\newcommand{\rs}{\co{reflective sign}} % 
\newcommand{\rss}{\co{reflective signs}} % 
\newcommand{\Rs}{\co{Reflective sign}} % 
\newcommand{\Rss}{\co{Reflective signs}} % 

\newcommand{\ic}{\co{consciousness}} % {immediate consciousness}
\newcommand{\hoa}{\co{horizon of actuality}} 
\newcommand{\herenow}{\co{here-and-now}}
\newcommand{\heno}{\co{here-and-now}}
\newcommand{\thth}{\co{there-and-then}}
\newcommand{\more}{\co{more}}
\newcommand{\pexp}{\co{proto-experience}}
\newcommand{\Pexp}{\co{Proto-experience}}

\newcommand{\sch}{\co{spiritual choice}}
\newcommand{\Sch}{\co{Spiritual choice}}

%%%%%%%%%%%%% used inside \pa for listing levels
\newcommand{\imm}{{\bf 1.\ }}
\newcommand{\act}{{\bf 2.\ }}
\newcommand{\mine}{{\bf 3.\ }}
\newcommand{\inv}{{\bf 4.\ }}
%%       ???
%% 1- immediacy/singularity
%% 2- relational/complexes
%% 3- totality/person(ality)
%% 4- infinity/unconditional



%%%%%%%%%%%%%%%%%%%%%%%%%%%%%%%%%%%%%%%%%%%%%%%%
%%%%%%%%%% Sections, Chapters, marking \ldots

\newcommand{\chapQ}[2]{\vspace*{-40ex}  
\ \hspace*{2em}\ \hfill{\parbox[b]{8cm}{\small{``{\em #1\/}''\hfill{#2}}\normalsize}} 
\\[30ex] 
}
%%% as above - variable breadth
\newcommand{\chapQQQ}[3]{\vspace*{-40ex}  
\ \hspace*{2em}\ \hfill{\parbox[b]{#1cm}{\small{``{\em #2\/}''\hfill{#3}}\normalsize}} 
\\[30ex] 
}

\renewcommand{\chaptermark}[1]{\markboth{{\small{(\today)}}\ 
\thechapter. #1}{}}
\renewcommand{\sectionmark}[1]{\markboth{\thechapter:\thesection.\ #1}{}}
\renewcommand{\subsectionmark}[1]{\markright{\thesubsection.\ #1}}

\newcommand{\secQpar}[4]{\vspace*{-#1ex}  \ \hspace*{2em}\ 
\hfill{\parbox[b]{#2cm}{\small{``{\em #3\/}''\hspace*{\fill{#4}}}\normalsize}} 
}
\newcommand{\secQparA}[4]{\vspace*{-#1ex}  \ \hspace*{2em}\ 
\hfill{\parbox[b]{#2cm}{\small{``{\em #3\/}''{\hspace*{4em}{#4}}}\normalsize}} 
} %In the beginning was Word

\newcommand{\secQ}[2]{\vspace*{-5ex}  \ \hspace*{2em}\ 
\hfill{\parbox[b]{8cm}{\small{``{\em #1\/}''\ \hfill{#2}}\normalsize}} 
\\[3ex] 
}
\newcommand{\secQQ}[1]{\vspace*{-6ex} \ \hspace*{2em}\hfill{\small{``{\em
        #1}''}}}

% variable quote at section start - varied width of the box
\newcommand{\secQQQ}[3]
{\vspace*{-5ex} \ \hspace*{2em}\hfill{\parbox[t]{#1cm}{\small{``{\em 
#2}'' \hspace*{\fill{#3}}}}}}


\newcommand{\subsecQ}[2]{\vspace*{-3ex}  \ \hspace*{2em}\ 
\hfill{\parbox[b]{6.5cm}{\small{``{\em #1\/}''\hspace*{\fill{#2}}\normalsize}}
%\\ 
}}
\newcommand{\secQv}[2]{\vspace*{-8ex}  \ \hspace*{2em}\ 
\hfill{\parbox[b]{8cm}{\small{{\em #1}\hfill{#2}}\normalsize}} 
\\[3ex] 
}

\newcommand{\ffrom}[1]{\\ \ \hspace*{2em} \hfill{#1}}
\newcommand{\from}[1]{\ \hfill{#1}}


%%%%%%%%%%%%%%%%%%%%%%%%%%%%%%%%%%%%%%%%%%%%%%%%
%%%%%%%%%% PARAGRAPHIES

\newcommand{\pbtable}[1]{\parbox{2.5cm}{#1}}

\newcommand{\MyLPar}{\parsep -.2ex plus.2ex minus.2ex\itemsep\parsep
   \vspace{-\topsep}\vspace{.5ex}}


\newcounter{PARAGRAPH}[chapter]

%%% Works - is fine
%\newcommand{\MyNumEnv}{\trivlist\refstepcounter{PARAGRAPH}\item[\hskip
%   \labelsep{\footnotesize[\arabic{PARAGRAPH}]}] } {\ignorespaces}

%%%  --- EXPLORE \marginpar makes WHAT YOU WANT
\newcommand{\MyNumEnv}{\refstepcounter{PARAGRAPH}
%  \setcounter{equation}{0}
%\trivlist\item[{}]
\ignorespaces\vspace*{1ex}\par\ignorespaces\noindent
\marginpar{{\footnotesize{\arabic{PARAGRAPH}.}}}  \hspace*{-.5em} }
%\marginpar{{\footnotesize{\arabic{PARAGRAPH}.}}}  \hspace*{-.5em} }

\setlength{\marginparwidth}{7mm} % dirty adjust of left to the text

%%% - here is some new hack to be tried:
%\newcommand{\mymarginpar}[1]{\ifodd{\marginpar{\begin{flushright}#1\end{flushleft}}}
%  \else{\marginpar{\begin{flushright}#1\end{flushright}}}}  

%%%  --- here is some hack
%\newcommand{\mymarginpar}[1]{%
%\vadjust{\smash{\llap{\parbox[t]{\marginparwidth}{#1}\kern\marginparsep}}}}

% standard paragraph:
\newenvironment{pa}{\ignorespaces\MyNumEnv}{\ignorespaces\par\ignorespaces\noindent\addvspace{0.5ex}}

% paragraph to be filled in:
\newcommand{\MyNumEnvS}{\refstepcounter{PARAGRAPH}
\vspace*{1ex}\par\noindent
\marginpar{{\footnotesize{{\bf\arabic{PARAGRAPH}.}}}}\hspace*{-.5em} }

\newenvironment{say}{\vspace*{1ex}\par\noindent \marginpar{\bf
    Say:}}{\vspace*{0.5ex}\par\noindent} 
%\newenvironment{say}{\MyNumEnvS {\bf Say:\ }}{\par\noindent\addvspace{0.5ex}}
\newenvironment{maybe}{\MyNumEnvS [{\bf Perhaps:\ }}{]\par\noindent\addvspace{0.5ex}}

\newcommand{\sayy}{\vspace*{1ex}\par\noindent\marginpar{\bf Say!}}


% margin mark for YES
\newcommand{\yeah}{\marginpar{\bf Yea!}}

\newcommand{\elab}[1]{\par\noindent !\dotfill{elab:~#1}\dotfill !\par\noindent}
\newcommand{\celab}[1]{!\dotfill{elab:~#1}\dotfill !\par\noindent}

% \bf paragraph with the title
\newcommand{\bpa}[1]{\trivlist\refstepcounter{PARAGRAPH}{\bf{\item[\hskip
  \labelsep{[\arabic{PARAGRAPH}]}] #1.}}\par} {\ignorespaces}

\newcounter{SUBPARAGRAPH}[PARAGRAPH]
\def\theSUBPARAGRAPH{\arabic{PARAGRAPH}.\arabic{SUBPARAGRAPH}}

%%% Works - is fine
%\newcommand{\MyNumEnvSub}{\trivlist\refstepcounter{SUBPARAGRAPH}
%\item[\hskip\labelsep{\footnotesize[\theSUBPARAGRAPH]}] } {\ignorespaces}

\newcommand{\MyNumEnvSub}{%\trivlist
\refstepcounter{SUBPARAGRAPH}
\vspace*{1ex}\par\noindent
\marginpar{{\footnotesize{\theSUBPARAGRAPH}}} } 

\newenvironment{subpa}{\MyNumEnvSub}{\par\noindent\addvspace{0.5ex}}

% unnumbered paragraph
\newcommand{\MyNumEnvN}{\trivlist\item[]} {\ignorespaces}
\newenvironment{pan}{\MyNumEnvN}{\par\noindent\addvspace{0.5ex}}

% comment -  small paragraph
\newcommand{\com}[1]{\vspace*{1ex}\par\noindent{\small{[\hfill{#1}\hfill]}}\normalsize\par\noindent}
\newcommand{\ccom}[1]{{\small{[\hfill{#1}\hfill]}}\normalsize\par\noindent}
% and numbered
\newenvironment{compa}[1]{\MyNumEnv{\small{#1}}}
       {\par\noindent\addvspace{0.5ex}\normalsize}

% Komment/structuring - to be removed
\newcounter{KOM}[section]%[subsection]
\newcommand{\kom}{\vspace*{1ex}\par\noindent\refstepcounter{KOM}{\bf \theKOM.} }

%minimal size -- notes after \kom's to be removed
\newcommand{\tin}[1]{{\tiny{#1}}}


\newenvironment{rephrase}{\vspace*{0.5ex}\par\noindent\dotfill{rephrase}\dotfill}
{\vspace*{0.5ex}\par\noindent\dotfill{end}\dotfill\normalsize\par\noindent}
       
%%% questionable paragraph:
%%% Works - is fine       
%\newcommand{\MyNumEnvQ}{\trivlist\item[\hskip
%   \labelsep{{\small{\bf ??}}}] } {\ignorespaces}

\newcommand{\MyNumEnvQ}{\vspace*{1ex}\par\noindent
   \marginpar{{\small{\bf ??}}} }% {\ignorespaces}
   
\newenvironment{qpa}[1]{\MyNumEnvQ {\small{#1}}}{\par\noindent\addvspace{0.5ex}}

\newcommand{\qpp}[1]{\vspace*{0.5ex}\par\noindent 
        ?\dotfill{quest: #1}\dotfill ?\vspace*{0.5ex}\par\noindent}

%%% illustrating/example paragraph
\newenvironment{ipa}{\MyNumEnv}{\par\noindent\addvspace{0.5ex}}

%%% used elsewhere paragraph: -- for quotes
%%% Works - is fine
%\newcommand{\MyNumEnvU}{\trivlist\refstepcounter{PARAGRAPH}\item[\hskip
%   \labelsep{{$\langle$\small{\arabic{PARAGRAPH}}$\rangle$}}] } {\ignorespaces}

\newcommand{\MyNumEnvU}{\refstepcounter{PARAGRAPH}
\vspace*{1ex}\par\noindent
\marginpar{{\footnotesize{(\thePARAGRAPH)}}} } % {\ignorespaces}

\newenvironment{upa}[1]
{\MyNumEnvU {\small{#1}}}{\par\noindent\addvspace{0.5ex}}

% Figurs to be numbered by Chapter.paragraph - if not more than one in a paragraph
% \newcounter{FIG}[PARAGRAPH]
% \def\thefigure{\Roman{chapter}:\arabic{PARAGRAPH}.\roman{FIG}}  %SMTH WRONG!!!

% ``Equations'' to be numbered as Figures
%\def\theequation{\Roman{chapter}:\arabic{PARAGRAPH}.\roman{equation}} %SMTH WRONG!!!
%\def\theequation{\Roman{chapter}.\arabic{equation}}
% well, not any longer
\def\theequation{\roman{equation}}

\def\thechapter{\Roman{chapter}} 
\def\thesection{\arabic{section}}
\newcounter{SUBSUB}[subsection]
\def\theSUBSUB{\thesubsection.\arabic{SUBSUB}}
\newcommand{\subsub}[1]{\refstepcounter{SUBSUB}
\addcontentsline{toc}{subsection}{\ \ \ \theSUBSUB. #1}
  \subsubsection{\bf\theSUBSUB. #1}}

\newcommand{\subsubnonr}[1]{
%\addcontentsline{toc}{subsection}{\ \ \ \ \ \ \ \ \ \ \ -- #1}
  \subsubsection*{#1}}
\newcommand{\subnonr}[1]{
\addcontentsline{toc}{section}{#1}
  \subsection*{#1}}

\newcounter{SUBSUBI}[SUBSUB]
\def\theSUBSUBI{\theSUBSUB.\roman{SUBSUBI}}
\newcommand{\subsubi}[1]{\refstepcounter{SUBSUBI}
%\addcontentsline{toc}{subsection}{\ \ \ \ \ \theSUBSUBI. #1}
  \subsubsection{\bf\theSUBSUBI. #1}}
  
% a separation, kind of subsub-section, but no title
\newcommand{\sep}{\begin{center}{\large{\bf * * *}}\end{center}}

%\newcounter{THESIS}[chapter]
%\def\theTHESIS{\Roman{chapter}.\arabic{THESIS}}
% thesis
%\newcommand{\thesis}[1]{\trivlist\refstepcounter{THESIS}\item[\hskip
%   \labelsep{{\scriptsize{\fbox{Thesis 
%   {\theTHESIS}}}}}] {\em #1} \vspace*{1ex}\par\noindent}
\newcommand{\thesis}[1]{
\vspace*{1ex}\par\noindent
%\begin{quote}
\refstepcounter{PARAGRAPH} 
\marginpar{\footnotesize{\thePARAGRAPH.}}
{\em #1}\vspace*{1ex}\par}
%#1 \end{quote} }

\newcommand{\thesisnonr}[1]%{\begin{quote} #1 \end{quote}}
{\vspace*{1ex}\par\noindent {\em #1}\vspace*{1ex} }
%{\vspace*{1ex}\par\noindent {\em #1}\vspace*{1ex}\par }


% thesis ad
\newcommand{\MyNumEnvTAD}{%\trivlist
  \refstepcounter{PARAGRAPH} 
%%\item[\hskip \labelsep{{\scriptsize{\fbox{\arabic{PARAGRAPH}}}}}]} 
%\item[\hskip \labelsep{{\scriptsize{\arabic{PARAGRAPH}}}}]}
\vspace*{1ex}\par\noindent\marginpar{\footnotesize{\thePARAGRAPH.}}
}%          {\ignorespaces}
\newenvironment{thesisad}[1]{\MyNumEnvTAD {{\bf{\em #1}}}\par }{\par\noindent\addvspace{0.5ex}}

% problem to be solved:
\newcommand{\MyNumEnvP}{\trivlist\refstepcounter{PARAGRAPH}\item[\hskip
%%%   \labelsep{{\small{\bf !{\thePARAGRAPH}!}}}] } {\ignorespaces}
   \labelsep{{\small{\bf !{\arabic{PARAGRAPH}}!}}}] } {\ignorespaces}
\newenvironment{prob}[1]{\MyNumEnvP {#1}}
        {\ignorespaces\par\noindent\addvspace{0.5ex}}

%  definition
\newcommand{\MyNumEnvD}{\trivlist\refstepcounter{PARAGRAPH}\item[\hskip
%%%   \labelsep{{\scriptsize{\fbox{\thePARAGRAPH}}}}] } {\ignorespaces}
   \labelsep{{\scriptsize{\fbox{\arabic{PARAGRAPH}}}}}] } {\ignorespaces}
\newenvironment{Def}[1]{\MyNumEnvD{\sf #1}\index{#1}}
        {\ignorespaces\par\noindent\addvspace{0.5ex}}

% notes:
% \newcounter{NOTE}
% \newcommand{\MyNumEnvN}{\trivlist\refstepcounter{NOTE}\item[\hskip
%   \labelsep{\footnotesize[\theNOTE]}] } {\ignorespaces}

\newenvironment{no}{\MyNumEnv}{\par\noindent\addvspace{0.5ex}}
% Title and LF
\newenvironment{ad}[1]
{%\vspace*{1ex}
  \MyNumEnv {{\bf{\em #1}}}\nopagebreak\par\nopagebreak}{\par\noindent\addvspace{0.5ex}}

% Title in-line: tit paragraph
\newenvironment{tpa}[1]
{%\vspace*{1ex}
  \MyNumEnv {{\bf{\em #1}}}}{\par\noindent\addvspace{0.5ex}}

%\newcommand{\tit}[1]{\paragraph{#1}} %{\vspace*{1ex}\par\noindent{\bf #1} }
\newcommand{\tit}[1]{\subsubsection*{#1}} 

\newcommand{\related}[1]{\footnote{#1}}


%%%%%%%%%%%%%%%%%%%%%%%%%%%%%%%%%%%%%%%%%%%%%%%%
%%%%%%%%%%%%%%  Arrays, LEVELS
\newcounter{LEVELS}
\def\theLEVELS{{\bf{\Roman{LEVELS}}}}
\newcommand{\levels}[1]{\trivlist\item[\hskip
   \labelsep] {\sf #1}}
% \newcommand{\levels}[1]{\trivlist\refstepcounter{LEVELS}\item[\hskip
%    \labelsep \theLEVELS.] {\sf #1}}
 %{\ignorespaces}
\newcommand{\are}[4]{\begin{enumerate}\MyLPar
  \item #1 \item #2 \item #3 \item #4 \end{enumerate}}
\newcommand{\aretab}[5]{\par\noindent \begin{tabular}{r#1}
  1.& #2 \\ 2.& #3 \\ 3.& #4 \\ 4.& #5 \end{tabular}\par\vspace{1ex}\noindent}

\newcommand{\aretabb}[5]{\par\noindent \begin{tabular}{r#1}
  1.& #2 \\ 2.& #3 \\ 3.& #4 \\ 4.& #5 \end{tabular}}
\newcommand{\aretabbINV}[5]{\par\noindent \begin{tabular}{r#1}
  4.& #2 \\ 3.& #3 \\ 2.& #4 \\ 1.& #5 \end{tabular}}

\newcommand{\levs}[6]{\parbox{#1cm} {
\trivlist\refstepcounter{LEVELS}\item[\hskip
   \labelsep \theLEVELS.] {\sf #2} {\ignorespaces}
   \aretabb{rl}{\co{immediacy}: & #3}
              {\co{actuality}: & #4}
              {\co{miness}: & #5}
              {\co{invisible}: & #6} }
}

\newcommand{\levsINV}[6]{\parbox{#1cm} {
\trivlist\refstepcounter{LEVELS}\item[\hskip
   \labelsep \theLEVELS.] {\sf #2} {\ignorespaces}
   \aretabbINV{rl}{\co{invisible}: & #3}
              {\co{miness}: & #4}
              {\co{actuality}: & #5}
              {\co{immediacy}: & #6} }
}

% used for additional 2-aspects at each level:
\newcommand{\levsTab}[6]{
\vspace*{0.5ex}\par\hspace*{1em}\parbox{12cm} {
%{\sf #1} {\ignorespaces} \par
   \begin{tabular}{l@{\ \ }r@{\ \ --\ \ }l}
      #2 \\ \hline
  \inv: & #6 \\      
  \mine: & #5 \\ 
  \act: & #4 \\ 
  \imm: & #3  
\end{tabular}}   \vspace*{0.5ex}\par\noindent
}

% used for onto-founding (up) and concrete-founding (down)
\newcommand{\found}[8]{\parbox{14cm} {
\trivlist\refstepcounter{LEVELS}\item[\hskip
   \labelsep \theLEVELS.] {\sf #1}\par
\begin{tabular}{rr@{\hspace*{2em}}l}   
4.& \multicolumn{2}{r}{#5\ \ \ $\downarrow$\ \ } \\
3.& #4 & #6 \\
2.& #3 & #7 \\
1.& #2 & #8
\end{tabular}
}}

\newcommand{\sekcja}[1]{\newpage\section{#1}}
\newcommand{\subsekcja}[1]{\newpage\subsection{#1}}

%\newcommand{\NOWY}[1]{\par\vspace*{1ex}\noindent\dotfill {\sc #1}\dotfill\par}
    % just to mark ako. section
\newcommand{\NOWY}[1]{\section{#1}}


%%%%%%%%%%%%%%%%%%%%%%%%%%%%%%%%%%%%%%%%%%%%%%%%
%%%%%%%%%% REFERENTIALS

\newcommand{\plan}[1]{\addcontentsline{toc}{subsubsection}{$\bullet$ #1}}
\newcommand{\planA}[1]{\addcontentsline{toc}{subsubsection}{\ \ \  -- #1}}
\newcommand{\planSub}[1]{\addcontentsline{toc}{subsection}{#1}}

\newcommand{\para}{\footnotesize{\S}}
\newcommand{\paras}{\para\para}
% \ref par
\newcommand{\refp}[1]{{\footnotesize{\S\ref{#1}}}}
%\ref par & p.
\newcommand{\refps}[1]{\refp{#1},~p.~\pageref{#1}}
%\ref par & ff.
\newcommand{\refpf}[1]{{\footnotesize{\S\S\ref{#1}.ff}}}
 %{{\footnotesize{[\ref{#1}]}}}
%\ref par & p.
\newcommand{\refpp}[1]{{\footnotesize{\S\ref{#1},~p.~\pageref{#1}}}}
\newcommand{\refpto}[2]{{\footnotesize{\S\S\ref{#1}.-\ref{#2}}}}
 %{{\footnotesize{[\ref{#1}]}},~p.~\pageref{#1}}
%\ref par & ff. & p.
\newcommand{\refppf}[1]{{\footnotesize{\S\S\ref{#1}.ff,~p.~\pageref{#1}}}}
% \ref to Section on page...
\newcommand{\refsp}[1]{\ref{#1},~p.~\pageref{#1}}
\newcommand{\refspf}[1]{\ref{#1},~p.~\pageref{#1}.ff}

%\newcommand{\re}[1]{{\footnotesize{[\ref{#1}]}}}
\newcommand{\re}[1]{\ref{#1}}


%%%%%%%%%%%%%%%%%%%%%%%%%%%%%%%%%%%%%%%%%%%%%%%%
%%%%%%%%%%%% CITATIONS:

%%%% old without \bibliography
\newcommand{\cit}[2]{\footnote{``#1'' [#2]}\ftntcit{#1}{#2}} %citation under the text
\newcommand{\citt}[2]{``#1''\footnote{#2}\ftntcit{#1}{#2}} % citation in the text
\newcommand{\citf}[2]{``#1'' [#2]} % citation in the footnote

%%% old specials:
\newcommand{\citp}[2]{\vspace*{.51ex}\par\noindent 
$\bullet$\ ``#1''\hspace*{\fill{[#2]}} } %cit-paragraph - aux at the end
\newcommand{\citu}[2]{\vspace*{.51ex}\par\noindent 
$\bullet$\ {\footnotesize{USED: ``#1''\hspace*{\fill{[#2]}}}} } %cit-paragraph - aux at the end

%%% NEW \bibliography, with \cite's:::::
%% just \cite in the footnote with no actual text to quote
\newcommand{\incitt}[2]{\ftnt{\citeauthor*{#1}}}
\newcommand{\incittib}[2]{\ftnt{Ibid.}}
%%\newcommand{\cit}[3]{\footnote{``#1'', \citeauthor*{#2}, #3}\ftntcit{#1}{#2}} %citation under the text
\renewcommand{\citet}[3]{``#1''\footnote{\citeauthor*{#2}. #3}}%\ftntcit{#1}{#2}}
                                %% citation in the text - only with
                                %% natbib.sty!!!
\newcommand{\citetin}[3]{``#1''\footnote{In \citeauthor*{#2}. #3}}%\ftntcit{#1}{#2}} % citation in the text - only with natbib.sty!!!
\newcommand{\citef}[3]{``#1'' [\citeauthor*{#2}. #3]} % citation in the footnote
%for reference to the non-original source from which quote is taken
\newcommand{\citaft}[2]{[after \citeauthor*{#1}.{#2}]}

%%%% keeping the argument hidden and saying only 'Ibid.'
\newcommand{\citetib}[3]{``#1''\footnote{{Ibid.} #3}} 
\newcommand{\citefib}[3]{``#1'' [{Ibid.} #3]} 

%%%% -- cite only the title of the work (and the chap./p.) : needs some fixing
\newcommand{\citit}[3]{``#1''\footnote{{\em \citeyear{#2}} #3.}}
\newcommand{\cititf}[3]{``#1'' [{\em \citeyear{#2}} #3]}



%%%% variants of the above, where the reference is put manually (Ibid., op.cit.)
%%%% -- not to be used
\newcommand{\citeti}[3]{``#1''\footnote{{#2} #3}} %\ftntcit{#1}{#2}} 
\newcommand{\citefi}[3]{``#1'' [{#2} #3]} % citation in the footnote

% extra command (to be fixed) to cite only the title of the work:
\newcommand{\citetit}[1]{\citeauthor*{#1}}

\newcommand{\btit}[1]{{\em #1\/}} % book title - to be used everywhere

%%%%%%%%%%%% and special FOOTNOTES
\newcommand{\ftn}[1]{\footnote{#1}}
% adding the current footnote to the list
\newcommand{\ftnt}[1]
 {\footnote{#1}}
% ...#1}+ \addtocontents{lot}{\small{{\bf ftn.\thefootnote/}p.\thepage:  #1\\[1ex] }} }
% used with \cit(t) for adding to the list
\newcommand{\ftntcit}[2]
 {\addtocontents{lot}{\small{{\bf cit.\thefootnote/}p.\thepage: ``#1'' [#2]\\[1ex] }} }
 

%%%%%%%%%%%%%%%%%%%%%%%%%%%%%%%%%%%%%%%%%%%%%%%%
%%%%%%%%% math-symbols
\newcommand{\simu}{~$\sim$~}
\newcommand{\impl}{~$\Rightarrow$~}
\newcommand{\isimp}{~$\Leftarrow$~}
\newcommand{\isnt}{~$\not=$~}

\newcommand{\Nat}{{\bf N}}

\newcommand{\mean}[1]{m(\mbox{#1})} %for ``formal'' meaning
\newcommand{\meant}[2]{m(\mbox{#1}) = \mbox{#2}} %for ``formal'' meaning = 2

\renewcommand{\;}{$\!\!{\scriptstyle ,}\,$}   % polish ou

%%%%%%%%%% points, equations

\newcommand{\equu}[2]{\begin{itemize}\item[#1] #2\end{itemize}}

\newcommand{\equl}[1]{\refstepcounter{equation}\begin{itemize}\item[(\theequation)]
    #1\end{itemize}} 

\newcommand{\equ}[1]{\begin{equation}\mbox{#1}\end{equation}}

\newcommand{\refeq}[1]{(\ref{#1})}

%%%%%%%%%%%%%%%%%%%%%%%%%%%%%%%%%%%%%%%%%%%
%%%%%%%%%% verse, poems
\newcommand{\lin}{/$\!$/} %linebreak in a poem, marked in-line


%%%%%%%%%%%%%%%%%%%%%%%%%%%%%%%%%%%%%%%%%%%%%%%%
%%%%%%%%%%%% XY-pic
\newcommand{\ci}[1]{\xy +\cir<#1mm>{}\endxy}

%%%%%%%%%%%% and drawings: for Wittgenstein in I:\subsubnonr{Signs and meaning}
% \newcommand{\cube}[1]{
% \setlength{\unitlength}{#1}  
%   \begin{picture}(1.6,1.2)(0,0.4)
% \put(0.5,1.5){\line(1,0){1}}  
% \put(0,1){\line(1,0){1}}
% \put(0,0){\line(1,0){1}}
% \put(0,0){\line(0,1){1}}
% \put(0,1){\line(1,1){0.5}}
% \put(1,1){\line(1,1){0.5}}
% \put(1,0){\line(0,1){1}}
% \put(1,0){\line(1,1){0.5}}
% \put(1.5,0.5){\line(0,1){1}}
% \end{picture}
% }
\newcommand{\beth}{\setlength{\unitlength}{0.1cm}
\begin{picture}(2,2)(0,0)
  \put(0,0){\line(1,0){3}}
  \put(0,2){\line(1,0){2}}
  \put(2,0){\line(0,1){2}}
\end{picture}  
}
\newcommand{\bethd}{\setlength{\unitlength}{0.1cm}
\begin{picture}(2,2)(0,0)
  \put(0,0){\line(1,0){3}}
  \put(0,2){\line(1,0){2}}
  \put(2,0){\line(0,1){2}}
  \put(.6,.8){.}
\end{picture}  
}
\newcommand{\be}{$\dot{b}$}

\newcommand{\cube}[1]{
\setlength{\unitlength}{#1}  
  \begin{picture}(0.9,0.5)(0,0.4)
\put(0.5,0.75){\line(1,0){0.5}}  
\put(0,0.5){\line(1,0){0.5}}
\put(0,0){\line(1,0){0.5}}

\put(0,0){\line(0,1){0.5}}
\put(0,0.5){\line(2,1){0.5}}
\put(0.5,0.5){\line(2,1){0.5}}

\put(0.5,0){\line(0,1){0.5}}
\put(0.5,0){\line(2,1){0.5}}

\put(1,0.25){\line(0,1){0.5}}
\end{picture}
}

%%%%%%%%%%%%%%%%%%%%%%%%%%%%%%%%%%%%%%%%%%%%%%%%
%%%%%%%%%%%% tricks for BibTeX
\newcommand{\NOOP}[1]{#1}
\newcommand{\NIX}{}

\newcommand{\Bible}[1]{#1}

\newcommand{\Ashtavakra}{}
\newcommand{\Bhagavad}{}
\newcommand{\Gospel}{}

\newcommand{\Anselm}{}
\newcommand{\Aquinas}{}
\newcommand{\Alphonsus}{}
\newcommand{\Arabi}{}
\newcommand{\August}{}
\newcommand{\Basil}{}
\newcommand{\Bernard}{}
\newcommand{\Bovelles}{}
\newcommand{\Bona}{}
\newcommand{\Clarembald}{}
\newcommand{\Clement}{}
\newcommand{\Cusanus}{}
\newcommand{\Dionysius}{}
\newcommand{\Eckhart}{}
\newcommand{\Eriugena}{}
\newcommand{\Francis}{}
\newcommand{\Fridugi}{}
\newcommand{\Gregory}{}
\newcommand{\Hrabanus}{}
\newcommand{\Isidore}{}
\newcommand{\John}{}
\newcommand{\Justin}{}
\newcommand{\NayaSutra}{}
\newcommand{\Philo}{}
\newcommand{\Trent}{}
\newcommand{\Theologia}{}
\newcommand{\ZZZ}{}
%\newcommand{\N.}{}

\newcommand{\Samkara}{\'{S}a\.{m}kara}
%\def\N.{}

%%%%%%%%%%%% not used now
\newcommand{\indexentry}[2]{\item[{}] {#1}\dotfill....{#2}}
\newcommand{\definitions}{\newpage\section*{Definitions}
\begin{itemize} \input{ggbIDX} \end{itemize} }

