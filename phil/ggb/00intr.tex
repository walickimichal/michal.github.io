\section*{Preface}\addcontentsline{toc}{section}{Preface}

More results, new views, novel perspectives, discoveries, insights, achievements
-- all promise to bring us closer to some ultimate state which they neither
specify nor, for the most, even mention. This omission is, however, not only
understandable but even amiable because how could anybody expect from a decently
modest person to aim at any higher tranquility than the simple satisfaction from
reaching the well-defined and plain, visible goals? We leave such higher
ambitions to others who happen to suffer from the immodesty of spirit, to
hysteric preachers if not dangerous demagogues.  But so, all the tense activity
leaves us where we started, with the same thirst which keeps driving us to
search for more results, new views, novel perspectives\ldots And leaves us,
perhaps, exhausted but not calmed.

To say something new, to inspire more
search, research and activity, to add one's co-operative push in the direction
which nobody sees or understands but everybody accepts, simply because things seem to be
drifting towards it, are ambitions which manage to preserve an appearance of
modesty only by means of their commonality and the proclaimed contributions to the
imagined goal towards which everything is, anyway, drifting.
% The intention is not to say anything new, least of all, to inspire any 
% search or activity but, on the contrary, to calm down.
Agitated construction of new perspectives and search for new revelations envelope the old
ones in the mist of irrelevance -- not only when they, indeed, become outdated
but usually {\em only because} they became old.
%
But agitation can at most exhaust, never calm down, and is only an apparent
medicine which soon requires an increased dose. 
% The intention is not to offer\noo{ -- in the lack of modesty covered by the
%   pretensions to its surplus --} any new perspective but, on the contrary, to
% recall an
The more spectacular breakthroughs, the less they seem to affect us. The more
novel discoveries, the more monotone their novelty and the more nagging the
suspicion of the presence of some 
old element, as archaic as, perhaps, the
thirst which once drove us from the caves to the self-constructed tents and
houses and which today drives us into the cosmic space or, as the case may be,
into the embraces of hysteric preachers if not of dangerous demagogues.
Tranquility, too, can require work -- not, however, intense work which, aiming at
specific goals, pacifies only by exhausting, but work which is not carried out
exclusively for the sake of the expected results but, rather, like a patient
polishing of optic lenses, as an expression
of the firm conviction that, particular results notwithstanding, the genuine
goals are not under our control and that doing the only things we can do, that
is, working towards visible results, we can at most help with, if not merely hope
for, the realisation of the invisible ones which are the ones that ultimately matter.

Agitating novelties are but idols of boredom -- our motto is \la{nihil novi}. If
any novelties are to be found here, they concern at most concepts, not the
object of discourse.  {\la{Mundus senescit}} -- \wo{The world has aged} --
noticed St.~Gregory of Tours in the VI-th century and some thousand years before
him a preacher observed: \citeti{The thing that hath been, it is that which
  shall be; and that which is done is that which shall be done: and there is no
  new thing under the sun.}{Eccl.}{I:9} When things appear new, it is either
because they are trifles -- perhaps, useful or powerful ones but, in existential terms, only
trifles -- or, if they are of any deeper significance, so simply because one
does not recognise this deeper value, which is the same as the lasting presence,
underneath the novel forms in which it must appear.  Now, repeating the old may
look like a waste of time and paper.  But this looks so only if we assume that
the old can be literally dissociated from the new, that the thing that hath been
is \thi{in itself} and appears in new forms only as if by accident.  These forms
are, however, essential for the presence of the deeper things which are not
unchangeable like dead pyramids, but 
\citeti{they become different things at different times, and are ever
  continuously the same.}{Empedocles}{DK 31B17} To understand something said
before one must say 
it oneself: first, only repeat it, preserving the old, until the time comes when
one can say it anew, giving thus to the old a new life. Every existence says
anew something said before, but saying this concretely amounts to drawing anew
the distinctions in the matter of life, the distinctions between the inherited
abstractions like \thi{love} and \thi{indifference}, \thi{indifference} and
\thi{impotence}, \thi{impotence} and \thi{thirst}, \thi{thirst} and \thi{lack},
\thi{lack} and \thi{illusion}, \thi{illusion} and \thi{lie}, \thi{lie} and
\thi{mistake}\ldots Infinite concreteness lies in such distinctions and the lack
of any final definitions is only another side of this concreteness.
%
% which, taking time
% which calms down but does not exhaust and time for exhaustion 
%

We will not try to offer any new perspectives but neither will we attempt to
draw all too precise distinctions. Attempting to capture and to define precisely
distinctions like those just mentioned will, more often than not, deprive them
of their concrete value. For this concreteness lies in the potential to appear
in so many and always new actual forms -- not in any imagined fact that their
appearances could be collected in an exhaustive list amenable to a precise case
analysis. Avoiding some distinctions one also avoids some questions. But it is
doubtful that every possible question is worth addressing just because one can
draw distinctions making it possible.  One can ask, for instance, if Empedocles
meant that the world, reappearing at every turn of the Cosmic Cycle, was really
new or only a repetition of the old and the same one. One can ask if even at
each single turn of the cycle, the world emerging through the dissolution of the
\gre{Sphairos}, the ultimate unity, under the influence of hatred (or was it
strife? or merely antipathy?), was distinct from or, perhaps, the same as the
one emerging after the power of love (or sympathy) begun to gather the
dissociated elements reestablishing the dissolved unity. One can even question
if he at all meant any cosmic cycle and not merely the constant flood and ebb
tides of vital energies which mark our lives with varied degrees of closeness
and intimacy or remoteness and frigidity.  (For although \wo{thrice ten thousand
  seasons} seem hardly compatible with the human existence, so the mood of the
lost heaven \citeti{from which I am too now a fugitive and wanderer, because I
  trusted in raging Strife,}{Empedocles}{DK 31B115} is easily discernible, and
the fact that men \citeti{at last become seers, and bards, and physicians, and
  princes [...] from which [state] they blossom forth as gods highest in
  honour}{Empedocles}{DK 31B146} sounds like a promise worth striving for on the
scale of one's own life and not of a cosmic cycle.)\noo{(For \citeti{I say the
    following about the Whole\ldots Man is that which we all
    know}{Democritus}{DK 68B165} and, perhaps, he is all that we know. In any
  case, \citet{in the philosopher [\ldots] there is nothing
    impersonal.}{BeyondGE}{I:6})} One can quote the old fragments against their
new appearances, or else, quote the new appearances as a confirmation of the old
ones.  But in the lack of sufficient sources and ultimately precise
distinctions, one is really left with the only possibility of rediscovering the
old formulations in one's own thought, or rather, one's own life. Even if an
interpretation in terms of a cosmic cycle might appear possible, it is only an
interpretation. The double process which \citeti{at one time increased so as to
  be a single One out of Many [and] at another time again grew apart so as to be
  Many out of One,}{Empedocles}{DK 31B17\label{Emp}} may provoke discordant
interpretations. But it is far from certain that any such interpretation is
better than the vague, and yet definite, nexus of the original formulation which
not only provoked them all but, perhaps, even {\em intended} them all.

Original thinking is able to see differences as manifestations of the same.  (We
will always use \wo{original} in the sense of originary, referring to the
archaic foundation, and not of a novel, even if genuine, invention, as is
suggested by the common usage along with the assumption that one can speak of
multiple origins.) Confronted with such thinking, one can attempt to determine
one of the involved aspects as \thi{the primordial one}, from which all others
arise as its reflections or projections. But such attempts, more often than not,
amount to a reduction.  Detailed treatment might require consideration of all
possibilities but, when exegesis is not the aim, such considerations would
confuse rather than clarify.  Although one should pay due respect to the
scholarly scrutiny,
%Although when writing (and this always means: simplifying and even abstracting),
%we can hardly do better than follow carefully the steps of scholarly scrutiny,
we should also watch for the point at which scrutiny becomes pedantry. Those who
try \citet{withholding their consent from any proposition that has not been
  proved}{CiceroGods}{I:1.\noo{[In the translation of F.~Brooks: \wo{refusing to
      make positive assertions upon uncertain data}.]}}  end up, perhaps, with
absolute certainties -- but only certainties about nothing. The rigidity of an
irrefutable argumentation which tries to arrive at indubitable precision and to
{\em force} its univocal judgment, seems rather to empty every phenomenon of its
concreteness and to yield only a residual site whose necessity, and that often
means applicability and usefulness, equals its existential hollowness.

% Having strong suggestions of the unity of vision \citet{we must be very careful
%   lest, wishing to perceive more, we do not stray away}{Plotinus}{V:8, 11,
%   23-24.} and believe to see more than what, being visible, remains only and
% nothing but visible.

% Often, saying more says too much and the distinctions asked
% for and possible to posit, simply confuse the issue if not also falsify it.
\noo{
We will leave many issues unresolved, if by resolving an issue one means
providing any decisive argument, bringing it down to a mechanic interaction of
precisely dissociated and arranged elements. Not because such a resolution 
would be impossible but only because it would be possible in several ways and,
usually, none of these ways is better than others. Each represents not a
clarification but a reduction of the original nexus, that is, each covers the
old by the new, sacrifices the clear to the precise.

This may, of course, indicate dilettantish simplifications. But is not
simplicity also the ultimate goal of scholarly thoroughness: to simplify the
complicated, to arrive at the unity which -- perhaps in a different sense but
still -- removes contradictions discovered and produced by the legions afraid of
being called simple-minded?  Every simplicity can be accused of simplification,
while many simplifications can be misunderstood as simplicity.  Where is the
line separating the two? Isn't it, at least to some extent, in the eye of the
beholder, in the way he draws the distinctions?  
}

%\section*{Summary}
%1
\pan The usefulness of rigidity and precision -- of arguments, proofs or
thinking in general -- is hardly disputable. It underlies the basic form of all
our activity, namely, construction -- not only of houses and cars, but also of
plans, arguments, theories. Construction amounts to collecting various,
preferably rigidly defined, bits and pieces, various fundamental building
blocks, and arranging them according to some rules in ways made possible by the
properties of these blocks and serving our purposes. This, however, obviously
presupposes availability of all such bits and pieces from which construction may
start. And here one encounters the main difficulty. What, eventually, are these
pieces, how can we characterise their properties in a sufficiently precise
manner? The ontological arguments concerned with such questions do not seem to
have been settled.  The existential hollowness of all too rigid argumentation
reflects simply its dependence on the availability and character of the rigid
pieces about which, however, one can always keep arguing.

We will not contribute to such discussions because we are primarily interested
%\marginpar{{\footnotesize{Book~I}}}
not in activities but in events, not in things we (can) do but in things which
happen. What is first for a construction are things, building blocks. But the
triviality that there are things harbours only an even greater one, namely, that
there are {\em different} things, that things are only to the extent they are
distinct from each other.  The primordial event, as we will describe it, is
distinction.  Distinctions lead to recognition of elements which, in various
contexts and for various purposes, can serve as the starting building blocks. But
the fact that they arise as a result of distinguishing which always, at least in
principle, might be continued, is probably the reason for the failure of all the
attempts to identify the ultimate atoms. Instead of looking for them, we will
rather describe the process of gradual emergence of solidified objects which,
having acquired identity and permanence, can serve as starting points of various
constructions. Naturally, what kind of staring points one is willing to accept,
depends on what kind of constructions one wishes or is able to perform. In the
most general way, we would say that the distinctions and their limits are
relative to the distinguishing being.

To avoid simple-minded accusations of subjectivisms and voluntarism, it is
important to emphasize that the distinctions are not relative merely to the
subjective will or consciousness but also, and most fundamentally, to the kind
of existence. (In the most trivial sense, distinctions affecting existence of
ants are often, though certainly not always, quite different from those
affecting human existence.) But we will go further, that is, we will begin
earlier than that. Distinction refers to something that is distinguished.
However, as distinctions are events which precede objects, they cannot\noo{, on
  the risk of ending in a vicious circle,} be accounted for in terms of any
given objects. The \thi{something} which is being distinguished is not any
object, is not something. Now, it should not be too daring to observe that
\citet{the word <<{\cco{nothing}}>> in no way differs in meaning from the
  expression <<{not something}>> [which] indicates that every thing, whatever
  expresses any reality, should be excluded from the mind\ldots}{AnselmFall}{XI}
We will therefore allow ourselves to use this word. Likewise, preceding all
distinctions, it can be called \wo{\cco{indistinct}}. But since these words can
also create wrong associations with the total void and emptiness, while they
point to \cco{that} which is distinguished and thus provides the ultimate origin
and material for all distinctions, we will also call it \wo{\cco{origin}} and,
as two \thi{indistinct} seems a plain contradiction, \wo{\cco{one}}.  Although
the words are hardly synonymous in the common language usage, we will
consistently use them for denoting the same.  The first Book will thus follow
the process of distinguishing the \cco{indistinct}, in which we will encounter
the basic aspects of every experience (like awareness and self-awareness,
actuality of consciousness, time, space and, first of all, the ineradicable
sense of \thi{objectivity}), arriving at the most \cco{precisely dissociated
  actual objects} of \cco{reflection}. It will be emphasized that this is not
any pantheism (or, what amounts to the same, any view that the fundamental unity
is a mere totality of distinct pieces), for the \cco{indistinct} remains
indistinct no matter how many distinctions have been made out of it. It remains
one and unmoved underneath all the variety of experience, as the eternal and
unchangeable origin -- discernible phenomenologically as the horizon -- of all
distinctions.

%2
In the second Book, we will turn this process up side down and sketch the
%\marginpar{{\footnotesize{\hspace*{-1em}Book~II}}}
general schema of \cco{reflective} construction.  Here, we will encounter
various elements moving traditionally within, or in the vicinity of,
epistemology (like sensations, concepts, particulars and universals, general
ideas, ego, self, forms of transcendence).  The general direction of this
movement is determined by the fact that, at each stage, \cco{reflection} is
surrounded by the horizon of \cco{transcendence} which, although given in
experience, escapes the possibilities of grasping it within the \cco{actuality}
of any single \cco{act}. We will observe possible \cco{actual} appearances of
higher, \cco{non-actual} aspects of experience which, dissolving gradually the
\cco{precision} of the immediate givens in the increasingly \cco{vague} and
general notions, lead (back) to the recognition of the \cco{presence} of the
ultimately \cco{transcendent}, because never distinguished, \cco{nothingness}.
In the concluding digression, we present a view of mathematics as the science of
pure distinctions, claiming that this fundamental aspect of every experience
(pure distinction being the mere fact of and awareness of distinguishing) lies
also at the bottom of the mathematical intuition, if not necessarily of all
advanced mathematical constructions.

%3
The third Book takes up the existential thread and begins with the recognition
%\marginpar{{\footnotesize{\hspace*{-1em}Book~III}}}
of \cco{thirst} as the fundamental mode, if not the content, of \cco{actual
  existence}; the \cco{thirst} which is just the experience of the
\cco{distance} between the actually given and the ultimately invisible, the
tension between the stimulating variety of the \cco{actual} world and its ever
\cco{present origin} which, threatening with empty boredom, attracts also with
the unforgettable promise of tranquility. The \cco{spiritual choice} is made
between these two faces of the \cco{one}. One alternative tends, even if not
necessarily leads, towards evil, which is characterised as the broken continuity
of being, \cco{alienation} from the \cco{origin}.\noo{(and hence, at first, only
  affects one, before it possibly begins to act).} The other opens up to the
\cco{gift} of the \cco{origin} establishing, if not any mystical union or direct
contact with it, so a \cco{concrete participation}.  Although the choice is not
made by the active \cco{reflection}, it is helped by it and then reflects back
on its \cco{actuality}. We will give several examples of such a \cco{concrete
  founding} in which \cco{actual} consciousness not only receives the general
structure of its objects and the world (as happens in the ontological
\cco{foundation} from the first Book), but where this world and its
objects obtain also specific character of \cco{concrete} continuity -- with
their \cco{origin}, and hence also with each other. Thus, not only objectivity
of the visible world but also its unity and concreteness, arise not from any
particulars found within it but from the ultimate objectivity which,
transcending it, remains forever as constant and unchangeable as it is
invisible.

\pan This brief summary should also suggest the answer to the reservations which
might, possibly, arise with respect to the subtitle.  What could possibly make
one write a \la{summa}, of any kind, nowadays?  Complexity -- not only of the
incomprehensible totality of the world, but even of every single issue -- makes
it look pretentious. Things fall apart and observing the dissolution or, as many
a one attempts, praising it as the openness unto plurality, is the only
reasonable attitude and, in some circles, the only politically correct way of
distancing oneself from the trauma and dangers of monism, that is,
totalitarianism.  Yet, philosophy which does not try to capture any unity ends
up as a catalogue of particular cases, particular concepts or just words, which
may be quite elaborate and intelligent, which even can address some particular
issues of existential relevance but which is unable to appeal to the {\em whole}
person.
% Philosophy which distances itself from any attempt at reaching some
% wisdom, and that means in particular, thinking and relating to the ultimate
% questions, ends up calling its impotency for modesty and glorifying the
% incessant questioning, that is, public scratching one's head with the emanating
% self-assurance that all genuine issues should be left to those who are
% unintelligent enough to expect any answers. After all, it takes an illiterate to
% believe that The Truth is written somewhere.

Only that which concerns the whole person can appeal to the whole person. A
particular issue can do it in a specific way but, to the extent it is a genuine
problem of human existence, it involves necessarily all others.  One meets
typically attempts to address a specific \thi{philosophical problem}, to
elaborate in the scope of a single paper the problem of free will, the problem
of meaning, the problem of truth.  Interesting\noo{(and, perhaps, even to some
  extent legitimate)} as such attempts may be, they reveal the prejudice that
such a division is not only possible, but also meaningful, that distinct
problems indeed can be treated in a relative independence; eventually, that only
the closest scrutiny of the most minute distinctions is able to give an adequate
description of any single issue. But the problem is that we not only do not
quite know how one problem could or should be addressed and approached -- we do
not even know precisely {\em what} any particular problem is. Attempting to
isolate any philosophical problem for a separate treatment, one fights
constantly with circumscribing it in any reasonable way which circumscription,
however, {\em never} becomes entirely satisfactory and final. In this process
the problem gets invested with all the relations and implications it carries to
all other problems. Trying to address, say, the issue of freedom without
at the same time illuminating the meaning of subjectivity and openness to truth,
the sense of meaningfulness, the presence of the absolute or its rejection, in
short, without addressing the integrity of the whole existence, one ends up with
the distinctions one started with and keeps opposing arbitrariness to
determinism, subjectivity to objectivity, spontaneity of feelings to the
rationality of some inviolable laws, etc.  Every issue is the sum of what it
excludes, is the border contracting the tension between this issue and others
into which it is interwoven. In an apparently strange (but, in fact, quite
understandable) dialectics, the tradition which had marked the XX-th century
with the missionary zeal of dissociating all the issues and bringing them under
objective, systematic and separate analysis, ends up with the holistic and
coherentist postulates, whether with respect to language, meaning or
truth.\noo{(Wittgenstein, Quine, Davidson)} Even though one can not forget the
idealistic origins of all such postulates, one would still like to deny their
idealistic connections. And one almost manages that, at least, as long as one
keeps dissociating, as long as one sticks to dissecting one particular issue at
a time.
% This, after all, can be handled by mere intelligence, with only minimum
% of integrity. 

We will not try to establish any totality which, in the presence of all too many
accepted distinctions, would indeed be a vanity. Paying attention to the
similarities rather than differences, we will often ignore possible distinctions
-- not because they would necessarily destroy any unity we might wish to find
but because they would (tend to) completely obscure it.\noo{We hope to avoid the
  accusations of not distinguish the distinct and relating the unrelated by
  making at least plausible that the intimate kinship of vague yet distinct
  aspects, their genuine unity precedes more rigid dissociations, and that the
  latter mark only the end -- or perhaps the middle, but certainly not the
  beginning -- of the road.} Philosophical anthropology addresses primarily the
unity of human existence. Placing it in the center, one is led to view all the
issues, also those present in any original thinking, from its perspective. (Thus
we would, indeed, view the fragments of Empedocles referred to on
page~\pageref{Emp}, as reflecting primarily, if not only, some fundamental
aspects of human existence.) This unity does not amount to any visible coherence
or demonstrable consistency of a totality of single issues. In a way which we
hope to make understandable, if not necessarily convincing to everybody, the
tradition which seems to have most to offer in this respect turns out to be
neo-Platonism, in its whole span from Plotinus and Pseudo-Dionysius, through
Eriugena, to Meister Eckhart and Cusanus. We do not, however, intend any new
interpretation of this tradition -- we only acknowledge its inspiration. We will
quote rather extensively and from rather varied traditions, and hence we will
not attempt any exegesis of the texts or analysis of the thoughts of others.
Any such attempt would make finishing this work impossible.  On few occasions,
more detailed statements illustrating differences should also serve as a
clarification of the presented views.  The variety of sources makes it natural
to limit the quotations to the most succinct statements which, hopefully,
express some essential idea.  Although the rules of conscientious exegesis may
be thus violated, and some quotations might have thus been not only drawn out of
their context but even adjusted to fit the present one, the intention is never
to violate the meaning of the quoted text.\noo{Besides, exegesis is not our
  objective.)}
% Asked by a true Irishman \wo{Are you a drinking man or are you a fighting man?}
% I could whole-heartedly confirm the former.  I like company, and

Variety of traditions is also a reason for focusing on affinities, and often
even only vague similarities, rather than differences and oppositions.  Was
St.~Augustine entitled to claim the presence of Christian truths in the
neo-Platonic texts, as he did in the much disputed and controverted passage in
\btit{Confessions} VII:9? Was St.~Clement of Alexandria right in the similar
claims about the affinity of the Greek philosophy and literature with the
Christian revelation? Was Philo Judaeus right claiming not only similarities
between but even the direct dependence of the Greek thought on the Biblical
tradition?  One may attempt proving that they were all wrong pointing out
significant differences and the lack of any verified continuity. The Greek
spirit was, after all, completely different from the Christian one.  Perhaps,
but this depends on how one draws the borders around the intuitions like
\thi{the Greek spirit} and \thi{the Christian spirit}.\noo{(Let us also notice
  that such abstracts, useful as they sometimes may be in philosophy, are
  primarily only of historical and sociological character.)}  One can always
find differences separating two views -- the question is at what level, and
then, what value one will attach to them as opposed to the similarities. After
all, the neo-Platonic culmination of the Greek spirit, with its severe critique 
of the emerging Christianity, provided the foundation for the depth of the
philosophical expression of Christian mysticism.  Opposing, say, the Greek
spirit to the Christian spirit, one should never forget that in both cases one
is speaking about spirit which, incarnated in different socio-historical and
political constellations, remains human spirit.  One can always
keep distinguishing, especially, if the insecurity of wisdom is attempted
compensated by the abundant precision of intelligence.  But the problem of
perspicacious thoroughness, as La Rochefoucauld observed, is not that it does
not reach the end but that it goes beyond it.
%only blathering and babbling has no limit while it 
We will for the most focus on the similarities and it is up to the reader 
to decide whether they are only due to the negligence in observing the important
distinctions or, perhaps, they are justified because the possible distinctions
are of negligible importance.

\subnonr{Sources and references} Two bibliographies are given at the end of the
text. The first one contains all the referenced works sorted by the name of the
(first) author. The second one is divided in some rough subgroups collecting,
respectively, the primary and secondary philosophical sources, the religious and
theological works, works on mathematics and computer science, psychology and
infant development, history and anthropology and, finally, literary and poetic
texts.

A few sources deserve a brief comment.  The authorship of {\em My Sister and I}
is the matter of dispute and scholars can not tell for sure (perhaps, rather
seriously doubt) that it is indeed, as is also claimed, autobiography written by
Nietzsche himself. The authorship of thoughts should not be that important.
However, in the academic context the issue can become a bit sensitive,
especially when the claimed author is Nietzsche.  (It might be so, in
particular, if one wanted to relate the contents of this autobiography to his
other works which, however, we are not doing.)

Even if it were not Nietzsche, it certainly could be, though
%As somebody said, it is \wo{how you imagined Nietzsche would sound if 
%you got him drunk}. 
the author might also have been more Nietzschean than Nietzsche himself. Facing
the lack of any decisive proofs or disproofs of purely textual, linguistic or
medical nature, we are left with the text which looks like it might have been
written, if not carefully re-read and edited, by Nietzsche.  The voice for or
against his authorship depends then on one's view of his thought -- whether one
rediscovers the message of the text in one's prior understanding, whether this
text \thi{fits} into the image one has of his whole thinking and, not least,
personality.  For me, there is a perfect match with the image I had formed
before reading this book. (Possible objections against the portrait arising from
it, should be confronted with less extreme, yet by no means incompatible,
impressions of the close friend in \citeauthor*{LouN}.) \citet{In the end,
  \btit{My Sister and I} reminds me of a true story.}{Sirens}{} Having made this
reservation, we will quote the text as if Nietzsche was its author.

Another referenced text, hopefully of much less dubious value, is a collection
of early Freiburg lectures by \citeauthor*{PhenomReligio} [\btit{Phenomenologie
  des religi\"{o}sen Lebens}, Gesamtausgabe, vol.~60]. Some of these have been
reconstructed almost exclusively from the notes of the students. Thus the reader
should be warned that the quoted formulations, although reflecting hopefully the
intentions, are hardly Heidegger's. (In any case, they are translated by me into
English, and that mostly from the Polish translation of the German text. Well...)

Likewise, \citeauthor*{Celsus}, is only reconstructed from the extensive
fragments quoted and criticized in \citeauthor*{aCelsus}. In this case, however,
the breadth and details of Origen's response give reasonable confidence into the
authenticity of the reconstruction. Much worse is the case of
\citeauthor*{Porphyry}, where even the attribution of authorship may be disputed
as the work is reconstructed mainly from the \btit{Apocriticus} of Macarius
Magnes which need not reflect the philosophy of Porphyry. These works are quoted
as if they were written by the authors to whom they are attributed by the
general (though not universal) opinion. For an investigation of the
associated doubts and controversies the reader may start by consulting the
referenced editions.

Two distinct editions of \citeauthor*{Periphyseon} have been used. The critical
edition (started by late I.~P.~Sheldon-Williams and continued by
\'{E}.~A.~Jeauneau) of volumes I, II and IV is referenced as just done, with the
number+letter identifying the page number and the manuscript as in the edition.
Volumes III and V are from the abbreviated translation by M.~L.~Uhlfelder and
are referenced in the same way, \citeauthor*{Periphy}, with only page numbers
in this single volume edition. In either case, the volume number identifies
uniquely the referenced edition. 


\sep
%
One encounters sometimes cases when, in an English text, quotations and longer
passages are given in French, German or some other language of the original --
sometimes even Latin or Greek. Unless this happens in a work addressed
exclusively to the scholarly audience, it may create the impression of
intellectual snobbery.  It is a great advantage to know German, French, Italian,
Latin and Greek, but few people do and I am not one of them. Although this work
will, hopefully, have something to offer to the scholars, it is intended also
for a broader audience of patient readers interested in a unified view rather
than single issues.  Since I have used extensively sources in other languages, I
have attempted to access -- and if I did not succeed then to translate -- all
the quotations into English. (A few exceptions concern passages of German poetry
which I did not dare to attempt translating.)  Sometimes, I ended thus
translating back into English texts translated originally from English into
another language.  Such cases are marked in the bibliography as \thi{my {\bf
    re}translation...}. The page references are then to the version, specified
in the bibliography, in the language from which it was retranslated.
%Hopefully, this will not cause any serious confusion.
%-- to fix it, I have to find some time with nothing better to do.

\subsection*{Some conventions}
All the works are referred by the English title, even if the used source is in
another language; this is then indicated in the bibliography. (A few exceptions
are made when the original source is referred after 
another author, as is often the case with collected works or fragments.)

The references to all the works look uniformly as
\begin{center}
  Author, \btit{Title} XI:1.5\ldots
\end{center}  
where the part before `:', typically a Roman numeral, refers to the main part
into which the source is divided (e.g., book, part, chapter), and the numerals
after `:' to the nested subparts.  The references to the Bible have no `Source',
thus `Mt. X:5' refers to \btit{The Gospel of Matthew}, chapter X, verse 5. 
(I have used primarily King James Version and commented occasional usage of
other translations in the footnotes.) 
Also the references to the pre-Socractics are usually given without any source by
merely specifying the author and the Diels-Kranz number, e.g., `Heraclitus, DK
22B45', where the number identifying the philosopher (here 22) is taken from the
fifth edition of Diels, \btit{Fragmente der Vorsokratiker}.

Identifying quotations by page numbers might have been reasonable in times when
most books existed only in one edition.  I have tried to avoid such references
but in a few cases, where the structuring and numbering of the text happens to
be very poor, I had to use this form. This is also sometimes the case with the
quotations borrowed from others which I did not verify (the source is then given
in the square braces ``[after...]'' following the reference).  The pagination
(possibly, the paragraph number or the like) 
follows then at the end of the reference as `Author, \btit{Title}
XI:1.5\ldots;p.21', where the numbers indicating part and subparts usually
involve only the main part (i.e., only `XI;p.21'), and may be totally absent, if
no such division of the work is given.  The edition is identified in the
bibliography.  Occasionally, the subparts may have a subtitle or a letter, as e.g.,
`II:d7.q1.a2'. These are only auxiliary and their meaning depends on the source.
Typically, these are used with the medieval authors and the reference above
might be to the {\bf d}istinction 7, {\bf q}uestion 1, {\bf a}nswer 2, in the
second, II, volume/book.

In few cases I do not know the origin of the quotation, or else I only (believe
to) know its author. I chose to indicate such incomplete pieces of information,
rather than skipping them all together. I have likewise indicated the use of
unauthorized, or in any case unedited, versions of the texts found on the
internet for which no bibliographical information, except for the title and the
author, is given in the bibliography. (For some, certainly very pragmatic reasons,
books printed in the USA do not carry explicitly the year of publication but only
the year of copyright. Consequently, the bibliographical information for such
books refers usually to this date.)

\sep
Words which are given some more specific, technical meaning are 
written with \co{slanted font}. \wo{Quotation marks} are used for 
words and quotations. \thi{Shudder-quotes} indicate, 
typically, either the referent of the word in the quotes, or else a 
concept or expression which is not given a technical meaning in the 
text but which is borrowed from somewhere else or even is only 
assumed to have some technical sense. Thus, for instance:
\begin{itemize}
\item 
\co{subject} -- is the subject in the technical sense introduced in
the text;
\item 
\wo{subject} -- refers to the word itself (quotations are also given
in the quotation marks);
\item 
\thi{subject} -- is subject in some, possibly technical sense of
somebody else; it may often indicate a slight irony over only apparently
precise meaning one might believe the word \wo{subject} to have;
\item 
subject -- this is just subject, with full ambiguity and with whatever meaning
the common usage might associate with it at the moment. 
\end{itemize}
I have tried to place more technical details in the footnotes which therefore
can be, for the most, skipped at first or casual reading. They are not, however,
addressed specifically to the scholars. Sometimes they elaborate the text but,
most often, will be useful for those who find some ideas interesting enough to
follow them in other authors.

