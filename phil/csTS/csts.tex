\documentclass[10pt]{article}
\usepackage{latexsym,amssymb}
\makeatletter

\ifcase \@ptsize
    % mods for 10 pt
    \oddsidemargin  0.15 in     %   Left margin on odd-numbered pages.
    \evensidemargin 0.35 in     %   Left margin on even-numbered pages.
    \marginparwidth 1 in        %   Width of marginal notes.
    \oddsidemargin 0.25 in      %   Note that \oddsidemargin = \evensidemargin
    \evensidemargin 0.25 in
    \marginparwidth 0.75 in
    \textwidth 5.875 in % Width of text line.
\or % mods for 11 pt
    \oddsidemargin 0.1 in      %   Left margin on odd-numbered pages.
    \evensidemargin 0.15 in    %   Left margin on even-numbered pages.
    \marginparwidth 1 in       %   Width of marginal notes.
    \oddsidemargin 0.125 in    %   Note that \oddsidemargin = \evensidemargin
    \evensidemargin 0.125 in
    \marginparwidth 0.75 in
    \textwidth 6.125 in % Width of text line.
\or % mods for 12 pt
    \oddsidemargin -10 pt      %   Left margin on odd-numbered pages.
    \evensidemargin 10 pt      %   Left margin on even-numbered pages.
    \marginparwidth 1 in       %   Width of marginal notes.
    \oddsidemargin 0 in      %   Note that \oddsidemargin = \evensidemargin
    \evensidemargin 0 in
    \marginparwidth 0.75 in
    \textwidth 6.375 true in % Width of text line.
\fi

\voffset -2cm
\textheight 22.5cm

\makeatother

%\makeatletter
%\show\
%\makeatother
\newcommand{\ite}[1]{\item[{\bf #1.}]}
\newcommand{\app}{\mathrel{\scriptscriptstyle{\vdash}}}
\newcommand{\estr}{\varepsilon}
\newcommand{\PSet}[1]{{\cal P}(#1)}
\newcommand{\ch}{\sqcup}
\newcommand{\into}{\to}
\newcommand{\Iff}{\Leftrightarrow}
\renewcommand{\iff}{\leftrightarrow}
\newcommand{\prI}{\vdash_I}
\newcommand{\pr}{\vdash}
\newcommand{\ovr}[1]{\overline{#1}}

\newcommand{\cp}{{\cal O}}

% update function/set
%\newcommand{\upd}[3]{#1\!\Rsh^{#2}_{\!\!#3}} % AMS
\newcommand{\upd}[3]{#1^{\raisebox{.5ex}{\mbox{${\scriptscriptstyle{\leftarrow}}\scriptstyle{#3}$}}}_{{\scriptscriptstyle{\rightarrow}}{#2}}} 
\newcommand{\rem}[2]{\upd{#1}{#2}{\bullet}}
\newcommand{\add}[2]{\upd {#1}{\bullet}{#2}}
%\newcommand{\mv}[3]{{#1}\!\Rsh_{\!\!#3}{#2}}
\newcommand{\mv}[3]{{#1}\:\raisebox{-.5ex}{$\stackrel{\displaystyle\curvearrowright}{\scriptstyle{#3}}$}\:{#2}}

\newcommand{\leads}{\rightsquigarrow} %AMS

\newenvironment{ites}{\vspace*{1ex}\par\noindent 
   \begin{tabular}{r@{\ \ }rcl}}{\vspace*{1ex}\end{tabular}\par\noindent}
\newcommand{\itt}[3]{{\bf #1.} & $#2$ & $\impl$ & $#3$ \\[1ex]}
\newcommand{\itte}[3]{{\bf #1.} & $#2$ & $\impl$ & $#3$ }
\newcommand{\itteq}[3]{\hline {\bf #1} & & & $#2=#3$ }
\newcommand{\itteqc}[3]{\hline {\bf #1} &  &  & $#2=#3$ \\[.5ex]}
\newcommand{\itteqq}[3]{{\bf #1} &  &  & $#2=#3$ }
\newcommand{\itc}[2]{{\bf #1.} & $#2$ &    \\[.5ex]}
\newcommand{\itcs}[3]{{\bf #1.} & $#2$ & $\impl$ & $#3$  \\[.5ex] }
\newcommand{\itco}[3]{   & $#1$ & $#2$  & $#3$ \\[1ex]}
\newcommand{\itcoe}[3]{   & $#1$ & $#2$  & $#3$}
\newcommand{\bit}{\begin{ites}}
\newcommand{\eit}{\end{ites}}
\newcommand{\na}[1]{{\bf #1.}}
\newenvironment{iten}{\begin{tabular}[t]{r@{\ }rcl}}{\end{tabular}}
\newcommand{\ass}[1]{& \multicolumn{3}{l}{\hspace*{-1em}{\small{[{\em Assuming:} #1]}}}}

%%%%%%%%% nested comp's
\newenvironment{itess}{\vspace*{1ex}\par\noindent 
   \begin{tabular}{r@{\ \ }lllcl}}{\vspace*{1ex}\end{tabular}\par\noindent}
\newcommand{\bitn}{\begin{itess}}
\newcommand{\eitn}{\end{itess}}
\newcommand{\comA}[2]{{\bf #1}& $#2$ \\ }
\newcommand{\comB}[3]{{\bf #1}& $#2$ & $#3$\\ }
\newcommand{\com}[3]{{\bf #1}& & & $#2$ & $\impl$ & $#3$\\[.5ex] }

\newcommand{\comS}[5]{{\bf #1} 
   & $#2$ & $#3$ & $#4$ & $\impl$ & $#5$\\[.5ex] }

%%%%%%%%%%%%%%%%
\newtheorem{CLAIM}{Proposition}[section]
\newtheorem{COROLLARY}[CLAIM]{Corollary}
\newtheorem{THEOREM}[CLAIM]{Theorem}
\newtheorem{LEMMA}[CLAIM]{Lemma}
\newcommand{\MyLPar}{\parsep -.2ex plus.2ex minus.2ex\itemsep\parsep
   \vspace{-\topsep}\vspace{.5ex}}
\newcommand{\MyNumEnv}[1]{\trivlist\refstepcounter{CLAIM}\item[\hskip
   \labelsep{\bf #1\ \theCLAIM\ }]\sf\ignorespaces}
\newenvironment{DEFINITION}{\MyNumEnv{Definition}}{\par\addvspace{0.5ex}}
\newenvironment{EXAMPLE}{\MyNumEnv{Example}}{\nopagebreak\finish}
\newenvironment{PROOF}{{\bf Proof.}}{\nopagebreak\finish}
\newcommand{\finish}{\hspace*{\fill}\nopagebreak 
     \raisebox{-1ex}{$\Box$}\hspace*{1em}\par\addvspace{1ex}}
\renewcommand{\abstract}[1]{ \begin{quote}\noindent \small {\bf Abstract.} #1
    \end{quote}}
\newcommand{\B}[1]{{\rm I\hspace{-.2em}#1}}
\newcommand{\Nat}{{\B N}}
\newcommand{\bool}{{\cal B}{\rm ool}}
\renewcommand{\c}[1]{{\cal #1}}
\newcommand{\Funcs}{{\cal F}}
%\newcommand{\Terms}{{\cal T}(\Funcs,\Vars)}
\newcommand{\Terms}[1]{{\cal T}(#1)}
\newcommand{\Vars}{{\cal V}}
\newcommand{\Incl}{\mathbin{\prec}}
\newcommand{\Cont}{\mathbin{\succ}}
\newcommand{\Int}{\mathbin{\frown}}
\newcommand{\Seteq}{\mathbin{\asymp}}
\newcommand{\Eq}{\mathbin{\approx}}
\newcommand{\notEq}{\mathbin{\Not\approx}}
\newcommand{\notIncl}{\mathbin{\Not\prec}}
\newcommand{\notCont}{\mathbin{\Not\succ}}
\newcommand{\notInt}{\mathbin{\Not\frown}}
\newcommand{\Seq}{\mathrel{\mapsto}}
\newcommand{\Ord}{\mathbin{\rightarrow}}
\newcommand{\M}[1]{\mathbin{\mathord{#1}^m}}
\newcommand{\Mset}[1]{{\cal M}(#1)}
\newcommand{\interpret}[1]{[\![#1]\!]^{A}_{\rho}}
\newcommand{\Interpret}[1]{[\![#1]\!]^{A}}
%\newcommand{\Comp}[2]{\mbox{\rm Comp}(#1,#2)}
\newcommand{\Comp}[2]{#1\diamond#2}
\newcommand{\Repl}[2]{\mbox{\rm Repl}(#1,#2)}
%\newcommand\SS[1]{{\cal S}^{#1}}
\newcommand{\To}[1]{\mathbin{\stackrel{#1}{\longrightarrow}}}
\newcommand{\TTo}[1]{\mathbin{\stackrel{#1}{\Longrightarrow}}}
\newcommand{\oT}[1]{\mathbin{\stackrel{#1}{\longleftarrow}}}
\newcommand{\oTT}[1]{\mathbin{\stackrel{#1}{\Longleftarrow}}}
\newcommand{\es}{\emptyset}
\newcommand{\C}[1]{\mbox{$\cal #1$}}
\newcommand{\Mb}[1]{\mbox{#1}}
\newcommand{\<}{\langle}
\renewcommand{\>}{\rangle}
\newcommand{\Def}{\mathrel{\stackrel{\mbox{\tiny def}}{=}}}
\newcommand{\impl}{\mathrel\Rightarrow}
\newcommand{\then}{\mathrel\Rightarrow}
\newfont{\msym}{msxm10}

\newcommand{\false}{\bot}
\newcommand{\true}{\top}

\newcommand{\restrict}{\mathbin{\mbox{\msym\symbol{22}}}}
\newcommand{\List}[3]{#1_{1}#3\ldots#3#1_{#2}}
\newcommand{\col}[1]{\renewcommand{\arraystretch}{0.4} \begin{array}[t]{c} #1
  \end{array}}
\newcommand{\prule}[2]{{\displaystyle #1 \over \displaystyle#2}}
\newcounter{ITEM}
\newcommand{\newITEM}[1]{\gdef\ITEMlabel{ITEM:#1-}\setcounter{ITEM}{0}}
\makeatletter
\newcommand{\Not}[1]{\mathbin {\mathpalette\c@ncel#1}}
\def\LabeL#1$#2{\edef\@currentlabel{#2}\label{#1}}
\newcommand{\ITEM}[2]{\par\addvspace{.7ex}\noindent
   \refstepcounter{ITEM}\expandafter\LabeL\ITEMlabel#1${(\roman{ITEM})}%
   {\advance\linewidth-2em \hskip2em %
   \parbox{\linewidth}{\hskip-2em {\rm\bf \@currentlabel\
   }\ignorespaces #2}}\par \addvspace{.7ex}\noindent\ignorespaces}
\def\R@f#1${\ref{#1}}
\newcommand{\?}[1]{\expandafter\R@f\ITEMlabel#1$}
\makeatother
\newcommand{\PROOFRULE}[2]{\trivlist\item[\hskip\labelsep {\bf #1}]#2\par
  \addvspace{1ex}\noindent\ignorespaces}
\newcommand{\PRULE}[2]{\displaystyle#1 \strut \over \strut \displaystyle#2}
%\setlength{\clauselength}{6cm}
%% \newcommand{\clause}[3]{\par\addvspace{.7ex}\noindent\LabeL#2${{\rm\bf #1}}%
%%   {\advance\linewidth-3em \hskip 3em
%%    \parbox{\linewidth}{\hskip-3em \parbox{3em}{\rm\bf#1.}#3}}\par 
%%    \addvspace{.7ex}\noindent\ignorespaces}
\newcommand{\clause}[3]{\par\addvspace{.7ex}\noindent
  {\advance\linewidth-3em \hskip 3em
   \parbox{\linewidth}{\hskip-3em \parbox{3em}{\rm\bf#1.}#3}}\par 
   \addvspace{.7ex}\noindent\ignorespaces}
\newcommand{\Cs}{\varepsilon}
\newcommand{\const}[3]{\Cs_{\scriptscriptstyle#2}(#1,#3)}
\newcommand{\Ein}{\sqsubset}%
\newcommand{\Eineq}{\sqsubseteq}%


\voffset -1cm


\begin{document}

\section{TM and complexity}
\begin{enu}
\item TM is ...
\item {\bf simplicity creates complexity}\\
reductionism: physics $\leadsto$ biology ?
\end{enu}

\section{Observability}\label{se:obs}

\begin{enu}
\item Complexity {\em is there}, but we have to abstract. Traditionally:
assumed full information; Now: essentially partial, incomplete information
\item Observation $\simeq$
chunk-of-reality $\simeq$ abstraction from complexity.
\item\label{obs} observation, abstraction and model
\begin{enu}
\item $a:1,1$ -- $a_i=1$\\
 $b:2,6$ -- $b_1=2, b_{i+1}=3*b_i$\\
 $c:5,11$ --  $c_1=5,c_{i+1}=2c_i+1$
\item $a:1,1,2,3,2,3,3,1,3,3,1,3,3,2,3,3,3...$ -- $\lim_a=3$\\
$b:2,6,7,3,6,7,6,7,7,3,7,6...$ --  $\lim_b=7$\\
 $c:5,11,11,13,15,7,14,15,14...$ -- $\lim_c=15$ ?
\item for $c$, let $B=4$ and ... apply formula below: \\
6-th tall for $c:21,22,23,24 \to 21=\frac{6^2+6}{2} \to 8*0+4*1+2*1+1*1=7$
\item $basis:a=2, b=3, c=4$; $n$-te tall ved basis $B$ given by:
\begin{itemize}
\item $f(n)=x_1x_2...x_B$ where
\item $x_1=Bn-B+1,x_2=Bn-B+2...x_B=Bn$; then 
\item $h(x_i)=0\iff\exists z:x_i=\frac{z^2+z}{2}$ and $h(x_i)=1$ otherwise; and 
\item $tall(n)= 2^{B-1}*h(x_1)+2^{B-2}*h(x_2)+...+h(x_B)$
\end{itemize}
It is a theory -- It gives full control! It identifies invariant across various
observation-series! Operationally it is all we may need to predict, apply, etc.
\item But it does not ``explain''! There are only interconnections between
observations but no ``hierarchy'' of entities; no ontology beyond
observations?
\item If one of the series, say $a$, could be used to construct the others, we
would feel that we have a kind of ``model'' ``explaining'' the rest, or else ...
\item\label{mod} construct a ``model'' 
\[\underline{\underline{\underline{01}0}1}101110111101111101111110...\]
Is it something ``more real'' than observations? Because it ``explains'' the
observations? It is just a more abstract construction!
\item Model is an abstraction from observations just like theory. It may
happen that it will lead to new observed entities but as well it may not.
\end{enu}
\item In sciences \refe{mod} may be never reached, or it may remain forever
an unobservable, construction. In CS, we rather start with \refe{mod} and
proceed towards more abstract observations -- but using it is an ontological
paradigm is unjustified.
\item ``What do we {\em really} observe?'' makes no sense. $a$ is as good as
$b$ and ant's observations may be as legitimate as ours.
\item In CS: abstraction $\simeq$ intuition, due to underlying simplicity
\begin{enu}
\item ``Black box'' -- observational -- approach to complexity
\item Instead of designing a concrete model, describe rather desired
behaviour of the system\\
Stack vs. $pop(push(s,x))=s$, $top(push(s,x))=x$
\item language of observations; granularity and memory: $a$ or $b$ or $c$
or...? Observations are, typically, sequences...
\item causality = necessary succession; how do we observe necessity? only by
inspecting the mechanism, the actual model
\item independence = possible reversal of order; how do we verify it, observe all
possibilities? only by inspecting the model
\item[{...}] unless we have a more powerful observational mechanisms and language
\item It is very natural to think in terms of a ``model'' rather than
properties (observations)\\
Yet, because of the need of abstraction, we would like to say that we
need only observable equivalence, indistinguishability, reversal of Leibniz
law
\item {\bf trade-off between observation (properties) and model (mechanism)}
\end{enu}
\item In CS, we know there may be different models realising the same
behaviour; in science, we have only observations...
\item {\bf sciences differ primarily by what they consider as relevant
observations}, object of interest, granularity... These should be reproducible
\end{enu}

\section{Identity and haecceity}
\begin{enu}
\item What did we observe in \ref{se:obs}.\ref{obs}? ``Something''.
\item They were ''chunks-of-reality'' -- not unreal phenomena!
\item But we tend to think that there is ``something'', a real ``substance''
which presents itself through various observations.
\item ``What are {\em the} observations, the basic things?'' does not make
sense. $a$ is as real and good as $b$, as $c$, as ... \\
In CS, choice of the observational language is determined by the application;
in science -- we do not know, and it is the basic activity to design
appropriate languages; \\
AI - sometimes reversal of Leibniz, sometimes ``nucleus''
\item Scientifically, we can get at most as far as reversal of Leibniz law
\item ...
\item {\bf Haecceity forever beyond sicence}, still, what is observed is real, only,
not the {\em whole} real\\
unity = one reality; diversity = accessible only partially through observations
\end{enu}

\section{Intertranslability}
\begin{enu}
\item Meaning is the invariant of translation; and so is truth
\begin{enu}
\item Tarski
\item BG
\item ...
\end{enu}
\item reductionism = translation into one, all-embracing language (also
``General Theory of Everything''); vs. partial mutual translations
\item there are ``isolated islands'' and larger, ``connected continents'' --
connected by translatability 
\item {\bf ant's science would be as legitimate as ours} -- even if not translable
into our language\\
{\bf the unity is not a GTE but a network of intertranslations}
\end{enu}

\section{Antireductionism}
\begin{enu}
\item I can compute with $\pi$ -- TM cannot; because I use the abstraction, a
symbol ``$\pi$'', which TM cannot use
\item infinite processes -- redefine the meaning of computation's result
\item emergent behaviour -- beyong Turing computability
\end{enu}

\end{document}