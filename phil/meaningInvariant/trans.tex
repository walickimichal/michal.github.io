
\documentclass[10pt]{article}
\makeatletter

\ifcase \@ptsize
    % mods for 10 pt
    \oddsidemargin  0.15 in     %   Left margin on odd-numbered pages.
    \evensidemargin 0.35 in     %   Left margin on even-numbered pages.
    \marginparwidth 1 in        %   Width of marginal notes.
    \oddsidemargin 0.25 in      %   Note that \oddsidemargin = \evensidemargin
    \evensidemargin 0.25 in
    \marginparwidth 0.75 in
    \textwidth 5.875 in % Width of text line.
\or % mods for 11 pt
    \oddsidemargin 0.1 in      %   Left margin on odd-numbered pages.
    \evensidemargin 0.15 in    %   Left margin on even-numbered pages.
    \marginparwidth 1 in       %   Width of marginal notes.
    \oddsidemargin 0.125 in    %   Note that \oddsidemargin = \evensidemargin
    \evensidemargin 0.125 in
    \marginparwidth 0.75 in
    \textwidth 6.125 in % Width of text line.
\or % mods for 12 pt
    \oddsidemargin -10 pt      %   Left margin on odd-numbered pages.
    \evensidemargin 10 pt      %   Left margin on even-numbered pages.
    \marginparwidth 1 in       %   Width of marginal notes.
    \oddsidemargin 0 in      %   Note that \oddsidemargin = \evensidemargin
    \evensidemargin 0 in
    \marginparwidth 0.75 in
    \textwidth 6.375 true in % Width of text line.
\fi

\voffset -2cm
\textheight 22.5cm

\makeatother

%% %\makeatletter
%\show\
%\makeatother
\newcommand{\ite}[1]{\item[{\bf #1.}]}
\newcommand{\app}{\mathrel{\scriptscriptstyle{\vdash}}}
\newcommand{\estr}{\varepsilon}
\newcommand{\PSet}[1]{{\cal P}(#1)}
\newcommand{\ch}{\sqcup}
\newcommand{\into}{\to}
\newcommand{\Iff}{\Leftrightarrow}
\renewcommand{\iff}{\leftrightarrow}
\newcommand{\prI}{\vdash_I}
\newcommand{\pr}{\vdash}
\newcommand{\ovr}[1]{\overline{#1}}

\newcommand{\cp}{{\cal O}}

% update function/set
%\newcommand{\upd}[3]{#1\!\Rsh^{#2}_{\!\!#3}} % AMS
\newcommand{\upd}[3]{#1^{\raisebox{.5ex}{\mbox{${\scriptscriptstyle{\leftarrow}}\scriptstyle{#3}$}}}_{{\scriptscriptstyle{\rightarrow}}{#2}}} 
\newcommand{\rem}[2]{\upd{#1}{#2}{\bullet}}
\newcommand{\add}[2]{\upd {#1}{\bullet}{#2}}
%\newcommand{\mv}[3]{{#1}\!\Rsh_{\!\!#3}{#2}}
\newcommand{\mv}[3]{{#1}\:\raisebox{-.5ex}{$\stackrel{\displaystyle\curvearrowright}{\scriptstyle{#3}}$}\:{#2}}

\newcommand{\leads}{\rightsquigarrow} %AMS

\newenvironment{ites}{\vspace*{1ex}\par\noindent 
   \begin{tabular}{r@{\ \ }rcl}}{\vspace*{1ex}\end{tabular}\par\noindent}
\newcommand{\itt}[3]{{\bf #1.} & $#2$ & $\impl$ & $#3$ \\[1ex]}
\newcommand{\itte}[3]{{\bf #1.} & $#2$ & $\impl$ & $#3$ }
\newcommand{\itteq}[3]{\hline {\bf #1} & & & $#2=#3$ }
\newcommand{\itteqc}[3]{\hline {\bf #1} &  &  & $#2=#3$ \\[.5ex]}
\newcommand{\itteqq}[3]{{\bf #1} &  &  & $#2=#3$ }
\newcommand{\itc}[2]{{\bf #1.} & $#2$ &    \\[.5ex]}
\newcommand{\itcs}[3]{{\bf #1.} & $#2$ & $\impl$ & $#3$  \\[.5ex] }
\newcommand{\itco}[3]{   & $#1$ & $#2$  & $#3$ \\[1ex]}
\newcommand{\itcoe}[3]{   & $#1$ & $#2$  & $#3$}
\newcommand{\bit}{\begin{ites}}
\newcommand{\eit}{\end{ites}}
\newcommand{\na}[1]{{\bf #1.}}
\newenvironment{iten}{\begin{tabular}[t]{r@{\ }rcl}}{\end{tabular}}
\newcommand{\ass}[1]{& \multicolumn{3}{l}{\hspace*{-1em}{\small{[{\em Assuming:} #1]}}}}

%%%%%%%%% nested comp's
\newenvironment{itess}{\vspace*{1ex}\par\noindent 
   \begin{tabular}{r@{\ \ }lllcl}}{\vspace*{1ex}\end{tabular}\par\noindent}
\newcommand{\bitn}{\begin{itess}}
\newcommand{\eitn}{\end{itess}}
\newcommand{\comA}[2]{{\bf #1}& $#2$ \\ }
\newcommand{\comB}[3]{{\bf #1}& $#2$ & $#3$\\ }
\newcommand{\com}[3]{{\bf #1}& & & $#2$ & $\impl$ & $#3$\\[.5ex] }

\newcommand{\comS}[5]{{\bf #1} 
   & $#2$ & $#3$ & $#4$ & $\impl$ & $#5$\\[.5ex] }

%%%%%%%%%%%%%%%%
\newtheorem{CLAIM}{Proposition}[section]
\newtheorem{COROLLARY}[CLAIM]{Corollary}
\newtheorem{THEOREM}[CLAIM]{Theorem}
\newtheorem{LEMMA}[CLAIM]{Lemma}
\newcommand{\MyLPar}{\parsep -.2ex plus.2ex minus.2ex\itemsep\parsep
   \vspace{-\topsep}\vspace{.5ex}}
\newcommand{\MyNumEnv}[1]{\trivlist\refstepcounter{CLAIM}\item[\hskip
   \labelsep{\bf #1\ \theCLAIM\ }]\sf\ignorespaces}
\newenvironment{DEFINITION}{\MyNumEnv{Definition}}{\par\addvspace{0.5ex}}
\newenvironment{EXAMPLE}{\MyNumEnv{Example}}{\nopagebreak\finish}
\newenvironment{PROOF}{{\bf Proof.}}{\nopagebreak\finish}
\newcommand{\finish}{\hspace*{\fill}\nopagebreak 
     \raisebox{-1ex}{$\Box$}\hspace*{1em}\par\addvspace{1ex}}
\renewcommand{\abstract}[1]{ \begin{quote}\noindent \small {\bf Abstract.} #1
    \end{quote}}
\newcommand{\B}[1]{{\rm I\hspace{-.2em}#1}}
\newcommand{\Nat}{{\B N}}
\newcommand{\bool}{{\cal B}{\rm ool}}
\renewcommand{\c}[1]{{\cal #1}}
\newcommand{\Funcs}{{\cal F}}
%\newcommand{\Terms}{{\cal T}(\Funcs,\Vars)}
\newcommand{\Terms}[1]{{\cal T}(#1)}
\newcommand{\Vars}{{\cal V}}
\newcommand{\Incl}{\mathbin{\prec}}
\newcommand{\Cont}{\mathbin{\succ}}
\newcommand{\Int}{\mathbin{\frown}}
\newcommand{\Seteq}{\mathbin{\asymp}}
\newcommand{\Eq}{\mathbin{\approx}}
\newcommand{\notEq}{\mathbin{\Not\approx}}
\newcommand{\notIncl}{\mathbin{\Not\prec}}
\newcommand{\notCont}{\mathbin{\Not\succ}}
\newcommand{\notInt}{\mathbin{\Not\frown}}
\newcommand{\Seq}{\mathrel{\mapsto}}
\newcommand{\Ord}{\mathbin{\rightarrow}}
\newcommand{\M}[1]{\mathbin{\mathord{#1}^m}}
\newcommand{\Mset}[1]{{\cal M}(#1)}
\newcommand{\interpret}[1]{[\![#1]\!]^{A}_{\rho}}
\newcommand{\Interpret}[1]{[\![#1]\!]^{A}}
%\newcommand{\Comp}[2]{\mbox{\rm Comp}(#1,#2)}
\newcommand{\Comp}[2]{#1\diamond#2}
\newcommand{\Repl}[2]{\mbox{\rm Repl}(#1,#2)}
%\newcommand\SS[1]{{\cal S}^{#1}}
\newcommand{\To}[1]{\mathbin{\stackrel{#1}{\longrightarrow}}}
\newcommand{\TTo}[1]{\mathbin{\stackrel{#1}{\Longrightarrow}}}
\newcommand{\oT}[1]{\mathbin{\stackrel{#1}{\longleftarrow}}}
\newcommand{\oTT}[1]{\mathbin{\stackrel{#1}{\Longleftarrow}}}
\newcommand{\es}{\emptyset}
\newcommand{\C}[1]{\mbox{$\cal #1$}}
\newcommand{\Mb}[1]{\mbox{#1}}
\newcommand{\<}{\langle}
\renewcommand{\>}{\rangle}
\newcommand{\Def}{\mathrel{\stackrel{\mbox{\tiny def}}{=}}}
\newcommand{\impl}{\mathrel\Rightarrow}
\newcommand{\then}{\mathrel\Rightarrow}
\newfont{\msym}{msxm10}

\newcommand{\false}{\bot}
\newcommand{\true}{\top}

\newcommand{\restrict}{\mathbin{\mbox{\msym\symbol{22}}}}
\newcommand{\List}[3]{#1_{1}#3\ldots#3#1_{#2}}
\newcommand{\col}[1]{\renewcommand{\arraystretch}{0.4} \begin{array}[t]{c} #1
  \end{array}}
\newcommand{\prule}[2]{{\displaystyle #1 \over \displaystyle#2}}
\newcounter{ITEM}
\newcommand{\newITEM}[1]{\gdef\ITEMlabel{ITEM:#1-}\setcounter{ITEM}{0}}
\makeatletter
\newcommand{\Not}[1]{\mathbin {\mathpalette\c@ncel#1}}
\def\LabeL#1$#2{\edef\@currentlabel{#2}\label{#1}}
\newcommand{\ITEM}[2]{\par\addvspace{.7ex}\noindent
   \refstepcounter{ITEM}\expandafter\LabeL\ITEMlabel#1${(\roman{ITEM})}%
   {\advance\linewidth-2em \hskip2em %
   \parbox{\linewidth}{\hskip-2em {\rm\bf \@currentlabel\
   }\ignorespaces #2}}\par \addvspace{.7ex}\noindent\ignorespaces}
\def\R@f#1${\ref{#1}}
\newcommand{\?}[1]{\expandafter\R@f\ITEMlabel#1$}
\makeatother
\newcommand{\PROOFRULE}[2]{\trivlist\item[\hskip\labelsep {\bf #1}]#2\par
  \addvspace{1ex}\noindent\ignorespaces}
\newcommand{\PRULE}[2]{\displaystyle#1 \strut \over \strut \displaystyle#2}
%\setlength{\clauselength}{6cm}
%% \newcommand{\clause}[3]{\par\addvspace{.7ex}\noindent\LabeL#2${{\rm\bf #1}}%
%%   {\advance\linewidth-3em \hskip 3em
%%    \parbox{\linewidth}{\hskip-3em \parbox{3em}{\rm\bf#1.}#3}}\par 
%%    \addvspace{.7ex}\noindent\ignorespaces}
\newcommand{\clause}[3]{\par\addvspace{.7ex}\noindent
  {\advance\linewidth-3em \hskip 3em
   \parbox{\linewidth}{\hskip-3em \parbox{3em}{\rm\bf#1.}#3}}\par 
   \addvspace{.7ex}\noindent\ignorespaces}
\newcommand{\Cs}{\varepsilon}
\newcommand{\const}[3]{\Cs_{\scriptscriptstyle#2}(#1,#3)}
\newcommand{\Ein}{\sqsubset}%
\newcommand{\Eineq}{\sqsubseteq}%


\voffset -1cm


\newcounter{EQ}
%\newcommand{\equ}[1]{\refstepcounter{EQ}\begin{center}\ \hfill #1\hfill{(\theEQ)}\end{center}}
\newcommand{\equ}[1]{\refstepcounter{EQ}\vspace{.5ex}\par\noindent\ 
    \hfill $#1$\hfill{(\theEQ)}\\[.5ex]}
\newcommand{\refe}[1]{(\ref{#1})}
\newcommand{\<}{\langle}
\renewcommand{\>}{\rangle}

\newcommand{\MyLPar}{\parsep -.2ex plus.2ex minus.2ex\itemsep\parsep
   \vspace{-\topsep}\vspace{.5ex}}

\newcommand{\co}[1]{{\sf{#1}}}
\newcommand{\wo}[1]{`#1'}
\newcommand{\thi}[1]{{\sl{#1}\/}}

\title{The Truth of `Meaning' and ... the Meaning of `Truth'}
\author{Micha{\l} Walicki}
\date{{}}
\begin{document}

\maketitle
\hfill\today

\noindent
All kinds of coherence and immanent theories of truth and meaning 
notwithstanding, the basic intuition remains the same: linguistic 
expressions 
refer, for the most, to some non-linguistic world and language's meaning is 
constituted by a relation to this world. What this relation consists of is 
the painful question, even more so, as it is hard to say what the world 
is without saying it in ... some language. A schematic illustration of this 
follows. %\vspace*{-1ex}

\begin{figure}[hbt]  \begin{center}
\setlength{\unitlength}{1cm}
\hspace*{6em}
\begin{picture}(4,3)  
\put(1,3){\oval(3,.6)}
\put(.2,2.9){L(anguage)}
\thicklines \put(1,1.4){\line(0,1){1.2}} \thinlines
\put(1.1,2){$M$}
\put(1,1){\oval(3,.6)}
\put(.3,.9){W(orld)}
\end{picture}  \vspace*{-4ex}\end{center}
\caption{Linguistic frame}\label{lang}
\end{figure}


\noindent
I won't elaborate too much on the painful question about the character of 
this relation. For our discussion it will do if we merely keep in mind that
it is a {\em relation} and not necessarily a function. (We draw a line at $M$, 
not an arrow!) An ambiguous 
expression may have different meanings while, on the other hand, the same 
meaning (not only the same object) can be expressed in different words. 


Given the relation $M$, it is tempting to make the following 
definitions. A linguistic expression (and I won't go into detailed 
distinctions between such), say \wo{dog}, has the associate semantic 
field in the world, namely all the \thi{dogs}. 
The meaning of an expression 
\wo{expr} is its afterset wrt. this relation, i.e., 
\equ{M({expr})=\{ {obj}:\<{expr},{obj}\> \in M\},}
whatever this set is and 
however it is constituted. Dually, any thing \thi{obj} has the associated 
set of linguistic expressions which can refer to it, namely, the foreset 
wrt. the relation \footnote{There is nothing wrong with allowing the language 
to be part 
of the world and let its expressions refer to other expressions. The above 
illustration captures only the schematic situation of a linguist for whom
the language is an independed object of studies in its relation to the world.
Also, I am by no means
limiting myself to extensional view. If you like, the meaning of an 
\wo{intensional expression} can be taken to be its \thi{intension} which, 
like \thi{dogs}, is in the world.)
Accepting the danger of oversimplification I will gloss over this, and many 
other niceties.} 
\equ{M^-({obj})=\{{expr}:\<{expr},{obj}\> \in M\}.}
It should be emphasized that meaning $M$, being a relation, may very well 
leave some expressions without any associated meaning and, vice versa, some 
meanings without associated expressions.
 
To fix terminology, let me call the pair 
\wo{word}-\thi{its meaning} a {\em concept}, e.g., the concept \co{dog} is 
the pair \wo{dog}-\thi{dogs}.
It is so clean and nice that there must be serious problems and I am sure 
that if philosophers got his into their hands, there would be an 
unpleasant slaughter. But I am not making any metaphysical or epistemic 
claims, I do not insist that this is what concepts {\em really 
are} -- after all, who can tell? I just stipulate the definition since this 
pair will play an important role in what follows.


Now, I believe that the figure~\ref{lang} reflects the universal situation 
in semantic: we have a language, its epressions refer to 
something-out-there, and our task is to investigate how that happens, what 
this relation consists of, how it is consituted. I am by no means 
unsymphatetic, this is certainly fruitful and demanding. But I think that 
there is also an underlying assumption which I will try to confront; the 
assumption that this figure is static and fixed, that concepts, even if 
fuzzy, are given, that meaning, even if ineffable, is constituted withing 
such a static linguistic frame. This fundamental fuzziness and ineffability  
come from even more basic fact that there are hardly any static sematnic 
frames and that what constitues meaning is as much a static relation within 
a frame as a dynamic relation of transition between frames.

\section{Indeterminacy and Invariants of Translation}
Translation is considered to be a transition between two frames. Ok, it is a 
transition from one language to another but, of course, not an arbirtary 
one but one ``preserving the meaning''. Again, we may draw a neat schema:
% \vspace*{-1ex}

\begin{figure}[hbt]  \begin{center}
\setlength{\unitlength}{1cm}
\begin{picture}(8,3)  
\put(1,3){\oval(2,.6)}   
\put(.9,2.9){$L_1$}
 \thicklines \put(2.2,3){\line(1,0){2.6}} \thinlines
 \put(3.4,3.2){$\tau_L$}
\thicklines \put(1,1.4){\line(0,1){1.2}} \thinlines
\put(1.1,2){$M_1$}
\put(1,1){\oval(2,.6)}
\put(.9,.9){$W_1$}
 \thicklines \put(2.2,1){\line(1,0){2.6}} \thinlines
 \put(3.4,1.2){$\tau_W$}

\put(6,3){\oval(2,.6)}
\put(5.9,2.9){$L_2$}
\thicklines \put(6,1.4){\line(0,1){1.2}} \thinlines
\put(6.1,2){$M_2$}
\put(6,1){\oval(2,.6)}
\put(5.9,.9){$W_2$}

\end{picture}  \vspace*{-4ex}\end{center}

\caption{Traslation}\label{trans}
\end{figure}

Typically, one would draw the upper horisontal arrow and not just line. However, 
although this is how translation is used in practice, this possibility is 
based on the existence -- and discovery -- of a correspondence between the 
respective languages. If one can translate language $L_1$ into $L_2$, 
then one can also translate, even if imperfectly, $L_2$ into $L_1$. 
Besides, as we know, already the first translation is imperfect. For us, 
this means that here too we have to do with {\em relation}.

What may strike one in the above figure is the presence of the associated 
translation of the worlds. Uuhh?! Yes. It does not show, if we stick to 
simple examples. We have no reservations when German \wo{Hund} gets 
translated as \wo{dog} -- \thi{dogs} are \thi{dogs}, whether in Germany or 
in England, and there is no need, at least no explicit need, to translate 
\thi{German dogs} to \thi{English dogs}. It would be a bit worse if we had, 
in Germany, a race of dogs not known in England. The actual 
\wo{Hund} might then refer to a \thi{dog}, sorry, \thi{Hund} of this race.
Ok, it is well-known that the semantic fields of words which we often take 
to be translations of each another are not necessarily the same in two 
languages. Or as we might say, using our definition: the concept 
\co{Hund} may not exactly coincide with the concept \co{dog}. And since 
one may try to look for another phrase describing the actual \thi{Hund} of 
the unknown race, indeterminacy of translation follows.

The point is only that translation does involve not only correspondence of 
the linguistic expressions but, equally, a correspondence between their 
semantic fields. When the fields diverge, one looks for other expressions. 
And what if one does not find any? Well, one always finds some, sometimes 
better and sometimes worse. And this is the whole meaning of the lower 
relation: when no exact linguistic counterpart is found, that is, there is 
no concept in the target language coinciding with the concept from the 
source language, one translates the respective world. This is much more 
clear if we ask for a translation of Nordic mythology (where there is a lot 
of snow) into a language of a small, isolated African tribe {\em Umba}, 
where no word 
for \thi{snow} exists. These are perhaps rare cases, but they show that, 
in fact, translation of the worlds takes place too. And when we think that 
it does not, it is because the two worlds are so close to each other that 
this translation takes care of itself.\footnote{Formally, if you do not 
like the idea of translating worlds, you may take both $W_1$ and $W_2$ to 
be the same world and $\tau_W$ to be identity.}
As in the case of $M$, both $\tau_L$ and $\tau_W$ are relations and so, 
formally, nothing precludes the cases where a word, \wo{snow}, in one 
language has no counterpart in the other. More significantly, \thi{snow} 
existing in one world may have no counterpart in the other world.

In fact, something more subtle can be said here. I am trying to translate 
an expression \wo{$e_1$} from $L_1$ to $L_2$. I find a candidate 
$e_2=\tau_L'(e_1)$ -- it says $\tau_L'$ to indicate that I 
have chosen a particular candidate among the possible ones given by the 
full afterset $\tau_L(e_1)$. I find its meaning $M_2(\tau_L'({e_1}))$ and 
check whether, translated back to the world $W_1$ it will yield the meaning 
of the original expression. The more subtle point concerns the fact that 
while the translation of the language goes from the source to the target, 
the associated -- verifying -- translation of the worlds goes in the 
opposite direction. Ideally, I should get what was there in the begining, 
that is, the formula for the ideal translation would be:
\equ{\label{inst} 
M_1({e_1}) = \tau_W^-(M_2(\tau_L'({e_1}))) }
Notice that another procedure of ideal translation would be expressed by 
the formula:
\equ{\label{ninst} 
\tau_W(M_1({e_1})) = M_2(\tau_L'({e_1})) }
If I discover that {\em Umba} has no word for \thi{snow}, I may decide for 
something like \wo{white pulver coming from the sky instead of rain}. 
According to \refe{inst}, I have lost, for, although the meaning of this 
phrase translated back to English has something to do with snow, it 
certainly isn't the same. However, according to \refe{ninst}, I am fine 
because what this phrase describes in {\em Umba} is what I get when I 
translate the English meaning of \wo{snow}, namely \thi{snow}, into 
the {\em Umba} world.

Since, mathematically \refe{inst} and \refe{ninst} are 
equivalent, this shows that our mathematics is too simple to capture all 
possible dimensions. Nevertheless, mathematics can say a lot about various 
compatibility criteria for the situations illustrated in 
figure~\ref{trans}, some of which will be indicated later on. In any case, 
mathematics serves us here merely as a model -- even if imperfect one. More 
importantly, this model should suffice to convey the fundamental point 
which is:

\section{Meaning is an invariant of translation}
If this sounds audacious, this section will try to make it less so. The 
setting of translation are two separate linguistic frames. And if we restrict 
our attention to this context, there isn't much more to say than that 
``translation attempts to preserve meaning''. Such a static view amounts to 
freezing and, as a matter of fact, absolutizing the static character of 
respective frames. The first claim is, that the relation, that is the 
relation within a frame between expressions and their referents, is by no 
means that static. I have intensionally been very 
vague as to what $L_1$, $L_2$ are and, more importantly, what $W_1$, $W_2$ 
may be. They certainly may be the frames of two separate language communities 
with distinc languages. But they also may be the frames within one 
language community, for instance, the frames of two persons. The 
relation $\tau$ would then be called \wo{communication} rather than 
\wo{translation}. They may also be frames of one and the same person at 
different moments of time, in which case, $\tau$ amounts to development of 
individual language capacities as well as world perception. \footnote{I am 
using \wo{world perception} to denote something like {\em Weltanschaung} -- not 
just perception but all the 
capacities of perceiving, recognizing, thinking distinctions and 
connections, as well as relating to them.} 
Probably, there are more plausible interpretations, but 
these will do for now.

To justify the first claim I only briefly mention the aspects of these two 
latter situations showing that all of the them share the structure represented 
in figure~\ref{trans}. Meeting new people, one often confronts a situation 
requiring a good deal of translation. And, provided that both speak the 
same language, even the language of the same social class, the translation 
will typically involve the translation of the respective worlds. Another 
tells me about his children, and his feelings for them, and I, who have no 
children, try to be emphatetic, which means, try to translate the setting 
he is sketching and incorporate it into my world. Or else he tells me about
his travels to foreign countries, or about his house, or problems at work, 
or whatever. Everything, which I understand pefectly well, has to be 
accommodated to my frame. Occasionally, he may use a word I do not 
understand. If I cannot translate it merely from the context, I ask him 
for explaining the word and, pretty adult and mature language users as we 
are, the conversation may go on. However, it may also turn out that he 
uses a word, which was familiar to me, in a way which does not make sense 
from the context. I haven't noticed it the first time, judged it a slip of 
the tounge, but then he did it again, and now again. Well, I cannot make it 
fit, I cannot translate it, so I ask him again. He gives me an 
astonishingly precise description of its meaning. I must admit that, in 
the past, I used to be occassionally uncertain about this word and now I 
realize why. I had slightly misunderstood it, I construed it in an 
approximate fashion which worked by and large, but sometimes caused 
uncertainty. Now I see why. I give him the right and he continues. It 
should pass without saying that I enter analogous scenario when reading a 
book, listen to a radio, or participate in a discussion between several 
people.

The story of a development of personal frame is indicated in the above 
conversation. Children development may illustrate it more 
clearly because here it is harder to deny that actual expansion 
and adjustements of both the language and the world and their mutual 
relation take place. Elimination of overgeneralizations is supposedly 
showing that language capacities are developed under the corrective of the 
environment. And surely, they are. However, they show development not only 
of language capacities but also of the world perception.
 If, at first, a child calles a 
\thi{plane} a \wo{bird}, a correction amounts to forcing a distinction which 
might not have been there before. And so on and on. Instead of multiplying 
examples, let me say at once that in this case, it will be typically the 
case that the relations $\tau$ effect sophistication of concepts. On the 
one hand, overgeneralizations get refined, and on the other, new 
abstactions appear. The latter corresponds to emergence of new words and 
expressions which collect various earlier meanings. Thus, having learned 
\wo{Europeans}, I do not find any word in the earlier language $L_1$ -- 
still, I may be able to find a group of entities (meanings) in the earlier 
world $W_1$ to which I can correctly translate the current meaning 
$M_2(Europeans_2)$.
The former corresponds to relating the word 
\wo{bird}$_1$ to word \wo{bird}$_2$ and, perhaps, also to the new word 
\wo{plane}$_2$. Notice that the corresponding $\tau^-_W$ will have no 
proper counterpart for \thi{plane}$_2$ in the earlier world $W_1$.
However, what is left from \thi{bird}$_1$ after it became \thi{bird}$_2$ is 
invariant under translation: any \thi{bird}$_2$ can be translated back to 
\thi{bird}$_1$.

Thus arises the second, and main claim, that meaning is what is invariant 
in various, yes, all transitions between frames. It is what persists
through all confrontations with new and other frames, whether my own or 
other people's, or other language's. Notice that, like the abstraction 
example above shows, new concepts may arise and thus, new meanings. They 
need not be entirely new -- I can translate the meaning of \wo{Europens} 
back -- but they are new as meanings of {\em new expressions}. 
One might, probably, try to think of meanings-in-themselves but for our 
purposes meanings are always meanings of some expressions. Thus, 
speaking about meaning as invariance under translation makes sense only from the 
point where meaning appeared in the semantic world {\em and} a 
corresponding expression appeared at the linguistic level.

\paragraph{Interdependency of meaning and translation.}
According to the main thesis, meaning of an expression is not fixed once 
and for all. For sure, it may be very stable, especially when living in a 
static world with few changes. But it also has a shifting aspect, any word 
can acquire a slight variation of meaning, a slight modification of 
emphasis, tone, flavour, when confronted with a new variation of its 
original meaning or else with a close but different meaning of a word in a 
different language. 

Only the assumption that each linguistic frame is static and closed leaves
us satisfied with the 
indeterminacy of translation. But there is more to translation 
than that. As any person using daily two different languages 
will know, it is creative and constitutive for meaning, it may endow old 
expressions with new meanings and old meanings with new expressions.
Taking as the basis figure \ref{trans} (or, rather, an innumerable bunch of such 
figures) instead of figure~\ref{lang}, allows us to fiddle not only with 
$\tau$'s, in order to accommodate them to $M$'s, but with $M$'s as well.
 This is what everybody learning a language does. (Linguists 
learning foreing languages do exactly that, too -- they change the frame.)
This is what a great translation of a difficult piece of text does.
And this is what everybody listening 
to other people does.\footnote{Beware, I do not mean that there are no bad, 
or incorrect translations -- in the literal, and not our general sense. 
In pratice, one does have two relatively stable 
frames, typically, sharing most of the world and separated by distinc 
languages. In practice, a lot of ingenuity is needed to make translation 
affect meaning in an acceptable way.} 

\paragraph{No language without language users.}
A point which may be worth noting is that linguistic frames are certainly 
anchored somewhere, namely, at the language users whether individual or 
collective. Meaning is inseparably linked to the use of the language. 
For conversation, discussion and any attempt to understand what another is 
saying, is a translation, that is, enrichment. It is where meaning of a 
language emerges. And since few people talk exclusively to 
themselves, meaning is a highly social phenomenon. 

I hope to have made it clear that no solipsistic constitution of meaning 
is assumed nor ensures. But it is not excluded either. It all depends on 
what one is willing to consider as frames. I, at least, would include here 
the examples from the begining of this section. Thus, since children 
certainly hear their parents and we certainly do talk and listen, the 
meaning is an invariant of a horrible sphagetti of mutual relations and 
translations. 

This, I believe, may account 
also for the impossibility to define the meaning of even simplest phrases. 
On the one hand, such a definition has to relate to the world which, we feel, is richer 
than language. But on the other hand, it has to accommodate all the 
modifications with which innumerable contexts and users, also all future poeats,
may  endow the meaning of the words.



\paragraph{Invariance, meaning...}
In a sense, I have dissociated meaning from expressions. It now resides 
beyond any single frame, at a meta-linguistic level. In a sense. And it is 
fine with me. But this smells, I am afraid, again the absolutization of semantic 
frames. I know, linguists love them but the whole point here is that, 
important and basic as they are, single frames are not that constitutive for 
meaning which emerges only between them. 
(Keep also in mind that frames are not just different languages, 
they exist withing every single language as well.) 
But, although meaning emerges between frames, there is 
no access to it except through some actual single frame. Only 
from such a frame, can I begin translating and, eventually, arrive 
at a new, larger frame encompassing the meaning of an expression, or else, 
more prosaically, elucidate the meanig of the exprerssion within my old 
frame.

More to the point, what does \wo{invariance} actually mean? I do not think 
I have said it very precisely. (If you feel I did, you may skip to the next 
section.) The point of departure is, as I just said the relation $M$ within 
one frame (right after I have heard or said the first words.) Eventually, 
it is the frame of an adult person who speaks the language he speaks and 
lives in the world he lives in. At this stage, there is apparently not so much 
new-meaning-acquisition, as only reuse of various meanings constituted in the 
course of development. But only apparently because, as I have argued, we 
do translate as long as we converse. If we ask about the meaning within a 
single frame, it must appear as something primitive, given, metaphysical, 
if not simply mysterious. The only experience revealing its ... well, 
meaning, is a transition to another frame. 

In terms of the 
figure~\ref{trans}, the invariant is this part of $M_1$ which makes the 
diagram commute, that is all these pairs $\<e_1,m_1\>\in M_1$ which survive the 
transition and hold after application of $\tau:$ 
\equ{\label{pres}
%e_1M_1m_1 \Rightarrow \tau_L(e_1)M_2\tau_W(m_1).
\<e_1,m_1\> \in M_1 \Rightarrow \<\tau_L(e_1),\tau_W(m_1)\> \in M_2.
} 
But even reducing the intuitive discussion to such 
a formal statement, we still have a lot of choices. What we have just said 
is that meaning is what is {\em preserved} by translation. Why not {\em 
reflected}, that is, why not that part of $M_2$ which originates from $M_1$, i.e.:
\equ{\label{refl}
\<e_2,m_2\> \in M_2 \Rightarrow \<\tau_L^-(e_2),\tau_W^-(m_2)\> \in M_1 \ \mbox{\rm ?}
%e_2M_2m_2 \Rightarrow \tau_L^-(e_2)M_1\tau_W^-(m_2)
}
These are not equivalent. So, perhaps, we should require both 
preservation {\em and} reflection? But why?

Things are even worse because all our $M$'s and $\tau$'s are 
relations. Thus, in general, $\tau_W(m_1)$ or $\tau_L^-(e_2)$ are {\em 
sets}. For instance, in \refe{pres} we start with singletons $e_1$ and 
$m_1$ but arrive at the sets $\tau_L(e_1)$ and $\tau_W(m_1)$. Should we 
require $M_2$ to hold between all individuals from these sets, some, at 
least one pair?
There are all kinds of possibilitites of combining 
preservation/reflection properties with the set-valued operations leading 
to a variety of possible formal invariance criteria. Snice I do not intend 
to settle this detail here, I refer interested reader to mathematical work 
on the subject.\footnote{\cite{Koh, Brink, WB}}

\section{Truth and meaning}
Speaking abut \thi{truth} should make one uncomfortable so I will start 
assumig the, by now quite common, position and treat it with symbols and 
formal systems. I haven't said that, but the figure~\ref{trans} has a 
special case (in fact is a generalization) where 
 $M$ is replaced by $\models$. This little known picture\footnote{It is 
widely known only in a tiny subculture of theoretical computer scientists 
who, themselves, constitue only a small branch of computer science; see~\cite{Inst}.} 
is as 
follows:

\begin{figure}[hbt]  \begin{center}
\setlength{\unitlength}{1cm}
\begin{picture}(8,3)  
\put(1,3){\oval(2,.6)}   
\put(.9,2.9){$L_1$}
 \thicklines \put(2.2,3){\vector(1,0){2.6}} \thinlines
 \put(3.4,3.2){$\tau_L$}
\thicklines \put(1,1.4){\line(0,1){1.2}} \thinlines
\put(1.1,2){$\models_1$}
\put(1,1){\oval(2,.6)}
\put(.9,.9){$W_1$}
 \thicklines \put(4.8,1){\vector(-1,0){2.6}} \thinlines
 \put(3.4,1.2){$\tau_W^-$}

\put(6,3){\oval(2,.6)}
\put(5.9,2.9){$L_2$}
\thicklines \put(6,1.4){\line(0,1){1.2}} \thinlines
\put(6.1,2){$\models_2$}
\put(6,1){\oval(2,.6)}
\put(5.9,.9){$W_2$}

\end{picture}  \vspace*{-4ex}\end{center}
\caption{Logical frame}\label{instF}
\end{figure}

\noindent
At the language level, we have formal theories: languages 
of some kind, for instance, all possible signatures (collections of 
non-logical symbols) for first-order logic with some axioms. 
With each such theory there
is associated its world -- the class of structures which interpret all the 
symbols. These (theories and worlds) are then related by $\models$-relation, 
in this case, the usual satisfaction relation of first-order logic. 
One 
defines translation of theories in some purposeful manner (for instance, 
so that it preserves logical symbols) and then shows, that for each 
language translation $\tau_L$ there is a corresponding, but contravariant, 
model translation $\tau_W^-$ at the model level. Given 
a translation $\tau_L:T_1\rightarrow T_2$ and any model $M_2\models T_2$ in $W_2$, one 
may recover a model $M_1=\tau_W^-(M2)$. The whole thing is of interest iff 
all these translations and $\models$-relations satisfy the condition for
every model $M_2$ and sentence $e_1:$
\equ{ \label{lf}
\tau_W^-(M_2)\models_1 e_1 \iff M_2\models_2 \tau_L(e_1)
}
Yes, roughly but not exactly, this is what we have seen in \refe{inst}. (Only here 
both $\tau_L$ and $\tau_W^-$ are functions.) 
This notion of logical frame generalizes the traditional notion of 
truth -- it is easy to show that most (if not all) standard logical 
systems known from the history of formal logic satisfy the 
condition~\refe{lf}. But now, one stands quite free to choose which 
parts of the picture to adjust -- one may fiddle with the definition of 
translation as well as with the definition of $\models$. And when one 
arrives at something which satisfies \refe{lf}, one is pretty sure that it 
isn't nonsense.\footnote{There exists quite an extensive theory of such 
logical frames.}

In this scenario, it is truth and only truth (well, $\models$) that is 
invariant under translation. So one may rightly say. But we do not have to 
yield to mathematical accidents. These people wanted to generalize Tarski's 
truth definition and they seem to have succeded quite well. But we want 
something different. Let me therefore say it at once. I certainly want to 
keep truth invariant under translation. However, meaning already is and 
the two do not seem to cincide. So ...? Truth is again a kind of (yes, I 
want to be vague) relation between linguistic expressions and the world. But 
not arbitrary linguistic expressions -- only propositional ones. Invariance 
of truth is a special case of the invariance of meaning, just as 
propositions (or whatever you assume can have truth-value) are special 
cases of linguistic expressions. 

Did I say that the meaning of a proposition is its truth conditions? Or even 
worse, its truth value? And therefore is preserved by translation, since 
meaning is... It is certainly a possible and simplest way. But I didn't mean 
it. I think the whole project of reducing meaning to truth is a dinosaur, 
so let's keep it where it belongs. I think, on the contrary, that it is 
truth of a proposition that depends on its meaning. I do not intend to 
define truth but I am prepared to claim that: preservation of 
meaning implies preservation of truth. If \wo{Ich bin ein Mann} is true, it 
is because of what it means. If I translate it and get \wo{I am a man} 
then what I got is true {\em provided that the translation preserved the 
meaning}. 

I hope that I said more than a few tautologies. That \wo{Ich bin ein Mann} 
is true because of what it means may be one. Given a static linguistic frame, 
it may be very difficult to say something more precise about the relation 
between truth and meaning. However, our dynamic view enables me to say: 
preservation of meaning implies preservation of truth. This, I hope, at 
least says something.


\section{Summarizing}
\begin{enumerate}
\item The normal situation of a language user is not to stay within a single, 
static linguistic frame but to constantly switch between them or, perhaps, constantly
extend his frame. In general, we call a transition between frames \wo{translation}.
\item Meaning is {\em defined as} an invariant of all translations. It manifests 
itself within a single (current) frame of a language user, but it emerges only through
the process of translation.
\item Since meaning of a proposition is what determines its truth-value, truth, too, 
is invariant under translation.
\end{enumerate}
We have left all technicalities for ``future research'' -- good luck!
\begin{thebibliography}{MM99}\MyLPar
\bibitem[BK86]{Koh} W. Bandler, L. Kohout, ``On the general theory of relational 
  morphisms'', {\em Int.~J.~General Systems}, vol. 13, (1986).
\bibitem[Br93]{Brink} Ch.~Brink, 
  ``Power Structures'', {\em Algebra Universalis}, 30, (1993).
\bibitem[GB83]{Inst} J. Goguen, R. Burstall, ``Introducing Institutions'', in
 {\em Logics of Programs}, LNCS, vol. 164, Springer, (1983).
\bibitem[WB97]{WB} M. Walicki, M. Bia{\l}asik, ``Categories of relational structures'', 
 in {\em Recent Trends in Algebraic Development Techniques}, LNCS, vol. 1376, Springer, 
 (1997).


\end{thebibliography}

\end{document}

\paragraph{Who defines whom?}
