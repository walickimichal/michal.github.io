
\documentclass[10pt,twocolumn,leqno]{article}

\newcounter{EQ}
\newcommand{\equ}[1]{\refstepcounter{EQ}\vspace{.5ex}\par\noindent\ 
    \hfill #1\hfill{(\theEQ)}\\[.5ex]}
\newcommand{\refp}[1]{(\ref{#1})}
\newcommand{\<}{\langle}
\renewcommand{\>}{\rangle}

\newcommand{\MyLPar}{\parsep -.2ex plus.2ex minus.2ex\itemsep\parsep
   \vspace{-\topsep}\vspace{.5ex}}

\hyphenation{be-visst-het ufull-stendig-het kon-fron-ta-sjon be-skri-vel-se}
\hyphenation{re-fe-re-re re-fe-ran-se etter-lig-ne gjel-de gjel-der ob-jekt-et ob-jekt 
 til-felle sys-tem sys-tem-et pro-gram pro-gram-mer-ing set-ning set-ning-en bes-temt
 bes-tem-me sann-het sann-hets-ver-di ver-di nod-ven-dig as-pekt de-ter-mi-nis-tisk
 de-ter-mi-nis-tis-ke eks-emp-el eks-emp-le-ne kunn-skap si-tu-asjon kon-klu-sjon
 Kon-klu-sjon-en spe-sielt ma-te-ma-tikk ma-te-ma-tikk-en}


\title{Mekanisk selv-referanse vs. menneskelig selv-bevissthet}
\author{Micha{\l} Walicki}
\date{{}}
\begin{document}

%\maketitle
\begin{center} 
 {\Large\bf\em Mekanisk selv-referanse vs. menneskelig selv-bevissthet} \\[2ex]
 Micha{\l} Walicki
\end{center}

Problemstillingen har en uhyggelig lukt av ``kunstig intelligens'' i dens 
ureflekterte optimisme, med trykket p{\aa} ``intelligens'' heller enn 
p{\aa} ``kunstig''. 
Prosjektet {\aa} bygge en maskin 
som kunne best{\aa} Turing test
% \footnote{Per kan stille 
% vilk{\aa}rlige sp{\o}rsm{\aa}l til to ``lukker''. Fra 
% hver lukke f{\aa}r han et svar og kan stille nye 
% sp{\o}rsm{\aa}l. Bak en av
% lukkene sitter et menneske mens bak den andre en maskin. Maskinen 
% best{\aa}r Turing test dersom Per ikke er istand til {\aa} 
% si hvilke svar som kommer fra maskinen og hvilke fra menneske. 
% Man har kritisert testens behavioristike karakter og reist tvil om 
% den var tilstrekkelig for {\aa} kalle maskin ``intelligent''.}, 
 er omtrent d{\o}dt.
I dag betyr utrykket enten en kommersiell fiasko eller en 
nisje i datateknologi som utnytter s{\ae}regne datastrukturer og
programmeringsteknikker.

Men man kan l{\ae}re selv fra de mest fatale erfaringer.
% En gang p{\aa} 50-60-tallet,
Det ble p{\aa}pekt at enhver ny innsikt innen ``kunstig intelligens'' 
f{\o}rte til forskyving av grensen for hva som 
mentes {\aa} skille mekanisk og menneskelig tenkning et hakk h{\o}yere -- 
menneskelig intelligens har hele tiden v{\ae}rt det som vi nettopp ikke hadde 
n{\aa}dd. 
 Mens en tilfredstillende definisjon 
av intelligens, for {\aa} ikke snakke om selv-bevissthet, 
 mangler
kan det v{\ae}re opplysende {\aa} sammenligne det udefinerbare med noe 
konkret som vi vet ikke holder m{\aa}l. 

%% Analyse av selv-referanse, tross 
%% uung{\aa}elige forenklinger, er noe alle kompetente personer burde
%% kunne si seg enig 
%% i.\footnote{Siden informatikk er bare en stedatter av matematikk mens 
%% datamaskiner
%% kun et teknologisk uttrykk av informatikk, vil jeg fritt blande maskiner,
%% formelle system og logiske teorier.}
%% Selv-bevissthet er en annen sak og her 
%% m{\aa}tte jeg i st{\o}rre grad ty til mine private meninger.


\section{Jeg lyver --} \vspace*{-1ex}
\hfill{\Large\bf alts{\aa} er jeg utenfor systemet}\\[1ex]
Er det 
sant at jeg lyver, s{\aa} lyver jeg om at jeg lyver, alts{\aa} lyver jeg ikke. 
Snakker jeg ikke sant, dvs. lyver jeg i det jeg sier ``jeg 
lyver'', s{\aa} er det usant at jeg lyver, alts{\aa} snakker jeg 
sant. 
Dette er Russells variant av Epimenides' paradoks (500~f.K.): er setningen
\equ{Denne\ setningen\ er\ usann.
}\label{pa:b}
sann, s{\aa} blir den usann og omvendt.
En selvmotsigelse viser at noen av antakelsene m{\aa} v{\ae}re gale, og 
eneste antakelse vi synes {\aa} ha gjort i dette tilfelle var at en 
setning m{\aa} enten v{\ae}re sann eller usann. 

Slike selv-referanse paradokser ble brukt for {\aa} vise 
umuligheten av forskjellige antakelser og her kommer et par eksempler 
fra dette {\aa}rhundret.
For {\aa} knytte formuleringer i naturlig spr{\aa}k
til mer formelle varianter omformulerer vi \refp{pa:b} slikt:
\equ{$E=$\ \ \ \ Setntningen\ $\<E\>$\ er\ usann.}\label{pa:f}
Det er en subtil men viktig forskjell. 
% mellom \refp{pa:b} og \refp{pa:f} 
%som vil spille en viktig rolle i det f{\o}lgende.
I naturlig spr{\aa}k oppn{\aa}s selv-referanse
 v.hj.a. indeksikalske ord som ``jeg'', ``denne'' hvis 
referanse avhenger av konteksten. 
I den mer formelle \refp{pa:f} 
er selv-referanse oppn{\aa}dd v.hj.a. et nytt navn $E$ som 
betegner hele setningen, samt en spesiell m{\aa}te {\aa} antyde at innenfor 
setningen refererer vi til denne -- 
%vi skriver $\<E\>$ istedenfor ``$E$''.
 notasjonen $\<E\>$ kan ses som et alternativ til ``$E$''.


I siste etappe av QEDs stafett siterte Jan Arne Telle et eksempel av 
l{\o}gner paradokset som viser uavgj{\o}rbarhet 
av stoppeproblemet. Anta at vi kan gi en maskin{\em beskrivelse} som input for en 
maskin (dette er ikke noe annet enn {\aa} programmere en
datamaskin) og at vi har et program ${\sf 
terminerer}\<P\>$ som tar en slik maskinbeskrivelse $\<P\>$ som input og 
alltid gir et korrekt svar {\sf Sant} dersom $P$ terminerer og {\sf 
Usant} 
dersom $P$ ikke terminerer. Da kan vi lage et nytt program:
\equ{ $E=$\ \ \ \ \ $l:$ if ${\sf terminerer}\<E\>$ goto 
$l$;}\label{pa:t}
Sp{\o}r vi om $E$ terminerer, har vi to muligheter. Hvis $E$ 
terminerer, er ${\sf terminerer}\<E\>$ sant og $E$ vil g{\aa} tilbake 
til $l$ og fortsette kj{\o}ring derifra i det uendelige -- dvs. den 
vil ikke terminere. Dersom $E$ ikke terminerer, er ${\sf 
terminerer}\<E\>$ usant, og dens kj{\o}ring vil stoppe umiddelbart -- 
 $E$ vil terminere.

Kanskje det mest kjente eksemplet
% av l{\o}gnerparadokset som ristet matematisk verden og 
%  f{\o}rte til oppl{\o}sning av et par matematisk-filosofiske skoler 
er G\"{o}dels ufullstendighetsteorem. 
Anta at vi har en aksiomatisk teori $T$ (et spr{\aa}k, identifiserbare aksiomer,
 inferensregler) som er motsigelsesfri og tillater oss 
{\aa} snakke om matematiske st{\o}rrelser -- i det minste om 
naturlige tall.
Disse, heller rimelige, antakelsene
tillater oss {\aa} gj{\o}re to ting. For det f{\o}rste, kan en
 vilk{\aa}rlig p{\aa}stand 
$P$ {\em kodes} som et objekt -- et naturlig tall -- $\<P\>$, 
som s{\aa} kan omtales av 
andre p{\aa}stander i $T$. For det andre, kan vi utrykke 
et predikat ${\sf bevisbar}_{T}\<P\>$ som er sant 
dersom tallet $\<P\>$ koder en p{\aa}stand $P$ som
kan bevises i $T$. 
G\"{o}del lager s{\aa} en setning:
\equ{$E=$\ \ \ ikke-${\sf bevisbar}_{T}\<E\>$}\label{pa:G}
Er ${\sf bevisbar}_{T}\<E\>$ sant, dvs. kan $E$ bevises i $T$, skal $E$ 
v{\ae}re sann, dvs. det skal 
v{\ae}re sant at $E$ ikke kan bevises i $T$.\footnote{Bland viktige 
detaljer som ble feid under teppe, er minimumskravet til enhver teori at 
det som kan bevises er ``sant''.}
Dermed m{\aa} ikke-${\sf bevisbar}_{T}\<E\>$ v{\ae}re sann,
dvs. $E$ er sann men ikke-bevisbar. Konklusjonen 
viser umuligheten av {\aa} lage en teori som er 
motsigelsesfri og som kan bevise alle sanne p{\aa}stander. Spesielt, er 
det umulig {\aa} formalisere {\em hele} matematikken i et konsistent
logisk system. \\[1.5ex]
%
%\subsubsection*
{\bf Selv-referanse} %\label{sub:sr}

La oss n{\aa} se litt p{\aa} det som er felles for de forskjellige 
eksemplene. Et formelt system har to klart adskilte aspekter: et domene av objekter 
som kan omtales (input for et program, naturlige tall) samt et 
spr{\aa}k som brukes for {\aa} omtale disse objektene (programmer, 
p{\aa}stander i en teori). Objektene er det som kan {\em refereres 
til} mens p{\aa}stander er det som {\em refererer} til objektene. 
I eksemplene har vi skilt mellom disse ved {\aa} skrive $E$ for en p{\aa}stand 
og $\<E\>$ for et omtalt objekt. 
Dermed er det ingen forskjell mellom referansen til objektet 5 i p{\aa}standen
``$P=$\ ikke ${\sf bevisbar}_{T}5$'' og til objektet $\<E\>$ i 
p{\aa}standen ``$E=$\ ikke ${\sf bevisbar}_{T}\<E\>$''. Begge sier at 
objektet enten ikke koder en p{\aa}stand eller koder en p{\aa}stand 
som ikke kan bevises. 

Selv-referanse oppn{\aa}es i to steg. For det 
f{\o}rste, systemets spr{\aa}k m{\aa} v{\ae}re uttryksfult nok for 
{\aa} kunne {\em kode} sine egne p{\aa}stander som objekter. 
Vanligvis bruker vi anf{\o}rselstegn for dette form{\aa}let:
%setningen -- ``Per'' har tre bokstaver. --
``Per'' refererer til et syntaktisk objekt, ordet ``Per''
og ikke til personen Per. Likes{\aa} er $\<E\>$ et tall, et input 
til et program, men ikke en p{\aa}stand. Spesielt {\em brukes} 
$\<E\>$ {\em som et objekt} referert til av setningen $E$.

Det andre steget er mer innviklet fordi det skjer p{\aa} et 
metaniv{\aa}. I ingen av v{\aa}re system gjelder 
likheten
\equ{$E\ =\ \<E\>$}\label{eq:sr}
Den kan ikke en gang uttrykkes -- $\<E\>$ er et objekt og
kan omtales av  p{\aa}stander men  $E$ er en p{\aa}stand 
som kan omtales kun p{\aa} metaniv{\aa} (${\sf bevisbar}_{T}E$ 
gir ingen mening!) Vi har trivielt likheten 
\equ{$\<E\>=\<E\>$}\label{eq:gn}
men det er 
\refp{eq:sr} som ville utrykke ``ekte'' selv-referanse {\em innenfor} systemet.

Selv-refererende karakter av $E$ kommer 
til syne utenfra $E$ -- i beste fall for resten av
systemet som {\em koder} $E$ som $\<E\>$, manipulerer det siste,
og etablerer likheten \refp{eq:gn}. I verste fall er den
sett kun  utenfra systemet av en som kan {\em 
tolke} likheten \refp{eq:gn} som \refp{eq:sr}, samt 
{\em tolke} resultater av en manipulasjon av $\<E\>$ som gjeldende for $E$.
Vi har f{\o}lgende hovedpunkter ang{\aa}ende 
selv-referanse i et formelt system: 
\hfill{\refstepcounter{EQ}(\theEQ)\label{no:sr}}
\begin{itemize}\MyLPar
\item[1.] Det er et skarpt skille mellom det som refererer 
%(program, formler) 
og det som refereres til. 
\item[1a.] Formler m{\aa} kunne {\em kodes} som objekter.
\item[1b.] Selv-referanse er dermed kun et spesielt tilfelle av 
referanse. 
% der formler refererer til objekter.
\item[2.] Det finnes ikke selv-refererende system -- kun system der 
 det er mulig {\aa} konstruere selv-refererende formler. 
\item[2a.] Identiteten av det refererende og det refererte sees kun 
utenfra selve referanse-relasjonen.
\end{itemize}  
%
%\subsubsection*
{\bf Usikkerhet og ufullstendighet}

Maskiner kan h{\aa}ndtere  veldefinerte problemer og 
meget avgrensede omr{\aa}der men er hjelpesl{\o}se 
 ovenfor problemer som de ikke p{\aa} forh{\aa}nd var 
programmert til {\aa} l{\o}se. Formell, mekanisk betraktning av uforutsigbarhet 
h{\o}res ut som {\em contradicto in adjecto} men det er nettopp objektet
av intens studie
innen felt som ikke-monoton, fuzzy og modal logikk,
ikke-deterministiske beregninger,
maskinl{\ae}ring, nevrale nettverk, genetiske og on-line algoritmer 
m.m. Felles for disse er fors{\o}k p{\aa} {\aa} definere algoritmer
for {\aa} handle ut fra ufullstendig informasjon samt reagere til 
uforutsatte situasjoner -- akkurat som vi handler hele tiden ut fra 
ufullstendig kunnskap om verden.

V{\aa}re eksempler antyder at ufullstendighet er 
et n{\o}dvendig aspekt ikke bare av v{\aa}r kunnskap men ogs{\aa} av selv-referanse.
Alle ledet til en form for ufullstendighet av 
systemet innenfor hvilket den oppsto. 
\refp{pa:b} 
%``Denne setningen er usann'' 
viser at noen setninger ikke har en bestemt sannhetsverdi;
% som vi ikke kan si er sanne eller usanne; 
det er umulig {\aa} avgj{\o}re om et program vil 
terminere fordi vi kan lage selv-refererende 
programmer \refp{pa:t}; en teori som er istand til {\aa} konstruere 
selv-refererende p{\aa}stander vil inneholde sanne setninger som den 
ikke kan bevise \refp{pa:G}. Alts{\aa}, litt spekulativt:
\equ{selv-referanse\ \ $\Rightarrow$\ \ ufullstendighet}\label{eq:ufu}
Med denne observasjonen forlater vi matematiske fakta
 og g{\aa}r over til ``filosofisk spekulasjon''.

\section{N{\aa}r er jeg meg selv bevisst?}
I daglig spr{\aa}kbruk betyr ``(selv-)bevissthet'' omtrent det samme som 
``(selv-)refleksjon''. I det jeg betrakter et maleri er jeg ikke bevisst 
(refleksivt) klar over at jeg gj{\o}r det -- jeg bare betrakter det. 
F{\o}rst n{\aa}r noen sp{\o}r meg ``Hva gj{\o}r du?'' vil jeg reflektere over
det {\em at} jeg betrakter dette maleriet. 
Dette {\aa}pner for et uendelig regress karakteristisk for 
selv-refleksjon: jeg betrakter et maleri, jeg tenker {\em at} jeg betrakter et 
maleri, jeg tenker {\em at} jeg tenker {\em at} jeg betrakter... I hvert 
tilfelle, er akten `jeg tenker {\em at}...' noe annet enn 
% det som st{\aa}r i
``...''. Jeg som tenker {\em at} jeg betrakter et maleri er `en annen' enn jeg 
som betrakter fordi i det jeg tenker {\em at} jeg betrakter, 
s{\aa} betrakter jeg ikke lenger men tenker {\em at}... Selv-refleksjon er 
alltid rettet mot noe som enten var eller kommer til {\aa} bli men ikke er 
tilstede her-og-n{\aa}.
%, den kommer alltid for sent 
% for {\aa} v{\ae}re med p{\aa} det den fors{\o}ker {\aa} fange.

Filosofer har lenge brukt et skille mellom slik selv-refleksiv
og umiddelbar selv-bevissthet som vi n{\aa} m{\aa} si et par ord 
om.\footnote{Skillet g{\aa}r tilbake til Middelalderen men 
analyser av selv-bevissthet er spesielt sentrale i tysk 
idealisme med Kant, Hegel of Fichte, samt i fenomenologi 
med Husserl og Sartre.} Poenget er at selv-refleksjon er umulig uten 
umiddelbar selv-bevissthet. Jeg kan besvare ``Hva 
gj{\o}r du?'' med opplagt ``Jeg betrakter et maleri'' bare fordi i det jeg 
betrakter uten {\aa} reflektere over det, er jeg umiddelbar, implisitt bevisst hva 
jeg gj{\o}r -- jeg bare fokuserer ikke p{\aa} {\em at} jeg gj{\o}r det.

I selv-refleksjon skiller man lett mellom refleksjonens objekt -- 
meg selv som betrakter (eller har nettopp betraktet) et maleri, fra refleksjonens
subjekt -- jeg som n{\aa} tenker {\em at}... Dette skillet minner
om det vi har sett i selv-referanse: setningen $E$ som refererer til 
sin egen koding $\<E\>$ og som trenger en utenforlliggende instans (systemet, og her 
en ny refleksjonsakt) for {\aa} etablere identiteten av begge \refp{eq:sr}.
 
Analyser av umiddelbar selv-bevissthet 
avsl{\o}rer en mye mer intim sammenheng. En av fenomenologiens sentrale 
teser lyder 
\equ{{\em{Bevissthet er n{\o}dvendigvis bevissthet {\em om noe}.}}}\label{eq:be}
Dette ``om noe'' markerer et brudd med kartesisk dualisme der 
bevissthet tenkes som en isolert, selv\-sten\-dig boks som p{\aa} en eller 
annen mystisk vis konfronteres med ytre objekt og verden. If{\o}lge 
\refp{eq:be} er det {\aa} ha et eller annet objekt et essensielt aspekt 
ved bevisstheten -- {\aa} tenke bevisstheten i seg selv, uten et objekt er 
feil, om overhodet mulig.

Maleriet er mitt objekt og i det jeg betrakter det er jeg meg dette 
objektet (umiddelbart) bevisst. Men hva vil det si at jeg er meg et objekt 
bevisst? Det vil si at jeg har dette objektet foran meg, at jeg vet at jeg 
betrakter et objekt som er der ute, p{\aa} veggen. 
Jeg vet at objektet er noe annet, at det ikke er {\em meg selv}.
Det {\aa} v{\ae}re bevisst, det {\aa} ha et objekt er det {\aa} vite 
at det er noe {\em annet enn meg selv} -- alts{\aa} {\aa} v{\ae}re umiddelbart 
seg selv bevisst! 
Det at jeg oppfatter maleriet som noe ytre, noe annet enn meg selv betyr
at jeg setter det opp mot meg selv, spesielt, at jeg er umiddelbart 
meg selv bevisst. Og p{\aa} den andre siden, det at jeg er meg selv bevisst 
betyr at jeg oppfatter meg selv som konfrontert med noe 
ytre, noe annet som ikke er meg selv.

G{\aa}r vi analytisk til verks kan vi skille mellom min 
selv-bevissthet, min
bevissthet om et objekt og selve objektet. Men fenomenologisk analyse viser 
at alle disse elementene er uadskillelige
og kan ikke betraktes som om de kunne inntreffe uavhengig av 
hverandre. 
% Bevisstheten er rettethet mot et objekt, som inneb{\ae}rer 
% umiddelbar kjennskap til forskjellen mellom dette objektet og seg selv, dvs.
% umiddelbar selv-bevissthet. Og p{\aa} den andre siden, 
% betyr ikke selv-bevisstheten noe annet enn bevisstheten om et eller annet som 
% ikke er meg selv.
%
Med litt god vilje kan vi knytte dette til \refp{eq:ufu}: 
\equ{selv-bevissthet\ \ $\Rightarrow$\ \ ytre objekt}
%
Implikasjonen betyr det vi nettopp sa: selv-be\-visst\-het inneb{\ae}rer bevissthet
om noe annet, en 
konfrontasjon med et {\em ytre} objekt. Den kan ikke
l{\o}s\-ri\-ves fra ytre verden som i denne
sammenhengen kan bety uforutsigbarhet, overraskelse, ufullstendighet
av bevissthetens kontrol og kunnskap.

Men her slutter analogien -- ingen av egenskapene
til mekanisk selv-referanse listet under \refp{no:sr} finnes igjen i 
selv-bevissthetens struktur: 
\hfill{\refstepcounter{EQ}(\theEQ)\label{no:sb}}
\begin{itemize}\MyLPar
\item[1.] Det som refererer (bevissthet) og det som refereres til 
(objekter) er n{\o}dvendigvis knyttet sammen og kan skilles kun i teoretisk 
refleksjon.
\item[1a.] Selv-bevissthet er umiddelbar og krever ingen {\em koding} 
eller gjenkjennelse.
\item[1b.] Selv-bevissthet er ikke et spesielt tilfelle av bevissthet 
 -- begge er aspekter av samme struktur.
\item[2.] Selv-bevissthet er ikke en spesiell akt innenfor et st{\o}rre 
system (mitt liv?) men utgj{\o}r min vedvarende v{\ae}rem{\aa}te.
\item[2a.] Selv'et av selv-bevisstheten er ikke i verden for {\em s{\aa}} {\aa} 
innse 
at det er identisk med seg selv. Selv-identiteten er en umiddelbar 
relasjon, en umiddelbar erkjennelse av bevisstheten konfrontert med 
eksternalitet.
\end{itemize}
Legg merke til at ingen av disse observasjonene gjelder for selv-refleksjon 
for hvilken beskrivelse av selv-referanse \refp{no:sr} 
er mye mer adekvat.

\section{Reduksjonisme ?}
Motsetningene mellom \refp{no:sb} og \refp{no:sr} kunne ikke v{\ae}rt st{\o}rre.
Men det er p{\aa} ingen m{\aa}te vist eller
bevist at det vi opplever i v{\aa}r selv-bevissthet ikke 
kan reduseres til noen element{\ae}re 
prosesser, kan hende, noen innviklede self-refererende strukturer.
Eddingtons forestilling om at jeg som betrakter et maleri og selve maleriet
bare er en illusjon, en manifestasjon av et kaotisk spill av atomer og 
element{\ae}re partikler har plaget mange. Mer 
schizofrent kan et sunt menneskes verdensbilde vanskelig bli men hva er 
man ikke villig til {\aa} akseptere i den vitenskapelige rasjonalitetens navn?

Forklaring, og spesielt en vitenskapelig forklaring, ender typisk med 
reduksjon der opplevelse erstattes med mekanisme. Ogs{\aa} i dette tilfelle 
st{\aa}r man fritt til 
{\aa} tro at en slik reduksjon er mulig.\footnote{For 
interesserte kan jeg anbefale bl.a. meget 
oppfinsomme teorier av D.Denett framlagt f.eks. i  b{\o}ker 
``Consciousness Explained'' eller ``Kinds of Minds''.} 
Men det er vanskelig {\aa} ane den minste mulighet til {\aa} etterligne 
menneskelig tenkem{\aa}te v.hj.a. n{\aa}v{\ae}rende (data)teknologi. 
Selv-bevissthet er ikke bare selv-bevissthet men ogs{\aa} bevissthet,
tenkning er ikke bare tenkning men tenkning om noen
bestemt virkelighet, psykologi er ikke bare psykologi men ogs{\aa} biologi. 
Ting som i prinsippet kan skilles 
-- kropp og sjel, selv-bevissthet og tenkning, tenkning og persepsjon -- 
er knyttet sammen mye n{\ae}rmere enn vi pleide {\aa} tro.
Mest sannsynlig, for at en maskin skal v{\ae}re selv-bevisst og 
kunne tenke som oss, m{\aa} 
den f{\o}rst kunne forholde seg til den samme virkeligheten som vi gj{\o}r. 
I dag er maskiners `virkelighet', deres `ytre verden' kun en liten 
redusert del av v{\aa}r virkelighet. At den, i en overskuelig framtid, kunne bli noe
kvalitativt mer er en dr{\o}m 
som bare mest optimistiske reduksjonister gidder {\aa} dr{\o}mme.


\end{document}