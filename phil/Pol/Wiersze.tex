\documentclass{article} %[a4paper]{article}
\newenvironment{wiersz}[1]
	{\begin{verse} \hspace*{-1em}{\bf #1}}
	{\end{verse} \begin{center}$\bullet\ \bullet\ \bullet$\end{center}}
\newenvironment{wierszd}[2]
	{\begin{verse} \hspace*{-1em}{\bf #1}\hfill{\small{(#2)}}\\[.5ex] }
	{\end{verse} \begin{center}$\bullet\ \bullet\ \bullet$\end{center}}
\newcommand{\chap}[1]{\newpage\section*{#1}}

\newcommand{\old}[1]{}

\usepackage[polish]{babel}

%\newcommand{\e}{\c{e}}
%\renewcommand{\a}{\c{a}}
\newcommand{\e}{\eob}
\renewcommand{\a}{\aob}

\renewcommand{\o}{\'{o}}
\newcommand{\z}{\'{z}}
\newcommand{\n}{\'{n}}
\newcommand{\s}{\'{s}}
\renewcommand{\S}{\'{S}}

\newcommand{\ci}{\'{c}}
\newcommand{\Z}{\'{Z}}
%\newcommand{\rz}{\.{z}}
\newcommand{\rz}{\zkb}
\newcommand{\Rz}{\Zkb}

\begin{document}

%\section*{}\vspace*{-3ex}
\chap{Czas rzeszywisto{\s}ci} 
%
\begin{wierszd} {Powierzchnie rzeczy}{1983}
Czego szuka szalbiercza niepewno{\s}{\ci} w wieku prawdy,\\
Wciele{\n} w trwanie rzeczy? \\
Czy\.{z} huk bomb, p{\e}kaj{\a}cych czerep{\o}w, \\
Krzyki {\s}mierci, \.{z}ar ca{\l}opale{\n}, \\
Nie wysuszy{\l}y {\z}r{\o}de{\l} {\l}ez, kt{\o}rymi B{\o}g \\
Przelewa{\l} serca? Odchodz{\a} spojrzenia, \\
Kobiety ze wzgardzon{\a} mi{\l}o{\s}ci{\a}; \\
Wiemy -- jeste{\s}my poza niewinno{\s}ci{\a}. 

Bracia z przypadku, spytajcie filozof{\o}w. \\
Zbyt wiele rzeczy nazwano ma{\l}ymi,  \\
Zbyt d{\l}ugo pr{\o}bowali{\s}my patrze{\ci} w niebo, \\
Wyprowadza{\ci} kszta{\l}ty z chmur. Jest, co musi by{\ci}. \\
Jak kamieniarze, ze szlachetnego marmuru \\
Wykuwajmy ko{\l}yski dom{\o}w -- by ros{\l}y \\
W ko{\s}cio{\l}y ziemi, wie\.{z}e {\s}wiadomo{\s}ci! \\
Lecz by je ogrza{\ci}, czy starczy wieczno{\s}ci? 

Fa{\l}szywe pi{\e}kno s{\l}{\o}w b{\o}lem jest tym wi{\e}kszym, \\
\.{Z}e od b{\o}lu ucieka. \\
Czy to z cieniem szacunku patrz{\a} na nas \\
Rzeczy, czy z ogniem nienasycenia?\\
A gdy wygasaj{\a}? Nie ciche ptak{\o}w umieranie \\
Lecz {\s}mier{\ci}, jak kamie{\n} obcy, o kt{\o}rej te\.{z} wiemy -- \\
Zwyci{\e}\.{z}aj{\a}cy, i znowu zwyci{\e}\.{z}eni. 

St{\e}\.{z}a{\l}a bry{\l}a \.{z}elaza wyj{\e}ta z ognia w rzeczwisto{\s}{\ci}, \\
Ostatni{\a} nami{\e}tno{\s}{\ci}, narkotyk. \\
Niech sprosta naszej sile, jak obc{\e}gi kowala.  \\
Atlasy milcz{\a}ce {\s}wiatem, \\
Ludzie zapomniani w nieodwracalnym \\
Odpowiadaj{\a} tylko za siebie. \\
Serca, jak powierzchnie rzeczy si{\e} {\s}cieraj{\a} -- \\
Zdumione -- i poza w{\l}asny niebyt trwaj{\a}. 
\vspace*{-1ex}\end{wierszd}\vspace*{-3ex}
%

\newpage
\begin{wierszd} {Boja{\z}{\n} i dr\.{z}enie}{198?}
T{\l}um zapatrzony w wyroki m{\e}drc{\o}w, \\ 
Grzej{\a}cy si{\e} przy ogniu o{\l}tarzy. \\ 
Jak uci{\e}ty w niesko{\n}czono{\s}ci wzrok schizofrenika \\ 
Goni{\a}cy wierzcho{\l}ek {\s}wiecy, \\ 
Przyt{\l}oczony ci{\e}\.{z}arem ko{\s}cielnych pie{\s}ni, \\ 
Kt{\o}re ze sklepienia wal{\a} si{\e} \\ 
Na g{\l}owy wiejskich prostaczk{\o}w, \\ 
Kobiecin w haftowanych chustkach, \\ 
Odnosz{\a}cych na r{\e}kach modlitwy do nieba. \\ 
Monolit {\s}wi{\a}tyni, wobec kt{\o}rej \\ 
Jedni daj{\a} pewno{\s}{\ci} drugim, \\ 
Kt{\o}rej nigdy nie przenika zak{\l}{\o}cenie, fa{\l}szywy ton, \\ 
Do kt{\o}rej nigdy nie dociera \\ 
Szelest li{\s}ci poruszanych niewidocznym cia{\l}em w{\e}\.{z}a, \\ 
Szmer, ruch dziwny pytaj{\a}cych oczu i serc, \\ 
Niepok{\o}j i zagubienie, szamoc{\a}ce si{\e} pod powierzchni{\a} ceremoni, \\ 
Jak nieodkryta istota rzeczy.

\end{wierszd}

\begin{wierszd}{}{1999}
Czym by{\l} j{\e}zyk? Jaki? \\
Gdy prz{\a}d{\l} s{\l}{\o}w poca{\l}unki\\
Na niesko{\n}czonej nici ... Z g{\l}{\e}bi,\\
Z nadmiaru serca jest bogactwo s{\l}{\o}w.

A z obfito{\s}ci s{\l}{\o}w -- niewys{\l}owionych -- \\
Rodzi si{\e} czyn, jak poca{\l}unek,\\
I stwarza to, co ju\.{z} by{\l}o.\\
Lecz gdy zamiera -- o czym {\s}wiadczy?

Gdzie s{\l}owa milkn{\a}, ziemia si{\e} ko{\n}czy, \\
A by \.{z}eglowa{\ci}, sk{\a}d wzi{\a}{\ci} {\l}{\o}d{\z}?\\
Czym by{\l} j{\e}zyk -- \.{z}e mog{\l}em zapomnie{\ci}?\\
Czy mo\.{z}na zapomnie{\ci} to, co by{\l}o?

Czy, co raz by{\l}o, jest na wieczno{\s}{\ci}?\\
Jak pier{\s}cie{\n} serca zatopiony w morzu,\\
Po{\s}r{\o}d wiecznych kr{\e}g{\o}w, fal\\
Co z czasem milkn{\a}, jak s{\l}owa ...

Kawa{\l}ki {\s}wiata, s{\l}owa, zapomniane\\
W morze bez skargi, jak zrozumienie\\
Na niesko{\n}czonej nici, milcz{\a}c\\
Przez u{\s}miechy fal wiecznie ... cicho i wiecznie.
\end{wierszd}
%
%%%% uspokojenie....
\begin{wierszd} {Przyzwyczajenie}{198?}
Do ciszy jezior, \\ 
Szumu drzew wierzcho{\l}k{\o}w, \\ 
I d{\l}ugodziobych czapli \\ 
Po wodzie brodz{\a}cych \\ 
W cieniu gwiazd \ldots \\ 
\ldots do tego si{\e} przyzwyczaj{\e}. \\ 
I dym{\o}w miast \\ 
Rozdrapywanych chmurami wie\.{z}owc{\o}w, \\ 
I samochod{\o}w za oknami, \\ 
W lampami gwiezdnym niebie \ldots \\ 
\ldots do tego si{\e} jeszcze przyzwyczaj{\e}. \\ 
Ale splecionych d{\l}oni \\ 
Wspieraj{\a}cych twarz wpatrzon{\a} w siebie \ldots \\ 
W pokoju, z kt{\o}rego mroku tylko oczy {\s}wiec{\a} \\ 
W podmiejskim barze, czy luksusowym hotelu. \\ 
Cho{\ci}by od tysi{\a}cleci jednym wpatrzeniem \\ 
Znajduje siebie - dzi{\s} tylko, \\ 
Mi{\e}dzy jutro, a wczoraj \ldots \\ 
\ldots I nie staje si{\e} to przyzwyczajeniem. 

\end{wierszd}
%
%\newpage
\begin{wierszd} {Wspomnienia}{198?}
{\S}lady zaszele{\s}ci{\l}y - znajomego kroku, \\ 
Jak martwy smutek li{\s}ci zapoznanych. \\ 
B{\l}ysk oczu zgasi{\l} {\s}wiat{\l}o w mroku \\ 
Niezgrabnych wspomnie{\n} w ciszy popl{\a}tanych. 

	Na wieczno{\s}{\ci}, kt{\o}ra przecie\.{z} \.{z}yciem kusi \\ 
	Niewydarzonych zdarze{\n} rocznice obchodzi{\ci}? \\ 
	Dlaczego wzrok spojrzeniem zabi{\ci} musi \\ 
	To wszystko, co pragn{\a}{\l}by odrodzi{\ci}? 

Dzie{\n} mnie wyprzedzi{\l}. Wieko nieba \\ 
Zamyka w czasie moje niebycie. \\ 
Komu, i czemu zaprzeczy{\ci} trzeba \\ 
By zn{\o}w mo\.{z}na by{\l}o rozpocz{\a}{\ci} \.{z}ycie? 

	Rozwia{\l} obrazy z mg{\l}y utkane brzask \\ 
	I przesta{\l} kusi{\ci} ich spe{\l}nieniem. \\ 
	Wzrokiem swym dzie{\n} wy{\l}uska{\l} z gwiazd \\ 
	I nadziej{\e}, kt{\o}ra te\.{z} jest wspomnieniem.
\end{wierszd}
%

%
 \newpage

\begin{wierszd}{}{1983}

Wzrok wpatrzony we mg{\l}{\e}. \\ 
Przed po{\l}udniem deszcz obrazy rozma\.{z}e \\ 
I pami{\e}{\ci}. \\ 
{\L}agodnym szelestem pokaleczy twarze \\ 
Nim zastygn{\a}, jak nieme zwierz{\e}ta, \\ 
Kt{\o}re nie zna{\l}y {\s}mierci, \\ 
A straci{\l}y zmys{\l}y z b{\o}lu trwa{\l}ego, \\ 
Bez {\z}r{\o}d{\l}a. 

I niebo sp{\l}ywa {\s}cie\.{z}kami deszczu  \\ 
W gar{\s}cie dolin. Obleka poszukiwaczy imion \\ 
Zw{\a}tpieniem. \\ 
Jakby bogowie znu\.{z}eni {\s}piewem \\ 
Z{\l}{\a}czyli r{\o}\.{z}ne kr{\o}lestwa \\ 
Wi{\e}zi{\a} silniejsz{\a} ni\.{z} w{\e}dr{\o}wka samotno{\s}ci \\ 
Bez pocz{\a}tku.  

A\.{z} cisza poch{\l}ania oddechy kwiat{\o}w, \\ 
{\S}pi{\a}cych kobiet uspokojonych mi{\l}o{\s}ci{\a} \\ 
I noc{\a}. \\ 
U{\s}miech ta{\n}czy po nagim ciele \\ 
Do bezg{\l}o{\s}nej muzyki ksi{\e}\.{z}yca \\ 
I schodzi w g{\l}{\e}bi{\e} milczenia, \\ 
Kt{\o}ra ucisza nawet krzyk duszy, \\ 
Bezgwiezdny. 

Co da miar{\e} przesz{\l}o{\s}ci \\ 
I z{\l}agodzi cierpienie, \\ 
Je{\s}li B{\o}g zdj{\a}{\l} zas{\l}on{\e} mi{\l}o{\s}ci \\ 
Ze {\s}wiata, i marzenie? 

\end{wierszd}

%\newpage
\begin{wierszd} {Niewolnicy}{1981}
Dni pop{\e}ka{\l}y w{\s}r{\o}d trz{\e}sienia ziemi, \\ 
Kt{\o}re po{\l}kn{\e}{\l}y ko{\s}cio{\l}y, rado{\s}{\ci} rze{\z}bionych cherubink{\o}w \\ 
Powieka morza zakry{\l}a tajemnic{\e} pocz{\a}tku, \\ 
Przymykaj{\a}c oczy Boga usypiaj{\a}cym szumem. 
 
	W podwodnych katedrach niewolnicy \\ 
	Sp{\e}dzaj{\a} stulecia samotno{\s}ci \\ 
	Poszukuj{\a}c w{\s}r{\o}d skarb{\o}w klucza, \\ 
	Kt{\o}rego tylko oni mog{\a} u\.{z}y{\ci}. 

A noc{\a} tylko B{\o}g patrzy {\l}agodnym okiem ksi{\e}\.{z}yca \\ 
Na ich rany migoc{\a}ce po o{\l}owianych wrotach fal \\ 
I wspomina ciep{\l}ym u{\s}miechem wiatru \\ 
Trz{\e}sienia ziemi, kt{\o}rych nigdy nie by{\l}o.

\end{wierszd}

\begin{wierszd}{}{2000}
%Jak ranne ptaki, w b{\o}lu zagubione,\\
Zranione ptaki, w b{\o}lu zagubione,\\
Wznosz{\a} ku niebu skrzydlate spojrzenia,\\
By w samotno{\s}ci niebem utulonej\\
Odnale{\z}{\ci} drog{\e} -- drog{\e} ukojenia?

W otwartym lustrze nieba, cisz{\a} czystym,\\
Szukaj{\a} odbi{\ci} -- ze {\s}wietln{\a} osnow{\a} -- \\
Rzeczy powykrzywianych, nieprzej\.{z}ystych,\\
Kt{\o}re przesta{\l}y si{\e} zgadza{\ci} ze sob{\a}.

I lec{\a} wy\.{z}ej, ku kra{\n}com wiatru, w gwiazd\\
Samotno{\s}{\ci}, co cierpieniu daje miar{\e},\\
Przeszukuj{\a}c puste szlaki, raz po raz, \\
Jakby ni{\ci}mi cierpienia tka{\l}y wiar{\e}.

Tam, hen, na dnie ciszy, gdzie si{\e} ko{\n}czy czas,\\
W oddali, w g{\l}{\e}bi ostatniego tchnienia,\\
Nim martwym deszczem spadn{\a} na pola, w las,\\
Dzi{\e}kuj{\a} za b{\o}l -- jak dow{\o}d istnienia.
\end{wierszd}

\chap{Uspokojenie}\vspace*{-3ex}

\begin{wierszd}{}{1983}
Zanurzony by{\l} w parskanie i dziki t{\e}tent koni, \\  
W oszala{\l}y kaszel suchotniczej ziemi. \\ 
Rozwiane ogony smaga{\l}y go po twarzy, \\ 
Wplata{\l}y si{\e} we w{\l}osy i wyrywa{\l}y je z kawa{\l}kami sk{\o}ry. \\ 
Wzniecony kurz opada{\l} w szczeliny kroplami krwi, \\ 
{\L}okcie i kolana otarte mia{\l} tward{\a} sier{\s}ci{\a} przewalaj{\a}cej si{\e} fali - \\ 
Fali zad{\o}w, grzbiet{\o}w i {\l}b{\o}w, \\ 
Fali za{\s}lepionej ob{\l}{\a}ka{\n}czym p{\e}dem cia{\l}. \\ 
.  .  .  \\ 
 I jeszcze przez chwil{\e} zwielokrotniony huk wype{\l}nia{\l} mu czaszk{\e}, \\ 
Mimo \.{z}e odg{\l}os powtarza{\l} si{\e} ju\.{z} tylko s{\l}abn{\a}cym echem, \\ 
{\S}wiadomy swojej bliskiej {\s}mierci. \\  
Rozwiane grzywy wtopi{\l}y si{\e} w s{\l}upy drgaj{\a}cego powietrza i py{\l}u. \\ 
Sta{\l} oszo{\l}omiony, pr{\o}buj{\a}c zatrzyma{\ci} ogrom zdumienia ... 

A teraz zag{\l}{\e}bia si{\e} w wilgotn{\a} ziele{\n} lasu, \\ 
Wdycha szmer wody i ptasie g{\l}osy, \\ 
I w stonowanych promieniach s{\l}o{\n}ca czuje puls, czuje \.{z}ycie, \\ 
Mimo \.{z}e wko{\l}o nie rusza si{\e} \.{z}adne zwierz{\e}.

\end{wierszd}

\begin{wierszd}{}{1981}

Deszcze spragnione sp{\e}kanej ziemi, \\ 
Wiatry zg{\l}odnia{\l}e spokojnych w{\l}os{\o}w, \\ 
I struna dr\.{z}{\a}ca w ruszeniu cieni, \\ 
T{\e}tent kaszl{\a}cy w rozgwarze g{\l}os{\o}w. 

	Eksplozja spok{\o}j przyogarnia{\l}a \\ 
	Ko{\s}cio{\l}a w palc{\o}w swych rozchylenie, \\ 
	Huk nadd{\z}wi{\e}kowc{\o}w, eskadra ca{\l}a \\ 
	Pomi{\e}dzy s{\l}owa, w ust rozchylenie. 
 
Zamknij bicie op{\e}tanych dzwon{\o}w, \\ 
Zamknij wiatry odwieczystej burzy \\ 
W sen g{\l}{\e}boki, r{\o}wny, m{\e}\.{z}czyzn nocnej str{\o}\.{z}y, \\ 
W splocie ramion, w pi{\a}stce dziecka, w ciszy zgon{\o}w. 

	Krzyk, jak jedwab cienki z szaty umierania, \\ 
	Sploty {\l}ez w rozpaczy wyrywane z dna \\ 
	{\S}wi{\a}tyni zanikania, wydumania \\ 
	W rozchylenie jasnych oczu, w {\s}wiat{\l}o dnia.

\end{wierszd}

\newpage
\begin{wierszd}{}{1983}
Niepokoj{\a}ce wirowanie  \\ 
Garncarskich k{\o}{\l} \\ 
Jakgdyby trwa{\l}o, p{\o}ki nie stanie, \\ 
Ruch st{\o}p \.{z}ylastych w g{\o}r{\e}, i w d{\o}{\l}. 
 
	Jak wszyscy, kt{\o}rzy potrafi{\a} zdradza{\ci}, \\ 
	Zerwane wi{\e}zy i rzeczy niechciane, \\ 
	Jak z{\l}o, co zwyk{\l}o samotnie chadza{\ci} \\ 
	Szalej{\a} ko{\l}a zbuntowane, 

Lecz zwyk{\l}ym ruchem si{\e} palce wczepiaj{\a} \\ 
W glin{\e}, i zwyk{\l}e si{\e} wczepia spojrzenie, \\ 
I krople potu na ko{\l}a padaj{\a}, \\ 
I z potem razem - uspokojenie.

\end{wierszd}

\begin{wierszd}{}{1999}
Ogarni{\e}ty przez morze, ci{\e}{\rz}kie jak przestrze{\n},\\
Milcz{\a}ce ci{\e}{\rz}arem g{\l}{\e}bi, z kt{\o}rej niepewne fale\\
Pr{\o}buj{\a} wyszumie{\ci} ukryte tajemnice,\\
Obudzi{\ci} potwora z wiecznego snu...

Ogarni{\e}ty przez si{\l}{\e} skupion{\a} jak wyczekiwanie\\
Dra{\rz}ni {\l}odzi{\a} powierzchni{\e}, ale przeczuwa wyroki,\\
Niepokoi fale, od brzegu do brzegu, nigdy w g{\l}{\a}b,\\
Czekaj{\a}c wybuchu, kt{\o}ry oby nie nast{\a}pi{\l}.

Ogarni{\e}ty tchnieniem oswojonego zwierz{\e}cia,\\
G{\l}aszcze sieciami cisz{\e} szukaj{\a}c skarb{\o}w\\
Lecz wie, {\rz}e przypadaj{\a}ca mu cz{\a}stka\\
Jest tylko darem obcej si{\l}y -- {\l}agodnej, niepokoj{\a}cej.
\end{wierszd}

\begin{wierszd} {Dwie {\s}wiece}{1981}
P{\l}omienie {\s}wiec \\ 
Ta{\n}cz{\a} nieprzytomnie, ob{\l}{\a}kane \\ 
Pragnieniem wyrwania si{\e} sobie. \\ 
W nienasyconych podskokach \\ 
Nie maj{\a} nawet czasu na zdziwienie, \\ 
Gdy ich gor{\a}cy rytua{\l} urywa si{\e} nagle \\ 
Zatrzymany bezmy{\s}lnym podmuchem \ldots \\ 
Pozostawiaj{\a} tylko chwil{\e} drgaj{\a}cego wspomnienia, \\ 
Smu\.{z}k{\e} dymu, \\ 
Kt{\o}ry bezwolnie oddaje swe cia{\l}o \\ 
Spokojnym obj{\e}ciom ciemno{\s}ci. 

Lecz je{\s}li pomy{\s}lny wiatr \\ 
Pozwoli im do ko{\n}ca \\ 
Cieszy{\ci} oczy \.{z}ywym {\s}wiat{\l}em, \\ 
Dopalaj{\a} si{\e} powoli \\ 
Rozwa\.{z}aj{\a}c pokornie sw{\o}j koniec \\ 
A\.{z} ich prosta {\s}wiadomo{\s}{\ci} \\ 
Przechodzi w niesamowite kszta{\l}ty ciep{\l}ego wosku \\ 
Zdradzaj{\a}ce przesz{\l}y niepok{\o}j, \\ 
Kt{\o}ry zastyga powoli, \\ 
Uk{\l}adaj{\a}c si{\e} z wieczno{\s}ci{\a}.

\end{wierszd}

%\newpage
%\vspace*{-4ex}

%\newpage
\begin{wierszd}{}{198?}

\.{Z}eglarze odm{\e}t{\o}w! \\ 
Dr{\o}g znanych z bajek i ksi{\a}g \\ 
Zapomnianych w dzeci{\n}stwie. \\ 
Twarde jak liny, \\ 
Wasze serce to kad{\l}ub okr{\e}tu. \\ 
\.{Z}eglarze odm{\e}t{\o}w! \\ 
Nie ciep{\l}ych piask{\o}w, lecz ciemnego morza, \\ 
Nie s{\l}odyczy fig i s{\l}{\o}w jasnych starc{\o}w, \\ 
Lecz szmeru powszedniej grozy, s{\l}onej wody, \\*
Uciszeni spojrzeniem w niesko{\n}czono{\s}{\ci} \\* 
Szukacie ... 
 
Mieszka{\n}cy wysp! \\ 
Spokojni, jak lot albatrosa \\ 
Oswoili{\s}cie lwy i ziemi{\e}. \\ 
W nadbrze\.{z}nych ska{\l}ach \\ 
Przechowali{\s}cie m{\a}dro{\s}ci ojc{\o}w, \\ 
Kt{\o}re chroni{\a} przed sztormem. \\ 
Mieszka{\n}cy wysp! \\ 
Uciszeni prac{\a}, w cieniu palm \\ 
Spo\.{z}ywacie owoce S{\l}o{\n}ca, \\ 
Gorzkie mleko i s{\l}odycz daktyli. \\ 
W sercach nie ma trwogi, \\ 
Wzrok nie szuka otch{\l}ani.

\end{wierszd}
\vspace*{-4ex}

%\newpage
\begin{wierszd}{}{198?}

Ws{\l}uchani jak tropiciele {\s}lad{\o}w \\ 
W niewidoczn{\a} obecno{\s}{\ci}, \\ 
W odciski st{\o}p na piasku \\ 
Zalewane cisz{\a} fal, \\ 
Gdy ofiarowuj{\a} si{\e} brzegom. 

Z twardych w{\l}{\o}kien rzeczy \\ 
Wysnuwaj{\a} lot ptak{\o}w. \\ 
Wie\.{z}ami si{\e}gaj{\a} po oddech wiatru, \\ 
Kt{\o}ry owiewa kolumnady s{\l}{\o}w \\ 
Cieniem niedotykalnego zdziwienia. 

W koszykach ognisk odnajduj{\a} \\ 
{\S}wiat{\l}o promieni s{\l}onecznych \\ 
Zebranych na zim{\e} wok{\o}{\l} osad, \\ 
Gdzie ta{\n}cz{\a} do muzyki, \\ 
Kt{\o}ra te\.{z} jest stamt{\a}d. 

Posiadaj{\a} to, co przeczuwaj{\a} \\ 
W oczach starc{\o}w bez skargi, \\ 
Niemych s{\l}owach matek, \\ 
I w mi{\l}o{\s}ci - znikome {\s}lady \\ 
Trwania poza ich trwaniem.
\end{wierszd} \vspace*{-3ex}




\begin{wierszd} {Cisza i {\s}piew I} {198?}
Martwi milcz{\a} - bo pustka nie rozumie s{\l}{\o}w. \\ 
\.{Z}ywi m{\o}wi{\a} z obfito{\s}ci serca. \\ 
Tych, kt{\o}rzy kochaj{\a} sta{\ci} na {\s}piew. \\ 

\end{wierszd}\vspace*{-3ex}

\begin{wierszd} {Cisza i {\s}piew II}{198?}
Przedwcze{\s}nie zmarli w milczeniu \\ 
Zapomnieli mi{\l}o{\s}ci, \\ 
Zamykaj{\a} cisz{\e} swojej duszy. \\ 
Sp{\e}tani strachem przed niemoc{\a} wys{\l}owienia nieznanego \\ 
Rzucaj{\a} zazdrosne spojrzenia 

	Ku pogodnym twarzom \.{z}yj{\a}cych, \\ 
	Kt{\o}rzy z pe{\l}ni serca m{\o}wi{\a} \\ 
	O wszystkim, co ma by{\ci} powiedziane, \\ 
	O tych, kt{\o}rzy odeszli, \\ 
	A w pokornym zachwycie 

O tych, kt{\o}rzy s{\a}cz{\a} wino cierpliwego zrozumienia \\ 
I \.{z}yj{\a} nieswoim \.{z}yciem, \\ 
Wyra\.{z}aj{\a} niwyra\.{z}alne \ldots \\ 
O kochaj{\a}cych - jedynych, kt{\o}rych sta{\ci} na {\s}piew \\ 
I modlitw{\e} za \.{z}ywych i umar{\l}ych.

\end{wierszd}\vspace*{-3ex}

\chap{Impresje}
%

\begin{wierszd} {Dzie{\n}} {1984}
{\S}wit, nim uczucia wczorajsze na nowo rozbudzi \\ 
Wyjmuje z ciszy kszta{\l}ty bez formy, na wp{\o}{\l} znane. \\ 
Ich o\.{z}ywieniem i spe{\l}nieniem {\l}udzi, \\ 
Lecz one wol{\a} swoje p{\o}{\l}\.{z}ycie nad ranem 

Ni\.{z} w dzie{\n} - oddany na w{\l}asno{\s}{\ci} pr{\o}bom uchwycenia \\ 
Sensu, pokory i mi{\l}o{\s}ci, kt{\o}rym grozi zdrada, \\ 
{\Z}r{\o}d{\l}a rado{\s}ci dzieci - pocz{\a}tk{\o}w istnienia \ldots \\ 
Bo wszystko nim si{\e} wypowie ju\.{z} znowu zapada 

W wiecz{\o}r, co odchodz{\a}cych ludzi \\ 
W ch{\l}odzie czerwonego wina powoli zatraca. \\ 
Kr{\o}tkim spojrzeniem chwil{\e} jeszcze trudzi, \\ 
A\.{z} wszystko w powracaj{\a}c{\a} ciemno{\s}{\ci} z u{\s}miechem obraca.
\end{wierszd}\vspace*{-2ex}

\begin{wierszd} {Pory roku}{1988}
Spopiela{\l}e dni, jak nagie drzewa \\ 
Powtykane w zimowy krajobraz, \\ 
Przykryte {\s}niegiem szkielet{\o}w. \\ 
Si{\l}a - z{\l}amana brakiem wiary  

W odrodzenie przebi{\s}nieg{\o}w \\ 
I rado{\s}{\ci} ka\.{z}dego kwiatu \\ 
Budz{\a}cego si{\e} w nadziei, \\ 
Kt{\o}ra jest jego m{\a}dro{\s}ci{\a}. 

Zieleni traw i dojrzewaj{\a}cym owocom \\ 
Niebo, rozmarzone gor{\a}cym powietrzem \\ 
U\.{z}ycza przestrzeni i - czasem - \\ 
Zakre{\s}la granice wolno{\s}ci.  

A\.{z} nasycone spe{\l}nieniem sady \\ 
Zrzucaj{\a} ci{\e}\.{z}kie owoce \\ 
W kosze cierpliwych zbieraczy \\ 
Gotuj{\a}cych si{\e} na d{\l}ugie rozmowy ze {\s}mierci{\a}.

\end{wierszd}

\newpage
\begin{wierszd}{Spacer zimowy}{1988}
Stalowe, ciemne niebo, \\ 
Stalowoszary l{\o}d, \\ 
Z{\l}owrog{\a} brzytw{\a} u{\s}miechu \\ 
Wdziera si{\e} w sk{\o}r{\e} ch{\l}{\o}d;

Przez futra nastroszone, \\ 
Swetry, we{\l}niane chustki, \\ 
Zcina cichn{\a}ce serca \\ 
W lodowe sople pustki.

\end{wierszd}

%\newpage
\begin{wierszd} 
{Msza barokowa} {1984}
Klawesynowe d{\z}wi{\e}ki \\ 
Sp{\l}ywaj{\a} po szatach barokowych anio{\l}k{\o}w \\ 
W r{\e}ce kap{\l}ana \\ 
Wznosz{\a}cego t{\l}um wiernych \\ 
Ku zagmatwanym jak boskie tajemnice \\ 
Zdobieniom stropu.

\end{wierszd}

\begin{wierszd}
{Praga}{1995}
Flety, fujarki, sypi{\a} koralami \\
Mi{\e}dzy b{\e}benki i dzwonki, w r{\e}kawy \\
\.{Z}ak{\o}w, muzykant{\o}w jarmarcznych \\
Wko{\l}o rynku zastyg{\l}ych \\
\hspace*{6em}{w kamienne fasady.}
\end{wierszd}


\chap{Portrety\hfill{\small{(1984)}}}\vspace*{2ex}

\begin{wiersz} {Heideggerowi  \\*[1ex]}
Cisz{\e} pr{\o}buj{\a} wype{\l}ni{\ci} \\ 
Dymem papierosa, \\ 
Muzyk{\a} nienasycenia. \\ 
W ukrytym strachu \\ 
Uchylaj{\a} si{\e} od upadku \\ 
pr{\o}buj{\a}c zaw{\l}adn{\a}{\ci} pustk{\a}. \\ 
Odwracaj{\a} wzrok \\ 
Od tchnienia {\s}mierci \\ 
I zab{\o}jczej niesko{\n}czno{\s}ci \.{z}ycia. \\ 
Lecz czy\.{z} nie istniej{\a}!? \\ 
Jak ka\.{z}da istota \\*
Nie mog{\a} sobie pozwoli{\ci} na k{\l}amstwo niebycia.

\end{wiersz}


\begin{wiersz}  
{Porter Kanta \\*[1ex]}
Bia{\l}e po{\n}czochy w czarnych lakierach stukaj{\a}cych miarowo, \\ 
Jeden przed drugim, jeden przed drugim, jeden ... \\ 
Jak bezosobowe obroty planet. \\ 
Asceza szczup{\l}ej sylwetki, w{\l}asnego cienia popychanego za{\l}o\.{z}onymi z ty{\l}u r{\e}kami. \\ 
Bia{\l}y ko{\l}nierzyk. \\ 
Lodowato blada maska z wyr\.{z}ni{\e}tymi kraw{\e}dziami ust, \\ 
Z bezosobowym dwojgiem oczu, kt{\o}re nie nale\.{z}{\a} do tego 
  przed\-sta\-wi\-cie\-la gatunku. \\ 
I nitki jedwabistej peruki \\ 
Przykrywaj{\a}ce rytmiczne uderzenia krwi, kt{\o}ra op{\l}ywa cienkimi \.{z}y{\l}kami m{\o}zg, pulsuje, jak j{\a}dra s{\l}o{\n}c. \\ 
I tylko oniemia{\l}e zdziwienie - gdzie znale{\z}{\ci} centrum, \\ 
Oplataj{\a}ce wszystkie te guziki, po{\l}y fraka, spodnie, g{\l}ow{\e}, \\ 
Przenikaj{\a}ce wszystko jednym, kategorycznym nakazem? \\ 
W prawym lakierze? W splecionych d{\l}oniach? {\Z}renicy? \\ 
W niesko{\n}czono{\s}ciach kosmosu?  \\ 

\end{wiersz}

\begin{wierszd} {Vivaldi - 4 pory roku}{1988}
Przedrze{\z}niaj{\a} skrzypce pierwsze drugie \\ 
Cicho, nie{\s}mia{\l}o, jak kwiat{\o}w pierwsze p{\a}ki. \\ 
Dr\.{z}{\a} w nadziei nim wybuchn{\a} d{\l}ugie, \\ 
Jak wiotkie {\l}odygi w{\s}r{\o}d wiosennej {\l}{\a}ki.  

Wnet si{\e} splot{\a} z upa{\l}em. O\.{z}ywi{\a} \\ 
Klawesynowi dni d{\l}ugie od gor{\a}ca, \\ 
Kt{\o}re ten przelicza klawiszami, \\ 
Kroplami, co ta{\n}cz{\a} po promieniach s{\l}o{\n}ca.  

Ruszy{\l}y smyki wiatr{\o}w jesiennych, \\ 
Zaszele{\s}ci{\l}y li{\s}{\ci}mi struny le{\s}nych harf. \\ 
{\S}wiat powoli schodzi do podziemi ciemnych, \\ 
Fale zb{\o}\.{z} p{\l}on{\a} rzek{\a} z{\l}otych szarf. 

Gruba wiolonczela ledwie drgn{\e}{\l}a \\ 
Nienawyk{\l}a do szybkiego skrzypiec biegu. \\ 
Szalone iskry nut zgarn{\e}{\l}a \\ 
I ju\.{z} --  \\ 
Zamar{\l}a orkiestra w bia{\l}ej ciszy {\s}niegu.

\end{wierszd}

\begin{wierszd}{}{1988} %\ \\
	Mozart ... \\* 
Gra do snu wszystkim - dzeciom i doros{\l}ym \\ 
Ma{\l}{\a}, wieczorn{\a} muzyk{\e}. I wype{\l}nia pok{\o}j \\ 
Raz menuetem przeci{\a}g{\l}ym, raz ta{\n}cem radosnym, \\ 
Co z g{\l}{\e}bi lekkie pi{\e}kno nios{\a}, a z pi{\e}knem spok{\o}j. 
 
	Berlioz ... \\* 
Hardy, zgorzknia{\l}y, {\s}wiadomy wolno{\s}ci \\ 
Marsz na szafot w ironii i {\s}miechu fanfar{\o}w. \\ 
Ci{\e}cie. I ju\.{z} b{\e}bny zapraszaj{\a} go{\s}ci \\ 
Na sabat, orgie psyche i do dusznych czar{\o}w. 
 
	Chopin ... \\*
Z t{\e}sknot{\a} d{\l}ugich klawiszy i dr\.{z}eniem \\ 
Wspomina wygnany ksi{\a}\.{z}{\e} falowanie p{\o}l. \\ 
Elegi{\e} smutku k{\l}adzie na fortepian cieniem \\ 
I sw{\o}j b{\o}l przywo{\l}uj{\a}c, koi {\s}wiata b{\o}l. 

	Bach ... \\*
Po stopniach fugi, w organowym szyku \\ 
Ci{\e}\.{z}kozbrojni rycerze w gmach katedry wstaj{\a}. \\ 
P{\l}yn{\a} konno pod sklepienia gotyku, \\ 
Kt{\o}re d{\z}wi{\e}ki Te Deum w niebo otwieraj{\a}. 
 
	Czajkowski ... \\*
Wiatr okrywa p{\l}aszczem dziewcz{\e}ta kwitn{\a}ce, \\ 
Gnie {\l}ab{\e}dzie szyje polnych drzew w zawieje, \\ 
Czasami wiosenn{\a} burz{\e} rozp{\e}ta na {\l}{\a}ce \\ 
I zaraz czystym okiem jeziora si{\e} {\s}mieje. 
\end{wierszd}

\chap{Ty}


\begin{wierszd}{Widzia{\l}em par{\e}}{1999}
W ksi{\e}\.{z}ycow{\a} noc, w{\s}r{\o}d\\
Spl{\a}tanych uliczek obcego miasta,\\
Skazani na obecno{\s}{\ci} -- kochaj{\a}cego.

Wysnuwaj{\a} z komin{\o}w\\
Smu\.{z}ki dymu dr\.{z}{\a}cym wzrokiem,\\
Uciekaj{\a}c przed spojrzeniem drugiego.

Odchodz{\a}, odlatuj{\a}, ponad dachy\\
Chyl{\a}c ku sobie niepewnie\\
Dum{\e} niespodziewanej samotno{\s}ci.

Jak trzciny zak{\l}{\o}cone wiatrem ...

\end{wierszd}


\begin{wiersz} {Poca{\l}unki \\*[1ex]}
S{\a} poca{\l}unki pierwsze, \\ 
Co ods{\l}aniaj{\a} {\s}wiat \\ 
Znany z woni marze{\n} ... \\ 
Z oddaniem wr{\e}czane  \\ 
W skupieniu wieczornych jezior, \\ 
W dr\.{z}enie le{\s}nej ciszy ... \\ 
Gdy dzwonki s{\l}owik{\o}w zacznaj{\a} gra{\ci} \\ 
Obejmuj{\a} czule usta obietnice \\ 
Tego, co si{\e} nie sta{\l}o, lecz musi si{\e} sta{\ci}. 
 
S{\a} poca{\l}unki chwil \\ 
Rzucane jak r{\o}\.{z}e w ziemi{\e}, \\ 
Z kt{\o}rej nie wyro{\s}nie nic ... \\ 
Czerwone rany duszy \\ 
Uk{\l}adane na piersiach \\ 
W kobierce niepami{\e}ci ... \\ 
Mi{\e}dzy korale naszyjnik{\o}w cienia,  \\ 
W rozgwar wieczor{\o}w i dni  \\ 
Wplataj{\a} spragnione usta warkocze zapomnienia. 

S{\a} poca{\l}unki zm{\e}czone, \\ 
Co g{\l}usz{\a} \\ 
Nieszcz{\e}{\s}cia innych czas{\o}w. \\ 
Nieszczere, bo zdradzone \\ 
Gin{\a} nienarodzone \\ 
W jednostajno{\s}ci dni ... \\ 
W bezwolnych chwilach rabunku \\ 
Wci{\a}\.{z} na nowo wydaj{\a} \\ 
Gorzki smak - gorzki smak poca{\l}unku. 

I s{\a} poca{\l}unki jedyne \\ 
Jak bukiet niezapominajek  \\ 
Wpi{\e}tych we w{\l}osy. \\ 
A ka\.{z}da jest {\s}wie\.{z}a wiosn{\a}, \\ 
Jak niespodzianka i dar \\ 
Dobrej wr{\o}\.{z}ki, o oczach skrz{\a}cych rado{\s}ci{\a} ... \\ 
I {\s}wiat zapomina krzywdy, \\ 
A usta kochane m{\o}wi{\a} \\ 
"Na wieczno{\s}{\ci}" - czyli - "teraz lub nigdy".

\end{wiersz}


\begin{wierszd}{}{1983}
Nie wspominam Ci{\e} s{\l}owami. One skoczne \\
Jak szaliki, latawce odlatuj{\a}. \\
Ani spojrzeniem, ani my{\s}lami, gdy mroczne  \\
Popi{\o}{\l} pami{\e}ci bezg{\l}o{\s}nie przesypuj{\a}.

Nie z front{\o}w samotnych {\s}mierci, odleg{\l}o{\s}ci, \\
Lecz z g{\l}{\e}bi domu, z ciszy przenikanej \\
Wzrokiem Boga gdy na nas patrzy, z chwil rado{\s}ci, \\
I z wdzi{\e}czno{\s}ci za przesz{\l}o{\s}{\ci} - zapoznanej.

Wstajesz nagle - nikt nie pyta, nikt nie przeczy \\
Z kapci, kt{\o}re w pod{\l}odze {\s}cie\.{z}k{\e} star{\l}y, \\
Z kartek, nad kt{\o}rymi u{\s}miech zawis{\l}, z rzeczy, \\
Kt{\o}re Twoim \.{z}yciem \.{z}y{\l}y - i zamar{\l}y,

\end{wierszd}

\old{

\begin{wierszd}{}{1999}
To by{\l}y dni bez liczb, bez godzin,\\
Twoje w{\l}osy {\s}piewa{\l}y ptakami\\
{\S}wit zwiastuj{\a}c z g{\l}{\e}bi parku\\
Milcz{\a}cego ciemno{\s}ci wiekami.

I twoje oczy patrzy{\l}y spo{\s}r{\o}d drzew\\
Skrzeniem stawu i {\z}r{\o}de{\l} z zieleni,\\
W cieniu marze{\n} gotowych do lotu,\\
Skrzyde{\l} ufnych nadziej{\a} przestrzeni.

Wi{\e}c by{\l}y dni bez liczb, bez godzin,\\
Gdy obietnic{\a} by{\l} najl\.{z}ejszy cie{\n},\\
A cisza dr\.{z}a{\l}a w samotno{\s}ci cia{\l}\\
Oczekiwaniem na poprzedni dzie{\n}.\\
\end{wierszd}

} %end \old

\begin{wierszd}{}{1999}
Za jeden gwiezdny b{\l}ysk u{\s}miechu,\\
Za jasne `witaj', gdy ul\.{z}y{\l}o nadzieji czekania\\
W spojrzeniu spod brwi spuszczonych,\\
Jak kaptur mnicha, kornie i zuchwale.

Za jedn{\a} rado{\s}\'c twoj{\a} odda{\l}bym\\
{\S}wiat  ...

A dusz{\e}? Zapatrzon{\a} w g{\l}{\e}bi{\e} niepokoju\\
Sp{\e}kanego rysami niewidzialnych znacze{\n},\\
Gdzie spo{\s}r{\o}d szczelin sp{\l}ywa woda \.{z}ycia,\\
Jak w{\l}osy, jak {\l}zy, kornie i cicho.

Za spok{\o}j twej rado{\s}ci odda{\l}bym ...
\end{wierszd}

\newpage
\begin{wierszd}{}{1999}
There are things which no man can ever own,\\
%Like things of which no man is the master,\\
Deep, quiet seas under the face of days,\\
Surrounding life by silent horison,\\
Whence love and peace into this world are sown.

Removing veils and fogg of confusion,\\
Incessant search for truths, the proud ``I know''\\
Breed only restless doubt for, in due time,\\
The new truths turn into old illusions.

Chaos is not calmed by ``I will'', ``I know''\\
Which scatter fancy toys around soul's room.\\
Simple peace is time's gift to the patience\\
Which humbly thanks for things it cannot own.

May you with joy millennium celebrate,\\
And Merry Christmas and Happy New Year.\\
I wasn't enough? Had only myself to give?\\
Greetings! With love -- which failed to incarnate.
\end{wierszd}

\end{document}