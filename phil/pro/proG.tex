\documentstyle[a4wide]{article}
%\makeatletter
%\voffset -2cm
%%\makeatletter
%\show\
%\makeatother
\newcommand{\ite}[1]{\item[{\bf #1.}]}
\newcommand{\app}{\mathrel{\scriptscriptstyle{\vdash}}}
\newcommand{\estr}{\varepsilon}
\newcommand{\PSet}[1]{{\cal P}(#1)}
\newcommand{\ch}{\sqcup}
\newcommand{\into}{\to}
\newcommand{\Iff}{\Leftrightarrow}
\renewcommand{\iff}{\leftrightarrow}
\newcommand{\prI}{\vdash_I}
\newcommand{\pr}{\vdash}
\newcommand{\ovr}[1]{\overline{#1}}

\newcommand{\cp}{{\cal O}}

% update function/set
%\newcommand{\upd}[3]{#1\!\Rsh^{#2}_{\!\!#3}} % AMS
\newcommand{\upd}[3]{#1^{\raisebox{.5ex}{\mbox{${\scriptscriptstyle{\leftarrow}}\scriptstyle{#3}$}}}_{{\scriptscriptstyle{\rightarrow}}{#2}}} 
\newcommand{\rem}[2]{\upd{#1}{#2}{\bullet}}
\newcommand{\add}[2]{\upd {#1}{\bullet}{#2}}
%\newcommand{\mv}[3]{{#1}\!\Rsh_{\!\!#3}{#2}}
\newcommand{\mv}[3]{{#1}\:\raisebox{-.5ex}{$\stackrel{\displaystyle\curvearrowright}{\scriptstyle{#3}}$}\:{#2}}

\newcommand{\leads}{\rightsquigarrow} %AMS

\newenvironment{ites}{\vspace*{1ex}\par\noindent 
   \begin{tabular}{r@{\ \ }rcl}}{\vspace*{1ex}\end{tabular}\par\noindent}
\newcommand{\itt}[3]{{\bf #1.} & $#2$ & $\impl$ & $#3$ \\[1ex]}
\newcommand{\itte}[3]{{\bf #1.} & $#2$ & $\impl$ & $#3$ }
\newcommand{\itteq}[3]{\hline {\bf #1} & & & $#2=#3$ }
\newcommand{\itteqc}[3]{\hline {\bf #1} &  &  & $#2=#3$ \\[.5ex]}
\newcommand{\itteqq}[3]{{\bf #1} &  &  & $#2=#3$ }
\newcommand{\itc}[2]{{\bf #1.} & $#2$ &    \\[.5ex]}
\newcommand{\itcs}[3]{{\bf #1.} & $#2$ & $\impl$ & $#3$  \\[.5ex] }
\newcommand{\itco}[3]{   & $#1$ & $#2$  & $#3$ \\[1ex]}
\newcommand{\itcoe}[3]{   & $#1$ & $#2$  & $#3$}
\newcommand{\bit}{\begin{ites}}
\newcommand{\eit}{\end{ites}}
\newcommand{\na}[1]{{\bf #1.}}
\newenvironment{iten}{\begin{tabular}[t]{r@{\ }rcl}}{\end{tabular}}
\newcommand{\ass}[1]{& \multicolumn{3}{l}{\hspace*{-1em}{\small{[{\em Assuming:} #1]}}}}

%%%%%%%%% nested comp's
\newenvironment{itess}{\vspace*{1ex}\par\noindent 
   \begin{tabular}{r@{\ \ }lllcl}}{\vspace*{1ex}\end{tabular}\par\noindent}
\newcommand{\bitn}{\begin{itess}}
\newcommand{\eitn}{\end{itess}}
\newcommand{\comA}[2]{{\bf #1}& $#2$ \\ }
\newcommand{\comB}[3]{{\bf #1}& $#2$ & $#3$\\ }
\newcommand{\com}[3]{{\bf #1}& & & $#2$ & $\impl$ & $#3$\\[.5ex] }

\newcommand{\comS}[5]{{\bf #1} 
   & $#2$ & $#3$ & $#4$ & $\impl$ & $#5$\\[.5ex] }

%%%%%%%%%%%%%%%%
\newtheorem{CLAIM}{Proposition}[section]
\newtheorem{COROLLARY}[CLAIM]{Corollary}
\newtheorem{THEOREM}[CLAIM]{Theorem}
\newtheorem{LEMMA}[CLAIM]{Lemma}
\newcommand{\MyLPar}{\parsep -.2ex plus.2ex minus.2ex\itemsep\parsep
   \vspace{-\topsep}\vspace{.5ex}}
\newcommand{\MyNumEnv}[1]{\trivlist\refstepcounter{CLAIM}\item[\hskip
   \labelsep{\bf #1\ \theCLAIM\ }]\sf\ignorespaces}
\newenvironment{DEFINITION}{\MyNumEnv{Definition}}{\par\addvspace{0.5ex}}
\newenvironment{EXAMPLE}{\MyNumEnv{Example}}{\nopagebreak\finish}
\newenvironment{PROOF}{{\bf Proof.}}{\nopagebreak\finish}
\newcommand{\finish}{\hspace*{\fill}\nopagebreak 
     \raisebox{-1ex}{$\Box$}\hspace*{1em}\par\addvspace{1ex}}
\renewcommand{\abstract}[1]{ \begin{quote}\noindent \small {\bf Abstract.} #1
    \end{quote}}
\newcommand{\B}[1]{{\rm I\hspace{-.2em}#1}}
\newcommand{\Nat}{{\B N}}
\newcommand{\bool}{{\cal B}{\rm ool}}
\renewcommand{\c}[1]{{\cal #1}}
\newcommand{\Funcs}{{\cal F}}
%\newcommand{\Terms}{{\cal T}(\Funcs,\Vars)}
\newcommand{\Terms}[1]{{\cal T}(#1)}
\newcommand{\Vars}{{\cal V}}
\newcommand{\Incl}{\mathbin{\prec}}
\newcommand{\Cont}{\mathbin{\succ}}
\newcommand{\Int}{\mathbin{\frown}}
\newcommand{\Seteq}{\mathbin{\asymp}}
\newcommand{\Eq}{\mathbin{\approx}}
\newcommand{\notEq}{\mathbin{\Not\approx}}
\newcommand{\notIncl}{\mathbin{\Not\prec}}
\newcommand{\notCont}{\mathbin{\Not\succ}}
\newcommand{\notInt}{\mathbin{\Not\frown}}
\newcommand{\Seq}{\mathrel{\mapsto}}
\newcommand{\Ord}{\mathbin{\rightarrow}}
\newcommand{\M}[1]{\mathbin{\mathord{#1}^m}}
\newcommand{\Mset}[1]{{\cal M}(#1)}
\newcommand{\interpret}[1]{[\![#1]\!]^{A}_{\rho}}
\newcommand{\Interpret}[1]{[\![#1]\!]^{A}}
%\newcommand{\Comp}[2]{\mbox{\rm Comp}(#1,#2)}
\newcommand{\Comp}[2]{#1\diamond#2}
\newcommand{\Repl}[2]{\mbox{\rm Repl}(#1,#2)}
%\newcommand\SS[1]{{\cal S}^{#1}}
\newcommand{\To}[1]{\mathbin{\stackrel{#1}{\longrightarrow}}}
\newcommand{\TTo}[1]{\mathbin{\stackrel{#1}{\Longrightarrow}}}
\newcommand{\oT}[1]{\mathbin{\stackrel{#1}{\longleftarrow}}}
\newcommand{\oTT}[1]{\mathbin{\stackrel{#1}{\Longleftarrow}}}
\newcommand{\es}{\emptyset}
\newcommand{\C}[1]{\mbox{$\cal #1$}}
\newcommand{\Mb}[1]{\mbox{#1}}
\newcommand{\<}{\langle}
\renewcommand{\>}{\rangle}
\newcommand{\Def}{\mathrel{\stackrel{\mbox{\tiny def}}{=}}}
\newcommand{\impl}{\mathrel\Rightarrow}
\newcommand{\then}{\mathrel\Rightarrow}
\newfont{\msym}{msxm10}

\newcommand{\false}{\bot}
\newcommand{\true}{\top}

\newcommand{\restrict}{\mathbin{\mbox{\msym\symbol{22}}}}
\newcommand{\List}[3]{#1_{1}#3\ldots#3#1_{#2}}
\newcommand{\col}[1]{\renewcommand{\arraystretch}{0.4} \begin{array}[t]{c} #1
  \end{array}}
\newcommand{\prule}[2]{{\displaystyle #1 \over \displaystyle#2}}
\newcounter{ITEM}
\newcommand{\newITEM}[1]{\gdef\ITEMlabel{ITEM:#1-}\setcounter{ITEM}{0}}
\makeatletter
\newcommand{\Not}[1]{\mathbin {\mathpalette\c@ncel#1}}
\def\LabeL#1$#2{\edef\@currentlabel{#2}\label{#1}}
\newcommand{\ITEM}[2]{\par\addvspace{.7ex}\noindent
   \refstepcounter{ITEM}\expandafter\LabeL\ITEMlabel#1${(\roman{ITEM})}%
   {\advance\linewidth-2em \hskip2em %
   \parbox{\linewidth}{\hskip-2em {\rm\bf \@currentlabel\
   }\ignorespaces #2}}\par \addvspace{.7ex}\noindent\ignorespaces}
\def\R@f#1${\ref{#1}}
\newcommand{\?}[1]{\expandafter\R@f\ITEMlabel#1$}
\makeatother
\newcommand{\PROOFRULE}[2]{\trivlist\item[\hskip\labelsep {\bf #1}]#2\par
  \addvspace{1ex}\noindent\ignorespaces}
\newcommand{\PRULE}[2]{\displaystyle#1 \strut \over \strut \displaystyle#2}
%\setlength{\clauselength}{6cm}
%% \newcommand{\clause}[3]{\par\addvspace{.7ex}\noindent\LabeL#2${{\rm\bf #1}}%
%%   {\advance\linewidth-3em \hskip 3em
%%    \parbox{\linewidth}{\hskip-3em \parbox{3em}{\rm\bf#1.}#3}}\par 
%%    \addvspace{.7ex}\noindent\ignorespaces}
\newcommand{\clause}[3]{\par\addvspace{.7ex}\noindent
  {\advance\linewidth-3em \hskip 3em
   \parbox{\linewidth}{\hskip-3em \parbox{3em}{\rm\bf#1.}#3}}\par 
   \addvspace{.7ex}\noindent\ignorespaces}
\newcommand{\Cs}{\varepsilon}
\newcommand{\const}[3]{\Cs_{\scriptscriptstyle#2}(#1,#3)}
\newcommand{\Ein}{\sqsubset}%
\newcommand{\Eineq}{\sqsubseteq}%


\voffset -1cm

\begin{document}
%\marginparwidth 2.5cm
\title{The ``Myth'' of Progress}
\author{{\em Micha{\l}\ Walicki}}
\date{February 1994}
\maketitle

\section{The Myth}
%\begin{verse}
%\reversemarginpar
\begin{tabular}[hb]{rl}
& Golden was the generation of weak Men who in the beginning \\
\footnotesize{\em 110}& created Eternity, who inhabit the palaces in Olymp.  \\
%  \marginpar{\hspace{5em}\footnotesize{\em 110}} \\  
& That happened when Kronos, as the king, ruled Heavens. \\
& They lived like Gods with tranquil mind, \\
& unworried and free from toils and suffering. They did not reach \\
& the miserable age; instead unchanged in hands and feet, \\
\footnotesize{\em 115} & they enjoyed the situation (Gelangen), free from all evil. \\
%  \marginpar{\hspace{5em} \footnotesize{\em 115}} \\
%
& They died as if enthralled by sleep. All that is good, \\
& was given unto them. The nourishing meadows bear, \\
& quite by itself, full harvest. As they wish \\
& they leave after work with an abundance of goods, \\
\footnotesize{\em 120} & friends of the holy Gods, richly blessed by the herd. \\
%  \marginpar{\hspace{5em}\footnotesize{\em 120}} \\
& And since the Earth had hidden this generation in its womb, \\
& have they acted as daemons according to Zeus', the high, will, \\
& good spirits of Earth, protectors of the mortal men, \\
& \{ who observe the right as well as the wrong deeds, \\
\footnotesize{\em 125}& while, hidden in fogg, they thoroughly przenikaja the world, \} \\
%   \marginpar{\hspace{5em}\footnotesize{\em 125}} \\
& Extending the blessing; theirs is the royal favour and right.\\
(2)& Then a second generation, a silver one, much lesser, \\
%             \marginpar{\footnotesize{\em (2)}} \\
& is  created, which inhabit the palaces high in Olymp, \\
& neither in look nor in mood comparable to the golden one. \\
\footnotesize{\em 130}& Hundred years grows up a child by the dearest mother \\
%   \marginpar{\hspace{5em}\footnotesize{\em 130}}\\
& gladly in its own house of kindered heart; \\
& when it reaches the blood of youth and grows adult, \\
& they lived only a short time, and through their own stupidity \\
& suffered pain. They were not, namely, able to abandon violence \\
\footnotesize{\em 135}& between each other. They did not want to respect Gods\\
%   \marginpar{\hspace{5em}\footnotesize{\em 135}}\\
& nor offer to the holy ones on the (geweihten) altars, \\
Zeus & as the local customs of men required. And Zeus, the Kronide, \\
%             \marginpar{\footnotesize{\em Zeus}} \\
& made them finally disappear, rozgniewany, because the holy Gods, \\
& who inhabit the Olymp, did not receive from them  resopect.\\
\footnotesize{\em 140}& Yet since the Earth had hidden also this generation in its womb, \\
%   \marginpar{\hspace{5em}\footnotesize{\em 140}} \\
& they have been called the holy mortals under the Earth, \\
& although lower, also they receive respect. \\
(3) & But another generation of weakly men, a third one, \\
%             \marginpar{\footnotesize{\em (3)}} \\
& powerful (new, everlasting), created the Father, in nothing reminding the 
   silver one, \\
\footnotesize{\em 145}& horrible and over-violent, from olcha, (derived) from Ares' \\
%   \marginpar{\hspace{5em}\footnotesize{\em 145}} \\
& (stoening) works and Ubermut; never eating corn, \\
& they had veiled (mysterious), steel-like houses, \\
& jerk. Violent power and arms, uncanny, grow \\
& over the strong and tight limbs from the shoulders. 
\end{tabular}

\begin{tabular}[ht]{rl}
\footnotesize{\em 150}& All their weapon was from Ruda, and from Ruda their houses, \\
%   \marginpar{\hspace{5em}\footnotesize{\em 150}} \\
& Ruda was used for work -- and yet there was no ice as black as that. \\
& But then also they go under, forced by their own arms, \\
& deep in the moisty house of icy Hades, \\
& nameless, restless; the black death, so horrible they were, \\
\footnotesize{\em 155}& eradicated them, and they (divide, veil from the shining people the Sun. \\
%   \marginpar{\hspace{5em}\footnotesize{\em 155}} \\
& Yet, when the Earth had hidden in its womb also this generation, \\
& created Zeus, the son of Kronos, on the allnourishing world \\
& yet another, the fourth one, better and more just: \\
%             \marginpar{\footnotesize{\em (4) Troya}} \\
& the generation of the Heros, equal to Gods, and so their name is \\
\footnotesize{\em 160}& half-gods, the generation before ours on the endless Earth. \\
%   \marginpar{\hspace{5em}\footnotesize{\em 160}} \\
& But also they started evil war, and terrifying butchery \\
& partly at the seven-gates' Theben, in the land of Kadmos, \\
& as the fight was for the Oedipus' cattle, \\
& partly in the ships over the wide depths of the see \\
\footnotesize{\em 165}& for the beautiful (gelockten) Helene they went to Troya. \\
%   \marginpar{\hspace{5em}\footnotesize{\em 165}} \\
& There had they faced the end of death, \\
& to others Zeus, the Kronide and Father, gave nourishment and dwellingpalce \\
\footnotesize{\em 168}& removed from the Men, and settled them at the  edge of the Earth. \\
%   \marginpar{\hspace{5em}\footnotesize{\em 168}} \\
\footnotesize{\em 170}& There, by the meanderous Ocean, they live, free \\
%   \marginpar{\hspace{5em}\footnotesize{\em 170}} \\
& from all dirtining of heart, removed on the holy island, \\
& the happy people of the Heros, for whom the nourishing fields \\
\footnotesize{\em 173}& bear the fruit, sweet like honney three times a year,\\
%   \marginpar{\hspace{5em}\footnotesize{\em 173}} \\
& \{ far away from Gods. They have Kronos for the king. \\
& Zeus himself has them freed, the Father of Men and Gods. \\
& Now he enjoys their respect as is due. \\
(5)& Zeus created yet another generation of weakly men: \\
%             \marginpar{\footnotesize{\em (5)}} \\
& all who are now feed by the nourishing fields.  \} \\
\footnotesize{\em 174}& I wish I did not belong to this fifth generation, \\
%   \marginpar{\hspace{5em}\footnotesize{\em 174}} \\
& but was already dead or not yet born! \\
& Now, namely, it is the generation of iron. Never during the day \\
& find they rest from needs and work, nor during the nights, \\
& completely exhausted. And the Gods send oppressing sorrows. \\
& However, we get also something good mixed to the evil. \\
\footnotesize{\em 180}& Zeus will annihilate also this generation, \\
%   \marginpar{\hspace{5em}\footnotesize{\em 180}} \\
& the day when they already at birth have grey hair: \\
& Then the father will be alien to the children and the children to him, \\
& a friendly guest will be alien to the host, and one partner to the other, \\
& then brother will no longer be a friend, as it used to be before. \\
\footnotesize{\em 185}& In a hurry, they avoid the duties toward the aged elders, \\
%   \marginpar{\hspace{5em}\footnotesize{\em 185}} \\
& blame them arrogantly and plague with hurting words, \\
& Dishonest, who disregard the punishment of Gods! So will they hardly give \\
& their aged elders the due and caring love. \\
& The right of the fist is what counts: one destroys the living place of 
another. \\
\footnotesize{\em 190}& No thanks will be given for justice or faithfulness \\
%   \marginpar{\hspace{5em}\footnotesize{\em 190}} \\
& or good deed -- rather, the protagonist of a ruthless work \\
& and violence will be honoured. Only in the fist there will be \\
& recognition and right. The bad will harm the better, \\
& speaking with twisted words and (obendrein nach beschworen). \\
\footnotesize{\em 195}& Sorrow of soul will everywhere accompany the crying men, \\
%   \marginpar{\hspace{5em}\footnotesize{\em 195}} \\
& sounding ugly, with evil look and malicious joy. 
\end{tabular}

\begin{tabular}[ht]{rl}
& Then comes the time when from the Earth crossed by the trails, \\
& beautiful skin wrapped in the white robes, \\
& up to the Olymp and the tribe of Gods, leaving the men \\
\footnotesize{\em 200} & go Aidos and Nemesis. But the bitter pain \\
%  \marginpar{\hspace{5em}\footnotesize{\em 200}} \\
& will stay with the men; nothing will help agains the evil. \\[1.5ex]
\multicolumn{2}{r}{\em Hesiod, ``Works and Days''} \\[3ex]
\end{tabular}
%\flushleft
%\normalmarginpar
%\end{verse}

\noindent 
The myth seems to tell a story of a succession in which mankind gradually declines 
from the grandeur of its origin into the miserable life of toils and sorrow. The Beginning
was maginificent -- apropriate for its divine nature -- but its depth has been
long forgotten. Each stage of the development which started with gods, brings 
us further away from them, further away from the happy
harmony and continuity of the divine life, towards the monotonous strive after
concrete, every-day goals.

We know the myth but we do not believe it any longer. Humbleness of its 
creators, their fascination with the invisible Beginning has been reversed.
We are less interested in the origins than in the ends. 
Above all, we do not believe in the story of gradual degradation. We rest
self-assured that what is now is better than what was before, and that what
will be tomorrow will be even better. 
Having reduced the beginning to the stage of an unenlightened, animal fear
confronted with the uncontrolled powers of nature, we have to imagine the end
as the site of all that we put in the place of divinity.
We 
are not feeling like the miserable creatures of the iron age. On the contrary,
we are feeling like the golden, or at least the silver generation.
We are convinced that we, humans independent of gods, are rising a wonderful 
construction which only malicious or envious spirit might compare to the
tower of Babel.

For the story of the marvellous 
Beginning we have substituted the image of even more marvellous end. 
For the past we have
substituted future, for the ontological order -- the order of knowledge and
achievements. The irony of history has changed the reality of the past into 
a crude primitivism, and made the shadow of the illusory future into a measure
of reality.
Then the irreversibility of time makes the origin less relevant. We
do not need the myths about our simple beginning and, above all, we do not 
want to
``confuse'' this beginning with the end. Like a person with an unhappy and 
miserable past does not want to remember but prefers to believe to be better
off now and to look forward, we prefer to conclude that progress brought us
incomparably better present and will bring us even better future. Our ``myth''
of progress is not a myth of beginning, not even of a continuous relation 
between the beginning and end. It is a ``myth'' of escaping from the beginning 
towards the end.


\section{When do we speak about progress?}
There seems to be an obvious answer -- progress presupposes measurement,
quantification. But what does that mean? Using the distinction quantity vs.
quality, we put an emphasis on some {\em precision} and {\em definiteness} 
which make the measurement posssible.
Qualities may be intuitive and understandable but they lack the sufficient
precision which alone admits objective comparison. We may say that ``blue is
colder than green'' but it is not the kind of comparison we make when realizing
that classical mechanics follows as a special case of relativity theory. In 
order to speak about progress we need a measure -- {\em precise enough 
criteria} allowing us to relate the scope of one stage of development to 
another.

The precision of these criteria can be founded only on the {\em definiteness}
of the objects being measured. The more precise, the better defined they are,
the easier it becomes to measure their changes. For instance, falsificationism,
studying the problem of demarcation by means of the possibilty of assessing
the relative value of different results, comes up exactly with the criteria
which make such an assessment precise enough. The criterion of falsifiability
(or, if you prefer, verifiability) reflects the degree of precision of the 
concepts and objects involved in a given theory. 

Precision or definiteness is
the first prerogative of progress in the modern sense of the word.\footnote{
Perhaps, ``precision'' itself is not a very precise notion, but we by no means
 aim at scientific exactitude.}
Definiteness of the object is what underlies the quantitative determinations.
Once some objects have been precisely defined and some results concerning
them established, they do not get lost -- the next stage may only add to this
depository. This is the crudest notion of progress but, in fact, all the 
others are only its derivative.

To this exactitude of definitions, there are closely related other phenomena
which we should mention. The first one is {\em controllability}. The more
precisely an object is defined, the more immanent it becomes, the more liable
to the control by the defining subject. The ``unclear'' feelings and moods
which enchance us, so to speak, ``from without'' are not underlied our control
in the same, easy way as the concrete material things. True, the minute
sensations, of pain or warmth, also come ``from without'' and are precisely
felt but not underlied directly our will. But just their minuteness, locality
makes makes it easy to relate them to external objects with the help of which
we can control them. This isn't so with, for instance, feelings which,
however revealing of the reality they may be, are ``unclear'' and hence 
uncontrollable -- because the objects they reveal are not of such a simple
and definite nature.

Thus controllability is intimately related to precision and it is only natural
that progress is essentially progress of power and control over the world.

The next one is {\em objectivity} in the sense of externality and 
intersubjectivity. The more precise and well-limited an object is, the more it
appears as stnding in front of, or against the subject. The more the long
lasting inquiry into the nature of the objects was focusing on their precise
definition, the more external and remote from the subjective intimacy the 
became. There is no contradiction between this process of externalisation and
the fact, mentioned above, that objects become more immanent. Externality is
a mode of immanence. Objectivisation, making the objects external to the 
subject, the more anchores them in the world and strips of the transcendetal
character.

This objectivisation through precision is reflected in the language which 
develops accordingly toward a more and more exact medium of direct 
communication. What is well defined can be precisely expressed. To say
that we make progress, we have to demonstrate it; to demonstrate it, we have
to express it in a clear language, and to express it in such a language we
have to limit ourselves to the reality which such a language can describe.
The precision of the language is only another side of the precision of the 
objects described. And the increased precision of objects is an increased
degree of objectivism. 

Our progress is not supposed to be some mystical
quality but an objectively characterised process. The progress of power and 
control is also the intersubjective process of externalisation.

Finally, the phenomenon related to controllability and objectivity, which also
rests directly on definiteness of the objects, is {\em necessity}. Again a 
comparison with the emotional life may illustrate the point. Although such a
life certainly has its logic, it is not the logic of necessity. Its objects 
are to vague, its processes to unpredictable to establish a network of
necessary relations. As the scope of objects diminishes and they become more
precise, such a network can be spawned around them. Sunrise or death are
well-defined phenomena and, all the disputes concerning the metaphysical
status of necessity notwithstanding, they may be related in such a manner to
other phenomena of similar kind. Whoever is born necessarily dies and the sun
will certainly rise tomorrow. These are clear, empirically obvious 
statements.\footnote{As remarked, I am not interested in their ``truth'' or
epistemological meaning. It does not concern me what necessity ``really is'',
only that its use is made under certain circumstances.} In order to ensure the
love of another person, on the other hand, one had to traditionally recourse
to spells and magic. And in spite of all the progress we have made, I can 
hardly believe that today one, who had run out of the possibilities to grant
it on the personal basis, is in the possession of any better means to do that.

Thus progressive thinking gets involved into considering the world in terms
of necesary laws. The most striking illustration of that is the fact that
virtually all preachers of progress, from Hegel via Marx and Comte, until 
today's scientism, considered it itself to be not only irreversible but also 
unavoidable process.

\section{The ``Myth''}
Progress emerges in the context of the quantitative view,  which, 
as its implications described in the previous section illustrate, applies only
to few lower layers of being. Where, then, does the ``Myth'', the believe in
its total and ultimate validity come from?

The first source of the ``myth'' of progress is the ``myth'' of science. That
is, absolutisation of science and identification of its development with the 
level of the development of a society, even the whole human kind. This
identification happened in the moment when the rationalisitc philosophy lifted
the finite reason from its condition of a servant of higher values or faith,
to the piedestal of the highest, most human faculty. True, similar postulates
were made earlier, for instance in Greece, to mention only Aristotles as the 
most profound and representative example. The Logos, which for the lack of 
better words we translate as reason, was the highest principle of Greek 
rationalists. The definition of man as a rational animal
made reason the most specifically human faculty. Concequently, to achieve 
perfection as a human being was to perfect one's reason. This, however, is a
strange argument involving a rare mixture of rationalistic arrogance and
irrational anarchism. In order to be a good man, one should concentrate on
what distinguishes him from other beings. What distinguishes a blind man from
others is blindness. Its cultivation will hardly make him more perfect - only
more blind. Even if what is the most profound in us is of a highly personal
and unique nature, it does not follow that differentia specifica is the most
profound.

Notwithstanding the strangenesss of this argument, there is an important 
difference between the reason of Aristotles and that of the latter 
rationalists and positivist. Logos was the manifestation of the ultimate,
eternal truth abiding beyond the practical world of men. The right attitude
towards it was contemplation, not action. Enlightment, and then positivism,
worshipped reason as the means of bringing about changes in the world, that
is, as science. Wisdom of the eternal was of no such practical importance.
But science, the knowledge of the termporal, was aimed exclusively at the 
visible results, ultimately, at the power to effect changes. The transition
from Greek to modern rationalism, being the transition from reason to science,
from the absoulte truth to the finite one, is also a transition from the 
static to the dynamic view of the world. Logos did not become completely
forgotten but it began to appear as an ultimate, in fact unreachable, goal
of scientific activity, as a static, noumenal measure of the changes occurring
in the relative sphere of science.
Since it could not be grasped and characterised in the same terms as the 
latter, eventually it become irrelevant. (And relevance is a measure of
reality.) Significantly, the idea of progress enters the stage only shortly
after the scientifc reason had conquered it, pushing Logos aside into the 
sphere of irrelevance.

We do not have to dwell here on the characteristics of this modern reason and,
particularly, science. Volumes have been written on this topic and what is
significant to us, has been mentioned in the previous section. Everything
which was said there about the conditions and correlates of progress applies
to the standards of scientific thinking. Identification of science and
society, made the scientific reason the measure of the world.
Thus, the indisputable progress of
science, initially coloured by the philosophical fascination and then by the 
influence on the practical life, become the measure of the progress of the 
mankind.

However, this identification explains only the mechanism which made
progress such a respectable idea. But it does not account for the reasons
why that happened. What was so attractive in viewing the progress as the 
 highest idea? 
It became that in the moment when the absolute standards,
 of Logos or religion, got eliminated from the intellectual vocabulary. 
And the lack of the absolute creates emptiness which man immediately tries to
fill with a new absolute. Killing the gods does not liberate the mortals --
it makes them slaves of idols or -- ``myths''.

A myth is a story about the absolute reality.
A ``myth'' begins when a limited and relative sphere of being becomes the 
absolute measure of the whole life, when it becomes an idol.
The ``myth'' of progress begins when progress ceases to be just a progress
of some specific field, where objects are prone to quantification, and becomes
Progress concerning the totality of life.

One of the obvious reason for the need of such a ``myth'' can be seen in the 
collapse of the faith in the absolute reality. Unable to give a precise description,
we chose to abandon the notion.
The element of tragedy here, being due to a mistake, appears as 
comi-tragical. It is the comi-tragedy of someone who catches a glimpse of a 
profound insight and tries to express it. But there are no unambigous 
formulations an no insights which, expressed by one, could not be misunderstood
by another. Attempting to express the insight as a thought, and the thought as a
universal formula, the epoch starts instantly to misinterpret it. The insufficiency 
of words is taken for the lack of reality, and
the uneasy feeling of this lack is suppressed by the 
pride of ``being a realist free and liberated from the prejudice''.

So, men ceased to live in the 
unchangable world, the world was no longer the ``best possible''.
Having mocked this last attempt to justify the actual world order as
the ``best possible among the possible worlds'', the best possibility (!?!)
 to find strenght for living in this present, far from perfect
world was to suggest that, eventually, it will turn better. 
A complete lack of psychological insight characterising this suggestion is by
no means an exception among the other invnetions of the enlightened 
philosophers. It does correspond well to the later events where people proved
dissatisfied with the eventual betterment of the world and, on the contrary,
impatiently insisted that it should occur now, tomorrow, during the next five
year's plan. Preaching the closeness of the ultimate kingdom of happiness 
-- or if you prefer, hope --
always follows the, no matter how idealised and exalted, idea of progress.

In fact, the relation bewteen the individual life and hopes, and the importance
of progress is quite intimate. An individual is immediately self-aware of his
identity and continuity of his being. Maintaining this feeling is a condition
of balance and psychic health and its disturbance is related to
some mental or psychic disorder. This feeling, which is also the feeling
of continuity of our world, is one of the fundamental spiritual needs.

An analogous need exists at the level of a group, be it a family, a nation,
culture. A group is also a spiritual entity and just like individual it has
its values, its past and its hopes for the future. However, it is not a
spiritual entity of entirely the same kind. An individual 
development from childhood through adolescence to maturity and old age, may
be, typically, considered a progress -- as a matter of fact, a progress in a
more profound than merely quantitative sense. It involves accumulation of
experiences but also their personal interpretations and conclusions. It 
involves learning from mistakes and, hopefully, moral improvement. In short,
it is not merely a progress of knowledge but also of being. 

Reasoning by a 
fatal analogy, one is led to ascribe similar features to the development of 
a group.
% , in particular a larger group like a society or even mankind.
The Jewish eschatology, inherited by the Christians,  provides an excellent 
example. Certainly,
St.John had mystical experiences of the wrath of God, of the annihilating
Apocalypse and the following resurrection of the purified spirit. So did many
others. But applying such individual experiences to the social context is an
act of unjustified transference. It is such a transference which brings the 
ideal of spiritual progress from the individual to the social context.

In modern terms one speaks of ``one generation rising on the shoulders of
the previous one''. I will refrain from drawing a detailed analogy of the 
picture thus suggested to the 
``Damnation'' of Rubens or obelisks of Vigeland. One imagines that those who
come later start from a different point than their predecessors did. They have
available all the knowlwedge and experiences that have been accumulated so 
far. This optimism goes as far as the claim that one can avoid the mistakes and
failures of previous generations. But can we really stretch validity of this 
talk about learning from history, from other's experiences that far? Isn't it
rather the case that, in spite of the thousands  years of history and
knowledge thereof, people are still prone to make the same mistakes, to 
commit the same crimes, to face the same uncertainity? What makes us believe
that a child born today has any better chances to live a happy life, to
become a noble, worthy human being, then children in the past had? Is it any
closer to the ideal of perfection only because it was born today and not 
thoudsand years ago. On these ways everybody must wander himself and it does 
not help that others where here before. The most they could have left are
ambiguous signs. We can benefit from their experiences only indirectly and
only to a very limited degree because what can be handed down is only 
knowledge about their life and being, not this being itself. All that can be
transmitted to posteriority has external character, is an object or a sign.
Only finite things can be accumulated and so we accumulate events, facts,
words, theories, knowledge. But a sign needs an interpretation,
and everything which can be interpreted can also be misinterpreted. Even the
most extensive knowledge does not ensure that facing a new situation the man 
will find enough tacit understanding, wit, compassion, wisdom, in short 
strength of being, to handle it.

And so, each generation, just like each individual, begins anew the life of 
possibilities, hopes and failures. The chance that it will produce Hitlers
rather than Einsteins is the same as it was for the emergence of Alexanders
rather than Aristotles.  We may allow ourselves to use the sign of ``human nature''
but we should not misunderstand it for a set of definite qualities granted to
every human being which can be improved in the long run of history. It
signifies rather a potentiality which in each case must find its own, new
realisation. It is an open possibility -- the possibility of attaining balance
and freedom, the state of grace, Nirvana, but also the possibility of
falling short of its ideal. The story of Faust may be old but is not, for this
reason, less actual. The soul is neither perfect nor corrupted by the Original Sin 
-- but it is prone to perfection as well as to corruption.
And the spiritual  possibility has
the importance of reality because it implies that such a reality may still
come true.

The thesis emerging from these considerations is that progress, as a substitute
for the absolute, is a special way of relating the presence to other, past
as well future, times. It is a substitute for the continuity of the world,
a dynamic, temporal reflection of the need for eternity. Of course, as any
idol, it cannot replace its god. Instead of establishing the continuity of
time, it breaks it, and in this break reaches its end.

\section{Twilight of the ``Myth''}
There is a malicious ambiguity in the relation of progressive thinking towards
 the past. On the one hand, one is ready to admit indebtedness
to the earlier generations -- without their contribution, {\em we} could not
reach the level we did. However, this is a deceitful indebtedness. It expresses
as much the feeling of superiority and contempt for the people of the old
times who couldn't enjoy the wonders of {\em our} times. For, after all,
our times are the best. They, poor, had primitive understanding, nourished 
illusions and prejudice, had to fight for food and survival, obey the lords, 
clean the clothes in hands - you mention it, they had to suffer it. This
rational progressive attitude towards the past hides, and sometimes not even 
that, the contempt for the underdeveloped ``primitive''. It is the ever 
self-content arrogance of reason which does create marvelous works but,
unfortunately, cannot avoid turning them into idols.

The arrogance of reason, like any arrogance, witnesses insecurity.
Although today's reason feels superior to the reason thousand years ago,
it has lost the other's humbleness. Destroying its master, it has exchanged
the static form of an ideal for a dynamic principle of action. And now it can
only dwell in, perhaps convincing, but then equally insufficient, partial and
relative determiniations. It is not any longer capable of positing the 
``certain knowledge of the totality of the world'' as its goal. 
When knowledge acquired its modern, scientific meaning such a claim should
sound ridiculous. Instead it
can only postulate ``science to extend the scope and certainty of knowledge''.
It sounds humble but, unless said with a real humbleness of some scientists,
reveals rather a philosophical despair. One ceased to compare each stage 
with an absolute and merely compares different stages with each other.
Certainly, 100 is greater than 1, but compared to the infinity both 
$\frac{100}{\infty}$ and $\frac{1}{\infty}$ yield zero. The postulate of relative
progress does have a descriptive value and is, perhaps, the highest expression
of the meaning of scientific progress. But seen from
outside it reduces to a crippled tautology: the meaning of work is more work, the
meaning of progress is to be able to progrss still further. We move away from
zero, from ignorance, through the stages, towards nothing. ``Always on the 
way'' is an appealing thought because it puts a free movement of wandering in
the place of a tense strive after goals. But the postulate urges us to move
faster and faster, with more efficiency, with more precision, with more, 
more~... There is a difference between being ``on the way'' and rushing on a 
highway.

The imperative of `more' becomes the ultimate imperative of the progress
expressing its quantitative character. Having substituted measure for finality,
we see progress only where there is an increase, not where there is a goal. 
We want the society,
law, freedom, human kind to progress and we are looking for the means of 
measuring them. And since it is not the intristic capability for being measured
but only our preconceived idea which forces the measurement, we
are getting blind to what lies beyond the limits of though, which became
the measure of the world.
It seems that the main, if not the only, contribution of the Western civilisation 
to the world culture lies in following this imperative of `more' without
a slightest reflection and reservation. We have the best developed science,
the most powerful economy, the highest standard of living, the most efficient
technology. We have achieved the highest skills in controlling the world. The
only thing which is left is to make them even greater.

But measuring the world, we lose the measure of things. Everything can be
compared to everything else and therefore no comparison matters. There 
emerges an apparent richness of differences without real significance - the
richness of {\em novelties}. Novelty is an ultimate pseudo-measure which a spirit
exhausted by the abundance of distinctions applies to their variety in the
last attempt to bring it in accordance with the demands of finite 
understanding. The inquiry into the structure and deep causes became, if not
impossible, then meaningless -- all one can still do is to recognize pure
differences. Two things are related only in so far as they are the same or
different. And what makes the difference is whether the thing is new or not.
The old is tiring -- it is known and boring -- and the oldest thing in the 
world, the life itself, is most boring of all. Novelty is a drug against the 
past. Escaping from the old, known past we are running into the future which,
imperceptibly, becomes merely a new present. The world advances only when
something new happens and nothing really happens unless it is something new.
So the world advances perpetually, every moment brings something new, but in
spite of all these novelties and revolutions it somehow remains in the same
place. We advance from one moment to the next, in a series of disrupted steps
breaking the continuity of time, the most profound mark of being. And a moment
devoid of the expectation of eterninty becomes a desperate expectation of the
next moment.

Heidegger might have been right in considering the superficial curiosity as
a constant feature of the inauthentic mode of being and, as such, an 
indispensable aspect of our existence. But the difference in proportions may
reveal a more fundamental difference. The difference between idle talk, casual
curiosity and thorough fascination with novelty makes such a difference: the
difference between stupidity or, sometimes, mere moments of relaxation and
insignificance, and the ultimate despair, metaphysical boredom.

But just like idols fall when facing the God, so does the insatiability of
the no more controllable process, the pride of the architects of the Babel
 end when facing the unlimited. Doing away with the
finality of finitude, we are left with the insatiable desire for `more'. 
But how can we postulate, lest require, that a finite being should grow and 
expand without limits? The unrestricted imperative of `more' contradicts the 
very essence of our finite being. 
Descartes
blamed our will for the errors we make because, while reason operates with
clear and distinct (finite and controllable) ideas, the will wants to be
unlimited. It urges the reason to never stop, to reach for everything which
imagination suggests to it. But outside the domain of finite certainity there
is nothing a sheer concept could take into possession and control. There is a deep
difference betwen what one can take oneself and what one can be given.
We may hope and ask for many things which turn into nothingness in the moment we 
begin to believe they can be conquered and taken by force.
Beyond the finite limits of visible goals
there is nothing to achieve; at best only something to be hoped for.

The overall imperative of `more' means not only more money, power, products,
but also more possibilities, freedom, knowledge, books. 
Scientist produce such a number of results that specialists are unable to read
even this portion which is related to their narrow field. And this for the 
simple reason (if only one among others) that also science must be measured but
this can be hardly done otherwise than by counting the number of publications.
The information overflow makes access to any particular piece of knowledge
increasingly difficult. Computer technology is in this respect like the Troyan
horse -- eventually it merely helps to store more data. The access to them,
however, becomes so complicated that it requires a new kind of knowledge. Like
the unproductive buying and selling of money is one of the most profitable 
financial activities, so does storing and retrieving of information becomes
one of the central branches of knowledge. The will to control of the world and
the wish for `more' are about to turn the society into a huge book-keeping
company. 

Overproduction of useful as well as useless material and conceptual
goods exceeds the capacity of any individual. We want to believe that, 
eventually, the whole mankind will benefit from the process. But mankind has
never been wiser than its wisest representatives. If no individual is capable
of understanding, how could the mankind understand? We work so much that
we cannot tell Monday from Sunday. The ritual and celebration of leisure, being
superfluous or waste, disappear under the amount of tasks. And when no days,
no occasions deserve particular attention, what is the significance of the 
passing time except, perhaps, that we are getting more tired and old.

The progressive activity, when carried to its extreme, turns into its opposite.
This does not simply mean a regress which, in a sense, is a special case of 
progress. This means termination of every movement, dissolution in the 
indifferent, unqualified homogenity of chaos which is turning into nothingness.
It is like following in the footsteps of catastrophy theory according to which
any growth has a limit, and that the faster it accelerates, the more
violently and abruptly it ends.
When progress becomes Progress it approaches its end - chaos, in the form of
insatiability, uncontrollability of the processes which started with the 
apparently  well defined goals and means, overflow of apparently true but
in fact insignificant meanings.

Too much light used to blind and too high flights used to end in destruction.
The sins of gluttony and avarice are variations on the theme of insatiable
desire, and we have also been warned against the vanity of the search for
knowledge. Constant struggle to obtain `more' only increases dissatisfaction
with the past and thirst for the future. Incomprehensibility of chaos is a 
form of infinity 
which we are learning to know. But its finite reflection is nothingness; 
nothingness of a finite being confronted with the overpowering. When we almost
reach the end, when the fulfillment seems closer than ever, we suddenly find 
that our hands are empty, that we are standing in void. We believe that we 
accumulate while we are wasting.

\section{Continuity}
A finite being confronted with chaos may do essentially two things. It may
say ``yes'', that is, accept it as a fact pertaining to its finitude, which 
means, ignore it. Consequently, it will
try to come up with some curious or even ingenious, but always meaningless
 constructions attempting to appropriate the phenomenon. Or it may
say ``no'', that is ``no more''. It may recognize the chaos as something 
transcending and overpowering, not belonging to the finite and, consequently,
give up and surrender to the dangers to which a finite reason is
exposed when enthralled by the breath of infinity.

Although futurologists seem still to have the scientific pretensions, 
we have left behind us the solution of utopian ``yes'', the arrogance of 
Enlightment and naivit\'{e} of positivism. One ``myth'' has fallen and 
historical consciousness is looking for a new one. It seems that we are 
beginning to experiment with the solution of ``no'':
of wild imagination and distorted feeling of reality, or rather unreality.
We are trying to get rid of the abolutistic illusions of objectivism by 
dissolving everything in a flux of uncontrollable processes - believing that 
in this way we remain true to our finitude. But thus we forget that it is an
object, a limited, definite entity which is the correlate of the finite reason.
Its dissolution leads not to truth but to confusion.

I do not want to draw yet another  picture of an apocalyptic end.
The world is  more sober than such exaggerated  creations, be they of guilt, 
dread or idealism.
Having established a new ``mythology'', one will probably continue to believe
that a progress has been made over the past generation of reason. History is
a succesion, raising and falling of ``myths''. Each new one believes itself
to be better than the earlier ones. Each one argues with the other ones as
they pass in the noise of quarrels and arguments. 

But an argument with the past is just an argument with the present. It is a
disturbed contact with the present which makes respectable past impossible.
The continuity broken by the
insatiability of progress is as much a break in the relations to the past as
to the present. The insecurity of reason, resulting from the 
destruction of the absolute, believes to find a firm basis
 in the iron necessity of the truths which have been proved by all available 
means of persuasion.  The break, like any other relation of being, manifests 
itself in the language, in what often seems
{\em the} virtue of reason -- the ``argumentative discourse of rational 
agents''.
The very need of argumentation is a sign of an opponent, or rather somebody
perceived as an opponent, who should be defeated. We are not satisfied with
merely communicating our meanings, we have to argue for them, they have to be
proved. And as the objects became more exact, so should the arguments. 
A mere declaration of a standpoint is something only fools allow themselves. A
respectable intellectual will always try to force his opinion on the 
interlocutor. True, force by means of the arguments but {\em force} 
nevertheless. And an argument which is not a sign of fear is a sign of lacking 
respect. It arrogantly disregards the ``subjective'' attitudes of the opponent
claiming all and the only  plausibility and rationality. It is not arbitrary and 
subjective, it has
the full power of the objective, that is, indisputable truth. One would like
to say that ``coercion begins where reason ends''. But this merely sets the use
of violence or physical pressure against intellectual discourse. As a matter
of fact, it is the use (or rather misuse) of reason, convinced of its 
objectivity and truth, which acts as a force disregarding human freedom and 
personal merits.

The obssesion
with the objectivity of argumentation goes so far that it denies the simple
truth that a sufficiently intelligent person can argue for, as well as against
{\em anything} with equal success. All things concerning something more than
purely scientific issues (and even some concerning such issues) may be argued
for and against with equal plausibility. This is certainly easier to grasp for
a human being than for a rational agent, and is definitely too much for such
an agent involved in a ``community of ideal, rational discourse''.
An argument concerned with a necessary law is fully natural and justified --
it is called a ``proof''. But there is a fundamental difference between the
mathematical or physical necessities with wich proofs may be concerned, and
the spiritual truths which only conditionally may be termed necessary. The former
are unavoidable -- one cannot prevent, by usual means, a body from falling when 
dropped from a hight. The latter can always be violated although such violations
will, typically, have their price. One may say ``you shall not kill'' but 
trying to force this maxim by means of arguments will fail unless the other too
has an access to the same values and moral intuitions. And even if he does, he
may still oppose them.
An argument concerning something more than plain matters of reason, especially
concerning the matters of existential relevance, is a way of depriving the
other of the dignity of his freedom. It simply does not leave any room for 
freedom -- only for seeing the correct solution or remaining blind and stupid.

``Myths'' pass in the skirmishes of history, but myths remain. For while ``myths''
argue, myths reveal.
And the only wisdom  is the same as in any, even the oldest, myth.
If we have made any progress whatsoever in 
a more profound sense, then what is it that still makes us, some of us, keep
looking for the right questions and answers in Veda, Bible, Summerian myths,
Plato? Should not these texts, thousands of years old, be ignored as so many
ingenious Voltaires and Russells would like us to do. And if the phrase
``right questions and answers'' sounds too naive, then what makes us read them
at all? It is not merely ``the exotic atmosphere of the myth'' or ``the 
original combination of phonems''. 
The language of such texts does not fit into the ideals of precision and 
exactitude, and yet it communicates most significant truths to the unbiased
readers. The language of a myth is not the language of the arguments. 
Revelation has no need for arguments.
It leaves it
to the freedom of the listener to accept its truth or to oppose it. The words
of such texts claim
to reveal the truth, the ultimate truth, but not a necessary and unavoidable
truth. They refer to a truth of being and acknowledge that it is different from
the words in which it is expressed. And just in that they realize their
insufficiency, their provisional and auxiliary character, consists their power.
They do not complain over being misused and misinterpreted. They even invite to
such misdeeds not caring about apparent consistency and intelligibility. They
argue for nothing and prove nothing. Whoever wants, or is inclined to do it, may
freely make his own misinterpretations. The words are there not to do the job
for a man, but to help him in the case he wants to receive such a help. They
talk about solutions but do not contain them - the solution cannot be merely 
said, it must  be re-created, re-constructed. They are at best hints, 
indications, allusions. They know the distance separating them from the reality
of human soul and do not pretend to be anything more. In this lies the great
power of things silent and invisible. They are not like superficial
distractions which try aggressively to attract one's attention, anxious to
lose it. They start attracting only when one  turns towards them 
realising the need for their guidance. Thus remaining outside, they come from
within.

Because they do not try to force their meaning on the reader but wait until he
himself discovers it, they remain the same. But each time a man freely finds
the meaning of this sameness, it becomes coloured by his personality, acquires
a new valor, appears in a new variation. In this continuity transcending the
actual differences, the new is not an enemy of the old but merely its new form.
The high things respect human freedom
because they, unlike concepts and ideologies, do not become less by being 
shared by more. 

Furthermore, such an indirect communication requiring every time a new consent
of the will, and a new reconstruction of the meaning, is for these very reasons
a relation of being. Let us call it {\em communion} to distinguish it from the
communication as a mere relation of knowledge.
It isn't a mere intellectual contest of two parts but a
participation of one in the other. Just like the apparent ``unclarity'' of
the words respects their reader, so the reader will hardly grasp their meaning
without an attitude of respect and openness.
It may be compared to other situtations where the meaning of an event
transcends the boundaries of the directly expressible. For instnace, the power 
of a personality may be so overwhelming that it engraves its mark on another 
person
without his active opening and understanding. If he tries to tell ``what
really happened'' he may find no appropriate words. And yet, there remains an
unmistakable feeling that something important has happened, that some earlier
attitiudes are no longer so obvious, perhaps not even possible, as they used
to be, in short, that something has been communicated even if the immediate
sense of this something remains unclear.

The continuity of historical or individual being, is 
the name we have given to the feeling of identity reflecting the totality
of the world in which one participates.
 This participation is not established by arguing with the others, not even by
interpreting the others' words, for interpretation still emphasizes the distinctness 
and distance from our own standards. It is established
through the communion of the truths which we recognise as {\em potentially 
our own}. Actually, they may be not ours for the reasons of differences in 
tradition, in the ethos of the respective cultures, in individual background.
But the understanding attitude towards the others, and to the past in 
particular, is possible only if these differences are recognised but not
absolutised. If one draws the conclusion that, because of these differences, we
are different as human beings or even, as it often happens, that  we are 
{\em better} than the others, the possibility of communion gives place to
the necessity of argumentation or interpretation.

Communion, being a relation of being, cannot be limited to a mere
intellectual exchange of concepts. It is a participation, that is {\em being},
being {\em with} and being with {\em another}. 
In order to appreciate a different culture, a different myth,
one has to participate in it, not necessarily in the direct physical  but, at
least, in the spiritual sense. One has to recognise its values and truths as
valuable and true, as humanly meaningful and respectable. 
This is often more than      % the simple universalism of 
reason, haunting for
``universal validity'' and therefore perceiving differences as contradictions, 
is capable of. Its universalism forgets that some questions are simply 
irrelevant. One myth need not be compared with another.
As long as I grow up and live in one culture, it is irrelevant for my 
existence how another one expresses its
ethos and faith. It becomes disturbing first when my own ethos and faith are
shaken and I am no longer able to believe and live them. This may lead to
a search for other myths which often will take the form of comparing various
cultures and measuring them against each other. This, however, is 
intellectualising, not participation. One can live in one, perhaps two
cultures at once. Comparing them all is an empty excercise of reason.

What establishes communion with the past is not interpretation which is an
intellectual activity biased, for the most, by one's own horizon of values and
prejudices. It is the kind of living participation, if
only spiritual, in its values. It is recognition of their fundamental meaning
for a human being as such, recognition of the fact that, potentially, they 
might have been my values, which is the basis of understanding that they may 
be the values of another. 
Communion is based on such a respect  which underlies
 the possibility of learning from the other. And even if we discover flaws and
mistakes, they  are mistakes which we too could make.
Reading the central texts of the past 
generations in this way establishes the mutual contact where new is only a new
form of the old, and where the reciprocal relation expresses the continuity of
human kind. 
%Such a relation is possible not only within a 
%particular tradition but also across different traditions and cultures. 


\section{The End}
``Myths'' have an important epistemological value. It consists in unveiling 
some fundamental need. Recognizing that the destruction of some values, 
the fall of
some aspect of life leads to the emergence of a ``myth'', indicates the 
essential importance of this aspect. An idol indicates the need for God.
How to distinguish between them is no subject for arguing. All one can say, 
if one can say it, is only ``this is God'' and ``that is an idol''.

The differentiating activity of reason and its ``myth'' made us so 
used to the image of ever changing reality that we are hardly able to look at
our ancestors as human beings of exactly the same kind as ourselves. 
Consequently, we lost the ability to communicate with them. We only 
got very clever in finding excuses for or formulating accusations against them.
Excuses and accusations of historical, technical, sociological, economical, 
psychological character. There is no doubt that things change. But only passing
from this fact to its absolute generalisation would mean that nothing remains 
the same, that there are only changes, perhaps even a progress, but no 
contnuity. 

Changes do not exclude continuity while continuity
without changes might turn monotone and unreal. 
There are different levels of spiritual
reality and only those who are unable to live at all of them have to find one
which they could endow with the absolute value.
{\Large \bf - - - - - - } 
There is a level of gold, the source of eternity preceding the world as well as the gods,
where humans live beyond the distinctions in a harnonious community like the gods 
themselves. The  level from which, after ``the eternity is created'',
the inaccessible and ``good spirits protect the mortals'' and dictate the course of their 
life.
There is a level of silver at which more distinct emotions begin to give rise to the 
opposition of good and evil. The level where pride and impiety may be unwilling to respect
the gods, but which still deserves the respect and is remembered for its high values and
 achievements; not so perfect as the golden one but still full of the potential for 
veneration and holiness.
There is a level of rudy, where evil, icy passions leave no room for sanctity although
still keep the individual above the concern for immediate goals, in the proximity of 
the supra-temporal. The level of merciless, cruel passions arosen not by the silver 
images of gods but by the ruthless pride of the great war-lords, who pass 
forgotten in the skirmishes of history. 
And there is a level furthest removed from the divine mysteries, the
level of iron necessities, of daily ``needs and work'' from which
one can never ``find a rest''. The level at which human life is filled with toils and 
sorrows, where the needs of everyday life make the world of gods seem unreal;
the level bereft of the greateness of the high values, departed even by those gods who 
otherwise would ashame and accuse the ``unjust rule of the brutal force''.


The end is not the same as the beginning just like a return to a place is not the same as
the first visit. But the place is still the same -- it is the world where one can attempt 
to rise to the level of gods or nourish the resignation and whish that one ``was already 
dead or not yet born''.  History is a perpetual 
reflection over the beginning -- each generation, longing for the golden age 
restlessly looks for the original which, in 
its historical language, it calls ``final''. But as the beginning remains always beyond
what follows it, so the successive ``myths'', convinced of their superiority over the
previous ones, can never fully grasp it. Beginning, being of a
golden and divine nature cannot be appropriated by the iron tools of the 
mortals. It is only reflected in their works and days, can be indicated or
hinted at, but not grasped and communicated by means of the hard  concepts.
Myth, on the other hand, is not a conceptual reflection {\em over} but a 
living reflection {\em of}
 the beginning. Therefore it does not change  with the history nor does it change the
history. It only reminds the history, which infatuated with ever new ``myths''
loses the measure of eternity,
about the Beginning which each finite being carries within itself until its very end.


%Either we live in the light of the absolute, or we cannot be satisfied.
%Either we accept the absolute or we have to accept dissatisfaction. 



\end{document}