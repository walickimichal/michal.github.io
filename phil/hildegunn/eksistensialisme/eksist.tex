\documentclass[12pt]{article}

\makeatletter
\input{a4wide}
\makeatother

\input{xypic}
\xyoption{curve}

\newcommand{\cici}{\setlength{\unitlength}{1cm}\begin{picture}(6.5,3)
\put(3,2){\circle{2}}
%\put(2,2){\circle{0.6}}
\put(3,2){\makebox(0,0)[c]{Jeg}}

\put(3,3.5){\makebox(0,0)[c]{\co{Intet}}}
\put(3,0.5){\makebox(0,0)[c]{\co{Intet}}}

\put(0,1.5){\makebox(0,0)[l]{\parbox{1.9cm}{1.som min\\ endelighet}}}
\put(6.5,1.5){\makebox(0,0)[r]{2.som tomhet}}

\put(2.8,1.8){\vector(-1,-1){1}}

\qbezier(3.2,1.7)(3.3,1.4)(3.6,1.4)
\qbezier(3.6,1.4)(4,1.7)(3.6,2)
\put(3.6,2){\vector(-1,0){0.2}}

%(2.5,2.0)(2.4,2.0)
\end{picture}
}

\newcommand{\no}[1]{$\{${\it{#1}}$\}$}

%\newcommand{\cit}[2]{\citt{#1}~\footnote{\cite{#2}} }
%\newcommand{\citt}[1]{{\em ``#1''}}
%\newcommand{\cita}[2]{\citt{#1}\footnote{#2}}

\newcommand{\nei}[1]{}
\newcommand{\com}[1]{{\footnotesize{[#1]}}}
\newcommand{\MyLPar}{\parsep -.2ex plus.2ex minus.2ex\itemsep\parsep
   \vspace{-\topsep}\vspace{.5ex}}

\newcommand{\co}[1]{{\bf\em #1\/}}
\newcommand{\citt}[1]{{\em #1}}
\newcommand{\ut}[1]{\\ \ \ .\dotfill{\small\bf #1}}
\newcommand{\und}[1]{\underline{#1}}
\newcommand{\ovr}[1]{\vspace*{2ex}\par\noindent{\bf...#1...}\vspace*{1ex}}
\newcommand{\ite}[1]{\item[{\bf #1})]}

\title{Eksistensialisme}
\author{Micha{\l} Walicki}
\date{April, 1999}
\begin{document}
%Merkelapper er farlige; spesielt for eksistensialisme som framhever det
%individuelle, det spesifikke. Vi m{\aa} abstrahere og generalisere her...

\section{Filosofi -- et fors{\o}k p{\aa} {\aa} ...}
forklare det konkrete, det individuelle utifra det abstrakte, det allmenne:
\begin{enumerate}
\ite{f1} finne det \co{Absolutte}, den {H{\o}yeste, Siste Sannhet}\\
Gud, Verdens{\AA}nd, Bevissthet, Ideen, VerdensSjel, Fornuft...
\ite{f2} fange det og uttrykke i ord og sammenhengdende lover, 
begreper\\
statisk, logisk, rasjonelt system
\ite{f3} avlede resten -- verden, tingene, enkelt menneske -- utfra de allmenne
lover\\
enkelt menneskes eksistens f{\o}lger (fra) disse lovene, er et {\em spesielt
tilfelle} av det allmenne \ut{det Allmenne -- det Individuelle}
\end{enumerate}

\subsection*{Ekstremt tilfelle av det...}
er rasjonalisme -- fornuftsfilosofi, der alt f{\o}lger faste lover -- med
ytrepunkt av objektivisme/determinisme/scientisme.

\subsection*{Psykologisk -- hva er motivasjon for et slik prosjekt?}
Man starter gjerne med ``fundamentale sp{\o}rsm{\aa}l'': 
\begin{itemize}\MyLPar
\item Hovrfor er jeg?
\item Hvorfor er jeg den jeg er?
\item Hvorfor skal jeg gj{\o}re det og ikke det?
\item Hvorfor er verden?
\item Hva som {\em egentlig} er?
\end{itemize}
som ikke  har entydige, allmengyldige svar. Dog, denne ``hvorfor'' ber om et {\em
svar}, uttrykker lengsel etter
\begin{enumerate}\setcounter{enumi}{3}
\ite{f4} trygghet -- urokkelig sikkerhet (Absolutt Sannhet)
\ite{f5} n{\o}dvendige lover -- hvis ikke determinisme, s{\aa} i det minste fornuftig, rasjonell
forklaring 
\end{enumerate}
Midt i s{\o}ken etter slike lover og sannheter, g{\aa}r menneske tapt --
lover (leder kanskje til vitenskap men) viser seg {\aa} v{\ae}re eksistensielt
irrelevante.

\ovr{Hvorfor? -- Slik er det bare, visse ting kan ikke defineres og uttrykkes direkte\\
En man som skulle for{\o}ke {\aa} forklare alt, f{\aa}r ikke tid til {\aa} leve...}

\section{Eksistensialisme...}
%
avviser alle 5 {\bf f}-punktene over. 
\citt{``Det er umulig {\aa} skape et system som fanger virkeligheten.''}

\begin{enumerate}
\ite{e1} eksistens er alltid  \und{individuell}: min, din -- ikke
ens; og \und{unik}: ikke-gjentakbar\ut{essens -- eksistens}
\ite{e2} eksistens er prim{\ae}rt et \und{sp{\o}rsm{\aa}l ved seg selv} -- Hvordan
skal jeg leve?\ut{man -- individet}
%\ut{sikkerhet -- usikkerhet}
\ite{e3} eksistens er alltid \und{konkret} -- \co{kastet inn i verden},
befinner seg i en bestemt \co{situasjon} som den ikke har skapt
selv\ut{altomfattende -- endelig}
\ite{e4} eksistens konfronteres med \und{muligheter} -- ikke entydige
fakta men egne \und{valg}\ut{n{\o}dvendighet -- mulighet}
\end{enumerate}
\subsection*{Eksistens g{\aa}r forut for essens = }
Man har ikke en oppskrift, et fast grunnlag i objektiv viten og verden for
sine valg og liv -- kun gjennom aktivt liv og valgene ``skaper'' man: seg
selv.

Personlig opplevelse, valg og erfaring g{\aa}r forut for argumenter,
begrunnelser og lover.

\ovr{Individ-problematikken har v{\ae}rt der f{\o}r men eksistensialisme for f{\o}rste
gang samler disse punktene og gjennomf{\o}rt velger det andre
alternativet...\vspace*{1ex}}
\begin{itemize}\MyLPar
\ite{e1-e2} Sokrates, \citt{``Kjenn deg selv''}
\ite{e4} humanisme: menneske styrer sitt liv selv
\ite{e3} marxisme (samtidig): et annet begrep av kontekst, \co{situasjon}
\end{itemize}
\section{Oppdagelse av individet skjer i Kristendommen} 
\begin{itemize}\MyLPar
\item individuell samvittighet %(Rom.14,23)
\item individuell skyld
\item individuelt forhold hvert menneske har til Gud, ...
\end{itemize}

\noindent
\begin{tabular}{r@{\ :\ \ }l@{\ -\ }l}
Tertullian & \citt{``Jeg tror fordi det er absurd''} & {\bf e4}\\
St.Augustin & \citt{``Finner du at din natur er vaklende, overg{\aa} den (deg
selv)''} & {\bf e1}\\
Pascal & \citt{``Mennesket er bare et siv i vinden''} & {\bf e2, e3} \\
Kierkegaard & \citt{``Subjektiviteten er sannheten''} & {\bf e1-e4}
\end{tabular}
\ut{fornuft -- tro}

\begin{tabular}{rcl}
``subjektivitet'' & = & ``objektiv usikkerhet''\\
& = & ikke-tinglihet/udefinerbarhet\\
& = & konfrontasjon med det {\aa}ndelige
\end{tabular}
\ut{tinglihet -- udefinerbarhet}

\section{Det uforklarlige, paradoksale...``absurde''}
\citt{``Man kan ikke \und{bevise} at Jesus var Gud -- fordi det `{\aa} bevise'
betyr {\aa} gj{\o}re noe tilgjengelig for fornuften, mens dette er
nettopp en paradoks som strider mot fornuften''.}
\begin{enumerate}\MyLPar
\item treenighet, kroppens gjenoppstandelse, transsubstansjon, ...
\item Hiob -- uforskyldt lidelse
\item H{\aa}p -- ikke fortvil selv om det er all grunn til det (og ingenting annet)
\item Abraham som (nesten) offrer Isaak -- Gen.22,1-14
\end{enumerate}

\section{\co{Valg}}
Abraham er bare et eksempel som, for Kierkegaard, gir en arketype, et
paradigme for den enkeltes ensomhet i valgsituasjon
\begin{enumerate}\MyLPar
\item Abraham m{\aa} velge i ensomhet, alene, uten hjelp fra andre -- fordi
andre ikke ville v{\ae}re i stand til {\aa} forst{\aa}... 
\ut{felleskap -- ensommhet}
\item Han velger faktisk {\em p{\aa} tross} av lover, normer, skikk...
\ut{normer, skikk -- individ}
\item Siden \co{valget} er ``absurd'' kan han ikke st{\o}tte seg p{\aa} noen
objektive kriterier: lover, kunnskap, fornuft...  
\ut{objektiv -- subjektiv}
\item Siden \co{valget} ikke har noen begrunnelse utenfor ham selv, b{\ae}rer
han alene fult ansvar \\
kan ikke skylde p{\aa} samfunnet, systemet, oppdragelse...
\end{enumerate}
%\ut{valg 3,4,5}
%
\ovr{\co{Valg} skjer ``utenfor verden'' -- den \co{individuerer!}}
%
\subsection*{og \co{Angst}}
Et slikt \co{valg} skjer ``utenfor den vanlige verden'': den objektive,
verifiserbare, felles, intersubjektive verden av ting og andre mennesker. Man
bytter `objektiv trygghet' av denne verden mot `objektiv usikkerhet',
dvs. subjektivitet, {\aa}ndelighet, et ``indre kall''.
\ut{trygghet -- usikkerhet}

\co{Angst} er en opplevels som markerer at man har mistet ``fast grunn'' i
``denne verden'' av ting. I denne verden kan man bare oppleve \co{frykt} --
ovenfor en bestemt situasjon: en fiende, en tiger, tap av penger. Angst har
ingen ``objekt'' som for{\aa}rsaker den, den er enkeltes konfrontasjon med en
``ytterste grense'', det ikke-objektive, udefinerbare, uforklarlige... Det er
et m{\o}te med Gud -- eller seg selv -- uten oppskrift eller regler som man
kan l{\ae}re fra andre. Man m{\aa} m{\o}te det -- kaste seg ut p{\aa} dypt
vann -- eller flykte og gjemme seg i trygghet av sm{\aa}borgerlig normalitet.
%
\ut{frykt -- angst}

\section{Det ytterste \co{valget}}
\begin{itemize}
\item[a)]
N{\aa}r jeg m{\o}ter min ytterste grense, hinsides hvilken ingenting mer kan
sees, gripes eller fornemmes...
\item[b)]
Foran et lukket rom: er det noe der eller ikke? 
\begin{enumerate}\MyLPar
\item Ja, jeg tror det
\item Nei, jeg tror det ikke
\item Jeg vet jo ikke, s{\aa} jeg vil ikke svare!
\end{enumerate}
Det siste kan man ikke si i en eksistensiell situasjon -- \citt{``Det {\aa} la
v{\ae}re {\aa} velge, er ogs{\aa} {\aa} velge.''}
Det {\aa} ikke svare betyr {\aa} ikke forholde seg til det, dvs. {\aa}
forholde seg som om ingenting (var der)!
\item[c)] Er det noe hinsides eller ikke? 
Er verden meaningsfull eller ikke? Er livet det? -- ingen allmenngyldige svar;
ethvert svar avsl{\o}rer han som svarer (hvordan {\em han} opplever {\em sitt} liv)
\end{itemize}
\[\cici \vspace*{-2ex}\]
\begin{center}
\begin{tabular}{l@{\hspace*{1.5em}}c@{\hspace*{1.5em}}l} %{l@{\hspace*{4em}}l}
min endelighet & -- & min absolutt frihet \\
{\aa}ndelig mening & -- & frav{\ae}r av mening\\
objektiv usikkerhet & -- & total usikkerhet \\
Gud er ubegripelig, & -- & \citt{``Gud er d{\o}d''},\\
men vi kan tro         & -- & vi er ikke lenger \\ & & \ \ {\em i stand} til
{\aa} tro
\end{tabular}
\end{center}
%\co{Intet} -- ikke som ``objektiv usikkerhet'' men som

\ovr{1.Kristen vs. 2.ateistisk eksistensialisme}

\newpage
\section{Ateistisk eksistensialisme}
tar opp hovedbegreper (fra Kierkegaard) men anvender de p{\aa} noe
annen m{\aa}te
\begin{itemize}\MyLPar
\item menneske konfronteres med faktisk verden og frav{\ae}r av objektiv mening
heller enn med {\aa}ndelig dimensjon av eksistens der objektiv verden kan
finne mening
\item grunnet frav{\ae}r av mening i verden, blir \co{valget} likegyldig, det
endrer ingenting vesentlig, og videre
\item det m{\aa} gj{\o}res hele tiden, i enhver situasjon
\end{itemize}

\begin{enumerate}\MyLPar
\item Jeg er \co{kastet inn i verden}
\begin{itemize}
\item
m{\o}ter situasjoner som jeg ikke har skapt og ikke har full kontroll
over
\item verden overg{\aa}r meg, er ``st{\o}rre en meg''
\item jeg er \co{endelig} -- ja, faktisk -- {\em fremmed} i denne verden
\item jeg m{\aa} velge uten {\aa} ha fullstendig oversikt
\end{itemize}
\item \co{Valget} m{\aa} gj{\o}res:
\begin{itemize}
\item i \co{grensesituasjoner} (d{\o}den, stor tragedie, tap, ...)
\item men ogs{\aa}, essensielt, hele tiden, i enhver situasjon, siden verden er
bare kaos av likestillte muligheter
\end{itemize}
\item Verden, tingene og situasjonene i verden, konfronterer meg som
\co{faktisistet}
\begin{itemize}
\item de har ingen iboende verdi eller mening
\item de er n{\o}ytrale og gir meg ingen grunnlag m.h.t. hva jeg skal velge
%\item jeg m{\aa} velge uten {\aa} kunne st{\o}tte meg til objektive begrunnelser
\item alternative \co{valg} er likegyldige, men man {\em m{\aa}} velge...
\item kun gjennom \co{valget} kan man muligens ``skape'' noe menigsfult (rebell)
\end{itemize}
\end{enumerate}
%\ut{valg: 1,3}
\co{Valget} gjelder ikke lenger (som hos Kierkegaard) {\aa}ndelig dimensjon, valget mellom verdier eller
livsformer men,
hovedsakelig, faktiske situasjoner i verden uten noen iboende mening.

Menigsl{\o}s verden; menneske konfronteres med absurd, likegyldige
muligheter, der \co{valget} m{\aa} gj{\o}res uten at det har hverken grunnlag
eller {\em betydning}:
\begin{itemize}
\item ``Sisyfosmyte'', Camus: likegyldighet av alle alternativene
\item ``Mens vi venter p{\aa} Godot'', Beckett / ``Nausea'', Sartre --
tilv{\ae}relsens bunnl{\o}se meningsl{\o}shet / eksistens, det ``{\aa} v{\ae}re'' er kvalmende
\item drap hos \\
\hspace*{-1em}\begin{tabular}{l|l}
Abraham & Den Fremmede \\ \hline
pinefull men n{\o}dvendig  & bare en tilfeldig realisering av
\'{e}n mulighet \\
gjelder en ytterst viktig sak -- & likegyldig og
gjelder ingenting \\
{\aa} f{\o}lge Guds ordre & kunne like godt velge noe annet \\
vekker Angst -- konfrontasjon med {\aa}nd & vekker ingenting
\end{tabular}
\end{itemize}

\newpage
\section{\co{Autentisk eksistens}: ``{\aa} leve gjennom \co{valg}''}
Stammer fra Kierkegaard, men ble noe modifisert pga. den underliggende
meningsl{\o}sheten. 
\begin{enumerate}
\item kritikk av tilv{\ae}relsen preget av `folket', `mengden', `media',
`man'
\item[{}] dvs. av \co{forfall} i det daglidagse og uegentlige, i \co{d{\aa}rlig tro}
av selv-bedrag, av flukt i tingenes og meningenes objektivitet  
%(\citt{``Jeg vet ikke hvem jeg {\em egentlig} er. Dette er ikke meg!''}) 
\ut{mengden -- individ}
\item selvstendig, ansvarsfult \co{valg}, der man konfronteres med
\item verdens likegyldighet som har overtatt klassisk rolle av \co{Det Onde}
(`det Ondes problem' har blitt problem av `det Likegyldige/Meningsl{\o}se')
\item som i ``Pesten'', ``Sisyfosmyten'' av Camus -- rebell mot verdens likegyldighet
\end{enumerate}

\ovr{tendens til {\aa} gjenfinne noen verdier `i menneske'
p{\aa} tross av den objektive n{\o}ytralitet (``Pesten'', Camus; marxisme av
sen Sartre; Abbagnano -- humanistisk eksistensialisme)}

\newpage
\section{Konklusjon...}
Ateistisk eksistensialisme -- Sisyfos, s{\o}ker men klarer ikke {\aa} finne
noen mening
\begin{enumerate}
\item Mot sine intensjoner -- preget av verdens \co{faktisitet}; \co{Intet} blir
mer og mer tomhet og frav{\ae}r av {\aa}ndelighet
\item \co{Valget} er meningsl{\o}s og vilk{\aa}rlig
\item En deskriptiv analyse av verdensanskuelse
som bygger p{\aa} underliggende frav{\ae}r av mening, konfrontasjon med kaos
av likegyldige muligehter
\item forf{\o}risk pga. skarpe beskrivelser av psykologiske mekanismer og
reaksjoner p{\aa} situasjoner som vi faktisk m{\o}ter i  v{\aa}r verden
\begin{itemize}\MyLPar
\item krise, n{\o}d, skuffelse, tap, dyp tragedie...
\item II Verdens Krig dannet grunnlag for fundamental tvil om meningen med
menneskelig samfunn og gjensidhet \\
\hspace*{-3em} I dag -- ``frihetens'' samfunn
\item overfl{\o}d av valgmuligheter som man ikke klarer {\aa} forholde seg til, ikke
har grunnlag til {\aa} velge mellom
\item ingen felles verdier av fundamental ({\aa}ndelig) karakter, ingen
allmenne normer som man kan ``vokse inn i'' gjennom oppdragelse
\end{itemize}
\end{enumerate}
\ovr{men verden er ikke bare fakta og faktiske muligheter}

\noindent
Kristen eksistensialisme -- ivaretar i st{\o}rre grad Kierkegaards dype intuisjoner
\begin{enumerate}
\item Det at noe ikke kan defineres, betyr ikke at det ikke er
virkelig. \co{Intet} er ikke tomhet men v{\aa}r \co{endelighet}.
\item \co{Valget} er ikke objektivt begrunnet men ikke dermed meningsl{\o}s eller vilk{\aa}rlig
\item Mening og verdier er ikke bare subjektive (i klassisk, negativ
forstand); menneske {\em m{\o}ter} de -- ikke som faktiske forhold men -- som
udefinerbare ({\aa}ndelige) og dermed desto sterkere intuisjoner, i {\em konfrontasjon} med 
sin endelighet (angst)
\item
Gir en rekke dype innsikt i menneskets
lengsel og streben etter verdier og holdninger som ikke kan begrunnes
objektivt, og som -- {\em av den grunn!} -- overg{\aa}r og er sterkere enn den
faktiske verden
\item \citt{``Virkelig er det du ikke kan leve uten.''}
\end{enumerate}

\newpage
\section{Noen av hovedpersoner og  litteratur}

\subsection*{Kristen eksistensialisme}
%
\begin{tabular}{lcll}
{\bf S{\o}ren Kierkegaard} & {\bf 1813-1855} & Danmark & ``Frygt og B{\ae}ven'', \\
    & & & ``Om Begrebet Ironi'' \\
    & & & ``Begrebet Angest''... og alt annet \\
Miguel de Unamuno & 1864-1936 & Spania & ``Livets tragiske mening''\\
Lev Shestov & 1866-1938 & Russland & ``Aten og Jerusalem'' \\
Karl Jaspers & 1883-1969 & Sveits & ``Veien til visdom'' \\
Gabriel Marcel & 1889-1973 & Frankrike & ``{\AA} v{\ae}re og {\aa} ha'',\\
    & & & ``Innf{\o}ring i h{\aa}pets metafysikk'' \\ 
\multicolumn{3}{l}{\und{Forgjengere/forfattere}} \\
%Tertullian & ?155-225? & Roma (Kartagina)  \\
St.Augustin & 354-430 & Hippo/Roma  & ``Bekjennelser'' \\
Blaise Pascal & 1623-1662 & Frankrike & ``Tanker'' \\
Fyodor Dostoyevsky & 1821-1881 & Russland & ``Forbrytelse og straff'', \\
    & & & ``Br{\o}drene Kramazov''\\
Simone Weil & 1909-1943 & Frankrike & ``Behov for r{\o}tter''
\end{tabular}


\subsection*{Ateistisk eksistensialisme}
%
\noindent
\begin{tabular}{lcll}
Martin Heidegger & 1889-1976 & Tyskland & ``V{\ae}ren og Tid'' \\
Jean-Paul Sartre & 1905-1980 & Frankrike & ``Ordene'', ``Nausea'' (Kvalmer),\\
  & & & ``Bak lukkede d{\o}rer''\\ 
\multicolumn{3}{l}{\und{Forgjengere/forfattere}} \\
Friedrich Nietzsche& 1844-1900 & Tyskland & ``Hinsides Det Gode og Det Onde''\\
Franz Kafka & 1883-1924 & {\O}sterike & ``Slottet'', ``Prosessen''\\
Albert Camus & 1913-1960 & Frankrike  & ``Sisyfosmyten'', ``Pesten''\\
   & & & ``Den Fremmede'' \\
Samuel Beckett & 1906-1989 & Irland & ``Mens vi venter p{\aa} Godot'', \\
   & & & ``Det Unavnelige'', ``Malone d{\o}r''
\end{tabular}


\newpage
\section*{``Motsettningspar''}
\begin{tabular}{r@{\ \ -\ \ }l}
det Allmenne & det Individuelle \\ \hline
essens & eksistens \\
man & individet \\
%sikkerhet & usikkerhet \\
altomfattende & endelig \\
n{\o}dvendig, allmenngyldig & mulig \\ \hline
{\bf fornuft} & {\bf tro} \\ 
tinglihet & udefinerbarhet \\
felleskap & ensomhet \\
(normer, skikk) \\
objektiv & subjektiv\\
trygghet & usikkerhet \\
frykt & angst \\
mengden, folket & individ
\end{tabular}
\newpage
\section*{Eksistensielt \co{valg}}
selvbevisst/resolute -- \co{valget} \co{indivdualiserer}
\begin{enumerate}
\item Jeg er \co{kastet inn i verden}
\begin{itemize}
\item
m{\o}ter situasjoner som jeg ikke har skapt og ikke har full kontroll
over
\item verden overg{\aa}r meg, er ``st{\o}rre en meg''
\item jeg er \co{endelig}
\item jeg m{\aa} velge uten {\aa} ha fullstendig oversikt
\end{itemize}
\item Verden, tingene og situasjonene i verden, konfronterer meg som
\co{faktisistet}
\begin{itemize}
\item de har ingen iboende verdi
\item er n{\o}trale og gir meg ingen grunnlag m.h.t. hva jeg skal velge
\item jeg m{\aa} velge uten {\aa} kunne st{\o}tte meg til objektive begrunnelser
\item \co{valget} er likegyldig, men man {\em m{\aa}} velge
\end{itemize}
\ \dotfill{Religi{\o}st}\dotfill
\item Jeg m{\aa} velge selv -- uten hjelp fra andre, lover, kunnskap, fornuft...
\item Jeg m{\aa} velge kanskje p{\aa} tross av normer, skikk...
\item Siden mitt \co{valg} ikke har noen begrunnelse utenfor meg selv, b{\ae}rer
jeg alene fult ansvar for mine \co{valg}\\
kan ikke skylde p{\aa} samfunnet, systemet, oppdragelse...
\end{enumerate}





\end{document}