\documentclass[12pt]{article}

\makeatletter
\input{a4wide}
\makeatother

\input{xypic}
\xyoption{curve}

\newcommand{\cici}{\setlength{\unitlength}{1cm}\begin{picture}(6.5,3)
\put(3,2){\circle{2}}
%\put(2,2){\circle{0.6}}
\put(3,2){\makebox(0,0)[c]{Jeg}}

\put(3,3.5){\makebox(0,0)[c]{\co{Intet}}}
\put(3,0.5){\makebox(0,0)[c]{\co{Intet}}}

\put(0,1.5){\makebox(0,0)[l]{\parbox{1.9cm}{1.som min\\ endelighet}}}
\put(6.5,1.5){\makebox(0,0)[r]{2.som tomhet}}

\put(2.8,1.8){\vector(-1,-1){1}}

\qbezier(3.2,1.7)(3.3,1.4)(3.6,1.4)
\qbezier(3.6,1.4)(4,1.7)(3.6,2)
\put(3.6,2){\vector(-1,0){0.2}}

%(2.5,2.0)(2.4,2.0)
\end{picture}
}

\newcommand{\no}[1]{$\{${\it{#1}}$\}$}

%\newcommand{\cit}[2]{\citt{#1}~\footnote{\cite{#2}} }
%\newcommand{\citt}[1]{{\em ``#1''}}
%\newcommand{\cita}[2]{\citt{#1}\footnote{#2}}

\newcommand{\nei}[1]{}
\newcommand{\com}[1]{{\footnotesize{[#1]}}}
\newcommand{\MyLPar}{\parsep -.2ex plus.2ex minus.2ex\itemsep\parsep
   \vspace{-\topsep}\vspace{.5ex}}

\newcommand{\co}[1]{{\bf\em #1\/}}
\newcommand{\citt}[1]{{\em #1}}
\newcommand{\ut}[2]{\\ \ \ .\dotfill{\small{\bf #1} vs. {\bf #2}}}
\newcommand{\und}[1]{\underline{#1}}
\newcommand{\ovr}[1]{\vspace*{2ex}\par\noindent{\bf...#1...}\vspace*{1ex}}
\newcommand{\ite}[1]{\item[{\bf #1})]}

%%%
\newcounter{SE}
\newcommand{\se}[1]{\stepcounter{SE}\vspace*{2ex}\par\noindent{\large\bf
\theSE. #1}\\[1ex] }

\title{Eksistensialisme}
\author{Micha{\l} Walicki}
\date{April, 1999}
\begin{document}
%Merkelapper er farlige; spesielt for eksistensialisme som framhever det
%individuelle, det spesifikke. Vi m{\aa} abstrahere og generalisere her...

\section{Kristendommen -- oppdagelse av individet...} 
\begin{itemize}
\item individuell samvittighet, %(Rom.14,23)
\item individuell skyld,
\item individuelt forhold hvert menneske har til Gud, 
\item ...
\end{itemize}

\noindent
\begin{tabular}{r@{\ :\ \ }l}
Tertullian & \citt{``Jeg tror p{\aa} det \und{fordi} det er absurd''} \\
St.Augustin & \citt{``Finner du at din natur er vaklende, overg{\aa} den (deg
selv)''} \\
Pascal & \citt{``Mennesket er bare et siv i vinden''} 
\end{tabular}
\ut{fornuft}{tro}


\subsection*{... og det uforklarlige, paradoksale...``absurde''}
\citt{``Man kan ikke \und{bevise} at Jesus var Gud -- fordi det `{\aa} bevise'
betyr {\aa} gj{\o}re noe tilgjengelig for fornuften, mens dette er
nettopp en paradoks som strider mot fornuften''.}
\begin{enumerate}
\item treenighet, kroppens gjenoppstandelse, ...
\item Job -- uforskyldt lidelse
\item H{\aa}p -- ikke fortvil selv om det er all grunn til det (og ingenting annet)
\item Abraham som (nesten) ofrer Isaak -- 1 Mos. 22,1-14
\end{enumerate}

\newpage
\section{\co{Valg}}
Abraham er bare et eksempel som, for Kierkegaard, gir en arketype, et
paradigme for den enkeltes ensomhet i valgsituasjon
\begin{enumerate}\MyLPar
\item Abraham m{\aa} velge i ensomhet, alene, uten hjelp fra andre -- fordi
andre ikke ville v{\ae}re i stand til {\aa} forst{\aa}... 
\ut{felleskap}{ensommhet}
\item Han velger faktisk {\em p{\aa} tross} av lover, normer, skikk...
\ut{normer, skikk}{individ}
\item Siden \co{valget} er ``absurd'' kan han ikke st{\o}tte seg p{\aa} noen
objektive kriterier: lover, kunnskap, fornuft...  
\ut{objektiv}{subjektiv}
\item Siden \co{valget} ikke har noen begrunnelse utenfor ham selv, b{\ae}rer
han alene fult ansvar \\
(kan ikke skylde p{\aa} samfunnet, systemet, oppdragelse...)
\end{enumerate}
%\ut{valg 3,4,5}
\subsection*{og \co{Angst}}
Et slikt \co{valg} skjer ``utenfor den vanlige verden'': den objektive,
verifiserbare, felles, intersubjektive verden av ting og andre mennesker. Man
bytter `objektiv trygghet' av denne verden mot `objektiv usikkerhet',
dvs. subjektivitet, {\aa}ndelighet, et ``indre kall''.
\ut{trygghet}{usikkerhet}

\co{Angst} er et  uttrykk for at menneske ikke helt er seg selv og
samtidig  har mistet ``fast grunn'' i
``denne verden''. Den gjelder intet i omverden, men de indre farer forbundet
med {\aa} virkeliggj{\o}re seg selv.
\co{Angst} har
ikke noe ``objekt'' som for{\aa}rsaker den, den er den enkeltes konfrontasjon med en
``ytterste grense'', det ikke-objektive, udefinerbare, {\aa}ndelige... Det er
et m{\o}te med Gud -- eller seg selv -- uten oppskrift eller regler som man
kunne l{\ae}re fra andre. Man m{\aa} m{\o}te det alene -- kaste seg ut p{\aa} dypt
vann -- eller flykte og gjemme seg i trygghet av sm{\aa}borgerlig normalitet.

I motsetning til \co{angst}, er \co{frykt} noe som man 
bare opplever i omverden, 
ovenfor et bestemt objekt, en bestemt situasjon: en fiende, en tiger, tap av penger. 
%
\ut{frykt}{angst}
\ovr{\co{Valg} skjer ``utenfor verden'' -- den \co{individuerer!}}

\section{\citt{``Subjektiviteten er sannheten''}}
\begin{tabular}{rcl}
``subjektivitet'' & = & ``objektiv usikkerhet''\\
& = & ikke gjentakbar, unik \\
& = & ikke-tinglighet/udefinerbarhet\\
& = & konfrontasjon med det {\aa}ndelige
\end{tabular}
\ut{tinglighet}{udefinerbarhet}

\newpage
\section{Det ytterste \co{valget}}
\begin{itemize}
\item[a)]
N{\aa}r jeg m{\o}ter min ytterste grense, hinsides hvor ingenting mer kan
sees, gripes eller fornemmes...
\item[b)]
Finnes det liv  p{\aa} andre planeter?
\begin{enumerate}\MyLPar
\item Ja, jeg tror det
\item Nei, jeg tror det ikke
\item Jeg vet jo ikke, s{\aa} jeg vil ikke svare!
\end{enumerate}
Det siste kan man ikke si ovenfor et eksistensielt \co{valg} -- \citt{``Det {\aa} la
v{\ae}re {\aa} velge, er ogs{\aa} {\aa} velge.''}
Det {\aa} ikke svare betyr {\aa} ikke forholde seg til det, dvs. {\aa}
forholde seg som om ingenting (var der)!
\item[c)] Er det noe hinsides eller ikke? 
Er verden meningsfull eller ikke? Er livet det? -- ingen allmenngyldige svar;
ethvert svar avsl{\o}rer han som svarer (hvordan {\em han} opplever {\em sitt} liv)
\vspace*{2ex}
%
\[\cici \vspace*{-2ex}\]
\begin{center}
\begin{tabular}{l@{\hspace*{1.5em}}c@{\hspace*{1.5em}}l} %{l@{\hspace*{4em}}l}
min endelighet & -- & min absolutte frihet \\
{\aa}ndelig mening & -- & frav{\ae}r av mening\\
objektiv usikkerhet & -- & total usikkerhet \\
Gud er ubegripelig, & -- & \citt{``Gud er d{\o}d''},\\
men vi kan tro         & -- & vi er ikke lenger \\ & & \ \ {\em i stand} til
{\aa} tro
\end{tabular}
\end{center}
\item[d)] For Kierkegaard: det ytterste \co{valget} er det som bringer
menneske over til det religi{\o}se stadiet i livet, det er valg av troen.
\end{itemize}

%\co{Intet} -- ikke som ``objektiv usikkerhet'' men som
\begin{center}
\ovr{1. Kristen vs. 2. ateistisk\\ eksistensialisme}
\end{center}
\newpage
\section{Ateistisk eksistensialisme}
tar opp hovedbegreper (fra Kierkegaard) men anvender de p{\aa} en litt
annen m{\aa}te. Hovedsaklig, menneske 
\begin{itemize}\MyLPar
\item er ikke lenger konfrontert med {\aa}ndelig \co{valg} (av verdier,
livsformer, tro) 
\item men med \co{valg} mellom
faktiske situasjoner og holdninger {\em i} verden
\end{itemize}
Utviklet mest inng{\aa}ende i litteraturen:
\begin{itemize}
\item ``Sisyfosmyte'', Camus: likegyldighet av alle alternativene
\item ``Mens vi venter p{\aa} Godot'', Beckett / ``Nausea'', Sartre --
tilv{\ae}relsens bunnl{\o}se meningsl{\o}shet / eksistens, det ``{\aa} v{\ae}re'' er kvalmende
\item drap hos \\
\hspace*{-1em}\begin{tabular}{l|l}
\multicolumn{1}{c|}{Abraham} & \multicolumn{1}{c}{Den Fremmede} \\ \hline
pinefull men n{\o}dvendig  & bare en tilfeldig realisering av
\'{e}n mulighet \\
gjelder en ytterst viktig sak -- & likegyldig og
gjelder ingenting \\
{\aa} f{\o}lge Guds ordre & kunne like godt velge noe annet \\
vekker Angst -- konfrontasjon med {\aa}nd & vekker kvalme
\end{tabular}
\end{itemize}

\begin{enumerate}\MyLPar
\item Jeg er \co{kastet inn i verden}
\begin{itemize}\MyLPar
\item
m{\o}ter situasjoner som jeg ikke har skapt og ikke har full kontroll
over
\item jeg er \co{endelig} -- ja, faktisk -- {\em fremmed} i denne verden
\item jeg m{\aa} velge {\em i} verden uten \\
-- {\aa} ha  oversikt og\\
--  uten at det ang{\aa}r meg
\end{itemize}
\item Verden, tingene og situasjonene i verden, konfronterer meg som
\co{faktisitet}
\begin{itemize}\MyLPar
\item de har ingen iboende verdi eller mening -- er {\em fremmed} for meg
\item de er n{\o}ytrale og gir meg ikke noe grunnlag m.h.t. hva jeg skal velge
%\item jeg m{\aa} velge uten {\aa} kunne st{\o}tte meg til objektive begrunnelser
\item pga. denne n{\o}ytraliteten er alternative \co{valg} likegyldige,
\co{valg}\\
-- har ikke noe grunnlag (i verdens \co{faktisitet})\\
-- og heller ikke noen {\em betydning} (for verdens \co{faktisitet})\\
-- men man {\em m{\aa}} velge...
\item gjennom \co{valget} kan man muligens ``skape'' noe menigsfullt (rebell)
\end{itemize}
\item \co{Valget} m{\aa} gj{\o}res:
\begin{itemize}
\item i \co{grensesituasjoner} (d{\o}den, stor tragedie, tap, ...)
\item men, essensielt, hele tiden, i enhver situasjon, siden verden er
bare kaos av likestilte muligheter
\end{itemize}
\end{enumerate}
%\ut{valg: 1,3}
\ovr{Menneske konfronteres kun med den faktiske verden og med frav{\ae}r av objektiv mening,
ikke med {\aa}ndelig dimensjon av eksistens der den objektive verden kan
finne mening}

%\newpage
\section{\co{Autentisk eksistens}: ``{\aa} leve gjennom \co{valg}''}
\co{Valg} mellom 
\begin{itemize}
\item \co{uegentlig} eksistens -- trygg, anonym, upersonlig, styrt
av felleskapets meninger og holdninger og 
\item \co{egentlig}/\co{autentisk}
eksistens -- som karakteriseres av selvstendighet, personlig ansvar og
uavhengighet fra `mengden', `folket'
\end{itemize}
m{\o}ter vi i alle former av eksistensialisme.

Den stammer fra Kierkegaard (hvor \co{egentlig} eksistens var preget av
religi{\o}sitet grunnet i selvstendig, opplevd valg av troen), men ble noe
modifisert pga. den underliggende meningsl{\o}sheten i ateistisk utgave.

\begin{enumerate}
\item kritikk av tilv{\ae}relsen preget av `folket', `mengden', `media'...
\item[{}] dvs. av \co{forfall} i det dagligdagse og uegentlige, i \co{d{\aa}rlig tro}
av selv-bedrag, av flukt i tingenes og meningenes objektivitet  
%(\citt{``Jeg vet ikke hvem jeg {\em egentlig} er. Dette er ikke meg!''}) 
\ut{mengden}{individ}
\item selvstendig, ansvarsfult \co{valg}, der man konfronteres med
\item verdens likegyldighet som har overtatt klassisk rolle av \co{Det Onde}
(`det Ondes problem' har blitt problem av `det Likegyldige/Meningsl{\o}se')
\item som i ``Pesten'', Camus -- rebell mot verdens likegyldighet
\end{enumerate}

\ovr{tendens til {\aa} gjenfinne noen verdier `i menneske'
p{\aa} tross av den objektive n{\o}ytralitet}
\\ (``Pesten'', Camus; marxisme av
sen Sartre; Abbagnano -- humanistisk eksistensialisme)

\newpage
\section{Oppsummering...}\vspace*{3ex}

\subsection{Eksistensialisme -- felles trekk}
\citt{``Det er umulig {\aa} skape et system som fanger virkeligheten.''}
\begin{enumerate}
\ite{e1} eksistens er alltid  \und{individuell}: min, din -- ikke
ens; og \und{unik}: ikke-gjentakbar\ut{essens}{eksistens}
\ite{e2} eksistens er prim{\ae}rt et \und{sp{\o}rsm{\aa}l ved seg selv} -- Hvordan
skal jeg leve?\ut{gjennomsnittlig `man'}{problematisk individualitet}
%\ut{sikkerhet}{usikkerhet}
\ite{e3} eksistens er alltid \und{konkret} -- \co{kastet inn i verden},
befinner seg i en bestemt \co{situasjon} som den ikke har skapt
selv\ut{altomfattende}{endelig}
\ite{e4} eksistens konfronteres med \und{muligheter} -- ikke entydige
fakta men egne {valg}\ut{n{\o}dvendighet}{mulighet}
\end{enumerate}
\subsection*{\citt{= ``Eksistens g{\aa}r forut for essens''} = }
Man har ikke en oppskrift, et fast grunnlag i objektiv viten og verden for
sine valg og liv -- kun gjennom aktivt liv og valgene ``skaper'' man: seg
selv.

Personlig opplevelse, valg og erfaring g{\aa}r forut for argumenter,
begrunnelser, regler.

\ovr{}
\newpage
\subsection{Ateistisk eksistensialisme --}
Sisyfos, s{\o}ker men klarer ikke {\aa} finne noen mening
\begin{enumerate}\MyLPar
\item Mot sine intensjoner -- preget av verdens \co{faktisitet}; \co{Intet} blir
mer og mer tomhet og frav{\ae}r av {\aa}ndelighet
\item \co{Valget} er meningsl{\o}st og vilk{\aa}rlig
\item En deskriptiv analyse av verdensanskuelsen
som bygger p{\aa} underliggende frav{\ae}r av mening, konfrontasjon med kaos
av likegyldige muligehter
\item forf{\o}risk pga. skarpe beskrivelser av psykologiske mekanismer og
reaksjoner p{\aa} situasjoner som vi faktisk m{\o}ter i  v{\aa}r verden
\begin{itemize}\MyLPar
\item krise, n{\o}d, skuffelse, tap, dyp tragedie...
\item II Verdens Krig dannet et grunnlag for fundamental tvil om meningen med
menneskelig samfunn og gjensidhet \\
\hspace*{-3em} I dag -- ``frihetens'' samfunn
\item overflod av valgmuligheter som man ikke klarer {\aa} forholde seg til, ikke
har grunnlag til {\aa} velge mellom
\item ingen felles verdier av fundamental ({\aa}ndelig) karakter, ingen
allmenne normer som man kan ``vokse inn i'' gjennom oppdragelse
\end{itemize}
\end{enumerate}
\ovr{men verden er ikke bare fakta og faktiske muligheter}

\subsection{Kristen eksistensialisme --}
 ivaretar i st{\o}rre grad Kierkegaards dype intuisjoner
\begin{enumerate}
\item Det at noe ikke kan defineres, betyr ikke at det ikke er
virkelig. \co{Intet} er ikke tomhet men v{\aa}r \co{endelighet}.
\item \co{Valget} er fritt -- ikke objektivt bestemt -- men ikke dermed meningsl{\o}st eller vilk{\aa}rlig
\item Mening og verdier er ikke bare subjektive (i klassisk, negativ
forstand); menneske {\em m{\o}ter} de -- ikke som faktiske forhold men -- som
udefinerbare ({\aa}ndelige) og dermed desto sterkere intuisjoner, i {\em konfrontasjon} med 
sin endelighet (angst)
\item
Gir en rekke dype innsikter i menneskets
lengsel og streben etter verdier og holdninger som ikke kan begrunnes
objektivt, og som -- {\em av den grunn!} -- overg{\aa}r og er sterkere enn den
faktiske verden
%\item \citt{``Virkelig er det du ikke kan leve uten.''}
\end{enumerate}

\newpage
\section{Noen av hovedpersoner og  litteratur}

\subsection*{Kristen eksistensialisme}
%
\begin{tabular}{lcll}
{\bf S{\o}ren Kierkegaard} & {\bf 1813-1855} & Danmark & ``Frygt og B{\ae}ven'', \\
    & & & ``Om Begrebet Ironi'' \\
    & & & ``Begrebet Angest''... og alt annet \\
Miguel de Unamuno & 1864-1936 & Spania & ``Livets tragiske mening''\\
Lev Shestov & 1866-1938 & Russland & ``Aten og Jerusalem'' \\
Karl Jaspers & 1883-1969 & Sveits & ``Veien til visdom'' \\
Gabriel Marcel & 1889-1973 & Frankrike & ``{\AA} v{\ae}re og {\aa} ha'',\\
    & & & ``Innf{\o}ring i h{\aa}pets metafysikk'' \\ 
\multicolumn{3}{l}{\und{Forgjengere/forfattere}} \\
%Tertullian & ?155-225? & Roma (Kartagina)  \\
St.Augustin & 354-430 & Hippo/Roma  & ``Bekjennelser'' \\
Blaise Pascal & 1623-1662 & Frankrike & ``Tanker'' \\
Fyodor Dostoyevsky & 1821-1881 & Russland & ``Forbrytelse og straff'', \\
    & & & ``Br{\o}drene Kramazov''\\
Simone Weil & 1909-1943 & Frankrike & ``Behov for r{\o}tter''
\end{tabular}


\subsection*{Ateistisk eksistensialisme}
%
\noindent
\begin{tabular}{lcll}
Martin Heidegger & 1889-1976 & Tyskland & ``V{\ae}ren og Tid'' \\
Jean-Paul Sartre & 1905-1980 & Frankrike & ``Ordene'', ``Nausea'' (Kvalme),\\
  & & & ``Bak lukkede d{\o}rer''\\ 
\multicolumn{3}{l}{\und{Forgjengere/forfattere}} \\
Friedrich Nietzsche& 1844-1900 & Tyskland & ``Hinsides Det Gode og Det Onde''\\
Franz Kafka & 1883-1924 & {\O}sterike & ``Slottet'', ``Prosessen''\\
Albert Camus & 1913-1960 & Frankrike  & ``Sisyfosmyten'', ``Pesten''\\
   & & & ``Den Fremmede'' \\
Samuel Beckett & 1906-1989 & Irland & ``Mens vi venter p{\aa} Godot'', \\
   & & & ``Det Unavnelige'', ``Malone d{\o}r''
\end{tabular}


\newpage
%\setcounter{section}{0}

\se{Kristendom oppdaget...}
I Vesten! 
\begin{itemize}\MyLPar
\item individ
\item paradoks \\
-- Job, h{\aa}p, ...\\
-- \citt{Jeg tror fordi det er absurd}
\end{itemize}

\se{Abraham velger}
paradigmet for eksistensielt valg:
\begin{itemize}\MyLPar
\item alene 
\item p{\aa} tross ...
\item uten objektiv grunn
\item b{\ae}rer hele ansvaret alene
\end{itemize}

\se{Angst}
uten objekt
\begin{itemize}\MyLPar
\item indivduerer
\item \citt{Subjektiviteten er sannheten}
\end{itemize}

\se{Ytterste valget}
mellom `tro' og `ikke-tro'

\se{Ateistisk eksistensialisme}
konfrontert ikke med {\AA}nd men kun verden
\begin{itemize}\MyLPar
\item kastet \\
-- har ikke kontroll\\
-- verden er ikke \und{min}, er fremmed 
\item faktisitet\\
-- ingen iboende mening \\
-- ingen grunnlag for valg\\
-- valg endrer ikke noe
\end{itemize}
Meningsl{\o}st....

\se{Autentisk eksistens}
gjennom valget
\begin{itemize}\MyLPar
\item Kierkegaard: tro vs. ikke-tro : kvalitet av valget, HVA som velges
\item her: den som velger vs. den som ikke gj{\o}r det : BARE det {\aa} velge
\end{itemize}
\end{document}