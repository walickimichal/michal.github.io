\documentclass[12pt]{article}

\makeatletter
\input{a4wide}
\makeatother

\newcommand{\pa}[1]{\begin{quote}{\small{#1}}\end{quote}}
\newcommand{\fr}[1]{\par\noindent{\footnotesize{#1}}}
\begin{document}

\footnotesize{\tableofcontents}


\section{Individualism}
The French political commentator Alexis de Tocqueville, who coined the
word, described it in terms of a kind of moderate selfishness,
disposing human beings to be concerned only with their own small
circle of family and friends. 

\subsection{Alexis de Tocqueville}
Tocqueville Alexis (Charles-Henri-Maurice Cl\'{e}rel) de

(b. July 29, 1805, Paris--d. April 16, 1859, Cannes, Fr.), political
scientist, historian, and politician, best-known for Democracy in
America, 4 vol. (1835-40), a perceptive analysis of the political
and social system of the United States in the early 19th century. (see
also Index: political science) 


\subsubsection{Early life}
Tocqueville was a great-grandson of the statesman Chr\'{e}tien de
                                      Malesherbes, a liberal aristocratic victim of the French Revolution
                                      and a political model for the young Tocqueville. Almost diminutive in
                                      stature, acutely sensitive, and plagued by severe bouts of anxiety
                                      since childhood, he remained close to his parents throughout his life. 




                                      Given the choice of a variety of careers, Tocqueville chose the one
                                      most congenial to him: that of politician. Despite a frail voice in a
                                      fragile body, distaste for the daily demands of parliamentary
                                      existence, and long periods of illness and nervous exhaustion, he
                                      adhered to his original choice until he was driven from office. His
                                      decision in favour of a public career was made with some assurance
                                      of success. His father was a loyal royalist prefect and in 1827 was
                                      made a peer of France by Charles X. At that time, young Tocqueville
                                      moved easily into government service as an apprentice magistrate.
                                      There he prepared himself for political life while observing the
                                      impending constitutional confrontation between the Conservatives
                                      and the Liberals with growing sympathy for the latter. He was
                                      strongly influenced by the lectures of the historian and statesman
                                      Francois Guizot, who asserted that the decline of aristocratic
                                      privilege was historically inevitable. And finally, after the manner of
                                      Liberals under the autocratic regime of the restored Bourbon kings,
                                      Tocqueville began to study English history as a model of political
                                      development. 




                                      He entered public life in the company of a close friend who was to
                                      become his alter ego--Gustave de Beaumont. Their life histories are
                                      virtual mirror images. Of similar backgrounds and positions, they
                                      were companions in their travels in America, England, and Algeria,
                                      coordinated their writings, and ultimately entered the legislature
                                      together. 




                                      The July Revolution of 1830 that put the "citizen king" Louis-Philippe
                                      of Orl\'{e}ans on the throne was a turning point for Tocqueville. It
                                      deepened his conviction that France was moving rapidly toward
                                      complete social equality. Breaking with the older liberal generation,
                                      he no longer compared France with the English constitutional
                                      monarchy but with democratic America. Of more personal concern,
                                      despite his oath of loyalty to the new monarch, his position had
                                      become precarious because of his family ties with the ousted Bourbon
                                      king. He and Beaumont, seeking to escape from their uncomfortable
                                      political situation, asked for and received official permission to
                                      study the uncontroversial problem of prison reforms in America. They
                                      also hoped to return with knowledge of a society that would mark
                                      them as especially fit to help mold France's political future. (see also
                                      Index: democracy ) 




\subsubsection{Visit to the United States}
The two men spent nine months in the United States during 1831 and
                                      1832, out of which came first their joint book, Du syst\`{e}me
                                      p\'{e}nitentiaire aux \'{E}tats-Unis et de son application en France
                                      (1833; On the Penitentiary System in the United States and Its
                                      Application in France); Beaumont's Marie, ou l'esclavage aux
                                      \'{E}tats-Unis (1835; Marie, or Slavery in the United States), on
                                      America's race problems; and the first part of Tocqueville's De la
                                      d\'{e}mocratie (1835-40; Democracy in America). On the basis of
                                      observations, readings, and discussions with a host of eminent
                                      Americans, Tocqueville attempted to penetrate directly to the
                                      essentials of American society and to highlight that aspect--equality
                                      of conditions--that was most relevant to his own philosophy.
                                      Tocqueville's study analyzed the vitality, the excesses, and the
                                      potential future of American democracy. Above all, the work was
                                      infused with his message that a society, properly organized, could
                                      hope to retain liberty in a democratic social order. 




                                      The first part of De la d\'{e}mocratie won an immediate reputation for
                                      its author as a political scientist. This period quite probably was the
                                      happiest and most optimistic of his life. Tocqueville was named to
                                      the Legion of Honour, the Academy of Moral and Political Sciences
                                      (1838), and the French Academy (1841). With the prizes and royalties
                                      from the book, he even found himself able to rebuild his ancestral
                                      chateau in Normandy. Within a few years his book had been published
                                      in England, Belgium, Germany, Spain, Hungary, Denmark, and Sweden.
                                      In the United States, while it was sometimes viewed as having been
                                      derived from politically biased sources, it was soon accorded the
                                      status of a classic. In 1836 Tocqueville married Mary Mottely, an
                                      Englishwoman. 




                                      Tocqueville spent four more years working on the final portion of De
                                      la d\'{e}mocratie, which was published in 1840. Its composition took
                                      far longer, moved farther afield, and ended far more soberly than
                                      Tocqueville had originally intended. American society slid into the
                                      background, while Tocqueville now attempted to complete a picture of
                                      the influence of equality itself on all aspects of modern society.
                                      France increasingly became his principal example, and what he saw
                                      there altered the tone of his work. He observed the curtailment of
                                      liberties by the Liberals, who had come to power in 1830, as well as
                                      the growth of state intervention in economic development. Most
                                      depressing to him was the increased political apathy and
                                      acquiescence of his fellow citizens in this rising paternalism. His
                                      chapters on democratic individualism and centralization in De la
                                      d\'{e}mocratie contained a new warning based on these observations. A
                                      mild, stagnant despotism now seemed to him democracy's greatest
                                      danger. 




\subsubsection{First political career}
This was also the period in which Tocqueville fulfilled his lifelong
                                      ambition to enter politics. He lost his first bid for the Chamber of
                                      Deputies in 1837, but he won the following election in 1839.
                                      Eventually, Tocqueville built up an enormous personal influence in his
                                      constituency. He won every election after the first by more than 70
                                      percent of the vote and became president of his departmental council
                                      (a local representative body). At the level of local politics his quest
                                      for preeminence was completely fulfilled, but his need for
                                      uncompromised dignity and independence deprived him of influence in
                                      the Chamber of Deputies for a much longer time. He was not able to
                                      follow the leadership of others, nor did his oratorical style win him
                                      quick recognition as a leader himself. As a result, he had no major
                                      legislative accomplishment to his credit during the reign of
                                      Louis-Philippe. His speech prophesying revolution only a few weeks
                                      before it took place in February 1848 fell on deaf ears. The biting
                                      sketches of friend, foe, and even himself in his Souvenirs (1893;
                                      "Recollections") reflect his feeling of the general mediocrity of
                                      political leadership before and after 1848. 




\subsubsection{Revolution of 1848}
The February Revolution brought about a new political situation for
                                      France and for Tocqueville. Having for years decried apathy as the
                                      chief danger for France, Tocqueville recognized even before the
                                      revolution that France was faced with a politically awakened working
                                      class that might well propel French politics into socialist and
                                      revolutionary channels. Tocqueville considered economic independence
                                      as necessary to the preservation of his own intellectual independence.
                                      He thus viewed pressures of the dependent poor for state welfare, and
                                      of the unemployed for state employment, as the initial steps to a
                                      universal and degrading dependence on the state by all social classes.
                                      Unsympathetic to revolutionaries and contemptuous of socialists
                                      before the revolution, Tocqueville opposed the demands of the
                                      Parisian workers during the June days of 1848, when their uprising
                                      was bloodily suppressed by the military dictator Gen. Louis
                                      Cavaignac, as well as in the debates over the constitution of 1848.
                                      The only intellectual change produced in Tocqueville by the events of
                                      1848 was a recognition of the strength of socialist ideas and of the
                                      problematic nature of the proprietary society. But while he had sought
                                      to reconcile the aristocracy to liberal democracy in De la
                                      d\'{e}mocratie, he rejected social democracy as it emerged in 1848 as
                                      incompatible with liberal democracy. 




                                      Politically, Tocqueville's own position was dramatically improved by
                                      the February Revolution. His electorate expanded from 700 to 160,000
                                      under universal manhood suffrage. He was elected as a conservative
                                      Republican to the Constituent Assembly by 79 percent of the voters
                                      and again in 1849 by more than 87 percent. Along with Beaumont, he
                                      was nominated to the committee that wrote the constitution of the
                                      Second Republic, and the following year he became vice president of
                                      the Assembly. He served as minister of foreign affairs between June
                                      and October of 1849. During his short tenure he worked cautiously to
                                      preserve the balance of power in Europe and to prevent France from
                                      extending its foreign involvements. His speeches were more
                                      successful and his self-confidence soared, but the results gave him
                                      little more durable satisfaction than those he had attained during the
                                      July monarchy under Louis-Philippe. 




                                      In June 1849 a government crisis produced by French armed
                                      intervention to restore papal authority in Rome was the occasion for
                                      Tocqueville's brief appointment to the Ministry of Foreign Affairs.
                                      Shortly after his dismissal from the ministry by Pres. Louis-Napol\'{e}on
                                      Bonaparte in October 1849 he suffered a physical collapse. After a
                                      slow recovery he performed a final service for the Second French
                                      Republic. As reporter for the constitutional revision committee he
                                      attempted to avert the final confrontation between the President and
                                      the legislature, which ended with an executive seizure of dictatorial
                                      power. Briefly imprisoned for opposing Louis-Napol\'{e}on's coup d'\'{e}tat
                                      on Dec. 2, 1851, Tocqueville was deprived of all political offices for
                                      refusing his oath to the new regime. He was thrown back on a small
                                      circle of political allies and friends and felt a deeper sense of
                                      isolation and political pessimism than ever before. (see also Index:
                                      Second Republic) 




\subsubsection{Return to politics}
Seeking to reenter politics, he reverted to the strategy of his
                                      youthful success--the publication of a book on the fundamental
                                      themes of liberty and equality. This time he chose as his subject the
                                      French Revolution. L'Ancien R\'{e}gime et la R\'{e}volution (The Old
                                      Regime and the Revolution) appeared in 1856, after years of research
                                      and intermittent illness, as the first part of his projected study. In
                                      this work, Tocqueville sought to demonstrate the continuity of
                                      political behaviour and attitudes that made postrevolutionary French
                                      society as prepared to accept despotism as that of the old regime. In
                                      this final study the traumatic events of the years 1848-51 were
                                      clearly the source of his emphasis on the durability of centralization
                                      and class hostility in French history. France seemed less the
                                      democratic society of the future he had glimpsed in America than the
                                      prisoner of its own past. Against the pessimism of his analysis of
                                      French political tendencies, L'Ancien R\'{e}gime reaffirmed the
                                      libertarian example of the Anglo-American world. The acclaim that
                                      greeted this study briefly dispelled the gloom of his last years. Once
                                      again his book made him a public figure. His visit to England in 1857,
                                      culminating in an audience with the Prince Consort, was the last
                                      public triumph of his life. He returned to his work; but before he could
                                      finish his study of the Revolution, he collapsed and died. 




\subsubsection{Reputation}
Tocqueville's reputation in the 19th century reached its high point
                                      during the decade following his death, as the great European powers
                                      accommodated themselves to universal suffrage. He died just at the
                                      onset of a revival of liberalism in France. The nine-volume
                                      publication of his works, edited by Beaumont (1860-66), was received
                                      as the legacy of a martyr of liberty. In England his name was invoked
                                      during the franchise reform debates of the 1860s, and in Germany it
                                      was linked to controversies over liberalization and federalization in
                                      the years preceding the empire devised by Bismarck. After 1870 his
                                      influence began to decline, a process not substantially reversed
                                      either by the posthumous publication of his Souvenirs in 1893 or
                                      that of his correspondence with his friend, the diplomatist and
                                      philosopher Arthur de Gobineau. By the turn of the century he was
                                      almost forgotten, and his works were generally regarded as outdated
                                      classics. They seemed too abstract and speculative for a generation
                                      that believed only in ascertained knowledge. Moreover, Tocqueville's
                                      prediction of democracy as a vast and uniformly levelling power
                                      seemed to have miscarried by not foreseeing both the extent of the
                                      new inequalities and conflicts produced by industrialization and those
                                      produced by European nationalisms and imperialism. The classless
                                      society had failed to appear in Europe; and America seemed to have
                                      become European by becoming nationalist and imperialist rather than
                                      vice versa. In France, Tocqueville's name was too closely identified
                                      with a narrowly defined Liberal tradition, which rapidly lost
                                      influence during the Third Republic. While his work as an innovative
                                      historian was acknowledged, it is significant that the revival of his
                                      ideas and reputation as a political sociologist owes so much to
                                      American, English, and German scholarship. 




                                      The 20th-century totalitarian challenge to the survival of liberal
                                      institutions produced by two world wars and by the Great Depression
                                      of the 1930s fostered a "Tocqueville renaissance." The outdated facts
                                      of his books seemed less significant than the political philosophy
                                      implicit in his search to preserve liberty in public life and his
                                      strategies for analyzing latent social tendencies. His work was found
                                      to display a wealth of fruitful philosophical and sociological
                                      hypotheses. At a popular level, the renewed upsurge of social
                                      democracy of Europe after 1945 combined with the polarization of the
                                      Cold War to produce a view of Tocqueville in the West as an
                                      alternative to Marx as a prophet of social change. Again, as in the late
                                      1850s and 1860s, Tocqueville rose to heights of popularity. Whether
                                      changes in social and intellectual climate will cause another
                                      levelling off only time will tell. It seems certain, however, that
                                      henceforth Tocqueville will continue to be invoked as an authority and
                                      inspiration by those sharing his contempt of static authoritarian
                                      societies as well as his belief in the final disappearance of class
                                      divisions and in liberty as the ultimate political value. (S.Dr.) 





\section{Personalism}
a school of philosophy, usually idealist, which asserts that the real is
                                      the personal, i.e., that the basic features of
                                      personality--consciousness, free self-determination, directedness
                                      toward ends, self-identity through time, and value
                                      retentiveness--make it the pattern of all reality. In the theistic form
                                      that it has often assumed, personalism has sometimes become
                                      specifically Christian, holding that not merely the person but the
                                      highest individual instance of personhood--Jesus Christ--is the
                                      pattern. 

                                      Personalism is thus in the tradition of the cogito, ergo sum ("I
                                      think, therefore I am") of Ren\'{e} Descartes in holding that, in the
                                      subjective flow of lived-through experience, one makes more direct
                                      soundings of the real than in anything arriving through the tortuous
                                      paths of perceptual processes. The word person comes from the Latin
                                      persona, which referred to the mask worn by an actor and thus to his
                                      role. Eventually, it came to mean the dignity of a man among men. The
                                      person is thus supreme both in reality (as substance) and in value (as
                                      dignity). 


                                      There are various kinds of personalism. Though most personalists
                                      are idealists, believing that reality is either of, in, or for
                                      consciousness, there are also realistic personalists, who hold that
                                      the natural order, though created by God, is not as such spiritual; and,
                                      again, though most personalists are theists, there are also atheistic
                                      personalists. Among the idealists there are absolutistic personalists
                                      (see absolute Idealism), panpsychistic personalists (see
                                      panpsychism), ethical personalists, and personal idealists, for whom
                                      reality comprises a society of finite persons or an ultimate person,
                                      God. 


                                      Though elements of personalistic thought can be discerned in many of
                                      the greatest philosophers of the Western tradition and even in the
                                      Orient -- as, for example, in Ramanuja, a 12th-century Hindu theist --
                                      Gottfried Wilhelm Leibniz, a 17th-18th-century German philosopher
                                      and mathematician, is usually singled out as the founder of the
                                      movement and George Berkeley, the 18th-century Anglo-Irish
                                      churchman and epistemologist, as another of its seminal sources. 


                                      Personalism has been strongly represented in France, usually under
                                      the name of spiritualism. Inspired by Maine de Biran, an
                                      18th-19th-century thinker who had taken as primordial the inner
                                      experience of acting against a resisting world, F\'{e}lix
                                      Ravaisson-Mollien, a 19th-century philosopher and archaeologist,
                                      drew a radical distinction between the spatial world of static
                                      necessary law and the world of living individuals, spontaneous,
                                      active, and developing. This led in turn to the personalism of Henri
                                      Bergson, a 19th-20th-century intuitionist, who stressed duration as a
                                      nonspatial experience in which subjective states both present and
                                      past intimately interpenetrate to form the free life of the spiritual
                                      person and who posited the \'{e}lan vital as a cosmic force expressing
                                      this life philosophy. 

                                      Personalism in the United States matured among 19th-20th-century
                                      philosophers of religion, often of the Methodist church, several of
                                      whom had studied in Germany under Rudolf Hermann Lotze, an erudite
                                      metaphysician and graduate in medicine. George Holmes Howison, for
                                      example, stressed the autonomy of the free moral person to the point
                                      of making him uncreated and eternal and hence free from an infinite
                                      person. Borden Parker Bowne, who made Boston University the citadel
                                      of personalism, was explicitly theistic, holding that men are
                                      creatures of God with many dimensions--moral, religious, emotional,
                                      logical--each worthy of consideration in its own right and each
                                      reflecting the rationality of the creator. Nature, too, for him, displays
                                      the energy and rational purpose of a God who is immanent in it as
                                      well as transcendent over it. 


                                      Through Bowne's disciples Edgar Brightman and Ralph Tyler Flewelling
                                      and many others, personalism was influential through the mid-20th
                                      century, and its impact upon existentialism and phenomenology has
                                      perpetuated its spirit and many of its insights. 


\subsection{Max Scheler}
                                      (b. Aug. 22, 1874, Munich, Ger.--d. May 19, 1928, Frankfurt am Main),
                                      German social and ethical philosopher, remembered for his
                                      phenomenological approach, after the philosophical method of the
                                      founder of phenomenology, Edmund Husserl. 




                                      In 1901 Scheler became a lecturer at the University of Jena; by that
                                      time he had already been influenced by Husserl. Scheler later met
                                      several of Husserl's disciples during his years (1907-10) as a
                                      professor at Munich. Retiring to Berlin in 1910, he wrote his major
                                      works before 1917, when he joined the German Foreign Office as a
                                      diplomat in Geneva and at The Hague. In 1919 he became professor of
                                      philosophy and sociology at Cologne. By 1920 he had become a pacifist
                                      and a convert to Roman Catholicism, but about 1924 he turned toward
                                      a more pantheistic view of man and the world. 




                                      As a phenomenologist, Scheler sought to discover the essence of
                                      mental attitudes and their relation to their objects. He differed from
                                      Husserl in his readiness to assign an independently real status to the
                                      objects. Scheler's work falls into two periods. During the first, his
                                      work contained a number of Christian orientations, as in the main
                                      work of this period, Der Formalismus in der Ethik und die materiale
                                      Wertethik (1913-16; Formalism in Ethics and Non-Formal Ethics of
                                      Values), which is in part a severe critique of Kant. Scheler shows
                                      that what one "ought to do" presupposes a feeling of the value of what
                                      ought to be done and divides all values into five ranks, which are
                                      given a priori and which are anchored in each person's ordo amoris,
                                      an "order, or logic, of the heart" that is not congruent with the logic
                                      of reason. In holding this view, Scheler followed the 17th-century
                                      French philosopher Blaise Pascal. According to this logic, moral acts
                                      and deeds are individual and originate in an individual's prerational
                                      preferring (or rejecting) of values. Moral experience lies in the "call
                                      of the hour," in which the a priori rankings among values become
                                      individually transparent, no matter how much the ordo amoris may
                                      be distorted by feelings of resentment, hate, or other passions. The
                                      only vehicle for attaining a higher moral status is the "exemplarity"
                                      of a person, which pulls the individual toward his exemplary
                                      self-value. 




                                      While the first period centred on the incontrovertible value of the
                                      individual person, in his second period Scheler set out to determine
                                      the "meta-anthropological" status of humanity. In Die Stellung des
                                      Menschen im Kosmos (1928; "Man's Place in the Universe") and in
                                      manuscripts edited after his death, he offers a grandiose view of
                                      Being: man, God, and world are one self-becoming cosmic process in
                                      absolute time. This process has two poles: spirit (Geist) and
                                      life-urge (Drang). By itself, spirit is powerless, unless its ideas can
                                      "functionalize" with life-factors (material conditions) allowing their
                                      realization, a concept similar to those of American pragmatism, in
                                      which Scheler took a lifelong interest. Divine spirit also needs
                                      human life and history to become real. Reality lies in the "resistance"
                                      between these two poles. Resistance qua reality is central not only in
                                      his phenomenology but also in his Versuche zu einer Soziologie des
                                      Wissens (1924; "Sociology of Knowledge"). 


\subsection{Carl Gustav Jung}
(b. July 26, 1875, Kesswil, Switz.--d. June 6, 1961, K\"{u}snacht), Swiss
                                      psychologist and psychiatrist who founded analytic psychology, in
                                      some aspects a response to Sigmund Freud's psychoanalysis. Jung
                                      proposed and developed the concepts of the extroverted and
                                      introverted personality, archetypes, and the collective unconscious.
                                      His work has been influential in psychiatry and in the study of
                                      religion, literature, and related fields. 


\subsubsection{Early life and career}
Jung was the son of a philologist and pastor. His childhood was lonely,
                                      though enriched by a vivid imagination, and from an early age he
                                      observed the behaviour of his parents and teachers, which he tried to
                                      resolve. Especially concerned with his father's failing belief in
                                      religion, he tried to communicate to him his own experience of God.
                                      Though the elder Jung was in many ways a kind and tolerant man,
                                      neither he nor his son succeeded in understanding each other. Jung
                                      seemed destined to become a minister, for there were a number of
                                      clergymen on both sides of his family. In his teens he discovered
                                      philosophy and read widely, and this, together with the
                                      disappointments of his boyhood, led him to forsake the strong family
                                      tradition and to study medicine and become a psychiatrist. He was a
                                      student at the universities of Basel (1895-1900) and Z\"{u}rich (M.D.,
                                      1902). 

                                      He was fortunate in joining the staff of the Burgh\"{o}lzli Asylum of the
                                      University of Z\"{u}rich at a time (1900) when it was under the direction
                                      of Eugen Bleuler, whose psychological interests had initiated what
                                      are now considered classical researches into mental illness. At
                                      Burgh\"{o}lzli, Jung began, with outstanding success, to apply association
                                      tests initiated by earlier researchers. He studied, especially,
                                      patients' peculiar and illogical responses to stimulus words and found
                                      that they were caused by emotionally charged clusters of
                                      associations withheld from consciousness because of their
                                      disagreeable, immoral (to them), and frequently sexual content. He
                                      used the now famous term complex to describe such conditions. 



\subsubsection{Association with Freud}
These researches, which established him as a psychiatrist of
                                      international repute, led him to understand Freud's investigations; his
                                      findings confirmed many of Freud's ideas, and, for a period of five
                                      years (between 1907 and 1912), he was Freud's close collaborator. He
                                      held important positions in the psychoanalytic movement and was
                                      widely thought of as the most likely successor to the inventor of
                                      psychoanalysis. But this was not to be the outcome of their
                                      relationship. Partly for temperamental reasons and partly because of
                                      differences of viewpoint, the collaboration ended. At this stage Jung
                                      differed with Freud largely over the latter's insistence on the sexual
                                      bases of neurosis. A serious disagreement came in 1912, with the
                                      publication of Jung's Wandlungen und Symbole der Libido
                                      (Psychology of the Unconscious, 1916), which ran counter to many
                                      of Freud's ideas. Though Jung had been elected president of the
                                      International Psychoanalytic Society in 1911, he resigned from the
                                      society in 1914. 




                                      His first achievement was to differentiate two classes of people
                                      according to attitude types: extroverted (outward-looking) and
                                      introverted (inward-looking). Later he differentiated four functions
                                      of the mind--thinking, feeling, sensation, and intuition--one or more
                                      of which predominate in any given person. The results of this study
                                      were embodied in Psychologische Typen (1921; Psychological
                                      Types, 1923). Jung's wide scholarship was well manifested here, as
                                      it also had been in The Psychology of the Unconscious. (see also
                                      Index: extrovert, introvert) 




                                      As a boy Jung had remarkably striking dreams and powerful fantasies
                                      that had developed with unusual intensity. After his break with Freud,
                                      he deliberately allowed this aspect of himself to function again and
                                      gave the irrational side of his nature free expression. At the same
                                      time, he studied it scientifically by keeping detailed notes of his
                                      strange experiences. He later developed the theory that these
                                      experiences came from an area of the mind that he called the
                                      collective unconscious, which he held was shared by everyone. This
                                      much contested conception was combined with a theory of archetypes
                                      that Jung believed were of fundamental importance for the study of
                                      the psychology of religion. In Jung's terms, archetypes are instinctive
                                      patterns, having a universal character, expressed in behaviour and
                                      images. (see also Index: religion, philosophy of) 

\subsubsection{Character of his psychotherapy}
The rest of his life was given over to the development of his ideas,
                                      especially those on the relation between psychology and religion. In
                                      his view, obscure and often neglected texts of writers in the past
                                      shed unexpected light not only on Jung's own dreams and fantasies but
                                      also on those of his patients; he thought it necessary for the
                                      successful prosecution of their art that psychotherapists become
                                      familiar with writings of the old masters. 




                                      Besides the development of new psychotherapeutic methods that
                                      derived from his own experience and the theories developed from
                                      them, Jung gave fresh importance to the so-called Hermetic tradition.
                                      He conceived that the Christian religion was part of a historic
                                      process necessary for the development of consciousness, but he
                                      thought that the heretical movements, starting with Gnosticism and
                                      ending in alchemy, were manifestations of unconscious archetypal
                                      elements not adequately expressed in the varying forms of
                                      Christianity. He was particularly impressed with his finding that
                                      alchemical-like symbols could be found frequently in modern dreams
                                      and fantasies, and he thought that alchemists had constructed a kind
                                      of textbook of the collective unconscious. He drove this home in four
                                      large volumes of his Collected Works. (see also Index: Hermeticism,
                                      alchemy) 




                                      His historical studies aided him in pioneering the psychotherapy of
                                      the middle-aged and elderly, especially those who felt their lives had
                                      lost meaning. He helped them to appreciate the place of their lives in
                                      the sequence of history. Most of these patients had lost their
                                      religious belief; Jung found that if they could discover their own myth
                                      as expressed in dream and imagination they would become more
                                      complete personalities. He called this process individuation. 


                                      In later years he became professor of psychology at the Federal
                                      Polytechnical University in Z\"{u}rich (1933-41) and professor of
                                      medical psychology at the University of Basel (1943). His personal
                                      experience, his continued psychotherapeutic practice, and his wide
                                      knowledge of history placed him in a unique position to comment on
                                      current events. As early as 1918 he had begun to think that Germany
                                      held a special position in Europe; the Nazi revolution was, therefore,
                                      highly significant for him, and he delivered a number of hotly
                                      contested views that led to his being wrongly branded as a Nazi
                                      sympathizer. Jung lived to the age of 85. 




                                      The authoritative English collection of all Jung's published writings
                                      is Herbert Read, Michael Fordham, and Gerhard Adler (eds.), The
                                      Collected Works of C.G. Jung, trans. by R.F.C. Hull, 20 vol., 2nd ed.
                                      (1966- ). Jung's The Psychology of the Unconscious appears in
                                      revised form as Symbols of Transformation in the Collected Works.
                                      His other major individual publications 
                                      include \"{U}ber die Psychologie
                                      der Dementia Praecox (1907; The Psychology of Dementia Praecox);
                                      Versuch einer Darstellung der psychoanalytischen Theorie (1913;
                                      The Theory of Psychoanalysis); Collected Papers on Analytical
                                      Psychology (1916); Two Essays on Analytical Psychology (1928);
                                      Das Geheimnis der goldenen Bl\"{u}te (1929; The Secret of the Golden
                                      Flower); Modern Man in Search of a Soul (1933), a collection of
                                      essays covering topics from dream analysis and literature to the
                                      psychology of religion; Psychology and Religion (1938); Psychologie
                                      und Alchemie (1944; Psychology and Alchemy); and Aion:
                                      Untersuchungen zur Symbolgeschichte (1951; Aion: Researches into
                                      the Phenomenology of the Self). Jung's Erinnerungen, Tr\"{a}ume,
                                      Gedanken (1962; Memories, Dreams, Reflections) is fascinating
                                      semiautobiographical reading, partly written by Jung himself and
                                      partly recorded by his secretary. (M.S.M.F. /F.Fo.) 
                                      
\subsection{Karl Jaspers}
This page is not complete. It is a collection of ideas from various works and authors, cited as frequently as possible and at the
end of this page. 

Do not rely on this site for more than the chronology and bibliography... Anything indented is a citation with the source listed
following the quote. This page does not contain my own opinions at this time. I simply have not had the time to complete
the massive undertaking of reading Jaspers' works and various biographies! 


Karl Jaspers' role in existentialism cannot be overstated. He coined the term "Existenzphilosophie" -- a forerunner of the term
existentialism -- and this alone makes his contribution unique. Jaspers viewed his philosophy as active, forever changing. This
approach compelled Jaspers to protest any attempt to group him with other philosophers. 
\begin{quote}
          It is in the work of Jaspers that the seeds sown by Kierkegaard and Nietzsche first grew into existentialism or, as he
          prefers to say, Existenzphilosophie. One reason for his opposition to the label "existentialism" is that it suggests a
          school of thought, a doctrine among others, a particular position.
          - Existentialism; Kaufmann, p. 22
\end{quote}

\subsubsection{Kierkegaard + Nietzsche}

As a philosopher who came upon the role along a circuitous path, Jaspers' 
legacy is a merging of S{\o}ren Kierkegaard and Friedrich
Nietzsche. Much like these two predecessors, Jaspers disliked formal philosophy, especially as taught at universities. However,
when merging the basics of Kierkegaard and Nietzsche into a foundation for existentialism, Jaspers did take liberties a "serious"
philosopher would not have. According to Walter Kaufmann: 
\begin{quote}
          To Jaspers the differences between Kierkegaard and Nietzsche seem much less important than that which they have
          in common. What mattered most to them, does not matter to Jaspers: he dismisses Kierkegaard's "forced
          Christianity" no less than Nietzsche's "forced anti-Christianity" as relatively unimportant; he discounts Nietzsche's
          ideas as absurdities, and he does not heed Kierkegaard's central opposition to philosophy. All the many philosophers
          since Hegel and Schelling, however, fare far worse: they are at best instructive but lack human substance: "The
          original philosophers of the age are Kierkegaard and Nietzsche." The crucial fact for Jaspers is that their thinking
          was not academically inspired but rooted in their Existenz. 
          - Existentialism, Kaufmann, p. 23 
\end{quote}
Maybe it was his willingness to discard the prominent themes of both men that allowed Jaspers to create something unique and
exciting. Kierkeaard's Christianity was central to his writings, yet Jaspers had no difficulty dismissing Kierkegaard's faith.
Nietzsche's "anti-Christian" tone was dismissed with equal ease by Jaspers. 

\subsubsection{Philosophy in Academia}
Jaspers shared an important opinion with Kierkegaard and Nietzsche: academic philosophy was unoriginal and lacked any true
value. Kierkegaard regarded philosophers at the universities with some suspicion, believing they distorted the concept of
Christianity. Nietzsche considered the professors state employees, and as such unwilling to challenge socially accepted beliefs,
no matter how wrong. Nietzsche and Kierkegaard both believed professors prone to attempt to justify whatever was currently
popular. Jaspers wrote: 
\begin{quote}
          As the realization overcame me that, at the time, there was no true philosophy at the universities, I thought that
          facing such a vaccuum even he, who was too weak to create his own philosophy, had the right to hold forth about
          philosophy, to declare what it once was and what it could be. -- from On My Philosophy, Jaspers 
\end{quote}
University professors are, to this day, under pressure to lecture upon popular topics and ideals. There is no such thing as an
"alternative" point of view; too often professors join to promote an "alternative" to such an extent that it become the norm within
the lecture halls. Quite simply, to be a respected academic it becomes essential to challenge societal norms. Since such
challenges are expected by the public at large and by other "educated" people, the challenges to accepted values and ideas are
hallow. A philosopher is expected to promote enlightened ideas knowing those ideas will be rejected as part of an elaborate social
ritual. 

While others occupied themselves with the study of philosophy, Jaspers encouraged his students to engage in the the act of
"philosophizing." For Jaspers, debate and discussion were more important than analyzing what was written in the past or how two
famous men might relate on a theoritical level. 
\begin{quote}
          The only significant content of philosophizing, however, consists in the impulses, the inner constitution, the way of
          seeing and judging, the readiness to react by making choices, the immersion in historical presentness, which grow in
          us, recognize themselves, and feel confirmed on the way past all objective contents. -- from Philosophie, Jaspers 
\end{quote}
It is true Jaspers published dozens of books, many nearly 1000 pages per volume. However, his books are a series of lectures and
thoughts. Jaspers did not use diaries or metaphors; he was not a novelist. Jaspers' works exist to encourage debate; he called this
an attemp to "light fires" for debate and discussion. According to Kaufamann, the real mark of Jaspers would be his willingness to
dismiss his own massive body of works as meaningless for anyone else. Jaspers was not trying to convice anyone to adopt a
philosophy, but to think and ponder his or her own Existenz. 

\subsubsection{States of Being}
Jaspers' works present a system in which there are two states of being: the Dasein and Existenz. Some students are confused by
these terms, as Dasein is the name used by Heidegger for a different conceptual framework. Dasein is existence in its most
minimal sense; Dasein is the realm of objectivity and science. Notice objectivity is considered a simplistic approach to
discovering the nature of existence and the self. 

Existenz is "authentic" being. As with Kierkegaard, Nietzsche, and Sartre, we find that Jaspers emphasizes the importance of
decision making a freedom in defining the individual. Total freedom for Jaspers translates into the same infinite possibility to
redefine the self Sartre would describe in his responses to psychoanalysis. Freedom to make a decision apart from all previous
decisions results in a sense of alienation and loneliness -- again, the responsibility of creating a self is a major one. 

There are limits, according to Jaspers, to our freedom. These limitations exist as "boundary situations" including death, suffering,
guilt, chance, and conflict. It should be noted that Jaspers did believe there was a certain randomness to fate; chance situations
arise forcing one to react in a manner not consistent with true freedom. Death stands apart from other boundaries as it is both
the source of dread and the reason many choose to experience pleasures. Without death, there might not be a reason to search for
pleasure. 

\subsubsection{Social Inputs}
If Existenz is a subjective state of being, how can it be evaluated and analyzed by the individual? Jaspers suggests social
interactions offer guidelines that individuals either adopt or reject. In other words, Existenz is a solitary state derived from the
values of society. As with Sartre's idea of "mirrors" ("Hell is other people!"), Jaspers writes of the self as "reflection in someone
else's authentic self." Unless we know what others think and expect of us, we cannot decide who we are or want to be. 

Jaspers, therefore, presents a view in which all people depend upon society for self-definition, even if the act of definition is a
rejection of society's values. No one is truly apart from society. In the extreme, a hermit defines his or her self as a complete
rejection of social structures, but here is no "hermit" without a society from which to seek shelter. As a result, individuals
experience a constant sensation of conflict: a desire to define the self freely while requiring society for that definition. 

\subsubsection{Leaps of Faith}
Karl Jaspers was a man of faith, but not a traditional Christian. His break with tradition was a rejection of the formality and
complex nature of organized religion, not a rejection of a supreme power or divine nature. Jaspers, much like Kierkegaard,
recognized his own faith lacked any basis in logic. This "leap of faith" for Jaspers represented a free choice to believe in an
existence greater than that detected by science. 

\subsubsection*{Resources}
Kaufmann, Walter; Existentialism from Dostoevsky to Sartre; (New York: Meridian, Penguin; 1989) ISBN: 0-452-00930-8 

\section{The Present Age: S{\o}ren Kierkegaard}
Translated from the Danish by Richard Hooker


    The present age is one of understanding, of reflection, devoid of
passion, an age which flies into enthusiasm for a moment only to decline
back into indolence.

   \ldots Not even a suicide does away with himself out of desperation,
he considers the act so long and so deliberately, that he kills himself with
thinking--one could barely call it suicide since it is thinking which takes
his life. He does not kill himself with deliberation but rather kills
himself because of deliberation. Therefore, one can not really prosecute
this generation, for its art, its understanding, its virtuosity and good
sense lies in reaching a judgement or a decision, not in taking action.

   Just as one might say about Revolutionary Ages that they run out of
control, one can say about the Present Age that it doesn't run at all. The
individual and the generation come between and stop each other; and
therefore the prosecuting attorney would find it impossible to admit any
fact at all, because nothing happens in this generation. From a flood of
indications one might think that either something extraordinary happened or
something extraordinary was just about to happen. But one will have thought
wrong, for indications are the only thing the present age achieves, and its
skill and virtuosity entirely consist in building magical illusions; its
momentary enthusiasms which use some projected change in the forms of things
as an escape for actually changing the forms of things, are the highest in
the scale of cleverness and the negative use of that strength which is the
passionate and creating energy during Revolutionary Ages. Eventually, this
present age tires of its chimerical attempts until it declines back into
indolence. Its condition is like one who has just fallen asleep in the
morning: first, great dreams, then laziness, and then a witty or clever
reason for staying in bed.

   The individual (no matter how well-meaning he might be, no matter how
much strength he might have, if only he would use it) does not have the
passion to rip himself away from either the coils of 
Reflection\footnote{This word has two meanings in Kierkegaard: 1.) reflection as "thinking,"
"deliberation," as opposed to acting and doing; 2.) most importantly,
reflection as "reflection," that is, becoming a kind of mirror in which you
derive your individuality from imitating the people around you. In Rousseau,
modern society was characterized by people getting their identity entirely
from the opinions of others; in Kierkegaard, reflection is a matter of
getting your identity solely by imitating others. This gives rise to "the
public."} or the
seductive ambiguities of Reflection; nor do the surroundings and times have
any events or passions, but rather provide a negative setting of a habit of
reflection, which plays with some illusory project only to betray him in the
end with a way out: it shows him that the most clever thing to do is nothing
at all. Vis inertiae\footnote{The way o fintertia} is the foundation of 
the tergiversation\footnote{Evasion, recusal} of the
times, and every passionless person congratulates himself for being the
first to discover it--and becomes, therefore, more clever. Weapons were
freely given out during Revolutionary Ages . . . but in the present age
everyone is given clever rules and calculators in order to aid one's
thinking. If any generation had the diplomatic task of postponing action so
that it might appear that something were about to happen, even though it
would never happen, then one would have to say that our age has achieved as
mightily as Revolutionary Ages. Someone should try an experiment with
himself: he should forget everything he knows about the times and its
relativity amplified by its familiarity, and then come into this age as if
he were from another planet, and read some book, or some article in the
newspaper: he will have this impression: "Something is going to happen
tonight, or else something happened last night!"

   A Revolutionary Age is an age of action; the present age is an age of
advertisement, or an age of publicity: nothing happens, but there is instant
publicity about it. A revolt in the present age is the most unthinkable act
of all; such a display of strength would confuse the calculating cleverness
of the times. Nevertheless, some political virtuoso might achieve something
nearly as great. He would write some manifesto or other which calls for a
General Assembly in order to decide on a revolution, and he would write it
so carefully that even the Censor himself would pass on it; and at the
General Assembly he would manage to bring it about that the audience
believed that it had actually rebelled, and then everyone would placidly go
home--after they had spent a very nice evening out. An enormous grounding in
scholarship is alien to the youth of today, in fact, they would find it
laughable. Nevertheless, some scientific virtuoso might achieve something
even greater. He would draw up some prospectus outlining systematically some
all-embracing, all-explaining system that he was about to write, and he
would manage to achieve the feat of convincing the reader (of the
prospectus) that he had in fact read the entire system. The Age of
Encyclopedists is gone, when with great pains men wrote large Folios; now we
have an age of intellectual tourists, small little encyclopedists, who, here
and there, deal with all sciences and all existence. And a genuine religious
rejection of the world, followed with constant self-denial, is equally
unthinkable among the youth of our time: nevertheless, some bible college
student has the virtuosity to achieve something even greater. He could
design some projected group or Society which aims to save those who are
lost. The age of great achievers is gone, the present age is an age of
anticipators. . . . Like a youth who plans to diligently study from
September 1 for an exam, and in order to solidify his resolve takes a
holiday for the entire month of August, such is our generation which has
decided resolutely that the next generation will work very hard, and in
order not to interfere with or delay the next generation, this generation
diligently--goes to parties. However, there is one difference in this
comparison: the youth understands that he is light-hearted, the present age
is on the contrary very serious--even at their parties.

   Action and passion is as absent in the present age as peril is absent
from swimming in shallow waters. . . .

   If a precious jewel, which all desired, lay out on a frozen lake,
where the ice was perilously thin, where death threatened one who went out
too far while the ice near the shore was safe, in a passionate age the
crowds would cheer the courage of the man who went out on the ice; they
would fear for him and with him in his resolute action; they would sorrow
over him if he went under; they would consider him divine if he returned
with the jewel. In this passionless, reflective age, things would be
different. People would think themselves very intelligent in figuring out
the foolishness and worthlessness of going out on the ice, indeed, that it
would be incomprehensible and laughable; and thereby they would transform
passionate daring into a display of skill . . . . The people would go and
watch from safety and the connoisseurs with their discerning tastes would
carefully judge the skilled skater, who would go almost to the edge (that
is, as far as the ice was safe, and would not go beyond this point) and then
swing back. The most skilled skaters would go out the furthest and venture
most dangerously, in order to make the crowds gasp and say: "Gods! He is
insane, he will kill himself!" But you will see that his skill is so
perfected that he will at the right moment swing around while the ice is
still safe and his life is not endangered. . . .

   Men, then, only desire money, and money is an abstraction, a form of
reflection . . . Men do not envy the gifts of others, their skill, or the
love of their women; they only envy each others' money. . . . These men
would die with nothing to repent of, believing that if only they had the
money, they might have truly lived and truly achieved something.

   The established order continues, but our reflection and
passionlessness finds its satisfaction in ambiguity. No person wishes to
destroy the power of the king, but if little by little it can be reduced to
nothing but a fiction, then everyone would cheer the king. No person wishes
to pull down the pre-eminent, but if at the same time pre-eminence could be
demonstrated to be a fiction, then everyone would be happy. No person wishes
to abandon Christian terminology, but they can secretly change it so that it
doesn't require decision or action. And so they are unrepentant, since they
have not pulled down anything. People do not desire any more to have a
strong king than they do a hero-liberator than they do religious authority,
for they innocently wish the established order to continue, but in a
reflective way they more or less know that the established order no longer
continues. . . .

   The reflective tension this creates constitutes itself into a new
principle, and just as in an age of passion enthusiasm is the unifying
principle, so in a passionless age of reflection envy 
(misundelse)\footnote{Besides "envy," etymologically misundelse means "contrariness" or "spite."
This is very similar to Nietzsche's ressentiment.} is the
negative-unifying principle. This must not be understood as a moral term,
but rather, the idea of reflection, as it were, is envy, and envy is
therefore twofold: it is selfish in the individual and in the society around
him. The envy of reflection in the individual hinders any passionate
decision he might make; and if he wishes to free himself from reflection,
the reflection of society around him re-captures him. . . .

   Envy (misundelse) constitutes the principle of characterlessness,
which from its misery sneaks up until it arrives at some position, and it
protects itself with the concession that it is nothing. The envy of
characterlessness never understands that distinction is really a
distinction, nor does it understand itself in recognizing distinction
negatively,\footnote{That is, it does not understand the exceptional in a positive sense as
being better than itself nor does it understand the exceptional in a
negative sense as being worse than itself.} 
but rather reduces it so that it is no longer distinction; and
envy defends itself not only from distinction, but against that distinction
which is to come.\footnote{The Final Judgement}

   Envy which is establishing itself is a levelling,\footnote{That is, it flattens everything to the same level; nothing is below this
level, nothing is above this level.} and while a
passionate age pushes forward, establishing new things and destroying
others, raising and tearing down, a reflective, passionless age does the
opposite, it stifles and hinders, it levels. This levelling is a silent,
mathematical, abstract process which avoids upheavals. . . . Levelling at
its maximum is like the stillness of death, where one can hear one's own
heartbeat, a stillness like death, into which nothing can penetrate, in
which everything sinks, powerless.

   One person can head a rebellion, but one person cannot head this
levelling process, for that would make him a leader and he would avoid being
levelled. Each individual can in his little circle participate in this
levelling, but it is an abstract process, and levelling is abstraction
conquering individuality. The levelling in modern times is the reflective
equivalent of fate in the ancient times. The dialectic of ancient times
tended towards leadership (the great man over the masses and the free man
over the slave); the dialectic of Christianity tends, at least until now,
towards representation (the majority views itself in the representative, and
is liberated in the knowledge that it is represented in that representative,
in a kind of self-knowledge); the dialectic of the present age tends towards
equality, and its most consequent but false result is levelling, as the
negative unity of the negative relationship between individuals.

   Everyone should see now that levelling has a fundamental meaning: the
category of "generation" supersedes the category of the "individual." During
ancient times the mass of individuals had this value: that it made valuable
the outstanding individual. . . . In ancient times, the single individual in
the masses signified nothing; the outstanding individual signified them all.
In the present age, the tendency is towards a mathematical equality . . .

   In order for levelling really to occur, first it is necessary to
bring a phantom into existence, a spirit of levelling, a huge abstraction,
an all-embracing something that is nothing, an illusion--the phantom of the
public. . . . The public is the real Levelling-Master, rather than the
leveller itself, for levelling is done by something, and the public is a
huge nothing.

   The public is an idea, which would never have occurred to people in
ancient times, for the people themselves en masse in corpore\footnote{"In mass, in a single body"} took steps
in any active situation, and bore responsibility for each individual among
them, and each individual had to personally, without fail, present himself
and submit his decision immediately to approval or disapproval. When first a
clever society makes concrete reality into nothing, then the 
Media\footnote{Danish Pressen , "the press," which in contemporary English is called "the
media."} creates
that abstraction, "the public," which is filled with unreal individuals, who
are never united nor can they ever unite simultaneously in a single
situation or organization, yet still stick together as a whole. The public
is a body, more numerous than the people which compose it, but this body can
never be shown, indeed it can never have only a single representation,
because it is an abstraction. Yet this public becomes larger, the more the
times become passionless and reflective and destroy concrete reality; this
whole, the public, soon embraces everything. . . .

   The public is not a people, it is not a generation, it is not a
simultaneity, it is not a community, it is not a society, it is not an
association, it is not those particular men over there, because all these
exist because they are concrete and real; however, no single individual who
belongs to the public has any real commitment; some times during the day he
belongs to the public, namely, in those times in which he is nothing; in
those times that he is a particular person, he does not belong to the
public. Consisting of such individuals, who as individuals are nothing, the
public becomes a huge something, a nothing, an abstract desert and
emptiness, which is everything and nothing. . . .

   The Media is an abstraction (because a newspaper is not concrete and
only in an abstract sense can be considered an individual), which in
association with the passionlessness and reflection of the times creates
that abstract phantom, the public, which is the actual leveller. . . . More
and more individuals will, because of their indolent bloodlessness, aspire
to become nothing, in order to become the public, this abstract whole, which
forms in this ridiculous manner: the public comes into existence because all
its participants become third parties.\footnote{That is, viewers, onlookers, people who watch what happens rather than
makes anything happen.} This lazy mass, which understands
nothing and does nothing, this public gallery seeks some distraction, and
soon gives itself over to the idea that everything which someone does, or
achieves, has been done to provide the public something to gossip about. . .
. The public has a dog for its amusement. That dog is the 
Media.\footnote{Danish: den literaire Foragtelighed , literally, the literary scandal
sheets; what we would call "tabloids."} If there
is someone better than the public, someone who distinguishes himself, the
public sets the dog on him and all the amusement begins. This biting dog
tears up his coat-tails, and takes all sort of vulgar liberties with his
leg--until the public bores of it all and calls the dog off. That is how the
public levels.

                                     
\section{Collectivism}
 of several types of social organization in which the individual is
                                      seen as being subordinate to a social collectivity such as a state, a
                                      nation, a race, or a social class. Collectivism may be contrasted with
                                      individualism (q.v.), in which the rights and interests of the
                                      individual are emphasized. (see also Index: individualism) 

                                      The earliest modern, influential expression of collectivist ideas in
                                      the West is in Jean-Jacques Rousseau's Du contrat social, of 1762
                                      (see social contract), in which it is argued that the individual finds
                                      his true being and freedom only in submission to the "general will" of
                                      the community. In the early 19th century the German philosopher
                                      G.W.F. Hegel argued that the individual realizes his true being and
                                      freedom only in unqualified submission to the laws and institutions
                                      of the nation-state, which to Hegel was the highest embodiment of
                                      social morality. Karl Marx later provided the most succinct statement
                                      of the collectivist view of the primacy of social interaction in the
                                      preface to his Contribution to the Critique of Political Economy: "It
                                      is not men's consciousness," he wrote, "which determines their being,
                                      but their social being which determines their consciousness." (see
                                      also Index: "Social Contract, The," ) 

                                      Collectivism has found varying degrees of expression in the 20th
                                      century in such movements as socialism, communism, and fascism.
                                      The least collectivist of these is social democracy, which seeks to
                                      reduce the inequities of unrestrained capitalism by government
                                      regulation, redistribution of income, and varying degrees of planning
                                      and public ownership. In communist systems collectivism is carried
                                      to its furthest extreme, with a minimum of private ownership and a
                                      maximum of planned economy. 


\subsection{Persona (Jung)}
in psychology, the personality that an individual projects to others,
as differentiated from the authentic self. The term, coined by Carl
Jung, is derived from the Latin persona, referring to the masks worn
by Etruscan mimes. According to Jung, the persona enables an
individual to interrelate with the world around him by reflecting the
role in life that the individual is playing. In this way one can arrive at
a compromise between one's innate psychological constitution and
society. Thus the persona enables the individual to adapt to society's
demands. 

%\documentclass{article}
%\makeatletter
%\input{a4wide}
%\makeatother
%\newcommand{\pa}[1]{\begin{quote}#1\end{quote}}
%\newcommand{\fr}[1]{\par\noindent{\footnotesize{#1}}}
%\begin{document}
%\subsection*{Persona}
\pa{The persona is a complicated system of relations between individual consciousness and society, fittingly enough a kind of
          mask, designed on the one hand to make a definite impression upon others, and, on the other, to conceal the true nature of
          the individual.}
\fr{"The Relations between the Ego and the Unconscious" (1928). In CW 7: Two 
Essays on Analytical Psychology. P.305}

\pa{Whoever looks into the mirror of the water will see first of all his own face. Whoever goes to himself risks a confrontation
          with himself. The mirror does not flatter, it faithfully shows whatever looks into it; namely, the face we never show to the
          world because we cover it with the persona, the mask of the actor. But the mirror lies behind the mask and shows the true
          face.}
\fr{"Archetypes of the Collective Unconscious" (1935). In CW 9, Part I: The 
Archetypes and the Collective Unconscious. P.43 } 

\pa{Every calling or profession has its own characteristic persona. It is easy to study these things nowadays, when the
          photographs of public personalities so frequently appear in the press. A certain kind of behaviour is forced on them by the
          world, and professional people endeavour to come up to these expectations. Only, the danger is that they become identical
          with their personas-the professor with his text-book, the tenor with his voice. Then the damage is done; henceforth he lives
          exclusively against the background of his own biography... The garment of Deianeira has grown fast to his skin, and a
          desperate decision like that of Heracles is needed if he is to tear this Nessus shirt from his body and step into the consuming
          fire of the flame of immortality, in order to transform himself into what he really is. One could say, with a little
          exaggeration, that the persona is that which in reality one is not, 
          but which oneself as well as others think one is. }
\fr{ "Concerning Rebirth" (1940). In CW 9, Part I: The Archetypes and the Collective Unconscious. P.221 }


\pa{          I once made the acquaintance of a very venerable personage - in fact, one might easily call him a saint. I stalked round him
          for three whole days, but never a mortal failing did I find in him. My feeling of inferiority grew ominous, and I was beginning
          to think seriously of how I might better myself. Then, on the fourth day, his wife came to consult me.... Well, nothing of the
          sort has ever happened to me since. But this I did learn: that any man who becomes one with his persona can cheerfully let
          all disturbances manifest themselves through his wife without her noticing it, though she pays for her self-sacrifice with a
          bad neurosis. }
\fr{ "The Relations between the Ego and the Unconscious" (1928). In CW 7: Two 
Essays on Analytical Psychology. P.306 }


\pa{Since the differentiated consciousness of civilized man has been granted an effective instrument for the practical realization
          of its contents through the dynamics of his will, there is all the more danger, the more he trains his will, of his getting lost
          in one-sidedness and deviating further and further from the laws and roots of his being.}
\fr{"The Psychology of the Child Archetype" 
(1940) In CW 9, Part I: The Archetypes and the Collective Unconscious. P.276 } 


\pa{ When there is a marked change in the individual's state of consciousness, the unconscious contents which are thereby
          constellated will also change. And the further the conscious situation moves away from a certain point of equilibrium, the
          more forceful and accordingly the more dangerous become the unconscious contents that are struggling to restore the
          balance. This leads ultimately to a dissociation: on the one hand, ego-consciousness makes convulsive efforts to shake off
          an invisible opponent (if it does not suspect its next-door neighbour of being the devil!), while on the other hand it
          increasingly falls victim to the tyrannical will of an internal "Government opposition" which displays all the characteristics
          of a daemonic subman and superman combined. When a few million people get into this state, it produces the sort of
          situation which has afforded us such an edifying object-lesson every day for the last ten years.* These contemporary events
          betray their psychological background by their very singularity. The insensate destruction and devastation are a reaction
          against the deflection of consciousness from the point of equilibrium. For an equilibrium does in fact exist between the
          psychic ego and non-ego, and that equilibrium is a religion a "careful consideration" of ever-present unconscious forces
          which we neglect at our peril. }
\fr{"The Psychology of Transference" (1946). In CW 16: The Practice of Psychotherapy. P.394,
       *The years 1935-1945}


\pa{          Nothing is so apt to challenge our self-awareness and alertness as being at war with oneself. One can hardly think of any
          other or more effective means of waking humanity out of the irresponsible and innocent half-sleep of the primitive mentality
          and bringing it to a state of conscious responsibility. }
\fr{                                          "Psychological Typology" (1936). 
In CW 6: Psychological Types. P. 964}


\pa{          Hidden in the neurosis is a bit of still undeveloped personality, a precious fragment of the psyche lacking which a man is
          condemned to resignation, bitterness, and everything else that is hostile to life. A psychology of neurosis that sees only the
          negative elements empties out the baby with the bath-water, since it neglects the positive meaning and value of these
          "infantile' i.e., creative-fantasies. }
\fr{       "The State of Psychotherapy Today" (1934). In CW 10: Civilization 
in Transition. P.355}


\pa{          We yield too much to the ridiculous fear that we are at bottom quite impossible beings, that if everyone were to appear as he
          really is a frightful social catastrophe would ensue. Many people today take "man as he really is" to mean merely the
          eternally discontented, anarchic, rapacious element in human beings, quite forgetting that these same human beings have
          also erected those firmly established forms of civilization which possess greater strength and stability than all the anarchic
          undercurrents. The strengthening of his social personality is one of the essential conditions for man's existence. Were it not
          so, humanity would cease to be. The selfishness and rebelliousness we meet in the neurotic's psychology are not "man as he
          really is" but an infantile distortion. In reality the normal man is "civic minded and moral"; he created his laws and observes
          them, not because they are imposed on him from without-that is a childish delusion-but because he loves law and order
          more than he loves disorder and lawlessness. }
\fr{                                "Archetypes and the Collective Unconscious" 
(1935). In CW 9, Part I: The Archetypes and the Collective Unconscious. P.442}


\pa{          The true genius nearly always intrudes and disturbs. He speaks to a temporal world out of a world eternal. He says the wrong
          things at the right time. Eternal truths are never true at any given moment in history. The process of transformation has to
          make a halt in order to digest and assimilate the utterly impractical things that the genius has produced from the
          storehouse of eternity. Yet the genius is the healer of his time, 
          because anything he reveals of eternal truth is healing. } 
\fr{                                                         "What India Can 
Teach Us" (1939). In CW 10: Civilization in Transition. P. 1004} 


\pa{          The genius will come through despite everything, for there is something absolute and indomitable in his nature. The
          so-called "misunderstood genius" is rather a doubtful phenomenon. Generally he turns out to be a good-for-nothing who is
          forever seeking a soothing explanation of himself. }
\fr{                        "The Gifted Child" (1943). In CW 17: The Development of Personality. P. 248 }


\pa{          Whoever speaks in primordial images speaks with a thousand voices; he enthrals and overpowers, while at the same time he
          lifts the idea he is seeking to express out of the occasional and the transitory into the realm of the ever enduring. He
          transmutes our personal destiny into the destiny of mankind, and evokes in us all those beneficent forces that ever and
          anon have enabled humanity to find a refuge from every peril and to outlive the longest night. 
}
\fr{                                      "On the Relation of Analytical Psychology 
of Poetry" (1922). In CW 15: The Spirit in Man, Art and Literature. P.129}


\pa{To be "normal" is the ideal aim for the unsuccessful, for all those who
are still below the general level of adaptation. But for
people of more than average ability, people who never found it difficult to
gain successes and to accomplish their share of the
world's work -- for them the moral compulsion to be nothing but normal signifies 
the bed of Procrustes -- deadly and
insupportable boredom, a hell of sterility and hopelessness. 
} \fr{"Problems of Modern Psychotherapy" 
 (1929). In CW 16: The Practice of Psychotherapy. P. 161 } 

%%%
\pa{          Nothing in us ever remains quite uncontradicted, and consciousness can take up no position which will not call up,
          somewhere in the dark corners of the psyche, a negation or a compensatory effect, approval or resentment. This process of
          coming to terms with the Other in us is well worth while, because in this way we get to know aspects of our nature which we
          would not allow anybody else to show us and which we ourselves would never have admitted. 
}
\fr{Mysterium Coniunctionis (1955) CW 14: P. 706} 


\pa{          The "other" in us always seems alien and unacceptable; but if we let ourselves be aggrieved the feeling sinks in, and we are
          the richer for this little bit of self-knowledge. 
}
\fr{                                       "Psychological Aspects of the Kore" 
(1941). In CW 9, Part I: The Archetypes of the Collective Unconscious. P. 918} 

\pa{          If we do not fashion for ourselves a picture of the world, we do not see ourselves either, who are the faithful reflections of
          that world. Only when mirrored in our picture of the world can we see ourselves in the round. Only in our creative acts do we
          step forth into the light and see ourselves whole and complete. Never shall we put any face on the world other than our own,
          and we have to do this precisely in order to find ourselves. For higher than science or art as an end in itself stands man, the
          creator of his instruments. }
\fr{                                       "Analytical Psychology and Weltanschauung" 
(1928). In CW 8: The Structure and Dynamics of the Psyche. P.737}

\subsection*{Individuation (Jung)}


\pa{          To find out what is truly individual in ourselves, profound reflection is needed; and suddenly we realize how uncommonly
          difficult the discovery of individuality in fact is. 
}
\fr{                                    "The Relations between the Ego and the Unconscious" (1928). In CW 7: Two Essays on Analytical Psychology. P. 242 
}

\pa{          We do not sufficiently distinguish between Individualism and individuation. Individualism means deliberately stressing and
          giving prominence to some supposed peculiarity, rather than to collective considerations and obligations. But individuation
          means precisely the better and more complete fulfilment of the collective qualities of the human being, since adequate
          consideration of the peculiarity of the individual is more conducive to better social achievement than when the peculiarity is
          neglected or suppressed. 
}
\fr{                                     "The Relations between the Ego and the Unconscious" (1928). In CW 7: Two Essays on Analytical Psychology. P. 267
}

\pa{          Every advance in culture is, psychologically, an extension of consciousness, a coming to consciousness that can take place
          only through discrimination. Therefore an advance always begins with individuation, that is to say with the individual,
          conscious of his isolation, cutting a new path through hitherto untrodden territory. To do this he must first return to the
          fundamental facts of his own being, irrespective of all authority and tradition, and allow himself to become conscious of his
          distinctiveness. If he succeeds in giving collective validity to his widened consciousness, he creates a tension of opposites
          that provides the stimulation which culture needs for its further progress. 
}
\fr{                                                    "On Psychic Energy" (1928). In CW 8: The Structure and Dynamics of the Psyche. P. 111
}

\pa{          Our life is like the course of the sun. In the morning it gains continually in strength until it reaches the zenith heat of high
          noon. Then comes the enantiodromia: the steady forward movement no longer denotes an increase, but a decrease, in
          strength. Thus our task in handling a young person is different from the task of handling an older person. In the former
          case, it is enough to clear away all the obstacles that hinder expansion and ascent; in the latter, we must nurture everything
          that assists the descent. 
}
\fr{                                             "On the Psychology of the Unconscious" (1912). In CW 7: Two Essays on Analytical Psychology P. 114 
}

\pa{          Neurosis is intimately bound up with the problem of our time and really represents an unsuccessful attempt on the part of
          the individual to solve the general problem in his own person. Neurosis is self-division. 
}
\fr{                                              "On the Psychology of the Unconscious" (1912). In CW 7: Two Essays on Analytical Psychology P. 18
}

\pa{          The small world of the child, the family milieu, is the model for the big world. The more intensely the family sets its stamp
          on the child, the more he will be emotionally inclined, as an adult, to see in the great world his former small world. Of
          course this must not be taken as a conscious intellectual process. On the contrary, the patient feels and sees the difference
          between now and then, and tries as well as he can to adapt himself. Perhaps he will even believe himself perfectly adapted,
          since he may be able to grasp the situation intellectually, but that does not prevent his emotions from lagging far behind his
          intellectual insight. 
}
\fr{                                                       "The Theory of Psychoanalysis" (1913). In CW 4: Freud and Psychoanalysis. P. 312
}

\pa{          Infantilism, however, is something extremely ambiguous. First, it can be either genuine or purely symptomatic; and second,
          it can be either residuary or embryonic. There is an enormous difference between something that has remained infantile and
          something that is in the process of growth. Both can take an infantile or embryonic form, and more often than not it is
          impossible to tell at first glance whether we are dealing with a regrettably persistent fragment of infantile life or with a vitally
          important creative beginning. To deride these possibilities is to act like a dullard who does not know that the future is more
          important than the past. 
}
\fr{                                                      "The State of Psychotherapy Today" (1934). In CW 10: Civilization in Transition. P.345
}

\pa{          Since the aims of the second half of life are different from those of the first, to linger too long in the youthful attitude
          produces a division of the will. Consciousness still presses forward in obedience, as it were, to its own inertia, but the
          unconscious lags behind, because the strength and inner resolve needed for further expansion have been sapped. This
          disunity with oneself begets discontent, and since one is not conscious of the real state of things one generally projects the
          reasons for it upon one's partner. A critical atmosphere thus develops, the necessary prelude to conscious realization. 
}
\fr{                                           "Marriage as a Psychological Relationship" (1925). In CW 17: The Development of the Personality. P. 331 
}

\pa{          It is not possible to live too long amid infantile surroundings, or in the bosom of the family, without endangering one's
          psychic health. Life calls us forth to independence, and anyone who does not heed this call because of childish laziness or
          timidity is threatened with neurosis. And once this has broken out, it becomes an increasingly valid reason for running away
          from life and remaining forever in the morally poisonous atmosphere of infancy. 
}
\fr{                                                                           Symbols of Transformation (1952). CW 5: P. 461
}

\pa{          If we try to extract the common and essential factors from the almost inexhaustible variety of individual problems found in
          the period of youth, we meet in all cases with one particular feature: a more or less patent clinging to the childhood level of
          consciousness, a resistance to the fateful forces in and around us which would involve us in the world. Something in us
          wishes to remain a child, to be unconscious or, at most, conscious only of the ego; to reject everything strange, or else
          subject it to our will; to do nothing, or else indulge our own craving for pleasure or power. In all this there is something of
          the inertia of matter; it is a persistence in the previous state whose range of consciousness is smaller, narrower, and more
          egoistic than that of the dualistic phase. For here the individual is faced with the necessity of recognizing and accepting
          what is different and strange as a part of his own life, as a kind of "also - I." 
}
\fr{                                                     "The Stages of Life" (1930). In CW 8: The Structure and Dynamics of the Psyche. P. 764
}

\pa{          Obviously it is in the youthful period of life that we have most to gain from a thorough recognition of the instinctual side. A
          timely recognition of sexuality, for instance, can prevent that neurotic suppression of it which keeps a man unduly
          withdrawn from life, or else forces him into a wretched and unsuitable way of living with which he is bound to come into
          conflict. Proper recognition and appreciation of normal instincts leads the young person into life and entangles him with
          fate, thus involving him in life's necessities and the consequent sacrifices and efforts through which his character is
          developed and his experience matured. For the mature person, however, the continued expansion of life is obviously not the
          right principle, because the descent towards life's afternoon demands simplification, limitation, and intensification-in other
          words, individual culture. 
}
\fr{                                                    "On Psychic Energy" (1928). In CW 8: The Structure and Dynamics of the Psyche. P. 113
}

\pa{It even seems as if young people who have had a hard struggle for 
existence are spared inner problems, while those who for
some reason or other have no difficulty with adaptation run into 
problems of sex or conflicts arising from a sense of
inferiority. 
}
\fr{ "The Stages of Life" (1930). In CW 8: The Structure and Dynamics of the Psyche. P. 762 
}

\pa{          The nearer we approach to the middle of life, and the better we have succeeded in entrenching ourselves in our personal
          attitudes and social positions, the more it appears as if we had discovered the right course and the right ideals and
          principles of behaviour. For this reason we suppose them to be eternally valid, and make a virtue of unchangeably clinging to
          them. We overlook the essential fact that the social goal is attained only at the cost of a diminution of personality. Many-far
          too many-aspects of life which should also have been experienced lie in the lumber-room among dusty memories; but
          sometimes, too, they are glowing coals under grey ashes. 
}
\fr{                                                     "The Stages of Life" (1930). In CW 8: The Structure and Dynamics of the Psyche. P. 771
}

\pa{          The discovery of the value of human personality is reserved for a riper age. For young people the search for personality values
          is very often a pretext for evading their biological duty. Conversely, the exaggerated longing of an older person for the sexual
          values of youth is a short-sighted and often cowardly evasion of a duty which demands recognition of the value of
          personality and submission to the hierarchy of cultural values. The young neurotic shrinks back in terror from the
          expansion of life's duties, the old one from the dwindling of the treasures he has attained. 
}
\fr{                                                                               CW 4: Freud and Psychoanalysis. P.664 
}

\pa{          The middle period of life is a time of enormous psychological importance. The child begins its psychological life within very
          narrow limits, inside the magic circle of the mother and the family. With progressive maturation it widens its horizon and its
          own sphere of influence; its hopes and intentions are directed to extending the scope of personal power and possessions;
          desire reaches out to the world in ever-widening range; the will of the individual becomes more and more identical with the
          natural goals pursued by unconscious motivations. Thus man breathes his own life into things, until finally they begin to
          live of themselves and to multiply; and imperceptibly he is overgrown by them. Mothers are overtaken by their children, men
          by their own creations, and what was originally brought into being only with labor and the greatest effort can no longer be
          held in check. First it was passion, then it became duty, and finally an intolerable burden, a vampire that fattens on the life
          of its creator. 
}
\fr{                                           "Marriage as a Psychological Relationship" (1925). In CW 17: The Development of the Personality. P. 331 
}

\pa{      Take for comparison the daily course of the sun-but a sun that is endowed with human feeling and man's limited
          consciousness. In the morning it rises from the nocturnal sea of unconsciousness and looks upon the wide, bright world
          which lies before it in an expanse that steadily widens the higher it climbs in the firmament. In this extension of its field of
          action caused by its own rising, the sun will discover its significance; it will see the attainment of the greatest possible
          height, and the widest possible dissemination of its blessings, as its goal. In this conviction the sun pursues its course to
          the unforeseen zenith-unforeseen, because its career is unique and individual, and the culminating point could not be
          calculated in advance. At the stroke of noon the descent begins. And the descent means the reversal of all the ideals and
          values that were cherished in the morning. 
}
\fr{                                                     "The Stages of Life" (1930). In CW 8: The Structure and Dynamics of the Psyche. P. 778
}

\pa{          Wholly unprepared, we embark upon the second half of life. Or are there perhaps colleges for forty-year-olds which prepare
          them for their coming life and its demands as the ordinary colleges introduce our young people to a knowledge of the world?
          No, thoroughly unprepared we take the step into the afternoon of life; worse still, we take this step with the false
          assumption that our truths and ideals will serve us as hitherto. But we cannot live the afternoon of life according to the
          programme of life's morning; for what was great in the morning will be little at evening, and what in the morning was true
          will at evening have become a lie. 
}
\fr{                                                     "The Stages of Life" (1930). In CW 8: The Structure and Dynamics of the Psyche. P. 784
}

\pa{          An inexperienced youth thinks one can let the old people go, because not much more can happen to them anyway: they have
          their lives behind them and are no better than petrified pillars of the past. But it is a great mistake to suppose that the
          meaning of life is exhausted with the period of youth and expansion; that, for example, a woman who has passed the
          menopause is "finished." The afternoon of life is just as full of meaning as the morning; only, its meaning and purpose are
          different. 
}
\fr{                                             "On the Psychology of the Unconscious" (1912). In CW 7: Two Essays on Analytical Psychology P. 114
}

\pa{          If we wish to stay on the heights we have reached, we must struggle all the time to consolidate our consciousness and its
          attitude. But we soon discover that this praiseworthy and apparently unavoidable battle with the years leads to stagnation
          and desiccation of soul. Our convictions become platitudes ground out on a barrel-organ, our ideals become starchy habits,
          enthusiasm stiffens into automatic gestures. The source of the water of life seeps away. We ourselves may not notice it, but
          everybody else does, and that is even more painful. If we should risk a little introspection, coupled perhaps with an energetic
          attempt to be honest for once with ourselves, we may get a dim idea of all the wants, longings, and fears that have
          accumulated down there-a repulsive and sinister sight. The mind shies away, but life wants to flow down into the depths.
          Fate itself seems to preserve us from this, for each of us has a tendency to become an immovable pillar of the past. 
}
\fr{                                                                           Symbols of Transformation (1952). CW 5: P. 553
}

\pa{          In my naturally limited experience there are, among people of maturer age, very many for whom the development of
          individuality is an indispensable requirement. Hence I am privately of the opinion that it is just the mature person who, in
          our times, has the greatest need of some further education in individual culture after his youthful education in school or
          university has moulded him on exclusively collective lines and thoroughly imbued him with the collective mentality. 
}
\fr{                                                    "On Psychic Energy" (1928). In CW 8: The Structure and Dynamics of the Psyche. P. 112 
}

\pa{          A human being would certainly not grow to be seventy or eighty years old if this longevity had no meaning for the species.
          The afternoon of human life must also have a significance of its own and cannot be merely a pitiful appendage to life's
          morning. The significance of the morning undoubtedly lies in the development of the individual, our entrenchment in the
          outer world, the propagation of our kind, and the care of our children. This is the obvious purpose of nature. But when this
          purpose has been attained -and more than attained-shall the earning of money, the extension of conquests, and the
          expansion of life go steadily on beyond the bounds of all reason and sense? Whoever carries over into the afternoon the law
          of the morning, or the natural aim, must pay for it with damage to his soul, just as surely as a growing youth who tries to
          carry over his childish egoism into adult life must pay for this mistake with social failure. 
}
\fr{                                                     "The Stages of Life" (1930). In CW 8: The Structure and Dynamics of the Psyche. P. 787 
}

\pa{Our personality develops in the course of our life from germs that are
hard or impossible to discern, and it is only our deeds that reveal
who we are. We are like the sun, which nourishes the life of the
          earth and brings forth every kind of strange, wonderful, and evil
          thing; we are like the mothers who bear in their wombs untold
          happiness and suffering. At first we do not know what deeds or
          misdeeds, what destiny, what good and evil we have in us, and only
          the autumn can show what the spring has engendered, only in the
          evening will it be seen what the morning began.  } \fr{ "The
          Development of the Personality" (1934). In CW 17: The Development
          of the Personality. P.290 }

\pa{          He who is rooted in the soil endures. Alienation from the unconscious and from its historical conditions spells rootlessness.
          That is the danger that lies in wait for the conqueror of foreign lands, and for every individual who, through one-sided
          allegiance to any kind of -ism, loses touch with the dark, maternal, earthy ground of his being. 
}
\fr{                                                               "Mind and Earth" (1927). In CW 10: Civilization in Transition. P. 103 
}

\pa{          In the case of psychological suffering, which always isolates the individual from the herd of so- called normal people, it is of
          the greatest importance to understand that the conflict is not a personal failure only, but at the same time a suffering
          common to all and a problem with which the whole epoch is burdened. This general point of view lifts the individual out of
          himself and connects him with humanity. 
}
\fr{                             Analytical Psychology: Its Theory and Practice: The Tavistock Lectures. (1935). In CW 18 (retitled) "The Tavistock Lectures" P.116
}

\pa{          On closer examination one is always astonished to see how much of our so-called individual psychology is really collective.
          So much, indeed, that the individual traits are completely overshadowed by it. Since, however, individuation is an
          ineluctable psychological necessity, we can see from the ascendancy of the collective what very special attention must be
          paid to this delicate plant "individuality" if it is not to be completely smothered. 
}
\fr{                                     "The Relations between the Ego and the Unconscious" (1928). In CW 7: Two Essays on Analytical Psychology. P. 241
}

\pa{          To be "normal" is the ideal aim for the unsuccessful, for all those who are still below the general level of adaptation. But for
          people of more than average ability, people who never found it difficult to gain successes and to accomplish their share of the
          world's work-for them the moral compulsion to be nothing but normal signifies the bed of Procrustes-deadly and
          insupportable boredom, a hell of sterility and hopelessness. 
}
\fr{                                                "The Principles of Practical Psychology" (1935). In CW 16: The Practice of Psychotherapy. P.161
}

\pa{          A person must pay dearly for the divine gift of creative fire. It is as though each of us was born with a limited store of energy.
          In the artist, the strongest force in his make-up, that is, his creativeness, will seize and all but monopolize this energy,
          leaving so little over that nothing of value can come of it. The creative impulse can drain him of his humanity to such a
          degree that the personal ego can exist only on a primitive or inferior level and is driven to develop all sorts of
          defects-ruthlessness, selfishness ("autoeroticism"), vanity, and other infantile traits. These inferiorities are the only means
          by which it can maintain its vitality and prevent itself from being wholly depleted. 
}
\fr{                                                   "Psychology and Literature" (1930). In CW 15: The Spirit in Man, Art and Literature. P. 158 
}
%%%
\pa{          There would appear to be a sort of conscience in mankind which severely punishes every one who does not somehow and at
          some time, at whatever cost to his virtuous pride, cease to defend and assert himself, and instead confess himself fallible
          and human. Until he can do this, an impenetrable wall shuts him off from the vital feeling that he is a man among other
          men. 
}
\fr{                                                  "Problems of Modern Psychotherapy" (1929). In CW 16: The Practice of Psychotherapy. P.132
}

\pa{          Every individual needs revolution, inner division, overthrow of the existing order, and renewal, but not by forcing these
          things upon his neighbours under the hypocritical cloak of Christian love or the sense of social responsibility or any of the
          other beautiful euphemisms for unconscious urges to personal power. Individual self-reflection, return of the individual to
          the ground of human nature, to his own deepest being with its individual and social destiny here is the beginning of a cure
          for that blindness which reigns at the present hour. 
}
\fr{                                               "On the Psychology of the Unconscious" (1912). In CW 7: Two Essays on Analytical Psychology P. 5
}

\pa{          It is the duty of one who goes his own way to inform society of what he finds on his voyage of discovery, be it cooling water
          for the thirsty or the sandy wastes of unfruitful error. The one helps, the other warns. Not the criticism of individual
          contemporaries will decide the truth or falsity of his discoveries, but future generations. There are things that are not yet
          true today, perhaps we dare not find them true, but tomorrow they may be. So every man whose fate it is to go his individual
          way must proceed with hopefulness and watchfulness, ever conscious of his loneliness and its dangers. 
}
\fr{                                             "On the Psychology of the Unconscious" (1912). In CW 7: Two Essays on Analytical Psychology P. 201
}

\pa{          If, as happens in long and difficult treatments, the analyst observes a series of dreams often running into hundreds, there
          gradually forces itself upon him a phenomenon which, in an isolated dream, would remain hidden behind the compensation
          of the moment. This phenomenon is a kind of developmental process in the personality itself. At first it seems that each
          compensation is a momentary adjustment of one-sidedness or an equalization of disturbed balance. But with deeper insight
          and experience, these apparently separate acts of compensation arrange themselves into a kind of plan. They seem to hang
          together and in the deepest sense to be subordinated to a common goal, so that a long dream-series no longer appears as a
          senseless string of incoherent and isolated happenings, but resembles the successive steps in a planned and orderly process
          of development. I have called this unconscious process spontaneously expressing itself in the symbolism of a long
          dream-series the individuation process. 
}
\fr{                                                 "On the Nature of Dreams" (1945). In CW 8: The Structure and Dynamics of the Psyche. P.550 
}

\subsection{Psyche (Jung)}

\pa{          Anyone who wants to know the human psyche will learn next to nothing from experimental psychology. He would be better
          advised to put away his scholar's gown, bid farewell to his study, and wander with human heart through the world. There, in
          the horrors of prisons, lunatic asylums and hospitals, in drab suburban pubs, in brothels and gambling-hells, in the salons
          of the elegant, the Stock Exchanges, Socialist meetings, churches, revivalist gatherings and ecstatic sects, through love and
          hate, through the experience of passion in every form in his own body, he would reap richer stores of knowledge than
          text-books a foot thick could give him, and he will know how to doctor the sick with real knowledge of the human soul. 
} \fr{                                                         "New Paths in Psychology" In CW 7: Two Essays on Analytical Psychology. P.409
}
\pa{          The totality of the psyche can never be grasped by intellect alone. Whether we will it or not, philosophy keeps breaking
          through, because the psyche seeks an expression that will embrace its total nature. 
} \fr{                                                  "On the Psychology of the Unconscious" In CW 7: Two Essays on Analytical Psychology. P.201
}
\pa{          The psyche is a self-regulating system that maintains its equilibrium just as the body does. Every process that goes too far
          immediately and inevitably calls forth compensations, and without these there would be neither a normal metabolism nor a
          normal psyche. In this sense we can take the theory of compensation as a basic law of psychic behaviour. Too little on one
          side results in too much on the other. 
} \fr{                                                 "The Practical Use of Dream Analysis" (1934). In CW 16: The Practice of Psychotherapy. P.330
}
\pa{          "All that is outside, also is inside," we could say with Goethe. But this "inside," which modern rationalism is so eager to
          derive from "outside," has an a priori structure of its own that antedates all conscious experience. It is quite impossible to
          conceive how "experience" in the widest sense, or, for that matter, anything psychic, could originate exclusively in the
          outside world. The psyche is part of the inmost mystery of life, and it has its own peculiar structure and form like every other
          organism. Whether this psychic structure and its elements, the archetypes, ever "originated" at all is a metaphysical
          question and therefore unanswerable. The structure is something given, the precondition that is found to be present in every
          case. And this is the mother, the matrix-the form into which all experience is poured. 
} \fr{                       "Psychological Aspects of the Mother Archetype" (1939.1959) In CW 9, Part I: The Archetypes and the Collective Unconscious. 1959. pp 187
}
\pa{          The tragic thing is that psychology has no self-consistent mathematics at its disposal, but only a calculus of subjective
          prejudices. Also, it lacks the immense advantage of an Archimedean point such as physics enjoys. The latter observes the
          physical world from the psychic standpoint and can translate it into psychic terms. The psyche, on the other hand, observes
          itself and can only translate the psychic back into the psychic. 
} \fr{                                                "On the Nature of the Psyche" (1947). In CW 8: The Structure and Dynamics of the Psyche. P.421
}
\pa{          A wrong functioning of the psyche can do much to injure the body, just as conversely a bodily illness can affect the psyche;
          for psyche and body are not separate entities, but one and the same life. 
} \fr{                                                  Two Essays on Analytical Psychology. CW 7: "On the Psychology of the Unconscious" pp.194
}
\pa{          The psyche consists essentially of images. It is a series of images in the truest sense, not an accidental juxtaposition or
          sequence, but a structure that is throughout full of meaning and purpose; it is a "picturing" of vital activities. And just as
          the material of the body that is ready for life has need of the psyche in order to be capable of life, so the psyche presupposes
          the living body in order that its images may live. 
} \fr{                                                      "Spirit and Life" (1926) In CW 8: The Structure and Dynamics of the Psyche. pp.618
}
\pa{          There is no difference in principle between organic and psychic formations. As a plant produces its flowers, so the psyche
          creates its symbols. 
} \fr{                     "Approaching the Unconscious" In Man and His Symbols, ed. C.G. Jung. In CW 18: retitled "Symbols and the Interpretations of Dreams" pp. 64
}
\pa{          Despite the materialistic tendency to understand the psyche as a mere reflection or imprint of physical and chemical
          processes, there is not a single proof of this hypothesis. Quite the contrary, innumerable facts prove that the psyche
          translates physical processes into sequences of images which have hardly any recognizable connection with the objective
          process. The materialistic hypothesis is much too bold and flies in the face of experience with almost metaphysical
          presumption. The only thing that can be established with certainty, in the present state of our knowledge, is our ignorance
          of the nature of the psyche. There is thus no ground at all for regarding the psyche as something secondary or as an
          epiphenomenon; on the contrary, there is every reason to regard it, at least hypothetically, as a factor sui generis, and to go
          on doing so until it has been sufficiently proved that psychic processes can be fabricated in a retort. 
} \fr{           "Concerning the Archetypes, with Special Reference to the Anima Concept" (1936.1954)In CW 9, Part I: The Archetypes and the Collective Unconscious. pp. 117
}
\pa{          A psychology that treats the psyche as an epiphenomenon would better call itself brain-psychology, and remain satisfied with
          the meager results that such a psycho-physiology can yield. The psyche deserves to be taken as a phenomenon in its own
          right; there are no grounds at all for regarding it as a mere epiphenomenon, dependent though it may be on the functioning
          of the brain. One would be as little justified in regarding life as an epiphenomenon of the chemistry of carbon compounds. 
} \fr{                                                    "On Psychic Energy" (1928). In CW 8: The Structure and Dynamics of the Psyche. pp.10
}
\pa{          Restriction to material reality carves an exceedingly large chunk out of reality as a whole, but it nevertheless remains a
          fragment only, and all round it is a dark penumbra which one would have to call unreal or surreal. This narrow perspective
          is alien to the Eastern view of the world, which therefore has no need of any philosophical conception of super-reality. Our
          arbitrarily delimited reality is continually menaced by the "supersensual," the "supernatural," the "superhuman," and a
          whole lot more besides. Eastern reality includes all this as a matter of course. For us the zone of disturbance already begins
          with the concept of the "psychic." In our reality the psychic cannot be anything except an effect at third hand, produced
          originally by physical causes; a "secretion of the brain," or something equally savoury. At the same time, this appendage of
          the material world is credited with the power to pull itself up by its own bootstraps, so to speak; and not only to fathom the
          secrets of the physical world, but also, in the form of 'mind,' to know itself. All this, without its being granted anything more
          than an indirect reality. 
} \fr{                                                  "The Real and Surreal" (1933). In CW 8: The Structure and Dynamics of the Psyche pp.743
}
\pa{          Since psyche and matter are contained in one and the same world, and moreover are in continuous contact with one another
          and ultimately rest on irrepresentable, transcendental factors, it is not only possible but fairly probable, even, that psyche
          and matter are two different aspects of one and the same thing. 
} \fr{                                               "On the Nature of the Psyche" (1947). In CW 8: The Structure and Dynamics of the Psyche. P.418
}
\pa{          Every science is a function of the psyche, and all knowledge is rooted in it. The psyche is the greatest of all cosmic wonders. 
} \fr{                                               "On the Nature of the Psyche" (1947). In CW 8: The Structure and Dynamics of the Psyche. P.357
}
\pa{          It does not surprise me that psychology debouches into philosophy, for the thinking that underlies philosophy is after all a
          psychic activity which, as such, is the proper study of psychology. I always think of psychology as encompassing the whole of
          the psyche, and that includes philosophy and theology and many other things besides. For underlying all philosophies and
          all religions are the facts of the human soul, which may ultimately be the arbiters of truth and error. 
} \fr{                                           "General Aspects of Dream Psychology" (1916). In CW 8: The Structure and Dynamics of the Psyche. P. 5
}
\pa{          Every science is descriptive at the point where it can no longer proceed experimentally, without on that account ceasing to
          be scientific. But an experimental science makes itself impossible when it delimits its field of work in accordance with
          theoretical concepts. The psyche does not come to an end where some physiological assumption or other stops. In other
          words, in each individual case that we observe scientifically, we have to consider the manifestations of the psyche in their
          totality. 
} \fr{                "Concerning the Archetypes, with Special Reference to the Anima Concept" (1936). In CW 9, Part I: The Archetypes and the Collective Unconscious. P.113
}
\pa{There is no Archimedean point from which to judge, since the psyche is indistinguishable from its manifestations. The
          psyche is the object of psychology, and-fatally enough-also its subject. There is no getting away from this fact. 
} \fr{                                                   "Psychology and Religion" (1938). In CW 11: Psychology and Religion: West and East. P.8
}
\pa{          Far from being a material world, this is a psychic world, which allows us to make only indirect and hypothetical inferences
          about the real nature of matter. The psychic, alone has immediate reality, and this includes all forms of the psychic, even
          "unreal" ideas and thoughts which refer to nothing "external." We may call them "imagination" or "delusion," but that does
          not detract in any way from their effectiveness. Indeed, there is no "real" thought that cannot, at times, be thrust aside by
          an "unreal" one, thus proving that the latter is stronger and more effective than the former. Greater than all physical
          dangers are the tremendous effects of delusional ideas, which are yet denied all reality by our world-blinded consciousness.
          Our much vaunted reason and our boundlessly overestimated will are sometimes utterly powerless in the face of "unreal"
          thoughts. The world powers that rule over all mankind, for good or ill, are unconscious psychic factors, and it is they that
          bring consciousness into being and hence create the sine qua non for the existence of any world at all. We are steeped in a
          world that was created by our own psyche. 
} \fr{                                                 "The Real and the Surreal" (1933). In CW 8: The Structure and Dynamics of the Psyche. P.747
}
\pa{          Since we do not know everything, practically every experience, fact, or object contains something unknown. Hence, if we
          speak of the totality of an experience, the word "totality" can refer only to the conscious part of it. As we cannot assume that
          our experience covers the totality of the object, it is clear that its absolute totality must necessarily contain the part that
          has not been experienced. The same holds true, as I have mentioned, of every experience and also of the psyche, whose
          absolute totality covers a greater area than consciousness. In other words, the psyche is no exception to the general rule
          that the universe can be established only so far as our psychic organism permits. 
} \fr{                                                   "Psychology and Religion" (1938). In CW 11: Psychology and Religion: West and East. P.68
}
\pa{          Not only in the psychic man is there something unknown, but also in the physical. We should be able to include this
          unknown quantity in a total picture of man, but we cannot. Man himself is partly empirical, partly transcendental ... Also,
          we do not know whether what we on the empirical plane regard as physical may not, in the Unknown beyond our experience,
          be identical with what on this side of the border we distinguish from the physical as psychic. 
} \fr{                                                                           Mysterium Coniunctionis (1955). CW 14: P.765
}
\pa{          All that is not encompassed by our knowledge, so that we are not in a position to make any statements about its total
          nature. Microphysics is feeling its way into the unknown side of matter, just as complex psychology is pushing forward into
          the unknown side of the psyche. Both lines of investigation have yielded findings which can be conceived only by means of
          antinomies, and both have developed concepts which display remarkable analogies. If this trend should become more
          pronounced in the future, the hypothesis of the unity of their subject matters would gain in probability. Of course there is
          little or no hope that the unitary Being can ever be conceived, since our powers of thought and language permit only of
          antinomian statements. But this much we do know beyond all doubt, that empirical reality has a transcendental
          background. 
} \fr{                                                                           Mysterium Coniunctionis (1955). CW 14: P.768
}
\pa{          It is a remarkable fact, which we come across again and again, that absolutely everybody, even the most unqualified layman,
          thinks he knows all about psychology as though the psyche were something that enjoyed the most universal understanding.
          But anybody who really knows the human psyche will agree with me when I say that it is one of the darkest and most
          mysterious regions of our experience. There is no end to what can be learned in this field. 
} \fr{                                                                            Psychology and Alchemy (1944). CW 12: P.2
}
\pa{          The psyche creates reality every day. The only expression I can use for this activity is fantasy. Fantasy is just as much
          feeling as thinking, as much intuition as sensation., There is no psychic function that, through fantasy, is not inextricably
          bound up with the other psychic functions. Sometimes it appears in primordial form, sometimes it is the ultimate and
          boldest product of all our faculties combined. Fantasy, therefore, seems to me the clearest expression of the specific activity
          of the psyche. It is, pre-eminently, the creative activity from which the answers to all answerable questions come; it is the
          mother of all possibilities, where, like all psychological opposites, the inner and outer worlds are joined together in living
          union. 
} \fr{                                                                               Psychological Types (1921). CW 6: P.78
}
\pa{          What we call fantasy is simply spontaneous psychic activity, and it wells up wherever the inhibitive action of the conscious
          mind abates or, as in sleep, ceases altogether. 
} \fr{                                                 "Problems of Modern Psychotherapy" (1929). In CW 16: The Practice of Psychotherapy. P.125
}
\pa{          Psychic existence is the only category of existence of which we have immediate knowledge, since nothing can be known
          unless it first appears as a psychic image. Only psychic existence is immediately verifiable. To the extent that the world does
          not assume the form of a psychic image, it is virtually nonexistent. 
} \fr{                 The Tibetan Book of the Great Liberation (1927). Psychological Commentary by C.G.Jung. In CW 11: Psychology and Religion: West and East. P.769
}
\pa{          What is "illusion"? By what criterion do we judge something to be an illusion? Does anything exist for the psyche that we
          are entitled to call illusion? What we are pleased to call illusion may be for the psyche an extremely important life-factor,
          something as indispensable as oxygen for the body-a psychic actuality of overwhelming significance. Presumably the psyche
          does not trouble itself about our categories of reality; for it, everything that works is real. The investigator of the psyche
          must not confuse it with his consciousness, else he veils from his sight the object of his investigation. On the contrary, to
          recognize it at all, he must learn to see how different it is from consciousness. Nothing is more probable than that what we
          call illusion is very real for the psyche-for which reason we cannot take psychic reality to be commensurable with conscious
          reality. 
} \fr{                                                       "The Aims of Psychotherapy" (1931) In CW 16: The Practice of Psychotherapy. P.III
}
\pa{          So far mythologists have always helped themselves out with solar, lunar, meteorological, vegetal, and other ideas of the kind.
          The fact that myths are first and foremost psychic phenomena that reveal the nature of the soul is something they have
          absolutely refused to see until now. Primitive man is not much interested in objective explanations of the obvious, but he
          has an imperative need or rather, his unconscious psyche has an irresistible urge-to assimilate all outer sense experiences to
          inner, psychic events. It is not enough for the primitive to see the sun rise and set; this external observation must at the
          same time be a psychic happening: the sun in its course must represent the fate of a god or hero who, in the last analysis,
          dwells nowhere except in the soul of man. All the mythologized processes of nature, such as summer and winter, the phases
          of the moon, the rainy seasons, and so forth, are in no sense allegories of these objective occurrences; rather they are
          symbolic expressions of the inner, unconscious drama of the psyche which becomes accessible to man's consciousness by
          way of projection-that is, mirrored in the events of nature. 
} \fr{                                  "Archetypes of the Collective Unconscious" (1935). In CW 9, Part I: The Archetypes and the Collective Unconscious. P.7 
}


\subsection{Consciousness (Jung)}

\pa{                    Man's capacity for consciousness alone makes him man. 
} \fr{                                                "On the Nature of the Psyche" (1947). In CW 8: The Structure and Dynamics of the Psyche. P.412
}
\pa{          Without consciousness there would, practically speaking, be no world, for the world exists for us only in so far as it is
          consciously reflected by a psyche. Consciousness is a precondition of being. Thus the psyche is endowed with the dignity of a
          cosmic principle, which philosophically and in fact gives it a position co-equal with the principle of physical being. The
          carrier of this consciousness is the individual, who does not produce the psyche of his own volition but is, on the contrary,
          preformed by it and nourished by the gradual awakening of consciousness during childhood. If therefore the psyche is of
          overriding empirical importance, so also is the individual, who is the only immediate manifestation of the psyche. 
} \fr{                                                            "The Undiscovered Self" (1957). In CW 10: Civilization in Transition. P. 528
}
\pa{          In the same way that the State has caught the individual, the individual imagines that he has caught the psyche and holds
          her in the hollow of his hand. He is even making a science of her in the absurd supposition that the intellect, which is but a
          part and a function of the psyche, is sufficient to comprehend the much greater whole. In reality the psyche is the mother
          and the maker, the subject and even the possibility of consciousness itself. It reaches so far beyond the boundaries of
          consciousness that the latter could easily be compared to an island in the ocean. Whereas the island is small and narrow,
          the ocean is immensely wide and deep and contains a life infinitely surpassing, in kind and degree, anything known on the
          island-so that if it is a question of space, it does not matter whether the gods are "inside" or "outside." It might be objected
          that there is no proof that consciousness is nothing more than an island in the ocean. Certainly it is impossible to prove
          this, since the known range of consciousness is confronted with the unknown extension of the unconscious, of which we
          only know that it exists and by the very fact of its existence exerts a limiting effect on consciousness and its freedom. 
} \fr{                                                  "Psychology and Religion" (1938). In CW 11: Psychology and Religion: West and East. P. 141
}
\pa{          Our consciousness does not create itself-it wells up from unknown depths. In childhood it awakens gradually, and all
          through life it wakes each morning out of the depths of sleep from an unconscious condition. 
} \fr{                                           "The Psychology of Eastern Meditation" (1943). In CW 11: Psychology and Religion: West and East. P.935
}
\pa{          Just as conscious contents can vanish into the unconscious, other contents can also arise from it. Besides a majority of
          mere recollections, really new thoughts and creative ideas can appear which have never been conscious before. They grow up
          from the dark depths like a lotus. 
} \fr{                                                         "Approaching the Unconscious" In Man and His Symbols (1964), In CW 18: P.37
}
\pa{          If one reflects upon what consciousness really is, one is deeply impressed by the extremely wonderful fact that an event
          which occurs outside in the cosmos produces simultaneously an inner image. Thus it also occurs within; in other words, it
          becomes conscious. 
} \fr{                                                                                     From the Basel Seminar (1934)
}
\pa{          It is just man's turning away from instinct-his opposing himself to instinct-that creates consciousness. Instinct is nature
          and seeks to perpetuate nature, whereas consciousness can only seek culture or its denial. Even when we turn back to
          nature, inspired by a Rousseauesque longing, we "cultivate" nature. As long as we are still submerged in nature we are
          unconscious, and we live in the security of instinct which knows no problems. Everything in us that still belongs to nature
          shrinks away from a problem, for its name is doubt, and wherever doubt holds sway there is uncertainty and the possibility
          of divergent ways. And where several ways seem possible, there we have turned away from the certain guidance of instinct
          and are handed over to fear. For consciousness is now called upon to do that which nature has always done for her children
          namely, to give a certain, unquestionable, and unequivocal decision. And here we are beset by an all-too-human fear that
          consciousness-our Promethean conquest-may in the end not be able to serve us as well as nature. 
} \fr{                                                     "The Stages of Life" (1930). In CW 8: The Structure and Dynamics of the Psyche. P. 750
}
\pa{          Every advance, every conceptual achievement of mankind, has been connected with an advance in self awareness: man
          differentiated himself from the object and faced Nature as something distinct from her. Any reorientation of psychological
          attitude will have to follow the same road. 
} \fr{                                          "General Aspects of Dream Psychology" (1916). In CW 8: The Structure and Dynamics of the Psyche. P. 523
}
\pa{          Whatever name we may put to the psychic background, the fact remains that our consciousness is influenced by it in the
          highest degree, and all the more so the less we are conscious of it. The layman can hardly conceive how much his
          inclinations, moods, and decisions are influenced by the dark forces of his psyche, and how dangerous or helpful they may be
          in shaping his destiny. Our cerebral consciousness is like an actor who has forgotten that he is playing a role. But when the
          play comes to an end, he must remember his own subjective reality, for he can no longer continue to live as Julius Caesar or
          as Othello, but only as himself, from whom he has become estranged by a momentary sleight of consciousness. He must
          know once again that he was merely a figure on the stage who was playing a piece by Shakespeare, and that there was a
          producer as well as a director in the background who, as always, will have something very important to say about his acting. 
} \fr{                       "Zur Umerziehung des deutschen Volkes" (On the Re-education of the Germans). In Basler Nachrichten, Nr. 486, 16 November 1946. P.332 
}
\pa{          It suits our hypertrophied and hubristic modern consciousness not to be mindful of the dangerous autonomy of the
          unconscious and to treat it negatively as an absence of consciousness. The hypothesis of invisible gods or daemons would
          be, psychologically, a far more appropriate formulation, even though it would be an anthropomorphic projection. But since
          the development of consciousness requires the withdrawal of all the projections we can lay our hands on, it is not possible
          to maintain any non-psychological doctrine about the gods. If the historical process of world despiritualization continues as
          hitherto, then everything of a divine or daemonic character outside us must return to the psyche, to the inside of the
          unknown man, whence it apparently originated. 
} \fr{                                                   "Psychology and Religion" (1938). In CW 11: Psychology and Religion: West and East. P 141
}
\pa{          The reason why consciousness exists, and why there is an urge to widen and deepen it, is very simple: without consciousness
          things go less well. This is obviously the reason why Mother Nature deigned to produce consciousness, that most remarkable
          of all nature's curiosities. Even the well-nigh unconscious primitive can adapt and assert himself, but only in his primitive
          world, and that is why under other conditions he falls victim to countless dangers which we on a higher level of
          consciousness can avoid without effort. True, a higher consciousness is exposed to dangers dreamt of by the primitive, but
          the fact remains that the conscious man has conquered the earth and not the unconscious one. Whether in the last
          analysis, and from a superhuman point of view, this is an advantage or a calamity we are not in a position to decide. 
} \fr{                                       "Analytical Psychology and Weltanshauung" (1928). In CW 8: The Structure and Dynamics of the Psyche. P. 695 
}
\pa{          And yet the attainment of consciousness was the most precious fruit of the tree of knowledge, the magical weapon which
          gave man victory over the earth, and which we hope will give him a still greater victory over himself. 
} \fr{                                                "The Meaning of Psychology for Modern Man" (1933). In CW 10: Civilization in Transition. P. 289
}
\pa{          "Reflection" should be understood not simply as an act of thought, but rather as an attitude. It is a privilege born of human
          freedom in contradistinction to the compulsion of natural law. As the word itself testifies ("reflection" means literally
          "bending back"), reflection is a spiritual act that runs counter to the natural process; an act whereby we stop, call something
          to mind, form a picture, and take up a relation to and come to terms with what we have seen. It should, therefore, be
          understood as an act of becoming conscious. 
} \fr{                                  "A Psychological Approach to the Dogma of the Trinity" (1942). In CW 11: Psychology and Religion: West and East. P. 235
}
\pa{          There is no other way open to us; we are forced to resort to conscious decisions and solutions where formerly we trusted
          ourselves to natural happenings. Every problem, therefore, brings the possibility of a widening of consciousness, but also the
          necessity of saying goodbye to childlike unconsciousness and trust in nature. This necessity is a psychic fact of such
          importance that it constitutes one of the most essential symbolic teachings of the Christian religion. It is the sacrifice of the
          merely natural man, of the unconscious, ingenuous being whose tragic career began with the eating of the apple in Paradise.
          The biblical fall of man presents the dawn of consciousness as a curse. And as a matter of fact it is in this light that we first
          look upon every problem that forces us to greater consciousness and separates us even further from the paradise of
          unconscious childhood. 
} \fr{                                                     "The Stages of Life" (1930). In CW 8: The Structure and Dynamics of the Psyche. P. 751
}
\pa{          There are many people who are only partially conscious. Even among absolutely civilized Europeans there is a
          disproportionately high number of abnormally unconscious individuals who spend a great part of their lives in an
          unconscious state. They know what happens to them, but they do not know what they do or say. They cannot judge of the
          consequences of their actions. These are people who are abnormally unconscious, that is, in a primitive state. What then
          finally makes them conscious? If they get a slap in the face, then they become conscious; something really happens, and that
          makes them conscious. They meet with something fatal and then they suddenly realize what they are doing. 
} \fr{                                                                                    From the "Basel Seminar" (1934)
}
\pa{The stirring up of conflict is a Luciferian virtue in the true sense of
the word. Conflict engenders fire, the fire of affects and
emotions, and like every other fire it has two aspects, that of combustion 
and that of creating light. On the one hand,
emotion is the alchemical fire whose warmth brings everything into existence
and whose heat burns all superfluities to ashes
(omnes superfluitates comburit). But on the other hand, emotion is the moment
when steel meets flint and a spark is struck
forth, for emotion is the chief source of consciousness. There is no change
from darkness to light or from inertia to
movement without emotion. 
} \fr{"Psychological Aspects of the Mother Archetype" (1939). In CW 9, Part
I: The Archetypes and the Collective Unconscious. P. 179
}
\pa{          The man we call modern, the man who is aware of the immediate present, is by no means the average man. He is rather the
          man who stands upon a peak, or at the very edge of the world, the abyss of the future before him, above him the heavens,
          and below him the whole of mankind with a history that disappears in primeval mists. The modern man-or, let us say again,
          the man of the immediate present-is rarely met with, for he must be conscious to a superlative degree. Since to be wholly of
          the present means to be fully conscious of one's existence as a man, it requires the most intensive and extensive
          consciousness, with a minimum of unconsciousness. It must be clearly understood that the mere fact of living in the present
          does not make a man modern, for in that case everyone at present alive would be so. He alone is modern who is fully
          conscious of the present. 
} \fr{                                                    "The Spiritual Problem of Modern Man" (1928) In CW 10: Civilization in Transition. P. 149
}
\pa{          Every one of us gladly turns away from his problems; if possible, they must not be mentioned, or, better still, their existence
          is denied. We wish to make our lives simple, certain, and smooth, and for that reason problems are taboo. We want to have
          certainties and no doubts-results and no experiments-without even seeing that certainties can arise only through doubt and
          results only through experiment. The artful denial of a problem will not produce conviction: on the contrary, a wider and
          higher consciousness is required to give us the certainty and clarity we need. 
} \fr{                                                     "The Stages of Life" (1930). In CW 8: The Structure and Dynamics of the Psyche. P. 751
}
\pa{          It seems to be very hard for people to live with riddles or to let them live, although one would think that life is so full of
          riddles as it is that a few more things we cannot answer would make no difference. But perhaps it is just this that is so
          unendurable, that there are irrational things in our own psyche which upset the conscious mind in its illusory certainties by
          confronting it with the riddle of its existence. 
} \fr{                                                              "The Philosophical Tree" (1945). In CW 13: Alchemical Studies. P. 307
}
\pa{          Everyone who becomes conscious of even a fraction of his unconscious gets outside his own time and social stratum into a
          kind of solitude. 
} \fr{                                                                            Mysterium Coniunctionis (1955) CW 14: P 258
}
\pa{          Genesis represents the act of becoming conscious as a taboo infringement, as though knowledge meant that a sacrosanct
          barrier had been impiously overstepped. I think that Genesis is right in so far as every step towards greater consciousness is
          a kind of Promethean guilt: through knowledge, the gods are as it were robbed of their fire, that is, something that was the
          property of the unconscious powers is torn out of its natural context and subordinated to the whims of the conscious mind.
          The man who has usurped the new knowledge suffers, however, a transformation or enlargement of consciousness, which no
          longer resembles that of his fellow men. He has raised himself above the human level of his age ("ye shall become like unto
          God"), but in so doing has alienated himself from humanity. The pain of this loneliness is the vengeance of the gods, for
          never again can he return to mankind. He is, as the myth says, chained to the lonely cliffs of the Caucasus, forsaken of God
          and man. 
} \fr{                                       "The Relations Between the Ego and the Unconscious" (1953) CW 7: Two Essays on Analytical Psychology. P. 243
}

\subsection{The Ego (Jung)}
\pa{          Nowhere are we closer to the sublime secret of all origination than in the recognition of our own selves, whom we always
          think we know already. Yet we know the immensities of space better than we know our own depths, where -even though we
          do not understand it-we can listen directly to the throb of creation itself. 
} \fr{                                       "Analytical Psychology and Weltanshauung" (1928). In CW 8: The Structure and Dynamics of the Psyche. P. 737
}
\pa{          An inflated consciousness is always egocentric and conscious of nothing but its own existence. It is incapable of learning
          from the past, incapable of understanding contemporary events, and incapable of drawing right conclusions about the
          future. It is hypnotized by itself and therefore cannot be argued with. It inevitably dooms itself to calamities that must strike
          it dead. 
} \fr{                                                                          Psychology and Alchemy (1944). In CW 12. P. 563
}
\pa{                                                                  You always become the thing you fight the most. 
} \fr{                                                     "Diagnosing the Dictators." In Hearst's International Cosmopolitan, January 1939 pp.22
}
\pa{          Only a life lived in a certain spirit is worth living. It is a remarkable fact that a life lived entirely from the ego is dull not only
          for the person himself but for all concerned. 
} \fr{                                                       "Spirit and Life" (1926). In CW 8: The Structure and Dynamics of the Psyche. P. 645 
}
\pa{          "But why on earth," you may ask, "should it be necessary for man to achieve, by hook or by crook, a higher level of
          consciousness?" This is truly the crucial question, and I do not find the answer easy. Instead of a real answer I can only
          make a confession of faith: 1 believe that, after thousands and millions of years, someone had to realize that this wonderful
          world of mountains and oceans, suns and moons, galaxies and nebulae, plants and animals, exists. From a low hill in the
          Athi plains of East Africa I once watched the vast herds of wild animals grazing in soundless stillness, as they had done from
          time immemorial, touched only by the breath of a primeval world. I felt then as if I were the first man, the first creature, to
          know that all this is. The entire world round me was still in its primeval state; it did not know that it was. And then, in that
          one moment in which I came to know, the world sprang into being; without that moment it would never have been. All
          Nature seeks this goal and finds it fulfilled in man, but only in the most highly developed and most fully conscious man. 
} \fr{                                    "Psychological Aspects of the Mother Archetype" (1939). In CW 8: The Structure and Dynamics of the Psyche. P. 177 
}
\pa{          The so-called "forces of the unconscious" are not intellectual concepts that can be arbitrarily manipulated, but dangerous
          antagonists which can, among other things, work frightful devastation in the economy of the personality. They are
          everything one could wish for or fear in a psychic "Thou." The layman naturally thinks he is the victim of some obscure
          organic disease; but the theologian, who suspects it is the devil's work, is appreciably nearer to the psychological truth. 
} \fr{                                       "Religion and Philosophy: A Reply to Martin Buber" (1952). In Jung, Gesammelte Werke, II: and in CW 18. P.659
}
\pa{          When the ego has been made a "seat of anxiety," someone is running away from himself and will not admit it. 
} \fr{                                            "The State of Psychotherapy Today" (1934). In CW 8: The Structure and Dynamics of the Psyche. P.360
}
\pa{          So far as we know, consciousness is always ego-consciousness. In order to be conscious of myself, I must be able to
          distinguish myself from others. Relationship can only take place where this distinction exists. 
} \fr{                                            "Marriage as a Psychological Relationship" (1925). In CW 17: The Development of the Personality. P.326
}
\pa{          The ego lives in space and time and must adapt itself to their laws if it is to exist at all. If it is absorbed by the unconscious
          to such an extent that the latter alone has the power of decision, then the ego is stifled, and there is no longer any medium
          in which the unconscious could be integrated and in which the work of realization could take place. The separation of the
          empirical ego from the "eternal" and universal man is therefore of vital importance, particularly today, when
          mass-degeneration of the personality is making such threatening strides. Mass-degeneration does not come only from
          without: it also comes from within, from the collective unconscious. Against the outside, some protection was afforded by
          the droits de I'homme which at present are lost to the greater part of Europe, and even where they are not actually lost we
          see political parties, as naive as they are powerful, doing their best to abolish them in favour of the slave state, with the bait
          of social security. Against the demonism from within, the Church offers some protection so long as it wields authority. But
          protection and security are only valuable when not excessively cramping to our existence; and in the same way the
          superiority of consciousness is desirable only if it does not suppress and shut out too much life. As always, life is a voyage
          between Scylla and Charybdis. 
} \fr{                                                    "The Psychology of Transference" (1946). In CW 16: The Practice of Psychotherapy. P.502
}
\pa{          The office I hold is certainly my special activity; but it is also a collective factor that has come into existence historically
          through the cooperation of many people and whose dignity rests solely on collective approval. When, therefore, I identify
          myself with my office or title, I behave as though I myself were the whole complex of social factors of which that office
          consists, or as though I were not only the bearer of the office, but also and at the same time the approval of society. I have
          made an extraordinary extension of myself and have usurped qualities which are not in me but outside Me. 
} \fr{                                     "The Relations Between the Ego and the Unconscious" (1953) In CW 7: Two Essays on Analytical Psychology. P.227 
}
\pa{          Although biological instinctive processes contribute to the formation of personality, individuality is nevertheless essentially
          different from collective instincts; indeed, it stands in the most direct opposition to them, just as the individual as a
          personality is always distinct from the collective. His essence consists precisely in this distinction. Every ego-psychology
          must necessarily exclude and ignore just the collective element that is bound to a psychology of instinct, since it describes
          that very process by which the ego becomes differentiated from collective drives. 
} \fr{                                                                               Psychological Types (1921). CW 6. P.88 
}
\pa{          The truth is that we do not enjoy masterless freedom; we are continually threatened by psychic factors which, in the guise of
          "natural phenomena," may take possession of us at any moment. The withdrawal of metaphysical projections leaves us
          almost defenseless in the face of this happening, for we immediately identify with every impulse instead of giving it the name
          of the "other," which would at least hold it at arm's length and prevent it from storming the citadel of the ego. 
} \fr{                                                  "Psychology and Religion" (1938). In CW 11: Psychology and Religion: West and East. P. 143
}
\pa{          If man were merely a creature that came into being as a result of something already existing unconsciously, he would have
          no freedom and there would be no point in consciousness. Psychology must reckon with the fact that despite the causal
          nexus man does enjoy a feeling of freedom, which is identical with autonomy of consciousness. However much the ego can
          be proved to be dependent and preconditioned, it cannot be convinced that it has no freedom. An absolutely preformed
          consciousness and a totally dependent ego would be a pointless farce, since everything would proceed just as well or even
          better unconsciously. The existence of ego consciousness has meaning only if it is free and autonomous. By stating these
          facts we have, it is true, established an antinomy, but we have at the same time given a picture of things as they are. There
          are temporal, local, and individual differences in the degree of dependence and freedom. In reality both are always present:
          the supremacy of the self and the hubris of consciousness. 
} \fr{                         "Transformation Symbolism in the Mass" (1942) Eranos Jahrbuch 1940/1941. In CW 11: Psychology and Religion: West and East. P.391
}
\pa{          Hysterical self-deceivers, and ordinary ones too, have at all times understood the art of misusing everything so as to avoid
          the demands and duties of life, and above all to shirk the duty of confronting themselves. They pretend to be seekers after
          God in order not to have to face the truth that they are ordinary egoists. 
} \fr{                                                              "The Visions of Zosimos" (1938). In CW 13: Alchemical Studies. P.142 
}
\pa{          Instead of waging war on himself it is surely better for a man to learn to tolerate himself, and to convert his inner difficulties
          into real experiences instead of expending them in useless fantasies. Then at least he lives, and does not waste his life in
          fruitless struggles. If people can be educated to see the lowly side of their own natures, it may be hoped that they will also
          learn to understand and to love their fellow men better. A little less hypocrisy and a little more tolerance towards oneself can
          only have good results in respect for our neighbour; for we are all too prone to transfer to our fellows the injustice and
          violence we inflict upon our own natures. 
} \fr{                                   Appendix I: "New Paths in Psychology" (1912) Variant Readings. In CW 7: Two Essays on Analytical Psychology. P. 439
}
\pa{          We always start with the naive assumption that we are masters in our own house. Hence we must first accustom ourselves
          to the thought that, in our most intimate psychic life as well, we live in a kind of house which has doors and windows to the
          world, but that, although the objects or contents of this world act upon us, they do not belong to us. For many people this
          hypothesis is by no means easy to conceive, just as they do not find it at all easy to understand and to accept the fact that
          their neighbour's psychology is not necessarily identical with their own. 
} \fr{                                     "The Relations Between the Ego and the Unconscious" (1953) In CW 7: Two Essays on Analytical Psychology. P.329
}
\pa{          The essential thing is that we should be able to stand up to our judgment of ourselves. From outside this attitude looks like
          self-righteousness, but it is so only if we are incapable of criticizing ourselves. If we can exercise self criticism, criticism from
          outside will affect us only on the outside and not pierce to the heart, for we feel that we have a sterner critic within us than
          any who could judge us from without. And anyway, there are as many opinions as there are heads to think them. We come
          to realize that our own judgment has as much value as the judgment of others. One cannot please everybody, therefore it is
          better to be at peace with oneself. 
} \fr{                                                 "The Swiss Line in the European Spectrum" (1928). In CW 10: Civilization in Transition. P.911
}
\pa{          We say that it is egoistic or "morbid" to be preoccupied with oneself; one's own company is the worst, "it makes you
          melancholy"-such are the glowing testimonials accorded to our human make-up. They are evidently deeply ingrained in our
          Western minds. Whoever thinks in this way has obviously never asked himself what possible pleasure other people could find
          in the company of such a miserable coward. 
} \fr{                                     "The Relations Between the Ego and the Unconscious" (1953) In CW 7: Two Essays on Analytical Psychology. P.323 
}
\pa{          The foremost of all illusions is that anything can ever satisfy anybody. That illusion stands behind all that is unendurable in
          life and in front of all progress, and it is one of the most difficult things to overcome. 
} \fr{                                An Introduction to Zen Buddhism (1949). Foreword by C.G. Jung. In CW 11: Psychology and Religion: West and East. P.905
}
\pa{          We all have a great need to be good ourselves, and occasionally we like to show it by the appropriate actions. If good can
          come of evil self-interest, then the two sides of human nature have co-operated. But when in a fit of enthusiasm we begin
          with the good, our deep-rooted selfishness remains in the background, unsatisfied and resentful, only waiting for an
          opportunity to take its revenge in the most atrocious way. 
} \fr{                                                                  "Return to the Simple Life" In DU I:3 (May 1941) In CW 18: P. 56
}
\pa{          "Love thy neighbors is wonderful, since we then have nothing to do about ourselves; but when it is a question of "love thy
          neighbour as thyself" we are no longer so sure, for we think it would be egoism to love ourselves. There was no need to
          preach "love thyself" to people in olden times, because they did so as a matter of course. But how is it nowadays? It would do
          us good to take this thing somewhat to heart, especially the phrase "as thyself." How can I love my neighbour if I do not love
          myself? How can we be altruistic if we do not treat ourselves decently? But if we treat ourselves decently, if we love ourselves,
          we make discoveries, and then we see what we are and what we should love. There is nothing for it but to put our foot into
          the serpent's mouth. He who cannot love can never transform the serpent, and then nothing is changed. 
} \fr{                                                                                    From the Basel Seminar (1934).
}


\subsection{Knowledge (Jung)}
\pa{Knowledge rests not upon truth alone, but on error also. 
}
\fr{                                                          "Freud and Jung: Contrasts" (1929) In CW 4: Freud and Psychoanalysis. P.774
}
\pa{          Mistakes are, after all, the foundations of truth, and if a man does not know what a thing is, it is at least an increase in
          knowledge if he knows what it is not. 
}
\fr{                                                                                   Aion (1951). CW 9, Part II: P. 429
}
\pa{          Widely accepted ideas are never the personal property of their so-called author; on the contrary, he is the bondservant of his
          ideas. Impressive ideas which are hailed as truths have something peculiar about them. Although they come into being at a
          definite time, they are and have always been timeless; they arise from that realm of creative psychic life out of which the
          ephemeral mind of the single human being grows like a plant that blossoms, bears fruit and seed, and then withers and dies.
          Ideas spring from something greater than the personal human being. Man does not make his ideas; we could say that man's
          ideas make him. 
}\fr{                                                          "Freud and Jung: Contrasts" (1929) In CW 4: Freud and Psychoanalysis. P.769
}
\pa{          It would be a ridiculous and unwarranted assumption on our part if we imagined that we were more energetic or more
          intelligent than the men of the past. Our material knowledge has increased, but not our intelligence. This means that we are
          just as bigoted in regard to new ideas, and just as impervious to them, as people were in the darkest days of antiquity. We
          have become rich in knowledge, but or in wisdom. 
}\fr{                                                                            Symbols of Transformation (1952). CW 5: P. 23
}
\pa{          I believe only what I know. Everything else is hypothesis and beyond that I can leave a lot of things to the Unknown. They do
          not bother me. But they would begin to bother me, I am sure, if I felt that I ought to know about them. 
}\fr{                                                    "Psychology and Religion" 
                                                    (1938). In CW 11: Psychology and Religion: West and East P.79
}
\pa{          We see colors but not wave-lengths. This well-known fact must nowhere be taken to heart more seriously than in psychology.
          The effect of the personal equation begins already in the act of observation. One sees what one can best see oneself. Thus,
          first and foremost, one sees the mote in one's brother's eye. No doubt the mote is there, but the beam sits in one's own-and
          may considerably hamper the act of seeing. I mistrust the principle of "pure observation" in so-called objective psychology
          unless one confines oneself to the eyepieces of chronoscopes and tachistoscopes and suchlike "psychological" apparatus.
          With such methods one also guards against too embarrassing a yield of empirical psychological facts. But the personal
          equation asserts itself even more in the presentation and communication of one's own observations, to say nothing of the
          interpretation and abstract exposition of the empirical material. Nowhere is the basic requirement so indispensable as in
          psychology that the observer should be adequate to his object, in the sense of being able to see not only subjectively but also
          objectively. The demand that he should see only objectively is quite out of the question, for it is impossible. We must be
          satisfied if he does not see too subjectively 
}\fr{                                                                                Psychological Types (1921). CW 6: P.91
}
\pa{          One can, it is true, understand many things with the heart, but then the head often finds it difficult to follow up with an
          intellectual formulation that gives suitable expression to what has been understood. There is also an understanding with the
          head, particularly of the scientific kind, where there is sometimes too little room for the heart. 
}\fr{                                           "The Psychology of Eastern Meditation" (1943). In CW 11: Psychology and Religion: West and East. P.934
}
\pa{          One of the greatest obstacles to psychological understanding is the inquisitive desire to know whether the psychological
          factor adduced is "true" or "correct." If the description of it is not erroneous or false, then the factor is valid in itself and
          proves its validity by its very existence. One might just as well ask if the duck-billed platypus is a "true" or "correct"
          invention of the Creator's will. 
}\fr{                                                "The Transcendent Function" (1916). In CW 8: The Structure and Dynamics of the Psyche. P.192
}
\pa{          Never do human beings speculate more, or have more opinions, than about things which they do not understand. 
}\fr{                                                                            Mysterium Coniuntionis (1955). CW 14: P.737
}
\pa{          Doubt alone is the mother of scientific truth. Whoever fights against dogma in high places falls victim, tragically enough, to
          the tyranny of a partial truth. 
}\fr{                                                  "In Memory of Sigmund Freud" (1939) In CW 15: The Spirit in Man, Art and Literature. P. 70
}
\pa{          Again, no psychological fact can ever be exhaustively explained in terms of causality alone; as a living phenomenon, it is
          always indissolubly bound up with the continuity of the vital process, so that it is not only something evolved but also
          continually evolving and creative. Anything psychic is Janus-faced: it looks both backwards and forwards. Because it is
          evolving, it is also preparing the future. Were this not so, intentions, aims, plans, calculations, predictions, and
          premonitions would be psychological impossibilities. 
}\fr{                                                                               Psychological Types (1921). CW 6: P.717
}
\pa{          Rational truths are not the last word, there are also irrational ones. In human affairs, what appears impossible by way of
          the intellect has often become true by way of the irrational. Indeed, all the greatest changes that have ever affected mankind
          have come not by way of intellectual calculation, but by ways which contemporary minds either ignored or rejected as
          absurd, and which were recognized only long afterwards because of their intrinsic necessity. More often than not they are
          never recognized at all, for the all-important laws of mental development are still a book with seven seals. 
}\fr{                                                                               Psychological Types (1921). CW 6: P.135
}
\pa{          We do not devalue statements that originally were intended to be metaphysical when we demonstrate their psychic nature;
          on the contrary, we confirm their factualcharacter. But, by treating them as psychic phenomena, we remove them from the
          inaccessible realm of metaphysics, about which nothing verifiable can be said, and this disposes of the impossible question
          as to whether thev are "true" or not. We take our stand simply and solely on the facts, recognizing that the archetypal
          structure of the unconscious will produce, over and over again and irrespective of tradition, those figures which reappear in
          the history of all epochs and all peoples, and will endow them with the same significance and numinosity that have been
          theirs from the beginning. 
}\fr{                                                                            Mysterium Coniuntionis (1955). CW 14: P.558
}
\pa{                                              All the true things must change and only that which changes remains true. 
}\fr{                                                                            Mysterium Coniuntionis (1955). CW 14: P.503 
}
\pa{          To speak of the morning and spring, of the evening and the autumn of life is not mere sentimental jargon. We thus give
          expression to psychological truths, and even more to physiological facts. 
}\fr{                                                     "The Stages of Life" (1930). In CW 8: The Structure and Dynamics of the Psyche. P. 780
}
\pa{          The high ideal of educating the personality is not for children: for what is usually meant by personality - a wellrounded
          psychic whole that is capable of resistance and abounding in energy-is an adult ideal. It is only in an age like ours, when the
          individual is unconscious of the problems of adult life, or-what is worse - when he consciously shirks them, that people
          could wish to foist this ideal on to childhood. 
}\fr{                                               "The Development of the Personality" (1934). In CW 17: The Development of the Personality. P.286
}
\pa{          If there is anything that we wish to change in our children, we should first examine it and see whether it is not something
          that could better be changed in ourselves. Take our enthusiasm for pedagogics. It may be that the boot is on the other leg. It
          may be that we misplace the pedagogical need because it would be an uncomfortable reminder that we ourselves are still
          children in many respects and still need a vast amount of educating. 
}\fr{                                               "The Development of the Personality" (1934). In CW 17: The Development of the Personality. P.287
}
\pa{          Our whole educational problem suffers from a one-sided approach to the child who is to be educated, and from an equally
          one-sided lack of emphasis on the uneducatedness of the educator. 
}\fr{                                               "The Development of the Personality" (1934). In CW 17: The Development of the Personality. P.284
}
\pa{          An inferior man is never a good teacher. But he can conceal his pernicious inferiority, which secretly poisons the pupil,
          behind an excellent method or an equally brilliant gift of gab. Naturally the pupil of riper years desires nothing better than
          the knowledge of useful methods, because he is already defeated by the general attitude, which believes in the all-conquering
          method. He has learnt that the emptiest head, correctly echoing a method, is the best pupil. His whole environment is an
          optical demonstration that all success and all happiness are outside, and that only the right method is needed to attain the
          haven of one's desires. Or does, perchance, the life of his religious instructor demonstrate the happiness which radiates from
          the treasure of the inner vision ? 
}\fr{                                                                               Psychological Types (1921). CW 6: P.665
}
\pa{          Aestheticism is not fitted to solve the exceedingly serious and difficult task of educating man, for it always presupposes the
          very thing it should create-the capacity to love beauty. It actually hinders a deeper investigation of the problem, because it
          always averts its face from anything evil, ugly, and difficult, and aims at pleasure, even though it be of an edifying kind.
          Aestheticism therefore lacks all moral force, because au fond it is still only a refined hedonism. 
}\fr{                                                                               Psychological Types (1921). CW 6: P.194
}
\pa{          The fact that by far the greater part of humanity not only needs guidance, but wishes for nothing better than to be guided
          and held in tutelage, justifies, in a sense, the moral value which the Church sets on confession. The priest, equipped with all
          the insignia of paternal authority, becomes the responsible leader and shepherd of his flock. He is the father confessor and
          the members of his parish are his penitent children. Thus priest and Church replace the parents, and to that extent they free
          the individual from the bonds of the family. In so far as the priest is a morally elevated personality with a natural nobility of
          soul and a mental culture to match, the institution of confession may be commended as a brilliant method of social
          guidance and education, which did in fact perform a tremendous educative task for more than fifteen hundred years. So long
          as the medieval Church knew how to be the guardian of art and science-a role in which her success was due, in part, to her
          wide tolerance of worldly interests-confession was an admirable instrument of education. But it lost its educative value, at
          least for more highly developed people, as soon as the Church proved incapable of maintaining her leadership in the
          intellectual sphere-the inevitable consequence of spiritual rigidity. 
}\fr{                                                        "The Theory of Psychoanalysis" (1913). In CW 4: Freud and Psychoanalysis. P.433
}
\pa{          Today we are convinced that in all fields of knowledge psychological premises exist which exert a decisive influence upon the
          choice of material, the method of investigation, the nature of the conclusions, and the formulation of hypotheses and
          theories. We have even come to believe that Kant's personality was a decisive conditioning factor of his Critique of Pure
          Reason. Not only our philosophers, but our own predilections in philosophy, and even what we are fond of calling our "best"
          truths are affected, if not dangerously undermined, by this recognition of a personal premise. All creative freedom, we cry
          out, is taken away from us! What? Can it be possible that a man only thinks or says or does what he himself is? 
}\fr{                               "Psychological Aspects of the Mother Archetype" (1939). In CW 9, Part I: The Archetypes and the Collective Unconscious. P.150
}
\pa{          Reduction to the natural condition is neither an ideal state nor a panacea. If the natural state were really the ideal one,
          then the primitive would be leading an enviable existence. But that is by no means so, for aside from all the other sorrows
          and hardships of human life the primitive is tormented by superstitions, fears, and compulsions to such a degree that, if he
          lived in our civilization, he could not be described as other than profoundly neurotic, if not mad. 
}\fr{                                                     "On Psychic Energy" (1928). In CW 8: The Structure and Dynamics of the Psyche. P.94
}
\pa{          No one can make history who is not willing to risk everything for it, to carry the experiment with his own life through to the
          bitter end, and to declare that his life is not a continuation of the past, but a new beginning. Mere continuation can be left
          to the animals, but inauguration is the prerogative of man, the one thing he can boast of that lifts him above the beasts. 
}\fr{                                                              "Woman in Europe" (1927). In CW 10: Civilization in Transition. P. 268 
}
\pa{          Sooner or later it will be found that nothing really new happens in history. There could be talk of something really novel only
          if the unimaginable happened: if reason, humanity and love won a lasting victory. 
}\fr{                                                                          "Return to the Simple Life" (1941) In CW 18: P. 56
}
\pa{          The idea wants changelessness and eternity. Whoever lives under the supremacy of the idea strives for permanence; hence
          everything that pushes towards change must be opposed to the idea. 
}\fr{                                                                               Psychological Types (1921). CW 6: P.153
}
\pa{          Never in any circumstances should one indulge in the unscientific illusion that one's own subjective prejudice is a universal
          and fundamental psychological truth. No true science can spring from this, only a faith whose shadow is intolerance and
          fanaticism. Contradictory views are necessary for the evolution of any science, only they must not be set up in rigid
          opposition to each other but should strive for the earliest possible synthesis. 
}\fr{                                                    Der Organismus der Seele (1932) Review by Carl Jung (1933). Included in CW 18: P.639
}
\pa{          Theories in psychology are the very devil. It is true that we need certain points of view for their orienting and heuristic value;
          but they should always be regarded as mere auxiliary concepts that can be laid aside at any time. We still know so very little
          about the psyche that it is positively grotesque to think we are far enough advanced to frame general theories. We have not
          even established the empirical extent of the psyche's phenomenology: how then can we dream of general theories? No doubt
          theory is the best cloak for lack of experience and ignorance, but the consequences are depressing: bigotedness,
          superficiality, and scientific sectarianism. 
}\fr{                                                    "Psychic Conflicts of a Child" (1910). In CW 17: The Development of the Personality. P.7 
}
\pa{          Our psychology is a science that can at most be accused of having discovered the dynamite terrorists work with. What the
          moralist and the general practitioner do with it is none of our business and we have no intention of interfering. Plenty of
          unqualified persons are sure to push their way in and commit the greatest follies, but that too does not concern us. Our aim
          is simply and solely scientific knowledge, and we do not have to bother with all the uproar it has provoked. If religion and
          morality are blown to pieces in the process, so much the worse for them for not having more stamina. Knowledge is a force of
          nature that goes its way irresistibly from inner necessity. 
}\fr{                                                                                    Essay Included in CW 18: P. 314
}
\pa{          Until recently psychology was a special branch of philosophy, but now we are coming to something which Nietzsche
          foresaw-the rise of psychology in its own right, so much so that it is even threatening to swallow philosophy. The inner
          resemblance between the two disciplines consists in this, that both are systems of opinion about objects which cannot be
          fully experienced and therefore cannot be adequately comprehended by a purely empirical approach. Both fields of study thus
          encourage speculation, with the result that opinions are formed in such variety and profusion that many heavy volumes are
          needed to contain them all. Neither discipline can do without the other, and the one invariably furnishes the unspoken-and
          generally unconscious-assumptions of the other. 
}\fr{                                        "Basic Postulates of Analytical Psychology" (1931). In CW 8: The Structure and Dynamics of the Psyche. P.659 
}
\pa{          There is not one modern psychology-there are dozens of them. This is curious enough when we remember that there is only
          one science of mathematics, of geology, zoology, botany, and so forth. But there are so many psychologies that an American
          university was able to publish a thick volume under the title Psychologies of 1930- I believe there are as many psychologies
          as philosophies, for there is also no single philosophy, but many. I mention this for the reason that philosophy and
          psychology are linked by indissoluble bonds which are kept in being by the interrelation of their subject-matters. Psychology
          takes the psyche for its subject, and philosophy - to put it briefly - takes the world. 
}\fr{                                                                               Psychological Types (1921). CW 6: P.655
}
\pa{          Dogma and science are incommensurable quantities which damage one another by mutual contamination. Dogma as a
          factor in religion is of inestimable value precisely because of its absolute standpoint. But when science dispenses with
          criticism and scepticism it degenerates into a sickly hot-house plant. One of the elements necessary to science is extreme
          uncertainty. Whenever science inclines towards dogma and shows a tendency to be impatient and fanatical, it is concealing
          a doubt which in all probability is justified and explaining away an uncertainty which is only too well founded. 
}\fr{                                             Secret Ways of the Mind (1932) Introduction by Carl Jung In CW 4: Freud and Psychoanalysis. P.746
}
\pa{          The danger that faces us today is that the whole of reality will be replaced by words. This accounts for that terrible lack of
          instinct in modern man, particularly the city-dweller. He lacks all contact with life and the breath of nature. He knows a
          rabbit or a cow only from the illustrated paper, the dictionary, or the movies, and thinks he knows whatit is really like-and
          is then amazed that cowsheds "smell," because the dictionary didn't say so. 
}\fr{                                                   "Good and Evil in Analytical Psychology" (1959). In CW 10: Civilization in Transition. P.882
}
\pa{          When I speak of the relation of psychology to art we are outside [art's] sphere, and it is impossible for us not to speculate.
          We must interpret, we must find meanings in things, otherwise we would be quite unable to think about them. We have to
          break down life and events, which are selfcontained processes, into meanings, images, concepts, well knowing that in doing
          so we are getting further away from the living mystery. As long as we ourselves are caught up in the process of creation, we
          neither see nor understand; indeed we ought not to understand, for nothing is more injurious to immediate experience than
          cognition. But for the purpose of cognitive understanding we must detach ourselves from the creative process and look at it
          from the outside; only then does it become an image that expresses what we are bound to call "meaning." 
}\fr{                                      "On the Relation of Analytical Psychology to Poetry" (1922). In CW 15: The Spirit in Man, Art and Literature. P.121
}
\pa{          Disappointment, always a shock to the feelings, is not only the mother of bitterness but the strongest possible incentive to a
          differentiation of feeling. The failure of a pet plan, the disappointing behaviour of someone one loves, can supply the impulse
          either for a more or less brutal outburst of affect or for a modification and adjustment of feeling, and hence for its higher
          development. This culminates in wisdom if feeling is supplemented by reflection and rational insight. Wisdom is never
          violent: where wisdom reigns there is no conflict between thinking and feeling. 
}\fr{                                                                            Mysterium Coniuntionis (1955). CW 14: P.334 
}
\pa{          We understand another person in the same way as we understand, or seek to understand, ourselves. What we do not
          understand in ourselves we do not understand in the other person either. So there is plenty to ensure that his image will be
          for the most part subjective. As we know, even an intimate friendship is no guarantee of objective knowledge. 
}\fr{                                          "General Aspects of Dreams Psychology" (1916). In CW 8: The Structure and Dynamics of the Psyche. P.508
}
\pa{          Only a fool is interested in other people's guilt, since he cannot alter it. The wise man learns only from his own guilt. He will
          ask himself: Who am I that all this should happen to me? To find the answer to this fateful question he will look into his
          own heart. 
}\fr{                                                                           Psychology and Alchemy (1944). CW 12: P. 152
}
\pa{          Paradox ... does more justice to the unknowable than clarity can do, for uniformity of meaning robs the mystery of its
          darkness and sets it up as something that is known. That is a usurpation, and it leads the human intellect into hybris by
          pretending that it, the intellect, has got hold of the transcendent mystery by a cognitive act and "grasped" it. The paradox
          therefore reflects a higher level of intellect and, by not forcibly representing the unknowable as known, gives a more faithful
          picture of the real state of affairs. 
}\fr{                                   "Archetypes of the Collective Unconscious" (1935). In CW 9, Part I: The Archetypes and the Collective Unconscious. P. 1
}
\pa{          The intellect is only one among several fundamental psychic functions and therefore does not suffice to give a complete
          picture of the world. For this another function feeling-is needed too. Feeling often arrives at convictions that are different
          from those of the intellect, and we cannot always prove that the convictions of feeling are necessarily inferior. 
}\fr{                            "The Psychological Foundations of Belief in Spirits" (1920). In CW 9, Part I: The Archetypes and the Collective Unconscious. P. 600
}
\pa{          Life is crazy and meaningful at once. And when we do not laugh over the one aspect and speculate about the other, life is
          exceedingly drab, and everything is reduced to the littlest scale. There is then little sense and little nonsense either. When
          you come to think about it, nothing has any meaning, for when there was nobody to think, there was nobody to interpret
          what happened. Interpretations are only for those who don't understand; it is only the things we don't understand that have
          any meaning. Man woke up in a world he did not understand, and that is why he tries to interpret it. 
}\fr{                                                                            Psychology and Alchemy (1944). CW 12: P. 75
}
\pa{          Knowledge of the universal origins builds the bridge between the lost and abandoned world of the past and the still largely
          inconceivable world of the future. How should we lay hold of the future, how should we assimilate it, unless we are in
          possession of the human experience which the past has bequeathed to us? Dispossessed of this, we are without root and
          without perspective, defenceless dupes of whatever novelties the future may bring. 
}\fr{                                                         "The Gifted Child" (1943) In CW 17: The Development of the Personality. P. 250
}



%\end{document}




\end{document}