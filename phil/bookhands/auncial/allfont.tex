% allfont.tex    Test Artificial Uncial fonts
\documentclass{article}
\usepackage{allauncl}

\newcommand{\romannum}[1]{\romannumeral #1}
\newcommand{\Romannum}[1]{\uppercase\expandafter{\romannumeral #1}}
\newcommand{\ABC}{ABCDEFGHIJKL MNOPQRSTUVWXYZ}
\newcommand{\abc}{abcdefghijkl mnopqrstuvwxyz}
\newcommand{\punct}{.,;:!?`' \&{} () []}
\newcommand{\dashes}{- -- ---}
\newcommand{\figs}{0123456789}
\newcommand{\sentence}{%
this is an example of the artificial uncial rustic font. now is the time for all good
men, and women, to come to the aid of the party while the quick brown fox
jumps over the lazy dog:}

\newcommand{\Sentence}{%
This is an example of the Artificial Uncial font. Now is the time for all good
men, and women, to come to the aid of the party while the quick brown fox
jumps over the lazy dog:}

\newcommand{\latin}{Te canit adcelebratqve polvs rex gazifer hymnis.
  Trans zephyriqve globvm scandvnt tva facta per axem.}

\title{Try Artificial Uncial Fonts}
\author{}
\date{}
\pagenumbering{roman}
\begin{document}
\maketitle

\tableofcontents

\section{First section}

    This provides a short test of the characters in the Artificial Uncial fonts
--- the \verb|auncl| font family.



\begin{center}
The Artificial Uncial Huge normal font. \\ \par
{\Huge \ABC\\ \abc\\ \punct{}\dashes\\ \figs\\ \par }
\end{center}


\begin{center}
The Artificial Uncial font in its normal size \\
{\ABC{} \abc{} \punct{} \dashes{} \figs} \\
\end{center}

\begin{center}
The bold normal font, the normal font, and the bold Computer Modern
Roman, all in the normal size \\
{\textbf{\abc{} \figs}} \\
{\abc{} \figs} \\
\textcmr{\textbf{\abc{} \figs}} \\
\end{center}

\begin{center}
The bold versions, in Huge and tiny sizes. \par
\bfseries
\Huge \abc{} \figs \par
\tiny \abc{} \figs \par
\end{center}

\begin{center}
The font in the tiny size \\ \par
{\tiny \ABC{} \\ \abc\\ \figs\\ \par } 
\end{center}

\begin{center}
    Some ligatures in the normal font \\
{``the brown \& lazy dog --- but quick \& red fox?''}
\end{center}

\section{Second section}

    First some well known English phrases in an abcedarian sentence.

\Sentence{}

    These are two Latin abecedarian sentences dating from about 
\Romannum{790}.

\latin{}

\textcmss{This is the end of the test file, with this sentence being typeset
using the Computer Modern Sans font in the point size as specified for this
document.}

\end{document}