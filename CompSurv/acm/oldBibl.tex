\bibitem[1]{c:1} [1] C. Aarts, R. C. Backhouse, P. Hoogendijk, E. Voermans, J. 
van der Woude, ``{\em A Relational Theory of Datatypes}'', 1992, 

{\small{[In preparation, 
draft available at URL=http://www.win.tue.nl/win/cs/wp/papers/papers.html.]}\normalsize}

\bibitem[2]{c:2} L. Aceto, R. DeNicola, A. Fantechi, ``Testing 
Equivalencies for Event Structures'', in {\em Mathematical models for the 
semantics of parallelism}, LNCS, vol.~280, Springer, 1986.

\bibitem[3]{c:3} K.R. Apt, G.D. Plotkin, ``A Cook's tour of countable 
nondeterminism'', in {\em Proceedings of 8-th International Colloquium on 
Automata, Languages and Programming}, LNCS vol.~115, Springer, 1981.

\bibitem[4]{c:33} K.R. Apt, ``Ten years of Hoare's logic: a survey.  Part 
II: Nondeterminism'', {\em Theoretical Ccomputer Science}, vol.~28, pp.~83-109, 
1984.

\bibitem[5]{c:4} E. Astesiano, G. Costa, ``Sharing in Nondeterminism'', in 
{\em Automata, Languages and Programming, $6^{th}$ Colloquium}, LNCS, vol.~71, 
Springer, 1979.

\bibitem[6]{c:5} R.J. Back, ``Semantics of Unbounded Nondeterminism'', in 
{\em Automata, Languages and Programming}, LNCS, vol.~85, Springer, 1980.

\bibitem[7]{c:6} R.J. Back, J. von Wright, ``Duality in Specification 
Languages: A Lattice-Theoretical Approach'', {\em Acta Informatica}, 
vol.~27, pp.~583-625, 1990.

\bibitem[8]{c:7} J.W. de Bakker, ``Semantics and Termination of 
Nondeterministic Recursive Programs'', in {\em Automata, Languages and 
Programming}, Edinburgh, pp.~435-477, 1976.

\bibitem[9]{c:BarT} M. Barr, C. Wells, ``{\em Toposes, Triples and 
Theories}'', Springer, 1985.

\bibitem[10]{c:Bar} M. Barr, C. Wells, ``{\em Category Theory for Computing 
Science}'', Prentice Hall, 1990.

\bibitem[11]{c:8} F. Bauer, ``{\em The Munich Project CIP: The Wide 
Spectrum Language CIP}'', LNCS vol.~183, Springer, 1985.

\bibitem[12]{c:9} H. Beki\u{c}, ``{\em Programming Languages and Their 
Definition}'', LNCS, vol.~177, Springer, 1984.

\bibitem[13]{c:10} M. Ben-Ari, Z. Manna, A. Pnueli, ``The temporal logic of 
branching time'', in {\em Proceedings of $8^{th}$ Annual Symposium on PoPL}, 
1981.

\bibitem[14]{c:11} J.A. Bergstra, J.W. Klop, ``{An Abstraction Mechanism 
for Process Algebras}'', Tech.  Rep.  IW 231/83, Dept.  of CS, Mathematisch 
Centrum, Amsterdam, 1983.

\bibitem[15]{c:12} J.A. Bergstra, J.V. Tucker, ``Initial and final algebra 
semantics for data type specifications'', {\em SIAM J. Comput.}, vol.~12, 
pp.~366-387, 1983.

\bibitem[16]{c:13} J.A. Bergstra, J.W. Klop, ``Algebra of communicating 
processes'', in {\em Proceedings of CWI Symposium on Mathematics and CS}, 
pp.~89-138, October 1986.

\bibitem[17]{c:14} J.A. Bergstra, J.V. Tucker, ``Algebraic specifications 
of computable and semicomputable data types'', {\em Theoretical Computer 
Science}, vol.~50, pp.~137-181, 1987.

\bibitem[18]{c:Bia} M. Bia{\l}asik, B. Konikowska, ``A Logic for Nondeterministic
Specifications'', to appear in {\em Essays Dedicated to the Memory of H.~Rasiowa}, Kluwer

{\small{[also: ``Reasoning with 
Nondeterministic Specifications'', Tech.  Rep.  no.~793, Polish Academy of 
Sciences, Institute of Computer Science, 1995.]}\normalsize}

\bibitem[19]{c:Bloom} S. Bloom, ``Varieties of ordered algebras'', {\em J. 
Comput.  System Sci.}, 13, pp.~200-212, 1976.

\bibitem[20]{c:15} S. Bloom, R. Tindell, ``Compatible orderings on the 
metric theory of trees'', {\em SIAM J. Comput.}, vol.~9, no.~4, pp.~683-691, 
1980.

\bibitem[21]{c:Bor} F. Borceux, ``{\em Handbook of Categorical Algebra}'', 
Cambridge, 1994.

\bibitem[22]{c:16} G. Boudol, ``{\em Calculus maximaux et semantique 
operationnelle des programmes non deterministes}'', Ph.D. thesis, 
University of Paris, 1980.

\bibitem[23]{c:17} M. Broy, W. Dosch, H. Parsch, P. Pepper, M. Wirsing, 
``Existential quantifiers in abstract data types'', in {\em Automata, 
Languages and Programming, $6^{th}$ Colloquium}, LNCS, vol.~71, Springer, 
1979.

\bibitem[24]{c:18} M. Broy, R. Gnatz, M. Wirsing, ``Semantics of 
Nondeterministic and Noncontinuous Constructs'', LNCS, vol.~69,  pp.~553-392, 
Springer, 1980.

\bibitem[25]{c:19} M. Broy, M. Wirsing, ``{\em Initial versus Terminal Algebra 
Semantics for Partially Defined Abstract Types}'', Tech.  Rep.  TUM-I 8018, 
Technische Universit\"{a}t M\"{u}nchen, December 1981. 

{\small{[revised version: 
Partial Abstract Types, {\em Acta Informatica\/} vol. 18, pp.~47-64, 1982.]}\normalsize}

\bibitem[26]{c:20} M. Broy, M. Wirsing, ``Programming Languages as Abstract 
Data Types'', {\em Lille Colloque 80}, 1980.

\bibitem[27]{c:21} M. Broy, M. Wirsing, ``On the Algebraic Specification of 
Nondeterministic Programming Languages'', in {\em CAAP'81}, LNCS,
vol.~112, pp.~162-179, Springer, 1981.

\bibitem[28]{c:22} M. Broy, ``Fixed point theory for communication and 
concurrency'', in {\em IFIP TC2 Working Conference on Formal Description of 
Programming Concepts II}, pp.~125-147, North-Holland, 1983.

\bibitem[29]{c:23} M. Broy, ``On the Herbrand Kleene universe for 
nondeterministic computations'', in {\em Proceedings of MFCS'84}, LNCS, 
vol.~176, Springer, 1984.

\bibitem[30]{c:24} M. Broy, ``A Theory for Nondeterminism, Parallelism, 
Communication and Concurrency'', {\em Theoretical Computer Science}, 
vol.~45, 1986.

\bibitem[31]{c:25} A.K. Chandra, ``Computable Nondeterministic Functions'', 
{\em $19^{th}$ Annual Symposium on Foundations of Computer Science}, 1978.

\bibitem[32]{c:26} W. Clinger, ``Nondeterministic call by need is neither 
lazy nor by name'', in {\em Proceedings of ACM Symp.  LISP and Functional 
Programming}, pp.~226-234, 1982.

\bibitem[33]{c:27} P.M. Cohn, ``{\em Universal Algebra, Mathematics and Its 
Applications}'', vol.~6, D.Reidel Publishing Company, 1965.

\bibitem[34]{c:28} O.J. Dahl, ``{\em Verifiable Programming}'', Prentice 
Hall, 1992.

\bibitem[35]{c:29} E.W. Dijkstra, ``Guarded Commands, Nondeterminacy and 
Formal Derivation of Programs'', {\em CACM}, vol.~18, 1975.

\bibitem[36]{c:30} E.W. Dijkstra, ``{\em A Discipline of Programming}, 
Prentice Hall, 1976.

\bibitem[37]{c:31} E.W. Dijkstra, ``A simple fix-point argument without the 
restriction to continuity'', in {\em Control Flow and Data Flow, Comp.  and 
System Sciences}, Springer, 1984.

\bibitem[38]{c:32} A. Dovier, E. Omodeo, E. Pontelli, G.F. Rossi, 
``Embedding Finite Sets in a Logic Programming Language'', LNAI, vol.~660, 
 pp.~150-167, Springer, 1993.

\bibitem[39]{c:33} H. Egli, ``{A mathematical model for nondeterministic 
computations}'', Tech.  Rep., ETH, 1975.

\bibitem[40]{c:34} J. Engelfriet, E.M. Schmidt, ``IO and OI. 1'', {\em 
Journal of Computer and System Sciences}, vol.~15, pp.~328-353, 1977.

\bibitem[41]{c:35} J. Engelfriet, E.M. Schmidt, ``IO and OI. 2'', {\em 
Journal of Computer and System Sciences}, vol.~16, pp.~67-99, 1978.

\bibitem[42]{c:36} Y.A. Feldman, D. Harel, ``A probabilistic dynamic 
logic'', in {\em Proceedings of $14^{th}$ Annual ACM Symposium on Theory of 
Computing}, 1982.

\bibitem[43]{c:37} J.A. Fern\'{a}ndez, J. Minker, ``Bottom-Up Evaluation of 
Hierarchical Disjunctive Dedutcive Databases'', in {\em Proceedings of the 
$8^{rh}$ International Conference on Logic Programming}, 1991.

\bibitem[44]{c:38} R.W. Floyd, ``Nondeterministic algorithms'', {\em 
Journal of the ACM}, vol.~14, no.~4, 1967.

\bibitem[45]{c:39} R.J. van Glabbeek, ``Bounded nondeterminism and the 
approximation induction principle in process algebra'', Tech.  Rep.  
CS-R8634, Centrum voor Wiskunde en Informatica, Amsterdam, 1986.  

{\small{[extended 
abstract in: F.J.~Brandenburg, G.~Vidal-Naquet and M.~Wirsing (Eds.) {\em 
Proceedings of STACS 87,\/} LNCS 247,  pp.~336-347, Springer, 1987.]}\normalsize}

\bibitem[46]{c:40} J.A. Goguen, J. Thatcher, E. Wagner, J. Wright, 
``Initial algebra semantics and continuous algebras'', {\em Journal Assoc.  
Comp.  Mach.}, vol.~24, pp.~68-95, 1977.

\bibitem[47]{c:41} J.A. Goguen, R.M. Burstall, ``Introducing 
Institutions'', in {\em Logics of Programs}, LNCS, vol.~164, Springer, 
1983.

\bibitem[48]{c:42} J.A. Goguen, ``What is unification?  {A} categorical 
view of substitution, equation, and solution'', in M. Nivat and A.-K. 
Hassan (Eds.)  {\em Resolution of Equations in Algebraic Structures, Volume 
1: Algebraic Techniques\/}, pp.~217-261, Academic Press, 1989.

\bibitem[49]{c:43} I. Guessarian, ``{\em Algebraic Semantics}'', LNCS,
vol.~99, Springer, 1981.

\bibitem[50]{c:44} J.V. Guttag, ``The Specification and Application to 
Programming of ADT'', Tech.  Rep.~CSR6-59, Computer Systems Research 
Group, University of Toronto, 1975.

\bibitem[51]{c:45} G.E. Hansoul, ``A subdirect decomposition theorem for 
multialgebras'', {\em Algebra Universalis}, vol.~16, pp.~275-281, 1983.

\bibitem[52]{c:46} D. Harel, A.R. Meyer, V.R. Pratt, ``Computability and 
Completeness in Logics of Programs'', in {\em Proceedings of $9^{th}$ Annual 
ACM Symposium on Theory of Computing}, 1977.

\bibitem[53]{c:47} D. Harel, V.R. Pratt, ``Nondeterminism in Logics of 
Programs'', in {\em Proceedings of $5^{th}$ Annual Symposium on PoPL}, 1978.

\bibitem[54]{c:48} R. Heckmann, ``Power Domains Supporting Recursion and 
Failure'', in {\em CAAP'92}, LNCS, vol.~581, Springer, 1992.

\bibitem[55]{c:49} M. Hennessy, ``The semantics of call-by-value and 
call-by-name in a nondeterministic environment'', {\em SIAM J. Comput.}, 
vol.~9, no.~1, 1980.

\bibitem[56]{c:50} M. Hennessy, R. Milner, ``On Observing Nondeterminism 
and Concurrency'', in {\em Automata, Languages and Programming}, LNCS, 
vol.~85, Springer, 1980.

\bibitem[57]{c:51} M. Hennessy, ``Powerdomains and nondeterministic 
recursive functions'', in {\em Proceedings of $5^{th}$ International Symp.  on 
Programming}, LNCS, vol.~137, Springer, 1982.

\bibitem[58]{c:52} M. Hennessy, ``Observing Processes'', in {\em Linear 
Time, Branching Time and Partial Order in Logics and Models for 
Concurrency}, LNCS, vol.~354, Springer, 1988.

\bibitem[59]{c:53} W.H. Hesselink, ``A Mathematical Approach to 
Nondeterminism in Data Types'', {\em ACM ToPLaS}, vol.~10, 1988.

\bibitem[60]{c:Hes} W.H. Hesselink, ``{\em Programs, Recursion and 
Unbounded Choice}'', Cambridge, 1992.

\bibitem[61]{c:54} C.A.R. Hoare, ``{\em Communicating Sequential 
Processes}'', Prentice Hall, 1985.

\bibitem[62]{c:55} P.F. Hoogendik, ``Relational programming laws in Boom 
hierarchy of types'', in {\em Proceedings of Mathematics of Program 
Construction}, Springer, 1992.

\bibitem[63]{c:56} G. Hornung, P. Raulefs, ``Terminal algebra semantics and 
retractions for abstract data types'', in {\em Automata, Languages and 
Programming}, LNCS, vol.~85, Springer, 1980.

\bibitem[64]{c:57} H. Hu{\ss}mann, ``Nondeterministic algebraic 
specifications and nonconfluent term rewriting'', in {\em Algebraic and 
Logic Programming}, LNCS, vol.~343, Springer, 1988.

\bibitem[65]{c:58} H. Hu{\ss}mann, ``{\em Nondeterministic Algebraic 
Specifications}'', Ph.D. thesis, Fakult\"{a}t f\"{u}r Mathematik und 
Informatik, Universit\"{a}t Passau, 1990.

\bibitem[66]{c:59} H. Hu{\ss}mann, ``{\em Nondeterminism in Algebraic 
Specifications and Algebraic Programs}'', Birkh\"{a}user, 1993.

\bibitem[67]{c:60} G. Huet, J. Hullot, ``Proofs by Induction in Equational 
Theories with Constructors'', {\em JCSS}, vol.~25, pp.~239-266, 1981.

\bibitem[68]{c:61} B. Jayaraman, ``Implementation of Subset-Equational 
Programs'', {\em Journal of Logic Programming}, vol.~12, no.~4, pp.~299-324, 
1992.

\bibitem[69]{c:62} S. Kaplan, ``Conditional Rewriting'', in {\em 
Conditional Term Rewriting Systems}, LNCS, vol.~308, Springer, 1987.

\bibitem[70]{c:63} S. Kaplan, ``Rewriting with a Nondeterministic Choice 
Operator'', {\em Theoretical Computer Science}, vol.~56, pp.~37-57, 1988.

\bibitem[71]{c:64} D. Kapur, ``{\em Towards a theory of abstract data 
types}'', Ph.D. thesis, Laboratory for CS, MIT, 1980.

\bibitem[72]{c:65} J.R. Kennaway, C.A.R. Hoare, ``A theory of 
nondeterminism'', in {\em Automata, Languages and Programming}, LNCS, vol.~
85, Springer, 1980.

\bibitem[73]{c:66} P.R. Kosinski, ``A Straightforward Denotational 
Semantics for Non-Determinate Data Flow Programs'', {\em $5^{th}$ Annual 
Symposium on PoPL}, 1978.

\bibitem[74]{c:67} P.R. Kosinski, Denotational Semantics of Determinate and 
Non-Determinate Data Flow Programs, Ph.D. thesis, MIT Laboratory of 
Computer Science, 1979.

\bibitem[75]{c:68} D. Kozen, ``Semantics of probabilistic programs'', {\em 
J. Comput.  Sys.  Sci.}, vol.~22, 1981.

\bibitem[76]{c:69} V. Kriau\u{c}iukas, M. Walicki, ``Reasoning and 
Rewriting with Set-Relations I: Ground Case'', in {\em Proceedings of 
CSL'94}, LNCS, vol.~933, 1995.

\bibitem[77]{c:69a} V. Kriau\u{c}iukas, M. Walicki, ``Reasoning and 
Rewriting with Set-Relations II: The Non-Ground Case'', to appear in {\em 
Recent Trends in Data Type Specification}, LNCS, 1996.

\bibitem[78]{c:70} R. Kuiper, ``An operational semantics for bounded 
nondeterminism equivalent to a denotational one'', in {\em Proceedings of 
the International Symposium on Algorithmic Languages,} 
(eds. J.W.de~Bakker, J.C. van~Vliet), IFIP,  pp.~373-398, North-Holland, 1981.

{\small{[also: Tech.  Rep.  IW 169/81, 
Stichtig Mathematisch Centrum, Dept.  of CS, Amsterdam, 1981.]}\normalsize}

\bibitem[79]{c:LR} D. Lehmann, M. Rabin, ``On the advantages of free 
choice'', {\em $8^{th}$ Annual Symposium on PoPL}, 1981.

\bibitem[80]{c:71} D. Lehmann, S. Shelah, ``Reasoning with Time and 
Chance'', in {\em Automata, Languages and Programming}, LNCS, vol.~154, 
Springer, 1983.

\bibitem[81]{c:72} J. Levy, J. Augusti, ``Bi-rewriting, a term rewriting 
technique for monotonic order relations'', in {\em RTA'93}, LNCS, vol.~690,  
pp.~17-31, Springer, 1993.

\bibitem[82]{c:73} U. de Liguoro, A. Piperno, ``Must pre-order in 
non-deterministic untyped $\lambda$-calculus'', in {\em CAAP'92}, LNCS, 
vol.~581, Springer, 1992.

\bibitem[83]{c:74} R.C. Lyndon, ``Properties Preserved under 
Homomorphism'', {\em Pacific Journal of Mathematics}, vol.~9, 1959.

\bibitem[84]{c:75} B. M\"{o}ller, ``On the Algebraic Specification of 
Infinite Objects -- Ordered and Continuous Models of Algebraic Types'', 
{\em Acta Informatica}, vol.~22, pp.~537-578, 1985.

\bibitem[85]{c:Mac} S. Mac Lane, ``{\em Categories for the Working 
Mathematician}'', Springer, 1971.

\bibitem[86]{c:76} B. Mahr, J.A. Makowsky, ``Characterizing specification 
languages which admit initial semantics'', in {\em CAAP'83}, LNCS, vol.~159, 
Springer, 1983, pp.~300-316.

\bibitem[87]{c:77} T.S.E. Maibaum, ``The Semantics of Nondeterminism'', 
Tech.  Rep.  CS-77-30, University of Waterloo, Ontario, Canada, December 
1977.

\bibitem[88]{c:78} J.A. Makowsky, ``Why Horn Formulas Matter in Computer 
Science'', {\em Journal of Computer and System Science}, vol.~34, pp.~266-292, 
1987.

\bibitem[89]{c:79} A.I. Mal'cev, ``{\em The Metamathematics of Algebraic 
Systems}'', Studies in Logic and the Foundations of Mathematics vol.~66, 
North-Holland, 1971.

\bibitem[90]{c:80} A.I. Mal'cev, ``{\em Algebraic Systems}'', Die 
Grundlehren der mathematischen Wissenschaften in Einzeldarstellungen, vol.~192, Springer, 1973.

\bibitem[91]{c:81} A. Manes, ``Fuzzy Theories'', {\em Journal of 
Mathematical Analysis and Applications}, 1982.

\bibitem[92]{c:82} Z. Manna, ``The Correctness of Nondeterministic 
Programs'', {\em Artificial Intelligence}, vol.~1, pp.~1-26, 1970.

\bibitem[93]{c:83} Z. Manna, A. Pnueli, ``{\em The Temporal Logic of 
Reactive and Concurrent Systems}'', Springer, 1992.

\bibitem[94]{c:84} J. McCarthy, ``A basis for a mathematical theory of 
computation'', in {\em Computer Programming and Formal Systems}, 
North-Holland, 1963.

\bibitem[95]{c:85} S. Meldal, ``An Abstract Axiomatization of Pointer 
Types'', in {\em Proceedings of The 22nd Annual Hawaii International 
Conference on System Sciences}, IEEE Computer Society Press, 1989.

\bibitem[96]{c:Mes} J. Meseguer, ``Order completion monads'', {\em Algebra 
Universalis}, 16, pp.~63-82, 1983.

\bibitem[97]{c:86} J. Meseguer, ``A Logical Theory of Concurrent Objects'', 
in N. Meyrowitz (Ed.) {\em OOPSLA ECOOP '90 Proceedings\/}, SIGLAN Notices, 
vol.~25 (10), pp.~101-115, October 1990. 

\bibitem[98]{c:87} J. Meseguer, ``Conditional rewriting logic as a unified 
model of concurrency'', {\em Theoretical Computer Science}, no.~96, 
pp.~73-155, 1992.

\bibitem[99]{c:88} J. Mezei, J.B. Wright, ``Algebraic automata and 
context-free sets'', {\em Information and Control}, vol.~11, 1967.

\bibitem[100]{c:89} R. Milner, ``Processes: a mathematical model of 
computing agents'', in {\em Proceedings of Logic Colloquium}, 1973.

\bibitem[101]{c:90} R. Milner, ``{\em Calculi for Communicating Systems}'', 
LNCS vol.~92, Springer, 1980.

\bibitem[102]{c:101a} R. Milner, ``{\em Communication and Concurrency}'', Prentice Hall, 1982.

\bibitem[103]{c:91} P.D. Mosses, ``Unified Algebras and Institutions'', in 
{\em Proceedings of LICS'89, $4^{th}$ Annual Symposium on Logic in Computer 
Science}, 1989.

\bibitem[104]{c:92} P.D. Mosses, ``Unified Algebras and Action Semantics'', 
in {\em Proceedings of STACS'89}, LNCS, vol.~349, Springer, 1989.

\bibitem[105]{c:Nel} G. Nelson, ``A generalization of Dijkstra's 
calculus'', {\em ACM ToPLaS}, 11, pp.~517-561, 1992.

\bibitem[106]{c:93} T. Nipkow, ``Non-deterministic Data Types: Models and 
Implementations'', {\em Acta Informatica}, vol.~22, pp.~629-661, 1986.

\bibitem[107]{c:94} T. Nipkow, ``Observing nondeterministic data types'', 
in {\em Recent Trends in Data Type Specification}, LNCS, vol.~332, 
Springer, 1987.

\bibitem[108]{c:95} T. Nipkow, ``{\em Behavioural Implementations Concepts 
for Nondeterministic Data Types}'', Ph.D. thesis, Dept.  of CS, The 
University of Manchester, 1987.

\bibitem[109]{c:96} M. Nivat, ``On the interpretation of recursive polyadic 
program schemes'', {\em Symposia Mathematica}, vol.~15, pp.~255-281, 1975.

\bibitem[110]{c:97} M. Nivat, ``Nondeterministic programs: an algebraic 
overview'', in {\em Information Processing '80, Proc.  of the IFJP Congress 
'80}, North-Holland, 1980.

\bibitem[111]{c:98} M. O'Donnell, ``{\em Computing in Systems Described by 
Equations}'', LNCS, vol.~58, Springer, 1977.

\bibitem[112]{c:99} D. Park, ``On the Semantics of Fair Parallellism'', in 
{\em Abstract Software Specifications}, LNCS, vol.~86, Springer, 1979.

\bibitem[113]{c:100} C.A. Petri, ``Non-sequential processes'', Tech.  Rep.  
ISF-77-05, Gesellschaft f\"{u}r Matematik und Datenverarbeitung, Sankt 
Augustin, 1977.

\bibitem[114]{c:101} G. Pickert, ``Bemerkungen zum Homomorphie-Begriff'', 
{\em Mathematische Zeitschrift}, vol.~53, 1950.

\bibitem[115]{c:102} H.E. Pickett, ``Homomorphisms and subalgebras of 
multialgebras'', {\em Pacific Journal of Mathematics}, vol.~21, pp.~327-342, 
1967.

\bibitem[116]{c:103} G. Plotkin, ``A power domain construction'', {\em SIAM 
J. Comput.}, vol.~5, no.~3, pp.~452-487, 1976.

\bibitem[117]{c:104} G. Plotkin, ``Dijksrta's Predicate Transformers and 
Smyth's Powerdomains'', in {\em Abstract Software Specifications}, LNCS, 
vol.~86, Springer, 1980.

\bibitem[118]{c:105} G. Plotkin, K.R. Apt, ``Countable Nondeterminism and 
Random Assignment'', Journal of the ACM, vol.~33, no.~4, pp.~724-767.

{\small{[also: Tech.~Rep., University of Edinburgh, 1982.]}\normalsize}

\bibitem[119]{c:106} G. Plotkin, ``Domains'', 1983, lecture notes, Dept.  
of Computer Science, University of Edinburgh. 

{\small{[available at 
URL=http://hypatia.dcs.qmw.ac.uk/authors/P/PlotkinGD/papers/dom.ps.Z.]}\normalsize}

\bibitem[120]{c:107} X. Qian, A. Goldberg, ``Referential Opacity In 
Nondeterministic Data Refinement'', {\em ACM LoPLaS}, vol.~2, no.~1-4, 
 pp.~233-241, 1993.

\bibitem[121]{c:108} J.H. Reif, ``Logics for probabilistic programming'', 
in {\em Proceedings of $12^{th}$ ACM Symposium on Theory of Computing}, 1980.

\bibitem[122]{c:109} W. Reisig, ``Petri Nets.  An Introduction'', {\em 
EATCS Monographs on Theoretical Computer Science}, vol.~4, Springer, 1985.

\bibitem[123]{c:110} H. S{\o}dergaard, P. Sestoft, ``Non-Determinism in Functional
Languages'', {\em Computer Journal,} vol.~35, no.~5,  pp.~514-523, 1992.

{\small{[also: ``Non-Determinacy and 
Its Semantics'', Tech.  Rep.  86/12, Datalogisk Institut, K{\o}benhavns 
Universitet, 1987.]}\normalsize}

\bibitem[124]{c:111} N. Saheb-Djahromi, ``Probabilistic CPO's for 
Nondeterminism'', Tech.  Rep.  CSR-37-79, University of Edinburgh, Dept.  
of CS, 1979.

\bibitem[125]{c:112} D. Sannella, A. Tarlecki, ``Specifications in 
Arbitrary Institutions'', {\em Information and Computation}, vol.~76, 
pp.~165-210, 1988.

\bibitem[126]{c:113} R.L. Schwartz, ``An axiomatic treatment of ALGOL 68 
routines'', in {\em Proceedings of $6^{th}$ Colloquium on Automata, Languages 
and Programming}, vol.~71, Springer, 1979.

\bibitem[127]{c:114} J. Schwartz, R. Dewar, E. Schonberg, E. Dubinsky, 
``{\em Programming with Sets; An Introduction to SETL}'', Springer, 1986.

\bibitem[128]{c:115} D. Scott, ``Continuous Lattices'', in {\em Proceedings 
of 1971 Dalhousie Conference}, LNM, vol.~274, Springer, 1972.

\bibitem[129]{c:116} D. Scott, ``Data Types as Lattices'', {\em SIAM J. 
Comput.}, vol.~5, no.~4, pp.~522-587, 1976.

\bibitem[130]{c:117} M.B. Smyth, ``Power domains'', {\em J. of Computer and 
System Sciences}, vol.~16, 1978.

\bibitem[131]{c:118} M.B. Smyth, ``Power Domains and Predicate 
Transformers: A Topological View'', in J.Diaz (ed.), {\em Automata,
Languages and Programming\/}, LNCS vol.154, Springer, pp.~225-241, 1983.

\bibitem[132]{c:genind} J. Spitzen, B. Wegbreit, ``The verification and synthesis of
data structures'', {\em Acta Informatica}, 4, pp.~127-144, 1975.

\bibitem[133]{c:119} E.W. Stark, ``Compositional relational semantics for 
indeterminate dataflow networks'', in {\em Category Theory and Computer 
Science}, LNCS, vol.~389, Springer, 1989.

\bibitem[134]{c:120} F. Stolzenburg, ``An Algorithm for General Set 
Unification'', {\em Workshop on Logic Programming with Sets}, ICLP'93, 
1993.

\bibitem[135]{c:121} P.A. Subrahmanyam, ``Nondeterminism in Abstract Data 
Types'', in {\em Automata, Languages and Programming}, LNCS, vol.~115, 
Springer, 1981.

\bibitem[136]{c:122} A. Tarlecki, ``Free constructions in algebraic 
institutions'', in {\em Proceedings of MFCS'84}, LNCS, vol.~176, Springer, 
1984.

\bibitem[137]{c:123} E. Voermans, ``Pers as types, inductive types and 
types with laws'', in {\em Proceedings of PHOENIX Seminar and Workshop on 
Declarative Programming}, Springer, 1991.

\bibitem[138]{c:124} H. Volger, ``The Semantics of Disjunctive Deductive 
Databases'', Tech.  Rep.  MIP-8931, Fakult\"{a}t f\"{u}r Mathematik und 
Informatik, Universit\"{a}t Passau, October 1989.

\bibitem[139]{c:125} H. Volger, ``The semantics of disjunctive deductive 
databases'', in {\em Proceedings of CSL'89}, LNCS, vol.~440, Springer, 
1989.

\bibitem[140]{c:126} M. Walicki, S. Meldal, ``A Complete Calculus for the 
Multialgebraic and Functional Semantics of Nondeterminism'', {\em ACM 
ToPLaS}, vol.~17, no.~2, 1995.

\bibitem[141]{c:127} M. Walicki, ``{\em Algebraic Specifications of 
Nondeterminism}'', Ph.D. thesis, University of Bergen, Department of 
Informatics, 1993.

\bibitem[142]{c:128} M. Walicki, S. Meldal, ``Initiality+Nondeterminism 
$\impl$ Junk'', in {\em Proceedings of NIK'93}, Tapir, 1993.

\bibitem[143]{c:129} M. Walicki, S. Meldal, ``Sets and Nondeterminism'', 
{\em Workshop on Logic Programming with Sets}, ICLP'93, 1993.

\bibitem[144]{c:130} M. Walicki, S. Meldal, ``Multialgebras, Power Algebras 
and Complete Calculi of Identities and Inclusions'', in {\em Recent Trends 
in Data Type Specifications}, LNCS, vol.~906, 1995.

\bibitem[145]{c:132} M. Walicki, M. Broy, ``Structured Specifications and 
Implementation of Nondeterministic Data Types'', {\em Nordic Journal of 
Computing}, no.~2,  pp.~358-395, 1995.

\bibitem[146]{c:133} M. Walicki, S. Meldal, `` Generated Models and the 
$\omega$-rule; the Nondeterministic Case'', in {\em Proceedings of 
TAPSOFT'95}, LNCS, vol.~915, 1995.

\bibitem[147]{c:134} G. Winskel, ``Event structure semantics of CCS and 
related languages'', in {\em Proceedings of ICALP'82}, LNCS, vol.~140, 
Springer, 1982.

\bibitem[148]{c:135} G. Winskel, ``{\em Event structures}'', LNCS vol.~255, Springer, 1987.

\bibitem[149]{c:136} G. Winskel, ``{\em An introduction to event 
structures}'', LNCS, vol.~354, Springer, 1988.

\bibitem[150]{c:137} U. Wolter, M. L\"{o}we, ``Beyond Conditional Equations'', 
in {\em CAAP'92}, LNCS, vol.~581, Springer, 1992.

