%\documentclass[10pt]{article}
%%%\documentstyle[a4wide,10pt]{article}
%\makeatletter
%
\ifcase \@ptsize
    % mods for 10 pt
    \oddsidemargin  0.15 in     %   Left margin on odd-numbered pages.
    \evensidemargin 0.35 in     %   Left margin on even-numbered pages.
    \marginparwidth 1 in        %   Width of marginal notes.
    \oddsidemargin 0.25 in      %   Note that \oddsidemargin = \evensidemargin
    \evensidemargin 0.25 in
    \marginparwidth 0.75 in
    \textwidth 5.875 in % Width of text line.
\or % mods for 11 pt
    \oddsidemargin 0.1 in      %   Left margin on odd-numbered pages.
    \evensidemargin 0.15 in    %   Left margin on even-numbered pages.
    \marginparwidth 1 in       %   Width of marginal notes.
    \oddsidemargin 0.125 in    %   Note that \oddsidemargin = \evensidemargin
    \evensidemargin 0.125 in
    \marginparwidth 0.75 in
    \textwidth 6.125 in % Width of text line.
\or % mods for 12 pt
    \oddsidemargin -10 pt      %   Left margin on odd-numbered pages.
    \evensidemargin 10 pt      %   Left margin on even-numbered pages.
    \marginparwidth 1 in       %   Width of marginal notes.
    \oddsidemargin 0 in      %   Note that \oddsidemargin = \evensidemargin
    \evensidemargin 0 in
    \marginparwidth 0.75 in
    \textwidth 6.375 true in % Width of text line.
\fi

\voffset -2cm
\textheight 22.5cm

%\makeatother

%%\makeatletter
%\show\
%\makeatother
\newcommand{\ite}[1]{\item[{\bf #1.}]}
\newcommand{\app}{\mathrel{\scriptscriptstyle{\vdash}}}
\newcommand{\estr}{\varepsilon}
\newcommand{\PSet}[1]{{\cal P}(#1)}
\newcommand{\ch}{\sqcup}
\newcommand{\into}{\to}
\newcommand{\Iff}{\Leftrightarrow}
\renewcommand{\iff}{\leftrightarrow}
\newcommand{\prI}{\vdash_I}
\newcommand{\pr}{\vdash}
\newcommand{\ovr}[1]{\overline{#1}}

\newcommand{\cp}{{\cal O}}

% update function/set
%\newcommand{\upd}[3]{#1\!\Rsh^{#2}_{\!\!#3}} % AMS
\newcommand{\upd}[3]{#1^{\raisebox{.5ex}{\mbox{${\scriptscriptstyle{\leftarrow}}\scriptstyle{#3}$}}}_{{\scriptscriptstyle{\rightarrow}}{#2}}} 
\newcommand{\rem}[2]{\upd{#1}{#2}{\bullet}}
\newcommand{\add}[2]{\upd {#1}{\bullet}{#2}}
%\newcommand{\mv}[3]{{#1}\!\Rsh_{\!\!#3}{#2}}
\newcommand{\mv}[3]{{#1}\:\raisebox{-.5ex}{$\stackrel{\displaystyle\curvearrowright}{\scriptstyle{#3}}$}\:{#2}}

\newcommand{\leads}{\rightsquigarrow} %AMS

\newenvironment{ites}{\vspace*{1ex}\par\noindent 
   \begin{tabular}{r@{\ \ }rcl}}{\vspace*{1ex}\end{tabular}\par\noindent}
\newcommand{\itt}[3]{{\bf #1.} & $#2$ & $\impl$ & $#3$ \\[1ex]}
\newcommand{\itte}[3]{{\bf #1.} & $#2$ & $\impl$ & $#3$ }
\newcommand{\itteq}[3]{\hline {\bf #1} & & & $#2=#3$ }
\newcommand{\itteqc}[3]{\hline {\bf #1} &  &  & $#2=#3$ \\[.5ex]}
\newcommand{\itteqq}[3]{{\bf #1} &  &  & $#2=#3$ }
\newcommand{\itc}[2]{{\bf #1.} & $#2$ &    \\[.5ex]}
\newcommand{\itcs}[3]{{\bf #1.} & $#2$ & $\impl$ & $#3$  \\[.5ex] }
\newcommand{\itco}[3]{   & $#1$ & $#2$  & $#3$ \\[1ex]}
\newcommand{\itcoe}[3]{   & $#1$ & $#2$  & $#3$}
\newcommand{\bit}{\begin{ites}}
\newcommand{\eit}{\end{ites}}
\newcommand{\na}[1]{{\bf #1.}}
\newenvironment{iten}{\begin{tabular}[t]{r@{\ }rcl}}{\end{tabular}}
\newcommand{\ass}[1]{& \multicolumn{3}{l}{\hspace*{-1em}{\small{[{\em Assuming:} #1]}}}}

%%%%%%%%% nested comp's
\newenvironment{itess}{\vspace*{1ex}\par\noindent 
   \begin{tabular}{r@{\ \ }lllcl}}{\vspace*{1ex}\end{tabular}\par\noindent}
\newcommand{\bitn}{\begin{itess}}
\newcommand{\eitn}{\end{itess}}
\newcommand{\comA}[2]{{\bf #1}& $#2$ \\ }
\newcommand{\comB}[3]{{\bf #1}& $#2$ & $#3$\\ }
\newcommand{\com}[3]{{\bf #1}& & & $#2$ & $\impl$ & $#3$\\[.5ex] }

\newcommand{\comS}[5]{{\bf #1} 
   & $#2$ & $#3$ & $#4$ & $\impl$ & $#5$\\[.5ex] }

%%%%%%%%%%%%%%%%
\newtheorem{CLAIM}{Proposition}[section]
\newtheorem{COROLLARY}[CLAIM]{Corollary}
\newtheorem{THEOREM}[CLAIM]{Theorem}
\newtheorem{LEMMA}[CLAIM]{Lemma}
\newcommand{\MyLPar}{\parsep -.2ex plus.2ex minus.2ex\itemsep\parsep
   \vspace{-\topsep}\vspace{.5ex}}
\newcommand{\MyNumEnv}[1]{\trivlist\refstepcounter{CLAIM}\item[\hskip
   \labelsep{\bf #1\ \theCLAIM\ }]\sf\ignorespaces}
\newenvironment{DEFINITION}{\MyNumEnv{Definition}}{\par\addvspace{0.5ex}}
\newenvironment{EXAMPLE}{\MyNumEnv{Example}}{\nopagebreak\finish}
\newenvironment{PROOF}{{\bf Proof.}}{\nopagebreak\finish}
\newcommand{\finish}{\hspace*{\fill}\nopagebreak 
     \raisebox{-1ex}{$\Box$}\hspace*{1em}\par\addvspace{1ex}}
\renewcommand{\abstract}[1]{ \begin{quote}\noindent \small {\bf Abstract.} #1
    \end{quote}}
\newcommand{\B}[1]{{\rm I\hspace{-.2em}#1}}
\newcommand{\Nat}{{\B N}}
\newcommand{\bool}{{\cal B}{\rm ool}}
\renewcommand{\c}[1]{{\cal #1}}
\newcommand{\Funcs}{{\cal F}}
%\newcommand{\Terms}{{\cal T}(\Funcs,\Vars)}
\newcommand{\Terms}[1]{{\cal T}(#1)}
\newcommand{\Vars}{{\cal V}}
\newcommand{\Incl}{\mathbin{\prec}}
\newcommand{\Cont}{\mathbin{\succ}}
\newcommand{\Int}{\mathbin{\frown}}
\newcommand{\Seteq}{\mathbin{\asymp}}
\newcommand{\Eq}{\mathbin{\approx}}
\newcommand{\notEq}{\mathbin{\Not\approx}}
\newcommand{\notIncl}{\mathbin{\Not\prec}}
\newcommand{\notCont}{\mathbin{\Not\succ}}
\newcommand{\notInt}{\mathbin{\Not\frown}}
\newcommand{\Seq}{\mathrel{\mapsto}}
\newcommand{\Ord}{\mathbin{\rightarrow}}
\newcommand{\M}[1]{\mathbin{\mathord{#1}^m}}
\newcommand{\Mset}[1]{{\cal M}(#1)}
\newcommand{\interpret}[1]{[\![#1]\!]^{A}_{\rho}}
\newcommand{\Interpret}[1]{[\![#1]\!]^{A}}
%\newcommand{\Comp}[2]{\mbox{\rm Comp}(#1,#2)}
\newcommand{\Comp}[2]{#1\diamond#2}
\newcommand{\Repl}[2]{\mbox{\rm Repl}(#1,#2)}
%\newcommand\SS[1]{{\cal S}^{#1}}
\newcommand{\To}[1]{\mathbin{\stackrel{#1}{\longrightarrow}}}
\newcommand{\TTo}[1]{\mathbin{\stackrel{#1}{\Longrightarrow}}}
\newcommand{\oT}[1]{\mathbin{\stackrel{#1}{\longleftarrow}}}
\newcommand{\oTT}[1]{\mathbin{\stackrel{#1}{\Longleftarrow}}}
\newcommand{\es}{\emptyset}
\newcommand{\C}[1]{\mbox{$\cal #1$}}
\newcommand{\Mb}[1]{\mbox{#1}}
\newcommand{\<}{\langle}
\renewcommand{\>}{\rangle}
\newcommand{\Def}{\mathrel{\stackrel{\mbox{\tiny def}}{=}}}
\newcommand{\impl}{\mathrel\Rightarrow}
\newcommand{\then}{\mathrel\Rightarrow}
\newfont{\msym}{msxm10}

\newcommand{\false}{\bot}
\newcommand{\true}{\top}

\newcommand{\restrict}{\mathbin{\mbox{\msym\symbol{22}}}}
\newcommand{\List}[3]{#1_{1}#3\ldots#3#1_{#2}}
\newcommand{\col}[1]{\renewcommand{\arraystretch}{0.4} \begin{array}[t]{c} #1
  \end{array}}
\newcommand{\prule}[2]{{\displaystyle #1 \over \displaystyle#2}}
\newcounter{ITEM}
\newcommand{\newITEM}[1]{\gdef\ITEMlabel{ITEM:#1-}\setcounter{ITEM}{0}}
\makeatletter
\newcommand{\Not}[1]{\mathbin {\mathpalette\c@ncel#1}}
\def\LabeL#1$#2{\edef\@currentlabel{#2}\label{#1}}
\newcommand{\ITEM}[2]{\par\addvspace{.7ex}\noindent
   \refstepcounter{ITEM}\expandafter\LabeL\ITEMlabel#1${(\roman{ITEM})}%
   {\advance\linewidth-2em \hskip2em %
   \parbox{\linewidth}{\hskip-2em {\rm\bf \@currentlabel\
   }\ignorespaces #2}}\par \addvspace{.7ex}\noindent\ignorespaces}
\def\R@f#1${\ref{#1}}
\newcommand{\?}[1]{\expandafter\R@f\ITEMlabel#1$}
\makeatother
\newcommand{\PROOFRULE}[2]{\trivlist\item[\hskip\labelsep {\bf #1}]#2\par
  \addvspace{1ex}\noindent\ignorespaces}
\newcommand{\PRULE}[2]{\displaystyle#1 \strut \over \strut \displaystyle#2}
%\setlength{\clauselength}{6cm}
%% \newcommand{\clause}[3]{\par\addvspace{.7ex}\noindent\LabeL#2${{\rm\bf #1}}%
%%   {\advance\linewidth-3em \hskip 3em
%%    \parbox{\linewidth}{\hskip-3em \parbox{3em}{\rm\bf#1.}#3}}\par 
%%    \addvspace{.7ex}\noindent\ignorespaces}
\newcommand{\clause}[3]{\par\addvspace{.7ex}\noindent
  {\advance\linewidth-3em \hskip 3em
   \parbox{\linewidth}{\hskip-3em \parbox{3em}{\rm\bf#1.}#3}}\par 
   \addvspace{.7ex}\noindent\ignorespaces}
\newcommand{\Cs}{\varepsilon}
\newcommand{\const}[3]{\Cs_{\scriptscriptstyle#2}(#1,#3)}
\newcommand{\Ein}{\sqsubset}%
\newcommand{\Eineq}{\sqsubseteq}%


\voffset -1cm


%\begin{document}


\section{$NEQ_4$ and admissibility of ($cut$).}
%
The appplications of ($E_s$) and ($E_a$) substituting a variable 
%in the RHS of the active formula
%, and those of ($E_s$) where the modified term is 
%a variable making up a {\em whole} RHS of an inclusion 
will be called {\em degenerate}:
\begin{itemize}\MyLPar
\item $(E_{ax})$\ \ \prule{\Gamma,x\Incl z, y\Incl r(x)\Seq\Delta}{\Gamma,
y\Incl r(z)\Seq\Delta}\ \
 $z\in\Vars$\ \ \ \ ($r(x)=x$ is a special case)
\item $(E_{sx})$\ \ \prule{\Gamma,x\Incl z\Seq\Delta}{\Gamma\Seq\Delta_z^x}\ \
 $z\in\Vars$
%\item $(E_{Rsd})$\ \ \prule{\Gamma,x\Incl t\Seq\Delta,p\preceq x}{\Gamma\Seq\Delta,p\preceq t}
\end{itemize}
These are excluded from $NEQ_4$. On the other hand, applications of 
various rules replacing a non-variable term will be called {\em 
non-reduced}. We also divide the applications of the 
equality rules explicitly into those modifying the LHS or RHS of 
inclusions. The a-equality rule modifying another equality has been 
removed.
Only reduced applications of the a-rule for modifying an LHS, $\eLar$,
are included in $NEQ_{4}$. 
The simple cut rule of $NEQ_{3}$ removing $t\Incl t$ is replaced by the 
elementary cut removing only variable $x\Incl x$.
The rules of $NEQ_{4}$ are shown in figure~\ref{fi:neq4}.


\documentstyle[a4wide,10pt]{article}
%% \makeatletter

%%%\setlength{\textwidth}{17cm}
%%%\setlength{\evensidemargin}{-.8cm}
%%%\setlength{\oddsidemargin}{-.8cm}

%% \makeatother
%\makeatletter
%\show\
%\makeatother
\newcommand{\ite}[1]{\item[{\bf #1.}]}
\newcommand{\app}{\mathrel{\scriptscriptstyle{\vdash}}}
\newcommand{\estr}{\varepsilon}
\newcommand{\PSet}[1]{{\cal P}(#1)}
\newcommand{\ch}{\sqcup}
\newcommand{\into}{\to}
\newcommand{\Iff}{\Leftrightarrow}
\renewcommand{\iff}{\leftrightarrow}
\newcommand{\prI}{\vdash_I}
\newcommand{\pr}{\vdash}
\newcommand{\ovr}[1]{\overline{#1}}

\newcommand{\cp}{{\cal O}}

% update function/set
%\newcommand{\upd}[3]{#1\!\Rsh^{#2}_{\!\!#3}} % AMS
\newcommand{\upd}[3]{#1^{\raisebox{.5ex}{\mbox{${\scriptscriptstyle{\leftarrow}}\scriptstyle{#3}$}}}_{{\scriptscriptstyle{\rightarrow}}{#2}}} 
\newcommand{\rem}[2]{\upd{#1}{#2}{\bullet}}
\newcommand{\add}[2]{\upd {#1}{\bullet}{#2}}
%\newcommand{\mv}[3]{{#1}\!\Rsh_{\!\!#3}{#2}}
\newcommand{\mv}[3]{{#1}\:\raisebox{-.5ex}{$\stackrel{\displaystyle\curvearrowright}{\scriptstyle{#3}}$}\:{#2}}

\newcommand{\leads}{\rightsquigarrow} %AMS

\newenvironment{ites}{\vspace*{1ex}\par\noindent 
   \begin{tabular}{r@{\ \ }rcl}}{\vspace*{1ex}\end{tabular}\par\noindent}
\newcommand{\itt}[3]{{\bf #1.} & $#2$ & $\impl$ & $#3$ \\[1ex]}
\newcommand{\itte}[3]{{\bf #1.} & $#2$ & $\impl$ & $#3$ }
\newcommand{\itteq}[3]{\hline {\bf #1} & & & $#2=#3$ }
\newcommand{\itteqc}[3]{\hline {\bf #1} &  &  & $#2=#3$ \\[.5ex]}
\newcommand{\itteqq}[3]{{\bf #1} &  &  & $#2=#3$ }
\newcommand{\itc}[2]{{\bf #1.} & $#2$ &    \\[.5ex]}
\newcommand{\itcs}[3]{{\bf #1.} & $#2$ & $\impl$ & $#3$  \\[.5ex] }
\newcommand{\itco}[3]{   & $#1$ & $#2$  & $#3$ \\[1ex]}
\newcommand{\itcoe}[3]{   & $#1$ & $#2$  & $#3$}
\newcommand{\bit}{\begin{ites}}
\newcommand{\eit}{\end{ites}}
\newcommand{\na}[1]{{\bf #1.}}
\newenvironment{iten}{\begin{tabular}[t]{r@{\ }rcl}}{\end{tabular}}
\newcommand{\ass}[1]{& \multicolumn{3}{l}{\hspace*{-1em}{\small{[{\em Assuming:} #1]}}}}

%%%%%%%%% nested comp's
\newenvironment{itess}{\vspace*{1ex}\par\noindent 
   \begin{tabular}{r@{\ \ }lllcl}}{\vspace*{1ex}\end{tabular}\par\noindent}
\newcommand{\bitn}{\begin{itess}}
\newcommand{\eitn}{\end{itess}}
\newcommand{\comA}[2]{{\bf #1}& $#2$ \\ }
\newcommand{\comB}[3]{{\bf #1}& $#2$ & $#3$\\ }
\newcommand{\com}[3]{{\bf #1}& & & $#2$ & $\impl$ & $#3$\\[.5ex] }

\newcommand{\comS}[5]{{\bf #1} 
   & $#2$ & $#3$ & $#4$ & $\impl$ & $#5$\\[.5ex] }

%%%%%%%%%%%%%%%%
\newtheorem{CLAIM}{Proposition}[section]
\newtheorem{COROLLARY}[CLAIM]{Corollary}
\newtheorem{THEOREM}[CLAIM]{Theorem}
\newtheorem{LEMMA}[CLAIM]{Lemma}
\newcommand{\MyLPar}{\parsep -.2ex plus.2ex minus.2ex\itemsep\parsep
   \vspace{-\topsep}\vspace{.5ex}}
\newcommand{\MyNumEnv}[1]{\trivlist\refstepcounter{CLAIM}\item[\hskip
   \labelsep{\bf #1\ \theCLAIM\ }]\sf\ignorespaces}
\newenvironment{DEFINITION}{\MyNumEnv{Definition}}{\par\addvspace{0.5ex}}
\newenvironment{EXAMPLE}{\MyNumEnv{Example}}{\nopagebreak\finish}
\newenvironment{PROOF}{{\bf Proof.}}{\nopagebreak\finish}
\newcommand{\finish}{\hspace*{\fill}\nopagebreak 
     \raisebox{-1ex}{$\Box$}\hspace*{1em}\par\addvspace{1ex}}
\renewcommand{\abstract}[1]{ \begin{quote}\noindent \small {\bf Abstract.} #1
    \end{quote}}
\newcommand{\B}[1]{{\rm I\hspace{-.2em}#1}}
\newcommand{\Nat}{{\B N}}
\newcommand{\bool}{{\cal B}{\rm ool}}
\renewcommand{\c}[1]{{\cal #1}}
\newcommand{\Funcs}{{\cal F}}
%\newcommand{\Terms}{{\cal T}(\Funcs,\Vars)}
\newcommand{\Terms}[1]{{\cal T}(#1)}
\newcommand{\Vars}{{\cal V}}
\newcommand{\Incl}{\mathbin{\prec}}
\newcommand{\Cont}{\mathbin{\succ}}
\newcommand{\Int}{\mathbin{\frown}}
\newcommand{\Seteq}{\mathbin{\asymp}}
\newcommand{\Eq}{\mathbin{\approx}}
\newcommand{\notEq}{\mathbin{\Not\approx}}
\newcommand{\notIncl}{\mathbin{\Not\prec}}
\newcommand{\notCont}{\mathbin{\Not\succ}}
\newcommand{\notInt}{\mathbin{\Not\frown}}
\newcommand{\Seq}{\mathrel{\mapsto}}
\newcommand{\Ord}{\mathbin{\rightarrow}}
\newcommand{\M}[1]{\mathbin{\mathord{#1}^m}}
\newcommand{\Mset}[1]{{\cal M}(#1)}
\newcommand{\interpret}[1]{[\![#1]\!]^{A}_{\rho}}
\newcommand{\Interpret}[1]{[\![#1]\!]^{A}}
%\newcommand{\Comp}[2]{\mbox{\rm Comp}(#1,#2)}
\newcommand{\Comp}[2]{#1\diamond#2}
\newcommand{\Repl}[2]{\mbox{\rm Repl}(#1,#2)}
%\newcommand\SS[1]{{\cal S}^{#1}}
\newcommand{\To}[1]{\mathbin{\stackrel{#1}{\longrightarrow}}}
\newcommand{\TTo}[1]{\mathbin{\stackrel{#1}{\Longrightarrow}}}
\newcommand{\oT}[1]{\mathbin{\stackrel{#1}{\longleftarrow}}}
\newcommand{\oTT}[1]{\mathbin{\stackrel{#1}{\Longleftarrow}}}
\newcommand{\es}{\emptyset}
\newcommand{\C}[1]{\mbox{$\cal #1$}}
\newcommand{\Mb}[1]{\mbox{#1}}
\newcommand{\<}{\langle}
\renewcommand{\>}{\rangle}
\newcommand{\Def}{\mathrel{\stackrel{\mbox{\tiny def}}{=}}}
\newcommand{\impl}{\mathrel\Rightarrow}
\newcommand{\then}{\mathrel\Rightarrow}
\newfont{\msym}{msxm10}

\newcommand{\false}{\bot}
\newcommand{\true}{\top}

\newcommand{\restrict}{\mathbin{\mbox{\msym\symbol{22}}}}
\newcommand{\List}[3]{#1_{1}#3\ldots#3#1_{#2}}
\newcommand{\col}[1]{\renewcommand{\arraystretch}{0.4} \begin{array}[t]{c} #1
  \end{array}}
\newcommand{\prule}[2]{{\displaystyle #1 \over \displaystyle#2}}
\newcounter{ITEM}
\newcommand{\newITEM}[1]{\gdef\ITEMlabel{ITEM:#1-}\setcounter{ITEM}{0}}
\makeatletter
\newcommand{\Not}[1]{\mathbin {\mathpalette\c@ncel#1}}
\def\LabeL#1$#2{\edef\@currentlabel{#2}\label{#1}}
\newcommand{\ITEM}[2]{\par\addvspace{.7ex}\noindent
   \refstepcounter{ITEM}\expandafter\LabeL\ITEMlabel#1${(\roman{ITEM})}%
   {\advance\linewidth-2em \hskip2em %
   \parbox{\linewidth}{\hskip-2em {\rm\bf \@currentlabel\
   }\ignorespaces #2}}\par \addvspace{.7ex}\noindent\ignorespaces}
\def\R@f#1${\ref{#1}}
\newcommand{\?}[1]{\expandafter\R@f\ITEMlabel#1$}
\makeatother
\newcommand{\PROOFRULE}[2]{\trivlist\item[\hskip\labelsep {\bf #1}]#2\par
  \addvspace{1ex}\noindent\ignorespaces}
\newcommand{\PRULE}[2]{\displaystyle#1 \strut \over \strut \displaystyle#2}
%\setlength{\clauselength}{6cm}
%% \newcommand{\clause}[3]{\par\addvspace{.7ex}\noindent\LabeL#2${{\rm\bf #1}}%
%%   {\advance\linewidth-3em \hskip 3em
%%    \parbox{\linewidth}{\hskip-3em \parbox{3em}{\rm\bf#1.}#3}}\par 
%%    \addvspace{.7ex}\noindent\ignorespaces}
\newcommand{\clause}[3]{\par\addvspace{.7ex}\noindent
  {\advance\linewidth-3em \hskip 3em
   \parbox{\linewidth}{\hskip-3em \parbox{3em}{\rm\bf#1.}#3}}\par 
   \addvspace{.7ex}\noindent\ignorespaces}
\newcommand{\Cs}{\varepsilon}
\newcommand{\const}[3]{\Cs_{\scriptscriptstyle#2}(#1,#3)}
\newcommand{\Ein}{\sqsubset}%
\newcommand{\Eineq}{\sqsubseteq}%


\voffset -1cm


\begin{document}
%\ \ \hspace*{1em}\ \  \hfill{\rabove
\begin{figure}[hbt]
\hspace*{2em}
\begin{tabular}{|r@{\ }l@{\ \ \ \ \ \ \ \ \ \ }r@{\ }ll|}
\multicolumn{5}{r}{{\small{\today}}} \\
\hline
\multicolumn{4}{|c}{{\bf Axioms}:} & \\[1ex]
\multicolumn{4}{|c}{$\begin{array}{l@{\ \ \ \ \ \ }l@{\ \ \ \ \ \ }r}
x\Incl t,\GSD, x\Incl t & s= t,\GSD, s= t & \GSD,x=x 
%%% \\[1ex]
%%% x\Incl y,\GSD,x=y & x=y,\GSD,x\Incl y & x\Incl y,\GSD,y\Incl x
\end{array}$} & \\[4ex]
%
\multicolumn{4}{|c}{{\bf Identity rules}:} & \\[1ex]
($=_{Ls}$) &
 \prule{t=s,\Gamma\Seq\Delta,p(s)\preceq q}{t=s,\Gamma\Seq\Delta,p(t)\preceq q} & 
 ($=_{Lar}$) & 
  \prule{x=t, p(x)\Incl q, \Gamma\Seq\Delta}{x=t, p(t)\Incl q,\Gamma\Seq\Delta} &\\[2ex]
% {\footnotesize - $t\not\in\Vars$} & & & \\[1ex]
($=_{Rs}$) & %\prule{t=s,\GSD,p\Incl q(s)}{t=s,\GSD,p\Incl q(t)} 
   & ($=_{Ra}$) &
 \prule{s=t, p\Incl q(s),\Gamma\Seq\Delta}{s=t,p\Incl q(t),\Gamma\Seq\Delta} & \\[2ex]
  & && {\footnotesize ($p\not\in\Vars$ needed for $\Incl_{Ra}$)}   & \\[1ex]
%
\multicolumn{4}{|c}{{\bf Inclusion rules}:} & \\[1ex]
($\Incl_{Ls}$) & \prule{s\Incl t,\GSD, p(t)\preceq q}{s\Incl t, \GSD, p(s)\preceq q}  &
     ($\Incl_{Lar}$) & 
     \prule{x\Incl t, p(x)\preceq q,\GSD}{x\Incl t, p(t)\preceq q,\GSD} & \\[2ex]
  & %would for Lsr: {\footnotesize - $s\not\in\Vars$} 
        && {\footnotesize ($t\in\Vars$ needed for $\preceq_{Lsx}$)}   & \\[1ex]
($\Incl_{Rs}$) & \prule{s\Incl t, \GSD,p\Incl q(s)}{s\Incl t,\GSD,p\Incl q(t)}  & 
   ($\Incl_{Ra}$) & & \\[3ex]
%would for Rsr: & {\footnotesize - $q(t)\not\in\Vars$} & & & \\[2ex]
%
\multicolumn{4}{|c}{{\bf Elimination rules}:} & \\[1ex]
($E_s$) & \prule{x\Incl t,\GSD,\phi[x]}{\GSD,\phi[t]}  &
($E_a$) & \prule{x\Incl t, y\Incl q(x),\GSD}{y\Incl q(t),\GSD} & \\[.5ex]
&  {\footnotesize - $x\not\in \Vars(\Gamma,t)$, $t\not\in\Vars$} &  
   &  {\footnotesize - $x\not\in \Vars(t,\Gamma,\Delta,y)$, $t\not\in\Vars$} & \\
& {\footnotesize - at most one $x$ in $\Delta$;} &  
   & {\footnotesize - at most one $x$ in $q(x)$} & \\
&  {\footnotesize - $\phi\not= p\preceq x$} &  && \\[1ex]
%
\multicolumn{4}{|c}{{\bf Elementary cut}:} & \\[1ex]
\multicolumn{4}{|c}{($cut_x$)\ \ \prule{x\Incl x,\Gamma\Seq\Delta}{\Gamma\Seq\Delta} } &
%%  ($cut_=$) & \prule{x= x,\Gamma\Seq\Delta}{\Gamma\Seq\Delta} & 
  \\[3ex]
\multicolumn{4}{|c}{{\bf Specific cut rules}:} & \\
\multicolumn{4}{|c}
{for each specific axiom $\Ax_k$: \(a_1,...,a_n\Seq s_1,...,s_m\), 
a  rule:} & \\[1ex]
\multicolumn{4}{|c}
{\prule{\Gamma\Seq\Delta,a_1\ ;...;\ \Gamma\Seq\Delta,a_n\ ;\ 
s_1,\Gamma\Seq\Delta\ ;...;\ s_m,\Gamma\Seq\Delta} 
{\Gamma\Seq\Delta}\ \ \ ($Sp.cut_k$)} & \\[1ex]
 \hline
\end{tabular} 
\caption{The rules of $NEQ_{4}$ ($x,y\in\Vars$)} %\label{fi:neq4}
\end{figure}

\noindent
\begin{enumerate}\MyLPar
\item Any derivation $D$ can be transformed into $D^*$ so that $D^*$ contains \\
\begin{tabular}{r@{:\ \ }l}
either & no $(=_{Lsx}), (\Incl_{Lsx}), (\Incl_{Rs})$ \\
or & no $(\Incl_{Larx})$ ... ? ... $(=_{Larx})$
\end{tabular}
\item $(=_{a=}), (\Incl_{Lanr}), (=_{Ranr}), (=_{Lanr}), (\Incl_{Ra})$ are admissible
without increasing  $\#(\Incl_{Lar} + =_{Lar})$ 
\item $(cut_t)$ and $(cut_{x=})$ are admissible
\end{enumerate}

\end{document}
%% \begin{REMARK}\label{re:cutx}
%% Actually, a more restricted
%% version of the elementary cut rule would suffice: 
%% \begin{center} \prule{x\Incl x,
%% x=p(r), \GSD}{x=p(r),\GSD} \end{center}
%% However, the more general formulation here makes
%% the proofs in this section easier, and the rule will be eliminated anyway in the
%% final version of the calculus in the next section.
%% \end{REMARK}
\noindent
The essential facts in the rest of this section are: 
\begin{enumerate}\MyLPar
\item The possibility of transforming $NEQ_{4}$ 
derivations in two different ways eliminating either some undesirable 
applications of a-rules or else of some s-rules. These proofs are in 
subsection~\ref{sub:der}. 
\item Admissibility of the rules 
excluded from $NEQ_{3}$ leading to the equivalence of the two 
calculi. These proofs are partly in subsection~\ref{sub:der} -- the 
crucial proofs of admissibility of the simple cut rule and
%redundancy of the $(cut_{x=})$ rule, as well as 
equivalence of the two calculi are in subsection \ref{sub:equiv}. 
\item Admissibility of $(cut)$ in $NEQ_{4}$ is 
shown in subsection~\ref{sub:cut}. The proof is based on the  
possibility of modifying $NEQ_{4}$ derivations as in point 1 above.
\end{enumerate}.

\subsection{Some facts about $NEQ_{4}$ derivations}\label{sub:der}

\begin{LEMMA}\label{fa:noeRs}
$\der{NEQ_4}DS\ \impl\ \der{NEQ_4}{D^*}S$
and $D^*$  does not contain $\eRs$.
%[Moreover, $D^*$ does not introduce any new applications of s-rules.
%This fact will be combined with lemma~\ref{le:nosx}.]
\end{LEMMA}
\begin{PROOF}
%By lemmas~\ref{le:noeqeq}-\ref{le:noLanr} we may assume that $D$ contains no
%applications of ($=_{a=}$), ($\Incl_{anr}$) or ($=_{Lanr}$).
By induction on $\<\#(\eRs,D),h(D)\>$.
For the axioms, we only have to consider the one with inclusion in the consequent.
%\begin{itemize}
%\item $x\Incl t,\GSD,x\Incl t$
%\end{itemize}
\[
\begin{array}{rl}
p=s, x\Incl t(s),\GSD,x\Incl t(s) \\ \cline{1-1}
p=s, x\Incl t(s),\GSD,x\Incl t(p) & \rabove{\eRs}
\end{array}
\conv
\begin{array}{rl}
p=s, x\Incl t(p),\GSD,x\Incl t(p) \\ \cline{1-1}
p=s, x\Incl t(s),\GSD,x\Incl t(p) & \rabove{\eRa}
\end{array}
\]
Consider the uppermost application of ($=_{Rs}$) in $D$:
\[ \begin{array}{cl}
\vdots          & \raisebox{-1.2ex}[1.5ex][0ex]{$R$} \\ \cline{1-1}
r=s, \Gamma\Seq\Delta, q\Incl t(r) &
\raisebox{-1.2ex}[1.5ex][0ex]{($=_{Rs}$)} \\ \cline{1-1}
r=s, \Gamma\Seq\Delta, q\Incl t(s) 
\end{array} \]
\begin{LS}
\item\label{it:RsEsA} $R=\Esr$ with $q\Incl t(r)$ as the modified formula and $r$ as
modified term. We have two subcases corresponding to the situation when the
term $r'$ substituted by $\Esr$ for its eigen-variable is a subterm or
superterm of (or equal to) $r$. (When the two are independent, the case is
trivial.) 
\begin{LSA}
\item  $r'$ is a subterm of $r$,\\
i.e., $t'(r')$ and $t(r)$ are identical with $r=f(r')$ and $t'(x)=t(f(x))$.
\[ D \left \{\begin{array}{cl}
%\vdots          \\
x\Incl r', f(r')=s, \GSD, q\Incl t(f(x)) & \raisebox{-1.2ex}[1.5ex][0ex]{$\Esr$} \\ \cline{1-1}
f(r')=s, \GSD, q\Incl t(f(r')) &
\raisebox{-1.2ex}[1.5ex][0ex]{($=_{Rs}$)} \\ \cline{1-1}
f(r')=s, \GSD, q\Incl t(s) 
\end{array} \right . \convd
D' \left \{ \begin{array}{cl}
x\Incl r', f(r')=s, f(x)=s, \Gamma\Seq\Delta,q\Incl t(f(x)) & \raisebox{-1.2ex}[1.5ex][0ex]{($=_{Rs}$)} \\ \cline{1-1}
x\Incl r', f(r')=s, f(x)=s, \Gamma\Seq\Delta, q\Incl t(s) &
\raisebox{-1.2ex}[1.5ex][0ex]{$\iLa$} \\ \cline{1-1}
x\Incl r', f(r')=s, f(r')=s, \Gamma\Seq\Delta, q\Incl t(s) &
\raisebox{-1.2ex}[1.5ex][0ex]{$\Esr$} \\ \cline{1-1}
f(r')=s, \Gamma\Seq\Delta, q\Incl t(s) 
\end{array} \right . \]
Induction hypothesis allows us to eliminate the application of ($=_{Rs}$)
from $D'$ since it occurs higher than in $D$.
\item $r'$ is a superterm of (or equal to) $r$,
i.e., $r'=f(r)$ and $t(x)=t'(f(x))$. \\ On the right we have the case when
$r'=r$ and $s$ is a variable $y$
\[ \sma{\begin{array}{c@{\hspace*{5em}}c}%D \left \{
\begin{array}[t]{cl}
%\vdots          \\
x\Incl f(r), r=s, \Gamma\Seq\Delta, q\Incl t'(x) & \\ \cline{1-1}
r=s, \Gamma\Seq\Delta, q\Incl t'(f(r)) & \rabove{\Esr} \\ \cline{1-1}
r=s, \Gamma\Seq\Delta, q\Incl t'(f(s)) & \rabove{\eRs} 
\end{array} %\right . 
& 
\prar{
x\Incl r, r=y, \GSD, q\Incl t'(x) \cl
          r=y, \GSD, q\Incl t'(r) & \rabove{\Esr} \cl
          r=y, \GSD, q\Incl t'(y) & \rabove{\eRs} 
} \\[.5ex]
\Downarrow & \Downarrow \\[.5ex]
%\convd %D \left \{
\begin{array}[t]{cl}
x\Incl f(r), r=s, \Gamma\Seq\Delta, q\Incl t'(x) & \\ \cline{1-1}
x\Incl f(s), r=s, \Gamma\Seq\Delta, q\Incl t'(x) & \rabove{\eRa}
\\ \cline{1-1}
r=s, \Gamma\Seq\Delta, q\Incl t'(f(s)) & \rabove{\Esr} 
\end{array} %\right . 
&
\prar{
x\Incl r, r=y, \GSD, q\Incl t'(x) \cl
x\Incl y, r=y, \GSD, q\Incl t'(x) & \rabove{\eRa} \cl
x\Incl y, r=y, \GSD, q\Incl t'(y) & \rabove{\iRs} \cl
x\Incl r, r=y, \GSD, q\Incl t'(y) & \rabove{\eRa} \cl
          r=y, \GSD, q\Incl t'(y) & \rabove{\Esr}
}
\end{array} }
\]
The case when the modified term is substituted into LHS of $\Incl$ allows simple
swap since then $\Esr$ and $\eRs$ modify different terms.%is treated analogously 
\end{LSA}
\item $R=\iLa$ with the modified formula $r=s(v)$ and $v$ as the modified
term:
\[ D \left \{ \begin{array}{cl}
 D_1 \left \{ \begin{array}{c}
               \vdots       \\ 
               r=s(u), u\Incl v, \Gamma\Seq\Delta,q\Incl t(r) 
           \end{array} \right . 
         & \raisebox{-3.2ex}[1.5ex][0ex]{$\iLa$}  \\ \cline{1-1}
 r=s(v), u\Incl v, \Gamma\Seq\Delta, q\Incl t(r) &
 \raisebox{-1.2ex}[1.5ex][0ex]{($=_{Rs}$)} \\ \cline{1-1}
 r=s(v), u\Incl v, \Gamma\Seq\Delta, q\Incl t(s(v)) 
 \end{array} \right . \convd
  D^* \left \{ \begin{array}{cl}
  D' \left \{ \begin{array}{cl}
    D_1 \left \{ \begin{array}{cl}
 \vdots       \\ 
 r=s(u), u\Incl v, \Gamma\Seq\Delta,q\Incl t(r) 
  \end{array} \right . & \raisebox{-3.2ex}[1.5ex][0ex]{($W_a$)}  \\
  \cline{1-1}
 r=s(u), r=s(v), u\Incl v, \Gamma \Seq \Delta, q\Incl t(r)
  & \raisebox{-1.2ex}[1.5ex][0ex]{($=_{Rs}$)} \\ \cline{1-1}
 r=s(u), r=s(v), u\Incl v, \Gamma\Seq\Delta, q\Incl t(s(v))  \end{array}
 \right . &  \\ \cline{1-1}
 r=s(v), r=s(v), u\Incl v, \Gamma\Seq\Delta, q\Incl t(s(v)) 
 & \raisebox{1.2ex}[1.5ex][0ex]{$\iLa$} 
 \end{array} \right . \]
  $h(D') < h(D)$, so $D^*$
  can be transformed into a desired derivation without ($=_{Rs}$). (Notice
  that the result of $D^*$ is the same as the result of $D$ by the implicit
  application of contraction.)
 \item $R=\iLs$ :
 \[ D \left \{ \begin{array}{cl}
  D_1 \left \{ \begin{array}{c}
               \vdots       \\ 
               r=s, q\Incl u, \Gamma\Seq\Delta, u\Incl t(r) 
            \end{array} \right . 
          & \raisebox{-3.2ex}[1.5ex][0ex]{$\iLs$}  \\ \cline{1-1}
 r=s, q\Incl u, \Gamma\Seq\Delta, q\Incl t(r) &
 \raisebox{-1.2ex}[1.5ex][0ex]{($=_{Rs}$)} \\ \cline{1-1}
 r=s, q\Incl u, \Gamma\Seq\Delta, q\Incl t(s) 
 \end{array} \right . \convd
  D^* \left \{ \begin{array}{cl}
  D' \left \{ \begin{array}{cl}
    D_1 \left \{ \begin{array}{cl}
 \vdots       \\ 
 r=s, q\Incl u, \Gamma\Seq\Delta, u\Incl t(r) 
  \end{array} \right . & \raisebox{-3.2ex}[1.5ex][0ex]{($=_{Rs}$)}  \\
  \cline{1-1}
 r=s, r=s(v), q\Incl u, \Gamma \Seq \Delta, u\Incl t(s) \end{array}
 \right . &  \\ \cline{1-1}
 r=s, q\Incl u, \Gamma\Seq\Delta, q\Incl t(s) 
 & \raisebox{1.2ex}[1.5ex][0ex]{$\iLs$} 
 \end{array} \right . \]
 Since $h(D')<h(D)$, the induction hypothesis yields a desired derivation.
 \item $R=\iRs$
\[\sma{\prarc{
f(p)=s, r\Incl p,\GSD, t\Incl q(f(r)) \cl
f(p)=s, r\Incl p,\GSD, t\Incl q(f(p)) & \rabove{\iRs} \cl
f(p)=s, r\Incl p,\GSD, t\Incl q(s) & \rabove{\eRs}
}
\conv
\prarc{
f(r)=s,f(p)=s, r\Incl p,\GSD, t\Incl q(f(r)) \cl
f(r)=s,f(p)=s, r\Incl p,\GSD, t\Incl q(s)  & \rabove{\eRs} \cl
f(p)=s,f(p)=s, r\Incl p,\GSD, t\Incl q(s)  & \rabove{\iLa} 
} }
\]
\[\sma{ \prarc{
p\Incl f(s), s=r, \GSD, t\Incl q(p) \cl
p\Incl f(s), s=r, \GSD, t\Incl q(f(s)) &\rabove{\iRs} \cl
p\Incl f(s), s=r, \GSD, t\Incl q(f(r)) &\rabove{\eRs} 
}
\conv
\prarc{
p\Incl f(r), p\Incl f(s), s=r, \GSD, t\Incl q(p) \cl
p\Incl f(r), p\Incl f(s), s=r, \GSD, t\Incl q(f(r)) &\rabove{\iRs} \cl
p\Incl f(s), p\Incl f(s), s=r, \GSD, t\Incl q(f(r)) &\rabove{\eRa}
} }
\]
 \item The other cases when $R$ is $\Ear$, $\eLs$, ($=_{Ra}$), ($=_{La}$), 
 ($Sp.cut$) or ($cut_t$) can be trivially swapped 
 with the application of ($=_{Rs}$) since they do not affect the active
 equation nor the modified inclusion of ($=_{Rs}$).
 \end{LS}
%% As we see, $D^*$ involves the same applications of the s-rules
%% as the original derivation $D$.
%Also,
% since $\#(\preceq_{anr},D)=0$ and no new such applications occurred, we do get
% $\#(\preceq_{anr},D^*)=0$.]
 \end{PROOF}

%\newpage
\subsubsection{Some admissible and redundant a-rules}
The underlying theme of this subsection are the reduced vs. non-reduced applications
of various a-rules. This facts, to be utilized in the proof of the $(cut)$-elimination,
appear in the side-conditions of the lemmas.
\begin{LEMMA}\label{le:noeqeq}
The following rule is admissible in $NEQ_4$:
\[
\mprule{s=t, f(s)=p, \GSD}{s=t, f(t)=p,\GSD}\ \ \eae
\]
\noindent
Morover, if $D^*$ derives $S$ using this
rule while $D$ derives $S$ without it then
$\#(\pLar,D^*)\leq \#(\pLar,D)$,
and $\#(\pLanr,D^*)\leq \#(\pLanr,D)$.
\end{LEMMA}
\begin{PROOF}
By lemma \ref{fa:noeRs} we assume no applications of $\eRs$ in $D$.
By induction on $\#(\eae,D)$ and $h(D)$. 
For $h(D)=1$ we have one possible axiom:
\[ \begin{array}{rl}
 s=t, f(s)=p,\GSD, f(s)=p \\ \cline{1-1}
 s=t, f(t)=p,\GSD, f(s)=p & \rabove{\eae}
\end{array} 
\conv
\begin{array}{rl}
 s=t, f(t)=p,\GSD, f(t)=p \\ \cline{1-1}
 s=t, f(t)=p,\GSD, f(s)=p & \rabove{\eLs}
\end{array}
\]
%
Consider the uppermost application of
$\eae$ and the rule $R$ applied just above it.  The elimination rules, 
$\iLs$, $\iRs$, ($Sp.cut$) and ($cut_x$) 
do not affect
the active or the principal formula of $\eae$ so these can be swapped
reducing the height at which the application of $\eae$ occurs. We have
then the following cases for $R$:
\begin{LS}
\item $\eLar$:
\[ D \left\{ \begin{array}{rl}
 D_1 \left\{ \begin{array}{r}
\multicolumn{1}{c}{\vdots} \\
r=x, x=t, s(x)\Incl p, \GSD \end{array} \right. \\ \cline{1-1}
r=x, x=t, s(t)\Incl p, \GSD & \rabove{\eLar} \\ \cline{1-1}
r=x, r=t, s(t)\Incl p, \GSD & \rabove{\eae} 
\end{array} \right.
\convd
D^* \left\{ \begin{array}{rl}
       D^*_1\left\{ \begin{array}{r}
       \multicolumn{1}{c}{\vdots} \\
       x\Incl t, r=x,x=t,s(x)\Incl p,\GSD \end{array} \right . \\ \cline{1-1}
x\Incl t, r=x, r=t,s(x)\Incl p,\GSD & \rabove{\eae} \cl
x\Incl t, r=x, r=t,s(t)\Incl p,\GSD &\rabove{\iLar} \cl
x\Incl r, r=x, r=t,s(t)\Incl p,\GSD &\rabove{\eRa} \cl
x\Incl x, r=x, r=t,s(t)\Incl p,\GSD &\rabove{\eRa} \cl
r=x, r=t, s(t)\Incl p,\GSD & \raisebox{1.2ex}[1.5ex][0ex]{($cut_x$)}
\end{array} \right .
\]
\noindent $h(D^*_1)< h(D_1)+1$ yields the conclusion. Exactly the same transformation 
is made when $\eae$ in $D$ modifies $t$ (or its subterm).
Since the transformation replaces an application of $\eLar$ by an application of 
$\iLar$, we have the unchanged total number of applications of $\pLar$ and $\pLanr$.

\item $\eLs$: The only case when simple swapping of the last two rules is
insufficient occurs when
the active formula of $\eLs$ is modified in the last application of
$\eae$. 
The end of the derivation has then the form:
%
\[D \left\{ \begin{array}{cl}
%\vdots \\
s=t, w(t)=q,\Gamma \Seq \Delta, p(q)=r  &  \\ \cline{1-1}
s=t, w(t)=q,\Gamma \Seq \Delta, p(w(t))=r & \rabove{\eLs} \\ \cline{1-1}
s=t, w(s)=q, \Gamma \Seq\Delta, p(w(t))=r & \rabove{\eae} 
\end{array} \right. \convd
%
%Nevertheless, swapping the applications of ($=_a$) and ($=_s$) we may obtain:
%
 D^* \left \{ \begin{array}{cl}
D_1 \left\{ \begin{array}{rl}
%\multicolumn{1}{c}{\vdots} \\
s=t, w(t)=q, \Gamma \Seq\Delta, p(q)=r  & \\ \cline{1-1}
s=t, w(s)=q,\Gamma \Seq\Delta, p(q)=r  & \rabove{\eae}
\end{array} \right.
\\ \cline{1-1}
s=t, w(s)=q, \Gamma \Seq\Delta, p(w(s))=r & \rabove{\eLs} \\ \cline{1-1}
s=t, w(s)=q, \Gamma \Seq\Delta, p(w(t))=r & \rabove{\eLs} 
\end{array}  \right . \]
%
$h(D_1) < h(D)$ so, by the induction hypothesis, there is a $NEQ_4$ derivation
corresponding to $D^*$ deriving the same  sequent without $\eae$. 
The case when in  $D$,$\eLs$ replaces $w(t)$ by $q$ is treated analogously.
%
\item $\iLa$ -- two cases%($\Incl_a$):
\[D \left\{ \begin{array}{rl} D_1\left\{\begin{array}{c} \vdots \\
s=t, w(p)=q, p\Incl r(t),\Gamma \Seq \Delta \end{array}\right. & \\ \cline{1-1}
s=t, w(r(t))=q, p\Incl r(t),\Gamma \Seq \Delta  &   
         \rabove{\iLa} \\ \cline{1-1}
s=t, w(r(s))=q, p\Incl r(t),\Gamma \Seq\Delta & \rabove{\eae} 
\end{array} \right. \convd
%
%We construct $D^*$:
D^* \left\{ \begin{array}{rl} D^*_1\left\{ \begin{array}{c} \vdots \\
s=t, w(p)=q, p\Incl r(t), p\Incl r(s),\Gamma \Seq \Delta\end{array}\right. \\ \cline{1-1}
s=t, w(r(s))=q, p\Incl r(t), p\Incl r(s), \Gamma \Seq \Delta  & 
   \rabove{\iLa} \\ \cline{1-1}
s=t, w(r(s))=q, p\Incl r(t),p\Incl r(t),\Gamma \Seq\Delta  &
   \rabove{\eRa}
\end{array} \right. \]
%
where the last sequent is the same as in $D$ by implicit contraction. The last sequent 
of $D^*_1$ can be obtained as in $D_1$ by weakening (lemma~\ref{le:noweak})
with $h(D^*_1)=h(D_1)$.
\[\sma{\prarc{
s=t(r), w(t(p))=q, p\Incl r, \GSD \cl
s=t(r), w(t(r))=q, p\Incl r, \GSD & \rabove{\iLa} \cl
s=t(r), w(s)=q, p\Incl r, \GSD & \rabove{\eae}
}\conv
\prarc{
s=t(p), s=t(r), w(t(p))=q, p\Incl r, \GSD \cl
s=t(p), s=t(r), w(s)=q, p\Incl r, \GSD & \rabove{\eae} \cl
s=t(r), s=t(r), w(s)=q, p\Incl r, \GSD & \rabove{\iLa} 
} }
\]
%
\item ($=_{Ra}$):
\[ \sma{
\begin{array}{rl}
\multicolumn{1}{c}{\vdots} \\
r=u, w(u)=t, p\Incl f(w(u)), \GSD \\ \cline{1-1}
r=u, w(u)=t, p\Incl f(t), \GSD & \rabove{\eRa} \\ \cline{1-1}
r=u, w(r)=t, p\Incl f(t), \GSD & \rabove{\eae} \end{array}
\conv
\begin{array}{rl}
\multicolumn{1}{c}{\vdots} \\
r=u, w(u)=t, p\Incl f(w(u)), \GSD \\ \cline{1-1}
r=u, w(r)=t, p\Incl f(w(u)), \GSD & \rabove{\eae} \\ \cline{1-1}
r=u, w(r)=t, p\Incl f(w(r)), \GSD & \rabove{\eRa} \\ \cline{1-1}
r=u, w(r)=t, p\Incl f(t), \GSD & \rabove{\eRa} 
\end{array} }
\]
Analogous transformation is applied when $\eae$ modifies $t$ rather than $w(u)$.
%% \item $\eRs$
%% \[\prar{
%% p(s)=r, t=s,\GSD, q\Incl f(p(s)) \cl
%% p(s)=r, t=s,\GSD, q\Incl f(r) & \rabove{\eRs} \cl
%% p(t)=r, t=s,\GSD, q\Incl f(r) & \rabove{\eae} 
%% }\conv
%% \prar{
%% p(s)=r, t=s,\GSD, q\Incl f(p(s)) \cl
%% p(t)=r, t=s,\GSD, q\Incl f(p(s)) & \rabove{\eae} \cl
%% p(t)=r, t=s,\GSD, q\Incl f(p(t)) & \rabove{\eRs} \cl
%% p(t)=r, t=s,\GSD, q\Incl f(r) & \rabove{\eRs} 
%% }
%% \]
\item All other cases involve only $\Incl$ in the antecedent so they admit trivial swap.
\end{LS}
In all cases no new applications of $\pLa$ appear in the transformed derivations 
$D^*$, and the ones which occur there are the same as in the original derivations $D$.
\end{PROOF}
%
\begin{LEMMA}\label{le:noar}\label{le:noRanr} 
$\der{NEQ_4}DS\ \impl\ \der{NEQ_4}{D^*}S$
and $D^*$ contains no applications of $\iLanr$ nor $\eRanr$.
Moreover, $\#(\pLar,D^*)\leq\#(\pLar,D)$ and
 $\#(\pLanr,D^*)\leq\#(\pLanr,D)$.
\end{LEMMA}
\begin{PROOF}
By induction on $\<(\#\iLanr+\#\eRanr,D),h(D)\>$.
By lemma~\ref{fa:noeRs}, we assume that the derivation $D$ does not 
contain any $\eRs$.
We first consider the case where the uppermost application of any of the
two rules is $\iLanr$.

%, where $n$ is the number of
%applications of ($\Incl_{anr}$) in $D$ and $h(D)$ is the height of $D$. 
The basis
case of $h(D)=1$ is as follows:
\[ \begin{array}{rl}
 s\Incl t, f(s)=p,\GSD, f(s)=p \\ \cline{1-1}
 s\Incl t, f(t)=p,\GSD, f(s)=p & \rabove{\iLanr}
\end{array} 
\conv
\begin{array}{rl}
 s\Incl t, f(t)=p,\GSD, f(t)=p \\ \cline{1-1}
 s\Incl t, f(t)=p,\GSD, f(s)=p & \rabove{\iLs}
\end{array}
\]
If we were to use ($=_{Ranr}$) the axiom must be of the form $x\Incl t,\GSD,x\Incl t$, 
so the application would be reduced. 

Consider the uppermost application of $\iLanr$ and the rule $R$ applied
immediately above it:
\[ q\notin \cal V\ \ \ \begin{array}{cl}
  \vdots & \\ \cline{1-1}
  q\prec t,s(q)\preceq p,\Gamma\Seq\Delta & \rabove{R} \\ \cline{1-1}
  q\prec t,s(t)\preceq p,\Gamma\Seq\Delta & \rabove{\iLanr} 
\end{array} \]
%
%where $q\notin \cal V$.
\begin{LS}
\item $R$ is $\iLs$ with the active formula $s(q)\Incl p$. 
  \[D \left\{ \begin{array}{cl}
%\vdots \\
 q\prec t,s(q)\Incl p,\Gamma \Seq \Delta,s_1(p)\preceq q_1  &
  \\ \cline{1-1}
q\prec t,s(q)\Incl p,\Gamma \Seq \Delta, s_1(s(q))\preceq q_1 &\rabove{\iLs} 
\\ \cline{1-1}
 q\prec t,s(t)\Incl p,\Gamma \Seq\Delta, s_1(s(q))\preceq q_1 &
 \rabove{\iLanr} 
\end{array} \right. \]
If the active formula of $\iLs$ does not contain $q$ in the way 
indicated above, the two applications can be
trivially swapped and the induction hypothesis allows to eliminate 
$\iLanr$. 

From the above  $D$, we may deduce the existence of the following derivation:
\[ \begin{array}{cl}
D_1\left\{
\begin{array}{cl}
%\vdots \\
 q\prec t,s(q)\Incl p,\GSD,s_1(p)\preceq q_1
         & \raisebox{-1.2ex}[1.5ex][0ex]{$\iLanr$} \\ \cline{1-1}
 q\prec t,s(t)\Incl p,\GSD,s_1(p)\preceq q_1
\end{array} \right. \\ \cline{1-1}
 q\prec t,s(t)\Incl p,\GSD, s_1(s(t))\preceq q_1
         & \rabove{\iLs} \\ \cline{1-1}
 q\prec t,s(t)\Incl p,\GSD, s_1(s(q))\preceq q_1
         & \rabove{\iLs}
\end{array}  \]
Since $h(D_1) < h(D)$, the induction hypothesis allows us to eliminate 
$\iLanr$.
\item $R$ is $\Ear$. Since
$q\not\in\cal V$ also $s(q)\not\in\cal V$. Therefore none of $q\Incl t$
and $s(q)\preceq p$ can be the modified formula of an application of $\Ear$.
Hence the two applications can be swapped and induction hypothesis applied.
%
\item\label{it:eqLa} $R$ is $\eLar$: %, i.e., ($=_{Lar}$):
\[ \begin{array}{rl}
%\multicolumn{1}{c}{\vdots} \\
q(x)\Incl t, s(q(r))\preceq p, x=r, \Gamma\Seq\Delta \\ \cline{1-1}
q(r)\Incl t, s(q(r))\preceq p, x=r, \Gamma\Seq\Delta  &
\raisebox{1.2ex}[1.5ex][0ex]{$(=_{Lar})$} \\ \cline{1-1}
q(r)\Incl t, s(t)\preceq p, x=r, \Gamma\Seq\Delta  &
\raisebox{1.2ex}[1.5ex][0ex]{$\iLanr$}
\end{array} \convd
%Using weakening (another axiom) we construct the following derivation $D'$:
 \begin{array}{rl}
%\multicolumn{1}{c}{\vdots} \\
q(r)\Incl t, q(x)\Incl t, s(q(r))\preceq p, x=r, \GSD \cl
q(r)\Incl t, q(x)\Incl t, s(t)\preceq p, x=r, \GSD  & \rabove{\iLanr} \cl
q(r)\Incl t, q(r)\Incl t, s(t)\preceq p, x=r, \GSD  & \rabove{\eLar}
\end{array} \] 
Analogous weakening is used when the modified formula of $\eLa$
is modified also in the subsequent application of $\iLanr$.
%
\item $R$ is $\eRar$ %($=_{Rar}$):
If the active formula of $\iLanr$ is not modified by ($=_{Rar}$) we
obtain the required reduction in height by swapping analogous to the previous
case. So, assume it was modified. The derivation ends then as follows:
\[ \begin{array}{rl}
%\multicolumn{1}{c}{\vdots} \\
t(y) \Incl q, y\Incl p(k), r=k, \Gamma\Seq\Delta \\ \cline{1-1}
t(y)\Incl q, y\Incl p(r), r=k, \Gamma\Seq\Delta &
\rabove{\eRar} \\ \cline{1-1}
t(p(r))\Incl q, y\Incl p(r), r=k, \Gamma\Seq\Delta &
\rabove{\iLar} 
\end{array} 
\]
Since we have ($=_{Rar}$), the LHS in $y\Incl p(k)$ must be a variable, which means 
that the final application is ($\Incl_{Lar}$).
%
\item $R$ is $\iLar$: This case is treated exactly as 
case \ref{it:eqLa}  with the applications of $\iLar$ instead:
\[ \sma{ \begin{array}{rl}
%\multicolumn{1}{c}{\vdots} \\
q(x)\Incl t, s(q(r))\preceq p, x\Incl r, \GSD \cl
q(r)\Incl t, s(q(r))\preceq p, x\Incl r, \GSD  & \rabove{\iLar} \cl
q(r)\Incl t, s(t)\preceq p, x\Incl r, \GSD  & \rabove{\iLanr}
\end{array} 
\conv
 \begin{array}{rl}
%\multicolumn{1}{c}{\vdots} \\
q(r)\Incl t, q(x)\Incl t, s(q(r))\preceq p, x\Incl r, \GSD \cl
q(r)\Incl t, q(x)\Incl t, s(t)\preceq p, x\Incl r, \GSD  & \rabove{\iLanr} \cl
q(r)\Incl t, q(r)\Incl t, s(t)\preceq p, x\Incl r, \GSD  & \rabove{\iLar}
\end{array} } \]
Analogous procedure yields the result if the modified formula of the
application of $\iLar$ is not active but modified in the subsequent $\iLanr$.
%
\item $R$ is $\iRs$
\[\prar{
q\Incl t, s(q)\Incl p, \GSD, r\Incl f(s(q)) \cl
q\Incl t, s(q)\Incl p, \GSD, r\Incl f(p) & \rabove{\iRs} \cl
q\Incl t, s(t)\Incl p, \GSD, r\Incl f(p) & \rabove{\iLanr} 
}
\conv
\prar{
q\Incl t, s(q)\Incl p, \GSD, r\Incl f(s(q)) \cl
q\Incl t, s(t)\Incl p, \GSD, r\Incl f(s(q)) & \rabove{\iLanr} \cl
q\Incl t, s(t)\Incl p, \GSD, r\Incl f(s(t)) & \rabove{\iRs} \cl
q\Incl t, s(t)\Incl p, \GSD, r\Incl f(p) & \rabove{\iRs} 
}
\]
\item $\Esr$, ($Sp.cut$) and ($cut_t$) can be swapped with $\iLanr$.
\end{LS}
Now the case when the uppermost application is of $\eRanr$
\begin{LS}
\item $R$ is $\eLar$: %($=_{La}$) $D$ ends as follows
\[ \hspace*{-1.5em}  \begin{array}{rl} 
x=u, s=t, q(x)\Incl p(t), \GSD \\ \cline{1-1}
x=u, s=t, q(u)\Incl p(t), \GSD & \rabove{\eLar} \cl
x=u, s=t, q(u)\Incl p(s), \GSD & \rabove{\eRanr} \end{array} \conv
% \] Swapping, we get ($=_{Ranr}$) at a lower height: \[ 
\begin{array}{rl}
x=u, s=t, q(x)\Incl p(t), \GSD \\ \cline{1-1}
x=u, s=t, q(x)\Incl p(s), \GSD & \rabove{\eRanr} \cl
x=u, s=t, q(u)\Incl p(s), \GSD & \rabove{\eLar} \end{array} \]
%
\item $R$ is $\iLar$: The side formula of $\eRanr$ cannot be active
here, since we are using reduced version $\iLar$:
\[ \begin{array}{rl}
x\Incl u, s=t, q(x)\Incl p(t), \GSD \\ \cline{1-1}
x\Incl u, s=t, q(u)\Incl p(t), \GSD & \rabove{\iLar} \cl
x\Incl u, s=t, q(u)\Incl p(s), \GSD & \rabove{\eRanr} \end{array} 
%Again, swapping gives reduction of height:
\conv
\begin{array}{rl}
x\Incl u, s=t, q(x)\Incl p(t), \GSD \\ \cline{1-1}
x\Incl u, s=t, q(x)\Incl p(s), \GSD & \rabove{\eRanr} \cl
x\Incl u, s=t, q(u)\Incl p(s), \GSD & \rabove{\iLar} \end{array} \]
%
\item $R$ is $\iLs$: %($\Incl_{s}$)
\[ \prar{
s=t, q\Incl p(t), \GSD, w(p(t))\preceq r \cl
s=t, q\Incl p(t), \GSD, w(q)\preceq r & \rabove{\iLs} \cl
s=t, q\Incl p(s), \GSD, w(q)\preceq r & \rabove{\eRanr} 
}
\conv
%We transform the derivation reducing the height of $\eRanr$
\prar{
s=t, q\Incl p(t), \GSD, w(p(t))\preceq r \cl
s=t, q\Incl p(s), \GSD, w(p(t))\preceq r & \rabove{\eRanr} \cl
s=t, q\Incl p(s), \GSD, w(p(s))\preceq r & \rabove{\eLs} \cl
s=t, q\Incl p(s), \GSD, w(q)\preceq r & \rabove{\iLs} 
} \]
%
\item $R$ is $\iRs$:
\[ \prar{
s=t, q\Incl p(t),\GSD, r\Incl f(q) \cl
s=t, q\Incl p(t),\GSD, r\Incl f(p(t)) & \rabove{\iRs} \cl
s=t, q\Incl p(s),\GSD, r\Incl f(p(t)) & \rabove{\eRanr}
} \conv
\prar{
s=t, q\Incl p(t),\GSD, r\Incl f(q) \cl
s=t, q\Incl p(s),\GSD, r\Incl f(q) & \rabove{\eRanr}\cl
s=t, q\Incl p(s),\GSD, r\Incl f(p(s)) & \rabove{\iRs} \cl
s=t, q\Incl p(s),\GSD, r\Incl f(p(t)) & \rabove{\eRs} %\wow{This is why I need $\eRs$}
}
\]
\item Other cases are trivial. \\
If $R$ is ($=_{Rar}$) or $\Ear$ then they do not affect the active or side
formula of ($=_{Ranr}$) since, otherwise, the LHS of the latter would have to
be a variable. \\
%% $\eRs$, 
$\eLs$, $\Esr$, ($Sp.cut$) and ($cut_x$) can be trivially swapped.
\end{LS}
As we can see, the applications of $\pLa$ 
occurring in the transformed derivations are the same as the original ones. 
Hence $\#(\pLa,D^*)\leq\#(\pLa,D)$ and $\#(\pLar,D^*)\leq
\#(\pLar,D^*)$.
\end{PROOF}
%
%%\subsubsection{Some rules admissible in $NEQ_4$ 
%%    without increasing $\#\pLar+\#\pLanr$}\label{sub:adm}
\begin{LEMMA}\label{le:noLanr}
The following rule is admissible in $NEQ_4$:
\[ \eLanr\ \ \mprule{s=t, p(s)\Incl q,\GSD}{s=t,p(t)\Incl q,\GSD}\ s\not\in\Vars \]
Moreover, if $\der{NEQ_{4}}DS$ and $\der{NEQ_{4}+\eLanr}{D'}S$, 
%$D'$ derives $S$ using this rule while $D$ derives $S$ not using it, 
we can construct $D'$ so that \\ $\#(\pLar,D')\leq\#(\pLar,D)$, and 
$\#(\pLanr,D')\leq\#(\pLanr,D)$.
\end{LEMMA}
\begin{PROOF}
By lemma %~\ref{fa:noeRs} and 
\ref{le:noar}, we assume that $D$ does not contain
any applications of $\iLanr$. %($\Incl_{anr}$). 
By induction on $\<\#(=_{Lanr},D),h(D)\>$. 
If $h(D)=1$, the only relevant possibility is the axiom with an inclusion in the
antecednet, but it has a variable in the LHS, % $x\Incl t,\GSD,x\Incl t$,
so an application of ($=_{La}$) %to $x\Incl t$ 
has to be reduced.

Consider the last rule above the
uppermost application of ($=_{Lanr}$). 
\begin{LS}
 \item $\iLa$, i.e, $\iLar$. \\
The first case is when both $\iLar$ and $\eLanr$ %and ($=_{Lanr}$) 
modify the same formula:
\[ %D \left \{ 
\begin{array}{rl}
f(x)\Incl p, r=u, x\Incl s(u), \Gamma \Seq\Delta \\ \cline{1-1}
f(s(u))\Incl p, r=u, x\Incl s(u), \Gamma \Seq\Delta  & \rabove{\iLar} \cl
f(s(r))\Incl p, r=u, x\Incl s(u), \Gamma \Seq\Delta  &
\rabove{\eLanr} \end{array} %\right .
 \convd
%We transform it into :
% D' \left \{ 
\begin{array}{rl}
f(x)\Incl p, r=u, x\Incl s(u), x\Incl s(r), \Gamma \Seq\Delta \\ \cline{1-1}
f(s(r))\Incl p, r=u, x\Incl s(u), x\Incl s(r),\Gamma \Seq\Delta  & \rabove{\iLar} \cl
f(s(r))\Incl p, r=u, x\Incl s(u), x\Incl s(u),\Gamma \Seq\Delta  &
\rabove{\eRar} \end{array} %\right .  
\]
Another case is the following:
\[\prarc{
f(g(x))\Incl p, x\Incl t, q=g(t),\GSD \cl
f(g(t))\Incl p, x\Incl t, q=g(t),\GSD & \rabove{\iLar} \cl
f(q)\Incl p, x\Incl t, q=g(t),\GSD & \rabove{\eLanr} 
}\convd
\prarc{
f(g(x))\Incl p, x\Incl t, q=g(t), q=g(x), \GSD \cl
f(q)\Incl p, x\Incl t, q=g(t), q=g(x), \GSD & \rabove{\eLanr} \cl
f(q)\Incl p, x\Incl t, q=g(t), q=g(t), \GSD & \rabove{\iLar} 
} 
\]
The last case is when the formula $\phi$ modified by $\eLanr$ in $D$ was active in the
preceding 
application of $\iLar$. But in this case, since we only have reduced applications
$\iLar$, $\phi$'s LHS must be a variable, so the application of $\eLa$ would not be
$\eLanr$ but $\eLar$.
%
\item $\iLs$: The two applications $\iLs$-$\eLanr$ are easily
replaced by $\eLanr$-$\eLs$.
\item $\iRs$:
\[\sma{ \prar{
f(s)\Incl t, s=r,\GSD,p\Incl q(f(s)) \cl
f(s)\Incl t, s=r,\GSD,p\Incl q(t) & \rabove{\iRs} \cl
f(r)\Incl t, s=r,\GSD,p\Incl q(t) & \rabove{\eLanr} 
}
\conv
\prar{
f(s)\Incl t, s=r,\GSD,p\Incl q(f(s)) \cl
f(r)\Incl t, s=r,\GSD,p\Incl q(f(s)) & \rabove{\eLanr} \cl
f(r)\Incl t, s=r,\GSD,p\Incl q(f(r)) & \rabove{\eRs} \cl
f(r)\Incl t, s=r,\GSD,p\Incl q(t) & \rabove{\iRs} 
} }
\]
%
\item ($=_{Lar}$): 
% \[ D \left \{ \begin{array}{rl}
% q(s(y)) \Incl p, y=u, s(u)=r, \Gamma\Seq\Delta  \\ \cline{1-1}
% q(s(u)) \Incl p, y=u, s(u)=r, \Gamma\Seq\Delta &
% \raisebox{1.2ex}[1.5ex][0ex]{($=_{Lar}$)} \\ \cline{1-1}
% q(r) \Incl p, y=u, s(u)=r, \Gamma\Seq\Delta &
% \raisebox{1.2ex}[1.5ex][0ex]{($=_{Lanr}$)} \end{array} \right . \convd
%  D' \left \{ \begin{array}{rl}
% q(s(y)) \Incl p, y=u, s(u)=r, s(y)=r, \Gamma\Seq\Delta \\ \cline{1-1}
% q(r) \Incl p, y=u, s(u)=r, s(y)=r, \Gamma\Seq\Delta &
% \raisebox{1.2ex}[1.5ex][0ex]{($=_{Lanr}$)} \\ \cline{1-1}
% q(r) \Incl p, y=u, s(u)=r, s(u)=r, \Gamma\Seq\Delta &
% \raisebox{1.2ex}[1.5ex][0ex]{$\eae$} \end{array} \right . \]
% OLD:
\[ D \left \{ \begin{array}{rl}
f(x)\Incl p, r=u, x= s(u), \Gamma \Seq\Delta \\ \cline{1-1}
f(s(u))\Incl p, r=u, x= s(u), \Gamma \Seq\Delta  &
\raisebox{1.2ex}[1.5ex][0ex]{($=_{Lar}$)} \\ \cline{1-1}
f(s(r))\Incl p, r=u, x= s(u), \Gamma \Seq\Delta  &
\raisebox{1.2ex}[1.5ex][0ex]{($=_{Lanr}$)} \end{array} \right . \convd
%We transform it into :
 D' \left \{ \begin{array}{rl}
f(x)\Incl p, r=u, x= s(u), x=s(r),\Gamma \Seq\Delta \\ \cline{1-1}
f(s(r))\Incl p, r=u, x= s(u), x=s(r),\Gamma \Seq\Delta  &
\raisebox{1.2ex}[1.5ex][0ex]{($=_{Lanr}$)} \\ \cline{1-1}
f(s(r))\Incl p, r=u, x= s(u), x=s(u),\Gamma \Seq\Delta  &
\raisebox{1.2ex}[1.5ex][0ex]{$\eae$} \end{array} \right . \]
By induction hypothesis ($=_{Lanr}$) can be eliminated from $D'$, and by 
lemma \ref{le:noeqeq} $D'$ can be transformed so that it
does not contain $\eae$ and this transformation does not
introduce any additional applications of $\pLa$. %($=_{Lanr}$).
The same applies in the other case:
\[
\prar{
f(s(x))\Incl p, x=u, s(u)=r, \GSD \cl
f(s(u))\Incl p, x=u, s(u)=r, \GSD & \rabove{\eLar} \cl
f(r)\Incl p, x=u, s(u)=r, \GSD & \rabove{\eLanr} 
}\convd
\prar{
f(s(x))\Incl p, x=u, s(u)=r, s(x)=r, \GSD \cl
f(r)\Incl p, x=u, s(u)=r, s(x)=r, \GSD & \rabove{\eLanr} \cl
f(r)\Incl p, x=u, s(u)=r, s(u)=r, \GSD & \rabove{\eae} 
} 
\]
%
\item Elimination rules are not relevant,
neither is ($Sp.cut$) and $(cut_x)$, and $\eRa$ and $\eRs$ are trivially swapped
since their active formula (equality) is not affected by $\eLanr$.
\end{LS}
We see that no new applications of
$\pLar$ or $\pLanr$ emerge.
\end{PROOF}
%
\begin{LEMMA}\label{le:inclaad} The following rule is admissible in $NEQ_4$:
\[\mprule{s\Incl t, q\Incl t,\GSD}{s\Incl t, q\Incl s,\GSD} %\ $s,t\not\in\Vars$
\ \ \ \ \iRa
\]
Moreover, if $D^*$ is a derivation using the rule $\iRa$ and $D$ is the corresponding
derivation not using this rule, then $\#(\pLar,D)\leq\#(\pLar,D^*)$, and
$\#(\pLanr,D)\leq\#(\pLanr,D^*)$. 
%  [Also, $D$ does not introduce new applications of $(cut_{x=})$ -- the fact to be used
%in lemma~\ref{le:noxx}.]
\end{LEMMA}
\begin{PROOF}
By induction on $\< \#(\iRa,D^*),h(D^*)\>$. 
By lemma  % \ref{fa:noeRs} and 
\ref{le:noar} % and \ref{le:noLanr}, 
we assume that the derivation does not 
contain any % $\eRs$, 
$\iLanr$ or $\eRanr$. % nor $\eLanr$.
The possible application to an axiom is: % $x\Incl t,\GSD,x\Incl t$
\[
\begin{array}{rl}
x\Incl t, s\Incl t,\GSD,x\Incl t &  \\ \cline{1-1}
x\Incl s, s\Incl t,\GSD,x\Incl t & \rabove{\iRa}
\end{array}
\conv
\begin{array}{rl}
x\Incl s, s\Incl t,\GSD,x\Incl s \\ \cline{1-1}
x\Incl s,s\Incl t, \GSD,x\Incl t & \rabove{\iRs}
\end{array}
\]
Consider the rule applied just above $\iRa$.
\begin{LS}
\item $\iRs$
\[ \prar{
q\Incl t, s\Incl t,\GSD,p\Incl f(q) \\ \cline{1-1}
q\Incl t, s\Incl t,\GSD,p\Incl f(t) & \rabove{\iRs} \\ \cline{1-1}
q\Incl s, s\Incl t,\GSD,p\Incl f(t) & \rabove{\iRa}
}
\conv
\prar{
q\Incl t, s\Incl t,\GSD,p\Incl f(q) \\ \cline{1-1}
q\Incl s, s\Incl t,\GSD,p\Incl f(q) & \rabove{\iRa} \\ \cline{1-1}
q\Incl s, s\Incl t,\GSD,p\Incl f(s) & \rabove{\iRs} \\ \cline{1-1}
q\Incl s, s\Incl t,\GSD,p\Incl f(t) & \rabove{\iRs}
}
\]
\item $\iLar$:
\[ \sma{ \begin{array}{rl}
 s(y) \Incl t, q\Incl t, y\Incl p, \GSD \\ \cline{1-1}
 s(p) \Incl t, q\Incl t, y\Incl p, \GSD & \rabove{\iLar} \\
 \cline{1-1}
 s(p) \Incl t, q\Incl s(p), y\Incl p, \GSD & \rabove{\iRa}
 \end{array} \conv
 \begin{array}{rl}
s(p)\Incl t, s(y)\Incl t, q\Incl t, y\Incl p, \GSD \\ \cline{1-1}
s(p)\Incl t, s(y)\Incl t, q\Incl s(p), y\Incl p, \GSD &
\rabove{\iRa} \\ \cline{1-1}
s(p)\Incl t, s(p)\Incl t, q\Incl s(p), y\Incl p, \GSD &
\rabove{\iLar} \end{array} 
}
\]
%
\item $\iLs$ %($\Incl_s$): $q\not\in\Vars$
% -- if $s\in\Vars$ we get $\Incl_{sx}$
\[ \prar{
s\Incl t, q\Incl t, \GSD, w(t)\preceq p \\ \cline{1-1}
s\Incl t, q\Incl t, \GSD, w(q)\preceq p & \rabove{\iLs} \\
\cline{1-1}
s\Incl t, q\Incl s, \GSD, w(q)\preceq p & \rabove{\iRa} 
}\conv
%We just swap:
\prar{
s\Incl t, q\Incl t, \GSD, w(t)\preceq p \\ \cline{1-1}
s\Incl t, q\Incl s, \GSD, w(t)\preceq p & \rabove{\iRa} \\
\cline{1-1}
s\Incl t, q\Incl s, \GSD, w(s)\preceq p & \rabove{\iLs} \\
\cline{1-1} 
s\Incl t, q\Incl s, \GSD, w(q)\preceq p & \rabove{\iLs} 
} \]
%% The obtained applications of $(\Incl_s)$ are not $(\Incl_{sx})$ -- the first
%% by restriction on $\iRa$ -- $s\not\in\Vars$, and the second since $q\not\in\Vars$.
\item $\Ear$:
\[ \sma{
\begin{array}{rl}
x\Incl t, y\Incl r(x), s\Incl r(t), \GSD \\ \cline{1-1}
          y\Incl r(t), s\Incl r(t), \GSD  & \rabove{\Ear} \\ \cline{1-1}
          y\Incl s, s\Incl r(t), \GSD  & \rabove{\iRa} \end{array}
  \conv
%Again, swapping after weakening, we reduce the height:
 \begin{array}{rl}
x\Incl t, y\Incl r(x), s\Incl r(t), s\Incl r(x),\GSD \\ \cline{1-1}
x\Incl t, y\Incl s, s\Incl r(t), s\Incl r(x),\GSD  & \rabove{\iRa} \\ \cline{1-1}
          y\Incl s, s\Incl r(t), s\Incl r(t),\GSD  & \rabove{\Ear} \end{array} 
}\]
%
We do exactly the same if $\Ear$ modifies the active (rather than the side)
formula of the application of $\iRa$.
\item $\eLar$:
\[ \sma{
\begin{array}{rl}
s(x)\Incl t, q\Incl t, x=p, \GSD \\ \cline{1-1}
s(p)\Incl t, q\Incl t, x=p, \GSD & \rabove{(=_{Lar})} \\ \cline{1-1}
s(p)\Incl t, q\Incl s(p), x=p, \GSD & \rabove{\iRa} \end{array} \conv
%Swap the applications, and clean up:
\begin{array}{rl}
s(p)\Incl t,s(x)\Incl t, q\Incl t, x=p, \GSD \\ \cline{1-1}
s(p)\Incl t,s(x)\Incl t, q\Incl s(p), x=p, \GSD & \rabove{\iRa} \\ \cline{1-1}
s(p)\Incl t,s(p)\Incl t, q\Incl s(p), x=p, \GSD & \rabove{(=_{Lar})} \end{array}
}
\]
\item $\eRar$:
\[ \sma{ \begin{array}{rl}
s\Incl t(p), x\Incl t(r), r=p, \GSD \\ \cline{1-1}
s\Incl t(p), x\Incl t(p), r=p, \GSD & \rabove{(=_{Rar})}\\ \cline{1-1}
s\Incl t(p), x\Incl s, r=p, \GSD & \rabove{\iRa} \end{array} \conv
%Weakening allows us to swap:
 \begin{array}{rl}
s\Incl t(r), s\Incl t(p), x\Incl t(r), r=p, \GSD \\ \cline{1-1}
s\Incl t(r), s\Incl t(p), x\Incl s, r=p, \GSD & \rabove{\iRa} \\ \cline{1-1}
s\Incl t(p), s\Incl t(p), x\Incl s, r=p, \GSD & \rabove{\eRa} \end{array} 
} \]
Analogous transformation is applied if $\iRa$ in the original derivation yields
$s\Incl x, x\Incl t(p)\ldots$.
%
\item The rules $\eLs$, $\eRs$, $\Esr$, $(Sp.cut)$, $(cut_x)$ and ($cut_{x=}$) 
can be trivially swapped, since they do
not affect the antecedent -- the active formula, which is either equality or
disappears, cannot be involved in the final application of $\iRa$.
\end{LS}
We see that no new applications of $\pLa$ appear in the transformed derivations 
$D$, and the ones which occur there are the same as in the original 
derivations $D^*$.
[Also, no new applications of $(cut_{x=})$ appear in the transformed derivations.]
\end{PROOF}
%

\subsubsection{Some redundant s-rules}\label{sub:nosx}
First lemma shows the possibility of eliminating most of the s-rules from the $NEQ_4$
derivations.
%% the following degenerate 
%% applications can be avoided in $NEQ_4$ derivations:
%% \[\sma{
%% \eRsx\ \ \mprule{s=x, \GSD, t\Incl f(s)}{s=x, \GSD, t\Incl f(x)} 
%% \hspace*{2em}
%% \iLsx\ \ \mprule{s\Incl x, \GSD, f(s)\preceq t}{s\Incl x, \GSD,f(x)\preceq t} 
%% \hspace*{3em}
%% \eLsx\ \ \mprule{s= x, \GSD, f(s)\preceq t}{s= x, \GSD, f(x)\preceq t}  
%% }
%% \]
\begin{LEMMA}\label{le:nosx}
$\der{NEQ_4}DS \impl \der{NEQ_4}{D^*}S$ and $D^*$ uses no applications
of $\eRs$, $\iLs$ or $\eLs$.
%% Applications $\eRsx$, $\iLsx$, $\eLsx$ can be eliminated.
%% [Morover, the resulting derivations do not introduce any new applications of
%% s-rules.]
\end{LEMMA}
\begin{PROOF}
We proceed by simultaneous induction on 
the number of applications of all the listed rules and by subinduction on the
height of the derivation. 
By lemma~\ref{fa:noeRs}, we may assume that derivation contains 
no applications of $\eRs$.
%, in particular, no applications of $\eRsx$.

Writing $\pLs$, we mean either $\iLs$ or $\eLs$ and use this notation when the
two cases are treated analogously. 
\begin{LS}
\item Each form of the axioms gives rise to several cases:
\begin{LSA}
\item $x\Incl t,\GSD, x\Incl t$
\begin{LSB}
\item If $s\not\in\Vars$, we choose a fresh variable $y:$
\[
\prar{
s\preceq x, x\Incl t, \GSD, x\Incl t \cl
s\preceq x, x\Incl t, \GSD, s\Incl t & \rabove{\pLs}
} \conv
\prar{
y\Incl s, s\preceq x, y\Incl t, \GSD, y\Incl t \cl
y\Incl s, s\preceq x, s\Incl t, \GSD, y\Incl t & \rabove{\iLar} \cl
     s\preceq x, s\Incl t, \GSD, s\Incl t & \rabove{\Esr}
}
\]
\item 
If $s$ is a variable, say $y$:
\[
\prarc{
y\preceq x, x\Incl t, \GSD, x\Incl t \cl
y\preceq x, x\Incl t, \GSD, y\Incl t & \rabove{\pLs}
} \conv
\prarc{
y\preceq x, y\Incl t, \GSD, y\Incl t \cl
y\preceq x, x\Incl t, \GSD, y\Incl t & \rabove{\pLar}
}
\]
\end{LSB}
\item $s=t,\GSD,s=t$ gives rise to four cases
\begin{LSB}
\item The substituted term is a variable $x$
\[
\prarc{
x\Incl p, s(p)= t, \GSD, s(p)= t \cl
x\Incl p, s(p)= t, \GSD, s(x)= t & \rabove{\iLs}
} \conv
\prarc{
x\Incl p, s(x)= t, \GSD, s(x)= t \cl
x\Incl p, s(p)= t, \GSD, s(x)= t & \rabove{\iLar}
}
\]
\item and variable substituted by $\eLs$:
\[\sma{
\prar{
x= p, s(p)= t, \GSD, s(p)= t \cl
x= p, s(p)= t, \GSD, s(x)= t & \rabove{\eLs}
} \conv
\prar{
x\Incl p, x= p, s(x)= t, \GSD, s(x)= t \cl
x\Incl p, x= p, s(p)= t, \GSD, s(x)= t & \rabove{\iLar} \cl
x\Incl x, x= p, s(p)= t, \GSD, s(x)= t & \rabove{\eRar} \cl
 x= p, s(p)= t, \GSD, s(x)= t & \rabove{(cut_x)} 
} }
\]
\item The substituted term $r\not\in\Vars$ -- we choose a fresh variable $y:$
\[\sma{ \prar{
r=p, s(p)=t, \GSD, s(p)=t \cl
r=p, s(p)=t, \GSD, s(r)=t & \rabove{\eLs}
}\conv
\prar{
y\Incl p, r=p, s(y)=t, \GSD, s(y)=t \cl
y\Incl p, r=p, s(p)=t, \GSD, s(y)=t & \rabove{\iLar} \cl
y\Incl r, r=p, s(p)=t, \GSD, s(y)=t & \rabove{\eRar} \cl
      r=p, s(p)=t, \GSD, s(r)=t & \rabove{\Esr} 
} }
\]
\item and non-variable substituted by $\iLs$
\[\prarc{
r\Incl p, s(p)=t, \GSD, s(p)=t \cl
r\Incl p, s(p)=t, \GSD, s(r)=t & \rabove{\iLs}
}\conv
\prar{
r\Incl p, s(r)=t, \GSD, s(r)=t \cl
r\Incl p, s(p)=t, \GSD, s(r)=t & \rabove{\iLa}
}
\]
\end{LSB}
\item $\GSD, x=x$.
The situation \[\prar{s=x,\GSD,x=x \cl s=x,\GSD,s=x& \rabove{\eLs}}\] gives an 
instance of an axiom. The case of $\iLs$ is treated as follows:
\[\prar{
t\Incl x, \GSD, x=x \cl 
t\Incl x,\GSD, t=x & \rabove{\iLs}
}\conv
\prar{
t=x, t\Incl x,\GSD, t=x \cl
x=x, t\Incl x,\GSD, t=x & \rabove{\iLa} \cl
t\Incl x, \GSD, t=x & \rabove{(cut_{x=})}
}
\]
\end{LSA}
\item $\iLa$
\[\sma{ \prarc{
q(t)\preceq r, t\Incl s, \GSD, p(r)\preceq t \cl
q(s)\preceq r, t\Incl s, \GSD, p(r)\preceq t & \rabove{\iLa} \cl
q(s)\preceq r, t\Incl s, \GSD, p(q(s))\preceq t & \rabove{\pLs} 
}\conv
\prarc{
q(s)\preceq r, q(t)\preceq r, t\Incl s, \GSD, p(r)\preceq t \cl
q(s)\preceq r, q(t)\preceq r, t\Incl s, \GSD, p(q(s))\preceq t & \rabove{\pLs} \cl
q(s)\preceq r, q(s)\preceq r, t\Incl s, \GSD, p(q(s))\preceq t & \rabove{\iLa} 
} }
\]
%% \item $\eLs$ % $s\not\in\Vars$
%% We consider the cases of $\iLsx$ and $\eLsx$ separately
%% \begin{LSA}
%% \item $\iLsx$
%% \[\prarc{
%% x\Incl s, s=t,\GSD, p(t)\preceq q \cl
%% x\Incl s, s=t,\GSD, p(s)\preceq q & \rabove{\eLs} \cl
%% x\Incl s, s=t,\GSD, p(x)\preceq q & \rabove{\iLsx} 
%% }\conv
%% \prarc{
%% x\Incl t, x\Incl s, s=t,\GSD, p(t)\preceq q \cl
%% x\Incl t, x\Incl s, s=t,\GSD, p(x)\preceq q & \rabove{\iLsx} \cl
%% x\Incl s, x\Incl s, s=t,\GSD, p(x)\preceq q & \rabove{\eRar} 
%% }
%% \]
%% Another case is as follows:
%% \[\sma{ \prarc{
%% x\Incl s, t=r(s), \GSD, p(t)\preceq q \cl
%% x\Incl s, t=r(s), \GSD, p(r(s))\preceq q & \rabove{\eLs} \cl
%% x\Incl s, t=r(s), \GSD, p(r(x))\preceq q & \rabove{\iLsx} 
%% }\conv
%% \prarc{
%% x\Incl s, t=r(s), t=r(x), \GSD, p(t)\preceq q \cl
%% x\Incl s, t=r(s), t=r(x), \GSD, p(r(x))\preceq q & \rabove{\eLs} \cl
%% x\Incl s, t=r(s), t=r(s), \GSD, p(r(x))\preceq q & \rabove{\iLar} 
%% } }
%% \]
%% \item 
%% The two cases of $\eLsx$
%% \[\prar{
%% x= s, s=t,\GSD, p(t)\preceq q \cl
%% x= s, s=t,\GSD, p(s)\preceq q & \rabove{\eLs} \cl
%% x= s, s=t,\GSD, p(x)\preceq q & \rabove{\eLsx} 
%% }\conv
%% \prar{
%% x\Incl t, x= s, s=t,\GSD, p(t)\preceq q \cl
%% x\Incl t, x= s, s=t,\GSD, p(x)\preceq q & \rabove{\iLsx} \cl
%% x\Incl s, x= s, s=t,\GSD, p(x)\preceq q & \rabove{\eRa} \cl
%% x\Incl x, x= s, s=t,\GSD, p(x)\preceq q & \rabove{\eRa} \cl
%%           x= s, s=t,\GSD, p(x)\preceq q & \rabove{(cut_x)} 
%% }
%% \]
%% \[\sma{ \prar{
%% x=s, r(s)=t,\GSD, p(t)\preceq q \cl
%% x=s, r(s)=t,\GSD, p(r(s))\preceq q & \rabove{\eLs} \cl
%% x=s, r(s)=t,\GSD, p(r(x))\preceq q & \rabove{\eLsx}
%% } } \convd \sma{
%% \prar{
%% x\Incl s, r(x)=t, x=s, r(s)=t,\GSD, p(t)\preceq q \cl
%% x\Incl s, r(x)=t, x=s, r(s)=t,\GSD, p(r(x))\preceq q & \rabove{\eLs} \cl
%% x\Incl s, r(s)=t, x=s, r(s)=t,\GSD, p(r(x))\preceq q & \rabove{\iLar} \cl
%% x\Incl x, r(s)=t, x=s, r(s)=t,\GSD, p(r(x))\preceq q & \rabove{\eRa} \cl
%%       r(s)=t, x=s, r(s)=t,\GSD, p(r(x))\preceq q & \rabove{(cut_x)}
%% } }
%% \]
%% \end{LSA}
\item $\Ear$
\[\sma{\prarc{
y\Incl t, x\Incl q(y),\GSD, f(q(t))\preceq p \cl
          x\Incl q(t),\GSD, f(q(t))\preceq p & \rabove{\Ear} \cl
          x\Incl q(t),\GSD, f(x)\preceq p & \rabove{\iLs}
}\conv
\prarc{
x\Incl q(t), y\Incl t, x\Incl q(y),\GSD, f(q(t))\preceq p \cl
x\Incl q(t), y\Incl t, x\Incl q(y),\GSD, f(x)\preceq p & \rabove{\iLs} \cl
x\Incl q(t), x\Incl q(t),\GSD, f(x)\preceq p & \rabove{\Ear}
} }
\]
\item $\eLar$
\[\sma{ \prarc{
s(y)\Incl q, t=y,\GSD, f(q)\preceq t \cl
s(t)\Incl q, t=y,\GSD, f(q)\preceq t & \rabove{\eLar} \cl
s(t)\Incl q, t=y,\GSD, f(s(t))\preceq t & \rabove{\iLs} 
}\conv
\prarc{
s(t)\Incl q, s(y)\Incl q, t=y,\GSD, f(q)\preceq t \cl
s(t)\Incl q, s(y)\Incl q, t=y,\GSD, f(s(t))\preceq t & \rabove{\iLs} \cl
s(t)\Incl q, s(t)\Incl q, t=y,\GSD, f(s(t))\preceq t & \rabove{\eLar} 
} }
\]
\item $\eRa$
\[\sma{\prarc{
x\Incl p(t), t=s,\GSD, f(p(s))\preceq q \cl
x\Incl p(s), t=s,\GSD, f(p(s))\preceq q & \rabove{\eRa} \cl
x\Incl p(s), t=s,\GSD, f(x)\preceq q & \rabove{\iLs}
}\conv
\prarc{
x\Incl p(s), x\Incl p(t), t=s,\GSD, f(p(s))\preceq q \cl
x\Incl p(s), x\Incl p(t), t=s,\GSD, f(x)\preceq q & \rabove{\iLs} \cl
x\Incl p(s), x\Incl p(s), t=s,\GSD, f(x)\preceq q & \rabove{\eRa} 
} }
\]
\item $\iRs$, $\eRs$, $\Esr$, $(cut_x)$, $(cut_{x=})$ and $(Sp.cut)$ can be 
trivially swapped.
\end{LS}
As we see, no new applications of s-rules appear in the resulting derivations.
\end{PROOF}
%

\noindent
In lemma~\ref{fa:noeRs} we have shown that rule $\eRs$ can be eliminated from the
derivations. % without introducing new applications of s-rules. 
However, new applications of a-rules appeared in the transformed derivations. 
Moreover, in lemma~\ref{le:noar}, eliminating non-reduced
applications of $\eRanr$ we used an application of $\eRs$ in the transformed derivation
(case 4). The final lemma shows now that, as a matter of fact, we can eliminate $\eRs$, 
as well as a non-reduced version of $\iRsnr$, from derivations without increasing the
number of $\pLanr$ and $\eRanr$. 
\begin{LEMMA}\label{le:noiRsnr}
$\der{NEQ_{4}}DS\ \impl\ \der{NEQ_{4}}{D^{*}}S$ and $D^*$ contains no 
applications of $\eRs$ nor
\[
\iRsnr\ \ \ \mprule{s\Incl t,\GSD, p\Incl q(s)}{s\Incl t,\GSD,p\Incl q(t)}\ \ 
s\not\in\Vars
\]
Moreover $\#(\pLanr,D^*) \leq \#(\pLanr,D)$ and $\#(\eRanr,D^*)\leq\#(\eRanr,D)$.
\end{LEMMA}
\begin{PROOF}
By lemma \ref{le:noar} 
we assume that no non-reduced applications $\iLanr$ or $\eRanr$
appear in $D$. Proceeding by simultaneous induction on the number of applications
of $\eRs$ and $\iRsnr$ and subinduction on the height of $D$, we consider first the
case when the uppermost application of one of these two rules is $\eRs$. (This part
of the proof is essentially revisting the proof of lemma~\ref{fa:noeRs} and checking
that the current side-conditions are satisfied.)

For the axioms, we only have to consider the one with inclusion in the consequent.
%\begin{itemize}
%\item $x\Incl t,\GSD,x\Incl t$
%\end{itemize}
\[
\begin{array}{rl}
p=s, x\Incl t(s),\GSD,x\Incl t(s) \\ \cline{1-1}
p=s, x\Incl t(s),\GSD,x\Incl t(p) & \rabove{\eRs}
\end{array}
\conv
\begin{array}{rl}
p=s, x\Incl t(p),\GSD,x\Incl t(p) \\ \cline{1-1}
p=s, x\Incl t(s),\GSD,x\Incl t(p) & \rabove{\eRar}
\end{array}
\]
The application is reduced  $\eRar$.
Consider the uppermost application of ($=_{Rs}$) in $D$:
\begin{LS}
\item\label{it:RsEsB} $R=\Esr$ with $q\Incl t(r)$ as the modified formula and $r$ as
modified term. We have two subcases corresponding to the situation when the
term $r'$ substituted by $\Esr$ for its eigen-variable is a subterm or
superterm of (or equal to) $r$. (When the two are independent, the case is
trivial.) 
\begin{LSA}
\item  $r'$ is a subterm of $r$,\\
i.e., $t'(r')$ and $t(r)$ are identical with $r=f(r')$ and $t'(x)=t(f(x))$.
\[ D \left \{\begin{array}{cl}
%\vdots          \\
x\Incl r', f(r')=s, \GSD, q\Incl t(f(x)) & \raisebox{-1.2ex}[1.5ex][0ex]{$\Esr$} \\ \cline{1-1}
f(r')=s, \GSD, q\Incl t(f(r')) &
\raisebox{-1.2ex}[1.5ex][0ex]{($=_{Rs}$)} \\ \cline{1-1}
f(r')=s, \GSD, q\Incl t(s) 
\end{array} \right . \convd
D' \left \{ \begin{array}{cl}
x\Incl r', f(r')=s, f(x)=s, \Gamma\Seq\Delta,q\Incl t(f(x)) & \raisebox{-1.2ex}[1.5ex][0ex]{($=_{Rs}$)} \\ \cline{1-1}
x\Incl r', f(r')=s, f(x)=s, \Gamma\Seq\Delta, q\Incl t(s) &
\raisebox{-1.2ex}[1.5ex][0ex]{$\iLar$} \\ \cline{1-1}
x\Incl r', f(r')=s, f(r')=s, \Gamma\Seq\Delta, q\Incl t(s) &
\raisebox{-1.2ex}[1.5ex][0ex]{$\Esr$} \\ \cline{1-1}
f(r')=s, \Gamma\Seq\Delta, q\Incl t(s) 
\end{array} \right . \]
The obtained application of $\iLar$ is reduced.
\item $r'$ is a superterm of (or equal to) $r$,\\
i.e., $r'=f(r)$ and $t(x)=t'(f(x))$.
\[ D \left \{\begin{array}{cl}
%\vdots          \\
x\Incl f(r), r=s, \Gamma\Seq\Delta, q\Incl t'(x) & \\ \cline{1-1}
r=s, \Gamma\Seq\Delta, q\Incl t'(f(r)) & \rabove{\Esr} \\ \cline{1-1}
r=s, \Gamma\Seq\Delta, q\Incl t'(f(s)) & \rabove{\eRs} 
\end{array} \right . \convd
D \left \{\begin{array}{cl}
x\Incl f(r), r=s, \Gamma\Seq\Delta, q\Incl t'(x) & \\ \cline{1-1}
x\Incl f(s), r=s, \Gamma\Seq\Delta, q\Incl t'(x) & \rabove{\eRar}
\\ \cline{1-1}
r=s, \Gamma\Seq\Delta, q\Incl t'(f(s)) & \rabove{\Esr} 
\end{array} \right . \]
Again, the resulting application of $\eRar$ is reduced.
The case when the modified term is substituted into LHS of $\Incl$ is treated
analogously.
\end{LSA}
\item $R=\iLar$ with the modified formula $r=s(v)$ and $v$ as the modified
term:
\[ D \left \{ \begin{array}{cl}
 D_1 \left \{ \begin{array}{c}
               \vdots       \\ 
               r=s(x), x\Incl v, \Gamma\Seq\Delta,q\Incl t(r) 
           \end{array} \right . 
         & \raisebox{-3.2ex}[1.5ex][0ex]{$\iLa$}  \\ \cline{1-1}
 r=s(v), x\Incl v, \Gamma\Seq\Delta, q\Incl t(r) &
 \raisebox{-1.2ex}[1.5ex][0ex]{($=_{Rs}$)} \\ \cline{1-1}
 r=s(v), x\Incl v, \Gamma\Seq\Delta, q\Incl t(s(v)) 
 \end{array} \right . \convd
  D^* \left \{ \begin{array}{cl}
  D' \left \{ \begin{array}{cl}
    D_1 \left \{ \begin{array}{cl}
 \vdots       \\ 
 r=s(x), x\Incl v, \Gamma\Seq\Delta,q\Incl t(r) 
  \end{array} \right . & \raisebox{-3.2ex}[1.5ex][0ex]{($W_a$)}  \\
  \cline{1-1}
 r=s(x), r=s(v), x\Incl v, \Gamma \Seq \Delta, q\Incl t(r)
  & \raisebox{-1.2ex}[1.5ex][0ex]{($=_{Rs}$)} \\ \cline{1-1}
 r=s(x), r=s(v), x\Incl v, \Gamma\Seq\Delta, q\Incl t(s(v))  \end{array}
 \right . &  \\ \cline{1-1}
 r=s(v), r=s(v), x\Incl v, \Gamma\Seq\Delta, q\Incl t(s(v)) 
 & \raisebox{1.2ex}[1.5ex][0ex]{$\iLar$} 
 \end{array} \right . \]
The resulting application of $\iLa$ is the same as the original one -- hence reduced.
The same transformation is applied if $\eRs$ substitutes $r$ for $s(v)$.
 \item $R=\iLs$ :
 \[ D \left \{ \begin{array}{cl}
  D_1 \left \{ \begin{array}{c}
               \vdots       \\ 
               r=s, q\Incl u, \Gamma\Seq\Delta, u\Incl t(r) 
            \end{array} \right . 
          & \raisebox{-3.2ex}[1.5ex][0ex]{$\iLs$}  \\ \cline{1-1}
 r=s, q\Incl u, \Gamma\Seq\Delta, q\Incl t(r) &
 \raisebox{-1.2ex}[1.5ex][0ex]{($=_{Rs}$)} \\ \cline{1-1}
 r=s, q\Incl u, \Gamma\Seq\Delta, q\Incl t(s) 
 \end{array} \right . \convd
  D^* \left \{ \begin{array}{cl}
  D' \left \{ \begin{array}{cl}
    D_1 \left \{ \begin{array}{cl}
 \vdots       \\ 
 r=s, q\Incl u, \Gamma\Seq\Delta, u\Incl t(r) 
  \end{array} \right . & \raisebox{-3.2ex}[1.5ex][0ex]{($=_{Rs}$)}  \\
  \cline{1-1}
 r=s, r=s(v), q\Incl u, \Gamma \Seq \Delta, u\Incl t(s) \end{array}
 \right . &  \\ \cline{1-1}
 r=s, q\Incl u, \Gamma\Seq\Delta, q\Incl t(s) 
 & \raisebox{1.2ex}[1.5ex][0ex]{$\iLs$} 
 \end{array} \right . \]
% Since $h(D')<h(D)$, the induction hypothesis yields a desired derivation.
\item $R=\iRsr$
\[\sma{\prarc{
f(p)=s, x\Incl p,\GSD, t\Incl q(f(x)) \cl
f(p)=s, x\Incl p,\GSD, t\Incl q(f(p)) & \rabove{\iRsr} \cl
f(p)=s, x\Incl p,\GSD, t\Incl q(s) & \rabove{\eRs}
}
\conv
\prarc{
f(p)=s,f(x)=s, x\Incl p,\GSD, t\Incl q(f(x)) \cl
f(p)=s,f(x)=s, x\Incl p,\GSD, t\Incl q(s)  & \rabove{\eRs} \cl
f(p)=s,f(p)=s, x\Incl p,\GSD, t\Incl q(s)  & \rabove{\iLar} 
} }
\]
\[\sma{ \prarc{
x\Incl f(s), s=r, \GSD, t\Incl q(x) \cl
x\Incl f(s), s=r, \GSD, t\Incl q(f(s)) &\rabove{\iRsr} \cl
x\Incl f(s), s=r, \GSD, t\Incl q(f(r)) &\rabove{\eRs} 
}
\conv
\prarc{
x\Incl f(r), x\Incl f(s), s=r, \GSD, t\Incl q(x) \cl
x\Incl f(r), x\Incl f(s), s=r, \GSD, t\Incl q(f(r)) &\rabove{\iRs} \cl
x\Incl f(s), x\Incl f(s), s=r, \GSD, t\Incl q(f(r)) &\rabove{\eRar}
} }
\]
In both cases, since the original application of $\iRsr$ is reduced, so are
the resulting applications of $\iLar$ and $\eRar$.
 \item The other cases are easily swapped as in lemma ~\ref{fa:noeRs}.
\end{LS}
%
So the case when the uppermost application is $\iRsnr$.
%\ref{fa:noeRs} we assume that $D$ contains no 
%$\eRs$.
For the axiom we have the following case ($y$ is a fresh variable):
\[
\prarc{
s\Incl t, x\Incl q(s),\GSD, x\Incl q(s) \cl
s\Incl t, x\Incl q(s),\GSD, x\Incl q(t) & \rabove{\iRsnr}
} \conv
\prarc{
y\Incl s, y\Incl t, x\Incl q(y), \GSD, x\Incl q(y) \cl
y\Incl s, y\Incl t, x\Incl q(y), \GSD, x\Incl q(t) & \rabove{\iRsr} \cl
y\Incl s, s\Incl t, x\Incl q(y), \GSD, x\Incl q(t) & \rabove{\iLar} \cl
          s\Incl t, x\Incl q(y), \GSD, x\Incl q(t) & \rabove{\Ear}
}
\]
Consider the rule applied just above $\iRsnr$.
\begin{LS}
\item $\iRsr$ -- we choose a fresh variable $y$
\[\prarc{
s\Incl t, x\Incl f(s),\GSD, p\Incl q(x) \cl
s\Incl t, x\Incl f(s),\GSD, p\Incl q(f(s)) & \rabove{\iRsr} \cl
s\Incl t, x\Incl f(s),\GSD, p\Incl q(f(t)) & \rabove{\iRsnr} 
} \convd
\prarc{
y\Incl s, y\Incl t, x\Incl f(y), s\Incl t, x\Incl f(s),\GSD, p\Incl q(x) \cl
y\Incl s, y\Incl t, x\Incl f(y), s\Incl t, x\Incl f(s),\GSD, p\Incl 
q(f(y)) & \rabove{\iRsr} \cl
y\Incl s, y\Incl t, x\Incl f(y), s\Incl t, x\Incl f(s),\GSD, p\Incl 
q(f(t)) & \rabove{\iRsr} \cl
y\Incl s, s\Incl t, x\Incl f(y), s\Incl t, x\Incl f(s),\GSD, p\Incl 
q(f(t)) & \rabove{\iLar} \cl
      s\Incl t, x\Incl f(s), s\Incl t, x\Incl f(s),\GSD, p\Incl 
q(f(t)) & \rabove{\Ear}
}
\]
Another case is as follows:
\[\sma{ \prarc{
f(s)\Incl f(t), x\Incl s,\GSD, p\Incl q(f(x)) \cl
f(s)\Incl f(t), x\Incl s,\GSD, p\Incl q(f(s)) & \rabove{\iRsr} \cl
f(s)\Incl f(t), x\Incl s,\GSD, p\Incl q(f(t)) & \rabove{\iRsnr} 
} }\convd
\sma{ \prarc{
f(x)\Incl f(t), f(s)\Incl f(t), x\Incl s,\GSD, p\Incl q(f(x)) \cl
f(x)\Incl f(t), f(s)\Incl f(t), x\Incl s,\GSD, p\Incl q(f(t)) & \rabove{\iRsnr} \cl
f(s)\Incl f(t), f(s)\Incl f(t), x\Incl s,\GSD, p\Incl q(f(t)) & \rabove{\iLar} 
} }
\]
The new application of $\iLar$ is reduced.
\item $\iLar$ or $\eLar$ are treated analogously
\[\sma{ \prarc{
p(x)\Incl q, x\preceq t,\GSD, r\Incl f(p(t)) \cl
p(t)\Incl q, x\preceq t,\GSD, r\Incl f(p(t)) &\rabove{\pLar} \cl
p(t)\Incl q, x\preceq t,\GSD, r\Incl f(q) &\rabove{\iRsnr} 
} \conv
\prarc{
p(t)\Incl q, p(x)\Incl q, x\preceq t,\GSD, r\Incl f(p(t)) \cl
p(t)\Incl q, p(x)\Incl q, x\preceq t,\GSD, r\Incl f(q) & \rabove{\iRsnr} \cl
p(t)\Incl q, p(t)\Incl q, x\preceq t,\GSD, r\Incl f(q) & \rabove{\pLar}
} }
\]
\item $\eRar$ -- since this requires a variable in the LHS of the modified 
formula, the subsequent application will be reduced $\iRsr$:
\[\prar{
x\Incl t(p), p=r, \GSD, w\Incl q(x) \cl
x\Incl t(r), p=r, \GSD, w\Incl q(x) & \rabove{\eRa} \cl
x\Incl t(r), p=r, \GSD, w\Incl q(t(r)) & \rabove{\iRsr} 
}
%\conv
%\prar{
%s\Incl t(r), s\Incl t(p), p=r, \GSD, w\Incl q(s) \cl
%s\Incl t(r), s\Incl t(p), p=r, \GSD, w\Incl q(t(r)) & \rabove{\iRsnr} \cl
%s\Incl t(r), s\Incl t(r), p=r, \GSD, w\Incl q(t(r)) & \rabove{\eRa}
%} 
\]
\item $\Esr$
\[\sma{ \prar{
x\Incl t, q(t)\Incl p,\GSD, r\Incl w(q(x)) \cl
x\Incl t, q(t)\Incl p,\GSD, r\Incl w(q(t))  & \rabove{\Esr} \cl
x\Incl t, q(t)\Incl p,\GSD, r\Incl w(p)  & \rabove{\iRsnr}
} \conv
\prar{
x\Incl t, q(x)\Incl p, q(t)\Incl p,\GSD, r\Incl w(q(x)) \cl
x\Incl t, q(x)\Incl p, q(t)\Incl p,\GSD, r\Incl w(p))  & \rabove{\iRsnr} \cl
x\Incl t, q(t)\Incl p, q(t)\Incl p,\GSD, r\Incl w(p))  & \rabove{\iLar} \cl
x\Incl t, q(t)\Incl p, q(t)\Incl p,\GSD, r\Incl w(p))  & \rabove{\Esr} 
} }
\]
\item The rules $\eLs$ and $\iLs$ can be trivially swapped since they affect only
the LHS of the inclusion in the consequent. The rule $\Ear$ requires
a variable in the LHS of the modified formula, so if this formula is active in a 
subsequent application of $\iRs$ this must be $\iRsr$. By the assumption, we do not
have applications of $\eRs$. $(cut_x)$ and $(Sp.cut)$ can be trivially swapped.
\end{LS}
The new applications of $\pLa$ are all reduced and in case 2) are of the same kind
as in the original derivation. Also, no new applications of $\eRa$ appear. 
Hence $\#\pLanr$ and $\#\eRanr$ are not increased.
\end{PROOF}

%%\begin{LEMMA}\label{le:noeRsnr}
%%$\der{NEQ_{4}}DS\ \impl\ \der{NEQ_{4}}{D^{*}}S$ and $D^*$ contains no 
%%applications of 
%%\[
%%\eRsnr\ \ \ \mprule{s=t,\GSD, p\Incl q(s)}{s=t,\GSD,p\Incl q(t)}\ \ 
%%s\not\in\Vars
%%\]
%%Moreover $\#(R,D^*) \leq \#(R,D)$ for $R$ any rule among
%%$\pLanr$, $\eRanr$ and $\iRsnr$.
%%\end{LEMMA}
%%\begin{PROOF}
%%By lemmas \ref{le:noar} and \ref{le:noiRsnr}, we assume that $D$ contains only 
%%reduced applications of  $\iLar$, $\eRar$ and $\iRsr$.
%%For an application to the axiom we have the following transformation:
%%\[\prarc{
%%s=t, x\Incl p(s),\GSD, x\Incl p(s) \cl
%%s=t, x\Incl p(s),\GSD, x\Incl p(t) & \rabove{\eRsnr}
%%} \conv
%%\prarc{
%%s=t, x\Incl p(t),\GSD, x\Incl p(t) \cl
%%s=t, x\Incl p(s),\GSD, x\Incl p(t) & \rabove{\eRar}
%%}
%%\]
%%Consider the last rule applied above the uppermost $\eRsnr$
%%\begin{LS}
%%\item $\iRsr$
%%\[\sma{ \prarc{
%%s=t, x\Incl p(s),\GSD, q\Incl f(x) \cl
%%s=t, x\Incl p(s),\GSD, q\Incl f(p(s)) & \rabove{\iRsr} \cl
%%s=t, x\Incl p(s),\GSD, q\Incl f(p(t)) & \rabove{\eRsnr} 
%%} \mconv
%%\prarc{
%%x\Incl p(t), s=t, x\Incl p(s),\GSD, q\Incl f(x) \cl
%%x\Incl p(t), s=t, x\Incl p(s),\GSD, q\Incl f(p(t)) &\rabove{\iRsr} \cl
%%x\Incl p(s), s=t, x\Incl p(s),\GSD, q\Incl f(p(t)) &\rabove{\eRar} 
%%} }
%%\]
%%\item $\iLar$
%%\[\sma{ \prarc{
%%s(x)=t, x\Incl r,\GSD, q\Incl f(s(r)) \cl
%%s(r)=t, x\Incl r,\GSD, q\Incl f(s(r)) & \rabove{\iLar} \cl
%%s(r)=t, x\Incl r,\GSD, q\Incl f(t) & \rabove{\eRsnr} 
%%}\mconv
%%\prarc{
%%s(r)=t, s(x)=t, x\Incl r,\GSD, q\Incl f(s(r)) \cl
%%s(r)=t, s(x)=t, x\Incl r,\GSD, q\Incl f(t) &\rabove{\eRsnr} \cl
%%s(r)=t, s(r)=t, x\Incl r,\GSD, q\Incl f(t) &\rabove{\iLar}
%%} }
%%\]
%%\item $\eRsr$
%%\[\sma{ \prarc{
%%x=q(s), t=s, \GSD, w\Incl f(x) \cl
%%x=q(s), t=s, \GSD, w\Incl f(q(s)) & \rabove{\eRsr} \cl
%%x=q(s), t=s, \GSD, w\Incl f(q(t)) & \rabove{\eRsnr} 
%%} \mconv
%%\prarc{
%%x\Incl q(t), x=q(s), t=s, \GSD, w\Incl f(x) \cl
%%x\Incl q(t), x=q(s), t=s, \GSD, w\Incl f(q(t)) & \rabove{\iRsr} \cl
%%x\Incl q(s), x=q(s), t=s, \GSD, w\Incl f(q(t)) & \rabove{\eRar} \cl
%%x\Incl x, x=q(s), t=s, \GSD, w\Incl f(q(t)) & \rabove{\eRar} \cl
%%x=q(s), t=s, \GSD, w\Incl f(q(t)) & \rabove{(cut_x)} 
%%} }
%%\]
%%\item $\Esr$
%%\[\sma{ \prar{
%%x\Incl q(s), s=t, \GSD, w\Incl f(x) \cl
%% s=t, \GSD, w\Incl f(q(s)) & \rabove{\Esr} \cl
%% s=t, \GSD, w\Incl f(q(t)) & \rabove{\eRsnr}
%%} \mconv
%%\prar{
%%x\Incl q(t), x\Incl q(s), s=t, \GSD, w\Incl f(x) \cl
%%x\Incl q(t), x\Incl q(s), s=t, \GSD, w\Incl f(q(t)) & \rabove{\iRsr} \cl
%%x\Incl q(s), x\Incl q(s), s=t, \GSD, w\Incl f(q(t)) & \rabove{\eRar} \cl
%%             x\Incl q(s), s=t, \GSD, w\Incl f(q(t)) \cl
%%              s=t, \GSD, w\Incl f(q(t)) & \rabove{\Esr} 
%%} }
%%\]
%%\item Other rules can be easily swapped: 
%%$\iLs$ and $\eLs$ modify only LHS of $\Incl$; $\Ear$, $\eRar$ and $\eLar$ modify an 
%%inclusion, i.e., not the active formula of a subsequent application of $\eRsnr$;
%%$(cut_x)$ and $(Sp.cut)$ are obvious.
%%\end{LS}
%%As we see, no new non-reduced applications appear in the transformed derivations, 
%%so the side condition of the lemma is satisfied.
%%\end{PROOF}
\begin{REMARK}\label{re:crucial}
%The side-conditions of lemmas \ref{fa:noeRs} and 
Lemma \ref{le:nosx} allows us to 
conclude that the applications of $\eRs$, $\eLs$ and $\iLs$ can be
eliminated from any derivation in $NEQ_4$.

Similarly, though alternatively, combining lemmas~\ref{le:noar} and \ref{le:noiRsnr} 
%and \ref{le:noeRsnr}
-- in this order, and with their side-conditions -- allows us to conclude 
that the applications of $\iLanr$, $\eRanr$, $\eRs$ and $\iRsnr$ can be 
eliminated. 

These two facts will be of crucial importance in the proof of cut-elimination in
\ref{sub:cut}.
\end{REMARK}
%
\subsection{Equivalence $NEQ_{4}\equiv NEQ_{3}$}\label{sub:equiv}
The rules of $NEQ_{4}$ are merely restricted versions of the rules of 
$NEQ_{3}$, so we trivially have $NEQ_{4}\impl NEQ_{3}$. For the 
opposite implication, we need to show admissibility in $NEQ_{4}$ of 
the unrestricted rules. Lemma~\ref{le:noeqeq} showed admissibility 
of $\eae$ and lemma~\ref{le:noLanr} of $\eLanr$. 
We now show admissibility in $NEQ_{4}$ of the remaining rules $(E_{ax})$, $(E_{sx})$ 
and $(cut_{t})$. % and $(cut_{x=})$ 

\begin{LEMMA}\label{le:noEad}\label{le:noEsd} 
% $\der{NEQ_4}DS\ \impl\ \der{NEQ_4}{D'}S$ and $D'$ uses no applications of 
The rules $\Eax$ and $\Esx$ are admissible in $NEQ_4$.
\end{LEMMA}
\begin{PROOF}
Consider first the uppermost application of $\Eax$.
Renaming the eigen-variable $z$ in the whole $D_1$ to $x$, and applying ($cut_x$)
to the result of $D_1'$, we get the desired result. (Remember that we may have $r(z)=z$.)
\[ \begin{array}{cl}
D_1 \left \{ \begin{array}{c}
 \vdots \\
 z\Incl x, y\Incl r(z), \Gamma\Seq\Delta \end{array} \right . \\ \cline{1-1}
 y \Incl r(x), \Gamma\Seq\Delta & \rabove{\Eax}
\end{array} 
\conv
\begin{array}{cl}
D_1' \left \{ \begin{array}{c}
 \vdots \\
 x\Incl x, y\Incl r(x), \Gamma\Seq\Delta \end{array} \right . \\ \cline{1-1}
 y \Incl r(x), \Gamma\Seq\Delta & \raisebox{1.2ex}[1.5ex][0ex]{$(cut_x)$}
\end{array} \]
In a similar way, we consider the uppermost application of $\Esx$:
\[ \begin{array}{cl}
D\left\{ \begin{array}{c}
 \vdots \\
 \Gamma,x\Incl y\Seq\Delta \end{array} \right. \\ \cline{1-1}
 \Gamma\Seq\Delta_y^x  &  \raisebox{1.2ex}[1.5ex][0ex]{($E_{sx}$)}
\end{array} \]
In all applications of $\Esr$ and $\Ear$ in $D$, replace the
eigen-variables so that they differ from the eigen-variable $x$ and from $y$. 
In the obtained
derivation, replace all the occurrences of $x$ by $y$. In this way, we 
obtain a derivation of $\Gamma,y\Incl y\Seq\Delta_y^x$. 
Application of ($cut_x$) gives the desired derivation $D^*$ of the
sequent $\Gamma\Seq\Delta_y^x$ 
without using degenerated applications of ($E_{sx}$).
\end{PROOF}
%
%% \begin{LEMMA}\label{le:noEsd} 
%% The rule ($E_{sx}$) is admissible in $NEQ_4$.
%% \end{LEMMA}
%% \begin{PROOF}
%% By induction on the number of applications of ($E_{sx}$) in a given
%% derivation $D$. Consider the uppermost application:
%% \[ \begin{array}{cl}
%% D\left\{ \begin{array}{c}
%%  \vdots \\
%%  \Gamma,x\Incl y\Seq\Delta \end{array} \right. \\ \cline{1-1}
%%  \Gamma\Seq\Delta_y^x  &  \raisebox{1.2ex}[1.5ex][0ex]{($E_{sx}$)}
%% \end{array} \]
%% In all applications of $\Esr$ and $\Ear$ in $D$, replace the
%% eigen-variables so that they differ from the eigen-variable $x$ and from $y$. 
%% In the obtained
%% derivation, replace all the occurrences of $x$ by $y$. In this way, we 
%% obtain a derivation of $\Gamma,y\Incl y\Seq\Delta_y^x$. 
%% Application of ($cut_x$) gives the desired derivation $D^*$ of the
%% sequent $\Gamma\Seq\Delta_y^x$ 
%% without using degenerated applications of ($E_{sx}$).
%% \end{PROOF}
%
%We show that ($cut_t$) is admissible in $NEQ_4$.
\begin{LEMMA}\label{le:nott} $\der{NEQ_4}D{t\Incl t,\Gamma\Seq\Delta}  
\ \ \Rightarrow \ \ \der{NEQ_4}{D^*}{\Gamma\Seq\Delta}$.
%[$D^*$ does not introduce new applications of $(cut_{x=})$.]
%, and $D^*$ does not use ($E_{sx}$) if $D$ does not use it.
\end{LEMMA}
\begin{PROOF} 
By lemma~\ref{le:noar} %\ref{le:noLanr}, $\eLanr$,
we assume that $D$ contains no applications of  $\iLanr$, $\eRanr$

By induction on $\<\gamma[t],h(D)\>$, where $\gamma[t]$ is the complexity of 
the term $t$ (defined in the usual way). 
If $\gamma[t]=0$, i.e., $t$ is a variable $x$,  application of ($cut_x$) yields the 
conclusion. This is all that may happen if $h(D)=1$, i.e. we merely modify once
some axiom. (Superposition of the antecedent in $x\Incl t,\GSD,x\Incl t$ leads to
$t\Incl t,x\Incl t,\GSD,x\Incl t$ -- removal of $t\Incl t$ yields the same axiom.)

So let $\gamma[t]>0$ and $R$ be the last rule applied.
 The only rules which may generate $t\Incl t$ in the antecedent
are $\iLar$, $\Ear$ and $\eRar$ or
$\eLar$. We consider these cases by looking at the rule $R_1$ applied just above
%
\begin{LS}
\item $R$ is $\iLar$ and $D$ ends as shown below, where $t$ is $t(s)$:
\[ D \left \{ \begin{array}{rl}
\multicolumn{1}{c}{\vdots} \\ \cline{1-1}
t(y)\Incl t(s), y\Incl s, \Gamma\Seq\Delta & \rabove{R_1} \\ \cline{1-1}
t(s) \Incl t(s), y\Incl s, \GSD & \rabove{\iLar} \end{array} \right .\]
%
\begin{LSA}
\item $R_1$ is $\iLar$: The two possibilities are that $t(y)\Incl t(s)$ was
\ref{it:tact} active
or \ref{it:tmod} modified in $R_1$. The former requires $t(y)$ to be the variable
 $y$ and so $t(s)$ to be $s$.
\begin{LSB}
\item\label{it:tact} $y\Incl t$ was active:
\[ \begin{array}{rl}
\multicolumn{1}{c}{\vdots} \\ 
y\Incl t, y\Incl t, q(y)\preceq p, \Gamma\Seq\Delta  \\ \cline{1-1}
y\Incl t, y\Incl t, q(t)\preceq p, \Gamma\Seq\Delta & \rabove{\iLar} \\ \cline{1-1}
t\Incl t, y\Incl t, q(t)\preceq p, \GSD & \rabove{\iLar} \end{array} \]
Swapping the two applications gives a derivation with an earlier
appearance of $t\Incl t$. 
%
\item\label{it:tmod} $y\Incl t$ was modified (we write it here in full as $t(y)\Incl t(p(q))$):
\[ D \left \{ \begin{array}{rl}
\multicolumn{1}{c}{\vdots} \\ 
y\Incl p(x), t(y)\Incl t(p(q)), x\Incl q, \GSD \\ \cline{1-1}
y\Incl p(x), t(p(x))\Incl t(p(q)), x\Incl q, \GSD & \rabove{\iLar} \\ \cline{1-1}
y\Incl p(x), t(p(q))\Incl t(p(q)), x\Incl q, \GSD & \rabove{\iLar} \end{array} \right
.\]
We first use a weakened derivation with $y\Incl p(q)$
\[ \begin{array}{cl}
 D_1 \left \{ \begin{array}{rl}
\multicolumn{1}{c}{\vdots} \\ 
y\Incl p(q), y\Incl p(x), t(y)\Incl t(p(q)), x\Incl q, \GSD \\ \cline{1-1}
y\Incl p(q), y\Incl p(x), t(p(q))\Incl t(p(q)), x\Incl q, \GSD & \rabove{\iLar} 
 \end{array} \right . \\ \cline{1-1}
p(x) \Incl p(q), y\Incl p(x), t(p(q))\Incl t(p(q)), x\Incl q, \GSD &
\rabove{\iLar}  \\ \cline{1-1}
p(q) \Incl p(q), y\Incl p(x), t(p(q))\Incl t(p(q)), x\Incl q, \GSD &
\rabove{\iLar} \end{array} \]
By induction on $h(D_1)<h(D)$ we can obtain the sequent concluding $D_1$ without
 $t(p(q))\Incl t(p(q))$.
Then, the final $p(q)\Incl p(q)$, can be eliminated by induction hypothesis
on complexity of terms, since $\gamma[p(q)]< \gamma[t(p(q))]$.
\end{LSB}
%
\item $R_1$ is $\Ear$:
\[ \begin{array}{rl}
x\Incl p, y\Incl t(x), y\Incl t(p), \GSD \\ \cline{1-1}
 y\Incl t(p), y\Incl t(p), \GSD & \rabove{\Ear} \\ \cline{1-1}
 t(p)\Incl t(p), y\Incl t(p), \GSD & \rabove{\iLar} \end{array} \convd
%Weaken the derivation as follows:
 \begin{array}{rl}
x\Incl p, y\Incl t(x),    y\Incl t(p), y\Incl t(x),\GSD \\ \cline{1-1}
x\Incl p, t(x)\Incl t(x), y\Incl t(p), y\Incl t(x),\GSD & \rabove{\iLar} \\ \cline{1-1}
 t(x)\Incl t(x), y\Incl t(p), y\Incl t(p),\GSD & \rabove{\Ear} \end{array} \]
Induction on complexity (or height) gives the conclusion (having 
eliminated $t(x)\Incl t(x)$ the application of $\Ear$ becomes correct).
%
\item $R$ is $\eRar$:
\[ \begin{array}{rl}
 y\Incl t(s), y\Incl t(p), p=s, \GSD \\ \cline{1-1}
 y\Incl t(p), y\Incl t(p), p=s, \GSD & \rabove{\eRar} \\ \cline{1-1}
 t(p)\Incl t(p), y\Incl t(p), p=s, \GSD & \rabove{\iLar} \end{array} \]
Weakening the derivation  with $y\Incl t(s)$ (if $\gamma[t(s)]<\gamma[t(p)]$,
and with $y\Incl t(p)$ otherwise), we obtain:
\[ \begin{array}{rl}
y\Incl t(s), y\Incl t(s), y\Incl t(p), p=s, \GSD \\ \cline{1-1}
 t(s)\Incl t(s), y\Incl t(s), y\Incl t(p), p=s, \GSD & \rabove{\iLar} \\ \cline{1-1}
 t(s)\Incl t(s), y\Incl t(p), y\Incl t(p), p=s, \GSD & \rabove{\eRar} \end{array} \]
Induction on height gives the conclusion. 
%
\item $R$ is $\eLar$:
\[ D \left \{ \begin{array}{rl}
t(x)\Incl t(s(p)), x=s(y), y\Incl p, \GSD \\ \cline{1-1}
t(s(y))\Incl t(s(p)), x=s(y), y\Incl p, \GSD & \rabove{\eLar} \\
\cline{1-1}
t(s(p))\Incl t(s(p)), x=s(y), y\Incl p, \GSD & \rabove{\iLar} \end{array}
\right . \convd
%Weakening $D$ we may obtain:
 \begin{array}{cl}
D_1 \left \{ \begin{array}{rl}
x\Incl s(p), t(x)\Incl t(s(p)), x=s(y), y\Incl p, \GSD \\ \cline{1-1}
x\Incl s(p), t(s(p))\Incl t(s(p)), x=s(y), y\Incl p, \GSD & \rabove{\iLar} \end{array}
\right . \\ \cline{1-1}
s(y)\Incl s(p), t(s(p))\Incl t(s(p)), x=s(y), y\Incl p, \GSD &
\rabove{\eLar} \\ \cline{1-1}
s(p)\Incl s(p), t(s(p))\Incl t(s(p)), x=s(y), y\Incl p, \GSD &
\rabove{\iLar} \end{array} \]
Occurence of $t(s(p))\Incl t(s(p))$ in $D_1$ can be eliminated by induction
hypothesis $h(D_1)<h(D)$, and of $s(p)\Incl s(p)$ by $\gamma[s(p)] < \gamma[t(s(p))]$.
%
\item All the remaining cases can be easily swapped yielding an earlier
occurrence of $t\Incl t$.

$\Esr$ :  Since the active and the side formula of the subsequent
$\iLar$ have to share a variable in the LHS, neither might be active in 
this application of $\Esr$.

$\iLs$, $\iRs$ : Activity of $t(y)\Incl t(s)$ in this rule, can be simulated by
the activity of $y\Incl s$ which also must be present.

$\eLs$, $\eRs$, ($Sp.cut$), ($cut_x$) and $(cut_{x=})$ are trivial.
\end{LSA}
%
\item  $R$ is $\eLar$ and $D$ ends as shown below :
\[ D \left \{ \begin{array}{rl}
\multicolumn{1}{c}{\vdots} \\ \cline{1-1}
t(y)\Incl t(s), y=s, \Gamma\Seq\Delta & \rabove{R_1} \\ \cline{1-1}
t(s) \Incl t(s), y=s, \GSD & \rabove{\eLar} \end{array} \right .\]
%
\begin{LSA}
%
\item $R_1$ is $\iLar$ The two possibilities are that $t'(y)\Incl t$ was
\ref{it:yact} active
or \ref{it:ymod} modified in $R_1$. The former requires $t'(y)$ to be the variable
$y$.
\begin{LSB}
\item\label{it:yact} $y\Incl t$ active:
\[ D \left \{ \begin{array}{rl}
\multicolumn{1}{c}{\vdots} \\ 
y\Incl t, y=t, q(y)\preceq p, \Gamma\Seq\Delta  \\ \cline{1-1}
y\Incl t, y=t, q(t)\preceq p, \Gamma\Seq\Delta & \rabove{\iLar} \\ \cline{1-1}
t \Incl t, y=t, q(t)\preceq p, \GSD & \rabove{\eLar} \end{array} \right
. \convd
%Instead, we may use
  \begin{array}{cl} D_1 \left \{ \begin{array}{rl}
\multicolumn{1}{c}{\vdots} \\ 
y\Incl t, y=t, q(y)\preceq p, \Gamma\Seq\Delta  \\ \cline{1-1}
t\Incl t, y=t, q(y)\preceq p, \Gamma\Seq\Delta & \rabove{\eLar} \end{array}
\right . \\ \cline{1-1}
t \Incl t, y=t, q(t)\preceq p, \GSD & \rabove{\eLar} \end{array} \]
and induction hypothesis, $h(D_1)<h(D)$, gives the conclusion. 
%\begin{REMARK}
%This merely exemplifies  the fact that activity of $y\Incl t$, can be
%simulated by activity of $y=t$: in 
 %\end{REMARK}
%
\item\label{it:ymod} $y\Incl t$ was modified (we write it as $q(y)\Incl q(p(t))$)
\[ D \left \{ \begin{array}{rl}
\multicolumn{1}{c}{\vdots} \\ 
y\Incl p(x), q(y)\Incl q(p(t)), x=t, \GSD \\ \cline{1-1}
y\Incl p(x), q(p(x))\Incl q(p(t)), x=t, \GSD & \rabove{\iLar} \\ \cline{1-1}
y\Incl p(x), q(p(t))\Incl q(p(t)), x=t, \GSD & \rabove{\eLar} \end{array} \right
. \convd
%Then we derive the same sequent
 D' \left \{ \begin{array}{cl} D_1 \left \{ \begin{array}{rl}
\multicolumn{1}{c}{\vdots} \\ 
y\Incl p(t), y\Incl p(x), q(y)\Incl q(p(t)), x=t, \GSD \\ \cline{1-1}
y\Incl p(t), y\Incl p(x), q(p(t))\Incl q(p(t)), x=t, \GSD & \rabove{\iLar} \end{array}
\right . \\ \cline{1-1}
y\Incl p(x), y\Incl p(x), q(p(t))\Incl q(p(t)), x=t, \GSD & \rabove{\eRar} \end{array} \right
.\]
\end{LSB}
%
\item $R_1$ is $\Ear$
\[ D \left \{ \begin{array}{rl}
x\Incl p, y\Incl t(x), t(p)=y, \GSD \\ \cline{1-1}
y\Incl t(p), t(p)=y, \GSD  & \rabove{\Ear} \\ \cline{1-1}
t(p)\Incl t(p), t(p)=y, \GSD  & \rabove{\eLar} \end{array} \right. \convd
%Construct:
 D' \left \{ \begin{array}{rl}
x\Incl p, y\Incl t(x), t(p)=y, \GSD \\ \cline{1-1}
y\Incl t(p), t(p)=y, \GSD  & \rabove{\Ear} \\ \cline{1-1}
y\Incl y, t(p)=y, \GSD  & \rabove{\eRar} \end{array} \right . \]
and $y\Incl y$ can be eliminated by induction hypothesis on $\gamma[t]$.
%
\item $R_1$ is $\eRar$ Then LHS of the side inclusion must be a variable,
i.e., 
\[ D \left \{ \begin{array}{rl}
y\Incl t(p), p=q, y=t(q), \GSD \\ \cline{1-1}
y\Incl t(q), p=q, y=t(q), \GSD & \rabove{\eRar} \\ \cline{1-1}
t(q)\Incl t(q), p=q, y=t(q), \GSD & \rabove{\eLar} \end{array} \right . \convd
%We construct $D'$ as follows:
 D' \left \{ \begin{array}{cl}
  D_1 \left \{ \begin{array}{rl}
 y\Incl t(p), p=q, y=t(q), y=t(p), \GSD \\ \cline{1-1}
t(p)\Incl t(p), p=q, y=t(q), y=t(p), \GSD & \rabove{\eLar} 
\end{array} \right . \\ \cline{1-1}
t(p)\Incl t(p), p=q, y=t(q), y=t(q), \GSD & \rabove{(=_{a=})} 
\end{array} \right . \]
 $h(D_1)<h(D)$ yields the conclusion.
%
\item $R_1$ is $\eLar$ 
\[ D \left \{ \begin{array}{rl}
 t(x)\Incl t(s(p)), y=p, x=s(y), \GSD \\ \cline{1-1}
 t(s(y))\Incl t(s(p)), y=p, x=s(y), \GSD & \rabove{\eLar} \\ \cline{1-1}
 t(s(p))\Incl t(s(p)), y=p, x=s(y), \GSD & \rabove{\eLar} \end{array}
 \right . \convd
%Weakening we get:
 \begin{array}{cl} D_1 \left \{ \begin{array}{rl}
 x=s(p), t(x)\Incl t(s(p)), y=p, x=s(y), \GSD \\ \cline{1-1} 
 x=s(p), t(s(p))\Incl t(s(p)), y=p, x=s(y), \GSD & \rabove{\eLar} 
\end{array} \right .  \\ \cline{1-1}
 x=s(y), t(s(p))\Incl t(s(p)), y=p, x=s(y), \GSD & \rabove{(=_{a=})} 
\end{array} \]
 $h(D_1)<h(D)$ gives the conclusion. \\
In the last two cases we have used lemma~\ref{le:noeqeq} which admits $\eae$.

%
\item All other cases of $R_1$ can be easily swapped leading to an earlier
occurrence of $t\Incl t$ (cf. the end of case 1. for $R$ = $\iLar$)
%
\end{LSA}
%
\item  $R$ is $\Ear$ This would require a degenerate application of this
rule which does not belong to $NEQ_4$ (cf. lemma \ref{le:noEad}). 
%
\item  $R$ is $\eRar$ Since this requires a variable in the LHS, we are
back in the ground case for $\gamma[t]=0$.
\end{LS}
%As we see, no new applications of $(cut_{x=})$ appear in the transformed derivations.
\end{PROOF}

\noindent
The remark at the very begining of this section \ref{sub:equiv}, together with lemmas
\ref{le:noeqeq}, \ref{le:noLanr}, \ref{le:noEsd} and
\ref{le:nott} (showing the admissibility of the 
missing $NEQ_{3}$ rules in $NEQ_{4}$), give us the first part of the 
following fact from which the second follows trivially.
\begin{CLAIM}\label{le:neq3isneq4}
$NEQ_3\equiv NEQ_4$, and thus, also $NEQ_{3}^{c}\equiv NEQ_{4}^{c}$.
\end{CLAIM}
% \noindent
% As an immediate corollary of this lemma, we have a strengthening of 
% lemma~\ref{le:neq3cisneq4c}: 
% \begin{COROLLARY}\label{co:neq3isneq4}
% $NEQ_3\equiv NEQ_4$.
% \end{COROLLARY}
% Thus, rules admissible in $NEQ_3$ are admissible in $NEQ_4$ too.

%
%% \begin{LEMMA}\label{le:noErsd} $\der{NEQ_4}DS\ \impl\ \der{NEQ_4}{D^*}S$
%% and $D^*$ contains no applications of ($E_{Rsd}$).
%% \end{LEMMA}
%% \begin{PROOF}
%% By induction on the number of applications of ($E_{Rsd}$) in a given
%% derivation $D$. 
%% By lemma~\ref{le:noar} we assume no applications of $\eRanr$ or $\iLanr$ in $D$.
%% %By lemma~\ref{le:noEad} %, resp. \ref{le:noEsd} 
%% %we assume that $D$ contains no applications of $(E_{ax})$.
%% %, resp. $(E_{sx})$ 
%% \[ \begin{array}{cl}
%% D_1\left\{ \begin{array}{cl}
%%  \vdots & \\ \cline{1-1}
%%  x\Incl t,\GSD,p\preceq x & \raisebox{1.2ex}[1.5ex][0ex]{$R$}\end{array} \right. \\ \cline{1-1}
%%  \GSD,p\preceq t  &  \raisebox{1.2ex}[1.5ex][0ex]{($E_{Rsd}$)}
%% \end{array} \]
%% \begin{LS}
%% \item $R$ is $\eRar$ %$(=_{Rar})$
%% \[ \begin{array}{cl}
%% s=r, x\Incl t(r),\GSD, p\preceq x  & \\ \cline{1-1}
%% s=r, x\Incl t(s),\GSD,p\preceq x & \rabove{\eRar} \\ \cline{1-1}
%% s=r,\GSD,p\preceq t(s)  &  \rabove{(E_{Rsd})}
%% \end{array} 
%% \conv 
%% \begin{array}{cl}
%%  s=r,x\Incl t(r),\GSD, p\preceq x  & \\ \cline{1-1}
%%  s=r, \GSD,p\preceq t(r) & \rabove{(E_{Rsd})} \\ \cline{1-1}
%%  s=r, \GSD,p\preceq t(s) & \rabove{\eRs} 
%% \end{array}
%% \]
%% ($E_{Rsd}$) can be eliminated by induction on $h(D)$. 
%% %The application of ($=_{Rs}$), admissible in $NEQ_3$ (lemma~\ref{le:noeqSD}) and, 
%% %by corollary~\ref{le:neq3cisneq4c} in $NEQ_4$, does not then introduce any new
%% %applications of $E_s$ by the qualification in lemma~\ref{le:noeqSD}.
%% \item $R$ is $\iLar$ 
%% \[ \begin{array}{cl}
%% x\Incl t, w(x)\Incl q,\GSD, p\preceq x  & \\ \cline{1-1}
%% x\Incl t, w(t)\Incl q,\GSD,p\preceq x & \rabove{\iLar} \\ \cline{1-1}
%% w(t)\Incl q, \GSD,p\preceq t  &  \rabove{(E_{Rsd})}
%% \end{array} 
%% \conv 
%% \begin{array}{cl}
%% x\Incl t, w(x)\Incl q,\GSD, p\preceq x  & \\ \cline{1-1}
%% x\Incl t, w(x)\Incl q,\GSD,p\preceq t & \rabove{(E_{Rsd})} \\ \cline{1-1}
%% w(t)\Incl q, \GSD,p\preceq t  &  \rabove{\iLar}
%% \end{array}
%% \]
%% \item $R$ is $(E_a)$ 
%% \[ \begin{array}{cl}
%% x\Incl t(z), z\Incl q,\GSD, p\preceq x  & \\ \cline{1-1}
%% x\Incl t(q), \GSD,p\preceq x & \raisebox{1.2ex}[1.5ex][0ex]{($E_{a}$)} \\ \cline{1-1}
%%  \GSD,p\preceq t(q)  &  \raisebox{1.2ex}[1.5ex][0ex]{($E_{Rsd}$)}
%% \end{array} 
%% \conv 
%% \begin{array}{cl}
%% x\Incl t(z), z\Incl q,\GSD, p\preceq x  & \\ \cline{1-1}
%% z\Incl t(q), \GSD,p\preceq t(z) & \raisebox{1.2ex}[1.5ex][0ex]{($E_{Rsd}$)} \\ \cline{1-1}
%%  \GSD,p\preceq t(q) & \raisebox{1.2ex}[1.5ex][0ex]{($E_{s}$)} 
%% \end{array}
%% \]
%% By lemma~\ref{le:noEad} the application of $(E_a)$ is not $(E_{ax})$, i.e. $t(z)\not=z$.
%% Hence, the resulting application of $(E_s)$ is not $(E_{Rsd})$.
%% \end{LS}
%% The rest is simple. $(=_{Lar})$ cannot result in $x\Incl t$, so it can be swapped.
%% Similarly, $(\preceq_{sr})$ modify at most LHS in the consequent, and $(E_s)$ involves 
%% another eigen-variable, so they can be swapped.
%% as well.
%% \end{PROOF}

\subsection{Admissibility of ($cut$).}\label{sub:cut}
Referring to the remark~\ref{re:crucial}, 
in the following proof of ($cut$)-elimination, 
we may introduce additional
\\[1ex]
\noindent
{\bf Assumptions:} If $\der{NEQ_4}DS$ then $\der{NEQ_4}{D^*}S$ where $D^*$ contains 
\vspace*{1ex}
%Any derivation $D$ in $NEQ_4$ can be transformed so that
%it contains

\hspace*{-1em}\begin{tabular}{rll}
1.& either no $\eRs$, $\eLs$, $\iLs$ 
 & lemma  \ref{le:nosx} (\ref{fa:noeRs}) \\ % subsection \\
2.& or no $\iLanr$, $\eRanr$, $\eRs$, $\iRsnr$
 & lemma \ref{le:noar}, \ref{le:noiRsnr} \\[.5ex]
% (3.)& no formula $x=x$ in the antecedent & lemma \ref{le:noxx} \\
% (4.)& no formula $t\Incl t$ in the antecedent & lemma \ref{le:nott} \\[.5ex]
\multicolumn{3}{l}{\hspace*{-1em}Also, $NEQ_4$ admits the rules} \\[.5ex]
3.& $\eae$, $\iLanr$, $\eRanr$, $\eLanr$  and $\iRa$ 
 & lemma \ref{le:noeqeq}, \ref{le:noar}, \ref{le:noLanr}, \ref{le:inclaad} 
 \\[.5ex]
& without increasing the total number of $\iLar$ and $\eLar$  \\[.5ex]
4.& $(cut_t)$    %and $(cut_{x=})$ 
  & lemma \ref{le:nott} % \ref{le:noxx}
\end{tabular} 

 
\begin{LEMMA}\label{le:elcut}
If $D$ is a $NEQ_4^c$ and $D_1$, $D_2$ are $NEQ_4$ derivations as follows:
%\begin{center}
\[ D\left\{ \begin{array}{cl}
 D_1\left\{ \begin{array}{c}
  \vdots \\   \Gamma\Seq\Delta, \phi
 \end{array} \right.
 D_2\left\{ \begin{array}{c}
  \vdots \\   \phi, \Gamma'\Seq\Delta'
 \end{array} \right. \\ \cline{1-1}
\Gamma,\Gamma' \Seq \Delta,\Delta'
&   \raisebox{1.2ex}[1.5ex][0ex]{$(cut)$}
\end{array}\right. \]
%\end{center}
then $\der{NEQ_4}{}{\Gamma,\Gamma'\Seq\Delta,\Delta'}$.
\end{LEMMA}
\begin{PROOF}
By induction on $\< \#(\preceq_a,D_2), h(D_1), h(D_2)\>$, 
where the first parameter is the total number of applications of ($\Incl_{ar}$) 
{\em or} ($=_{Lar}$) {\em which modify the cut formula} $\phi$ in $D_2$ (this
qualification is essential in case \ref{it:cutactive} on page
\pageref{it:cutactive}).
%  $h(D_1)$ is the height of $D_1$ and $h(D_2)$ the height of  $D_2$. 
\\[1ex]
\noindent 
If $h(D_1)=0$, then the resulting sequent of $D_1$ is an axiom. If
the cut formula is not the one mentioned explicitly in the axioms (Fig.~\ref{fi:neq4}) 
then we obtain a cut-free derivation directly by choosing
another instance of the same axiom. In the other case, if the cut formula is
 $t\Incl t$ or $x=x$, the assumption 4. or 3., respectively, allows us to conclude the
 existence of a derivation of the conclusion of $D_2$ without this formula in
 the antecedent. Finally, if the axiom in $D_1$ is
 $\Gamma,s=t\Seq s=t,\Delta$ and the cut formula is $s=t$, then it will also appear in the
 antecedent of the conclusion of ($cut$). Then this conclusion can be
 obtained without ($cut$) directly from $D_2$ by starting it with the
 instance of the axiom extended with $\Gamma$ and $\Delta$. \\[1ex]
\noindent
For $h(D_1)>0$, we consider the last rule $R$ applied in $D_1$. \\[1ex]
\noindent
{\bf I.} ($=_{sr}$):
\[ D \left \{ \begin{array}[t]{cl}
 \begin{array}{cl}
 D_1^*\left\{ \begin{array}{c}
  \vdots \\   t=s,\Gamma\Seq\Delta, w(s)\preceq q
 \end{array} \right. \\ \cline{1-1}
t=s,\Gamma\Seq\Delta, w(t)\preceq q & \raisebox{1.2ex}[1.5ex][0ex]{($=_{sr}$)}
 \end{array}
 D_2\left\{ \begin{array}{c}
  \vdots \\ \vdots \\  w(t)\preceq q, \Gamma'\Seq\Delta'
 \end{array} \right. \\ \cline{1-1}
t=s,\Gamma,\Gamma' \Seq \Delta,\Delta'
&   \raisebox{1.2ex}[1.5ex][0ex]{($cut$)}
\end{array} \right . \convd
%
%Instead, we drop the last application in $D_1$ and construct following
%derivation from $D_2$:
 D' \left \{ \begin{array}[t]{cl}
 D_1^*\left\{ \begin{array}{c}
  \vdots \\ \vdots \\  t=s,\Gamma\Seq\Delta, w(s)\preceq q
 \end{array} \right. 
D'_2\left\{\begin{array}{cl}
 D_2\left\{ \begin{array}{c}
  \vdots \\   w(t)\preceq q, t=s, \Gamma'\Seq\Delta'
 \end{array} \right. \\ \cline{1-1}
w(s)\preceq q, t=s, \Gamma'\Seq\Delta' & \rabove{(=_a)}
 \end{array} \right.\\ \cline{1-1}
t=s,\Gamma,\Gamma' \Seq \Delta,\Delta'
&   \raisebox{1.2ex}[1.5ex][0ex]{($cut$)}
\end{array} \right . \]
 $t$ cannot be a variable, since this would imply that the application in
 $D_1$ was ($=_{sx}$), which isn't a $NEQ_4$ rule.
Hence, the application of ($=_a$) is either ($=_{a=}$) or ($=_{Lanr}$) (if it was 
($=_{Lar}$), i.e. if $t\in\Vars$, we would get increas of the induction parameter). 
The former can
be eliminated by lemma \ref{le:asinNEQ3}, and the latter by \ref{le:noLanr} 
(and corollary~\ref{co:neq3isneq4}) --
both without intcreasing the number of applications of ($\preceq_a$).
Thus, $\#(\preceq_a, D'_2)$ is not greater
than in the original $D_2$, while the height of $D_1$ at which to
perform ($cut$) has been reduced.  \\[1ex]
%
\noindent
{\bf II.} ($\Incl_s$): We do a similar trick as in the previous case:
%\item  
\[ D \left \{\begin{array}[t]{cl}
 \begin{array}{cl}
 D_1^*\left\{ \begin{array}{c}
  \vdots \\   s\Incl p,\Gamma\Seq\Delta, w(p)\preceq q
 \end{array} \right. \\ \cline{1-1}
s\Incl p,\Gamma\Seq\Delta, w(s)\preceq q & \raisebox{1.2ex}[1.5ex][0ex]{($\Incl_s$)}
 \end{array}
 D_2\left\{ \begin{array}{c}
  \vdots \\   w(s)\preceq q, \Gamma'\Seq\Delta'
 \end{array} \right. \\ \cline{1-1}
s\Incl p,\Gamma,\Gamma' \Seq \Delta,\Delta'
&   \raisebox{1.2ex}[1.5ex][0ex]{($cut$)}
\end{array} \right . \convd
%
%We do a similar trick as in the previous case:
 D' \left \{ \begin{array}[t]{cl}
 D_1^*\left\{ \begin{array}{c}
  \vdots \\ \vdots \\  s\Incl p ,\Gamma\Seq\Delta, w(p)\preceq q
 \end{array} \right. 
D'_2\left\{ \begin{array}{cl}
 D_2\left\{ \begin{array}{c}
  \vdots \\   w(s)\preceq q, s\Incl p, \Gamma'\Seq\Delta'
 \end{array} \right. \\ \cline{1-1}
w(p)\preceq q, s\Incl p, \Gamma'\Seq\Delta' & \rabove{(\Incl_a)}
 \end{array} \right. \\ \cline{1-1}
s\Incl p ,\Gamma,\Gamma' \Seq \Delta,\Delta'
&   \raisebox{1.2ex}[1.5ex][0ex]{($cut$)}
\end{array} \right . \]
 $s$ cannot be a variable because then the application of ($\Incl_s$) in
 $D_1$ would be ($\Incl_{sx}$) which is not a $NEQ_4$ rule 
(cf. lemma \ref{le:noInclsx}). Thus, the application of
 ($\Incl_a$) after $D_2$ in $D'_2$ is not reduced, and this can be eliminated
 by lemma \ref{le:noar} (and corollary~\ref{co:neq3isneq4}) 
without increasing the number 
 of applications of ($\preceq_a$) in $D'_2$. So ($cut$) can be eliminated
 using the second parameter of induction hypothesis. \\[1ex]
%
\noindent
{\bf III.} ($\Incl_{ar}$), ($E_a$), ($=_{Lar}$), ($=_{Rar}$), ($Sp.cut$) or ($cut_x$): 
Since none of
these rules modifies the succedent, their application as $R$ in $D_1$ means
that we can apply ($cut$) with $D_1^*$ instead, and the induction on $h(D_1$)
gives the conclusion. \\[1ex]
{\bf IV.} ($E_s$):
%
\[ \begin{array}{cl}
D_1\left\{ \begin{array}{cl}
 D_1^*\left\{ \begin{array}{cl}
  \vdots \\ 
  x\Incl t,\Gamma\Seq\Delta, \phi_x^x  
 \end{array} \right. \\ \cline{1-1}
\Gamma\Seq\Delta, \phi_t^x & \raisebox{1.2ex}[1.5ex][0ex]{($E_s$)}
 \end{array} \right .
 D_2\left\{ \begin{array}{cl}
  \vdots \\ \cline{1-1}  \phi_t^x, \Gamma'\Seq\Delta' & \raisebox{1.2ex}[1.5ex][0ex]{$R'$}
 \end{array} \right. \\ \cline{1-1}
\Gamma,\Gamma' \Seq \Delta,\Delta'
&   \raisebox{1.2ex}[1.5ex][0ex]{($cut$)}
\end{array} \]
%
By restrictions on ($E_s$) $t\not\in\Vars$ and, furthermore, 
$x\not\in\C V(\Gamma,\Delta,t)$ so we can choose
 $x$ so that $x\not\in\C V(\Gamma',\Delta')$. We consider two cases: \ref{it:Ar} when
in $D_1$, $x$ in $\phi$ is in the RHS of $\Incl$, 
and \ref{it:Bl} when $x$ occurs in
a LHS of $\Incl$ or when $\phi$ is an equality. 
\begin{LS}
%
%
\item\label{it:Ar} Let $\phi$ be $p\Incl s(t)$  \\[.5ex]
\noindent
-- $t$ indicating the only occurrence
which has been substituted for $x$. We can then 
%construct the following derivation 
extend $D_2$ and construct the following derivation:
%
\[
 \begin{array}{cl}
 D_1^*\left\{ \begin{array}{cl}
  \ \\ \ \\ \vdots \\ 
  x\Incl t,\Gamma\Seq\Delta, p\Incl s(x) 
         \end{array} \right. \ \ \ 
%
D'_2\left\{ \begin{array}{cl}
  D_2\left\{ \begin{array}{cl}
  \vdots \\ 
  p\Incl s(t),\Gamma'\Seq\Delta' \end{array} \right.  \\ \cline{1-1}
 s(x)\Incl s(t),p\Incl s(t),\Gamma'\Seq\Delta' 
   & \raisebox{1.2ex}[1.5ex][0ex]{(W)} \\ \cline{1-1}
s(x)\Incl s(t), p\Incl s(x),\Gamma'\Seq\Delta' 
& \raisebox{1.2ex}[1.5ex][0ex]{($\Incl_a^*$)}
         \end{array} \right.
%     \end{array} \right. 
%
\\ \cline{1-1}
x\Incl t, s(x)\Incl s(t), \Gamma',\GSD,\Delta' 
  & \raisebox{1.2ex}[1.5ex][0ex]{($cut$)} \\ \cline{1-1}
x\Incl t, s(t)\Incl s(t), \Gamma',\GSD,\Delta' 
  & \raisebox{1.2ex}[1.5ex][0ex]{($\Incl_{ar}$)} \\ \cline{1-1}
s(t)\Incl s(t), \Gamma',\GSD,\Delta' 
  & \raisebox{1.2ex}[1.5ex][0ex]{($E_s$)} \\ \cline{1-1}
\Gamma',\GSD,\Delta' 
& \raisebox{1.2ex}[1.5ex][0ex]{($cut_t$)} 
\end{array} 
 \]
\noindent
Notice that $x\not=s(x)\not\in\Vars$ since, otherwise, the application of 
($E_s$) in $D_1$ would be ($E_{Rsd}$) which is excluded by lemma~\ref{le:noErsd}.
Thus the rule ($\Incl_a^*$) is admissible (lemma \ref{le:inclaad}) and
does not increase the number of
applications of ($\preceq_a$), and ($cut$)
can be eliminated by induction on $h(D_1^*)<h(D_1)$.
The resulting derivation is  in $NEQ_4$ so, by lemma~\ref{le:nott},
($cut_t$) is admissible.
%
\item\label{it:Bl} Let $\phi$ be $s(t)\preceq p$ \\[.5ex]
\noindent
Here we have three subcases, depending on
whether $\phi$ in the application of the last rule $R'$ in $D_2$ was 
\ref{it:cutneither} neither modified nor active, \ref{it:cutmodified} modified, 
or \ref{it:cutactive} active.
%
\begin{LSA}
%
\item\label{it:cutneither} The cut formula $\phi$ is neither modified nor active.\\
We may then swap the
applications of $R'$ and ($cut$), and the induction on the height of $D_2$
yields the required elimination of ($cut$). 
%
\item\label{it:cutmodified}  $\phi$ is modified by $R'$.\\
%%\begin{LSA}
%%  \item $\phi$ is inclusion $s(t)\Incl p$ (modifed by $R'$).  \\[.5ex]
The two possibilities of $\phi$ being an inclusion and equality are entirely 
analogous. This case depends only on the rule $R'$ -- if $\phi$ is an equality, 
$R'$ cannot be ($=_a$), but except for that the two cases are identical.
%  {\sf II.a) 
%
 \begin{LSB}
   \item $R'$ is ($E_a$) or ($=_{Rar}$): \\
%II.a.$\alpha$ 
  This would require $\phi$ to be an inclusion with  a single
   variable in the LHS, i.e, $y\Incl p$. %$s(t)=y$. 
Since in the current case \ref{it:Bl},
  ($E_s$) in $D_1$
   replaces $x$ in the LHS of $\Incl$, this would mean that this application is
   actually degenerate ($E_{sd}$), what is excluded by lemma \ref{le:noEsd}. 
  \item  $R'$ is ($\Incl_{ar}$) or ($=_{Lar}$): \\
%II.a.$\beta$ 
  \noindent
   These two cases are identical so we treat them jointly and
  write ($\preceq_{ar}$). In each subcase, all occurrences of ($\preceq$) must be
  replaced consistently either by ($\Incl_{ar}$) or by ($=_{Lar}$), unless we
  mention the latter rules explicitly. The cut formula $\phi$ 
has the form $\phi[x]$ and $\phi[t]$ after the application of ($E_s$) in $D_1$, while in $D_2$ we 
write it as $\phi'[y]$, and as $\phi'[t']$ after the final application of ($\preceq_a$), i.e., 
 $\phi[t] = \phi = \phi'[t']$. This is to indicate that the term $t$ introduced in $D_1$ and
$t'$ introduced in $D_2$ into $\phi$ may be different. The relation between the two
 does not matter, however -- what makes it possible to treat all the (sub)cases in
the same way, is the fact that if $\phi'[y]$ is an inclusion the modified term is introduced into
its LHS.

$D_2$ ends as follows:
\[ D_2 \left \{ \begin{array}{cl}
 D_2^* \left \{ \begin{array}{cl}
 \vdots \\
 \phi'[y], y\preceq t', \Gamma'\Seq\Delta' \end{array} \right . \\ \cline{1-1}
 \phi'[t'], y\preceq t', \Gamma'\Seq\Delta' & \rabove{(\preceq_{ar})}
 \end{array} \right . \]
We drop this last application of ($\preceq_{ar}$) and, instead, extend $D_1$
weakened with $y\preceq t'$. % \\[.5ex]
%
% \mbox{ 
{ \footnotesize 
\[\begin{array}{cl}
D_1' \left \{ \begin{array}{cl}
D_1 \left \{ \begin{array}{cl}
  \vdots \\ 
  y\preceq t', x\Incl t,\Gamma\Seq\Delta, \phi[x]  \\ \cline{1-1}
y\preceq t', \Gamma\Seq\Delta, \phi[t]  & \raisebox{1.2ex}[1.5ex][0ex]{($E_s$)}
 \end{array} \right . \\ \cline{1-1}
y\preceq t', \Gamma\Seq\Delta, \phi'[y] &
\raisebox{1.2ex}[1.5ex][0ex]{($\preceq_s$)}
 \end{array} \right .
%
 D_2^*\left\{ \begin{array}{cl}
 \vdots \\
\vdots \\
\phi'[y], y\preceq t', \Gamma'\Seq\Delta' \end{array} \right .
 \\ \cline{1-1}
y\preceq t', \Gamma,\Gamma' \Seq \Delta,\Delta' &   \raisebox{1.2ex}[1.5ex][0ex]{($cut$)}
\end{array} \] }
%  \) }} \\[.5ex]
The obtained application of ($\preceq_{sx}$) is admissible by assumption 5. 
(lemma~\ref{le:noInclsx}), and so
($cut$) can be eliminated by induction hypothesis on $\#(\preceq_a,D_2)$.
%
\end{LSB}
%
%
\item\label{it:cutactive} %LSA 
 $\phi$ is active in $R'$
\begin{LSB}
\item\label{it:inact} $\phi$ is inclusion $s(t)\Incl p$ (active in $R'$),
\\
%  {\sf III.a) 
that is $R'$ is either ($\Incl_s$) or ($\Incl_{ar}$).
The latter case is excluded because it would require $s(t)$ to be a variable.
Then we would have a degenerate application of ($E_s$) in $D_1$, what is excluded by the
assumption. So let
 $R'$ be $(\Incl_s$). We have the following derivation: 
{\footnotesize \[ \begin{array}{cl} %\hspace*{-4em}
D_1 \left \{ \begin{array}{cl}
 D_1^*\left\{ \begin{array}{cl}
  \vdots \\ 
  x\Incl t,\Gamma\Seq\Delta, s(x)\Incl p  
 \end{array} \right. \\ \cline{1-1}
\Gamma\Seq\Delta, s(t)\Incl p & \raisebox{1.2ex}[1.5ex][0ex]{($E_s$)}
 \end{array} \right .
 D_2\left\{ \begin{array}{cl}
  D_2^* \left \{ \begin{array}{c}
\vdots \\
s(t)\Incl p, \Gamma'\Seq\Delta', w(p)\preceq q \end{array} \right .
\\ \cline{1-1}  
s(t)\Incl p, \Gamma'\Seq\Delta', w(s(t))\preceq q & \raisebox{1.2ex}[1.5ex][0ex]{($\Incl_s$)}
 \end{array} \right. \\ \cline{1-1}
\Gamma,\Gamma' \Seq \Delta,\Delta', w(s(t))\preceq q
 &   \raisebox{1.2ex}[1.5ex][0ex]{($cut$)}
\end{array} \] }
%} \\[.5ex]
%
First, construct the derivation $M'$ by cutting $s(t)\Incl p$ after $D_1$ and
 $D_2^*$. Since $h(D_2^*)<h(D_2)$ this ($cut$) can be eliminated using the third argument of 
induction. Then extend $M'$ to $M$  as follows:
\[ \hspace*{-1em} M \left \{ \begin{array}{cl} 
  M' \left \{ \begin{array}{cl}
 D_1 \left \{ \begin{array}{c}
 \vdots \\
 \GSD, s(t)\Incl p \end{array} \right .
 D_2^* \left \{ \begin{array}{c}
 \vdots \\
 s(t)\Incl p, \Gamma'\Seq\Delta', w(p)\preceq q \end{array} \right . \\ \cline{1-1}
 \Gamma,\Gamma'\Seq\Delta,\Delta', w(p)\preceq q & \rabove{(cut)} \end{array} \right . \\
 \cline{1-1}
 s(x)\Incl p, \Gamma,\Gamma'\Seq\Delta,\Delta',w(p)\preceq q
 &   \raisebox{1.2ex}[1.5ex][0ex]{($W_a$)} \\ \cline{1-1}
 s(x)\Incl p, \Gamma,\Gamma'\Seq\Delta,\Delta',w(s(x))\preceq q
 &   \raisebox{1.2ex}[1.5ex][0ex]{($\Incl_s$)}
 \end{array} \right . \]
The application of ($cut$) to $M$ with $D_1^*$ can be eliminated
by induction hypothesis $h(D_1^*)<h(D_1)$ -- notice that we are using here
the fact that $\#(\preceq_a,M)$ {\em modifying the cut formula} is not greater
than $\#(\preceq_a,D_2^*)$, even if the total number of arbitrary applications
of ($\preceq_a$) in $M$ may be far greater than in $D_2^*$ (due to
applications in $D_1$).
It yields the following sequent leading
to the desired conclusion by an application of ($E_s$) -- $x$ may be chosen so
that $x\Not\in\Vars(\Gamma',\Delta',w,q$):
\[ \begin{array}{cl}
 x\Incl t, \Gamma, \Gamma' \Seq \Delta, \Delta', w(s(x))\preceq q \\
 \cline{1-1}
 \Gamma, \Gamma' \Seq \Delta, \Delta', w(s(t))\preceq q
 &   \raisebox{1.2ex}[1.5ex][0ex]{($E_s$)}
\end{array} \]
\noindent
%
\item % LSB {\sf III.b) 
 $\phi$ is equality $s(t)= p$ (active in $R'$). \\
 We have three cases for $R'$ which are all treated analogously to
the previous case \ref{it:inact}.
% (III.a).
\begin{LSC}
\item $R'$ is ($=_s$) \\
This is treated exactly as \ref{it:inact} with applications of
($=_s$) instead of ($\Incl_s$). 
%
\item  $R'$ is ($=_{Lar}$) or ($=_{Rar}$)\\
We proceed as above with the construction of $M'$, $M$ and ($cut$). The
differences occur only in the last step, so we make the following generic
description where $\psi$ is the side and $\psi'$ the modified formula
of $R'$: 
% \\[.5ex]  \hspace*{-.5em} \mbox{ 
{\footnotesize
\[ \begin{array}{cl}  \hspace*{-1.5em}
D_1 \left \{ \begin{array}{cl}
 D_1^*\left\{ \begin{array}{cl}
  \vdots \\ 
  x\Incl t,\Gamma\Seq\Delta, s(x)= p  
 \end{array} \right. \\ \cline{1-1}
\Gamma\Seq\Delta, s(t)= p & \raisebox{1.2ex}[1.5ex][0ex]{($E_s$)}
 \end{array} \right .
 D_2\left\{ \begin{array}{cl}
  D_2^* \left \{ \begin{array}{c}
\vdots \\
s(t)= p, \psi, \Gamma'\Seq\Delta' \end{array} \right .
\\ \cline{1-1}  
s(t)= p, \psi', \Gamma'\Seq\Delta',  & \raisebox{1.2ex}[1.5ex][0ex]{($R'$)}
 \end{array} \right. \\ \cline{1-1}
\psi', \Gamma,\Gamma' \Seq \Delta,\Delta'
 &   \raisebox{1.2ex}[1.5ex][0ex]{($cut$)}
\end{array} \] }
% } } \\[.5ex] \noindent
First, construct the derivation $M'$ by cutting $s(t)=p$ after $D_1$ and
 $D_2^*$ which is weakened with $s(x)=p$. $h(D_2^*)<h(D_2)$ means that this ($cut$)
 can be eliminated. This
yields the following sequent
\[  s(x)=p, \psi, \Gamma, \Gamma' \Seq \Delta, \Delta' \]
leading to the desired conclusion by the procedure depending on
 $R'$. (Remember that $x$ may be chosen so
that $x\Not\in\Vars(\Gamma',\Delta',\psi$), and $\#(\preceq_a,M')$ is not
greater than in $D_2^*$):
\begin{LSD}
\item $R'$ is ($=_{Lar}$)\\
 $s(t)$ cannot be a variable (since then ($E_s$) in $D_1$ would be
 degenerate), so $p$ must be a variable $y$ and $\psi$ is $f(y)\Incl q$, and $\psi'$ is $f(s(t))\Incl q$.
\[ \begin{array}{cl}
D_1^* \left \{ \begin{array}{c} \vdots \\ 
   x \Incl t, \GSD, s(x)=y \end{array} \right . \ \ \ \ \ 
\begin{array}{rl}
 s(x)=y, f(y)\Incl q, \Gamma, \Gamma' \Seq \Delta, \Delta' \\
 \cline{1-1}
 s(x)=y, f(s(x))\Incl q, \Gamma, \Gamma' \Seq \Delta, \Delta' 
 &   \raisebox{1.2ex}[1.5ex][0ex]{($=_{Lar}$)}  
 \end{array} \\ \cline{1-1}
x\Incl t, f(s(x))\Incl q, \Gamma, \Gamma' \Seq \Delta, \Delta' 
 &   \raisebox{1.2ex}[1.5ex][0ex]{($cut$)} \\ \cline{1-1}
x\Incl t, f(s(t))\Incl q, \Gamma, \Gamma' \Seq \Delta, \Delta' 
 &   \raisebox{1.2ex}[1.5ex][0ex]{($\Incl_{ar}$)} \\ \cline{1-1}
f(s(t))\Incl q, \Gamma, \Gamma' \Seq \Delta, \Delta' 
 &   \raisebox{1.2ex}[1.5ex][0ex]{($E_s$)}
\end{array} \]
% \[ \begin{array}{rl}
%  s(x)=y, f(y)\Incl q, \Gamma, \Gamma' \Seq \Delta, \Delta' \\
%  \cline{1-1}
%  s(x)=y, f(s(x))\Incl q, \Gamma, \Gamma' \Seq \Delta, \Delta' 
%  &   \raisebox{1.2ex}[1.5ex][0ex]{($=_{Lar}$)} \\ \cline{1-1}
% x\Incl t, f(s(x))\Incl q, \Gamma, \Gamma' \Seq \Delta, \Delta' 
%  &   \raisebox{1.2ex}[1.5ex][0ex]{($cut$)\ with\ $s(x)=y$\ after\ $D_1^*$} \\ \cline{1-1}
% x\Incl t, f(s(t))\Incl q, \Gamma, \Gamma' \Seq \Delta, \Delta' 
%  &   \raisebox{1.2ex}[1.5ex][0ex]{($\Incl_a$)} \\ \cline{1-1}
% f(s(t))\Incl q, \Gamma, \Gamma' \Seq \Delta, \Delta' 
%  &   \raisebox{1.2ex}[1.5ex][0ex]{($E_s$)}
% \end{array} \]
This ($cut$) can be eliminated since $h(D_1^*)<h(D_1)$.
%
\item $R'$ is ($=_{Rar}$)\\
substituting $s(t)$ for $p$, i.e., $\psi$ is $y\Incl f(p)$, and $\psi'$ is
$y\Incl f(s(t))$ for a $y\in\Vars$:
\[ \begin{array}{cl}
D_1^* \left \{ \begin{array}{c} \vdots \\ 
 x\Incl t, \GSD, s(x)=p \end{array} \right . \ \ \ \ \ \ 
\begin{array}{rl}
 s(x)=p, y\Incl f(p), \Gamma, \Gamma' \Seq \Delta, \Delta' \\
 \cline{1-1}
 s(x)=p, y\Incl f(s(x)), \Gamma, \Gamma' \Seq \Delta, \Delta' 
 &   \raisebox{1.2ex}[1.5ex][0ex]{($=_{Rar}$)} 
 \end{array} \\ \cline{1-1}
x\Incl t, y\Incl f(s(x)), \Gamma, \Gamma' \Seq \Delta, \Delta' 
 &   \raisebox{1.2ex}[1.5ex][0ex]{($cut$)} \\ \cline{1-1}
y\Incl f(s(t)), \Gamma, \Gamma' \Seq \Delta, \Delta' 
 &   \raisebox{1.2ex}[1.5ex][0ex]{($E_a$)}
\end{array} \]
This ($cut$) can be eliminated since $h(D_1^*)<h(D_1)$.
%
\item $R'$ is ($=_{Rar}$)\\
substituting $p$ for $s(t)$, i.e., $\psi$ is $y\Incl f(s(t))$, and $\psi'$ is
 $y\Incl f(p)$ for a $y\in\Vars$. Here we have to weaken $D_2^*$ also with 
$f(s(x))\Incl f(s(t))$:
{\footnotesize
\[ \begin{array}{cl} \hspace*{-2em}
D_1^* \left \{ \begin{array}{c} \vdots \\ 
 x\Incl t, \GSD, s(x)=p \end{array} \right . \ \ \ \ 
\begin{array}{rl}
f(s(x))\Incl f(s(t)), s(x)=p, y\Incl f(s(t)), \Gamma, \Gamma' \Seq \Delta, \Delta' \\
 \cline{1-1}
f(s(x))\Incl f(s(t)), s(x)=p, y\Incl f(s(x)), \Gamma, \Gamma' \Seq \Delta, \Delta' 
 &   \raisebox{1.2ex}[1.5ex][0ex]{($\Incl_a^*$)} \\ \cline{1-1}
f(s(x))\Incl f(s(t)), s(x)=p, y\Incl f(p), \Gamma, \Gamma' \Seq \Delta, \Delta' 
 &   \raisebox{1.2ex}[1.5ex][0ex]{($=_{Rar}$)} 
 \end{array} \\ \cline{1-1}
x\Incl t, f(s(x))\Incl f(s(t)), y\Incl f(p), \Gamma, \Gamma' \Seq \Delta, \Delta' 
 &   \raisebox{1.2ex}[1.5ex][0ex]{($cut$)} \\ \cline{1-1}
x\Incl t, f(s(t))\Incl f(s(t)), y\Incl f(p), \Gamma, \Gamma' \Seq \Delta, \Delta' 
 &   \raisebox{1.2ex}[1.5ex][0ex]{($\Incl_{ar}$)} \\ \cline{1-1}
x\Incl t, y\Incl f(p), \Gamma, \Gamma' \Seq \Delta, \Delta' 
 &   \raisebox{1.2ex}[1.5ex][0ex]{($cut_t$)} \\ \cline{1-1}
y\Incl f(p), \Gamma, \Gamma' \Seq \Delta, \Delta' 
 &   \raisebox{1.2ex}[1.5ex][0ex]{($E_s$)}
\end{array} \]
}
Again, ($cut$) can be eliminated since $h(D_1^*)<h(D_1)$.
By the same argument as in the case \ref{it:Ar}, $s(x)\not\in\Vars$ and
hence also $f(s(x))\not\in\Vars$ and $f(s(t))\not\in\Vars$, 
so we can apply ($\Incl_a^*$).
\end{LSD}
\end{LSC}
\end{LSB}
\end{LSA}
\end{LS}
\end{PROOF}

%
\noindent
The proposition yields the problematic implication of the 
\begin{THEOREM}\label{th:neq4cisneq4}
 $NEQ_4^c \equiv NEQ_4$ \end{THEOREM}
Combined with proposition~\ref{le:neq3isneq4} (and the equivalences 
from propositions~\ref{le:neqisneq1}, \ref{le:neq1isneq2} and \ref{le:neq2isneq3})
we have also:
\begin{COROLLARY}\label{co:Allequiv}
$NEQ\equiv NEQ_3^c \equiv NEQ_4^c \equiv NEQ_4 \equiv NEQ_3$.
\end{COROLLARY}

