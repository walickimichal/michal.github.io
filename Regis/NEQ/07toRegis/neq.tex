\documentclass[10pt]{article}
%%\documentstyle[a4wide,10pt]{article}
\makeatletter

\ifcase \@ptsize
    % mods for 10 pt
    \oddsidemargin  0.15 in     %   Left margin on odd-numbered pages.
    \evensidemargin 0.35 in     %   Left margin on even-numbered pages.
    \marginparwidth 1 in        %   Width of marginal notes.
    \oddsidemargin 0.25 in      %   Note that \oddsidemargin = \evensidemargin
    \evensidemargin 0.25 in
    \marginparwidth 0.75 in
    \textwidth 5.875 in % Width of text line.
\or % mods for 11 pt
    \oddsidemargin 0.1 in      %   Left margin on odd-numbered pages.
    \evensidemargin 0.15 in    %   Left margin on even-numbered pages.
    \marginparwidth 1 in       %   Width of marginal notes.
    \oddsidemargin 0.125 in    %   Note that \oddsidemargin = \evensidemargin
    \evensidemargin 0.125 in
    \marginparwidth 0.75 in
    \textwidth 6.125 in % Width of text line.
\or % mods for 12 pt
    \oddsidemargin -10 pt      %   Left margin on odd-numbered pages.
    \evensidemargin 10 pt      %   Left margin on even-numbered pages.
    \marginparwidth 1 in       %   Width of marginal notes.
    \oddsidemargin 0 in      %   Note that \oddsidemargin = \evensidemargin
    \evensidemargin 0 in
    \marginparwidth 0.75 in
    \textwidth 6.375 true in % Width of text line.
\fi

\voffset -2cm
\textheight 22.5cm

\makeatother

%%%\setlength{\textwidth}{17cm}
%%%\setlength{\evensidemargin}{-.8cm}
%%%\setlength{\oddsidemargin}{-.8cm}

%% \makeatother
%\makeatletter
%\show\
%\makeatother
\newcommand{\ite}[1]{\item[{\bf #1.}]}
\newcommand{\app}{\mathrel{\scriptscriptstyle{\vdash}}}
\newcommand{\estr}{\varepsilon}
\newcommand{\PSet}[1]{{\cal P}(#1)}
\newcommand{\ch}{\sqcup}
\newcommand{\into}{\to}
\newcommand{\Iff}{\Leftrightarrow}
\renewcommand{\iff}{\leftrightarrow}
\newcommand{\prI}{\vdash_I}
\newcommand{\pr}{\vdash}
\newcommand{\ovr}[1]{\overline{#1}}

\newcommand{\cp}{{\cal O}}

% update function/set
%\newcommand{\upd}[3]{#1\!\Rsh^{#2}_{\!\!#3}} % AMS
\newcommand{\upd}[3]{#1^{\raisebox{.5ex}{\mbox{${\scriptscriptstyle{\leftarrow}}\scriptstyle{#3}$}}}_{{\scriptscriptstyle{\rightarrow}}{#2}}} 
\newcommand{\rem}[2]{\upd{#1}{#2}{\bullet}}
\newcommand{\add}[2]{\upd {#1}{\bullet}{#2}}
%\newcommand{\mv}[3]{{#1}\!\Rsh_{\!\!#3}{#2}}
\newcommand{\mv}[3]{{#1}\:\raisebox{-.5ex}{$\stackrel{\displaystyle\curvearrowright}{\scriptstyle{#3}}$}\:{#2}}

\newcommand{\leads}{\rightsquigarrow} %AMS

\newenvironment{ites}{\vspace*{1ex}\par\noindent 
   \begin{tabular}{r@{\ \ }rcl}}{\vspace*{1ex}\end{tabular}\par\noindent}
\newcommand{\itt}[3]{{\bf #1.} & $#2$ & $\impl$ & $#3$ \\[1ex]}
\newcommand{\itte}[3]{{\bf #1.} & $#2$ & $\impl$ & $#3$ }
\newcommand{\itteq}[3]{\hline {\bf #1} & & & $#2=#3$ }
\newcommand{\itteqc}[3]{\hline {\bf #1} &  &  & $#2=#3$ \\[.5ex]}
\newcommand{\itteqq}[3]{{\bf #1} &  &  & $#2=#3$ }
\newcommand{\itc}[2]{{\bf #1.} & $#2$ &    \\[.5ex]}
\newcommand{\itcs}[3]{{\bf #1.} & $#2$ & $\impl$ & $#3$  \\[.5ex] }
\newcommand{\itco}[3]{   & $#1$ & $#2$  & $#3$ \\[1ex]}
\newcommand{\itcoe}[3]{   & $#1$ & $#2$  & $#3$}
\newcommand{\bit}{\begin{ites}}
\newcommand{\eit}{\end{ites}}
\newcommand{\na}[1]{{\bf #1.}}
\newenvironment{iten}{\begin{tabular}[t]{r@{\ }rcl}}{\end{tabular}}
\newcommand{\ass}[1]{& \multicolumn{3}{l}{\hspace*{-1em}{\small{[{\em Assuming:} #1]}}}}

%%%%%%%%% nested comp's
\newenvironment{itess}{\vspace*{1ex}\par\noindent 
   \begin{tabular}{r@{\ \ }lllcl}}{\vspace*{1ex}\end{tabular}\par\noindent}
\newcommand{\bitn}{\begin{itess}}
\newcommand{\eitn}{\end{itess}}
\newcommand{\comA}[2]{{\bf #1}& $#2$ \\ }
\newcommand{\comB}[3]{{\bf #1}& $#2$ & $#3$\\ }
\newcommand{\com}[3]{{\bf #1}& & & $#2$ & $\impl$ & $#3$\\[.5ex] }

\newcommand{\comS}[5]{{\bf #1} 
   & $#2$ & $#3$ & $#4$ & $\impl$ & $#5$\\[.5ex] }

%%%%%%%%%%%%%%%%
\newtheorem{CLAIM}{Proposition}[section]
\newtheorem{COROLLARY}[CLAIM]{Corollary}
\newtheorem{THEOREM}[CLAIM]{Theorem}
\newtheorem{LEMMA}[CLAIM]{Lemma}
\newcommand{\MyLPar}{\parsep -.2ex plus.2ex minus.2ex\itemsep\parsep
   \vspace{-\topsep}\vspace{.5ex}}
\newcommand{\MyNumEnv}[1]{\trivlist\refstepcounter{CLAIM}\item[\hskip
   \labelsep{\bf #1\ \theCLAIM\ }]\sf\ignorespaces}
\newenvironment{DEFINITION}{\MyNumEnv{Definition}}{\par\addvspace{0.5ex}}
\newenvironment{EXAMPLE}{\MyNumEnv{Example}}{\nopagebreak\finish}
\newenvironment{PROOF}{{\bf Proof.}}{\nopagebreak\finish}
\newcommand{\finish}{\hspace*{\fill}\nopagebreak 
     \raisebox{-1ex}{$\Box$}\hspace*{1em}\par\addvspace{1ex}}
\renewcommand{\abstract}[1]{ \begin{quote}\noindent \small {\bf Abstract.} #1
    \end{quote}}
\newcommand{\B}[1]{{\rm I\hspace{-.2em}#1}}
\newcommand{\Nat}{{\B N}}
\newcommand{\bool}{{\cal B}{\rm ool}}
\renewcommand{\c}[1]{{\cal #1}}
\newcommand{\Funcs}{{\cal F}}
%\newcommand{\Terms}{{\cal T}(\Funcs,\Vars)}
\newcommand{\Terms}[1]{{\cal T}(#1)}
\newcommand{\Vars}{{\cal V}}
\newcommand{\Incl}{\mathbin{\prec}}
\newcommand{\Cont}{\mathbin{\succ}}
\newcommand{\Int}{\mathbin{\frown}}
\newcommand{\Seteq}{\mathbin{\asymp}}
\newcommand{\Eq}{\mathbin{\approx}}
\newcommand{\notEq}{\mathbin{\Not\approx}}
\newcommand{\notIncl}{\mathbin{\Not\prec}}
\newcommand{\notCont}{\mathbin{\Not\succ}}
\newcommand{\notInt}{\mathbin{\Not\frown}}
\newcommand{\Seq}{\mathrel{\mapsto}}
\newcommand{\Ord}{\mathbin{\rightarrow}}
\newcommand{\M}[1]{\mathbin{\mathord{#1}^m}}
\newcommand{\Mset}[1]{{\cal M}(#1)}
\newcommand{\interpret}[1]{[\![#1]\!]^{A}_{\rho}}
\newcommand{\Interpret}[1]{[\![#1]\!]^{A}}
%\newcommand{\Comp}[2]{\mbox{\rm Comp}(#1,#2)}
\newcommand{\Comp}[2]{#1\diamond#2}
\newcommand{\Repl}[2]{\mbox{\rm Repl}(#1,#2)}
%\newcommand\SS[1]{{\cal S}^{#1}}
\newcommand{\To}[1]{\mathbin{\stackrel{#1}{\longrightarrow}}}
\newcommand{\TTo}[1]{\mathbin{\stackrel{#1}{\Longrightarrow}}}
\newcommand{\oT}[1]{\mathbin{\stackrel{#1}{\longleftarrow}}}
\newcommand{\oTT}[1]{\mathbin{\stackrel{#1}{\Longleftarrow}}}
\newcommand{\es}{\emptyset}
\newcommand{\C}[1]{\mbox{$\cal #1$}}
\newcommand{\Mb}[1]{\mbox{#1}}
\newcommand{\<}{\langle}
\renewcommand{\>}{\rangle}
\newcommand{\Def}{\mathrel{\stackrel{\mbox{\tiny def}}{=}}}
\newcommand{\impl}{\mathrel\Rightarrow}
\newcommand{\then}{\mathrel\Rightarrow}
\newfont{\msym}{msxm10}

\newcommand{\false}{\bot}
\newcommand{\true}{\top}

\newcommand{\restrict}{\mathbin{\mbox{\msym\symbol{22}}}}
\newcommand{\List}[3]{#1_{1}#3\ldots#3#1_{#2}}
\newcommand{\col}[1]{\renewcommand{\arraystretch}{0.4} \begin{array}[t]{c} #1
  \end{array}}
\newcommand{\prule}[2]{{\displaystyle #1 \over \displaystyle#2}}
\newcounter{ITEM}
\newcommand{\newITEM}[1]{\gdef\ITEMlabel{ITEM:#1-}\setcounter{ITEM}{0}}
\makeatletter
\newcommand{\Not}[1]{\mathbin {\mathpalette\c@ncel#1}}
\def\LabeL#1$#2{\edef\@currentlabel{#2}\label{#1}}
\newcommand{\ITEM}[2]{\par\addvspace{.7ex}\noindent
   \refstepcounter{ITEM}\expandafter\LabeL\ITEMlabel#1${(\roman{ITEM})}%
   {\advance\linewidth-2em \hskip2em %
   \parbox{\linewidth}{\hskip-2em {\rm\bf \@currentlabel\
   }\ignorespaces #2}}\par \addvspace{.7ex}\noindent\ignorespaces}
\def\R@f#1${\ref{#1}}
\newcommand{\?}[1]{\expandafter\R@f\ITEMlabel#1$}
\makeatother
\newcommand{\PROOFRULE}[2]{\trivlist\item[\hskip\labelsep {\bf #1}]#2\par
  \addvspace{1ex}\noindent\ignorespaces}
\newcommand{\PRULE}[2]{\displaystyle#1 \strut \over \strut \displaystyle#2}
%\setlength{\clauselength}{6cm}
%% \newcommand{\clause}[3]{\par\addvspace{.7ex}\noindent\LabeL#2${{\rm\bf #1}}%
%%   {\advance\linewidth-3em \hskip 3em
%%    \parbox{\linewidth}{\hskip-3em \parbox{3em}{\rm\bf#1.}#3}}\par 
%%    \addvspace{.7ex}\noindent\ignorespaces}
\newcommand{\clause}[3]{\par\addvspace{.7ex}\noindent
  {\advance\linewidth-3em \hskip 3em
   \parbox{\linewidth}{\hskip-3em \parbox{3em}{\rm\bf#1.}#3}}\par 
   \addvspace{.7ex}\noindent\ignorespaces}
\newcommand{\Cs}{\varepsilon}
\newcommand{\const}[3]{\Cs_{\scriptscriptstyle#2}(#1,#3)}
\newcommand{\Ein}{\sqsubset}%
\newcommand{\Eineq}{\sqsubseteq}%


\voffset -1cm

\input xypic

\title{Cut-Elimination and Other Gimmicks\\ for Logic of Nondeterminism}

\author{ {\em Regimantas ~Pliu\v skevi\v cius}\ \thanks{Department of Mathematical Logic,
Vilnius Institute of Mathematics and Informatics, Lithuania,
{\sl \{logica@sedcs.mii2.lt\}}} \\
{\em Micha{\l}~Walicki}\ \thanks{Department of Informatics, University of Bergen, 5020
Bergen, Norway, {\sl \{michal@ii.uib.no\}}} \\
{\em Jurate~Sakalauskait\.e }$^*$ \\
{\em Aida~Pliu\v skevi\v cien\.e }$^*$ 
}

\begin{document}




\section{Specifications of nondeterministic operations}\label{se:specs}

\begin{DEFINITION} {\bf (Specifications)} \label{de:terms-vars-forms}
A specification SP is a pair ($\Sigma,\Ax$), where $\Sigma$ is a signature, i.e.
a pair ($\Sorts,\Funcs$) of sort and operation symbols (the latter with specified
arities and rank). $\Ax$ is a a set of sequents, $\Gamma\Seq\Delta$, 
where $\Gamma, \Delta$ are finite sets of atomic formulae.

%There is a countable set of variables \C V.
The set of terms\footnote{We will not mention sorting explicitly
and always assume that terms and formulae are correctly sorted.}
 built in the standard way from the variables and
operation symbols is denoted $\Terms{\Funcs,\C V}$. An {\em atomic formula} 
is an {\em equality}, $s=t$, or an {\em inclusion}, $s\Incl t$, of terms
$s,t\in\Terms{\Funcs,\Vars}$ (where $t$ is called the 
right-hand-side (RHS) and $s$ the left-hand-side (LHS) of inclusion).
\end{DEFINITION}
\noindent
We will write $\C V(\sigma)$ to denote the variables occurring in $\sigma$ 
(which may be any syntactic entity or a set thereof), and $x\in\Vars$ to say that
$x$ is a variable.

The non-standard aspect of this definition concerns the atomic formulae which are
inclusions (typically, one has only equalities). This originates from the fact that
the structures to be specified are not standard algebras but {\em multialgebras},
where operations may return {\em sets} of individuals rather than only individuals.
%
\begin{DEFINITION} {\bf (Multialgebras)} \label{de:multialgebras}
Given a signature $\Sigma=(\Sorts,\Funcs)$, a {\em $\Sigma$-multialgebra} $A$
is given by:
\begin{itemize}\MyLPar
\item  the carrier $|A|$ of $A$ consists of a set $S^A$ for each $S\in\Sorts$
\item for each operation symbol $f: \List Sn\times \rightarrow S$, an operation
$f^A: \List {S^A}n\times \rightarrow \C P^+(S^A)$, where $\C P^+(X)$ denotes the 
power set of the set $X$ with empty set excluded.
\end{itemize}
\end{DEFINITION}
\noindent
Given a multialgebra $A$ and an assignment to variables $\beta:X\rightarrow |A|$, 
there is a unique function $\beta [\_]$ which gives, to each term $t$ 
with $\C V(t)\subseteq X$, its interpretation in $A$ defined in the usual way as:
\[ \begin{array}{ccccc} 
\beta [x] = \beta (x) & & \beta [c] = c^A
& & \beta [f(\List tn,)] = f^A(\beta[t_1]\ldots \beta[t_n]) 
\end{array} \]
Unlike in classical algebra, this function need not be a homomorphism of multialgebras.
An extensive study of multialgebraic homomorphisms can be found in \cite{WB}.
%
\begin{DEFINITION} {\bf (Semantics)} \label{de:semantics}
A $\Sigma$-multialgebra $A$ satisifes an atomic formula $\phi$ under assignment
 $\beta:X\rightarrow |A|$, written $A,\beta\models \phi$, iff:
\begin{itemize}\MyLPar
\item $\phi$ is $s=t$, and $\beta[s]=\beta[t]=\{a\}$ for some $a\in |A|$
\item $\phi$ is $s\Incl t$, and $\beta[s]\subseteq\beta[t]$.
\end{itemize}
 $A$ satisifes a sequent, $A\models\Gamma\Seq\Delta$ iff,
for each assignment $\beta$, there exists a $\gamma\in\Gamma : A,\beta\not\models
\gamma$, or there exists a $\delta\in\Delta: A,\beta\models \delta$. 
%satisfies at least one of the atoms in $\Delta$ or $A$ does not satisfy at least
%one of the atoms in $\Gamma$.
A $\Sigma$-multialgebra $A$ is an SP{\em-multimodel}, $A\in \MMod(SP)$, iff it satisfies 
all axioms $\Ax$ of SP.
\end{DEFINITION}
\noindent
The natural deduction-like calculus $NEQ$ consists of the rules given in
figure \ref{fi:neq} \cite{WM}. Each specification contains a set
$\Ax$ of {\em specific axioms} \reff{ru:spax}.
\begin{figure}[hbt]
\hspace*{3em}\begin{tabular}{|ll|}
\hline
\multicolumn{2}{|c|}{{\bf Axioms} :}\\[.5ex]
\TABRUL{\Seq x=x : x\in\Vars} \label{ru:neqx} & 
\TABRUL{s\odot t \Seq s\odot t} \label{ru:neqid}\ \ \ \ $\odot\in\{=,\Incl\}$  \\[2ex]
%%& {\footnotesize \ \ \ \ \ \ \ \ \ - $\odot\in\{=,\Incl\}$} \\[2ex]
%
\multicolumn{2}{|c|}{{\bf Identity and inclusion rules} :}\\[.5ex]
\TABRULE{
\Gamma_t^x\Seq\Delta_t^x \ \ ; \ \ \Gamma'\Seq s=t,\Delta'}
{\Gamma_s^x,\Gamma'\Seq\Delta_s^x,\Delta' } \label{ru:neqeq} & 
\TABRULE{
\Gamma\Seq\Delta, w(t)\preceq q \ \ ; \ \ \Gamma'\Seq s\Incl t,\Delta'}
{\Gamma,\Gamma'\Seq\Delta',\Delta, w(s)\preceq q} \label{ru:neqincl} \\[4ex]
%& {\footnotesize \ \ \ - $x$ not  in a RHS of $\Incl$ in $\Delta$.} \\[2ex]
%
\multicolumn{2}{|c|}{{\bf Elimination rules}:}\\[.5ex]
\TABRULE{\Gamma, x\Incl t\Seq\Delta} 
{\Gamma_t^x\Seq\Delta_t^x}\ ($elim1$) \label{ru:elim1} 
& \TABRULE{x\Incl t, y\Incl r(x), \Gamma\Seq\Delta} 
{y\Incl r(t),\Gamma\Seq\Delta}\ ($E_a$)  \label{ru:elim2}  \\
%\noindent
{\footnotesize \ \ \ - $x\not\in \Vars(t)$;} & {\footnotesize \ \ \ -
$x\not\in\Vars(t,y,\Gamma,\Delta), y\in\Vars$}  \\
{\footnotesize \ \ \ - $x$ occurs at most once
in $\Gamma\Seq\Delta$ } & {\footnotesize \ \ \ - $x$ occurs at most once in $r$}  \\
{\footnotesize \ \ \ - and not in a RHS of $\Incl$ in $\Gamma$} &  \\[2ex]
% 
\multicolumn{2}{|c|}{{\bf Weakening rules} ($W$) :}\\[.5ex]
\multicolumn{1}{|r}{\prule{\Gamma\Seq\Delta}{\Gamma\Seq\Delta,\atom}}  & 
\TABRULE{\Gamma\Seq\Delta}{\Gamma, \atom\Seq\Delta}\label{ru:weak}   \\[4ex]
%%
\multicolumn{2}{|c|}{{\bf Cut rule} ($cut$) :}\\[.5ex]
\multicolumn{2}{|c|}{\TABRULE{\Gamma\Seq\Delta,\atom\ \ ;\ \ \atom,\Gamma'\Seq\Delta'}
{\Gamma,\Gamma'\Seq\Delta,\Delta'}\label{ru:neqcut} } \\[2ex] %\hline
\multicolumn{2}{|c|}{{\bf Specific axioms} :}\\[.5ex]
\multicolumn{2}{|c|}{\TABRUL{\lis\atom\Seq \lism\batom{m}}\label{ru:spax}
 ($\Ax_k$)\ \ \ \  where $k$ identifies a particular axiom in the set $\Ax$.}\\[1ex]
\hline
\end{tabular}
\caption{The rules and axioms of $NEQ$}\label{fi:neq}
\end{figure}

\noindent
$\atom$, $\batom$ stand for atoms, i.e., $s\odot t$ with $\odot\in\{=,\Incl\}$.
 The symbol $\preceq$ abbreviates either $=$ or $\Incl$. 
$\Gamma_t^x$ indicates replacement of one or more $x$ by $t$.
Sometimes, we may write $\Gamma[x]$, $\phi[x]$ for a (set of) atom(s) containing
$x$, and then $\Gamma[t]$, $\phi[t]$ for the result of replacing these occurrences
of $x$ by $t$.

%
\begin{REMARK} \label{re:1}
{From the definition of a sequent it follows that $NEQ$ implicitly contains the 
structural rules ``exchange'' and ``contraction''.}
$NEQ$ also assumes implicitly symmetry of equations, i.e., $s=t$ is treated as an 
unorded set $\{s,t\}$. 

Two peculiar features of this calculus are the absence of the unrestricted 
substitutivity,
and that $=$ is only a partial equivalence.
  The equality $t=t$ represents ``deterministic equality'' -- it holds only for terms which
denote individual elements of the carrier.
 Thus, in general, $NEQ\ \not\vdash\ \Seq t=t$. 
But since
variables are assigned only individual elements, we do have 
$NEQ\vdash\ \Seq x=x$  for any variable $x$. Consequently, since
terms may denote {\em sets} of individuals, unrestricted substitutivity would not
be sound wrt. the multialgebraic semantics. For motivations and examples the reader
is referred to \cite{WM,Top,Broy}.
\end{REMARK}
%
\begin{THEOREM}\label{th:cmpl} {\em \cite{WM}}
$NEQ$ is sound and complete with respect to the multialgebraic semantics, i.e.,
for any sequent $S$:
\[\MMod(\Sigma,\Ax)\models S \ \ \iff \ \ \der{\Ax, NEQ}{}S\]
\end{THEOREM}
\noindent
 $\der{\Ax,NEQ}{}S$,  indicating that $S$ is derivable from $\Ax$
using the rules of $NEQ$, might be written in a more standard fashion as
 $\Ax\der{}{NEQ}S$. However, we will consider variants of the calculus
 where the axioms $\Ax$ are built into the rules, and the latter notation will be used
 for a different purpose.

%We  assume given a fixed signature and a set $\Ax$ of
%specific axioms of the general form:
%\PROOFRUL{r_i \odot_i s_i \Seq u_j\odot_j v_j}\label{ru:spax}
% ($\Ax_k$)\ \ \ \  where $k$ identifies a particular axiom in the set $\Ax$.
%
\begin{DEFINITION} Let $I, J$ be arbitrary
calculi. We use the following notational convention:
\begin{itemize}\MyLPar
\item $I \Rightarrow J$ if, for any sequent $S : \der{\Ax, I}{}S 
\ \Rightarrow\  \der{\Ax, J}{}S;$
\item $I\equiv J$ if $I\Rightarrow J$ and $J\Rightarrow I$;
\item $I^c$ denotes $I$ extended with the 
($cut$) rule;
\item $\der IDS$ indicates that $D$ is a derivation of $S$ 
using the rules of calculus $I$;
\item for derivations $D, D'$, we write $D\impl D'$ to indicate that $D'$
may be constructed, assuming that $D$ is given.
\item Variable $x$ in the applications of any of the elimination rules will be called
{\em eigen-variable}.
\end{itemize}
\end{DEFINITION}
%
%
\section{Some intermediary calculi}\label{se:inter}
In this section we transform $NEQ$ to an equivalent calculus 
$NEQ_{3}$ which will provide a basis for the final calculus and 
(cut)-elimination.
Instead of introducing the resulting calculus at once, we proceed 
step by step to better visualize the crucial transformations. The most 
significant one occurs in subsection~\ref{sub:Spcut} where specific 
axioms are replaced by appropriate inference rules.

\subsection{Kanger-like calculus -- $NEQ_1$}\label{se:neq1}
We transform $NEQ$ to an equivalent calculus $NEQ_1$ 
with Kanger-like rules for equality \cite{K} and inclusion.

\begin{DEFINITION} \label{de:neq1}
The calculus $NEQ_1$ is obtained from $NEQ$ by the following transformations:\\
\noindent For any $t\in\Terms{\Funcs,\C V}$ and $x\in\Vars$, the following axiom 
is added:
\PROOFRUL{x\Incl t \Seq x\Incl t}\ \label{ru:neq1tint}\\[.5ex]
\noindent The axiom \reff{ru:neqid}, $s\odot t\Seq s\odot t$ is replaced by:
\PROOFRUL{s=t\Seq s=t}\label{ru:neq1id}\\[.5ex]
 % \ \ \ where: $s,t\not\in\C V$\\[4pt]
\noindent The inference rules for equality and inclusion~\reff{ru:neqeq}, 
\reff{ru:neqincl} are replaced by:
\PROOFRULE{s=t,\Gamma_t^x\Seq\Delta_t^x}{s=t,\Gamma_s^x\Seq\Delta_s^x} ($=_1$)
\label{ru:K12} \ \ \ \ \ \ \ \ 
\prule{t=s,\Gamma_t^x\Seq\Delta_t^x}{t=s,\Gamma_s^x\Seq\Delta_s^x} ($=_2$)
\PROOFRULE{\Gamma, s\Incl t\Seq \Delta, w(t)\preceq q}{\Gamma, s\Incl t\Seq
\Delta, w(s)\preceq q} $\iLs$
\label{ru:Kincl}
%\ \ \ \ where: $x$ in $\Delta$ isn't in the RHS of $\Incl$.
\end{DEFINITION}
\noindent
The cases for $=_1$ and $=_2$ are entirely symmetric. Since
the reader is going to suffer a lot of notational abbreviations, we
are allowing ourselves to gloss over this difference in the sequel and treat these
two cases uniformly -- we write merely one rule $(=)$ which is to be understood
as a shortened form for the two rules $(=_1)$ and $(=_2)$.


\begin{LEMMA}\label{le:neqtoneq1} $NEQ \Rightarrow NEQ_1$ \end{LEMMA}
\begin{PROOF}
We show that the axioms~\reff{ru:neqid} and inference rules~\reff{ru:neqeq},
\reff{ru:neqincl} are derivable in $NEQ_1$.
%$NEQ_1$ proves all the axioms $S$ and rules $R$ removed from $NEQ$.
\begin{LS}\MyLPar
%%\item Let $S$ be $s=x\Seq s=x$, where $x\in\C V$ (the case
%%$x=s\Seq x=s$ is analogous). \\ We derive it in $NEQ_1$: \ \ \ 
%%\( \begin{array}{cl}
%%  \Seq x=x & \raisebox{-1.2ex}[1.5ex][0ex]{(W)} \\ \cline{1-1}
%%  s=x\Seq x=x & \raisebox{-1.2ex}[1.5ex][0ex]{($=$)} \\ \cline{1-1}
%%  s=x\Seq s=x
%%\end{array} \) 
\item Let $S$ be $s\Incl t\Seq s\Incl t$. The derivation in $NEQ_1$
is obtained by choosing a fresh variable $x$:\ \ \ 
\[\prar{
  x\Incl t\Seq x\Incl t \cl
  \Seq t\Incl t & \rabove{(elim1)}   \cl
  s\Incl t\Seq t\Incl t & \rabove{(W)}  \cl
  s\Incl t\Seq s\Incl t & \rabove{\iLs}
}
\]
\item Suppose that $NEQ_1$ proves the premises of \reff{ru:neqeq}. Weakening the
first premise with the formula $s=t$ allows us to get $s=t,\Gamma_s^x\Seq\Delta_s^x$ by
($=$) and ($cut$) with the second premise yields the conclusion of 
\reff{ru:neqeq} in $NEQ_1$:
\begin{center} \( \begin{array}{ccl}
 & \Gamma_t^x\Seq\Delta_t^x & \raisebox{-1.2ex}[1.5ex][0ex]{(W)} \\ \cline{2-2}
 & s=t,\Gamma_t^x\Seq\Delta_t^x & \raisebox{-1.2ex}[1.5ex][0ex]{($=$)} 
      \\ \cline{2-2}
\Gamma'\Seq s=t,\Delta' & s=t,\Gamma_s^x\Seq\Delta_s^x 
   & \raisebox{-1.2ex}[1.5ex][0ex]{($cut$)} \\ \cline{1-2}
\multicolumn{2}{c}{\Gamma_s^x,\Gamma'\Seq\Delta_s^x,\Delta'}
\end{array} \) \end{center}
\item An analogous procedure as above allows us to derive \reff{ru:neqincl} in $NEQ_1$
\end{LS}
\end{PROOF}

\begin{LEMMA}\label{neq1toneq} $NEQ_1 \Rightarrow NEQ$ \end{LEMMA}
\begin{PROOF}
\begin{LS}\MyLPar
%% \item The axiom \reff{ru:neq1tint} is $NEQ$ derivable by ($elim1$):
%%\prule{x\Incl t\Seq x\Incl t}{\Seq t\Incl t}
\item 
Both inference rules \reff{ru:K12} are derived by a 
single application of \reff{ru:neqeq}:

\prule{s=t,\Gamma_t^x\Seq\Delta_t^x\ \ ;\ \ s=t\Seq s=t}
      {s=t,\Gamma_s^x\Seq\Delta_s^x}
\item 
and the rule \reff{ru:Kincl} is obtained by an analogous application 
of \reff{ru:neqincl}.
\end{LS}
\end{PROOF}
The two lemmas yield:
\begin{CLAIM}\label{le:neqisneq1} $NEQ\equiv NEQ_1$.\end{CLAIM}

\subsection{$NEQ_2$ -- some restrictions on ($cut$) and 
($elim1$).}\label{sub:Spcut}
A new equivalent of $NEQ$ is introduced in which the general ($cut$) rule is
replaced by the {\em specific-cut rules} depending on the specific axioms
of the specification. Recall that we have fixed a set of axioms $\Ax$, of the 
general form \reff{ru:spax}: \(r_i \odot_i s_i \Seq u_j\odot_j v_j\).

\begin{DEFINITION} Calculus $NEQ_2$ is obtained from $NEQ_1$ by:
\begin{itemize}\MyLPar
\item[1.] replacing the inference rule \reff{ru:elim1} with:
\PROOFRULE{\Gamma,x\Incl t\Seq\Delta}{\Gamma\Seq\Delta_t^x}\ \ 
 ($E_s$)\ \ 
 $x\not\in\Vars(\Gamma,t)$ and at most one occurence of $x$ in
$\Delta$.
\item[2.]  adding the following inference rules for inclusion:
\PROOFRULE{s\Incl t,w(s)\preceq q, \GSD}{s\Incl t,w(t)\preceq q,\GSD}
 \label{ru:Kincla} $\iLa$  \ \ \ \ \ \ \ \ 
\prule{s\Incl t, \GSD, w\Incl q(s)}{s\Incl t, \GSD, w\Incl q(t)} $\iRs$ \\
(In the case of equality $w(s)=q$, we implicitly allow $q=w(t)$ in the conclusion.)
\item[3.] removing the cut-rule \reff{ru:neqcut};
\item[4.] replacing each specific axiom $\Ax_k$ of the form \reff{ru:spax} by 
the ($Sp.cut_k$) rule:
\PROOFRULE{\GSD,A_1\ ;\ ...\ ;\ \GSD,A_n\ ;\ B_1,\GSD\ ;\ ...\ ;\ B_m,\GSD
%\Gamma\Seq\Delta,A_1;\ \ ;\ \ u_j\odot_j v_j,\Gamma\Seq\Delta
} {\Gamma\Seq\Delta}
 \label{ru:spcut} ($Sp.cut_k$)
\end{itemize}
Formulae $\atom_i$ and $\batom_j$ in applications of ($Sp.cut$) will be 
called the {\em specific cut formulae}.
\end{DEFINITION} 
Before proving the equivalence of $NEQ_1$ and $NEQ_2^c$ we introduce the
following conventions to be used extensively.

\begin{DEFINITION} In an application of a rule we distinguish the following
formulae:
\begin{itemize}\MyLPar
\item {\em active} -- the atomic formula explicitly mentioned in the premise(s)
and conclusion of the rule but not modified by its application;
\item {\em side} -- the formulae in the premise of the rule which are modified
(or removed) by its application;
\item {\em modified} -- the new formulae obtained in the conclusion of the rule
(either as the modified side formulae or as a new formulae introduced in the 
conclusion).\footnote{This is also called {\em principal}.}
\end{itemize}
\end{DEFINITION}
\noindent
The term which appears in a modified formula as a 
result of the application of the rule is called the {\em resulting
term}. Sometimes we will refer by ``modified'' to the term from the side
formula which was changed by the application of the rule, by 
``replaced'' to this part of it which was replaced and by ``substituted'' 
to this part of the resulting term which appeard for the replaced 
(sub)term of the modified term.

It is implicitly assumed that each rule modifies at most one term, even if
their formulation might indicate that several terms are modified
simultaneously (like for instance \reff{ru:neqeq} and \reff{ru:neqincl} in figure \ref{fi:neq}). This
convention is used throughout the whole paper.
\begin{EXAMPLE}
For instance, each of the weakening rules \reff{ru:weak}
has only one modified formula ($s\odot t$), and no side formula.
The ($cut$) rule \reff{ru:neqcut} has only side formula which is
also called the {\em cut} formula. The atom $s\odot t$ in the rules \reff{ru:K12}, 
\reff{ru:Kincl} is the active formula of these rules. 
Term $t$ is modified and $s$ substituted in
\reff{ru:K12} and in \reff{ru:Kincl}. The number of modified and side formulae in
these rules depends on the number of atoms in which $t$ is replaced by $s$. 
In the following application of ($elim1$) rule \reff{ru:elim1}
\begin{center}
\prule{\Gamma, \phi_x^x,x\Incl t\Seq\Delta}{\Gamma,\phi_t^x\Seq\Delta}
\end{center}
\noindent $\phi_t^x$ is the modified formula, $\phi_x^x, x\Incl t$ are side formulae, 
and $t$ is the substituted term.\\
We apply this notions in the top-down fashion, i.e., the formula above the
line will have a side formula, which is modified in the line below.
If one thinks of the bottom-up applications of the rules the notions remain the same but
what was ``modified'' in top-down will now be ``side'' and vice versa.
\end{EXAMPLE}

\noindent
The rules will be tagged by $(\odot_{Xyz})$ according to the following 
convention: 
\begin{itemize}\MyLPar
\item $\odot\in\{=,\Incl\}$ reflects the operator of the active 
formula; 
\item $X\in\{L,R\}$ identifies the side of the inclusion in which 
the modified term occured: $L$ for LHS  or equality and $R$ for RHS;
\item $y\in\{a,s\}$ -- $a$ if the side formula is in the 
antecedent of the sequent, and $s$ if it is in the consequent; the 
rules will be called ``a-rules'', respectively, ``s-rules'';
\item $z$ may be an additional annotation, typically, $x$ if the 
substituted term was a variable, $r$ if the replaced term was a 
variable, $nr$ if the replaced term was not a variable.
\end{itemize}
Thus, for instance, $\iLs$ is the rule with active inclusion which 
modifies a term in either an equality or LHS of an inclusion in the 
succedent.

We will also use following notions for a derivation $D$:
\begin{itemize}\MyLPar
\item the {\em height} of $D$, denoted $h(D)$, with the obvious meaning;
\item the {\em number of applications} of a rule $R$ in $D$, denoted $\#(R,D)$;
%\item the {\em grade of an application} of a rule $R$ in (a branch of) $D$ is
%the consequtive number of this application (in this branch) -- i.e., the first application has
%grade 1, the second 2, and so on;
%\item the {\em grade of a rule} $R$, denoted $gr(R,D)$, is the sum of the grades of all its
%applications in $D$.
\end{itemize}
%
\begin{LEMMA}\label {le:neq1to2}
 $NEQ_1 \Rightarrow NEQ_2^c$.\end{LEMMA}
\begin{PROOF}
%\begin{LS}
%\item 
All specific axioms $\Ax$ are derivable in $NEQ_2$, using ($Sp.cut$) 
rules and derivability of the axiom $s\odot t\Seq s\odot t$ in $NEQ_2$. 

%\item 
The rule ($E_s$) is a restricted version of ($elim1$) when the modified
formula occurs in the succedent.
To see that ($elim1$) is derivable in $NEQ_2$, first observe 
that application of ($elim1$) with the modified 
formula in the succedent yields the same result as application of ($E_s$). 
So, consider an application of ($elim1$) with the modified formula in the 
antecedent. 
\begin{center}
\prule{\Gamma, s(x)\preceq p, x\Incl t\Seq\Delta} 
  {\Gamma, s(t)\preceq p \Seq\Delta}  ($elim1$)
\end{center}
By the restrictions on ($elim1$), $x$ occurs either in $=$ or in a LHS of $\Incl$,
which is indicated by the symbol $\preceq$. We may apply $\iLa$ and 
($elim_s$) to obtain the same conclusion:
\[\begin{array}{cl}
\Gamma, s(x)\preceq p, x\Incl t\Seq\Delta \\ \cline{1-1}
\Gamma, s(t)\preceq p, x\Incl t\Seq\Delta
  & \raisebox{1.2ex}[1.5ex][0ex]{$\iLa$} \\ \cline{1-1}
\Gamma, s(t)\preceq p\Seq\Delta
  & \raisebox{1.2ex}[1.5ex][0ex]{($E_s$)} 
\end{array} \] %\samepage
%\end{LS}       %\samepage
\end{PROOF} 

\begin{CLAIM}\label{le:neq1isneq2}
$NEQ_2^c \equiv NEQ_1$.\end{CLAIM}
\begin{PROOF}
$\Leftarrow$ is lemma~\ref{le:neq1to2}. For the opposite implication we have 
the following situations.
\begin{LS}
\item The ($Sp.cut$) rules \reff{ru:spcut} are derivable in $NEQ_1$.
For each ($Sp.cut_k$), use ($cut$) to its premises and the 
corresponding specific axiom $\Ax_k$.
\item  The rule $\iLa$ is derivable in $NEQ_1$.
Assume that $NEQ_1$ proves the premise 
$s\Incl t, w(s)\preceq q,\GSD$ %$s\Incl t,\Gamma^t_s\Seq\Delta$
 of the rule. (For simplicity, 
consider only the case with one occurrence of $s$ in this premise, i.e., when
it has the form $s\Incl t$.)
The following is a derivation of
the conclusion of the rule in $NEQ_1$:
\[ \begin{array}{rcc}
\raisebox{-1.2ex}[1.5ex][0ex]{(W)} & w(t)\preceq q\Seq w(t)\preceq q &  \\ \cline{2-2}
\raisebox{-1.2ex}[1.5ex][0ex]{
$x$ not in the RHS of $\Incl$ in $\phi$\ \ $\iLs$} %($\Incl_s$)} 
 & s\Incl t,w(t)\preceq q\Seq w(s)\preceq q & 
      \\ \cline{2-2}
 \raisebox{-1.2ex}[1.5ex][0ex]{($cut$)} 
& s\Incl t, w(t)\preceq q\Seq w(s)\preceq q & s\Incl t, w(s)\preceq q,\GSD 
   \\ \cline{2-3}
& \multicolumn{2}{c}{\ \ \ \ \ s\Incl t,w(t)\preceq q,\GSD}
\end{array} \]
\item In a similar way, we show admissibility of $\iRs$ in $NEQ_1$. Assume it
proves the premise of the rule $s\Incl t, \GSD, w\Incl q(s)$. We then construct the 
following derivation of its conclusion in $NEQ_1$:
\[ \begin{array}{rrl}
& x\Incl q(t) \Seq x\Incl q(t) \\ \cline{2-2}
& \Seq q(t)\Incl q(t) & \rabove{(elim1)} \\ \cline{2-2}
& s\Incl t, \Seq q(t)\Incl q(t) & \rabove{(W)} \\ \cline{2-2}
& s\Incl t, \Seq q(s)\Incl q(t) & \rabove{\iLs} \\ \cline{2-2}
& w\Incl q(s), s\Incl t, \Seq q(s)\Incl q(t) & \rabove{(W)} \\ \cline{2-2}
s\Incl t, \GSD, w\Incl q(s) & w\Incl q(s), s\Incl t, \Seq w \Incl q(t) & \rabove{\iLs} \\ \cline{1-2}
\multicolumn{2}{c}{ s\Incl t, \GSD, w\Incl q(t) } & \rabove{(cut)}
\end{array} \]
\end{LS} \vspace*{-1ex}
\end{PROOF} \vspace*{-2ex}

\subsection{Calculus $NEQ_3$ -- no weakening rules.}
Applying the standard technique, we will now eliminate the weakening rules by
replacing each axiom of $NEQ_2$ of the form $a\Seq s$ by
the set of axioms $\Gamma,a \Seq\Delta, s$ for all $\Gamma, \Delta$.
This simplifies the proofs,
nevertheless, it is often easier to understand the arguments if
the applications of weakening are made explicit. We will occasionally
formulate the proofs using such an explicit mention of weakening which is to
be understood as starting the whole derivation with another instance of the
same axiom -- in particular, it does not increase the height of the
derivation. More precisely, letting $NEQ_2^*$ denote $NEQ_2$ with axioms 
modified as indicated above and without the weakening rules, we have the following fact:

\begin{LEMMA}\label{le:noweak}
 $\der{NEQ_2}DS\ \impl\ \der{NEQ_2^*}{D^*}S$ and $h(D^*)\leq h(D)$.
\end{LEMMA}
\begin{PROOF}
We have to show that weakening rules are admissible in $NEQ_2^*$. Proceeding
by induction on the number of their applications and $h(D)$, consider the
last rule $R$ applied above weakening. The cases for ($W_a$) and ($W_s$) are
exactly the same, so we only formulate the proof for ($W_a$).
\begin{LS}
\item If no rules are applied above ($W_a$) in $D$, the result is an axiom
 $s\Seq s$ (with LHS possibly empty) of
 $NEQ_2$ weakened by a formula $\phi$. In $NEQ_2^*$ we obtain it directly as
 an axiom $\phi,s \Seq s$. The height of this new derivation is less then $h(D)$.
\item $R$ is ($=$), and $D$ ends as follows
\[ D \left \{ \begin{array}{rl}
 \multicolumn{1}{c}{\vdots} \\
 s=t, f(s)\odot q, \GSD \\ \cline{1-1}
 s=t, f(t)\odot q, \GSD & \rabove{(=)} \\ \cline{1-1}
 \phi, s=t, f(t)\odot q, \GSD & \rabove{(W)} \end{array} \right . \conv
%Obviously, we can swapp the two applications:
 \begin{array}{rl}
 \multicolumn{1}{c}{\vdots} \\
 s=t, f(s)\odot q, \GSD \\ \cline{1-1}
 \phi, s=t, f(s)\odot q, \GSD & \rabove{(W)} \\ \cline{1-1}
 \phi, s=t, f(t)\odot q, \GSD & \rabove{(=)} \end{array} \]
Induction hypothesis on $h(D)$ yields the conclusion.
\item The argument is exactly the same for all the remaining rules, with the
exception of the elimination rules.
\item $R$ is ($E_s$) -- 
If $\phi$ does not contain the eigen-variable $x$ of ($E_s$), we can
trivially swap the two. If it does, however, we have to rename $x$ in the
whole $D_1$ to a new variable, say $y$, not occuring there. We then obtain
equivalent derivation in which we can swap the applications of ($E_s$) and ($W$).
\[ \begin{array}{lll}
D \left \{ \begin{array}{cl}
 D_1 \left \{ \begin{array}{rl}
 \multicolumn{1}{c}{\vdots} \\
 x\Incl t,  \GSD \\ \cline{1-1}
  \GSD_t^x & \rabove{(E_s)} \end{array} \right . \\ \cline{1-1}
 \phi[x], \GSD_t^x & \rabove{(W)} \end{array} \right . 
%\]
&\Rightarrow &
%\[ 
D' \left \{ \begin{array}{cl}
 D_1' \left \{ \begin{array}{rl}
 \multicolumn{1}{c}{\vdots} \\
 y\Incl t,  \GSD \\ \cline{1-1}
  \GSD_t^y & \rabove{(E_s)} \end{array} \right . \\ \cline{1-1}
 \phi[x], \GSD_t^y & \rabove{(W)} \end{array} \right . 
\end{array}\]
\item The same procedure is applied when the last rule was ($E_a$).
\end{LS}
\end{PROOF}


\begin{DEFINITION} Calculus $NEQ_3$ is obtained from $NEQ_2$ by:
\begin{itemize}\MyLPar
\item replacing weakening rules by extended axioms as indicated in lemma \ref{le:noweak};
\item replacing rules \reff{ru:K12} ($=$) by:
\PROOFRULE{s=t,\Gamma\Seq\Delta_t^x}{s=t,\Gamma\Seq\Delta_s^x} ($=_{s}$)
\label{ru:K12s} \ \ \ 
and \ \ \ 
\prule{s=t,\Gamma_t^x\Seq\Delta}{s=t,\Gamma_s^x\Seq\Delta} ($=_{a}$)
\label{ru:K12a} 
\item adding the inference rules of {\em simple cut}
\PROOFRULE{t\Incl t, \Gamma\Seq\Delta}{\Gamma\Seq\Delta} ($cut_t$)\label{ru:sicut}
\ \ \ \ \ \ \ \ \ 
\TABRULE{x=x, \Gamma\Seq\Delta}{\Gamma\Seq\Delta} ($cut_{x=}$)\label{ru:cutxeq}
\end{itemize}
\end{DEFINITION}
%\noindent
%The rules of $NEQ_3$ are shown in figure \ref{fi:neq3}.
%
\begin{figure}[hbt]
\hspace*{6em}\begin{tabular}{|lcl|}
\hline
\multicolumn{3}{|c|}{{\bf Axioms} :}\\[.5ex]
\multicolumn{3}{|c|}{%$\GSD, t\Incl t$
$x\Incl t,\GSD, x\Incl t$ \ \ \ \ \ \ \ 
$\GSD,x=x$ \ \ \ \ \ \ \
$\Gamma, s= t\Seq s= t, \Delta$ % : $s,t\not\in\Vars$
} \\[2ex]
%
\multicolumn{3}{|c|}{{\bf Identity rules} :}\\[.5ex]
\prule{s=t,\Gamma\Seq\Delta_t^x}{s=t,\Gamma\Seq\Delta_s^x} ($=_{s}$) & &
\prule{s=t,\Gamma_t^x,\Seq\Delta}{s=t,\Gamma_s^x,\Seq\Delta} ($=_{a}$) \\[3ex]
%
\multicolumn{3}{|c|}{{\bf Inclusion rules} :}\\[.5ex]
\prule{s\Incl t, \Gamma\Seq \Delta, w(t)\preceq q}{s\Incl t, \Gamma\Seq
\Delta, w(s)\preceq q} 
$\iLs$ & &
\prule{s\Incl t, w(s)\preceq q,\Gamma\Seq\Delta}{s\Incl t, w(t)\preceq q,\Gamma\Seq\Delta}
 $\iLa$ \\[3ex]
%
\prule{s\Incl t, \GSD, w\Incl q(s)}{s\Incl t, \GSD, w\Incl q(t)} $\iRs$ && \\[3ex]
\multicolumn{3}{|c|}{{\bf Elimination rules} :}\\[.5ex]
\prule{x\Incl t, \Gamma \Seq\Delta} 
  {\Gamma\Seq\Delta_t^x}  ($E_s$) & &
\prule{x\Incl t, y\Incl r(x), \Gamma\Seq\Delta}
  {y\Incl r(t),\Gamma\Seq\Delta}  ($E_a$) \\[1.5ex]
{\footnotesize \ \ \ - $x\not\in \Vars(\Gamma,t)$;} & & 
   {\footnotesize \ \ \ - $x$ is fresh $: x\not\in \C V(t,\Gamma,\Delta)$;} \\
{\footnotesize \ \ \ - at most one $x$ in $\Delta$;} & & 
   {\footnotesize \ \ \ - at most one $x$ in $r$ } \\[2ex]
%
\multicolumn{3}{|c|}{{\bf Simple cut rules} :}\\[.5ex]
\multicolumn{3}{|c|}{\prule{t\Incl t,\Gamma\Seq\Delta}{\Gamma\Seq\Delta}\ ($cut_t$)
\ \ \ \ \ \ \ \ \ 
\prule{x=x, \Gamma\Seq\Delta}{\Gamma\Seq\Delta}\ ($cut_{x=}$) } 
\\[3ex]
%
\multicolumn{3}{|c|}{{\bf Specific cut rules} :}\\[.5ex]
\multicolumn{3}{|c|}
{for each specific axiom $\Ax_k$: \(\lis\atom\Seq \lism\batom{m}\), 
a  rule:}\\[1ex]
\multicolumn{3}{|c|}
{\prule{\Gamma\Seq\Delta,\atom_1\ ;...;\ \Gamma\Seq\Delta,\atom_n\ ;\ 
\batom_1,\Gamma\Seq\Delta\ ;...;\ \batom_m,\Gamma\Seq\Delta} 
{\Gamma\Seq\Delta}\ \ \ ($Sp.cut_k$)} \\[1ex]
\hline
\end{tabular}
\caption{The rules of $NEQ_3$ ($x,y\in\Vars$).}\label{fi:neq3}
\end{figure}

\noindent
Any application of the equality rule \reff{ru:K12} in $NEQ_2$ can be simulated
by repeated applications of the rules \reff{ru:K12s} -- recall that each of the rules
\reff{ru:K12s} abbreviates two rules: $(=_1)$ substituting the first and $(=_2)$
substituting the second term from the active equality $s=t$.
% and \reff{ru:K12a}. 
Weakening rules are admissible in $NEQ_3$ according to lemma~\ref{le:noweak} and, in
the presence of the ($cut$) rule, the simple cut ($cut_t$), ($cut_{x=}$) 
do not add anything to the power of the calculus. 
We thus have the obvious:
\begin{CLAIM}\label{le:neq2isneq3}
 $NEQ_2^c \equiv NEQ_3^c$.
\end{CLAIM}

%\documentclass[10pt]{article}
%%%\documentstyle[a4wide,10pt]{article}
%\makeatletter
%
\ifcase \@ptsize
    % mods for 10 pt
    \oddsidemargin  0.15 in     %   Left margin on odd-numbered pages.
    \evensidemargin 0.35 in     %   Left margin on even-numbered pages.
    \marginparwidth 1 in        %   Width of marginal notes.
    \oddsidemargin 0.25 in      %   Note that \oddsidemargin = \evensidemargin
    \evensidemargin 0.25 in
    \marginparwidth 0.75 in
    \textwidth 5.875 in % Width of text line.
\or % mods for 11 pt
    \oddsidemargin 0.1 in      %   Left margin on odd-numbered pages.
    \evensidemargin 0.15 in    %   Left margin on even-numbered pages.
    \marginparwidth 1 in       %   Width of marginal notes.
    \oddsidemargin 0.125 in    %   Note that \oddsidemargin = \evensidemargin
    \evensidemargin 0.125 in
    \marginparwidth 0.75 in
    \textwidth 6.125 in % Width of text line.
\or % mods for 12 pt
    \oddsidemargin -10 pt      %   Left margin on odd-numbered pages.
    \evensidemargin 10 pt      %   Left margin on even-numbered pages.
    \marginparwidth 1 in       %   Width of marginal notes.
    \oddsidemargin 0 in      %   Note that \oddsidemargin = \evensidemargin
    \evensidemargin 0 in
    \marginparwidth 0.75 in
    \textwidth 6.375 true in % Width of text line.
\fi

\voffset -2cm
\textheight 22.5cm

%\makeatother

%%\makeatletter
%\show\
%\makeatother
\newcommand{\ite}[1]{\item[{\bf #1.}]}
\newcommand{\app}{\mathrel{\scriptscriptstyle{\vdash}}}
\newcommand{\estr}{\varepsilon}
\newcommand{\PSet}[1]{{\cal P}(#1)}
\newcommand{\ch}{\sqcup}
\newcommand{\into}{\to}
\newcommand{\Iff}{\Leftrightarrow}
\renewcommand{\iff}{\leftrightarrow}
\newcommand{\prI}{\vdash_I}
\newcommand{\pr}{\vdash}
\newcommand{\ovr}[1]{\overline{#1}}

\newcommand{\cp}{{\cal O}}

% update function/set
%\newcommand{\upd}[3]{#1\!\Rsh^{#2}_{\!\!#3}} % AMS
\newcommand{\upd}[3]{#1^{\raisebox{.5ex}{\mbox{${\scriptscriptstyle{\leftarrow}}\scriptstyle{#3}$}}}_{{\scriptscriptstyle{\rightarrow}}{#2}}} 
\newcommand{\rem}[2]{\upd{#1}{#2}{\bullet}}
\newcommand{\add}[2]{\upd {#1}{\bullet}{#2}}
%\newcommand{\mv}[3]{{#1}\!\Rsh_{\!\!#3}{#2}}
\newcommand{\mv}[3]{{#1}\:\raisebox{-.5ex}{$\stackrel{\displaystyle\curvearrowright}{\scriptstyle{#3}}$}\:{#2}}

\newcommand{\leads}{\rightsquigarrow} %AMS

\newenvironment{ites}{\vspace*{1ex}\par\noindent 
   \begin{tabular}{r@{\ \ }rcl}}{\vspace*{1ex}\end{tabular}\par\noindent}
\newcommand{\itt}[3]{{\bf #1.} & $#2$ & $\impl$ & $#3$ \\[1ex]}
\newcommand{\itte}[3]{{\bf #1.} & $#2$ & $\impl$ & $#3$ }
\newcommand{\itteq}[3]{\hline {\bf #1} & & & $#2=#3$ }
\newcommand{\itteqc}[3]{\hline {\bf #1} &  &  & $#2=#3$ \\[.5ex]}
\newcommand{\itteqq}[3]{{\bf #1} &  &  & $#2=#3$ }
\newcommand{\itc}[2]{{\bf #1.} & $#2$ &    \\[.5ex]}
\newcommand{\itcs}[3]{{\bf #1.} & $#2$ & $\impl$ & $#3$  \\[.5ex] }
\newcommand{\itco}[3]{   & $#1$ & $#2$  & $#3$ \\[1ex]}
\newcommand{\itcoe}[3]{   & $#1$ & $#2$  & $#3$}
\newcommand{\bit}{\begin{ites}}
\newcommand{\eit}{\end{ites}}
\newcommand{\na}[1]{{\bf #1.}}
\newenvironment{iten}{\begin{tabular}[t]{r@{\ }rcl}}{\end{tabular}}
\newcommand{\ass}[1]{& \multicolumn{3}{l}{\hspace*{-1em}{\small{[{\em Assuming:} #1]}}}}

%%%%%%%%% nested comp's
\newenvironment{itess}{\vspace*{1ex}\par\noindent 
   \begin{tabular}{r@{\ \ }lllcl}}{\vspace*{1ex}\end{tabular}\par\noindent}
\newcommand{\bitn}{\begin{itess}}
\newcommand{\eitn}{\end{itess}}
\newcommand{\comA}[2]{{\bf #1}& $#2$ \\ }
\newcommand{\comB}[3]{{\bf #1}& $#2$ & $#3$\\ }
\newcommand{\com}[3]{{\bf #1}& & & $#2$ & $\impl$ & $#3$\\[.5ex] }

\newcommand{\comS}[5]{{\bf #1} 
   & $#2$ & $#3$ & $#4$ & $\impl$ & $#5$\\[.5ex] }

%%%%%%%%%%%%%%%%
\newtheorem{CLAIM}{Proposition}[section]
\newtheorem{COROLLARY}[CLAIM]{Corollary}
\newtheorem{THEOREM}[CLAIM]{Theorem}
\newtheorem{LEMMA}[CLAIM]{Lemma}
\newcommand{\MyLPar}{\parsep -.2ex plus.2ex minus.2ex\itemsep\parsep
   \vspace{-\topsep}\vspace{.5ex}}
\newcommand{\MyNumEnv}[1]{\trivlist\refstepcounter{CLAIM}\item[\hskip
   \labelsep{\bf #1\ \theCLAIM\ }]\sf\ignorespaces}
\newenvironment{DEFINITION}{\MyNumEnv{Definition}}{\par\addvspace{0.5ex}}
\newenvironment{EXAMPLE}{\MyNumEnv{Example}}{\nopagebreak\finish}
\newenvironment{PROOF}{{\bf Proof.}}{\nopagebreak\finish}
\newcommand{\finish}{\hspace*{\fill}\nopagebreak 
     \raisebox{-1ex}{$\Box$}\hspace*{1em}\par\addvspace{1ex}}
\renewcommand{\abstract}[1]{ \begin{quote}\noindent \small {\bf Abstract.} #1
    \end{quote}}
\newcommand{\B}[1]{{\rm I\hspace{-.2em}#1}}
\newcommand{\Nat}{{\B N}}
\newcommand{\bool}{{\cal B}{\rm ool}}
\renewcommand{\c}[1]{{\cal #1}}
\newcommand{\Funcs}{{\cal F}}
%\newcommand{\Terms}{{\cal T}(\Funcs,\Vars)}
\newcommand{\Terms}[1]{{\cal T}(#1)}
\newcommand{\Vars}{{\cal V}}
\newcommand{\Incl}{\mathbin{\prec}}
\newcommand{\Cont}{\mathbin{\succ}}
\newcommand{\Int}{\mathbin{\frown}}
\newcommand{\Seteq}{\mathbin{\asymp}}
\newcommand{\Eq}{\mathbin{\approx}}
\newcommand{\notEq}{\mathbin{\Not\approx}}
\newcommand{\notIncl}{\mathbin{\Not\prec}}
\newcommand{\notCont}{\mathbin{\Not\succ}}
\newcommand{\notInt}{\mathbin{\Not\frown}}
\newcommand{\Seq}{\mathrel{\mapsto}}
\newcommand{\Ord}{\mathbin{\rightarrow}}
\newcommand{\M}[1]{\mathbin{\mathord{#1}^m}}
\newcommand{\Mset}[1]{{\cal M}(#1)}
\newcommand{\interpret}[1]{[\![#1]\!]^{A}_{\rho}}
\newcommand{\Interpret}[1]{[\![#1]\!]^{A}}
%\newcommand{\Comp}[2]{\mbox{\rm Comp}(#1,#2)}
\newcommand{\Comp}[2]{#1\diamond#2}
\newcommand{\Repl}[2]{\mbox{\rm Repl}(#1,#2)}
%\newcommand\SS[1]{{\cal S}^{#1}}
\newcommand{\To}[1]{\mathbin{\stackrel{#1}{\longrightarrow}}}
\newcommand{\TTo}[1]{\mathbin{\stackrel{#1}{\Longrightarrow}}}
\newcommand{\oT}[1]{\mathbin{\stackrel{#1}{\longleftarrow}}}
\newcommand{\oTT}[1]{\mathbin{\stackrel{#1}{\Longleftarrow}}}
\newcommand{\es}{\emptyset}
\newcommand{\C}[1]{\mbox{$\cal #1$}}
\newcommand{\Mb}[1]{\mbox{#1}}
\newcommand{\<}{\langle}
\renewcommand{\>}{\rangle}
\newcommand{\Def}{\mathrel{\stackrel{\mbox{\tiny def}}{=}}}
\newcommand{\impl}{\mathrel\Rightarrow}
\newcommand{\then}{\mathrel\Rightarrow}
\newfont{\msym}{msxm10}

\newcommand{\false}{\bot}
\newcommand{\true}{\top}

\newcommand{\restrict}{\mathbin{\mbox{\msym\symbol{22}}}}
\newcommand{\List}[3]{#1_{1}#3\ldots#3#1_{#2}}
\newcommand{\col}[1]{\renewcommand{\arraystretch}{0.4} \begin{array}[t]{c} #1
  \end{array}}
\newcommand{\prule}[2]{{\displaystyle #1 \over \displaystyle#2}}
\newcounter{ITEM}
\newcommand{\newITEM}[1]{\gdef\ITEMlabel{ITEM:#1-}\setcounter{ITEM}{0}}
\makeatletter
\newcommand{\Not}[1]{\mathbin {\mathpalette\c@ncel#1}}
\def\LabeL#1$#2{\edef\@currentlabel{#2}\label{#1}}
\newcommand{\ITEM}[2]{\par\addvspace{.7ex}\noindent
   \refstepcounter{ITEM}\expandafter\LabeL\ITEMlabel#1${(\roman{ITEM})}%
   {\advance\linewidth-2em \hskip2em %
   \parbox{\linewidth}{\hskip-2em {\rm\bf \@currentlabel\
   }\ignorespaces #2}}\par \addvspace{.7ex}\noindent\ignorespaces}
\def\R@f#1${\ref{#1}}
\newcommand{\?}[1]{\expandafter\R@f\ITEMlabel#1$}
\makeatother
\newcommand{\PROOFRULE}[2]{\trivlist\item[\hskip\labelsep {\bf #1}]#2\par
  \addvspace{1ex}\noindent\ignorespaces}
\newcommand{\PRULE}[2]{\displaystyle#1 \strut \over \strut \displaystyle#2}
%\setlength{\clauselength}{6cm}
%% \newcommand{\clause}[3]{\par\addvspace{.7ex}\noindent\LabeL#2${{\rm\bf #1}}%
%%   {\advance\linewidth-3em \hskip 3em
%%    \parbox{\linewidth}{\hskip-3em \parbox{3em}{\rm\bf#1.}#3}}\par 
%%    \addvspace{.7ex}\noindent\ignorespaces}
\newcommand{\clause}[3]{\par\addvspace{.7ex}\noindent
  {\advance\linewidth-3em \hskip 3em
   \parbox{\linewidth}{\hskip-3em \parbox{3em}{\rm\bf#1.}#3}}\par 
   \addvspace{.7ex}\noindent\ignorespaces}
\newcommand{\Cs}{\varepsilon}
\newcommand{\const}[3]{\Cs_{\scriptscriptstyle#2}(#1,#3)}
\newcommand{\Ein}{\sqsubset}%
\newcommand{\Eineq}{\sqsubseteq}%


\voffset -1cm


%\begin{document}


\section{$NEQ_4$ and admissibility of ($cut$).}
%
The appplications of ($E_s$) and ($E_a$) substituting a variable 
%in the RHS of the active formula
%, and those of ($E_s$) where the modified term is 
%a variable making up a {\em whole} RHS of an inclusion 
will be called {\em degenerate}:
\begin{itemize}\MyLPar
\item $(E_{ax})$\ \ \prule{\Gamma,x\Incl z, y\Incl r(x)\Seq\Delta}{\Gamma,
y\Incl r(z)\Seq\Delta}\ \
 $z\in\Vars$\ \ \ \ ($r(x)=x$ is a special case)
\item $(E_{sx})$\ \ \prule{\Gamma,x\Incl z\Seq\Delta}{\Gamma\Seq\Delta_z^x}\ \
 $z\in\Vars$
%\item $(E_{Rsd})$\ \ \prule{\Gamma,x\Incl t\Seq\Delta,p\preceq x}{\Gamma\Seq\Delta,p\preceq t}
\end{itemize}
These are excluded from $NEQ_4$. On the other hand, applications of 
various rules replacing a non-variable term will be called {\em 
non-reduced}. We also divide the applications of the 
equality rules explicitly into those modifying the LHS or RHS of 
inclusions. The a-equality rule modifying another equality has been 
removed.
Only reduced applications of the a-rule for modifying an LHS, $\eLar$,
are included in $NEQ_{4}$. 
The simple cut rule of $NEQ_{3}$ removing $t\Incl t$ is replaced by the 
elementary cut removing only variable $x\Incl x$.
The rules of $NEQ_{4}$ are shown in figure~\ref{fi:neq4}.


\documentstyle[a4wide,10pt]{article}
%% \makeatletter

%%%\setlength{\textwidth}{17cm}
%%%\setlength{\evensidemargin}{-.8cm}
%%%\setlength{\oddsidemargin}{-.8cm}

%% \makeatother
%\makeatletter
%\show\
%\makeatother
\newcommand{\ite}[1]{\item[{\bf #1.}]}
\newcommand{\app}{\mathrel{\scriptscriptstyle{\vdash}}}
\newcommand{\estr}{\varepsilon}
\newcommand{\PSet}[1]{{\cal P}(#1)}
\newcommand{\ch}{\sqcup}
\newcommand{\into}{\to}
\newcommand{\Iff}{\Leftrightarrow}
\renewcommand{\iff}{\leftrightarrow}
\newcommand{\prI}{\vdash_I}
\newcommand{\pr}{\vdash}
\newcommand{\ovr}[1]{\overline{#1}}

\newcommand{\cp}{{\cal O}}

% update function/set
%\newcommand{\upd}[3]{#1\!\Rsh^{#2}_{\!\!#3}} % AMS
\newcommand{\upd}[3]{#1^{\raisebox{.5ex}{\mbox{${\scriptscriptstyle{\leftarrow}}\scriptstyle{#3}$}}}_{{\scriptscriptstyle{\rightarrow}}{#2}}} 
\newcommand{\rem}[2]{\upd{#1}{#2}{\bullet}}
\newcommand{\add}[2]{\upd {#1}{\bullet}{#2}}
%\newcommand{\mv}[3]{{#1}\!\Rsh_{\!\!#3}{#2}}
\newcommand{\mv}[3]{{#1}\:\raisebox{-.5ex}{$\stackrel{\displaystyle\curvearrowright}{\scriptstyle{#3}}$}\:{#2}}

\newcommand{\leads}{\rightsquigarrow} %AMS

\newenvironment{ites}{\vspace*{1ex}\par\noindent 
   \begin{tabular}{r@{\ \ }rcl}}{\vspace*{1ex}\end{tabular}\par\noindent}
\newcommand{\itt}[3]{{\bf #1.} & $#2$ & $\impl$ & $#3$ \\[1ex]}
\newcommand{\itte}[3]{{\bf #1.} & $#2$ & $\impl$ & $#3$ }
\newcommand{\itteq}[3]{\hline {\bf #1} & & & $#2=#3$ }
\newcommand{\itteqc}[3]{\hline {\bf #1} &  &  & $#2=#3$ \\[.5ex]}
\newcommand{\itteqq}[3]{{\bf #1} &  &  & $#2=#3$ }
\newcommand{\itc}[2]{{\bf #1.} & $#2$ &    \\[.5ex]}
\newcommand{\itcs}[3]{{\bf #1.} & $#2$ & $\impl$ & $#3$  \\[.5ex] }
\newcommand{\itco}[3]{   & $#1$ & $#2$  & $#3$ \\[1ex]}
\newcommand{\itcoe}[3]{   & $#1$ & $#2$  & $#3$}
\newcommand{\bit}{\begin{ites}}
\newcommand{\eit}{\end{ites}}
\newcommand{\na}[1]{{\bf #1.}}
\newenvironment{iten}{\begin{tabular}[t]{r@{\ }rcl}}{\end{tabular}}
\newcommand{\ass}[1]{& \multicolumn{3}{l}{\hspace*{-1em}{\small{[{\em Assuming:} #1]}}}}

%%%%%%%%% nested comp's
\newenvironment{itess}{\vspace*{1ex}\par\noindent 
   \begin{tabular}{r@{\ \ }lllcl}}{\vspace*{1ex}\end{tabular}\par\noindent}
\newcommand{\bitn}{\begin{itess}}
\newcommand{\eitn}{\end{itess}}
\newcommand{\comA}[2]{{\bf #1}& $#2$ \\ }
\newcommand{\comB}[3]{{\bf #1}& $#2$ & $#3$\\ }
\newcommand{\com}[3]{{\bf #1}& & & $#2$ & $\impl$ & $#3$\\[.5ex] }

\newcommand{\comS}[5]{{\bf #1} 
   & $#2$ & $#3$ & $#4$ & $\impl$ & $#5$\\[.5ex] }

%%%%%%%%%%%%%%%%
\newtheorem{CLAIM}{Proposition}[section]
\newtheorem{COROLLARY}[CLAIM]{Corollary}
\newtheorem{THEOREM}[CLAIM]{Theorem}
\newtheorem{LEMMA}[CLAIM]{Lemma}
\newcommand{\MyLPar}{\parsep -.2ex plus.2ex minus.2ex\itemsep\parsep
   \vspace{-\topsep}\vspace{.5ex}}
\newcommand{\MyNumEnv}[1]{\trivlist\refstepcounter{CLAIM}\item[\hskip
   \labelsep{\bf #1\ \theCLAIM\ }]\sf\ignorespaces}
\newenvironment{DEFINITION}{\MyNumEnv{Definition}}{\par\addvspace{0.5ex}}
\newenvironment{EXAMPLE}{\MyNumEnv{Example}}{\nopagebreak\finish}
\newenvironment{PROOF}{{\bf Proof.}}{\nopagebreak\finish}
\newcommand{\finish}{\hspace*{\fill}\nopagebreak 
     \raisebox{-1ex}{$\Box$}\hspace*{1em}\par\addvspace{1ex}}
\renewcommand{\abstract}[1]{ \begin{quote}\noindent \small {\bf Abstract.} #1
    \end{quote}}
\newcommand{\B}[1]{{\rm I\hspace{-.2em}#1}}
\newcommand{\Nat}{{\B N}}
\newcommand{\bool}{{\cal B}{\rm ool}}
\renewcommand{\c}[1]{{\cal #1}}
\newcommand{\Funcs}{{\cal F}}
%\newcommand{\Terms}{{\cal T}(\Funcs,\Vars)}
\newcommand{\Terms}[1]{{\cal T}(#1)}
\newcommand{\Vars}{{\cal V}}
\newcommand{\Incl}{\mathbin{\prec}}
\newcommand{\Cont}{\mathbin{\succ}}
\newcommand{\Int}{\mathbin{\frown}}
\newcommand{\Seteq}{\mathbin{\asymp}}
\newcommand{\Eq}{\mathbin{\approx}}
\newcommand{\notEq}{\mathbin{\Not\approx}}
\newcommand{\notIncl}{\mathbin{\Not\prec}}
\newcommand{\notCont}{\mathbin{\Not\succ}}
\newcommand{\notInt}{\mathbin{\Not\frown}}
\newcommand{\Seq}{\mathrel{\mapsto}}
\newcommand{\Ord}{\mathbin{\rightarrow}}
\newcommand{\M}[1]{\mathbin{\mathord{#1}^m}}
\newcommand{\Mset}[1]{{\cal M}(#1)}
\newcommand{\interpret}[1]{[\![#1]\!]^{A}_{\rho}}
\newcommand{\Interpret}[1]{[\![#1]\!]^{A}}
%\newcommand{\Comp}[2]{\mbox{\rm Comp}(#1,#2)}
\newcommand{\Comp}[2]{#1\diamond#2}
\newcommand{\Repl}[2]{\mbox{\rm Repl}(#1,#2)}
%\newcommand\SS[1]{{\cal S}^{#1}}
\newcommand{\To}[1]{\mathbin{\stackrel{#1}{\longrightarrow}}}
\newcommand{\TTo}[1]{\mathbin{\stackrel{#1}{\Longrightarrow}}}
\newcommand{\oT}[1]{\mathbin{\stackrel{#1}{\longleftarrow}}}
\newcommand{\oTT}[1]{\mathbin{\stackrel{#1}{\Longleftarrow}}}
\newcommand{\es}{\emptyset}
\newcommand{\C}[1]{\mbox{$\cal #1$}}
\newcommand{\Mb}[1]{\mbox{#1}}
\newcommand{\<}{\langle}
\renewcommand{\>}{\rangle}
\newcommand{\Def}{\mathrel{\stackrel{\mbox{\tiny def}}{=}}}
\newcommand{\impl}{\mathrel\Rightarrow}
\newcommand{\then}{\mathrel\Rightarrow}
\newfont{\msym}{msxm10}

\newcommand{\false}{\bot}
\newcommand{\true}{\top}

\newcommand{\restrict}{\mathbin{\mbox{\msym\symbol{22}}}}
\newcommand{\List}[3]{#1_{1}#3\ldots#3#1_{#2}}
\newcommand{\col}[1]{\renewcommand{\arraystretch}{0.4} \begin{array}[t]{c} #1
  \end{array}}
\newcommand{\prule}[2]{{\displaystyle #1 \over \displaystyle#2}}
\newcounter{ITEM}
\newcommand{\newITEM}[1]{\gdef\ITEMlabel{ITEM:#1-}\setcounter{ITEM}{0}}
\makeatletter
\newcommand{\Not}[1]{\mathbin {\mathpalette\c@ncel#1}}
\def\LabeL#1$#2{\edef\@currentlabel{#2}\label{#1}}
\newcommand{\ITEM}[2]{\par\addvspace{.7ex}\noindent
   \refstepcounter{ITEM}\expandafter\LabeL\ITEMlabel#1${(\roman{ITEM})}%
   {\advance\linewidth-2em \hskip2em %
   \parbox{\linewidth}{\hskip-2em {\rm\bf \@currentlabel\
   }\ignorespaces #2}}\par \addvspace{.7ex}\noindent\ignorespaces}
\def\R@f#1${\ref{#1}}
\newcommand{\?}[1]{\expandafter\R@f\ITEMlabel#1$}
\makeatother
\newcommand{\PROOFRULE}[2]{\trivlist\item[\hskip\labelsep {\bf #1}]#2\par
  \addvspace{1ex}\noindent\ignorespaces}
\newcommand{\PRULE}[2]{\displaystyle#1 \strut \over \strut \displaystyle#2}
%\setlength{\clauselength}{6cm}
%% \newcommand{\clause}[3]{\par\addvspace{.7ex}\noindent\LabeL#2${{\rm\bf #1}}%
%%   {\advance\linewidth-3em \hskip 3em
%%    \parbox{\linewidth}{\hskip-3em \parbox{3em}{\rm\bf#1.}#3}}\par 
%%    \addvspace{.7ex}\noindent\ignorespaces}
\newcommand{\clause}[3]{\par\addvspace{.7ex}\noindent
  {\advance\linewidth-3em \hskip 3em
   \parbox{\linewidth}{\hskip-3em \parbox{3em}{\rm\bf#1.}#3}}\par 
   \addvspace{.7ex}\noindent\ignorespaces}
\newcommand{\Cs}{\varepsilon}
\newcommand{\const}[3]{\Cs_{\scriptscriptstyle#2}(#1,#3)}
\newcommand{\Ein}{\sqsubset}%
\newcommand{\Eineq}{\sqsubseteq}%


\voffset -1cm


\begin{document}
%\ \ \hspace*{1em}\ \  \hfill{\rabove
\begin{figure}[hbt]
\hspace*{2em}
\begin{tabular}{|r@{\ }l@{\ \ \ \ \ \ \ \ \ \ }r@{\ }ll|}
\multicolumn{5}{r}{{\small{\today}}} \\
\hline
\multicolumn{4}{|c}{{\bf Axioms}:} & \\[1ex]
\multicolumn{4}{|c}{$\begin{array}{l@{\ \ \ \ \ \ }l@{\ \ \ \ \ \ }r}
x\Incl t,\GSD, x\Incl t & s= t,\GSD, s= t & \GSD,x=x 
%%% \\[1ex]
%%% x\Incl y,\GSD,x=y & x=y,\GSD,x\Incl y & x\Incl y,\GSD,y\Incl x
\end{array}$} & \\[4ex]
%
\multicolumn{4}{|c}{{\bf Identity rules}:} & \\[1ex]
($=_{Ls}$) &
 \prule{t=s,\Gamma\Seq\Delta,p(s)\preceq q}{t=s,\Gamma\Seq\Delta,p(t)\preceq q} & 
 ($=_{Lar}$) & 
  \prule{x=t, p(x)\Incl q, \Gamma\Seq\Delta}{x=t, p(t)\Incl q,\Gamma\Seq\Delta} &\\[2ex]
% {\footnotesize - $t\not\in\Vars$} & & & \\[1ex]
($=_{Rs}$) & %\prule{t=s,\GSD,p\Incl q(s)}{t=s,\GSD,p\Incl q(t)} 
   & ($=_{Ra}$) &
 \prule{s=t, p\Incl q(s),\Gamma\Seq\Delta}{s=t,p\Incl q(t),\Gamma\Seq\Delta} & \\[2ex]
  & && {\footnotesize ($p\not\in\Vars$ needed for $\Incl_{Ra}$)}   & \\[1ex]
%
\multicolumn{4}{|c}{{\bf Inclusion rules}:} & \\[1ex]
($\Incl_{Ls}$) & \prule{s\Incl t,\GSD, p(t)\preceq q}{s\Incl t, \GSD, p(s)\preceq q}  &
     ($\Incl_{Lar}$) & 
     \prule{x\Incl t, p(x)\preceq q,\GSD}{x\Incl t, p(t)\preceq q,\GSD} & \\[2ex]
  & %would for Lsr: {\footnotesize - $s\not\in\Vars$} 
        && {\footnotesize ($t\in\Vars$ needed for $\preceq_{Lsx}$)}   & \\[1ex]
($\Incl_{Rs}$) & \prule{s\Incl t, \GSD,p\Incl q(s)}{s\Incl t,\GSD,p\Incl q(t)}  & 
   ($\Incl_{Ra}$) & & \\[3ex]
%would for Rsr: & {\footnotesize - $q(t)\not\in\Vars$} & & & \\[2ex]
%
\multicolumn{4}{|c}{{\bf Elimination rules}:} & \\[1ex]
($E_s$) & \prule{x\Incl t,\GSD,\phi[x]}{\GSD,\phi[t]}  &
($E_a$) & \prule{x\Incl t, y\Incl q(x),\GSD}{y\Incl q(t),\GSD} & \\[.5ex]
&  {\footnotesize - $x\not\in \Vars(\Gamma,t)$, $t\not\in\Vars$} &  
   &  {\footnotesize - $x\not\in \Vars(t,\Gamma,\Delta,y)$, $t\not\in\Vars$} & \\
& {\footnotesize - at most one $x$ in $\Delta$;} &  
   & {\footnotesize - at most one $x$ in $q(x)$} & \\
&  {\footnotesize - $\phi\not= p\preceq x$} &  && \\[1ex]
%
\multicolumn{4}{|c}{{\bf Elementary cut}:} & \\[1ex]
\multicolumn{4}{|c}{($cut_x$)\ \ \prule{x\Incl x,\Gamma\Seq\Delta}{\Gamma\Seq\Delta} } &
%%  ($cut_=$) & \prule{x= x,\Gamma\Seq\Delta}{\Gamma\Seq\Delta} & 
  \\[3ex]
\multicolumn{4}{|c}{{\bf Specific cut rules}:} & \\
\multicolumn{4}{|c}
{for each specific axiom $\Ax_k$: \(a_1,...,a_n\Seq s_1,...,s_m\), 
a  rule:} & \\[1ex]
\multicolumn{4}{|c}
{\prule{\Gamma\Seq\Delta,a_1\ ;...;\ \Gamma\Seq\Delta,a_n\ ;\ 
s_1,\Gamma\Seq\Delta\ ;...;\ s_m,\Gamma\Seq\Delta} 
{\Gamma\Seq\Delta}\ \ \ ($Sp.cut_k$)} & \\[1ex]
 \hline
\end{tabular} 
\caption{The rules of $NEQ_{4}$ ($x,y\in\Vars$)} %\label{fi:neq4}
\end{figure}

\noindent
\begin{enumerate}\MyLPar
\item Any derivation $D$ can be transformed into $D^*$ so that $D^*$ contains \\
\begin{tabular}{r@{:\ \ }l}
either & no $(=_{Lsx}), (\Incl_{Lsx}), (\Incl_{Rs})$ \\
or & no $(\Incl_{Larx})$ ... ? ... $(=_{Larx})$
\end{tabular}
\item $(=_{a=}), (\Incl_{Lanr}), (=_{Ranr}), (=_{Lanr}), (\Incl_{Ra})$ are admissible
without increasing  $\#(\Incl_{Lar} + =_{Lar})$ 
\item $(cut_t)$ and $(cut_{x=})$ are admissible
\end{enumerate}

\end{document}
%% \begin{REMARK}\label{re:cutx}
%% Actually, a more restricted
%% version of the elementary cut rule would suffice: 
%% \begin{center} \prule{x\Incl x,
%% x=p(r), \GSD}{x=p(r),\GSD} \end{center}
%% However, the more general formulation here makes
%% the proofs in this section easier, and the rule will be eliminated anyway in the
%% final version of the calculus in the next section.
%% \end{REMARK}
\noindent
The essential facts in the rest of this section are: 
\begin{enumerate}\MyLPar
\item The possibility of transforming $NEQ_{4}$ 
derivations in two different ways eliminating either some undesirable 
applications of a-rules or else of some s-rules. These proofs are in 
subsection~\ref{sub:der}. 
\item Admissibility of the rules 
excluded from $NEQ_{3}$ leading to the equivalence of the two 
calculi. These proofs are partly in subsection~\ref{sub:der} -- the 
crucial proofs of admissibility of the simple cut rule and
%redundancy of the $(cut_{x=})$ rule, as well as 
equivalence of the two calculi are in subsection \ref{sub:equiv}. 
\item Admissibility of $(cut)$ in $NEQ_{4}$ is 
shown in subsection~\ref{sub:cut}. The proof is based on the  
possibility of modifying $NEQ_{4}$ derivations as in point 1 above.
\end{enumerate}.

\subsection{Some facts about $NEQ_{4}$ derivations}\label{sub:der}

\begin{LEMMA}\label{fa:noeRs}
$\der{NEQ_4}DS\ \impl\ \der{NEQ_4}{D^*}S$
and $D^*$  does not contain $\eRs$.
%[Moreover, $D^*$ does not introduce any new applications of s-rules.
%This fact will be combined with lemma~\ref{le:nosx}.]
\end{LEMMA}
\begin{PROOF}
%By lemmas~\ref{le:noeqeq}-\ref{le:noLanr} we may assume that $D$ contains no
%applications of ($=_{a=}$), ($\Incl_{anr}$) or ($=_{Lanr}$).
By induction on $\<\#(\eRs,D),h(D)\>$.
For the axioms, we only have to consider the one with inclusion in the consequent.
%\begin{itemize}
%\item $x\Incl t,\GSD,x\Incl t$
%\end{itemize}
\[
\begin{array}{rl}
p=s, x\Incl t(s),\GSD,x\Incl t(s) \\ \cline{1-1}
p=s, x\Incl t(s),\GSD,x\Incl t(p) & \rabove{\eRs}
\end{array}
\conv
\begin{array}{rl}
p=s, x\Incl t(p),\GSD,x\Incl t(p) \\ \cline{1-1}
p=s, x\Incl t(s),\GSD,x\Incl t(p) & \rabove{\eRa}
\end{array}
\]
Consider the uppermost application of ($=_{Rs}$) in $D$:
\[ \begin{array}{cl}
\vdots          & \raisebox{-1.2ex}[1.5ex][0ex]{$R$} \\ \cline{1-1}
r=s, \Gamma\Seq\Delta, q\Incl t(r) &
\raisebox{-1.2ex}[1.5ex][0ex]{($=_{Rs}$)} \\ \cline{1-1}
r=s, \Gamma\Seq\Delta, q\Incl t(s) 
\end{array} \]
\begin{LS}
\item\label{it:RsEsA} $R=\Esr$ with $q\Incl t(r)$ as the modified formula and $r$ as
modified term. We have two subcases corresponding to the situation when the
term $r'$ substituted by $\Esr$ for its eigen-variable is a subterm or
superterm of (or equal to) $r$. (When the two are independent, the case is
trivial.) 
\begin{LSA}
\item  $r'$ is a subterm of $r$,\\
i.e., $t'(r')$ and $t(r)$ are identical with $r=f(r')$ and $t'(x)=t(f(x))$.
\[ D \left \{\begin{array}{cl}
%\vdots          \\
x\Incl r', f(r')=s, \GSD, q\Incl t(f(x)) & \raisebox{-1.2ex}[1.5ex][0ex]{$\Esr$} \\ \cline{1-1}
f(r')=s, \GSD, q\Incl t(f(r')) &
\raisebox{-1.2ex}[1.5ex][0ex]{($=_{Rs}$)} \\ \cline{1-1}
f(r')=s, \GSD, q\Incl t(s) 
\end{array} \right . \convd
D' \left \{ \begin{array}{cl}
x\Incl r', f(r')=s, f(x)=s, \Gamma\Seq\Delta,q\Incl t(f(x)) & \raisebox{-1.2ex}[1.5ex][0ex]{($=_{Rs}$)} \\ \cline{1-1}
x\Incl r', f(r')=s, f(x)=s, \Gamma\Seq\Delta, q\Incl t(s) &
\raisebox{-1.2ex}[1.5ex][0ex]{$\iLa$} \\ \cline{1-1}
x\Incl r', f(r')=s, f(r')=s, \Gamma\Seq\Delta, q\Incl t(s) &
\raisebox{-1.2ex}[1.5ex][0ex]{$\Esr$} \\ \cline{1-1}
f(r')=s, \Gamma\Seq\Delta, q\Incl t(s) 
\end{array} \right . \]
Induction hypothesis allows us to eliminate the application of ($=_{Rs}$)
from $D'$ since it occurs higher than in $D$.
\item $r'$ is a superterm of (or equal to) $r$,
i.e., $r'=f(r)$ and $t(x)=t'(f(x))$. \\ On the right we have the case when
$r'=r$ and $s$ is a variable $y$
\[ \sma{\begin{array}{c@{\hspace*{5em}}c}%D \left \{
\begin{array}[t]{cl}
%\vdots          \\
x\Incl f(r), r=s, \Gamma\Seq\Delta, q\Incl t'(x) & \\ \cline{1-1}
r=s, \Gamma\Seq\Delta, q\Incl t'(f(r)) & \rabove{\Esr} \\ \cline{1-1}
r=s, \Gamma\Seq\Delta, q\Incl t'(f(s)) & \rabove{\eRs} 
\end{array} %\right . 
& 
\prar{
x\Incl r, r=y, \GSD, q\Incl t'(x) \cl
          r=y, \GSD, q\Incl t'(r) & \rabove{\Esr} \cl
          r=y, \GSD, q\Incl t'(y) & \rabove{\eRs} 
} \\[.5ex]
\Downarrow & \Downarrow \\[.5ex]
%\convd %D \left \{
\begin{array}[t]{cl}
x\Incl f(r), r=s, \Gamma\Seq\Delta, q\Incl t'(x) & \\ \cline{1-1}
x\Incl f(s), r=s, \Gamma\Seq\Delta, q\Incl t'(x) & \rabove{\eRa}
\\ \cline{1-1}
r=s, \Gamma\Seq\Delta, q\Incl t'(f(s)) & \rabove{\Esr} 
\end{array} %\right . 
&
\prar{
x\Incl r, r=y, \GSD, q\Incl t'(x) \cl
x\Incl y, r=y, \GSD, q\Incl t'(x) & \rabove{\eRa} \cl
x\Incl y, r=y, \GSD, q\Incl t'(y) & \rabove{\iRs} \cl
x\Incl r, r=y, \GSD, q\Incl t'(y) & \rabove{\eRa} \cl
          r=y, \GSD, q\Incl t'(y) & \rabove{\Esr}
}
\end{array} }
\]
The case when the modified term is substituted into LHS of $\Incl$ allows simple
swap since then $\Esr$ and $\eRs$ modify different terms.%is treated analogously 
\end{LSA}
\item $R=\iLa$ with the modified formula $r=s(v)$ and $v$ as the modified
term:
\[ D \left \{ \begin{array}{cl}
 D_1 \left \{ \begin{array}{c}
               \vdots       \\ 
               r=s(u), u\Incl v, \Gamma\Seq\Delta,q\Incl t(r) 
           \end{array} \right . 
         & \raisebox{-3.2ex}[1.5ex][0ex]{$\iLa$}  \\ \cline{1-1}
 r=s(v), u\Incl v, \Gamma\Seq\Delta, q\Incl t(r) &
 \raisebox{-1.2ex}[1.5ex][0ex]{($=_{Rs}$)} \\ \cline{1-1}
 r=s(v), u\Incl v, \Gamma\Seq\Delta, q\Incl t(s(v)) 
 \end{array} \right . \convd
  D^* \left \{ \begin{array}{cl}
  D' \left \{ \begin{array}{cl}
    D_1 \left \{ \begin{array}{cl}
 \vdots       \\ 
 r=s(u), u\Incl v, \Gamma\Seq\Delta,q\Incl t(r) 
  \end{array} \right . & \raisebox{-3.2ex}[1.5ex][0ex]{($W_a$)}  \\
  \cline{1-1}
 r=s(u), r=s(v), u\Incl v, \Gamma \Seq \Delta, q\Incl t(r)
  & \raisebox{-1.2ex}[1.5ex][0ex]{($=_{Rs}$)} \\ \cline{1-1}
 r=s(u), r=s(v), u\Incl v, \Gamma\Seq\Delta, q\Incl t(s(v))  \end{array}
 \right . &  \\ \cline{1-1}
 r=s(v), r=s(v), u\Incl v, \Gamma\Seq\Delta, q\Incl t(s(v)) 
 & \raisebox{1.2ex}[1.5ex][0ex]{$\iLa$} 
 \end{array} \right . \]
  $h(D') < h(D)$, so $D^*$
  can be transformed into a desired derivation without ($=_{Rs}$). (Notice
  that the result of $D^*$ is the same as the result of $D$ by the implicit
  application of contraction.)
 \item $R=\iLs$ :
 \[ D \left \{ \begin{array}{cl}
  D_1 \left \{ \begin{array}{c}
               \vdots       \\ 
               r=s, q\Incl u, \Gamma\Seq\Delta, u\Incl t(r) 
            \end{array} \right . 
          & \raisebox{-3.2ex}[1.5ex][0ex]{$\iLs$}  \\ \cline{1-1}
 r=s, q\Incl u, \Gamma\Seq\Delta, q\Incl t(r) &
 \raisebox{-1.2ex}[1.5ex][0ex]{($=_{Rs}$)} \\ \cline{1-1}
 r=s, q\Incl u, \Gamma\Seq\Delta, q\Incl t(s) 
 \end{array} \right . \convd
  D^* \left \{ \begin{array}{cl}
  D' \left \{ \begin{array}{cl}
    D_1 \left \{ \begin{array}{cl}
 \vdots       \\ 
 r=s, q\Incl u, \Gamma\Seq\Delta, u\Incl t(r) 
  \end{array} \right . & \raisebox{-3.2ex}[1.5ex][0ex]{($=_{Rs}$)}  \\
  \cline{1-1}
 r=s, r=s(v), q\Incl u, \Gamma \Seq \Delta, u\Incl t(s) \end{array}
 \right . &  \\ \cline{1-1}
 r=s, q\Incl u, \Gamma\Seq\Delta, q\Incl t(s) 
 & \raisebox{1.2ex}[1.5ex][0ex]{$\iLs$} 
 \end{array} \right . \]
 Since $h(D')<h(D)$, the induction hypothesis yields a desired derivation.
 \item $R=\iRs$
\[\sma{\prarc{
f(p)=s, r\Incl p,\GSD, t\Incl q(f(r)) \cl
f(p)=s, r\Incl p,\GSD, t\Incl q(f(p)) & \rabove{\iRs} \cl
f(p)=s, r\Incl p,\GSD, t\Incl q(s) & \rabove{\eRs}
}
\conv
\prarc{
f(r)=s,f(p)=s, r\Incl p,\GSD, t\Incl q(f(r)) \cl
f(r)=s,f(p)=s, r\Incl p,\GSD, t\Incl q(s)  & \rabove{\eRs} \cl
f(p)=s,f(p)=s, r\Incl p,\GSD, t\Incl q(s)  & \rabove{\iLa} 
} }
\]
\[\sma{ \prarc{
p\Incl f(s), s=r, \GSD, t\Incl q(p) \cl
p\Incl f(s), s=r, \GSD, t\Incl q(f(s)) &\rabove{\iRs} \cl
p\Incl f(s), s=r, \GSD, t\Incl q(f(r)) &\rabove{\eRs} 
}
\conv
\prarc{
p\Incl f(r), p\Incl f(s), s=r, \GSD, t\Incl q(p) \cl
p\Incl f(r), p\Incl f(s), s=r, \GSD, t\Incl q(f(r)) &\rabove{\iRs} \cl
p\Incl f(s), p\Incl f(s), s=r, \GSD, t\Incl q(f(r)) &\rabove{\eRa}
} }
\]
 \item The other cases when $R$ is $\Ear$, $\eLs$, ($=_{Ra}$), ($=_{La}$), 
 ($Sp.cut$) or ($cut_t$) can be trivially swapped 
 with the application of ($=_{Rs}$) since they do not affect the active
 equation nor the modified inclusion of ($=_{Rs}$).
 \end{LS}
%% As we see, $D^*$ involves the same applications of the s-rules
%% as the original derivation $D$.
%Also,
% since $\#(\preceq_{anr},D)=0$ and no new such applications occurred, we do get
% $\#(\preceq_{anr},D^*)=0$.]
 \end{PROOF}

%\newpage
\subsubsection{Some admissible and redundant a-rules}
The underlying theme of this subsection are the reduced vs. non-reduced applications
of various a-rules. This facts, to be utilized in the proof of the $(cut)$-elimination,
appear in the side-conditions of the lemmas.
\begin{LEMMA}\label{le:noeqeq}
The following rule is admissible in $NEQ_4$:
\[
\mprule{s=t, f(s)=p, \GSD}{s=t, f(t)=p,\GSD}\ \ \eae
\]
\noindent
Morover, if $D^*$ derives $S$ using this
rule while $D$ derives $S$ without it then
$\#(\pLar,D^*)\leq \#(\pLar,D)$,
and $\#(\pLanr,D^*)\leq \#(\pLanr,D)$.
\end{LEMMA}
\begin{PROOF}
By lemma \ref{fa:noeRs} we assume no applications of $\eRs$ in $D$.
By induction on $\#(\eae,D)$ and $h(D)$. 
For $h(D)=1$ we have one possible axiom:
\[ \begin{array}{rl}
 s=t, f(s)=p,\GSD, f(s)=p \\ \cline{1-1}
 s=t, f(t)=p,\GSD, f(s)=p & \rabove{\eae}
\end{array} 
\conv
\begin{array}{rl}
 s=t, f(t)=p,\GSD, f(t)=p \\ \cline{1-1}
 s=t, f(t)=p,\GSD, f(s)=p & \rabove{\eLs}
\end{array}
\]
%
Consider the uppermost application of
$\eae$ and the rule $R$ applied just above it.  The elimination rules, 
$\iLs$, $\iRs$, ($Sp.cut$) and ($cut_x$) 
do not affect
the active or the principal formula of $\eae$ so these can be swapped
reducing the height at which the application of $\eae$ occurs. We have
then the following cases for $R$:
\begin{LS}
\item $\eLar$:
\[ D \left\{ \begin{array}{rl}
 D_1 \left\{ \begin{array}{r}
\multicolumn{1}{c}{\vdots} \\
r=x, x=t, s(x)\Incl p, \GSD \end{array} \right. \\ \cline{1-1}
r=x, x=t, s(t)\Incl p, \GSD & \rabove{\eLar} \\ \cline{1-1}
r=x, r=t, s(t)\Incl p, \GSD & \rabove{\eae} 
\end{array} \right.
\convd
D^* \left\{ \begin{array}{rl}
       D^*_1\left\{ \begin{array}{r}
       \multicolumn{1}{c}{\vdots} \\
       x\Incl t, r=x,x=t,s(x)\Incl p,\GSD \end{array} \right . \\ \cline{1-1}
x\Incl t, r=x, r=t,s(x)\Incl p,\GSD & \rabove{\eae} \cl
x\Incl t, r=x, r=t,s(t)\Incl p,\GSD &\rabove{\iLar} \cl
x\Incl r, r=x, r=t,s(t)\Incl p,\GSD &\rabove{\eRa} \cl
x\Incl x, r=x, r=t,s(t)\Incl p,\GSD &\rabove{\eRa} \cl
r=x, r=t, s(t)\Incl p,\GSD & \raisebox{1.2ex}[1.5ex][0ex]{($cut_x$)}
\end{array} \right .
\]
\noindent $h(D^*_1)< h(D_1)+1$ yields the conclusion. Exactly the same transformation 
is made when $\eae$ in $D$ modifies $t$ (or its subterm).
Since the transformation replaces an application of $\eLar$ by an application of 
$\iLar$, we have the unchanged total number of applications of $\pLar$ and $\pLanr$.

\item $\eLs$: The only case when simple swapping of the last two rules is
insufficient occurs when
the active formula of $\eLs$ is modified in the last application of
$\eae$. 
The end of the derivation has then the form:
%
\[D \left\{ \begin{array}{cl}
%\vdots \\
s=t, w(t)=q,\Gamma \Seq \Delta, p(q)=r  &  \\ \cline{1-1}
s=t, w(t)=q,\Gamma \Seq \Delta, p(w(t))=r & \rabove{\eLs} \\ \cline{1-1}
s=t, w(s)=q, \Gamma \Seq\Delta, p(w(t))=r & \rabove{\eae} 
\end{array} \right. \convd
%
%Nevertheless, swapping the applications of ($=_a$) and ($=_s$) we may obtain:
%
 D^* \left \{ \begin{array}{cl}
D_1 \left\{ \begin{array}{rl}
%\multicolumn{1}{c}{\vdots} \\
s=t, w(t)=q, \Gamma \Seq\Delta, p(q)=r  & \\ \cline{1-1}
s=t, w(s)=q,\Gamma \Seq\Delta, p(q)=r  & \rabove{\eae}
\end{array} \right.
\\ \cline{1-1}
s=t, w(s)=q, \Gamma \Seq\Delta, p(w(s))=r & \rabove{\eLs} \\ \cline{1-1}
s=t, w(s)=q, \Gamma \Seq\Delta, p(w(t))=r & \rabove{\eLs} 
\end{array}  \right . \]
%
$h(D_1) < h(D)$ so, by the induction hypothesis, there is a $NEQ_4$ derivation
corresponding to $D^*$ deriving the same  sequent without $\eae$. 
The case when in  $D$,$\eLs$ replaces $w(t)$ by $q$ is treated analogously.
%
\item $\iLa$ -- two cases%($\Incl_a$):
\[D \left\{ \begin{array}{rl} D_1\left\{\begin{array}{c} \vdots \\
s=t, w(p)=q, p\Incl r(t),\Gamma \Seq \Delta \end{array}\right. & \\ \cline{1-1}
s=t, w(r(t))=q, p\Incl r(t),\Gamma \Seq \Delta  &   
         \rabove{\iLa} \\ \cline{1-1}
s=t, w(r(s))=q, p\Incl r(t),\Gamma \Seq\Delta & \rabove{\eae} 
\end{array} \right. \convd
%
%We construct $D^*$:
D^* \left\{ \begin{array}{rl} D^*_1\left\{ \begin{array}{c} \vdots \\
s=t, w(p)=q, p\Incl r(t), p\Incl r(s),\Gamma \Seq \Delta\end{array}\right. \\ \cline{1-1}
s=t, w(r(s))=q, p\Incl r(t), p\Incl r(s), \Gamma \Seq \Delta  & 
   \rabove{\iLa} \\ \cline{1-1}
s=t, w(r(s))=q, p\Incl r(t),p\Incl r(t),\Gamma \Seq\Delta  &
   \rabove{\eRa}
\end{array} \right. \]
%
where the last sequent is the same as in $D$ by implicit contraction. The last sequent 
of $D^*_1$ can be obtained as in $D_1$ by weakening (lemma~\ref{le:noweak})
with $h(D^*_1)=h(D_1)$.
\[\sma{\prarc{
s=t(r), w(t(p))=q, p\Incl r, \GSD \cl
s=t(r), w(t(r))=q, p\Incl r, \GSD & \rabove{\iLa} \cl
s=t(r), w(s)=q, p\Incl r, \GSD & \rabove{\eae}
}\conv
\prarc{
s=t(p), s=t(r), w(t(p))=q, p\Incl r, \GSD \cl
s=t(p), s=t(r), w(s)=q, p\Incl r, \GSD & \rabove{\eae} \cl
s=t(r), s=t(r), w(s)=q, p\Incl r, \GSD & \rabove{\iLa} 
} }
\]
%
\item ($=_{Ra}$):
\[ \sma{
\begin{array}{rl}
\multicolumn{1}{c}{\vdots} \\
r=u, w(u)=t, p\Incl f(w(u)), \GSD \\ \cline{1-1}
r=u, w(u)=t, p\Incl f(t), \GSD & \rabove{\eRa} \\ \cline{1-1}
r=u, w(r)=t, p\Incl f(t), \GSD & \rabove{\eae} \end{array}
\conv
\begin{array}{rl}
\multicolumn{1}{c}{\vdots} \\
r=u, w(u)=t, p\Incl f(w(u)), \GSD \\ \cline{1-1}
r=u, w(r)=t, p\Incl f(w(u)), \GSD & \rabove{\eae} \\ \cline{1-1}
r=u, w(r)=t, p\Incl f(w(r)), \GSD & \rabove{\eRa} \\ \cline{1-1}
r=u, w(r)=t, p\Incl f(t), \GSD & \rabove{\eRa} 
\end{array} }
\]
Analogous transformation is applied when $\eae$ modifies $t$ rather than $w(u)$.
%% \item $\eRs$
%% \[\prar{
%% p(s)=r, t=s,\GSD, q\Incl f(p(s)) \cl
%% p(s)=r, t=s,\GSD, q\Incl f(r) & \rabove{\eRs} \cl
%% p(t)=r, t=s,\GSD, q\Incl f(r) & \rabove{\eae} 
%% }\conv
%% \prar{
%% p(s)=r, t=s,\GSD, q\Incl f(p(s)) \cl
%% p(t)=r, t=s,\GSD, q\Incl f(p(s)) & \rabove{\eae} \cl
%% p(t)=r, t=s,\GSD, q\Incl f(p(t)) & \rabove{\eRs} \cl
%% p(t)=r, t=s,\GSD, q\Incl f(r) & \rabove{\eRs} 
%% }
%% \]
\item All other cases involve only $\Incl$ in the antecedent so they admit trivial swap.
\end{LS}
In all cases no new applications of $\pLa$ appear in the transformed derivations 
$D^*$, and the ones which occur there are the same as in the original derivations $D$.
\end{PROOF}
%
\begin{LEMMA}\label{le:noar}\label{le:noRanr} 
$\der{NEQ_4}DS\ \impl\ \der{NEQ_4}{D^*}S$
and $D^*$ contains no applications of $\iLanr$ nor $\eRanr$.
Moreover, $\#(\pLar,D^*)\leq\#(\pLar,D)$ and
 $\#(\pLanr,D^*)\leq\#(\pLanr,D)$.
\end{LEMMA}
\begin{PROOF}
By induction on $\<(\#\iLanr+\#\eRanr,D),h(D)\>$.
By lemma~\ref{fa:noeRs}, we assume that the derivation $D$ does not 
contain any $\eRs$.
We first consider the case where the uppermost application of any of the
two rules is $\iLanr$.

%, where $n$ is the number of
%applications of ($\Incl_{anr}$) in $D$ and $h(D)$ is the height of $D$. 
The basis
case of $h(D)=1$ is as follows:
\[ \begin{array}{rl}
 s\Incl t, f(s)=p,\GSD, f(s)=p \\ \cline{1-1}
 s\Incl t, f(t)=p,\GSD, f(s)=p & \rabove{\iLanr}
\end{array} 
\conv
\begin{array}{rl}
 s\Incl t, f(t)=p,\GSD, f(t)=p \\ \cline{1-1}
 s\Incl t, f(t)=p,\GSD, f(s)=p & \rabove{\iLs}
\end{array}
\]
If we were to use ($=_{Ranr}$) the axiom must be of the form $x\Incl t,\GSD,x\Incl t$, 
so the application would be reduced. 

Consider the uppermost application of $\iLanr$ and the rule $R$ applied
immediately above it:
\[ q\notin \cal V\ \ \ \begin{array}{cl}
  \vdots & \\ \cline{1-1}
  q\prec t,s(q)\preceq p,\Gamma\Seq\Delta & \rabove{R} \\ \cline{1-1}
  q\prec t,s(t)\preceq p,\Gamma\Seq\Delta & \rabove{\iLanr} 
\end{array} \]
%
%where $q\notin \cal V$.
\begin{LS}
\item $R$ is $\iLs$ with the active formula $s(q)\Incl p$. 
  \[D \left\{ \begin{array}{cl}
%\vdots \\
 q\prec t,s(q)\Incl p,\Gamma \Seq \Delta,s_1(p)\preceq q_1  &
  \\ \cline{1-1}
q\prec t,s(q)\Incl p,\Gamma \Seq \Delta, s_1(s(q))\preceq q_1 &\rabove{\iLs} 
\\ \cline{1-1}
 q\prec t,s(t)\Incl p,\Gamma \Seq\Delta, s_1(s(q))\preceq q_1 &
 \rabove{\iLanr} 
\end{array} \right. \]
If the active formula of $\iLs$ does not contain $q$ in the way 
indicated above, the two applications can be
trivially swapped and the induction hypothesis allows to eliminate 
$\iLanr$. 

From the above  $D$, we may deduce the existence of the following derivation:
\[ \begin{array}{cl}
D_1\left\{
\begin{array}{cl}
%\vdots \\
 q\prec t,s(q)\Incl p,\GSD,s_1(p)\preceq q_1
         & \raisebox{-1.2ex}[1.5ex][0ex]{$\iLanr$} \\ \cline{1-1}
 q\prec t,s(t)\Incl p,\GSD,s_1(p)\preceq q_1
\end{array} \right. \\ \cline{1-1}
 q\prec t,s(t)\Incl p,\GSD, s_1(s(t))\preceq q_1
         & \rabove{\iLs} \\ \cline{1-1}
 q\prec t,s(t)\Incl p,\GSD, s_1(s(q))\preceq q_1
         & \rabove{\iLs}
\end{array}  \]
Since $h(D_1) < h(D)$, the induction hypothesis allows us to eliminate 
$\iLanr$.
\item $R$ is $\Ear$. Since
$q\not\in\cal V$ also $s(q)\not\in\cal V$. Therefore none of $q\Incl t$
and $s(q)\preceq p$ can be the modified formula of an application of $\Ear$.
Hence the two applications can be swapped and induction hypothesis applied.
%
\item\label{it:eqLa} $R$ is $\eLar$: %, i.e., ($=_{Lar}$):
\[ \begin{array}{rl}
%\multicolumn{1}{c}{\vdots} \\
q(x)\Incl t, s(q(r))\preceq p, x=r, \Gamma\Seq\Delta \\ \cline{1-1}
q(r)\Incl t, s(q(r))\preceq p, x=r, \Gamma\Seq\Delta  &
\raisebox{1.2ex}[1.5ex][0ex]{$(=_{Lar})$} \\ \cline{1-1}
q(r)\Incl t, s(t)\preceq p, x=r, \Gamma\Seq\Delta  &
\raisebox{1.2ex}[1.5ex][0ex]{$\iLanr$}
\end{array} \convd
%Using weakening (another axiom) we construct the following derivation $D'$:
 \begin{array}{rl}
%\multicolumn{1}{c}{\vdots} \\
q(r)\Incl t, q(x)\Incl t, s(q(r))\preceq p, x=r, \GSD \cl
q(r)\Incl t, q(x)\Incl t, s(t)\preceq p, x=r, \GSD  & \rabove{\iLanr} \cl
q(r)\Incl t, q(r)\Incl t, s(t)\preceq p, x=r, \GSD  & \rabove{\eLar}
\end{array} \] 
Analogous weakening is used when the modified formula of $\eLa$
is modified also in the subsequent application of $\iLanr$.
%
\item $R$ is $\eRar$ %($=_{Rar}$):
If the active formula of $\iLanr$ is not modified by ($=_{Rar}$) we
obtain the required reduction in height by swapping analogous to the previous
case. So, assume it was modified. The derivation ends then as follows:
\[ \begin{array}{rl}
%\multicolumn{1}{c}{\vdots} \\
t(y) \Incl q, y\Incl p(k), r=k, \Gamma\Seq\Delta \\ \cline{1-1}
t(y)\Incl q, y\Incl p(r), r=k, \Gamma\Seq\Delta &
\rabove{\eRar} \\ \cline{1-1}
t(p(r))\Incl q, y\Incl p(r), r=k, \Gamma\Seq\Delta &
\rabove{\iLar} 
\end{array} 
\]
Since we have ($=_{Rar}$), the LHS in $y\Incl p(k)$ must be a variable, which means 
that the final application is ($\Incl_{Lar}$).
%
\item $R$ is $\iLar$: This case is treated exactly as 
case \ref{it:eqLa}  with the applications of $\iLar$ instead:
\[ \sma{ \begin{array}{rl}
%\multicolumn{1}{c}{\vdots} \\
q(x)\Incl t, s(q(r))\preceq p, x\Incl r, \GSD \cl
q(r)\Incl t, s(q(r))\preceq p, x\Incl r, \GSD  & \rabove{\iLar} \cl
q(r)\Incl t, s(t)\preceq p, x\Incl r, \GSD  & \rabove{\iLanr}
\end{array} 
\conv
 \begin{array}{rl}
%\multicolumn{1}{c}{\vdots} \\
q(r)\Incl t, q(x)\Incl t, s(q(r))\preceq p, x\Incl r, \GSD \cl
q(r)\Incl t, q(x)\Incl t, s(t)\preceq p, x\Incl r, \GSD  & \rabove{\iLanr} \cl
q(r)\Incl t, q(r)\Incl t, s(t)\preceq p, x\Incl r, \GSD  & \rabove{\iLar}
\end{array} } \]
Analogous procedure yields the result if the modified formula of the
application of $\iLar$ is not active but modified in the subsequent $\iLanr$.
%
\item $R$ is $\iRs$
\[\prar{
q\Incl t, s(q)\Incl p, \GSD, r\Incl f(s(q)) \cl
q\Incl t, s(q)\Incl p, \GSD, r\Incl f(p) & \rabove{\iRs} \cl
q\Incl t, s(t)\Incl p, \GSD, r\Incl f(p) & \rabove{\iLanr} 
}
\conv
\prar{
q\Incl t, s(q)\Incl p, \GSD, r\Incl f(s(q)) \cl
q\Incl t, s(t)\Incl p, \GSD, r\Incl f(s(q)) & \rabove{\iLanr} \cl
q\Incl t, s(t)\Incl p, \GSD, r\Incl f(s(t)) & \rabove{\iRs} \cl
q\Incl t, s(t)\Incl p, \GSD, r\Incl f(p) & \rabove{\iRs} 
}
\]
\item $\Esr$, ($Sp.cut$) and ($cut_t$) can be swapped with $\iLanr$.
\end{LS}
Now the case when the uppermost application is of $\eRanr$
\begin{LS}
\item $R$ is $\eLar$: %($=_{La}$) $D$ ends as follows
\[ \hspace*{-1.5em}  \begin{array}{rl} 
x=u, s=t, q(x)\Incl p(t), \GSD \\ \cline{1-1}
x=u, s=t, q(u)\Incl p(t), \GSD & \rabove{\eLar} \cl
x=u, s=t, q(u)\Incl p(s), \GSD & \rabove{\eRanr} \end{array} \conv
% \] Swapping, we get ($=_{Ranr}$) at a lower height: \[ 
\begin{array}{rl}
x=u, s=t, q(x)\Incl p(t), \GSD \\ \cline{1-1}
x=u, s=t, q(x)\Incl p(s), \GSD & \rabove{\eRanr} \cl
x=u, s=t, q(u)\Incl p(s), \GSD & \rabove{\eLar} \end{array} \]
%
\item $R$ is $\iLar$: The side formula of $\eRanr$ cannot be active
here, since we are using reduced version $\iLar$:
\[ \begin{array}{rl}
x\Incl u, s=t, q(x)\Incl p(t), \GSD \\ \cline{1-1}
x\Incl u, s=t, q(u)\Incl p(t), \GSD & \rabove{\iLar} \cl
x\Incl u, s=t, q(u)\Incl p(s), \GSD & \rabove{\eRanr} \end{array} 
%Again, swapping gives reduction of height:
\conv
\begin{array}{rl}
x\Incl u, s=t, q(x)\Incl p(t), \GSD \\ \cline{1-1}
x\Incl u, s=t, q(x)\Incl p(s), \GSD & \rabove{\eRanr} \cl
x\Incl u, s=t, q(u)\Incl p(s), \GSD & \rabove{\iLar} \end{array} \]
%
\item $R$ is $\iLs$: %($\Incl_{s}$)
\[ \prar{
s=t, q\Incl p(t), \GSD, w(p(t))\preceq r \cl
s=t, q\Incl p(t), \GSD, w(q)\preceq r & \rabove{\iLs} \cl
s=t, q\Incl p(s), \GSD, w(q)\preceq r & \rabove{\eRanr} 
}
\conv
%We transform the derivation reducing the height of $\eRanr$
\prar{
s=t, q\Incl p(t), \GSD, w(p(t))\preceq r \cl
s=t, q\Incl p(s), \GSD, w(p(t))\preceq r & \rabove{\eRanr} \cl
s=t, q\Incl p(s), \GSD, w(p(s))\preceq r & \rabove{\eLs} \cl
s=t, q\Incl p(s), \GSD, w(q)\preceq r & \rabove{\iLs} 
} \]
%
\item $R$ is $\iRs$:
\[ \prar{
s=t, q\Incl p(t),\GSD, r\Incl f(q) \cl
s=t, q\Incl p(t),\GSD, r\Incl f(p(t)) & \rabove{\iRs} \cl
s=t, q\Incl p(s),\GSD, r\Incl f(p(t)) & \rabove{\eRanr}
} \conv
\prar{
s=t, q\Incl p(t),\GSD, r\Incl f(q) \cl
s=t, q\Incl p(s),\GSD, r\Incl f(q) & \rabove{\eRanr}\cl
s=t, q\Incl p(s),\GSD, r\Incl f(p(s)) & \rabove{\iRs} \cl
s=t, q\Incl p(s),\GSD, r\Incl f(p(t)) & \rabove{\eRs} %\wow{This is why I need $\eRs$}
}
\]
\item Other cases are trivial. \\
If $R$ is ($=_{Rar}$) or $\Ear$ then they do not affect the active or side
formula of ($=_{Ranr}$) since, otherwise, the LHS of the latter would have to
be a variable. \\
%% $\eRs$, 
$\eLs$, $\Esr$, ($Sp.cut$) and ($cut_x$) can be trivially swapped.
\end{LS}
As we can see, the applications of $\pLa$ 
occurring in the transformed derivations are the same as the original ones. 
Hence $\#(\pLa,D^*)\leq\#(\pLa,D)$ and $\#(\pLar,D^*)\leq
\#(\pLar,D^*)$.
\end{PROOF}
%
%%\subsubsection{Some rules admissible in $NEQ_4$ 
%%    without increasing $\#\pLar+\#\pLanr$}\label{sub:adm}
\begin{LEMMA}\label{le:noLanr}
The following rule is admissible in $NEQ_4$:
\[ \eLanr\ \ \mprule{s=t, p(s)\Incl q,\GSD}{s=t,p(t)\Incl q,\GSD}\ s\not\in\Vars \]
Moreover, if $\der{NEQ_{4}}DS$ and $\der{NEQ_{4}+\eLanr}{D'}S$, 
%$D'$ derives $S$ using this rule while $D$ derives $S$ not using it, 
we can construct $D'$ so that \\ $\#(\pLar,D')\leq\#(\pLar,D)$, and 
$\#(\pLanr,D')\leq\#(\pLanr,D)$.
\end{LEMMA}
\begin{PROOF}
By lemma %~\ref{fa:noeRs} and 
\ref{le:noar}, we assume that $D$ does not contain
any applications of $\iLanr$. %($\Incl_{anr}$). 
By induction on $\<\#(=_{Lanr},D),h(D)\>$. 
If $h(D)=1$, the only relevant possibility is the axiom with an inclusion in the
antecednet, but it has a variable in the LHS, % $x\Incl t,\GSD,x\Incl t$,
so an application of ($=_{La}$) %to $x\Incl t$ 
has to be reduced.

Consider the last rule above the
uppermost application of ($=_{Lanr}$). 
\begin{LS}
 \item $\iLa$, i.e, $\iLar$. \\
The first case is when both $\iLar$ and $\eLanr$ %and ($=_{Lanr}$) 
modify the same formula:
\[ %D \left \{ 
\begin{array}{rl}
f(x)\Incl p, r=u, x\Incl s(u), \Gamma \Seq\Delta \\ \cline{1-1}
f(s(u))\Incl p, r=u, x\Incl s(u), \Gamma \Seq\Delta  & \rabove{\iLar} \cl
f(s(r))\Incl p, r=u, x\Incl s(u), \Gamma \Seq\Delta  &
\rabove{\eLanr} \end{array} %\right .
 \convd
%We transform it into :
% D' \left \{ 
\begin{array}{rl}
f(x)\Incl p, r=u, x\Incl s(u), x\Incl s(r), \Gamma \Seq\Delta \\ \cline{1-1}
f(s(r))\Incl p, r=u, x\Incl s(u), x\Incl s(r),\Gamma \Seq\Delta  & \rabove{\iLar} \cl
f(s(r))\Incl p, r=u, x\Incl s(u), x\Incl s(u),\Gamma \Seq\Delta  &
\rabove{\eRar} \end{array} %\right .  
\]
Another case is the following:
\[\prarc{
f(g(x))\Incl p, x\Incl t, q=g(t),\GSD \cl
f(g(t))\Incl p, x\Incl t, q=g(t),\GSD & \rabove{\iLar} \cl
f(q)\Incl p, x\Incl t, q=g(t),\GSD & \rabove{\eLanr} 
}\convd
\prarc{
f(g(x))\Incl p, x\Incl t, q=g(t), q=g(x), \GSD \cl
f(q)\Incl p, x\Incl t, q=g(t), q=g(x), \GSD & \rabove{\eLanr} \cl
f(q)\Incl p, x\Incl t, q=g(t), q=g(t), \GSD & \rabove{\iLar} 
} 
\]
The last case is when the formula $\phi$ modified by $\eLanr$ in $D$ was active in the
preceding 
application of $\iLar$. But in this case, since we only have reduced applications
$\iLar$, $\phi$'s LHS must be a variable, so the application of $\eLa$ would not be
$\eLanr$ but $\eLar$.
%
\item $\iLs$: The two applications $\iLs$-$\eLanr$ are easily
replaced by $\eLanr$-$\eLs$.
\item $\iRs$:
\[\sma{ \prar{
f(s)\Incl t, s=r,\GSD,p\Incl q(f(s)) \cl
f(s)\Incl t, s=r,\GSD,p\Incl q(t) & \rabove{\iRs} \cl
f(r)\Incl t, s=r,\GSD,p\Incl q(t) & \rabove{\eLanr} 
}
\conv
\prar{
f(s)\Incl t, s=r,\GSD,p\Incl q(f(s)) \cl
f(r)\Incl t, s=r,\GSD,p\Incl q(f(s)) & \rabove{\eLanr} \cl
f(r)\Incl t, s=r,\GSD,p\Incl q(f(r)) & \rabove{\eRs} \cl
f(r)\Incl t, s=r,\GSD,p\Incl q(t) & \rabove{\iRs} 
} }
\]
%
\item ($=_{Lar}$): 
% \[ D \left \{ \begin{array}{rl}
% q(s(y)) \Incl p, y=u, s(u)=r, \Gamma\Seq\Delta  \\ \cline{1-1}
% q(s(u)) \Incl p, y=u, s(u)=r, \Gamma\Seq\Delta &
% \raisebox{1.2ex}[1.5ex][0ex]{($=_{Lar}$)} \\ \cline{1-1}
% q(r) \Incl p, y=u, s(u)=r, \Gamma\Seq\Delta &
% \raisebox{1.2ex}[1.5ex][0ex]{($=_{Lanr}$)} \end{array} \right . \convd
%  D' \left \{ \begin{array}{rl}
% q(s(y)) \Incl p, y=u, s(u)=r, s(y)=r, \Gamma\Seq\Delta \\ \cline{1-1}
% q(r) \Incl p, y=u, s(u)=r, s(y)=r, \Gamma\Seq\Delta &
% \raisebox{1.2ex}[1.5ex][0ex]{($=_{Lanr}$)} \\ \cline{1-1}
% q(r) \Incl p, y=u, s(u)=r, s(u)=r, \Gamma\Seq\Delta &
% \raisebox{1.2ex}[1.5ex][0ex]{$\eae$} \end{array} \right . \]
% OLD:
\[ D \left \{ \begin{array}{rl}
f(x)\Incl p, r=u, x= s(u), \Gamma \Seq\Delta \\ \cline{1-1}
f(s(u))\Incl p, r=u, x= s(u), \Gamma \Seq\Delta  &
\raisebox{1.2ex}[1.5ex][0ex]{($=_{Lar}$)} \\ \cline{1-1}
f(s(r))\Incl p, r=u, x= s(u), \Gamma \Seq\Delta  &
\raisebox{1.2ex}[1.5ex][0ex]{($=_{Lanr}$)} \end{array} \right . \convd
%We transform it into :
 D' \left \{ \begin{array}{rl}
f(x)\Incl p, r=u, x= s(u), x=s(r),\Gamma \Seq\Delta \\ \cline{1-1}
f(s(r))\Incl p, r=u, x= s(u), x=s(r),\Gamma \Seq\Delta  &
\raisebox{1.2ex}[1.5ex][0ex]{($=_{Lanr}$)} \\ \cline{1-1}
f(s(r))\Incl p, r=u, x= s(u), x=s(u),\Gamma \Seq\Delta  &
\raisebox{1.2ex}[1.5ex][0ex]{$\eae$} \end{array} \right . \]
By induction hypothesis ($=_{Lanr}$) can be eliminated from $D'$, and by 
lemma \ref{le:noeqeq} $D'$ can be transformed so that it
does not contain $\eae$ and this transformation does not
introduce any additional applications of $\pLa$. %($=_{Lanr}$).
The same applies in the other case:
\[
\prar{
f(s(x))\Incl p, x=u, s(u)=r, \GSD \cl
f(s(u))\Incl p, x=u, s(u)=r, \GSD & \rabove{\eLar} \cl
f(r)\Incl p, x=u, s(u)=r, \GSD & \rabove{\eLanr} 
}\convd
\prar{
f(s(x))\Incl p, x=u, s(u)=r, s(x)=r, \GSD \cl
f(r)\Incl p, x=u, s(u)=r, s(x)=r, \GSD & \rabove{\eLanr} \cl
f(r)\Incl p, x=u, s(u)=r, s(u)=r, \GSD & \rabove{\eae} 
} 
\]
%
\item Elimination rules are not relevant,
neither is ($Sp.cut$) and $(cut_x)$, and $\eRa$ and $\eRs$ are trivially swapped
since their active formula (equality) is not affected by $\eLanr$.
\end{LS}
We see that no new applications of
$\pLar$ or $\pLanr$ emerge.
\end{PROOF}
%
\begin{LEMMA}\label{le:inclaad} The following rule is admissible in $NEQ_4$:
\[\mprule{s\Incl t, q\Incl t,\GSD}{s\Incl t, q\Incl s,\GSD} %\ $s,t\not\in\Vars$
\ \ \ \ \iRa
\]
Moreover, if $D^*$ is a derivation using the rule $\iRa$ and $D$ is the corresponding
derivation not using this rule, then $\#(\pLar,D)\leq\#(\pLar,D^*)$, and
$\#(\pLanr,D)\leq\#(\pLanr,D^*)$. 
%  [Also, $D$ does not introduce new applications of $(cut_{x=})$ -- the fact to be used
%in lemma~\ref{le:noxx}.]
\end{LEMMA}
\begin{PROOF}
By induction on $\< \#(\iRa,D^*),h(D^*)\>$. 
By lemma  % \ref{fa:noeRs} and 
\ref{le:noar} % and \ref{le:noLanr}, 
we assume that the derivation does not 
contain any % $\eRs$, 
$\iLanr$ or $\eRanr$. % nor $\eLanr$.
The possible application to an axiom is: % $x\Incl t,\GSD,x\Incl t$
\[
\begin{array}{rl}
x\Incl t, s\Incl t,\GSD,x\Incl t &  \\ \cline{1-1}
x\Incl s, s\Incl t,\GSD,x\Incl t & \rabove{\iRa}
\end{array}
\conv
\begin{array}{rl}
x\Incl s, s\Incl t,\GSD,x\Incl s \\ \cline{1-1}
x\Incl s,s\Incl t, \GSD,x\Incl t & \rabove{\iRs}
\end{array}
\]
Consider the rule applied just above $\iRa$.
\begin{LS}
\item $\iRs$
\[ \prar{
q\Incl t, s\Incl t,\GSD,p\Incl f(q) \\ \cline{1-1}
q\Incl t, s\Incl t,\GSD,p\Incl f(t) & \rabove{\iRs} \\ \cline{1-1}
q\Incl s, s\Incl t,\GSD,p\Incl f(t) & \rabove{\iRa}
}
\conv
\prar{
q\Incl t, s\Incl t,\GSD,p\Incl f(q) \\ \cline{1-1}
q\Incl s, s\Incl t,\GSD,p\Incl f(q) & \rabove{\iRa} \\ \cline{1-1}
q\Incl s, s\Incl t,\GSD,p\Incl f(s) & \rabove{\iRs} \\ \cline{1-1}
q\Incl s, s\Incl t,\GSD,p\Incl f(t) & \rabove{\iRs}
}
\]
\item $\iLar$:
\[ \sma{ \begin{array}{rl}
 s(y) \Incl t, q\Incl t, y\Incl p, \GSD \\ \cline{1-1}
 s(p) \Incl t, q\Incl t, y\Incl p, \GSD & \rabove{\iLar} \\
 \cline{1-1}
 s(p) \Incl t, q\Incl s(p), y\Incl p, \GSD & \rabove{\iRa}
 \end{array} \conv
 \begin{array}{rl}
s(p)\Incl t, s(y)\Incl t, q\Incl t, y\Incl p, \GSD \\ \cline{1-1}
s(p)\Incl t, s(y)\Incl t, q\Incl s(p), y\Incl p, \GSD &
\rabove{\iRa} \\ \cline{1-1}
s(p)\Incl t, s(p)\Incl t, q\Incl s(p), y\Incl p, \GSD &
\rabove{\iLar} \end{array} 
}
\]
%
\item $\iLs$ %($\Incl_s$): $q\not\in\Vars$
% -- if $s\in\Vars$ we get $\Incl_{sx}$
\[ \prar{
s\Incl t, q\Incl t, \GSD, w(t)\preceq p \\ \cline{1-1}
s\Incl t, q\Incl t, \GSD, w(q)\preceq p & \rabove{\iLs} \\
\cline{1-1}
s\Incl t, q\Incl s, \GSD, w(q)\preceq p & \rabove{\iRa} 
}\conv
%We just swap:
\prar{
s\Incl t, q\Incl t, \GSD, w(t)\preceq p \\ \cline{1-1}
s\Incl t, q\Incl s, \GSD, w(t)\preceq p & \rabove{\iRa} \\
\cline{1-1}
s\Incl t, q\Incl s, \GSD, w(s)\preceq p & \rabove{\iLs} \\
\cline{1-1} 
s\Incl t, q\Incl s, \GSD, w(q)\preceq p & \rabove{\iLs} 
} \]
%% The obtained applications of $(\Incl_s)$ are not $(\Incl_{sx})$ -- the first
%% by restriction on $\iRa$ -- $s\not\in\Vars$, and the second since $q\not\in\Vars$.
\item $\Ear$:
\[ \sma{
\begin{array}{rl}
x\Incl t, y\Incl r(x), s\Incl r(t), \GSD \\ \cline{1-1}
          y\Incl r(t), s\Incl r(t), \GSD  & \rabove{\Ear} \\ \cline{1-1}
          y\Incl s, s\Incl r(t), \GSD  & \rabove{\iRa} \end{array}
  \conv
%Again, swapping after weakening, we reduce the height:
 \begin{array}{rl}
x\Incl t, y\Incl r(x), s\Incl r(t), s\Incl r(x),\GSD \\ \cline{1-1}
x\Incl t, y\Incl s, s\Incl r(t), s\Incl r(x),\GSD  & \rabove{\iRa} \\ \cline{1-1}
          y\Incl s, s\Incl r(t), s\Incl r(t),\GSD  & \rabove{\Ear} \end{array} 
}\]
%
We do exactly the same if $\Ear$ modifies the active (rather than the side)
formula of the application of $\iRa$.
\item $\eLar$:
\[ \sma{
\begin{array}{rl}
s(x)\Incl t, q\Incl t, x=p, \GSD \\ \cline{1-1}
s(p)\Incl t, q\Incl t, x=p, \GSD & \rabove{(=_{Lar})} \\ \cline{1-1}
s(p)\Incl t, q\Incl s(p), x=p, \GSD & \rabove{\iRa} \end{array} \conv
%Swap the applications, and clean up:
\begin{array}{rl}
s(p)\Incl t,s(x)\Incl t, q\Incl t, x=p, \GSD \\ \cline{1-1}
s(p)\Incl t,s(x)\Incl t, q\Incl s(p), x=p, \GSD & \rabove{\iRa} \\ \cline{1-1}
s(p)\Incl t,s(p)\Incl t, q\Incl s(p), x=p, \GSD & \rabove{(=_{Lar})} \end{array}
}
\]
\item $\eRar$:
\[ \sma{ \begin{array}{rl}
s\Incl t(p), x\Incl t(r), r=p, \GSD \\ \cline{1-1}
s\Incl t(p), x\Incl t(p), r=p, \GSD & \rabove{(=_{Rar})}\\ \cline{1-1}
s\Incl t(p), x\Incl s, r=p, \GSD & \rabove{\iRa} \end{array} \conv
%Weakening allows us to swap:
 \begin{array}{rl}
s\Incl t(r), s\Incl t(p), x\Incl t(r), r=p, \GSD \\ \cline{1-1}
s\Incl t(r), s\Incl t(p), x\Incl s, r=p, \GSD & \rabove{\iRa} \\ \cline{1-1}
s\Incl t(p), s\Incl t(p), x\Incl s, r=p, \GSD & \rabove{\eRa} \end{array} 
} \]
Analogous transformation is applied if $\iRa$ in the original derivation yields
$s\Incl x, x\Incl t(p)\ldots$.
%
\item The rules $\eLs$, $\eRs$, $\Esr$, $(Sp.cut)$, $(cut_x)$ and ($cut_{x=}$) 
can be trivially swapped, since they do
not affect the antecedent -- the active formula, which is either equality or
disappears, cannot be involved in the final application of $\iRa$.
\end{LS}
We see that no new applications of $\pLa$ appear in the transformed derivations 
$D$, and the ones which occur there are the same as in the original 
derivations $D^*$.
[Also, no new applications of $(cut_{x=})$ appear in the transformed derivations.]
\end{PROOF}
%

\subsubsection{Some redundant s-rules}\label{sub:nosx}
First lemma shows the possibility of eliminating most of the s-rules from the $NEQ_4$
derivations.
%% the following degenerate 
%% applications can be avoided in $NEQ_4$ derivations:
%% \[\sma{
%% \eRsx\ \ \mprule{s=x, \GSD, t\Incl f(s)}{s=x, \GSD, t\Incl f(x)} 
%% \hspace*{2em}
%% \iLsx\ \ \mprule{s\Incl x, \GSD, f(s)\preceq t}{s\Incl x, \GSD,f(x)\preceq t} 
%% \hspace*{3em}
%% \eLsx\ \ \mprule{s= x, \GSD, f(s)\preceq t}{s= x, \GSD, f(x)\preceq t}  
%% }
%% \]
\begin{LEMMA}\label{le:nosx}
$\der{NEQ_4}DS \impl \der{NEQ_4}{D^*}S$ and $D^*$ uses no applications
of $\eRs$, $\iLs$ or $\eLs$.
%% Applications $\eRsx$, $\iLsx$, $\eLsx$ can be eliminated.
%% [Morover, the resulting derivations do not introduce any new applications of
%% s-rules.]
\end{LEMMA}
\begin{PROOF}
We proceed by simultaneous induction on 
the number of applications of all the listed rules and by subinduction on the
height of the derivation. 
By lemma~\ref{fa:noeRs}, we may assume that derivation contains 
no applications of $\eRs$.
%, in particular, no applications of $\eRsx$.

Writing $\pLs$, we mean either $\iLs$ or $\eLs$ and use this notation when the
two cases are treated analogously. 
\begin{LS}
\item Each form of the axioms gives rise to several cases:
\begin{LSA}
\item $x\Incl t,\GSD, x\Incl t$
\begin{LSB}
\item If $s\not\in\Vars$, we choose a fresh variable $y:$
\[
\prar{
s\preceq x, x\Incl t, \GSD, x\Incl t \cl
s\preceq x, x\Incl t, \GSD, s\Incl t & \rabove{\pLs}
} \conv
\prar{
y\Incl s, s\preceq x, y\Incl t, \GSD, y\Incl t \cl
y\Incl s, s\preceq x, s\Incl t, \GSD, y\Incl t & \rabove{\iLar} \cl
     s\preceq x, s\Incl t, \GSD, s\Incl t & \rabove{\Esr}
}
\]
\item 
If $s$ is a variable, say $y$:
\[
\prarc{
y\preceq x, x\Incl t, \GSD, x\Incl t \cl
y\preceq x, x\Incl t, \GSD, y\Incl t & \rabove{\pLs}
} \conv
\prarc{
y\preceq x, y\Incl t, \GSD, y\Incl t \cl
y\preceq x, x\Incl t, \GSD, y\Incl t & \rabove{\pLar}
}
\]
\end{LSB}
\item $s=t,\GSD,s=t$ gives rise to four cases
\begin{LSB}
\item The substituted term is a variable $x$
\[
\prarc{
x\Incl p, s(p)= t, \GSD, s(p)= t \cl
x\Incl p, s(p)= t, \GSD, s(x)= t & \rabove{\iLs}
} \conv
\prarc{
x\Incl p, s(x)= t, \GSD, s(x)= t \cl
x\Incl p, s(p)= t, \GSD, s(x)= t & \rabove{\iLar}
}
\]
\item and variable substituted by $\eLs$:
\[\sma{
\prar{
x= p, s(p)= t, \GSD, s(p)= t \cl
x= p, s(p)= t, \GSD, s(x)= t & \rabove{\eLs}
} \conv
\prar{
x\Incl p, x= p, s(x)= t, \GSD, s(x)= t \cl
x\Incl p, x= p, s(p)= t, \GSD, s(x)= t & \rabove{\iLar} \cl
x\Incl x, x= p, s(p)= t, \GSD, s(x)= t & \rabove{\eRar} \cl
 x= p, s(p)= t, \GSD, s(x)= t & \rabove{(cut_x)} 
} }
\]
\item The substituted term $r\not\in\Vars$ -- we choose a fresh variable $y:$
\[\sma{ \prar{
r=p, s(p)=t, \GSD, s(p)=t \cl
r=p, s(p)=t, \GSD, s(r)=t & \rabove{\eLs}
}\conv
\prar{
y\Incl p, r=p, s(y)=t, \GSD, s(y)=t \cl
y\Incl p, r=p, s(p)=t, \GSD, s(y)=t & \rabove{\iLar} \cl
y\Incl r, r=p, s(p)=t, \GSD, s(y)=t & \rabove{\eRar} \cl
      r=p, s(p)=t, \GSD, s(r)=t & \rabove{\Esr} 
} }
\]
\item and non-variable substituted by $\iLs$
\[\prarc{
r\Incl p, s(p)=t, \GSD, s(p)=t \cl
r\Incl p, s(p)=t, \GSD, s(r)=t & \rabove{\iLs}
}\conv
\prar{
r\Incl p, s(r)=t, \GSD, s(r)=t \cl
r\Incl p, s(p)=t, \GSD, s(r)=t & \rabove{\iLa}
}
\]
\end{LSB}
\item $\GSD, x=x$.
The situation \[\prar{s=x,\GSD,x=x \cl s=x,\GSD,s=x& \rabove{\eLs}}\] gives an 
instance of an axiom. The case of $\iLs$ is treated as follows:
\[\prar{
t\Incl x, \GSD, x=x \cl 
t\Incl x,\GSD, t=x & \rabove{\iLs}
}\conv
\prar{
t=x, t\Incl x,\GSD, t=x \cl
x=x, t\Incl x,\GSD, t=x & \rabove{\iLa} \cl
t\Incl x, \GSD, t=x & \rabove{(cut_{x=})}
}
\]
\end{LSA}
\item $\iLa$
\[\sma{ \prarc{
q(t)\preceq r, t\Incl s, \GSD, p(r)\preceq t \cl
q(s)\preceq r, t\Incl s, \GSD, p(r)\preceq t & \rabove{\iLa} \cl
q(s)\preceq r, t\Incl s, \GSD, p(q(s))\preceq t & \rabove{\pLs} 
}\conv
\prarc{
q(s)\preceq r, q(t)\preceq r, t\Incl s, \GSD, p(r)\preceq t \cl
q(s)\preceq r, q(t)\preceq r, t\Incl s, \GSD, p(q(s))\preceq t & \rabove{\pLs} \cl
q(s)\preceq r, q(s)\preceq r, t\Incl s, \GSD, p(q(s))\preceq t & \rabove{\iLa} 
} }
\]
%% \item $\eLs$ % $s\not\in\Vars$
%% We consider the cases of $\iLsx$ and $\eLsx$ separately
%% \begin{LSA}
%% \item $\iLsx$
%% \[\prarc{
%% x\Incl s, s=t,\GSD, p(t)\preceq q \cl
%% x\Incl s, s=t,\GSD, p(s)\preceq q & \rabove{\eLs} \cl
%% x\Incl s, s=t,\GSD, p(x)\preceq q & \rabove{\iLsx} 
%% }\conv
%% \prarc{
%% x\Incl t, x\Incl s, s=t,\GSD, p(t)\preceq q \cl
%% x\Incl t, x\Incl s, s=t,\GSD, p(x)\preceq q & \rabove{\iLsx} \cl
%% x\Incl s, x\Incl s, s=t,\GSD, p(x)\preceq q & \rabove{\eRar} 
%% }
%% \]
%% Another case is as follows:
%% \[\sma{ \prarc{
%% x\Incl s, t=r(s), \GSD, p(t)\preceq q \cl
%% x\Incl s, t=r(s), \GSD, p(r(s))\preceq q & \rabove{\eLs} \cl
%% x\Incl s, t=r(s), \GSD, p(r(x))\preceq q & \rabove{\iLsx} 
%% }\conv
%% \prarc{
%% x\Incl s, t=r(s), t=r(x), \GSD, p(t)\preceq q \cl
%% x\Incl s, t=r(s), t=r(x), \GSD, p(r(x))\preceq q & \rabove{\eLs} \cl
%% x\Incl s, t=r(s), t=r(s), \GSD, p(r(x))\preceq q & \rabove{\iLar} 
%% } }
%% \]
%% \item 
%% The two cases of $\eLsx$
%% \[\prar{
%% x= s, s=t,\GSD, p(t)\preceq q \cl
%% x= s, s=t,\GSD, p(s)\preceq q & \rabove{\eLs} \cl
%% x= s, s=t,\GSD, p(x)\preceq q & \rabove{\eLsx} 
%% }\conv
%% \prar{
%% x\Incl t, x= s, s=t,\GSD, p(t)\preceq q \cl
%% x\Incl t, x= s, s=t,\GSD, p(x)\preceq q & \rabove{\iLsx} \cl
%% x\Incl s, x= s, s=t,\GSD, p(x)\preceq q & \rabove{\eRa} \cl
%% x\Incl x, x= s, s=t,\GSD, p(x)\preceq q & \rabove{\eRa} \cl
%%           x= s, s=t,\GSD, p(x)\preceq q & \rabove{(cut_x)} 
%% }
%% \]
%% \[\sma{ \prar{
%% x=s, r(s)=t,\GSD, p(t)\preceq q \cl
%% x=s, r(s)=t,\GSD, p(r(s))\preceq q & \rabove{\eLs} \cl
%% x=s, r(s)=t,\GSD, p(r(x))\preceq q & \rabove{\eLsx}
%% } } \convd \sma{
%% \prar{
%% x\Incl s, r(x)=t, x=s, r(s)=t,\GSD, p(t)\preceq q \cl
%% x\Incl s, r(x)=t, x=s, r(s)=t,\GSD, p(r(x))\preceq q & \rabove{\eLs} \cl
%% x\Incl s, r(s)=t, x=s, r(s)=t,\GSD, p(r(x))\preceq q & \rabove{\iLar} \cl
%% x\Incl x, r(s)=t, x=s, r(s)=t,\GSD, p(r(x))\preceq q & \rabove{\eRa} \cl
%%       r(s)=t, x=s, r(s)=t,\GSD, p(r(x))\preceq q & \rabove{(cut_x)}
%% } }
%% \]
%% \end{LSA}
\item $\Ear$
\[\sma{\prarc{
y\Incl t, x\Incl q(y),\GSD, f(q(t))\preceq p \cl
          x\Incl q(t),\GSD, f(q(t))\preceq p & \rabove{\Ear} \cl
          x\Incl q(t),\GSD, f(x)\preceq p & \rabove{\iLs}
}\conv
\prarc{
x\Incl q(t), y\Incl t, x\Incl q(y),\GSD, f(q(t))\preceq p \cl
x\Incl q(t), y\Incl t, x\Incl q(y),\GSD, f(x)\preceq p & \rabove{\iLs} \cl
x\Incl q(t), x\Incl q(t),\GSD, f(x)\preceq p & \rabove{\Ear}
} }
\]
\item $\eLar$
\[\sma{ \prarc{
s(y)\Incl q, t=y,\GSD, f(q)\preceq t \cl
s(t)\Incl q, t=y,\GSD, f(q)\preceq t & \rabove{\eLar} \cl
s(t)\Incl q, t=y,\GSD, f(s(t))\preceq t & \rabove{\iLs} 
}\conv
\prarc{
s(t)\Incl q, s(y)\Incl q, t=y,\GSD, f(q)\preceq t \cl
s(t)\Incl q, s(y)\Incl q, t=y,\GSD, f(s(t))\preceq t & \rabove{\iLs} \cl
s(t)\Incl q, s(t)\Incl q, t=y,\GSD, f(s(t))\preceq t & \rabove{\eLar} 
} }
\]
\item $\eRa$
\[\sma{\prarc{
x\Incl p(t), t=s,\GSD, f(p(s))\preceq q \cl
x\Incl p(s), t=s,\GSD, f(p(s))\preceq q & \rabove{\eRa} \cl
x\Incl p(s), t=s,\GSD, f(x)\preceq q & \rabove{\iLs}
}\conv
\prarc{
x\Incl p(s), x\Incl p(t), t=s,\GSD, f(p(s))\preceq q \cl
x\Incl p(s), x\Incl p(t), t=s,\GSD, f(x)\preceq q & \rabove{\iLs} \cl
x\Incl p(s), x\Incl p(s), t=s,\GSD, f(x)\preceq q & \rabove{\eRa} 
} }
\]
\item $\iRs$, $\eRs$, $\Esr$, $(cut_x)$, $(cut_{x=})$ and $(Sp.cut)$ can be 
trivially swapped.
\end{LS}
As we see, no new applications of s-rules appear in the resulting derivations.
\end{PROOF}
%

\noindent
In lemma~\ref{fa:noeRs} we have shown that rule $\eRs$ can be eliminated from the
derivations. % without introducing new applications of s-rules. 
However, new applications of a-rules appeared in the transformed derivations. 
Moreover, in lemma~\ref{le:noar}, eliminating non-reduced
applications of $\eRanr$ we used an application of $\eRs$ in the transformed derivation
(case 4). The final lemma shows now that, as a matter of fact, we can eliminate $\eRs$, 
as well as a non-reduced version of $\iRsnr$, from derivations without increasing the
number of $\pLanr$ and $\eRanr$. 
\begin{LEMMA}\label{le:noiRsnr}
$\der{NEQ_{4}}DS\ \impl\ \der{NEQ_{4}}{D^{*}}S$ and $D^*$ contains no 
applications of $\eRs$ nor
\[
\iRsnr\ \ \ \mprule{s\Incl t,\GSD, p\Incl q(s)}{s\Incl t,\GSD,p\Incl q(t)}\ \ 
s\not\in\Vars
\]
Moreover $\#(\pLanr,D^*) \leq \#(\pLanr,D)$ and $\#(\eRanr,D^*)\leq\#(\eRanr,D)$.
\end{LEMMA}
\begin{PROOF}
By lemma \ref{le:noar} 
we assume that no non-reduced applications $\iLanr$ or $\eRanr$
appear in $D$. Proceeding by simultaneous induction on the number of applications
of $\eRs$ and $\iRsnr$ and subinduction on the height of $D$, we consider first the
case when the uppermost application of one of these two rules is $\eRs$. (This part
of the proof is essentially revisting the proof of lemma~\ref{fa:noeRs} and checking
that the current side-conditions are satisfied.)

For the axioms, we only have to consider the one with inclusion in the consequent.
%\begin{itemize}
%\item $x\Incl t,\GSD,x\Incl t$
%\end{itemize}
\[
\begin{array}{rl}
p=s, x\Incl t(s),\GSD,x\Incl t(s) \\ \cline{1-1}
p=s, x\Incl t(s),\GSD,x\Incl t(p) & \rabove{\eRs}
\end{array}
\conv
\begin{array}{rl}
p=s, x\Incl t(p),\GSD,x\Incl t(p) \\ \cline{1-1}
p=s, x\Incl t(s),\GSD,x\Incl t(p) & \rabove{\eRar}
\end{array}
\]
The application is reduced  $\eRar$.
Consider the uppermost application of ($=_{Rs}$) in $D$:
\begin{LS}
\item\label{it:RsEsB} $R=\Esr$ with $q\Incl t(r)$ as the modified formula and $r$ as
modified term. We have two subcases corresponding to the situation when the
term $r'$ substituted by $\Esr$ for its eigen-variable is a subterm or
superterm of (or equal to) $r$. (When the two are independent, the case is
trivial.) 
\begin{LSA}
\item  $r'$ is a subterm of $r$,\\
i.e., $t'(r')$ and $t(r)$ are identical with $r=f(r')$ and $t'(x)=t(f(x))$.
\[ D \left \{\begin{array}{cl}
%\vdots          \\
x\Incl r', f(r')=s, \GSD, q\Incl t(f(x)) & \raisebox{-1.2ex}[1.5ex][0ex]{$\Esr$} \\ \cline{1-1}
f(r')=s, \GSD, q\Incl t(f(r')) &
\raisebox{-1.2ex}[1.5ex][0ex]{($=_{Rs}$)} \\ \cline{1-1}
f(r')=s, \GSD, q\Incl t(s) 
\end{array} \right . \convd
D' \left \{ \begin{array}{cl}
x\Incl r', f(r')=s, f(x)=s, \Gamma\Seq\Delta,q\Incl t(f(x)) & \raisebox{-1.2ex}[1.5ex][0ex]{($=_{Rs}$)} \\ \cline{1-1}
x\Incl r', f(r')=s, f(x)=s, \Gamma\Seq\Delta, q\Incl t(s) &
\raisebox{-1.2ex}[1.5ex][0ex]{$\iLar$} \\ \cline{1-1}
x\Incl r', f(r')=s, f(r')=s, \Gamma\Seq\Delta, q\Incl t(s) &
\raisebox{-1.2ex}[1.5ex][0ex]{$\Esr$} \\ \cline{1-1}
f(r')=s, \Gamma\Seq\Delta, q\Incl t(s) 
\end{array} \right . \]
The obtained application of $\iLar$ is reduced.
\item $r'$ is a superterm of (or equal to) $r$,\\
i.e., $r'=f(r)$ and $t(x)=t'(f(x))$.
\[ D \left \{\begin{array}{cl}
%\vdots          \\
x\Incl f(r), r=s, \Gamma\Seq\Delta, q\Incl t'(x) & \\ \cline{1-1}
r=s, \Gamma\Seq\Delta, q\Incl t'(f(r)) & \rabove{\Esr} \\ \cline{1-1}
r=s, \Gamma\Seq\Delta, q\Incl t'(f(s)) & \rabove{\eRs} 
\end{array} \right . \convd
D \left \{\begin{array}{cl}
x\Incl f(r), r=s, \Gamma\Seq\Delta, q\Incl t'(x) & \\ \cline{1-1}
x\Incl f(s), r=s, \Gamma\Seq\Delta, q\Incl t'(x) & \rabove{\eRar}
\\ \cline{1-1}
r=s, \Gamma\Seq\Delta, q\Incl t'(f(s)) & \rabove{\Esr} 
\end{array} \right . \]
Again, the resulting application of $\eRar$ is reduced.
The case when the modified term is substituted into LHS of $\Incl$ is treated
analogously.
\end{LSA}
\item $R=\iLar$ with the modified formula $r=s(v)$ and $v$ as the modified
term:
\[ D \left \{ \begin{array}{cl}
 D_1 \left \{ \begin{array}{c}
               \vdots       \\ 
               r=s(x), x\Incl v, \Gamma\Seq\Delta,q\Incl t(r) 
           \end{array} \right . 
         & \raisebox{-3.2ex}[1.5ex][0ex]{$\iLa$}  \\ \cline{1-1}
 r=s(v), x\Incl v, \Gamma\Seq\Delta, q\Incl t(r) &
 \raisebox{-1.2ex}[1.5ex][0ex]{($=_{Rs}$)} \\ \cline{1-1}
 r=s(v), x\Incl v, \Gamma\Seq\Delta, q\Incl t(s(v)) 
 \end{array} \right . \convd
  D^* \left \{ \begin{array}{cl}
  D' \left \{ \begin{array}{cl}
    D_1 \left \{ \begin{array}{cl}
 \vdots       \\ 
 r=s(x), x\Incl v, \Gamma\Seq\Delta,q\Incl t(r) 
  \end{array} \right . & \raisebox{-3.2ex}[1.5ex][0ex]{($W_a$)}  \\
  \cline{1-1}
 r=s(x), r=s(v), x\Incl v, \Gamma \Seq \Delta, q\Incl t(r)
  & \raisebox{-1.2ex}[1.5ex][0ex]{($=_{Rs}$)} \\ \cline{1-1}
 r=s(x), r=s(v), x\Incl v, \Gamma\Seq\Delta, q\Incl t(s(v))  \end{array}
 \right . &  \\ \cline{1-1}
 r=s(v), r=s(v), x\Incl v, \Gamma\Seq\Delta, q\Incl t(s(v)) 
 & \raisebox{1.2ex}[1.5ex][0ex]{$\iLar$} 
 \end{array} \right . \]
The resulting application of $\iLa$ is the same as the original one -- hence reduced.
The same transformation is applied if $\eRs$ substitutes $r$ for $s(v)$.
 \item $R=\iLs$ :
 \[ D \left \{ \begin{array}{cl}
  D_1 \left \{ \begin{array}{c}
               \vdots       \\ 
               r=s, q\Incl u, \Gamma\Seq\Delta, u\Incl t(r) 
            \end{array} \right . 
          & \raisebox{-3.2ex}[1.5ex][0ex]{$\iLs$}  \\ \cline{1-1}
 r=s, q\Incl u, \Gamma\Seq\Delta, q\Incl t(r) &
 \raisebox{-1.2ex}[1.5ex][0ex]{($=_{Rs}$)} \\ \cline{1-1}
 r=s, q\Incl u, \Gamma\Seq\Delta, q\Incl t(s) 
 \end{array} \right . \convd
  D^* \left \{ \begin{array}{cl}
  D' \left \{ \begin{array}{cl}
    D_1 \left \{ \begin{array}{cl}
 \vdots       \\ 
 r=s, q\Incl u, \Gamma\Seq\Delta, u\Incl t(r) 
  \end{array} \right . & \raisebox{-3.2ex}[1.5ex][0ex]{($=_{Rs}$)}  \\
  \cline{1-1}
 r=s, r=s(v), q\Incl u, \Gamma \Seq \Delta, u\Incl t(s) \end{array}
 \right . &  \\ \cline{1-1}
 r=s, q\Incl u, \Gamma\Seq\Delta, q\Incl t(s) 
 & \raisebox{1.2ex}[1.5ex][0ex]{$\iLs$} 
 \end{array} \right . \]
% Since $h(D')<h(D)$, the induction hypothesis yields a desired derivation.
\item $R=\iRsr$
\[\sma{\prarc{
f(p)=s, x\Incl p,\GSD, t\Incl q(f(x)) \cl
f(p)=s, x\Incl p,\GSD, t\Incl q(f(p)) & \rabove{\iRsr} \cl
f(p)=s, x\Incl p,\GSD, t\Incl q(s) & \rabove{\eRs}
}
\conv
\prarc{
f(p)=s,f(x)=s, x\Incl p,\GSD, t\Incl q(f(x)) \cl
f(p)=s,f(x)=s, x\Incl p,\GSD, t\Incl q(s)  & \rabove{\eRs} \cl
f(p)=s,f(p)=s, x\Incl p,\GSD, t\Incl q(s)  & \rabove{\iLar} 
} }
\]
\[\sma{ \prarc{
x\Incl f(s), s=r, \GSD, t\Incl q(x) \cl
x\Incl f(s), s=r, \GSD, t\Incl q(f(s)) &\rabove{\iRsr} \cl
x\Incl f(s), s=r, \GSD, t\Incl q(f(r)) &\rabove{\eRs} 
}
\conv
\prarc{
x\Incl f(r), x\Incl f(s), s=r, \GSD, t\Incl q(x) \cl
x\Incl f(r), x\Incl f(s), s=r, \GSD, t\Incl q(f(r)) &\rabove{\iRs} \cl
x\Incl f(s), x\Incl f(s), s=r, \GSD, t\Incl q(f(r)) &\rabove{\eRar}
} }
\]
In both cases, since the original application of $\iRsr$ is reduced, so are
the resulting applications of $\iLar$ and $\eRar$.
 \item The other cases are easily swapped as in lemma ~\ref{fa:noeRs}.
\end{LS}
%
So the case when the uppermost application is $\iRsnr$.
%\ref{fa:noeRs} we assume that $D$ contains no 
%$\eRs$.
For the axiom we have the following case ($y$ is a fresh variable):
\[
\prarc{
s\Incl t, x\Incl q(s),\GSD, x\Incl q(s) \cl
s\Incl t, x\Incl q(s),\GSD, x\Incl q(t) & \rabove{\iRsnr}
} \conv
\prarc{
y\Incl s, y\Incl t, x\Incl q(y), \GSD, x\Incl q(y) \cl
y\Incl s, y\Incl t, x\Incl q(y), \GSD, x\Incl q(t) & \rabove{\iRsr} \cl
y\Incl s, s\Incl t, x\Incl q(y), \GSD, x\Incl q(t) & \rabove{\iLar} \cl
          s\Incl t, x\Incl q(y), \GSD, x\Incl q(t) & \rabove{\Ear}
}
\]
Consider the rule applied just above $\iRsnr$.
\begin{LS}
\item $\iRsr$ -- we choose a fresh variable $y$
\[\prarc{
s\Incl t, x\Incl f(s),\GSD, p\Incl q(x) \cl
s\Incl t, x\Incl f(s),\GSD, p\Incl q(f(s)) & \rabove{\iRsr} \cl
s\Incl t, x\Incl f(s),\GSD, p\Incl q(f(t)) & \rabove{\iRsnr} 
} \convd
\prarc{
y\Incl s, y\Incl t, x\Incl f(y), s\Incl t, x\Incl f(s),\GSD, p\Incl q(x) \cl
y\Incl s, y\Incl t, x\Incl f(y), s\Incl t, x\Incl f(s),\GSD, p\Incl 
q(f(y)) & \rabove{\iRsr} \cl
y\Incl s, y\Incl t, x\Incl f(y), s\Incl t, x\Incl f(s),\GSD, p\Incl 
q(f(t)) & \rabove{\iRsr} \cl
y\Incl s, s\Incl t, x\Incl f(y), s\Incl t, x\Incl f(s),\GSD, p\Incl 
q(f(t)) & \rabove{\iLar} \cl
      s\Incl t, x\Incl f(s), s\Incl t, x\Incl f(s),\GSD, p\Incl 
q(f(t)) & \rabove{\Ear}
}
\]
Another case is as follows:
\[\sma{ \prarc{
f(s)\Incl f(t), x\Incl s,\GSD, p\Incl q(f(x)) \cl
f(s)\Incl f(t), x\Incl s,\GSD, p\Incl q(f(s)) & \rabove{\iRsr} \cl
f(s)\Incl f(t), x\Incl s,\GSD, p\Incl q(f(t)) & \rabove{\iRsnr} 
} }\convd
\sma{ \prarc{
f(x)\Incl f(t), f(s)\Incl f(t), x\Incl s,\GSD, p\Incl q(f(x)) \cl
f(x)\Incl f(t), f(s)\Incl f(t), x\Incl s,\GSD, p\Incl q(f(t)) & \rabove{\iRsnr} \cl
f(s)\Incl f(t), f(s)\Incl f(t), x\Incl s,\GSD, p\Incl q(f(t)) & \rabove{\iLar} 
} }
\]
The new application of $\iLar$ is reduced.
\item $\iLar$ or $\eLar$ are treated analogously
\[\sma{ \prarc{
p(x)\Incl q, x\preceq t,\GSD, r\Incl f(p(t)) \cl
p(t)\Incl q, x\preceq t,\GSD, r\Incl f(p(t)) &\rabove{\pLar} \cl
p(t)\Incl q, x\preceq t,\GSD, r\Incl f(q) &\rabove{\iRsnr} 
} \conv
\prarc{
p(t)\Incl q, p(x)\Incl q, x\preceq t,\GSD, r\Incl f(p(t)) \cl
p(t)\Incl q, p(x)\Incl q, x\preceq t,\GSD, r\Incl f(q) & \rabove{\iRsnr} \cl
p(t)\Incl q, p(t)\Incl q, x\preceq t,\GSD, r\Incl f(q) & \rabove{\pLar}
} }
\]
\item $\eRar$ -- since this requires a variable in the LHS of the modified 
formula, the subsequent application will be reduced $\iRsr$:
\[\prar{
x\Incl t(p), p=r, \GSD, w\Incl q(x) \cl
x\Incl t(r), p=r, \GSD, w\Incl q(x) & \rabove{\eRa} \cl
x\Incl t(r), p=r, \GSD, w\Incl q(t(r)) & \rabove{\iRsr} 
}
%\conv
%\prar{
%s\Incl t(r), s\Incl t(p), p=r, \GSD, w\Incl q(s) \cl
%s\Incl t(r), s\Incl t(p), p=r, \GSD, w\Incl q(t(r)) & \rabove{\iRsnr} \cl
%s\Incl t(r), s\Incl t(r), p=r, \GSD, w\Incl q(t(r)) & \rabove{\eRa}
%} 
\]
\item $\Esr$
\[\sma{ \prar{
x\Incl t, q(t)\Incl p,\GSD, r\Incl w(q(x)) \cl
x\Incl t, q(t)\Incl p,\GSD, r\Incl w(q(t))  & \rabove{\Esr} \cl
x\Incl t, q(t)\Incl p,\GSD, r\Incl w(p)  & \rabove{\iRsnr}
} \conv
\prar{
x\Incl t, q(x)\Incl p, q(t)\Incl p,\GSD, r\Incl w(q(x)) \cl
x\Incl t, q(x)\Incl p, q(t)\Incl p,\GSD, r\Incl w(p))  & \rabove{\iRsnr} \cl
x\Incl t, q(t)\Incl p, q(t)\Incl p,\GSD, r\Incl w(p))  & \rabove{\iLar} \cl
x\Incl t, q(t)\Incl p, q(t)\Incl p,\GSD, r\Incl w(p))  & \rabove{\Esr} 
} }
\]
\item The rules $\eLs$ and $\iLs$ can be trivially swapped since they affect only
the LHS of the inclusion in the consequent. The rule $\Ear$ requires
a variable in the LHS of the modified formula, so if this formula is active in a 
subsequent application of $\iRs$ this must be $\iRsr$. By the assumption, we do not
have applications of $\eRs$. $(cut_x)$ and $(Sp.cut)$ can be trivially swapped.
\end{LS}
The new applications of $\pLa$ are all reduced and in case 2) are of the same kind
as in the original derivation. Also, no new applications of $\eRa$ appear. 
Hence $\#\pLanr$ and $\#\eRanr$ are not increased.
\end{PROOF}

%%\begin{LEMMA}\label{le:noeRsnr}
%%$\der{NEQ_{4}}DS\ \impl\ \der{NEQ_{4}}{D^{*}}S$ and $D^*$ contains no 
%%applications of 
%%\[
%%\eRsnr\ \ \ \mprule{s=t,\GSD, p\Incl q(s)}{s=t,\GSD,p\Incl q(t)}\ \ 
%%s\not\in\Vars
%%\]
%%Moreover $\#(R,D^*) \leq \#(R,D)$ for $R$ any rule among
%%$\pLanr$, $\eRanr$ and $\iRsnr$.
%%\end{LEMMA}
%%\begin{PROOF}
%%By lemmas \ref{le:noar} and \ref{le:noiRsnr}, we assume that $D$ contains only 
%%reduced applications of  $\iLar$, $\eRar$ and $\iRsr$.
%%For an application to the axiom we have the following transformation:
%%\[\prarc{
%%s=t, x\Incl p(s),\GSD, x\Incl p(s) \cl
%%s=t, x\Incl p(s),\GSD, x\Incl p(t) & \rabove{\eRsnr}
%%} \conv
%%\prarc{
%%s=t, x\Incl p(t),\GSD, x\Incl p(t) \cl
%%s=t, x\Incl p(s),\GSD, x\Incl p(t) & \rabove{\eRar}
%%}
%%\]
%%Consider the last rule applied above the uppermost $\eRsnr$
%%\begin{LS}
%%\item $\iRsr$
%%\[\sma{ \prarc{
%%s=t, x\Incl p(s),\GSD, q\Incl f(x) \cl
%%s=t, x\Incl p(s),\GSD, q\Incl f(p(s)) & \rabove{\iRsr} \cl
%%s=t, x\Incl p(s),\GSD, q\Incl f(p(t)) & \rabove{\eRsnr} 
%%} \mconv
%%\prarc{
%%x\Incl p(t), s=t, x\Incl p(s),\GSD, q\Incl f(x) \cl
%%x\Incl p(t), s=t, x\Incl p(s),\GSD, q\Incl f(p(t)) &\rabove{\iRsr} \cl
%%x\Incl p(s), s=t, x\Incl p(s),\GSD, q\Incl f(p(t)) &\rabove{\eRar} 
%%} }
%%\]
%%\item $\iLar$
%%\[\sma{ \prarc{
%%s(x)=t, x\Incl r,\GSD, q\Incl f(s(r)) \cl
%%s(r)=t, x\Incl r,\GSD, q\Incl f(s(r)) & \rabove{\iLar} \cl
%%s(r)=t, x\Incl r,\GSD, q\Incl f(t) & \rabove{\eRsnr} 
%%}\mconv
%%\prarc{
%%s(r)=t, s(x)=t, x\Incl r,\GSD, q\Incl f(s(r)) \cl
%%s(r)=t, s(x)=t, x\Incl r,\GSD, q\Incl f(t) &\rabove{\eRsnr} \cl
%%s(r)=t, s(r)=t, x\Incl r,\GSD, q\Incl f(t) &\rabove{\iLar}
%%} }
%%\]
%%\item $\eRsr$
%%\[\sma{ \prarc{
%%x=q(s), t=s, \GSD, w\Incl f(x) \cl
%%x=q(s), t=s, \GSD, w\Incl f(q(s)) & \rabove{\eRsr} \cl
%%x=q(s), t=s, \GSD, w\Incl f(q(t)) & \rabove{\eRsnr} 
%%} \mconv
%%\prarc{
%%x\Incl q(t), x=q(s), t=s, \GSD, w\Incl f(x) \cl
%%x\Incl q(t), x=q(s), t=s, \GSD, w\Incl f(q(t)) & \rabove{\iRsr} \cl
%%x\Incl q(s), x=q(s), t=s, \GSD, w\Incl f(q(t)) & \rabove{\eRar} \cl
%%x\Incl x, x=q(s), t=s, \GSD, w\Incl f(q(t)) & \rabove{\eRar} \cl
%%x=q(s), t=s, \GSD, w\Incl f(q(t)) & \rabove{(cut_x)} 
%%} }
%%\]
%%\item $\Esr$
%%\[\sma{ \prar{
%%x\Incl q(s), s=t, \GSD, w\Incl f(x) \cl
%% s=t, \GSD, w\Incl f(q(s)) & \rabove{\Esr} \cl
%% s=t, \GSD, w\Incl f(q(t)) & \rabove{\eRsnr}
%%} \mconv
%%\prar{
%%x\Incl q(t), x\Incl q(s), s=t, \GSD, w\Incl f(x) \cl
%%x\Incl q(t), x\Incl q(s), s=t, \GSD, w\Incl f(q(t)) & \rabove{\iRsr} \cl
%%x\Incl q(s), x\Incl q(s), s=t, \GSD, w\Incl f(q(t)) & \rabove{\eRar} \cl
%%             x\Incl q(s), s=t, \GSD, w\Incl f(q(t)) \cl
%%              s=t, \GSD, w\Incl f(q(t)) & \rabove{\Esr} 
%%} }
%%\]
%%\item Other rules can be easily swapped: 
%%$\iLs$ and $\eLs$ modify only LHS of $\Incl$; $\Ear$, $\eRar$ and $\eLar$ modify an 
%%inclusion, i.e., not the active formula of a subsequent application of $\eRsnr$;
%%$(cut_x)$ and $(Sp.cut)$ are obvious.
%%\end{LS}
%%As we see, no new non-reduced applications appear in the transformed derivations, 
%%so the side condition of the lemma is satisfied.
%%\end{PROOF}
\begin{REMARK}\label{re:crucial}
%The side-conditions of lemmas \ref{fa:noeRs} and 
Lemma \ref{le:nosx} allows us to 
conclude that the applications of $\eRs$, $\eLs$ and $\iLs$ can be
eliminated from any derivation in $NEQ_4$.

Similarly, though alternatively, combining lemmas~\ref{le:noar} and \ref{le:noiRsnr} 
%and \ref{le:noeRsnr}
-- in this order, and with their side-conditions -- allows us to conclude 
that the applications of $\iLanr$, $\eRanr$, $\eRs$ and $\iRsnr$ can be 
eliminated. 

These two facts will be of crucial importance in the proof of cut-elimination in
\ref{sub:cut}.
\end{REMARK}
%
\subsection{Equivalence $NEQ_{4}\equiv NEQ_{3}$}\label{sub:equiv}
The rules of $NEQ_{4}$ are merely restricted versions of the rules of 
$NEQ_{3}$, so we trivially have $NEQ_{4}\impl NEQ_{3}$. For the 
opposite implication, we need to show admissibility in $NEQ_{4}$ of 
the unrestricted rules. Lemma~\ref{le:noeqeq} showed admissibility 
of $\eae$ and lemma~\ref{le:noLanr} of $\eLanr$. 
We now show admissibility in $NEQ_{4}$ of the remaining rules $(E_{ax})$, $(E_{sx})$ 
and $(cut_{t})$. % and $(cut_{x=})$ 

\begin{LEMMA}\label{le:noEad}\label{le:noEsd} 
% $\der{NEQ_4}DS\ \impl\ \der{NEQ_4}{D'}S$ and $D'$ uses no applications of 
The rules $\Eax$ and $\Esx$ are admissible in $NEQ_4$.
\end{LEMMA}
\begin{PROOF}
Consider first the uppermost application of $\Eax$.
Renaming the eigen-variable $z$ in the whole $D_1$ to $x$, and applying ($cut_x$)
to the result of $D_1'$, we get the desired result. (Remember that we may have $r(z)=z$.)
\[ \begin{array}{cl}
D_1 \left \{ \begin{array}{c}
 \vdots \\
 z\Incl x, y\Incl r(z), \Gamma\Seq\Delta \end{array} \right . \\ \cline{1-1}
 y \Incl r(x), \Gamma\Seq\Delta & \rabove{\Eax}
\end{array} 
\conv
\begin{array}{cl}
D_1' \left \{ \begin{array}{c}
 \vdots \\
 x\Incl x, y\Incl r(x), \Gamma\Seq\Delta \end{array} \right . \\ \cline{1-1}
 y \Incl r(x), \Gamma\Seq\Delta & \raisebox{1.2ex}[1.5ex][0ex]{$(cut_x)$}
\end{array} \]
In a similar way, we consider the uppermost application of $\Esx$:
\[ \begin{array}{cl}
D\left\{ \begin{array}{c}
 \vdots \\
 \Gamma,x\Incl y\Seq\Delta \end{array} \right. \\ \cline{1-1}
 \Gamma\Seq\Delta_y^x  &  \raisebox{1.2ex}[1.5ex][0ex]{($E_{sx}$)}
\end{array} \]
In all applications of $\Esr$ and $\Ear$ in $D$, replace the
eigen-variables so that they differ from the eigen-variable $x$ and from $y$. 
In the obtained
derivation, replace all the occurrences of $x$ by $y$. In this way, we 
obtain a derivation of $\Gamma,y\Incl y\Seq\Delta_y^x$. 
Application of ($cut_x$) gives the desired derivation $D^*$ of the
sequent $\Gamma\Seq\Delta_y^x$ 
without using degenerated applications of ($E_{sx}$).
\end{PROOF}
%
%% \begin{LEMMA}\label{le:noEsd} 
%% The rule ($E_{sx}$) is admissible in $NEQ_4$.
%% \end{LEMMA}
%% \begin{PROOF}
%% By induction on the number of applications of ($E_{sx}$) in a given
%% derivation $D$. Consider the uppermost application:
%% \[ \begin{array}{cl}
%% D\left\{ \begin{array}{c}
%%  \vdots \\
%%  \Gamma,x\Incl y\Seq\Delta \end{array} \right. \\ \cline{1-1}
%%  \Gamma\Seq\Delta_y^x  &  \raisebox{1.2ex}[1.5ex][0ex]{($E_{sx}$)}
%% \end{array} \]
%% In all applications of $\Esr$ and $\Ear$ in $D$, replace the
%% eigen-variables so that they differ from the eigen-variable $x$ and from $y$. 
%% In the obtained
%% derivation, replace all the occurrences of $x$ by $y$. In this way, we 
%% obtain a derivation of $\Gamma,y\Incl y\Seq\Delta_y^x$. 
%% Application of ($cut_x$) gives the desired derivation $D^*$ of the
%% sequent $\Gamma\Seq\Delta_y^x$ 
%% without using degenerated applications of ($E_{sx}$).
%% \end{PROOF}
%
%We show that ($cut_t$) is admissible in $NEQ_4$.
\begin{LEMMA}\label{le:nott} $\der{NEQ_4}D{t\Incl t,\Gamma\Seq\Delta}  
\ \ \Rightarrow \ \ \der{NEQ_4}{D^*}{\Gamma\Seq\Delta}$.
%[$D^*$ does not introduce new applications of $(cut_{x=})$.]
%, and $D^*$ does not use ($E_{sx}$) if $D$ does not use it.
\end{LEMMA}
\begin{PROOF} 
By lemma~\ref{le:noar} %\ref{le:noLanr}, $\eLanr$,
we assume that $D$ contains no applications of  $\iLanr$, $\eRanr$

By induction on $\<\gamma[t],h(D)\>$, where $\gamma[t]$ is the complexity of 
the term $t$ (defined in the usual way). 
If $\gamma[t]=0$, i.e., $t$ is a variable $x$,  application of ($cut_x$) yields the 
conclusion. This is all that may happen if $h(D)=1$, i.e. we merely modify once
some axiom. (Superposition of the antecedent in $x\Incl t,\GSD,x\Incl t$ leads to
$t\Incl t,x\Incl t,\GSD,x\Incl t$ -- removal of $t\Incl t$ yields the same axiom.)

So let $\gamma[t]>0$ and $R$ be the last rule applied.
 The only rules which may generate $t\Incl t$ in the antecedent
are $\iLar$, $\Ear$ and $\eRar$ or
$\eLar$. We consider these cases by looking at the rule $R_1$ applied just above
%
\begin{LS}
\item $R$ is $\iLar$ and $D$ ends as shown below, where $t$ is $t(s)$:
\[ D \left \{ \begin{array}{rl}
\multicolumn{1}{c}{\vdots} \\ \cline{1-1}
t(y)\Incl t(s), y\Incl s, \Gamma\Seq\Delta & \rabove{R_1} \\ \cline{1-1}
t(s) \Incl t(s), y\Incl s, \GSD & \rabove{\iLar} \end{array} \right .\]
%
\begin{LSA}
\item $R_1$ is $\iLar$: The two possibilities are that $t(y)\Incl t(s)$ was
\ref{it:tact} active
or \ref{it:tmod} modified in $R_1$. The former requires $t(y)$ to be the variable
 $y$ and so $t(s)$ to be $s$.
\begin{LSB}
\item\label{it:tact} $y\Incl t$ was active:
\[ \begin{array}{rl}
\multicolumn{1}{c}{\vdots} \\ 
y\Incl t, y\Incl t, q(y)\preceq p, \Gamma\Seq\Delta  \\ \cline{1-1}
y\Incl t, y\Incl t, q(t)\preceq p, \Gamma\Seq\Delta & \rabove{\iLar} \\ \cline{1-1}
t\Incl t, y\Incl t, q(t)\preceq p, \GSD & \rabove{\iLar} \end{array} \]
Swapping the two applications gives a derivation with an earlier
appearance of $t\Incl t$. 
%
\item\label{it:tmod} $y\Incl t$ was modified (we write it here in full as $t(y)\Incl t(p(q))$):
\[ D \left \{ \begin{array}{rl}
\multicolumn{1}{c}{\vdots} \\ 
y\Incl p(x), t(y)\Incl t(p(q)), x\Incl q, \GSD \\ \cline{1-1}
y\Incl p(x), t(p(x))\Incl t(p(q)), x\Incl q, \GSD & \rabove{\iLar} \\ \cline{1-1}
y\Incl p(x), t(p(q))\Incl t(p(q)), x\Incl q, \GSD & \rabove{\iLar} \end{array} \right
.\]
We first use a weakened derivation with $y\Incl p(q)$
\[ \begin{array}{cl}
 D_1 \left \{ \begin{array}{rl}
\multicolumn{1}{c}{\vdots} \\ 
y\Incl p(q), y\Incl p(x), t(y)\Incl t(p(q)), x\Incl q, \GSD \\ \cline{1-1}
y\Incl p(q), y\Incl p(x), t(p(q))\Incl t(p(q)), x\Incl q, \GSD & \rabove{\iLar} 
 \end{array} \right . \\ \cline{1-1}
p(x) \Incl p(q), y\Incl p(x), t(p(q))\Incl t(p(q)), x\Incl q, \GSD &
\rabove{\iLar}  \\ \cline{1-1}
p(q) \Incl p(q), y\Incl p(x), t(p(q))\Incl t(p(q)), x\Incl q, \GSD &
\rabove{\iLar} \end{array} \]
By induction on $h(D_1)<h(D)$ we can obtain the sequent concluding $D_1$ without
 $t(p(q))\Incl t(p(q))$.
Then, the final $p(q)\Incl p(q)$, can be eliminated by induction hypothesis
on complexity of terms, since $\gamma[p(q)]< \gamma[t(p(q))]$.
\end{LSB}
%
\item $R_1$ is $\Ear$:
\[ \begin{array}{rl}
x\Incl p, y\Incl t(x), y\Incl t(p), \GSD \\ \cline{1-1}
 y\Incl t(p), y\Incl t(p), \GSD & \rabove{\Ear} \\ \cline{1-1}
 t(p)\Incl t(p), y\Incl t(p), \GSD & \rabove{\iLar} \end{array} \convd
%Weaken the derivation as follows:
 \begin{array}{rl}
x\Incl p, y\Incl t(x),    y\Incl t(p), y\Incl t(x),\GSD \\ \cline{1-1}
x\Incl p, t(x)\Incl t(x), y\Incl t(p), y\Incl t(x),\GSD & \rabove{\iLar} \\ \cline{1-1}
 t(x)\Incl t(x), y\Incl t(p), y\Incl t(p),\GSD & \rabove{\Ear} \end{array} \]
Induction on complexity (or height) gives the conclusion (having 
eliminated $t(x)\Incl t(x)$ the application of $\Ear$ becomes correct).
%
\item $R$ is $\eRar$:
\[ \begin{array}{rl}
 y\Incl t(s), y\Incl t(p), p=s, \GSD \\ \cline{1-1}
 y\Incl t(p), y\Incl t(p), p=s, \GSD & \rabove{\eRar} \\ \cline{1-1}
 t(p)\Incl t(p), y\Incl t(p), p=s, \GSD & \rabove{\iLar} \end{array} \]
Weakening the derivation  with $y\Incl t(s)$ (if $\gamma[t(s)]<\gamma[t(p)]$,
and with $y\Incl t(p)$ otherwise), we obtain:
\[ \begin{array}{rl}
y\Incl t(s), y\Incl t(s), y\Incl t(p), p=s, \GSD \\ \cline{1-1}
 t(s)\Incl t(s), y\Incl t(s), y\Incl t(p), p=s, \GSD & \rabove{\iLar} \\ \cline{1-1}
 t(s)\Incl t(s), y\Incl t(p), y\Incl t(p), p=s, \GSD & \rabove{\eRar} \end{array} \]
Induction on height gives the conclusion. 
%
\item $R$ is $\eLar$:
\[ D \left \{ \begin{array}{rl}
t(x)\Incl t(s(p)), x=s(y), y\Incl p, \GSD \\ \cline{1-1}
t(s(y))\Incl t(s(p)), x=s(y), y\Incl p, \GSD & \rabove{\eLar} \\
\cline{1-1}
t(s(p))\Incl t(s(p)), x=s(y), y\Incl p, \GSD & \rabove{\iLar} \end{array}
\right . \convd
%Weakening $D$ we may obtain:
 \begin{array}{cl}
D_1 \left \{ \begin{array}{rl}
x\Incl s(p), t(x)\Incl t(s(p)), x=s(y), y\Incl p, \GSD \\ \cline{1-1}
x\Incl s(p), t(s(p))\Incl t(s(p)), x=s(y), y\Incl p, \GSD & \rabove{\iLar} \end{array}
\right . \\ \cline{1-1}
s(y)\Incl s(p), t(s(p))\Incl t(s(p)), x=s(y), y\Incl p, \GSD &
\rabove{\eLar} \\ \cline{1-1}
s(p)\Incl s(p), t(s(p))\Incl t(s(p)), x=s(y), y\Incl p, \GSD &
\rabove{\iLar} \end{array} \]
Occurence of $t(s(p))\Incl t(s(p))$ in $D_1$ can be eliminated by induction
hypothesis $h(D_1)<h(D)$, and of $s(p)\Incl s(p)$ by $\gamma[s(p)] < \gamma[t(s(p))]$.
%
\item All the remaining cases can be easily swapped yielding an earlier
occurrence of $t\Incl t$.

$\Esr$ :  Since the active and the side formula of the subsequent
$\iLar$ have to share a variable in the LHS, neither might be active in 
this application of $\Esr$.

$\iLs$, $\iRs$ : Activity of $t(y)\Incl t(s)$ in this rule, can be simulated by
the activity of $y\Incl s$ which also must be present.

$\eLs$, $\eRs$, ($Sp.cut$), ($cut_x$) and $(cut_{x=})$ are trivial.
\end{LSA}
%
\item  $R$ is $\eLar$ and $D$ ends as shown below :
\[ D \left \{ \begin{array}{rl}
\multicolumn{1}{c}{\vdots} \\ \cline{1-1}
t(y)\Incl t(s), y=s, \Gamma\Seq\Delta & \rabove{R_1} \\ \cline{1-1}
t(s) \Incl t(s), y=s, \GSD & \rabove{\eLar} \end{array} \right .\]
%
\begin{LSA}
%
\item $R_1$ is $\iLar$ The two possibilities are that $t'(y)\Incl t$ was
\ref{it:yact} active
or \ref{it:ymod} modified in $R_1$. The former requires $t'(y)$ to be the variable
$y$.
\begin{LSB}
\item\label{it:yact} $y\Incl t$ active:
\[ D \left \{ \begin{array}{rl}
\multicolumn{1}{c}{\vdots} \\ 
y\Incl t, y=t, q(y)\preceq p, \Gamma\Seq\Delta  \\ \cline{1-1}
y\Incl t, y=t, q(t)\preceq p, \Gamma\Seq\Delta & \rabove{\iLar} \\ \cline{1-1}
t \Incl t, y=t, q(t)\preceq p, \GSD & \rabove{\eLar} \end{array} \right
. \convd
%Instead, we may use
  \begin{array}{cl} D_1 \left \{ \begin{array}{rl}
\multicolumn{1}{c}{\vdots} \\ 
y\Incl t, y=t, q(y)\preceq p, \Gamma\Seq\Delta  \\ \cline{1-1}
t\Incl t, y=t, q(y)\preceq p, \Gamma\Seq\Delta & \rabove{\eLar} \end{array}
\right . \\ \cline{1-1}
t \Incl t, y=t, q(t)\preceq p, \GSD & \rabove{\eLar} \end{array} \]
and induction hypothesis, $h(D_1)<h(D)$, gives the conclusion. 
%\begin{REMARK}
%This merely exemplifies  the fact that activity of $y\Incl t$, can be
%simulated by activity of $y=t$: in 
 %\end{REMARK}
%
\item\label{it:ymod} $y\Incl t$ was modified (we write it as $q(y)\Incl q(p(t))$)
\[ D \left \{ \begin{array}{rl}
\multicolumn{1}{c}{\vdots} \\ 
y\Incl p(x), q(y)\Incl q(p(t)), x=t, \GSD \\ \cline{1-1}
y\Incl p(x), q(p(x))\Incl q(p(t)), x=t, \GSD & \rabove{\iLar} \\ \cline{1-1}
y\Incl p(x), q(p(t))\Incl q(p(t)), x=t, \GSD & \rabove{\eLar} \end{array} \right
. \convd
%Then we derive the same sequent
 D' \left \{ \begin{array}{cl} D_1 \left \{ \begin{array}{rl}
\multicolumn{1}{c}{\vdots} \\ 
y\Incl p(t), y\Incl p(x), q(y)\Incl q(p(t)), x=t, \GSD \\ \cline{1-1}
y\Incl p(t), y\Incl p(x), q(p(t))\Incl q(p(t)), x=t, \GSD & \rabove{\iLar} \end{array}
\right . \\ \cline{1-1}
y\Incl p(x), y\Incl p(x), q(p(t))\Incl q(p(t)), x=t, \GSD & \rabove{\eRar} \end{array} \right
.\]
\end{LSB}
%
\item $R_1$ is $\Ear$
\[ D \left \{ \begin{array}{rl}
x\Incl p, y\Incl t(x), t(p)=y, \GSD \\ \cline{1-1}
y\Incl t(p), t(p)=y, \GSD  & \rabove{\Ear} \\ \cline{1-1}
t(p)\Incl t(p), t(p)=y, \GSD  & \rabove{\eLar} \end{array} \right. \convd
%Construct:
 D' \left \{ \begin{array}{rl}
x\Incl p, y\Incl t(x), t(p)=y, \GSD \\ \cline{1-1}
y\Incl t(p), t(p)=y, \GSD  & \rabove{\Ear} \\ \cline{1-1}
y\Incl y, t(p)=y, \GSD  & \rabove{\eRar} \end{array} \right . \]
and $y\Incl y$ can be eliminated by induction hypothesis on $\gamma[t]$.
%
\item $R_1$ is $\eRar$ Then LHS of the side inclusion must be a variable,
i.e., 
\[ D \left \{ \begin{array}{rl}
y\Incl t(p), p=q, y=t(q), \GSD \\ \cline{1-1}
y\Incl t(q), p=q, y=t(q), \GSD & \rabove{\eRar} \\ \cline{1-1}
t(q)\Incl t(q), p=q, y=t(q), \GSD & \rabove{\eLar} \end{array} \right . \convd
%We construct $D'$ as follows:
 D' \left \{ \begin{array}{cl}
  D_1 \left \{ \begin{array}{rl}
 y\Incl t(p), p=q, y=t(q), y=t(p), \GSD \\ \cline{1-1}
t(p)\Incl t(p), p=q, y=t(q), y=t(p), \GSD & \rabove{\eLar} 
\end{array} \right . \\ \cline{1-1}
t(p)\Incl t(p), p=q, y=t(q), y=t(q), \GSD & \rabove{(=_{a=})} 
\end{array} \right . \]
 $h(D_1)<h(D)$ yields the conclusion.
%
\item $R_1$ is $\eLar$ 
\[ D \left \{ \begin{array}{rl}
 t(x)\Incl t(s(p)), y=p, x=s(y), \GSD \\ \cline{1-1}
 t(s(y))\Incl t(s(p)), y=p, x=s(y), \GSD & \rabove{\eLar} \\ \cline{1-1}
 t(s(p))\Incl t(s(p)), y=p, x=s(y), \GSD & \rabove{\eLar} \end{array}
 \right . \convd
%Weakening we get:
 \begin{array}{cl} D_1 \left \{ \begin{array}{rl}
 x=s(p), t(x)\Incl t(s(p)), y=p, x=s(y), \GSD \\ \cline{1-1} 
 x=s(p), t(s(p))\Incl t(s(p)), y=p, x=s(y), \GSD & \rabove{\eLar} 
\end{array} \right .  \\ \cline{1-1}
 x=s(y), t(s(p))\Incl t(s(p)), y=p, x=s(y), \GSD & \rabove{(=_{a=})} 
\end{array} \]
 $h(D_1)<h(D)$ gives the conclusion. \\
In the last two cases we have used lemma~\ref{le:noeqeq} which admits $\eae$.

%
\item All other cases of $R_1$ can be easily swapped leading to an earlier
occurrence of $t\Incl t$ (cf. the end of case 1. for $R$ = $\iLar$)
%
\end{LSA}
%
\item  $R$ is $\Ear$ This would require a degenerate application of this
rule which does not belong to $NEQ_4$ (cf. lemma \ref{le:noEad}). 
%
\item  $R$ is $\eRar$ Since this requires a variable in the LHS, we are
back in the ground case for $\gamma[t]=0$.
\end{LS}
%As we see, no new applications of $(cut_{x=})$ appear in the transformed derivations.
\end{PROOF}

\noindent
The remark at the very begining of this section \ref{sub:equiv}, together with lemmas
\ref{le:noeqeq}, \ref{le:noLanr}, \ref{le:noEsd} and
\ref{le:nott} (showing the admissibility of the 
missing $NEQ_{3}$ rules in $NEQ_{4}$), give us the first part of the 
following fact from which the second follows trivially.
\begin{CLAIM}\label{le:neq3isneq4}
$NEQ_3\equiv NEQ_4$, and thus, also $NEQ_{3}^{c}\equiv NEQ_{4}^{c}$.
\end{CLAIM}
% \noindent
% As an immediate corollary of this lemma, we have a strengthening of 
% lemma~\ref{le:neq3cisneq4c}: 
% \begin{COROLLARY}\label{co:neq3isneq4}
% $NEQ_3\equiv NEQ_4$.
% \end{COROLLARY}
% Thus, rules admissible in $NEQ_3$ are admissible in $NEQ_4$ too.

%
%% \begin{LEMMA}\label{le:noErsd} $\der{NEQ_4}DS\ \impl\ \der{NEQ_4}{D^*}S$
%% and $D^*$ contains no applications of ($E_{Rsd}$).
%% \end{LEMMA}
%% \begin{PROOF}
%% By induction on the number of applications of ($E_{Rsd}$) in a given
%% derivation $D$. 
%% By lemma~\ref{le:noar} we assume no applications of $\eRanr$ or $\iLanr$ in $D$.
%% %By lemma~\ref{le:noEad} %, resp. \ref{le:noEsd} 
%% %we assume that $D$ contains no applications of $(E_{ax})$.
%% %, resp. $(E_{sx})$ 
%% \[ \begin{array}{cl}
%% D_1\left\{ \begin{array}{cl}
%%  \vdots & \\ \cline{1-1}
%%  x\Incl t,\GSD,p\preceq x & \raisebox{1.2ex}[1.5ex][0ex]{$R$}\end{array} \right. \\ \cline{1-1}
%%  \GSD,p\preceq t  &  \raisebox{1.2ex}[1.5ex][0ex]{($E_{Rsd}$)}
%% \end{array} \]
%% \begin{LS}
%% \item $R$ is $\eRar$ %$(=_{Rar})$
%% \[ \begin{array}{cl}
%% s=r, x\Incl t(r),\GSD, p\preceq x  & \\ \cline{1-1}
%% s=r, x\Incl t(s),\GSD,p\preceq x & \rabove{\eRar} \\ \cline{1-1}
%% s=r,\GSD,p\preceq t(s)  &  \rabove{(E_{Rsd})}
%% \end{array} 
%% \conv 
%% \begin{array}{cl}
%%  s=r,x\Incl t(r),\GSD, p\preceq x  & \\ \cline{1-1}
%%  s=r, \GSD,p\preceq t(r) & \rabove{(E_{Rsd})} \\ \cline{1-1}
%%  s=r, \GSD,p\preceq t(s) & \rabove{\eRs} 
%% \end{array}
%% \]
%% ($E_{Rsd}$) can be eliminated by induction on $h(D)$. 
%% %The application of ($=_{Rs}$), admissible in $NEQ_3$ (lemma~\ref{le:noeqSD}) and, 
%% %by corollary~\ref{le:neq3cisneq4c} in $NEQ_4$, does not then introduce any new
%% %applications of $E_s$ by the qualification in lemma~\ref{le:noeqSD}.
%% \item $R$ is $\iLar$ 
%% \[ \begin{array}{cl}
%% x\Incl t, w(x)\Incl q,\GSD, p\preceq x  & \\ \cline{1-1}
%% x\Incl t, w(t)\Incl q,\GSD,p\preceq x & \rabove{\iLar} \\ \cline{1-1}
%% w(t)\Incl q, \GSD,p\preceq t  &  \rabove{(E_{Rsd})}
%% \end{array} 
%% \conv 
%% \begin{array}{cl}
%% x\Incl t, w(x)\Incl q,\GSD, p\preceq x  & \\ \cline{1-1}
%% x\Incl t, w(x)\Incl q,\GSD,p\preceq t & \rabove{(E_{Rsd})} \\ \cline{1-1}
%% w(t)\Incl q, \GSD,p\preceq t  &  \rabove{\iLar}
%% \end{array}
%% \]
%% \item $R$ is $(E_a)$ 
%% \[ \begin{array}{cl}
%% x\Incl t(z), z\Incl q,\GSD, p\preceq x  & \\ \cline{1-1}
%% x\Incl t(q), \GSD,p\preceq x & \raisebox{1.2ex}[1.5ex][0ex]{($E_{a}$)} \\ \cline{1-1}
%%  \GSD,p\preceq t(q)  &  \raisebox{1.2ex}[1.5ex][0ex]{($E_{Rsd}$)}
%% \end{array} 
%% \conv 
%% \begin{array}{cl}
%% x\Incl t(z), z\Incl q,\GSD, p\preceq x  & \\ \cline{1-1}
%% z\Incl t(q), \GSD,p\preceq t(z) & \raisebox{1.2ex}[1.5ex][0ex]{($E_{Rsd}$)} \\ \cline{1-1}
%%  \GSD,p\preceq t(q) & \raisebox{1.2ex}[1.5ex][0ex]{($E_{s}$)} 
%% \end{array}
%% \]
%% By lemma~\ref{le:noEad} the application of $(E_a)$ is not $(E_{ax})$, i.e. $t(z)\not=z$.
%% Hence, the resulting application of $(E_s)$ is not $(E_{Rsd})$.
%% \end{LS}
%% The rest is simple. $(=_{Lar})$ cannot result in $x\Incl t$, so it can be swapped.
%% Similarly, $(\preceq_{sr})$ modify at most LHS in the consequent, and $(E_s)$ involves 
%% another eigen-variable, so they can be swapped.
%% as well.
%% \end{PROOF}

\subsection{Admissibility of ($cut$).}\label{sub:cut}
Referring to the remark~\ref{re:crucial}, 
in the following proof of ($cut$)-elimination, 
we may introduce additional
\\[1ex]
\noindent
{\bf Assumptions:} If $\der{NEQ_4}DS$ then $\der{NEQ_4}{D^*}S$ where $D^*$ contains 
\vspace*{1ex}
%Any derivation $D$ in $NEQ_4$ can be transformed so that
%it contains

\hspace*{-1em}\begin{tabular}{rll}
1.& either no $\eRs$, $\eLs$, $\iLs$ 
 & lemma  \ref{le:nosx} (\ref{fa:noeRs}) \\ % subsection \\
2.& or no $\iLanr$, $\eRanr$, $\eRs$, $\iRsnr$
 & lemma \ref{le:noar}, \ref{le:noiRsnr} \\[.5ex]
% (3.)& no formula $x=x$ in the antecedent & lemma \ref{le:noxx} \\
% (4.)& no formula $t\Incl t$ in the antecedent & lemma \ref{le:nott} \\[.5ex]
\multicolumn{3}{l}{\hspace*{-1em}Also, $NEQ_4$ admits the rules} \\[.5ex]
3.& $\eae$, $\iLanr$, $\eRanr$, $\eLanr$  and $\iRa$ 
 & lemma \ref{le:noeqeq}, \ref{le:noar}, \ref{le:noLanr}, \ref{le:inclaad} 
 \\[.5ex]
& without increasing the total number of $\iLar$ and $\eLar$  \\[.5ex]
4.& $(cut_t)$    %and $(cut_{x=})$ 
  & lemma \ref{le:nott} % \ref{le:noxx}
\end{tabular} 

 
\begin{LEMMA}\label{le:elcut}
If $D$ is a $NEQ_4^c$ and $D_1$, $D_2$ are $NEQ_4$ derivations as follows:
%\begin{center}
\[ D\left\{ \begin{array}{cl}
 D_1\left\{ \begin{array}{c}
  \vdots \\   \Gamma\Seq\Delta, \phi
 \end{array} \right.
 D_2\left\{ \begin{array}{c}
  \vdots \\   \phi, \Gamma'\Seq\Delta'
 \end{array} \right. \\ \cline{1-1}
\Gamma,\Gamma' \Seq \Delta,\Delta'
&   \raisebox{1.2ex}[1.5ex][0ex]{$(cut)$}
\end{array}\right. \]
%\end{center}
then $\der{NEQ_4}{}{\Gamma,\Gamma'\Seq\Delta,\Delta'}$.
\end{LEMMA}
\begin{PROOF}
By induction on $\< \#(\preceq_a,D_2), h(D_1), h(D_2)\>$, 
where the first parameter is the total number of applications of ($\Incl_{ar}$) 
{\em or} ($=_{Lar}$) {\em which modify the cut formula} $\phi$ in $D_2$ (this
qualification is essential in case \ref{it:cutactive} on page
\pageref{it:cutactive}).
%  $h(D_1)$ is the height of $D_1$ and $h(D_2)$ the height of  $D_2$. 
\\[1ex]
\noindent 
If $h(D_1)=0$, then the resulting sequent of $D_1$ is an axiom. If
the cut formula is not the one mentioned explicitly in the axioms (Fig.~\ref{fi:neq4}) 
then we obtain a cut-free derivation directly by choosing
another instance of the same axiom. In the other case, if the cut formula is
 $t\Incl t$ or $x=x$, the assumption 4. or 3., respectively, allows us to conclude the
 existence of a derivation of the conclusion of $D_2$ without this formula in
 the antecedent. Finally, if the axiom in $D_1$ is
 $\Gamma,s=t\Seq s=t,\Delta$ and the cut formula is $s=t$, then it will also appear in the
 antecedent of the conclusion of ($cut$). Then this conclusion can be
 obtained without ($cut$) directly from $D_2$ by starting it with the
 instance of the axiom extended with $\Gamma$ and $\Delta$. \\[1ex]
\noindent
For $h(D_1)>0$, we consider the last rule $R$ applied in $D_1$. \\[1ex]
\noindent
{\bf I.} ($=_{sr}$):
\[ D \left \{ \begin{array}[t]{cl}
 \begin{array}{cl}
 D_1^*\left\{ \begin{array}{c}
  \vdots \\   t=s,\Gamma\Seq\Delta, w(s)\preceq q
 \end{array} \right. \\ \cline{1-1}
t=s,\Gamma\Seq\Delta, w(t)\preceq q & \raisebox{1.2ex}[1.5ex][0ex]{($=_{sr}$)}
 \end{array}
 D_2\left\{ \begin{array}{c}
  \vdots \\ \vdots \\  w(t)\preceq q, \Gamma'\Seq\Delta'
 \end{array} \right. \\ \cline{1-1}
t=s,\Gamma,\Gamma' \Seq \Delta,\Delta'
&   \raisebox{1.2ex}[1.5ex][0ex]{($cut$)}
\end{array} \right . \convd
%
%Instead, we drop the last application in $D_1$ and construct following
%derivation from $D_2$:
 D' \left \{ \begin{array}[t]{cl}
 D_1^*\left\{ \begin{array}{c}
  \vdots \\ \vdots \\  t=s,\Gamma\Seq\Delta, w(s)\preceq q
 \end{array} \right. 
D'_2\left\{\begin{array}{cl}
 D_2\left\{ \begin{array}{c}
  \vdots \\   w(t)\preceq q, t=s, \Gamma'\Seq\Delta'
 \end{array} \right. \\ \cline{1-1}
w(s)\preceq q, t=s, \Gamma'\Seq\Delta' & \rabove{(=_a)}
 \end{array} \right.\\ \cline{1-1}
t=s,\Gamma,\Gamma' \Seq \Delta,\Delta'
&   \raisebox{1.2ex}[1.5ex][0ex]{($cut$)}
\end{array} \right . \]
 $t$ cannot be a variable, since this would imply that the application in
 $D_1$ was ($=_{sx}$), which isn't a $NEQ_4$ rule.
Hence, the application of ($=_a$) is either ($=_{a=}$) or ($=_{Lanr}$) (if it was 
($=_{Lar}$), i.e. if $t\in\Vars$, we would get increas of the induction parameter). 
The former can
be eliminated by lemma \ref{le:asinNEQ3}, and the latter by \ref{le:noLanr} 
(and corollary~\ref{co:neq3isneq4}) --
both without intcreasing the number of applications of ($\preceq_a$).
Thus, $\#(\preceq_a, D'_2)$ is not greater
than in the original $D_2$, while the height of $D_1$ at which to
perform ($cut$) has been reduced.  \\[1ex]
%
\noindent
{\bf II.} ($\Incl_s$): We do a similar trick as in the previous case:
%\item  
\[ D \left \{\begin{array}[t]{cl}
 \begin{array}{cl}
 D_1^*\left\{ \begin{array}{c}
  \vdots \\   s\Incl p,\Gamma\Seq\Delta, w(p)\preceq q
 \end{array} \right. \\ \cline{1-1}
s\Incl p,\Gamma\Seq\Delta, w(s)\preceq q & \raisebox{1.2ex}[1.5ex][0ex]{($\Incl_s$)}
 \end{array}
 D_2\left\{ \begin{array}{c}
  \vdots \\   w(s)\preceq q, \Gamma'\Seq\Delta'
 \end{array} \right. \\ \cline{1-1}
s\Incl p,\Gamma,\Gamma' \Seq \Delta,\Delta'
&   \raisebox{1.2ex}[1.5ex][0ex]{($cut$)}
\end{array} \right . \convd
%
%We do a similar trick as in the previous case:
 D' \left \{ \begin{array}[t]{cl}
 D_1^*\left\{ \begin{array}{c}
  \vdots \\ \vdots \\  s\Incl p ,\Gamma\Seq\Delta, w(p)\preceq q
 \end{array} \right. 
D'_2\left\{ \begin{array}{cl}
 D_2\left\{ \begin{array}{c}
  \vdots \\   w(s)\preceq q, s\Incl p, \Gamma'\Seq\Delta'
 \end{array} \right. \\ \cline{1-1}
w(p)\preceq q, s\Incl p, \Gamma'\Seq\Delta' & \rabove{(\Incl_a)}
 \end{array} \right. \\ \cline{1-1}
s\Incl p ,\Gamma,\Gamma' \Seq \Delta,\Delta'
&   \raisebox{1.2ex}[1.5ex][0ex]{($cut$)}
\end{array} \right . \]
 $s$ cannot be a variable because then the application of ($\Incl_s$) in
 $D_1$ would be ($\Incl_{sx}$) which is not a $NEQ_4$ rule 
(cf. lemma \ref{le:noInclsx}). Thus, the application of
 ($\Incl_a$) after $D_2$ in $D'_2$ is not reduced, and this can be eliminated
 by lemma \ref{le:noar} (and corollary~\ref{co:neq3isneq4}) 
without increasing the number 
 of applications of ($\preceq_a$) in $D'_2$. So ($cut$) can be eliminated
 using the second parameter of induction hypothesis. \\[1ex]
%
\noindent
{\bf III.} ($\Incl_{ar}$), ($E_a$), ($=_{Lar}$), ($=_{Rar}$), ($Sp.cut$) or ($cut_x$): 
Since none of
these rules modifies the succedent, their application as $R$ in $D_1$ means
that we can apply ($cut$) with $D_1^*$ instead, and the induction on $h(D_1$)
gives the conclusion. \\[1ex]
{\bf IV.} ($E_s$):
%
\[ \begin{array}{cl}
D_1\left\{ \begin{array}{cl}
 D_1^*\left\{ \begin{array}{cl}
  \vdots \\ 
  x\Incl t,\Gamma\Seq\Delta, \phi_x^x  
 \end{array} \right. \\ \cline{1-1}
\Gamma\Seq\Delta, \phi_t^x & \raisebox{1.2ex}[1.5ex][0ex]{($E_s$)}
 \end{array} \right .
 D_2\left\{ \begin{array}{cl}
  \vdots \\ \cline{1-1}  \phi_t^x, \Gamma'\Seq\Delta' & \raisebox{1.2ex}[1.5ex][0ex]{$R'$}
 \end{array} \right. \\ \cline{1-1}
\Gamma,\Gamma' \Seq \Delta,\Delta'
&   \raisebox{1.2ex}[1.5ex][0ex]{($cut$)}
\end{array} \]
%
By restrictions on ($E_s$) $t\not\in\Vars$ and, furthermore, 
$x\not\in\C V(\Gamma,\Delta,t)$ so we can choose
 $x$ so that $x\not\in\C V(\Gamma',\Delta')$. We consider two cases: \ref{it:Ar} when
in $D_1$, $x$ in $\phi$ is in the RHS of $\Incl$, 
and \ref{it:Bl} when $x$ occurs in
a LHS of $\Incl$ or when $\phi$ is an equality. 
\begin{LS}
%
%
\item\label{it:Ar} Let $\phi$ be $p\Incl s(t)$  \\[.5ex]
\noindent
-- $t$ indicating the only occurrence
which has been substituted for $x$. We can then 
%construct the following derivation 
extend $D_2$ and construct the following derivation:
%
\[
 \begin{array}{cl}
 D_1^*\left\{ \begin{array}{cl}
  \ \\ \ \\ \vdots \\ 
  x\Incl t,\Gamma\Seq\Delta, p\Incl s(x) 
         \end{array} \right. \ \ \ 
%
D'_2\left\{ \begin{array}{cl}
  D_2\left\{ \begin{array}{cl}
  \vdots \\ 
  p\Incl s(t),\Gamma'\Seq\Delta' \end{array} \right.  \\ \cline{1-1}
 s(x)\Incl s(t),p\Incl s(t),\Gamma'\Seq\Delta' 
   & \raisebox{1.2ex}[1.5ex][0ex]{(W)} \\ \cline{1-1}
s(x)\Incl s(t), p\Incl s(x),\Gamma'\Seq\Delta' 
& \raisebox{1.2ex}[1.5ex][0ex]{($\Incl_a^*$)}
         \end{array} \right.
%     \end{array} \right. 
%
\\ \cline{1-1}
x\Incl t, s(x)\Incl s(t), \Gamma',\GSD,\Delta' 
  & \raisebox{1.2ex}[1.5ex][0ex]{($cut$)} \\ \cline{1-1}
x\Incl t, s(t)\Incl s(t), \Gamma',\GSD,\Delta' 
  & \raisebox{1.2ex}[1.5ex][0ex]{($\Incl_{ar}$)} \\ \cline{1-1}
s(t)\Incl s(t), \Gamma',\GSD,\Delta' 
  & \raisebox{1.2ex}[1.5ex][0ex]{($E_s$)} \\ \cline{1-1}
\Gamma',\GSD,\Delta' 
& \raisebox{1.2ex}[1.5ex][0ex]{($cut_t$)} 
\end{array} 
 \]
\noindent
Notice that $x\not=s(x)\not\in\Vars$ since, otherwise, the application of 
($E_s$) in $D_1$ would be ($E_{Rsd}$) which is excluded by lemma~\ref{le:noErsd}.
Thus the rule ($\Incl_a^*$) is admissible (lemma \ref{le:inclaad}) and
does not increase the number of
applications of ($\preceq_a$), and ($cut$)
can be eliminated by induction on $h(D_1^*)<h(D_1)$.
The resulting derivation is  in $NEQ_4$ so, by lemma~\ref{le:nott},
($cut_t$) is admissible.
%
\item\label{it:Bl} Let $\phi$ be $s(t)\preceq p$ \\[.5ex]
\noindent
Here we have three subcases, depending on
whether $\phi$ in the application of the last rule $R'$ in $D_2$ was 
\ref{it:cutneither} neither modified nor active, \ref{it:cutmodified} modified, 
or \ref{it:cutactive} active.
%
\begin{LSA}
%
\item\label{it:cutneither} The cut formula $\phi$ is neither modified nor active.\\
We may then swap the
applications of $R'$ and ($cut$), and the induction on the height of $D_2$
yields the required elimination of ($cut$). 
%
\item\label{it:cutmodified}  $\phi$ is modified by $R'$.\\
%%\begin{LSA}
%%  \item $\phi$ is inclusion $s(t)\Incl p$ (modifed by $R'$).  \\[.5ex]
The two possibilities of $\phi$ being an inclusion and equality are entirely 
analogous. This case depends only on the rule $R'$ -- if $\phi$ is an equality, 
$R'$ cannot be ($=_a$), but except for that the two cases are identical.
%  {\sf II.a) 
%
 \begin{LSB}
   \item $R'$ is ($E_a$) or ($=_{Rar}$): \\
%II.a.$\alpha$ 
  This would require $\phi$ to be an inclusion with  a single
   variable in the LHS, i.e, $y\Incl p$. %$s(t)=y$. 
Since in the current case \ref{it:Bl},
  ($E_s$) in $D_1$
   replaces $x$ in the LHS of $\Incl$, this would mean that this application is
   actually degenerate ($E_{sd}$), what is excluded by lemma \ref{le:noEsd}. 
  \item  $R'$ is ($\Incl_{ar}$) or ($=_{Lar}$): \\
%II.a.$\beta$ 
  \noindent
   These two cases are identical so we treat them jointly and
  write ($\preceq_{ar}$). In each subcase, all occurrences of ($\preceq$) must be
  replaced consistently either by ($\Incl_{ar}$) or by ($=_{Lar}$), unless we
  mention the latter rules explicitly. The cut formula $\phi$ 
has the form $\phi[x]$ and $\phi[t]$ after the application of ($E_s$) in $D_1$, while in $D_2$ we 
write it as $\phi'[y]$, and as $\phi'[t']$ after the final application of ($\preceq_a$), i.e., 
 $\phi[t] = \phi = \phi'[t']$. This is to indicate that the term $t$ introduced in $D_1$ and
$t'$ introduced in $D_2$ into $\phi$ may be different. The relation between the two
 does not matter, however -- what makes it possible to treat all the (sub)cases in
the same way, is the fact that if $\phi'[y]$ is an inclusion the modified term is introduced into
its LHS.

$D_2$ ends as follows:
\[ D_2 \left \{ \begin{array}{cl}
 D_2^* \left \{ \begin{array}{cl}
 \vdots \\
 \phi'[y], y\preceq t', \Gamma'\Seq\Delta' \end{array} \right . \\ \cline{1-1}
 \phi'[t'], y\preceq t', \Gamma'\Seq\Delta' & \rabove{(\preceq_{ar})}
 \end{array} \right . \]
We drop this last application of ($\preceq_{ar}$) and, instead, extend $D_1$
weakened with $y\preceq t'$. % \\[.5ex]
%
% \mbox{ 
{ \footnotesize 
\[\begin{array}{cl}
D_1' \left \{ \begin{array}{cl}
D_1 \left \{ \begin{array}{cl}
  \vdots \\ 
  y\preceq t', x\Incl t,\Gamma\Seq\Delta, \phi[x]  \\ \cline{1-1}
y\preceq t', \Gamma\Seq\Delta, \phi[t]  & \raisebox{1.2ex}[1.5ex][0ex]{($E_s$)}
 \end{array} \right . \\ \cline{1-1}
y\preceq t', \Gamma\Seq\Delta, \phi'[y] &
\raisebox{1.2ex}[1.5ex][0ex]{($\preceq_s$)}
 \end{array} \right .
%
 D_2^*\left\{ \begin{array}{cl}
 \vdots \\
\vdots \\
\phi'[y], y\preceq t', \Gamma'\Seq\Delta' \end{array} \right .
 \\ \cline{1-1}
y\preceq t', \Gamma,\Gamma' \Seq \Delta,\Delta' &   \raisebox{1.2ex}[1.5ex][0ex]{($cut$)}
\end{array} \] }
%  \) }} \\[.5ex]
The obtained application of ($\preceq_{sx}$) is admissible by assumption 5. 
(lemma~\ref{le:noInclsx}), and so
($cut$) can be eliminated by induction hypothesis on $\#(\preceq_a,D_2)$.
%
\end{LSB}
%
%
\item\label{it:cutactive} %LSA 
 $\phi$ is active in $R'$
\begin{LSB}
\item\label{it:inact} $\phi$ is inclusion $s(t)\Incl p$ (active in $R'$),
\\
%  {\sf III.a) 
that is $R'$ is either ($\Incl_s$) or ($\Incl_{ar}$).
The latter case is excluded because it would require $s(t)$ to be a variable.
Then we would have a degenerate application of ($E_s$) in $D_1$, what is excluded by the
assumption. So let
 $R'$ be $(\Incl_s$). We have the following derivation: 
{\footnotesize \[ \begin{array}{cl} %\hspace*{-4em}
D_1 \left \{ \begin{array}{cl}
 D_1^*\left\{ \begin{array}{cl}
  \vdots \\ 
  x\Incl t,\Gamma\Seq\Delta, s(x)\Incl p  
 \end{array} \right. \\ \cline{1-1}
\Gamma\Seq\Delta, s(t)\Incl p & \raisebox{1.2ex}[1.5ex][0ex]{($E_s$)}
 \end{array} \right .
 D_2\left\{ \begin{array}{cl}
  D_2^* \left \{ \begin{array}{c}
\vdots \\
s(t)\Incl p, \Gamma'\Seq\Delta', w(p)\preceq q \end{array} \right .
\\ \cline{1-1}  
s(t)\Incl p, \Gamma'\Seq\Delta', w(s(t))\preceq q & \raisebox{1.2ex}[1.5ex][0ex]{($\Incl_s$)}
 \end{array} \right. \\ \cline{1-1}
\Gamma,\Gamma' \Seq \Delta,\Delta', w(s(t))\preceq q
 &   \raisebox{1.2ex}[1.5ex][0ex]{($cut$)}
\end{array} \] }
%} \\[.5ex]
%
First, construct the derivation $M'$ by cutting $s(t)\Incl p$ after $D_1$ and
 $D_2^*$. Since $h(D_2^*)<h(D_2)$ this ($cut$) can be eliminated using the third argument of 
induction. Then extend $M'$ to $M$  as follows:
\[ \hspace*{-1em} M \left \{ \begin{array}{cl} 
  M' \left \{ \begin{array}{cl}
 D_1 \left \{ \begin{array}{c}
 \vdots \\
 \GSD, s(t)\Incl p \end{array} \right .
 D_2^* \left \{ \begin{array}{c}
 \vdots \\
 s(t)\Incl p, \Gamma'\Seq\Delta', w(p)\preceq q \end{array} \right . \\ \cline{1-1}
 \Gamma,\Gamma'\Seq\Delta,\Delta', w(p)\preceq q & \rabove{(cut)} \end{array} \right . \\
 \cline{1-1}
 s(x)\Incl p, \Gamma,\Gamma'\Seq\Delta,\Delta',w(p)\preceq q
 &   \raisebox{1.2ex}[1.5ex][0ex]{($W_a$)} \\ \cline{1-1}
 s(x)\Incl p, \Gamma,\Gamma'\Seq\Delta,\Delta',w(s(x))\preceq q
 &   \raisebox{1.2ex}[1.5ex][0ex]{($\Incl_s$)}
 \end{array} \right . \]
The application of ($cut$) to $M$ with $D_1^*$ can be eliminated
by induction hypothesis $h(D_1^*)<h(D_1)$ -- notice that we are using here
the fact that $\#(\preceq_a,M)$ {\em modifying the cut formula} is not greater
than $\#(\preceq_a,D_2^*)$, even if the total number of arbitrary applications
of ($\preceq_a$) in $M$ may be far greater than in $D_2^*$ (due to
applications in $D_1$).
It yields the following sequent leading
to the desired conclusion by an application of ($E_s$) -- $x$ may be chosen so
that $x\Not\in\Vars(\Gamma',\Delta',w,q$):
\[ \begin{array}{cl}
 x\Incl t, \Gamma, \Gamma' \Seq \Delta, \Delta', w(s(x))\preceq q \\
 \cline{1-1}
 \Gamma, \Gamma' \Seq \Delta, \Delta', w(s(t))\preceq q
 &   \raisebox{1.2ex}[1.5ex][0ex]{($E_s$)}
\end{array} \]
\noindent
%
\item % LSB {\sf III.b) 
 $\phi$ is equality $s(t)= p$ (active in $R'$). \\
 We have three cases for $R'$ which are all treated analogously to
the previous case \ref{it:inact}.
% (III.a).
\begin{LSC}
\item $R'$ is ($=_s$) \\
This is treated exactly as \ref{it:inact} with applications of
($=_s$) instead of ($\Incl_s$). 
%
\item  $R'$ is ($=_{Lar}$) or ($=_{Rar}$)\\
We proceed as above with the construction of $M'$, $M$ and ($cut$). The
differences occur only in the last step, so we make the following generic
description where $\psi$ is the side and $\psi'$ the modified formula
of $R'$: 
% \\[.5ex]  \hspace*{-.5em} \mbox{ 
{\footnotesize
\[ \begin{array}{cl}  \hspace*{-1.5em}
D_1 \left \{ \begin{array}{cl}
 D_1^*\left\{ \begin{array}{cl}
  \vdots \\ 
  x\Incl t,\Gamma\Seq\Delta, s(x)= p  
 \end{array} \right. \\ \cline{1-1}
\Gamma\Seq\Delta, s(t)= p & \raisebox{1.2ex}[1.5ex][0ex]{($E_s$)}
 \end{array} \right .
 D_2\left\{ \begin{array}{cl}
  D_2^* \left \{ \begin{array}{c}
\vdots \\
s(t)= p, \psi, \Gamma'\Seq\Delta' \end{array} \right .
\\ \cline{1-1}  
s(t)= p, \psi', \Gamma'\Seq\Delta',  & \raisebox{1.2ex}[1.5ex][0ex]{($R'$)}
 \end{array} \right. \\ \cline{1-1}
\psi', \Gamma,\Gamma' \Seq \Delta,\Delta'
 &   \raisebox{1.2ex}[1.5ex][0ex]{($cut$)}
\end{array} \] }
% } } \\[.5ex] \noindent
First, construct the derivation $M'$ by cutting $s(t)=p$ after $D_1$ and
 $D_2^*$ which is weakened with $s(x)=p$. $h(D_2^*)<h(D_2)$ means that this ($cut$)
 can be eliminated. This
yields the following sequent
\[  s(x)=p, \psi, \Gamma, \Gamma' \Seq \Delta, \Delta' \]
leading to the desired conclusion by the procedure depending on
 $R'$. (Remember that $x$ may be chosen so
that $x\Not\in\Vars(\Gamma',\Delta',\psi$), and $\#(\preceq_a,M')$ is not
greater than in $D_2^*$):
\begin{LSD}
\item $R'$ is ($=_{Lar}$)\\
 $s(t)$ cannot be a variable (since then ($E_s$) in $D_1$ would be
 degenerate), so $p$ must be a variable $y$ and $\psi$ is $f(y)\Incl q$, and $\psi'$ is $f(s(t))\Incl q$.
\[ \begin{array}{cl}
D_1^* \left \{ \begin{array}{c} \vdots \\ 
   x \Incl t, \GSD, s(x)=y \end{array} \right . \ \ \ \ \ 
\begin{array}{rl}
 s(x)=y, f(y)\Incl q, \Gamma, \Gamma' \Seq \Delta, \Delta' \\
 \cline{1-1}
 s(x)=y, f(s(x))\Incl q, \Gamma, \Gamma' \Seq \Delta, \Delta' 
 &   \raisebox{1.2ex}[1.5ex][0ex]{($=_{Lar}$)}  
 \end{array} \\ \cline{1-1}
x\Incl t, f(s(x))\Incl q, \Gamma, \Gamma' \Seq \Delta, \Delta' 
 &   \raisebox{1.2ex}[1.5ex][0ex]{($cut$)} \\ \cline{1-1}
x\Incl t, f(s(t))\Incl q, \Gamma, \Gamma' \Seq \Delta, \Delta' 
 &   \raisebox{1.2ex}[1.5ex][0ex]{($\Incl_{ar}$)} \\ \cline{1-1}
f(s(t))\Incl q, \Gamma, \Gamma' \Seq \Delta, \Delta' 
 &   \raisebox{1.2ex}[1.5ex][0ex]{($E_s$)}
\end{array} \]
% \[ \begin{array}{rl}
%  s(x)=y, f(y)\Incl q, \Gamma, \Gamma' \Seq \Delta, \Delta' \\
%  \cline{1-1}
%  s(x)=y, f(s(x))\Incl q, \Gamma, \Gamma' \Seq \Delta, \Delta' 
%  &   \raisebox{1.2ex}[1.5ex][0ex]{($=_{Lar}$)} \\ \cline{1-1}
% x\Incl t, f(s(x))\Incl q, \Gamma, \Gamma' \Seq \Delta, \Delta' 
%  &   \raisebox{1.2ex}[1.5ex][0ex]{($cut$)\ with\ $s(x)=y$\ after\ $D_1^*$} \\ \cline{1-1}
% x\Incl t, f(s(t))\Incl q, \Gamma, \Gamma' \Seq \Delta, \Delta' 
%  &   \raisebox{1.2ex}[1.5ex][0ex]{($\Incl_a$)} \\ \cline{1-1}
% f(s(t))\Incl q, \Gamma, \Gamma' \Seq \Delta, \Delta' 
%  &   \raisebox{1.2ex}[1.5ex][0ex]{($E_s$)}
% \end{array} \]
This ($cut$) can be eliminated since $h(D_1^*)<h(D_1)$.
%
\item $R'$ is ($=_{Rar}$)\\
substituting $s(t)$ for $p$, i.e., $\psi$ is $y\Incl f(p)$, and $\psi'$ is
$y\Incl f(s(t))$ for a $y\in\Vars$:
\[ \begin{array}{cl}
D_1^* \left \{ \begin{array}{c} \vdots \\ 
 x\Incl t, \GSD, s(x)=p \end{array} \right . \ \ \ \ \ \ 
\begin{array}{rl}
 s(x)=p, y\Incl f(p), \Gamma, \Gamma' \Seq \Delta, \Delta' \\
 \cline{1-1}
 s(x)=p, y\Incl f(s(x)), \Gamma, \Gamma' \Seq \Delta, \Delta' 
 &   \raisebox{1.2ex}[1.5ex][0ex]{($=_{Rar}$)} 
 \end{array} \\ \cline{1-1}
x\Incl t, y\Incl f(s(x)), \Gamma, \Gamma' \Seq \Delta, \Delta' 
 &   \raisebox{1.2ex}[1.5ex][0ex]{($cut$)} \\ \cline{1-1}
y\Incl f(s(t)), \Gamma, \Gamma' \Seq \Delta, \Delta' 
 &   \raisebox{1.2ex}[1.5ex][0ex]{($E_a$)}
\end{array} \]
This ($cut$) can be eliminated since $h(D_1^*)<h(D_1)$.
%
\item $R'$ is ($=_{Rar}$)\\
substituting $p$ for $s(t)$, i.e., $\psi$ is $y\Incl f(s(t))$, and $\psi'$ is
 $y\Incl f(p)$ for a $y\in\Vars$. Here we have to weaken $D_2^*$ also with 
$f(s(x))\Incl f(s(t))$:
{\footnotesize
\[ \begin{array}{cl} \hspace*{-2em}
D_1^* \left \{ \begin{array}{c} \vdots \\ 
 x\Incl t, \GSD, s(x)=p \end{array} \right . \ \ \ \ 
\begin{array}{rl}
f(s(x))\Incl f(s(t)), s(x)=p, y\Incl f(s(t)), \Gamma, \Gamma' \Seq \Delta, \Delta' \\
 \cline{1-1}
f(s(x))\Incl f(s(t)), s(x)=p, y\Incl f(s(x)), \Gamma, \Gamma' \Seq \Delta, \Delta' 
 &   \raisebox{1.2ex}[1.5ex][0ex]{($\Incl_a^*$)} \\ \cline{1-1}
f(s(x))\Incl f(s(t)), s(x)=p, y\Incl f(p), \Gamma, \Gamma' \Seq \Delta, \Delta' 
 &   \raisebox{1.2ex}[1.5ex][0ex]{($=_{Rar}$)} 
 \end{array} \\ \cline{1-1}
x\Incl t, f(s(x))\Incl f(s(t)), y\Incl f(p), \Gamma, \Gamma' \Seq \Delta, \Delta' 
 &   \raisebox{1.2ex}[1.5ex][0ex]{($cut$)} \\ \cline{1-1}
x\Incl t, f(s(t))\Incl f(s(t)), y\Incl f(p), \Gamma, \Gamma' \Seq \Delta, \Delta' 
 &   \raisebox{1.2ex}[1.5ex][0ex]{($\Incl_{ar}$)} \\ \cline{1-1}
x\Incl t, y\Incl f(p), \Gamma, \Gamma' \Seq \Delta, \Delta' 
 &   \raisebox{1.2ex}[1.5ex][0ex]{($cut_t$)} \\ \cline{1-1}
y\Incl f(p), \Gamma, \Gamma' \Seq \Delta, \Delta' 
 &   \raisebox{1.2ex}[1.5ex][0ex]{($E_s$)}
\end{array} \]
}
Again, ($cut$) can be eliminated since $h(D_1^*)<h(D_1)$.
By the same argument as in the case \ref{it:Ar}, $s(x)\not\in\Vars$ and
hence also $f(s(x))\not\in\Vars$ and $f(s(t))\not\in\Vars$, 
so we can apply ($\Incl_a^*$).
\end{LSD}
\end{LSC}
\end{LSB}
\end{LSA}
\end{LS}
\end{PROOF}

%
\noindent
The proposition yields the problematic implication of the 
\begin{THEOREM}\label{th:neq4cisneq4}
 $NEQ_4^c \equiv NEQ_4$ \end{THEOREM}
Combined with proposition~\ref{le:neq3isneq4} (and the equivalences 
from propositions~\ref{le:neqisneq1}, \ref{le:neq1isneq2} and \ref{le:neq2isneq3})
we have also:
\begin{COROLLARY}\label{co:Allequiv}
$NEQ\equiv NEQ_3^c \equiv NEQ_4^c \equiv NEQ_4 \equiv NEQ_3$.
\end{COROLLARY}



\section{The final cut-free calculus $NEQ_5$}\label{se:lastNEQ}
Although the elementary cut rules
%\reff{ru:cutx} from definition \ref{de:neq4}
are relatively innocent,
we make the final step showing that they are redundant. 
We do it in two steps. First, in \ref{sub:onecut}, we remove the rule $(cut_x)$ 
obtaining $NEQ'_5$ which is equivalent to $NEQ_4$.
This step isn't really problematic.
The rule $(cut_{x=})$, however, expresses the specific character of variables as
compared to arbitrary other terms. In \ref{sub:nocut} we show that it can be
replaced by two other rules which capture this specificity of variables, thus
obtaining the final cut-free version of the calculus $NEQ_5$.

%
\subsection{$NEQ'_5$ -- the calculus with one elementary cut}\label{sub:onecut}
\begin{DEFINITION}\label{de:neq5}
 $NEQ'_5$ is obtained from $NEQ_4$ by 
\begin{itemize}\MyLPar
\item replacing the axiom $x\Incl t,\GSD,x\Incl t$ by $\GSD,t\Incl t$;
%\item adding axiom $x\Incl y,\GSD, y\Incl x$ for $x,y\in\Vars$
\item admitting non-reduced applications of $\eLanr$ but \\
 removing the degenerate applications $\eRax$ and $\eLax$ which result in a single 
variable; % and $\iLax$;
\item removing the elementary cut rule ($cut_x$) and $\eRs$;
%\item imposing additional restrictions on $\eLa$ and $\eRa$
\end{itemize}
\end{DEFINITION}
\noindent
The rules of $NEQ'_5$ are given in figure \ref{fi:neq5}.
\begin{figure}[hbt]
\hspace*{3em}\begin{tabular}{|l@{\ \ \ \ \ \ \ \ \ \ \ \ }ll|}
\hline
\multicolumn{2}{|c}{{\bf Axioms}:} & \\[1ex]
\multicolumn{3}{|c|}{$\GSD, t\Incl t$\ \ \ \ \ \ $\GSD, x=x$\ \ \
\ \ \  %$x\Incl y,\GSD, y\Incl x$\ \ \ \ \ \ 
$s=t,\GSD, s=t$ }\\[2ex]
%
\multicolumn{2}{|c}{{\bf Identity rules}:} & \\[1ex]
$\eLs$\ \prule{t=s,\GSD, p(s)\preceq q}{t=s,\GSD, p(t)\preceq q} 
& 
$\eLanx$\ \prule{s=t, p(s)\Incl q,\GSD}{s=t, p(t)\Incl q,\GSD} 
       \ \ {\footnotesize{$p(t)\not\in\Vars$}} 
& \\[2.5ex]
 & 
%$\eRs$\ \prule{s=t, \GSD, p\Incl q(s)}{s=t,\GSD, p\Incl q(t)}   & 
 $\eRanx$\ \prule{s=t, p\Incl q(s), \GSD}{s=t, p\Incl q(t),\GSD} 
 \ \ {\footnotesize{$q(t)\not\in\Vars$}} 
 & 
\\[2ex]
%
\multicolumn{2}{|c}{{\bf Inclusion rules}:}
& \\[1ex]
 $\iLs$\ \prule{t\Incl s, \Gamma\Seq \Delta, p(s)\preceq q}{t\Incl s, \Gamma\Seq
\Delta, p(t)\preceq q} & 
$\iLa$\ \prule{s\Incl t, p(s)\preceq q, \GSD}{s\Incl t, p(t)\preceq q, \GSD} & \\[2ex]
%       \ \ {\footnotesize{$t\not\in\Vars$}} & \\[2ex]
%
 $\iRs$\ \prule{s\Incl t, \GSD, p\Incl q(s)}{s\Incl t,\GSD, p\Incl q(t)} & 
%% $p(t)\not\in\Vars$ & 
& \\[2ex]
\multicolumn{2}{|c}{{\bf Elimination rules}:} & \\[1ex]
 $\Esr$\ \prule{\Gamma, x\Incl t\Seq\Delta,\phi[x]} 
  {\Gamma\Seq\Delta,\phi[t]}  & 
\multicolumn{2}{l|}{ $\Ear$\ \prule{x\Incl t, y\Incl r(x), \Gamma\Seq\Delta}
  {y\Incl r(t),\Gamma\Seq\Delta} } %($E_{ar}^*$)} 
   \\[.5ex]
{\footnotesize \ \ \ - $x\not\in \Vars(\Gamma,\Delta,t),\ t\not\in\Vars$}
    &  \multicolumn{2}{l|}{{\footnotesize \ \ \
 - $x\not\in \Vars(t,\Gamma,\Delta,y),\ t\not\in\Vars$}} \\
 {\footnotesize \ \ \ - at most one $x$ in $\phi$;} 
  &  \multicolumn{2}{l|}{{\footnotesize \ \
 \ - at most one occurrence of $x$ in $r$ }} \\[2ex]
%
\multicolumn{3}{|c|}{{\bf Elementary cut}:}\\
%\multicolumn{3}{|c|}{
$(cut_{x=})$\ \ \prule{x=x,\GSD}{\GSD} &&  %}
\\[2ex]
\multicolumn{3}{|c|}{{\bf Specific cut rules}:}\\
\multicolumn{3}{|c|}
{for each specific axiom $\Ax_k$: \(a_1,...,a_n\Seq s_1,...,s_m\), 
a  rule:}\\[1ex]
\multicolumn{3}{|c|}
{\prule{\Gamma\Seq\Delta,a_1\ ;...;\ \Gamma\Seq\Delta,a_n\ ;\ 
s_1,\Gamma\Seq\Delta\ ;...;\ s_m,\Gamma\Seq\Delta} 
{\Gamma\Seq\Delta}\ \ \ ($Sp.cut_k$)} \\
 \hline
\end{tabular} 
\caption{The calculus $NEQ'_5$ ($x,y\in\Vars$)}\label{fi:neq5}
\end{figure}
%
\begin{LEMMA}\label{le:neq5toneq4}
$NEQ'_5\impl NEQ_4$.
\end{LEMMA}
\begin{PROOF}
Obviously, the axiom $\GSD, t\Incl t$ is derivable in $NEQ_4$ by choosing a fresh
variable in the axiom $x\Incl t,\GSD, x\Incl t$ and applying $(E_{sr})$. 
If $t$ is a variable $x\in\Vars$,
then one starts with $x\Incl x,\GSD, x\Incl x$ and applies $(cut_x)$.
%%% Another new axiom is derivable in $NEQ_4$ as follows:
%%% \[\prarc{
%%% x=y, x\Incl y, \GSD, x\Incl y \cl
%%% x=y, x\Incl y, \GSD, y\Incl y & \rabove{\eLs} \cl
%%% x=y, x\Incl y, \GSD, y\Incl x & \rabove{\eRs} \cl
%%% y=y, x\Incl y, \GSD, y\Incl x & \rabove{\iLa} \cl
%%%      x\Incl y, \GSD, y\Incl x & \rabove{(cut_{x=})}
%%% }
%%% \]
The non-reduced applications $\eLanr$ are admissible in $NEQ_4$ by lemma~\ref{le:noLanr}.
\end{PROOF}

\noindent
For the other implication $NEQ_4\impl NEQ'_5$ we show admissibility in $NEQ'_5$
of the missing rules: %. By corollary~\ref{co:noxeq}, we have to show
%admissibility of 
$(cut_x)$ % $(cut_{x=})$ 
and unrestricted $\eLar$ and $\eRa$.

\begin{LEMMA}\label{le:noxeq}
The rule $(cut_x)$ is admissible in $NEQ'_5$.
\end{LEMMA}
\begin{PROOF}
%% By lemma \ref{le:noax} we assume that derivation $D$ above $(cut_x)$ 
%% contains no $\eLarx$ or $\eRax$
%% and proceed by induction on $D$'s height.
Instead of any axiom with $x\Incl x$ in the antecedent, we may
take the one without it. The only rules which might possibly generate this formula in the
antecedent are $\Ear$, $\eLanx$, $\eRanx$ or $\iLa$. 
But since the first three of these rules are non-degenerate, they cannot introduce 
a single variable, 
i.e. they cannot result in a formula $x\Incl x$. Thus the only possibility is the
following:
%% \begin{LS}
%% \item $\Ear$ -- cannot generate $x\Incl x$ due to the
%% restriction that $\Ear$ does not result in a variable.
%% \item $\eLar$ and $\eRa$ are non-degenerate -- $\eLanx$ and $\eRanx$ cannot 
%% introduce a variable resulting in $x\Incl x$.
%% \item $\iLanx$ -- does not introduce a variable either.
%% %would require the following situation
\[\prar{t\Incl x, t\Incl x, \GSD \cl
x\Incl x, t\Incl x, \GSD & \rabove{\iLa}
}
\]
But the resulting sequent is the same as the premise weakened with $x\Incl x$.
%% \end{LS}
\end{PROOF}

\begin{LEMMA}\label{le:noax}
%$\der{NEQ'_5}DS\impl\der{NEQ'_5}{D^*}S$ and $D^*$ contains no 
$NEQ'_5$ admits degenerate applications of $\eLar$, namely, 
\[
\eLarx\ \ \mprule{y=x, y\Incl t,\GSD}{y=x, x\Incl t,\GSD}\ \ \ \ \ \ \ \ \ \ \ 
%\iLax\ \ \mprule{s\Incl x, s\preceq t,\GSD}{s\Incl x, x\preceq t,\GSD}\ \ \ \ 
%\eRax\ \ \mprule{t=x, s\Incl t,\GSD}{t=x, s\Incl x,\GSD}\ \ \ \ 
 y,x\in\Vars
\]
\end{LEMMA}
\begin{PROOF}
By %simultaneous 
induction on the number of applications of the rule and the height
of derivation. The rules is not applicable to any axiomatic formula, so
%% \[\prarc{
%% y=z, y\Incl x,\GSD, x\Incl y \cl
%% y=z, z\Incl x,\GSD, x\Incl y & \rabove{\eLarx}
%% }\conv
%% \prarc{
%% y=z, z\Incl x,\GSD, x\Incl z \cl
%% y=z, z\Incl x,\GSD, x\Incl y & \rabove{\eRs}
%% }
%% \]
consider the rule applied above.
%We consider first the case of when the uppermost application was of $\eLarx$.
\begin{LS}
\item $\eLanx$ -- the two applications cannot interact and can be swapped %for $\eLarx$ 
since the subsequent application $\eLarx$ is 
reduced, i.e.,
it replaces a variable which couldn't have been introduced just above it by a 
non-degenerate application of $\eLanx$. 
\item $\iLanx$ -- two cases 
%% can be swapped since their interaction with the
%% subsequent $\eLarx$ would require
%% a degenerate application $\iLax$
%% \[\prar{
%%  x=y, s\Incl y, s\Incl q,\GSD \cl
%%  x=y, s\Incl y, y\Incl q,\GSD & \rabove{\iLax} \cl
%%  x=y, s\Incl y, x\Incl q,\GSD & \rabove{\eLarx}
%%  }\ \ \ \ \ \ \ 
%% \prar{
%%  x=s, s\Incl y, y\Incl q, \GSD \cl
%%  x=y, s\Incl y, y\Incl q, \GSD & \rabove{\iLax} \cl
%%  x=y, s\Incl y, x\Incl q, \GSD & \rabove{\eLarx}
%%  }
%% \]
\[\sma{ \prar{
  x=y, s\Incl y, s\Incl q,\GSD \cl
  x=y, s\Incl y, y\Incl q,\GSD & \rabove{\iLa} \cl
  x=y, s\Incl y, x\Incl q,\GSD & \rabove{\eLarx}
}\conv
\prar{
  x=s, x=y, s\Incl y, s\Incl q,\GSD \cl
  x=s, x=y, s\Incl y, x\Incl q,\GSD & \rabove{\eLarx} \cl
  x=y, x=y, s\Incl y, x\Incl q,\GSD & \rabove{\iLa} 
} }
\]
Analogous weakening when $\iLa$ modifies the formula 
active in the application of $\eLarx$;
\[\prarc{
 x=s, s\Incl y, y\Incl q, \GSD \cl
 x=y, s\Incl y, y\Incl q, \GSD & \rabove{\iLa} \cl
 x=y, s\Incl y, x\Incl q, \GSD & \rabove{\eLarx}
 }\conv
\prarc{
  x=y, x=s, s\Incl y, y\Incl q, \GSD \cl
  x=y, x=s, s\Incl y, x\Incl q, \GSD & \rabove{\eLarx} \cl
  x=y, x=y, s\Incl y, x\Incl q, \GSD & \rabove{\iLa}
 }
\]
\item 
$\Ear$ -- swapping is possible since $x\in\Vars$
\[\prarc{
z\Incl p, x=y, y\Incl r(z),\GSD \cl
          x=y, y\Incl r(p),\GSD & \rabove{\Ear} \cl
          x=y, x\Incl r(p),\GSD & \rabove{\eLarx} 
}\conv
\prarc{
z\Incl p, x=y, y\Incl r(z),\GSD \cl
z\Incl p, x=y, x\Incl r(z),\GSD & \rabove{\eLarx} \cl
          x=y, x\Incl r(p),\GSD & \rabove{\Ear} 
}
\]
\item $\iLs$ 
\[\prar{
y\Incl t, y=x, \GSD, p(t)\preceq q \cl
y\Incl t, y=x, \GSD, p(y)\preceq q & \rabove{\iLs} \cl
x\Incl t, y=x, \GSD, p(y)\preceq q & \rabove{\eLarx} 
}\conv
\prar{
y\Incl t, y=x, \GSD, p(t)\preceq q \cl
x\Incl t, y=x, \GSD, p(t)\preceq q & \rabove{\eLarx} \cl
x\Incl t, y=x, \GSD, p(x)\preceq q & \rabove{\iLs} \cl
x\Incl t, y=x, \GSD, p(y)\preceq q & \rabove{\eLs} 
}
\]
\item $\iRs$ -- is treated analogously to $\iLs$.
\item Neither active nor modified formulae of $\eLs$ %$\eRs$ 
or $\Esr$ can be
affected by $\eLarx$, so these, as well as $(Sp.cut)$,
allow trivial swap with $\eLarx$.
$\eRa$ modifies at most the RHS of inclusion modified then by $\eLarx$, so these 
two can be swapped, too.
\end{LS}
% As we can see, no applications of $\eRax$ appear in the transformed derivations.
%
\end{PROOF}
%
To show admissibility of the unrestricted $\eRa$ we need the following lemma.
\begin{LEMMA}\label{le:yeseae}
The rule $\eae$ is admissible in $NEQ'_5$
\[ \eae\ \ \mprule{s=t, p(s)=q,\GSD}{s=t,p(t)=q,\GSD}\ \ p(t)\not\in\Vars \]
\end{LEMMA}
\begin{PROOF}
Application to an axiom is transformed as follows:
\[\prarc{
s=t, p(s)=q,\GSD, p(s)=q \cl
s=t, p(t)=q,\GSD, p(s)=q & \rabove{\eae}
}\conv
\prarc{
s=t, p(t)=q,\GSD, p(t)=q \cl
s=t, p(t)=q,\GSD, p(s)=q & \rabove{\eLs}
}
\]
The trivial differences from the proof of lemma~\ref{le:noeqeq} concern the presence
of non-reduced version of $\eLa$. In addition this case used there
$(cut_x)$. Here it will be treated as follows:
\begin{LS}
\item $\eLanx$ -- $p(f(s)), f(t)\not\in\Vars$ \vspace*{-1ex}
\[\sma{ \prar{
s=t, f(s)=r, p(r)\Incl q, \GSD \cl
s=t, f(s)=r, p(f(s))\Incl q, \GSD & \rabove{\eLanx} \cl
s=t, f(t)=r, p(f(s))\Incl q, \GSD & \rabove{\eae}
}\conv
\prar{
s=t, f(s)=r, p(r)\Incl q, \GSD \cl
s=t, f(t)=r, p(r)\Incl q, \GSD & \rabove{\eae} \cl
s=t, f(t)=r, p(f(t))\Incl q, \GSD & \rabove{\eLanx} \cl
s=t, f(t)=r, p(f(s))\Incl q, \GSD & \rabove{\eLanx} 
} } %\vspace*{-2ex}
\]
The first application is $\eLanx$ because $f(t)\not\in\Vars\impl p(f(t))\not\in\Vars$.

And the other possibility -- $p(r), f(t)\not\in\Vars$
\[ \sma{ \prar{
s=t, f(s)=r, p(f(s))\Incl q, \GSD \cl
s=t, f(s)=r, p(r)\Incl q, \GSD & \rabove{\eLanx} \cl
s=t, f(t)=r, p(r)\Incl q, \GSD & \rabove{\eae} 
}\conv
\prar{
s=t, f(s)=r, p(f(s))\Incl q, \GSD \cl
s=t, f(t)=r, p(f(s))\Incl q, \GSD & \rabove{\eae} \cl
s=t, f(t)=r, p(f(t))\Incl q, \GSD & \rabove{\eLanx} \cl
s=t, f(t)=r, p(r)\Incl q, \GSD & \rabove{\eLanx} 
} }  %\vspace*{-4ex}
\] 
Again, the first application is $\eLanx$ because 
$f(t)\not\in\Vars\impl p(f(t))\not\in\Vars$.
%
\item $\iLa$ -- $p(t)\not\in\Vars$
\[\sma{\prarc{
s\Incl t, p(s)=q, p(t)=r, \GSD \cl
s\Incl t, p(t)=q, p(t)=r, \GSD & \rabove{\iLa} \cl
s\Incl t, r=q, p(t)=r, \GSD & \rabove{\eae} 
}\conv
\prarc{
p(s)= r, s\Incl t, p(s)=q, p(t)=r, \GSD \cl
p(s)= r, s\Incl t, p(s)=q, p(t)=r, \GSD & \rabove{\eae} \cl
p(t)= r, s\Incl t, p(s)=q, p(t)=r, \GSD & \rabove{\iLa} 
} }
\]
Furthermore, we have the case when $t(k)\not\in\Vars$ which needs $\eRanx$
\[\sma{ \prarc{
s\Incl t(k), p(s)=q, k=r, \GSD \cl
s\Incl t(k), p(t(k))=q, k=r, \GSD & \rabove{\iLa} \cl
s\Incl t(k), p(t(r))=q, k=r, \GSD & \rabove{\eae}
}\conv
\prarc{
s\Incl t(r), s\Incl t(k), p(s)=q, k=r, \GSD \cl
s\Incl t(r), s\Incl t(k), p(t(r))=q, k=r, \GSD & \rabove{\eae} \cl
s\Incl t(k), s\Incl t(k), p(t(r))=q, k=r, \GSD & \rabove{\eRanx}
} }
\]
If $t(k)\in\Vars$, i.e. it is an $x$ we have the following situation with 
$p(r)\not\in\Vars$:
\[\prarc{
s\Incl x, p(s)=q, x=r, \GSD \cl
s\Incl x, p(x)=q, x=r, \GSD & \rabove{\iLa} \cl
s\Incl x, p(r)=q, x=r, \GSD & \rabove{\eae}
}\conv
\prarc{
s\Incl x, p(s)=q, x=r, s=r, \GSD \cl
s\Incl x, p(r)=q, x=r, s=r, \GSD & \rabove{\eae} \cl
s\Incl x, p(r)=q, x=r, x=r, \GSD & \rabove{\iLa}
}
\]
\item $\eRanx$ -- $q(t), f(r)\not\in\Vars$
\[\sma{ \prar{
t=s, p\Incl f(q(s)), q(s)=r, \GSD \cl
t=s, p\Incl f(r), q(s)=r, \GSD & \rabove{\eRanx} \cl
t=s, p\Incl f(r), q(t)=r, \GSD & \rabove{\eae} 
}\conv
\prar{
t=s, p\Incl f(q(s)), q(s)=r, \GSD \cl
t=s, p\Incl f(q(s)), q(t)=r, \GSD & \rabove{\eae} \cl
t=s, p\Incl f(q(t)), q(t)=r, \GSD & \rabove{\eRanx} \cl
t=s, p\Incl f(r), q(t)=r, \GSD & \rabove{\eRanx}
} }
\]
The first resulting application is $\eRanx$ since 
$q(t)\not\in\Vars\impl f(q(t))\not\in\Vars$.
Analogous transformation is applied when $\eae$ modifies $r$ rather than $q(s)$.
%% \item $\eRs$
%% \[\prar{
%% s=t, q(s)=r, \GSD, p\Incl f(q(s)) \cl
%% s=t, q(s)=r, \GSD, p\Incl f(r)  & \rabove{\eRs} \cl
%% s=t, q(t)=r, \GSD, p\Incl f(r)  & \rabove{\eae} 
%% }\conv
%% \prar{
%% s=t, q(s)=r, \GSD, p\Incl f(q(s)) \cl
%% s=t, q(t)=r, \GSD, p\Incl f(q(s)) & \rabove{\eae} \cl
%% s=t, q(t)=r, \GSD, p\Incl f(q(t)) & \rabove{\eRs} \cl
%% s=t, q(t)=r, \GSD, p\Incl f(r) & \rabove{\eRs} 
%% }
%% \]
%% Another case is analogous
%% \[\prar{
%% s=t, q(s)=r, \GSD, p\Incl f(r) \cl
%% s=t, q(s)=r, \GSD, p\Incl f(q(s)) & \rabove{\eRs} \cl
%% s=t, q(t)=r, \GSD, p\Incl f(q(s)) & \rabove{\eae} 
%% }\conv
%% \prar{
%% s=t, q(s)=r, \GSD, p\Incl f(r) \cl
%% s=t, q(t)=r, \GSD, p\Incl f(r) & \rabove{\eae} \cl
%% s=t, q(t)=r, \GSD, p\Incl f(q(t)) & \rabove{\eRs} \cl
%% s=t, q(t)=r, \GSD, p\Incl f(q(s)) & \rabove{\eRs} 
%% }
%% \]
\item $\eLs$ %-- the cases here are entirely analogous to those for $\eRs$.
\[\prar{
s=t, q(s)=r, \GSD,  f(q(s))\Incl p \cl
 s=t, q(s)=r, \GSD, f(r)\Incl p  & \rabove{\eLs} \cl
 s=t, q(t)=r, \GSD, f(r)\Incl p  & \rabove{\eae} 
}\conv
\prar{
 s=t, q(s)=r, \GSD, f(q(s))\Incl p \cl
 s=t, q(t)=r, \GSD, f(q(s))\Incl p & \rabove{\eae} \cl
 s=t, q(t)=r, \GSD, f(q(t))\Incl p & \rabove{\eLs} \cl
 s=t, q(t)=r, \GSD, f(r)\Incl p & \rabove{\eLs} 
}
\]
Another case is analogous
\[\prar{
 s=t, q(s)=r, \GSD, f(r)\Incl p \cl
 s=t, q(s)=r, \GSD, f(q(s))\Incl p & \rabove{\eLs} \cl
 s=t, q(t)=r, \GSD, f(q(s))\Incl p & \rabove{\eae} 
}\conv
\prar{
 s=t, q(s)=r, \GSD, f(r)\Incl p \cl
 s=t, q(t)=r, \GSD, f(r)\Incl p & \rabove{\eae} \cl
 s=t, q(t)=r, \GSD, f(q(t))\Incl p & \rabove{\eLs} \cl
 s=t, q(t)=r, \GSD, f(q(s))\Incl p & \rabove{\eLs} 
}
\]
\item Other cases are trivial. The active formula of $\iLs$ and $\iRs$ is an inclusion
and, similarly, elimination rules involve only inclusions. $(Sp.cut)$ can be trivially
swapped.
\end{LS}
\end{PROOF}
%
\begin{LEMMA}\label{le:eRax}
%$\der{NEQ'_5}DS\impl\der{NEQ'_5}{D^*}S$ and $D^*$ contains no 
$NEQ'_5$ admits degenerate applications of $\eRa$, namely, 
\[
\eRax\ \ \mprule{t=x, s\Incl t,\GSD}{t=x, s\Incl x,\GSD}\ \ \ \ 
 x\in\Vars
\]
\end{LEMMA}
\begin{PROOF}
%% Application to an axiomatic formula is treated as follows:
%% \[\prarc{
%% x=z, y\Incl x, \GSD, x\Incl y \cl
%% x=z, y\Incl z, \GSD, x\Incl y & \rabove{\eRax}
%% }\conv
%% \prarc{
%% x=z, y\Incl z, \GSD, x\Incl y \cl
%% x=z, y\Incl z, \GSD, x\Incl y & \rabove{\eLs}
%% }
%% \]
The rule isn't applicable to any axiomatic formula, so we consider the
rule applied just above.
\begin{LS}
\item $\eRanx$ -- $q(t)\not\in\Vars$
\[\sma{ \prarc{
s=t, q(t)=x, p\Incl q(s), \GSD \cl
s=t, q(t)=x, p\Incl q(t), \GSD & \rabove{\eRanx} \cl
s=t, q(t)=x, p\Incl x, \GSD & \rabove{\eRax} 
}\conv
\prarc{
q(s)= x, s=t, q(t)=x, p\Incl q(s), \GSD \cl
q(s)= x, s=t, q(t)=x, p\Incl x, \GSD & \rabove{\eRax} \cl
q(t)= x, s=t, q(t)=x, p\Incl x, \GSD & \rabove{\eae}
} }
\]
Since $q(t)\not\in\Vars$, the rule $\eae$ is admissible by lemma~\ref{le:yeseae}.
\item $\iLa$  %-- $t\not\in\Vars$
\[\sma{ \prarc{
x= p(s), s\Incl t, q\Incl p(t), \GSD \cl % 
x= p(t), s\Incl t, q\Incl p(t), \GSD & \rabove{\iLa} \cl % 
x= p(t), s\Incl t, q\Incl x, \GSD & \rabove{\eRax}
}\conv
\prarc{
x=p(t), x= p(s), s\Incl t, q\Incl p(t), \GSD \cl
x=p(t), x= p(s), s\Incl t, q\Incl x, \GSD & \rabove{\eRax} \cl
x=p(t), x= p(t), s\Incl t, q\Incl x, \GSD & \rabove{\iLa} 
} }
\]
Similar weakening when $\iLanx$ introduces the subsequently substituted variable:
\[\prarc{
r=p, r\Incl x, q\Incl p, \GSD \cl
x=p, r\Incl x, q\Incl p, \GSD & \rabove{\iLa} \cl
x=p, r\Incl x, q\Incl x, \GSD & \rabove{\eRax}
}\conv
\prarc{
x=p, r=p, r\Incl x, q\Incl p, \GSD \cl
x=p, r=p, r\Incl x, q\Incl x, \GSD & \rabove{\eRax} \cl
r=p, r=p, r\Incl x, q\Incl x, \GSD & \rabove{\iLa}
}
\]
Interaction of both formulae gives two cases; the first with $t\not\in\Vars$
\[\prar{
x=p, p\Incl t, \GSD \cl
x=t, p\Incl t, \GSD  & \rabove{\iLa} \cl
x=t, p\Incl x, \GSD  & \rabove{\eRax}
}\conv
\prar{
x=t, x=p, p\Incl t, \GSD \cl
x=t, x=p, p\Incl x, \GSD & \rabove{\eRax} \cl
x=t, x=x, p\Incl x, \GSD & \rabove{\iLa} \cl
x=t, x=t, p\Incl x, \GSD & \rabove{\eae} 
}
\]
where $\eae$ is admissible by lemma~\ref{le:yeseae} since $t\not\in\Vars$. 
%(It does not introduce any applications of $\eRanx$.)
If $t$ is an $y\in\Vars$
\[\prar{
x=p, p\Incl y, \GSD \cl
x=y, p\Incl y, \GSD & \rabove{\iLa} \cl
x=y, p\Incl x, \GSD & \rabove{\eRax}
}\conv
\prar{
p\Incl y, x=p, p\Incl x, \GSD \cl
p\Incl y, x=y, p\Incl x, \GSD  & \rabove{\iLa} \cl
x\Incl y, x=y, p\Incl x, \GSD  & \rabove{\iLa} \cl
y\Incl y, x=y, p\Incl x, \GSD  & \rabove{\eLax} \cl
          x=y, p\Incl x, \GSD  & \rabove{(cut_x)}
}
\]
The rule $\eLax$ is admissible by lemma \ref{le:noax} and $(cut_x)$ by \ref{le:noxeq}.
%-- both without introducing new applications of $\eRa$.
\item $\iLs$
\[\sma{ \prar{
x=q, p\Incl q, \GSD, r(q))\preceq s \cl
x=q, p\Incl q, \GSD, r(p)\preceq s & \rabove{\iLs} \cl
x=q, p\Incl x, \GSD, r(p)\preceq s & \rabove{\eRax}
}\conv
\prar{
x=q, p\Incl q, \GSD, r(q)\preceq s \cl
x=q, p\Incl x, \GSD, r(q)\preceq s & \rabove{\eRax} \cl
x=q, p\Incl x, \GSD, r(x)\preceq s & \rabove{\eLs} \cl
x=q, p\Incl x, \GSD, r(p)\preceq s & \rabove{\iLs} 
} }
\]
\item $\iRs$ is treated analogously with applications of $\iRs$ and $\eRs$.
\item\label{case:Ear} $\Ear$ -- $t\not\in\Vars$ %two cases
\[\sma{ \prar{
z\Incl t,    r(t)=x, y\Incl r(z), \GSD \cl
             r(t)=x, y\Incl r(t), \GSD & \rabove{\Ear} \cl
             r(t)=x, y\Incl x, \GSD & \rabove{\eRax}
}\conv
\prar{
z\Incl t,  r(z)=x,  r(t)=x, y\Incl r(z), \GSD \cl
z\Incl t,  r(z)=x,  r(t)=x, y\Incl x, \GSD & \rabove{\eRax} \cl
z\Incl t,  r(t)=x,  r(t)=x, y\Incl x, \GSD & \rabove{\iLa}  \cl
           r(t)=x,  r(t)=x, y\Incl x, \GSD & \rabove{\Ear}
} }
\]
%% was general \eRa
%% \[\sma{ \prar{
%% z\Incl r(t), t=s, y\Incl q(z), \GSD \cl
%%              t=s, y\Incl q(r(t)), \GSD & \rabove{\Ear} \cl
%%              t=s, y\Incl q(r(s)), \GSD & \rabove{\eRa}
%% }\conv
%% \prar{
%% z\Incl r(t), t=s, y\Incl q(z), \GSD \cl
%% z\Incl r(s), t=s, y\Incl q(z), \GSD & \rabove{\eRa} \cl
%%        t=s, y\Incl q(r(s)), \GSD & \rabove{\Ear} 
%% } }
%% \]
%% PROBLEM -- if $r(t)=t$ and $s=x\in\Vars$
\item All other rules allow trivial swap. $\eLanx$ modifies LHS of an inclusion so 
it does not interact with $\eRax$. The latter cannot affect the active equalities
of $\eLs$. % or $\eRs$.
\end{LS}
\end{PROOF}

\noindent
We have thus obtained:
\begin{CLAIM}\label{pr:neq5isneq4}
$NEQ'_5\equiv NEQ_4$.
\end{CLAIM}
\begin{PROOF}
$\impl$ is lemma~\ref{le:neq5toneq4}. For the opposite implication we use
lemma~\ref{fa:noeRs}, i.e. the fact that $NEQ_4\equiv NEQ_4-\eRs$.
%, we 
%use corollary~\ref{co:noxeq}, i.e. assuming the first of the following imlications, 
%we show the second:
%$NEQ_4\impl\ NEQ_4-\{(cut_{x=}),\iLax\}\ \impl NEQ'_5$.
The $NEQ_4$ axiom $x\Incl t,\GSD,x\Incl t$ is derivable in $NEQ'_5$ from the
axiom $x\Incl t,\GSD, t\Incl t$ by a single application of $\iLs$. 
Admissibility of all the missing rules was shown above: 
$(cut_x)$ in lemma~\ref{le:noxeq}, $\eLarx$ in \ref{le:noax} and 
$\eRax$ in \ref{le:eRax}. 
(By lemma~\ref{fa:noeRs} wee don't need to show admissibility of $\eRs$ in $NEQ'_5$.)
%e -- though we did not show it --
%by the proof which is a repetition of the proof of lemma~\ref{fa:noeRs}. (The resulting
%applications of a-rules, which were introduced there, are either now the same as in
%$NEQ'_5$ or else admissible by the preceding lemmas.
\end{PROOF}


\subsection{$NEQ_5$ -- the calculus with no cut}\label{sub:nocut}
We now remove the rule $(cut_{x=})$ and introduce two new 
rules $\Lsix$ and $\Laix$ -- see fig.~\ref{fi:neq51}.


\begin{figure}[hbt]
\hspace*{3em}\begin{tabular}{||l@{\ \ \ \ \ \ \ \ \ \ \ \ }ll||}
\hline\hline
\multicolumn{2}{||c}{{\bf Axioms}:} & \\[1ex]
\multicolumn{3}{||c||}{$\GSD, t\Incl t$\ \ \ \ \ \ $\GSD, x=x$\ \ \
\ \ \  %$x\Incl y,\GSD, y\Incl x$\ \ \ \ \ \ 
$s=t,\GSD, s=t$ }\\[2ex]
%
\multicolumn{2}{||c}{{\bf Identity rules}:} & \\[1ex]
$\eLs$\ \prule{t=s,\GSD, p(s)\preceq q}{t=s,\GSD, p(t)\preceq q} 
& 
$\eLanx$\ \prule{s=t, p(s)\Incl q,\GSD}{s=t, p(t)\Incl q,\GSD} 
       \ \ {\footnotesize{$p(t)\not\in\Vars$}} 
& \\[2.5ex]
 & 
%$\eRs$\ \prule{s=t, \GSD, p\Incl q(s)}{s=t,\GSD, p\Incl q(t)}   & 
 $\eRanx$\ \prule{s=t, p\Incl q(s), \GSD}{s=t, p\Incl q(t),\GSD} 
 \ \ {\footnotesize{$q(t)\not\in\Vars$}} 
 & 
\\[2ex]
%
\multicolumn{2}{||c}{{\bf Inclusion rules}:}
& \\[1ex]
 $\iLs$\ \prule{t\Incl s, \Gamma\Seq \Delta, p(s)\preceq q}{t\Incl s, \Gamma\Seq
\Delta, p(t)\preceq q} & 
$\iLa$\ \prule{s\Incl t, p(s)\preceq q, \GSD}{s\Incl t, p(t)\preceq q, \GSD} & \\[2ex]
%       \ \ {\footnotesize{$t\not\in\Vars$}} & \\[2ex]
%
 $\iRs$\ \prule{s\Incl t, \GSD, p\Incl q(s)}{s\Incl t,\GSD, p\Incl q(t)} & 
%% $p(t)\not\in\Vars$ & 
& \\[3ex]
$\Lsix$\ \prule{t\Incl x,\GSD, p(t)\preceq q}{t\Incl x,\GSD, p(x)\preceq q} & 
$\Laix$\ \prule{t\Incl x, p(x)\Incl q,\GSD}{t\Incl x,p(t)\Incl q,\GSD} 
       \ \ {\footnotesize{$p(t)\not\in\Vars$}} 
& \\[3ex]
\multicolumn{2}{||c}{{\bf Elimination rules}:} & \\[1ex]
 $\Esr$\ \prule{\Gamma, x\Incl t\Seq\Delta,\phi[x]} 
  {\Gamma\Seq\Delta,\phi[t]}  & 
\multicolumn{2}{l||}{ $\Ear$\ \prule{x\Incl t, y\Incl r(x), \Gamma\Seq\Delta}
  {y\Incl r(t),\Gamma\Seq\Delta} } %($E_{ar}^*$)} 
   \\[.5ex]
{\footnotesize \ \ \ - $x\not\in \Vars(\Gamma,\Delta,t),\ t\not\in\Vars$}
    &  \multicolumn{2}{l||}{{\footnotesize \ \ \
 - $x\not\in \Vars(t,\Gamma,\Delta,y),\ t\not\in\Vars$}} \\
 {\footnotesize \ \ \ - at most one $x$ in $\phi$;} 
  &  \multicolumn{2}{l||}{{\footnotesize \ \
 \ - at most one occurrence of $x$ in $r$ }} \\[2ex]
%
\multicolumn{3}{||c||}{{\bf Specific cut rules}:}\\
\multicolumn{3}{||c||}
{for each specific axiom $\Ax_k$: \(a_1,...,a_n\Seq s_1,...,s_m\), 
a  rule:}\\[1ex]
\multicolumn{3}{||c||}
{\prule{\Gamma\Seq\Delta,a_1\ ;...;\ \Gamma\Seq\Delta,a_n\ ;\ 
s_1,\Gamma\Seq\Delta\ ;...;\ s_m,\Gamma\Seq\Delta} 
{\Gamma\Seq\Delta}\ \ \ ($Sp.cut_k$)} \\
 \hline\hline
\end{tabular} 
\caption{The calculus $NEQ_5$ ($x,y\in\Vars$)}\label{fi:neq51}
\end{figure}

\begin{LEMMA}\label{le:primeto5}
Rules $\Lsix$ and $\Laix$ are 
admissible in $NEQ'_5$, i.e., $NEQ_5\impl NEQ'_5$.
\end{LEMMA}
\begin{PROOF}
The activity of these rules can be simulated by the activity
of $t=x$ which, then, can be reduced to $x=x$ and eliminated by $(cut_{x=})$.
The cases for both rules are entirely analogous.
%Thus, 
\[\prar{
\multicolumn{1}{c}{\vdots} \\
t\Incl x, \GSD, p(t)\preceq q \cl
t\Incl x, \GSD, p(x)\preceq q & \rabove{\Lsix\ \ \ \ \ \impl\ \ \ \ \ }
}%\conv
\prar{
\multicolumn{1}{c}{\vdots} \\
t=x, t\Incl x, \GSD, p(t)\preceq q \cl
t=x, t\Incl x, \GSD, p(x)\preceq q & \rabove{\eLs} \cl
x=x, t\Incl x, \GSD, p(x)\preceq q & \rabove{\iLa} \cl
     t\Incl x, \GSD, p(x)\preceq q & \rabove{(cut_{x=})}
}
\]
and analogous case for $\Laix$ -- here $p(t)\not\in\Vars:$
\[\prar{
\multicolumn{1}{c}{\vdots} \\
t\Incl x, p(x)\Incl q,\GSD \cl
t\Incl x, p(t)\Incl q,\GSD & \rabove{\Laix\ \ \ \ \ \impl\ \ \ \ \ }
}%\conv
\prar{
\multicolumn{1}{c}{\vdots} \\
t=x, t\Incl x, p(x)\Incl q, \GSD \cl
t=x, t\Incl x, p(t)\Incl q, \GSD & \rabove{\eLanx} \cl
x=x, t\Incl x, p(t)\Incl q, \GSD & \rabove{\iLa} \cl
     t\Incl x, p(t)\Incl q, \GSD & \rabove{(cut_{x=})}
}
\]
%% and entirely analogous case for $\Rsix$.
%% \[\prar{
%% \multicolumn{1}{c}{\vdots} \\
%% t\Incl x, \GSD, p\Incl q(x) \cl
%% t\Incl x, \GSD, p\Incl q(t) & \rabove{\Rsix\ \ \ \ \ \impl\ \ \ \ \ }
%% }%\conv
%% \prar{
%% \multicolumn{1}{c}{\vdots} \\
%% t=x, t\Incl x, \GSD, p\Incl q(x) \cl
%% t=x, t\Incl x, \GSD, p\Incl q(t) & \rabove{\eRs} \cl
%% x=x, t\Incl x, \GSD, p\Incl q(t) & \rabove{\iLa} \cl
%%      t\Incl x, \GSD, p\Incl q(t) & \rabove{(cut_{x=})}
%% }
%% \]
\end{PROOF}

\noindent
To show admissibility of $(cut_{x=})$ in $NEQ_5$, we will need
\begin{LEMMA}\label{le:xok}
If $\der{NEQ_5}D{x\Incl x,\GSD}$ then $\der{NEQ_5}{D^*}{\GSD}$ and $h(D^*)\leq h(D)$.
\end{LEMMA}
\begin{PROOF}
The argument is exactly the same as in the proof of lemma~\ref{le:noxeq}. The new rule
$\Laix$ cannot generate $x\Incl x$ because of the restriction $p(t)\not\in\Vars$.
\end{PROOF}


\begin{LEMMA}\label{le:noxx}\label{le:nocutx}
Admissibility of $(cut_{x=}):$ 
If $\der{NEQ_5}D{x=x,\GSD}$ then $\der{NEQ_5}{}{\GSD}$.
\end{LEMMA}
\begin{PROOF} 
By induction on $h(D)$. 
Instead of any axiom with $x=x$ in the antecedent, we may
take the one without it. 
So, consider the last rule applied before the appearance
of $x=x$, i.e., when it was the modified formula. 
The only possibility is $\iLa$.
 First we have the case when this rule was applied to an axiom,
and the only problematic case is the following:
%axiom $x=y,\GSD,y\Incl x$, with 
%two possibilities:
\[
\prarc{
x=t, t\Incl x,\GSD,x=t \cl
x=x, t\Incl x,\GSD,x=t& \rabove{\iLa} 
}\conv
\prarc{
t\Incl x,\GSD,x=x  \cl 
t\Incl x,\GSD,x=t & \rabove{\iLs} 
}
\]
(Notice that this case shows that $(cut_{x=})$ and $\iLs$ cannot be removed
simultaneously -- the case 1.3) of the proof of lemma~\ref{le:nosx} needed $(cut_{x=})$
to eliminate $\iLs$.)
Consider the rule applied just above $\iLa$.
\begin{LS}
\item $\iLa$ -- first there are two similar cases
\[\prarc{
x=p(t), p(s)\Incl x, t\Incl s, \GSD \cl
x=p(s), p(s)\Incl x, t\Incl s, \GSD & \rabove{\iLa} \cl
x=x, p(s)\Incl x, t\Incl s, \GSD & \rabove{\iLa} 
}\conv
\prarc{
x=p(t), p(t)\Incl x, p(s)\Incl x, t\Incl s, \GSD \cl
x=x, p(t)\Incl x, p(s)\Incl x, t\Incl s, \GSD & \rabove{\iLa} \cl
x=x, p(s)\Incl x, p(s)\Incl x, t\Incl s, \GSD & \rabove{\iLa} 
}
\]
\[\prarc{
x=p(s), p(t)\Incl x, t\Incl s, \GSD \cl
x=p(s), p(s)\Incl x, t\Incl s, \GSD & \rabove{\iLa} \cl
x=x, p(s)\Incl x, t\Incl s, \GSD & \rabove{\iLa} 
}\conv
\prarc{
x=p(s), p(s)\Incl t, p(t)\Incl x, t\Incl s, \GSD \cl
x=x, p(s)\Incl t, p(t)\Incl x, t\Incl s, \GSD & \rabove{\iLa} \cl
x=x, p(s)\Incl t, p(s)\Incl x, t\Incl s, \GSD & \rabove{\iLa} 
}
\]
Then we have the following case:
\[\prar{
\multicolumn{1}{c}{\vdots} \cl
t=t,t\Incl x,\GSD & \rabove R \cl
x=t,t\Incl x,\GSD & \rabove{\iLa} \cl
x=x,t\Incl x,\GSD & \rabove{\iLa} 
}
\]
We consider the rule $R$ -- the only relevant cases are when it modified a formula
resulting in $t=t$ or $t\Incl x$. 
\begin{LSA}
\item The first case is the axiom:
\[\prar{
t=t, t\Incl x,\GSD, t=t \cl
x=t,t\Incl x,\GSD, t=t & \rabove{\iLa} \cl
x=x,t\Incl x,\GSD, t=t & \rabove{\iLa} 
}\conv
\prar{
t\Incl x, \GSD, x=x \cl
t\Incl x, \GSD, x=t & \rabove{\iLs} \cl
t\Incl x, \GSD, t=t & \rabove{\iLs} 
}
\]
\item $\iLa$ -- two analogously treated cases:
\begin{LSB}
\item modified $t=t$
\[\sma{ \prarc{
t(s)=t(p), p\Incl s, t(s)\Incl x, \GSD \cl
t(s)=t(s), p\Incl s, t(s)\Incl x, \GSD & \rabove{\iLa} \cl
x=t(s), p\Incl s, t(s)\Incl x, \GSD & \rabove{\iLa} \cl
x=x, p\Incl s, t(s)\Incl x, \GSD & \rabove{\iLa} 
}
\conv
\prarc{
t(s)=t(p), p\Incl s, t(s)\Incl x, t(p)\Incl x, \GSD \cl
x=t(p), p\Incl s, t(s)\Incl x, t(p)\Incl x,  \GSD & \rabove{\iLa} \cl
x=x, p\Incl s, t(s)\Incl x, t(p)\Incl x, \GSD & \rabove{\iLa} \cl
x=x, p\Incl s, t(s)\Incl x, t(s)\Incl x, \GSD & \rabove{\iLa} 
} }
\]
\item\label{lst} modified $t\Incl x$
\[\sma{\prarc{
t(s)=t(s), p\Incl s, t(p)\Incl x, \GSD \cl
t(s)=t(s), p\Incl s, t(s)\Incl x, \GSD & \rabove{\iLa} \cl
x =t(s), p\Incl s, t(s)\Incl x, \GSD & \rabove{\iLa} \cl
x =x, p\Incl s, t(s)\Incl x, \GSD & \rabove{\iLa} 
}
\conv
\prarc{
t(s)=t(s), p\Incl s, t(p)\Incl x, t(s)\Incl x, \GSD \cl
x=t(s), p\Incl s, t(p)\Incl x, t(s)\Incl x, \GSD & \rabove{\iLa} \cl
x=x, p\Incl s, t(p)\Incl x, t(s)\Incl x, \GSD & \rabove{\iLa} \cl
x=x, p\Incl s, t(s)\Incl x, t(s)\Incl x, \GSD & \rabove{\iLa} 
} }
\]
\end{LSB}
\item $\eLanx$ -- this case is entirely analogous to the case \ref{lst} of $\iLa$ and
is treated by the same form of weakening.
\item $\Laix$ -- modifying $t\Incl x$, is again entirely analogous to the case \ref{lst}.
\item All other rules allow trivial swaps.
\end{LSA}
\item $\eLanx$ -- we have following cases 
\begin{LSA}
\item\label{csa} -- the modified formula was $t\Incl x$
\[\sma{ \prarc{
x=p(t), p(y)\Incl x, y=t, \GSD \cl
x=p(t), p(t)\Incl x, y=t, \GSD & \rabove{\eLanx} \cl
x=x, p(t)\Incl x, y=t, \GSD & \rabove{\iLa} 
}\conv
\prarc{
x=p(t), p(t)\Incl x, p(y)\Incl x, y=t, \GSD \cl
x=x, p(t)\Incl x, p(y)\Incl x, y=t, \GSD & \rabove{\iLa} \cl
x=x, p(t)\Incl x, p(t)\Incl x, y=t, \GSD & \rabove{\eLanx} 
} }
\]
\item -- again with modified $t\Incl x$
\[\prar{
x=t, x\Incl x, \GSD \cl
x=t, t\Incl x, \GSD & \rabove{\eLanx} \cl
x=x, t\Incl x, \GSD & \rabove{\iLa} 
}\conv
\prar{
x=t, t\Incl x, \GSD \cl
x=x, t\Incl x, \GSD & \rabove{\iLa}
}
\]
The conversion is possible by weakening and admissibility of $(cut_x)$, 
lemma~\ref{le:xok}, without increasing the length of the derivation.
\item -- another modfied formula
\[\prarc{
x=t, t\Incl x, p(x)\Incl q, \GSD \cl
x=t, t\Incl x, p(t)\Incl q, \GSD & \rabove{\eLanx} \cl
x=x, t\Incl x, p(t)\Incl q, \GSD & \rabove{\iLa} 
}\conv
\prarc{
x=t, t\Incl x, p(x)\Incl q, \GSD \cl
x=x, t\Incl x, p(x)\Incl q, \GSD & \rabove{\iLa} \cl
x=x, t\Incl x, p(t)\Incl q, \GSD & \rabove{\Laix} 
}
\]
The application of $\Laix$ is legal since $p(t)\not\in\Vars$.
\item -- the case symmetric to the above
\[\prarc{
x=t, t\Incl x, p(t)\Incl q, \GSD \cl
x=t, t\Incl x, p(x)\Incl q, \GSD & \rabove{\eLanx} \cl
x=x, t\Incl x, p(x)\Incl q, \GSD & \rabove{\iLa} 
}\conv
\prarc{
x=t, t\Incl x, p(t)\Incl q, \GSD \cl
x=x, t\Incl x, p(t)\Incl q, \GSD & \rabove{\iLa} \cl
x=x, t\Incl x, p(x)\Incl q, \GSD & \rabove{\iLa} 
}
\]
\end{LSA}
\item $\Laix$ -- this is similar to the case \ref{csa}:
\[\sma{ \prarc{
x=t(s), t(y)\Incl x, s\Incl y, \GSD \cl
x=t(s), t(s)\Incl x, s\Incl y, \GSD & \rabove{\Laix} \cl
x=x, t(s)\Incl x, s\Incl y, \GSD & \rabove{\iLa}
}\conv
\prarc{
x=t(s), t(y)\Incl x, s\Incl y, t(s)\Incl y, \GSD \cl
x=x, t(y)\Incl x, s\Incl y, t(s)\Incl y, \GSD & \rabove{\iLa} \cl
x=x, t(s)\Incl x, s\Incl y, t(s)\Incl y, \GSD & \rabove{\Laix}
} }
\]
\item $\eLs$ -- replacing $x$ for $t$ 
\[\prarc{
x=t, t\Incl x, \GSD, p(t)\preceq q \cl
x=t, t\Incl x, \GSD, p(x)\preceq q & \rabove{\eLs} \cl
x=x, t\Incl x, \GSD, p(x)\preceq q & \rabove{\iLa} 
}\conv
\prarc{
x=t, t\Incl x, \GSD, p(t)\preceq q \cl
x=x, t\Incl x, \GSD, p(t)\preceq q & \rabove{\iLa} \cl
x=x, t\Incl x, \GSD, p(x)\preceq q & \rabove{\Lsix}
}
\]
The dual case is even simpler not requiring the rule $\Lsix$
\[\prarc{
x=t, t\Incl x, \GSD, p(x)\preceq q \cl
x=t, t\Incl x, \GSD, p(t)\preceq q & \rabove{\eLs} \cl
x=x, t\Incl x, \GSD, p(t)\preceq q & \rabove{\eLs}
}\conv
\prarc{
x=t, t\Incl x, \GSD, p(x)\preceq q \cl
x=x, t\Incl x, \GSD, p(x)\preceq q & \rabove{\iLa} \cl
x=x, t\Incl x, \GSD, p(t)\preceq q & \rabove{\iLs} 
}
\]
\item Other rules allow trivial swaps: $\eRanx$ and $\Ear$ 
cannot result in $t\Incl x$, $\iLs$ and
$\iRs$ may involve active $t\Incl x$ which remains unchanged and hence can be swapped; 
$\Esr$ and $(Sp.cut)$ can be trivially swapped.
%\item $\eRs$ -- replacing $t$ for $x$ ???
\end{LS}
\end{PROOF}

\begin{CLAIM}\label{pr:lastneq}
$NEQ_5\equiv NEQ'_5$.
\end{CLAIM}
\begin{PROOF}
$\impl$ is lemma \ref{le:primeto5}. $\Leftarrow$ follows from lemma~\ref{le:noxx} -- 
admissibility of $(cut_{x=})$
\end{PROOF}

\noindent
This is the end of the cut-elimination story. Combining the chain of equivalences of the
intermediary calculi, summarised in
% given in lemmas \ref{le:neqisneq1}, \ref{le:neq1isneq2}, \ref{le:neq2isneq3}, 
theorem \ref{th:neq4cisneq4}, and
the propositions \ref{pr:neq5isneq4}, \ref{pr:lastneq} from this section,
we conclude that $NEQ_5$ is a cut-free
equivalent of $NEQ$.

In the following section, we study the special case of the calculus
$NEQ_5$ without any specific axioms, showing further possibility of
simplifications. 


%\newpage\input{trash}

%\newpage
\section{No specific axioms -- no elimination rules}\label{se:noax}
We have shown that calculus $NEQ$ with specific axioms can be replaced by an
equivalent calculus with specific cut rules.
In the presence of the specific cut rules, the general ($cut$)
can be eliminated. 
Obviously, derivations of tautologies in the absence of any specific axioms
will not require the specific cut rules. Below we show a much less obvious
and stronger result
-- when no specific axioms are present, 
one can even dispense with the elimination rules! 
Observe that these rules do have a cut-like flavor.
Thus, in spite of the
complications in the world of multialgebras, one can
obtain the multialgebraic tautologies with quite a simple means of the identity
and inclusion rules. Of course, when specific axioms are present, the
elimination rules are indispensable since they play the role of substitution rules.

%Assuming the absence of specific axioms, and l
Letting $\spNEQ$ be $NEQ_5$
without the ($Sp.cut$) rules, we have the obvious:
\begin{LEMMA}
If there are no specific axioms then $\spNEQ\equiv NEQ_5$.
\end{LEMMA}
%
Redundancy of elimination rules will be shown using the following fact.
\begin{LEMMA}\label{le:ssp}
Assume that $\der{\spNEQ}D{\GSD}$ and for an $x\in\Vars$
\begin{enumerate}\MyLPar
\item\label{ca:ax} if an equation $p(x)=q \in \Delta$ then there is only this one
occurrence of $x$ in $\Delta$,
\item\label{ca:el2} if $x\Incl r \in\Gamma$ then $x$ is different from the LHS of any
inclusion modified by an application of $\Ear$ in $D$,
% this does not seem necessary \item $x\not\in\Vars(t)$,
\item\label{ca:six} if $s\Incl x \in\Gamma$ then $x$ is different from any variable
modified by $\Lsix$ in $D$, and
%$x$ does not occur in any equation
%nor LHS of any inclusion in $\Delta$
\item\label{ca:aix} if $s\Incl x \in\Gamma$ then $x$ is different from any variable
modified by $\Laix$ in $D$
\end{enumerate}
\noindent
then $\der{\spNEQ}{D^*}{(\GSD)_t^x}$ (with {\em all} $x$'s replaced by $t$)
\end{LEMMA}
\begin{PROOF}
The intention of these conditions is to enable exactly the same derivations
with $t$ occurring instead of $x$.
The case \ref{ca:ax} prevents us from concluding derivability of $\GSD, t=t$
from the fact that $\GSD,x=x$, being an axiom, is derivable. 
In all other cases, we can
 start derivation with an axiom where $t$ occurs instead of $x$. All the
 rules applied in $D$ and involving $x$ can be applied 
in the same order in $D^*$ with $t$ instead of $x$.
The possible exceptions  are: % $\Ear$: %($E_{ar}^*$):
\[ \sma{ \begin{array}{c@{\ \ \ \ \ \ \ }c@{\ \ \ \ \ \ \ }c}
D \left \{ \begin{array}{rl}
 \multicolumn{1}{c}{\vdots} \\
 y\Incl q, x\Incl r(y),\GSD  \\ \cline{1-1}
 x\Incl r(q),\GSD  % & \rabove{\Ear} 
 \end{array} \right . 
&
D \left \{ \begin{array}{rl}
 \multicolumn{1}{c}{\vdots} \\
 s\Incl x,\GSD, p(s)\preceq q  \\ \cline{1-1}
 s\Incl x,\GSD, p(x)\preceq q  % & \rabove{\Lsix} 
 \end{array} \right . 
&
D \left \{ \begin{array}{rl}
 \multicolumn{1}{c}{\vdots} \\
 s\Incl x, p(x)\Incl q,\GSD  \\ \cline{1-1}
 s\Incl x, p(s)\Incl q, \GSD  % & \rabove{\Laix} 
\end{array} \right . \\[5ex]
 \Ear &  \Lsix &  \Laix
\end{array} }
\]
The sequent resulting from $\Ear$ cannot be obtained with $t$ substituted for $x$ by the
application of $\Ear$ unless $t$ is a variable. This case, however, is
excluded by the condition \ref{ca:el2}.
Similarly, sequents resulting from $\Lsix$, resp. $\Laix$, cannot be obtained with
$t$ instead of $x$ 
if $t$ isn't a variable. Conditions \ref{ca:six}, resp. \ref{ca:aix}, 
exclude these cases.
\end{PROOF}

\begin{DEFINITION}\label{de:neq6}
 $\elNEQ'$ is $\spNEQ$ but without the elimination rules $\Esr$
 and $\Ear$. %($E_{ar}^*$).
\end{DEFINITION}
\noindent
The rules of the calculus  $\elNEQ'$ for the case without specific axioms 
are given in figure \ref{fi:neq6}. 

\begin{figure}[hbt]
\hspace*{2em}
\begin{tabular}{|l@{\ \ \ \ \ \ \ \ \ \ \ \ }ll|}
\hline
\multicolumn{2}{|c}{{\bf Axioms}:} & \\[1ex]
\multicolumn{3}{|c|}{$\GSD, t\Incl t$;\ \ \ \ \ \ $\GSD, x=x$;\ \ \
\ \ \ 
$s=t,\GSD, s=t$ }\\[2ex]
%
\multicolumn{2}{|c}{{\bf Identity rules}:} & \\[1ex]
 $\eLs$\ \prule{t=s,\GSD, p(s)\preceq q}{t=s,\GSD, p(t)\preceq q}
& 
$\eLanx$\ \prule{s=t, p(s)\Incl q,\GSD}{s=t, p(t)\Incl q,\GSD} 
       \ \ {\footnotesize{$p(t)\not\in\Vars$}}    & \\[2.5ex]
& $\eRanx$\ \prule{s=t, p\Incl q(s), \GSD}{s=t, p\Incl q(t),\GSD} 
        \ \ {\footnotesize{$q(t)\not\in\Vars$}}   &  \\[3ex]
% & 
%$\eRs$\ \prule{s=t, \GSD, p\Incl q(s)}{s=t,\GSD, p\Incl q(t)}   & & \\[2ex]
%
\multicolumn{2}{|c}{{\bf Inclusion rules}:}
& \\[1ex]
$\iLs$\ \prule{t\Incl s, \Gamma\Seq \Delta, p(s)\preceq q}{t\Incl s, \Gamma\Seq
\Delta, p(t)\preceq q} & 
$\iLa$\ \prule{s\Incl t, p(s)\preceq q, \GSD}{s\Incl t, p(t)\preceq q, \GSD} 
%       \ \ {\footnotesize{$t\not\in\Vars$}}
   & \\[3ex]
%
$\iRs$\ \prule{s\Incl t, \GSD, p\Incl q(s)}{s\Incl t,\GSD, p\Incl q(t)}  & & \\[3ex]
$\Lsix$\ \prule{t\Incl x,\GSD, p(t)\preceq q}{t\Incl x,\GSD, p(x)\preceq q} & 
$\Laix$\ \prule{t\Incl x, p(x)\Incl q,\GSD}{t\Incl x,p(t)\Incl q,\GSD} 
       \ \ {\footnotesize{$p(t)\not\in\Vars$}} 
& \\[2.5ex]
 \hline
\end{tabular} 
\caption{The rules of $\elNEQ'$ ($x\in\Vars$) -- no specific axioms}\label{fi:neq6}
\end{figure}
%
\begin{REMARK}\label{re:oldhold}
Revisiting the proof of lemma \ref{le:nott}, %\ref{le:nocutx}, 
we see that the transformations used 
there did not introduce any new applications of the elimination rules 
(neither did lemma~\ref{le:noeqeq} used in this proof).
The same holds (trivially) for the proof of lemma~\ref{le:xok} 
(cf.~lemma~\ref{le:noxeq}).
Hence: %, these lemmas hold also for $\elNEQ'$, in particular:
\begin{enumerate}\MyLPar
% \item $\der{\elNEQ'}{}{x=x, \GSD}\ \impl\ \der{\elNEQ'}{}{\GSD}$;
\item ($cut_x$) is admissible in $\elNEQ'$ (cf. lemma \ref{le:xok}, \ref{le:noxeq});
\item ($cut_t$) is admissible in $\elNEQ'$ (cf. lemma \ref{le:nott});
%% $\der{\elNEQ'}{}{t\Incl t, \GSD}\ \impl\ \der{\elNEQ'}{}{\GSD}$ (lemma \ref{le:nott});
%\item ($cut$) is admissible in $\elNEQ'$;
\item lemma \ref{le:ssp} holds for $\elNEQ'$.
\end{enumerate}
\end{REMARK}
\noindent
Using this observation, we get:
%
\begin{CLAIM}\label{le:spneqiselneq}
In the absence of specific axioms $\spNEQ \equiv \elNEQ'$
\end{CLAIM}
\begin{PROOF}
 $\Leftarrow$ is obvious, and for $\impl$ we have to show admissibility of
 the elimination rules in $\elNEQ'$. Consider the uppermost application of an
 elimination rule $E$:
\begin{LS}
\item $E$ is $\Ear$: %($E_{ar}^*$):
 \[ \begin{array}{rl}
D_1 \left \{ \begin{array}{c}
\vdots \\
 x\Incl t, y\Incl r(x),\GSD \end{array} \right . \\ \cline{1-1}
 y\Incl r(t), \GSD & \rabove{\Ear} \end{array} 
\conv 
\begin{array}{rl}
D_1' \left \{ \begin{array}{c}
\vdots \\
 t\Incl t, y\Incl r(t),\GSD \end{array} \right . \\ \cline{1-1}
 y\Incl r(t), \GSD & \rabove{(cut_t)} \end{array} 
\]
The application of $\Ear$ implies $x\not\in\Vars(t,y,\Gamma,\Delta)$, and hence the
conditions of lemma \ref{le:ssp} are satisfied. Applying it to $D_1$, we
obtain $D'_1$ and $(cut_t)$ is admissible by remark \ref{re:oldhold}.
%%  \[ \begin{array}{rl}
%% D_1' \left \{ \begin{array}{c}
%% \vdots \\
%%  t\Incl t, y\Incl r(t),\GSD \end{array} \right . \\ \cline{1-1}
%%  y\Incl r(t), \GSD & \rabove{(cut_t)} \end{array} \]
\item $E$ is $\Esr$:
 \[ \begin{array}{rl}
      D_1 \left \{ \begin{array}{c}
         \vdots \\
        x\Incl t,\GSD, \phi[x] \end{array} \right . \\ \cline{1-1}
     \GSD, \phi[t] & \rabove{\Esr} \end{array} 
\conv
\begin{array}{rl}
      D_1 \left \{ \begin{array}{c}
         \vdots \\
        t\Incl t,\GSD, \phi[t] \end{array} \right . \\ \cline{1-1}
     \GSD, \phi[t] & \rabove{(cut_t)} \end{array} 
\]
By the restrictions on $\Esr$, there is at most one occurrence of $x$ in 
$\Gamma,\Delta,\phi[x]$ so, again, the
conditions of lemma \ref{le:ssp} hold and we can apply it to $D_1$ obtaining $D'_1$, 
% $t\Incl t,\GSD, \phi[t]$, 
which leads to the desired conclusion 
% $\der{\elNEQ}{}{\GSD,\phi[t]}$ 
by admissibility of $(cut_t)$. \vspace*{-1ex}
\end{LS}
\end{PROOF}


\begin{LEMMA}\label{le:noiRs}
The rule $\iRs$ is redundant.
\end{LEMMA}
\begin{PROOF}
\begin{LS}
\item Axiom
\[\prarc{
s\Incl t, \GSD, q(s)\Incl q(s) \cl
s\Incl t, \GSD, q(s)\Incl q(t) & \rabove{\iRs}
}\conv
\prarc{
s\Incl t, \GSD, q(t)\Incl q(t) \cl
s\Incl t, \GSD, q(s)\Incl q(t) & \rabove{\iLs}
}
\]
\item $\Laix$
\[\sma{\prarc{
t\Incl x, p(x)\Incl q, \GSD, r\Incl f(p(t)) \cl
t\Incl x, p(t)\Incl q, \GSD, r\Incl f(p(t)) & \rabove{\Laix} \cl
t\Incl x, p(t)\Incl q, \GSD, r\Incl f(q) & \rabove{\iRs}
}\conv
\prarc{
t\Incl x, p(x)\Incl q, p(t)\Incl q, \GSD, r\Incl f(p(t)) \cl
t\Incl x, p(x)\Incl q, p(t)\Incl q, \GSD, r\Incl f(q) & \rabove{\iRs} \cl
t\Incl x, p(t)\Incl q, p(t)\Incl q, \GSD, r\Incl f(q) & \rabove{\Laix} 
} }
\]
\item $\iLa$, $\eLanx$, $\eRanx$ -- these are treated by the same kind of weakening
as the above case of $\Laix$
\item The remaining rules modify at most LHS of a formula in the consequent and thus
allow trivial swaps.
\end{LS}
\end{PROOF}

\noindent
Thus all multialgebraic tautologies are derivable with the calculus $\elNEQ$ given
in figure~\ref{fi:neq6a}.

\begin{figure}[hbt]
\hspace*{2em}
\begin{tabular}{||l@{\ \ \ \ \ \ \ \ \ \ \ \ }ll||}
\hline\hline
\multicolumn{2}{||c}{{\bf Axioms}:} & \\[1ex]
\multicolumn{3}{||c||}{$\GSD, t\Incl t$;\ \ \ \ \ \ $\GSD, x=x$;\ \ \
\ \ \ 
$s=t,\GSD, s=t$ }\\[2ex]
%
\multicolumn{2}{||c}{{\bf Identity rules}:} & \\[1ex]
 $\eLs$\ \prule{t=s,\GSD, p(s)\preceq q}{t=s,\GSD, p(t)\preceq q}
& 
$\eLanx$\ \prule{s=t, p(s)\Incl q,\GSD}{s=t, p(t)\Incl q,\GSD} 
       \ \ {\footnotesize{$p(t)\not\in\Vars$}}    & \\[2.5ex]
& $\eRanx$\ \prule{s=t, p\Incl q(s), \GSD}{s=t, p\Incl q(t),\GSD} 
        \ \ {\footnotesize{$q(t)\not\in\Vars$}}   &  \\[3ex]
% & 
%$\eRs$\ \prule{s=t, \GSD, p\Incl q(s)}{s=t,\GSD, p\Incl q(t)}   & & \\[2ex]
%
\multicolumn{2}{||c}{{\bf Inclusion rules}:}
& \\[1ex]
$\iLs$\ \prule{t\Incl s, \Gamma\Seq \Delta, p(s)\preceq q}{t\Incl s, \Gamma\Seq
\Delta, p(t)\preceq q} & 
$\iLa$\ \prule{s\Incl t, p(s)\preceq q, \GSD}{s\Incl t, p(t)\preceq q, \GSD} 
%       \ \ {\footnotesize{$t\not\in\Vars$}}
   & \\[3ex]
%
%$\iRs$\ \prule{s\Incl t, \GSD, p\Incl q(s)}{s\Incl t,\GSD, p\Incl q(t)}  & & \\[3ex]
$\Lsix$\ \prule{t\Incl x,\GSD, p(t)\preceq q}{t\Incl x,\GSD, p(x)\preceq q} & 
$\Laix$\ \prule{t\Incl x, p(x)\Incl q,\GSD}{t\Incl x,p(t)\Incl q,\GSD} 
       \ \ {\footnotesize{$p(t)\not\in\Vars$}} 
& \\[2.5ex]
 \hline\hline
\end{tabular} 
\caption{The rules of $\elNEQ$ ($x\in\Vars$) -- no specific axioms}\label{fi:neq6a}
\end{figure}
%



\section{Other connectives}\label{se:connectives}
Since satisfaction of a sequent is defined in the usual way, soundness/completeness
%sound\-ness/com\-ple\-te\-ness 
of $NEQ$ and its equivalence with $NEQ_5$ imply that the entailment
relation in $NEQ_5$ coincides with the semantic consequence. If we restrict
our formulae $\phi$ to the atomic ones (i.e., $s\Incl t$ or $s=t$), this
means that, given a specification with specific axioms $\Ax = \{\phi_1,...,\phi_n\}$, 
 $\MMod(\Ax)\models \phi \Leftrightarrow \der{NEQ_5}{}{\Ax\Seq\phi}$. Since the specific
 axioms are thus incorporated into the actual sequents to be proved, we need
 no ($Sp.cut$) rules and can
 conclude that  \[ \MMod(\Ax)\models \phi \iff \der{\elNEQ}{}{\Ax\Seq\phi}\]
\noindent
Notice that this formulation, however simple and convincing, requires us to
``guess'' the variable names which have to be used in $\phi$ in order to
``match'' the appropriate variable names in the axioms.

Furthermore, we may extend the language introducing other logical operators
in the usual way.
We obtain a sound and complete system $NEQ_i^+$ (for $i\leq 6$) by extending
 $NEQ_i$ with the standard rules \\[1ex]
%
\begin{tabular}{rl@{\hspace*{6em}}rl}
($\neg\ \Seq$) & \PROOFR{\Gamma\Seq\Delta,\phi}{\Gamma,\neg\phi\Seq\Delta} & 
($\Seq\ \neg$) & \PROOFR{\Gamma,\phi\Seq\Delta}{\Gamma\Seq\Delta,\neg\phi} \\
($\lor\ \Seq$) & \PROOFR{\Gamma,\phi_1\Seq\Delta\ \ ;\ \
\Gamma,\phi_2\Seq\Delta}{\Gamma,\phi_1\lor\phi_2\Seq\Delta}
& ($\Seq\ \lor$) &
\PROOFR{\Gamma\Seq\Delta,\phi_1,\phi_2}{\Gamma\Seq\Delta,\phi_1\lor\phi_2} \\
($\land\ \Seq$) &
\PROOFR{\Gamma,\phi_1,\phi_2\Seq\Delta}{\Gamma,\phi_1\land\phi_2\Seq\Delta} &
($\Seq\ \land$) & \PROOFR{\Gamma\Seq\Delta,\phi_1\ \ ;\ \
\Gamma\Seq\Delta,\phi_2}{\Gamma\Seq\Delta,\phi_1\land\phi_2} \\
($\impl\ \Seq$) & \PROOFR{\Gamma,\phi_1\Seq\Delta\ \ ;\ \
\Gamma\Seq\Delta,\phi_2}{\Gamma,\phi_2\impl\phi_1\Seq\Delta}  & ($\Seq\ \impl$) &
\PROOFR{\Gamma,\phi_1\Seq\Delta,\phi_2}{\Gamma\Seq\Delta,\phi_1\impl\phi_2} 
\end{tabular} \\[1ex]
%
We then have that, for a set of specific axioms $\Ax=\{\Phi_1,...,\Phi_n\}$ and a
formula $\Phi$, all built from the atomic equalities and
inclusions with the logical operators,  $\MMod(\Ax)\models \Phi \Leftrightarrow
 \der{NEQ_5^+}{}{\Ax\Seq\Phi}$. A proof in $NEQ_5^+$ will typically consist of a series
 of applications of the above eight rules leading to a sequent
 $\Gamma\Seq\Delta$, to which one then applies the rules of $NEQ_5$.
Here again axioms are incorporated into the formulae, so we can use $\elNEQ$
instead:
 \[ \MMod(\Ax)\models \Phi \iff  \der{\elNEQ^+}{}{\Ax\Seq\Phi} \]

%\section{Further specialization of $Sp.cut$}
%\PROOFRULE{\Gamma_i\Seq\Delta_i,r_i\odot_i s_i\ \ ;\ \ 
%u_j\odot_j v_j,\Gamma_j\Seq\Delta_j} {\bigc_i\Gamma_i,\bigc_j\Gamma_j
%\Seq \bigc_i\Delta_i,\bigc_j\Delta_j}
%





\section{Related work.}\label{se:related}
The first, to our knowledge, reasoning system for multialgebras was introduced 
by Hu{\ss}mann in
\cite{Hus}. It dealt only with simple (atomic) inclusions and, more significantly, 
was not complete.
A complete extension of this calculus was given in \cite{Mich}.
The present system $NEQ_5$ is a significant improvement of the original
calculi from \cite{Mich} and \cite{Top}, particularly, due to cut-freeness.

Another cut-free, Gentzen-style calculus was introduced in \cite{BK95}.
It addresses proofs of tautologies (no specific axioms) and so it is a counterpart
of our $NEQ_6$.
% -- $NEQ_5$ is essentially an extension of this system.
\begin{itemize}\MyLPar
\item 
The main difference is that \cite{BK95} address partial multialgebras, i.e. ones where
some terms may denote empty set. Our calculi work only for total algebras. 
Nevertheless, it seems that partiality is captured by ...
\item
There are two differences in the language. The first one concerns
the special ``let'' term-construction: `$let\ x:=t\ in\ f(x,x)$' is a possible term. 
Although absent from our language, it is
expressible, except for the cases when a term with $let$ occurs in the RHS
of inclusion. 
\item 
On the other hand, our system
deals not only with inclusion, but also with the determinisitc equality which
is not a part of the language in \cite{BK95}.
Thus, our results show that the
only rules needed for their case are the first axiom and the inclusion rules
from figure \ref{fi:neq6}.
\item
The proofs in the calculus from \cite{BK95} proceed essentially in a bottom-up 
fashion by 1) replacing (sub)terms of concern by variables and introducing 
additional assumptions about the relations between them, and then 2) performing 
the deductions using rules handling variables. In general, this leads to complex
proof-{\em trees} -- in $NEQ_6$ all the proofs are linear.
\item
The restrictions on the inclusion and elimination rules are essentially the same
in both systems.
However, we have
made it explicit that the elimination rules (which are present in \cite{BK95}, 
even though no specific axioms are admitted)
are needed only in the presence of 
specific axioms -- when no such axioms are involved they can be eliminated and $NEQ_6$
is a complete system for deriving all multialgebraic tautologies.
\end{itemize}
The sufficiency of $\elNEQ$ is a significant strengthening of these earlier results.
In addition, the cut-freeness of the system from \cite{BK95} corresponds to the 
situations when
no specific axioms are present (i.e., cut-freeness of $\elNEQ$). 
We have shown how such axioms can be treated explicitly by
the more specific cut rules instead of general cut. Finally, other rules of the 
calculus have
been given very specific restrictions. All these results enhance significantly 
proof search.

\subsection*{Acknowledgment}
We thank Valentinas Kriau\v ciukas and Sigurd
Meldal for the comments and help they provided during our work on this paper.

The first three authors were supported by a grant from the Norwegian Research Council.

\begin{thebibliography}{MM99}\MyLPar
\bibitem[BK95]{BK95} M.~Bia{\l}asik, B.~Konikowska. 
   {\em Reasoning with Nondeterministic Specifications,} Tech.~Rep. no.~793, IPI PAN, Warszawa,
   (1995).
\bibitem[Hus90]{Hus} H. Hu{\ss}mann, {\em Nondeterministic Algebraic Specifications},
Ph.D. thesis, Fakult\"{a}t f\"{u}r Mathematik und Informatik, Universit\"{a}t Passau, 
 (1990).
\bibitem[Kan63]{K} S.~Kanger,
   {\em A simplified proof method for elementary logic,}
   in Comput. Progr. and Formal Systems, North-Holland, Amsterdam, (1963).
\bibitem[Pli71]{Aida1} A.~Pliu\v skevi\v cien\.e, 
   {\em Elimination of cut-type rules in axiomatic theories with equality,}
   Seminars in mathematics V.A.Steklov Mathem. Inst., Leningrad, 16, 90-94, (1971).
\bibitem[Pli73]{Aida2} A.~Pliu\v skevi\v cien\.e,
   {\em Specialization of the use of axioms for deduction search in 
   axiomatic theories with equality,}
   J. Soviet Math., 1, (1973).
\bibitem[PWM94]{LFCS} A.~Pliu\v skevi\v cien\.e, R.~Pliu\v skevi\v cius,
   M.~Walicki, S.~Meldal,
   {\em On specialization of derivations in axiomatic equality theories,}
   in Proc. of LFCS'94, LNCS vol. 813, (1994).
\bibitem[Wal93]{Mich} M.~Walicki,
   {\em Algebraic Specifications of Nondeterminism,}
   Ph.D. thesis, Institute of Informatics, University of Bergen, (1993).
\bibitem[WB97]{WB} M.~Walicki, M.~Bia{\l}asik, {\em Relations, Multialgebras and 
   Homomorphisms}, Tech.Rep., no~838, Polish Academy of Sciences, Institute of
  CS, (1997).
\bibitem[WB95]{Broy} M.~Walicki, M.~Broy, Structured Specifications and Implementation of
   Nondeterministic Data Types, {\em Noric Journal of Computing}, vol.~2, (1995).
\bibitem[WM95a]{WM} M.~Walicki, S.~Meldal, 
   Multialgebras, Power Algebras and Complete Calculi of Identities and 
       Inclusions, {\em Recent Trends in Data Type  Specification}, LNCS, vol. 906 (1995).
\bibitem[WM95b]{Top} M.~Walicki, S.~Meldal, 
   A Complete Calculus for the Multialgebraic and Functional Semantics of Nondeterminism,  
       {\em ACM ToPLaS}, vol.~17, no.~2, (1995).
\end{thebibliography} 


\end{document}
