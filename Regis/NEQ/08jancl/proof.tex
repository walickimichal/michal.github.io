
\begin{CLAIM}\label{le:elcut}
If $D$ is a $NEQ_4^c$ and $D_1$, $D_2$ are $NEQ_4$ derivations as follows:
%\begin{center}
\[ D\left\{ \begin{array}{cl}
 D_1\left\{ \begin{array}{c}
  \vdots \\   \Gamma\Seq\Delta, \phi
 \end{array} \right.
 D_2\left\{ \begin{array}{c}
  \vdots \\   \phi, \Gamma'\Seq\Delta'
 \end{array} \right. \\ \cline{1-1}
\Gamma,\Gamma' \Seq \Delta,\Delta'
&   \raisebox{1.2ex}[1.5ex][0ex]{$(cut)$}
\end{array}\right. \]
%\end{center}
then $\der{NEQ_4}{}{\Gamma,\Gamma'\Seq\Delta,\Delta'}$.
\end{CLAIM}
\begin{PROOF}
By the assumption 1. we let $D_1$ contain no $\eRs$, $\eLs$ or $\iLs$. 
By the assumption 2. we let $D_2$ contain only reduced applications $\iLar$, 
$\eRar$, $\iRsr$ -- $\eLanr$ does not belong to $NEQ_4$ -- and no $\eRs$.

The induction is then on $\< \#(\pLar,D_2), h(D_1), h(D_2)\>$, 
where the first parameter is the total number of applications of $\iLar$
{\em or} $\eLar$ {\em which modify the cut formula} $\phi$ in $D_2$ (this
qualification is essential in case \ref{it:cutactive} on page
\pageref{it:cutactive}).
%  $h(D_1)$ is the height of $D_1$ and $h(D_2)$ the height of  $D_2$. 
\\[1ex]
\noindent 
If $h(D_1)=0$, then the resulting sequent of $D_1$ is an axiom. If
the cut formula is not the one mentioned explicitly in the axioms (Fig.~\ref{fi:neq4}) 
then we obtain a cut-free derivation directly by choosing
another instance of the same axiom. In the other case, if the cut formula is
 $t\Incl t$ or $x=x$, the assumption 4. allows us to conclude the
 existence of a derivation of the conclusion of $D_2$ without this formula in
 the antecedent. Finally, if the axiom in $D_1$ is
 $s=t,\GSD,s=t$ or $x\Incl t,\GSD,x\Incl t$ and the cut formula is $s=t$, resp. 
$x\Incl t$, then it will also appear in the
 antecedent of the conclusion of ($cut$). Then this conclusion can be
 obtained without ($cut$) directly from $D_2$ by starting it with the
 instance of the axiom extended with $\Gamma$ and $\Delta$. \\[1ex]
\noindent
For $h(D_1)>0$, we consider the last rule $R$ applied in $D_1$. \\[1ex]
\noindent
{\bf I.} $\iLa$, $\Ear$, $\eLar$, $\eRa$, ($Sp.cut$), $(cut_{x=})$ or ($cut_x$): 
Since none of
these rules modifies the succedent, their application as $R$ in $D_1$ means
that we can apply ($cut$) with $D_1^*$ -- before $R$ -- instead. The 
induction on $h(D_1)$ and a subsequent application of $R$
gives the conclusion. \\[1ex]
\noindent
{\bf II.} The rules $\eRs$, $\eLs$ and $\iLs$ are not used 
in $D_1$ by assumption 1.\\[1ex]
%
\noindent
{\bf III.} $\iRs$: and the derivations end as follows:
\[ D \left \{\begin{array}{cl}
 D_1\left\{ \begin{array}{cl}
 D_1^*\left\{ \begin{array}{c}
  \vdots \\   s\Incl t,\GSD, w\Incl q(s)
 \end{array} \right. \\ \cline{1-1}
s\Incl t,\GSD, w\Incl q(t) & \rabove{\iRs}
 \end{array} \right.
 D_2\left\{ \begin{array}{c}
  \vdots \\   w\Incl q(t), \Gamma'\Seq\Delta'
 \end{array} \right. \\ \cline{1-1}
s\Incl t,\Gamma,\Gamma' \Seq \Delta,\Delta'
&   \raisebox{1.2ex}[1.5ex][0ex]{($cut$)}
\end{array} \right . 
%\convd
\]
We drop this last application of $\iRs$ in $D_1$ and extend weakened $D_2$ as follows:
\[
 D' \left \{ \begin{array}{cl}
 D_1^*\left\{ \begin{array}{c}
  \vdots \\ \vdots \\  s\Incl t,\Gamma\Seq\Delta, w\Incl q(s)
 \end{array} \right. 
D'_2\left\{ \begin{array}{cl}
 D_2\left\{ \begin{array}{c}
  \vdots \\   w\Incl q(t), q(s)\Incl q(t), \Gamma'\Seq\Delta'
 \end{array} \right. \\ \cline{1-1}
w\Incl q(s), q(s)\Incl q(t), \Gamma'\Seq\Delta' & \rabove{\iRa}
 \end{array} \right. \\ \cline{1-1}
s\Incl t,q(s)\Incl q(t)\Gamma,\Gamma' \Seq \Delta,\Delta'
&   \raisebox{1.2ex}[1.5ex][0ex]{($cut$)} \cl
s\Incl t,q(t)\Incl q(t),\Gamma,\Gamma' \Seq \Delta,\Delta' & \rabove{\iLa} \cl
s\Incl t,\Gamma,\Gamma' \Seq \Delta,\Delta' & \rabove{(cut_t)} 
\end{array} \right .
\]
By assumption 3. (lemma~\ref{le:inclaad}) the application of $\iRa$ is admissible
without increasing the number of $\pLar$, so $(cut)$ can be eliminated by the 
second argument of induction. The result will be a $NEQ_4$ derivation which
admits $(cut_t)$ by assumption 4 (lemma~\ref{le:nott}). \\[1ex]
%
{\bf IV.} $\Esr$ :
%
\[ \begin{array}{cl}
D_1\left\{ \begin{array}{cl}
 D_1^*\left\{ \begin{array}{cl}
  \vdots \\ 
  x\Incl t,\Gamma\Seq\Delta, \phi[x]
 \end{array} \right. \\ \cline{1-1}
\Gamma\Seq\Delta, \phi[t] & \rabove{\Esr}
 \end{array} \right .
 D_2\left\{ \begin{array}{cl}
  \vdots \\ \cline{1-1}  \phi[t], \Gamma'\Seq\Delta' & \rabove{R'}
 \end{array} \right. \\ \cline{1-1}
\Gamma,\Gamma' \Seq \Delta,\Delta'
&   \raisebox{1.2ex}[1.5ex][0ex]{($cut$)}
\end{array} \]
%
By restrictions on $\Esr$, $t\not\in\Vars$ and, furthermore, 
$x\not\in\C V(\Gamma,\Delta,t)$ so we can choose
 $x$ so that $x\not\in\C V(\Gamma',\Delta')$. We consider two cases: \ref{it:Ar} when
in $D_1$, $x$ in $\phi$ is in the RHS of $\Incl$, 
and \ref{it:Bl} when $x$ occurs in
a LHS of $\Incl$ or when $\phi$ is an equality. 
\begin{LS}
%
%
\item\label{it:Ar} Let $\phi$ be $p\Incl s(t)$  \\[.5ex]
\noindent
-- $t$ indicating the only occurrence
which has been substituted for $x$. We can then 
%construct the following derivation 
extend $D_2$ and construct the following derivation:
%
\[
 \begin{array}{cl}
 D_1^*\left\{ \begin{array}{cl}
  \ \\ \ \\ \vdots \\ 
  x\Incl t,\Gamma\Seq\Delta, p\Incl s(x) 
         \end{array} \right. \ \ \ 
%
D'_2\left\{ \begin{array}{cl}
  D_2\left\{ \begin{array}{cl}
  \vdots \\ 
  p\Incl s(t),\Gamma'\Seq\Delta' \end{array} \right.  \\ \cline{1-1}
 s(x)\Incl s(t),p\Incl s(t),\Gamma'\Seq\Delta' 
   & \raisebox{1.2ex}[1.5ex][0ex]{(W)} \\ \cline{1-1}
s(x)\Incl s(t), p\Incl s(x),\Gamma'\Seq\Delta' & \rabove{\iRa}
         \end{array} \right.
\\ \cline{1-1}
x\Incl t, s(x)\Incl s(t), \Gamma',\GSD,\Delta' 
  & \raisebox{1.2ex}[1.5ex][0ex]{($cut$)} \\ \cline{1-1}
x\Incl t, s(t)\Incl s(t), \Gamma',\GSD,\Delta' 
  & \rabove{\iLar} \\ \cline{1-1}
s(t)\Incl s(t), \Gamma',\GSD,\Delta' 
  & \rabove{\Esr} \\ \cline{1-1}
\Gamma',\GSD,\Delta' 
& \raisebox{1.2ex}[1.5ex][0ex]{($cut_t$)} 
\end{array} 
 \]
\noindent
% Notice that $x\not=s(x)\not\in\Vars$ since, otherwise, the application of 
% ($E_s$) in $D_1$ would be ($E_{Rsd}$) which is excluded by lemma~\ref{le:noErsd}.
The rule $\iRa$ is admissible (assumption 3, lemma \ref{le:inclaad}) and
does not increase the number of
applications of $\pLar$. So ($cut$)
can be eliminated by induction on $h(D_1^*)<h(D_1)$.
The resulting derivation is  in $NEQ_4$ so, by assumption 4. (lemma~\ref{le:nott})
($cut_t$) is admissible.
%
\item\label{it:Bl} Let $\phi$ be $s(t)\preceq p$ \\[.5ex]
\noindent
Here we have three subcases, depending on
whether $\phi$ in the application of the last rule $R'$ in $D_2$ was 
\ref{it:cutneither} neither modified nor active, \ref{it:cutmodified} modified, 
or \ref{it:cutactive} active.
%
\begin{LSA}
%
\item\label{it:cutneither} The cut formula $\phi$ is neither modified nor active.\\
We may then swap the
applications of $R'$ and ($cut$), and the induction on the height of $D_2$
yields the required elimination of ($cut$). 
%
\item\label{it:cutmodified}  $\phi$ is modified by $R'$.\\
%%\begin{LSA}
%%  \item $\phi$ is inclusion $s(t)\Incl p$ (modifed by $R'$).  \\[.5ex]
The two possibilities of $\phi$ being an inclusion and equality are entirely 
analogous. This case depends only on the rule $R'$ -- if $\phi$ is an equality, 
$R'$ cannot be $\eae$, but except for that the two cases are identical.
%  {\sf II.a) 
%
 \begin{LSB}
   \item $R'$ is $\Ear$ or ($=_{Rar}$): \\
%II.a.$\alpha$ 
  This would require $\phi$ to be an inclusion with  a single
   variable in the LHS, i.e, $y\Incl p$. %$s(t)=y$. 
Since in the current case \ref{it:Bl}, $\Esr$ in $D_1$
   replaces $x$ in the LHS of $\Incl$, this would mean that this application is
   actually degenerate ($E_{sx}$), what is excluded by lemma \ref{le:noEsd}. 
\item  $R'$ is $\iLar$ or $\eLar$: \\
These two cases are identical so we treat them jointly and
  write $\pLar$. In each subcase, all occurrences of $\pLar$ must be
  replaced consistently either by $\iLar$ or by $\eLar$, unless we
  mention the latter rules explicitly. The cut formula $\phi$ 
has the form $\phi[x]$ and $\phi[t]$ after the application of $\Esr$ in $D_1$, 
while in $D_2$ we 
write it as $\phi'[y]$, and as $\phi'[t']$ after the final application 
of $\pLar$, i.e. 
 $\phi[t] = \phi = \phi'[t']$. This is to indicate that the term $t$ introduced in $D_1$ and
$t'$ introduced in $D_2$ into $\phi$ may be different. The relation between the two
 does not matter, however -- what makes it possible to treat all the (sub)cases in
the same way, is the fact that if $\phi'[y]$ is an inclusion the modified term is introduced into
its LHS.

$D_2$ ends as follows:
\[ D_2 \left \{ \begin{array}{cl}
 D_2^* \left \{ \begin{array}{cl}
 \vdots \\
 \phi'[y], y\preceq t', \Gamma'\Seq\Delta' \end{array} \right . \\ \cline{1-1}
 \phi'[t'], y\preceq t', \Gamma'\Seq\Delta' & \rabove{\pLar}
 \end{array} \right . \]
We drop this last application of $\pLar$ and, instead, extend $D_1$
weakened with $y\preceq t'$. % \\[.5ex]
%
% \mbox{ 
{ \footnotesize 
\[\begin{array}{cl}
D_1' \left \{ \begin{array}{cl}
D_1 \left \{ \begin{array}{cl}
  \vdots \\ 
  y\preceq t', x\Incl t,\Gamma\Seq\Delta, \phi[x]  \\ \cline{1-1}
y\preceq t', \Gamma\Seq\Delta, \phi[t]  & \rabove{\Esr}
 \end{array} \right . \\ \cline{1-1}
y\preceq t', \Gamma\Seq\Delta, \phi'[y] & \rabove{\pLs}
 \end{array} \right .
%
 D_2^*\left\{ \begin{array}{cl}
 \vdots \\
\vdots \\
\phi'[y], y\preceq t', \Gamma'\Seq\Delta' \end{array} \right .
 \\ \cline{1-1}
y\preceq t', \Gamma,\Gamma' \Seq \Delta,\Delta' &   \raisebox{1.2ex}[1.5ex][0ex]{($cut$)}
\end{array} \] }
%  \) }} \\[.5ex]
The obtained application of $\pLs$ is admissible and here harmless.
%(lemma~\ref{le:noInclsx}).) So
($cut$) can be eliminated by induction hypothesis on $\#(\pLa,D_2)$.
%
\end{LSB}
%
%
\item\label{it:cutactive} %LSA 
 $\phi$ is active in $R'$
\begin{LSB}
\item\label{it:inact} $\phi$ is inclusion $s(t)\Incl p$ (active in $R'$),
\\
%  {\sf III.a) 
that is $R'$ is either $\iLs$, $\iRsr$ or $\iLar$.
The last two cases are excluded because they would require $s(t)$ to be a variable.
Then we would have a degenerate application of $\Esx$ in $D_1$, which does not 
belong to $NEQ_4$.
So let
 $R'$ be $\iLs$. We have the following derivation: 
{\footnotesize \[ \begin{array}{cl} %\hspace*{-4em}
D_1 \left \{ \begin{array}{cl}
 D_1^*\left\{ \begin{array}{cl}
  \vdots \\ 
  x\Incl t,\Gamma\Seq\Delta, s(x)\Incl p  
 \end{array} \right. \\ \cline{1-1}
\Gamma\Seq\Delta, s(t)\Incl p & \rabove{\Esr}
 \end{array} \right .
 D_2\left\{ \begin{array}{cl}
  D_2^* \left \{ \begin{array}{c}
\vdots \\
s(t)\Incl p, \Gamma'\Seq\Delta', w(p)\preceq q \end{array} \right .
\\ \cline{1-1}  
s(t)\Incl p, \Gamma'\Seq\Delta', w(s(t))\preceq q & \rabove{\iLs}
 \end{array} \right. \\ \cline{1-1}
\Gamma,\Gamma' \Seq \Delta,\Delta', w(s(t))\preceq q
 &   \raisebox{1.2ex}[1.5ex][0ex]{($cut$)}
\end{array} \] }
%} \\[.5ex]
%
First, construct the derivation $M'$ by cutting $s(t)\Incl p$ after $D_1$ and
 $D_2^*$. Since $h(D_2^*)<h(D_2)$ this ($cut$) can be eliminated using the third argument of 
induction. Then extend $M'$ to $M$  as follows:
\[ \hspace*{-1em} M \left \{ \begin{array}{cl} 
  M' \left \{ \begin{array}{cl}
 D_1 \left \{ \begin{array}{c}
 \vdots \\
 \GSD, s(t)\Incl p \end{array} \right .
 D_2^* \left \{ \begin{array}{c}
 \vdots \\
 s(t)\Incl p, \Gamma'\Seq\Delta', w(p)\preceq q \end{array} \right . \\ \cline{1-1}
 \Gamma,\Gamma'\Seq\Delta,\Delta', w(p)\preceq q & \rabove{(cut)} \end{array} \right . \\
 \cline{1-1}
 s(x)\Incl p, \Gamma,\Gamma'\Seq\Delta,\Delta',w(p)\preceq q
 &   \raisebox{1.2ex}[1.5ex][0ex]{($W_a$)} \\ \cline{1-1}
 s(x)\Incl p, \Gamma,\Gamma'\Seq\Delta,\Delta',w(s(x))\preceq q
 &   \rabove{\iLs}
 \end{array} \right . \]
The application of ($cut$) to $M$ with $D_1^*$ can be eliminated
by induction hypothesis $h(D_1^*)<h(D_1)$ -- notice that we are using here
the fact that $\#(\pLa,M)$ {\em modifying the cut formula} is not greater
than $\#(\pLa,D_2^*)$, even if the total number of arbitrary applications
of $\pLa$ in $M$ may be far greater than in $D_2^*$ (due to
applications in $D_1$).
It yields the following sequent leading
to the desired conclusion by an application of $\Esr$ -- $x$ may be chosen so
that $x\Not\in\Vars(\Gamma',\Delta',w,q$):
\[ \begin{array}{cl}
 x\Incl t, \Gamma, \Gamma' \Seq \Delta, \Delta', w(s(x))\preceq q \\
 \cline{1-1}
 \Gamma, \Gamma' \Seq \Delta, \Delta', w(s(t))\preceq q & \rabove{\Esr}
\end{array} \]
\noindent
%
\item % LSB {\sf III.b) 
 $\phi$ is equality $s(t)= p$ (active in $R'$). \\
 We have only three cases for $R'$ which are all treated analogously to
the previous case \ref{it:inact}. By assumption 2., the case of $\eRs$ for $R'$ 
is excluded.
% (III.a).
\begin{LSC}
\item $R'$ is $\eLs$ \\
This is treated exactly as \ref{it:inact} with applications of
$\eLs$ instead of $\iLs$. 
%
%% \item $R'$ is $\eRs$ \\
%% By assumption 2. this case is excluded.
%% must be a reduced application, i.e., either $s(t)$ or $p$ 
%% must be a variable.
%% The former case is excluded since then we would have a degenerate application 
%% of $(E_s)$ in $D_1$. If $p$ is a variable, say $y$, we proceed as in case \ref{it:inact}.
%% The result of $M'$ -- $(cut)$ between $D_1$ and $D^*_2$ is eliminated by 
%% induction on $h(D_2)$ --
%% is weakened with $s(x)=y$, i.e., 
%% \[ \hspace*{-1em} M \left \{ \begin{array}{cl} 
%%   M' \left \{ \begin{array}{cl}
%%  D_1 \left \{ \begin{array}{c}
%%  \vdots \\
%%  \GSD, s(t)=y \end{array} \right .
%%  D_2^* \left \{ \begin{array}{c}
%%  \vdots \\
%%  s(t)=y, \Gamma'\Seq\Delta', w\Incl q(y) \end{array} \right . \\ \cline{1-1}
%%  \Gamma,\Gamma'\Seq\Delta,\Delta', w\Incl q(y) & \rabove{(cut)} \end{array} \right . \\
%%  \cline{1-1}
%%  s(x)=y, \Gamma,\Gamma'\Seq\Delta,\Delta',w\Incl q(y)
%%  &   \raisebox{1.2ex}[1.5ex][0ex]{($W_a$)} \\ \cline{1-1}
%%  s(x)=y, \Gamma,\Gamma'\Seq\Delta,\Delta',w\Incl q(s(x))
%%  &   \rabove{\eRsr}
%%  \end{array} \right . \]
%% The $(cut)$ of this sequent with $D^*_1$ can be eliminated by induction on $h(D_1)$ -- 
%% the first argument of induction is here $0$ --
%% and leads to the following 
%% \[ \begin{array}{cl}
%%  x\Incl t, \Gamma, \Gamma' \Seq \Delta, \Delta', w\Incl q(s(x)) \\
%%  \cline{1-1}
%%  \Gamma, \Gamma' \Seq \Delta, \Delta', w\Incl q(s(t))& \rabove{(E_s)}
%% \end{array} \]
\item  $R'$ is ($=_{Lar}$) or ($=_{Rar}$)\\
We proceed as above with the construction of $M'$, $M$ and ($cut$). The
differences occur only in the last step, so we make the following generic
description where $\psi$ is the side and $\psi'$ the modified formula
of $R'$: 
% \\[.5ex]  \hspace*{-.5em} \mbox{ 
{\footnotesize
\[ \begin{array}{cl}  \hspace*{-1.5em}
D_1 \left \{ \begin{array}{cl}
 D_1^*\left\{ \begin{array}{cl}
  \vdots \\ 
  x\Incl t,\Gamma\Seq\Delta, s(x)= p  
 \end{array} \right. \\ \cline{1-1}
\Gamma\Seq\Delta, s(t)= p & \rabove{\Esr}
 \end{array} \right .
 D_2\left\{ \begin{array}{cl}
  D_2^* \left \{ \begin{array}{c}
\vdots \\
s(t)= p, \psi, \Gamma'\Seq\Delta' \end{array} \right .
\\ \cline{1-1}  
s(t)= p, \psi', \Gamma'\Seq\Delta',  & \raisebox{1.2ex}[1.5ex][0ex]{($R'$)}
 \end{array} \right. \\ \cline{1-1}
\psi', \Gamma,\Gamma' \Seq \Delta,\Delta'
 &   \raisebox{1.2ex}[1.5ex][0ex]{($cut$)}
\end{array} \] }
% } } \\[.5ex] \noindent
First, construct the derivation $M'$ by cutting $s(t)=p$ after $D_1$ and
 $D_2^*$ which is weakened with $s(x)=p$. $h(D_2^*)<h(D_2)$ means that this ($cut$)
 can be eliminated. This
yields the following sequent
\[  s(x)=p, \psi, \Gamma, \Gamma' \Seq \Delta, \Delta' \]
leading to the desired conclusion by the procedure depending on
 $R'$. (Remember that $x$ may be chosen so
that $x\Not\in\Vars(\Gamma',\Delta',\psi$), and $\#(\preceq_a,M')$ is not
greater than in $D_2^*$):
\begin{LSD}
\item $R'$ is ($=_{Lar}$)\\
 $s(t)$ cannot be a variable (since then $\Es$ in $D_1$ would be
 degenerate), so $p$ must be a variable $y$ and $\psi$ is $f(y)\Incl q$, and $\psi'$ is $f(s(t))\Incl q$.
\[ \begin{array}{cl}
D_1^* \left \{ \begin{array}{c} \vdots \\ 
   x \Incl t, \GSD, s(x)=y \end{array} \right . \ \ \ \ \ 
\begin{array}{rl}
 s(x)=y, f(y)\Incl q, \Gamma, \Gamma' \Seq \Delta, \Delta' \\
 \cline{1-1}
 s(x)=y, f(s(x))\Incl q, \Gamma, \Gamma' \Seq \Delta, \Delta' 
 &   \raisebox{1.2ex}[1.5ex][0ex]{($=_{Lar}$)}  
 \end{array} \\ \cline{1-1}
x\Incl t, f(s(x))\Incl q, \Gamma, \Gamma' \Seq \Delta, \Delta' 
 &   \raisebox{1.2ex}[1.5ex][0ex]{($cut$)} \\ \cline{1-1}
x\Incl t, f(s(t))\Incl q, \Gamma, \Gamma' \Seq \Delta, \Delta' 
 &   \rabove{\iLar} \\ \cline{1-1}
f(s(t))\Incl q, \Gamma, \Gamma' \Seq \Delta, \Delta' 
 &   \rabove{\Esr}
\end{array} \]
% \[ \begin{array}{rl}
%  s(x)=y, f(y)\Incl q, \Gamma, \Gamma' \Seq \Delta, \Delta' \\
%  \cline{1-1}
%  s(x)=y, f(s(x))\Incl q, \Gamma, \Gamma' \Seq \Delta, \Delta' 
%  &   \raisebox{1.2ex}[1.5ex][0ex]{($=_{Lar}$)} \\ \cline{1-1}
% x\Incl t, f(s(x))\Incl q, \Gamma, \Gamma' \Seq \Delta, \Delta' 
%  &   \raisebox{1.2ex}[1.5ex][0ex]{($cut$)\ with\ $s(x)=y$\ after\ $D_1^*$} \\ \cline{1-1}
% x\Incl t, f(s(t))\Incl q, \Gamma, \Gamma' \Seq \Delta, \Delta' 
%  &   \raisebox{1.2ex}[1.5ex][0ex]{($\Incl_a$)} \\ \cline{1-1}
% f(s(t))\Incl q, \Gamma, \Gamma' \Seq \Delta, \Delta' 
%  &   \raisebox{1.2ex}[1.5ex][0ex]{($E_s$)}
% \end{array} \]
This ($cut$) can be eliminated since $h(D_1^*)<h(D_1)$.
%
\item $R'$ is ($=_{Rar}$)\\
substituting $s(t)$ for $p$, i.e., $\psi$ is $y\Incl f(p)$, and $\psi'$ is
$y\Incl f(s(t))$ for a $y\in\Vars$:
\[ \begin{array}{cl}
D_1^* \left \{ \begin{array}{c} \vdots \\ 
 x\Incl t, \GSD, s(x)=p \end{array} \right . \ \ \ \ \ \ 
\begin{array}{rl}
 s(x)=p, y\Incl f(p), \Gamma, \Gamma' \Seq \Delta, \Delta' \\
 \cline{1-1}
 s(x)=p, y\Incl f(s(x)), \Gamma, \Gamma' \Seq \Delta, \Delta' 
 &   \raisebox{1.2ex}[1.5ex][0ex]{($=_{Rar}$)} 
 \end{array} \\ \cline{1-1}
x\Incl t, y\Incl f(s(x)), \Gamma, \Gamma' \Seq \Delta, \Delta' 
 &   \raisebox{1.2ex}[1.5ex][0ex]{($cut$)} \\ \cline{1-1}
y\Incl f(s(t)), \Gamma, \Gamma' \Seq \Delta, \Delta' 
 &   \rabove{\Ear}
\end{array} \]
This ($cut$) can be eliminated since $h(D_1^*)<h(D_1)$.
%
\item $R'$ is ($=_{Rar}$)\\
substituting $p$ for $s(t)$, i.e., $\psi$ is $y\Incl f(s(t))$, and $\psi'$ is
 $y\Incl f(p)$ for a $y\in\Vars$. Here we have to weaken $D_2^*$ also with 
$f(s(x))\Incl f(s(t))$:
{\footnotesize
\[ \begin{array}{cl} \hspace*{-2em}
D_1^* \left \{ \begin{array}{c} \vdots \\ 
 x\Incl t, \GSD, s(x)=p \end{array} \right . \ \ \ \ 
\begin{array}{rl}
f(s(x))\Incl f(s(t)), s(x)=p, y\Incl f(s(t)), \Gamma, \Gamma' \Seq \Delta, \Delta' \\
 \cline{1-1}
f(s(x))\Incl f(s(t)), s(x)=p, y\Incl f(s(x)), \Gamma, \Gamma' \Seq \Delta, \Delta' 
 &   \rabove{\iRa} \\ \cline{1-1}
f(s(x))\Incl f(s(t)), s(x)=p, y\Incl f(p), \Gamma, \Gamma' \Seq \Delta, \Delta' 
 &   \raisebox{1.2ex}[1.5ex][0ex]{($=_{Rar}$)} 
 \end{array} \\ \cline{1-1}
x\Incl t, f(s(x))\Incl f(s(t)), y\Incl f(p), \Gamma, \Gamma' \Seq \Delta, \Delta' 
 &   \raisebox{1.2ex}[1.5ex][0ex]{($cut$)} \\ \cline{1-1}
x\Incl t, f(s(t))\Incl f(s(t)), y\Incl f(p), \Gamma, \Gamma' \Seq \Delta, \Delta' 
 &   \rabove{\iLar} \\ \cline{1-1}
x\Incl t, y\Incl f(p), \Gamma, \Gamma' \Seq \Delta, \Delta' 
 &   \raisebox{1.2ex}[1.5ex][0ex]{($cut_t$)} \\ \cline{1-1}
y\Incl f(p), \Gamma, \Gamma' \Seq \Delta, \Delta' 
 &   \rabove{\Esr}
\end{array} \]
}
Again, ($cut$) can be eliminated since $h(D_1^*)<h(D_1)$.
% By the same argument as in the case \ref{it:Ar}, $s(x)\not\in\Vars$ and
% hence also $f(s(x))\not\in\Vars$ and $f(s(t))\not\in\Vars$, 
% so we can apply $\iRa$.
\end{LSD}
\end{LSC}
\end{LSB}
\end{LSA}
\end{LS}
\end{PROOF}
