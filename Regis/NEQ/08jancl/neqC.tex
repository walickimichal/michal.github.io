\section{The final cut-free calculus $NEQ_5$}\label{se:lastNEQ}
Although the elementary cut rules
%\reff{ru:cutx} from definition \ref{de:neq4}
are relatively innocent,
we make the final step showing that they are redundant. 
We do it in two steps. First, in \ref{sub:onecut}, we remove the rule $(cut_x)$ 
obtaining $NEQ'_5$ which is equivalent to $NEQ_4$.
This step isn't really problematic.
The rule $(cut_{x=})$, however, expresses the specific character of variables as
compared to arbitrary other terms. In \ref{sub:nocut} we show that it can be
replaced by two other rules which capture this specificity of variables, thus
obtaining the final cut-free version of the calculus $NEQ_5$.

%
\subsection{$NEQ'_5$ -- the calculus with one elementary cut}\label{sub:onecut}
\begin{DEFINITION}\label{de:neq5}
 $NEQ'_5$ is obtained from $NEQ_4$ by 
\begin{itemize}\MyLPar
\item replacing the axiom $x\Incl t,\GSD,x\Incl t$ by $\GSD,t\Incl t$;
%\item adding axiom $x\Incl y,\GSD, y\Incl x$ for $x,y\in\Vars$
\item admitting non-reduced applications of $\eLanr$ but \\
 removing the degenerate applications $\eRax$ and $\eLax$ which result in a single 
variable; % and $\iLax$;
\item removing the elementary cut rule ($cut_x$) and $\eRs$;
%\item imposing additional restrictions on $\eLa$ and $\eRa$
\end{itemize}
\end{DEFINITION}
\noindent
The rules of $NEQ'_5$ are given in figure \ref{fi:neq5}.
\begin{figure}[hbt]
\hspace*{3em}\begin{tabular}{|l@{\ \ \ \ \ \ \ \ \ \ \ \ }ll|}
\hline
\multicolumn{2}{|c}{{\bf Axioms}:} & \\[1ex]
\multicolumn{3}{|c|}{$\GSD, t\Incl t$\ \ \ \ \ \ $\GSD, x=x$\ \ \
\ \ \  %$x\Incl y,\GSD, y\Incl x$\ \ \ \ \ \ 
$s=t,\GSD, s=t$ }\\[2ex]
%
\multicolumn{2}{|c}{{\bf Identity rules}:} & \\[1ex]
$\eLs$\ \prule{t=s,\GSD, p(s)\preceq q}{t=s,\GSD, p(t)\preceq q} 
& 
$\eLanx$\ \prule{s=t, p(s)\Incl q,\GSD}{s=t, p(t)\Incl q,\GSD} 
       \ \ {\footnotesize{$p(t)\not\in\Vars$}} 
& \\[2.5ex]
 & 
%$\eRs$\ \prule{s=t, \GSD, p\Incl q(s)}{s=t,\GSD, p\Incl q(t)}   & 
 $\eRanx$\ \prule{s=t, p\Incl q(s), \GSD}{s=t, p\Incl q(t),\GSD} 
 \ \ {\footnotesize{$q(t)\not\in\Vars$}} 
 & 
\\[2ex]
%
\multicolumn{2}{|c}{{\bf Inclusion rules}:}
& \\[1ex]
 $\iLs$\ \prule{t\Incl s, \Gamma\Seq \Delta, p(s)\preceq q}{t\Incl s, \Gamma\Seq
\Delta, p(t)\preceq q} & 
$\iLa$\ \prule{s\Incl t, p(s)\preceq q, \GSD}{s\Incl t, p(t)\preceq q, \GSD} & \\[2ex]
%       \ \ {\footnotesize{$t\not\in\Vars$}} & \\[2ex]
%
 $\iRs$\ \prule{s\Incl t, \GSD, p\Incl q(s)}{s\Incl t,\GSD, p\Incl q(t)} & 
%% $p(t)\not\in\Vars$ & 
& \\[2ex]
\multicolumn{2}{|c}{{\bf Elimination rules}:} & \\[1ex]
 $\Esr$\ \prule{\Gamma, x\Incl t\Seq\Delta,\phi[x]} 
  {\Gamma\Seq\Delta,\phi[t]}  & 
\multicolumn{2}{l|}{ $\Ear$\ \prule{x\Incl t, y\Incl r(x), \Gamma\Seq\Delta}
  {y\Incl r(t),\Gamma\Seq\Delta} } %($E_{ar}^*$)} 
   \\[.5ex]
{\footnotesize \ \ \ - $x\not\in \Vars(\Gamma,\Delta,t),\ t\not\in\Vars$}
    &  \multicolumn{2}{l|}{{\footnotesize \ \ \
 - $x\not\in \Vars(t,\Gamma,\Delta,y),\ t\not\in\Vars$}} \\
 {\footnotesize \ \ \ - at most one $x$ in $\phi$;} 
  &  \multicolumn{2}{l|}{{\footnotesize \ \
 \ - at most one occurrence of $x$ in $r$ }} \\[2ex]
%
\multicolumn{3}{|c|}{{\bf Elementary cut}:}\\
%\multicolumn{3}{|c|}{
$(cut_{x=})$\ \ \prule{x=x,\GSD}{\GSD} &&  %}
\\[2ex]
\multicolumn{3}{|c|}{{\bf Specific cut rules}:}\\
\multicolumn{3}{|c|}
{for each specific axiom $\Ax_k$: \(a_1,...,a_n\Seq s_1,...,s_m\), 
a  rule:}\\[1ex]
\multicolumn{3}{|c|}
{\prule{\Gamma\Seq\Delta,a_1\ ;...;\ \Gamma\Seq\Delta,a_n\ ;\ 
s_1,\Gamma\Seq\Delta\ ;...;\ s_m,\Gamma\Seq\Delta} 
{\Gamma\Seq\Delta}\ \ \ ($Sp.cut_k$)} \\
 \hline
\end{tabular} 
\caption{The calculus $NEQ'_5$ ($x,y\in\Vars$)}\label{fi:neq5}
\end{figure}
%
\begin{LEMMA}\label{le:neq5toneq4}
$NEQ'_5\impl NEQ_4$.
\end{LEMMA}
\begin{PROOF}
Obviously, the axiom $\GSD, t\Incl t$ is derivable in $NEQ_4$ by choosing a fresh
variable in the axiom $x\Incl t,\GSD, x\Incl t$ and applying $(E_{sr})$. 
If $t$ is a variable $x\in\Vars$,
then one starts with $x\Incl x,\GSD, x\Incl x$ and applies $(cut_x)$.
%%% Another new axiom is derivable in $NEQ_4$ as follows:
%%% \[\prarc{
%%% x=y, x\Incl y, \GSD, x\Incl y \cl
%%% x=y, x\Incl y, \GSD, y\Incl y & \rabove{\eLs} \cl
%%% x=y, x\Incl y, \GSD, y\Incl x & \rabove{\eRs} \cl
%%% y=y, x\Incl y, \GSD, y\Incl x & \rabove{\iLa} \cl
%%%      x\Incl y, \GSD, y\Incl x & \rabove{(cut_{x=})}
%%% }
%%% \]
The non-reduced applications $\eLanr$ are admissible in $NEQ_4$ by lemma~\ref{le:noLanr}.
\end{PROOF}

\noindent
For the other implication $NEQ_4\impl NEQ'_5$ we show admissibility in $NEQ'_5$
of the missing rules: %. By corollary~\ref{co:noxeq}, we have to show
%admissibility of 
$(cut_x)$ % $(cut_{x=})$ 
and unrestricted $\eLar$ and $\eRa$.

\begin{LEMMA}\label{le:noxeq}
The rule $(cut_x)$ is admissible in $NEQ'_5$.
\end{LEMMA}
\begin{PROOF}
%% By lemma \ref{le:noax} we assume that derivation $D$ above $(cut_x)$ 
%% contains no $\eLarx$ or $\eRax$
%% and proceed by induction on $D$'s height.
Instead of any axiom with $x\Incl x$ in the antecedent, we may
take the one without it. The only rules which might possibly generate this formula in the
antecedent are $\Ear$, $\eLanx$, $\eRanx$ or $\iLa$. 
But since the first three of these rules are non-degenerate, they cannot introduce 
a single variable, 
i.e. they cannot result in a formula $x\Incl x$. Thus the only possibility is the
following:
%% \begin{LS}
%% \item $\Ear$ -- cannot generate $x\Incl x$ due to the
%% restriction that $\Ear$ does not result in a variable.
%% \item $\eLar$ and $\eRa$ are non-degenerate -- $\eLanx$ and $\eRanx$ cannot 
%% introduce a variable resulting in $x\Incl x$.
%% \item $\iLanx$ -- does not introduce a variable either.
%% %would require the following situation
\[\prar{t\Incl x, t\Incl x, \GSD \cl
x\Incl x, t\Incl x, \GSD & \rabove{\iLa}
}
\]
But the resulting sequent is the same as the premise weakened with $x\Incl x$.
%% \end{LS}
\end{PROOF}

\begin{LEMMA}\label{le:noax}
%$\der{NEQ'_5}DS\impl\der{NEQ'_5}{D^*}S$ and $D^*$ contains no 
$NEQ'_5$ admits degenerate applications of $\eLar$, namely, 
\[
\eLarx\ \ \mprule{y=x, y\Incl t,\GSD}{y=x, x\Incl t,\GSD}\ \ \ \ \ \ \ \ \ \ \ 
%\iLax\ \ \mprule{s\Incl x, s\preceq t,\GSD}{s\Incl x, x\preceq t,\GSD}\ \ \ \ 
%\eRax\ \ \mprule{t=x, s\Incl t,\GSD}{t=x, s\Incl x,\GSD}\ \ \ \ 
 y,x\in\Vars
\]
\end{LEMMA}
\begin{PROOF}
By %simultaneous 
induction on the number of applications of the rule and the height
of derivation. The rules is not applicable to any axiomatic formula, so
%% \[\prarc{
%% y=z, y\Incl x,\GSD, x\Incl y \cl
%% y=z, z\Incl x,\GSD, x\Incl y & \rabove{\eLarx}
%% }\conv
%% \prarc{
%% y=z, z\Incl x,\GSD, x\Incl z \cl
%% y=z, z\Incl x,\GSD, x\Incl y & \rabove{\eRs}
%% }
%% \]
consider the rule applied above.
%We consider first the case of when the uppermost application was of $\eLarx$.
\begin{LS}
\item $\eLanx$ -- the two applications cannot interact and can be swapped %for $\eLarx$ 
since the subsequent application $\eLarx$ is 
reduced, i.e.,
it replaces a variable which couldn't have been introduced just above it by a 
non-degenerate application of $\eLanx$. 
\item $\iLanx$ -- two cases 
%% can be swapped since their interaction with the
%% subsequent $\eLarx$ would require
%% a degenerate application $\iLax$
%% \[\prar{
%%  x=y, s\Incl y, s\Incl q,\GSD \cl
%%  x=y, s\Incl y, y\Incl q,\GSD & \rabove{\iLax} \cl
%%  x=y, s\Incl y, x\Incl q,\GSD & \rabove{\eLarx}
%%  }\ \ \ \ \ \ \ 
%% \prar{
%%  x=s, s\Incl y, y\Incl q, \GSD \cl
%%  x=y, s\Incl y, y\Incl q, \GSD & \rabove{\iLax} \cl
%%  x=y, s\Incl y, x\Incl q, \GSD & \rabove{\eLarx}
%%  }
%% \]
\[\sma{ \prar{
  x=y, s\Incl y, s\Incl q,\GSD \cl
  x=y, s\Incl y, y\Incl q,\GSD & \rabove{\iLa} \cl
  x=y, s\Incl y, x\Incl q,\GSD & \rabove{\eLarx}
}\conv
\prar{
  x=s, x=y, s\Incl y, s\Incl q,\GSD \cl
  x=s, x=y, s\Incl y, x\Incl q,\GSD & \rabove{\eLarx} \cl
  x=y, x=y, s\Incl y, x\Incl q,\GSD & \rabove{\iLa} 
} }
\]
Analogous weakening when $\iLa$ modifies the formula 
active in the application of $\eLarx$;
\[\prarc{
 x=s, s\Incl y, y\Incl q, \GSD \cl
 x=y, s\Incl y, y\Incl q, \GSD & \rabove{\iLa} \cl
 x=y, s\Incl y, x\Incl q, \GSD & \rabove{\eLarx}
 }\conv
\prarc{
  x=y, x=s, s\Incl y, y\Incl q, \GSD \cl
  x=y, x=s, s\Incl y, x\Incl q, \GSD & \rabove{\eLarx} \cl
  x=y, x=y, s\Incl y, x\Incl q, \GSD & \rabove{\iLa}
 }
\]
\item 
$\Ear$ -- swapping is possible since $x\in\Vars$
\[\prarc{
z\Incl p, x=y, y\Incl r(z),\GSD \cl
          x=y, y\Incl r(p),\GSD & \rabove{\Ear} \cl
          x=y, x\Incl r(p),\GSD & \rabove{\eLarx} 
}\conv
\prarc{
z\Incl p, x=y, y\Incl r(z),\GSD \cl
z\Incl p, x=y, x\Incl r(z),\GSD & \rabove{\eLarx} \cl
          x=y, x\Incl r(p),\GSD & \rabove{\Ear} 
}
\]
\item $\iLs$ 
\[\prar{
y\Incl t, y=x, \GSD, p(t)\preceq q \cl
y\Incl t, y=x, \GSD, p(y)\preceq q & \rabove{\iLs} \cl
x\Incl t, y=x, \GSD, p(y)\preceq q & \rabove{\eLarx} 
}\conv
\prar{
y\Incl t, y=x, \GSD, p(t)\preceq q \cl
x\Incl t, y=x, \GSD, p(t)\preceq q & \rabove{\eLarx} \cl
x\Incl t, y=x, \GSD, p(x)\preceq q & \rabove{\iLs} \cl
x\Incl t, y=x, \GSD, p(y)\preceq q & \rabove{\eLs} 
}
\]
\item $\iRs$ -- is treated analogously to $\iLs$.
\item Neither active nor modified formulae of $\eLs$ %$\eRs$ 
or $\Esr$ can be
affected by $\eLarx$, so these, as well as $(Sp.cut)$,
allow trivial swap with $\eLarx$.
$\eRa$ modifies at most the RHS of inclusion modified then by $\eLarx$, so these 
two can be swapped, too.
\end{LS}
% As we can see, no applications of $\eRax$ appear in the transformed derivations.
%
\end{PROOF}
%
To show admissibility of the unrestricted $\eRa$ we need the following lemma.
\begin{LEMMA}\label{le:yeseae}
The rule $\eae$ is admissible in $NEQ'_5$
\[ \eae\ \ \mprule{s=t, p(s)=q,\GSD}{s=t,p(t)=q,\GSD}\ \ p(t)\not\in\Vars \]
\end{LEMMA}
\begin{PROOF}
Application to an axiom is transformed as follows:
\[\prarc{
s=t, p(s)=q,\GSD, p(s)=q \cl
s=t, p(t)=q,\GSD, p(s)=q & \rabove{\eae}
}\conv
\prarc{
s=t, p(t)=q,\GSD, p(t)=q \cl
s=t, p(t)=q,\GSD, p(s)=q & \rabove{\eLs}
}
\]
The trivial differences from the proof of lemma~\ref{le:noeqeq} concern the presence
of non-reduced version of $\eLa$. In addition this case used there
$(cut_x)$. Here it will be treated as follows:
\begin{LS}
\item $\eLanx$ -- $p(f(s)), f(t)\not\in\Vars$ \vspace*{-1ex}
\[\sma{ \prar{
s=t, f(s)=r, p(r)\Incl q, \GSD \cl
s=t, f(s)=r, p(f(s))\Incl q, \GSD & \rabove{\eLanx} \cl
s=t, f(t)=r, p(f(s))\Incl q, \GSD & \rabove{\eae}
}\conv
\prar{
s=t, f(s)=r, p(r)\Incl q, \GSD \cl
s=t, f(t)=r, p(r)\Incl q, \GSD & \rabove{\eae} \cl
s=t, f(t)=r, p(f(t))\Incl q, \GSD & \rabove{\eLanx} \cl
s=t, f(t)=r, p(f(s))\Incl q, \GSD & \rabove{\eLanx} 
} } %\vspace*{-2ex}
\]
The first application is $\eLanx$ because $f(t)\not\in\Vars\impl p(f(t))\not\in\Vars$.

And the other possibility -- $p(r), f(t)\not\in\Vars$
\[ \sma{ \prar{
s=t, f(s)=r, p(f(s))\Incl q, \GSD \cl
s=t, f(s)=r, p(r)\Incl q, \GSD & \rabove{\eLanx} \cl
s=t, f(t)=r, p(r)\Incl q, \GSD & \rabove{\eae} 
}\conv
\prar{
s=t, f(s)=r, p(f(s))\Incl q, \GSD \cl
s=t, f(t)=r, p(f(s))\Incl q, \GSD & \rabove{\eae} \cl
s=t, f(t)=r, p(f(t))\Incl q, \GSD & \rabove{\eLanx} \cl
s=t, f(t)=r, p(r)\Incl q, \GSD & \rabove{\eLanx} 
} }  %\vspace*{-4ex}
\] 
Again, the first application is $\eLanx$ because 
$f(t)\not\in\Vars\impl p(f(t))\not\in\Vars$.
%
\item $\iLa$ -- $p(t)\not\in\Vars$
\[\sma{\prarc{
s\Incl t, p(s)=q, p(t)=r, \GSD \cl
s\Incl t, p(t)=q, p(t)=r, \GSD & \rabove{\iLa} \cl
s\Incl t, r=q, p(t)=r, \GSD & \rabove{\eae} 
}\conv
\prarc{
p(s)= r, s\Incl t, p(s)=q, p(t)=r, \GSD \cl
p(s)= r, s\Incl t, p(s)=q, p(t)=r, \GSD & \rabove{\eae} \cl
p(t)= r, s\Incl t, p(s)=q, p(t)=r, \GSD & \rabove{\iLa} 
} }
\]
Furthermore, we have the case when $t(k)\not\in\Vars$ which needs $\eRanx$
\[\sma{ \prarc{
s\Incl t(k), p(s)=q, k=r, \GSD \cl
s\Incl t(k), p(t(k))=q, k=r, \GSD & \rabove{\iLa} \cl
s\Incl t(k), p(t(r))=q, k=r, \GSD & \rabove{\eae}
}\conv
\prarc{
s\Incl t(r), s\Incl t(k), p(s)=q, k=r, \GSD \cl
s\Incl t(r), s\Incl t(k), p(t(r))=q, k=r, \GSD & \rabove{\eae} \cl
s\Incl t(k), s\Incl t(k), p(t(r))=q, k=r, \GSD & \rabove{\eRanx}
} }
\]
If $t(k)\in\Vars$, i.e. it is an $x$ we have the following situation with 
$p(r)\not\in\Vars$:
\[\prarc{
s\Incl x, p(s)=q, x=r, \GSD \cl
s\Incl x, p(x)=q, x=r, \GSD & \rabove{\iLa} \cl
s\Incl x, p(r)=q, x=r, \GSD & \rabove{\eae}
}\conv
\prarc{
s\Incl x, p(s)=q, x=r, s=r, \GSD \cl
s\Incl x, p(r)=q, x=r, s=r, \GSD & \rabove{\eae} \cl
s\Incl x, p(r)=q, x=r, x=r, \GSD & \rabove{\iLa}
}
\]
\item $\eRanx$ -- $q(t), f(r)\not\in\Vars$
\[\sma{ \prar{
t=s, p\Incl f(q(s)), q(s)=r, \GSD \cl
t=s, p\Incl f(r), q(s)=r, \GSD & \rabove{\eRanx} \cl
t=s, p\Incl f(r), q(t)=r, \GSD & \rabove{\eae} 
}\conv
\prar{
t=s, p\Incl f(q(s)), q(s)=r, \GSD \cl
t=s, p\Incl f(q(s)), q(t)=r, \GSD & \rabove{\eae} \cl
t=s, p\Incl f(q(t)), q(t)=r, \GSD & \rabove{\eRanx} \cl
t=s, p\Incl f(r), q(t)=r, \GSD & \rabove{\eRanx}
} }
\]
The first resulting application is $\eRanx$ since 
$q(t)\not\in\Vars\impl f(q(t))\not\in\Vars$.
Analogous transformation is applied when $\eae$ modifies $r$ rather than $q(s)$.
%% \item $\eRs$
%% \[\prar{
%% s=t, q(s)=r, \GSD, p\Incl f(q(s)) \cl
%% s=t, q(s)=r, \GSD, p\Incl f(r)  & \rabove{\eRs} \cl
%% s=t, q(t)=r, \GSD, p\Incl f(r)  & \rabove{\eae} 
%% }\conv
%% \prar{
%% s=t, q(s)=r, \GSD, p\Incl f(q(s)) \cl
%% s=t, q(t)=r, \GSD, p\Incl f(q(s)) & \rabove{\eae} \cl
%% s=t, q(t)=r, \GSD, p\Incl f(q(t)) & \rabove{\eRs} \cl
%% s=t, q(t)=r, \GSD, p\Incl f(r) & \rabove{\eRs} 
%% }
%% \]
%% Another case is analogous
%% \[\prar{
%% s=t, q(s)=r, \GSD, p\Incl f(r) \cl
%% s=t, q(s)=r, \GSD, p\Incl f(q(s)) & \rabove{\eRs} \cl
%% s=t, q(t)=r, \GSD, p\Incl f(q(s)) & \rabove{\eae} 
%% }\conv
%% \prar{
%% s=t, q(s)=r, \GSD, p\Incl f(r) \cl
%% s=t, q(t)=r, \GSD, p\Incl f(r) & \rabove{\eae} \cl
%% s=t, q(t)=r, \GSD, p\Incl f(q(t)) & \rabove{\eRs} \cl
%% s=t, q(t)=r, \GSD, p\Incl f(q(s)) & \rabove{\eRs} 
%% }
%% \]
\item $\eLs$ %-- the cases here are entirely analogous to those for $\eRs$.
\[\prar{
s=t, q(s)=r, \GSD,  f(q(s))\Incl p \cl
 s=t, q(s)=r, \GSD, f(r)\Incl p  & \rabove{\eLs} \cl
 s=t, q(t)=r, \GSD, f(r)\Incl p  & \rabove{\eae} 
}\conv
\prar{
 s=t, q(s)=r, \GSD, f(q(s))\Incl p \cl
 s=t, q(t)=r, \GSD, f(q(s))\Incl p & \rabove{\eae} \cl
 s=t, q(t)=r, \GSD, f(q(t))\Incl p & \rabove{\eLs} \cl
 s=t, q(t)=r, \GSD, f(r)\Incl p & \rabove{\eLs} 
}
\]
Another case is analogous
\[\prar{
 s=t, q(s)=r, \GSD, f(r)\Incl p \cl
 s=t, q(s)=r, \GSD, f(q(s))\Incl p & \rabove{\eLs} \cl
 s=t, q(t)=r, \GSD, f(q(s))\Incl p & \rabove{\eae} 
}\conv
\prar{
 s=t, q(s)=r, \GSD, f(r)\Incl p \cl
 s=t, q(t)=r, \GSD, f(r)\Incl p & \rabove{\eae} \cl
 s=t, q(t)=r, \GSD, f(q(t))\Incl p & \rabove{\eLs} \cl
 s=t, q(t)=r, \GSD, f(q(s))\Incl p & \rabove{\eLs} 
}
\]
\item Other cases are trivial. The active formula of $\iLs$ and $\iRs$ is an inclusion
and, similarly, elimination rules involve only inclusions. $(Sp.cut)$ can be trivially
swapped.
\end{LS}
\end{PROOF}
%
\begin{LEMMA}\label{le:eRax}
%$\der{NEQ'_5}DS\impl\der{NEQ'_5}{D^*}S$ and $D^*$ contains no 
$NEQ'_5$ admits degenerate applications of $\eRa$, namely, 
\[
\eRax\ \ \mprule{t=x, s\Incl t,\GSD}{t=x, s\Incl x,\GSD}\ \ \ \ 
 x\in\Vars
\]
\end{LEMMA}
\begin{PROOF}
%% Application to an axiomatic formula is treated as follows:
%% \[\prarc{
%% x=z, y\Incl x, \GSD, x\Incl y \cl
%% x=z, y\Incl z, \GSD, x\Incl y & \rabove{\eRax}
%% }\conv
%% \prarc{
%% x=z, y\Incl z, \GSD, x\Incl y \cl
%% x=z, y\Incl z, \GSD, x\Incl y & \rabove{\eLs}
%% }
%% \]
The rule isn't applicable to any axiomatic formula, so we consider the
rule applied just above.
\begin{LS}
\item $\eRanx$ -- $q(t)\not\in\Vars$
\[\sma{ \prarc{
s=t, q(t)=x, p\Incl q(s), \GSD \cl
s=t, q(t)=x, p\Incl q(t), \GSD & \rabove{\eRanx} \cl
s=t, q(t)=x, p\Incl x, \GSD & \rabove{\eRax} 
}\conv
\prarc{
q(s)= x, s=t, q(t)=x, p\Incl q(s), \GSD \cl
q(s)= x, s=t, q(t)=x, p\Incl x, \GSD & \rabove{\eRax} \cl
q(t)= x, s=t, q(t)=x, p\Incl x, \GSD & \rabove{\eae}
} }
\]
Since $q(t)\not\in\Vars$, the rule $\eae$ is admissible by lemma~\ref{le:yeseae}.
\item $\iLa$  %-- $t\not\in\Vars$
\[\sma{ \prarc{
x= p(s), s\Incl t, q\Incl p(t), \GSD \cl % 
x= p(t), s\Incl t, q\Incl p(t), \GSD & \rabove{\iLa} \cl % 
x= p(t), s\Incl t, q\Incl x, \GSD & \rabove{\eRax}
}\conv
\prarc{
x=p(t), x= p(s), s\Incl t, q\Incl p(t), \GSD \cl
x=p(t), x= p(s), s\Incl t, q\Incl x, \GSD & \rabove{\eRax} \cl
x=p(t), x= p(t), s\Incl t, q\Incl x, \GSD & \rabove{\iLa} 
} }
\]
Similar weakening when $\iLanx$ introduces the subsequently substituted variable:
\[\prarc{
r=p, r\Incl x, q\Incl p, \GSD \cl
x=p, r\Incl x, q\Incl p, \GSD & \rabove{\iLa} \cl
x=p, r\Incl x, q\Incl x, \GSD & \rabove{\eRax}
}\conv
\prarc{
x=p, r=p, r\Incl x, q\Incl p, \GSD \cl
x=p, r=p, r\Incl x, q\Incl x, \GSD & \rabove{\eRax} \cl
r=p, r=p, r\Incl x, q\Incl x, \GSD & \rabove{\iLa}
}
\]
Interaction of both formulae gives two cases; the first with $t\not\in\Vars$
\[\prar{
x=p, p\Incl t, \GSD \cl
x=t, p\Incl t, \GSD  & \rabove{\iLa} \cl
x=t, p\Incl x, \GSD  & \rabove{\eRax}
}\conv
\prar{
x=t, x=p, p\Incl t, \GSD \cl
x=t, x=p, p\Incl x, \GSD & \rabove{\eRax} \cl
x=t, x=x, p\Incl x, \GSD & \rabove{\iLa} \cl
x=t, x=t, p\Incl x, \GSD & \rabove{\eae} 
}
\]
where $\eae$ is admissible by lemma~\ref{le:yeseae} since $t\not\in\Vars$. 
%(It does not introduce any applications of $\eRanx$.)
If $t$ is an $y\in\Vars$
\[\prar{
x=p, p\Incl y, \GSD \cl
x=y, p\Incl y, \GSD & \rabove{\iLa} \cl
x=y, p\Incl x, \GSD & \rabove{\eRax}
}\conv
\prar{
p\Incl y, x=p, p\Incl x, \GSD \cl
p\Incl y, x=y, p\Incl x, \GSD  & \rabove{\iLa} \cl
x\Incl y, x=y, p\Incl x, \GSD  & \rabove{\iLa} \cl
y\Incl y, x=y, p\Incl x, \GSD  & \rabove{\eLax} \cl
          x=y, p\Incl x, \GSD  & \rabove{(cut_x)}
}
\]
The rule $\eLax$ is admissible by lemma \ref{le:noax} and $(cut_x)$ by \ref{le:noxeq}.
%-- both without introducing new applications of $\eRa$.
\item $\iLs$
\[\sma{ \prar{
x=q, p\Incl q, \GSD, r(q))\preceq s \cl
x=q, p\Incl q, \GSD, r(p)\preceq s & \rabove{\iLs} \cl
x=q, p\Incl x, \GSD, r(p)\preceq s & \rabove{\eRax}
}\conv
\prar{
x=q, p\Incl q, \GSD, r(q)\preceq s \cl
x=q, p\Incl x, \GSD, r(q)\preceq s & \rabove{\eRax} \cl
x=q, p\Incl x, \GSD, r(x)\preceq s & \rabove{\eLs} \cl
x=q, p\Incl x, \GSD, r(p)\preceq s & \rabove{\iLs} 
} }
\]
\item $\iRs$ is treated analogously with applications of $\iRs$ and $\eRs$.
\item\label{case:Ear} $\Ear$ -- $t\not\in\Vars$ %two cases
\[\sma{ \prar{
z\Incl t,    r(t)=x, y\Incl r(z), \GSD \cl
             r(t)=x, y\Incl r(t), \GSD & \rabove{\Ear} \cl
             r(t)=x, y\Incl x, \GSD & \rabove{\eRax}
}\conv
\prar{
z\Incl t,  r(z)=x,  r(t)=x, y\Incl r(z), \GSD \cl
z\Incl t,  r(z)=x,  r(t)=x, y\Incl x, \GSD & \rabove{\eRax} \cl
z\Incl t,  r(t)=x,  r(t)=x, y\Incl x, \GSD & \rabove{\iLa}  \cl
           r(t)=x,  r(t)=x, y\Incl x, \GSD & \rabove{\Ear}
} }
\]
%% was general \eRa
%% \[\sma{ \prar{
%% z\Incl r(t), t=s, y\Incl q(z), \GSD \cl
%%              t=s, y\Incl q(r(t)), \GSD & \rabove{\Ear} \cl
%%              t=s, y\Incl q(r(s)), \GSD & \rabove{\eRa}
%% }\conv
%% \prar{
%% z\Incl r(t), t=s, y\Incl q(z), \GSD \cl
%% z\Incl r(s), t=s, y\Incl q(z), \GSD & \rabove{\eRa} \cl
%%        t=s, y\Incl q(r(s)), \GSD & \rabove{\Ear} 
%% } }
%% \]
%% PROBLEM -- if $r(t)=t$ and $s=x\in\Vars$
\item All other rules allow trivial swap. $\eLanx$ modifies LHS of an inclusion so 
it does not interact with $\eRax$. The latter cannot affect the active equalities
of $\eLs$. % or $\eRs$.
\end{LS}
\end{PROOF}

\noindent
We have thus obtained:
\begin{CLAIM}\label{pr:neq5isneq4}
$NEQ'_5\equiv NEQ_4$.
\end{CLAIM}
\begin{PROOF}
$\impl$ is lemma~\ref{le:neq5toneq4}. For the opposite implication we use
lemma~\ref{fa:noeRs}, i.e. the fact that $NEQ_4\equiv NEQ_4-\eRs$.
%, we 
%use corollary~\ref{co:noxeq}, i.e. assuming the first of the following imlications, 
%we show the second:
%$NEQ_4\impl\ NEQ_4-\{(cut_{x=}),\iLax\}\ \impl NEQ'_5$.
The $NEQ_4$ axiom $x\Incl t,\GSD,x\Incl t$ is derivable in $NEQ'_5$ from the
axiom $x\Incl t,\GSD, t\Incl t$ by a single application of $\iLs$. 
Admissibility of all the missing rules was shown above: 
$(cut_x)$ in lemma~\ref{le:noxeq}, $\eLarx$ in \ref{le:noax} and 
$\eRax$ in \ref{le:eRax}. 
(By lemma~\ref{fa:noeRs} wee don't need to show admissibility of $\eRs$ in $NEQ'_5$.)
%e -- though we did not show it --
%by the proof which is a repetition of the proof of lemma~\ref{fa:noeRs}. (The resulting
%applications of a-rules, which were introduced there, are either now the same as in
%$NEQ'_5$ or else admissible by the preceding lemmas.
\end{PROOF}


\subsection{$NEQ_5$ -- the calculus with no cut}\label{sub:nocut}
We now remove the rule $(cut_{x=})$ and introduce two new 
rules $\Lsix$ and $\Laix$ -- see fig.~\ref{fi:neq51}.


\begin{figure}[hbt]
\hspace*{3em}\begin{tabular}{||l@{\ \ \ \ \ \ \ \ \ \ \ \ }ll||}
\hline\hline
\multicolumn{2}{||c}{{\bf Axioms}:} & \\[1ex]
\multicolumn{3}{||c||}{$\GSD, t\Incl t$\ \ \ \ \ \ $\GSD, x=x$\ \ \
\ \ \  %$x\Incl y,\GSD, y\Incl x$\ \ \ \ \ \ 
$s=t,\GSD, s=t$ }\\[2ex]
%
\multicolumn{2}{||c}{{\bf Identity rules}:} & \\[1ex]
$\eLs$\ \prule{t=s,\GSD, p(s)\preceq q}{t=s,\GSD, p(t)\preceq q} 
& 
$\eLanx$\ \prule{s=t, p(s)\Incl q,\GSD}{s=t, p(t)\Incl q,\GSD} 
       \ \ {\footnotesize{$p(t)\not\in\Vars$}} 
& \\[2.5ex]
 & 
%$\eRs$\ \prule{s=t, \GSD, p\Incl q(s)}{s=t,\GSD, p\Incl q(t)}   & 
 $\eRanx$\ \prule{s=t, p\Incl q(s), \GSD}{s=t, p\Incl q(t),\GSD} 
 \ \ {\footnotesize{$q(t)\not\in\Vars$}} 
 & 
\\[2ex]
%
\multicolumn{2}{||c}{{\bf Inclusion rules}:}
& \\[1ex]
 $\iLs$\ \prule{t\Incl s, \Gamma\Seq \Delta, p(s)\preceq q}{t\Incl s, \Gamma\Seq
\Delta, p(t)\preceq q} & 
$\iLa$\ \prule{s\Incl t, p(s)\preceq q, \GSD}{s\Incl t, p(t)\preceq q, \GSD} & \\[2ex]
%       \ \ {\footnotesize{$t\not\in\Vars$}} & \\[2ex]
%
 $\iRs$\ \prule{s\Incl t, \GSD, p\Incl q(s)}{s\Incl t,\GSD, p\Incl q(t)} & 
%% $p(t)\not\in\Vars$ & 
& \\[3ex]
$\Lsix$\ \prule{t\Incl x,\GSD, p(t)\preceq q}{t\Incl x,\GSD, p(x)\preceq q} & 
$\Laix$\ \prule{t\Incl x, p(x)\Incl q,\GSD}{t\Incl x,p(t)\Incl q,\GSD} 
       \ \ {\footnotesize{$p(t)\not\in\Vars$}} 
& \\[3ex]
\multicolumn{2}{||c}{{\bf Elimination rules}:} & \\[1ex]
 $\Esr$\ \prule{\Gamma, x\Incl t\Seq\Delta,\phi[x]} 
  {\Gamma\Seq\Delta,\phi[t]}  & 
\multicolumn{2}{l||}{ $\Ear$\ \prule{x\Incl t, y\Incl r(x), \Gamma\Seq\Delta}
  {y\Incl r(t),\Gamma\Seq\Delta} } %($E_{ar}^*$)} 
   \\[.5ex]
{\footnotesize \ \ \ - $x\not\in \Vars(\Gamma,\Delta,t),\ t\not\in\Vars$}
    &  \multicolumn{2}{l||}{{\footnotesize \ \ \
 - $x\not\in \Vars(t,\Gamma,\Delta,y),\ t\not\in\Vars$}} \\
 {\footnotesize \ \ \ - at most one $x$ in $\phi$;} 
  &  \multicolumn{2}{l||}{{\footnotesize \ \
 \ - at most one occurrence of $x$ in $r$ }} \\[2ex]
%
\multicolumn{3}{||c||}{{\bf Specific cut rules}:}\\
\multicolumn{3}{||c||}
{for each specific axiom $\Ax_k$: \(a_1,...,a_n\Seq s_1,...,s_m\), 
a  rule:}\\[1ex]
\multicolumn{3}{||c||}
{\prule{\Gamma\Seq\Delta,a_1\ ;...;\ \Gamma\Seq\Delta,a_n\ ;\ 
s_1,\Gamma\Seq\Delta\ ;...;\ s_m,\Gamma\Seq\Delta} 
{\Gamma\Seq\Delta}\ \ \ ($Sp.cut_k$)} \\
 \hline\hline
\end{tabular} 
\caption{The calculus $NEQ_5$ ($x,y\in\Vars$)}\label{fi:neq51}
\end{figure}

\begin{LEMMA}\label{le:primeto5}
Rules $\Lsix$ and $\Laix$ are 
admissible in $NEQ'_5$, i.e., $NEQ_5\impl NEQ'_5$.
\end{LEMMA}
\begin{PROOF}
The activity of these rules can be simulated by the activity
of $t=x$ which, then, can be reduced to $x=x$ and eliminated by $(cut_{x=})$.
The cases for both rules are entirely analogous.
%Thus, 
\[\prar{
\multicolumn{1}{c}{\vdots} \\
t\Incl x, \GSD, p(t)\preceq q \cl
t\Incl x, \GSD, p(x)\preceq q & \rabove{\Lsix\ \ \ \ \ \impl\ \ \ \ \ }
}%\conv
\prar{
\multicolumn{1}{c}{\vdots} \\
t=x, t\Incl x, \GSD, p(t)\preceq q \cl
t=x, t\Incl x, \GSD, p(x)\preceq q & \rabove{\eLs} \cl
x=x, t\Incl x, \GSD, p(x)\preceq q & \rabove{\iLa} \cl
     t\Incl x, \GSD, p(x)\preceq q & \rabove{(cut_{x=})}
}
\]
and analogous case for $\Laix$ -- here $p(t)\not\in\Vars:$
\[\prar{
\multicolumn{1}{c}{\vdots} \\
t\Incl x, p(x)\Incl q,\GSD \cl
t\Incl x, p(t)\Incl q,\GSD & \rabove{\Laix\ \ \ \ \ \impl\ \ \ \ \ }
}%\conv
\prar{
\multicolumn{1}{c}{\vdots} \\
t=x, t\Incl x, p(x)\Incl q, \GSD \cl
t=x, t\Incl x, p(t)\Incl q, \GSD & \rabove{\eLanx} \cl
x=x, t\Incl x, p(t)\Incl q, \GSD & \rabove{\iLa} \cl
     t\Incl x, p(t)\Incl q, \GSD & \rabove{(cut_{x=})}
}
\]
%% and entirely analogous case for $\Rsix$.
%% \[\prar{
%% \multicolumn{1}{c}{\vdots} \\
%% t\Incl x, \GSD, p\Incl q(x) \cl
%% t\Incl x, \GSD, p\Incl q(t) & \rabove{\Rsix\ \ \ \ \ \impl\ \ \ \ \ }
%% }%\conv
%% \prar{
%% \multicolumn{1}{c}{\vdots} \\
%% t=x, t\Incl x, \GSD, p\Incl q(x) \cl
%% t=x, t\Incl x, \GSD, p\Incl q(t) & \rabove{\eRs} \cl
%% x=x, t\Incl x, \GSD, p\Incl q(t) & \rabove{\iLa} \cl
%%      t\Incl x, \GSD, p\Incl q(t) & \rabove{(cut_{x=})}
%% }
%% \]
\end{PROOF}

\noindent
To show admissibility of $(cut_{x=})$ in $NEQ_5$, we will need
\begin{LEMMA}\label{le:xok}
If $\der{NEQ_5}D{x\Incl x,\GSD}$ then $\der{NEQ_5}{D^*}{\GSD}$ and $h(D^*)\leq h(D)$.
\end{LEMMA}
\begin{PROOF}
The argument is exactly the same as in the proof of lemma~\ref{le:noxeq}. The new rule
$\Laix$ cannot generate $x\Incl x$ because of the restriction $p(t)\not\in\Vars$.
\end{PROOF}


\begin{LEMMA}\label{le:noxx}\label{le:nocutx}
Admissibility of $(cut_{x=}):$ 
If $\der{NEQ_5}D{x=x,\GSD}$ then $\der{NEQ_5}{}{\GSD}$.
\end{LEMMA}
\begin{PROOF} 
By induction on $h(D)$. 
Instead of any axiom with $x=x$ in the antecedent, we may
take the one without it. 
So, consider the last rule applied before the appearance
of $x=x$, i.e., when it was the modified formula. 
The only possibility is $\iLa$.
 First we have the case when this rule was applied to an axiom,
and the only problematic case is the following:
%axiom $x=y,\GSD,y\Incl x$, with 
%two possibilities:
\[
\prarc{
x=t, t\Incl x,\GSD,x=t \cl
x=x, t\Incl x,\GSD,x=t& \rabove{\iLa} 
}\conv
\prarc{
t\Incl x,\GSD,x=x  \cl 
t\Incl x,\GSD,x=t & \rabove{\iLs} 
}
\]
(Notice that this case shows that $(cut_{x=})$ and $\iLs$ cannot be removed
simultaneously -- the case 1.3) of the proof of lemma~\ref{le:nosx} needed $(cut_{x=})$
to eliminate $\iLs$.)
Consider the rule applied just above $\iLa$.
\begin{LS}
\item $\iLa$ -- first there are two similar cases
\[\prarc{
x=p(t), p(s)\Incl x, t\Incl s, \GSD \cl
x=p(s), p(s)\Incl x, t\Incl s, \GSD & \rabove{\iLa} \cl
x=x, p(s)\Incl x, t\Incl s, \GSD & \rabove{\iLa} 
}\conv
\prarc{
x=p(t), p(t)\Incl x, p(s)\Incl x, t\Incl s, \GSD \cl
x=x, p(t)\Incl x, p(s)\Incl x, t\Incl s, \GSD & \rabove{\iLa} \cl
x=x, p(s)\Incl x, p(s)\Incl x, t\Incl s, \GSD & \rabove{\iLa} 
}
\]
\[\prarc{
x=p(s), p(t)\Incl x, t\Incl s, \GSD \cl
x=p(s), p(s)\Incl x, t\Incl s, \GSD & \rabove{\iLa} \cl
x=x, p(s)\Incl x, t\Incl s, \GSD & \rabove{\iLa} 
}\conv
\prarc{
x=p(s), p(s)\Incl t, p(t)\Incl x, t\Incl s, \GSD \cl
x=x, p(s)\Incl t, p(t)\Incl x, t\Incl s, \GSD & \rabove{\iLa} \cl
x=x, p(s)\Incl t, p(s)\Incl x, t\Incl s, \GSD & \rabove{\iLa} 
}
\]
Then we have the following case:
\[\prar{
\multicolumn{1}{c}{\vdots} \cl
t=t,t\Incl x,\GSD & \rabove R \cl
x=t,t\Incl x,\GSD & \rabove{\iLa} \cl
x=x,t\Incl x,\GSD & \rabove{\iLa} 
}
\]
We consider the rule $R$ -- the only relevant cases are when it modified a formula
resulting in $t=t$ or $t\Incl x$. 
\begin{LSA}
\item The first case is the axiom:
\[\prar{
t=t, t\Incl x,\GSD, t=t \cl
x=t,t\Incl x,\GSD, t=t & \rabove{\iLa} \cl
x=x,t\Incl x,\GSD, t=t & \rabove{\iLa} 
}\conv
\prar{
t\Incl x, \GSD, x=x \cl
t\Incl x, \GSD, x=t & \rabove{\iLs} \cl
t\Incl x, \GSD, t=t & \rabove{\iLs} 
}
\]
\item $\iLa$ -- two analogously treated cases:
\begin{LSB}
\item modified $t=t$
\[\sma{ \prarc{
t(s)=t(p), p\Incl s, t(s)\Incl x, \GSD \cl
t(s)=t(s), p\Incl s, t(s)\Incl x, \GSD & \rabove{\iLa} \cl
x=t(s), p\Incl s, t(s)\Incl x, \GSD & \rabove{\iLa} \cl
x=x, p\Incl s, t(s)\Incl x, \GSD & \rabove{\iLa} 
}
\conv
\prarc{
t(s)=t(p), p\Incl s, t(s)\Incl x, t(p)\Incl x, \GSD \cl
x=t(p), p\Incl s, t(s)\Incl x, t(p)\Incl x,  \GSD & \rabove{\iLa} \cl
x=x, p\Incl s, t(s)\Incl x, t(p)\Incl x, \GSD & \rabove{\iLa} \cl
x=x, p\Incl s, t(s)\Incl x, t(s)\Incl x, \GSD & \rabove{\iLa} 
} }
\]
\item\label{lst} modified $t\Incl x$
\[\sma{\prarc{
t(s)=t(s), p\Incl s, t(p)\Incl x, \GSD \cl
t(s)=t(s), p\Incl s, t(s)\Incl x, \GSD & \rabove{\iLa} \cl
x =t(s), p\Incl s, t(s)\Incl x, \GSD & \rabove{\iLa} \cl
x =x, p\Incl s, t(s)\Incl x, \GSD & \rabove{\iLa} 
}
\conv
\prarc{
t(s)=t(s), p\Incl s, t(p)\Incl x, t(s)\Incl x, \GSD \cl
x=t(s), p\Incl s, t(p)\Incl x, t(s)\Incl x, \GSD & \rabove{\iLa} \cl
x=x, p\Incl s, t(p)\Incl x, t(s)\Incl x, \GSD & \rabove{\iLa} \cl
x=x, p\Incl s, t(s)\Incl x, t(s)\Incl x, \GSD & \rabove{\iLa} 
} }
\]
\end{LSB}
\item $\eLanx$ -- this case is entirely analogous to the case \ref{lst} of $\iLa$ and
is treated by the same form of weakening.
\item $\Laix$ -- modifying $t\Incl x$, is again entirely analogous to the case \ref{lst}.
\item All other rules allow trivial swaps.
\end{LSA}
\item $\eLanx$ -- we have following cases 
\begin{LSA}
\item\label{csa} -- the modified formula was $t\Incl x$
\[\sma{ \prarc{
x=p(t), p(y)\Incl x, y=t, \GSD \cl
x=p(t), p(t)\Incl x, y=t, \GSD & \rabove{\eLanx} \cl
x=x, p(t)\Incl x, y=t, \GSD & \rabove{\iLa} 
}\conv
\prarc{
x=p(t), p(t)\Incl x, p(y)\Incl x, y=t, \GSD \cl
x=x, p(t)\Incl x, p(y)\Incl x, y=t, \GSD & \rabove{\iLa} \cl
x=x, p(t)\Incl x, p(t)\Incl x, y=t, \GSD & \rabove{\eLanx} 
} }
\]
\item -- again with modified $t\Incl x$
\[\prar{
x=t, x\Incl x, \GSD \cl
x=t, t\Incl x, \GSD & \rabove{\eLanx} \cl
x=x, t\Incl x, \GSD & \rabove{\iLa} 
}\conv
\prar{
x=t, t\Incl x, \GSD \cl
x=x, t\Incl x, \GSD & \rabove{\iLa}
}
\]
The conversion is possible by weakening and admissibility of $(cut_x)$, 
lemma~\ref{le:xok}, without increasing the length of the derivation.
\item -- another modfied formula
\[\prarc{
x=t, t\Incl x, p(x)\Incl q, \GSD \cl
x=t, t\Incl x, p(t)\Incl q, \GSD & \rabove{\eLanx} \cl
x=x, t\Incl x, p(t)\Incl q, \GSD & \rabove{\iLa} 
}\conv
\prarc{
x=t, t\Incl x, p(x)\Incl q, \GSD \cl
x=x, t\Incl x, p(x)\Incl q, \GSD & \rabove{\iLa} \cl
x=x, t\Incl x, p(t)\Incl q, \GSD & \rabove{\Laix} 
}
\]
The application of $\Laix$ is legal since $p(t)\not\in\Vars$.
\item -- the case symmetric to the above
\[\prarc{
x=t, t\Incl x, p(t)\Incl q, \GSD \cl
x=t, t\Incl x, p(x)\Incl q, \GSD & \rabove{\eLanx} \cl
x=x, t\Incl x, p(x)\Incl q, \GSD & \rabove{\iLa} 
}\conv
\prarc{
x=t, t\Incl x, p(t)\Incl q, \GSD \cl
x=x, t\Incl x, p(t)\Incl q, \GSD & \rabove{\iLa} \cl
x=x, t\Incl x, p(x)\Incl q, \GSD & \rabove{\iLa} 
}
\]
\end{LSA}
\item $\Laix$ -- this is similar to the case \ref{csa}:
\[\sma{ \prarc{
x=t(s), t(y)\Incl x, s\Incl y, \GSD \cl
x=t(s), t(s)\Incl x, s\Incl y, \GSD & \rabove{\Laix} \cl
x=x, t(s)\Incl x, s\Incl y, \GSD & \rabove{\iLa}
}\conv
\prarc{
x=t(s), t(y)\Incl x, s\Incl y, t(s)\Incl y, \GSD \cl
x=x, t(y)\Incl x, s\Incl y, t(s)\Incl y, \GSD & \rabove{\iLa} \cl
x=x, t(s)\Incl x, s\Incl y, t(s)\Incl y, \GSD & \rabove{\Laix}
} }
\]
\item $\eLs$ -- replacing $x$ for $t$ 
\[\prarc{
x=t, t\Incl x, \GSD, p(t)\preceq q \cl
x=t, t\Incl x, \GSD, p(x)\preceq q & \rabove{\eLs} \cl
x=x, t\Incl x, \GSD, p(x)\preceq q & \rabove{\iLa} 
}\conv
\prarc{
x=t, t\Incl x, \GSD, p(t)\preceq q \cl
x=x, t\Incl x, \GSD, p(t)\preceq q & \rabove{\iLa} \cl
x=x, t\Incl x, \GSD, p(x)\preceq q & \rabove{\Lsix}
}
\]
The dual case is even simpler not requiring the rule $\Lsix$
\[\prarc{
x=t, t\Incl x, \GSD, p(x)\preceq q \cl
x=t, t\Incl x, \GSD, p(t)\preceq q & \rabove{\eLs} \cl
x=x, t\Incl x, \GSD, p(t)\preceq q & \rabove{\eLs}
}\conv
\prarc{
x=t, t\Incl x, \GSD, p(x)\preceq q \cl
x=x, t\Incl x, \GSD, p(x)\preceq q & \rabove{\iLa} \cl
x=x, t\Incl x, \GSD, p(t)\preceq q & \rabove{\iLs} 
}
\]
\item Other rules allow trivial swaps: $\eRanx$ and $\Ear$ 
cannot result in $t\Incl x$, $\iLs$ and
$\iRs$ may involve active $t\Incl x$ which remains unchanged and hence can be swapped; 
$\Esr$ and $(Sp.cut)$ can be trivially swapped.
%\item $\eRs$ -- replacing $t$ for $x$ ???
\end{LS}
\end{PROOF}

\begin{CLAIM}\label{pr:lastneq}
$NEQ_5\equiv NEQ'_5$.
\end{CLAIM}
\begin{PROOF}
$\impl$ is lemma \ref{le:primeto5}. $\Leftarrow$ follows from lemma~\ref{le:noxx} -- 
admissibility of $(cut_{x=})$
\end{PROOF}

\noindent
This is the end of the cut-elimination story. Combining the chain of equivalences of the
intermediary calculi, summarised in
% given in lemmas \ref{le:neqisneq1}, \ref{le:neq1isneq2}, \ref{le:neq2isneq3}, 
theorem \ref{th:neq4cisneq4}, and
the propositions \ref{pr:neq5isneq4}, \ref{pr:lastneq} from this section,
we conclude that $NEQ_5$ is a cut-free
equivalent of $NEQ$.

In the following section, we study the special case of the calculus
$NEQ_5$ without any specific axioms, showing further possibility of
simplifications. 


%\newpage\input{trash}

%\newpage
\section{No specific axioms -- no elimination rules}\label{se:noax}
We have shown that calculus $NEQ$ with specific axioms can be replaced by an
equivalent calculus with specific cut rules.
In the presence of the specific cut rules, the general ($cut$)
can be eliminated. 
Obviously, derivations of tautologies in the absence of any specific axioms
will not require the specific cut rules. Below we show a much less obvious
and stronger result
-- when no specific axioms are present, 
one can even dispense with the elimination rules! 
Observe that these rules do have a cut-like flavor.
Thus, in spite of the
complications in the world of multialgebras, one can
obtain the multialgebraic tautologies with quite a simple means of the identity
and inclusion rules. Of course, when specific axioms are present, the
elimination rules are indispensable since they play the role of substitution rules.

%Assuming the absence of specific axioms, and l
Letting $\spNEQ$ be $NEQ_5$
without the ($Sp.cut$) rules, we have the obvious:
\begin{LEMMA}
If there are no specific axioms then $\spNEQ\equiv NEQ_5$.
\end{LEMMA}
%
Redundancy of elimination rules will be shown using the following fact.
\begin{LEMMA}\label{le:ssp}
Assume that $\der{\spNEQ}D{\GSD}$ and for an $x\in\Vars$
\begin{enumerate}\MyLPar
\item\label{ca:ax} if an equation $p(x)=q \in \Delta$ then there is only this one
occurrence of $x$ in $\Delta$,
\item\label{ca:el2} if $x\Incl r \in\Gamma$ then $x$ is different from the LHS of any
inclusion modified by an application of $\Ear$ in $D$,
% this does not seem necessary \item $x\not\in\Vars(t)$,
\item\label{ca:six} if $s\Incl x \in\Gamma$ then $x$ is different from any variable
modified by $\Lsix$ in $D$, and
%$x$ does not occur in any equation
%nor LHS of any inclusion in $\Delta$
\item\label{ca:aix} if $s\Incl x \in\Gamma$ then $x$ is different from any variable
modified by $\Laix$ in $D$
\end{enumerate}
\noindent
then $\der{\spNEQ}{D^*}{(\GSD)_t^x}$ (with {\em all} $x$'s replaced by $t$)
\end{LEMMA}
\begin{PROOF}
The intention of these conditions is to enable exactly the same derivations
with $t$ occurring instead of $x$.
The case \ref{ca:ax} prevents us from concluding derivability of $\GSD, t=t$
from the fact that $\GSD,x=x$, being an axiom, is derivable. 
In all other cases, we can
 start derivation with an axiom where $t$ occurs instead of $x$. All the
 rules applied in $D$ and involving $x$ can be applied 
in the same order in $D^*$ with $t$ instead of $x$.
The possible exceptions  are: % $\Ear$: %($E_{ar}^*$):
\[ \sma{ \begin{array}{c@{\ \ \ \ \ \ \ }c@{\ \ \ \ \ \ \ }c}
D \left \{ \begin{array}{rl}
 \multicolumn{1}{c}{\vdots} \\
 y\Incl q, x\Incl r(y),\GSD  \\ \cline{1-1}
 x\Incl r(q),\GSD  % & \rabove{\Ear} 
 \end{array} \right . 
&
D \left \{ \begin{array}{rl}
 \multicolumn{1}{c}{\vdots} \\
 s\Incl x,\GSD, p(s)\preceq q  \\ \cline{1-1}
 s\Incl x,\GSD, p(x)\preceq q  % & \rabove{\Lsix} 
 \end{array} \right . 
&
D \left \{ \begin{array}{rl}
 \multicolumn{1}{c}{\vdots} \\
 s\Incl x, p(x)\Incl q,\GSD  \\ \cline{1-1}
 s\Incl x, p(s)\Incl q, \GSD  % & \rabove{\Laix} 
\end{array} \right . \\[5ex]
 \Ear &  \Lsix &  \Laix
\end{array} }
\]
The sequent resulting from $\Ear$ cannot be obtained with $t$ substituted for $x$ by the
application of $\Ear$ unless $t$ is a variable. This case, however, is
excluded by the condition \ref{ca:el2}.
Similarly, sequents resulting from $\Lsix$, resp. $\Laix$, cannot be obtained with
$t$ instead of $x$ 
if $t$ isn't a variable. Conditions \ref{ca:six}, resp. \ref{ca:aix}, 
exclude these cases.
\end{PROOF}

\begin{DEFINITION}\label{de:neq6}
 $\elNEQ'$ is $\spNEQ$ but without the elimination rules $\Esr$
 and $\Ear$. %($E_{ar}^*$).
\end{DEFINITION}
\noindent
The rules of the calculus  $\elNEQ'$ for the case without specific axioms 
are given in figure \ref{fi:neq6}. 

\begin{figure}[hbt]
\hspace*{2em}
\begin{tabular}{|l@{\ \ \ \ \ \ \ \ \ \ \ \ }ll|}
\hline
\multicolumn{2}{|c}{{\bf Axioms}:} & \\[1ex]
\multicolumn{3}{|c|}{$\GSD, t\Incl t$;\ \ \ \ \ \ $\GSD, x=x$;\ \ \
\ \ \ 
$s=t,\GSD, s=t$ }\\[2ex]
%
\multicolumn{2}{|c}{{\bf Identity rules}:} & \\[1ex]
 $\eLs$\ \prule{t=s,\GSD, p(s)\preceq q}{t=s,\GSD, p(t)\preceq q}
& 
$\eLanx$\ \prule{s=t, p(s)\Incl q,\GSD}{s=t, p(t)\Incl q,\GSD} 
       \ \ {\footnotesize{$p(t)\not\in\Vars$}}    & \\[2.5ex]
& $\eRanx$\ \prule{s=t, p\Incl q(s), \GSD}{s=t, p\Incl q(t),\GSD} 
        \ \ {\footnotesize{$q(t)\not\in\Vars$}}   &  \\[3ex]
% & 
%$\eRs$\ \prule{s=t, \GSD, p\Incl q(s)}{s=t,\GSD, p\Incl q(t)}   & & \\[2ex]
%
\multicolumn{2}{|c}{{\bf Inclusion rules}:}
& \\[1ex]
$\iLs$\ \prule{t\Incl s, \Gamma\Seq \Delta, p(s)\preceq q}{t\Incl s, \Gamma\Seq
\Delta, p(t)\preceq q} & 
$\iLa$\ \prule{s\Incl t, p(s)\preceq q, \GSD}{s\Incl t, p(t)\preceq q, \GSD} 
%       \ \ {\footnotesize{$t\not\in\Vars$}}
   & \\[3ex]
%
$\iRs$\ \prule{s\Incl t, \GSD, p\Incl q(s)}{s\Incl t,\GSD, p\Incl q(t)}  & & \\[3ex]
$\Lsix$\ \prule{t\Incl x,\GSD, p(t)\preceq q}{t\Incl x,\GSD, p(x)\preceq q} & 
$\Laix$\ \prule{t\Incl x, p(x)\Incl q,\GSD}{t\Incl x,p(t)\Incl q,\GSD} 
       \ \ {\footnotesize{$p(t)\not\in\Vars$}} 
& \\[2.5ex]
 \hline
\end{tabular} 
\caption{The rules of $\elNEQ'$ ($x\in\Vars$) -- no specific axioms}\label{fi:neq6}
\end{figure}
%
\begin{REMARK}\label{re:oldhold}
Revisiting the proof of lemma \ref{le:nott}, %\ref{le:nocutx}, 
we see that the transformations used 
there did not introduce any new applications of the elimination rules 
(neither did lemma~\ref{le:noeqeq} used in this proof).
The same holds (trivially) for the proof of lemma~\ref{le:xok} 
(cf.~lemma~\ref{le:noxeq}).
Hence: %, these lemmas hold also for $\elNEQ'$, in particular:
\begin{enumerate}\MyLPar
% \item $\der{\elNEQ'}{}{x=x, \GSD}\ \impl\ \der{\elNEQ'}{}{\GSD}$;
\item ($cut_x$) is admissible in $\elNEQ'$ (cf. lemma \ref{le:xok}, \ref{le:noxeq});
\item ($cut_t$) is admissible in $\elNEQ'$ (cf. lemma \ref{le:nott});
%% $\der{\elNEQ'}{}{t\Incl t, \GSD}\ \impl\ \der{\elNEQ'}{}{\GSD}$ (lemma \ref{le:nott});
%\item ($cut$) is admissible in $\elNEQ'$;
\item lemma \ref{le:ssp} holds for $\elNEQ'$.
\end{enumerate}
\end{REMARK}
\noindent
Using this observation, we get:
%
\begin{CLAIM}\label{le:spneqiselneq}
In the absence of specific axioms $\spNEQ \equiv \elNEQ'$
\end{CLAIM}
\begin{PROOF}
 $\Leftarrow$ is obvious, and for $\impl$ we have to show admissibility of
 the elimination rules in $\elNEQ'$. Consider the uppermost application of an
 elimination rule $E$:
\begin{LS}
\item $E$ is $\Ear$: %($E_{ar}^*$):
 \[ \begin{array}{rl}
D_1 \left \{ \begin{array}{c}
\vdots \\
 x\Incl t, y\Incl r(x),\GSD \end{array} \right . \\ \cline{1-1}
 y\Incl r(t), \GSD & \rabove{\Ear} \end{array} 
\conv 
\begin{array}{rl}
D_1' \left \{ \begin{array}{c}
\vdots \\
 t\Incl t, y\Incl r(t),\GSD \end{array} \right . \\ \cline{1-1}
 y\Incl r(t), \GSD & \rabove{(cut_t)} \end{array} 
\]
The application of $\Ear$ implies $x\not\in\Vars(t,y,\Gamma,\Delta)$, and hence the
conditions of lemma \ref{le:ssp} are satisfied. Applying it to $D_1$, we
obtain $D'_1$ and $(cut_t)$ is admissible by remark \ref{re:oldhold}.
%%  \[ \begin{array}{rl}
%% D_1' \left \{ \begin{array}{c}
%% \vdots \\
%%  t\Incl t, y\Incl r(t),\GSD \end{array} \right . \\ \cline{1-1}
%%  y\Incl r(t), \GSD & \rabove{(cut_t)} \end{array} \]
\item $E$ is $\Esr$:
 \[ \begin{array}{rl}
      D_1 \left \{ \begin{array}{c}
         \vdots \\
        x\Incl t,\GSD, \phi[x] \end{array} \right . \\ \cline{1-1}
     \GSD, \phi[t] & \rabove{\Esr} \end{array} 
\conv
\begin{array}{rl}
      D_1 \left \{ \begin{array}{c}
         \vdots \\
        t\Incl t,\GSD, \phi[t] \end{array} \right . \\ \cline{1-1}
     \GSD, \phi[t] & \rabove{(cut_t)} \end{array} 
\]
By the restrictions on $\Esr$, there is at most one occurrence of $x$ in 
$\Gamma,\Delta,\phi[x]$ so, again, the
conditions of lemma \ref{le:ssp} hold and we can apply it to $D_1$ obtaining $D'_1$, 
% $t\Incl t,\GSD, \phi[t]$, 
which leads to the desired conclusion 
% $\der{\elNEQ}{}{\GSD,\phi[t]}$ 
by admissibility of $(cut_t)$. \vspace*{-1ex}
\end{LS}
\end{PROOF}


\begin{LEMMA}\label{le:noiRs}
The rule $\iRs$ is redundant.
\end{LEMMA}
\begin{PROOF}
\begin{LS}
\item Axiom
\[\prarc{
s\Incl t, \GSD, q(s)\Incl q(s) \cl
s\Incl t, \GSD, q(s)\Incl q(t) & \rabove{\iRs}
}\conv
\prarc{
s\Incl t, \GSD, q(t)\Incl q(t) \cl
s\Incl t, \GSD, q(s)\Incl q(t) & \rabove{\iLs}
}
\]
\item $\Laix$
\[\sma{\prarc{
t\Incl x, p(x)\Incl q, \GSD, r\Incl f(p(t)) \cl
t\Incl x, p(t)\Incl q, \GSD, r\Incl f(p(t)) & \rabove{\Laix} \cl
t\Incl x, p(t)\Incl q, \GSD, r\Incl f(q) & \rabove{\iRs}
}\conv
\prarc{
t\Incl x, p(x)\Incl q, p(t)\Incl q, \GSD, r\Incl f(p(t)) \cl
t\Incl x, p(x)\Incl q, p(t)\Incl q, \GSD, r\Incl f(q) & \rabove{\iRs} \cl
t\Incl x, p(t)\Incl q, p(t)\Incl q, \GSD, r\Incl f(q) & \rabove{\Laix} 
} }
\]
\item $\iLa$, $\eLanx$, $\eRanx$ -- these are treated by the same kind of weakening
as the above case of $\Laix$
\item The remaining rules modify at most LHS of a formula in the consequent and thus
allow trivial swaps.
\end{LS}
\end{PROOF}

\noindent
Thus all multialgebraic tautologies are derivable with the calculus $\elNEQ$ given
in figure~\ref{fi:neq6a}.

\begin{figure}[hbt]
\hspace*{2em}
\begin{tabular}{||l@{\ \ \ \ \ \ \ \ \ \ \ \ }ll||}
\hline\hline
\multicolumn{2}{||c}{{\bf Axioms}:} & \\[1ex]
\multicolumn{3}{||c||}{$\GSD, t\Incl t$;\ \ \ \ \ \ $\GSD, x=x$;\ \ \
\ \ \ 
$s=t,\GSD, s=t$ }\\[2ex]
%
\multicolumn{2}{||c}{{\bf Identity rules}:} & \\[1ex]
 $\eLs$\ \prule{t=s,\GSD, p(s)\preceq q}{t=s,\GSD, p(t)\preceq q}
& 
$\eLanx$\ \prule{s=t, p(s)\Incl q,\GSD}{s=t, p(t)\Incl q,\GSD} 
       \ \ {\footnotesize{$p(t)\not\in\Vars$}}    & \\[2.5ex]
& $\eRanx$\ \prule{s=t, p\Incl q(s), \GSD}{s=t, p\Incl q(t),\GSD} 
        \ \ {\footnotesize{$q(t)\not\in\Vars$}}   &  \\[3ex]
% & 
%$\eRs$\ \prule{s=t, \GSD, p\Incl q(s)}{s=t,\GSD, p\Incl q(t)}   & & \\[2ex]
%
\multicolumn{2}{||c}{{\bf Inclusion rules}:}
& \\[1ex]
$\iLs$\ \prule{t\Incl s, \Gamma\Seq \Delta, p(s)\preceq q}{t\Incl s, \Gamma\Seq
\Delta, p(t)\preceq q} & 
$\iLa$\ \prule{s\Incl t, p(s)\preceq q, \GSD}{s\Incl t, p(t)\preceq q, \GSD} 
%       \ \ {\footnotesize{$t\not\in\Vars$}}
   & \\[3ex]
%
%$\iRs$\ \prule{s\Incl t, \GSD, p\Incl q(s)}{s\Incl t,\GSD, p\Incl q(t)}  & & \\[3ex]
$\Lsix$\ \prule{t\Incl x,\GSD, p(t)\preceq q}{t\Incl x,\GSD, p(x)\preceq q} & 
$\Laix$\ \prule{t\Incl x, p(x)\Incl q,\GSD}{t\Incl x,p(t)\Incl q,\GSD} 
       \ \ {\footnotesize{$p(t)\not\in\Vars$}} 
& \\[2.5ex]
 \hline\hline
\end{tabular} 
\caption{The rules of $\elNEQ$ ($x\in\Vars$) -- no specific axioms}\label{fi:neq6a}
\end{figure}
%



\section{Other connectives}\label{se:connectives}
Since satisfaction of a sequent is defined in the usual way, soundness/completeness
%sound\-ness/com\-ple\-te\-ness 
of $NEQ$ and its equivalence with $NEQ_5$ imply that the entailment
relation in $NEQ_5$ coincides with the semantic consequence. If we restrict
our formulae $\phi$ to the atomic ones (i.e., $s\Incl t$ or $s=t$), this
means that, given a specification with specific axioms $\Ax = \{\phi_1,...,\phi_n\}$, 
 $\MMod(\Ax)\models \phi \Leftrightarrow \der{NEQ_5}{}{\Ax\Seq\phi}$. Since the specific
 axioms are thus incorporated into the actual sequents to be proved, we need
 no ($Sp.cut$) rules and can
 conclude that  \[ \MMod(\Ax)\models \phi \iff \der{\elNEQ}{}{\Ax\Seq\phi}\]
\noindent
Notice that this formulation, however simple and convincing, requires us to
``guess'' the variable names which have to be used in $\phi$ in order to
``match'' the appropriate variable names in the axioms.

Furthermore, we may extend the language introducing other logical operators
in the usual way.
We obtain a sound and complete system $NEQ_i^+$ (for $i\leq 6$) by extending
 $NEQ_i$ with the standard rules \\[1ex]
%
\begin{tabular}{rl@{\hspace*{6em}}rl}
($\neg\ \Seq$) & \PROOFR{\Gamma\Seq\Delta,\phi}{\Gamma,\neg\phi\Seq\Delta} & 
($\Seq\ \neg$) & \PROOFR{\Gamma,\phi\Seq\Delta}{\Gamma\Seq\Delta,\neg\phi} \\
($\lor\ \Seq$) & \PROOFR{\Gamma,\phi_1\Seq\Delta\ \ ;\ \
\Gamma,\phi_2\Seq\Delta}{\Gamma,\phi_1\lor\phi_2\Seq\Delta}
& ($\Seq\ \lor$) &
\PROOFR{\Gamma\Seq\Delta,\phi_1,\phi_2}{\Gamma\Seq\Delta,\phi_1\lor\phi_2} \\
($\land\ \Seq$) &
\PROOFR{\Gamma,\phi_1,\phi_2\Seq\Delta}{\Gamma,\phi_1\land\phi_2\Seq\Delta} &
($\Seq\ \land$) & \PROOFR{\Gamma\Seq\Delta,\phi_1\ \ ;\ \
\Gamma\Seq\Delta,\phi_2}{\Gamma\Seq\Delta,\phi_1\land\phi_2} \\
($\impl\ \Seq$) & \PROOFR{\Gamma,\phi_1\Seq\Delta\ \ ;\ \
\Gamma\Seq\Delta,\phi_2}{\Gamma,\phi_2\impl\phi_1\Seq\Delta}  & ($\Seq\ \impl$) &
\PROOFR{\Gamma,\phi_1\Seq\Delta,\phi_2}{\Gamma\Seq\Delta,\phi_1\impl\phi_2} 
\end{tabular} \\[1ex]
%
We then have that, for a set of specific axioms $\Ax=\{\Phi_1,...,\Phi_n\}$ and a
formula $\Phi$, all built from the atomic equalities and
inclusions with the logical operators,  $\MMod(\Ax)\models \Phi \Leftrightarrow
 \der{NEQ_5^+}{}{\Ax\Seq\Phi}$. A proof in $NEQ_5^+$ will typically consist of a series
 of applications of the above eight rules leading to a sequent
 $\Gamma\Seq\Delta$, to which one then applies the rules of $NEQ_5$.
Here again axioms are incorporated into the formulae, so we can use $\elNEQ$
instead:
 \[ \MMod(\Ax)\models \Phi \iff  \der{\elNEQ^+}{}{\Ax\Seq\Phi} \]

%\section{Further specialization of $Sp.cut$}
%\PROOFRULE{\Gamma_i\Seq\Delta_i,r_i\odot_i s_i\ \ ;\ \ 
%u_j\odot_j v_j,\Gamma_j\Seq\Delta_j} {\bigc_i\Gamma_i,\bigc_j\Gamma_j
%\Seq \bigc_i\Delta_i,\bigc_j\Delta_j}
%



