
\begin{LEMMA}\label{le:elcut}
If $D_1$, $D_2$ are $NEQ_4$ derivations as follows:
%\begin{center}
\[ \begin{array}{cl}
 D_1\left\{ \begin{array}{c}
  \vdots \\   \Gamma\Seq\Delta, \phi
 \end{array} \right.
 D_2\left\{ \begin{array}{c}
  \vdots \\   \phi, \Gamma'\Seq\Delta'
 \end{array} \right. \\ \cline{1-1}
\Gamma,\Gamma' \Seq \Delta,\Delta'
&   \raisebox{1.2ex}[1.5ex][0ex]{($cut$)}
\end{array} \]
%\end{center}
then $\der{NEQ_4}{}{\Gamma,\Gamma'\Seq\Delta,\Delta'}$.
\end{LEMMA}
\begin{PROOF}
By induction on $\< \#(\preceq_a,D_2), h(D_1), h(D_2)\>$, 
where the first parameter is the total number of applications of ($\Incl_a$) 
{\em or} ($=_{La}$) {\em which modify the cut formula} $\phi$ in $D_2$ (this
qualification is essential in case \ref{it:cutactive} on page
\pageref{it:cutactive}).
%  $h(D_1)$ is the height of $D_1$ and $h(D_2)$ the height of  $D_2$. 
\\[1ex]
\noindent 
If $h(D_1)=0$, then the resulting sequent of $D_1$ is an axiom. If
the cut formula is not the one mentioned explicitly in the axioms (Fig.~\ref{fi:neq4}) 
then we obtain a cut-free derivation directly by choosing
another instance of the same axiom. In the other case, if the cut formula is
 $t\Incl t$ or $x=x$, the assumption 4. or 3., respectively, allows us to conclude the
 existence of a derivation of the conclusion of $D_2$ without this formula in
 the antecedent. Finally, if the axiom in $D_1$ is
 $\Gamma,s=t\Seq s=t,\Delta$ and the cut formula is $s=t$, then it will also appear in the
 antecedent of the conclusion of ($cut$). Then this conclusion can be
 obtained without ($cut$) directly from $D_2$ by starting it with the
 instance of the axiom extended with $\Gamma$ and $\Delta$. \\[1ex]
\noindent
For $h(D_1)>0$, we consider the last rule $R$ applied in $D_1$. \\[1ex]
\noindent
{\bf I.} ($=_{sr}$):
\[ D \left \{ \begin{array}[t]{cl}
 \begin{array}{cl}
 D_1^*\left\{ \begin{array}{c}
  \vdots \\   t=s,\Gamma\Seq\Delta, w(s)\preceq q
 \end{array} \right. \\ \cline{1-1}
t=s,\Gamma\Seq\Delta, w(t)\preceq q & \raisebox{1.2ex}[1.5ex][0ex]{($=_{sr}$)}
 \end{array}
 D_2\left\{ \begin{array}{c}
  \vdots \\ \vdots \\  w(t)\preceq q, \Gamma'\Seq\Delta'
 \end{array} \right. \\ \cline{1-1}
t=s,\Gamma,\Gamma' \Seq \Delta,\Delta'
&   \raisebox{1.2ex}[1.5ex][0ex]{($cut$)}
\end{array} \right . \convd
%
%Instead, we drop the last application in $D_1$ and construct following
%derivation from $D_2$:
 D' \left \{ \begin{array}[t]{cl}
 D_1^*\left\{ \begin{array}{c}
  \vdots \\ \vdots \\  t=s,\Gamma\Seq\Delta, w(s)\preceq q
 \end{array} \right. 
\begin{array}{cl}
 D_2\left\{ \begin{array}{c}
  \vdots \\   w(t)\preceq q, t=s, \Gamma'\Seq\Delta'
 \end{array} \right. \\ \cline{1-1}
w(s)\preceq q, t=s, \Gamma'\Seq\Delta' & \rabove{(=_a)}
 \end{array} \\ \cline{1-1}
t=s,\Gamma,\Gamma' \Seq \Delta,\Delta'
&   \raisebox{1.2ex}[1.5ex][0ex]{($cut$)}
\end{array} \right . \]
 $t$ cannot be a variable, since this would imply that the application in
 $D_1$ was ($=_{sx}$), what is excluded (by lemma \ref{le:noInclsx}, cf. Fig.~\ref{fi:neq4}).
Hence, the application of ($=_a$) is either ($=_{a=}$) or ($=_{Lanr}$). The former can
be eliminated by lemma \ref{le:noeqeq}, and the latter by \ref{le:noLanr} 
(and corollary~\ref{co:neq3isneq4}) --
both without intcreasing the number of applications of ($\preceq_a$).
Thus, $\#(\preceq_a, D'_2)$ is not greater
than in the original $D_2$, while the height of $D_1$ at which to
perform ($cut$) has been reduced.  \\[1ex]
%
\noindent
{\bf II.} ($\Incl_s$): We do a similar trick as in the previous case:
%\item  
\[ D \left \{\begin{array}[t]{cl}
 \begin{array}{cl}
 D_1^*\left\{ \begin{array}{c}
  \vdots \\   s\Incl p,\Gamma\Seq\Delta, w(p)\preceq t
 \end{array} \right. \\ \cline{1-1}
s\Incl p,\Gamma\Seq\Delta, w(s)\preceq t & \raisebox{1.2ex}[1.5ex][0ex]{($\Incl_s$)}
 \end{array}
 D_2\left\{ \begin{array}{c}
  \vdots \\   w(s)\preceq t, \Gamma'\Seq\Delta'
 \end{array} \right. \\ \cline{1-1}
s\Incl p,\Gamma,\Gamma' \Seq \Delta,\Delta'
&   \raisebox{1.2ex}[1.5ex][0ex]{($cut$)}
\end{array} \right . \convd
%
%We do a similar trick as in the previous case:
 D' \left \{ \begin{array}[t]{cl}
 D_1^*\left\{ \begin{array}{c}
  \vdots \\ \vdots \\  s\Incl p ,\Gamma\Seq\Delta, w(p)\preceq q
 \end{array} \right. 
\begin{array}{cl}
 D_2\left\{ \begin{array}{c}
  \vdots \\   w(p)\preceq q, s\Incl p, \Gamma'\Seq\Delta'
 \end{array} \right. \\ \cline{1-1}
w(s)\preceq q, s\Incl p, \Gamma'\Seq\Delta' & \rabove{(\Incl_a)}
 \end{array} \\ \cline{1-1}
s\Incl p ,\Gamma,\Gamma' \Seq \Delta,\Delta'
&   \raisebox{1.2ex}[1.5ex][0ex]{($cut$)}
\end{array} \right . \]
 $s$ cannot be a variable because then the application of ($\Incl_s$) in
 $D_1$ would be ($\Incl_{sx}$) -- cf. lemma \ref{le:noInclsx}. Thus, the application of
 ($\Incl_a$) after $D_2$ in $D'$ is not reduced, and this can be eliminated
 by lemma \ref{le:noar} (and corollary~\ref{co:neq3isneq4}) 
without increasing the number 
 of applications of ($\preceq_a$) in $D_2$. So ($cut$) can be eliminated
 using the second parameter of induction hypothesis. \\[1ex]
%
\noindent
{\bf III.} ($\Incl_{ar}$), ($E_a$), ($=_{Lar}$) or ($=_{Rar}$): Since none of
these rules modifies the succedent, their application as $R$ in $D_1$ means
that we can apply ($cut$) with $D_1^*$ instead, and the induction on $h(D_1$)
gives the conclusion. \\[1ex]
\noindent
{\bf IV.} ($Sp.cut$) or ($cut_x$): Obvious induction on $h(D_1$). \\[1ex]
\noindent
{\bf V.} ($E_s$):
%
\[ \begin{array}{cl}
D_1\left\{ \begin{array}{cl}
 D_1^*\left\{ \begin{array}{cl}
  \vdots \\ 
  x\Incl t,\Gamma\Seq\Delta, \phi_x^x  
 \end{array} \right. \\ \cline{1-1}
\Gamma\Seq\Delta, \phi_t^x & \raisebox{1.2ex}[1.5ex][0ex]{($E_s$)}
 \end{array} \right .
 D_2\left\{ \begin{array}{cl}
  \vdots \\ \cline{1-1}  \phi_t^x, \Gamma'\Seq\Delta' & \raisebox{1.2ex}[1.5ex][0ex]{$R'$}
 \end{array} \right. \\ \cline{1-1}
\Gamma,\Gamma' \Seq \Delta,\Delta'
&   \raisebox{1.2ex}[1.5ex][0ex]{($cut$)}
\end{array} \]
%
By restrictions on ($E_s$) $x\not\in\C V(\Gamma,\Delta,t)$ and we can choose
 $x$ so that $x\not\in\C V(\Gamma',\Delta')$. We consider two cases: \ref{it:Ar} when
in $D_1$, $x$ in $\phi$ is in the RHS of $\Incl$, 
and \ref{it:Bl} when $x$ occurs in
a LHS of $\Incl$ or when $\phi$ is an equality. 
\begin{LS}
%
%
\item\label{it:Ar} Let $\phi$ be $p\Incl s(t)$  \\[.5ex]
\noindent
-- $t$ indicating the only occurrence
which has been substituted for $x$. We can then construct the following 
derivation extending $D_2$:
%
\[D_2^*\left\{ \begin{array}{cl}
  D_2\left\{ \begin{array}{cl}
  \vdots \\ 
  p\Incl s(t),\Gamma'\Seq\Delta' \end{array} \right.  \\ \cline{1-1}
 x\Incl t,p\Incl s(t),\Gamma'\Seq\Delta' 
   & \raisebox{1.2ex}[1.5ex][0ex]{(W)} \\ \cline{1-1}
x\Incl t, p\Incl s(x),\Gamma'\Seq\Delta' 
& \raisebox{1.2ex}[1.5ex][0ex]{($\Incl_a^*$)}
\end{array} \right. \]
\noindent
The rule ($\Incl_a^*$) is admissible by lemma \ref{le:inclaad} (actually, 
lemma~\ref{le:asinNEQ3}) and 
 it does not increase the number of
applications of ($\preceq_a$). Thus,
conclusion of this derivation can be ($cut$) with the conclusion of $D_1^*$ at
the height $h(D_1^*)<h(D_1)$, which can be eliminated by induction on
$h(D_1)$.  Final application of ($E_s$) removes
$x\Incl t$ (since, by the assumption, $x\not\in\C V(t,\Gamma,\Gamma',\Delta,
\Delta'$)), and gives a cut-free derivation of $\Gamma,\Gamma'\Seq\Delta,\Delta'$.
%
%
\item\label{it:Bl} Let $\phi$ be $s(t)\preceq p$ \\[.5ex]
\noindent
Here we have three subcases, depending on
whether $\phi$ in the application of the last rule $R'$ in $D_2$ was 
\ref{it:cutneither} neither modified nor active, \ref{it:cutmodified} modified, 
or \ref{it:cutactive} active.
%
\begin{LSA}
%
\item\label{it:cutneither} The cut formula $\phi$ is neither modified nor active.\\
We may then swap the
applications of $R'$ and ($cut$), and the induction on the height of $D_2$
yields the required elimination of ($cut$). 
%
\item\label{it:cutmodified}  $\phi$ is modified by $R'$.\\
%%\begin{LSA}
%%  \item $\phi$ is inclusion $s(t)\Incl p$ (modifed by $R'$).  \\[.5ex]
The two possibilities of $\phi$ being an inclusion and equality are entirely 
analogous. This case depends only on the rule $R'$ -- if $\phi$ is an equality, 
$R'$ cannot be ($=_a$), but except for that the two cases are identical.
%  {\sf II.a) 
%
 \begin{LSB}
   \item $R'$ is ($E_a$) or ($=_{Rar}$): \\
%II.a.$\alpha$ 
  This would require $\phi$ to be an inclusion with  a single
   variable in the LHS, i.e, $y\Incl p$. %$s(t)=y$. 
Since in the current case \ref{it:Bl},
  ($E_s$) in $D_1$
   replaces $x$ in the LHS of $\Incl$, this would mean that this application is
   actually degenerate ($E_{sd}$), what is excluded by lemma \ref{le:noEsd}. 
  \item  $R'$ is ($\Incl_{ar}$) or ($=_{Lar}$): \\
%II.a.$\beta$ 
  \noindent
   These two cases are identical so we treat them jointly and
  write ($\preceq_{ar}$). In each subcase, all occurrences of ($\preceq$) must be
  replaced consistently either by ($\Incl_{ar}$) or by ($=_{Lar}$), unless we
  mention the latter rules explicitly. The cut formula $\phi$ 
has the form $\phi[x]$ and $\phi[t]$ after the application of ($E_s$) in $D_1$, while in $D_2$ we 
write it as $\phi'[y]$, and as $\phi'[t']$ after the final application of ($\preceq_a$), i.e., 
 $\phi[t] = \phi = \phi'[t']$. This is to indicate that the term $t$ introduced in $D_1$ and
$t'$ introduced in $D_2$ into $\phi$ may be different. The relation between the two
 does not matter, however -- what makes it possible to treat all the (sub)cases in
the same way, is the fact that if $\phi'[y]$ is an inclusion the modified term is introduced into
its LHS.

$D_2$ ends as follows:
\[ D_2 \left \{ \begin{array}{cl}
 D_2^* \left \{ \begin{array}{cl}
 \vdots \\
 \phi'[y], y\preceq t', \Gamma'\Seq\Delta' \end{array} \right . \\ \cline{1-1}
 \phi'[t'], y\preceq t', \Gamma'\Seq\Delta' & \rabove{(\preceq_{ar})}
 \end{array} \right . \]
We drop this last application of ($\preceq_{ar}$) and, instead, extend $D_1$
weakened with $y\preceq t'$. % \\[.5ex]
%
% \mbox{ 
{ \scriptsize 
\[\begin{array}{cl}
D_1' \left \{ \begin{array}{cl}
D_1 \left \{ \begin{array}{cl}
  \vdots \\ 
  y\preceq t', x\Incl t,\Gamma\Seq\Delta, \phi[x]  \\ \cline{1-1}
y\preceq t', \Gamma\Seq\Delta, \phi[t]  & \raisebox{1.2ex}[1.5ex][0ex]{($E_s$)}
 \end{array} \right . \\ \cline{1-1}
y\preceq t', \Gamma\Seq\Delta, \phi'[y] &
\raisebox{1.2ex}[1.5ex][0ex]{($\preceq_s$)}
 \end{array} \right .
%
 D_2^*\left\{ \begin{array}{cl}
 \vdots \\
\vdots \\
\phi'[y], y\preceq t', \Gamma'\Seq\Delta' \end{array} \right .
 \\ \cline{1-1}
y\preceq t', \Gamma,\Gamma' \Seq \Delta,\Delta' &   \raisebox{1.2ex}[1.5ex][0ex]{($cut$)}
\end{array} \] }
%  \) }} \\[.5ex]
($cut$) can be eliminated by induction hypothesis on $\#(\preceq_a,D_2)$.
%
%%\end{LSB}
%
\end{LSB}
%
%
%
\item\label{it:cutactive} %LSA 
 $\phi$ is active in $R'$
\begin{LSB}
\item\label{it:inact} $\phi$ is inclusion $s(t)\Incl p$ (active in $R'$),
\\
%  {\sf III.a) 
that is $R'$ is either ($\Incl_s$) or ($\Incl_{ar}$).
The latter case is excluded because it would require $s(t)$ to be a variable.
Then we would have a degenerate application of ($E_s$) in $D_1$, what is excluded by the
assumption. So let
 $R'$ be $(\Incl_s$). We have the following derivation: 
% \\[0.5ex]
%\hspace*{-2.5em}
{\scriptsize \[ \begin{array}{cl} \hspace*{-4em}
D_1 \left \{ \begin{array}{cl}
 D_1^*\left\{ \begin{array}{cl}
  \vdots \\ 
  x\Incl t,\Gamma\Seq\Delta, s(x)\Incl p  
 \end{array} \right. \\ \cline{1-1}
\Gamma\Seq\Delta, s(t)\Incl p & \raisebox{1.2ex}[1.5ex][0ex]{($E_s$)}
 \end{array} \right .
 D_2\left\{ \begin{array}{cl}
  D_2' \left \{ \begin{array}{c}
\vdots \\
s(t)\Incl p, \Gamma'\Seq\Delta', w(p)\preceq q \end{array} \right .
\\ \cline{1-1}  
s(t)\Incl p, \Gamma'\Seq\Delta', w(s(t))\preceq q & \raisebox{1.2ex}[1.5ex][0ex]{($\Incl_s$)}
 \end{array} \right. \\ \cline{1-1}
\Gamma,\Gamma' \Seq \Delta,\Delta', w(s(t))\preceq q
 &   \raisebox{1.2ex}[1.5ex][0ex]{($cut$)}
\end{array} \] }
%} \\[.5ex]
%
First, construct the derivation $M'$ by cutting $s(t)\Incl p$ after $D_1$ and
 $D_2'$. Since $h(D_2')<h(D_2)$ this ($cut$) can be eliminated using the third argument of 
induction. Then extend $M'$ to $M$  as follows:
\[ \hspace*{-1em} M \left \{ \begin{array}{cl} 
  M' \left \{ \begin{array}{cl}
 D_1 \left \{ \begin{array}{c}
 \vdots \\
 \GSD, s(t)\Incl p \end{array} \right .
 D_2' \left \{ \begin{array}{c}
 \vdots \\
 s(t)\Incl p, \Gamma'\Seq\Delta', w(p)\preceq q \end{array} \right . \\ \cline{1-1}
 \Gamma,\Gamma'\Seq\Delta,\Delta', w(p)\preceq q & \rabove{(cut)} \end{array} \right . \\
 \cline{1-1}
 s(x)\Incl p, \Gamma,\Gamma'\Seq\Delta,\Delta',w(p)\preceq q
 &   \raisebox{1.2ex}[1.5ex][0ex]{($W_a$)} \\ \cline{1-1}
 s(x)\Incl p, \Gamma,\Gamma'\Seq\Delta,\Delta',w(s(x))\preceq q
 &   \raisebox{1.2ex}[1.5ex][0ex]{($\Incl_s$)}
 \end{array} \right . \]
The application of ($cut$) to $M$ with $D_1^*$ can be eliminated
by induction hypothesis $h(D_1^*)<h(D_1)$ -- notice that we are using here
the fact that $\#(\preceq_a,M)$ {\em modifying the cut formula} is not greater
than $\#(\preceq_a,D_2')$, even if the total number of arbitrary applications
of ($\preceq_a$) in $M$ may be far greater than in $D_2'$ (due to
applications in $D_1$).
It yields the following sequent leading
to the desired conclusion by an application of ($E_s$) -- $x$ may be chosen so
that $x\Not\in\Vars(\Gamma',\Delta',w,q$):
\[ \begin{array}{cl}
 x\Incl t, \Gamma, \Gamma' \Seq \Delta, \Delta', w(s(x))\preceq q \\
 \cline{1-1}
 \Gamma, \Gamma' \Seq \Delta, \Delta', w(s(t))\preceq q
 &   \raisebox{1.2ex}[1.5ex][0ex]{($E_s$)}
\end{array} \]
\noindent
%
\item % LSB {\sf III.b) 
 $\phi$ is equality $s(t)= p$ (active in $R'$). \\
 We have three cases for $R'$ which are all treated analogously to
the previous case \ref{it:inact}.
% (III.a).
\begin{LSC}
\item $R'$ is ($=_s$) \\
This is treated exactly as \ref{it:inact} with applications of
($=_s$) instead of ($\Incl_s$). 
%
\item  $R'$ is ($=_{Lar}$) or ($=_{Rar}$)\\
We proceed as above with the construction of $M'$, $M$ and ($cut$). The
differences occur only in the last step, so we make the following generic
description where $\varphi$ is the side and $\varphi'$ the modified formula
of $R'$: 
% \\[.5ex]  \hspace*{-.5em} \mbox{ 
{\scriptsize
\[ \begin{array}{cl}  \hspace*{-1.5em}
D_1 \left \{ \begin{array}{cl}
 D_1^*\left\{ \begin{array}{cl}
  \vdots \\ 
  x\Incl t,\Gamma\Seq\Delta, s(x)= p  
 \end{array} \right. \\ \cline{1-1}
\Gamma\Seq\Delta, s(t)= p & \raisebox{1.2ex}[1.5ex][0ex]{($E_s$)}
 \end{array} \right .
 D_2\left\{ \begin{array}{cl}
  D_2' \left \{ \begin{array}{c}
\vdots \\
s(t)= p, \varphi, \Gamma'\Seq\Delta' \end{array} \right .
\\ \cline{1-1}  
s(t)= p, \varphi', \Gamma'\Seq\Delta',  & \raisebox{1.2ex}[1.5ex][0ex]{($R'$)}
 \end{array} \right. \\ \cline{1-1}
\varphi', \Gamma,\Gamma' \Seq \Delta,\Delta'
 &   \raisebox{1.2ex}[1.5ex][0ex]{($cut$)}
\end{array} \] }
% } } \\[.5ex] \noindent
First, construct the derivation $M'$ by cutting $s(t)=p$ after $D_1$ and
 $D_2'$ which is weakened with $s(x)=p$. $h(D_2')<h(D_2)$ means that this ($cut$)
 can be eliminated. This
yields the following sequent
\[  s(x)=p, \varphi, \Gamma, \Gamma' \Seq \Delta, \Delta' \]
leading to the desired conclusion by the procedure depending on
 $R'$. (Remember that $x$ may be chosen so
that $x\Not\in\Vars(\Gamma',\Delta',\varphi$), and $\#(\preceq_a,M')$ is not
greater than in $D_2'$):
\begin{LSD}
\item $R'$ is ($=_{Lar}$)\\
 $s(t)$ cannot be a variable (since then ($E_s$) in $D_1$ would be
 degenerate), so $p$ must be a variable $y$ and $\varphi$ is $f(y)\Incl q$, and $\varphi'$ is $f(s(t))\Incl q$.
\[ \begin{array}{cl}
D_1^* \left \{ \begin{array}{c} \vdots \\ 
   x \Incl t, \GSD, s(x)=y \end{array} \right . \ \ \ \ \ 
\begin{array}{rl}
 s(x)=y, f(y)\Incl q, \Gamma, \Gamma' \Seq \Delta, \Delta' \\
 \cline{1-1}
 s(x)=y, f(s(x))\Incl q, \Gamma, \Gamma' \Seq \Delta, \Delta' 
 &   \raisebox{1.2ex}[1.5ex][0ex]{($=_{Lar}$)}  
 \end{array} \\ \cline{1-1}
x\Incl t, f(s(x))\Incl q, \Gamma, \Gamma' \Seq \Delta, \Delta' 
 &   \raisebox{1.2ex}[1.5ex][0ex]{($cut$)} \\ \cline{1-1}
x\Incl t, f(s(t))\Incl q, \Gamma, \Gamma' \Seq \Delta, \Delta' 
 &   \raisebox{1.2ex}[1.5ex][0ex]{($\Incl_{ar}$)} \\ \cline{1-1}
f(s(t))\Incl q, \Gamma, \Gamma' \Seq \Delta, \Delta' 
 &   \raisebox{1.2ex}[1.5ex][0ex]{($E_s$)}
\end{array} \]
% \[ \begin{array}{rl}
%  s(x)=y, f(y)\Incl q, \Gamma, \Gamma' \Seq \Delta, \Delta' \\
%  \cline{1-1}
%  s(x)=y, f(s(x))\Incl q, \Gamma, \Gamma' \Seq \Delta, \Delta' 
%  &   \raisebox{1.2ex}[1.5ex][0ex]{($=_{Lar}$)} \\ \cline{1-1}
% x\Incl t, f(s(x))\Incl q, \Gamma, \Gamma' \Seq \Delta, \Delta' 
%  &   \raisebox{1.2ex}[1.5ex][0ex]{($cut$)\ with\ $s(x)=y$\ after\ $D_1^*$} \\ \cline{1-1}
% x\Incl t, f(s(t))\Incl q, \Gamma, \Gamma' \Seq \Delta, \Delta' 
%  &   \raisebox{1.2ex}[1.5ex][0ex]{($\Incl_a$)} \\ \cline{1-1}
% f(s(t))\Incl q, \Gamma, \Gamma' \Seq \Delta, \Delta' 
%  &   \raisebox{1.2ex}[1.5ex][0ex]{($E_s$)}
% \end{array} \]
This ($cut$) can be eliminated since $h(D_1^*)<h(D_1)$.
%
\item $R'$ is ($=_{Rar}$)\\
substituting $s(t)$ for $p$, i.e., $\varphi$ is $y\Incl f(p)$, and $\varphi'$ is
$y\Incl f(s(t))$ for a $y\in\Vars$:
\[ \begin{array}{cl}
D_1^* \left \{ \begin{array}{c} \vdots \\ 
 x\Incl t, \GSD, s(x)=p \end{array} \right . \ \ \ \ \ \ 
\begin{array}{rl}
 s(x)=p, y\Incl f(p), \Gamma, \Gamma' \Seq \Delta, \Delta' \\
 \cline{1-1}
 s(x)=p, y\Incl f(s(x)), \Gamma, \Gamma' \Seq \Delta, \Delta' 
 &   \raisebox{1.2ex}[1.5ex][0ex]{($=_{Rar}$)} 
 \end{array} \\ \cline{1-1}
x\Incl t, y\Incl f(s(x)), \Gamma, \Gamma' \Seq \Delta, \Delta' 
 &   \raisebox{1.2ex}[1.5ex][0ex]{($cut$)} \\ \cline{1-1}
y\Incl f(s(t)), \Gamma, \Gamma' \Seq \Delta, \Delta' 
 &   \raisebox{1.2ex}[1.5ex][0ex]{($E_a$)}
\end{array} \]
This ($cut$) can be eliminated since $h(D_1^*)<h(D_1)$.
%
\item $R'$ is ($=_{Rar}$)\\
substituting $p$ for $s(t)$, i.e., $\varphi$ is $y\Incl f(s(t))$, and $\varphi'$ is
 $y\Incl f(p)$ for a $y\in\Vars$. Here we have to weaken $D_2'$ also with $x\Incl t$:
\[ \begin{array}{cl} \hspace*{-2em}
D_1^* \left \{ \begin{array}{c} \vdots \\ 
 x\Incl t, \GSD, s(x)=p \end{array} \right . \ \ \ \ 
\begin{array}{rl}
 x\Incl t, s(x)=p, y\Incl f(s(t)), \Gamma, \Gamma' \Seq \Delta, \Delta' \\
 \cline{1-1}
x\Incl t, s(x)=p, y\Incl f(s(x)), \Gamma, \Gamma' \Seq \Delta, \Delta' 
 &   \raisebox{1.2ex}[1.5ex][0ex]{($\Incl_a^*$)} \\ \cline{1-1}
x\Incl t, s(x)=p, y\Incl f(p), \Gamma, \Gamma' \Seq \Delta, \Delta' 
 &   \raisebox{1.2ex}[1.5ex][0ex]{($=_{Rar}$)} 
 \end{array} \\ \cline{1-1}
x\Incl t, y\Incl f(p), \Gamma, \Gamma' \Seq \Delta, \Delta' 
 &   \raisebox{1.2ex}[1.5ex][0ex]{($cut$)} \\ \cline{1-1}
y\Incl f(p), \Gamma, \Gamma' \Seq \Delta, \Delta' 
 &   \raisebox{1.2ex}[1.5ex][0ex]{($E_s$)}
\end{array} \]
Again, ($cut$) can be eliminated since $h(D_1^*)<h(D_1)$.
\end{LSD}
\end{LSC}
\end{LSB}
\end{LSA}
\end{LS}
\end{PROOF}
