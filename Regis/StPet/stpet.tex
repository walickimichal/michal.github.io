\magnification=\magstep1
\hsize=13.5truecm
\vsize=19truecm
\tolerance=10000
\font\tit=cmbx12 scaled\magstep1
\font\titl=cmbx12
\font\titli=cmbx10
\font\normalfont=cmr10
\font\abstractsfont=cmr9
\font\otherboldfont=cmbx9
\def\rut{{\scriptstyle\bigcirc}}
\def\kva{{\scriptstyle\square}}
\def\ddr{\Rightarrow}
\def\rro{\omega}\def\rrO{\Omega}
\def\rrG{\Gamma}
\def\rrl{\lambda}
\def\cW{{\cal W}}
\def\abzac{\hskip 0.5cm}
\def\sat{{\rm Sat}}
\def\g{G_{L\rro}}
\def\tus{\varnothing}
\def\rightheadline{\headsfont\hfil \null \hfil\otherfont\folio}
\def\leftheadline{\otherfont\folio\headsfont\hfil \null\hfil}
\def\nothing{\hfill}
\def\skyrius#1{{\it#1}}
\def\skyr#1{\vskip .75cm{\bf #1}\medskip}
\def\sky#1{\vskip .5truecm{\titli #1}\vskip 0.35truecm}
%\nopagenumbers
\headline={\null}
\def\bruks{\hbox to 4truecm{\dotfill}}
\def\s{\v s}
\def\e{\. e}
\def\cv{\v c}
\def\abzac{\hskip 0.75cm}
\def\proof{{\it Proof.\ }}
\def\ref{\parindent=0.5truecm\noindent\hang}
\def\br{\t \nobreak --\allowbreak \t }
\def\mly{\leqslant}\def\dly{\geqslant}
\def\cut{{\rm cut}}
\def\sp{{\rm Sp.}}

%\vskip2truecm
\parindent=0pt

\normalfont
\centerline{\tit On Specialization of Derivations}
\centerline{\tit in Axiomatic Equality Theories}
\vskip 0.5cm
\centerline{A. Pliu\s kevi\cv ien\e ,\  R. Pliu\s kevi\cv ius}
\vskip 0.3cm
\centerline{\vtop{\hbox{Institute of Mathematics and Informatics}
\hbox{Akademijos 4, Vilnius 2600, LITHUANIA}
\hbox{email: logica@sedcs.mii2.lt}}}
\vskip 0.5truecm
\centerline{M. Walicki, S. Meldal}
\vskip 0.3truecm
\centerline{\vtop{\hbox{University of Bergen}
\hbox{Department of Informatics}
\hbox{HiB, N-5020 Bergen, NORWAY}
\hbox{e-mail: $\{$michal,sigurd$\}$@ii.uib.no}}}
%\vskip 0.75truecm
%spell_date
%spell_from

\skyr{Abstract}
\vbox{\abstractsfont
\baselineskip=9pt
\lineskip=0pt
Walicki and Meldal have defined a calculus $DEQ$ (``Disjunctive
EQuational calculus'') for reasoning about
nondeterministic operators when specifying nondeterministic systems in
an equation-oriented style.

A variant of $DEQ$, the calculus $DEQ^\ast$ for axiomatic equality theories
with cut-like rules introducing as cut formulas only the negative equalities
of a specific axiom, is constructed. For pure positive specific axioms
(i.e. with empty antecedents) and for so called non-contrary equality
theories $DEQ^\ast$ does not contain cut-like rules at all. The variant of
the calculus $DEQ^\ast$ without structural rules of contraction and exchange
is constructed.  A simple cut-elimination procedure for axiomatic equality
theories is presented.  }

\skyr{1. Introduction}

The motivation for introducing $DEQ$ was the need for the disjunctive
formulae when specifying and reasoning about nondeterministic
operations [9].
The notion of nondeterminism arises naturally in describing concurrent
systems. Nondeterminism is also a natural concept in describing sequential
programs, either as a means of indicating a {\it don't care} attitude as to
which among a number of computational paths will actually be utilized in a
particular computation or as a means of increasing
the level of abstraction [3]. In terms of satisfiability,
allowing nondeterminism is quite distinct from underspecification. The
latter, though admitting distinct models, still insists on a function always
returning a particular value, given a particular list of arguments. In the
case of nondeterministic operators this is no longer true.

In [9, 10] we presented a function oriented view of nondeterministic
operators.  The result of a (non-)deterministic operator is considered a
single (rather than set) value, namely the result of a {\it particular
application}. Distinct occurrences of a particular term each denote a
single value, but not necessarily the same one for each. For instance,
specifying an operator $\sqcup.s$ which chooses 
nondeterministically an element from the set $s$, one needs to say
that for any application $i$, $\sqcup_i.\{0,1\}=0  or
\sqcup_i.\{0,1\}=1$. In [9,10] it is shown how viewing
nondeterministic operations as determinisitc functions dependent on an
additional index (application) argument, allows one to translate
nondeterministic specifications into deterministic ones using
the language of conditional equations extended with disjunction.
Then we presented a natural deduction system, $DEQ$, as a
reasoning tool for transformed algebraic specifications of nondeterministic
operators, using axiomatic equality theories.  Along with the natural
aspect and modularity, the system $DEQ$ contains a cut rule which is not
amenable to automated reasoning. This paper introduces an equivalent
variant $DEQ^\ast$ of the $DEQ$ calculus where cut is replaced by
cut-like rules making $DEQ^\ast$ more amenable to automation.

A general theory of specialization of derivations in axiomatic
equality theories was proposed by M. Rogava in [8] and A. Pliu\s kevi\cv
ien\e \ in [4, 5]. In [8] specific axioms of axiomatic equality
theories were divided into two types: The axioms having a single
equality, i.e. only one item-equality, were identified as being of
{\it type 1}. The other specific axioms were identified as axioms
of {\it type 2}.  For specific axioms of type 1 equality-like rules
were introduced in [8].  For specific axioms of type 2 unspecialized
cut-like rules were introduced in [8]. In [4, 5] we introduced
specialized cut-like rules for specific axioms of type 2, introducing
positive equalities of specific axioms as cut formulas.

There we also introduced the complex notion of variants of equality formulas
by specific axioms with a view to eliminate the introduction of the negative
equalities of specific axioms as cut formulas.  The main purpose of this
paper is to construct the calculus $DEQ^\ast$ which is equivalent to
$DEQ$. $DEQ^\ast$ contains a cut-like rule introducing as cut formulas only
the negative equalities of specific axioms.  Therefore in $DEQ^\ast$ the
complex notion of variant of [4, 5] is not necessary.  For pure positive
specific axioms (i.e. with empty antecedents) and for so-called non-contrary
equality theories one can construct the calculus $DEQ^\ast$ not containing
cut-like rules at all.  In this process no variant notion is used. The
alternate version of the calculus $DEQ^\ast$ not containing structural rules
of contraction and exchange is also constructed. In the paper a simple
cut-elimination procedure for axiomatic equality theories is presented.

\skyr{2. The natural deduction-like calculi $DEQ$ and $DEQ^\prime$}

\skyrius{Definition 2.1} (sequent) A sequent is an expression $\rrG\to \Delta,$
where $\rrG, \Delta$ are finite sets of equalities $r=s,$ where
$r, s$ are arbitrary terms.  $\bigtriangledown$

\skyrius{Definition 2.2} (calculus $DEQ$). A calculus $DEQ$ is
defined by the following postulates (see e.g. [9, 10]):

\parindent=25pt
\item{(1)} specific axioms (specifications)
\medskip
\item{\ } (Sp.\  $AX_k$) $r_1=s_1, \ldots , r_m=s_m\to
u_1=v_1, \ldots , u_n=v_n,$
\medskip
\item{\ } where $k, m, n$ are some natural numbers; the index $k$
identifies the specific axioms;
\item{(2)} equality postulates (where $e$ represents a single equation,
$t, t_1, t_2$ are arbitrary terms)
\medskip
\item{\ } R1.\hskip 5truecm  $\displaystyle{\to t=t}$
\medskip
\item{\ } R2.\hskip 4truecm ${\displaystyle\rrG^x_{t_1}\to
\Delta^x_{t_1};\quad \rrG^\prime\to t_1=t_2\over
\displaystyle\rrG^x_{t_2},\ \rrG^\prime\to \Delta^x_{t_2}}$
\medskip
\item{\ } R3.\hskip 5truecm ${\displaystyle\rrG\to \Delta\over\displaystyle
\rrG^x_t\to \Delta^x_t}$
\medskip
\item{\ } R4.\hskip 6truecm $\displaystyle {e\to e}$
\medskip
\item{(3)} Structural rules
\medskip
\item{\ } R5. \hskip 4truecm ${\displaystyle\rrG\to
\Delta\over\displaystyle \rrG\to \Delta, e}\qquad
{\displaystyle\rrG\to \Delta\over
\displaystyle e, \rrG\to \Delta}\, ({\rm weakening})$
\medskip
\item{\ } R6.\hskip 4truecm  ${\displaystyle\rrG\to \Delta, e;\quad e,
\rrG^\prime\to \Delta^\prime\over
\displaystyle\rrG, \rrG^\prime\to \Delta, \Delta^\prime}\, ({\rm cut})$
$\bigtriangledown$
\parindent=0pt
\medskip

\skyrius{Remark 2.1.}

(i) $DEQ$ is sound and complete w.r.t. standard semantics (see [9]
with reference to [1]).

(ii) $DEQ$ represents a variant of the  natural deduction system for the
applied axiomatic theory of equations.

(iii) In $DEQ$  the rule
$${\rrG^x_{t_1}\to \Delta^x_{t_1};\quad \rrG^\prime\to t_1=t_2,
\Delta^\prime\over
\rrG^x_{t_2}, \rrG^\prime\to \Delta^x_{t_2}, \Delta^\prime}$$
is derivable (using R2, R4 and R6).

(iv) From  definition 2.2 it follows that $DEQ$ implicitly contains
the structural rules ``exchange'' and ``contraction''.
$\bigtriangledown$

In the following assume given a set of specific axioms $({\rm Sp.}\ AX_k)$.

\skyrius{Definition 2.3} (the calculus $DEQ^\prime$). A calculus
$DEQ^\prime$ is constructed from $DEQ$ by removing rule R3 and
changing $({\rm Sp.}\ AX_k)$ of $DEQ$ in the following manner: let $S$
be $({\rm Sp.}\ AX_k)$ of $DEQ,$ then $S^{x_1,\ldots,
x_n}_{t_1,\ldots, t_n}$ is the set of specific axioms $({\rm Sp.}\
AX^\prime_k)$ of $DEQ^\prime,$ where $t_1, \ldots , t_n$ are arbitrary
terms.  $\bigtriangledown$

\skyrius{Definition 2.4} (inclusion and equivalence of calculi). Let $I_1, I_2$
be arbitrary calculi then the notation $I_1 \ddr I_2$ indicates that
if $I_1\vdash S$ then $I_2\vdash S$ (where $S$ is a sequent) and the
notation $I_1\equiv I_2$ means $I_1\ddr I_2$ and $I_2\ddr I_1.$
$\bigtriangledown$

\skyrius{Lemma 2.1.} $DEQ\equiv DEQ^\prime.$

\skyrius{Proof:} Follows from the fact that $DEQ^\prime\vdash \rrG\to
\Delta\ddr DEQ^\prime\vdash \rrG^x_t\to \Delta^x_t.$ $\bigtriangledown$


\skyr{3. Kanger-like calculus $DEQ_1$}

\skyrius{Definition 3.1} (calculus $DEQ_1$). A calculus $DEQ_1$ is
obtained from $DEQ^\prime$ by replacing axiom R4 and rule R2 with the
rules (see e.g. [2]): $${r=s, \rrG^x_s\to \Delta^x_s\over r=s,
\rrG^x_r\to \Delta^x_r}\, (=_1)\qquad {s=r, \rrG^x_s\to
\Delta^x_s\over s=r, \rrG^x_r\to \Delta^x_r}\, (=_2)$$

\skyrius{Lemma 3.1.} $DEQ_1\equiv DEQ^\prime.$

\skyrius{Proof:} Follows from the following facts: (1) rules $(=_1), (=_2)$ are
derivable in $DEQ^\prime;$  (2) R4 and R2 are derivable in $DEQ_1.$
$\bigtriangledown$
\vskip -6pt

\skyr{4. Rogava-like calculi $DEQ_2, \ DEQ_3$}

\skyrius{Definition 4.1} (calculus $DEQ_2$). A calculus $DEQ_2$ is
obtained from
$DEQ_1$ by removing (cut) and replacing every specific axiom $(\sp\
AX^\prime_k)$
by the following
$(\sp\ \cut_k)$ rule  (see e.g. [8, 4]):
$${S_{11};\ldots; S_{1m};\ S_{21}; \ldots ; S_{2n}\over
\rrG\to \Delta}\, (\sp\ \cut_k),$$
where

$\rrG\to \Delta:=\rrG_{11}, \ldots, \rrG_{1m}, \rrG_{21}, \ldots, \rrG_{2n}
\to \Delta_{11}, \ldots, \Delta_{1m}, \Delta_{21}, \ldots , \Delta_{2n}$
$$\eqalignno{&S_{11}:=\rrG_{11}\to \Delta_{11},\ r_1=s_1\cr
&\quad \bruks\cr
&S_{1m}:=\rrG_{1m}\to \Delta_{1m}, \ r_m=s_m\cr
&S_{21}:=u_1=v_1,\ \rrG_{21}\to \Delta_{21}\cr
&\quad \bruks\cr
&S_{2n}:=u_n=v_n,\ \rrG_{2n}\to \Delta_{2n}\cr}$$~$\bigtriangledown$

\skyrius{Definition 4.2} (calculus $I+\cut$). Let $I$ be some calculus  not
containing (cut), then $I+\cut$  denotes the calculus obtained from $I$
by adding (cut). $\bigtriangledown$

\skyrius{Lemma 4.1.} ($\sp \ AX^\prime_k$) are derivable in $DEQ_2.$

\skyrius{Proof:} Using rules $(\sp\ \cut_k).$  $\bigtriangledown$

\skyrius{Lemma 4.2.} The rules $(\sp \ \cut_k)$ are derivable in $DEQ_1.$

\skyrius{Proof:} Using the (cut) rule. $\bigtriangledown$

\skyrius{Theorem 4.1.} $DEQ_1\equiv DEQ_2+(\cut).$

\skyrius{Proof:} From Lemmas 4.1 and 4.2. $\bigtriangledown$

\skyrius{Definition 4.3} (calculus $DEQ_3$). A calculus $DEQ_3$ is obtained from
the calculus $DEQ_2$ by replacing rules $(=_1), (=_2)$ by the following
rules (see e.g. [7]):
$${r=s,\ \rrG\to \Delta^x_s\over r=s,\ \rrG\to \Delta^x_r}\,
(=_{1s})\qquad {s=r, \ \rrG\to \Delta^x_s\over s=r, \ \rrG\to
\Delta^x_r}\, (=_{2s})$$
$\bigtriangledown$

\skyrius{Remark 4.1.} Sometimes the
notation $(=_s)$ will be used instead of the notation $(=_{is})\
(i=1,2)$.
$\bigtriangledown$

\skyrius{Lemma 4.3.} $DEQ_3\vdash S\ddr DEQ_2\vdash S.$

\proof: Obvious.   $\bigtriangledown$

\skyrius{Definition 4.4.} The notation $I\vdash^D S$ will represent the
fact that one
can construct a derivation $D$ of the sequent $S$ in calculus $I.$
  $\bigtriangledown$

\skyrius{Lemma 4.4.}  $DEQ_2\vdash^D S\ddr DEQ_3\vdash^{D^\ast} S.$

\skyrius{Proof.}  The proof is carried out
by induction on $n\rro+h(D),$ where $n$ is the
number of antecedent applications of $(=_1), (=_2)$ and  $h(D)$ is the
height of the
given derivation $D.$  Consider the upper application of
$(=_i)\ (i=1, 2)$ in $D.$ We consider only the following case:
$$D^+\left \{\eqalign{&\ u=v,\ r=s(u), \ \rrG^x_{s(u)}\to \Delta^x_{s(v)}\cr
\noalign{\vskip -10pt}
&\hbox to 6truecm{\hrulefill}\, (=_{2s})\cr
\noalign{\vskip -6pt}
&\ u=v, r=s(u), \rrG^x_{s(u)}\to \Delta^x_{s(u)}\cr
\noalign{\vskip -10pt}
&\hbox to 6truecm{\hrulefill}\, (=_1)\cr
\noalign{\vskip -6pt}
&\ u=v,\ r=s(u),\ \rrG^x_r\to \Delta^x_r\cr}\right. $$

We construct the following derivation $D^\prime$ in $DEQ_2:$
$$D^\prime\left \{ \eqalign{D_1\Big \{ &{\ u=v,\ r=s(u), \rrG^x_{s(u)}\to
\Delta^x_{s(v)}\over
u=v,\ r=s(u), \rrG^x_r\to \Delta^x_{s(v)}}\, (=_1)\cr
\noalign{\vskip -10pt}
&\hbox to 6truecm{\hrulefill}\, (=_{2s})\cr
\noalign{\vskip -6pt}
&\ u=v,\ r=s(u),\ \rrG^x_r\to \Delta^x_{s(u)}\cr
\noalign{\vskip -10pt}
&\hbox to 6truecm{\hrulefill}\, (=_{1s})\cr
\noalign{\vskip -6pt}
&\ u=v, r=s(u),\ \rrG^x_r\to \Delta^x_r\cr}\right.$$

As $h(D_1)<h(D^+),$ then $DEQ_2\vdash ^{D_1} S_1\ddr
DEQ_3\vdash ^{D^\ast_1} S_1,$ where $S_1:=u=v, r=s(u), \rrG^x_r\to
\Delta^x_{s(v)}$ and $DEQ_2\vdash^{D^\prime} S_2\ddr DEQ_3 \vdash
^{D^{\prime\ast}}
S_2,$ where $S_2:=u=v,\ r=s(u), \ \rrG^x_r\to \Delta^x_r.$
 $\bigtriangledown$

\skyrius{Theorem 4.2.} $DEQ_2\equiv DEQ_3.$

\skyrius{Proof:} Follows from Lemmas 4.3 and 4.4. $\bigtriangledown$

\skyrius{Remark 4.1.} From the proof of Lemma 4.4 it follows that
$h(D^\ast)>h (D),$
where $D^\ast$ is the derivation of a sequent $S$ in $DEQ_3,\ D$ is the
derivation of the same sequent $S$ in $DEQ_2.$ $\bigtriangledown$

\skyrius{Lemma 4.5.} $DEQ_3\vdash r=r, \ \rrG\to \Delta\ddr DEQ_3\vdash \rrG\to
\Delta.$

\skyrius{Proof:} By induction on the height of the given derivation.
$\bigtriangledown$


\skyrius{Lemma 4.6.} Consider the derivation $D:$
$${D_1\big \{ \rrG\to \Delta, e;\quad D_2\big \{e, \Pi\to \Theta\over
\rrG, \Pi\to \Delta, \Theta}\, (\cut),$$
where $D_1, D_2$ are the derivations in $DEQ_3,$ then $DEQ_3\vdash \rrG,
\Pi\to \Delta, \Theta.$

\skyrius{Proof.} The proof is carried out by induction on $h(D_1).$
 Let $h(D_1)=0,$ then $\rrG\to\Delta,\ e := \to r=r.$ Consider  the
derivation $D_2$ in $DEQ_3.$ By Lemma 4.5 we get $DEQ_3\vdash^{D^\prime_2}
\Pi\to \Theta$ and $D^\prime_2$ is the desired derivation.
 Let $h(D_1)>0,$ then we consider only the following case:
$${{\displaystyle D^\prime_1\big \{u=v, \ \rrG\to \Delta, \ r=s(v)\over
\displaystyle u=v, \ \rrG\to \Delta, \ r=s(u)}\, (=_s)\quad
D_2\big \{ r=s(u), \ \Pi\to \Theta\over
u=v, \ \rrG, \Pi\to \Delta, \Theta}\, (\cut)$$

Applying (weakening) and $(=)$ to $D_2$ weget $DEQ_2\vdash^{D^\prime_2}
S^\prime_2 := r=s(v),\ u=v, \ \Pi\to \Theta.$  Using Lemma 4.4 we get $DEQ_3
\vdash^{D^{\prime\ast}_2} S^\prime_2.$ Applying (cut) to
$D^\prime_1, D^{\prime\ast}_2$ we get
$DEQ_3+\cut\vdash^{D^\prime} S := u=v, \ \rrG, \Pi\to \Delta, \Theta.$ As
$h(D^\prime_1)<h(D_1)$ we get $DEQ_3\vdash  S.$  $\bigtriangledown$

\skyrius{Theorem 4.3} (elemination theorem for $DEQ_2$). $DEQ_2+\cut \ddr
DEQ_2.$

\skyrius{Proof:} Follows from Lemma  4.6 and Theorem 4.2. $\bigtriangledown$

\skyrius{Theorem 4.4.} $DEQ_1\equiv DEQ_2.$

\skyrius{Proof:} From Theorems 4.1 and 4.3. $\bigtriangledown$


\skyr{5. Specialization of $(\sp \ \cut_k)$ rules}

\skyrius{Definition 5.1} (calculus $DEQ_4$). A calculus $DEQ_4$ is obtained
from
$DEQ_2$ by replacing rules $(\sp \ \cut_k)$ by the rules
$${S_{11};\ldots; S_{1m}; S_{21};\ldots ; S_{2n}\over
\rrG\to \Delta}\, (\sp =_k),$$

where

$\matrix{\rrG\to &\Delta:=\hfill\cr
                 &\rrG_{11}, \ldots, \rrG_{1m}, \rrG_{21}, \ldots , \rrG_{2n}
\to \Delta_{11}, \ldots, \Delta_{1m}, \Delta_{21} {x_1\atop \tau_{11}},
\ldots , \Delta_{2n}
{x_n\atop \tau_{n1}}}$
$$\eqalignno{&S_{11}:=\rrG_{11}\to \Delta_{11},\ r_1=s_1\cr
&\quad \bruks\cr
&S_{1m}:=\rrG_{1m}\to \Delta_{1m},\ r_m=s_m\cr
&S_{21}:=\rrG_{21}\to \Delta_{21}{x_1\atop\tau_{12}}\cr
&\quad \bruks\cr
&S_{2n}:=\rrG_{2n}\to \Delta_{2n}{x_n\atop \tau_{n2}}\cr}$$
where $\tau_{j1}, \tau_{j2}\in \{u_j, v_j\},$
$j\in \{1, \ldots, n\}$ and $\tau_{j1}\neq \tau_{j2}.$
$\bigtriangledown$

\skyrius{Remark 5.1.} Special cases of $(\sp =_k)$ rules, corresponding
to specific axioms of the form $\to u=v,$ were presented in [8, 4,
5]. In [4, 5] the complex notion of variant of formula by specific
axioms was used instead of the introduction of negative
equalities $r_j=s_j\ (j=1, \ldots , m)$ of specific axioms as cut
formulas in the premises $s_{11}, \ldots, s_{1m}$;
equalities
$u_j=v_j\ (j=1, \ldots, n)$ in the premises $s_{21}, \ldots, s_{2n}$
were introduced as cut formulas, besides. In rules $(\sp =_k)$ only the
negative equalities of specific axioms are introduced as cut
formulas. Therefore the complex notion of the variant of a formula is
unnecessary. If $m=0$ then the $(\sp =k)$ rules are purely cut-free.
Analogous rules (corresponding to the case when $m=0$) were presented
in [6] for arithmetical robinson system and at the same time no
cut-like rule was not introduced for the specific axiom containing a
negative equality.

\skyrius{Example 5.1.} Let $DEQ^\prime$ have the following three
specific axioms $(\sp \ AX_k)\ (k\in \{1, 2, 3\}):$

$\matrix{(1)\,\hfill&\hfill r=s&\to u_1(c_1)=v_1(c_2),\
u_2(c_3)=v_2(c_4)\hfill\cr
(2)\,\hfill&\hfill u_1(c_5)=v_1(c_6)&\to \hfill\cr
(3)\,\hfill&\hfill u_2(c_7)=v_2(c_8)&\to \hfill}$

Then the calculus $DEQ_4$ has the following three $(\sp =_k)\ (k\in \{1, 2,
3\})$
rules:
$${\rrG_1\to \Delta_1, r=s;\quad \rrG_2\to \Delta_2 {x_1\atop \tau_{12}};\quad
\rrG_3\to \Delta_3{x_2\atop \tau_{22}}\over
\rrG_1, \rrG_2, \rrG_3\to \Delta_1, \Delta_2 {x_1\atop \tau_{11}},
\Delta_3{x_2\atop \tau_{21}}}\,
(\sp =_1)$$
$${\rrG\to \Delta, u_1(c_5)=v_1(c_6)\over \rrG\to \Delta}\,
(\sp =_2)$$
$${\rrG\to \Delta, u_2(c_7)=v_2(c_8)\over \rrG\to \Delta}\,
(\sp =_3),$$
where
$$\eqalignno{&\tau_{11}, \tau_{12}\in \{u_1(c_1), v_1(c_2)\}\cr
&\tau_{21}, \tau_{22}\in \{u_2(c_3), v_2(c_4)\}\cr}$$
Let $S:=\rrG\to, $ where $\rrG:=s=r,$ $c_1=c_5, c_2=c_6, c_3=c_7, c_4=c_8,$ then
applying $(\sp =_3), (\sp =_2), (=_s), (\sp =_1)$ bottom-up we get
$DEQ_4\vdash S.$   $\bigtriangledown$

\skyrius{Lemma 5.1.} $DEQ_4\ddr DEQ_3.$

\skyrius{Proof:} Follows from the fact  that rules $(\sp =_k)$ are derivable in
$DEQ_3.$   $\bigtriangledown$

\skyrius{Example 5.2.} Let $DEQ_4$ be as in Example 5.1. Consider
the rule $({\rm Sp.}=_1).$ Let us show that $({\rm Sp.}=_1)$ is derivable in
$DEQ_3$. Let

$\matrix{S_1:=\rrG_1\to \Delta_1, r=s\hfill\cr
S_2:= \rrG_2\to \Delta_2 {x_1\atop \tau_{12}}\hfill\cr
S_3:=\rrG_3\to \Delta_3 {x_2\atop \tau_{22}}\hfill}$

Applying $(W),\ ( =_{is})\ (i=1,2)$ to $S_2, S_3$ we get

$\matrix{S^\prime_2:=u_1(c_1)=v_1(c_2), \rrG_2\to \Delta_2 {x_1\atop
\tau_{11}}\hfill\cr
S^\prime_3:=u_2 (c_3)=v_2 (c_4), \rrG_3\to \Delta_3 {x_2\atop \tau_{21}}}$

Applying (Sp. ${\rm cut}_1$) to
$S_1, S^\prime_2, S^\prime_3$ we get the conclusion $S$ of (Sp.$=_1$).
$\bigtriangledown$

\skyrius{Lemma 5.2.} Consider the derivation $D$
$${\buildrel \vdots \over S_{11}; \ldots; \buildrel \vdots \over S_{1m};
\buildrel \vdots \over S_{21}; \ldots; \buildrel \vdots \over S_{2n} \over
\rrG\to \Delta }\, (\sp \ {\rm cut}_k)$$
and let  $D_{11}, \ldots, D_{1m}, D_{21}, \ldots , D_{2n}$
be the derivations of the sequents $S_{11}, \ldots, S_{1m}, S_{21}, \ldots
, S_{2n},$ respectively, in the
calculus  $DEQ_4,$ then $DEQ_4\vdash \rrG\to \Delta.$

\skyrius{Proof:} By induction on $h(D_{21})+\cdots +h(D_{2n}).$
   $\bigtriangledown$

\skyrius{Example 5.3.} Let $DEQ_4$ be the same as in Example 5.1 and let
$S_1, S^\prime_2, S^\prime_3$ be the same as in Example 5.2. Then (Sp.
${\rm cut}_1$) has
the following form:
$$\eqalign{&D_1\{ S_1; \ D_2\{ S^\prime_2; \ D_3\{ S^\prime_3\cr
\noalign{\vskip -10pt}
&\hbox to 4.3truecm {\hrulefill}\ \ ({\rm Sp.} {\rm cut}_1)\cr
\noalign{\vskip -6pt}
&\hskip 1.7truecm S\cr}$$

Let us consider only the case when $D_2, D_3$ have the following forms:

$$D_2\left \{  \eqalign{ D^\prime_2 \Big \{ & u_1 (c_1)=v_1 (c_2),\
\rrG_2\to \Delta_2{x_1\atop \tau_{12}}\cr
\noalign{\vskip -10pt}
&\hbox to 6truecm{\hrulefill}\, (=_{is})\cr
\noalign{\vskip -6pt}
&u_1 (c_1)=v_1(c_2),\ \rrG_2\to \Delta_2 {x_1\atop \tau_{11}}\cr}\right. $$
$$D_3\left \{  \eqalign {D^\prime_3 \Big \{& u_2 (c_3)=v_2 (c_4),\
\rrG_3\to \Delta_3{x_2\atop \tau_{22}}\cr
\noalign{\vskip -10pt}
&\hbox to 6truecm{\hrulefill}\, (=_{is})\cr
\noalign{\vskip -6pt}
&u_2 (c_3)=v_2(c_4),\ \rrG_3\to \Delta_3 {x_2\atop \tau_{21}}\cr}\right. $$

Having set $\rrG:=\rrG_1,\ \rrG_2,\ \rrG_3$ and
$S_2:=u_1(c_1)=v_1(c_2),\ \rrG_2\to \Delta_2{x_1\atop \tau_{12}},\ \allowbreak
S_3:=u_2(c_3)=v_2(c_4),\
\rrG_3\to \Delta_3{x_2\atop \tau_{22}}$,
we construct the following derivation $D^\ast$:

$$D^\ast\,\left \{
\eqalign{
&\hskip 1.3truecmS_1;\ D^\prime_2\{ S_2;\ D_3 \{ S^\prime_3
\hskip 1.5truecmS_1;\ D_2\{ S^\prime_2;\ D^\prime_3 \{ S_3\cr
\noalign{\vskip -10pt}
&\hskip 1.1truecm\hbox to 4.2truecm{\hrulefill}\ %\, ({\rm Sp. cut}_1)
\hskip 0.5truecm\hbox to 4.2truecm{\hrulefill}\cr %\ \,({\rm Sp. cut}_1)\cr
\noalign{\vskip -6pt}
&S_1;\hskip 0.5truecm \rrG\to
\Delta_1, \Delta_2{x_1\atop \tau_{12}}, \Delta_3 {x_2\atop \tau_{21}}
\hskip 0.5truecm\rrG\to
\Delta_1, \Delta_2{x_1\atop \tau_{11}}, \Delta_3 {x_2\atop \tau_{22}}\cr
\noalign{\vskip -10pt}
&\hbox to 10.5truecm{\hrulefill}\ ({\rm Sp.} =_1)\cr
\noalign{\vskip -6pt}
&\hskip 3.5truecm  \rrG\to
\Delta_1, \Delta_2{x_1\atop \tau_{11}}, \Delta_3 {x_2\atop \tau_{21}}\cr}
\right.$$


As $h(D^\prime_2)<h(D_2)$ and $h(D^\prime_3)<h(D_3)$, The (Sp. ${\rm cut}_1$)
steps (unlabeled above) can be eliminated from the derivation~$D^\ast.$


\skyrius{Theorem 5.1.} $DEQ_4\equiv DEQ_3.$

\skyrius{Proof:} Follows from Lemmas 5.1 and 5.2.    $\bigtriangledown$

\skyrius{Definition 5.2} (calculus $DEQ_5$). A calculus $DEQ_5$ is
obtained from calculus $DEQ_4$ by  replacing rules $(\sp =_k)$ by
the  rules $(\sp =_k^\ast):$
$${S_{11};\ldots ; S_{1m};\ S_{21};\ldots ; S_{2n}\over \rrG\to \Delta}\,
(\sp =^\ast_k),$$
where
$$\eqalignno{&\rrG\to\Delta:=\rrG \to \Delta {x_1\ldots x_n\atop \tau_{11}
\ldots \tau_{n1}}\cr
&S_{11}:=\rrG\to \Delta {x_1\ldots x_n\atop \tau_{11}\ldots \tau_{n1}},
r_1=s_1\cr
&\quad \bruks \cr
&S_{1m}:=\rrG\to \Delta {x_1\ldots x_n\atop \tau_{11}\ldots \tau_{n1}},
r_m=s_m\cr
&S_{21}:=\rrG\to \Delta {x_1 x_2\ldots x_n\atop \tau_{12}\tau_{21} \ldots
\tau_{n1}}\cr
&\quad \bruks \cr
&S_{2n}:=\rrG\to \Delta {x_1\ldots x_{n-1}x_n\atop \tau_{11}\ldots \tau_{n-11}
\tau_{n2}}\cr}$$
   $\bigtriangledown$


\skyrius{Lemma 5.2.} $DEQ_4\equiv DEQ_5.$

\skyrius{Proof:} Using structural rule weakening.  $\bigtriangledown$

\skyrius{Definition 5.3} (calculus $DEQ^\ast$). A calculus $DEQ^\ast$ is
obtained from the calculus $DEQ_5$ by dropping structural rule weakening and
replacing the axiom $\to r=r$ by the axiom $\rrG\to \Delta, r=r.$
    $\bigtriangledown$

\skyrius{Lemma 5.3.} Structural rule weakening is admissible in $DEQ^\ast.$

\skyrius{Proof:} By induction on the  height of a given derivation.
    $\bigtriangledown$

\skyrius{Theorem 5.2.} $DEQ^\ast\equiv DEQ.$

\skyrius{Proof:} Follows from Lemmas 2.1, 3.1, Theorems 4.2, 4.4, 5.1 and
Lemmas 5.2,~5.3.     $\bigtriangledown$

\skyr{6. Some specializations of $(\sp =_k)$ rules}

\skyrius{Definition 6.1} (contrary calculus $DEQ$). A calculus
$DEQ$ will
be called {\it contrary} if  there  exists some
substitution  $\bar x / \bar t$ (where $\bar x := x_1, \ldots , x_n;\
\allowbreak
\bar t :=t_1, \ldots , t_n$)  and some specific axiom $k$ containing a
negative occurrence of an equality $r^{k}_i=s^{k}_i,$ where $i\in \{1, \ldots,
m_k\},$ such that using only equality rules $(=_{is})$ it is possible to
construct a derivation of the sequent $\rrG^{\bar x}_{\bar t}\to
(r^{k}_i=s^{k}_i)^{\bar x}_{\bar t}$ where $\rrG$ consists of positive
occurrences of equality in specific axioms of $DEQ$ and maybe some equalities
constituted only from symbols of constants entering in axioms (Sp. $AX_k$) In
the opposite case $DEQ$ will be called {\it non-contrary}.  $\bigtriangledown$

\skyrius{Example 6.1.} Let $DEQ$ have the following specific axioms (Sp.
$AX_k$)  $(k\in \{1, 2, 3\}):$

$\matrix{(1)\,\hfill& \hfill f(u)=r(a)&\to t_1=t_2 (c_1)\hfill\cr
(2)\hfill& &\to f(c_1)=g(v)\hfill\cr
(3)\hfill& &\to r(b)=g(c_2)\hfill}$

where $a, b, c_1, c_2$ are some constants.

$DEQ$ is contrary because it is possible to construct the derivation $D$ of
the sequent $\rrG^{\bar x}_{\bar t}\to (r^{k}_i=s^{k}_i)^{\bar x}_{\bar t}$,
where $\rrG:= f(c_1)=g(v),\ r(b)=g(c_2),\ a=b;\  \ (r^{k}_i=s^{k}_i):=
f(u)=r(a)$
and the substitution $\bar x/\bar t,$ where $\bar x:=u, v; \ \bar t:=c_1, c_2$.
The  derivation  $D$ has the form:
$$\eqalign{&\ f(c_1)=g(c_2),\ r(b)=g(c_2),\ a=b\to r(a)=r(a)\cr
\noalign{\vskip -10pt}
&\hbox to 9.5truecm{\hrulefill}\, (=_s)\cr
\noalign{\vskip -6pt}
&\ f(c_1)=g(c_2),\ r(b)=g(c_2),\ a=b\to r(b)=r(a)\cr
\noalign{\vskip -10pt}
&\hbox to 9.5truecm{\hrulefill}\, (=_s)\cr
\noalign{\vskip -6pt}
&\ f(c_1)=g(c_2),\ r(b)=g(c_2),\ a=b\to g(c_2)=r(a)\cr
\noalign{\vskip -10pt}
&\hbox to 9.5truecm{\hrulefill}\, (=_s)\cr
\noalign{\vskip -6pt}
&\ f(c_1)=g(c_2),\ r(b)=g(c_2),\ a=b\to f(c_1)=r(a)\cr}$$
$\bigtriangledown$

\skyrius{Example 6.2.} Let $DEQ$ has the following (Sp. $AX_k$)  $(k\in
\{1, 2\}):$

$\matrix{
(1)\,\hfill&f(u)=g(v)&\to t_1=t_2 (c_1)\hfill\cr
(2)\hfill& &\to g(c_1)=f(c_2)\hfill}$

where $c_1, c_2$ are some constants.

It is easy to see that $DEQ$ is contrary under the substitution
$\bar x/\bar t$, where $\bar x:=u, v; \bar t:=c_2, c_1$.
The following derivation of the sequent   $y=c_1,\  u=c_2,\  v=c_1\to
t_1=t_2 (y)$
shows the necessity of introducing as cut formula the negative occurrence of
equality $f(u)=g(v):$
$$\eqalign{&\ \rrG\to f(c_2)=f(c_2)\cr
\noalign{\vskip -10pt}
&\hbox to 4.truecm{\hrulefill}\, ({\rm Sp.}=_2)\cr
\noalign{\vskip -6pt}
&\ \rrG\to f(c_2)=g(c_1)\cr
\noalign{\vskip -10pt}
&\hbox to 3.7truecm{\hrulefill}\, (=_s), (=_s)\cr
\noalign{\vskip -6pt}
&\ \rrG\to f(u)=g(v)\qquad \qquad \qquad \rrG\to t_1=t_1\cr
\noalign{\vskip -10pt}
&\hbox to 8truecm{\hrulefill}\, ({\rm Sp.}=_1)\cr
\noalign{\vskip -6pt}
&\hskip 2.3truecm \rrG\to t_1=t_2 (c_1)\cr
\noalign{\vskip -10pt}
&\hskip 2truecm\hbox to 3.5truecm{\hrulefill}\, (=_s)\cr
\noalign{\vskip -6pt}
&\hskip 2.3truecm \rrG\to t_1=t_2 (y)\cr}$$

where $\rrG:= y=c_1,\ u=c_2,\ v=c_1.$ $\bigtriangledown$

\skyrius{Definition 6.2} (calculus $DEQ^{\ast\ast}$).
The calculus $DEQ^{\ast\ast}$ is obtained from the calculus $DEQ^\ast$ (1)
replacing rules
$(\sp =^\ast_k)$ by following ones:
$${S^\prime_{21};\ldots ; S^\prime_{2n}\over
\Pi, \rrG\to \Delta^{x_1\ldots x_n}_{\tau_{11}\ldots \tau_{n1}}}\,
(\sp =^{\ast\ast}_k),$$
where $\Pi:=r^\prime_1=s^\prime_1, \ldots , r^\prime_m=s^\prime_m;\ \ r^\prime_i
\in\{r_i, s_i\}, \ s^\prime_i\in \{r_i, s_i\},\  i\in \{1, \ldots, m\}, \
r^\prime_i\neq s^\prime_i;$
$S^\prime_{2j}$ is obtained from $S_{2j}$ of rules $(\sp =^\ast_k)$ changing
$\rrG$ by $\Pi, \rrG;$  (2) replacing rules $(=_{is})$ by $(=_i)\ (i=1,2).$
$\bigtriangledown$

\skyrius{Theorem 6.1.} Let $DEQ$ be a non-contrary calculus, then
$DEQ\equiv DEQ^{\ast\ast}.$

\skyrius{Proof:} Applying the definition of contrary calculus.
  $\bigtriangledown$

\skyrius{Example 6.3.} Let $DEQ$ be the following (Sp. $AX_k$) ($k\in \{
1,2\}):$

$\matrix{(1)\,\hfill&g(v)=f(u)&\to t_1=t_2(c_1)\hfill\cr
(2)\hfill& &\to g(c_1)=g(c_2)\hfill}$

where $c_1, c_2$ are some constants.

It is easy to see that $DEQ$ is
non-contrary. We construct the following derivations of the sequent
$y=c_1,\, v=s, u=r,\, f(u)=g(s)\to t_1=t_2(y)$,
which shows how  the rule (Sp. $=^{\ast\ast}$)works :

$$\eqalign{ &\hskip 1truecm \rrG\to t_1=t_1\cr
\noalign{\vskip -10pt}
&\hbox to 5truecm{\hrulefill}\, ({\rm Sp.}=^{\ast\ast})\cr
\noalign{\vskip -6pt}
&\hskip 1truecm \rrG\to t_1=t_2(c_1)\cr
\noalign{\vskip -10pt}
&\hbox to 5truecm{\hrulefill}\, (=_s)\cr
\noalign{\vskip -6pt}
&\hskip 1truecm \rrG\to t_1=t_2(y)\cr
\noalign{\vskip -10pt}
&\hbox to 5truecm{\hrulefill}\, (=)\cr
\noalign{\vskip -6pt}
&\hskip 1truecm \rrG^\ast \to t_1=t_2(y)\cr}$$
where $\rrG:=y=c_1, \ v=s, u=r, f(u)=g(v); \ \rrG^\ast :=y=c_1, v=s,
u=r, f(u)=g(s).$
 $\bigtriangledown$

\skyrius{Remark 6.1.} To prove that $DEQ \equiv DEQ^{\ast\ast}$ in case
$DEQ$  is contrary it is necessary to change the rules $(\sp =_k^{\ast\ast})$
in such a way that they involve the notion of variant of formula by
specific axioms (see [4,5]).   $\bigtriangledown$

\skyr{7. Elimination of contraction and exchange}

\skyrius{Definition 7.1} (calculus $DEQ_6$). A calculus $DEQ_6$ is
obtained from the calculus $DEQ^\ast$ in the following way:

(i) the $\rrG, \Delta$
in sequent $\rrG\to \Delta$ are considered arbitrary lists rather than sets;

(ii) adding the structural rules of contraction and exchange:
$${\rrG, e_1, e_2, \Pi\to \Delta \over \rrG, e_2, e_1, \Pi\to \Delta}\,
(EX\to)\qquad {\rrG\to \Delta, e_1, e_2, \Theta\over \rrG\to\Delta, e_2, e_1,
\Theta}\,(\to EX)$$
$${e, e, \rrG\to\Delta\over e, \rrG\to \Delta}\, (C\to)\qquad
{\rrG\to \Delta, e, e\over \rrG\to \Delta, e}\, (\to C)$$
 $\bigtriangledown$

\skyrius{Lemma 7.1.} $DEQ_6\equiv DEQ^\ast.$

\skyrius{Proof:}  Applying definition 7.1.  $\bigtriangledown$

\skyrius{Definition 7.2.} (calculus $DEQ_7$). The calculus $DEQ_7$ is
obtained from the calculus $DEQ_6$ by dropping structural rules
$(EX\to), (\to EX).$  $\bigtriangledown$

\skyrius{Lemma 7.2.} $DEQ_6\equiv DEQ_7.$

\skyrius{Proof:}d Aanalogously to the proof in [7]. $\bigtriangledown$

\skyrius{Definition 7.3} (Calculus $DEQ_8$). A calculus $DEQ_8$ is
obtained from the calculus $deq_7$ by dropping the structural rules $(C\to),
(\to C)$ and changingthe rules $(=)$ and $(\sp =_k)$ in such a way that the
main formulas are duplicated in the premises of the rule.
$\bigtriangledown$

For example, instead
of the rule $(=_{1s})$ the following rule: $${r=s, \rrG\to\Delta^x_s,
\Delta^x_r\over r=s, \rrG\to \Delta^x_r}\, (=^\ast_{1s})$$ is in the calculus
$DEQ_8.$

\skyrius{Definition 7.4} (calculus $DEQ^\prime_7$). A calculus
$DEQ^\prime_7$ is obtained from the calculus $DEQ_7$ by dropping the
structural rule $(C\to).$ $\bigtriangledown$

\skyrius{Lemma 7.3.} $DEQ_7\vdash ^D S\ddr DEQ^\prime_7\vdash^{D^\ast} S.$

\skyrius{Proof:} By induction on $n\rro+h(D),$ where $n$ is the number of
applications of $(C\to)$ in $D.$ $\bigtriangledown$

\skyrius{Lemma 7.4.} $DEQ^\prime_7\vdash^D\ddr DEQ_8\vdash ^{D^\ast} S.$

\skyrius{Proof.} The proof is carried out by $n\rro+h(D),$ where $n$ is the
number
of applications of $(\to C)$ in $D.$ We consider only the
following case

$$D^\prime\left \{ \eqalign{D^\prime_1 \Big \{  &\ \rrG,\ r=s\to \Delta, \
p(s)=q, p(r)=q, p(r)=q\cr
\noalign{\vskip -12pt}
&\ \hbox to 8truecm{\hrulefill}\, (=^\ast_s)\cr
\noalign{\vskip -6pt}
&\ \rrG,\ r=s\to \Delta, \ p(r)=q, \ p(r)=q\cr
\noalign{\vskip -10pt}
&\ \hbox to 6.5truecm{\hrulefill}\, (\to C)\cr
\noalign{\vskip -6pt}
&\ \rrG,\ r=s\to\Delta,\ p(r)=q\cr}\right.$$

where $D^\prime_1$ is the derivation in $DEQ_8.$ Let us construct the
following derivation $D^{\prime\ast}:$

$$D^{\prime\ast}\left \{ \eqalign{D^{\prime\ast}_1\Big \{ &{\ \rrG,\ r=s\to
\Delta, \
p(s)=q, p(r)=q, p(r)=q\over
\rrG,\ r=s\to \Delta, \ p(s)=q, \ p(r)=q}\
\, (\to C)\cr
\noalign{\vskip -10pt}
&\ \hbox to 8truecm{\hrulefill}\, (=^\ast_s)\cr
\noalign{\vskip -6pt}
&\ \rrG,\ r=s\to\Delta,\ p(r)=q\cr}\right.$$

As $h(D^{\prime\ast}_1)<h(D^\prime_1)$ then $DEQ^\prime_7\vdash^{D^{\prime\ast}}
S^\ast\ddr DEQ_8\vdash^{D^{\prime\ast\ast}} S^\ast,$ where
$S^\ast:=\rrG, r=s\to \Delta, p(r)=q.$ $\bigtriangledown$


\skyrius{Theorem 7.2.} $DEQ_7\equiv DEQ_8.$

\skyrius{Proof:} Follows from Lemmas 7.3 and 7.4.  $\bigtriangledown$

\skyrius{Remark 7.1.}  Let $p=q$ be any explicitly indicated term in
(Sp. $AX_k$) of $DEQ^\prime,$  then $p,q$ will be called {\it axiom terms}.
We introduce an ordering of terms:
Let $r(t)$ stand for the rank of a term $t$
which is defined in the following way:

(1) $r(t)=0,$ if $t$ is any axiom term;

(2) $r(t_1)<r(t_2),$ if $t_1, t_2$ are not axiom terms and $t_1$ is less
then $t_2$
in a lexicograpfic ordering  of terms.

Then using the ordering of terms introduced above we can
drop duplication of the main formulas in the equality rules  and leave this
duplication  in the rules $(\sp =^\ast_k)$ only.
 $\bigtriangledown$
\def\refa#1#2{\parindent=0.4cm\item{#1}{#2}\noindent\hang}
\bigskip

\centerline{\bf References}
\medskip

\ref {1.} {S. Feferman, Lectures on Proof Theory, LNM, 70, Springer, 1969.}

\ref {2.} {S. Kanger, A simplified proof method for elementary logic.
Comput. Progr. and  Formal Systems, North-Holland, Amsterdam,  87--94, 1963.}

\ref {3.} {S. Meldal, An abstract axiomatization of pointer types,
in Proc. of the 22nd Annual Hawaii International Conference on
System Sciences, B.D. Shriver (ed.), IEEE Computer Society Press, vol. 2,
129--134, 1989.}

\ref {4.} {A. Pliu\v skevi\v cien\e ,\ Elimination of cut-type rules in
axiomatic theories with equality. Seminars in mathematics V.A. Steklov Mathem.
Institute, Leningrad, 16,  90--94, 1971.}

\ref {5.} {A. Pliu\s kevi\cv ien\e ,\ Specialization of the use of axioms
for deduction
search in axiomatic theories with equality, Seminars in mathematics
V.A. Steklov Mathem. Institute, Leningrad, J. Soviet Math.  1,  110--116, 1973.}

\ref {6.} {A. Pliu\s kevi\cv ien\e ,\ A sequential variant of R.M.Robinson's
arithmetic system not  containing  cut rules. Proc.  Steklov  Inst. Math.,
Leningrad,
 121,  121--150, 1972.}

\ref {7.} {R. Pliu\s kevi\cv ius, Sequential variant of the calculus of
constructive logic
for normal formulas. Proc. Steklov Inst. Math., 98,  175--229, 1968.}

\ref {8.} {M. Rogava, Sequential variants of applied predicate calculi without
structural deductive rules. Proc. Steklov Inst. Math. 121,  136--164, 1972.}

\ref {9.} {M. Walicki,  Algebraic Specifications of nondeterminism,
Ph. D thesis, University of Bergen, 1993.}

\ref {10.} {M. Walicki, S. Meldal, A complete calculus for the
multialgebraic and functional semantics of nondeterminism,
(submitted for publ.) 1993.}
%spell_to

\bye






