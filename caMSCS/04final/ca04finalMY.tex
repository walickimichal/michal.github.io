%%%%%%%%%%%%%% LaTeX reports some errors, also when formatting 
%%%%%%%%%%%%%% some Xy-pic figures - just type r and let it continue
%%%%%%%%%%%%%%

\documentclass{article}
%\documentclass{mscs}
\usepackage{amssymb}
\usepackage{latexsym}

\input defs

\makeatletter
\input a4wide.tex
\makeatother

\input xypic
\xyoption{graph}
\xyoption{frame}

\title{Computation Algebras}
\author{{Micha{\l} Walicki\thanks{The authors gratefully acknowledge the financial support received
from the Norwegian Research Council (NFR).}\\ 
\small University of Bergen\\[-.5ex] 
\small Department of Informatics\\[-.5ex] 
\small 5020 Bergen, Norway\\[-.5ex] 
\small michal@ii.uib.no}
\and {Magne Haveraaen$^*$\\ 
\small University of Bergen\\[-.5ex] 
\small Department of Informatics\\[-.5ex]  
\small 5020 Bergen, Norway\\[-.5ex] 
\small magne@ii.uib.no}
\and {Sigurd Meldal$^*$\\ 
\small CalPoly\\[-.5ex] 
\small Computer Science Department\\[-.5ex] 
\small San Luis Obispo, CA 93407, USA\\[-.5ex] 
\small sigurd@calpoly.edu}
} 

\date{{}} 

\begin{document}
\maketitle 

\begin{abstract}
\noindent
We introduce a framework which
generalizes algebraic specifications by equipping algebras
with descriptions of {\em evaluation strategies\/}. The resulting abstract
mathematical description  allows one to model the
{\em implementation\/} of algebras on various platforms in a way independent
of the function-oriented specifications.

We are studying
algebras with associated data dependencies.  The
latter provide separate means for modeling computational aspects apart from
the functional specifications captured by an algebra.  
The formalization of evaluation strategies 
(1) introduces increased portability among different hardware
platforms, and (2) allows a potential increase in execution
efficiency, since a chosen evaluation strategy may be tailored to a
particular platform. We present the development process where
algebraic specifications are equipped with data dependencies, the
latter are refined and, finally, mapped to actual hardware architectures.
\end{abstract}

\section{Introduction}
The creed of algebraic specification is {\em abstraction}: one models
programs as algebras, thus abstracting from the operational details and
focusing on the abstract functionality of the system.  A single such
specification may be implemented on different platforms.  The
implementation design, as well as the particular platform on which it is
realized, need not be the primary issues.  It is one of the central tenets
of the algebraic specification that the designer may initially model his
system at a high level of abstraction without involving himself in low
level implementation details, only introducing implementation concerns as
appropriate, much later in the development process.  At some point during
the development process the choice of the set of constructors, the
computation strategy, the target language, and, perhaps, the hardware
platform begin to play a more important role. In particular, because of
its intentional abstraction from the operational aspects, the standard
algebraic approach to specification has difficulties when faced with the
possibilities of parallel implementations.  It is at this point that our
proposed method comes into successful play.

We propose a framework of {\em computation algebras\/} addressing the
issue of how to incorporate computation-related aspects within the
framework of classical algebraic semantics.  
%%%%% Added
Computation algebras allow the developer to transform abstract requirements
(expressed in an algebraic specification) into designs by adding
implementation-specific information concerning the order and dependency
between various subcomputations. 
%%%%% end Added
Though occasionally tempting,
we would like to avoid rebuilding the theory from scratch -- we want to
take advantage of the existing theoretical and tool-oriented results of
algebraic specifications.  Our aim is only to {\em extend\/} the classical
framework with  constructs which in an algebraic way addresses the choice
of computational strategies.

This is achieved by associating additional structures, {\em data
dependency} graphs, with standard algebras.  The data depency graphs carry
information about evaluation strategies for the operations of the algebras.
We have to warn the reader that our data dependencies are not exactly
what is meant by that in the literature, especially on imperative 
programs, e.g.
\cite{MW83,FOW,KK}.
We proceed from the algebraic, that is, functional descriptions and 
dependencies
relate functional terms and not program statements.
In particular, a variable
is only a (sub)term denoting a specific value -- it is not an actual program
variable of a program capable of storing various values at different points
of computation.

We   mainly address the semantic issues of this approach, without
entering into a discussion of possible specification {\em formalisms\/}.
Thus we will assume that data dependencies are given along with algebras,
and study their interaction.  Since data dependencies are standard
algebraic objects we expect that they can easily be described in a standard
algebraic fashion.  (We intentionally refrain from including
an analysis of control dependencies -- unlike data dependencies, they
seem to be inherent features of actual programs rather than of abstract
data types.)
  In some cases data dependencies may be recovered directly from
an algebraic specification and section~\ref{se:cr} describes a way to do it.

By allowing computational structures into the semantics of a specification
we provide the means  not only for development of programs, but for the
development of {\em efficient\/} programs.  We will show that data
dependencies provide such a means, enabling one to assess the time/space
resources required and to pursue optimization strategies such as syntactic
and semantic memoisation.  Above all, the computational structures carry
the information necessary to achieve {\em parallel\/} implementations:
the identification of dependencies among various parts of a computation
enables us to determine its possible parallel distributions.  In addition to
algebras and data dependencies, we will introduce a representation
of (parallel) machine architectures and show how mappings from dependencies
to architectures may be utilized to yield parallel implementations.  The
framework enables us then to {\em port the implementations of computation
algebras\/} along the morphisms between various hardware architectures.  It
should be emphasized, however, that we are {\em not} concerned with a
general theory of concurrency.  In this context, we view concurrency
merely as a means of increasing the efficiency of the implementation.

Section \ref{se:defs} introduces the basic notions of computation
algebras and their implementations and illustrates them on a simple
example. It also gives a general view of the envisaged development process
using computation algebras.
Section~\ref{se:cr} addresses the issue of constructing dependencies 
for a function
defined by a set of recurrence equations over a given algebra.
It  illustrates the ideas of the
programming language {\sc Sapphire} developed at the University of Bergen
for which computation algebras provide the semantic framework.

Section~\ref{se:difSign} refines the picture of development sketched in
section~\ref{se:defs} -- its results suggest a more
specific  metodology for development using computation
algebras. Section~\ref{se:nd} shows how nondeterminism can be included into
the proposed framework. 
Section \ref{se:conclusion} summarizes the results and indicates some
  directions for future work.

The paper focuses on the description of the {\em concepts\/}; its technical
content is limited to definitions and a few results.
It gives an improved version of the results from \cite{WHM96}.
The more technical aspects  assume some elementary knowledge of the category
theory (e.g., \cite{BW90,RB88,McL}) with particular emphasis on fibrations.

\section{The Framework of Computation Algebras}\label{se:defs}
This section introduces the framework of computation algebras. We
define first some basic notions in Section \ref{sub:basic}, then data dependencies in
Section \ref{sub:dd},
computation algebras in Section \ref{sub:ca}, and their implementation in
Section \ref{sub:imp}. Section~\ref{sub:dev} summarizes these pieces presenting the
view of the development process using computation algebras.
%
\subsection{Signatures, Algebras and Graphs}\label{sub:basic}

We use the standard notion of a signature $\Sigma$ as a pair
$\<\Sorts,\Funcs\>$ of sort and operation symbols.  In Section
\ref{sub:difSign} we will use the category $\Sig$ with signatures as
objects and {\em injective} signature morphisms.

  For a signature $\Sigma$, $\AlgS$ will denote a category of
  $\Sigma$-algebras.  It is ``{\em a} category'' because we do not make any
  assumptions as to what the algebras are, except that they are objects
  where one can interpret the $\Sigma$-terms.  One may think of standard
  $\Sigma$-algebras with $\Sigma$-homomorphisms defined in the usual
  way.  But one may as well think of the category of multiagebras with
  various multihomomorphisms, power algebras, unified algebras, etc.  We
  will define various notions relatively to an arbitrary full subcategory
  $\Aclass\subseteq\AlgS$.\footnote{Using the {\Cat {different font}} for
  categories, we hope that the use of the subset symbol $\subseteq$ also for
  subcategory won't create any confusion.}

The primary objects of our
interest will be {\em simple directed acyclic graphs} which
\begin{itemize}\MyLPar
\item have at most one edge between any pair of nodes, and
\item have no isolated nodes (each node has at least one incident edge).
\end{itemize}
We call them {\em simple DAGs}.
We will, however, need a more general notion of a mapping of such graphs than
a simple graph homomorphism.
Mapping an edge of the source graph to a {\em path} in the target graph
corresponds to splitting the computation into several steps. On the other hand,
mapping an edge to a {\em set} of edges (or paths) corresponds to
distributing the computation.

To define this concept of a morphism, we introduce three categories:
$\DAG$ -- of DAGs, $\DAG^\ecomp$ -- of path closures, and
$\pat\DAG$ -- of distributed paths (or just sets of paths). The 
latter two categories are
used merely for the purpose of defining more general morphisms between
$\DAG$-objects. An object in  $\DAG^\ecomp$ contains, in addition to the
underlying graph $G$, also all the paths over $G$ -- we will want to map an
edge of one graph  onto a path in a target graph.\footnote{The symbol
`$\ecomp$' is used for (sequential) composition in diagrammatic order -- not
for functional composition.}
 An object in
$\pat\DAG$ contains, in addition, ``parallel composition'' of
paths which, too, will be possible images of edges.
%%%% Added
The construction described in the rest of this subsection starts with 
the category $\DAG$: the free functor $F^\ecomp$
adds freely to each object all the paths and then, the free functor
$\patt F$ adds all the sets of paths. The resulting composition $\pat
F=\Comp{F^\ecomp}{\patt F}$ has left adjoint and our morphisms are Kleisli
morphisms obtained from this adjunction.


\begin{DEFINITION}{}\label{de:Gr}
$\DAG$ is the category of {\em directed acyclic graphs} where
\begin{enumerate}\MyLPar
\item objects are {\em DAGs} -- algebras over signature
$\CSig= \<I, E; s,t:E\into I\>$, where
\begin{itemize}\MyLPar
\item $I$ is a set of nodes and $E$ a set of
edges;
\item $s/t$ return the source/target node of an edge;
\item $E$ viewed as a relation ($\subseteq I\times I$) is acyclic;
\item there are no isolated nodes: for each node $v$, there is an edge $e$
with $s(e)=v$ or $t(e)=v$.
\end{itemize}
\item morphisms are $\CSig$-homomorphisms. \vspace*{-4.5ex}
\end{enumerate}
\end{DEFINITION}
\noindent
We will often write $i\ddep j$ to indicate the existence of an edge 
$e$ from $i$
to $j$ (i.e., with $s(e)=i$ and $t(e)=j$). A $\CSig$-algebra is a DAG 
(an object of \DAG)
if the transitive closure $\tc{\ddep}$ of this relation is irreflexive.

A DAG may have several edges between a pair of nodes. Of particular
importance to us will be {\em simple} DAGs -- the ones with at most one
edge between a pair of nodes.
\begin{DEFINITION}
A {\em simple DAG} is a $DAG$ satisfying:\ 
$s(e_1)=s(e_2)\land t(e_1)=t(e_2) \impl e_1=e_2$;
\end{DEFINITION} \vspace*{-4ex}

\begin{DEFINITION}{}\label{de:pathcl}
$\DAG^\ecomp $ is the category of {\em path closures} of DAGs where
\begin{enumerate}\MyLPar
\item objects are  algebras over signature $\CSig^\ecomp =\CSig\cup
\{\_\ecomp \_ :E\times E\into E\}$ whose $\CSig$-reducts belong to
$\DAG$ -- for any $P\in\Obj{\DAG^\ecomp}:P|_\CSig\in\Obj{\DAG}$.
$\_\ecomp \_$ is a partial operation of path formation
defined for any edges $e_1, e_2$ with $t(e_1)=s(e_2)$, and satisfying
\begin{itemize}\MyLPar
\item $e_1\ecomp (e_2\ecomp e_3)= (e_1\ecomp e_2)\ecomp e_3$
\item $s(e_1\ecomp e_2)=s(e_1)$ and  $t(e_1\ecomp e_2)=t(e_2)$;
\end{itemize}
\item morphisms are $\CSig^\ecomp$-homomorphisms.
\end{enumerate} \vspace*{-4.5ex}
\end{DEFINITION}

\newpage
\noindent
Thus, a $\DAG^\ecomp$-morphism $f:G\into H$  maps an edge $\dep xy$ from
$G$ onto en edge from $f(x)$ to $f(y)$ in $H$ -- this latter edge, however,
may correspond to a path obtained by composition of several edges in $H$.
The following fact is easy to verify:
\begin{CLAIM}\label{cl:adj1}
Let
\begin{itemize}\MyLPar
\item $F^\ecomp:\DAG\into\DAG^\ecomp$ be the free functor sending a
$G\in\Obj{\DAG}$ onto its {\em path closure} $G^\ecomp$ by adding freely
all possible compositions of edges from $G$; for a $\DAG$-morphisms $h$, we
let $F^\ecomp(h)=h$~\footnote{Strictly speaking, $F^\ecomp(h)$ is $h$ extended to
paths: $F^\ecomp(h)(e_1\ecomp e_2) = h(e_1)\ecomp h(e_2)$.}
\item
$U^\ecomp:\DAG^\ecomp\into \DAG$ be the forgetful functor sending a 
graph $G^\ecomp$
onto its $\CSig$-reduct $G^\ecomp|_\CSig$;  for a 
$\DAG^\ecomp$-morphisms $h:U^\ecomp(h)=h$.
\end{itemize}
There is an adjunction
$F^\ecomp\adj U^\ecomp$ with the unit being the inclusion of the 
source graph into
its path closure.
\end{CLAIM}
%
The objects in the next category will allow two nodes to be connected not
only by paths but also by all finite, non-empty sets of paths.
\begin{DEFINITION}{}\label{de:distpath}
$\pat\DAG$ is the category of {\em distributed} path closures of DAGs where
\begin{enumerate}\MyLPar
\item objects are  algebras over signature $\pat\CSig=\CSig^\ecomp \cup
\{\_\oplus\_ :E\times E\into E\}$ whose $\CSig^\ecomp$-reducts belong to
$\DAG^\ecomp$ -- for any 
$P\in\Obj{\pat\DAG}:P|_{\CSig^{\ecomp}}\in\Obj{\DAG^\ecomp}$.
$\_\oplus\_$ is a partial operation of parallel composition
  defined for any edges $e_1, e_2$ with $s(e_1)=s(e_2)$
and $t(e_1)=t(e_2)$, which is associative, commutative and idempotent
\begin{itemize}\MyLPar
\item $e_1\oplus(e_2\oplus e_3)=(e_1\oplus e_2)\oplus e_3$
\item $e_1\oplus e_2=e_2\oplus e_1$
\item $e\oplus e=e$
\end{itemize}
composition distributes over $\oplus$:
\begin{itemize}\MyLPar
\item
$e_1\ecomp (e_2\oplus e_3)= (e_1\ecomp  e_2)\oplus(e_1\ecomp e_3)$
\item
$(e_1\oplus e_2)\ecomp e_3= (e_1\ecomp e_3)\oplus (e_2\ecomp e_3)$;
\end{itemize}
and since $\oplus$ is defined only for edges with the same source and target:
\begin{itemize}\MyLPar
\item $s(e_1\oplus e_2)=s(e_1)=s(e_2)$ and 
$t(e_1\oplus e_2)=t(e_1)=t(e_2)$;
\end{itemize}
\item morphisms are $\pat\CSig$-homomorphisms
\end{enumerate} \vspace*{-4.5ex}
\end{DEFINITION}

\noindent
The first three axioms allow formation of sets of edges (which here may be
paths from the ``underlying'' DAG); the next two ensure that composition of
sets of paths corresponds to forming sets of composite paths.
Canonically, an edge $\dep xy$ is a non-empty set of edges (paths in the
``underlying'' DAG) from $x$ to $y$.
Hence, a morphism $f:G\into H$ maps en edge $\dep xy$ from $G$ onto a 
non-empty set
of paths between $f(x)$ and $f(y)$ in $H$.
Again, we have an easy fact:
\begin{CLAIM}\label{cl:adj2}
Let
\begin{itemize}\MyLPar
\item $\patt F:\DAG^\ecomp\into\pat\DAG$ be the free functor sending a
$G^\ecomp\in\Obj{\DAG^\ecomp}$ onto  $\pat G\in\Obj{\pat{\DAG}}$ by 
adding freely
all possible parallel ($\oplus$) compositions of edges from 
$G^\ecomp$ subject to the
restrictions and axioms from the above Definition;
\item
$\patt U:\pat\DAG\into\DAG^\ecomp$ be the forgetful functor  sending 
a  graph $\pat G$
onto its $\CSig^\ecomp$-reduct $\pat G|_{\CSig^{\ecomp}}$;
\item both functors map morphisms onto identical morphisms in their
respective target categories.
\end{itemize}
There is an adjunction $\patt F\adj \patt U$ with the unit being the
inclusion of the source graph into its closure under parallel composition.
\end{CLAIM}
Now define two functors
\Func{\pat F = \Comp{F^\ecomp}{\patt F}:\DAG\into\pat\DAG\ \ {\rm
and}\ \ \pat U = \Comp{\patt U}{U^\ecomp}:\pat\DAG\into\DAG.}
The two adjunctions
  from Propositions~\ref{cl:adj1} and \ref{cl:adj2} give an adjunction 
$\pat F\adj \pat U$. The
morphisms we are looking for can be now defined as
Kleisli morphisms between objects of $\DAG$ induced by this last adjunction.
The resulting category is given by: the objects are simple DAGs, a
morphism $f:G\into H$ is a \DAG-morphisms $f:G\into \pat U({\pat
F}(H))$. Morphisms are composed pointwise: for $f:A\into B$ and $g:B\into C$,
$f$ maps an edge $e=\depo Axy$ onto a set of paths 
$f(e)=\{p_1,p_2...p_n\}$ from
$f(x)$ to $f(y)$ in $B$. Then, $g$ maps each path $p_i$ onto a set of
paths $g(p_i)=\{r_{i1},r_{i2}...r_{ik}\}$ from $g(f(x))$ to $g(f(y))$ in $C$,
obtained by mapping each edge of $p_i$ onto a set of paths and composing
these paths along the $p_i$. The composition is then
$g(f(e))=\bigcup_{p_{i}\in f(e)}g(p_i)$. Formally
%
\begin{DEFINITION}{}\label{de:clgr}
The category $\Gr$ is the full subcategory -- with simple DAGs as objects
-- of the Kleisli category induced by the adjunction
$\pat F\adj \pat U$. \footnote{This concise formulation expresses what we
have described above. For the construction of Kleisli category see 
e.g. \cite{McL}.}
\end{DEFINITION}
\noindent
Since we will be working with the category \Gr,
in the sequel, we will use the notation $\pat G$ -- where $G\in\Obj{\Gr}$ --
as an abbreviation for $\pat U(\pat F(G))$.
%
\begin{EXAMPLE}\label{ex:morGr}
Three examples of \Gr-morphisms are given in Figures~\ref{fi:Demb} and
\ref{fi:exCCC}. The former maps edges onto paths  while the latter
maps an edge onto distributed paths.
\end{EXAMPLE} \vspace*{-3ex}
%

\begin{figure}[hbt]
\spcol{-.5}
\sprow{-1}
\def\objectstyle{\scriptstyle}
\def	\c#1{
	\save \go[r]\merge\go+C\merge\go="c#1"
	\restore}
\hspace*{4em}
\diagram %compileto{Demb}
4 & & & & & \bullet & \bullet
    &&&& 4  & \bullet \\
& 3\ulto\morphism\solid{}{}[u] &&&& \bullet\uto\urto & \bullet\uto\ulto 
    &&&& \bullet \uto \xdotted[ur] & 3 \morphism\solid{}{}[u] \ulto \\
2 \uuto \urto & & & & & \bullet \uto \urto & \bullet \uto \ulto  
   &&&& 2 \morphism\solid{}{}[u] \urto & \bullet \uto \xdotted[ul] \\
& 1 \ulto \uuto	&&&& \bullet \uto \urto & \bullet \uto \ulto
   &&&& \circ \udotted \xdotted[ur] & 1 \morphism\solid{}{}[u] \ulto \\
 & D & & & & & W 
      & & & & f:D \rto & W \\
\enddiagram
\caption{A \Gr-morphism from $D$ to $W$.}\label{fi:Demb}
\end{figure}

\begin{figure}[hbt]
\spcol{-0.5}
\sprow{-1}
\def\objectstyle{\scriptstyle}
\hspace*{3em} \diagram %compileto{RefA}
2 && &  \bullet &  &&&  2 & &&& 2 \\
   && \bullet \urto && \bullet \ulto  && \bullet\urto && \bullet\ulto
     && \bullet\urto && \circ \xdotted[ul] \\
   && \bullet \uto && \bullet \uto  && \bullet \uto && \bullet \uto
     && \bullet \uto && \circ \xdotted[u] \\
1\xto[uuu] && & \bullet \ulto \urto \xto[uuu] &&& & 1 \ulto \urto 
\xdotted[uuu]
    &&& & 1 \ulto \xto[uuu] \xdotted[ur] \\
G && & H & &&  g:G\rto & H & && h:G\rto & H
\enddiagram
{\small \caption{Two $\Gr$-morphisms from $G$ to $H$.}\label{fi:exCCC}}
\end{figure} \vspace*{-2ex}

\noindent
A \Gr-morphism $h:A\into B$ expresses the fact that $B$ can {\em
simulate} $A$ -- edges of $A$ are represented in $B$ not necessarily as
edges, but possibly as paths, or even {\em sets\/} of paths. $B$ may 
also contain
additional edges not in the image of $h$.
%
\begin{SREMARK}{Remark.}
  $\DAG$-embedding of $A$ into $B$ is a special case of a $\Gr$-morphism
$A\into B$. One
might expect that a surjective (on the nodes) $\DAG$-morphism $B\into A$
would yield a simpler notion.
However, it would require our graphs to be reflexive (this is not a big
problem) and,
more importantly, would exclude many cases where $B$ is a richer structure
than $A$.
For instance, there is no surjective $\DAG$-morphism from
$B= \spcol{-1.2}\sprow{-1.6}
\def\objectstyle{\scriptstyle}\def\labelstyle{\scriptstyle}
\diagramcompileto{AA}
& \bullet \ddto \dlto \\ \bullet \drto \\ & \bullet
\enddiagram $
onto
$A=\spcol{-1.2}\sprow{-1.8}
\def\objectstyle{\scriptstyle}\def\labelstyle{\scriptstyle}
\diagramcompileto{BB}
\bullet \rto & \bullet \rto & \bullet
\enddiagram $.
But we would like to say that $B$ does reflect the
edges of $A$ (and introduces
some more) -- and so there is a $\Gr$-morphism $A\into B$.
\end{SREMARK}

\noindent
 From now on, we will treat ``graph'' as a synonym for ``DAG'' since
DAGs are the only graphs we are going to deal with.
Also, our graphs are primarily connected {\em simple} DAGs, in which case we
will often identify a graph $G$ and its edge relation $E^G$, written 
also $\ddepo G$.
Then, we say that $G$ is {\em well-founded\/} when the relation 
$\ddepo G$ is, and
denote by $\tc G$ the transitive closure $\tc{\ddepo G}$.
Notice that, in general, $G^\ecomp\not=\ \tc G\not=\pat G$. The following
example (suggested by an anonymous referee) illustrates the difference between the three:
\[
\xymatrix@C=0.15cm@R=0.4cm{
   &&& \bullet &&  &&         
   &&& \bullet &&  &&
   &&& \bullet &&  &&
   &&& \bullet &&&  && \\
G= & \bullet \ar[rru] &&&& \bullet \ar[llu] && 
   G^\ecomp= & \bullet \ar[rru] &&&& \bullet \ar[llu] &&
   \pat{G} = & \bullet \ar[rru] &&&& \bullet \ar[llu] && 
   \tc{G} = & \bullet \ar[rru] &&&& \bullet \ar[llu] && \\
& && \bullet \ar[llu]\ar[rru] &&  && 
   \Drop{F^\ecomp(G)} & && \bullet \ar[llu]\ar[rru] \ar@<-3pt>[uu] \ar@<3pt>[uu]&&  && 
   \Drop{\patt F(F^\ecomp(G))}& && \bullet \ar[llu]\ar[rru] \ar@<-3pt>[uu] \ar@<3pt>[uu] \ar[uu] &&  && 
   & && \bullet \ar[llu]\ar[rru] \ar[uu] &&  && 
}
\]
Occasionally, we may use the following terms:
\begin{itemize}\MyLPar
\item A graph is {\em sequential} if $\tc\ddep$ is a total ordering of the
nodes.
\item A {\em partitioning} of a graph is a partitioning of the set of nodes
${I = \bigcup_{k=1}^{k=n} I_k}$ with $1 < n < \omega$ such
that each $I_k$ is sequential.
\item An {\em st-graph} (``space-time'') is a simple DAG with nodes indexed by
pairs $\<s,t\>\in
  S\times \Nat$, such that:
  \begin{enumerate}\MyLPar
   \item $\forall s,t: \dep {\<s,t\>}{\<s,t+1\>}$~;
   \item $\forall s, s', t, t': \dep{\<s,t\>}{\<s',t'\>} \impl t'=t+1$~;
   \item $\exists k\in\Nat_+: \forall s, s', t: \dep{\<s,t\>}{\<s',t+1\>} \iff
\dep{\<s,t+k\>}{\<s',t+k+1\>}$.
  \end{enumerate}
\end{itemize}

\noindent
The $st$-graphs are singled out because they will be used as the
{\bf s}pace-{\bf t}ime representation of hardware architectures
\cite{MW83}.  The $s$ component identifies a processor and the $t$
component a time-point during a computation.  Edges represent possible
communications.  The first condition says that each processor performs a
sequential computation with the possibility of maintaining its local data
from one step of the computation to the next one.  The second condition
requires all elementary communications to happen in a single time-step.
The last condition says that the communication topology is fixed throughout
the whole computation, i.e., the graph is actually a repetition of the
pattern of the $k$-steps communications (for some fixed $k>0$).

We note the following characterization:
%
\begin{CLAIM}\label{pr:mono}
Let $k:A\into B \in \Mor{\Gr}$. $k$ is mono
$\iff$ $k$ is injective on the edges.
\end{CLAIM}
\begin{PROOF} Let $T$ be an arbitrary object and
$f,g:T\into A$ two arbitrary morphisms.
\begin{itemize}\MyLPar
\item[$\impl )$]
If $k$ is mono, then for any $f,g:T\into A$, $\Comp
fk=\Comp gk \impl f=g$.
In particular, for any edge $e\in E^T : k(f(e))=k(g(e))\impl f(e)=g(e)$.
Taking $T$ as a one-edge
graph and considering all $f,g:T\into A$, implies that $k$ must be
injective on the edges.
\item[$\Leftarrow )$]
Suppose $k$ is injective on the edges and $\Comp fk = \Comp gk$.  For each
edge $e\in E^T$ we then have $k(f(e))=k(g(e))\impl f(e)=g(e)$.  To verify
that $f=g$, we have to check that also for each node $i$ in $T$,
$f(i)=g(i)$.  Assume that for some node $f(i)\not=g(i)$. Since $T$ is a DAG
(according to Definition~\ref{de:Gr} it has no isolated nodes),
there is an edge $e$ incident to $i$ and then we would have
$f(e)\not=g(e)$. Because of the injectivity on the edges, we then get
$k(f(e))\not=k(g(e))$, contrary to the assumption $\Comp fk=\Comp gk$.
\end{itemize} \vspace*{-4ex}
\end{PROOF}

\subsection{Data Dependencies}\label{sub:dd}
As remarked in introduction, our data dependencies are intended to capture
the information concerning what resources are going to be used in performing
new steps of computation. Resources in this context refer exclusively to the
values of some functions and computations compute such values. Thus 
our data dependencies
relate functional terms and not program
statements. For instance, computation of the value of the term $f(s(x),2)$
may depend on the value of $s(x)$ which,
in turn, may depend on $x$. $x$, however
is just a term -- a mathematical variable which, in any instance, has 
some specific value.
It never depends on itself, as it might do if it was a program
variable occurring  in  a statement $x:=x+1$.
%

For a signature $\Sigma$, a $\Sigma$ {\em data dependency} is a
graph (simple DAG) with nodes labeled by ground $\Sigma$-terms 
$\GTerms$.  The only
requirement we put on the morphisms between such graphs is that they 
respect the sorting of
labels.

\begin{DEFINITION}{[Data dependencies]}\label{de:DDS}
$\DDS$ is the category of $\Sigma$ {\em data dependencies} where:
%($\Gamma$ is as in definition \ref{de:Gr}):
\begin{enumerate}\MyLPar
\item objects are  $\Sigma$ data dependencies -- pairs $\<G,lab\>$ where
$G\in\Obj\Gr$ is a simple DAG and $lab:I^G\into\GTerms$ is a function
labeling nodes of this graph with ground $\Sigma$-terms;
\item a $\DDS$-morphism $m:\dd C=\<C,lab_{\dd C}\>\into \dd 
D=\<D,lab_{\dd D}\>$ is a \Gr-morphism between the
	underlying graphs $m:C\into D$, such that $\forall i\in
I^C:Sort(lab_{\dd D}(m(i))) =
	Sort(lab_{\dd C}(i))$.\footnote{For a signature 
$\Sigma=\<\Sorts,\Funcs\>$, the
function
$Sort:\Ter{\Sigma,X}\into\Sorts$ returns the sort of the argument term.}
\end{enumerate}\vspace*{-3.5ex}
\end{DEFINITION}
\noindent
In the sequel, $\Dclass$ will denote an arbitrary subcategory of 
$\DDS$. The functor
\Func{Gr:{\DDS}\into \Gr\label{fu:gr}} is the obvious forgetful 
functor which, for a given
  $\Sigma$ data dependency $\dd D=\<D,lab\>$, returns its underlying 
graph $D$. We
call such a graph  the {\em shape of $\dd D$}.

The relation $\dep ij$ (in the shape $Gr(\dd D)$ of $\dd 
D=\<D,lab\>$) indicates
that the computation at node $j$
depends on the computation at node $i$. If $lab(i)=s$ and $lab(j)=t$ this means
that the result of the evaluation of $t$ at $j$ depends on the result of
the evaluation of $s$ at $i$.
%
\begin{EXAMPLE}\label{ex:Fib}
To give a flavor of our aims, consider a specification
of the Fibonacci function:\\[1ex]
$ \Spec{Fib}{\Sorts:}{& Nat}
	{\\ \Funcs:&}{\multicolumn{4}{c}{{\rm \ldots the\ usual\ functions\
for\ }Nat}\\
		 F: & Nat & \into & Nat}
	{\\ \Axs:&}{ & F(0) & = & 1 \\ & F(1) & = & 1 \\
		& F(n+2) & = & F(n) + F(n+1)} $

\noindent
Ignoring for the moment details (to be discussed later), we may
easily recognize the
dependency given by the tree of recursive calls $\dd R$ in Figure~\ref{fi:rec}.
Certainly, Definition~\ref{de:DDS} allows us to map this tree, for instance,
  onto a linear order (respecting the ordering in the tree and avoiding loops).
In the following subsection we will couple dependencies with algebras and thus
specify the means of avoiding uninteresting morphisms admitted, in 
principle, by
Definition~\ref{de:DDS}.
In Figure~\ref{fi:rec} we have indicated two
dependency morphisms: $d'$ corresponding to syntactic memoisation (identifying
the nodes with identical labels), and $d''$
corresponding to semantic memoisation (identifying $F(0)$ with $F(1)$)
given the above
specification.\footnote{In Section~\ref{sub:compatible}, we suggest
a more specific notion of a dependency morphism {\em compatible} with a given
class of algebras of which $d''$ is an example.}

Here, we may consider the dependency $\dd D$ as a refinement (improvement)
of $\dd R$.
Notice that $\dd D$ is sequential under the ordering $F(n) < F(m)$ for $n<m$.
Nevertheless, the partitioning of $\dd D$ indicated  by the vertical
arrows implies serious savings of storage space, even if the
computations along both lines
are heavily synchronized and far from independent.
\end{EXAMPLE}\vspace*{-2ex}

\begin{figure}[hbt]
\def\objectstyle{\scriptstyle}
  \spcol{-1.5}
  \sprow{-1.3}
  \diagram
  & & & F(4) & & & & 		& & & & 			F(4) & & & & & & F(4) & \\
  & & F(3) \urto&  & & & &       & & & & 			& F(3) \ulto \udotted
							& & & & & & F(3) \ulto \udotted \\
  & F(2) \urto&  & & & F(2) \xto[uull] & \xto[rrr]^{d'} & 	& & & &
	F(2) \uuto \urto
					& &\xto[rrr]^{d''} & & &  & F(2) \uuto \urto &  \\
  F(1) \urto &  & & F(1) \uulto & F(1) \urto & &  &
			& & & & 		& F(1) \ulto \uuto
						& & & & & & F(1) \ulto \uuto \\
  & & F(0) \uulto & & & & F(0) \uulto  & 	& & & & F(0) \uuto			 & \\
  & & & \text{{\normalsize $\dd R$}} & & & &
			& & & & 			\text{{\normalsize
  $\dd D'$}} &
							& & & & &
  \text{{\normalsize $\dd D$}}
  \enddiagram
\caption{The recursive tree dependency $\dd R$, together with the ``actual''
dependencies for computation of Fibonacci, $\dd D'$ and $\dd D$, and the
dependency morphism $d=\Comp{d'}{d''}$.}\label{fi:rec}
\end{figure}
%
\begin{SREMARK}{Remark.}
As a matter of fact, there is also a dependency morphisms $g:
\dd D\into \dd R$
which embeds $\dd D$ into the leftmost branch of $\pat{\dd R}$. This
illustrates the
fact that refinement of data dependencies is not necessarily an
``improvement'' in the narrow
sense of memoisation -- it rather reflects the relation of one dependency
being able to
perform the computations of which another one is capable.
\end{SREMARK}
%
\subsection{Computation Algebras}\label{sub:ca}
A $\Sigma$-computation algebra is a $\Sigma$ algebra with an associated
$\Sigma$-dependency.
The latter provides a clue for how various operations of the algebra are to
be computed.
In general, for $\Aclass\subseteq\AlgS$, $\Dclass\subseteq\DDS$, an
$\AD$-computation algebra
is an $\Aclass$-algebra with an associated $\Dclass$-dependency.
%
\begin{DEFINITION}{[Computation algebra]}\label{de:CA}
An $\AD$-{\em computation algebra} $C$ is a triple $\<A,\dd B,ev\>$ where
\begin{enumerate}\MyLPar
\item $A\in\Obj{\Aclass}$ ;
\item $\dd B\in\Obj{\Dclass}$ ;
\item $\ev:I^{Gr(\dd B)}\into A$ is an evaluation function
such that $lab_{\dd B}(i)=t \impl \ev(i)=t^A$.
\end{enumerate}\vspace*{-4ex}
\end{DEFINITION}
\noindent
The $\ev$ function is uniquely determined by the algebra $A$ and the 
labeling of
$\dd B$, so we will
usually write a computation algebra as a pair $\<A,\dd B\>$.
When $\AD$ does not matter or is clear from the context, we will simply
write
  {\em computation algebra\/} (instead of
  $\AD$-computation algebra).

Dependencies may (and will) be interpreted as the requirements on the
communications between different space-time points on an actual
architecture.  Consequently, we will refer to the dependency part
$\dd B$ of a computation algebra $\<A,\dd B\>$ as its {\em 
communication} part.
%
\begin{DEFINITION}{[Computation homomorphisms]}\label{de:Chom}
An $\AD$-computation {\em homomorphism} $m:\<A,\dd A,ev_A\>\into 
\<B,\dd B,ev_B\>$ is
a pair $\<h,g\>$ such that
\begin{enumerate}\MyLPar
\item $h : A\into B\ \in\Mor{\Aclass}$ ;
\item $g : \dd B\into \dd A\ \in \Mor{\Dclass}$, and
\item $h(\ev_A(g(i))) = \ev_B(i)$, for all $i\in I^{Gr(\dd 
B)}$.\footnote{The diagram in Figure \ref{fi:homC} commutes, but not in a 
category (at
least none we
have defined explicitly), and so we write the composition of functions
explicitly as $a(b(i))$.}

\begin{figure}[hbt]
\spreaddiagramcolumns{1pc}\spreaddiagramrows{1pc}
\diagramcompileto{Chom}
& & & \dd A \save\go[0,1]\merge\framed<5pt>\restore \rdotted|{\ev_A}|>\tip
		& A \dto^{\textstyle h} \comr{-3}{-6} & \<A,\dd A,ev_A\> 
\dto^{\textstyle m} \\
& & & \dd B \uto^{\textstyle g}
	\save\go[0,1]\merge\framed<5pt>\restore \rdotted|{\ev_B}|>\tip & B &
	\<B,\dd B,ev_B\>
\enddiagram
\caption{A homomorphism between two computation algebras.}\label{fi:homC}
\end{figure} \vspace*{-5ex}
\end{enumerate}
\end{DEFINITION}

\noindent
As with computation algebras,
when $\AD$ does not matter, or is clear from the context, we will speak
about {\em computation homomorphisms\/}.

The primary intuition behind this definition of homomorphism is that it is to
be used to model a refinement process -- in the above definition, we would say
that $\<A,\dd A\>$ refines $\<B,\dd B\>$. Refinement concerns both the
algebra and dependency part. The usual algebra homomorphism $h:A\into B$
corresponds to the classical notion of data refinement -- $A$ may have
multiple representations for a single value from $B$. The (contravariant)
homomorphism $g:\dd B\into \dd A$ represents the possible improvement  of the
dependency $\dd B$ by the dependency $\dd A$ (cf. Figure~\ref{fi:rec}) --
$\dd A$ may introduce paths, even sets of paths, in the place of single edges
in $\dd B$.

In the example~\ref{ex:Fib},
the standard algebra of natural numbers $\Nata$ with function $F$, may give rise
to three different computation algebras
$\<\Nata,\dd R\>$,
$\<\Nata,\dd D'\>$ and $\<\Nata,\dd D\>$. The respective morphisms obtained from Figure~\ref{fi:rec}, e.g.
$\<id_{\Nata},d\>:\<\Nata,\dd D\>\into\<\Nata,\dd R\>$, will represent gradual refinements
of $\<\Nata,\dd R\>$. On the other hand, a 
dependency morphism which
sends, for instance, a node labeled with $F(2)$ to a node labeled with $F(4)$
will not yield a computation homomorphism unless algebra $\Nata$ satisfies some additional 
(and here, unintended) equations.

Thus, refinement corresponds to a standard (data) refinement on the algebra
part, while dependencies may be refined by ``stretching'' (to paths),
``distributing'' (to sets of paths) or augmenting with new dependencies. 
The following two examples illustrate these aspects of refinement.

\begin{EXAMPLE}
We let $\hc C$ and $\hc D$ be as in Figure~\ref{fi:exC}, with $\hc h(a) = a$, $\hc h(b)=d$, and $\hc h(f)=f$.

The commutativity requirement, $\ha h(\ev_C(\hc h(i)))=\ev_D(i)$, restricts
the legal
combinations of $\hc h$ and $\ha h$. In particular, the labels in $\hc C$
may differ from
those in $\hc D$ like with $\hc h(b)=d$ (taking, for simplicity, the names of
the nodes in the Figure to be
their labels), only if $d$ evaluates in $\ha C$
to a value
$\ev_C(d)=x$ representing $\ev_D(b)$, i.e., such that $\ha h(x)=\ev_D(b)$.

Also, refinement ``does not care'' about what is computed in $C$ at the
nodes which are not in the image of $\hc h(\GGr(D))$.  There may be new nodes
and dependencies, also on nodes labeled by terms not occurring in the
labeling of $\hc D$ (like $\dep cf$ in $\hc C$).  In particular, if we
admit a computation algebra with an empty dependency part (i.e., empty
carriers in the category $\Dclass$), a standard algebra $\ha A$ may be
considered a computation algebra with empty communication part, $A=\<\ha
A,\es\>$.  Then any algebra $B$ with non-empty communication part will be a
refinement of $A$ provided there is a homomorphism $\ha h:\ha B \into
\ha A$. 
\end{EXAMPLE}\vspace*{-3ex}
%
\begin{figure}[hbt]
\spcol{-1}
\sprow{-1.5}
\def\objectstyle{\scriptstyle}
\hspace*{12em}\diagramcompileto{Ref}
& \hc C:& f & & & & \\
& \bullet \dear{\urto} & c \dear{\uto} & \bullet \dear{\ulto}
&\morphism\dotted\tip{}[rrr]^{\ev_C}
& & & \ha C \xto[dddd]^{\ha h} & & C \xto[dddd]^h \\
a \dear{\urto} & & & & d \dear{\ulto} \\ \\
& & \uuto^{\hc h} \\
& \hc D:  & f &\morphism\dotted\tip{}[rrrr]^{\ev_D}  & & & & \ha D & & D \\
& a \dear{\urto} & & b \dear{\ulto} & & &
\enddiagram
{\small \caption{Refinement of a computation algebra by adding new dependencies.}\label{fi:exC}}
\end{figure}\vspace*{-3ex}
%
\begin{EXAMPLE}
As an example of possible ``distribution'', consider an
  algebra $D$ with the sorts $S, S_1, S_2$, and the
  dependency $\dep t{f(t)}$ for two terms of sort $S$ (Figure~\ref{fi:exCC}).  Furthermore, let the
  sort $S$ be the Cartesian product of the two other sorts $S=S_1\times S_2$,
  with the obvious projections $\pi_i:S\into S_i$ and pairing function
  $\<\_,\_\>:S_1\times S_2\into S$.  Let $\ha D$ satisfy the ordinary
  equalities for these functions and, in addition, let $\ha D\models t =
  \<t_1,t_2\>$, and $\ha D\models f(t) = \<f_1(t_1),f_2(t_2)\>$.  This
  allows us to design the refinement of the dependency
  structure illustrated in Figure~\ref{fi:exCC} (cf. Figure~\ref{fi:exCCC})
which together with the
identity on $\ha D$ yields a computation homomorphism $\<id_{\ha D},\hc h\>$.
The morphism $\hc h$ refines the dependency $\dep t{f(t)}$ in
$\hc D$  to {\em both\/} paths (through $t_1$ and $t_2$).
\end{EXAMPLE}\vspace*{-3ex}
%
\begin{figure}[hbt]
\spcol{-1}
\sprow{-1.2}
\def\objectstyle{\scriptstyle}
\hspace*{12em}\diagramcompileto{RefA}
& \ha D \xto[rrr]^{id_{\ha D}} & & & \ha D \\ \\
& \<f_1(t_1),f_2(t_2)\> \uudotted_{ev}|>\hole|>\tip & & & f(t)
\uudotted_{ev}|>\hole|>\tip \\
f_1(t_1) \dear{\urto} & & f_2(t_2) \dear{\ulto} & &  \\
  & & & & \llto_{\hc h} \\
t_1 \dear{\uuto} & & t_2 \dear{\uuto} \\
& t \dear{\ulto}^{\pi_1} \dear{\urto}_{\pi_2} & & & t \dear{\xto[uuuu]}
\enddiagram
{\small \caption{Refinement of a computation algebras by distributing dependencies.}\label{fi:exCC}}
\end{figure}\vspace*{-2ex}

\noindent
Computation algebras and their homomorphisms form a category.
\begin{DEFINITION}{[Category of computation algebras]}
$\CA\AD$ is the category with objects being $\AD$-algebras and morphisms
being $\AD$-homomorphisms (and obvious pointwise composition).  When
talking about the entire category $\CA(\AlgS,\DDS)$, we will instead write
$\CAS$.
\end{DEFINITION}
%
We define two obvious functors:
\Func{\hc{\_}:\op{\CA\AD}\into \Dclass \label{fu:dep}}
projects computation algebras onto
their communication part, i.e., $\hc{\<A,\dd A\>}=\dd A$ and
$\hc{\<h,g\>}=g$ with the notation from 
Definition~\ref{de:Chom}.\footnote{Of course, each $\AD$ has its own 
functor
$\hc{\_}_{\AD}$.  But we may as well view each such a functor as a restriction
of $\hc{\_}_{\CAS}$ to the respective source subcategory.}
Similarly
\Func{\ha{\_}:\CA\AD\into \Aclass \label{fu:alg}}
projects computation algebras onto their algebra part ($\ha{\<A,\dd A\>}=A$
and $\ha{\<h,g\>}=h$).

We will now use these functors to simplify the notation for the 
components of a computation
algebra: $A$ representing such an algebra will denote the triple $\comA A$,
and a computation homomorphism $h$ a pair $\<\ha h,\hc h\>$.


\subsection{Machines and Distribution}\label{sub:imp}
We introduce the model of machine architectures and their mappings 
(simulations)
in Section \ref{sub:march}, and then define mappings of a computation algebra
to a given architecture called ``distributions'', Section~\ref{sub:impl}.

\subsubsection{Machine architectures}\label{sub:march}
In order to relate the dependencies to hardware architectures we will
consider the latter as space-time unfoldings of the actual inter-processor
communication structure \cite{MW83}.  Thus, an 
architecture is
simply an $st$-graph, where each node represents a unique time-point
at a given processor and edges represent the possible single-step
communications.  This perspective may be extended to cover dynamically
changing architectures as well (it would merely require graphs rather
than the $st$-graphs) but here we restrict our attention to
the static case.

A morphism between two architectures
  expresses the possibility of using one architecture, $Z$, to {\em
simulate} another, $W$, i.e., that each communication of which $W$ is
capable can be simulated by a set of chains of communications in $Z$.  This is
captured by the existence of a mapping of $W$ into the (possibly
  distributed) path closure of
$Z$, that is, a ${\Gr}$-morphism $W\into Z$.  Consequently, we
define:
\begin{DEFINITION}{[Machine architectures]}\label{de:Arch}
The category of {\em machine architectures} is ${\Gr}$.
\end{DEFINITION}
\noindent
Thus, similarly to the dependency morphisms, an architecture morphisms
$W\into Z$ tells us that, and how, $Z$ can simulate communications of $W$.

We might have given a more restrictive definition of machine 
architectures.  For instance, the
unfoldings of actual processor-architectures will be $st$-graphs
which are simply repeating patterns of one-step communications.  Also,
simulation morphisms might be restricted to $\Gr$-monomorphisms.
Nevertheless, we choose this more generous definition because it will
result in a more uniform treatment -- it will allow us to consider also
shapes of data dependencies as objects, and distributions as morphisms of
the same category $\Gr$.
%
\begin{EXAMPLE}
An $n$-dimensional vector $W$ has $n$ processors connected so that each
processor, except for $W_1$ and $W_n$, has two-ways communication channels
to its left and right neighbor: $W_1\rl W_2\rl\cdots\rl W_n$.  A
space-time unfolding for a 2-dimensional vector is illustrated in Figure
\ref{fi:ZW}.a).  The vertical arrows $\dep
{(W_i,k)}{(W_i,k+1)}$ are added to represent the fact that each processor
can carry its local data from one computation step to the next.

A ring $Z$ with 3 processors where, in each step,
processor $Z_1$ can communicate to $Z_2$, $Z_2$ to $Z_3$, and $Z_3$ to $Z_1$
is given in Figure \ref{fi:ZW}.b).
  This architecture may simulate the 2-dimensional
vector $W$ by the morphism $m$ indicated in Figure \ref{fi:ZW}.c).
\end{EXAMPLE}\vspace*{-3ex}
%
\begin{figure}[hbt]
\spcol{-0.6}
\sprow{-1}
\def\objectstyle{\scriptstyle}
\hspace*{2em}
\diagram %compileto{ZW}
\bullet & & \bullet & &  &
	\bullet & \bullet & \bullet & &	&
	\bullet & \circ & \bullet \\
& & & & & \bullet \uto \urto & \bullet \uto \urto & \bullet \uto \ullto & & &
	\bullet \uto \urdotted & \bullet \udotted \urto & \bullet \uto \ullto \\
\bullet \uuto \uurrto & & \bullet \uuto \uullto & & &
	\bullet \uto \urto & \bullet \uto \urto & \bullet \uto \ullto & & &
	\bullet \morphism\solid{}{}[u] \morphism\solid{}{}[ur] & \circ \udotted
\urdotted &
		\bullet \morphism\solid{}{}[u]\ulldotted \\
& & & & & \bullet \uto \urto & \bullet \uto \urto & \bullet \uto \ullto & & &
	\bullet \uto \urdotted & \bullet \udotted \urto & \bullet \uto \ullto \\
% bottom line
\bullet \uuto \uurrto & & \bullet \uuto \uullto & & &
	\bullet \uto \urto & \bullet \uto \urto & \bullet \uto \ullto & & &
	\bullet \morphism\solid{}{}[u] \morphism\solid{}{}[ur] & \circ \udotted
\urdotted &
		\bullet \morphism\solid{}{}[u] \ulldotted \\
W_1\rrto<.5ex>& & W_2\llto<.5ex>& & & Z_1 \rto & Z_2 \rto & Z_3
\lltod^{\text{\normalsize b)}}
		& & & &	\save [0,0]\Drop{m:W\into Z} \restore \\
& \text{{\normalsize a)}} & & & & & &
	& & & & \text{{\normalsize c)}}
%
\enddiagram
\caption{Space-time representations of a) 2-dimensional vector $W$, b)
3-processor ring $Z$,
and c) a simulation $m$ of $W$ by $Z$.}\label{fi:ZW}
\end{figure}\vspace*{-3ex}
%
\subsubsection{Distribution}\label{sub:impl}
Assume that we would like to implement a computation algebra on 
some specific machine
architecture. {\em Distribution} is what, in our abstract setting, 
represents an
implementation. It certainly does not capture all aspects of an actual
implementation, and for this reason we have chosen to call it
``distribution''. This term reflects the intuition of distributing the
dependencies of an algebra onto the communication structure of a 
given architecture.

The composition of functors (\ref{fu:gr}) and (\ref{fu:dep}) 
$\Comp{\hc{\_}}{Gr}:
\op{\CA\AD}\into{\Dclass}\into\Arch$ applied to an algebra returns the
underlying graph of its communication part. We will write this composition as
$\GGr$, i.e., $\GGr(X)= Gr(\hc X)$, for an object or morphism $X$ in $\CAS$.

Distribution of an algebra $\shcomA C$ on a given architecture $W$
amounts to a simulation of the (shape of the) dependency structure given
by $Gr(\hc{C})$ on the communication structure $W$.
%
\begin{DEFINITION}{[Distribution]}\label{de:implem}
A {\em distribution} $\imca CmW$ of a computation algebra $C$ on an architecture
$W$is a
  $\Arch$-morphism $m:{\GGr}(C)\into W$.
\end{DEFINITION}
%
\begin{EXAMPLE}
The dependency $\dd D$ (with the shape $Gr(\dd D)$ from  Figure~\ref{fi:rec})
is mapped onto a
2-processors vector $W$ as it was illustrated in example~\ref{ex:morGr},
Figure~\ref{fi:Demb}.
Figure~\ref{fi:Dembb} is just a repetition of this earlier example:
a) is the data dependency $\dd D$ with the shape $Gr(\dd D)$,
b) is a 2-processors vector $W$ and c) is a distribution  $f$ of $\dd D$
on $W$.
\end{EXAMPLE}\vspace*{-3ex}
%
\begin{figure}[hbt]
\spcol{-.5}
\sprow{-1}
\def\objectstyle{\scriptstyle}
\def	\c#1{
	\save \go[r]\merge\go+C\merge\go="c#1"
	\restore}
\hspace*{4em}
\diagram %compileto{Demb}
F(4) & & & & & \bullet & \bullet & 4 & & & & F(4)  & \bullet \\
  & F(3) \ulto \morphism\solid{}{}[u]	& & & & \bullet \uto \urto & 
\bullet \uto
\ulto   & 3 & &
	& & \bullet \uto \xdotted[ur] & F(3) \morphism\solid{}{}[u] \ulto \\
F(2) \uuto \urto		& & & & & \bullet \uto \urto & \bullet \uto 
\ulto & 2 & &
	& & F(2) \morphism\solid{}{}[u] \urto & \bullet \uto \xdotted[ul] \\
  &F(1) \ulto \uuto	& & & & \bullet \uto \urto & \bullet \uto \ulto   &
				1 \xto '[u]'[uu][uuu]  & &
	& & \circ \udotted \xdotted[ur] &F(1) \morphism\solid{}{}[u] \ulto \\
\c1 &  & & & & W_1 \rto<.5ex>& W_2 \lto<.5ex> & & & & & \c2 & \\
\c3 &  & & & & & \text{{\normalsize b)}} & & & & & \c4 &
\save \go"c2" \Drop{f:Gr(\dd D)\into W} \restore
\save \go"c1" \Drop{\dd D} \restore
\save \go"c3" \Drop{\text{\normalsize a)}} \restore
\save \go"c4" \Drop{\text{\normalsize c)}} \restore
\enddiagram
\caption{a) Data dependency $\dd D$, b) space-time
representation of a
2-dimensional vector $W$, and c) a distribution $f$ of $\dd D$ on $W$.}\label{fi:Dembb}
\end{figure}\vspace*{-2ex}


\noindent
Combining such distribution morphisms with the simulation morphisms
between various architectures, leads to the possibility of porting the
distributions.

\begin{EXAMPLE}
The mapping $f$ from Figure \ref{fi:Dembb}.c) is a
distribution of
a computation algebra $D=\<N,\dd D\>$ (with $N$  a $\Sigma_{\mbox{\small\sc
Fib}}$-algebra satisfying
{\sc{Fib}}) on the vector $W$.
Then, the mapping $m$ (Figure \ref{fi:ZW}.c) gives a ported distribution of $D$
on the ring $Z$.
The results are shown in Figure \ref{fi:port}.
\end{EXAMPLE}\vspace*{-3ex}
%
\begin{figure}[hbt]
\sprow{-1.5}
\spcol{-0.5}
\def\objectstyle{\scriptstyle}
\def	\k#1{
	\save \go[r]\merge\go+C\merge\go="k#1"
	\restore}
\hspace*{4em}
\diagram %compileto{Port}
F(4) & & & & F(4)  & \bullet & & & F(4)  & \circ  & 	\bullet  \\
% 4
& & & & & & & & \bullet \uto\urdotted & \circ \udotted\urdotted &
	\bullet\morphism\solid{}{}[u] \ullto\\
% 5
& F(3) \uulto \morphism\solid{}{}[uu] & & &
	\bullet \uuto \uurdotted & F(3) \uulto \morphism\solid{}{}[uu] & & &
	\bullet \morphism\solid{}{}[u] \urdotted & \circ \udotted \urdotted &
		F(3) \morphism\solid{}{}[u] \ulldotted \\
% 6
& & & & & & & & \bullet \morphism\solid{}{}[u] \urdotted & \bullet \udotted
\urto &
	\bullet \uto \ulldotted \\
% 7
F(2) \morphism\solid\tip{}[uuuu] \uurto & & & &
	F(2) \morphism\solid{}{}[uu] \uurto & \bullet \uuto \uuldotted & & &
	F(2) \morphism\solid{}{}[u] \morphism\solid{}{}[ur] & \circ \udotted
\urdotted &
		\bullet \morphism\solid{}{}[u] \ulldotted \\
% 4-8
& & & & & & & & \circ \udotted\urdotted & \circ\udotted\urdotted &
	\bullet\morphism\solid{}{}[u] \ullto\\
% 5-9
& F(1) \uulto \morphism\solid\tip{}[uuuu] & & &
	\circ \uudotted \uurdotted & F(1) \uulto \morphism\solid{}{}[uu] & & &
	\circ \udotted \urdotted & \circ \udotted \urdotted &
		F(1) \morphism\solid{}{}[u] \ulldotted \\
\k1 & & & & \k2 & & & & & \save [0,0]\Drop{\Comp fm:Gr(\dd D)\into Z}
\restore \\
\k3 & & & & \k4 & & & & & \text{{\normalsize c)}}
\save \go"k1" \Drop{\dd D} \restore
\save \go"k2" \Drop{{f:Gr(\dd D)\into W}} \restore
\save \go"k3" \Drop{\text{\normalsize a)}} \restore
\save \go"k4" \Drop{\text{\normalsize b)}} \restore
\enddiagram
\caption{a) Data dependency $\dd D$, b) the distribution $f$ of $\dd D$ on a 2-dimensional
vector $W$,
and c) distribution $f$ ported to the ring $Z$.}\label{fi:port}
\end{figure}\vspace*{-2ex}

\noindent
A morphism between two distributions is defined as follows:
\begin{DEFINITION}{[Morphisms of distributions]}\label{de:imor}
A morphism  between two $\AD$ distributions,
$m:\imca AfW \into \imca BgZ$, is a pair $\<h,h'\>$,
where:
\begin{enumerate}\MyLPar
\item $h:A\into B \in \Mor{\CA\AD}$ ;
\item $h':Z\into W \in \Mor{\Arch}$, and 
\item $\Comp{\GGr(h)}f = \Comp g{h'}$.
\end{enumerate}\vspace*{-4.5ex}
\end{DEFINITION}
The situation is depicted in Figure \ref{fi:homid}. The third condition is
the requirement of the commutativity of the rightmost square in
$\Gr$.\footnote{We let the distribution morphisms go in the opposite 
direction as
the respective $\Gr$-morphisms between the distributions. This choice is to
keep with the convention that morphisms between structures reflect some form
of ``refinement'' of the target by the source.}

\begin{figure}[hbt]
\spreaddiagramcolumns{1pc}\spreaddiagramrows{1pc}
\hspace*{4em}
\diagramcompileto{IDhom}
& \ha A \save\go[0,1]\merge\framed<5pt>\restore \dto_{\textstyle \ha h} &
    \hc{A} \ldotted|{\ev_A}|>\tip 
	& Gr(\hc{A}) \rto^f  \symb{\stackrel{\GGr}{\Large\Longrightarrow}}{-3}{-6}
	& W \coml{-3}{-6} && \imca AfW \dto^{\<h,h'\>} \\
& \ha B \save\go[0,1]\merge\framed<5pt>\restore &
   \hc{B} \ldotted|{\ev_B}|>\tip \uto^{\textstyle \hc{h}} 
& Gr(\hc{B}) \uto|{Gr(\hc h)} \rto_g & Z \uto_{h'}  && \imca BgZ
\enddiagram
\caption{A morphism $\<h,h'\>$ of distributions.}\label{fi:homid}
\end{figure}
\begin{DEFINITION}{}
We let $\IM\AD$ be the category
\begin{itemize}\MyLPar
\item objects -- distributions of $\AD$ computation algebras (Definition
\ref{de:implem}),
\item morphisms given in Definition \ref{de:imor} and pointwise
composition.\footnote{This works because the commutativity of the square in
Figure \ref{fi:homid} -- i.e., for each edge $e$ in $\GGr(B)$, the (sets
  of) paths
obtained in $W$ by $h'(g(e))$ and by $f(\GGr(h)(e))$ are identical --
implies commutativity for each (set of) paths in $\GGr(B)$.  Hence 
composition with
another such commuting square, given by $\GGr(n):\GGr(C)\into \GGr(B)$,
$n':Y\into Z$, and $k:Gr(\hc C)\into Y$, will yield a commuting
square $\Comp{(\Comp{\GGr(n)}{\GGr(h)})}f = \Comp k{(\Comp{n'}{h'})}$.}
\end{itemize} \vspace*{-4.5ex}
\end{DEFINITION}
%
The forgetful functor 
\Func{\ArchF:\IM\AD\into\op\Arch\label{fu:march}} is the obvious
projection of distributions onto their architecture part, i.e. $\ArchF(\imca
AfW)=W$ and $\ArchF(\<h,h'\>)=h'$.
\footnote{Again,
we may view each $\ArchF_{\IM\AD}$ as a restriction of 
$\ArchF_{\IM{(\CAS)}}$ to
$\IM{(\CA\AD)}$.}

\subsection{The Development Process}\label{sub:dev}
The introduced framework is motivated by the view of
the development of (possibly parallel) implementations by means of
computation
algebras.  We envision the process starting with the usual refinement of
algebraic specifications which proceeds through a series of subclasses of
$\AlgS$ and results in one class of isomorphic algebras (an algebra $\ha A$
in Figure \ref{fi:method} is a chosen representative from this 
resulting class).
At this point, when the specification has become sufficiently concrete, one
introduces the dependencies, i.e., constructs a computation algebra
$\chom B$.  Dependencies can then be refined along with a further
refinement of the algebra itself or, perhaps, keeping the algebra fixed.
%
\begin{SREMARKno}{Remark.}\label{re:some}
Since there is no surjectivity requirement on $\ev$, the communication part
$\hc A$ need not contain the full information about all operations of the
algebra $\ha A$.  (In the extreme case, the shape of $\hc A$ might be the
empty graph.)  There are several reasons for the lack of such a
requirement.  Firstly, it allows us to consider a standard algebra $\ha A$ as
a computation algebra with empty communication part.  More importantly,
it allows us to model the development process in which more and more
operations of the algebra are gradually equipped with more detailed
computational information.  Furthermore, we may imagine a specification of
computation algebras as a kind of algebraic specification with 
hidden functions: only
the visible operations are of interest to the implementation, and the
resulting computation algebra need only provide the dependency 
structure for these
operations.
Finally, we can use this setting in  situations where (parts of) an algebra
are implemented on some machine (e.g. all built-in types constitute a
given algebra), and one need to compute some additional, newly defined
functions -- the communication part will then be needed only for 
these new functions.
\end{SREMARKno}
%
When all the required operations of the algebra have been associated with
the satisfactory dependencies, $\shcomA C$, one may choose an architecture
on which to distribute the dependencies $\hc C$,  $\imca CfW$.
Once this is done, there remains only the question of portability, that is,
of moving the distribution to other architectures.

One may tend to imagine that a transition from one level (category) to the
next means that the structure resulting from the development at the
previous level is complete and remains fixed for the rest of the process.
Thus, having reached the algebra $\ha B$ we begin to introduce and refine
dependencies while $\ha B$ remains unchanged (and so, $\ha C$ is actually
$\ha B$).  Similarly, having designed $\shcomA C$ and distributing it on
$W$, we may move to another architecture $Z$ but $\shcomA C$ will be kept
fixed (i.e., $\shcomA D = \shcomA C$).  Although this is probably the most
natural scenario, Figure \ref{fi:method} illustrates the process in its full
generality, where {\em all\/} the aspects may change at {\em all\/} levels.

\begin{figure}[hbt]
\spcol{-0.5}
\sprow{-0.5}
\diagramcompileto{Method}
& & &  & \rline^{\textstyle \IM\AD} &  & & \rline^{\textstyle \CA\AD}
	& & & \rline^{\textstyle \Aclass} & \\
& & \op\Aclass &\ha{D} & \lto|<\hole|>\hole \rdotted|<\hole|>\hole
& & \lto|<\hole|>\hole \ha{C} & \lto|<\hole|>\hole \rdotted|<\hole|>\hole
& & \lto|<\hole|>\hole \ha{B} & \lto|<\hole|>\hole \rdotted|<\hole|>\hole
& & \lto|<\hole|>\hole \ha{A}\\
& & \Dclass & \hc{D} \uto|<\hole|>\hole_{\ev_{D}} \dto|<\hole|>\hole^g
    & \lto|<\hole|>\hole  \rdotted|<\hole|>\hole
    & & \lto|<\hole|>\hole  \hc{C} \uto|<\hole|>\hole_{\ev_{C}} 
\dto|<\hole|>\hole^f
      & \lto|<\hole|>\hole \rdotted|<\hole|>\hole
    & & \lto|<\hole|>\hole \hc{B}\uto|<\hole|>\hole_{\ev_{B}} \\
& & \Arch &  Z & \lto|<\hole|>\hole  \rdotted|<\hole|>\hole
	& & \lto|<\hole|>\hole  W
\enddiagram
\caption{The development process.}\label{fi:method}
\end{figure}
%
Although the technical results below
are rather straightforward and hardly surprising, they provide a sound
justification for the above methodology as well as a clue as to how it may
be applied.  They yield a pleasing conceptual structure in which algebras can
be thought of as indexed by various dependencies, and computation
algebras by machine architectures.  More importantly, they give us the
possibility of modularizing the implementation process and reusing its
results.

The simulation mappings between architectures (the $\Arch$-morphisms) may be
defined and stored independently of the actual programs implemented on them.
Suppose that we have distributed a computation algebra $D$ on an
architecture $W$, $\imca DdW$ (Definition~\ref{de:implem}). In order to
distribute $D$ on another architecture $Z$, we need to define an appropriate
distribution $\imca DcZ$. However, if we have a $\Arch$-morphism $h:W\into
Z$, it will tell us how $Z$ can simulate any computation on $W$. Thus it will
allow us to port the distribution $\imca DdW$ by just using this 
general simulation $h$.

Such a reuse of (stored) simulation morphisms between architectures for
porting distributions will be formalized by
  showing that the functor $\ArchF$ from (\ref{fu:march})
yields a {\em fibration}. Intuitively, this means that we can view the category
$\IM\AD$ as classes of
computation algebras indexed by machine architectures -- the {\em fiber}
  over a given architecture $W$ represents all computation
algebras which can be distributed on $W$. Then, given an architecture
morphism $h:W\into Z$ and an algebra $D$ distributed on $Z$ (in $Z$'s
fiber), there is a canonical way of defining a {\em cleavage}: an algebra
$C$ and its distribution $\imca CcZ$  on $Z$ together with a distribution
morphism $cl(h,D):\imca CcZ\into \imca DdW$ which is ``compatible''
with the simulation $h$.  As a
matter of fact, the triviality of the proof (that $\ArchF$ is a fibration)
coincides with the desired fact that $D=C$. Thus, cleavage gives explicitly
the distribution resulting from porting
a distribution of $D$ from $W$ to $Z$ ``along'' a given simulation $h$.

The concept of fibration seems a very natural model for porting
distributions and we will encounter it again later on. We now recall the
notion of fibration, 
Definition~\ref{de:fib},
(for more on  fibrations, see \cite{Ben, Her, Jac}) and show that
$\ArchF$ is a fibration in Proposition \ref{le:MAr}.
\begin{DEFINITION}{[Fibration]}\label{de:fib}
Given two categories $\Cat A$ and $\Cat X$ and a functor $F:\Cat A\into \Cat
X$:

\hspace*{7em}
\spcol{0.3}
%\sprow{-0.9}
\diagram%compileto{FibM}
& A \dto_f \drto^g  & &  \\
\Cat A \dto^{F} & B \rto^h  & C & \\
\Cat X & Y  \rto^{u} & X  &
\enddiagram

\begin{itemize}\MyLPar
\item
A morphism $h:B\into C$ in $\Cat A$ is {\em cartesian} if for any
morphism $g:A\into C$ with $F(g)=F(h)$, there exists a unique $f:A\into B$
such that $\Comp fh=g$.
\item A functor $F$ is a {\em fibration} if for every $C\in\Obj{\Cat
A}$ and $u:Y\into F(C)\in\Mor{\Cat X}$, there exists a cartesian morphism
$h:B\into C$, and
the composite of two cartesian morphisms is cartesian.
\item $F$ is a {\em cofibration} if $F^{op}:\Cat A^{op}\into \Cat X^{op}$ is
a fibration; it is {\em bifibration} if it is both fibration and cofibration.
\item An  $A\in\Obj{\Cat A}$ is {\em over} $Y$ if $F(A)=Y$; a
morphism $g\in\Mor{\Cat A}$ is {\em over} $u$ if $F(g)=u$. For an object
$Y\in\Obj{\Cat X}$, {\em the fiber} over $Y$, $\fibre FY$, is the 
subcategory of $\Cat A$
with objects all those over $Y$ and morphisms all those over $id_Y$.
\item A particular choice of a
cartesian lifting for every appropriate morphism $u\in\Mor{\Cat X}$ is called
a {\em cleavage} for $F$. Given a cleavage $cl$, a morphism $u:Y\into 
X\in\Mor{\Cat
X}$ and an object $C\in\Obj{\Cat A}$ over $X$, we denote by $cl(u,C)$ the
cartesian lifting of $u$ whose codomain is $C$.
\end{itemize}
We will mention {\em split} fibrations:
\begin{itemize}\MyLPar
\item Given a fibration $F$ with a cleavage $cl$, any $u:Y\into X$ in $\Cat
X$ determines a reindexing functor $\reind u:\fibre FX\into\fibre FY$ 
as follows:
  \begin{itemize}\MyLPar
  \item for an object $C$ in $\fibre FX$, $\reind u(C)$ is the domain 
of $cl(u,C)$
  \item for a morphism $f:D\into C$ in $\fibre FX$, $\reind u(f)$ is the unique
  morphism in $\fibre FY$ making $\Comp{\reind u(f)}{cl(u,C)} = 
\Comp{cl(u,D)}f$.
  \end{itemize}
\item For every $Y\in\Obj{\Cat X}$ there is a natural isomorphism
$\nattr{id_{\fibre F{Y}}}{\reind{(id_Y)}}$, determined by the universal
property of $cl(id_Y,A)$ for each $A\in\Obj{\Cat A}$. For all $u:Y\into X$ and
$w:X\into W$ there is a natural isomorphism $\nattr{\Comp{\reind w}{\reind
u}}{\reind{(\Comp uw)}}$, determined by $cl(\Comp uw,A)$. If these isomorphisms
are identities then the fibration
is {\em split}.
\end{itemize} \vspace*{-4ex}
\end{DEFINITION}
%
\begin{CLAIM}\label{le:MAr}
For any $\AD$, $\ArchF:\IM\AD\into\op\Arch$ is a (split) fibration.
\end{CLAIM}
\begin{PROOF}
The diagram to the left illustrates the fibration situation and the one to
the right gives details for the present context (dashed arrows indicate
the directions in the $\_^{op}$ categories):

\noindent
\hspace*{1em}
\spcol{0.3}
\sprow{-0.9}
\def	\a#1{ \ddotted|<{\rotate\tip}
	\save \go[d]\merge\go+C\merge\go="a#1"\framed<5pc>
	\restore}
\diagramcompileto{FibM}
& & &  					& \a1 \ha B \\
& \imca BbZ \ddto_f \ddrto^g  & &  	& \hc B \xto[ddddr]_b \\
& &  & 					& &  \a2 \ha{D} & \a3 \ha{D}  \\
\IM\AD \ddto^{\ArchF} & \imca CcZ \rto^h  & \imca DdW &
					& &\hc{D} \ddto^{\Comp d{h'}} & \hc{D} \ddto^d \\ \\
\op\Arch & Z  \rdashed^{\op{h'}}|>\tip & W  &  	& & Z  & W  \lto^{h'}
\save \go"a1" \xto"a2"^{f} \xto"a3"^{g}\restore
\save \go"a2" \xto"a3"_{id_D} \restore
\enddiagram
%
\pp{1ex}
For an $\op\Arch$-morphism $\op{h'}:Z\dinto W$ and distribution
$d=\imca DdW$, the
cleavage is defined as
\begin{equation}\label{eq:clM}
cl(\op{h'},\imca DdW) \Def h:\imca DcZ\into \imca DdW\ where
	\left\{ \begin{array}[h]{l}
	h= \<id_D,{h'}\> \\
	\imca DcZ = \Comp d{h'} 
	\end{array}\right .\
\end{equation}
%
\noindent
It is trivially a morphism in $\Mor{\IM\AD}$ since
$\Comp{id_{\GGr(D)}}c = \Comp d{h'}$.

Let $g:\imca BbZ\into \imca DdW$ be $\<\chom g,h'\>$ -- a distribution over
$\op{h'}$.
The unique $f:b\into c$ with
$\ArchF(f)=id_{Z}$ making $\Comp{f}{h}=g$ is then $\<\chom g, id_Z\>$.
$f\in\Mor{\IM\AD}$
because $\Comp{\hc g}b=\Comp d{h'} = \Comp c{id_Z}$, where the first
equality
holds since $g\in\Mor{\IM\AD}$.

Composition of cleavages is a cleavage: for $h':W\into Z$, $g':Z\into V$
and $\imca DdW\in \fibre{\ArchF}{W}$ we have
$cl(\op{h'}, \imca DdW)=\<id_D,h'\>$, $cl(\op{g'},\imca D{(\Comp 
d{h'})}Z)=\<id_D,g'\>$, so

$\Comp{cl(\op{g'},\imca D{(\Comp d{h'})}Z)}{cl(\op{h'},\imca DdW)}
= \<id_D,\Comp{h'}{g'}\> = cl(\Comp{\op{g'}}{\op{h'}},\imca DdW).$ \\
Hence $\ArchF$ is a fibration.

It is split:
we have $\reind{(id_{W})}(\imca DdW)= \imca DdW$, so the natural transformation
$\nattr{id_{\fibre{\ArchF}{W}}}{\reind{(id_W)}}$ is identity.
Composition of cleavages gives a cleavage,
so we get $\Comp{\reind{(\op{h'})}}{\reind{(\op{g'})}} =
\reind{(\Comp{\op{g'}}{\op{h'}})}$.
\end{PROOF}
\noindent
%
\begin{EXAMPLE}
The collection of our examples so far illustrates this idea
of development. In example~\ref{ex:Fib} we designed a dependency $\dd 
R$ which was
then refined to $\dd D$ by the morphism $d:\dd R\into \dd D$ -- assuming
standard algebra for natural numbers $\Nata$. The dependency $\dd
D$ was then distributed on a vector $W$ by $\imca DfW$ (Figure~\ref{fi:Dembb})
and ported to $Z$ (Figure \ref{fi:port}) using the simulation $m:W\into Z$
from Figure~\ref{fi:ZW}.
Provided that the simulation
$m$ and distribution $f$ were available from a library, all the work 
we had to do was to
design the mapping $d$.
%
\end{EXAMPLE}\vspace*{-3ex}
\begin{figure}[hbt]
\hspace*{16em}
\diagramcompileto{Sum}
\Nata   \ar@{-}[r]^{id_\Nata}    & \Nata \ar@{-}[r]^{id_\Nata}  & \Nata \\
\dd D  \ar[d]_{\Comp fm}  \udotted|{ev}|>\hole|>\tip &
	\dd D \ar@{-}[l]_{id_D} \ar[d]^f
	\udotted|{ev}|>\hole|>\tip &
\dd R \lto_d  \udotted|{ev}|>\hole|>\tip \\
Z 	& W \ar[l]^m
\enddiagram
\caption{The summary of the example.}\label{fi:sum}
\end{figure}\vspace*{-3ex}

\section{Constructing dependencies for recursive functions}\label{se:cr}
So far we have considered relationships between various aspects --
functional specification (algebra), dependencies (communication part) and
distribution -- given independently from each other.
In this section we show how -- under certain conditions -- dependencies can
be derived from specifications using grammars. 
Although the described construction may
indicate the possibilities of reconsidering implementation techniques for
functional languages, it should be emphasized that we are here concerned only
with {\em a particular example} of utilizing computation algebra framework for desigining
efficient implementations of algebraic specifications. 


We recall that a 
  \emph{context free grammar} $G$ is given by a 4-tuple 
$(T,N,\prd,S)$ where
	$T$ is a set of terminal symbols, $N$ is a set of non-terminal
	symbols auch that 
	$N\cap T=\emptyset$, the set of production rules is 
	$\prd\subseteq N\times (T\cup N)^{*}$, and
	$S\in N$ is the start symbol.
     From the grammar $G$ we get the \emph{derives} relation
     $\Longrightarrow\subseteq (N\cup T)^{*}\times (N\cup T)^{*}$ by
     $vXw \Longrightarrow vuw$ for strings $v,w\in (N\cup T)^{*}$ and
     productions $X \prd u$.
     The grammar defines the \emph{language}, the set of strings,
     ${\cal L}(G) = \{w\in T^{*} | S\Longrightarrow^{*} w\}$ where
     $\Longrightarrow^{*}$ is the reflexive, transitive closure of
     $\Longrightarrow$.

In terms of Figure~\ref{fi:method}, we are at the transition point from
$\Aclass$ to $\CA(\Aclass,\Dclass)$: we 
have obtained a specific algebra $A$ and introduce a new recursive 
function to be
computed over $A$. For this new function, we construct the
dependency relatively to the algebra $A$. The difference from the general
setting described in Section \ref{sub:dev} is
that we want to compute a new
function over a given algebra and not construct a dependency for the whole
algebra (cf. Remark~\ref{re:some}).
This is quite a common scenario in practice and we will
give a few examples.

Let $\Sigma$ be a signature, $S_1...S_n,S$ be some sorts from
$\Sigma$,
and $R:S_1\times...\times S_n\into S$ a new function symbol not in $\Sigma$.
A {\em generalized $\Sigma$-term recurrence} for $R$
is given by a set of equations of the form
\begin{eqnarray}\label{def:recur}
  R(\ovl{\tau_0}(X)) = \Phi_{\ovl{\tau_{0}}(X)}
(\  X,R(\ovl{\tau_{1}}(X)),R(\ovl{\tau_{2}}(X)),\ldots,
    R(\ovl{\tau_{{z}}}(X))\ )
\end{eqnarray}
where $X$ is a (sequence of) variable(s), for each $0\leq k\leq
z:\ovl{\tau_{k}}(X)$ is a sequence
of terms $t_{1,k}(X)...t_{n,k}(X)$ of appropriate sorts, i.e.,
$t_{i,k}(X) \in\Ter{\Sigma}(X)_{S_{i}}$, and
$\Phi_{\ovl{\tau_{0}}(X)}(X,y_1...y_n)\in \Ter{\Sigma}(X,y_1...y_n)_S$
with all $y_i$ of sort $S$. $R$ is typically given by a {\em set} of such
equations, i.e.,
different equations for different (sequences of) term(s)
$\ovl{\tau_0}(X)$. (This form of generalized recurrence was studied in
\cite{Hav90,Hav93,CH95,HS98} as a definition schema for
``constructive recursive functions''.)

Given a recurrence \re{def:recur} and a $\Sigma$-algebra $A$, we take 
as the set of
nodes for the shape of the dependency, the domain of $R$ in $A$, i.e.,
the cartesian
product $V=S_1^A\times...\times S_n^A$.

We now construct a collection of context free grammars 
$G_{\Sigma,V}$, which will define the data
dependency. As the non-terminal symbols of the grammar, we take the set
$N=\{N_v:v\in V\}$ of distinct symbols indexed by the nodes ($N_u=N_v\impl
u=v$).
Treating the non-terminals $N_{v}$ as place-holders of type $S$, we may
now define substitution rules from (\ref{def:recur}).
For each variable assignment $\alpha:X\into A$, let $\alpha_i\in V$ denote the
interpretation
of $\ovl{\tau_i}(X)$ under $\alpha$, i.e., $\alpha_i =
\ovl{\tau_i}^A(\alpha(X)) =
\<t_{1,i}^A(\alpha(X)),...,t_{n,i}^A(\alpha(X))\>$.
For every expression form ${\ovl{\tau_{0}}}$ (from the set of equations
(\ref{def:recur})) and every assignment $\alpha:X\into A$, we get the 
production:
\begin{eqnarray}\label{eq:recur}
  N_{\alpha_{0}} \prd_{G_{\Sigma,V}}
\Phi_{\alpha_{0}}^A(\alpha(X),N_{\alpha_{1}},N_{\alpha_{2}},\ldots,N_{{\alpha
_{z}}})
\end{eqnarray}
Thus, for each node $v$, we obtain a node-specific grammar
$G_{\Sigma,v}=\<\Funcs,N,\prd_{G_{\Sigma,V}},N_v\>$, i.e., where
terminals are the function symbols $\Funcs$ from $\Sigma$,
non-terminals are the $N$'s,
productions are given by~(\ref{eq:recur}), and the start symbol is $N_v$.

We define the shape of the dependency $\ddepo{G_{\Sigma,V}}$ by 
introducing an edge for all
pairs $u,v\in V$, according to the rule:
\equ{\label{eq:dep}
         u \ddepo{G_{\Sigma,V}} v\ \ \ \iff\ \ \
         N_v \prd_{G_{\Sigma,V}} r N_u t, \ \ {\rm for\ some\ }r,t\ 
{\rm with\ }rN_ut\in\Ter{\Sigma}(N).
}
This gives us a graph
structure on $V$, and we can define a ``compatible'' $lab$eling function by
requiring that
\equ{\label{eq:uni}
\forall v\in V : lab_{G_{\Sigma,V}}(v)\in {\cal L}(G_{\Sigma,v})
}
where ${\cal L}(G_{\Sigma,v})$ is the language generated by the grammar
$G_{\Sigma,v}$ of node $v$.

The result may not be a data dependency: the
graph $\<V,\ddepo{G_{\Sigma,V}}\>$ obtained from \re{eq:dep} may be cyclic,
or else
the labeling \re{eq:uni} may not be unique, problems
often due to the recurrence \re{def:recur} not being well-defined to begin
with.
The following  conditions on $G_{\Sigma,V}$ ensure that the result will be a
data dependency -- a (simple) DAG with unique labeling.
\begin{CLAIM}\label{cl:uniq}
If a collection of node-specific grammars $G_{\Sigma,V}$ is such that
\begin{enumerate}\MyLPar
\item\label{it:onerule}
for every non-terminal
$N_v$ there is exactly one rule $N_v\prd_{G_{\Sigma,V}}t$, and
\item\label{it:welf}
the resulting graph
$\Gamma_{G_{\Sigma,V}}=\<V,\ddepo{G_{\Sigma,V}}\>$ is well founded,
\end{enumerate}
then $G_{\Sigma,V}$ determines a unique $\Sigma$ data dependency.
\end{CLAIM}
\begin{PROOF}
Point 2.\ implies that $\Gamma_{G_{\Sigma,V}}$ is acyclic.  For every $v\in V$, the
language  ${\cal L}(G_{\Sigma,v})$ is a one-element set
since by Point 1.\ we have exactly one $N_v\prd_{G_{\Sigma,V}} t$ for each $v$,
and, since $\Gamma_{G_{\Sigma,V}}$ is well founded, each path from $t$ will
uniquely lead to a terminal node.  Thus the unique data dependency is
$D_{G_{\Sigma,V}}=\<V,\ddepo{G_{\Sigma,V}},lab_{G_{\Sigma,V}}\>$ where
$lab_{G_{\Sigma,V}}(v)=q$, where $q$ is the unique string in 
${\cal L}(G_{\Sigma,v})$.
\end{PROOF}
\noindent
In each particular case of a given recurrence \re{def:recur} and an algebra
$A$, the well-definedness of the resulting dependency has to be checked, for
instance, by verifying the conditions of the above proposition. \vspace*{-.5ex}

Another obvious fact following from the assumptions of 
Proposition~\ref{cl:uniq}
is that if $\depo{G_{\Sigma,V}}uv$
then $lab_{G_{\Sigma,V}}(u)$ is a subterm of $lab_{G_{\Sigma,V}}(v)$.

\begin{SREMARK}{Remark.}
The described construction obviously subsumes primitive recursion for 
which we have
$\Sigma=\<Nat,\Funcs\>$,
the only model $\Nata$ being that of the naturals such that $\Funcs$ 
are the basic
  primitive recursive operations on naturals.
Given recursive equation schema for an $R$ with $n+1$ variables
\begin{eqnarray*}
         R(0,y_1,\ldots,y_n) &=& g(y_1,\ldots,y_n) \\
         R(\suc(i),y_1,\ldots,y_n) &=& h(i,y_1,\ldots,y_n,R(i,y_1,\ldots,y_n)),
\end{eqnarray*}
where $g$ and $h$ are primitive recursive operations, we get
the set of nodes $V=\Nat^{n+1}$, nodes of the form 
$y=\<y_0,y_1,\ldots,y_n\>$, and the
  productions for any fixed values of $y_1...y_n:$
\begin{eqnarray*}
    N_{\<0,y_1,\ldots,y_n\>} &\prd & g(y_1,\ldots,y_n) \\
    N_{\<i+1,y_1,\ldots,y_n\>} &\prd &
h(i,y_1,\ldots,y_n,N_{\<i,y_1,\ldots,y_n\>}) \vspace*{-8ex}
\end{eqnarray*}
\end{SREMARK}
\noindent
	This approach to the recursive programming of functions is a
	derivative of the semantic (i.e. $\mu$-recursive) approach to
	computability. 
Most functional languages, on the other hand, are compiled using (parallel)
	graph reduction techniques \cite{PJ87} or, more generally, a 
	syntax-oriented computability model based on term substitution,
	e.g. $\lambda$-calculus. Unlike these substitution
	based approaches, the construction described here leads often
	immediately to an efficient dependency structure which is amenable to
	a direct translation to efficient code on (parallel) machine
	hardware. The two following subsections illustrate this claim on two examples. 

\subsection{A simple example -- the Fibonacci function}
Take again the Fibonacci function from
example~\ref{ex:Fib}, which we want to compute in a standard algebra $\Nata$ of
natural numbers. That is, we have the signature
$\Sigma=\<\{Nat\},\{s:Nat\into Nat, \oplus:Nat\times Nat\rightarrow 
Nat,z:Nat\}\>$, and $\Nata$
is given by $Nat^\Nata=\Nat$, the set of natural numbers,
$s^\Nata=$ the successor function, $\oplus^\Nata=$ addition, and $z^\Nata=0$.

The recursive specification of the Fibonacci function  $F:Nat\into
Nat$ is:
\begin{eqnarray}
F(z) & = & s(z) \nonumber \\
F(s(z)) & = & s(z) \label{eq:fib} \\
F(i\oplus ss(z)) & = & F(i) \oplus F(i\oplus s(z)) \nonumber
\end{eqnarray}
We choose $\Nat$ as the set of nodes $V$ (corresponding to the arity of $F$),
i.e., $V=\Nat$.
With $N_{k}$, for $k\in \Nat$, as non-terminal symbols, we now obtain the
production rules for the set of node-specific grammars (using that
$z^{\Nata}=0$, $s(z)^{\Nata}=1$ and $ss(z)^{\Nata}=2$ for the indices):
%
\begin{eqnarray}
N_{0} &\prd  & s(z) \nonumber \\
N_{1} &\prd  & s(z) \label{eq:fibG} \\
N_{i+2} &\prd  &( N_{i} \oplus N_{i+1}) \nonumber
\end{eqnarray}
\noindent
Each non-terminal has only one production and so
each node will be labeled by a unique ground term generated by the
corresponding grammar.
The resulting graph  and labeling (with parentheses added for readability)
are shown in Figure~\ref{fi:recsp}.
\begin{figure}[hbt]
\def\objectstyle{\scriptstyle}
  \spcol{-.5}
  \sprow{-1.5}
\[ \diagramcompileto{FibCR}
    N_4 &               & & & &       (1\oplus 1)\oplus((1\oplus 
1)\oplus 1) & \\
    & N_3\ulto \udotted & & & &       & (1\oplus 1)\oplus 1 \ulto \udotted \\
    N_2 \uuto \urto &   & & & &       1\oplus 1 \uuto \urto &  \\
    & N_1 \ulto \uuto   & & & &       & 1 \ulto \uuto \\
    N_0 \uuto &         & & & &       1 \uuto \\
\save \Drop{\text{{\normalsize $\Gamma_{\mathit{Fib}}$ with non-terminals}}}
    \restore&     & & & &
\save \Drop{\text{{\normalsize $\Gamma_{\mathit{Fib}}$ with labels}}} \restore
  \enddiagram  \]
\caption[Fibonacci graph]{The graph derived from the Fibonacci recurrence
(\ref{eq:fib}): on the left with the non-terminals, on the right with
the language derived for each node (abbreviating $s(z)$ by $1$).}
\label{fi:recsp}
\end{figure}

\noindent
Notice that although labeling uses fully unfolded terms, the graph itself is
not the usual tree of recursive calls as, e.g. $\dd R$ in
Figure~\ref{fi:rec}, but a much more efficient graph $\Gamma_{\mathit{Fib}}$.

\subsection{An example of parallelization - Fast Fourier Transform}\label{se:parallel}
To illustrate the application to actual parallelization of
programs, we review the technique of the fast Fourier Transform, FFT, due to
\cite{CT65}. We first present the background problem and its solution given
by a recurrence. The dependency for this recurrence obtained by our technique  turns out to have
the well known shape of the {\em butterfly graph}. We then show a standard
distribution of the butterfly on a hypercube architecture.

\subsubsection{The numerical problem}
In many areas, computing of  $n$-order polynomials $p(x) =
\sum^{n}_{k=0} p_k x^k$, $0\le n$, for certain argument values
$x=x_1,\ldots,x_k$ is important. These areas include, for instance,
signal processing,
where evaluation of the polynomials for the {\em complex roots of unity\/}
plays a central role.

An $n$'th complex root of unity is a complex number $\omega$ such that
$\omega^n=1$.  For every $n\ge 1$ there are $n$ distinct complex roots,
spread at $n$ regular intervals along the unit circle in complex
space, numbered counterclockwise starting at the number 1 which is
root number 0. The $j$'th such root is
denoted $\omega^j_n$, and satisfies
\[    \omega^0_n = 1 \hspace*{3em}
       \omega^{j+1}_n = \omega^{j}_n \omega^1_n \hspace*{3em} 
       \omega^j_n     = \omega^{j \bmod n}_n
\]
Useful facts about these roots are that $(\omega^j_{2n})^2=
\omega^{2j}_{2n}= \omega^j_n$ and $\omega^{j+n}_{2n}= -\omega^j_{2n}$.
The roots may be computed by $\omega^{j}_{n}= e^{(2\pi i/n)*j} = cos(2\pi j/n)
  + sin(2\pi j/n) i$, where $i=\omega^1_4$ is the imaginary unit, i.e.,
  $i^2 = -1$.

Cooley and Tukey \cite{CT65} discovered a very efficient algorithm, the FFT,
for computing an $n$-order polynomial for {\em all} $n$'th complex roots of unity.
Let us consider this method for the case where $n=2^M$, i.e., the
one-dimensional binary version. Then the polynomial $p(x)$ can be split into
``even'' $p_{e}$ and ``odd'' $p_{o}$ parts such that 
$p(x)=p_e(x^2)+x*p_o(x^2)$, where
$p_e(x)=\sum^{2^{n-1}}_{k=0} p_{2k} * x^k$ and $p_o(x)=\sum^{2^{n-1}}_{k=0}
p_{2k+1} * x^k$.  Taking into account the properties of the roots of
unity, we get
\begin{eqnarray*}
         p(\omega^j_{2^m})
                 &=& p_e(\omega^{j}_{2^{m-1}})
                 + \omega^{j}_{2^{m}}  p_o(\omega^{j}_{2^{m-1}})
\\      p_e(\omega^{j}_{2^{m-1}})
                 &=& p_{e,e}(\omega^{j}_{2^{m-2}})
                 + \omega^{j}_{2^{m-1}}  p_{e,o}(\omega^{j}_{2^{m-2}})
\\      p_o(\omega^{j}_{2^{m-1}})
                 &=& p_{o,e}(\omega^{j}_{2^{m-2}})
                 + \omega^{j}_{2^{m-1}}  p_{o,o}(\omega^{j}_{2^{m-2}})
\end{eqnarray*}
This can be formulated as a recurrence equation
\begin{eqnarray}
%         p(\omega^j_{2^M}) &=& F(j,M) \nonumber \\      
     F(j,0) &=& p_j, \label{eq:roots} \\
F(j,m+1) &=& F(s_0(j,m+1),m) + \omega^{j}_{2^{m+1}}F(s_1(j,m+1),m)   \nonumber
\end{eqnarray}
where $s_0(j,m)= j - (j \bmod 2^m) + (j \bmod 2^{m-1})$ sets the $m$'th bit of
  $j$ to $0$ and $s_1(j,m)= j - (j \bmod 2^m) + 2^m + (j \bmod 2^{m-1})$ sets
the $m$'th bit of $j$ to $1$. Bits are the 0's and 1's in the binary
representation of a number, and are counted with the least significant bit
as number 1. The value of the polynomial at the $j$-th complex root of unity
equals then $p(\omega^j_{2^{M}})=F(j,M)$.

\subsubsection{Constructing a dependency}
Applying our strategy we start with the
equations \re{eq:roots} defining the recurrence.
We work with the algebra of natural and complex numbers, and as the set of
nodes $V$ we take the cartesian product
$\{\<j,m\>\in\Nat\times\Nat\mid 0\le j < 2^M, 0\le m\le M\}$ -- since $F$
has two natural number arguments which are within these limits.
The grammar resulting from the equations \re{eq:roots} is
then (using $\oplus$ as the complex addition operation and $\otimes$
as the complex multiplication):
\begin{eqnarray}
\ \ \ \ \ \ \ \         N_{\<j,0\>} &\prd & p_j \nonumber\\
N_{\<j,m+1\>} &\prd & N_{\<s_{0}(j,m+1),m\>} \oplus
(\omega^j_{2^{m+1}} \otimes N_{\<s_{1}(j,m+1),m\>}) \label{eq:fft}
\end{eqnarray}
with the coefficients $p_j$ of the polynomial treated as input since the roots
of unity $\omega^j_{2^{m+1}}$ are fixed once $M$ is known.
The resulting shape of the data dependency for $M=3$ is given
in Figure~\ref{fi:bbf}.
Again, observe how this method leads directly to an ``efficient'' graph
instead of generating
the full tree of recursive calls.

\begin{figure}[hbt]
\def\objectstyle{\scriptstyle}
\spcol{-.5}
\sprow{-.7}
%\everyentry={\bullet}
\hspace*{6em}\diagramcompileto{Butterfly}
3& & \bullet & \bullet & \bullet & \bullet & \bullet
   & \bullet & \bullet & \bullet \\
2 \uto^{\textstyle m} & &
\bullet \uline \xline[urrrr]
& \circ \uline \xline[urrrr]
& \bullet \uline \xline[urrrr]
& \bullet \uline \xline[urrrr]
& \bullet \uline \xline[ullll]
& \bullet \uline \xline[ullll]
& \bullet \uline \xline[ullll]
& \bullet \uline \xline[ullll] \\
1 \uto & &
\bullet \uline \urrline
& \bullet \urrline \uline
& \bullet \ullline \uline
& \bullet \ullline \uline
& \bullet \uline \urrline
& \bullet \urrline \uline
& \bullet \ullline \uline
& \bullet \ullline \uline \\
0 \uto  & &
\bullet \uline \urline
& \bullet \uline \ulline
& \bullet \uline \urline
& \bullet \uline \ulline
& \bullet \uline \urline
& \bullet \uline \ulline
& \bullet \uline \urline
& \bullet \uline \ulline \\
& & 0 \rline & 1 \rline & 2 \rrto^{\textstyle j} & & &
\enddiagram
\caption{The butterfly pattern of the recurrence $F$.}\label{fi:bbf}
\end{figure}
\noindent
Columns correspond to the $j$ and rows to the $m$ arguments with the leftmost
column and bottom row numbered 0.  The bottom row will be labeled with
$p_0,p_1,p_2,...$ according to the first kind of productions from
\re{eq:fft}. The labels in the upper rows will be generated using the 
second kind of
productions from \re{eq:fft}. We do not write it explicitly (lengthy 
labeling can be
easily recovered from the grammar) but only observe that the label at node
$\<j,m\>$ will represent the value of $F(j,m)$. For instance, the 
node at row 2, column
1 (marked with the circle instead of bullet on the drawing) will represent
the value of $F(1,2)$.

\subsubsection{Distribution on a hypercube}
The shape from Figure~\ref{fi:bbf} is known as a {\em butterfly graph}.
Increasing $M$ by one, i.e., taking $n=2^{M+1}$, amounts to
putting two butterflies of height $M$ side by side, adding a new set of
nodes on top (row $M+1$), and connecting
uppermost nodes (in the $M$'th row) of the one butterfly to its own
and to the respective
nodes of the other
in the row $M+1$.

Now, Figure \ref{hc.hw} shows the topology of hypercubes of dimension
$M=$ 3 and 4. Each of
the  $2^M$ processors has $M$ bidirectional communication channels. A
hypercube of dimension $M+1$ is obtained from two hypercubes of dimension $M$
by connecting each node of one hypercube to corresponding node of the other.

\begin{figure}[hbt]
\sprow{-2}
\spcol{-1.2}
\hspace*{10em}\diagramcompileto{Hyper}
& & &  & &     & \bullet \xline[ddd] \xline[dl] \xline[drr] & & \\
% 2
& \bullet \xline[ddd] \xline[dl] \xline[drr] \xdotted[urrrrr] & &   & &
\bullet \xline[ddd] \xline[drr] & & & \bullet \xline[ddd] \xline[dl] \\
% 3
\bullet \xline[ddd] \xline[drr] \xdotted[urrrrr] & & & \bullet \xline[ddd]
         \xline[dl] \xdotted[urrrrr] & &
& & \bullet \xline[ddd] & \\
% 4
& & \bullet \xline[ddd] \xdotted[urrrrr] & & &
& \bullet \xline[dl] \xline[drr] & & \\
% 5
& \bullet \xline[dl] \xline[drr] \xdotted[urrrrr] & &  & &
\bullet \xline[drr] & & & \bullet \xline[dl] \\
% 6
\bullet \xline[drr] \xdotted[urrrrr] & & & \bullet \xline[dl]
      \xdotted[urrrrr] & &
& & \bullet & \\
& & \bullet \xdotted[urrrrr] &
\enddiagram
\caption{Connections for hypercubes of dimension 3 (solid), and 4 (solid
and dotted). }\label{hc.hw}
\end{figure}

The hypercube parallel program will then, on hypercube node $j$
	at time-step $m+1$, compute $F(j,m+1)$ using input data supplied
	by nodes $s_0(j,m+1)$ and $s_1(j,m+1)$ at time-step $m$. One of the
	data supplying nodes will be the same as the node $j$, the other will be 
	its neighbor along dimension $m$.�

The  Figure
\ref{hc.sta} shows a distribution of the butterfly dependency of height 3
from Figure
\ref{fi:bbf} on the $st$-graph of the hypercube of dimension
3 from Figure \ref{hc.hw}. (Thicker arrows indicate the embedded edges of
the butterfly; thinner lines the remaining edges of the hypercube.)
Using the regularity in building a butterfly and a
hypercube of higher dimensions, an analogous distribution is obtained for a
butterfly of arbitrary height $M$ into a hypercube of dimension $M$.

\input{rowsAll}

\section{More about development process}\label{se:difSign}
This section adds a few results concerning the development process using
computation algebras. First, in Section~\ref{sub:difSign}, we show the counterpart of
the standard result allowing one to view computation algebras over
different signatures as objects of one category. Thus we obtain the 
means of refining
the picture of the development process from Figure \ref{fi:method} by also
admitting transitions between computation algebras over different
signatures.
Section \ref{sub:specific} identifies then two subcategories of $\CAS$
with more specific morphisms which make the coupling of the algebra- and the
communication-part tighter than in the general case of computation
homomorphisms from Definition~\ref{de:Chom}. The latter of these
subcategories was used in the design of the actual specification formalism
{\sc Sapphire}.
Section \ref{sub:general} states two further properties of the category of
computation algebras which are of relevance for the methodology of
development.

\subsection{Computation Algebras over Different
Signatures}\label{sub:difSign}
So far we have only considered computation algebras over one given signature.  Thus,
the development process illustrated in Figure \ref{fi:method} is not quite
realistic since it is rather certain that such a process will involve
change of signatures.  We will now extend the standard way of
flattening the categories of algebras indexed by signatures to the
computation algebras and their distributions.  

Let $\Sig$ be the category of
signatures with injective signature morphisms.\footnote{We restrict
ourselves to the
injective
signature morphisms for one main reason: defining a reduct functor over
data dependencies
for non-injective morphisms implies several choices which seem slightly
arbitrary. We
therefore prefer to postpone such a decision until we have more experience and
reasons to choose one alternative rather than another.}
%
We write a dependency $\dd D$ as a triple $\<I^D,\ddepo D,lab_{\dd D}\>$.
\begin{DEFINITION}{[$\sigma$-reduct]}
Let $\sigma:\Sigma'\into\Sigma\in\Mor{\Sig}$ and $\dd D\in\DD(\Sigma)$. The
{\em $\sigma$-reduct} of
$\dd D$, $\reduct {\dd D}\sigma\in\DD(\Sigma')$, is defined in two steps
 (we assume the ordinary extension of $\sigma$ to $\sigma:\GTerms\into \Ter{\Sigma'}$): \\
\noindent
1. let $\dd X$ be $\dd D$ with restricted re-labeling $:I^X=I^D$, $\ddepo X\ =\
\ddepo D$ and
  $lab_X(i)=\sigma^-(lab_{\dd D}(i))$, where $\sigma^-:\GTerms\into
  \Ter{\Sigma'}$ is given by $\sigma^-(t) = \left\{\begin{array}{ll} t' &
  {\rm such\ that\ }\sigma(t')=t\ {\rm if\ it\ exists}\\
   \bot & {\rm otherwise} \end{array}\right.$ \\
\noindent
2. $\reduct {\dd D}\sigma$ is then given by
\begin{itemize}\MyLPar
	\item $I^{\reduct D\sigma} = \{i\in I^D: \sigma^-(lab_X(i))\not = \bot\}$,
	\item $lab_{\reduct D\sigma}(i)=lab_X(i)$, and
	\item $\depo {\reduct D\sigma}ij \iff \left \{
	\begin{array}{l} \depo Dij \ \ \ or \\
	\exists\ {\rm path\ }\depo Xi{k_1} \cdots \depo X{k_z}j\ {\rm 
and\ } 1\leq n\leq z \impl
lab_X(k_n)=\bot
	\end{array}\right .$
\end{itemize} \vspace*{-5ex}
\end{DEFINITION}

\newpage
\noindent In the first step, we remove all the labels from $D$ which are
not in the image of $\sigma$, and re-label the remaining ones to their
$\sigma$ pre-image.  In the second we convert paths with unlabeled ($\bot$-labeled)
intermediary nodes into edges between the labeled end-points.

For $A\in\AlgS$ and $\sigma:\Sigma'\into\Sigma\in\Mor{\Sig}$, we let
$\reduct A\sigma$
denote the usual reduct. We then have the extension of the standard result:
%
\begin{CLAIM}
$\sigma:\Sigma'\into\Sigma\in\Mor{\Sig}$ induces the functor $\reduct{\_}\sigma
:\CA(\Sigma)\into\CA(\Sigma').$
\end{CLAIM}
\begin{PROOF}
For an $A\in\CA(\Sigma)$,
define $\reduct A\sigma=\<\reduct{\ha A}\sigma,\reduct{\hc A}\sigma\>$.
For a morphism $h:A\into B\in\Mor{\CA(\Sigma)}$, $\reduct h\sigma
=\<\reduct{\ha h}\sigma,\reduct{\hc h}\sigma\>$
where $\reduct{\ha h}\sigma$ is the image of $\ha h$ under the usual reduct
functor
for algebras, while $\reduct{\hc h}\sigma$ is the restriction of $\hc h$ to
$\reduct{\hc B}\sigma$ under changed labeling:
each node $i$ of $\reduct{\hc B}\sigma$ is sent to the node $\reduct{\hc
h(i)}\sigma=\hc h(i)$,
and edge $e$ of $\reduct{\hc B}\sigma$ onto the edge (path/set of paths)
$\reduct{\hc h(e)}\sigma$ in
$\reduct{\hc A}\sigma$.
This is a well defined $\DD(\Sigma')$-morphism: commutativity
$\Comp{\hc h}{\Comp{\ev_A}{\ha h}}=\ev_B$ implies the corresponding
commutativity in the
reduct, since the latter operations are restrictions of their pre-images.
\end{PROOF}
\noindent Thus $\reduct{\_}{\_}:\op{\Sig}\into{\Cat {CAT}}$ sending
$\Sigma$ to $\CA(\Sigma)$ and $\sigma$ to $\reduct{\_}\sigma$ is an indexed
category.  By the standard application of the Grothendieck construction we
may define the flattened category $\CA$ with 1) objects $\<\Sigma,A\>$
where $A\in\Obj{\CA(\Sigma)}$, 2) morphisms of the form
$\<\sigma,f\>:\<\Sigma',A\>\into\<\Sigma,B\>$, where $\sigma\in\Mor{\Sig}$
and $f:A\into\reduct B\sigma\in\Mor{\CA(\Sigma')}$, and 3) composition
$\Comp{\<\sigma,f\>}{\<\rho,g\>}=\<\Comp\sigma\rho, \Comp f{\reduct
g\rho}\>$.

Quite an analogous construction yields an indexed category
$\rreduct{\_}{\_}:\op\Sig\into \Cat{CAT}$ which sends $\Sigma$ to
$\IM(\Sigma)$.  For $\sigma:\Sigma'\into\Sigma$, the induced functor
$\rreduct{\_}\sigma:\IM(\Sigma')\into\IM(\Sigma)$, sends a distribution
$\imca AfW\in\IM(\Sigma)$ onto $\imca {\reduct A\sigma}{(\rreduct
f\sigma)}W\in\IM(\Sigma')$, where $\rreduct f\sigma$ is the restriction of
$f$ to $\reduct{\hc A}\sigma$.  The result of the flattening is the category
$\IM$.  We do not dwell on this general category, because we consider 
the categories
$\IM(\Sigma)$ for particular $\Sigma$ more central -- porting {\em 
distributions\/} does not
typically involve changes of signature.



\subsection{More specific dependency morphisms}\label{sub:specific}
The class $\CAS$ provides the general framework for possible specializations.
Arbitrary dependency morphisms in $\DDS$ make the
connection between
the algebra- and the communication-part of the objects in $\CAS$ rather loose
and indefinite. We mention two possible restrictions.

\subsubsection{Computation algebras with compatible
$\Dclass$-morphisms}\label{sub:compatible}
Let $\Aclass$ be a fixed class of
$\Sigma$-algebras.  It is natural to think of it as a model class of some
(equational) specification.  One may postulate that such a choice of
$\Aclass$ should restrict the relevant dependency morphisms.  In
particular, if $\Aclass\not\models s=t$, the $\Dclass$-morphisms should not
be allowed to map nodes labeled by $s$ to ones labeled by $t$ (nor vice
versa), since such mappings are not compatible with the algebras in
$\Aclass$.  Thus, compatibility of a $\DDS$ morphism $h: \dd C\into \dd
D$ wrt.  an algebra $A$ means that $h$ does not effect a re-labeling
which is inconsistent with valuation in $A$.  More precisely:
\begin{DEFINITION}{}\label{de:compat}
Let $h:\dd C\into \dd D\in\Mor{\DDS}$ and $A\in\Aclass\subseteq\AlgS$.
\begin{enumerate}\MyLPar
\item $h$ is {\em $A$-compatible} iff $\forall i\in I^{C} :
(lab_{\dd D}(h(i)))^A =
(lab_{\dd C}(i))^A$~;
\item $h$ is $\Aclass${\em -compatible\/} iff it is $A$-compatible for all
$A\in\Obj{\Aclass}$~;
\item $\DD(\Aclass)$ is a wide subcategory of $\DDS$ with
$\Aclass$-compatible morphisms.
\end{enumerate}\vspace*{-4.5ex}
\end{DEFINITION}
\noindent
Alternatively, $h:\dd C\into \dd D$ is $\Aclass$-compatible iff for each
$A\in\Obj{\Aclass}$,
  $\<id_{A}, h\>$ is a computation homomorphism $\<A, \dd D\>\into 
\<A, \dd C\>$.
We have the obvious
\begin{CLAIM}
Suppose that $\Aclass$ has an initial object $T_\Aclass$, and let
  $h: \dd C\into \dd D \in \Mor{\DDS}$.\\ $h$ is $\Aclass$-compatible
  $\iff$ $h$ is $T_\Aclass$-compatible.
\end{CLAIM}
\begin{PROOF}
($\impl$) is obvious. For ($\Leftarrow$) assume $h$ is
$T_\Aclass$-compatible.
This means that the lower
triangle of the following diagram commutes,
$\Comp h{ev_{T_{\Aclass}}^{\dd D}}=ev_{T_{\Aclass}}^{\dd C}$, where
  $\ev_X^{\dd Y}$ assigns the values
from an algebra $X$ to
the nodes of a dependency $\dd Y$ according to their labels (i.e. $ev_X^{\dd
Y}(i)=(lab_{\dd Y}(i))^X$).

\hspace*{12em}
\spcol{.6}
\diagramcompileto{TK}
& A \\
& T_\Aclass \uto_<<<<a \\
\dd C \uurto^{\ev_A^{\dd C}} \urto_{ev_{T_{\Aclass}}^{\dd C}} \rrto_{h} & &
\dd D \uulto_{\ev_A^{\dd D}}
\ulto^{ev_{T_{\Aclass}}^{\dd D}}
\enddiagram

\noindent
Since $T_\Aclass$ is initial, for each $A\in\Aclass$ and dependency $\dd Y$,
$\ev_A^{\dd Y}=\Comp{\ev_{T_{\Aclass}}^{\dd Y}}a$
where $a$ is the
unique homomorphism $T_\Aclass\into A$. But then $\Comp
{h}{\ev_{T_{\Aclass}}^{\dd D}}=\ev_{T_{\Aclass}}^{\dd C} \impl
\Comp{\Comp{h}{\ev_{T_{\Aclass}}^{\dd D}}}a=\Comp{\ev_{T_{\Aclass}}^{\dd
C}}a \impl \Comp {h}{\ev_A^{\dd D}}=\ev_A^{\dd C}$.
\end{PROOF}
\noindent
As a special case, the morphisms compatible with $\Aclass=\AlgS$ are the ones
which do not change the labeling at all, since they have to be compatible
with the
initial word algebra $T_\Sigma$.

For computation algebras with compatible dependency morphism, we also have
the following fact about the restriction of the functor $\hc{(\_)}$ 
from (\ref{fu:dep}):
%
\begin{CLAIM}\label{le:Dep}
$\hc{(\_)}:\op{\CA(\Aclass,\DD(\Aclass))}\into {\DD(\Aclass)}$ is a 
split fibration.
\end{CLAIM}
%
\begin{PROOF}
We use the notation from the diagram to the right
which illustrates the fibration situation in the present context (the dashed
arrows indicate the directions in $\_^{op}$ categories):
\pp{1ex}
\hspace*{1em} \spreaddiagramcolumns{1pc}
\diagramcompileto{FibC}
& B\framed<5pt>   \ddashed<.5ex>|>\tip  \drdashed<0.5ex>|>\tip
   &  & \ha{B}  \drrdashed<.5ex>|>\tip  \drdashed<0.5ex>|>\tip  \\
\op{\CA(\Aclass,\DD(\Aclass))} \dto^{\hc{(\_)}} & C\framed<5pt>
\rdashed<-.5ex>|>\tip_{cl(d,D)}  \uto<.5ex>^f
    &  D\framed<5pt> \lto<-.5ex> \ulto<-1.5ex>_g
    &  & \ha{D} \rdashed<0.5ex>|>\tip  \ulto<.5ex>^<<<<{\ha{f}}
    & \ha{D} \lto<0.5ex>^{id_{\ha{D}}} \ullto<-1.5ex>_{\ha{g}} \\
\DD(\Aclass) & \dd{C}\rto_{d} & \dd{D}
    &  & \dd{C} \save\go[u]\merge\framed<5pt>\restore
    \uuldotted<.5ex>|<<<<<<<<{\ev_{B}}|>\tip \udotted|{\ev_{C}}|>\tip
\rto_{d} & \dd{D} \save\go[u]\merge\framed<5pt>\restore
	\udotted|{\ev_{D}}|>\tip
\enddiagram
\pp{1ex}
For a $\DD(\Aclass)$-morphism $d:\dd{C}\into\dd{D}$ and $D=\<\ha 
D,\dd D,ev_D\>$,
the cleavage is defined as
\begin{equation}\label{eq:ccl}
cl(d,D) \Def \<id_{\ha{D}}, d\>:C\dinto D,\ {\rm
where\ }C=\<\ha D,\dd C,\ev_C\>
\end{equation}
with $\ev_C$ given by the last point of Definition \ref{de:CA}. Since
$d$ is $\Aclass$-compatible,
$\Comp{\Comp{d}{\ev_D}}{id_{\ha
D}}=ev_C$, and so the cleavage is a morphism in
$\op{\CA(\Aclass,\DD(\Aclass))}$.

If $g:D\into B$ is over $d$ ($\hc{\op g}=d$), the unique $f:C\into B$ over
$id_{\dd{C}}$ making $\Comp{cl(d,D)}{f}=g$ is $\<\ha g, id_{\dd{C}}\>$ (i.e.,
  $\Comp{\<id_{\ha{D}}, d\>}{\<\ha g, id_{\dd{C}}\>}=\<\ha g, d\>$).

Obviously, the cleavages compose:
for $c:\dd B\into \dd C$ and $d:\dd C\into\dd D$, we have $cl(d,D)=\<id_{\ha
D},d\>$ and $cl(c,\<\ha D,\dd C\>)=\<id_{\ha D},c\>$. So $\Comp{cl(c,\<\ha
D,\dd C\>)}{cl(d,D)}= \Comp{\<id_{\ha D},c\>}{\<id_{\ha D},d\>}={\<id_{\ha
D},\Comp cd\>}=cl(\Comp cd,D)$. Hence $\hc{(\_)}$ is a fibration.

It is split:
  $\reind{(id_{\dd D})}(\<\ha D,\dd D\>)=\<\ha D,\dd D\>$, so the 
natural transformation
$\nattr{id_{\hc{(\dd D)}^{-1}} } {\reind{(id_{\dd D})}}$ is identity
($\hc{(\dd D)}^{-1}$ denotes the fiber over $\dd D$).
Also, since cleavages compose :
$\Comp{\reind{(d)}}{\reind{(c)}} = \reind{(\Comp{c}{d})}$.
\end{PROOF}
\noindent
For the compatible dependency morphisms, we also have the additional fact:

\begin{CLAIM}\label{le:coDep}
  $\hc{(\_)}:\op{\CA(\Aclass,\DD(\Aclass))}\into {\DD(\Aclass)}$ is a 
cofibration.
\end{CLAIM}
\begin{PROOF}
The proof and definition of cocleavage are essentially the same as in the
Proposition \ref{le:Dep}.
We have the following picture:

\hspace*{1em}
\spcol{1}
\diagramcompileto{CoFibC}
  & B\framed<5pt>   \dto_f \drto^g   &
  & & \ha{B} \drrto^{\ha{g}} \drto|{\ \ \ha{f}=\ha g} \\ 
\CA(\Aclass,\DD(\Aclass)) \dto^{\op{\hc{(\_)}}} & C\framed<5pt>
    \rto_{ccl(\op d,D)}  & D\framed<5pt>
  & &  & \ha{D} \rto_{id_{\ha{D}}} & \ha{D}  \\
\op{\DD(\Aclass)} & \dd{C}  \rdashed^{\op d}|>\tip & \dd{D}
  & & & \dd{C} \save\go[u]\merge\framed<5pt>\restore
		\uuldotted|{\ev_{B}}|>\tip \udotted|{\ev_C}|>\tip
  & \dd{D} \save\go[u]\merge\framed<5pt>\restore \udotted|{\ev_{D}}|>\tip
\lto_{d}
\enddiagram
\pp{1ex}
For a $\op{\DD(\Aclass)}$-morphism $\op d:\dd{C}\dinto\dd{D}$ and
$D=\<\ha{D},\dd{D},\ev_D\>$, the
cocleavage is defined as
\begin{equation}\label{eq:cl}
ccl(\op d,D) \Def \<id_{\ha D},d\>:C\into D\ {\rm where\ }
	C = \<\ha{D},\dd{C},\ev_C\>,
\end{equation}
with $\ev_C$ induced by $lab_{\dd C}$ according to Definition
\ref{de:CA}.3.
Trivial repetition of the arguments from the proof of Proposition \ref{le:Dep}
yields the conclusion.
\end{PROOF}
%
\noindent
This, together with the Proposition \ref{le:Dep}, means that $\hc{(\_)}$ is a
bifibration and gives
\begin{COROLLARY}
Reindexing functors induced by $(\hc{(\_)},cl)$ have left adjoints.
\end{COROLLARY}
\noindent
Finally, we may
refer back to Section~\ref{sub:difSign} and observe that,
for an injective signature morphism $\sigma:\Sigma'\into\Sigma$,
  if $d:\dd C\into \dd D\in\Mor{\DDS}$ is $A$-compatible for
an $A\in\Obj{\AlgS}$, then $\reduct{d}\sigma$ is $\reduct 
{A}\sigma$-compatible.
Hence, the image under $\reduct{\_}\sigma$ of a $\CA(\Aclass,\DD(\Aclass))$
is a category $\CA(\reduct{\Aclass}\sigma,\Dclass)$ where
$\Obj{\reduct\Aclass\sigma} =
\{\reduct A\sigma:A\in\Obj{\Aclass}\}$ and morphisms in $\Dclass$ are
$\reduct{\Aclass}\sigma$-compatible. (In general, $\Dclass$ may be only a
subcategory of $\DD(\reduct\Aclass\sigma)$.)

\subsubsection{Graph-morphisms}\label{sub:Saphire}
We believe that refinement of data dependencies -- at least as expressed by
the compatible dependency morphisms in Definition~\ref{de:compat} -- may be
a practical notion for the development of implementations and, perhaps, more
abstract algorithms.  Nevertheless, one could argue that it is not needed in
such generality since data dependencies are quite concrete objects
introduced towards the very end of the development process.  In particular,
passing from one computation algebra to another should not imply the
introduction of new dependencies, only (and at most) purely syntactic
memoisation.  It is only the distribution morphisms which adjust the
dependency graph to match it to the communication structure of a given
architecture.

Accepting this view, one could work with a class of
computation algebras $\AD$ where $\Aclass\subseteq\AlgS$ and the
dependencies in $\Dclass$ are restricted so that the target of the
forgetful functor $Gr$ projecting dependencies onto their graph-part
(cf.~\re{fu:gr} after Definition~\ref{de:DDS}),
is no longer the category $\Gr$ but its wide subcategory $\Gr^\ecomp$:
morphisms in $\Gr^\ecomp$ may map edges on paths but not
on ``parallel compositions'' of paths. ($\Gr^\ecomp$ can be defined as $\Gr$
in Definition~\ref{de:clgr} but with the adjunction
${F^\ecomp}\adj{U^\ecomp}$ from Proposition~\ref{cl:adj1} instead of 
${\pat F}\adj{\pat U}$.)
Thus objects in $\Dclass$ are simple DAGs and
morphisms  are the $\Aclass$-compatible dependency morphisms $h$ such that
$Gr(h)$ is a $\Gr^\ecomp$-morphism.

In fact, restricting our attention to merely syntactic memoisation, we restrict
further the morphisms in $\Dclass$ to those which do not perform any
relabeling but can, at most, identify some nodes from the source graph with
identical labels. Thus, the morphisms are in fact $\AlgS$-compatible.
Let us denote this class of computation algebras by
$(\Aclass,\DSaph)$.
Since $\DSaph$-morphisms do not perform any relabeling
whatsoever, so in this class, any pair
$h \in \Mor{\DSaph}$ and $g \in \Mor{\Aclass}$ will give rise to a
$(\Aclass,\DSaph)$-computation morphism.

Since $\Mor{\DSaph}\subset\Mor{\DD(\Aclass)}$, we can trivially repeat the
constructions from Propositions \ref{le:Dep} and \ref{le:coDep}, to see that
  $\hc{(\_)}:\op{(\Aclass,\DSaph)}\into {\DSaph}$ is a bifibration.

The class $(\Aclass,\DSaph)$, although theoretically perhaps not the most
fascinating one, deserves to be mentioned because it was used in practical
applications and underlies the design of the specification formalism {\sc
Sapphire} \cite{CH95,HS98} (see section~\ref{se:cr}).

\subsection{Two more facts about 
$\CA(\Aclass,\Dclass)$.}\label{sub:general} The following two facts show
further properties of $\CA(\Aclass,\Dclass)$ which can be relevant in the
development process. The one in Section \ref{sub:dc} allows us to treat refinement
along the algebra- and communication-part relatively independently from each
other. This indicates the possibility that once developed chains of refinements of, say,
dependency structures, could be reused in new contexts by coupling them with
new actual algebras. The second one, Section \ref{sub:init}, gives (some) conditions
for the existence of initial computation algebras relatively to the
exisitence of initial objects in $\Aclass$ and $\Dclass$.

\subsubsection{$\CA(\Aclass,\Dclass)$ as a double category.}\label{sub:dc}
A morphism between computation algebras consists of a pair of one
standard
algebra-morphism and one dependency-morphism (with additional compatibility
criterion, Definition~\ref{de:Chom}). Yet, the coupling of the two is
loose enough to allow us to separate them in a development
process. More precisely, any morphism in $\CA\AD$ can be seen as applying an
algebra morphism and then a dependency morphism (or vice versa).
In a sense, $\CA\AD$ ``consists of'' two categories:
the ``horizontal'' one being
$\Aclass$ and the ``vertical'' one $\Dclass$.

Expressing this relative independence categorically, we can show that
computation algebras form  a double category
with horizontal and vertical arrows, respectively, of the form  $\<\ha h,
id\>$ and $\<id, \hc h\>$ (and cells being the commuting squares
\(\def\objectstyle{\scriptstyle}\spcol{-1}\sprow{-1}\diagram 
A\rto_{h}\dto_{v} & B\dto^{u}\\
C\rto^{g}&D\enddiagram\), where $h,g$ are horizontal and $u,v$ vertical
arrows).

We will not spell out the details of this construction which is rather
straightforward (for double categories, the reader may consult 
\cite{Ehr63, McL}).
We only show the simple proposition illustrating that any morphism in
$\CA\AD$ can be seen as
consisting of two, relatively independent steps: one mapping the algebra-part
and another mapping the communication-part.
The nice implication of this is
that we can work relatively independently with the algebra- and the
communication-part of the objects in $\CA\AD$.
%
\begin{CLAIM}\label{pr:factorC}
Each morphism $h:C\into D$ in $\CA\AD$ can be factored so that
\begin{enumerate}\MyLPar
\item $h=\Comp{\<\ha h, id_{\hc{C}}\>}{\<id_{\ha{D}}, \hc h\>}$,
\item $h=\Comp{\<id_{\ha{C}}, \hc h\>}{\<\ha h, id_{\hc{D}}\>}$.
\end{enumerate}
\end{CLAIM}
\begin{PROOF}
The following diagram details the first case:

\hspace*{8em}
\sprow{-0.3}
\spcol{-1}
\diagramcompileto{FactorC}
\hc{C} \save\go[d]\merge\framed<5pt>\restore \ddotted|{\ev_C}|>\tip
	& & & &	\hc{D} \save\go[d]\merge\framed<5pt>\restore   \xto[llll]_{\hc{h}}
		\ddotted|{\ev_D}|>\tip  \xto[ddll]^>>>>>{\hc{h}} \\
\ha{C} \xto[ddrr]_{\ha{h}} \xto '[r]'[rrr]^{\ha{h}}[rrrr] & & &  & \ha{D} \\
  & & \hc{C} \save\go[d]\merge\framed<5pt>\restore \ddotted|{\ev}|>\tip
	\xto[uull]^<<<<<<{id_{\hc{C}}}  \\
  & & \ha{D} \xto[uurr]_{id_{\ha{D}}}
\enddiagram

The intermediary computation algebra is as shown in the Figure with $\ev =
\Comp{\ev_C}{\ha h}$.  This makes $\<\ha h, id_{\hc C}\>$ a 
computation homomorphism
$\comA C \into \<\ha D,\hc C, \ev\>$.  Also, since $h$ is a 
computation homomorphism,
we have for all $i\in I^{\hc D}: \ha h(\ev_C(\hc h(i))) = \ev_D(i)$, 
i.e., $ev(\hc h(i))=ev_D(i)$ which means that
$\<id_{\ha D}, \hc h\>$ is a
computation homomorphism $\<\ha D, \hc C, \ev\>\into \comA D$.
The second case is entirely analogous with the intermediary algebra being
$\<\ha C, \hc D, \Comp{\hc h}{\ev_C}\>$.
\end{PROOF}
\vspace*{-6ex}

\subsubsection{Initial objects}\label{sub:init}
Define functor $\hcop{(\_)}:\CA\AD\into\op\Dclass$ analogously to
$\hc{(\_)}$ in
  \re{fu:dep}, i.e., $\hcop{A} = \hc A$ and $\hcop{\<f,g\>}=g$ (the
  communication part $g$ of a morphism $\<h,g\>$ in $\CA\AD$ is contravariant
  to the whole morphism). We then have the following general fact:
\begin{CLAIM}\label{pr:adj}
If $\Aclass$ has an initial object $T_\Aclass$, then the functor
$F:\op\Dclass\into{\CA\AD}$ defined by $F(\dd C)=\<T_\Aclass,\dd C\>$ and
$F(g)=\<id_{T_{\Aclass}},g\>$ is left adjoint to $\hcop{({\_})}$,
$F\adj \hcop{(\_)}$, with identity as unit.
\end{CLAIM}
  %
\noindent Thus, if $\op\Dclass$ has an initial object, so does $\CA\AD$.
However, in general, this will not be the case since interesting subcategories
of $\DDS$ do not have terminal objects ($\op\Dclass$ do not have initial
objects).  In particular $\es$ is initial in $\Gr$ and in $\DDS$, but
there are no terminal objects there.  (If $Z_\Aclass$ is terminal in
$\Aclass$, then $\<Z_\Aclass,\es\>$ is terminal in $\CA(\Aclass,\DDS)$.)

We do not focus on the notion of initiality and the relevance of
empty communication part is probably limited to the fact that it
allows us to treat standard algebras as computation algebras with an empty
dependency.

Nevertheless, the proposition gives us useful information in a slightly more
restricted context. Typically, initial object in a model class $\Aclass$ of a
specification is obtained as (or from) a term model. Then, given a particular
(non-empty) dependency, the proposition provides an ``initial'' evaluation
strategy over such a term model.


\section{Nondeterminism}\label{se:nd}
Since parallelism and distributed programs often are a source of nondeterminism, we suggest here how
this phenomenon can be incorporated into the present framework.
There are at least two ways to do it.
  The first, in Section \ref{sub:ndB}, utilizes
the notion of multialgebras, which are designed for incorporating
nondeterminism into algebraic specification as an abstraction
mechanism \cite{Hus,ADT,Survey}.
The second, Section \ref{sub:ndA}, is biased
towards the operational view, according to which nondeterminism is
  something which arises only during the actual computations.

\subsection{The multialgebraic view}\label{sub:ndB}
Revisiting the Definition~\ref{de:CA} of computation algebras and the
subsequent constructions, one can observe that we have not made any
significant assumptions about the underlying category of
$\Sigma$-algebras. The only essential requirement was that $ev$-function
assigns values from an algebra to each node of the dependency in a way
respecting the $lab$eling of the node. In fact,
instead of the usual $\Sigma$-algebras,
we can work with  a different category for $\AlgS$.

For modeling nondeterminism, a natural choice is (some) category of
multialgebras with multihomomorphisms, \cite{WB97}.  A multialgebra is an
algebra where operations applied to single arguments may return {\em sets} of
elements, corresponding to the possible results of a nondeterministic
operation.  Composition of operations is defined pointwise.  A (ground) term
is interpreted in a multialgebra as the set (of its possible results).  The
notion of a homomorphism $h:A\into B$ may be generalized in various ways
(\cite{WB97} surveys the possibilities), but the most common one replaces the
usual homomorphism condition by the requirement $h(f(x)^A) \subseteq
f^B(h(x))$, for each function symbol $f\in\Sigma$.  There is also a choice of
the primitive operations for writing specifications. The most common
approaches use a primitive predicate, $s\Incl t$ (interpreted in an algebra $A$
as set inclusion $s^A\subseteq t^A$, i.e., as $t$ being at least as
nondeterministic as $s$), and/or the nondeterministic binary choice
$\_\choice\_:S\times S\into S$, for some/every sort $S$. We do not discuss
the specification languages here, and in the examples to follow we will use
the above, most common, definition of multihomomorphism.

The change of the category of algebras is the only modification needed.
The computation homomorphism $h$
(with $\ha h$
being a multihomomorphism) allows us, for instance,
to refine an algebra $B$ to a more deterministic algebra $A$
  as illustrated in Figure \ref{fi:ndB}.
All operations except $\choice$ are deterministic.  $\hc
h$ may relabel
  $s\choice t$ to $s$ because both terms have the same sort
  and, furthermore, because $\ha A$ is more deterministic than
  $\ha B$: for $i\in I^B$ with $lab(i)= s\choice t$, we have
that
  $\ha h(\ev_A(\hc h(i))) = \ha h(s^A)= s^B \subseteq s^B\cup t^B =
(s\choice t)^B=ev_B(i)$, i.e., $\ha h$ is a multihomomorphism. 

\begin{figure}[hbt]
\def\objectstyle{\scriptstyle}
\spcol{-1.5}
\sprow{-.5}
\hspace*{7em}
\diagram %compileto{NonD}
& \ha A & \xto[rrrr]^{\ha h} & & & & 	& \ha B & & & & &
\xto[llll]_{\ha g} & \ha C \\
& f(s) & & & & & 			& f(s)\choice f(t) & & & & & 	& f(t) \\
& s \dear\uto &  & & & & \xto[llll]_{\hc h}	& s \choice t \dear\uto  &
\xto[rrrr]^{\hc g} & & & & 	& t \dear\uto \\
s \dear\urto & & t \dear\ulto & & & & 		s \dear\urto & & t \dear\ulto & & &
& 	s \dear\urto & & t \dear\ulto
\enddiagram
\caption{}\label{fi:ndB} 
\end{figure}


\subsection{The operational view}\label{sub:ndA}
One source of nondeterminism is purely operational -- the differences in
relative speed of various processors on various machines being
unpredictable.  A paradigmatic example of this is a processor $P$ waiting
for an input from one of several other processors.  $P$ processes the data
which arrives first -- without discriminating against any of the sources --
and discards any later arrivals.
The dependency
$\def\objectstyle{\scriptstyle}\spcol{-1.5}\sprow{-1.5}\diagram & 
s\choice t \\ s
\dear\urto & & t\dear\ulto \enddiagram$ from Figure~\ref{fi:ndB} can be read
as expressing just that: the value at the
topmost node depends on the values at the other two nodes -- it will simply
be the one arriving first.
Multialgebras provide an abstract model for this kind of phenomena.
We will now sketch a more detailed, low-level way of including them into
  computation algebra framework.

The computation related information for a computation algebra 
$A=\shcomA A$ is
represented by its communication part.  Thus we may retain the
standard notion of algebra $\ha A$ and try to model nondeterminism in $\hc A$.
We simply allow for labeling the nodes of $\hc A$ by {\em sets of terms},
rather than unique terms.  The set labeling a node represents the possible
values computed at the node.  The source of the evaluation function $\ev_A$
is then a set of pairs $\<i,t\>$ where $i$ is some node and $t$ one of the
terms labeling $i : \bigcup_{i\in {I^{\GGr(A)}}} \{\<i,t\>:t\in lab^A(i)\}$.
The nodes act as oracles -- in an actual computation each one will produce
a unique value.  But this value is not known in advance, and so the
labeling  merely indicates the range of
possibilities.
%
\begin{DEFINITION}{}\label{de:DDSn}
$\DDSn$ is the category where ($\CSig$ is as in Definition \ref{de:Gr}):
\begin{enumerate}\MyLPar
\item objects are $\Sigma$ n-data dependencies -- pairs $\dd G=\<G,lab\>$
where $G\in\Obj\Gr$ is a simple DAG and $lab:I^G\into\Psett(\GTerms)$ is a
function labeling each node of this graph with a non-empty set of ground
$\Sigma$-terms of the same sort, i.e., $\forall i\ \forall s,t\in
lab(i):Sort(s)=Sort(t)$~;
\item an $\DDSn$-morphism $h:\dd B\into \dd A$ is a pair $\<\hcg h, 
\hcl h\>$ where
	\begin{itemize}\MyLPar
	\item $\hcg h$ is a (sort compatible) $\Gr$-morphism
	$\hcg h:Gr(\dd B)\into Gr(\dd A)$,
	\item $\hcl h$ is a family $\{h_i:lab_{\dd A}(\hcg h(i))\into
lab_{\dd B}(i)\}_{i\in{I^B}} $.
	\end{itemize} \vspace*{-5ex}
\end{enumerate}
\end{DEFINITION}
\noindent
Sort compatibility means that the morphisms have to respect the sorting of
labels:
if labels of $i$ are of
sort $S$, then so must be the labels of $\hcg h(i)$. The $\hcl h$ mapping
(which we can treat as a partial function
$I^{A}\times\GTerms\into\GTerms$ with the domain given by all the pairs
$\<\hcg h(i),t\>$ for all $i\in I^{B}$ with $t\in lab_{\dd A}(\hcg h(i)$)
picks the subset of
labels of $lab_{\dd B}(i)$ which is the pre-image of the label set 
$lab_{\dd A}(\hcg h(i))$.
For instance, in Figure \ref{fi:nd}, $\dd B$ is a dependency for
nondeterministic choice between
  $s$ and $t$ which is then propagated as the argument to the function
  $f$. (The relabeling is merely a restriction and $\hcl h$ is the obvious
inclusion.)

\begin{figure}[hbt]
  \def\objectstyle{\scriptstyle}
\spcol{-1.5}
\sprow{-1}
\hspace*{7em}
\diagram %compileto{NonD}
& \dd A &  & & & & 	                &  \dd B & & & & &             &  \dd C \\
& f(s) & & & & & 			& \{f(s),f(t)\} & & & & & 	& f(t) \\
& s  \dear{\uto} & & & & & \xto[llll]_{h} 	& \{s,t\} \dear{\uto}  &
\xto[rrrr]^{g} & & & &	& s \dear{\uto} \\
s \dear\urto & & t \dear\ulto & & & & 		s \dear\urto & & t \dear\ulto & & &
& 	s \dear\urto & & t \dear\ulto
\enddiagram
\caption{}\label{fi:nd}
\end{figure}

The node $i$ in $\dd B$ with the label $\{s,t\}$ may compute $s$ or $t$.
Similarly, the topmost
node $j$ may compute $f(s)$ or $f(t)$. The dependency $\dep ij$ says that
when $i$ computes
one of $s$ or $t$, $j$ will compute one of $f(s)$ or $f(t)$.
A possible deterministic refinement of $\dd B$ which always chooses $s$ and
computes only $f(s)$ is given in $\dd A$ with the morphism $h$.

Observe, however, that the definitions of an n-dependency and an
$\DDSn$-morphism do not take
into account the
possible semantic information about {\em how} the computation at $j$
depends on the
result delivered by $i$. In principle, it admits the situations like $\dd C$,
where we obtain $\dep s{f(t)}$.
  The morphism
only captures the preservation of dependencies (in terms of the underlying
graphs) and the non-increasing nondeterminism.
That the computation of $f(t)$ at $j$
may result only if the node below produces $t$ is the kind of semantic
information -- {\em how}
the computation at $j$ depends on the input from $i$ --
which might (should) be incorporated into the specification of dependencies.
In short, if the dependency $\dep s{f(t)}$ in $\dd C$ is not of the intended
kind, it should be
prohibited by the specification (this topic is to be explored more fully in
a future paper).
%
\begin{DEFINITION}{}\label{de:CAn}
Given $\Aclass\subseteq\AlgS$ and $\Dclass\subseteq\DDSn$; an $\AD$
{\em n-computation algebra}
$C$ is $\comA C$ where
\begin{enumerate}\MyLPar
\item $\hc{C}\in\Obj{\Dclass}$~;
\item $\ha{C}\in\Obj{\Aclass}$~;
\item $\ev_C:\hc{C}\times\GTerms\into \ha{C}$ is a (partial)
function
such that $\forall t\in lab_{\hc{C}}(i) : \ev_C(i,t)=t^{\ha{C}}$.
\end{enumerate} \vspace*{-5ex}
\end{DEFINITION}
\noindent
As for computation algebras, $\ev$ is uniquely determined by $\ha C$ and
the labeling of $\hc C$.
\begin{DEFINITION}{}
An $\AD$ {\em n-computation homomorphism} $h:C\into D$ is 
$\<\ha h, \<\hcg h, \hcl
h\>\>$ where
\begin{enumerate}\MyLPar
\item $\ha h: \ha C\into \ha D \in \Mor{\Aclass}$ ;
\item $\<\hcg h, \hcl h\>: \hc D\into \hc C \in \Mor{\Dclass}$, and
\item for each $i\in I^{\GGr(D)}$ and $t\in lab_{\dd C}(\hcg
h(i))$, we have $\ha h(ev_{\dd C}(\hcg h(i),t))=ev_{\dd D}(i,\hcl 
h(\hcg h(i),t))$.
\end{enumerate} \vspace*{-3ex}
\end{DEFINITION}
%
The last condition (illustrated in Figure \ref{fi:homCn}) is a
generalization of the
commutativity condition for the computation homomorphisms from the 
Definition \ref{de:Chom}.
In the deterministic case, the two coincide since then $\hcl h$ is uniquely
determined by
$\hcg h$.

\begin{figure}[hbt]
\spreaddiagramcolumns{1pc}\spreaddiagramrows{1pc}
\diagramcompileto{Chomn}
& & & \hc{C}\times\GTerms \save\go[0,1]\merge\framed<5pt>\restore 
\rdotted|{\ev_{\dd C}}|>\tip
		\dto<.1ex>^{\textstyle {\hcl h}}
		& \ha{C} \dto^{\textstyle \ha{h}}
     & C \dto^{\textstyle h} \\
& & & \hc{D}\times\GTerms \uto<2.5ex>^{\textstyle \hcg{h}}
	\save\go[0,1]\merge\framed<5pt>\restore \rdotted|{\ev_{\dd 
D}}|>\tip & \ha{D} & D
\enddiagram
\caption{An n-computation homomorphism.}\label{fi:homCn}
\end{figure}

\noindent
The category $\CAn\AD$ has $\AD$ n-computation algebras as objects and $\AD$
n-computation homomorphisms as morphisms.
Distributions of $\CAn\AD$ are defined as before (Definition
\ref{de:implem}), and
so are the projection functors $\hc{(\_)}$ and $\ArchF$.
The Proposition~\ref{le:MAr} generalizes trivially to the nondeterministic
context by an entirely analogous proof using the construction \re{eq:clM}.
Modifying the first point of Definition \ref{de:compat} to the requirement
$\forall i\in I^{Gr(\dd C)}, t\in lab_{\dd D}(\hcg h(i)): t^A= (\hcl h(\hcg
h(i),t))^A$,
we may
define the class $\CAn(\Aclass,\DDn(\Aclass))$ of n-algebras with
compatible dependency
morphisms. The Lemma \ref{le:Dep} generalizes then equally easy to this
class by a
construction analogous to \re{eq:cl}.
We therefore do not do it explicitly here but only emphasize
that the earlier results remain valid when we put $\Cat n$-'s
at all relevant places.


\section{Conclusions and further work}\label{se:conclusion}
We have introduced a general notion of computation algebras which
extends the standard algebraic semantics with additional structures, data
dependencies, carrying information concerning the evaluation strategy.
The dependency morphisms can be used to design more efficient
implementations, in particular, by syntactic memoisation.  The combination of
dependencies with actual algebras and the notion of computation 
homomorphisms allows
us to perform semantic memoisation as well.

The explicit information about the computation structure
opens the possibility for constructing parallel implementations from
algebraic specifications.
Dependency and distribution morphisms express, respectively, 
two  aspects of parallelization: (a) the
logical dependency decomposition and (b) partitioning and routing on a
physical medium.
We have shown how indexing computation algebras with
machine architectures yields a fibration.  Its explicitly defined cleavage
provides a way of porting distributions between various (parallel)
architectures.


The general framework admits the definition of various subclasses of
computation algebras.  One such class -- the algebras with compatible
dependency morphisms, $\CA(\Aclass,\DD(\Aclass))$ -- has been singled out
and studied in more detail.

Extending algebra(s) with the computation and control information 
resembles the otherwise widely studied attempts of adding such an
information to term rewriting systems \cite{ELANa, ELANb, Maude, Maudeb, Stratego}. In this
respect the main difference is that we are not con\-sider\-ing executable
specifications (like term rewriting systems) and their efficiency or
capabilities for modeling computations, but
efficiency of an actual implementation of a (low level, detailed)
specification. In particular, dependencies can be adjusted to actual
(parallel) machine architectures which are included as a part of the overall
picture in our model of development process.

In Section \ref{se:cr} we have suggested a method allowing one to construct data dependency from a
given specification of recursive function to be computed in a given algebra.
A special case of this method, combined with a subcategory $(\Aclass,
\DSaph)$ of computation algebras (cf. Section~\ref{sub:Saphire}) 
and embedded in
the programming language {\sc Sapphire}, has been worked out in more detail
and used in practical applications for parallelization of programs
\cite{Hav93,CH95,HS98,Hav99}.
This construction of efficient data dependencies for recursive functions
leads towards
consideration of efficient implementations of functional programming
languages, in particular, of basing such implementations on the semantic
($\mu$-recursive), rather then syntactic (substitution oriented)
computational model. This, however, is an open question for future research
which falls outside the scope of the present paper. 

\subsection*{Open issues}
Computation algebras offer a
mathematical framework for modeling efficient implementation of algebras.
Application of this framework to actual architectures and
(parallel) programming languages is the main topic deserving further study. 
We mention here three issues of more detailed character.

\paragraph{1. Space-time representation does not capture all aspects of parallel
architectures.} For instance, the simulation $m$ of 2-dimensional vector on a
3-processor ring from Figure~\ref{fi:ZW}.c) assumes that multiple data can be
sent along the channels. (The third node in the rightmost column in
Figure~\ref{fi:ZW}.c) receives
two pieces of data which {\em both} have to be passed to the node above it.)
In general, this may be precluded on machines where sending multiple data
requires several computation steps. Thus actual simulation morphisms
have to respect such additional restrictions.

Similar issue concerns the
kind of allowed communications. Figure~\ref{fi:broad}.b) shows
identical space-time representation of two different
architectures: $B$ -- with broadcats, and $S$ with stepwise communication
(i.e., a processor can communicate only along one channel at a
time). The distribution c) of $\dd D$ on $B$ is not meaningful on $S$ which
can be achieved in two steps as illustrated in d).%\vspace*{-3ex}

\begin{figure}[hbt]
\spcol{-1.5}
\sprow{-1.5}
%\hspace*{6em}
\[
\diagramcompileto{Broad}
j_1 & j_2 & j_3 & &&
 \bullet & \bullet & \bullet & &&
	j_1 & j_2 & j_3 & &&
	j_1 & j_2 & \circ \\
& & & &&
  & & & && 	& & & &&
	\circ \udotted \urdotted \urrdotted & \bullet \uto \ulto \urdotted &
		j_3 \ulldotted \uldotted \udotted \\
 & i \uulto \uuto \uurto &  & &&
  \bullet \uuto \uurto \uurrto & \bullet \uulto \uuto \uurto & \bullet
  \uullto \uulto \uuto & &&
	\circ \uudotted \uurdotted \uurrdotted & i \uulto \uuto \uurto &
		\circ \uulldotted \uuldotted \uudotted & &&
	\circ \udotted \urdotted \urrdotted & i \uldotted \morphism\solid{}{}[u]
\urto &
		\circ \ulldotted \udotted \uldotted \\
& \dd D & & && 
& B\ and\ S & & & && \dd D\into B & & & && \dd D\into S  \\
& \text{{\normalsize a)}} & & & && \text{{\normalsize b)}} & & & &&
\text{{\normalsize c)}} & & & &&\text{{\normalsize d)}}
\enddiagram
\]
\caption{a) Data dependency $\dd D$, b) space-time representation of a broadcast, $B$, and a stepwise,
$S$, architecture,
c) distribution of $\dd D$ on $B$, and d) on $S$.}\label{fi:broad}
\end{figure}

%\noindent
Distribution morphisms can distinguish between these situations but since
this information is not part of the st-representation, this is
something that has to be done when designing actual
morphisms. One should verify that space-time graphs, together with the simulation and distribution
morphisms,  offer sufficient representational power.

\paragraph{2. Portability vs. Efficiency.}
Another issue concerns the efficiency of the ported implementations
(distributions).  For instance, the implementation achieved in Figure
\ref{fi:port}.c) as the result of porting an implementation from a vector to
a ring, is certainly not the most efficient one.  There is, for instance, the
unnecessary step from $(Z_3,1)$ -- holding $F(1)$ -- to $(Z_3,2)$.  This delay occured because we have used
a {\em generic} simulation $m$ of $W$ by $Z$ (from Figure \ref{fi:ZW}.c), i.e., one that
allows us to port {\em any} implementation from $W$ to $Z$.  Even if such a
simulation is generically optimal (i.e., uses minimum time on the target
architecture for all possible simulations from the source architecture), for a particular
case it may be possible to find a more efficient mapping.  However, though
efficiency of computation always is a concern, general portability and
re-usability of software is at least equally important.  What we have
proposed is a sound methodology which guarantees {\em correctness
preservation} of an implementation which is ported from one platform to
another.  We expect that the possible loss in computational efficiency will,
hopefully, be richly compensated by increased {\em engineering efficiency\/}.
And, having successfully ported an implementation, there is nothing
precluding some final optimization in order to achieve a more efficient
product.

\paragraph{3. Code generation.}
The most important issue concerns generation of actual code.
It is not at all obvious that the communication
dependencies and the data dependencies should have the same or similar
interpretation in terms of evaluation strategies.
We may imagine that the hardware dependency $\dep ij$ reflects the fact
that $j$ has to wait for a piece of data from $i$ -- once it is produced and
received, $j$ begins its computation.  But we may also imagine a different scenario in which $j$
ignores the input from $i$ and simply evaluates whatever it is supposed to
as soon as it has sufficient information to do so (i.e., a non-strict
interpretation of the dependency which is determined at run-time).  
A unifying view is to think of the nodes (in a dependency) as 
``active agents'' who make
their local decisions as to how to treat the incoming information.  Then
the presence of an edge identifies a {\em potential\/} dependency. 
 As a particularly interesting case, if we
admit nondeterminism in computations, it
may turn out that the dependencies are introduced exactly in order to allow
non-strict evaluations where the relative speed of processing and
communication determines which dependencies are realized.  The resulting
implementation may turn out to be more time-efficient, allowing
simultaneous computation of possible inputs.

Although not sufficiently detailed,
our notion of distribution (Definition \ref{de:implem}) does capture several
aspects of parallelization.  Firstly, different ``columns'' in the
$st$-graph (the sets of nodes $\{\<s,t\>:t\in\Nat\}$ for different $s$)
of a given architecture  will represent
different processors and so a distribution mapping amounts to {\em
partitioning} of the code onto these processors.
Furthermore, the links between different components (``columns'') of the partition
correspond to possible communications between processors.  Mapping a
particular edge of a data dependency onto a path in a space-time graph
amounts then to a particular choice of {\em routing}.  (This is why we want
to work with the path, and not transitive closures of graphs.)
Finally, distribution morphisms represent (part of) the generation of the actual
code which consists of two parts: the communication part and the
computation part.  The shapes of dependencies serve as the source only of
the first part which is related to the communication structure, while their
labeling indicates distribution of the actual code.  $\dep ij$ could
correspond to the following commands being executed at $i$ and $j$,
respectively: ``$i:$ {\tt compute X;} {\bf send to} {\tt j;}'', and ``$j:$
{\bf wait for} {\tt i; compute Y;}''\footnote{The directions of
sending/receiving will often be identified not by the (name of) 
another processor but by
the local (name of a) channel. This is another specific distinction,
depending on the actual machine and programming language, which we did not
address at all.}, where $X,Y$ are
the pieces of code resulting from the labeling of $i$ and $j$. 

Our model comprises only the abstract notion, not such a
concrete view of implementation. Code generation for a special class of 
computation algebras
specified equationally with the explicit specification of the dependency
structures was studied in connection with the {\sc Sapphire}
methodology for development of parallel programs,
\cite{Hav93,HS98}. 

\subsubsection*{Other questions for further research}
\begin{itemize}
\item
Although we did not discuss any specification formalism, it
should be obvious that all the aspects of our setting are amenable to a
description using algebraic techniques.  In particular, data dependencies,
architectures, and morphisms between them are standard algebraic objects. 
Future research should integrate description of
data dependencies into an actual specification framework.
\item
An interesting alternative would be to investigate the
possibilities of extracting data
dependencies from a given algebraic specification.
Section~\ref{se:cr} opens a way in this direction.
Open questions concern, in particular, the syntactic
conditions on the recursive definitions such as (\ref{def:recur}) 
which would ensure
well-definedness of the resulting function in specific algebras. 
General construction of graphs and
labelings from specifications is a broader issue worth closer investigation.
\item
A variety of dependency and, especially, simulation morphisms between existing
architectures could be defined and stored for the future use in
concrete applications.  It would be desirable to develop such a library and
the tools for its use.
\item
We have indicated how the operational and
multialgebraic notions of nondeterminism may be included into the
framework.  However, the role of nondeterminism will be explored further.
\item
We would like to  make a closer study of the notion of
dependency refinement and its possible use for the development of
algorithms.  It is possible that the techniques of graph transformation
might be relevant for this purpose.
\end{itemize}

\subsection*{Acknowledgments}
We thank Andrzej Tarlecki and Eric Wagner for interesting suggestions and
comments on earlier versions of this paper. An anonymous referee suggested to
extend the definition of $\DAG^\ecomp$-morphisms by equipping graphs with the
``parallel composition'' using Kleisli construction. We also thank another anonymous
referee for suggesting improvements of the structure of the paper. Kristoffer Rose deserves thanks
for designing and making available the {\tt XY-pic} package for drawing
nice diagrams.


\begin{thebibliography}{MW95}\MyLPar
\bibitem[Ben85]{Ben} J.~B{\'e}nabou. Fibred Categories and the Foundation
   of Naive Category Theory. {\em Journal of Symbolic Logic}, 50, (1985).

\bibitem[BKK98a]{ELANa} P. Borovansky, C. Kirchner, H. Kirchner. A
functional view of rewriting and strategies for a semantics of ELAN. In {\em The
Third Fuji International Symposium on Functional and Logic Programming},
pp. 143-167, Kyoto (Japan), World Scientific, (1998). 
\bibitem[BKK98b]{ELANb} P. Borovansky, C. Kirchner, 
H. Kirchner. Rewriting as a Unified Specification Tool for Logic and Control:
The ELAN Language. In {\em Second International Workshop on the Theory and
Practice of Algebraic Specifications ASF+SDF'97}, Electronic Workshops in
Computing, eWiC web site: http://ewic.springer.co.uk/, Amsterdam (The
Netherlands), Springer, (1998). 

\bibitem[BGT91]{BGT91} R.~Burstall, J.~Goguen, A.~Tarlecki.
   Some Fundamental Algebraic Tools for the Semantics of Computation. Part 3:
   Indexed Categories. {\em TCS}, 91:239-264, (1991).
\bibitem[BW90]{BW90} M.~Barr, C.~Wells. {\em Category Theory for Computing
Science}.  Prentice Hall, (1990).
\bibitem[CDEM99]{Maudeb} M.~Clavel, F.~Dur\'{a}n, S.~Eker, J.~Meseguer,
M.-O.~Stehr. Maude as a Formal Meta-Tool. In the {\em Proceedings of FM'99}, The
World Congress On Formal Methods In The Development Of Computing
Systems, Toulouse (France), (1999).

\bibitem[CELM96]{Maude} M.~Clavel, S.~Eker, P.~Lincoln and
J. Meseguer. Principles of Maude.  In Proc. 1st Intl. Workshop on Rewriting Logic
and its
 Applications, Electronic Notes in Theoretical Computer Science,
Elsevier Sciences, (1996).

\bibitem[CT65]{CT65}
         J.~W.~Cooley, J.~W.~Tukey. An algorithm for the machine computation
         of complex Fourier series, {\em Math. Comp.} vol. 19, pp. 297-301,
         (1965).
\bibitem[\v CH95]{CH95} V.~\v Cyras, M.~Haveraaen. Modular Programming
of Recurrencies: a Comparison of Two Approaches, {\em Informatica}, vol.~6,
no.~4, (1995).
\bibitem[Ehr63]{Ehr63} C.~Ehresmann. {\em Cat\'{e}gories Structur\'{e}es,}
Ann.Sci.Ecole Norm.Sup. 80, (1963).
\bibitem[FOW87]{FOW} J.~Ferrante, K.J.~Ottenstein, J.D.~Warren. The Program
Dependence Graph and its use in Optimization. {\em ACM ToPLaS}, vol.~9, no.~3,
(1987).
\bibitem[Hav90]{Hav90} M.~Haveraaen. Distributing Programs on Different
Parallel Architectures. {\em Proceedings of the International Conference on
Parallel Processing, Software}, vol. II, (1990).
\bibitem[Hav93]{Hav93} M.~Haveraaen. {\em How to Create a Parallel Program
without Knowing it.}
   Proceedings of the $4^{th}$ Nordic Workshop on Program Correctness,
Tech.~Rep. no. 78,
   University of Bergen, Department of Informatics, (1993).
\bibitem[Hav99]{Hav99} M. Haveraaen: {An algebra of data dependencies and
	embeddings for parallel programming}. To appear in
	{\em Formal Aspects of Computing}, 1999.
\bibitem[HS98]{HS98} M. Haveraaen, S. S\o reide: {\em Solving recursive
	problems in linear time using Constructive Recursion}.
	Proceedings of {\em Norsk Informatikk
	Konferanse NIK'98\/}, Tapir, Norway, 1998, pp.~310--321.
\bibitem[Her93]{Her} C.A.~Hermida. {\em Fibrations, Logical Predicates and
   Indeterminates.} Technical Report~DAIMI PB-462, Computer Science
   Department, Aarhus University, (1993).
\bibitem[Hus93]{Hus} H.~Hu{\ss}mann. {\em Nondeterminism in Algebraic
Specifications
   and Algebraic Programs.} Birkh\"{a}user, (1993).
\bibitem[Jac91]{Jac} B.~Jacobs. {\em Categorical Type Theory.} PhD thesis,
   University of Nijmegen, (1991).
\bibitem[KK90]{KK} K.~Kennedy, K.~McKinley, Loop Distribution with Arbitrary
  Control Flow, {\em Supercomputing}, (1990).
\bibitem[MacL71]{McL} S.~Mac Lane. {\em Categories for the Working
Mathematician}, Springer, (1971).

\bibitem[MW83]{MW83} W.L.~Miranker, A.~Winkler. Spacetime Representations of
   Computational Structures. {\em Computing}, vol.~32, (1984).
\bibitem[PJ87]{PJ87} J.~Peyton, L.~Simon.
	{\em The Implementation of Functional Programming Languages},
	Prentice-Hall, (1987).
\bibitem[RB88]{RB88} D.~Rydeheard, R.~Burstall. {\em Computational Category
Theory}.
   Prentice Hall, (1988).

\bibitem[Vi99]{Stratego} E. Visser. Strategic Pattern Matching, in {\em
Rewriting Techniques and Applications (RTA'99)}, LNCS vol. 1631, pp. 30-44, Springer, (1999).

\bibitem[WB97]{WB97} M.~Walicki, M.~Bia{\l}asik. Categories of Relational
Structures.  In {\em Recent Trends in Data
Type Specification,} LNCS, vol.~1376, (1997).
\bibitem[WHM96]{WHM96} M.~Walicki, M.~Haveraaen, S.~Meldal, {\em Communication
   Algebras}, Tech. Rep. no~117, Dept. of Informatics, University of Bergen,
(1996).
\bibitem[WM94]{ADT} M.~Walicki, S.~Meldal. Multialgebras, Power Algebras and
Complete Calculi of Identities and Inclusions. In {\em Recent Trends in Data
Type Specification,} LNCS, vol.~906, (1994).
\bibitem[WM97]{Survey} M.~Walicki, S.~Meldal. Algebraic Approaches to
Nondeterminism -- an Overview. {\em ACM Computing Surveys,} 29, 1, (1997).

\end{thebibliography}


\end{document}

\section{Actual Machines and Parallel Code}\label{se:implem}
This section discusses some issues concerning actual implementation on actual
machine architectures which are not addressed by the general framework. These
issues need further study and we are including this discussion primarily to
suggest the required development and the abstract character of the framework
developed so far. 

Simulation, modeled by the $\Gr$-morphisms, plays a crucial role in 
defining appropriate
ways of porting implementations. Although the category of machine
architectures (i.e. $\Gr$, Definition~\ref{de:Arch}) allows arbitrary
$\Gr$-morphisms, in practice they will have to reflect more detailed
properties of the actual architectures. Such properties may be directly
expressible by
the space-time representations but, in many cases, will appear as
additional restrictions
on the admissible morphisms -- the space-time representation does not always
capture all the relevant information concerning the possible computations on a
given architecture. The first two subsections comment on this issue.

\subsection{Sending Multiple Data}
Recall the morphism $m:W\into Z$ from Figure \ref{fi:ZW}.c simulating
a 2-dimensional
vector $W$ by a 3-processor ring $Z$.
An alternative simulation $n$ is illustrated in Figure \ref{fi:alt}.
\begin{figure}[hbt]
\sprow{-1}
\spcol{-0.5}
\hspace*{14em}
\diagramcompileto{Alt}
5 & \bullet & \circ & \bullet  \\
4 \uto & \bullet \xdouble[u]|>\hole|>\tip \urdotted & \bullet \udotted\urto &
	\bullet \uto \ulldotted\\
3 \morphism\solid{}{}[u]
	&  \bullet \morphism\solid{}{}[u] \morphism\solid{}{}[ur] & \circ \udotted
\urdotted &
	\bullet \morphism\solid{}{}[u] \ullto \\
2 \morphism\solid{}{}[u]
	& \bullet \xdouble[u]|>\hole|>\tip \urdotted & \bullet \udotted \urto &
	\bullet \uto \ulldotted \\
1 \morphism\solid{}{}[u]
	& \bullet \morphism\solid{}{}[u] \morphism\solid{}{}[ur] & \circ \udotted
\urdotted &
	\bullet \morphism\solid{}{}[u] \ullto \\
& Z_1 & Z_2 & Z_3
\enddiagram
\caption{An alternative simulation $n:W\into Z$.}\label{fi:alt}
\end{figure}

\noindent
The simulation $n$ maps distinct edges of $W$ onto distinct but 
non-disjoint paths of $Z$.
This will yield an adequate simulation provided that $Z$ allows
us to join the data received from multiple sources and send them in one
package at the time.
The double arrows from the nodes $(Z_1,2)$ and $(Z_1,4)$ indicate
that, at these
points, data sent from different sources (e.g., from $(Z_1,1)$ and
$(Z_3,1)$) have to be
collected and sent together in one step. Of course, the same happens in
Figure \ref{fi:ZW}.c)
at the points $(Z_3,3)$ and $(Z_3,5)$. Whether this is possible, and how,
will depend on the
actual architecture. Consequently, the simulation morphisms have to take
this phenomenon into
account.
This does not seem to cause
any problems in this example
because the collected data are not communicated to another processor but
kept at the one
receiving them.

\subsection{Broadcast vs. Stepwise Communication}
A similar distinction will have to be made when a dependency is distributed
on an architecture
with broadcast vs. one where each processor can perform at most one
communication at each
time step (let's call the latter 1-ctp:
one-{\bf c}ommunication-(at-){\bf t}ime-(per-){\bf p}rocessor).

For the sake of the example, let's consider 3-processor architectures:
broadcast allows arbitrary communications between arbitrary
processors at each time step, while 1-ctp allows each processor to send data in
at most one direction at each time step (but at each step, any of the
communications is possible.)

Suppose that a dependency $\dd D$ in question has point(s), $i$, such
that there are several $j_n$ with $\dep i{j_n}$ for, let say, $n=3$.
It may be trivially distributed on a broadcast architecture $U$ with 3
processors as
illustrated in Figure \ref{fi:broad}.b).
A graphical space-time representation of a 3-processor 1-ctp $V$ will be
the same
as that of $U$ (\ref{fi:broad}.a).
However, for a given processor, $V$ allows at most one communication at a time,
and so
distribution
of $\dd D$ on $V$ cannot look like in \ref{fi:broad}.b) but will require at
least two time steps,
as exemplified in \ref{fi:broad}.c).

\begin{figure}[hbt]
\spcol{-1.5}
\sprow{-1}
\hspace*{6em}
\diagramcompileto{Broad}
\bullet & \bullet & \bullet & &
	j_1 & j_2 & j_3 & &
	j_1 & j_2 & \circ \\
  & & & & 	& & & &
	\circ \udotted \urdotted \urrdotted & \bullet \uto \ulto \urdotted &
		j_3 \ulldotted \uldotted \udotted \\
\bullet \uuto \uurto \uurrto & \bullet \uulto \uuto \uurto & \bullet
\uullto \uulto \uuto & &
	\circ \uudotted \uurdotted \uurrdotted & i \uulto \uuto \uurto &
		\circ \uulldotted \uuldotted \uudotted & &
	\circ \udotted \urdotted \urrdotted & i \uldotted \morphism\solid{}{}[u]
\urto &
		\circ \ulldotted \udotted \uldotted \\
& U / V & & & & \dd D\into U & & & & \dd D\into V  \\
& \text{{\normalsize a)}} & & & & \text{{\normalsize b)}} & & & &
\text{{\normalsize c)}}
\enddiagram
\caption{a) Space-time representation of a broadcast $U$ and a 1-ctp $V$,
b) distribution of $\dd D$ on $U$, and c) on $V$.}\label{fi:broad}
\end{figure}

\noindent
It is easier to think in terms of broadcast and so one may expect that
distributions
of dependencies will be made in terms of such architectures.
Having a simulation $m:U\into V$, we will be able to port such a
distribution to
a 1-ctp architecture. The only point is that such an $m$ must respect the
fact that
a space-time graph does not provide a complete representation of the
communication
structure.\footnote{The same problem will occur in many other cases. For
instance, if we take
the ring $Z$ in Figure \ref{fi:ZW}.b) to be
a token ring (only one processor ``holding the token'' can send data at a
time passing,
simultaneously, token to the receiving processor, which so can send data and
the token to the next one). In spite of the same space-time representation,
$m$ would not then be a correct
simulation of $W$ since the full communication history on a token ring must
form a path,
which is not the case in \ref{fi:ZW}.c).}

\subsection{Portability and Efficiency}
Another issue concerns the efficiency of the ported
implementations (distributions).  For instance, the implementation 
achieved in Figure
\ref{fi:port}.c) as the result of porting an implementation from a vector
to a ring, is certainly not the most efficient one.  There is, for
instance, the unnecessary step from $(Z_3,1)$ to $(Z_3,2)$ and, in general,
from $(Z_3,1+(4*n))$ to $(Z_3,2+(4*n))$.  $Z_3$ does not have to wait at
the point $1+(4*n)$ to send data to $Z_1$, but can do it immediately.  This
delay occurs here because we have used a generic simulation $m$ of $W$ by
$Z$.  It is generic in the sense that it allows us to port {\em any}
implementation from $W$ to $Z$.  In each particular case, such as the present
one, it may be possible to find a more efficient mapping which is tailored
for this particular application.  However, though efficiency of computation
always is a concern,
general portability and re-usability of software is at least equally 
as important
an issue.
What we have proposed is a sound methodology which
guarantees {\em correctness preservation} of an implementation which is ported
from one platform to another.  We expect that the possible loss in
computational efficiency will, hopefully, be richly compensated by
increased {\em engineering efficiency\/}.  And, having successfully ported
an implementation, there is nothing precluding some final optimization in
order to achieve a more efficient product.

\subsection{Code generation}
We called ``distribution'' a
mapping from a data dependency to an architecture.
We chose this, rather specific, term because we
do not want to claim that what we have
obtained is an actual implementation.

It is not at all obvious that the communication
dependencies and the data dependencies should have the same or similar
interpretation in terms of evaluation strategies.
We may imagine that the hardware dependency $\dep ij$ reflects the fact
that $j$ has to wait for a piece of data from $i$ -- once it is produced and
received, $j$ begins its computation.  This view is more data dependency
oriented -- as exemplified by the Fibonacci function in Figure
\ref{fi:rec}.  But we may also imagine a different scenario in which $j$
ignores the input from $i$ and simply evaluates whatever it is supposed to
as soon as it has sufficient information to do so (i.e., a non-strict
interpretation of the dependency).  The latter is often the approach used
in many parallel implementations.

A unifying view is to think of the nodes (in a dependency) as 
``active agents'' who make
their local decisions as to how to treat the incoming information.  Then
the presence of an edge identifies a {\em potential\/} dependency.  If a
term is evaluated in a strict manner, where all the data being referenced
have to be gathered before evaluation can proceed, then the edge identifies
the actual dependencies.  But if the evaluation is non-strict, e.g., it is
determined dynamically during actual computation whether a particular piece
of data is required or not, then the computation will realize a subset
of the data dependency edges.  As a particularly interesting case, if we
admit nondeterminism in computations, it
may turn out that the dependencies are introduced exactly in order to allow
non-strict evaluations where the relative speed of processing and
communication determines which dependencies are realized.  The resulting
implementation may turn out to be more time-efficient, allowing
simultaneous computation of possible inputs.

Although not sufficiently detailed,
our notion of distribution (Definition \ref{de:implem}) does capture several
aspects of parallelization.  Firstly, different threads in the $st$-graph
of a given architecture ($\{\<s,t\>:t\in\Nat\}$) will represent
different processors and so a distribution mapping amounts to {\em
partitioning} of the code onto these processors.

Furthermore, the links between different components of the partition
correspond to possible communications between processors.  Mapping a
particular edge of a data dependency onto a path in a space-time graph
amounts then to a particular choice of {\em routing}.  (This is why we want
to work with the path, and not transitive closures of graphs.)

Finally, distribution morphisms represent the generation of the actual
code which consists of two parts: the communication part and the
computation part.  The shapes of dependencies serve as the source only of
the first part which is related to the communication structure, while their
labeling indicates distribution of the actual code.  $\dep ij$ could
correspond to the following commands being executed at $i$ and $j$,
respectively: ``$i:$ {\tt compute X;} {\bf send to} {\tt j;}'', and ``$j:$
{\bf wait for} {\tt i; compute Y;}''.\footnote{The directions of
sending/receiving will often be identified not by the (name of) 
another processor but by
the local (name of a) channel. This is another specific distinction,
depending on the actual machine and programming language, which we did not
address at all.} $X$ and $Y$ are
the pieces of code resulting from the labeling of $i$ and $j$, together
with the semantics of these labels implied by the specification.  Notice,
however, that the separation is not absolute since the semantics of $Y$
may, in addition, indicate that the successful completion of the ``{\bf
wait for} {\tt i;}'' is either necessary or else merely optional for the
execution of $Y$.  In the former case the {\bf wait} will be strict, in the
latter, it may be time dependent.

Of course, our model comprises only the abstract notion, not such a
concrete view of implementation.  For instance, the actual code for each
processor has to be generated by ``sequentializing'' the pieces of code
assigned to the subsequent nodes on the path corresponding to the
processor.  Also, data dependencies contain explicit information about what
is computed at each node, but this obviously does not apply to the
communicating processors.  The sending node $i$ knows what the value of $X$
it is sending, but $X$ remains unknown to the recipient $j$ until it has
received it.  Although one can imagine how this transition from a data
dependency to the actual code might be accomplished, it certainly warrants
a closer analysis.  Code generation for a special class of 
communication algebras
specified equationally with the explicit specification of the dependency
structures was studied in connection with the {\sc Sapphire}
methodology for development of parallel programs,
\cite{Hav93,HS98}.  A more detailed
investigation of both the specification formalism and parallel code
generation is required to demonstrate the general applicability of the
present theory.

