\documentstyle{llncs}

%\newtheorem{claim}{Proposition}[section]
%\newtheorem{corollary}[claim]{Corollary}
%\newtheorem{theorem}[claim]{Theorem}
%\newtheorem{lemma}[claim]{Lemma}
%\newtheorem{definition}[claim]{Definition}
\newcommand{\smallerspaces}{\parsep -.2ex plus.2ex minus.2ex\itemsep\parsep
   \vspace{-\topsep}\vspace{.5ex}}
\newcommand{\MyNumEnv}[1]{\parskip 0pt\partopsep 0pt\trivlist \refstepcounter 
   {claim}\item [\hskip \labelsep {\bf #1\ \theclaim\ }]\ignorespaces}
%\newenvironment{example}{\MyNumEnv{Example}}{\nopagebreak \finish}
%\newenvironment{proof}{{\bf Proof.}}{\nopagebreak\finish}
\newcommand{\finish}{\hspace*{\fill}\nopagebreak 
     \raisebox{-1ex}{$\Box$}\hspace*{1em}\par\addvspace{1ex}}
\newcommand{\II}[1]{{\rm I\!#1}}
\newcommand{\Nat}{{\II N}}

\newcommand{\strategy}{{\sc maximal literal}}
\newcommand{\rev}[1]{\overline{#1}}
\newcommand{\subdom}[2]{#1\lceil #2}
\newcommand{\subterm}[2]{#1|_{#2}}
\newcommand{\frontier}[1]{\lfloor#1\rfloor}
\newcommand{\dt}[2]{\delta(#1,#2)}
\newcommand{\ft}[1]{\tau(#1)}
\newcommand{\fp}[1]{\pi(#1)}
\newcommand{\fl}[1]{\ell(#1)}
\newcommand{\CC}{{\sf CC}}
\newcommand{\Sub}{\sigma}
\newcommand{\Bin}{{\sf B}}
\newcommand{\At}{{\sf A}}
\newcommand{\Lit}{{\sf L}}
\newcommand{\Var}[1]{{\sl{\cal V}\!ar}(#1)}
\newcommand{\Pos}[1]{{\sl{\cal P}\!os}(#1)}
\newcommand{\Top}{\mbox{\footnotesize$\Lambda$}}
\newcommand{\mgu}{{\sf mgu}}
\newcommand{\Cl}[2]{#1\cdot#2}
\newcommand{\Funcs}{\Sigma}
\newcommand{\Terms}{{\cal T}}
\newcommand{\Dterms}{{\cal D}}
\newcommand{\D}{{\cal D}}
\newcommand{\Vars}{{\cal V}}
\newcommand{\Fvars}{\Phi}
\newcommand{\TD}{{\cal TD}}
\newcommand{\VV}[1]{\Vars(#1)}
\newcommand{\FV}[1]{\Fvars(#1)}

\newcommand{\Incl}{\mathbin{\prec}}
\newcommand{\Cont}{\mathbin{\succ}}
\newcommand{\Int}{\mathbin{\frown}}
\newcommand{\Seteq}{\mathbin{\asymp}}
\newcommand{\Eq}{\mathbin{\approx}}
\newcommand{\Din}{\mathbin{\ll}}
\newcommand{\Nid}{\mathbin{\gg}}

\newcommand{\notEq}{\mathbin{\Not\approx}}
\newcommand{\notIncl}{\mathbin{\Not\prec}}
\newcommand{\notCont}{\mathbin{\Not\succ}}  
\newcommand{\notInt}{\mathbin{\Not\frown}}
\newcommand{\notSeteq}{\mathbin{\Not\asymp}}
\newcommand{\notDin}{\mathbin{\Not\ll}}
\newcommand{\notNid}{\mathbin{\Not\gg}}

\newcommand{\Seq}{\mathrel{\mapsto}}
\newcommand{\Ord}{\mathop{\rightarrow}}
\newcommand{\M}[1]{\mathop{\mathop{#1}_{mul}}}
\newcommand{\Mset}[1]{{\cal M}(#1)}
\newcommand{\Value}[1]{[\![#1]\!]_{\upsilon}}
\newcommand{\Comp}[2]{#1\diamond#2}
\newcommand{\Repl}[2]{\mbox {\sf Repl}(#1,#2)}
\newcommand{\Sup}{\mbox {\sf Sup}}
\newcommand\Ss[1]{{\cal S}^{#1}}
\newcommand{\To}[1]{\mathbin {\stackrel {#1}{\longrightarrow}}}
\newcommand{\TTo}[1]{\mathbin {\stackrel {#1}{\Longrightarrow}}}
\newcommand{\oT}[1]{\mathbin {\stackrel {#1}{\longleftarrow}}}
\newcommand{\oTT}[1]{\mathbin {\stackrel {#1}{\Longleftarrow}}}
\newcommand{\es}{\emptyset}
\newcommand{\C}[1]{\mbox {$\cal #1$}}
%\newcommand{\Mb}[1]{\mbox {#1}}
\newcommand{\<}{\langle}
\renewcommand{\>}{\rangle}
\newcommand{\Def}{\mathrel {\stackrel {\mbox {\tiny def}}{=}}}
\newcommand{\Uni}{\stackrel {\mbox {\tiny uni}}{\sim}}
\newcommand{\impl}{\mathrel\Rightarrow}
\newcommand{\then}{\mathrel\Rightarrow}
\newcommand{\List}[3]{#1_{1}#3\ldots#3#1_{#2}}
\newcommand{\prule}[2]{{\displaystyle #1 \over \displaystyle#2}}
\newcounter{ITEM}
\newcommand{\newITEM}[1]{\gdef\ITEMlabel {ITEM:#1-}\setcounter {ITEM}{0}}
\makeatletter
\newcommand{\Not}[1]{\mathbin {\mathpalette\C@ncel#1}}
\newcommand{\C@ncel}[2]{\m@th \ooalign {$\hfil #1|\hfil $\crcr $#1#2$}}
\def\l@bel#1@{\edef\@currentlabel{(\roman {ITEM})}\label {#1}}
\newcommand{\ITEM}[2]{\par\addvspace{.7ex}\noindent\refstepcounter{ITEM}%
   \expandafter \l@bel \ITEMlabel #1@{\advance \linewidth-2em \hskip2em 
   \parbox{\linewidth}{\hskip-2em{\rm\bf \@currentlabel\ }\ignorespaces#2}}%
   \par \addvspace {.7ex}\noindent\ignorespaces}
\def\R@f#1@{\ref {#1}}
\newcommand{\?}[1]{\expandafter\R@f\ITEMlabel#1@}
\makeatother
\newcommand{\epsi}{\varepsilon}
\newcommand{\dom}{{\sl{\cal D}om}}
\newcommand{\true}{\bf T}
\newcommand{\false}{\bf F}

%\catcode`*=\active
%\def*#1*{{\sf #1}}
%\def*{\cdot}


\voffset -1cm
\begin{document}


\title{Rewriting and Reasoning with Set-Relations II: \\
The Non-Ground Case Completeness}

\author{\begin{tabular}[t]{c}
\it Valentinas Kriau\v ciukas\thanks
{Both authors gratefully acknowledge the financial support received from the
Norwegian Research Council.}
\\
\small Department of Mathematical Logic\\
\small  Institute of Mathematics and Informatics\\
\small  Vilnius, Lithuania\\
\footnotesize valentinas.kriauciukas@mlats.mii.lt
\and
\it Micha{\l} Walicki\( ^\star\)\\
\small Department of Computer Science\\
\small University of Bergen\\
\small Bergen, Norway\\
\footnotesize michal@ii.uib.no
\end{tabular}
}
\institute{{}}

%
%\author{ \begin{tabular}[t]{c}
%{\it Valentinas Kriau\v ciukas\thanks{Both authors gratefully 
%acknowledge the financial support received from the Norwegian Research Council.}}\\
%\small Department of Mathematical Logic\\
%\small Institute of Mathematics and Informatics\\
%\small Vilnius, Lithuania\\
%\footnotesize valentinas.kriauciukas@mlats.mii.lt 
%\and 
%{\it Micha{\l} Walicki\( ^*\)}\\
%\small Department of Computer Science\\
%\small University of Bergen\\
%\small Bergen, Norway\\
%\footnotesize michal@ii.uib.no}
%
\maketitle

\begin{abstract}
We consider reasoning and rewriting with
set-relations: intersection, inclusion and identity of 1-element sets, each
of which satisfies only two among the three properties of the equivalence
relations.  The paper presents a complete inference system which is a
generalization of ordered paramodulation and superposition calculi.  Notions
of rewriting proof and confluent rule system are defined for such
non-equivalence relations. Together with the notions of forcing and redundancy
they are applied in the completeness proof.  
Ground completeness cannot be lifted to the non-ground case
because substitution for variables is restricted to
deterministic terms.
To overcome the problems of restricted
substitutivity and hidden (in relations) existential quantification,
unification is defined as a three step process: substitution of determistic 
terms, introduction of bindings and ``on-line'' skolemisation.
The inference rules based on this unification derive non-ground clauses
even from the ground ones, thus making an application of a standard
lifting lemma impossible. The completness theorem is proved directly without 
use of such a lemma.

\noindent{\bf KEYWORDS:}
rewriting, theorem proving, binary relations, nondeterminism, 
completeness, ordered superposition
\end{abstract}


\section{Introduction}

Reasoning with sets becomes an important issue in different areas of computer
science. Its relevance can be noticed in constraint and logic programming
e.g.  \cite {SD,DO,Jay,Sto}, in algebraic approach to nondeterminism e.g.
\cite {HusB,PS1,MW}, in term rewriting e.g. \cite {LA,Kap,HusB}.

In an earlier paper \cite{KW}, we proved ground-completeness of reasoning
system for three set relations:  inclusion, intersection  and
identity of 1-element sets. The use of these originated from the study
of specification of nondeterminism in \cite{Mich,MW-II,MW}. The reader is referred
to these works for more detailed motivation and background.
In the present paper we extend the results from \cite{KW} to the non-ground case.

The
set-relations we are considering are not congruences -- not even equivalences: 
equality is not reflexive, inclusion is  not symmetric,
 and intersection is not transitive. 
We study the rewriting proofs in the presence of these
relations generalizing several classical notions (critical pair, confluence,
rewriting proof) to the present context. Our results on rewriting extend
bi-rewriting of Levy and Agusti \cite{LA} in that we consider three different
set-relations. We also take a step beyond the framework of Bachmair and
Ganzinger \cite{BG249} in that we study more general composition of relations
than chaining of transitive relations, albeit, in a very similar way as \cite
{BG-Oslo}.  But we have several other things complicating application of usual
techniques. \\[1ex]
%
\noindent 1) 
The essential feature of calculus of nondeterministic operations is
unsoundness of unrestricted substitution. Terms and variables denote objects
of different kind: variables always mean single elements, while terms mean
sets of possible values even if values of their variables are fixed.  This
excludes unification of terms by substitutions.  Therefore, we apply rules
like {\em relaxed paramodulation} \cite {relaxed-par}, which make terms
identical by ``cutting out'' some subterms, but which introduce new literals
into derived clauses, like in the following derivation:
\[
\prule {y\notIncl g(c); \quad h(x,x)\Incl g(x)}{y\notIncl h(x,x), x\notInt c}
\] 
Here \(s\Incl t\) means that each possible value of $s$ is also a possible
value of $t$, \(s\Int t\) means that $s,t$ have a common possible value,
comma between literals means disjunction and $x,y$ are the only variables.
The variable $x$ can not be replaced by $c$ in \(h(x,x)\Incl g(x)\) because
\(h(c,c)\Incl g(c)\) is equivalent to \(x\Int c\land y\Int c\then h(x,y)\Incl
g(c)\), but not to \(x\Int c\then h(x,x)\Incl g(c)\), when $c$ has more than
one possible value.  Fortunately, \(y\notIncl g(c)\) is equivalent to \(x\Int
c\then y\notIncl g(x)\), what is used in the derivation.  Literals 
\(x\notInt c\), where $x$ is a variable, are called {\em bindings}.\\[1ex]
%
\noindent 2) 
Another complicating circumstance is that
if some nondeterministic term occured instead of $y$ in the left premise,
the derivation, although  correct, would in some cases yield too weak a 
conclusion (Example~\ref {famous}).  \\[1ex]
%\noindent 
3) Yet another problem is illustrated by
an attempt to apply transitivity of $\Incl$ to atoms \(f(y)\Incl g(c,y)\) and
\( g(x,h(x,x))\Incl e(x)\).  The term $h(x,x)$ may be moved into a binding,
but $c$ cannot because \(f(y)\Incl g(c,y)\) means \(\exists x(x\Incl c\land
f(y)\Incl g(x,y))\) but not \(\forall x(x\Incl c\then f(y)\Incl g(x,y))\).
Bindings do not help in this situation, because the occurrence of the term $g(c,y)$ 
in the literal \(f(y)\Incl g(c,y)\) is, as we call it, of type $\exists$.  
Following Skolem, we know that
there exists a function $\alpha$ satisfying \(\alpha(y)\Incl c\) and \(y\Int
h(\alpha(y),\alpha(y))\then f(y)\Incl e(\alpha(y))\).  The function $\alpha$
is a semantical object, and we will introduce notation rules for such
functions since they will be needed in the derivations in
our inference system.\\[1ex]
%
We consider {\em refutational} proofs with a strategy analogous
to {\em ordered resolution} and {\em ordered superposition} \cite{BG,PP,S-A},
in which term ordering is used to restrict proof search space.  According to this
strategy, only maximal terms may be involved in the applications of
the inference rules. Our results are valid for any {\em simplification} ordering
\cite {Der} of terms.  In the following example the maximal terms are underlined.

\begin{example} \label {famous}
\begin{eqnarray}
\label{cl:f-h}  &\underline{f(x)}\Incl h(x,x); \\
\label{cl:g-h}  &\underline{g(x)}\Cont h(x,x);\\
\label{cl:g-f}  &\underline{g(c)}\notCont f(c);\\
\noalign
{This set of clauses has no model \cite{Hus,Mich}.  The ordering of 
the functional symbols, $g>f>h>c$, gives rise to the term ordering used here.
There is only one possibility to start: to unify terms with
$g$. But if $c$ were moved from $g(c)$ into a binding just now, the clause
\(x\notInt c, g(c)\notCont f(c)\) would be obtained, which does not
contradict the first two clauses.  First, the other side of the literal
must be made deterministic. A new deterministic constant $d$ denotes an element
of $f(c)$ which is not an element of $g(c)$:}
\label{cl:f-d}  &\underline{f(c)}\Cont d; &  (\ref {cl:g-f})\\
\label{cl:g-d}  &\underline{g(c)}\notInt d; &  (\ref {cl:g-f})\\
\noalign{Only now $c$ may be moved into a binding and transitivity of 
$\Cont$ applied:}
\label{cl:h-d}  & c\notInt x,\,\underline{h(x,x)}\notCont d; &  (\ref{cl:g-d},\ref{cl:g-h})\\
\label{cl:c-e}   &\underline c\Cont e; &  (\ref {cl:f-d})\\
\label{cl:fe-d}   &\underline{f(e)}\Cont d; &  (\ref {cl:f-d})\\
\noalign{In the similar way Clause~\ref {cl:f-d} was prepared to be resolved
with Clause~\ref {cl:f-h}, because deterministic terms can be substituted
without any restrictions:}
\label{cl:he-d}  &\underline{h(e,e)}\Cont d; &  (\ref {cl:fe-d},\ref{cl:f-h})\\
\label{cl:c-e2}  & \underline c\notInt e,\, d\notCont d; &  (\ref {cl:he-d},\ref{cl:h-d})\\
\label{cl:d-d}   & \underline e\notInt e,\,d\notCont d; &  (\ref {cl:c-e},\ref{cl:c-e2})\\
&\underline{d}\notCont d; &  (\ref {cl:d-d})\\
& \Box
\end{eqnarray}
The clauses~\ref {cl:g-d} and \ref{cl:fe-d} are {\em dual bindings} of new
deterministic terms (constants in this example) $d,e$, which are some kind of
{\em Skolem} functions, introduced to break down existential binding present
inside of some terms.  Their introduction, like introduction of variable
bindings, is an effect of term unification, and can not be avoided.
\end{example}
%
The example shows how complicated unification is in the case of
nondeterministic operations.  The definition of this unification was the
main problem to be solved on the way to the completeness result.

In a standard completeness proof one
proves completeness of
the ground case (we have done that in \cite {KW}), and a lifting lemma, which is
then applied to show completeness of non-ground case.  Lifting
lemma amounts to the fact 
that a conclusion of an inference rule applied to ground
instances of clauses is a ground instance of the conclusion of the same rule
applied to the clauses.  
In our case, this cannot be done.
For the first, rules introducing bindings do
not preserve groundness of clauses. The same is the case for
relaxed paramodulation rule \cite {relaxed-par} and therefore
authors constructed there a syntactic proof of completeness.  
Secondly, as shown by Example~\ref{famous}, ground
instances of a contradictory set of clauses may be consistent.
The problem is the {\em lack}
of deterministic constants and functions, which could be used to
form enough  ground terms.
We give a direct semantic proof of completeness not using a lifting lemma.
The same method can be also applied in the case of relaxed paramodulation.

Section~\ref{se:nd-specs} defines the syntax and the multialgebraic semantics 
of the language and lists some basic properties of the set-relations. 
Section \ref{se:unification} defines the unification of nondeterministic terms.
Section~\ref{se:reasoning} introduces the reasoning system \C I, ordering of 
words and specifies the \strategy\ proof strategy for using \C I.
%Section~\ref{se:Grewrite} discusses ground term rewriting with the introduced 
%set-relations which forms the basis for t
A sketch of the completeness proof is presented in
Section~\ref{se:completeness}.
Due to space limitations, only the general ideas of proofs of the main theorems
are included in this version of the paper.


\section{Syntax and Semantics}\label{se:nd-specs}

%\subsection{Syntax}
%
Specifications are written using a countable set of variables $\Vars$ and a
finite non-empty set of functional symbols $\Funcs$ having arity
\(ar:\Funcs\to \Nat\).\footnote{To simplify the notation we are treating only
the unsorted case. Extensions to many sorts are straightforward.}  A symbol
\(f\in\Funcs\) with \(ar(f)=0\) is called a {\em constant}.  Terms $\Terms$
over signature $\Funcs$ are defined in usual way. 

There are only three atomic forms of formulae built using binary predicates: 
{\em equation} $s\Eq t$, {\em inclusion} $s\Incl t$ and {\em intersection} 
$s\Int t$. Atoms and their negations form the set of {\em literals}. An atom 
$a$ is a {\em positive} literal, and a negated atom $\neg a$ is a {\em 
negative} literal. Negative literals are written as $s\notEq t$, 
$s\notIncl t$ and $s\notInt t$.  \(\rev{s\oplus t}\) is the reversed literal
\(t\rev \oplus s\), which is different from \(t\oplus s\) only in
the cases \(\oplus\in\{\Incl,\notIncl,\Cont,\notCont\}\), because these signs
are not symmetric by their shape and meaning.
A {\em clause} is a finite set of literals,
% ({\em i.e.}, multiplicity and ordering of the atoms do not matter), 
a {\em specification} is a set of {\em
clauses}.  
%In \cite{MW,Mich} a restricted language is used, allowing only
%negated intersections and only positive inclusions and equations in clauses
%
%Add more references, Michal!
%
By {\em words} we will mean the union of the sets of terms, literals and
clauses.  \(\VV w\) denotes the set of variables occuring in a word $w$.

%\subsection{Semantics}
%
Syntactic expressions are interpretated in {\em multialgebras}
\cite{Kap,Hus,Mich} which, unlike usual algebras, allow
functions to have multiple values.
(\(\true\) and  \(\false\) denote the boolean values.)
%
\begin{definition}
A $\Funcs$-{\em multialgebra} $A$ is a tuple \(\<S^A,\Funcs^A\>\) where
$S^A$ is a non empty {\em carrier} set, and $\Funcs^A$ is a set of
set-valued functions \(f^A: (S^A)^{ar(f)}\to\C P^+(S^A)\), where \(f\in
\Funcs\), and \(\C P^+(S^A)\) is the power-set of \(S^A\) with the empty set
excluded.
\end{definition}
%
\begin{definition} \label {def:semantics}
Let $A$ be a \(\Funcs\)-multialgebra, \(\upsilon:\Vars \to S^A\) be an
interpretation of variables, then value \(\Value w\) of a word $w$ is
defined for :
\begin{enumerate}\smallerspaces
\item a variable \(x\in\Vars\), \(\Value x \Def \upsilon(x)\);
  \label {semantics-v}
\item a constant $c\in\Funcs$, \(\Value c \Def c^A\);
\item a term \(t=f(\List tn,)\in \Terms\),  
  \(\displaystyle
  \Value t \Def \bigcup_{{\alpha_i} \in \Value{t_i}}
  f^A(\List{\alpha}n,)\), 
  \label {semantics1}
\item a literal \(l=s\oplus t\), \(\Value l \Def
  F_\oplus(\Value s, \Value t)\), where
  \vspace{1ex}\newline \(
  \begin{array}{r@{\ \equiv\ } l@{\quad}r@{\ \equiv\ } l}
      F_{\Eq}(U,V) & \forall \alpha\in U\; \forall \beta\in V\;\alpha=\beta, &
      F_{\Incl}(U,V) & U\subseteq V,\\
      F_{\Int}(U,V) & U\cap V\neq \es & %\mbox{, and } &
      F_{\neg\otimes}(U,V) & \neg F_\otimes(U,V);
  \end{array}\)
  \label {semantics3}
\item a clause \(C=\List ln,\), \(\Value C \Def \displaystyle
  \bigvee_{j=1}^n \Value{l_j} \), if \(n>0\) and \(\Value C \Def \false\), if $n=0$.
  \label {semantics4}
\vspace{-1ex}
\end{enumerate}
\noindent
$A$ {\em satisfies} a clause $C$ if \(\Value C =
\true\) for every interpretation \(\upsilon:\Vars \to S^A\).
It {\em satisfies} a literal $l$ iff it satisfies the clause
\(\{ l\}\), and {\em satisfies} a specification \C S iff it satisfies all
the clauses in \C S.
\end{definition}
%
%\begin{definition}\label {def:satisfies}
%A \(\Funcs\)-multialgebra 
%\end{definition}
%
%\subsection
\paragraph{Basic properties of atoms.}
%
%We just repeat here some facts from \cite {KW} in order to have a more
%self-containing paper.  
The relation \(s\Seteq t\Def s\Incl t\land s\Cont t\) expresses equality of term
value sets.
%and is the usual interpretation of equality in the set-valued
%approach to nondeterminism \cite{PS1,Kap}.
%\begin{equation} \label{eq:Seteq-definition}
%\end{equation}
%As can be expected, it does not increase expressibility and therefore is not
%used in the language.  For a discussion of the intended meaning and
%difference between `$\Eq$' and `$\Seteq$' in the context of nondeterminism
%see \cite{MW,Mich}.
%
\noindent
The positive, resp. negative, relations are totaly ordered by strength:
\begin{equation} \label{eq:rel-order}
u\Eq v  \impl u\Seteq v \impl u\Incl v \impl u\Int v
\hspace{2em} and \hspace{2em}
 u\notEq v \Leftarrow u \notSeteq \Leftarrow
 u\notIncl v \Leftarrow u\notInt v 
\end{equation}
Derivations and lemmas below refer always to the strongest possible relation.
%In derivations we always try to obtain the strongest possible relation, the
%strongest resulting relation always is presented in lemmas below.
%
%\subsubsection{Term replacement}

Replacement of terms --
``equals by equals'' -- is possible only in the case of equations, nevertheless
the following lemmas will allow later to develop techniques of term-rewriting.
%
\begin{lemma}\label{le:replacement}
The following term replacement properties hold: % for the introduced predicates:
\begin{equation}\label{eq:rep}
s\Eq t  \impl   u[s]_p \Seteq  u[t]_p\ \ \ \ \ \ 
s\Incl t  \impl  u[s]_p \Incl  u[t]_p\ \ \ \ \ \  
s\Int t \impl  u[s]_p \Int  u[t]_p  
\end{equation}
%\vspace{-1ex} \end{eqnarray*}
\end{lemma}
%
The following lemma describes  the rules for replacement of
subterms in literals and is related to the appearance of
{\em critical peaks} \cite{Der} and generation of {\em
critical pairs}
%
\begin{lemma} \label{le:replacement-in-atoms}
The defined predicates satisfy the rules
given in Table~\ref {tbl:replacement}.
\begin{table}[hbt]
\[\begin{array}{|c||c|c|c|c|c|c|c|c|}
\hline
 \Repl\_\_   & u[s]\Eq v   & u[s]\Incl v & u[s]\Cont v & u[s]\Int v & u[s]\notEq v & u[s]\notIncl v & u[s]\notCont v & u[s]\notInt v\\
\hline
\hline
s\Eq t   & u[t]\Eq v   & u[t]\Incl v & u[t]\Cont v & u[t]\Int v & u[t]\notEq v & u[t]\notIncl v & u[t]\notCont v & u[t]\notInt v\\
\hline
s\Incl t & u[t]\Cont v & u[t]\Int v  & u[t]\Cont v & u[t]\Int v & u[t]\notEq v & u[t]\notIncl v & u[t]\notEq v   & u[t]\notIncl v \\
\hline
s\Cont t & u[t]\Eq v   & u[t]\Incl v & -           & -          & -            & -              & u[t]\notCont v & u[t]\notInt v \\
\hline
s\Int t  & u[t]\Cont v & u[t]\Int v  & -           & -          & -            & -              & u[t]\notEq v   & u[t]\notIncl v \\
\hline 
\end{array}\]
\caption{Rules for subterm replacement} \label{tbl:replacement}
\end{table}
\end{lemma}

The table may be encoded as a partial function \(\Repl
\_\_\): \((s\oplus t\land u[s]_p\otimes v\impl u[t]_p \Repl \oplus \otimes
v)\) for any terms $s,t,u,v$ and position $p$ at $u$.  

When $u[s]=s$, we may use the table for {\em chaining} or {\em composing} relations,
e.g., $s\Eq t\land s\Incl u \impl t\Incl u$. We write it as $\Comp\Eq\Incl = \Incl$.
Notice that in (\ref{eq:rep}) the predicate `$\Eq$' was not inherited after substitution
unlike the other two. This means that, sometimes, composition may produce
stronger result than replacement.
% when composing, rather than replacing, we can obtain stronger result. 
For instance, \(s\Eq t\land s\Cont u\impl t\Eq u\)
(the table would yield $t\Cont u$).
The differences occur in row 1, columns 3 through 6, where the result of chaining will be
\begin{equation}\label{eq:chain}
\Comp\Eq{\Cont}=\Eq\ \ \ \ \ 
\Comp\Eq\Int =\Incl\ \ \ \ \ 
\Comp\Eq\notEq =\notCont\ \ \ \ \ 
\Comp\Eq\notIncl=\notInt 
\end{equation}
\noindent
%Thus composition $\Comp\oplus\otimes$ yields relation $\ominus$ which is the same as in 
%the table above except for these four cases. 
It is easy to see that $\Comp\_\_$ is transitive.
We join the two operations into one: % superposition function:
 \[\Sup(p ,\oplus ,\otimes) = \left \{ \begin{array}{cl}
 \Comp \oplus \otimes & {\ \ \rm if}\ p\ {\rm is\ the\ top\ position} \\
 \Repl \oplus \otimes & {\ \ \rm otherwise} \end{array} \right . \]
%This fact causes differences between the two tables
%below.
$\Sup$(erposition) of negative and positive atoms is symmetric to the superposition of
the positive and the negative ones given in the table. Superposition of two
negative atoms does not allow one to draw any specific conclusion and
therefore is not mentioned at all.
%
%The next lemma describes possibilities of {\em chaining} or {\em composing}
%the introduced relations.  Because of quite a big number of cases we present
%results in the form of table.
%
%\begin{lemma} \label{le:composition}
%For any given two relations sharing the same argument $s$, Table~\ref
%{tbl:composition} specifies the strongest relation that can be derived from
%them.  The bar `$-$' means trivial relation (relating all terms).  
%\begin{table}[hbt]
%\[\begin{array}{|c||c|c|c|c|c|c|c|c|}
%\hline
%\Comp{\_}{\_} & s\Eq u   & s\Incl u & s\Cont u & s\Int u   & s\notEq u   & s\notIncl u & s\notCont u & s\notInt u\\
%\hline \hline
%s\Eq t    & t\Eq u   & t\Incl u & t\Eq u   & t\Incl u  & t\notCont u & t\notInt u  & t\notCont u & t\notInt u\\
%\hline 
%s\Incl t  & t\Cont u & t\Int u  & t\Cont u & t\Int u   & t\notEq u   & t\notIncl u & t\notEq u   & t\notIncl u \\
%\hline 
%s\Cont t  & t\Eq u   & t\Incl u & -        & -         & -           &    -        & t\notCont u & t\notInt u\\
%\hline 
%s\Int t   & t\Cont u & t\Int u  & -        & -         & -           &    -        & t\notEq u   & t\notIncl u\\
%\hline 
%\end{array}\]
%\caption{Rules for literal composition}\label {tbl:composition}\vspace{-1ex}
%\end{table}
%\end{lemma}
%
%For convenience we will write the partial function coded in Table~\ref
%{tbl:composition} as \(\Comp\oplus\otimes=\odot\), meaning that $\odot$ is
%the strongest relation obtained by composing \(\oplus\) and \(\otimes\) for
%any terms.  Because of the ordering (\ref {eq:rel-order}) the fact that, for
%instance, \(\Comp\Eq\notInt=\notInt\) will imply that also \(\notIncl\) can
%be obtained from composing \(\Eq\) and \(\notInt\).
%
%
%\begin{lemma} \label {le:composition-transitivity}
%The composition function \(\Comp\_\_\) is transitive. 
%\end{lemma}
%.
%
%\subsubsection{Term types in literals}
\paragraph{Term types $\exists$ and $\forall$ in literals.\ }
The point~\ref {semantics3} of Definition \ref{def:semantics} involved
existential quantification in some predicates. For example,
\(F_{\Incl}(U,V)\) means \(\forall\alpha\in U\/ \exists \beta\in V\,
\alpha=\beta\),  \(F_{\Int}(U,V)\) means
\(\exists\alpha\in U\/ \exists \beta\in V\, \alpha=\beta\), and so on.
The meaning of atoms is defined according to the following schema:
$\Value{s\oplus t} = Q^\oplus_1 \alpha\in \Value s\, Q^\oplus_2\beta \in
\Value t\, \alpha = \beta,$ where \(Q^\oplus_i\in \{\forall ,\exists \}\).
It explains why terms
in literals can have types $\forall$ or $\exists$.
%\begin{equation}\label {eq:quantifiers}
%\Value{s\oplus t} = Q^\oplus_1 \alpha\in \Value s\, Q^\oplus_2\beta \in
%\Value t\, \alpha = \beta,
%\end{equation}
%where the notation \(\Value .\) of Definition~\ref {def:semantics} is used,
%and \(Q^\oplus_i\in \{\forall ,\exists \}\) are
%quintifiers. 
%
There are four possibilities of arranging the quantifiers $Q_1^\oplus Q_2^\oplus :
\forall\forall$ corresponds to $\oplus=\Eq$, $\forall\exists$ to $\Incl$ and
$\exists\exists$ to $\Int$. 
 The fourth $\exists\forall$, say $\Nid$, has not been used but it can be
defined as $s\Nid t\Leftrightarrow s\Cont t\land t\Eq t$.
It is the exact counterpart of the relation `$:$' used in {\em unified algebras} 
\cite {uni-al}.
%
%\begin{table}[hbt]
%\[\begin{array}{|c||c|c|c|c|}
%\hline
%\oplus     & \Eq     &  \Incl  &  \Int   &  \Nid   \\
%\hline
%Q^\oplus_1 & \forall & \forall & \exists & \exists \\
%\hline 
%Q^\oplus_2 & \forall & \exists & \exists & \forall \\
%\hline 
%\end{array}\]
%\caption{Term types in atoms} \label{tbl:types}
%\end{table}
%
%Table~\ref {tbl:types} says for instance that both terms $s$ and $t$ in
%\(s\Eq t\) are of type $\forall$, in \(s\Int t\) are of type $\exists$, for
%negated atoms \(s\notEq t\) and \(s\notInt t\) the types are opposite. All
%other literals have terms of different types.
%
%The relation \(s\Nid t\) is
%definable as \(s\Cont t\land t\Eq t\), so by strength this relation is
%between \(s\Cont t\) and \(s\Eq t\). 
%(This is the exact counterpart of the relation `$:$' used in {\em unified algebras} 
%\cite {uni-al}.)
% It defines \(s\Eq t\) in the form \(s\Din t\land s\Nid t\), like
%`$\Incl$' defines `$\Seteq$' in (\ref {eq:Seteq-definition}). However,
%`$\Din$' and `$\Nid$' taken separately cannot model `$\Eq$', like `$\Incl$'
%and `$\Cont$' do with `$\Seteq$'.  It is enough to check Table~\ref
%{tbl:composition} to see that any result of composition of an atom \(s\Seteq
%t\) with some other atom can be obtained by composition of either \(s\Incl
%t\) or \(s\Cont t\), what justifies our ignoring of `$\Seteq$'.  In the case
%of `$\Din$' and `$\Eq$', composition of \(r\Incl s\) with \(s\Din t\) gives
%\(r\Din t\), with \(s\Nid t\) gives nothing, but with \(s\Eq t\) produces
%\(r\Eq t\), a stronger relation than two previous.  Of course, from \(r\Din
%t\) and \(s\Nid t\) it follows that \(r\Eq t\) because of determinism of $t$ implied
%by \(s\Nid t\).  The composition rules cannot help here -- some others would have 
%to be used.
%
\section{Unification of nondeterministic terms}\label{se:unification}

\subsection{Introduction of f-variables}

In proofs we extend syntax by additional functional symbols from an infinite
set \(\Fvars\) called {\em f-variables}.  They are related with some literals
and {\em positions} in them. 
%
%\subsubsection{Positions in terms}
%
%Informally, a position in a term is a place where some symbol or subterm
%occurs.  Formally, we define it as follows:
%
\begin{definition}\label {def:position}
For convenience, we define two kinds of positions:
\begin{enumerate}\smallerspaces
\item A {\em position} is any finite sequence \(f_1n_1\ldots f_kn_k\) or
\(f_1n_1\ldots f_kn_kc\), where $f_i$ %\(\List fk,\) 
are functional sybols, $c$ is a
constant, and $n_i$ %\(\List nk,\) 
are natural numbers such that \(0< n_i\leq
ar(f_i)\).  The empty sequence, the {\em top position}, is denoted
\(\Top\).
\item A {\em variable position} is a position without the trailing constant $c$.
\end{enumerate}
\end{definition}
We include the functional symbols in positions in order to be able to relate
positions directly with specific terms. The lack of any (variable) symbol at the 
end of a variable position corresponds to the ``hole'' represented by a variable
-- appending any term, or position, will correspond to substitution.

The top position exists in every term $t$ and is the position of the outermost
functional symbol, if $t$ is a non-variable term, otherwise, there is only
one position $\Top$ in $t$.  For any function symbol $f$ occurring in $t$ at
a position $p$, if \(ar(f)>0\), then any position \(pfm\) with \(0<m\leq
ar(f)\) is in $t$, otherwise ($f$ is a constant), only \(pf\) is in $t$.  The
set of all positions of the term $t$, denoted \(\Pos t\), is the smallest set
of positions satisfying the two previous sentences.  Positions are partially
ordered by the prefix order, a position $pq$ is below $p$, if $q$ is
non-empty.  The top position is the top element of their semilattice.  For a
set of positions $P$, \(\min(P)\) and \(\max(P)\) denote,
respectively, the sets of minimal and maximal positions in $P$.  The set 
\(\Var t\) of variable positions in a term $t$ is a subset of \(\min (\Pos
t)\), the remaining minimal positions (if any) are occupied by constants.
%  A set
%of positions $P$ is called a {\em frontier}, if all positions in $P$ are
%pairwise incomparable, {\it i.e.}, \(P=\max(P)=\min(P)\).  
For a given
position $p$ and a term $t$, \(\subterm tp\) denotes the subterm of $t$
occuring at the position $p$, $t[s]_p$ denotes the term $t$ with the subterm
\(\subterm tp\) replaced by $s$ (the case \(s=\subterm tp\) is 
possible).  For a set of positions $P$, \(\subterm tP\) denotes the set of
subterms \(\{\subterm tp:p\in P\cap \Pos t\}\).
%
\paragraph{When are f-variables introduced?\ }
%
The f-variables correspond to {\em Skolem} functions and are needed in
unification of the terms of type $\exists$. 
 There are four general forms of literals whith a non-variable 
term $s$ of type $\exists$, 
in which new f-variables may be introduced by unification.
\begin{equation} \label{eq:exist-literals}
 s\Int t,\quad s\notEq t, \quad s\notIncl t, \quad  s\Cont t,
\end{equation} 
The unification process will introduce  f-variables for
such literals $l$ which, in addition, satisfy:
%\begin{enumerate}\smallerspaces
1) all variables occur in $l$ at most once, and
2) for $l= s\Cont t$, $t$ is a variable.
%\end{enumerate}
To describe all possible appearences of f-variables in any proof, we establish
a 1-1 correspondence  between the set $\Fvars$ and the set of all
pairs \(\<[l],p\>\), where a literal $l$ satisfying the above two conditions
has one of the forms presented in
(\ref {eq:exist-literals}), $p$ is a non-variable position in $s$ , and
\([l]\) is a pair of position sets \(\<\Pos s ,\Pos t\>\) presenting a class
of literals equal to $l$ up to renaming of variables.  Let $e$ be an
f-variable corresponding to the pair \(\<[l],p\>\) and let
%variables occur in $l$ at most once and let 
$\vec x$ be the list of all variables in $l$ (each occurring once).  Then
arity of $e$ is equal \(|\Var s\cup\Var t|\) (\(=|\vec x|\)), and we let
%\[
$\dt lp\Def e(\vec x).$ 
%\] 
Now, for any literal $l'=l\sigma$ obtained from
$l$ by a substitution $\sigma$, \(\dt {l'}p=e(\vec x\sigma)\).  Any literal
$l'$ with \([l']=[l]\) has the form \(l\rho\) for some renaming of
variables $\rho$.  Thanks to these notational conventions,
 it is possible to forget about
pairs of position sets and work directly with terms and literals.  We also
denote by
%\[
$\ft{e(\vec x)}\Def \subterm sp$, 
the subterm which can be safely
replaced by $e(\vec x)$ in the sense of Lemma~\ref {le:f-variables}, 
the replacement position by
$\fp{e(\vec x)}\Def p$, 
and the {\em only} literal
where the f-variable $e$ can be introduced by
$\fl{e(\vec x)}\Def l$.
% All these notations will be used in the case of substituted variables $\vec x$.

Thus, we related the literals from (\ref{eq:exist-literals}) 
and non-variable positions in their first terms with
specific terms containing only one functional symbol from \( \Fvars\).

\paragraph{Deterministic terms and substitutions.\ }
%
%F-variables are {\em Skolem} functions which have to cope with elimination of
%existential quantification hidden in some predicates.  
Lemma~\ref {le:f-variables}
demonstrates the characteristic properties of f-variables.  
Terms constructed completely
of variables and f-variables are called {\em d-terms} (shorthand for {\em
deterministic terms}), their set is denoted $\Dterms$, so \(\Terms \cap
\Dterms =\Vars\).  Terms of this kind are used to construct a model in the
completeness proof.  Only d-terms are allowed to be substituted into
variables.

\begin{definition}\label{def:substitution}
Call a {\em substitution} any function \(\sigma:\Vars \to \Dterms\).  The
domain \(\dom (\sigma)\) of $\sigma$ is the set \(\{x: \sigma(x) \neq x\}\)
of variables on which $\sigma$ is non-trivial.  As a set the substitution
$\sigma$ is considered as the set of pairs \(\{\<x,\sigma(x)\>: x\in\dom
(\sigma)\}\).
\end{definition}

\paragraph{Frontiers and skeletons.\ }  

Variables are the only d-terms in the
set $\Terms$.  After instantiation of some variables by d-terms, the
obtained terms become divided into two parts: the top is nondeterminitic, and
bottom is determinitic.  The natural way to present this division is to write
such terms in the form \(t\sigma\), where $t$ does not contain f-variables
and $\sigma$ is a substitution.  The same applies to other words, like
literals and clauses.  Words in such presentation remind of {\em
closures} \(w\cdot \sigma\) from \cite {Basic-par}.  We sometimes use this
form of presentation.  In our case, unlike in \cite {Basic-par},
this form is derivable from the word structure because of non-intersection of
classes \(\Terms \setminus \Vars\) and \(\Dterms \setminus \Vars\).  We
borrow some terminology from \cite {Basic-par}: $w$ is called a {\em
skeleton} and \(\Var w\) is called the {\em frontier} of a word \(w\sigma\),
also denoted \(\frontier {w\sigma}\).  It is supposed that all variables in
any skeleton are different, therefore all skeletons of the word \(w\sigma\)
are equal up to renaming of variables.  (This is relevant to introduction of
f-variables discussed above.)  In spite of non-uniqueness of skeletons, we
will write \(w\sigma=w\cdot \sigma\), as if interpreting the sign
`$\cdot $' as an application of substitution to the word $w$.

The restriction to substitute only d-terms, restricts the possibility to
unify terms. As a kind of compensation for that, it  
 is allowed to {\em replace} some subterms by d-terms.
Soundness of this replacement
is based on the semantics of f-variables.
%
%\subsection{Semantics for f-variables}
\paragraph{Semantics for f-variables.\ }
For a  \(\Funcs\)- multialgebra $A$, each f-variable $e$ is
interpretated as a determinitic function \(\phi(e): (S^A)^{ar(e)}\to S^A\).
Let $\phi$ be such an interpretation of f-variables. It
extends  $A$ to a \(\Funcs\cup\Fvars\)-multialgebra, denoted
 \(A_\phi\).  Values of d-terms in the multialgebra \(A_\phi\) are
evaluated according to the usual rules of (deterministic) algebras.  
The interpretations of f-variables must satisfy the additional conditions
given in:

\begin{lemma}\label{le:f-variables}
Any \(\Funcs\)-multialgebra $A$ can be extended with an intepretation $\phi$
of f-variables in such a way that for any d-term $d:$ 
%the following statements are true:
\begin{itemize}\smallerspaces
\item $A_\phi$ satisfies the atom \(d\Incl \ft d\);
\item $A_\phi$ satisfies \(\fl d\) iff satisfies \(\fl d [d]_{\fp d}\) \\
(If \(\fl d\) is of the form $f(s)\Cont t$, then this condition is satisfied
for $t\in\Vars$.)
\vspace{-2ex}
\end{itemize}
\end{lemma}
The last condition says that the subterm \(\ft d\) can be replaced by $d$ in
\(\fl d\) without changing the meaning of \(\fl d\). 

\subsection{Unification}

Success in unifying terms depends not only on terms, but also on the literals in which
they occur.  In usual unification, one tries to make terms identical by
some {\em unifying substitution}.  In our case, only d-subterms can be substituted
for variables.  If a variable  should be replaced by a
non-deterministic subterm, then the inverse action is made --- the subterm is
replaced by the variable, we say that the subterm is {\em ejected} and put
into a new literal, called a {\em binding}.  In clauses, this kind of
replacement is legal only inside of terms of type $\forall$.  In terms of
type $\exists$, the ejected subterms must be replaced by new f-variables,
which are then bound by {\em assumptions}, the special kind of clauses.  To
be shorter in some places below, we call {\em unifying sets} collections
consisting of a substitution (presented as a set), a set of bindings and a
set of assumptions.  In general, the process of unification can be presented
as a sequence of three phases: 1)~ejection of some subterms, 2)~formation of
the unifying sets, and 3)~usual unification.
%
\paragraph{Ejection.\ }
%
To perform an ejection from a term, it is sufficient to know the frontier
of the other term, so we formulate ejection relatively to some given
set $Q$ of positions.  Let \(l= s\oplus t\) and \(l'=l\cdot \sigma\) be
literals, $l'$ be the one where ejection should occur, and let \(P
\Def \max(Q\cap\Pos s)\setminus \Var s\) be the set of non-variable positions of
$s$ wich are maximal in $Q$.  If $P$ is empty, then there are no
subterms to be ejected.  If not, then we proceed as follows.  All subterms
\(\subterm sP\) are ejected and replaced by new (all distinct) variables,
which are neither from \(\dom (\sigma)\) nor from \(\VV l'\).  Let $s_Q$ be
the term obtained from $s$ by this replacement. 

The rest depends on the form of the literal $l$ and is presented in the next
phases. 
%
\paragraph{Formation of unifying sets.\ }
%
Let \(B =\{\subterm {s_Q} p\notInt \subterm sp: p\in P\}\) be a set of
bindings, \(A =\{\dt {l} p\Incl \subterm sp: p\in P\}\) be a set of atoms,
and \(S =\{\<\subterm {s_Q}p ,\dt {l'}p\>: p\in P\}\) be a substitution. All
these sets are obtained by the replacement of \(\subterm sP\).

Different cases to consider are presented in 
of Table~\ref {tbl:unification}.  The second term of $l$ ({\it i.e.}, $t$)
can be changed to a new variable $y$, so the final forms of $l$ and $t$ are
denoted $l_Q$ and $t_Q$, while \(\Sub(l',Q)\) and \(\At(l',Q)\) denote the
obtained substitution and the set of atoms, respectively.  The first line of
the table describes the trivial case \(P=\emptyset\).

\begin{table}[hbt]
\begin{center}
\(
\begin{array}{|c||c|c|c|c|}
\hline
   \mbox{for }Q\mbox{ and }l'=(s\oplus t)\cdot \sigma & \Bin(l',Q) & l_Q & \Sub(l',Q) & \At(l',Q) \\
\hline\hline
 (Q\cap\Pos s)\setminus \Var s=\es & \es & l & \es & \es \\ 
\hline
 \oplus=\Cont\land t\notin \Vars & \{y\notInt t\} & s_Q\Cont y & \sigma\cup S & A \\
\hline
 \oplus\in\{\Cont,\Int,\notEq,\notIncl\}\land & & & & \\
 \land (\oplus=\Cont\then t\in \Vars) & \es & s_Q\oplus t & \sigma\cup S & A \\
\hline
 \oplus=\notCont\land t\notin \Vars & B & s_Q\notInt y & \sigma\cup\{\<y,\dt {t\notIncl s}{\Top}\>\} & \{\dt {t\notIncl s}{\Top}\Incl t\} \\
\hline
  \oplus\in\{\Eq,\Incl,\notCont,\notInt\}\land & & & & \\
  \land( \oplus=\notCont\then t\in \Vars) & B & s_Q\oplus t & \sigma & \es \\
\hline
\end{array}
\)
\end{center}
\caption{Ejection cases} \vspace{-3ex} \label{tbl:unification}
\end{table}
%
\paragraph{Deterministic unification.\ }
%
The ejection and formation of unifying sets were formulated relatively to
some unspecified set of positions $Q$.  In unification of two terms $s',s''$,
$Q$ is the union of their frontiers, \(\frontier {s'}\cup \frontier
{s''}\).  In the first step, every non-variable term from \(\subterm {s'}
Q\cup \subterm {s''}Q\) was ejected and replaced by a new variable, in the
second step, the literals $l'\cdot\sigma'$ and $l''\cdot\sigma''$ containing
$s',s''$ were considered and, if necessary, transformed.  $s',s''$ became now
\(s'_Q\Sub(l',Q)\) and \(s''_Q\Sub(l'',Q)\).  The whole question is reduced
now to unification of the latter two terms by a substitution, say $\rho$.
The previous transformations were needed only to ensure that no 
nondeterministic term appears in $\rho$.  Practical unification algorithms 
could, of course, proceed in other way, but this is another story.

\section{Inference system}\label{se:reasoning}
%
%\subsection{Overlapping of literals}
%
Unification is used in inference rules, as the
case of {\em literal overlapping}.  \(\mgu(s,t)\) denotes the {\em most
general unifier} of terms $s,t$.  For a position $p$ and a set of
positions $P$ we denote \(pP\Def \{pq: q\in P\}\).

\begin{definition}\label {def:literal-overalap}
Let \(l'=s\oplus t\) and \(l''=u \otimes v\) be literals, $p$ be a position
in $u$ above the frontier, \(Q=\max(\frontier s \cup \frontier {\subterm
up})\) be a set of positions, \( s'\oplus' t' = l'_Q \cdot \Sub(l',Q)\),
\(u'\otimes' v' = l''_{pQ} \cdot \Sub(l'',pQ)\).  Then the literal $l'$
{\em overlaps} the literal $l''$ at a position $p$, if the substitution
\(\Sub(l',l'',p) =\mgu (s\Sub(l',Q), \subterm up\Sub(l'',Q))\)
called the {\em unifying substitution} exists.
\end{definition}

Literal overlapping is not sufficient to derive new literal from $l'$ and
$l''$, the additional condition being that the relation \(\ominus
=\Sup(p ,\oplus' ,\otimes')\) is non-trivial.  In this case,
\begin{itemize}\smallerspaces
\item the literal \(\Lit (l',l'',p)\Def u'[t']_p\ominus v\) is called the {\em
    critical literal} formed by  \(l'\) and \(l''\); 
\item the literal set \(\Bin(l',l'',p)\Def \Bin(l',Q) \cdot \Sub(l',Q )\cup
   \Bin(l'',pQ) \cdot \Sub(l'',pQ)\) is called the {\em binding set};
\item the clause \(\CC(l',l'',p)\Def \Bin(l',l'',p) \cup \{\Lit (l,l',p)\}\), 
   is called the {\em critical clause} formed by the  \(l'\) and \(l''\);
\item the set of single clauses \(\At(l',l'',p)\Def \{\{a \cdot \Sub(l',Q)\}:
   a\in\At(l',Q)\} \cup \{\{a \cdot \Sub(l'',pQ)\}: a\in\At(l'',pQ)\}\) is
   called the {\em assumption set}.
\end{itemize}
%
%So, the process of unification of overlapping terms produces a set of atoms
%called {\em assumptions} and a clause called critical and consisting of
%obtained bindings and of the critical literal.  
In the ground case \cite{KW} 
we only had the critical literal without any bindings or
assumptions.  The critical clause \(\CC(l',l'',p)\), the assumption set
\(\At(l',l'',p)\) and the unifying substitution \(\Sub(l',l'',p)\) are
now used in the  inference rules.

\subsection{Inference rules}

\begin{description}
\item[Reflexivity resolution]\quad\(\prule {C,s\oplus s'}
  {\{(B,C)\sigma\}\cup \C A}\), 
\quad
where \(\oplus\) is one of \(\notIncl\), \(\notInt\) or \(\notCont\),\\[.5ex]
\C A \(=\At(l,\rev l,\Top)\),
\(B=\Bin(l,\rev l,\Top)\) and
$\sigma$ is a substitution \(\Sub(l,\rev l,\Top)\) for \(l= s\oplus s'\).

\item[Superposition]\quad \(\prule {C,a \qquad D,l}
{(C,D,\CC(a,l,p))\Sub(a,l,p)}\) \quad 
atom \(a\) overlaps literal \(l\) at $p$.

\item[Compositionality resolution]
\quad \(\prule {C,s\oplus t \qquad D,s'\odot u,s''\ominus w}
{(C,t\otimes u,s\odot u)\sigma}\) \quad
 \(\odot = \Comp {\rev \oplus}{\neg\otimes}\) and
\(\sigma=\mgu\{s,s',s''\}\).
\end{description}

\begin{theorem} \label{th:soundness}
The inference system $\C I$ is sound.
\end{theorem}
\begin{proof} 
To prove soundness of inference rules, which use so complicated
unification, is not a trivial task. It consists of two subtask: 1) to prove
soundness of unification on which the rules are based, 2) for each rule, to
demonstrate particular property of predicates which is applied in the rule.
\end{proof}

\subsection{Ordering of words}

We assume the existence of a {\em simplification ordering} `$<$` on 
the ground terms \cite{Der}
to define a more specific proof
strategy for the system $\C I$, to study the possibility of rewriting
wrt. the introduced predicates and, finally, to define the model in the
completeness proof. 
%
%Various
%orderings of terms and atoms are used extensively in the study of automated
%deduction. We will apply such an ordering to define a more specific proof
%strategy for the system $\C I$, to study the possibility of rewriting
%wrt. the introduced predicates and, finally, to define the model in the
%completeness proof. We assume the existence of a {\em simplification
%ordering} `$>$' \cite{Der} on ground terms which is {\em total} (\(s>t\lor
%t>s\), for any ground terms $s,t$), {\em well-founded} (any strictly
%descending sequence of ground terms \(t_1 > t_2 > t_3 > \cdots\) is finite),
%{\em monotone} (\(s>t\Rightarrow u[s]_p>u[t]_p\)) and {\em simplifying} {\it
%i.e.}, \(u[s]_p>s\) for any position \(p\neq \Top\).  
The partial ordering of
non-ground terms is derived from the ordering of ground ones
according to the rule: $u > v \Leftrightarrow u\sigma > v\sigma$
%\begin{equation} \label{eq:ord-non-ground}
%u > v \iff u\sigma > v\sigma
%\end{equation}
for any ground substitution $\sigma$.

Our specific assumption about the orderings is that any
deterministic term (from $\Dterms$) is strictly smaller than any non-variable
non-deterministic term (from \(\Terms\)). Thus
any variable is smaller than non-variable term from
$\Terms$.  But in the set $\Dterms$ we have usual picture of term ordering.

Literals and clauses are identified with multisets and their ordering is defined by the
{\em multiset extension} \cite{DM} of the term ordering.  
%Considering
%multisets over some set $T$ as functions of type \(T\to \Nat\) we use
%\begin{definition} \label{def:multiset-ordering}
%For an ordering `$\Ord$' on a given set $T$, an ordering `\(\M\Ord\)' on the
%set \(T\to \Nat\) is a {\em multiset extension} of `$\Ord$', if
%\[\beta \M\Ord \gamma \iff \forall d\in  T\,\exists c\in T\/  \left( (\beta
%(c)>\gamma (c) \land (\beta (d)\geq \gamma (d)\lor c\Ord d  )\right).\]
%\end{definition}
%In the general case it is known \cite{DM} that `$\M\Ord$' is total if
%`$\Ord$' is total and `$\M\Ord$' is well-founded if `$\Ord$' is well-founded.
%
A literal $s\oplus t$ is represented by the multiset \(\{\{s,\oplus\},
\{t,\rev \oplus\}\}\).  We assume that any term is bigger than any predicate
symbol.  A stronger positive predicate is bigger than a weaker one, the order
between negative predicates is reversed, and all negative predicates are
bigger than the positive ones:
\begin{equation} \label{eq:predicate-order}
\notEq\ >\ \notIncl\ >\ \notCont\ >\ \notInt\ >\ \Eq\ >\ \Incl\ >\ \Cont\ >\
\Int.
\end{equation}

The ordering of the predicates will make the negated form of an atom bigger
than the atom itself.  Whenever possible, we suppose in a written literal
$s\oplus t$ that is not the case \(s < t\). It explains why both signs
`$\Incl$' and `$\Cont$' are used. This rule, of course, is not applied to the
conclusions of the proof rules.  The ordering of literals is the twofold
extension of `$<$' because each literal is a multiset of two multisets.
Clauses are compared as multisets of literals, so their ordering is the
multiset extension of the ordering of literals (threefold multiset extension
of `$<$'). Although we have here three different orderings, we will use the
same symbol `$<$' to denote any of them. This should not introduce any
confusion as the sets of terms, literals and clauses are disjoint.

\subsection {The \strategy\ proof strategy} \label {se:strategy}
%
The proof strategy
which we call \strategy\ srategy 
is known in the equational case as {\em ordered
paramodulation} 
%
%(and others. Check that!)
 
The literals mentioned explicitly in the premises of the proof rules are
called {\em active}. Various ways of selecting the active literals will lead
to different proof strategies. The \strategy\ strategy requires that the
active literals in the premise clauses are the ones which are maximal {\it
wrt}. the ordering defined above.  Stated explicitly the strategy amounts to
the following restrictions on the application of the rules:
\begin{description}\smallerspaces
\item[Reflexivity resolution:] \((s\oplus s')\sigma\) is maximal
in the clause \((C,s\oplus s')\sigma\).
\item [Superposition:] the atom \(a\sigma\) and the literal \(l\sigma\),
where \(\sigma=\Sub(a,l,p)\), are maximal in the respective clauses
\((C,a)\sigma\) and \((D,l)\sigma\).
\item [Compositionality resolution:] the atom \((s\oplus t)\sigma\) is maximal
in the clause \((C,s\oplus t )\sigma\). The maximal atom in the clause
\((D, s'\odot u, s''\ominus w)\sigma\) is \((s''\ominus w)\sigma\), and
\(s'\odot u\) is an atom.
\end{description}
%
The important observation for our proof of completeness concerns the ordering of
clauses in premisses and conclusions of the proof rules.  The
nondeterministic terms from bindings that appear in the conclusions are subterms
of the active literals, and therefore binding literals are smaller than
the active literals from premisses.  So, if other new (``not contained
in premisses'') literals are smaller than the (maximal) active literals
then the conclusion clause is smaller than the maximal of premisses clauses.
Furthermore, new variables may be introduced instead of non-deterministic terms.
But then the assumption that $\Dterms$ (including variables!) 
are smaller than non-variable $\Terms$ makes the conclusion smaller than the 
respective premisses.
%
%\section{Ground literal rewriting}\label{se:Grewrite}
%
%This section mainly repeats the relevant 
%The definitions and lemmas listed here are essentially the same as in \cite{KW}.
%They introduce the concepts and results used in the completeness proof.
%
%\begin{definition} \label{def:rewriting-step}
%A literal $r$ is a {\em rewriting step} in \C L if either \(r\in\C L\), or
%$r$ is an atom \(u[s]_p\oplus u[t]_p\) for some term $u$, a position $p$ in
%$u$, and an atom \(s\otimes t\in \C L\), where \(\oplus\in\{\Incl,\Cont\}\),
%if $\otimes=\Eq$, or $\oplus=\otimes$, otherwise.
%\end{definition}
%
%A sequence of rewriting steps \(\<s\oplus_1 t_1,\: t_1\oplus_2 t_2,\:
%t_2\oplus_3t_3,\: ...\:,\:t_n\oplus_n t\>\) is called a {\em rewriting
%sequence}, the predicate sign of the derived literal \(s\oplus t\) is
%computed using the function \(\Comp\_\_\): \(\oplus=\Comp {\rev {\Comp {\rev {\Comp
%{\rev {\oplus _1}}{\rev {\oplus _2}}}}{\cdots }}}{\oplus _n}\).
%
%\begin{definition} \label{def:rewriting-proof}
%A rewriting sequence is a {\em rewriting-proof} if it does not contain a {\em
%peak} ({\it w.r.t.} to an ordering of terms $<$), {\em i.e.}, a pair of
%consecutive rewriting steps \(s\oplus t\),\(t\otimes u\) such that \(s\leq
%t\geq u\).
%\end{definition}
%
%\begin{definition}\label{def:rewriting-closure}
%For a set \C L of ground literals, the {\em rewriting closure} of \C L is the
%set of ground literals, $\C L^\ast$, defined as follows:
%\begin{itemize}\smallerspaces
%\item  all atoms of the form $s\Incl s$ or $s\Int s$, where $s$ is a ground
%  term, belong to $\C L^\ast$;
%\item if an atom \(s\oplus t\in\C L\) and a literal \(u[s]_p\otimes v\in\C
%  A^\ast\), then the literal \(u[t]_p\odot v\in\C L^\ast\), if
%  \(\odot=\Sup(p,\oplus,\otimes)\);
%\end{itemize}
%\end{definition}
%
%\begin{definition} \label{def:critical-literal}
%A ground rule \(r_1 = s\To\oplus t\) {\em overlaps} a ground rule \(r_2 =
%u[s]_p \To \otimes v\). In this case the literal \(l = u[t]_p \odot v\),
%where \(\odot = \Sup(s,u[s]_p,\oplus,\otimes)\), is called a {\em critical
%literal} formed by the rules \(r_1,r_2\), if $l$ is different from \(r_1\)
%and \(r_2\).
%\end{definition}
%
%\begin{definition} \label{def:confluent-system}
%A set \C R of ground rewriting rules is {\em confluent} if \(\C R^\ast\)
%contains all critical literals formed by overlapping rules from \C R.
%\end{definition}
%
%\begin{definition} \label{def:forcing}
%A set of ground atoms \C A {\em forces}
%\begin{itemize}\smallerspaces
%\item a ground atom  $a$ if \(a\in\C A^\ast\), and the literal \(\neg a\) if 
%\(a\notin \C A^\ast\);
%\item a ground clause $C$ if it forces some literal \(l\in C\);
%\item a non-ground clause $C$ if it forces any ground instance of $C$;
%\item a set of clauses \C S if it forces all clauses from \C S.
%\end{itemize}
%\end{definition}
%
\section{Completeness} \label{se:completeness}
In this section $\C L$ denotes a set of ground literals and $\C A$ a set of ground atmos.
Such sets can be viewed as rewriting systems.
%We record the crucial definitions and lemmas used in the completeness proof:
\begin{definition}\label{def:forcing}
Let $\C L$ be a set of ground literals, and $r,r_1,r_2$ be ground literals
\begin{enumerate}\smallerspaces
\item 
$r$ is a {\em rewriting step} in \C L if either \(r\in\C L\), or
$r$ is an atom \(u[s]_p\oplus u[t]_p\) for some term $u$, a position $p$ in
$u$, and an atom \(s\otimes t\in \C L\), where \(\oplus\in\{\Incl,\Cont\}\),
if $\otimes=\Eq$, or $\oplus=\otimes$, otherwise.
\item
A sequence of rewriting steps \(\<s\oplus_1 t_1,\: t_1\oplus_2 t_2,\:
t_2\oplus_3t_3,\: ...\:,\:t_n\oplus_n t\>\) is a {\em rewriting
sequence} -- the result \(s\oplus t\) is
%computed using the function 
\(\Comp\_\_ : \oplus=\Comp {\rev {\Comp {\rev {\Comp
{\rev {\oplus _1}}{\rev {\oplus _2}}}}{\cdots }}}{\oplus _n}\).
\item
A rewriting sequence is a {\em rewriting-proof} if it does not contain a {\em
peak}
(a pair of consecutive rewriting steps \(s\oplus t\),\(t\otimes u\) such that \(s\leq
t\geq u\).)
\item
The {\em rewriting closure} of \C L is the
set of ground literals, $\C L^\ast$, defined as follows:
\begin{itemize}\smallerspaces
\item  all ground atoms  $s\Incl s$ and $s\Int s$ belong to $\C L^\ast$;
\item if an atom \(s\oplus t\in\C L\) and a literal \(u[s]_p\otimes v\in\C
  A^\ast\), then the literal \(u[t]_p\odot v\in\C L^\ast\), if
  \(\odot=\Sup(p,\oplus,\otimes)\);
\end{itemize}
%
\item
\(r_1 = s\To\oplus t\) {\em overlaps} \(r_2 = u[s]_p \To \otimes v\). 
%Both ground rules !
\(l = u[t]_p \odot v\), where \(\odot = \Sup(s,u[s]_p,\oplus,\otimes)\), 
is the {\em critical literal}, if it is different from \(r_1\)
and \(r_2\).
\item
\C L  is {\em confluent} if \(\C L^\ast\)
contains all critical literals formed by overlapping rules from \C L.
\item
A set of ground atoms \C A {\em forces}
\begin{itemize}\smallerspaces
\item a ground atom  $a$ if \(a\in\C A^\ast\), and the literal \(\neg a\) if 
\(a\notin \C A^\ast\);
\item a ground clause $C$ if it forces some literal \(l\in C\);
\item a non-ground clause $C$ if it forces any ground instance of $C$;
\item a set of clauses \C S if it forces all clauses from \C S.
\end{itemize}
\end{enumerate}
\end{definition}
%
%
\begin{lemma} \label {le:first-rule}
Let \C L be confluent, \(a\in \C L\)
and \(b\in\C L^\ast\), \(b\leq a\) and $b$ can not be derived
without $a$. Then $a$ and $b$ have the same maximal term, say $s$, and, if $P$
is a rewriting proof of $b$ in $\C L$, then the rule $a$
in $P$ rewrites $s$ and is used only once in $P$,
{\em i.e.}, \(P=\<a,P'\>\), where $a$ is not used in $P'$. If \(a\ne b\), then the
literal \(c\), proved by the rest of proof $P'$, is smaller than $b$.
\end{lemma}
%\begin{proof}
%Let \(a=s\otimes t\), \(b=s'\oplus u\), (the first terms not smaller than the
%second ones). The relation \(b<a\) is defined using the multiset
%presentation:
%\[a=\{\{s,\oplus\},\{t,\rev \oplus\}\} \geq
%    b=\{\{s',\otimes\},\{u,\rev \otimes\}\}.\] 
%
%From this it follows that $s'\leq s$.  In the rewriting proof of $b$ all
%terms are not bigger than $s'$, but the rule $a$ can be applied only to a
%term not smaller than $s$ (Condition~O4 of simplification orderings is used
%here). Hence, the first terms must be syntactically identical: $s=s'$. In
%this case second terms yields the same order as literals: \(t\geq u\).
%
%It is obvious, that $a$ can rewrite only the first term of $b$. It could
%rewrite the second term in the case \(s=u\). In this case, by the condition
%of the lemma, \(\otimes\) must be \(\Eq\) or \(\notEq\), other cases mean
%that $b$ is reflexivity literal, accepted or rejected (by consistency
%condition) without any proof.  Formally, the proof of such a trivial atom
%consists of one step $b$ and $a$ is not used at all--- this contradicts the
%condition of the lemma.  In any case, if $b$ is reflexive atom, then $a$ must
%be such one, too. The only possible case is \(b=a\) and lemma is true in this
%case.
%
%A rewriting step occurring before $a$ and preserving the term $s$ may only be
%only reflexive literal. As was noted after the definition of reducing
%sequences, such literals, if they are used in a rewriting proof, have
%smallest terms of the proof. This again gives us the considered case of
%reflexive $b$. The same conclusion, {\em i.e.}, \(b=a\), we get in the case
%of reflexive $a$.  This means, that $a$ in any case is the first step in $P$.
%
%Suppose now, that \(a\ne b\), {\em i.e.}, the proof \(P\) can not consists
%only of one step \(a\). By Definition~\ref {def:rewriting-closure}, the
%literal \(c=t\odot u\), is such that \(\otimes =\Comp {\oplus^{-s}}\odot\).
%In the case \(s>t\), the literal \(b>c\), because \(s>u\). The case \(s=t\)
%means reflexivity of $a$ and \(b=a\), as was noted above.
%\end{proof}
%
%\begin{lemma} \label{le:preserve-confluency}
%For a confluent and consistent system \C R and a rule \(r\notin\C R^\ast\)
%the system \(\C R\cup \{r\}\) is confluent and consistent iff
%\begin{itemize}\smallerspaces
%\item $r$ does not have the form  \(s\To\oplus s\), where
% \(\oplus\in\{\notIncl,\notCont,\notInt\}\),
%\item for any critical literal $l$ formed by any \(r'\in \C R
%\cup\{r\}\) overlapping (or overlapped by) $r$, $l\in \C R^\ast$,
%\end{itemize}
%\end{lemma}
%
%
We are sketching the proof of {\em refutational completeness} of the inference system
\C I, {\em i.e.}, that there exists a model (multialgebra) satisfying all the
clauses from \C S if the empty clause is not derivable from \C S.  We call a
set of clauses \C S {\em consistent} if it does not contain the empty clause.
The main result is

\begin{theorem}\label{completeness}
If a set of clauses \C S is consistent then it has a model.
\end{theorem}
\noindent
The construction proceeds in two main steps. Given a consistent set \C S,
we select a set of atoms \C R %(section~\ref{se:forcing-set})
and show that \C R is a {\em forcing set}\/ for \C S.
%(section~\ref{se:main-R}).
Then %there is left to show show that
\C R can be used to construct a multimodel which satisfies \C S in the
way it was done in \cite{KW}. 

%\subsection{Model}\label{se:forcing-set}\label{se:main-R}

Let \C G denote the set of ground instances of form \C S, {\it i.e.}, of
clauses \(C\sigma\), where \(C\in \C S\) and $\sigma$ is a ground
substitution (involving only $\Fvars$ functional symbols.).  
The starting point of the model construction is the set of
maximal literals of ground instances of \C S :
\begin{equation} \label{eq:max-literals}
\C L_0 \Def \{\max(C) : C\in \C G\}.
\end{equation}
%
For an $\C L$ and an $l\in\C L$, we define the set \(\C
L_l\Def \{a\in\C L:a<l\}\)
% contains all the literal from \C L that are
%smaller than $l$.
%
\begin{definition} \label{def:redundant-clause}\label{def:redundant-literal}
Let $\C A$, $\C G$  be  sets of ground atoms and clauses, respectively, 
$C$ be a ground clause and $l$ a ground literal.
\begin{enumerate}\smallerspaces
\item
\(C\) with \(\max(C)=l\) is {\em redundant} in \C A and \C G if either
\begin{itemize}\smallerspaces
\item $C$ is forced by \(\C A_l\)  or 
\item the set \C G contains another clause \(C'<C\) with \(\max(C')=l\), and
\(C'\) is not forced by \(\C A_l\).
\end{itemize}
%\end{definition}
%
%\begin{definition} 
\item
$l$ is {\em redundant} in \C A and \C G if either
\begin{itemize}\smallerspaces 
\item \(l=s\oplus s'\), where \(\oplus \in \{\notIncl ,\notCont ,\notInt\}\),
  $l$ overlaps \(\rev l\) at the top position, and the clause \(\Bin(l,\rev
  l,\Top)\) is not forced by \(\C A\), or
\item \(\C A\cup\{l\}\) contains a rule $r$ overlapping $l$ at a position $p$,
  and such that the critical clause \(\CC(r,l,p) <\{l\}\) and is not forced
  by \(\C A\), or 
\item every clause $C\in \C G$ with $\max(C)=l$ is redundant in \C A.
\end{itemize}
\end{enumerate}
\end{definition}
\noindent
Definitions~\ref {def:redundant-literal}.2 and \ref {def:forcing}.7
of redundancy and forcing are so related, that all negative literals that are
not forced are redundant. Since any forced literal makes redundant all
clauses containing it, any negative literal appears redundant.

If $\C L_i$ is known, and \(l_i\) is the minimal redundant
literal in \(\C L_i\), we let:
\begin{equation} \label{eq:atoms-model}
\C L_{i+1} \Def \C L_i \setminus \{l_i\}, \hspace{7em}
\C R \Def \bigcap_{i\in \Nat} \C L_i.
\end{equation}
%
Before sketching the proof that \C R is the forcing set for $\C S$, 
we list some of its properties
%A ground clause $C$ is redundant in \C G if it is redundant in \C R and \C G.
%
\begin{lemma} \label{le:redundancy-limit}
\(l\in\C L_0\) (a clause \(C\in \C G\) with \(\max(C)=l\)) is
redundant in some \(\C L_i\) with \(l_i>l\) $\Leftrightarrow$ it is redundant in every
\(\C L_j\) with $j>i$ $\Leftrightarrow$ it is redundant in \C R.
\end{lemma}

\begin{corollary} \label{co:model-confluent}
\C R is confluent.
\end{corollary}

\begin{definition} \label{def:productive}
\(C,l\in \C G\) with \(\max(C,l)=l\) is {\em productive} for $l$
in \C A if \C A does not force $C$.
\end{definition}

\begin{lemma} \label{le:productive-clause}
For any \(l\in\C R\), there is a clause \(C\in\C G\) productive
for $l$ in \(\C R_l\).
\end{lemma}
%\begin{proof}
%If $l\in \C R$ then $l$ is non-redundant. The negated form of the redundancy
%Definition~\ref{def:redundant-literal} is a conjunction including the
%condition that there is a non-redundant clause $C\in \C G$ with $\max(C)=l$.
%Non-redundancy of $C$, {\em i.e.}, negation of
%Definition~\ref{def:redundant-clause} implies that $C$ is not forced by \(\C
%R_l\), what means productivity of $C$ for $l$ in \(\C R_l\).
%\end{proof}
%
For any literal \(l\in\C R\) we denote by \(\C R_{l'}\) the set \(\C
R_l\cup\{l\}\), while for \(l\notin\C R\) \(\C R_{l'}=\C R_l\).

\begin{definition}
Call a set \C G {\em relatively closed} if any application of a
rule from \C I with premises from \C G produces a clause whose each ground
instance is either in \C G or redundant in \C G.
\end{definition}
%
The main technical result, giving the completeness theorem is
%
\begin{theorem} \label{le:main-theorem}
Let \C G be consistent and relatively closed set of ground clauses, \(\C
L_0\) and \C R be as defined by (\ref {eq:max-literals}) and (\ref
{eq:atoms-model}).  Any literal \(l\in \C L_0\) satisfies the following
conditions:
\begin{description}\smallerspaces
\item[I1.] if $l$ is not redundant in \C R, then for any \(a\in\C R_l\) there
  exists a clause \(C\in \C G\) productive for $a$ in \(\C R_l\),
\item[I2.] if $l$ is redundant in \C R, then \(\C R_l\) forces any clause
  \(C\in \C G\) with \(\max(C)=l\).
\end{description}
\end{theorem}
\begin{proof} By contradiction, assuming that $l$ is a minimal literal 
in \(\C L_0\) not satisfying the theorem, i.e., either
%We assume that there exist some literal $l$ not satisfying
%the theorem and suppose $l$ is minimal in \(\C L_0\) with this property.
%Observe that any literal from \(\C L_0\) must satisfy one of the conditions 
%I1 or I2, therefore we have two non-intersecting cases:
\begin{description}\smallerspaces
\item[B1.] $l$, being included in \C R, ``spoils'' productiveness of some 
  clause \(C\in\C G\) with \(a=\max (C) \leq l\), {\em i.e.}, $C$ is
  productive for $a$ in \(\C R_l\), but is not productive in \(\C
  R_l\cup\{l\}\), or
\item[B2.] $l$, being not included in \C R, leaves unforced by \(\C R_l\) 
  some clause \(C\in\C G\) with \(\max (C)=l\).
\end{description}
The following lemma allows us to obtain contradiction in both cases:
\begin{lemma}\label {le:contradiction-way}
The following conditions about a clause $D$ and a literal $l$ cannot be 
satisfied simultaneously in a relatively closed \C G:
\newITEM C
\ITEM{i}{$D$ is a ground instance of conclusion of some proof rule
from \C I with premises from \C G,}
\ITEM{ii}{ for any clause \(D'\leq D\), if \(D'\in\C G\), then \(\C R_l\)
forces $D'$,} 
\ITEM{iii}{ \(\C R_l\) does not force $D$,}
\ITEM{iv}{\(\max(D)<l\).}
\end{lemma}
%\begin{proof}
%The condition \?{i} and relative-closeness of \C G mean that the clause $D$
%must be in \C G or be redundant in \C G.  By \?{iii} and \?{ii}, the clause
%\(D\notin\C G\), so it is redundant in \C G. The redundancy of $D$ in \C G,
%by Lemma~\ref {le:redundancy-limit} and \?{iv}, means redundancy of $D$ in
%\(\C R_l\). The redundancy definition includes two cases: either 
%\ITEM{v}{$D$ is forced by $\C R_l$, or}
%\ITEM{vi}{ $\C G$ contains a clause \(D'<D\) with \(\max(D')=l\) that is not
%forced by $\C R_l$.} The case \?{v} is excluded by \?{iii}. The case \?{vi}
%contradicts \?{ii}.
%\end{proof}
%and
%\begin{corollary} \label{cor:contradiction-way}
%Lemma~\ref {le:contradiction-way} holds if Condition~\?{ii} is replaced by the
%following  statement:  $l$ is the minimal literal in \(\C L_0\) satisfying
%B1 or B2.
%\end{corollary}
%\begin{proof}
%Minimality of $l$ with respect to Condition~B2 means, that for all clauses
%$C\in\C G$ from \(\max(C)<l\) follows that \(\C R_l\) forces $C$. Any clause
%\(D'\leq D\) by \?{iv} has \(\max(D')<l\), and therefore satisfies \?{ii}.
%\end{proof}
%
Assumption B1 means that
 \(\max(C)=a\), and \(C\setminus\{a\}\) is not forced by \(\C R_l\),
but there exists \(b\in C\) such that \(b\neq a\) and \(b\in \C
  R_{l'}^\ast\).
%Productiveness conditions for $C$ in B1 means that
%\ITEM{2}{ \(\max(C)=a\), and \(C\setminus\{a\}\) is not forced by \(\C R_l\),}
%\ITEM{4}{ but there exists \(b\in C\) such that \(b\neq a\) and \(b\in \C
%  R_{l'}^\ast\).}
By Lemma~\ref {le:productive-clause}, there is a clause $D,l$ such that
\(l=\max(D)\) and $D$ is not forced by \(\C R_l\).
%\ITEM{1}{\(l=\max(D)\), $D$ is not forced by \(\C R_l\).}
%We want to prove that the factoring rule can be applied to
%clauses \(D,l\) and $C$. The order of the atoms \(a,b,l\) is important:
%\ITEM{6}{ \(b< a\leq l\) (it follows from \?{2} and B1).}
%The strong inequality between $a$ and $b$ follows from maximality of $a$ in
%$C$. By Lemma~\ref {le:first-rule} applied to atoms \(l\neq b\), we get that,
%if \(l = s\oplus t\) and \(b = s\odot u\), then there exists $\otimes$ such
%that
%\ITEM{FA}{\(\odot = \Comp {\rev \oplus}\otimes\) and \(c = t\otimes u< b\).}
%The first condition in \?{FA} is sufficient to apply the factoring rule to
%\(D,l\) and $C$ and derive the clause \(E = (D,b,\neg c)\).  From \?{1},
%\?{6} and \?{FA} it follows \(\max(E)<l\).  From Lemma~\ref {le:first-rule}
%we also have, that $l$ is used only once in the proof of $b$, hence \(c\in\C
%R_l^\ast\) {\it i.e.}, \C R does not force \(\neg c\).  This condition
%together with \?{1}, \?{2}, and \?{4} implies that $E$ is not forced by \(\C
%R_l\).
Lemma~\ref{le:first-rule} implies that we can apply the factoring rule to
$C$ and $D,l$ producing the clause $E=D,b,\neg c$ with $\max(E)<l$ which
%The literal $l$ and the clause $E$ 
satisfy the conditions of Lemma~\ref
{le:contradiction-way} thus leading to a contradiction. \\
%
\noindent Assuming B2, we have a clause $D$ such that
%That means that there exists a clause $D$ such that
\(C=(D,l)\), \(\max(D)<l\) and $C$ is not forced by $\C R_l$.
%\ITEM{2i}{\(C=(D,l)\), \(\max(D)<l\) and $C$ is not forced by $\C R_l$.}
%Assume that $C$ is minimal with this property.
Then $l$ is redundant and we analyse the three alternatives of
Definition~\ref{def:redundant-literal}.2.
The last one is impossible since $C$ is non-redundant. In the first case,
contradiction follows from Lemma~\ref{le:contradiction-way} after
application of reflexivity resolution. In the second case, 
Lemma~\ref{le:productive-clause} and superposition rule give again a pair
satisfying the conditions of Lemma~\ref{le:contradiction-way}, thus yielding a
contradiction.
%
%The literal $l$ is redundant, and by Definition~\ref
%{def:redundant-literal} there are three alternatives:
%\ITEM{21}{\(l=s\oplus s'\), where \(\oplus \in \{\notIncl ,\notCont
%  ,\notInt\}\), $l$ overlaps \(\rev l\) at the top position, and the clause
%  \(\Bin(l,l',\Top)\) is not forced by \(\C A\),}
%\ITEM{22}{\(\C A\cup\{l\}\) contains a rule $r$ overlapping $l$ at a position
%  $p$, and the critical clause \(\CC(r,l,p) <\{l\}\) is not forced by \(\C
%  A\),}
%\ITEM{23}{ every clause $B\in \C G$ with $\max(B)=l$ is redundant in $\C R_l$.}
%
%The alternative \?{23} is false because $C$ is non-redundant ---
%Definition~\ref {def:redundant-clause} of redundancy subsumes the negated form
%of the minimality assumption about $C$ which we have just made.
%
%In the case of \?{21}, the reflexivity resolution rule can be applied to the
%clause $C$ to produce the clause \(D,\Bin(l,l',\Top)\).  The all bindings
%literals are smaller than $l$ (see note in Section~\ref {se:strategy}, why by
%Lemma~\ref {le:contradiction-way} we derive contradiction in this case.
%
%Let consider the alternative \?{22} and let \((C',r)\) be a productive clause
%for $r$ in \(\C R_r\) (that exists by Lemma~\ref {le:productive-clause}).
%From minimality of $l$ it follows, that no a literal between $r$ and $l$
%destroys productivity of $(C',r)$. (By the way, \(r=l\) is possible.)  So,
%$(C',r)$ is also productive for $r$ in \(\C R_l\):
%\ITEM{PCl}{ \(C'\cap \C R_l^\ast =\es\).}
%By \?{22}, there exists a clause
%\ITEM{28}{\((D' = D,\,C',\CC(r,l,p))\Sub(r,l,p)\)}
%deduced from clauses $D,\,l$ and $C',\,r$ by the superposition rule.  Now,
%all conditions of Lemma~\ref {le:contradiction-way} for $D'$ hold, and
%contradiction follows.
\end{proof}
\noindent
Thus $\C R$ is the forcing set for $\C S$.  The construction
of a multialgebra from $\C R$ is given in \cite {KW}.

\subsection*{Conclusions and directions for future work}

Possible apply rewritings in relations others than congruences.

Possible to prove completeness without the lifting lemma, but, perhaps, some
another lemma can be formulated.

Possible to unify even in the situations where usual substitutions are not
allowed.

On-line skolemisation

Other applications, like ``rewriting logic'', or ``constraint logic
programming''

May be ``basic strategy''

Going to tools

Going to extensions, like relational algebra


\begin{thebibliography}{MM99}\smallerspaces
\bibitem[Bez90]{Bez} M.~Bezem. 
   Completeness of Resolution Revisited. 
   {\em Theoretical Computer Science}, 74, pp.27-237, (1990).
\bibitem[BG91]{BG} L.~Bachmair, H.~Ganzinger. 
   {\em Rewrite-Based Equational Theorem Proving with 
                           Selection and Simplification.}
   Technical Report MPI-I-91-208, Max-Planck-Institut f. Informatik, 
   Saarbr\"ucken, (1991).
\bibitem[BG93]{BG249} L.~Bachmair, H.~Ganzinger. 
   {\em Rewrite Techniques for Transitive Relations.}
   Technical Report MPI-I-93-249, Max-Planck-Institut f. Informatik, 
   Saarbr\"ucken, (1993). [to appear in LICS'94]
\bibitem[BG95]{BG-Oslo} L.~Bachmair, H.~Ganzinger. 
   {\it Ordered Chaining Calculi for First-Order Theories of Binary Relations}.
   Technical Report MPI-I-95-2-009, Max-Planck-Institut f. Informatik, 
   Saarbr\"ucken, (1995).
\bibitem[BGLS93]{Basic-par} L.~Bachmair, H.~Ganzinger, C.~Lynch,  W.~Snyder. 
   {\em Basic paramodulation}.
   Technical Report MPI-I-93-236, Max-Planck-Institut f. Informatik, 
   Saarbr\"ucken, (September1993). [appeared in LNCS???]
\bibitem[DJ90]{Der} N.~Dershowitz, J.-P.~Jouannaud. 
   Rewrite systems. In: J.~van Leeuwen (ed.) 
   {\em Handbook of theoretical computer science}, vol. B,
   chap. 6, pp.243-320. Amsterdam: Elsevier, (1990).
\bibitem[DM79]{DM} N.~Dershowitz, Z.~Manna. 
   Proving termination with multiset orderings. 
   {\em Communications of the ACM}, 22:8,pp.465-476, (1979).
\bibitem[DO92]{DO} A.~Dovier,E.~Omodeo,E.~Pontelli,G.-F.~Rossi. 
   Embedding finite sets in a logic programming language. 
   {\em LNAI}, 660, pp.150-167, Springer Verlag, (1993).
\bibitem[Hes88]{PS1} W.H.~Hesselink. A Mathematical Approach to Nondeterminism
   in Data Types. {\em ACM Transactions on Programming Languages and Systems},
   10, pp.87-117, (1988).
\bibitem[Hus92]{Hus} H.~Hussmann. Nondeterministic algebraic
   specifications and nonconfluent term rewriting. {\em Journal of Logic
   Programming}, 12, pp.237-235, (1992).
\bibitem[Hus93]{HusB} H.~Hussmann. 
   {\em Nondeterminism in Algebraic Specifications and Algebraic Programs.}
   Birkh\"auser Boston, (1993).
\bibitem[Jay92]{Jay} B.~Jayaraman. Implementation of Subset-Equational 
   Programs. {\em Journal of Logic Programming}, 12:4, pp.299-324, (1992).
\bibitem[Kap88]{Kap} S.~Kaplan. Rewriting with a Nondeterministic Choice
   Operator. {\it Theoretical Computer Science}, 56:1, pp.37-57, (1988).
\bibitem[KW94]{KW} V.~Kriau\v ciukas, M.~Walicki.  Reasoning and Rewriting
   with Set-Relations I: Ground-Completeness.  In {\it Proceedings of
   CSL'94}, LNCS~933, pp.~264--278. Also: Report no.96, Dept. of Informatics,
   University of Bergen (1994).
\bibitem[LA93]{LA} J.~Levy, J.~Agust\'i. Bi-rewriting, a term rewriting
   technique for monotonic order relations. In {\em RTA'93, LNCS}, 
   690, pp.17-31. Springer-Verlag, (1993).
\bibitem[Mos89]{uni-al} P.D.~Mosses. Unified algebras and institutions. In
   {\it LICS'89, Proc. 4th Ann. Symp. on Logic in Computer Science},
   pp.~304--312, IEEE, (1989).
\bibitem[SL91]{relaxed-par} W.~Snyder, C.~Lynch. Goal directed strategies for
   paramodulation. In {\it Proc. 4th Int. Conf, on Rewriting Techniques and
   Applications}, Lect. Notes in Comput. Sci., vorl.~488, pp.~150--161,
   Berlin, 1991. Springer-Verlag.
\bibitem[PP91]{PP} J.~Pais, G.E.~Peterson. Using Forcing to Prove Completeness
   of Resolution and Paramodulation. {\em Journal of Symbolic Computation}, 
   11:(1/2), pp.3-19, (1991).
\bibitem [S-A92]{S-A} R.~Socher-Ambrosius. 
   {\em Completeness of Resolution and Superposition Calculi.}
   Technical Report
   MPI-I-92-224, Max-Planck-Institut f. Informatik, Saarbr\"ucken, (1992).
\bibitem [SD86]{SD} J.~Schwartz,R.~Dewar,E.~Schonberg,E.~Dubinsky. 
   {\em Programming with sets, an introduction to SETL. }
   Springer Verlag, New York, (1986).
\bibitem[Sto93]{Sto} F.~Stolzenburg. 
   {\em An Algorithm for General Set Unification.}
   Workshop on Logic Programming with Sets, ICLP'93, (1993).
\bibitem[Wal93]{Mich} M.~Walicki. 
   {\em Algebraic Specifications of Nondeterminism.}
   Ph.D. thesis, Institute of Informatics, University of Bergen, (1993).
\bibitem[WM95a]{MW-II} M.~Walicki, S.~Meldal. Multialgebras, Power algebras
   and Complete Calculi of Identities and Inclusions. In {\it Recent Trends
   in Data Type Specification, Selected papers of joint 10th ADT and 5th
   COMPASS workshops, S.~Margherita, Italy, May 30 - June 3, 1994},
   Springer-Verlag, LNCS~906, pp.~453--468 (1995).
\bibitem[WM95b]{MW} M.~Walicki, S.~Meldal. A Complete Calculus for 
   Multialgebraic and Functional Semantics of Nondeterminism. 
   {\it ACM Trans. on Programming Languages and Systems}, vol.~17, N0.~2,
   pp.~366--393 (1995).
\end{thebibliography} 
\end{document}
