\documentstyle[a4wide]{article}
%%\documentstyle[11pt,a4wide]{article}
% these pages are quite small. These are Springer LNCS pages for 12pt
%\textwidth 16cm \textheight 23.5cm
%%\textwidth 16cm \textheight 24cm
\voffset=-.5in
%\hoffset=-.7in


\newtheorem{example}{Example}[section]
\newtheorem{definition}[example]{Definition}
\newtheorem{theorem}[example]{Theorem}
\newtheorem{lemma}[example]{Lemma}
\newtheorem{fact}[example]{Fact}
\newtheorem{proposition}[example]{Proposition}
\newtheorem{corollary}[example]{Corollary}

%equalities
\newcommand{\eequal}{\stackrel{e}{=}}%existential equality
\newcommand{\sequal}{\stackrel{s}{=}}%strong equality
\newcommand{\wequal}{\stackrel{w}{=}}%weak equality
\newcommand{\eleq}{\mathrel{\dot{=} }}%element equality
\newcommand{\seq}{\mathrel{\asymp}}% set equality 

\newcommand{\exist}{\downarrow}% defindnes in partial algebra
\newcommand{\st}{\ast}% extra star element


%different Categorical, arrows...
\newcommand{\embd}{\hookrightarrow}% embedding of institutions 
\newcommand{\pfunc}{\hookrightarrow}% partial function
\newcommand{\ring}{\circ}% composition of arrows

\newlength{\listskip}\setlength{\listskip}{1ex plus .2ex minus .5ex}
\newlength{\eqpskip}\setlength{\eqpskip}{.5ex plus .2ex minus .2ex}
\newenvironment{eqp}{\par %\topsep\eqpskip \parskip 0ex
   \begin{tabbing}
   \quad\=\+\quad\=$\{$\ \=\kill\(}{\)\end{tabbing}
   \addvspace{-1ex plus.2ex minus.3ex}} 
\newcommand{\comment}[2]{\)\-\\[\eqpskip]$#1$\>\>$\{$\>\+\+\ignorespaces#2$\}$\-\\[\eqpskip]\(}
\newcommand{\nline}{\)\\\(\mbox{}}

%%%%%%% MICHAL

%%% comments to be removed eventually
\newcommand{\fix}{\par\noindent\hspace*{-2em}{\large\bf fix} $\uparrow$ \dotfill}
\newcommand{\fixx}[1]{\par\noindent\hspace*{-2em}{\large\bf fix} $\uparrow$ {\small
#1} \dotfill}
\newcommand{\fixd}[1]{\par\noindent\hspace*{-2em}{\large\bf fix} $\downarrow$ {\small
#1} \dotfill}
\newcommand{\isit}[1]{\par\noindent\hspace*{-2em}{\small{{\bf [is it right?} #1{\bf ]}}}\par\noindent}
\newcommand{\how}[1]{\par\noindent\hspace*{-2em}{\small{{\bf [how?} #1{\bf
]}}}\par\noindent}
\newcommand{\todo}[1]{\par\noindent\hspace*{-2em}{\small{{\bf [To Do:} #1{\bf ]}}}\par\noindent}
\newcommand{\noo}[1]{\par\noindent\hspace*{-2em}{\large\bf NO} $\uparrow$ {\small
#1} \dotfill}
%%% environments
\newtheorem{schem}[example]{Schema}
\newcommand{\MyLPar}{\parsep -.2ex plus.2ex minus.2ex\itemsep\parsep
   \vspace{-\topsep}\vspace{.5ex}}
\newenvironment{PROOF}{{\bf Proof.}}{\nopagebreak\finish}
\newenvironment{PROOFs}{{\bf Proof (sketch)}}{\nopagebreak\finish}
\newcommand{\finish}{\hspace*{\fill}\nopagebreak 
     \raisebox{-1pt}{$\Box$}\par\addvspace{1.5ex}\noindent}

%%% Specs:
\newcommand{\spec}[1]{\begin{array}[t]{rrl}#1\end{array}\vspace*{1ex}}
\newcommand{\tit}[1]{\multicolumn{3}{l}{#1}}

%%% symbols
\newcommand{\eeq}{\eequal}
\newcommand{\PSet}[1]{{\mathcal{P}}(#1)}
\newcommand{\ovr}[1]{\overline{#1}}
\newcommand{\To}{\Rightarrow}
\newcommand{\Tod}{\Leftarrow}
\newcommand{\Iff}{\Longleftrightarrow}
\newcommand{\hviss}{\Iff}
\newcommand{\ok}[1]{d_{#1}} % OK part of a sort #1

\newcommand{\by}[2]{\stackrel{#1}{#2}}
\newcommand{\Toby}[1]{\ \by{#1}{\Longrightarrow}\ }

\newcommand{\adj}{\mathrel{\small\dashv}} %adjunction


%%% use these for signatures
\newcommand{\Sorts}{{S}} 
\newcommand{\Ops}{\Omega}
\newcommand{\ndc}{\Pi}
\newcommand{\err}{E}
\newcommand{\sign}{(\Sorts,\Ops,\ndc)}
\newcommand{\POps}{P\Omega}
\newcommand{\Pops}{\POps}
\newcommand{\Terms}[1]{T(#1)}
\newcommand{\TermsS}{\Terms\Sigma}
\newcommand{\TermsSX}{\Terms{\Sigma,X}}
\newcommand{\dom}{{\bf dom}}

%%% general categorical concepts
\newcommand{\cat}[1]{{\bf #1}} %put it around any category 
\newcommand{\inst}[1]{{\mathcal{#1}}} %put it around any institution: needs $_$
\newcommand{\fu}[1]{{\sf {#1}}} %put it around any functor
\newcommand{\thr}[1]{{\bf #1}} %put it around any Theory
\newcommand{\obj}[1]{|#1|} %objects of a category
\newcommand{\natt}{\mathrel{\Longrightarrow}} %natural transformation
%%% Categories
%multialgebras
\newcommand{\MA}[1]{\cat{MAlg}_{#1}}
\newcommand{\MAS}{\cat{MAlg}_{\Sigma}} %partial multialgebras
\newcommand{\PMA}[1]{\cat{PMAlg}_{#1}}
\newcommand{\PMAS}{\cat{PMAlg}_{\Sigma}} %partial algebras
\newcommand{\PAl}[1]{\cat{PAlg}_{#1}}
%used little but still...
\newcommand{\PAlo}{\PAl{(\Sorts,\Ops)}}
\newcommand{\MAlo}{\MA{(\Sorts,\Ops,\emptyset)}}
%standard
\newcommand{\Sign}{\cat{Sign}}
\newcommand{\Set}{\cat{Set}}

%%% Functors
\newcommand{\Mod}{\fu{Mod}}
\newcommand{\Sen}{\fu{Sen}}
\newcommand{\sen}{\Sen}

%%% Institutions
\newcommand{\MAH}{\inst{MAH}}
\newcommand{\MAC}{\inst{MAC}}
%\newcommand{\MAPC}{\inst{MAPC}}
\newcommand{\MAP}{\inst{MAP}}%partial multialgebras
\newcommand{\PA}{\inst{PA}}%partial algebras

%%% Equivalences 
\newcommand{\quot}{\sim} % ekvivalens tegn
\newcommand{\kernel}[1]{\quot_{#1}} %% ekvivalens med subskript
\newcommand{\qu}[2]{#1/\!_{#2}}

%%% some special functions used in specifications
\newcommand{\ite}[3]{\mathit{if}\ #1\ \mathit{then}\ #2\ \mathit{else}\ #3}%if then else
% and
\newcommand{\band}{\mathrel{\mathit{and}}}% ``and''
\newcommand{\choice}{\sqcup} % Nondeterministic choice



\title{Complete Reasoning and Rewriting with Set-Relations}

%changes:  interlines spaces decreased
\author{ {\it Valentinas Kriau\v ciukas\thanks{Both authors gratefully 
acknowledge the financial support received from the Norwegian Research Council.}}\\[-.5ex]
\small Department of Mathematical Logic\\[-.5ex]
\small Institute of Mathematics and Informatics\\[-.5ex]
\small Vilnius, LITHUANIA\\[-.5ex]
\footnotesize valentinas.kriauciukas@mlats.mii.lt 
\and 
{\it Micha{\l} Walicki\/\(^\ast\)}\\[-.5ex]
\small Department of Computer Science\\[-.5ex]
\small University of Bergen\\[-.5ex]
\small Bergen, NORWAY\\[-.5ex]
\footnotesize michal@ii.uib.no}


\begin{document}

\maketitle
\abstract {The paper investigates reasoning with set-relations: intersection, 
inclusion and identity of 1-element sets. A language is introduced which, 
interpreted in a multialgebraic semantics, allows one to specify such 
relations. 
Each of these set-relations satisfies only two among the 
three properties of the equivalence relations - we study rewriting with such 
non-equivalence relations and point out differences from the equational case. 
The introduced complete inference systems
generalize ordered paramodulation and superposition calculi.  Notions
of rewriting proof and confluent rule system are defined for such
non-equivalence relations.
 Together with the notions of forcing and redundancy
they are applied  in the completeness proofs.  

In the first part, an inference system is given and shown sound and 
refutationally ground-complete for 
a particular proof strategy which selects only maximal literals from the premise 
clauses. 
As a corollary of the ground-completeness theorem we obtain 
ground-completeness of the introduced rewriting technique.

The second part studies rewriting with the non-ground clauses.
Ground completeness cannot be lifted to the non-ground case
because substitution for variables is restricted to
deterministic terms.
To overcome the problems of restricted
substitutivity and hidden (in relations) existential quantification,
unification is defined as a three step process: substitution of determistic 
terms, introduction of bindings and ``on-line'' skolemisation.
The inference rules based on this unification derive non-ground clauses
even from the ground ones, thus making an application of a standard
lifting lemma impossible. The completness theorem is proved directly without 
use of such a lemma.}


\noindent{\bf Keywords:}
rewriting, theorem proving, binary relations, nondeterminism, 
completeness, ordered superposition



%\documentstyle[a4wide]{article}
%\makeatletter
%\show\


\section{Introduction}

Reasoning with sets becomes an important issue in different areas of computer 
science. Its relevance can be noticed in constraint and logic programming e.g. 
\cite{SD,DO,Jay,Sto}, in algebraic approach to nondeterminism e.g. 
\cite{HusB,PS1,MW}, in term rewriting e.g. \cite{LA,Kap,HusB}.

Our interest in the set concepts originates from an earlier study of 
specifications of nondeterministic operations. Such operations are naturally 
modelled as set-valued functions. The semantic structures serving this purpose 
- {\em multialgebras} - generalize the traditional algebras allowing operations 
which, for a given argument, return not necessarily a single value but a set of 
values (namely, the set of all possible values returned by an arbitrary 
application of the operation). In \cite{MW,Mich} we defined a specification 
language using set-relations and its multialgebraic semantics. The 
set-relations we considered were: inclusion, intersection and identity of 
1-element sets. The first two are the usual set relations. Inclusion allows one 
to define set equality which, for that reason, is not included in the language. 
The third relation is particularly important: it provides the syntactic 
\mbox{means}
of distinguishing between sets and their elements, and is indispensable for 
obtaining a complete reasoning system. Such a system is also given in the above 
works.

In the present paper we use the same set-relations but introduce a new 
reasoning system -- first for the ground case, and then for the general, 
non-ground case.
% It is less general than the earlier one - we are studying only 
% the ground case - but it is much more prone to automation. 
Rewriting with 
non-congruence relations becomes also an issue of increasing importance. The 
set-relations we are considering are not even equivalences: equality is 
symmetric and transitive (but not reflexive), inclusion is reflexive and 
transitive (but not symmetric) and intersection is reflexive and symmetric (but 
not transitive). We study the rewriting proofs in the presence of these 
relations generalizing several classical notions (critical pair, confluence, 
rewriting proof) to the present context. Our results on rewriting extend
bi-rewriting of Levy and Agusti \cite{LA} in that we consider three different 
set-relations. We also take a step beyond the framework of Bachmair and 
Ganzinger \cite{BG249} in that we study more general composition of relations 
than chaining of transitive relations.


Section~\ref{se:nd-specs} defines the syntax and the multialgebraic semantics 
of the language and lists some basic properties of the set-relations. 
Section~\ref{se:reasoning} introduces the reasoning system \C I, ordering of 
words and specifies the {\em maximal literal} proof strategy for using \C I.
Section~\ref{se:rewrite} discusses term rewriting with the introduced 
set-relations. In section~\ref{se:completeness} we prove the first main 
theorem - refutational
ground completeness of \C I with the maximal literal strategy and, as a simple 
corollary, ground completeness of rewriting.

The second part of the paper, sections~\ref{se:ng} through \ref{se:completenessNG},
extend these results to the non-ground case.
In a standard completeness proof one can utilize
completeness of
the ground case by establishing a lifting lemma.
It amounts to the fact 
that a conclusion of an inference rule applied to ground
instances of clauses is a ground instance of the conclusion of the same rule
applied to the clauses.  
In our case, this cannot be done.
For the first, some rules do
not preserve groundness of clauses. The same is the case for
relaxed paramodulation rule \cite {relaxed-par} and therefore
authors constructed there a syntactic proof of completeness.  
Secondly, % as shown by Example~\ref{famous}, 
ground instances of a contradictory set of clauses may happen to be consistent.
The problem is the {\em lack}
of deterministic constants and functions, which could be used to
form enough  ground terms.
We give a direct semantic proof of completeness not using a lifting lemma.
The same method can be also applied in the case of relaxed paramodulation.

Section~\ref{se:ng} introduces the definition of interpretation
of non-ground expressions and lists the implied refinements of some other notions
from the first part.
%properties of the set-relations.  
Section \ref{se:unification} discusses the
non-standard difficulties with and defines the unification of
nondeterministic terms.  Section~\ref{se:reasoningNG} introduces the
reasoning system \C J, extending \C I, and specifies the generalization of
the \strategy\ proof strategy for using \C J.
%Section~\ref{se:Grewrite} discusses ground term rewriting with the introduced 
%set-relations which forms the basis for t
A sketch of the completeness proof is presented in
Section~\ref{se:completenessNG}.

%changes
The present paper is an improved and shortened version of the report \cite{KW}.
Because of the space limitations, we only include the proofs of the central 
lemmas and theorems.

\section{Specifications of set-relations} \label{se:nd-specs}

% \subsection{Syntax}
Specifications are written using a finite set of function symbols $\Funcs$
having arity $ar:\Funcs\to \Nat$.\footnote
{We are treating only the unsorted case --
extension to many sorts is straightforward.}
A symbol $f\in\Funcs^0$ %with $ar(f)=0$ is
is called a {\em constant}. In this part, we consider only ground case, and so we do not
introduce any variables (see sections following~\ref{se:ng} for the non-ground case).  
We denote by $\GTerms$ the set of all (ground)
terms.  There are only three atomic forms using binary predicates: {\em
equation} $s\Eq t$, {\em inclusion} $s\Incl t$ and {\em intersection} $s\Int
t$. A {\em specification} is a set of {\em clauses}
%changes
--- finite sets of {\em
literals}, where a {\em literal} is an atom or a negated atom written \(\neg 
a\).
%changes
%CHANGE
 (In \cite{MW,Mich} a 
restricted language is used, allowing only negated intersections, positive 
inclusions and equations in clauses.) We will usually write negated atoms 
explicitly as $s\notEq t$, $s\notIncl t$ and $s\notInt t$, and 
assume \(\neg(\neg a)=a\).
%
By
{\em words} we will mean the union of the sets of terms, literals and
clauses.  We will write $u[s]_p$ to denote that a term $s$ is a subterm of a
term $u$ at a position $p$. Often the position will be omitted for the sake of
simplicity.
%CHANGE  - removed from here and moved to the begining of Section 5.
%

Syntactic expressions of the language are interpretated in {\em multialgebras}
\cite{Kap,Hus,Mich}.
\begin{DEFINITION}
An $\Funcs$-{\em multialgebra} $A$ is a tuple \(\<S^A,\Funcs^A\>\) where $S^A$
is a non empty {\em carrier} set, and $\Funcs^A$ is a set of set-valued functions
\(f^A: (S^A)^{ar(f)}\to\C P^+(S^A)\), where \(f\in \Funcs\), and \(\C P^+(S^A)\)
is the power-set of \(S^A\) with the empty set excluded.
\end{DEFINITION}
We are dealing with total multialgebras and therefore exclude the empty set.
Its admission, for instance for modelling partiality as in \cite{BK},
%Admission of the empty set for modelling partiality (as it is done, for
%instance, in \cite{BK}), 
would require modification of the
inference system we are introducing in this paper.

Defining the meaning of the words we follow \cite{MW,Mich}.
\begin{DEFINITION} \label{def:semantics}
An interpretation
\(\Interpret d\) of any expression $d$ of the language is defined as follows: 
\begin{itemize}\MyLPar
\item \(\Interpret c \Def c^A\), if $c$ is a constant;
\item \(\Interpret{f(\List tn,)} \Def \bigcup\{f^A(\List\alpha n,):
  \alpha_i\in\Interpret {t_i}\}\) 
for any \(f\in \Funcs^n\) and \(\{\List
  tn,\}\subset \GTerms\);
\item \(\Interpret{s\Eq t}\) is true if \(\Interpret s=\Interpret
  t=\{\alpha\}\) for some $\alpha\in S^A$, and false otherwise;
\item \(\Interpret{s\Incl t}\) is true if \(\Interpret s\subseteq\Interpret
  t\), and false otherwise;
\item \(\Interpret{s\Int t}\) is true if \(\Interpret s\cap \Interpret
  t \not = \es\), and false otherwise;
\item for an atom $a$, \(\Interpret{\neg a}\) is true if \(\Interpret{a}\) 
 is false, and is false otherwise;
\item \(\Interpret{\List an,}\) is true if some
  \(\Interpret{a_i}\) is true, and false otherwise.
% or some \(\Interpret{b_i}\) is true, and false  otherwise.
% (so, empty clause is always false). 
\end{itemize}
\end{DEFINITION}
% We say that a multialgebra $A$ {\em satisfies} an atom $a$ (a clause $C$) if
% \(\Interpret a\) (respectively, \(\Interpret C\)) is true, and $A$ {\em
% satisfies} a specification \C S if it satisfies all clauses in \C S.
Definition~\ref {def:semantics} implies that for each \(f\in \Funcs\), \(f^A\)
is \(\subseteq\)-monotone (because it is defined by pointwise extension).
Interpretation of a constant $c$ is, according to the definition of
multialgebra, a non-empty set. Observe also that equality is not reflexive ---
\(t\Eq t\) is not true in general. A term $t$ for which this equality is true
is called {\em deterministic} because then it has only one possible value. The
equality is merely a symmetric and transitive relation. An inclusion \(s\Incl
t\) means that the term $s$ has the value set which is included in the value
set of $t$. This relation is a partial preorder --- it is transitive and
reflexive, but not symmetric. The intersection is reflexive (because of
nonemptiness of term values) and symmetric, but lacks the transitivity
property. Thus each of these relations satisfies two of the three properties
of equivalence relations.  Now we present some other properties of these
relations.\\[8pt]

\subsection{Basic properties of literals}

The following relation expresses equality of term value sets, and is the usual
 interpretation of equality
 in the set-valued approach to nondeterminism \cite{PS1,Kap}. 
\MyLPar \begin{equation}\MyLPar \label{eq:Seteq-definition}
s\Seteq t\Def s\Incl t\land s\Cont t.
\end{equation}
As can be expected, it does not increase 
expressibility and therefore is not used in the language.
For a discussion of the intended meaning and difference 
between `$\Eq$' and `$\Seteq$' in the context of nondeterminism 
see \cite{MW,Mich}. 

The positive, resp. negative, relations are totally ordered by strength:
\MyLPar\begin{equation}\MyLPar \label{eq:rel-order}\label{eq:rel-orderNG}
u\Eq v \impl u \Seteq v \impl u\Incl v \impl u\Int v
\hspace{2em} and \hspace{2em}
 u\notEq v \Leftarrow u \notSeteq v \Leftarrow
 u\notIncl v \Leftarrow u\notInt v 
\end{equation}
Derivations and lemmas below refer always to the strongest possible relation.

Replacement of terms --
``equals by equals'' -- is possible only in equations, nevertheless
the following lemmas will allow later to develop techniques of term-rewriting.
($u[t]_p$ denotes term $u$ with term $t$ substituted at the position $p$.)

\begin{LEMMA} \label{le:replacement}%\label{le:replacementNG}
The following properties hold for the introduced predicates:

\(s\Eq t \impl   u[s]_p \Seteq  u[t]_p, \ \ 
s\Seteq t  \impl  u[s]_p \Seteq  u[t]_p, \ \ 
s\Incl t  \impl  u[s]_p \Incl  u[t]_p, \ \   
s\Int t  \impl  u[s]_p \Int  u[t]_p.\) 
\end{LEMMA}
In general, we will need to perform replacement of a (sub)term $s$ occurring in
one literal $u[s]\otimes v$ by a term $t$ related to $s$ in another literal
$s\oplus t$. 
Rules for such a replacement of terms in literals, related to the appearance 
of {\em critical peaks} \cite{Der} and generation of {\em
critical pairs}, are given in the following lemma.

 \begin{LEMMA} \label{le:replacement-in-atoms} The defined predicates
satisfy the rules given in Table~\ref {tbl:replacement}.
\begin{table}[hbt]
\[\begin{array}{|c||c|c|c|c|c|c|c|c|}
\hline
 \Repl\_\_   & u[s]\Eq v  \rule{0pt}{9pt} \rule{0pt}{-2pt}  & u[s]\Incl v & u[s]\Cont v & u[s]\Int v & u[s]\notEq v & u[s]\notIncl v & u[s]\notCont v & u[s]\notInt v\\
\hline
\hline
s\Eq t \rule{0pt}{9pt} \rule{0pt}{-2pt} & u[t]\Eq v   & u[t]\Incl v & u[t]\Cont v & u[t]\Int v & u[t]\notEq v & u[t]\notIncl v & u[t]\notCont v & u[t]\notInt v\\
\hline
s\Incl t \rule{0pt}{9pt} \rule{0pt}{-2pt} & u[t]\Cont v & u[t]\Int v  & u[t]\Cont v & u[t]\Int v & u[t]\notEq v & u[t]\notIncl v & u[t]\notEq v   & u[t]\notIncl v \\
\hline
s\Cont t \rule{0pt}{9pt} \rule{0pt}{-2pt} & u[t]\Eq v   & u[t]\Incl v & -           & -          & -            & -              & u[t]\notCont v & u[t]\notInt v \\
\hline
s\Int t \rule{0pt}{9pt} \rule{0pt}{-2pt} & u[t]\Cont v & u[t]\Int v  & -           & -          & -            & -              & u[t]\notEq v   & u[t]\notIncl v \\
\hline 
\end{array}\vspace{-1ex}\]
\caption{Rules for subterm replacement} \label{tbl:replacement}
\end{table}
\end{LEMMA}
%
The table may be encoded as a partial function \[
\Repl \_\_: (s\oplus
t\land u[s]_p\otimes v\impl u[t]_p \Repl \oplus \otimes v)\] 
for any terms $s,t,u,v$ and position $p$ at $u$.

When $u[s]=s$, we may use a similar table for {\em chaining} or {\em composing}
relations, e.g., $s\Eq t\land s\Incl u \impl t\Incl u$. 
%We write it as $\Comp\Eq\Incl = \Incl$.  
Notice that in lemma \ref{le:replacement} the predicate `$\Eq$'
was not inherited after substitution unlike the other two. Thus,
sometimes, composition may produce stronger result than replacement.
For instance, \(s\Eq t\land t\Cont u\impl s\Eq u\)
(the table would yield $s\Cont u$).
The next lemma describes the possibilities of such a chaining the introduced relations.  
%
\begin{LEMMA} \label{le:composition}
The introduced predicates satisfy the composition properties, giv\-en in
Table~\ref {tbl:composition}.\vspace{-2ex}
\begin{table}[hbt]\MyLPar
\[\begin{array}{|c||c|c|c|c|c|c|c|c|}
\hline
\Comp\_\_        & s\Eq u   & s\Incl u & s\Cont u & s\Int u  
& s\notEq u   & s\notIncl u & s\notCont u & s\notInt u\\
\hline \hline
s\Eq t    & t\Eq u   & t\Incl u & t\Eq u   & t\Incl u 
& t\notCont u & t\notInt u  & t\notCont u & t\notInt u\\
\hline 
s\Incl t  & t\Cont u & t\Int u  & t\Cont u & t\Int u  
& t\notEq u   & t\notIncl u & t\notEq u   & t\notIncl u \\
\hline 
s\Cont t  & t\Eq u   & t\Incl u & -        & -        
& -           &    -        & t\notCont u & t\notInt u\\
\hline 
s\Int t   & t\Cont u & t\Int u  & -        & -        
& -           &    -        & t\notEq u   & t\notIncl u\\
\hline 
\end{array}\]\MyLPar
\caption{Rules for literal composition} \label{tbl:composition}\vspace{-1ex}
\end{table}
\end{LEMMA}
%
For convenience we will write the
partial function coded in this table as \(\Comp\oplus\otimes=\odot\), meaning
that $\odot$ is the strongest relation obtained by composing \(\oplus\) and
\(\otimes\) for any terms, {\em i.e.}:
\[\Comp\oplus\otimes=\odot\iff\forall s,t,u: t\rev\oplus s\land s\otimes u\impl t\odot u\] 
%for any terms $s,t,u$.
Note that the table defines only the strongest
composite of the arguments. Because of the ordering (\ref{eq:rel-order}) 
the fact that, for instance, \(\Comp\Eq\notInt=\notInt\)
will imply that also \(\notIncl\) can be obtained from composing \(\Eq\) and 
\(\notInt\).

Composition of negative and positive atoms is symmetric to the composition
of the positive and the negative ones given in the table. Composition of
two negative atoms does not allow one to draw any conclusion and therefore
is not mentioned at all.
%
\begin{LEMMA} \label {le:composition-transitivity}
The composition function \(\Comp\_\_\) is transitive. 
\end{LEMMA}
%
The two tables differ in predicate signs at four places --- 1:3 - 1:6
(row 1, columns 3 through 6), where the relation resulting from
Table~\ref{tbl:composition} is stronger than the one from
Table~\ref{tbl:replacement}.  These cases must be distinguished when the 
superposition rule (see below) is applied. Therefore we introduce the
function \(\Sup(p,\oplus,\otimes)\), which joins the two by selecting the appropriate
table. Its 
value is \(\Comp\oplus\otimes\) if $p$ is the top position, $u[s]=s$, 
and \(\Repl\oplus\otimes\), otherwise.
\[
\Sup(p ,\oplus ,\otimes) = \cases{\Comp {\oplus} \otimes, & if $p$ is the top position, \cr  \Repl \oplus \otimes, & otherwise.}
\]
$\Sup$(erposition) of negative and positive atoms is symmetric to the
superposition of the positive and the negative ones given in the
table. Superposition of two negative atoms does not allow one to draw any
specific conclusion and therefore is not mentioned at all.


\section{The Inference System \C I} \label{se:reasoning}

% \subsection{Proof rules}
The following set of rules was constructed in analogy to the inference
systems for first-order predicate calculus with equality \cite{BG,S-A}.
However, there are some additional restrictions due to the composition laws
as compared with the equational case.
Very similar rules are presented in \cite{BG249} for transitive relations.

\begin{tabular}{r@{\ :\ \ }l} 
%\PROOFRULE
{R\bf eflexivity resolution} & 
{\quad \(\prule{C, s\oplus s }{C}\),
where \(\oplus\in\{\notIncl,\notCont,\notInt\}\).} \\[3ex]
%
%\PROOFRULE
{\bf Superposition} & {\quad \(\prule
{C,s\oplus t \qquad D,u[s]_p\otimes v}{C,D,u[t]_p\odot v}\),
where   \(\odot=\Sup(p,\oplus,\otimes)\).}
\\[3ex]
%\PROOFRULE%{Contextual factoring
{\bf Compositionality resolution} & {\quad \(\prule
{C,s\oplus t \qquad D,s\odot u}{C,t\otimes u,s\odot u}\),
where \(\odot = \Comp{\rev\oplus}{\neg\otimes}\). } \\[2ex]
\end{tabular}
\\
The analogous rule for equality called {\em equality factoring} \cite{BG,S-A}
is a special case of our rule when both premise clauses coincide. In 
\cite{BG249} analogous rule is called {\em transitivity resolution}.

Let $\C I$ denote the inference system consisting of the above rules. 
\begin{THEOREM} \label{th:soundness}
The inference system $\C I$ is sound.
\end{THEOREM}
\begin{PROOF}
Soundness of reflexivity resolution follows from reflexivity of
 `$\Incl$' and `$\Int$'. 
Soundness of superposition is a direct consequence of the replacement
and composition laws (Lemmas~\ref{le:replacement-in-atoms}, \ref{le:composition}).
Soundness of the
compositionality
rule is based on the following short deduction. Suppose that the first premise 
clause and the implication
\(s\oplus t \land \neg(t\otimes u)\impl s\odot u\)
both are true. The later is equivalent to
\(s\oplus t \impl t\otimes u \lor s\odot u \).
A single application of the (usual) resolution rule gives the conclusion of the 
rule. The second premise clause is not used in this step --- it only shows the
goal atom.

\end{PROOF}

\subsection{Ordering of words and the proof strategy}\label{sub:ordMax}

Various orderings of terms and atoms are used extensively in the study of 
automated deduction. We will apply such an ordering to define a more specific
proof strategy for the system $\C I$, to study the possibility of rewriting
wrt. the introduced predicates and, finally, to define the model in the
completeness proof. We assume the existence of 
a {\em simplification ordering} `$>$' \cite{Der} on ground terms which is
\begin{itemize}\MyLPar
\item {\em total} (\(\forall s\not=t\in\GTerms : s>t\lor t>s\)), 
\item {\em well-founded} (\(\forall t\in\GTerms : \{s:s<t\}\) is finite), 
\item {\em monotone} (\(\forall u,s,t\in\GTerms :s>t\Rightarrow u[s]>u[t]\)) 
and 
%CHANGE
\item {\em increasing} (\(\forall s\in\GTerms : u[s]\not= s \Rightarrow u[s]>s\)).
\end{itemize}
Ordering of other words is defined by the {\em multiset extension} \cite{DM} of this
ordering.
Let
$\Mset T$ denote the set of all finite multisets of elements from $T$. Each
element of $\Mset T$ can be represented by a function \(\beta:T\to \Nat\)
such that \(\beta\equiv 0\) except for some finite number of elements of $T$.
\(\beta(d)\) is a number of copies of $d$ in the multiset $\beta$.
\begin{DEFINITION} \label{def:multiset-ordering}
For an ordering `$\Ord$' on a given set $T$, an ordering `\(\M\Ord\)' on the
set \(\Mset T\) is a {\em multiset extension} of `$\Ord$', if
\[\beta\M\Ord\gamma\iff\forall d\in  T\,\exists c\in T\/ \left((\beta(c)>\gamma(c)
\land (\beta(d)\geq \gamma(d) \lor c\Ord d)\right).\]
\end{DEFINITION}
In the particular case of total ordering of $T$,
% \[T=\{t_1\Ord t_2\Ord\cdots\},\] 
which is the only one considered here, \(\alpha\M\Ord\beta\) 
means that there is some $c\in T$ such that:
\(\alpha(c)>\beta(c)\land \forall d\Ord c\; \alpha(d)=\beta(d).\)
This is a lexicographic ordering comparing biggest components first. In the
general case it is known \cite{DM} that `$\M\Ord$' is total if `$\Ord$' is
total and `$\M\Ord$' is well-founded if `$\Ord$' is well-founded.

% \subsubsection{Ordering of atoms and clauses}
Writing a literal $s\oplus t$, we indicate that \(s\geq 
t\). It explains why both
signs `$\Incl$' and  `$\Cont$' are used. This rule, of course, is not applied to the
conclusions of the proof rules. We assume that any
term is bigger than any predicate symbol.
A stronger positive predicate is bigger than a weaker one, the order between 
negative predicates is reversed, and all negative predicates are bigger than 
the positive ones:
 \begin{equation} \label{eq:predicate-order}
\notEq\ >\ \notIncl\ >\ \notCont\ >\ \notInt \ >\ \Eq\ >\ \Incl\ >\ \Cont\ >\ \Int.
 \end{equation}
The ordering of the predicates will make the negated form of an atom bigger
than the atom itself.  Whenever possible, we suppose in a written literal
$s\oplus t$ that  \(s < t\) is not the case. It explains why both signs
`$\Incl$' and `$\Cont$' are used. This rule, of course, is not applied to the
conclusions of the proof rules. 

By analogy with the commonly used approach in equational reasoning, we
identify literals with multisets.
%changes
A literal
$s\oplus t$ is represented by the multiset
\(\{\{s,\oplus\}, \{t,\oplus^{-1}\}\}\), 
% and negation of this atom by the multiset 
% \(\{\{s,s,\oplus\}, \{t,t,\oplus^{-1}\}\}.\) 
%This distinction  makes the negated form of an atom
%bigger than the atom itself. Even more, the negated atom is bigger than
%any atom with the same maximal term. 
The ordering of the predicates will make the negated form of an atom bigger
than the atom itself.

The ordering of literals is the twofold extension of `$<$' because each
literal is a multiset of two multisets. The biggest literal in a clause $C$ w.r.t.
this ordering is denoted by \(\max(C)\).  Clauses are compared as multisets of
literals, so their ordering is the multiset extension of the
ordering of literals (threefold multiset extension of `$<$'). Although
we have here three different orderings, we will use the same symbol `$<$' to
denote any of them. This should not introduce any confusion as the
 sets of terms, literals and clauses are disjoint.

\subsubsection{The \strategy\ proof strategy}\label{sub:Max}
This proof strategy is known in the equational case as {\em ordered
paramodulation}.
The literals
mentioned explicitly in the premises of the proof rules are called
{\em active}. Various ways of selecting the active
%changes
literals will lead to
different proof strategies. The {\em maximal literal} strategy
requires that the active
%changes
literals in the premise clauses are the ones
which are maximal wrt. the ordering defined above.
Stated explicitly the strategy amounts
to the following restrictions on the application of the rules:
\begin{description} %\MyLPar
\item[Reflexivity resolution:] the literal \(s\oplus s\) is maximal 
in the premise clause.
\item [Superposition:]
%changes
the atom \(s\oplus t\) and the literal \(u[s]_p\otimes v\)
are maximal in their clauses.
\item [Compositionality resolution:] the atom \(s\oplus t\) is maximal
in its clause. The atom \(s\odot u\) {\bf is not} maximal in 
the second premise clause,
but the term $s$ {\bf is} maximal in this clause,  
and the maximal literal in this clause is positive. Both \(\oplus\) and
\(\odot\) are positive.
\end{description}
%CHANGE
The restriction on the last rule is the only case where some active atom
($s\odot u$) is not maximal in its clause. However, it is almost
maximal because the maximal term $s$ of the clause occurs in it.  Another
reason why this weakening of the strategy is not essential is that 
the second
clause in the premise provides merely the context allowing application
of the rule, and in fact the atom $s\odot u$ is not so ``active''. 
A particular consequence of this restriction is that the rule can be applied only
when its second premise is a non-Horn clause.


\section{Rewriting proofs} \label{se:rewrite}

In the next section we show that if the empty clause can not be deduced using 
the maximal literal strategy, then a model exists satisfying the initial set of 
clauses. The model is constructed from an appropriate set of ground atoms 
which force all the initial clauses to be true. The notion of forcing requires 
construction of a deductive closure of a given set of literals. This section 
investigates the rewriting proofs in which ground literals are rewritten to 
ground literals. The obtained results will serve as a basis for the 
construction of forcing set in the completeness proof.

%CHANGE
Although, eventually, only atoms will be used in the model construction 
 we give
a more general account -- our definitions and lemmas
apply to rewriting with both negative and positive literals.
We let \C A denote a set of atoms and \C L a set of literals (in this and the next 
sections, these are ground atoms, resp. literals).

Rewriting of literals with the set-relations is based on the fact that the 
relations satisfy  replacement properties from Lemma~\ref 
{le:replacement-in-atoms}. For example, the implication: \(s\Int t\impl
u[s]_p \Int  u[t]_p\) means that the atom \(u[s]_p \Int u[t]_p\) can be derived 
applying the rule \(s\To\Int t\) to the term \(u[s]_p\).
%changes
%CHANGE
The following definition states what kind of literals can be derived directly 
applying the replacement property to some set \C L of ground literals (also
called {\em axioms}). 
%As a matter of fact, only atoms can be derived 
%with nontrivial applications of this property, and only such case will be
%needed in the proof of completeness. 
\begin{DEFINITION} \label{def:rewriting-step}
A literal 
$r$ is a {\em rewriting step} in \C L if either 
\begin{enumerate}\MyLPar
\item \(r\in\C L\), or 
\item $r$ is an
atom \(u[s]_p\oplus u[t]_p\) for some term $u$, a position $p$ in $u$, and an
atom \(s\otimes t\in \C L\), where \(\oplus\in\{\Incl,\Cont\}\), if
$\otimes=\Eq$, or $\oplus=\otimes$, otherwise.
\end{enumerate}
% $r$ is a {\em rewriting step} in \C A if either 1) \(r\in\C A\), or 2)
% $r$ is an atom $u[s]_p\oplus u[t]_p$,
% and  \(s\otimes t\in \C A\), where $\oplus=\otimes$, if
% $\otimes\not=\Eq$, or \(\oplus\in\{\Incl,\Cont\}\), otherwise.
\end{DEFINITION}
%CHANGE
The rule based
directly on this kind of term-rewriting is superposition.
The forcing set in the completeness proof will consist of ground 
positive literals,
which can be derived using only this one rule.
% this rule is the only rule applicable to derive from them new literals.
%changes
The superposition rule takes two rewriting steps and composes them into one. 
Consequtive applications of the superposition correspond to composition 
of the finite 
sequence of the corresponding rewriting steps \(\<s\oplus_1 t_1,\: t_1\oplus_2 
t_2,\: t_2\oplus_3t_3,\: ...\:,\:t_n\oplus_n t\>\).
Such a sequence is called a {\em rewriting sequence}, 
and the predicate sign $\oplus$ of the resulting literal is
computed using the function \(\Comp\_\_\): \(\oplus=\Comp {(\Comp {(\Comp 
{\oplus _1}{\oplus _2})}{\cdots })}{\oplus _n}\). The next 
definition puts all such literals into {\em rewriting closure} of $\C L$. This 
closure also contains atoms that are trivially true.
\begin{DEFINITION}\label {def:rewriting-closure}
For a set \C L of ground literals, the {\em rewriting closure} of \C L is the set
of ground literals, $\C L^\ast$, defined as follows:
\begin{itemize}\MyLPar
\item  all atoms of the form $s\Incl s$ or $s\Int s$ belong to $\C L^\ast$;
\item if an atom \(s\oplus t\in\C L\) and a literal \(u[s]_p\otimes v\in\C L^\ast\),
then the literal \(u[t]_p\odot v\in\C L^\ast\), if 
\(\odot=\Sup(p,\oplus,\otimes)\).
\end{itemize}
\end{DEFINITION}
%changes: removed graph+what follows
%It follows from Lemma~\ref{le:replacement} that rewriting steps with equality
%predicate can not be produced by replacement. This means that equality
%steps, if they appear in proofs, are equality axioms and are
% used without replacement.
Primarily, we are interested in {\em reducing} rewriting sequences, {\em i.e.}, 
such that rewriting is used to produce terms of lower complexity in some 
well-founded ordering.  The term ordering is used to orient literals but it
does not allow us to orient the {\em reflexive} literals of the form $s\oplus s$.
%changes
%(there no sense to use other reflexive literals in rewriting sequences)
However, the orientation problem of these particular literals
turns out to be inessential for the following arguments (except the next 
definition), and so we allow them to have orientation that is appropriate for 
the context in which it is used. A literal \(s\oplus t\) can be written in the 
form \(s\To\oplus t\) to emphasize that $s\geq t$, then it is also called a {\em
rule}. The fact that this literal is derived by a rewriting sequence in which 
terms do not increase in any step is written as \(s \TTo \oplus t\) or (the 
same) as \(t \oTT{{\rev\oplus} \rlap{${}$}}s\).
\begin{DEFINITION} \label{def:reducing-proof}
A rewriting sequence is {\em reducing} (w.r.t. to an ordering of terms
$<$) if it does not contain a {\em peak}, {\em i.e.}, a pair of consecutive
rewriting steps \(s\oplus t,t\otimes u\) such that \(s\leq t\geq u\).
A {\em rewriting proof} is a reducing rewriting sequence.
\end{DEFINITION}
%CHANGE
The non-strong inequalities in the last condition capture the cases of
reflexive steps. A rewriting step \(s\oplus s\) does not form a peak
in a rewriting sequence only at a locally minimal point,
{\em i.e.}, in a rewriting proof where $s$ is the smallest term.
Definition~\ref {def:reducing-proof} means that any reducing proof consists of 
two decreasing branches like \(s\TTo{}u\oTT{}t\), or has only one \(s\TTo{}t\) 
or \(s\oTT{}t\). The table from Lemma~\ref{le:composition} can be written as a 
summary of all the possible combinations of the resulting predicate signs 
appearing in two-branches rewriting proofs:
\[\begin{array}%
 {@{(s}c@{u}c@{t\lor s}c@{u}c@{t\lor s}c@{u}c@{t)\ \impl\ s}c@{t}c}
\TTo\Eq   & \oTT\Eq   & \TTo\Eq   & \oTT\Cont & \TTo\Incl & \oTT\Eq   & \Eq   &,\\
\TTo\Eq   & \oTT\Incl & \TTo\Eq   & \oTT\Int  & \TTo\Incl & \oTT\Incl & \Incl &,\\
\TTo\Cont & \oTT\Cont & \TTo\Cont & \oTT\Eq   & \TTo\Int  & \oTT\Eq   & \Cont &,\\
\TTo\Cont & \oTT\Int  & \TTo\Cont & \oTT\Incl & \TTo\Int  & \oTT\Incl & \Int  &.
\end{array}\]
Lemma~\ref{le:replacement-in-atoms} describes how the peaks can be eliminated
from rewriting sequences. Let us take, for example, one implication from this lemma:
\(s\Int t\land u[s]\Eq v\impl u[t]\Cont v.\)
The premise can be interpreted as a possibility to have a peak \(u[t] \oT\Int
u[s] \To\Eq v \) in proofs, if both atoms \(s\Int t\) and \(u[s]\Eq t\) are
axioms. This peak can be ``cut down'' changing it by the consequence
\(u[t]\Cont v\), if it is also among the axioms. The following notions are
commonly used in similar situations. 
%
\begin{DEFINITION} \label {def:critical-atom}
A rule \(r_1 = s\To\oplus t\) {\em overlaps} a rule \(r_2 = u[s]_p \To \otimes 
v\). In this case the literal \(l = u[t]_p \odot v\), where \(\odot = 
\Sup(p,\oplus,\otimes)\), is called a {\em critical literal} formed by 
the rules \(r_1,r_2\), if $l$ is different from \(r_1\) and \(r_2\).
\end{DEFINITION}
Critical literals correspond to {\em critical pairs} from equational reasoning 
\cite{Der}. In our case the definition is more complex because the predicate 
sign is important and replacement is not merely of ``equals by equals''. Also, 
when the rule \(r_1\) is
%changes
reflexive (which is not necessarily a tautology in our case) then
the critical literal $l$ may be the same as \(r_2\). It is better to exclude
such cases because they would complicate our model construction.
\begin{DEFINITION} \label{def:confluent-system}
A set \C L of ground literals (rewriting rules) is {\em confluent} if $\C L^\ast$ 
contains all critical literals formed by overlapping rules from \C L.
\end{DEFINITION}
In term-rewriting theory \cite{Der} such systems are called {\em 
locally-confluent}. Confluent systems have slightly different definition, but 
both these notions are proved to be equivalent.  In \cite{LA} a similar 
definition introduces bi-confluent systems.

In completeness proofs like ours, {\em fully-reduced} \cite{PP} or {\em
left-reduced} \cite{S-A,BG} rewriting systems are used.  We are not able to
define the analogous notion, since deduction and reduction 
% (see the proof of Theorem~\ref{th:soundness}) 
are not the same in our language, and will apply
Definition~\ref{def:confluent-system} instead. Its direct consequence is
\begin{LEMMA} \label{le:proofs-in-confluent}
Any literal derivable by a rewriting sequence in a confluent system \C L has
a rewriting proof in \C L.
\end{LEMMA}
% \begin{PROOF}
% Using induction on rewriting sequences as multisets of terms ordered by a
% multiset extension of the term ordering. 
%
% When composition of literals is stronger than replacement,
% the question may arise, why the definition of critical literals does not take in
% account these cases. The answer is that in
% these cases both rules overlap each other, and we have two possible cases.
% One of them gives the same result as one got by composition.
% \end{PROOF}

%changes
%CHANGE
Since we have allowed both negative and positive literals to occur in 
one set of axioms $\C L$,  the 
unpleasant situation, when both an atom $a$ and its negation $\neg a$ 
belong to
$\C L^\ast$, is possible. The set $\C L$ is {\em consistent} if no such 
atom 
exists. A set containing only atoms is obviously consistent. 
Although such a set will be of main importance in the following section,
we again formulate stronger results, 
taking into account the general situation of possibly inconsistent
sets of literals.
%sets of that kind are considered in the rest of the paper, we
%present here more general results than we need further.
Next lemma, to be used in the completeness proof to construct confluent 
systems incrementally, characterizes rules that can be added to 
a confluent and consistent system preserving both these properties.
Here we  have again the situation different  from the usual 
equational reasoning, because
%changes
the rule \(s\Eq t\) overlaps itself and produces the critical atom \(t \Eq t\) 
which need not be always true. 

\begin{LEMMA} \label{le:preserve-confluency}
For a confluent and consistent system \C L and a rule $r\notin\C L^\ast$ the 
system \(\C L\cup \{r\}\) is confluent and consistent iff
%changes
\newITEM{coco}
\ITEM{triv}{$r$ does not have the form  \(s\To\oplus s\), where
 \(\oplus\in\{\notIncl,\notCont,\notInt\}\),}
\ITEM{critical}{ for any critical literal $l$ formed by any $r'\in \C L
%changes
\cup\{r\}
$ overlapping (or overlapped by) $r$, $l\in \C L^\ast$,}
\end{LEMMA}
%
%changes: new proof
%\begin{PROOF} We first prove a simpler case --- that the lemma conditions
%imply
%\ITEM{simpler}{\(\neg r\notin\C R^\ast\),}
%and later will show that the general case is reducible to this one. Unfolding
%rewriting closure Definition~\ref {def:rewriting-closure} we get from
%\?{simpler} that either
%\ITEM{refl}{$\neg r$ has the form $s\Incl s$ or $s\Int s$, or}
%\ITEM{crit}{there are an atom $s\oplus t\in \C R$ and a literal \(l=
%   u[s]\otimes v\in\C R^\ast\) such  that \(\neg r=u[t]\odot v\), where \(\odot
%   =\Sup (s,u[s],\oplus,\otimes )\).}
%The case \?{refl} is excluded by \?{triv}. In the case of
%\?{crit} we may assume \(u[t]>v\) and \(t\geq s\), because the rewriting
%sequence deriving \(\neg r\) must be reducing. By \?{critical}, the critical literal
%\(u[s]\ominus v\) produced by $r$ and the rule \(t\oplus^{-1} s\) belongs to
%\(\C R^\ast\), here \(\ominus=\Sup(t,u[t],\oplus^{-1},\neg\odot)\). This
%critical literal exists, because \?{crit} means implication \(s\oplus t \land
%l \impl \neg r\), which is equivalent to \(s\oplus t \land r \impl \neg
%l\). In most cases \(\neg l=u[s]\ominus v\), {\em i.e.}, \(\neg \otimes
%=\ominus\), but sometimes the predicate \(\ominus\) may be stronger than
%\(\neg\otimes\). Both cases contradict consistency of $\C R$:  in the first one
%both \(\neg l\) and $l$ are in \(\C R^\ast\), in the second one the rewriting
%sequence \(\<v\ominus ^{-1} u[s], u[s]\otimes v\>\) proves one of literals
%\(v\notIncl v\) or \(v\notInt v\).
%
%Let now consider the general case: both some atom \(a= u\oplus v\) and its
%negation \(\neg a=u(\neg\oplus)v\) are in \((\C R\cup\{r\})^\ast\). Assume that
%$a$ is minimal. If \(u>v\), then concatenation of rewriting sequences deriving
%\(v\oplus^{-1} u\) and \(u(\neg\oplus)v\) gives the sequence proving the
%literal \(v\notIncl v\) (check Table~\ref {tbl:composition}) that is smaller
%than $a$. So, \(v=u\). Let assume
%\ITEM{noteq}{ \(\oplus\ne\Eq\).}
%The atom $a$ is in the rewriting closure by definition, and $r$ is used only in
%the proof of \(\neg a\). Since \(\C R\cup \{r\}\) is obviously confluent,
%\(\neg a\) has a rewriting proof. If it consists of two branches \(u\TTo{} z\) and
%\(z\oTT{} u\) with \(z<u\), then the sequence \(z\oTT{}u\TTo{} z\) (with
%these branches interchanged) derives \(z\notInt z\) or \(z\notIncl
%z\). To be sure in that is enough to check nine cases in Table~\ref
%{tbl:composition}. Both atoms are smaller than $a$, what contradicts minimality
%of $a$.
%
%In ``flat'' rewriting sequences (whit all steps being reflexive
%literals) nontrivial rewriting steps are only \(u\Eq u\) and \(u\notEq u\).
%One of these literals must be $r$, the another must belong to \C R.
%
%In the opposite case to \?{noteq}, the similar consideration based on
%minimality of atom $a$ shows that both literals  \(a=u\Eq u\) and \(\neg a=u\notEq u\)
%must have ``flat'' proofs. Only in this case proof branches are not
%interchanged, but composed with the sequence proving another literal. In the case
%of ``flat'' proofs $r$ is $a$ or \(\neg a\), so \(\neg r\in\C R\).
%\end{PROOF}
%
Finally, we have the following technical result to be 
used in the next section. It is the only place where the ordering of predicates
has direct application.
 \begin{LEMMA} \label {le:first-rule} 
If: \C L is consistent and confluent, \(a\in \C L\), an atom $b$, with \(b\leq a\), has a
rewriting proof $P$ in \C L, but \(b\notin(\C L\setminus \{a\})^\ast\). 

 Then:
$a$ and $b$ have the same maximal term, $s$, and the proof $P$ has the
form $a,P'$, where $a$ is not used in the proof $P'$. If \(a\ne b\), then the
literal derived by $P'$ is smaller than $b$.
 \end{LEMMA}
%
%\begin{LEMMA} \label {le:first-rule}
%Let \C L be a confluent and consistent system of ground literals, \(a\in \C L\)
%and \(b\in\C L^\ast\) be literals, \(b\leq a\) and $b$ can not be derived
%without $a$. Then $a$ and $b$ have the same maximal term, say $s$, and, if $P$
%is a rewriting proof of $b$ in $\C L$, then the rule $a$
%in $P$ rewrites $s$ and is used only once in $P$,
%{\em i.e.}, \(P=\<a,P'\>\), where $a$ is not used in $P'$. If \(a\not= b\), 
%then the literal \(c\), proved by the rest $P'$ of the proof $P$, 
%is smaller than $b$.
%\end{LEMMA}
\begin{PROOF}
Let \(a=s\otimes t\), \(b=s'\oplus u\), (the first terms not smaller than the
second ones). The relation \(b<a\) is defined using the multiset
presentation:
\[a=\{\{s,\oplus\},\{t,\rev \oplus\}\} \geq
    b=\{\{s',\otimes\},\{u,\rev \otimes\}\}.\] 
From this it follows that $s'\leq s$.  In the rewriting proof of $b$ all
terms are not bigger than $s'$, but the rule $a$ can be applied only to a
term not smaller than $s$ (Condition~O4 of simplification orderings is used
here). Hence, the first terms must be syntactically identical: $s=s'$. In
this case second terms yields the same order as literals: \(t\geq u\).

It is obvious, that $a$ can rewrite only the first term of $b$. It could
rewrite the second term in the case \(s=u\). In this case, by the condition
of the lemma, \(\otimes\) must be \(\Eq\) or \(\notEq\), other cases mean
that $b$ is reflexivity literal, accepted or rejected (by consistency
condition) without any proof.  Formally, the proof of such a trivial atom
consists of one step $b$ and $a$ is not used at all--- this contradicts the
condition of the lemma.  In any case, if $b$ is reflexive atom, then $a$ must
be such one, too. The only possible case is \(b=a\) and lemma is true in this
case.

A rewriting step occurring before $a$ and preserving the term $s$ may only be
only reflexive literal. As was noted after the definition of reducing
sequences, such literals, if they are used in a rewriting proof, have
smallest terms of the proof. This again gives us the considered case of
reflexive $b$. The same conclusion, {\em i.e.}, \(b=a\), we get in the case
of reflexive $a$.  This means, that $a$ in any case is the first step in $P$.

Suppose now, that \(a\ne b\), {\em i.e.}, the proof \(P\) can not consists
only of one step \(a\). By Definition~\ref {def:rewriting-closure}, the
literal \(c=t\odot u\), is such that 
%%%WAS: \(\otimes =\Comp {\oplus^{-s}}\odot\).
\(\otimes =\Comp {\oplus}\odot\).
In the case \(s>t\), the literal \(b>c\), because \(s>u\). The case \(s=t\)
means reflexivity of $a$ and \(b=a\), as was noted above.
\end{PROOF}



\input{cslProof}

\section{Transition to the Non-Ground Case}\label{se:ng}
We recast the definitions and concepts from section~\ref{se:nd-specs} introducing
variables. The interpretation of terms and clauses in a multialgebra is a
strightforward extension of the definition~\ref{def:semantics}. Notice that
variables are assigned only {\em individual} elements of the carrier -- not sets
thereof.
%\section{Syntax and Semantics}\label{se:nd-specsNG}
%% Specifications are written using a countable set of variables $\Vars$ and a
%% finite non-empty set of functional symbols $\Funcs$ having arity
%% \(ar:\Funcs\to \Nat\).\footnote{To simplify the notation we are treating only
%% the unsorted case. Extensions to many sorts are straightforward.}  An
%% \(f\in\Funcs\) with \(ar(f)=0\) is a {\em constant}.  Terms $\Terms$
%% over signature $\Funcs$ are defined in the usual way. 
%% 
%% There are only three atomic forms of formulae built using binary predicates: 
%% {\em equation} $s\Eq t$, {\em inclusion} $s\Incl t$ and {\em intersection} 
%% $s\Int t$. Atoms and their negations form the set of {\em literals}. An atom 
%% $a$ is a {\em positive} literal, and a negated atom $\neg a$ is a {\em 
%% negative} literal. Negative literals are written as $s\notEq t$, 
%% $s\notIncl t$ and $s\notInt t$.  \(\rev{s\oplus t}\) is the reversed literal
%% \(t\rev \oplus s\), which is different from \(t\oplus s\) only in
%% the cases \(\oplus\in\{\Incl,\notIncl,\Cont,\notCont\}\), because these signs
%% are not symmetric by their shape and meaning.
%% A {\em clause} is a finite set of literals,
% ({\em i.e.}, multiplicity and ordering of the atoms do not matter), 
%% a {\em specification} is a set of {\em
%% clauses}.  
%% 
%% By {\em words} we will mean the union of the sets of terms, literals and
%% clauses.  \(\VV w\) denotes the set of variables occuring in a word $w$.
%
%\subsection{Semantics}
%
%% Words are interpretated in {\em multialgebras}
%% \cite{Kap,Hus,Mich} which, unlike usual algebras, allow
%% functions to have multiple values.
%% \(\true\) and  \(\false\) denote the boolean values.
%
%% 
%% \begin{DEFINITION}
%% A $\Funcs$-{\em multialgebra} $A$ is a pair \(\<S^A,\Funcs^A\>\) where
%% $S^A$ is a non empty {\em carrier} set, and $\Funcs^A$ is a set of
%% set-valued functions \(f^A: (S^A)^{ar(f)}\to\C P^+(S^A)\), where \(f\in
%% \Funcs\), and \(\C P^+(S^A)\) is the power-set of \(S^A\) with the empty set
%% excluded.
%% \end{DEFINITION}
%
%
 \begin{DEFINITION} \label {def:semanticsNG}
Let $A$ be a \(\Funcs\)-multialgebra, \(\upsilon:\Vars \to S^A\) be an
interpretation of variables, then value \(\Value w\) of a word $w$ is
defined for :
\begin{enumerate}%\smallerspaces
\item a variable \(x\in\Vars\), \ \(\Value x \Def \upsilon(x)\);
  \label {semantics-v}
\item a constant $c\in\Funcs$, \ \(\Value c \Def c^A\);
\item a term \(t=f(\List tn,)\in \Terms\), \
  \(\Value t \Def \bigcup_{{\alpha_i} \in \Value{t_i}} f^A(\List{\alpha}n,)\);
  \label {semantics1}
\item a literal \(l=s\oplus t\), \ \(\Value l \Def
  F_\oplus(\Value s, \Value t)\), where
  \vspace{1ex}\newline \(
  \begin{array}{r@{\ \equiv\ } l@{\quad}r@{\ \equiv\ } l}
   F_{\Eq}(U,V)  & \forall \alpha\in U\; \forall \beta\in V\;\alpha=\beta, &
   F_{\Incl}(U,V)& \forall \alpha\in U\; \exists \beta\in V\;\alpha=\beta,\\
   F_{\Int}(U,V) & \exists \alpha\in U\; \exists \beta\in V\;\alpha=\beta, &
   F_{\neg\otimes}(U,V) & \neg F_\otimes(U,V);
  \end{array}\)
  \label {semantics3}
\item a clause \(C\), \ \(\Value C \Def 
  \bigvee_{l\in C} \Value{l} \) if \(C\neq \emptyset\), and \(\Value C \Def
  \false\) if $C=\emptyset$.
  \label {semantics4}
\end{enumerate}
\noindent
The multialgebra $A$ {\em satisfies} a clause $C$ if \(\Value C =
\true\) for every interpretation \(\upsilon:\Vars \to S^A\).
It {\em satisfies} a literal $l$ iff it satisfies the clause
\(\{ l\}\), and {\em satisfies} a specification \C S iff it satisfies all
the clauses in \C S.
\end{DEFINITION}


\subsection{Term types $\exists$ and $\forall$ in literals.}
%
The point~\ref {semantics3} of Definition \ref{def:semanticsNG} involved
existential quantification in some predicates.  The meaning of atoms is
defined according to the following schema: $\Value{s\oplus t} = Q^\oplus_1
\alpha\in \Value s\, Q^\oplus_2\beta \in \Value t\, \alpha = \beta,$ where
\(Q^\oplus_i\in \{\forall ,\exists \}\). 
We thus say that (the occurrence of) the term $s$ {\em has type} $Q_1^\oplus$ and
of $t$ the type $Q_2^\oplus$.
% It explains why terms in literals can have types $\forall$ or $\exists$.
There are four possibilities of arranging the quantifiers $Q_1^\oplus
Q_2^\oplus : \forall\forall$ corresponds to $\oplus=\Eq$, $\forall\exists$ to
$\Incl$ and $\exists\exists$ to $\Int$.  

The fourth $\exists\forall$, say
$\Nid$, has not been used but it can be defined as 
\(s\Cont t\land t\Eq t\), so by strength it is
between \(s\Cont t\) and \(s\Eq t\). 
(This is the exact counterpart of the relation `$:$' used in {\em unified algebras} 
\cite {uni-al}.)
 It defines \(s\Eq t\) in the form \(s\Din t\land s\Nid t\), like
`$\Incl$' defines `$\Seteq$' in (\ref {eq:Seteq-definition}). However,
`$\Din$' and `$\Nid$' taken separately cannot model `$\Eq$', like `$\Incl$'
and `$\Cont$' do with `$\Seteq$'.  It is enough to check Table~\ref
{tbl:composition} to see that any result of composition of an atom \(s\Seteq
t\) with some other atom can be obtained by composition of either \(s\Incl
t\) or \(s\Cont t\), what justifies our ignoring of `$\Seteq$'.  In the case
of `$\Din$' and `$\Eq$', composition of \(r\Incl s\) with \(s\Din t\) gives
\(r\Din t\), with \(s\Nid t\) gives nothing, but with \(s\Eq t\) produces
\(r\Eq t\), a stronger relation than two previous.  Of course, from \(r\Din
t\) and \(s\Nid t\) it follows that \(r\Eq t\) because of determinism of $t$ implied
by \(s\Nid t\).  The composition rules cannot help here -- some others would have 
to be used.


%%\subsection{Basic properties of atoms.}
%% 
%%  \begin{LEMMA}\label{le:replacementNG}
%% The following term replacement properties hold: % for the introduced predicates:
%% \begin{equation}\label{eq:rep}
%% s\Eq t  \impl   u[s]_p \Seteq  u[t]_p,\ \ \ \ \ \ 
%% s\Incl t  \impl  u[s]_p \Incl  u[t]_p,\ \ \ \ \ \  
%% s\Int t \impl  u[s]_p \Int  u[t]_p.
%% \end{equation}
%%  \end{LEMMA}
%
% The following lemma describes  the rules for replacement of
% subterms in literals. It is related to the appearance of
% {\em critical peaks} \cite{Der} and generation of {\em
% critical pairs}.
%%  The differences occur in row 1, columns 3 through 6, where the result of chaining will be
%%  \begin{equation}\label{eq:chain}
%%  \Comp\Eq{\Cont}=\Eq,\ \ \ \ \ \ 
%%  \Comp\Eq\Int =\Incl,\ \ \ \ \ \ 
%%  \Comp\Eq\notEq =\notCont,\ \ \ \ \ \ 
%%  \Comp\Eq\notIncl=\notInt. 
%%  \end{equation}
%%  \noindent
%%  It is easy to see that $\Comp\_\_$ is transitive (so the first relation is
%%  reversed comparing with entries of Table~\ref {tbl:replacementNG}).
%%  We join the two operations into one: 
%
%
%
\subsection{Problems with unification}\label{se:unification}
%
%\subsubsection{Deterministic terms, substitutions, frontiers and skeletons.}
%
There are several things hindering us from the
application of the usual unification techniques and we begin here with a brief example illustrating these difficulties. \\[1ex]
%
\noindent 1) 
The essential feature of calculus of nondeterministic operations is
unsoundness of unrestricted substitution. Terms and variables denote objects
of different kind: variables always mean single elements, while terms mean
sets of possible values even if values of their variables are fixed.  This
excludes unification of terms by substitutions.  Therefore, we apply rules
like {\em relaxed paramodulation} \cite {relaxed-par}, which make terms
identical by ``cutting out'' some subterms, but which introduce new literals
into derived clauses, like in the following derivation:
\[
\prule {y\notIncl g(c); \quad h(x,x)\Incl g(x)}{y\notIncl h(x,x), x\notInt c}
\] 
%Here \(s\Incl t\) means that each possible value of $s$ is also a possible
%value of $t$, \(s\Int t\) means that $s,t$ have a common possible value,
(Comma between literals means disjunction and $x,y$ are the only variables.)
The variable $x$ can not be replaced by $c$ in \(h(x,x)\Incl g(x)\) because
\(h(c,c)\Incl g(c)\) is equivalent to \(x\Int c\land y\Int c\then h(x,y)\Incl
g(c)\), but not to \(x\Int c\then h(x,x)\Incl g(c)\), when $c$ has more than
one possible value.  Fortunately, \(y\notIncl g(c)\) is equivalent to \(x\Int
c\then y\notIncl g(x)\), what is used in the derivation.  Literals 
\(x\notInt c\), where $x$ is a variable, are called {\em bindings}.\\[1ex]
%
\noindent 2) 
Another complicating circumstance is that
if some nondeterministic term occured instead of $y$ in the left premise,
the derivation, although  correct, would in some cases yield too weak a 
conclusion (Example~\ref {famous}).  \\[1ex]
%\noindent 
3) Yet another problem is illustrated by
an attempt to apply transitivity of $\Incl$ to atoms \(f(y)\Incl g(c,y)\) and
\( g(x,h(x,x))\Incl e(x)\).  The term $h(x,x)$ may be moved into a binding,
but $c$ cannot because \(f(y)\Incl g(c,y)\) means \(\exists x(x\Incl c\land
f(y)\Incl g(x,y))\) but not \(\forall x(x\Incl c\then f(y)\Incl g(x,y))\).
Bindings do not help in this situation, because the occurrence of the term $g(c,y)$ 
in the literal \(f(y)\Incl g(c,y)\) is, as we call it, of type $\exists$.  
Following Skolem, we know that
there exists a function $\alpha$ satisfying \(\alpha(y)\Incl c\) and \(y\Int
h(\alpha(y),\alpha(y))\then f(y)\Incl e(\alpha(y))\).  The function $\alpha$
is a semantical object, and we will introduce notation rules for such
functions since they will be needed in the derivations in
our inference system.\\[1ex]
%
Unification is used during the proof process but, in order to solve the
above problems, it will not only unify two terms but also produce some
additional assumptions to be included in the processed clauses.
 We consider {\em refutational} proofs with a strategy
analogous to {\em ordered resolution} and {\em ordered superposition}
\cite{BG,PP}, in which term ordering is used to restrict the proof
search space.  According to this strategy, only maximal terms may be
involved in the applications of the inference rules. Our results are
valid for any {\em simplification} ordering
\cite {Der} of terms.  In the following example the maximal terms are underlined.

\begin{EXAMPLE} \label{famous}
\begin{eqnarray}
\label{cl:f-h}  &\underline{f(x)}\Incl h(x,x); \\
\label{cl:g-h}  &\underline{g(x)}\Cont h(x,x);\\
\label{cl:g-f}  &\underline{g(c)}\notCont f(c);\\
\noalign
{This set of clauses has no model \cite{Hus,Mich}.  The ordering of 
the functional symbols, $g>f>h>c$, gives rise to the term ordering used here.
There is only one possibility to start: to unify terms with
$g$. But if $c$ were moved from $g(c)$ into a binding just now, the clause
\(x\notInt c, g(x)\notCont f(c)\) would be obtained, which does not
contradict the first two clauses.  First, the other side of the literal
must be made deterministic. A new deterministic constant $d$ denotes an element
of $f(c)$ which is not an element of $g(c)$:}
\label{cl:f-d}  &\underline{f(c)}\Cont d; &  (\ref {cl:g-f})\\
\label{cl:g-d}  &\underline{g(c)}\notInt d; &  (\ref {cl:g-f})\\
\noalign{Only now $c$ may be moved into a binding and transitivity of 
$\Cont$ applied:}
\label{cl:h-d}  & c\notInt x,\,\underline{h(x,x)}\notCont d &  (\ref{cl:g-d},\ref{cl:g-h})\\
\label{cl:c-e}   &\underline c\Cont e; &  (\ref {cl:f-d})\\
\label{cl:fe-d}   &\underline{f(e)}\Cont d; &  (\ref {cl:f-d})\\
\noalign{In the similar way clause~\ref {cl:f-d} was prepared to be resolved
with clause~\ref {cl:f-h}, because deterministic terms can be substituted
without any restrictions:}
\label{cl:he-d}  &\underline{h(e,e)}\Cont d;\quad &  (\ref {cl:fe-d},\ref{cl:f-h})\\
\label{cl:c-e2}  & \underline c\notInt e,\, d\notCont d; &  (\ref {cl:he-d},\ref{cl:h-d})\\
\label{cl:d-d}   & \underline e\notInt e,\,d\notCont d; &  (\ref {cl:c-e},\ref{cl:c-e2})\\
&\underline{d}\notCont d; &  (\ref {cl:d-d})\\
& \Box \vspace{-2ex}
\end{eqnarray}
The clauses~\ref {cl:g-d} and \ref{cl:fe-d} are {\em assumptions} about new
deterministic terms (constants in this example) $d,e$, which are some kind of
{\em Skolem} functions, introduced to break down existential binding present
inside of some terms.  Their introduction, like introduction of variable
bindings, is an effect of term unification, and can not be avoided in this
strategy.
\end{EXAMPLE}
%
The example shows how complicated unification is in the case of
nondeterministic operations -- its definition was the
main problem to be solved on the way to the completeness result. The
next section presents the solution to this problem.

\section{Unification of nondeterministic terms}

\subsection{Substitutions are deterministic}

In proofs we extend syntax by additional functional symbols from an infinite
set \(\Fvars\) called {\em f-variables}.  They are always interpreted as
deterministic functions, therefore terms constructed completely of variables
and f-variables are called {\em d-terms} (shorthand for {\em deterministic
terms}), their set is denoted $\Dterms$, so \(\Terms \cap \Dterms =\Vars\).
Terms of this kind are used to construct a model in the completeness proof.
Only d-terms are allowed to be substituted into variables. \vspace{.5ex}

\begin{DEFINITION}\label{def:substitution}
Call a {\em substitution} any function \(\sigma:\Vars \to \Dterms\).  The
domain \(\dom (\sigma)\) of $\sigma$ is the set \(\{x: \sigma(x) \neq x\}\)
of variables on which $\sigma$ is non-trivial.  As a set the substitution
$\sigma$ is considered as the set of pairs \(\{\<x,\sigma(x)\>: x\in\dom
(\sigma)\}\).
\end{DEFINITION} \vspace{.5ex}

Variables are the only d-terms in the set $\Terms$.  After instantiation of
some variables by d-terms, the obtained terms become divided into two parts:
the top is nondeterministic, and bottom is deterministic.  The natural way to
present this division is to write such terms in the form \(t\sigma\), where
$t$ does not contain f-variables and $\sigma$ is a substitution.  The same
applies to other words, like literals and clauses.  Words in such
presentation remind of {\em closures} \(w\cdot \sigma\) from \cite
{Basic-par}.  We sometimes use this form of presentation.  In our case,
unlike in \cite {Basic-par}, this form is derivable from the word structure
because of non-intersection of classes \(\Terms \setminus \Vars\) and
\(\Dterms \setminus \Vars\).  We borrow some terminology from \cite
{Basic-par}: $w$ is called a {\em skeleton} and \(\Var w\) is called the {\em
frontier} of a word \(w\sigma\), also denoted \(\frontier {w\sigma}\).  It is
supposed that all variables in any skeleton are different, therefore all
skeletons of the word \(w\sigma\) are equal up to renaming of variables.
(This is relevant to introduction of f-variables discussed below.)  In spite
of non-uniqueness of skeletons, we will write \(w\sigma=w\cdot \sigma\), as
if interpreting the sign `$\cdot $' as an application of substitution to the
word $w$.

The restriction to substitute only d-terms restricts the possibility to
unify terms. As a kind of compensation for that, it is allowed to {\em
replace} some subterms by d-terms.  Soundness of this replacement is based on
special properties of f-variables formulated in Lemma~\ref {le:f-variables}.
%
\subsection{Introduction of f-variables}
%
They are related with some literals and {\em positions} in them.  We include
the functional symbols in positions in order to be able to relate positions
directly with specific terms.

 \begin{DEFINITION}\label {def:position}
A {\em position} is any finite sequence \(f_1n_1\ldots f_kn_k\), where $f_i$ are functional sybols and $n_i$ are natural numbers such that \(0< n_i\leq ar(f_i)\).
The empty sequence, the {\em top position}, is denoted \(\Top\).
\end{DEFINITION}

{\em Concatenation} of positions $p,q$ is denoted by juxstaposition \(pq\),
with unit \(\Top\).  
For a set of positions $Q$, \(pQ\) denotes \(\{pq: q\in Q\}\).  
Positions form upper semilattice
with $\Top$ as the top {\it wrt}. the {\em prefix order}:
position \(pq\) is {\em below} $p$ for \(q\neq \Top\).  For a set of
positions $P$, \(\min(P)\) and \(\max(P)\) denote, respectively, the sets of
minimal and maximal positions in $P$.  The set of positions in $t$, \(\Pos
t\), is the smallest set of positions such that \(\Top \in \Pos t\) and, if
\(t=f(\List tn,)\), then \(\Pos t\Def \bigcup_{i=1}^n fi\cdot\Pos {t_i}\).
$\subterm tp$ denotes a subterm of $t$ which {\em occurs at the position} $p:$
\(\subterm t\Top= t\) and \(\subterm t{qfi}=t_i\) if \(\subterm tq=f(\List
tn,)\), otherwise \(\subterm t{qfi}\) is undefined.
$t[s]_p$ denotes the term $t$ with the
subterm \(\subterm tp\) replaced by $s$ (the case \(s=\subterm tp\) is
possible).
For a set of positions $P$, \(\subterm tP\) denotes the set of subterms
\(\{\subterm tp:p\in P\cap \Pos t\}\).
\(\Var t\) denotes the set of {\em variable positions} in a term $t$,
{\it i.e.}, \(\{p\in\Pos t:\subterm tp\) is a variable$\}$.  % for \(p\in\Var t\).
\(\Var t\) is a subset of \(\min
(\Pos t)\), the remaining minimal positions (if any) are occupied by
constants.
%The lack of any (variable) symbol at the end of a variable position
%corresponds to the ``hole'' represented by a variable -- appending any term,
%or position, will correspond to substitution.
%
\subsubsection{When are f-variables introduced?}
%
The f-variables correspond to {\em Skolem} functions and are needed in
unification of the terms of type $\exists$.  There are four general forms of
literals whith a non-variable term $s$ of this type in which new
f-variables may be introduced by unification:
\begin{equation} \label{eq:exist-literals}
 s\Int t,\quad s\notEq t, \quad s\notIncl t, \quad  s\Cont t,
\end{equation} 
\(s,t \in \Terms\).  To describe all possible appearences of f-variables
in any proof, we establish a bijection $\dt\_\_$ between the following sets:
\begin{itemize}%\smallerspaces
\item the set of all pairs \(\<l,p\>\), where a literal $l$ has one of the
forms presented in (\ref {eq:exist-literals}), $t$ is
a variable in the case \(l= s\Cont t\), $p$ is a non-variable position
in $s$,
\item the set of d-terms with exactly one f-variable.
\end{itemize}
For \(e(\vec x)=\dt lp\), \(\vec x\) is the list of all variables occuring in
$l$, in the same order from left to right and with all repetitions.  Thanks
to this requirement, \(\dt{l\cdot \sigma}p=\dt lp\sigma\) for any
substitution $\sigma$.  The inverse bijection $\expandafter\inverse\dt
\hide\_$ is a pair of functions \(\<\fl\_, \fp\_\>\), {\it i.e.}, \(\fl {\dt
lp}=l\), \(\fp {\dt lp}=p\) and \(\dt {\fl d}{\fp d}=d\).  We also denote by
$\ft{e(\vec x)}\Def \subterm sp$, the subterm of $l$ which can be safely
replaced in it by $e(\vec x)$ in the sense of Lemma~\ref {le:f-variables}.
%
\subsubsection{Semantics for f-variables.}
%
For a  \(\Funcs\)-multialgebra $A$, each f-variable $e$ is
interpretated as a deterministic function \(\phi(e): (S^A)^{ar(e)}\to S^A\).
Let $\phi$ be such an interpretation of f-variables. It
extends  $A$ to a \(\Funcs\cup\Fvars\)-multialgebra, denoted
 \(A_\phi\).  Values of d-terms in the multialgebra \(A_\phi\) are
evaluated according to the usual rules of (deterministic) algebras.  
The interpretations of f-variables must satisfy the additional conditions
given in:

\begin{LEMMA}\label{le:f-variables}
Any \(\Funcs\)-multialgebra $A$ can be extended with an intepretation $\phi$
of f-variables in such a way that for any d-term $d:$ 
%the following statements are true:
\begin{itemize}%\smallerspaces
\item $A_\phi$ satisfies the atom \(d\Incl \ft d\);
\item $A_\phi$ satisfies \(\fl d\) iff it satisfies \(\fl d [d]_{\fp d}\) 
%\\(If \(\fl d\) is of the form $f(s)\Cont t$, then this condition is satisfied for $t\in\Vars$.)
\end{itemize}
 \end{LEMMA}
The last condition says that the subterm \(\ft d\) can be replaced by $d$ in
\(\fl d\) without changing the meaning of \(\fl d\). 

\subsection{Unification}

Success in unifying terms depends not only on the terms, but also on the literals 
in which
they occur.  In usual unification, one tries to make terms identical by
some unifying substitution.  In our case, only d-subterms can be substituted
for variables.  If a variable  should be replaced by a
non-deterministic subterm, then the inverse action is made --- the subterm is
replaced by the variable, we say that the subterm is {\em ejected} and put
into a new literal, called a {\em binding}.  In clauses, this kind of
replacement is legal only inside of terms of type $\forall$.  In terms of
type $\exists$, the ejected subterms must be replaced by new f-variables,
which are then bound by {\em assumptions}, the special kind of clauses.  To
be shorter in some places below, we call {\em unifying sets} collections
consisting of a substitution (presented as a set), a set of bindings and a
set of assumptions.  In general, the process of unification can be presented
as a sequence of three phases: 1)~ejection of some subterms, 2)~formation of
the unifying sets, and 3)~usual unification.
%
\subsubsection{Ejection.}
%
To perform an ejection from a term, it is sufficient to know the frontier of
the other term, so we formulate ejection relatively to some given set $Q$ of
positions.  Let \(l= s\oplus t\) and \(l'=l\cdot \sigma\) be literals, $l'$
be the one where ejection should occur, and let \(P \Def \max(Q\cap\Pos
s)\setminus \Var s\) be the set of non-variable positions of $s$ which are
maximal in $Q$.  If $P$ is empty, then there are no subterms to be ejected.
If not, then we proceed as follows.  All subterms \(\subterm sP\) are ejected
and replaced by new, neither from \(\dom (\sigma)\) nor from \(\VV {l'}\), all
distinct variables.  Let $s_Q$ be the term obtained from $s$ by this
replacement.

The rest depends on the form of the literal $l$ and is presented in the next
phases. 
%
\subsubsection{Formation of unifying sets.}
%
Let \(B =\{\subterm {s_Q} p\notInt \subterm sp: p\in P\}\) be a set of
bindings, \(A =\{\dt {l} p\Incl \subterm sp: p\in P\}\) be a set of atoms,
and \(S =\{\<\subterm {s_Q}p ,\dt {l'}p\>: p\in P\}\) be a substitution. All
these sets are obtained by the replacement of \(\subterm sP\).

Different cases to consider are presented in Table~\ref {tbl:unification}.
The second term of $l$ ({\it i.e.}, $t$) can be changed to a new variable
$y$, so the final forms of $l$ and $t$ are denoted $l_Q$ and $t_Q$, while
\(\Sub(l',Q)\) and \(\At(l',Q)\) denote the obtained substitution and the set
of assumptions, respectively.  The first line of the table describes the
trivial case \(P=\emptyset\).

\begin{table}[hbt]
\begin{center}
\(
\begin{array}{|c||c|c|c|c|}
\hline
   \mbox{for }Q\mbox{ and }l'=(s\oplus t)\cdot \sigma & \Bin(l',Q) & l_Q & \Sub(l',Q) & \At(l',Q) \\
\hline\hline
 (Q\cap\Pos s)\setminus \Var s=\es & \es & l & \es & \es \\ 
\hline
 \oplus=\Cont\ \land\ t\notin \Vars & \{y\notInt t\} & s_Q\Cont y & \sigma\cup S & A \\
\hline
 \oplus\in\{\Cont,\Int,\notEq,\notIncl\}\ \land & & & & \\
 (\oplus=\Cont\ \then\ t\in \Vars) & \es & s_Q\oplus t & \sigma\cup S & A \\
\hline
 \oplus=\notCont\land t\notin \Vars & B & s_Q\notInt y & \sigma\cup\{\<y,\dt {t\notIncl s}{\Top}\>\} & \{\dt {t\notIncl s}{\Top}\Incl t\} \\
\hline
  \oplus\in\{\Eq,\Incl,\notCont,\notInt\}\ \land & & & & \\
  ( \oplus=\notCont\ \then\ t\in \Vars) & B & s_Q\oplus t & \sigma & \es \\
\hline
\end{array}
\)
\end{center}
\caption{Ejection cases} \label{tbl:unification}
\end{table}
%
\subsubsection{Deterministic unification.}
%
The ejection and formation of unifying sets were formulated relatively to
some unspecified set of positions $Q$.  In unification of two terms $s',s''$,
$Q$ is the union of their frontiers, \(\frontier {s'}\cup \frontier
{s''}\).  In the first step, every non-variable term from \(\subterm {s'}
Q\cup \subterm {s''}Q\) was ejected and replaced by a new variable, in the
second step, the literals $l'\cdot\sigma'$ and $l''\cdot\sigma''$ containing
$s',s''$ were considered and, if necessary, transformed.  $s',s''$ became now
\(s'_Q\Sub(l',Q)\) and \(s''_Q\Sub(l'',Q)\).  The whole question is reduced
now to unification of the latter two terms by a substitution, say $\rho$.
The previous transformations were needed only to ensure that no 
nondeterministic term appears in $\rho$.  Practical unification algorithms 
could, of course, proceed in other way, but this is another story.

\section{The Inference System \C J}\label{se:reasoningNG}
%
%\subsection{Overlapping of literals}
%
Unification is used in inference rules, as the
case of {\em literal overlapping}.  \(\mgu(s,t)\) denotes the 
usual {\em most general unifier} of terms $s,t$.

\begin{DEFINITION}\label {def:literal-overalap}
Let \(l'=s\oplus t\) and \(l''=u \otimes v\) be literals, $p$ be a position
in $u$ above the frontier, \(Q=\max(\frontier s \cup \frontier {\subterm
up})\) be a set of positions, \( s'\oplus' t' = l'_Q \cdot \Sub(l',Q)\),
\(u'\otimes' v' = l''_{pQ} \cdot \Sub(l'',pQ)\).  Then the literal $l'$
{\em overlaps} the literal $l''$ at a position $p$, if the substitution
\(\Sub(l',l'',p) =\mgu (s\Sub(l',Q), \subterm up\Sub(l'',Q))\)
called the {\em unifying substitution} exists.
\end{DEFINITION}

Literal overlapping is not sufficient to derive new literal from $l'$ and
$l''$, the additional condition being that the relation \(\ominus
=\Sup(p ,\oplus' ,\otimes')\) is non-trivial.  In this case,
\begin{itemize}%\smallerspaces
\item the literal \(\Lit (l',l'',p)\Def u'[t']_p\ominus v\) is called the {\em
    critical literal} formed by  \(l'\) and \(l''\); 
\item the literal set \(\Bin(l',l'',p)\Def \Bin(l',Q) \cdot \Sub(l',Q )\cup
   \Bin(l'',pQ) \cdot \Sub(l'',pQ)\) is called the {\em binding set};
\item the clause \(\CC(l',l'',p)\Def \Bin(l',l'',p) \cup \{\Lit (l,l',p)\}\), 
   is called the {\em critical clause} formed by the  \(l'\) and \(l''\);
\item the set of single clauses \(\At(l',l'',p)\Def \{\{a \cdot \Sub(l',Q)\}:
   a\in\At(l',Q)\} \cup \{\{a \cdot \Sub(l'',pQ)\}: a\in\At(l'',pQ)\}\) is
   called the {\em assumption set}.
\end{itemize}
%
%So, the process of unification of overlapping terms produces a set of atoms
%called {\em assumptions} and a clause called critical and consisting of
%obtained bindings and of the critical literal.  
In the ground case 
we only had the critical literal without any bindings or
assumptions.  The critical clause \(\CC(l',l'',p)\), the assumption set
\(\At(l',l'',p)\) and the unifying substitution \(\Sub(l',l'',p)\) are
now used in the  inference system \C J, with the following rules: \\[2ex]
%
\begin{tabular}{r@{\ :\ }l}
{\bf Reflexivity resolution} & \quad\(\prule {C,s\oplus s'}
  {\{(B,C)\sigma\}\cup \C A}\) 
\quad
where \(\oplus\) is one of \(\notIncl\), \(\notInt\) or \(\notCont\),\\[2ex]
\multicolumn{2}{r}{\C A \(=\At(l,\rev l,\Top)\),
\(B=\Bin(l,\rev l,\Top)\) and
$\sigma$ is a substitution \(\Sub(l,\rev l,\Top)\) for \(l= s\oplus s'\).} \\[3ex]
%
{\bf Superposition} & \quad \(\prule {C,a \qquad D,l}
{(C,D,\CC(a,l,p))\Sub(a,l,p)}\) \quad 
atom \(a\) overlaps literal \(l\) at $p$.\\[3ex]
%
{\bf Compositionality resolution} & 
\quad \(\prule {C,s\oplus t \qquad D,s'\odot u,s''\ominus w}
{(C,t\otimes u,s\odot u)\sigma}\) \quad  \\[2ex] 
\multicolumn{2}{r}{where
\(\odot = \Comp \oplus {\neg\otimes}\) and \(\sigma=\mgu\{s,s',s''\}\).}
\end{tabular}

\begin{THEOREM} \label{th:soundnessNG}
The inference system $\C J$ is sound.
\end{THEOREM}
\begin{PROOF} 
Proving soundness of inference rules, which use so complicated
unification, is not a trivial task. It consists of two subtasks: 1) proving
soundness of unification on which the rules are based, and 2) for each rule, 
showing the particular property of predicates which is applied in the rule.
\end{PROOF}

\subsection{Ordering of words and the proof strategy}\label {se:strategy}
We extend the definitions of ordering from \ref{sub:ordMax} to non-ground terms.
The partial ordering of
non-ground terms is derived from the ordering of the ground ones
according to the rule: %$u > v \Leftrightarrow u\sigma > v\sigma$
\begin{equation} \label{eq:ord-non-ground}
u > v \iff u\sigma > v\sigma
\end{equation}
for any ground substitution $\sigma$.

Our specific assumption about the orderings is that any deterministic term
(from $\Dterms$) is strictly smaller than any non-variable non-deterministic
term (from \(\Terms\)). Thus any variable is smaller than non-variable term
from $\Terms$.  But in the set $\Dterms$ of d-terms we have the sam
term ordering as before.
Literals and clauses are identified with multisets and their ordering
is defined by the {\em multiset extension} \cite{DM} of the term
ordering as before.
%% \footnote{ It is possible to use sets instead of multisets
%% but this would require definition of different new orderings on
%% sets. For instance, if $t<s$ we want $t\Eq s < s\Eq s$. This is
%% obtained directly using the multiset extension but not using extension
%% to sets. Furthermore, multisets work uniformely when extending the
%% ordering to the level of literals and then clauses. It is easier to
%% work with such uniform extensions than with possibly different
%% extensions to sets.}
%Considering
%multisets over some set $T$ as functions of type \(T\to \Nat\) we use
%\begin{DEFINITION} \label{def:multiset-ordering}
%For an ordering `$\Ord$' on a given set $T$, an ordering `\(\M\Ord\)' on the
%set \(T\to \Nat\) is a {\em multiset extension} of `$\Ord$', if
%\[\beta \M\Ord \gamma \iff \forall d\in  T\,\exists c\in T\/  \left( (\beta
%(c)>\gamma (c) \land (\beta (d)\geq \gamma (d)\lor c\Ord d  )\right).\]
%\end{DEFINITION}
%In the general case it is known \cite{DM} that `$\M\Ord$' is total if
%`$\Ord$' is total and `$\M\Ord$' is well-founded if `$\Ord$' is well-founded.
%
%% A literal $s\oplus t$ is represented by the multiset \(\{\{s,\oplus\},
%% \{t,\rev \oplus\}\}\).  We assume that any term is bigger than any predicate
%% symbol.  A stronger positive predicate is bigger than a weaker one, the order
%% between negative predicates is reversed, and all negative predicates are
%% bigger than the positive ones:
%% \begin{equation} \label{eq:predicate-orderNG}
%% \notEq\ >\ \notIncl\ >\ \notCont\ >\ \notInt\ >\ \Eq\ >\ \Incl\ >\ \Cont\ >\
%% \Int.
%% \end{equation}
%
%%  The ordering of literals is the twofold
%% extension of `$<$' because each literal is a multiset of two multisets.
%% Clauses are compared as multisets of literals, so their ordering is the
%% multiset extension of the ordering of literals (threefold multiset extension
%% of `$<$'). Although we have here three different orderings, we will use the
%% same symbol `$<$' to denote any of them. This should not introduce any
%% confusion as the sets of terms, literals and clauses are disjoint.

%\subsubsection {The \strategy\ proof strategy.} 
%% 
%% The literals mentioned explicitly in the premises of the proof rules are
%% called {\em active}. Various ways of selecting the active literals will lead
%% to different proof strategies. The \strategy\ strategy requires that the
%% active literals in the premise clauses are the ones which are maximal {\it
%% wrt}. the ordering defined above.  Stated explicitly the strategy amounts to
%% the following restrictions on the application of the rules:
Using this extension of the ordering, we adapt the \strategy\ proof strategy
for \C I from \ref{sub:Max} to the present system \C J:
\begin{description}%\smallerspaces
\item[Reflexivity resolution:] \((s\oplus s')\sigma\) is maximal
in the clause \((C,s\oplus s')\sigma\).
\item [Superposition:] the atom \(a\sigma\) and the literal \(l\sigma\),
where \(\sigma=\Sub(a,l,p)\), are maximal in the respective clauses
\((C,a)\sigma\) and \((D,l)\sigma\).
\item [Compositionality resolution:] the atom \((s\oplus t)\sigma\) is maximal
in the clause \((C,s\oplus t )\sigma\). The maximal atom in the clause
\((D, s'\odot u, s''\ominus w)\sigma\) is \((s''\ominus w)\sigma\), and
\(s'\odot u\) is an atom.
\end{description}
%
The important observation for our proof of completeness concerns the ordering of
clauses in premisses and conclusions of the proof rules.  The
nondeterministic terms from bindings that appear in the conclusions are subterms
of the active literals, and therefore binding literals are smaller than
the active literals from premisses.  So, if other new (``not contained
in premisses'') literals are smaller than the (maximal) active literals
then the conclusion clause is smaller than the maximal of premisses clauses.
Furthermore, new variables may be introduced instead of non-deterministic terms.
But then the assumption that $\Dterms$ (including variables!) 
are smaller than non-variable $\Terms$ makes the conclusion smaller than the 
respective premisses.
%
%\section{Ground literal rewriting}\label{se:Grewrite}
%
%This section mainly repeats the relevant 
%The definitions and lemmas listed here are essentially the same as in \cite{KW}.
%They introduce the concepts and results used in the completeness proof.
%
%\begin{DEFINITION} \label{def:rewriting-step}
%A literal $r$ is a {\em rewriting step} in \C L if either \(r\in\C L\), or
%$r$ is an atom \(u[s]_p\oplus u[t]_p\) for some term $u$, a position $p$ in
%$u$, and an atom \(s\otimes t\in \C L\), where \(\oplus\in\{\Incl,\Cont\}\),
%if $\otimes=\Eq$, or $\oplus=\otimes$, otherwise.
%\end{DEFINITION}
%
%A sequence of rewriting steps \(\<s\oplus_1 t_1,\: t_1\oplus_2 t_2,\:
%t_2\oplus_3t_3,\: ...\:,\:t_n\oplus_n t\>\) is called a {\em rewriting
%sequence}, the predicate sign of the derived literal \(s\oplus t\) is
%computed using the function \(\Comp\_\_\): \(\oplus=\Comp {\rev {\Comp {\rev {\Comp
%{\rev {\oplus _1}}{\rev {\oplus _2}}}}{\cdots }}}{\oplus _n}\).
%
%\begin{DEFINITION} \label{def:rewriting-proof}
%A rewriting sequence is a {\em rewriting-proof} if it does not contain a {\em
%peak} ({\it w.r.t.} to an ordering of terms $<$), {\em i.e.}, a pair of
%consecutive rewriting steps \(s\oplus t\),\(t\otimes u\) such that \(s\leq
%t\geq u\).
%\end{DEFINITION}
%
%\begin{DEFINITION}\label{def:rewriting-closure}
%For a set \C L of ground literals, the {\em rewriting closure} of \C L is the
%set of ground literals, $\C L^\ast$, defined as follows:
%\begin{itemize}%\smallerspaces
%\item  all atoms of the form $s\Incl s$ or $s\Int s$, where $s$ is a ground
%  term, belong to $\C L^\ast$;
%\item if an atom \(s\oplus t\in\C L\) and a literal \(u[s]_p\otimes v\in\C
%  A^\ast\), then the literal \(u[t]_p\odot v\in\C L^\ast\), if
%  \(\odot=\Sup(p,\oplus,\otimes)\);
%\end{itemize}
%\end{DEFINITION}
%
%\begin{DEFINITION} \label{def:critical-literal}
%A ground rule \(r_1 = s\To\oplus t\) {\em overlaps} a ground rule \(r_2 =
%u[s]_p \To \otimes v\). In this case the literal \(l = u[t]_p \odot v\),
%where \(\odot = \Sup(s,u[s]_p,\oplus,\otimes)\), is called a {\em critical
%literal} formed by the rules \(r_1,r_2\), if $l$ is different from \(r_1\)
%and \(r_2\).
%\end{DEFINITION}
%
%\begin{DEFINITION} \label{def:confluent-system}
%A set \C R of ground rewriting rules is {\em confluent} if \(\C R^\ast\)
%contains all critical literals formed by overlapping rules from \C R.
%\end{DEFINITION}
%
%\begin{DEFINITION} \label{def:forcingNG}
%A set of ground atoms \C A {\em forces}
%\begin{itemize}%\smallerspaces
%\item a ground atom  $a$ if \(a\in\C A^\ast\), and the literal \(\neg a\) if 
%\(a\notin \C A^\ast\);
%\item a ground clause $C$ if it forces some literal \(l\in C\);
%\item a non-ground clause $C$ if it forces any ground instance of $C$;
%\item a set of clauses \C S if it forces all clauses from \C S.
%\end{itemize}
%\end{DEFINITION}
%
%
\section{Completeness} \label{se:completenessNG} 
%
As we have seen in section~\ref{se:rewrite}, sets of ground atoms \C A, or literals
\C L, can be viewed as rewriting systems. We will now use
the same definitions \ref{def:rewriting-step} through \ref{def:confluent-system}
of rewriting step, critical literal, etc. as in
section~\ref{se:rewrite}. We also use the definition~\ref{def:forcing} of forcing
but extend it with the case of forcing a non-ground clause.
\begin{DEFINITION}\label{def:ground-rewriting}\label {def:forcingNG}
A set of ground atoms \C A {\em forces}
\begin{itemize}\MyLPar%\smallerspaces
\item a ground atom  $a$ if \(a\in\C A^\ast\), and the literal \(\neg a\) if 
\(a\notin \C A^\ast\);
\item a ground clause $C$ if it forces some literal \(l\in C\);
\item a non-ground clause $C$ if it forces any ground instance of $C$;
\item a set of clauses \C S if it forces all clauses from \C S.
\end{itemize}
 \end{DEFINITION}
%
The next lemma is the exact counterpart of lemma~\ref{le:first-rule}.
 \begin{LEMMA} \label {le:first-ruleNG} 
If: \C L is confluent, \(a\in \C L\), an atom $b$, with \(b\leq a\), has a
rewriting proof $P$ in \C L, but \(b\notin(\C L\setminus \{a\})^\ast\).  Then:
$a$ and $b$ have the same maximal term, $s$, and the proof $P$ has the
form $a,P'$, where $a$ is not used in the proof $P'$. If \(a\ne b\), then the
literal derived by $P'$ is smaller than $b$.
 \end{LEMMA}
%% 
%% \begin{LEMMA} \label{le:preserve-confluency}
%% For a confluent and consistent system \C R and a rule \(r\notin\C R^\ast\)
%% the system \(\C R\cup \{r\}\) is confluent and consistent iff
%% \begin{itemize}%\smallerspaces
%% \item $r$ does not have the form  \(s\To\oplus s\), where
%%  \(\oplus\in\{\notIncl,\notCont,\notInt\}\),
%% \item for any critical literal $l$ formed by any \(r'\in \C R
%% \cup\{r\}\) overlapping (or overlapped by) $r$, $l\in \C R^\ast$,
%% \end{itemize}
%% \end{LEMMA}
%
We are sketching the proof of {\em refutational completeness} of the inference system
\C J, {\em i.e.}, that there exists a model (multialgebra) satisfying all the
clauses from a consisten set \C S (the empty clause is not derivable from \C S). 
% A set of clauses \C S is {\em consistent} if it does not contain the empty clause.
%The main result is

\begin{THEOREM}\label{completeness}
If a set of clauses \C S is consistent then it has a model.
\end{THEOREM}
\noindent
The construction proceeds in two main steps as in the ground case. 
Given a consistent set \C S,
we select a set of atoms \C R %(section~\ref{se:forcing-set})
and show that \C R is a {\em forcing set}\/ for \C S.
Then \C R can is used to construct a multimodel which satisfies \C S in the
way it was done in \ref{se:multimodel}. We list the steps of the construction
but give only the proofs which are significantly different from those in the
ground case.
%\cite{KW}. 

%\subsection{Model}\label{se:forcing-set}\label{se:main-R}

Let \C G denote the set of ground instances of form \C S, {\it i.e.}, of
clauses \(C\sigma\), where \(C\in \C S\) and $\sigma$ is a ground
substitution (involving only $\Fvars$ functional symbols.).  
The starting point of the model construction is now the set of
maximal literals of ground instances of \C S :
\begin{equation} \label{eq:max-literals}
\C L_0 \Def \{\max(C) : C\in \C G\}.
\end{equation}
%
As before, for an $\C L$ and an $l\in\C L$, we define the set \(\C
L_l\Def \{a\in\C L:a<l\}\)
% contains all the literal from \C L that are
%smaller than $l$.
%

\begin{DEFINITION}\label{def:redundancy}
Let $\C A$, $\C G$  be  sets of ground atoms and clauses, respectively, 
$C$ be a ground clause and $l$ a ground literal.
\begin{enumerate}\MyLPar%\smallerspaces
\item \label{def:redundant-clauseNG}
  \(C\) with \(\max(C)=l\) is {\em redundant} in \C A and \C G if either
  \begin{itemize}%\smallerspaces
  \item $\forc{\C A_l}C$, i.e., $C$ is forced by \(\C A_l\)  or 
  \item the set \C G contains another clause \(C'<C\) with \(\max(C')=l\), and
    $\notforc{\C A_l}{C'}$.%    \(C'\) is not forced by \(\C A_l\).
   \end{itemize}
\item \label {def:redundant-literal}
$l$ is {\em redundant} in \C A and \C G if either
  \begin{itemize}%\smallerspaces 
  \item \(l=s\oplus s'\), where \(\oplus \in \{\notIncl ,\notCont ,\notInt\}\),
  $l$ overlaps \(\rev l\) at the top position, and the clause \(\Bin(l,\rev
  l,\Top)\) is not forced by \(\C A\), or
  \item \(\C A\cup\{l\}\) contains an atom $r$ overlapping $l$ at a position
  $p$, and such that the critical clause \(\CC(r,l,p) <\{l\}\) and is not
  forced by \(\C A\), or
  \item every clause $C\in \C G$ with $\max(C)=l$ is redundant in \C A.
  \end{itemize}
\item \label{def:productiveNG}
\(C,l\) with \(\max(C,l)=l\) is {\em productive} for $l$
in \C A if $\notforc{\C A}C$. %\C A does not force $C$.
\end{enumerate}
 \end{DEFINITION}
\noindent
Definitions~\ref {def:forcingNG} (of forcing) and
\ref {def:redundancy}.\ref {def:redundant-literal} 
(of redundancy) are so related, that all negative literals that are
not forced are redundant. Since any forced literal makes redundant all
clauses containing it, any negative literal appears redundant.

If $\C L_i$ is known, and \(l_i\) is the minimal redundant
literal in \(\C L_i\), we define as in the ground case:
\begin{equation} \label{eq:atoms-modelNG}
\C L_{i+1} \Def \C L_i \setminus \{l_i\}, \hspace{7em}
\C R \Def{\bigcap_{i\in \Nat}} \C L_i.
\end{equation}
%
For a literal \(l\in\C R\) we denote by \(\C R_{l'}\) the set \(\C
R_l\cup\{l\}\), while for \(l\notin\C R :\C R_{l'}=\C R_l\).
%% Before sketching the proof that \C R is the forcing set for $\C S$, 
%% we list some of its properties.
The following three lemmas are the counterparts of (with proofs analogous to)
\ref{le:redundancy-limit}, 
\ref{co:model-confluent} and \ref{le:productive-clause}. Recall the notation
$\red{\C L}l$ for redundancy of a literal $l$ in a set of literals \C L.

 \begin{LEMMA} \label{le:redundancy-limitNG}
For a literal $l\in\C L_0$ (a clause $C\in \C G$ with \(\max(C)=l\)) the three
conditions are equivalent: 
%\[ 
\begin{itemize}\MyLPar
\item $\exists i, l_i>l : \red{\C L_i}l$
\item $\forall j>i: \red{\C L_j}l$
\item $\red{\C R}l$.
\end{itemize}
%% 
%% \(l\in\C L_0\) (a clause \(C\in \C G\) with \(\max(C)=l\)) is
%% redundant in some \(\C L_i\) with \(l_i>l\) $\Leftrightarrow$ it is redundant in every
%% \(\C L_j\) with $j>i$ $\Leftrightarrow$ it is redundant in \C R.
\end{LEMMA}

 \begin{LEMMA} \label{le:model-confluent}
\C R is confluent.
\end{LEMMA}

\begin{LEMMA} \label{le:productive-clauseNG}
For any \(l\in\C R\), there is a clause \(C\in\C G\) productive
for $l$ in \(\C R_l\).
 \end{LEMMA}
%% \begin{PROOF}
%%  If $l\in \C R$ then $l$ is non-redundant. The negated form of the redundancy
%% Definition~\ref{def:redundancy}.\ref{def:redundant-literal} is a conjunction including the
%% condition that there is a non-redundant clause $C\in \C G$ with $\max(C)=l$.
%% Non-redundancy of $C$, {\em i.e.}, negation of
%% Definition~\ref{def:redundancy}.\ref{def:redundant-clauseNG} implies that $C$ is not forced by \(\C
%%  R_l\), what means productivity of $C$ for $l$ in \(\C R_l\).
%%  \end{PROOF}
%
 \begin{DEFINITION}
A set \C G is {\em relatively closed} if any application of a
rule from \C J with premises from \C G yields a clause whose each ground
instance is in \C G or is redundant in \C G.
 \end{DEFINITION}
%
The main technical result, giving the completeness theorem is
%
\begin{THEOREM} \label{le:main-theoremNG}
Let \C G be consistent and relatively closed set of ground clauses, \(\C
L_0\) and \C R be defined by (\ref {eq:max-literals}) and (\ref
{eq:atoms-modelNG}). Each \(l\in \C L_0\) satisfies the following
conditions:
\begin{description}%\smallerspaces
\item[I1.] if $\red{\C R}l$, %$l$ is not redundant in \C R, 
  then for any \(a\in\C R_l\) there
  exists a clause \(C\in \C G\) productive for $a$ in \(\C R_l\),
\item[I2.] if $\red{\C R}l$, %$l$ is redundant in \C R, 
  then \(\C R_l\) forces any clause  \(C\in \C G\) with \(\max(C)=l\).
\end{description}
\end{THEOREM}
\begin{PROOF} 
The structure of the proof is essentially the same as the proof of the ground case
\ref{th:ground-completeness}, proceeding 
by contradiction from the assumption that $l$ is a minimal literal 
in \(\C L_0\) not satisfying the theorem, i.e., either
%We assume that there exist some literal $l$ not satisfying
%the theorem and suppose $l$ is minimal in \(\C L_0\) with this property.
%Observe that any literal from \(\C L_0\) must satisfy one of the conditions 
%I1 or I2, therefore we have two non-intersecting cases:
\begin{description}%\smallerspaces
\item[B1.] $l$, being included in \C R, ``spoils'' productiveness of some 
  clause \(C\in\C G\) with \(a=\max (C) \leq l\), {\em i.e.}, $C$ is
  productive for $a$ in \(\C R_l\), but is not productive in \(\C
  R_l\cup\{l\}\), or
\item[B2.] $l$, being not included in \C R, leaves unforced by \(\C R_l\) 
  some clause \(C\in\C G\) with \(\max (C)=l\).
\end{description}
The following lemma and corollary (cf.~\ref{le:contradiction-way}, \ref{cor:contradiction-way}) allow us to obtain contradiction in both cases:

 \begin{LEMMA}\label {le:contradiction-wayNG}
The following conditions about a clause $D$ and a literal $l$ cannot be 
satisfied simultaneously in a relatively closed \C G:
%\begin{enumerate}
%\item\label{i} 
\setcounter{ITEM}{0}
\ITEM{i}{$D$ is a ground instance of conclusion of some proof rule
from \C J with premises from \C G,}
%\item\label{ii} 
\ITEM{ii} {for any clause \(D'\leq D\), if \(D'\in\C G\), then \(\C R_l\)
forces $D'$,}
%\item\label{iii} 
\ITEM{iii} {\(\C R_l\) does not force $D$,}
%\item\label{iv}
\ITEM{iv} {\(\max(D)<l\).}
%\end{enumerate}
 \end{LEMMA}
%% \begin{PROOF}
%% The condition \?{i} and relative-closeness of \C G mean that the clause $D$
%% must be in \C G or be redundant in \C G.  By \?{iii} and \?{ii}, the clause
%% \(D\notin\C G\), so it is redundant in \C G. The redundancy of $D$ in \C G,
%% by Lemma~\ref {le:redundancy-limitNG} and \?{iv}, means redundancy of $D$ in
%% \(\C R_l\). The redundancy definition includes two cases: either 
%% \ITEM{v}{$D$ is forced by $\C R_l$, or}
%% \ITEM{vi}{ $\C G$ contains a clause \(D'<D\) with \(\max(D')=l\) that is not
%% forced by $\C R_l$.} The case \?{v} is excluded by \?{iii}. The case \?{vi}
%% contradicts \?{ii}.
%% \end{PROOF}
%
\begin{COROLLARY} \label{cor:contradiction-wayNG}
Lemma~\ref {le:contradiction-wayNG} holds if Condition~\?{ii} is replaced by the
following  statement:  $l$ is the minimal literal in \(\C L_0\) satisfying
B1 or B2.
\end{COROLLARY}
%% \begin{PROOF}
%% Minimality of $l$ with respect to Condition~B2 means, that for all clauses
%% $C\in\C G$ from \(\max(C)<l\) follows that \(\C R_l\) forces $C$. Any clause
%% \(D'\leq D\) by \?{iv} has \(\max(D')<l\), and therefore satisfies \?{ii}.
%% \end{PROOF}
%
\underline{Assumption B1} means that
 \(\max(C)=a\), and \(C\setminus\{a\}\) is not forced by \(\C R_l\),
but there exists \(b\in C\) such that \(b\neq a\) and \(b\in \C
  R_{l'}^\ast\).
Productiveness conditions for $C$ in B1 means that
\ITEM{2}{ \(\max(C)=a\), and \(C\setminus\{a\}\) is not forced by \(\C R_l\),}
\ITEM{4}{ but there exists \(b\in C\) such that \(b\neq a\) and \(b\in \C
  R_{l'}^\ast\).}
By Lemma~\ref {le:productive-clauseNG}, there is a clause $D,l$ such that
\(l=\max(D)\) and $D$ is not forced by \(\C R_l\).
\ITEM{1}{\(l=\max(D)\), $D$ is not forced by \(\C R_l\).}
We want to prove that the factoring rule can be applied to
clauses \(D,l\) and $C$. The order of the atoms \(a,b,l\) is important:
\ITEM{6}{ \(b< a\leq l\) (it follows from \?{2} and B1).}
The strong inequality between $a$ and $b$ follows from maximality of $a$ in
$C$. By Lemma~\ref {le:first-ruleNG} applied to atoms \(l\neq b\), we get that,
if \(l = s\oplus t\) and \(b = s\odot u\), then there exists $\otimes$ such
that
\ITEM{FA}{\(\odot = \Comp {\rev \oplus}\otimes\) and \(c = t\otimes u< b\).}
The first condition in \?{FA} is sufficient to apply the factoring rule to
\(D,l\) and $C$ and derive the clause \(E = (D,b,\neg c)\).  From \?{1},
\?{6} and \?{FA} it follows \(\max(E)<l\).  From Lemma~\ref {le:first-ruleNG}
we also have, that $l$ is used only once in the proof of $b$, hence \(c\in\C
R_l^\ast\) {\it i.e.}, \C R does not force \(\neg c\).  This condition
together with \?{1}, \?{2}, and \?{4} implies that $E$ is not forced by \(\C
R_l\).

Lemma~\ref{le:first-ruleNG} implies that we can apply the compositionality
resolution to $C$ and $D,l$ producing the clause $E=D,b,\neg c$ with
$\max(E)<l$ which
%The literal $l$ and the clause $E$ 
satisfy the conditions of Lemma~\ref
{le:contradiction-wayNG} thus leading to a contradiction. \\[1ex]
%
\underline{Assuming B2}, we have a clause $D$ such that
%That means that there exists a clause $D$ such that
\(C=(D,l)\), \(\max(D)<l\) and $C$ is not forced by $\C R_l$.
\ITEM{2i}{\(C=(D,l)\), \(\max(D)<l\) and $C$ is not forced by $\C R_l$.}
%Assume that $C$ is minimal with this property.
Then $l$ is redundant and we analyse the three alternatives of
Definition~\ref{def:redundancy}.\ref{def:redundant-literal}.
The last one is impossible since $C$ is non-redundant. In the first case,
contradiction follows from Lemma~\ref{le:contradiction-wayNG} after
application of reflexivity resolution. In the second case, 
Lemma~\ref{le:productive-clauseNG} and superposition rule give again a pair
satisfying the conditions of Lemma~\ref{le:contradiction-wayNG}, thus yielding a
contradiction.

The literal $l$ is redundant, and by Definition~\ref {def:redundancy}.\ref
{def:redundant-literal} there are three alternatives:
\ITEM{21}{\(l=s\oplus s'\), where \(\oplus \in \{\notIncl ,\notCont
  ,\notInt\}\), $l$ overlaps \(\rev l\) at the top position, and the clause
  \(\Bin(l,l',\Top)\) is not forced by \(\C A\),}
\ITEM{22}{\(\C A\cup\{l\}\) contains a rule $r$ overlapping $l$ at a position
  $p$, and the critical clause \(\CC(r,l,p) <\{l\}\) is not forced by \(\C
  A\),}
\ITEM{23}{ every clause $B\in \C G$ with $\max(B)=l$ is redundant in $\C R_l$.}

The alternative \?{23} is false because $C$ is non-redundant ---
Definition~\ref {def:redundancy}.\ref{def:redundant-clauseNG} 
of redundancy subsumes the negated form
of the minimality assumption about $C$ which we have just made.

In the case of \?{21}, the reflexivity resolution rule can be applied to the
clause $C$ to produce the clause \(D,\Bin(l,l',\Top)\).  The all bindings
literals are smaller than $l$ (see conclding note in subsection~\ref {se:strategy}, 
why by Lemma~\ref {le:contradiction-wayNG} we derive contradiction in this case.)

Consider the alternative \?{22} and let \((C',r)\) be a productive clause
for $r$ in \(\C R_r\) (that exists by Lemma~\ref {le:productive-clauseNG}).
From minimality of $l$ it follows, that no a literal between $r$ and $l$
destroys productivity of $(C',r)$. (By the way, \(r=l\) is possible.)  So,
$(C',r)$ is also productive for $r$ in \(\C R_l\):
\ITEM{PCl}{ \(C'\cap \C R_l^\ast =\es\).}
By \?{22}, there exists a clause
\ITEM{28}{\((D' = D,\,C',\CC(r,l,p))\Sub(r,l,p)\)}
deduced from clauses $D,\,l$ and $C',\,r$ by the superposition rule.  Now,
all conditions of Lemma~\ref {le:contradiction-wayNG} for $D'$ hold, and
contradiction follows.
\end{PROOF}
\noindent
Thus $\C R$ is the forcing set for $\C S$. 
A multialgebraic model of \C S can be now constructed from $\C R$ 
in the same way as in subsection~\ref{se:multimodel}.
%theorem~\ref{th:multialgebra-exists}.

\section{Conclusions and directions for future work}

We presented a theoretical case study of transferring the ideas from ordered
paramodulation to a particular non-equational reasoning system.  We
concentrated on a particular logic, but the obtained results are of general
interest. Firstly, because reasoning with set-valued relations is very common
and, secondly, because the introduced techniques should be applicable to the
more general binary relations.  Together with L.~Bachmair and H.~Ganzinger,
\cite {BG249,BG-Oslo}, we have shown how term-rewriting techniques for
theorem proving and methods of restricting inference rules can be applied to
more general binary relations than congruences.  Specifically, we
demonstrated how term-rewriting can work in a languge with a very restricted
substitutivity into variables and in the presence of existential
quantification.

We were not concerned here with inference restrictions other than those
imposed by the term ordering.  Probably, also the ideas from {\em basic
paramodulation} \cite{Basic-par} could be transferred to our logic, because
the level of deterministic terms is entirely analogous to the equational
case.

Our results may be particularly relevant to the area of relational
specifications \cite {rel-spec}, where composition of relations also hides
existential quantification.  We intend to investigate the possible
applications and extensions in this direction \cite {KMW}.  Another
interesting possibility would be to extend our results to typing relations.




\begin{thebibliography}{MM99}\MyLPar
\bibitem[Bez90]{Bez} M.~Bezem. 
   Completeness of Resolution Revisited. 
   {\em Theoretical Computer Science}, 74, pp.27-237, (1990).
\bibitem[BG91]{BG} L.~Bachmair, H.~Ganzinger. 
   {\em Rewrite-Based Equational Theorem Proving with 
                           Selection and Simplification.}
   Technical Report MPI-I-91-208, Max-Planck-Institut f. Informatik, 
   Saarbr\"ucken, (1991).
\bibitem[BG93]{BG249} L.~Bachmair, H.~Ganzinger. 
   {\em Rewrite Techniques for Transitive Relations.}
   Technical Report MPI-I-93-249, Max-Planck-Institut f. Informatik, 
   Saarbr\"ucken, (1993). [to appear in LICS'94]

\bibitem[BG95]{BG-Oslo} L.~Bachmair, H.~Ganzinger. 
   {\it Ordered Chaining Calculi for First-Order Theories of Binary Relations}.
   Technical Report MPI-I-95-2-009, Max-Planck-Institut f. Informatik, 
   Saarbr\"ucken, (1995).
\bibitem[BGLS95]{Basic-par} L.~Bachmair, H.~Ganzinger, C.~Lynch,  W.~Snyder. 
   Basic paramodulation. {\it Information and Computation}, 121(2),
   pp.~172--192 (1995). 
\bibitem[BGS94]{rel-spec} R.~Berghammer, T.~Gritzner, G.~Schmidt. Prototyping
   relational specifications using higher-order objects. In {\it
   Proc.\ Int.\ Workshop on Higher Order Algebra, Logic and Term Rewriting
   (HOA'93)}, Amsterdam, The Netherlands, Sept.\ 1993, LNCS, vol.~816,
   pp.~56--75 (1994).
\bibitem[BK95]{BK} M.~Bia{\l}asik, B.~Konikowska, {\it Reasoning with 
   Nondeterministic Specifications}, ICS PAS Report no.~793, Institute
  of Computer Science, Polish Academy of Sciences, (1995).

\bibitem[DJ90]{Der} N.~Dershowitz, J.-P.~Jouannaud. 
   Rewrite systems. In: J.~van Leeuwen (ed.) 
   {\em Handbook of theoretical computer science}, vol. B,
   chap. 6, pp.243-320. Amsterdam: Elsevier, (1990).
\bibitem[DM79]{DM} N.~Dershowitz, Z.~Manna. 
   Proving termination with multiset orderings. 
   {\em Communications of the ACM}, 22:8,pp.465-476, (1979).
\bibitem[DO92]{DO} A.~Dovier,E.~Omodeo,E.~Pontelli,G.-F.~Rossi. 
   Embedding finite sets in a logic programming language. 
   {\em LNAI}, 660, pp.150-167, Springer Verlag, (1993).
\bibitem[Hes88]{PS1} W.H.~Hesselink. A Mathematical Approach to Nondeterminism
   in Data Types. {\em ACM Transactions on Programming Languages and Systems},
   10, pp.87-117, (1988).
\bibitem[Hus92]{Hus} H.~Hussmann. Nondeterministic algebraic
   specifications and nonconfluent term rewriting. {\em Journal of Logic
   Programming}, 12, pp.237-235, (1992).
\bibitem[Hus93]{HusB} H.~Hussmann. 
   {\em Nondeterminism in Algebraic Specifications and Algebraic Programs.}
   Birkh\"auser Boston, (1993).
\bibitem[Jay92]{Jay} B.~Jayaraman. Implementation of Subset-Equational 
   Programs. {\em Journal of Logic Programming}, 12:4, pp.299-324, (1992).
\bibitem[Kap88]{Kap} S.~Kaplan. Rewriting with a Nondeterministic Choice
   Operator. {\it Theoretical Computer Science}, 56:1, pp.37-57, (1988).
\bibitem[KW94]{KW} V.~Kriau\v ciukas, M.~Walicki
   {\em Reasoning and Rewriting with Set-Relations I: Ground-completeness.}
   Proceedings of CSL'94, LNCS 933, (1995).
\bibitem[KW95]{KW} V.~Kriau\v ciukas, M.~Walicki
   {\em Reasoning and Rewriting with Set-Relations II: Completeness of the non-ground case.}
   in Recent Trends in Data Type Specification, LNCS 1130, (1995).
\bibitem[KW95a]{KMW}V.~Kriau\v ciukas, M.~Walicki.
   Nondeterministic Algebraic Specifications in Relational Syntax, in
   {\it Proc. of 7th Nordic Workshop on Programming Theory}, Report
   No.~86, Programming Methodology Group, G\"oteborg University, (1995).
\bibitem[LA93]{LA} J.~Levy, J.~Agust\'i. Bi-rewriting, a term rewriting
   technique for monotonic order relations. In {\em RTA'93, LNCS}, 
   690, pp.17-31. Springer-Verlag, (1993).
\bibitem[Mos89]{uni-al} P.D.~Mosses. Unified algebras and institutions. In
   {\it LICS'89, Proc. 4th Ann. Symp. on Logic in Computer Science},
   pp.~304--312, IEEE, (1989).
\bibitem[SL91]{relaxed-par} W.~Snyder, C.~Lynch. Goal directed strategies for
   paramodulation. In {\it Proc. 4th Int. Conf, on Rewriting Techniques and
   Applications}, LNCS, vol.~488, pp.~150--161,
   Berlin, (1991).
\bibitem[PP91]{PP} J.~Pais, G.E.~Peterson. Using Forcing to Prove Completeness
   of Resolution and Paramodulation. {\em Journal of Symbolic Computation}, 
   11:(1/2), pp.3-19, (1991).
\bibitem [S-A92]{S-A} R.~Socher-Ambrosius. 
   {\em Completeness of Resolution and Superposition Calculi.}
   Technical Report
   MPI-I-92-224, Max-Planck-Institut f. Informatik, Saarbr\"ucken, (1992).
\bibitem [SD86]{SD} J.~Schwartz,R.~Dewar,E.~Schonberg,E.~Dubinsky. 
   {\em Programming with sets, an introduction to SETL. }
   Springer Verlag, New York, (1986).
\bibitem[Sto93]{Sto} F.~Stolzenburg. 
   {\em An Algorithm for General Set Unification.}
   Workshop on Logic Programming with Sets, ICLP'93, (1993).
\bibitem[Wal93]{Mich} M.~Walicki. 
   {\em Algebraic Specifications of Nondeterminism.}
   Ph.D. thesis, Institute of Informatics, University of Bergen, (1993).
\bibitem[WM95a]{MW-II} M.~Walicki, S.~Meldal. Multialgebras, Power algebras
   and Complete Calculi of Identities and Inclusions. In {\it Recent Trends
   in Data Type Specification}, 
\bibitem[WM95b]{MW} S.~Meldal, M.~Walicki. A Complete Calculus for 
   Multialgebraic and Functional Semantics of Nondeterminism. 
   {\em ACM TOPLAS}, Vol.~17, No.~2, (1993).
\end{thebibliography} 

\end{document}

%%%%%%%%%% adt-bibliography
\begin{thebibliography}{MM99}
\bibitem[Bez90]{Bez} M.~Bezem. 
   Completeness of Resolution Revisited. 
   {\it Theoretical Computer Science}, 74, pp.27-237, (1990).
\bibitem[BG94a]{BG} L.~Bachmair, H.~Ganzinger. 
   Rewrite-Based Equational Theorem Proving with Selection and
   Simplification. {\it Journal of Logic and Computation}, 4(3),
   pp.~217--247 (1994).
\bibitem[BG94b]{BG249} L.~Bachmair, H.~Ganzinger. 
   Rewrite Techniques for Transitive Relations.
   {\it Proc. 9th IEEE Symposium on Logic in Computer Science}, pp.~384--393.
   IEE Computer Society Press (1994).
\bibitem[DJ90]{Der} N.~Dershowitz, J.-P.~Jouannaud. 
   Rewrite systems. In: J.~van Leeuwen (ed.)  {\it Handbook of
   theoretical computer science}, vol. B, chap. 6,
   pp.243-320. Amsterdam: Elsevier, (1990).
\bibitem[DM79]{DM} N.~Dershowitz, Z.~Manna. 
   Proving termination with multiset orderings. 
   {\it Communications of the ACM}, 22:8,pp.465-476, (1979).
\bibitem[DO92]{DO} A.~Dovier, E.~Omodeo, E.~Pontelli, G.-F.~Rossi. 
   Embedding finite sets in a logic programming language. 
   LNAI, vol.~660, pp.150-167, (1993).
\bibitem[Hes88]{PS1} W.H.~Hesselink. A Mathematical Approach to Nondeterminism
   in Data Types. {\it ACM Trans. on Programming Languages and Systems},
   10, pp.87-117, (1988).
\bibitem[Hus92]{Hus} H.~Hussmann. Nondeterministic algebraic
   specifications and nonconfluent term rewriting. {\it Journal of Logic
   Programming}, 12, pp.237-235, (1992).
\bibitem[Hus93]{HusB} H.~Hussmann. 
   {\it Nondeterminism in Algebraic Specifications and Algebraic Programs.}
   Birkh\"auser Boston, (1993).
\bibitem[Jay92]{Jay} B.~Jayaraman. Implementation of Subset-Equational 
   Programs. {\it Journal of Logic Programming}, 12:4, pp.299-324, (1992).
\bibitem[Kap88]{Kap} S.~Kaplan. Rewriting with a Nondeterministic Choice
   Operator. {\it Theoretical Computer Science}, 56:1, pp.37-57, (1988).
\bibitem[KW94]{KW} V.~Kriau\v ciukas, M.~Walicki.  Reasoning and Rewriting
   with Set-Relations I: Ground-Completeness.  In {\it Proceedings of
   CSL'94}, LNCS, vol.~933, pp.~264--278, (1995).
\bibitem[LA93]{LA} J.~Levy, J.~Agust\'i. Bi-rewriting, a term rewriting
   technique for monotonic order relations. In {\it RTA'93}, LNCS, 
   vol.~690, pp.17-31, (1993).
\bibitem[PP91]{PP} J.~Pais, G.E.~Peterson. Using Forcing to Prove Completeness
   of Resolution and Paramodulation. {\it Journal of Symbolic Computation}, 
   11:(1/2), pp.3-19, (1991).
%\bibitem [S-A92]{S-A} R.~Socher-Ambrosius. 
%   {\it Completeness of Resolution and Superposition Calculi}.
%   Technical Report
%   MPI-I-92-224, Max-Planck-Institut f\"ur Informatik, Saarbr\"ucken, (1992).
\bibitem [SD86]{SD} J.~Schwartz, R.~Dewar, E.~Schonberg, E.~Dubinsky. 
   {\it Programming with sets, an introduction to SETL. }
   Springer Verlag, New York, (1986).
%\bibitem[Sto93]{Sto} F.~Stolzenburg. 
%   {\it An Algorithm for General Set Unification.}
%   Workshop on Logic Programming with Sets, ICLP'93, (1993).
\bibitem[Wal93]{Mich} M.~Walicki. 
   {\it Algebraic Specifications of Nondeterminism.}
   Ph.D. thesis, Institute of Informatics, University of Bergen, (1993).
% Selected papers of joint 10th ADT and 5th COMPASS workshops, S.~Margherita, Italy, May 30 - June 3, 1994},
   LNCS, vol.~906, pp.~453--468 (1995).
\bibitem[WM95b]{MW} M.~Walicki, S.~Meldal. A Complete Calculus for 
   Multialgebraic and Functional Semantics of Nondeterminism. 
   {\it ACM Trans. on Programming Languages and Systems}, vol.~17, No.~2,
   pp.~366--393 (1995).
\end{thebibliography} 
