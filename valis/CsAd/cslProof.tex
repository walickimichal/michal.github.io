\section{The Ground-Completeness Theorem} \label{se:completeness}

The proof system \C I  is used to derive a clause from a given
specification \C S ``by contradiction'':
to prove that a clause
$C = \{\List an,\}$ follows from \C S, one takes the
negation of $C$, namely the set of unary clauses $neg(C)\Def\{ \List{\neg a}n;\}$,
% \[neg(C)\Def \{\List{\Seq a}n{;\ };\ \List bm{\Seq;\ }\Seq;\}\]
adds it to \C S, and tries to derive the empty clause from the resulting
set of clauses.
Proving {\em refutational completeness}, we have to show that 
if some set of ground clauses \C S 
has no model, then the empty clause is derivable using rules from \C I. 
The
usual way to prove this is to show that there exists a model satisfying all the
clauses from \C S if the empty clause is not derivable from \C S. In our proof
we follow ideas of R.~Socher-Ambrosius \cite{S-A}, L.~Bachmair and 
H.~Ganzinger \cite{BG249} which, in
 turn, develop the ideas of M.~Bezem~\cite{Bez}. Similar
proof using forcing is given by Pais and Peterson \cite{PP}. All these
 works are concerned with first-order predicate calculus with equality.
In \cite{BG249}, a similar proof method is used with respect to 
transitive relations.

Our construction proceeds in two main steps. Given a consistent set \C S
of clauses, we select a set of atoms \C R (section~\ref{se:forcing-set})
and show that \C R is a {\em forcing set}\/ for  the clauses from \C S
(section~\ref{se:main-R}). Then (section~\ref{se:multimodel}) we show that
\C R can be used to construct a multimodel which satisfies \C S.

We call a set of clauses \C S {\em consistent} if it does not contain the
empty clause. The {\em redundancy} of clauses in \C S will be defined during
the model construction. Redundancy notion was developed by Bachmair and
Ganzinger \cite{BG} to cover simplification techniques commonly used in
theorem provers.  Referring to this notion we fix a set \C S and assume it is
consistent and {\em relatively closed}, meaning that any application of a
rule from \C I with premises from \C S produces a clause that is in \C S or is
redundant in \C S. The main result is

\begin{THEOREM}[Ground-completeness] 
\label{th:ground-completeness}
If a set of ground clauses \C S is consistent and relatively closed then it
has a model.
\end{THEOREM}

\noindent The following section introduces the notions of forcing set,
redundancy and productive clauses.

\subsection{Forcing set, redundancy and productive clauses} \label{se:forcing-set}

We borrow the notion of forcing from \cite{PP} where it is also used in a
completeness proof:
\begin{DEFINITION} \label{def:forcing}
A set of ground atoms \C A {\em forces}
\begin{itemize}\MyLPar
\item  a ground atom  $a$ if \(a\in\C A^\ast\), and the literal \(\neg a\) if 
\(a\notin \C A^\ast\);
\item  a clause $C$ if it forces some literal \(l\in C\);
\item  a set of clauses \C S if it forces all clauses from \C S.
\end{itemize}
\end{DEFINITION}

In the last case we say that \C A is a {\em forcing set} for \C S.
We write $\forc{\C A}w$ if \C A forces $w$.
For a consistent set \C S of ground clauses we will construct a set \C R of
ground atoms forcing \C S.
All such atoms can be oriented into rules because of our assumption about an
ordering of terms, therefore we can treat \C R  as a
system of rules. We will show that \C R is confluent.
In this case it is enough to consider only rewriting proofs.

%changes : no S+, no S-
The starting point of the model construction is
%changes
the set of maximal literals
\begin{equation} \label{eq:max-atoms-set}
\C L_0 \Def \{\max(C) : C\in \C S\}.
\end{equation}
%
In rewriting proofs all terms are not bigger than the maximal term of the 
literal being proved.  This admits an incremental construction of the model, 
starting with $\C L_0$ and removing redundant literals. % !!!

% \subsubsection{Redundant clauses and atoms}

Redundancy of clauses is defined relatively to two sets: one set of clauses \C
S and one of ground atoms \C A. This is an intermediate notion, the final one
refers only to \C S. We have already fixed the set of clauses \C S to
shorten our formulations. For a given literal $l$ and a set \C L of literals
the set
\(\C L_l\Def \{a\in\C L:a<l\}\)
contains all the literals from \C L that are smaller than $l$.
%
\begin{DEFINITION} \label{def:redundant-clause}
A clause \(C\in\C S\) with \(\max(C)=l\) is {\em redundant} in a set
of ground atoms \C A if either
\begin{itemize}\MyLPar
\item $\forc{\C L_l}C$ or 
%$C$ is forced by $\C L_l$  or 
\item the set \C S contains  another clause \(C'<C\) with \(\max(C')=l\),
such that $\notforc{\C L_l}{C'}$.
% \(C'\) is not forced by $\C L_l$.
\end{itemize}
\end{DEFINITION}
%
The nature of the second condition of the definition may not be very clear, but
thanks to this condition, the whole definition is a negated assertion about
some minimality of a clause. Statements of this kind are very appropriate in
inductive proofs, like our proof of completeness. 
The redundancy of literals is based on redundancy of clauses and Lemma~\ref
{le:preserve-confluency}.
\begin{DEFINITION} \label{def:redundant-atom}
A ground literal $l$ is {\em redundant} in a set of ground atoms \C A if either
\begin{itemize}\MyLPar
%changes
\item $l=s\oplus s$, where \(\oplus \in \{\notIncl,\notCont ,\notInt\}\), or
\item $\C A\cup\{l\}$ contains a rule $r$ overlapping $l$ and forming
 the critical literal $a$, such that $\notforc{\C A}a$, or
%where $a$ is the critical literal formed by $l$ and $r$, or
% is not forced by \(\C A\), or
\item every clause $C\in \C S$ with $\max(C)=l$ is redundant in \C A.
\end{itemize}
\end{DEFINITION}
We will write $\red{\C L}w$ to indicate that $w$ is redundant in \C L.
Observe that Definitions~\ref {def:redundant-atom} and \ref {def:forcing} of
redundancy and forcing are so related, that all negative literals that are not 
forced are redundant. Since any forced literal makes all clauses 
containing it redundant, any negative literal appears redundant. 
This overlapping of notions shortens our proofs.

After all the preliminary definitions the definition of the forcing set is quite
short. The set is defined as a limit of a decreasing sequence of sets, which 
begins with $\C L_0$ defined in (\ref{eq:max-atoms-set}). Succeeding sets are 
obtained removing minimal redundant literals. Suppose $\C L_i$ is already 
known, and let \(l_i\) be the minimal redundant literal in \(\C L_i\):
\begin{equation} \label{eq:atoms-model}
\C L_{i+1} \Def \C L_i \setminus \{l_i\}, \hspace{7em}
\C R \Def \bigcap_{i\in \Nat} \C L_i.
\end{equation}
%changes:
%Notice that $\C L_0$ may be empty and, even if it is not, \C R may be empty.
%These special cases do not cause much trouble but must be taken care of. If
%\(\C L_0 = \es\) then \(\SS+ = \es\), what implies that all clauses in \C S
%have non-trivial negative atoms (trivial ones like \(s\notIncl s\) or
%\(s\notInt s\) can be deleted by the reflexivity rule). Empty set \C R forces
%negation of any non-trivial atom.
%This makes any non-trivial atom in the antecedent of a clause false. 
%Hence, in this case, empty \C R forces all the
%clauses. The case when $\C L_0$ is not empty but \C R is, will be covered by
%Lemma~\ref {le:clauses-fromSplus}.
%
The next lemma shows that redundancy
is preserved when taking the limit in the definition of $\C R$,
and that redundancy of a word in some \(\C L_i\) is equivalent
to its redundancy in $\C R$.
\begin{LEMMA} \label{le:redundancy-limit}
For a literal $l\in\C L_0$ (a clause $C$ with \(\max(C)=l\)), the three conditions are
equivalent: 
\begin{itemize}\MyLPar
\item $\exists i, l_i>l : \red{\C L_i}l$
\item $\forall j>i: \red{\C L_j}l$
\item $\red{\C R}l$.
\end{itemize}
%\[ \exists i, l_i>l : \red{\C L_i}l \iff
% \forall j>i: \red{\C L_j}l \iff \red{\C R}l .\]
%is redundant in some
%$\C L_i$ with $l_i>l \iff$  it is redundant in every $\C L_j$ with
%$j>i \iff$ it is redundant in $\C R$.
\end{LEMMA}
As an immediate consequence of this lemma
%changes
and Lemma~\ref {le:preserve-confluency}
we obtain
\begin{COROLLARY} \label{co:model-confluent}
\C R is confluent.
\end{COROLLARY}

% \subsubsection{Productive clauses}

With every atom $a$ from \C R there is associated a clause from $\C S$ which
causes $a$ to be included in \C R. In \cite{S-A} such clauses are called {\em
regular}, in \cite{BG} {\em productive} because they produce atoms being
included in the forcing set. In \cite{PP}, where the forcing method is presented,
no special notion for clauses of this kind is used.
\begin{DEFINITION} \label{def:productive}
A clause $C,l$ with $\max(C,l)=l$ is {\em productive} for $l$ in a set \C L 
if $\notforc{\C L}C$.
% \C L does not force $C$.
\end{DEFINITION}
%In such situations we write $\prod{\C L}Cl$.
The last lemma of this subsection 
states the existence of productive clauses in somehow weaker form than
we need, but the stronger form will be given in the proof of the theorem.

\begin{LEMMA} \label{le:productive-clause}
For any $l\in\C R$, there exists a clause \(C\in\C S\)
which is productive for $l$ in \(\C R_l\).
\end{LEMMA}
\begin{PROOF}
 If $l\in \C R$ then $l$ is non-redundant. The negated form of the redundancy
 of a literal, 
 Definition~\ref{def:redundant-atom}, is a conjunction including the (last)
 condition that there is a non-redundant clause $C\in \C S$ with $\max(C)=l$.
 Non-redundancy of $C$, {\em i.e.}, negation of
 Definition~\ref{def:redundant-clause} implies that $C$ is not forced by \(\C
 R_l\), what means productivity of $C$ for $l$ in \(\C R_l\).
\end{PROOF}

\subsection{The main properties of \protect\C R} \label{se:main-R}
%changes
For any \(l\in\C R\) we denote by \(\C R_{l'}\) the set \(\C R_l\cup\{l\}\);
if \(l\notin\C R\) then $\C R_{l'}$ denotes simply \(\C R_l\).
%the same for \(l\notin\C R\) denotes simply \(\C R_l\). 
We can interpret \(l'\)
as the next literal after \(l\) in the set of words with respect to the order 
of literals.

%The following theorem about the relation between \C R and the 
%clauses from \C S is the main technical result of the paper:
%changes: almost new things
\begin{THEOREM} \label{le:main-theorem}
Let \C S be consistent and relatively closed set of ground clauses, \(\C L_0\) 
and \C R be as defined by (\ref {eq:max-atoms-set}) and (\ref {eq:atoms-model}).
Any literal \(l\in \C L_0\) satisfies the following conditions:
\begin{description}\MyLPar
\item[I1.] if \(\neg\redC{R}{l}\), then for any \(a\in\C R_l\) there
%$l$ is not redundant in \C R, 
  exists a clause $C\in \C S$ productive for $a$ in \(\C R_l\),
\item[I2.] if \(\red{\C R}l\), then for any clause   
  \(C\in \C S\) with \(\max(C)=l\) : $\forc{\C R_l}C$.
%$l$ is redundant in \C R, then \(\C R_l\) forces any clause
%  \(C\in \C S\) with \(\max(C)=l\).
\end{description}
\end{THEOREM}
The theorem is formulated in the form of an induction statement. 
After taking the
limit \(\C R=\bigcup_{l\in\C L_0} R_l\), the theorem
means that any atom in \C R has productive clause in \C S
(this is an auxiliary assertion) and that \(\forc{\C R}{\C S}\).
% is forced by \C R. 
This is the main technical result of this part.
%
\begin{PROOF} We assume that there exists some literal $l$ not satisfying
the theorem and that $l$ is minimal in \(\C L_0\) with this property.
Observe that any literal from \(\C L_0\) must satisfy one of the conditions 
I1 or I2, therefore we have two non-intersecting cases:
\begin{description}\MyLPar
\item[B1.] $l$, being included in \C R, ``spoils'' productiveness of some 
  clause \(C\in\C S\) with \(a=\max (C) \leq l\), {\em i.e.}, $C$ is productive
  for $a$ in \(\C R_l\), but is not productive in \(\C R_l\cup\{l\}\);
\item[B2.] $l$, being not included in \C R, leaves 
 some clause \(C\in\C S\) with \(\max (C)=l\) unforced by \(\C R_l\) 
(\(=\C R_{l'}\) in this case).
\end{description}

For both these cases we must get contradiction. The next lemma
will provide a uniform way to reach this goal.
\begin{LEMMA}\label {le:contradiction-way}
The following conditions about a clause $D$ and a literal $a$ cannot be 
satisfied simultaneously in a relatively closed \C S:
\newITEM C
\ITEM{iG}{ clause $D$ is a conclusion of some proof rule from \C I with
premises from \C S,}
\ITEM{iiG}{ for any clause \(D'\leq D\), if \(D\in\C S\), then 
 \(\forc{\C R_a}{D'}\),}
\ITEM{iiiG}{ \(\notforc{\C R_a}D\),}
%{ \(\C R_a\) does not force $D$,}
\ITEM{ivG}{\(\max(D)<a\).}
\end{LEMMA}
\begin{PROOF}
The condition \?{iG} and relative-closeness of \C S mean that the clause $D$
must be in \C S or be redundant in \C S. 
By \?{iiiG}  and \?{iiG}, the
clause \(D\notin\C S\), so it is redundant in \C S. The
redundancy of $D$ in \C S, by Lemma~\ref {le:redundancy-limit} and \?{ivG}, means
redundancy of $D$ in \(\C R_a\). The redundancy definition includes two cases:
either
\ITEM {vG}{$D$ is forced by $\C R_a$, or}
\ITEM {viG}{ $\C S$ contains a clause \(D'<D\) with \(\max(D')=l\) that is not
forced by $\C R_a$.} The case \?{vG} is excluded by \?{iiiG}. The case \?{viG} 
contradicts \?{iiG}.
\end{PROOF}

\begin{COROLLARY} \label{cor:contradiction-way}
Lemma~\ref {le:contradiction-way} holds if Condition~\?{iiG} is replaced by the
following  statement:  $a$ is the minimal literal in \(\C L_0\) satisfying
B1 or B2.
\end{COROLLARY}
\begin{PROOF}
Minimality of $a$ with respect to Condition~B2
means, that  for all clauses
$C\in\C S$ from \(\max(C)<a\) follows that \(\C R_a\) forces $C$. Any clause
\(D'\leq D\) by \?{ivG} has \(\max(D')<a\), and therefore satisfies \?{iiG}.
\end{PROOF}
\vspace{1ex}

\noindent %{\bf B1.} 
We continue the proof of the theorem by first:\\
\underline{Assuming B1.}
Productiveness conditions for $C$ in B1 mean that
\ITEM{2}{ $\max(C)=a$, and \(\notforc{\C R_l}{C\setminus\{a\}}\),}
%\(C\setminus\{a\}\) is not forced by \(\C R_l\),}
\ITEM{4}{ but there exists \(b\in C\) such that \(b\not = a\) and \(b\in \C
  R_{l'}^\ast\).}
Let $D,l$ be a clause that exists by Lemma~\ref {le:productive-clause}, {\em
i.e.},
\ITEM{1}{\(l=\max(D)\) and \(\notforc{\C R_l}D\).}
% $D$ is not forced by $\C R_l$.}
We want to prove that the factoring rule can be applied to
clauses \(D,l\) and $C$. The order of the atoms \(a,b,l\) is important:
\ITEM{6}{ $b< a\leq l$ (it follows from \?{2} and B1).}
The strong inequality between $a$ and $b$ follows from maximality of $a$ in 
$C$. By Lemma~\ref {le:first-rule} applied to atoms $l\not= b$, we get that,
if $l = s\oplus t$ and $b = s\odot u$, then there exists $\otimes$ such that
\ITEM{FA}{\(\odot = \Comp {\oplus^{-1}}\otimes\) and \(c = t\otimes u< b\).}
The first condition in \?{FA} is sufficient to apply  the factoring rule to 
\(D,l\) and $C$ and derive the clause \(E = (D,b,\neg c)\).
From \?{1}, \?{6} and \?{FA} it follows \(\max(E)<l\).
From Lemma~\ref {le:first-rule} we also have, that $l$ is used only once in
the proof of $b$, hence \(c\in\C R_l^\ast\) {\em i.e.}, 
\(\notforc{\C R}{\neg c}\).
This condition together with \?{1}, \?{2}, and \?{4} implies that
\(\notforc{\C R}E\).
% $E$ is not forced by \(\C R_l\).

The literal $l$ and the clause $E$ satisfy the conditions of Lemma~\ref
{le:contradiction-way}. Contradiction.\\[1ex]
%
\underline{Assuming B2,}
%\noindent {\bf B2.} Let us now assume that B2 is true.
there exists a clause $D$ such that
\ITEM{2i}{\(C=(D,l)\), \(\max(D)<l\) and \(\notforc{\C R_l}C\).} % ????
% $C$ is not forced by $\C R_l$.}
Assume that $C$ is minimal with this property.
The literal $l$ is redundant, and by Definition~\ref
{def:redundant-atom} there are three alternatives:
\ITEM{21}{ $l=s\oplus s$, where \(\oplus \in \{\notIncl,\notCont ,\notInt\}\),}
\ITEM{22}{ there is some atom $r\in \C R_l\cup\{l\}$ that overlaps $l$ 
forming the critical literal $a$ such that \(\notforc{\C R_l}a\),}
%   critical literal \(a\) formed by $l$ and $r$ is not forced by \(\C R_l\),}
\ITEM{23}{ for every clause $B\in \C S$ with $\max(B)=l$ : \(\red{\C R_l}B\).}
% is redundant in $\C R_l$.}

In the case of \?{21}, the reflexivity resolution rule can be applied to the
clause $C$ to produce $D$. By Lemma~\ref {le:contradiction-way} we derive 
contradiction in this case.

The alternative \?{23} is false because $C$ is non-redundant ---
Definition~\ref {def:redundant-clause} of redundancy subsumes the negated form
of the minimality assumption about $C$ which we have just made.

The alternative \?{22} after
unfolding Definition~\ref {def:critical-atom} of critical literals looks as
\ITEM{CA}{ $r=s\otimes t\in \C R_l$, $l=u[s]_p\oplus v$, and \(a = u[t]_p\odot 
v\), where $\odot = \Sup(s,u[s],\otimes,\oplus)$.}

Let \((C',r)\) be a productive clause for $r$ in \(\C R_r\) (that
exists by Lemma~\ref {le:productive-clause}).
From minimality of $l$ it follows, that no literal between $r$ and $l$
destroys productivity of $(C',r)$. (By the way, \(r=l\)  is possible.)
So, $(C',r)$ is also productive for $r$ in \(\C R_l\):
\ITEM{PCl}{ \(C'\cap \C R^\ast_l=\es\).}
 
By \?{CA}, there exists a clause
\ITEM{28}{\(D' = D,C',a\)}
deduced from clauses $C$ and $C',\,r$ by the superposition rule.
 The condition
\ITEM{9}{ $a<l$} is now derivable, but requires some care. The critical
literal definition forbids \(a=l\), therefore we must only exclude the case 
\(a>l\). If \(s>t\) and \(u[s]>v\), then by the properties of the simplification 
ordering the maximal term in $l$ is bigger than the maximal one in $a$ (and
thus \(l>a\)).

In the remaining cases at least one of $r$ or $l$ is a reflexive literal. 
Since \(\neg\red{\C R}r\) the only possible reflexive form is \(s\Eq s\).
%$r$ is non-redundant in \C R, it allows only \(s\Eq s\) as a reflexive form. 
By \?{2i}, \(\notforc{\C R_l}l\), so the only possibility for a positive 
reflexive $l$ is \(u[s]\Eq u[s]\).
%$l$ is not forced by \(\C R_l\) (by \?{2i}), so $l$ also allows only 
% \(u[s]\Eq u[s]\) as a positive reflexive form. 
If $l$ is negative, but not \(u[s]\notEq 
u[s]\), then we have already considered the case allowing us to apply to 
$C$ the reflexivity resolution rule.

It follows from this short analysis, that we must check 
the first and fifth columns (also at the first rows)
in Tables~\ref {tbl:composition} and \ref {tbl:replacement},
and verify that there is no  case when the predicate 
sign is changed from a smaller to a bigger one. 
In Table~\ref {tbl:composition} 
(the case \(u[s]=s\)) all nontrivial changes are presented. The case when both 
$r$ and $l$ are positive admits a sign change \(\Eq \to \Cont\).
Other cases with 
positive predicates are excluded by observation that \(l\geq r\), for example, 
cases like \(r=s\Eq s\) and \(l=s\Incl v\) are impossible thanks to \(\Eq > 
\Incl\). In the negative part of the table we find changes \(\notEq\to 
\notCont\) and \(\notIncl\to\notInt\). All these changes are decreasing 
according to the order defined in (\ref {eq:predicate-order}), what implies 
\(l\geq a\), and, finally, \?{9}.

Maximality of \(l\) and \(r\) in their clauses and \?{28} are sufficient to 
conclude $\max(D')<l$. Productiveness of clauses $C$, $C'$ and \?{22} imply 
that $D'$ is not forced in \(\C R_a\).  All conditions of Lemma~\ref 
{le:contradiction-way} hold, and contradiction follows.
\end{PROOF}


\subsection{From the forcing set to a multialgebra} \label{se:multimodel}

We have shown that for a consistent and relatively closed set \C S of ground
clauses, the set of ground atoms \C R is a model of \C S in the sense that it
forces all the clauses from \C S. To complete the proof of Theorem~\ref
{th:ground-completeness} we need to show that the existence of such an \C R
implies the existence of a multialgebra $A$ which satisfies all the atoms from
$\C R^\ast$ and only these ones. Then, from the definition of forcing 
it follows that $A$ also satisfies all the clauses \C S.
% \subsubsection{Partial orders in multialgebras}

The rewriting closure \(\C R^\ast\) defines a reflexive transitive relation
 `$\Incl$' on the set of terms \(\GTerms\).
% , namely
% \begin{itemize}\MyLPar
%\item for each term \(s\in\GTerms\), \(s\Incl s\in\C R^\ast\) (reflexivity of
% preorder); 
% \item if \(s\Incl t,\,t\Incl u\in\C R^\ast\), then \(s\Incl u\in \C R^\ast\)
% (transitivity of preorder).
% \end{itemize}
In multialgebras this partial (pre)order `\(\Incl\)' on terms is interpreted as
set inclusion: an atom \(s\Incl t\) means that
\(\Interpret s\subseteq \Interpret t\).
% , where for every term \(w\in\GTerms\),
% \(\Interpret w\) denotes some nonempty subset of the carrier. 
Two other predicates
of our language also have natural interpretation in partial order terms:
\begin{itemize}\MyLPar
\item \(s\Eq t\) means that both sets \(\Interpret s\) and \(\Interpret t\) are
equal {\em minimal} elements in the partial order of nonempty sets, 
{\em i.e.}, they denote the same set {\em with one element};
\item \(s\Int t\) means that there exists some minimal element $\alpha$ such that
\(\alpha\in\Interpret s\) and \(\alpha\in\Interpret t\).
\end{itemize}
% Hence the inclusion as partial order and the property of being a 
% 1-element set are the main relations of our language. 
The relation `$\Incl$' on the
set \(\GTerms\) is partial preorder, because different terms may have the
same value set. To turn it into a partial order 
we have to take the quotient of \(\C R^\ast\) modulo
`$\Seteq$' that was defined in (\ref {eq:Seteq-definition}).  
Since $\Seteq$ denotes set equality, it is obviously a congruence: 
reflexivity and
transitivity follow from the analogous properties of inclusion `$\Incl$',
symmetricity from (\ref {eq:Seteq-definition}), and
the replacement property is given in (\ref {le:replacement}).

We will first construct, from a set \C R of ground atoms, a partially
ordered set \(PO(\C R)=\<\C C,\Eineq\>\). 
% where \(\C C\) is a set of elements
% and `$\Eineq$' is a partial order relation on \C C. 
Then, we will extend
signature with new constants and \C R with new atoms so that \(PO(\C R)\)
will define a multialgebra, satisfying all the atoms from \C R.

Let \(\C C\Def \C R^\ast/_{\Seteq}\) be the quotient of \(\C R^\ast\) modulo
`$\Seteq$'.
\([t]\Def \{s\in\GTerms: s\Incl t\in\C R^\ast\land t\Incl s\in\C R^\ast\}\)
denotes the equivalence class in \C C of a term $t$, and  
\(\Eineq\ \Def\{\<[s],[t]\>: s\Incl t\in\C R^\ast \}\)
is a partial order on \C C (`$\Ein$' is the irreflexive part of this
relation).
(That \(\<\C C,\Eineq\>\) is well defined, i.e, that all atoms in one
equivalence class stand in the same relations, follows from Table~\ref 
{tbl:composition}: for any atom
\(s\oplus t\), any \(t\Incl u\) or \(t\Cont u\) is enough to derive
also \(s\oplus u\).)
Let \(\C D\Def \{[s]:s\in\GTerms,\, s\Eq s\in\C R^\ast\}\)
be the set of {\em deterministic} elements in \C C,
\(\C D(T)\Def \{S\in\C D:S\Eineq T\}\) be
the set of deterministic elements which are smaller than or equal to $T\in\C
C$, and 
\(\C M\Def \{S\in\C C:\forall T\in\C C\;T\Not\Ein S\}\)
be the set of {\em minimal} elements in \C C. \(\C M(S)\) denotes the set of
elements from \C M that are smaller than or equal to $S$. We will write \(\C
M[t]\) (or \(\C D[t]\)) instead \(\C M([t])\) (respectively, \(\C D([t])\)). 

The set \C D is a subset of \C M, because from \(s\Eq s\) and \(s\Cont u\)
it follows that \(s\Eq u\), meaning that no elements lie below the class \([s]\) if
\([s]\in\C D\). The set \C D is a ``candidate'' to be a carrier of a
multialgebra, and \(\C D(T)\) should be an interpretation mapping defining the
multialgebra.  Definition~\ref {def:semantics} tells us what properties
$\C D[\_]$ should have in order to be an interpretation mapping:
\begin{description}\MyLPar
\item [PO1.] \(\C D [f(\List tn,)] =
\bigcup\{\C D [f(\List\alpha n,)]:\alpha_i\in \C D [t_i]\}\) 
for any \(f\in \Funcs^n\), \(\{\List tn,\}\subset
\GTerms\);
\item [PO2.] \(\C D [s]=\{[s]\}\iff s\Eq s\in\C R^\ast\);
\item [PO3.] \(\C D [s]\subseteq \C D [t] \iff s\Incl t\in\C R^\ast\);
\item [PO4.] \(\C D [s]\cap \C D [t]\not=\es \iff s\Int t\in\C R^\ast\).
\end{description}
\noindent If the mapping \(\C D[\_]\) satisfies PO1--PO4, then a multialgebra $A$
satisfying \C R can be defined:
\begin{description}\MyLPar
\item[MA1.] the carrier \(S^A\Def \C D\);
\item[MA2.] for any constant \(c\in\Funcs\): \(c^A\Def \C D[c]\);
\item[MA3.] for any \(f\in\Funcs^n\) and
\(\{[d_1] ,\ldots, [d_n]\}\subseteq\C D\):
\(f^A([d_1] ,\ldots, [d_n])\Def \C D[f(\List dn,)]\).
\end{description}
In this case \(\Interpret\_=\C D[\_]\), what follows from PO1, and we say that
\C R {\em defines} the multialgebra $A$. The next result is important, because
the multialgebra is constructed from positive atoms only, while clauses from
\C S may also contain negative atoms. We must be sure that the multialgebra
makes true only atoms derivable from the forcing set \C R.
\begin{LEMMA} \label{le:MA-exact}
If \C R defines a multialgebra $A$ then, for any atom \(a :
a\in\C R^\ast \iff \Interpret a\) is true.
\end{LEMMA}
However, the mapping $\C D[\_]$ defined from the forcing set \C R may violate any
of the Requirements~PO1--PO4. We might therefore consider the mapping \(\C M[\_]\)
instead, but also this one may violate these requirements.
To meet these problems we will extend
 \C C (and mapping \(\C D[\_]\)) with new minimal elements. Making
 all new elements deterministic, will also make mappings \(\C D[\_]\) and
\(\C M[\_]\) coincide.
%\subsubsection{Conservative extension of the forcing set}
\begin{DEFINITION} \label{def:conservative-extension}
Let \(\Funcs\subset\Funcs_1\) be two signatures and \(\C R, \C R_1\) two sets of
atoms over signature $\Funcs,$ resp. \(\Funcs_1.\)
\(\C R_1\) is a {\em conservative extension} of \C R if for any 
($\Funcs$-)atom \(a\), \(a\in\C R_1^\ast\iff a\in\C R^\ast\).
\end{DEFINITION} 

\newITEM H
\begin{THEOREM} \label{th:multialgebra-exists}
For any atom set \(\C R\) there exists an atom set \(\C R_1\) that is a
conservative extension of \(\C R\) and defines a multialgebra.
\end{THEOREM}
\begin{PROOF}
Let \(M=PO(\C R)\) be a partially ordered set as defined above. We first extend
\C R with some new (deterministic) constants and atoms obtaining \(\C R_1\). Then
we show that \(\C R_1\) is a conservative extension of \C R and, finally, that
\(PO(\C R_1)\) satisfies the conditions PO1--PO4. This will imply that 
\(\C R_1\) defines a multialgebra satisfying exactly the atoms from \C R.

For each nondeterministic element
\(T\in\C C\setminus \C D\) add to $\C M$ a new deterministic element \(\chi(T)\),
called the {\em characteristic} element for $T$. Each of these elements is 
considered as an
equivalence class \([c]=\{c\}\) containing only one new constant. Let \(\Psi\Def
\{c: [c]=\chi(T), T\in\C C\setminus\C D\}\) be the set of these new constants. 
The main
property of the characteristic elements is
\begin{equation} \label{eq:charact-constant}
\chi(T)\Ein V\iff T\Eineq V.
\end{equation}

This corresponds to inclusion in \(\C R_1\) of all atoms
\ITEM{char-elem}{ \(c\Eq c\) and \(c\Incl t\), where \(c\in\Psi\), \(t\in T\)
and \(\chi(T)=[c]\).} 
The mapping \(\chi\) is injective, so \(\chi^{-1}\) is a partial function. We
extend the domain of \(\chi\) to the whole set \C C defining, for any \(D\in\C D\),
\(\chi(D)\Def D\). This extension preserves injectivity of \(\chi\).

This addition of characteristic elements makes PO2 and PO3 true.

To meet the requirement PO4, we choose a new set of constants \(\Phi\),
\(\Phi\cap(\Funcs\cup\Psi)=\es\). 
For any terms \(s,t\) violating PO4, a new constant \(c\in\Phi\)
and the atoms
\ITEM{int-atoms}{\(c\Eq c\), \(c\Incl s\) and \(c\Incl t\)}
are added to \(\C R_1\).This will make PO4 true.
% Each \(c\in\Phi\) corresponds to one pair of corresponding
% classes \(\{[s], [t]\}\). 

We are left with PO1. In fact, as a consequence of the extensions, new cases
violating PO1 may occur. To satisfy this last requirement, observe that
we can define the function-values on the new elements in an arbitrary way
(as long as we do not violate monotonicity). We need to include the following
 new atoms:
\ITEM{char-atoms}{\(t\Incl s\) for any terms \(t=f(\List tn,
)\in\GTerms\) and \(s=f(\List dn,)\notin\GTerms\), where \(\chi[t_i]
= [d_i]\).}
% for any \(i\in[1,n]\).} 
This will imply inclusion of the term $s$ in the
equivalence class $[t]$ (because the opposite inclusion \(s\Incl t\) will hold
in the rewriting closure of \(\C R_1\) by definition).

To show that the set \(\C R_1\) is a conservative extension of \(\C R\),
notice that all atoms in \(\C R_1\setminus\C R\) contain terms not
belonging to \(\GTerms\). Furthermore, in the cases \?{char-elem} and
\?{int-atoms} these new atoms have the form \(c\Eq c\) or \(c\Incl t\) for some
new constant $c$ and some nondeterministic term \(t\in\GTerms\). Non-trivial
rewriting steps produced by such atoms (rules) have the form \(u[c]_p\Incl
u[t]_p\). If such a step is used in a rewriting proof, but the constant \(c\)
does not occur in the resulting atom, at least two steps of this kind must be
in the proof and they can be made consecutive. Any such two-step proof derives a
(nontrivial) atom \(u[v]_p\Int u[t]_p\), where \(c\Incl t\) and \(c\Incl v\)
are in \(\C R_1\). But this does not add anything to \(\C R^\ast\), if \(t\) and
\(v\) are in the same equivalence class. If they are not, then we have the
case \?{int-atoms}, and \(v\Int t\in\C R^\ast\), so again noting new is added
to \C R. 

In the case \?{char-atoms}, all we can prove about
\(s=f(\List dn,)\) without using the fact \(s\Cont t\), are inclusions like
\(u[s]_p\Incl u[v]_p\) for \(v\Cont t\). This observation is based on the
uniqueness of characteristic elements. But this means, that the fact \(s\Cont
t\) does not add anything new to the knowledge about \(t\) already contained in \C
R. This ends the proof of conservativeness of \(\C R_1\) with respect to \C R.

Before the final proof of the properties PO1--PO4, we have to include
in \(\C R_1\) all atoms 
\ITEM{min-terms}{\(t\Eq t\) for any class \([t]\in PO(\C R_1)\) that is
minimal (but not yet deterministic),} thus forcing all minimal elements in \(PO(\C
R_1)\) to be deterministic. Because of the characteristic elements, 
all these \([t]\) are not in \(PO(\C R)\). Hence the extension preserves
conservativeness of \(\C R_1\). \\[6pt]
\noindent {\bf PO2.} After this last extension PO2 is satisfied by \(PO(\C R_1)\).\\[6pt]
\noindent {\bf PO4.} There are no atoms of the form \(s\Int t\) in \(\C R_1\setminus
\C R\), so \(PO(\C R_1)\) satisfies PO4.\\[6pt]
\noindent {\bf PO3.} PO3 is satisfied for any \(S\Ein T\), if \(T\in PO(\C R)\), 
because of the addition of the characteristic elements. 
If \(T\notin PO(\C R)\), there are terms 
% atoms
\(s\in S\) and \(t\in T\), and in a rewriting proof of \(s\Incl t\) there is a
rewriting step \(u[v]\Incl u[w]\), such that the atom \(v\Incl w\) is in \(\C
R_1\), but the atom \(w\Incl v\notin\C R_1^\ast\). If \(w\in\GTerms\) is
nondeterministic in \(\C C\) and \([c]=\chi[w]\), then \(u[z]\Incl u[w]\) is
provable in \(\C R_1\), but \(u[z]\Incl u[v]\) is not, because of (\ref
{eq:charact-constant}). From this it also follows that the atom \(u[z]\Incl s\)
is not provable in \(\C R_1\), but \(u[z]\Incl t\) is.  If \(v\Incl w\) is in
\C R, then \(w\) is nondeterministic, otherwise, it is easy to check all the cases
 \?{char-elem}, \?{int-atoms}, \?{char-atoms}, and \?{min-terms} and to see
that \(w\in\GTerms\) and is nondeterministic, too. Thus
\(PO(\C R_1)\) satisfies PO3.\\[6pt]
\noindent {\bf PO1.} Let
\ITEM{singularity}{\(t=f(\List tn,)\), \(s=f(\List zn,)\), \(z_1\Incl t_1
,\ldots,z_n\Incl t_n\in\C R_1^\ast\), and all \(\{\List zn,\}\) be
deterministic.} 
If $z$ is deterministic and \(z\Incl s\in\C R_1^\ast\), then also \(z\Incl
t\in\C R_1^\ast\), because any \(z_i\) may be rewritten to
\(t_i\). This proves one inclusion for PO1. Let now $z$ be a deterministic
term, \(z\Incl t\in\C R_1^\ast\). We must find \(s,\List zn,\) such as in
\?{singularity}, and show \(z\Incl s\in\C R^\ast_1\). For this it is
enough to take \(z_i=\chi(t_i)\) and use \?{char-atoms}, which states \(t\Incl
s\) for this case. This proves PO1 for \(PO(\C R_1)\).
\end{PROOF}

This ends the proof of the completeness theorem which also yields:
\begin{COROLLARY} \label{co:rew-completeness}
Any ground atom valid in all multimodels of a given set of ground atoms \C R 
has a rewriting proof in \C R.
\end{COROLLARY}


