% This is LLNCS.DOC the documentation file of
% the LaTeX macro package from Springer-Verlag
% for Lecture Notes in Computer Science, version 1.2
\documentstyle[llncsdoc]{llncs}
%
\newfont{\titelfont}{cmr10 scaled 1728}
\newfont{\titelbffont}{cmbx10 scaled 2074}
\newfont{\titelbigfont}{cmr10 scaled 2488}
\markboth{Style File for Authors Coding with \LaTeX{}}{Style File
for Authors Coding with \LaTeX{}}
%
\begin{document}
\thispagestyle{empty}
\begin{flushleft}
{\titelbffont Style File}\\[6pt]
{\titelbffont for Authors Coding with \LaTex{}}\\[2cm]
\end{flushleft}
\rule{\textwidth}{1pt}
\vspace{2pt}
\begin{flushright}
\begin{tabular}{@{}l}
{\titelbigfont Lecture Notes}\\[7pt]
{\titelbigfont in Computer Science}\\[10pt]
{\titelfont Version 1.1}
\end{tabular}
\end{flushright}
\rule{\textwidth}{1pt}
\vfill
\begin{flushright}
\begin{tabular}{@{}l}
{\titelfont Springer-Verlag}\\[8pt]
{\titelfont Berlin\enspace Heidelberg\enspace New\kern0.1em York}\\[5pt]
{\titelfont London\enspace Paris\enspace Tokyo}\\[5pt]
{\titelfont Hong Kong\enspace Barcelona\enspace Budapest}
\end{tabular}
\end{flushright}
\newpage
%
\section*{For further information please contact us:}
%
\begin{flushleft}
\begin{tabular}{l@{\quad}l@{\hspace{3mm}}l}
$\bullet$  & \multicolumn{2}{l}{\bf Springer-Verlag Heidelberg}\\[1mm]
& \multicolumn{2}{l}{Department New Technologies/Product Development}\\
& \multicolumn{2}{l}{Springer-Verlag, Postfach 105280, W-6900 Heidelberg
1, FRG}\\[0.5mm]
 & Telex:     & 461723\\
 & Telefax:   & (06221)487648\\
 &bitnet/EARN:& SPRINGER @ DHDSPRI6
\end{tabular}
\end{flushleft}
\rule{\textwidth}{1pt}
%
\section*{Acceptable formats of your disk/magnetic tape and output:}
%
The following formats are acceptable: 5.25$^{\prime\prime}$ diskette
MS-DOS, 5.25$^{\prime\prime}$ CP/M, 3.5$^{\prime\prime}$ diskette
MS-DOS, 3.5$^{\prime\prime}$ diskette Apple MacIntosh, 9-track 1600
bpi magnetic tape VAX/VMS, 9-track 1600 bpi magnetic tape ANSI with
label, SUN-Streamer Tape.

Once you have completed your work using this macro package,
please submit your own printout of the {\it final
version together with the disk or magnetic tape}, containing your
\LaTeX{} input (source) file und the final DVI-file and make sure
that the text is {\it identical in both cases.}

\bigskip
This macro package, as well as all other macro packages and style
files that Springer distributes, are also available through our
mailserver

SVSERV @ DHDSPRI6.bitnet
%
\newpage
\tableofcontents
\newpage
%
\section{Introduction}
%
Authors wishing to code their contribution
with \LaTeX{}, as well as those who have already coded with \LaTeX{},
will be provided with a style file that will give the text the
desired layout. Authors are requested to
adhere strictly to these instructions; {\it the style
file must not be changed}.

The text output area is automatically set within an area of
12.2\,cm horizontally  and 19.3\,cm vertically.

If you are already familiar with \LaTeX{}, then the
LLNCS style file should not give you any major difficulties.
This style file will change the layout to the required LLNCS style
(it will for instance  define the layout of \verb|\section|).
We had to invent some extra commands,
which are not provided by \LaTeX{} (e.g.\
\verb|\institute|, see also Sect.\,\ref{contbegin})

For the main body of the paper (the text) you
should use the commands of the standard \LaTeX{} ``article'' style.
Even if you are familiar with those commands, we urge you to read
this entire documentation thoroughly. It contains many suggestions on
how to use our commands properly; thus  your paper
will be formatted exactly to LLNCS standard.
For the input of the references at the end of your contribution,
please follow our instructions given in Sect.\,\ref{refer} References.

The majority of these hints are not specific for LLNCS; they may improve
your use of \LaTeX{} in general.
Furthermore, the documentation provides suggestions about the proper
editing and use
of the input files (capitalization, abbreviation etc.) (see
Sect.\,\ref{refedit} How to Edit Your Input File).
%
\section{How to Proceed}
%
Please insert the enclosed diskette or tape into your computer. You will
find the following files:
\begin{flushleft}
\begin{tabular}{@{}p{2.5cm}l}
{\it llncs.doc}  & General instructions (this document),\\
        & {\it llncs.doc} means llncs latex documentation\\
{\it llncs.dem}  & An example showing how to code the text\\
{\it llncs.sty}  & Style file to invoke  \LaTeX{}
\end{tabular}
\end{flushleft}
You have to run each file twice with: \verb|latex file.name|.
%
\subsection{How to Invoke the LLNCS Style File}
%
The LLNCS style file is an adaptation of the \LaTeX{} ``article'' style.
Therefore you may use all ``article'' style commands to prepare your
manuscript.
LLNCS style is invoked by replacing ``article'' by ``llncs'' in the
first
line of your document:
\begin{verbatim}
\documentstyle{llncs}
%
\begin{document}
  <Your contribution>
\end{document}
\end{verbatim}
%
\subsection{Contributions Already  Coded with \protect\LaTeX{} without
the \protect\\ LLNCS Style File}
%
If your file is already coded with \LaTeX{} you can easily
adapt it a posteriori to LLNCS style.

Please refrain from using any \LaTeX{} or \TeX{} commands
that affect the layout or formatting of your document (i.e. commands
like \verb|\textheight|, \verb|\vspace|, \verb|\hoffset| etc.).
There may nevertheless be exceptional occasions on which to
use some of them.

The LLNCS style has been carefully designed to produce the right layout
from your \LaTeX{} input. If there is anything specific you would like
to do and  for which the style file does not provide a command,
{\em please contact us.\/}
%
\section{General Rules for Coding Formulas}
%
With mathematical formulas you may proceed as described
in Sect.\,3.3 of the {\em \LaTeX{} User's Guide \& Reference
Manual\/} by Leslie Lamport (1986), Addison-Wesley Publishing
Company, Reading, Mass.

Equations are automatically numbered sequentially throughout your
contribution using arabic numerals in parentheses on the right-hand
side.

When you are working in math mode everything is typeset in italics.
Sometimes you need to insert non-mathematical elements (e.g.\
words or phrases). Such insertions should be coded in  roman
(with \verb|\mbox|) as illustrated in the following example:
\begin{flushleft}
{\it Sample Input}
\end{flushleft}
\begin{verbatim}
\begin{equation}
  \left(\frac{a^{2} + b^{2}}{c^{3}} \right) = 1 \quad
  \mbox{ if } c\neq 0 \mbox{ and if } a,b,c\in \bbbr \enspace .
\end{equation}
\end{verbatim}
{\it Sample Output}
\begin{equation}
  \left(\frac{a^{2} + b^{2}}{c^{3}} \right) = 1 \quad
  \mbox{ if } c\neq 0 \mbox{ and if } a,b,c\in \bbbr \enspace .
\end{equation}

If you wish to start a new paragraph immediately after a displayed
equation, insert a blank line so as to produce the required
indentation. If there is no new paragraph either do not insert
a blank line or code \verb|\noindent| immediately before
continuing the text. Titles have no end punctuation.

Please punctuate a displayed equation in the same way as other
ordinary  text but with an \verb|\enspace| before end punctuation.

Note that the sizes of the parentheses or other delimiter
symbols used in equations should ideally match the height of the
formulas being enclosed. This is automatically taken care of by
the following \LaTeX{} commands:\\[2mm]
\verb|\left(| or \verb|\left[| and
\verb|\right)| or \verb|\right]|.
%
\subsection{Italic and Roman Type in Math Mode}
%
\begin{alpherate}
\item
In math mode \TeX{} and \LaTeX{} treat all letters as though they
were mathematical or physical variables, hence they are typeset in
italics. However, for certain components of formulas, like short texts,
this would be incorrect and therefore coding in  roman is required.
Roman should also be used for
subscripts and superscripts {\it in formulas\/} where these are
merely labels and not in themselves variables,
e.g.\ $T_{\rm eff}$ \ not \ $T_{eff}$,
$T_{\rm K}$ \ not \ $T_K$ (K = Kelvin),
$m_{\rm e}$ \ not \ $m_e$ (e = electron).
However, do not code for roman
if the sub/superscripts represent variables,
e.g.\ $\sum_{i=1}^{n} a_{i}$.
\item
Please ensure that {\em physical units\/} (e.g.\ pc, erg s$^{-1}$
K, cm$^{-3}$, W m$^{-2}$ Hz$^{-1}$, m kg s$^{-2}$ A$^{-2}$) and
{\it abbreviations} such as Ord, Var, GL, SL, sgn, const.\
are always set in roman type. To ensure
this use the \verb|\rm| command: \verb|{\rm Hz}|.
On p.\ 46 of the {\em \LaTeX{} User's Guide \& Reference
Manual\/} by Leslie Lamport you will find the names of
common mathe\-matical functions, such as log, sin, exp, max and sup.
These should be coded as \verb|\log|,
\verb|\sin|, \verb|\exp|, \verb|\max|, \verb|\sup|
and will appear in roman.
\item
Chemical symbols and formulas should be coded for roman,
e.g.\ Fe not $Fe$, H$_2$O not {\em H$_2$O\/}.
\item
Familiar foreign words and phrases, e.g.\ et al.,
a priori, in situ, brems\-strah\-lung, eigenvalues should not be
italicized.
\end{alpherate}
%
\section{How to Edit Your Input (Source) File}
\label{refedit}
%
\subsection{Headings}
%
All words in  headings should be capitalized except for
conjunctions,  prepositions
(e.g.\ on, of, by, and, or, but,  from, with, without, under)
and definite and indefinite articles (the, a, an) unless they appear at
the beginning. Formula letters must be typeset as in the text.
%
\subsection{Capitalization and Non-capitalization}
%
\begin{alpherate}
\item
The following should always be capitalized:
\begin{itemize}
\item
Headings (see preceding Sect.\,4.1)
\item
Abbreviations and expressions
in the text such as  Fig(s)., Table(s), Sect(s)., Chap(s).,
Theorem, Corollary, Definition etc. when used with numbers, e.g.\
Fig.\,3, Table\,1, Theorem 2.
\end{itemize}
Please follow the special rules in Sect.\,4.3 for referring to
equations.
\item
The following should {\it not\/} be capitalized:
\begin{itemize}
\item
The words  figure(s), table(s), equation(s),
theorem(s) in the text when used without an accompanying number
\item
Figure legends and table captions except for names and
abbreviations.
\end{itemize}
\end{alpherate}
%
\subsection{Abbreviation of Words}
%
\begin{alpherate}
\item
The following {\it should} be abbreviated when they appear in running
text {\it unless\/} they come at the beginning of a sentence: Chap.,
Sect., Fig.; e.g.\ The results are depicted in Fig.\,5. Figure 9 reveals
that \dots .\\
{\it Please note\/}: Equations should usually be referred to solely by
their number in parentheses: e.g.\ (14). However, when the reference
comes at the beginning of a sentence, the unabbreviated word
``Equation'' should be used: e.g.\ Equation (14) is very important.
However, (15) makes it clear that \dots .
\item
If abbreviations of names or concepts are used
throughout the text, they should be defined at first occurrence,
e.g.\ Plurisubharmonic (PSH) Functions, Strong Optimization (SOPT)
Problem.
\end{alpherate}
%
\section{How to Code the Beginning of Your Contribution}
\label{contbegin}
%
The title of your contribution should be coded as follows:
\begin{verbatim}
\title{<Your contribution title>}
\end{verbatim}
All words in  titles should be capitalized except for
conjunctions, prepositions
(e.g.\ on, of, by, and, or, but,  from, with, without, under)
and definite and indefinite articles (the, a, an) unless they appear at
the beginning. Formula letters must be typeset as in the text.
Titles have no end punctuation.

If a long \verb|\title| must be divided please use the
code  \verb|\\| (for new line).\\[6mm]
Now the name(s) of the authors(s) must be given:
\begin{verbatim}
\author{<author(s) name(s)>}
\end{verbatim}
Numbers referring to different addresses are
to be attached to each author.
If you have done this correctly, the entry now reads, for example:
\begin{verbatim}
\author{Ivar Ekeland\inst{1} and Roger Temam\inst{2}}
\end{verbatim}
The first name\footnote{Other initials are optional
and may be inserted if this is the usual
way of writing your name, e.g.\ Alfred J.~Holmes, E.~Henry Green.}
is followed by the surname.

If there is more than one author, the order is up to you.
However, if there are more than two authors, you must separate the names
by commas. If the authors have different affiliations,
each name has to be followed by:
\begin{verbatim}
\inst{<no>}
\end{verbatim}
\newpage
Next the address(es) of institute(s), company etc. is (are) required.
If there is more than one address, the entries are numbered
automatically with \verb|\and|, in the order in which you type them.
Please make sure that the numbers match those placed next to
to the authors' names.
\begin{verbatim}
\institute{<name of an institute>
\and <name of the next institute>
\and <name of the next institute>}
\end{verbatim}
Unlike usual \LaTeX{} the \verb|\and| command is used with the
\verb|\institute| and not with the \verb|\author| command.

\medskip
If footnotes are needed in \verb|\title| please code
(immediately after the word where the footnote
indicator should be placed):
\begin{verbatim}
\thanks{<text>}
\end{verbatim}
\verb|\thanks| may only appear in \verb|\title|, \verb|\author|
and \verb|\institute| (see below) to footnote anything.

\medskip\noindent
The command
\begin{verbatim}
\maketitle
\end{verbatim}
formats the complete heading of your article. If you leave
it out the work done so far will produce {\bf no} text,
see {\it Sample Input\/} on p.~\pageref{samppage}.

Then the abstract should follow. Please refer to the
demonstration file {\tt llncs.dem} for an example or
to the {\it Sample Input\/} like above.
%
\section{How to Code Your Text}
%
The contribution title and all headings should be capitalized
except for conjunctions, prepositions
(e.g.\ on, of, by, and, or, but,  from, with, without, under)
and definite and indefinite articles (the, a, an) unless they appear at
the beginning. Formula letters must be typeset as in the text.

Headings will be automatically numbered by the following codes.\\[2mm]
{\it Sample Input}
\begin{verbatim}
\section{This is a First-Order Title}
\subsection{This is a Second-Order Title}
\subsubsection{This is a Third-Order Title.}
\paragraph{This is a Fourth-Order Title.}
\end{verbatim}
\verb|\section| and \verb|\subsection| have no end punctuation.\\
\verb|\subsubsection| and  \verb|\paragraph|
need to be punctuated at the end.

In addition to the above-mentioned headings your text may be structured
by subsections indicated by run-in headings (theorem-like environments).
All the theorem-like environments are numbered automatically
throughout the sections of your file.
If you call lemma once, this will be numbered 1; if corollary follows,
this will be numbered 2; if you then call lemma again, this will be
numbered 3.
\newpage
But in case you want to reset this counter at 1 in each section,
please give the  document style option \verb|envcountreset|:
\begin{verbatim}
\documentstyle[envcountreset]{llncs}
\end{verbatim}

\vspace{.5cm}\noindent
The following possibilities for run-in headings are available:
\begin{flushleft}
\verb|\begin{lemma} Text of the lemma \end{lemma}|\quad (see Output
Sample 1)\\[2mm]
%
\verb|\begin{lemma}[additional explanation of lemma] Text of
lemma \end{lemma}| \quad (see Output Sample 2)\\[2mm]
%
\verb|\begin{lemma}(overriding automatic numbering.) Text of the lemma
\end{lemma}| \quad (see Output Sample 3)\\[2mm]
%
\verb|\begin{lemma}(x)[additional explanation of lemma] Text of
lemma\end{lemma}| \quad (see Output Sample 4)\\[2mm]
%
\verb|\begin{lemma}* Text of lemma \end{lemma}| \quad (see Output
Sample 5)\\[2mm]
%
\verb|\begin{lemma}*[additional explanation of lemma] Text of
lemma\end{lemma}| \quad (see Output Sample 6)
\end{flushleft}
%
{\it Output Sample 1}
\begin{lemma}Text of lemma\end{lemma}
%
{\it Output Sample 2}
\begin{lemma}[additional explanation of lemma] Text of lemma\end{lemma}
%
{\it Output Sample 3}
\begin{lemma}(overriding automatic numbering.) Text of lemma\end{lemma}
%
{\it Output Sample 4}
\begin{lemma}(x)[additional explanation of lemma] Text of
lemma\end{lemma}
%
{\it Output Sample 5}
\begin{lemma}* Text of lemma\end{lemma}
%
{\it Output Sample 6}
\begin{lemma}*[additional explanation of lemma] Text of lemma\end{lemma}
\noindent
The following variety of run-in headings are at your disposal:
\begin{alpherate}
\item
{\bf Bold} run-in headings with italicized text
as built-in environments:
\begin{verbatim}
\begin{proposition} <text> \end{proposition}
\begin{corollary} <text> \end{corollary}
\begin{lemma} <text> \end{lemma}
\begin{theorem} <text> \end{theorem}
\end{verbatim}
\item
The following must generally appear as {\it italic} run-in heading:
\begin{verbatim}
\begin{proof} <text>    \qed    \end{proof}
\end{verbatim}
\item
Further {\it italic} or {\bf bold} run-in headings may also occur:
\begin{verbatim}
\begin{definition} <text> \end{definition}
\begin{example} <text> \end{example}
\begin{remark} <text> \end{remark}
\begin{exercise} <text> \end{exercise}
\begin{problem} <text> \end{problem}
\begin{solution} <text> \end{solution}
\begin{note} <text> \end{note}
\begin{question} <text> \end{question}
\end{verbatim}
\end{alpherate}
%
\subsubsection*{Defining Your Own Environments.}
%
You can define additional environments using the command
\verb|\newstytheorem| which has five parameters. The first is the name
your environment should have (e.g.\ \verb|conjecture|).
After this follows the
font family used for this heading (please use only \verb|\bf| for bold
or \verb|\it| for italic) and the font family to use for the text of
this new environment (e.g.\ \verb|\it| or \verb|\rm|). Then the name of
an already known environment should be given in brackets (e.g.\
\verb|[theorem]|). Your new environment will be numbered like the old
one. Finally comes the real text of the new run-in heading (e.g.\
\verb|Conjecture|).\\
Sample definition:
\begin{verbatim}
\newstytheorem{conjecture}{\bf}{\it}[theorem]{Conjecture}
\end{verbatim}
\newstytheorem{conjecture}{\bf}{\it}[theorem]{Conjecture}
Use of that definition:
\begin{verbatim}
\begin{conjecture} <text> \end{conjecture}
\end{verbatim}
e.g.
\begin{verbatim}
\begin{conjecture} It is clear that ...\end{conjecture}
\end{verbatim}
Its output:
\begin{conjecture}
It is clear that \dots
\end{conjecture}
%
\noindent
{\it Sample Input}
\label{samppage}
\begin{verbatim}
\title{Hamiltonian Mechanics}

\author{Ivar Ekeland\inst{1} and Roger Temam\inst{2}}

\institute{Princeton University, Princeton NJ 08544, USA
\and
Universit\'{e} de Paris-Sud,
Laboratoire d'Analyse Num\'{e}rique, B\^{a}timent 425,\\
F-91405 Orsay Cedex, France}

\maketitle
%
\begin{abstract}
This paragraph shall summarize the contents of the paper
in short terms.
\end{abstract}
%
\section{Fixed-Period Problems: The Sublinear Case}
%
With this chapter, the preliminaries are over, and we begin the
search for periodic solutions \dots
%
\subsection{Autonomous Systems}
%
In this section we will consider the case when the Hamiltonian
$H(x)$ \dots
%
\subsubsection*{The General Case: Nontriviality.}
%
We assume that $H$ is
$\left(A_{\infty}, B_{\infty}\right)$-subqua\-dra\-tic
at infinity, for some constant \dots
%
\paragraph{Notes and Comments.}
The first results on subharmonics were \dots
%
\begin{proposition}
Assume $H'(0)=0$ and $ H(0)=0$. Set \dots
\end{proposition}
\begin{proof}[of proposition]
Condition (8) means that, for every $\delta'>\delta$, there is
some $\varepsilon>0$ such that \dots \qed
\end{proof}
%
\begin{example}[\rm (External forcing)]
Consider the system \dots
\end{example}
\begin{corollary}
Assume $H$ is $C^{2}$ and
$\left(a_{\infty}, b_{\infty}\right)$-subquadratic
at infinity. Let \dots
\end{corollary}
\end{verbatim}
\newpage
\begin{verbatim}
\begin{lemma}
Assume that $H$ is $C^{2}$ on $\bbbr^{2n}\backslash \{0\}$
and that $H''(x)$ is \dots
\end{lemma}
\begin{theorem}[(Ghoussoub-Preiss)]
Let $X$ be a Banach Space and $\Phi:X\to\bbbr$ \dots
\end{theorem}
\begin{definition}
We shall say that a $C^{1}$ function $\Phi:X\to\bbbr$
satisfies \dots
\end{definition}
\end{verbatim}
{\it Sample Output\/} (follows on the next page together with
examples of the above run-in headings)
%
\title{Hamiltonian Mechanics}

\author{Ivar Ekeland\inst{1} and Roger Temam\inst{2}}

\institute{Princeton University, Princeton NJ 08544, USA
\and
Universit\'{e} de Paris-Sud,
Laboratoire d'Analyse Num\'{e}rique, B\^{a}timent 425,\\
F-91405 Orsay Cedex, France}

\maketitle
%
\begin{abstract}
This paragraph shall summarize the contents of the paper
in short terms.
\end{abstract}
%
\section*{1\quad Fixed-Period Problems: The Sublinear Case}
%
With this chapter, the preliminaries are over, and we begin the search
for periodic solutions \dots
%
\subsection*{1.1\quad Autonomous Systems}
%
In this section we will consider the case when the Hamiltonian
$H(x)$ \dots
%
\subsubsection*{The General Case: Nontriviality.}
%
We assume that $H$ is
$\left(A_{\infty}, B_{\infty}\right)$-subqua\-dra\-tic at
infinity, for some constant \dots
%
\paragraph{Notes and Comments.}
The first results on subharmonics were \dots
%
\begin{proposition}
Assume $H'(0)=0$ and $ H(0)=0$. Set \dots
\end{proposition}
\begin{proof}[of proposition]
Condition (8) means that, for every $\delta'>\delta$, there is
some $\varepsilon>0$ such that \dots \qed
\end{proof}
%
\begin{example}[\rm (External forcing)]
Consider the system \dots
\end{example}
\begin{corollary}
Assume $H$ is $C^{2}$ and
$\left(a_{\infty}, b_{\infty}\right)$-subquadratic
at infinity. Let \dots
\end{corollary}
\begin{lemma}
Assume that $H$ is $C^{2}$ on $\bbbr^{2n}\backslash \{0\}$
and that $H''(x)$ is \dots
\end{lemma}
\begin{theorem}[(Ghoussoub-Preiss)]
Let $X$ be a Banach Space and $\Phi:X\to\bbbr$ \dots
\end{theorem}
\begin{definition}
We shall say that a $C^{1}$ function $\Phi:X\to\bbbr$ satisfies \dots
\end{definition}
%
\section{Fine Tuning of the Text}
%
The following should be used to improve the readability of the text:
\begin{flushleft}
\begin{tabular}{@{}p{.19\textwidth}p{.79\textwidth}}
\verb|\,|   & a thin space, e.g.\ between numbers or between units
              and num\-bers; a line division will not be made
              following this space\\
\verb|--|   & en dash; two strokes, without a space at either end\\
\verb*| -- |& en dash; two strokes, with  a space at either end\\
\verb|-|    & hyphen; one stroke, no space at either end\\
\verb|$-$|  & minus, in the text {\em only} \\[8mm]
{\em Input} & \verb|21\,$^{\circ}$C etc.,|\\
            &  \verb|Dr h.\,c.\,Rockefellar-Smith \dots|\\
            & \verb|20,000\,km and  Prof.\,Dr Mallory \dots|\\
            & \verb|1950--1985 \dots|\\
            & \verb|this -- written on a computer -- is now printed|\\
            & \verb|$-30$\,K \dots|\\[3mm]
{\em Output}& 21\,$^{\circ}$C etc., Dr h.\,c.\,Rockefellar-Smith \dots\\
            & 20,000\,km and  Prof.\,Dr Mallory \dots\\
            & 1950--1985 \dots\\
            & this -- written on a computer -- is now printed\\
            & $-30$\,K \dots
\end{tabular}
\end{flushleft}
%
\section {Special Typefaces}
%
Normal type (roman) need not be coded. {\it Italic}
(not {\sl slanted}) or, if necessary, {\bf boldface}
should be used for emphasis in the text.
\begin{flushleft}
\begin{tabular}{@{}p{.19\textwidth}p{.79\textwidth}}
\verb|{\it Text}|   & {\it Italicized Text}\\
\verb|{\em Text}|   & {\em Emphasized Text}\\
 & {\it If you would like to emphasize a {\em definition} within an
   italicized text (e.g.\ of a {\em theorem)} you should code the
   expression to be emphasized by} \verb|\em|.\\
\verb|{\bf Text}|   & {\bf Important Text}\\
\verb|\vec{Symbol}| & Vectors may only appear in math mode. The default
   \LaTeX{} vector symbol has been adapted to LLNCS conventions.\\
 & \verb|$\vec{A \times B\cdot C}| yields $\vec{A\times B\cdot C}$\\
 & \verb|$\vec{A}^{T} \otimes \vec{B} \otimes \vec{\hat{D}}$|\\
 & yields $\vec{A}^{T} \otimes \vec{B} \otimes \vec{\hat{D}}$
\end{tabular}
\end{flushleft}
%
\section {Footnotes}
%
Footnotes within the text should be coded:
\begin{verbatim}
\footnote{Text}
\end{verbatim}
{\it Sample Input}
\begin{flushleft}
Text with a footnote\verb|\footnote{The footnote is automatically
numbered.}| and text continues \dots
\end{flushleft}
{\it Sample Output}
\begin{flushleft}
Text with a footnote\footnote{The footnote is automatically numbered.}
and text continues \dots
\end{flushleft}
%
\section {Lists}
%
Please code lists as described below:\\[2mm]
{\it Sample  Input}
\begin{verbatim}
\begin{enumerate}
  \item First item
  \item Second item
  \begin{enumerate}
    \item First nested item
    \item Second nested item
  \end{enumerate}
  \item Third item
\end{enumerate}
\end{verbatim}
{\it Sample Output}
 \begin{enumerate}
\item First item
\item Second item
  \begin{enumerate}
    \item First nested item
    \item Second nested item
  \end{enumerate}
\item Third item
\end{enumerate}
%
\section {Figures}
%
Figure legends should be inserted after (not in)
the  paragraph in which the figure is first mentioned.
They will be numbered automatically.

{\it The figures\/} (line drawings and those containing halftone inserts
as well as halftone figures) {\it should not be pasted into your
laserprinter output}. They should be enclosed separately in camera-ready
form (original artwork, glossy prints, photographs and/or slides). The
lettering should be suitable for reproduction, and after reduction the
height of capital letters should be at least
1.8\,mm and not more than 2.5\,mm.
Check that lines and other details are uniformly black and
that the lettering on figures is clearly legible.

To leave the desired amount of space for the height of
your figures, please use the coding described below.
As can be seen in the output, we will automatically
provide 1\,cm space above and below the figure,
so that you should only leave the space equivalent to the size of the
figure itself. Please note that ``\verb|x|'' in the following
coding stands for the actual height of the figure:
\begin{verbatim}
\begin{figure}
\vspace{x cm}
\caption[ ]{...text of caption...}          (Do type [ ])
\end{figure}
\end{verbatim}
\begin{flushleft}
{\it Sample Input}
\end{flushleft}
\begin{verbatim}
\begin{figure}
\vspace{2.5cm}
\caption{This is the caption of the figure displaying a white
eagle and a white horse on a snow field}
\end{figure}
\end{verbatim}
\begin{flushleft}
{\it Sample Output}
\end{flushleft}
\begin{figure}
\vspace{2.5cm}
\caption{This is the caption of the figure displaying a white eagle and
a white horse on a snow field}
\end{figure}
%
\section{Tables}
%
Table captions should be treated
in the same way as figure legends, except that
the table captions appear {\it above} the tables. The  tables
will be numbered automatically.
%
\subsection{Tables Coded with \protect\LaTeX{}}
%
Please use the following coding:\\[2mm]
{\it Sample Input}
\begin{verbatim}
\begin{table}
\caption{This is the example table taken out of {\it The
\TeX{}book,} p.\,246}
\vspace{2pt}
\begin{tabular}{r@{\quad}rl}
\hline
\multicolumn{1}{l}{\rule{0pt}{12pt}
                Year}&\multicolumn{2}{l}{World population}\\[2pt]
\hline\rule{0pt}{12pt}
8000 B.C.  &     5,000,000& \\
  50 A.D.  &   200,000,000& \\
1650 A.D.  &   500,000,000& \\
1945 A.D.  & 2,300,000,000& \\
1980 A.D.  & 4,400,000,000& \\[2pt]
\hline
\end{tabular}
\end{table}

Before continuing your text you need an empty line. \dots
\end{verbatim}
{\it Sample Output}
\begin{table}
\caption{This is the example table taken out of {\it The
\TeX{}book,} p.\,246}
\vspace{2pt}
\begin{tabular}{r@{\quad}rl}
\hline
\multicolumn{1}{l}{\rule{0pt}{12pt}
                Year}&\multicolumn{2}{l}{World population}\\[2pt]
\hline\rule{0pt}{12pt}
8000 B.C.  &     5,000,000& \\
  50 A.D.  &   200,000,000& \\
1650 A.D.  &   500,000,000& \\
1945 A.D.  & 2,300,000,000& \\
1980 A.D.  & 4,400,000,000& \\[2pt]
\hline
\end{tabular}
\end{table}

Before continuing your text you need an empty line. \dots

\vspace{3mm}
For further information you will find a complete description of
the tabular environment
on p.~63~ff. and p.~182 of the {\em \LaTeX{} User's Guide \& Reference
Manual\/} by Leslie Lamport.
%
\subsection{Tables Not Coded with \protect\LaTeX{}}
%
If you do not wish to code your table using \LaTeX{}
but prefer to have it reproduced separately,
proceed as for figures and use the following coding:\\[2mm]
{\it Sample Input}
\begin{verbatim}
\begin{table}
\caption{text of your caption}
\vspace{x cm}     % the actual height needed for your table
\end{table}
\end{verbatim}
%
\subsection{Signs and Characters}
%
\subsubsection*{Special Signs.}
%
You may need to use special signs.  The available ones are listed in the
{\em \LaTeX{} User's Guide \& Reference Manual\/} by Leslie Lamport,
pp.~44\,ff.
We have created further symbols for math mode (enclosed in \$):
\begin{center}
\begin{tabular}{l@{\hspace{1em}yields\hspace{1em}}
c@{\hspace{3em}}l@{\hspace{1em}yields\hspace{1em}}c}
\verb|\grole| & $\grole$ & \verb|\getsto| & $\getsto$\\
\verb|\lid|   & $\lid$   & \verb|\gid|    & $\gid$
\end{tabular}
\end{center}
%
\subsubsection*{Gothic (Fraktur).}
%
If gothic letters are {\it necessary}, please use those of the
relevant \AmSTeX{} alphabet which are available from the
American Mathematical Society.

In \LaTeX{} only the following gothic letters are available:
\verb|$\Re$| yields $\Re$ and \verb|$\Im$| yields $\Im$. These should
{\it not\/} be used when you need gothic letters for your contribution.
Use \AmSTeX{} gothic as explained above. For the real and the imaginary
parts of a complex number within math mode you should use instead:
\verb|$\rm Re$| (which yields Re) or \verb|$\rm Im$| (which yields Im).
%
\subsubsection*{Script.}
%
For script capitals use the coding
\begin{center}
\begin{tabular}{l@{\hspace{1em}which yields\hspace{1em}}c}
\verb|$\cal AB$| & $\cal AB$
\end{tabular}
\end{center}
(see p.~43 of  the \LaTeX{} book).
%
\subsubsection*{Special Roman.}
%
If you need other symbols than those below, you could use
the blackboard bold characters of \AmSTeX{},  but there might arise
capacity problems
in loading additional \AmSTeX{} fonts. Therefore  we created
the blackboard bold characters listed below.
Some of them are not esthetically
satisfactory. This need not deter you from using them:
in the final printed form they will be
replaced by the well-designed MT (monotype) characters of
the phototypesetting machine.
\begin{flushleft}
\begin{tabular}{@{}ll@{ yields }
c@{\hspace{1.1em}}ll@{ yields }c}
\verb|\bbbc| & (complex numbers)   & $\bbbc$
  & \verb|\bbbf| & (blackboard bold F) & $\bbbf$\\
\verb|\bbbh| & (blackboard bold H) & $\bbbh$
  & \verb|\bbbk| & (blackboard bold K) & $\bbbk$\\
\verb|\bbbm| & (blackboard bold M) & $\bbbm$
  & \verb|\bbbn| & (natural numbers N) & $\bbbn$\\
\verb|\bbbp| & (blackboard bold P) & $\bbbp$
  & \verb|\bbbq| & (rational numbers)  & $\bbbq$\\
\verb|\bbbr| & (real numbers)      & $\bbbr$
  & \verb|\bbbs| & (blackboard bold S) & $\bbbs$\\
\verb|\bbbt| & (blackboard bold T) & $\bbbt$
  & \verb|\bbbz| & (whole numbers)     & $\bbbz$\\
\verb|\bbbone| & (symbol one)      & $\bbbone$
\end{tabular}
\end{flushleft}
\begin{displaymath}
\begin{array}{c}
\bbbc^{\bbbc^{\bbbc}} \otimes
\bbbf_{\bbbf_{\bbbf}} \otimes
\bbbh_{\bbbh_{\bbbh}} \otimes
\bbbk_{\bbbk_{\bbbk}} \otimes
\bbbm^{\bbbm^{\bbbm}} \otimes
\bbbn_{\bbbn_{\bbbn}} \otimes
\bbbp^{\bbbp^{\bbbp}}\\[2mm]
\otimes
\bbbq_{\bbbq_{\bbbq}} \otimes
\bbbr^{\bbbr^{\bbbr}} \otimes
\bbbs^{\bbbs_{\bbbs}} \otimes
\bbbt^{\bbbt^{\bbbt}} \otimes
\bbbz \otimes
\bbbone^{\bbbone_{\bbbone}}
\end{array}
\end{displaymath}
%
\subsubsection*{Sans Serif.}
%
Using our macros you can also choose this font family;
use the command \verb|\sf| for {\sf sans serif}
(like \verb|\it| for {\it italic style}).
%
\section{References}
\label{refer}
%
There are three reference systems available; only one, of course,
should be used for your contribution. With each system (by
number only, by letter-number or by author-year) a reference list
containing all citations in the
text, should be included at the end of your contribution placing the
\LaTeX{} environment \verb|thebibliography| there.
For an overall information on that environment
see the {\em \LaTeX{} User's Guide \& Reference
Manual\/} by Leslie Lamport, p.~73.
%
\subsection{References by Letter-Number or by Number Only}
%
References are cited in the text -- using the \verb|\cite|
command of \LaTeX{} -- by number or by letter-number in square
brackets, e.g.\ [1] or [E1, S2], [P1], according to your use of the
\verb|\bibitem| command in the \verb|thebibliography| environment. The
coding is as follows: if you choose your own label for the sources by
giving an optional argument to the \verb|\bibitem| command the citations
in the text are marked with the label you supplied. Otherwise a simple
numbering is done, which is preferred.
\begin{verbatim}
The results in this section are a refined version
of \cite{clar:eke}; the minimality result of Proposition~14
was the first of its kind.
\end{verbatim}
The above input produces the citation: ``\dots\ refined version of
[CE1]; the min\-i\-mality\dots''. Then the \verb|\bibitem| entry of
the \verb|thebibliography| environment should read:
\begin{verbatim}
\begin{thebibliography}{[MT1]}
.
.
\bibitem[CE1]{clar:eke}
Clarke, F., Ekeland, I.:
Nonlinear oscillations and boundary-value problems for
Hamiltonian systems.
Arch. Rat. Mech. Anal. {\bf 78} (1982) 315--333
.
.
\end{thebibliography}
\end{verbatim}
The complete bibliography looks like this:
%
\begin{thebibliography}{[MT1]}
%
\bibitem[CE1]{clar:eke}
Clarke, F., Ekeland, I.:
Nonlinear oscillations and
boundary-value problems for Hamiltonian systems.
Arch. Rat. Mech. Anal. {\bf 78} (1982) 315--333
%
\bibitem[CE2]{clar:eke:2}
Clarke, F., Ekeland, I.:
Solutions p\'{e}riodiques, du
p\'{e}riode donn\'{e}e, des \'{e}quations hamiltoniennes.
Note CRAS Paris {\bf 287} (1978) 1013--1015
%
\bibitem[MT1]{mich:tar}
Michalek, R., Tarantello, G.:
Subharmonic solutions with prescribed minimal
period for nonautonomous Hamiltonian systems.
J. Diff. Eq. {\bf 72} (1988) 28--55
%
\bibitem[Ta1]{tar}
Tarantello, G.:
Subharmonic solutions for Hamiltonian
systems via a $\bbbz_{p}$ pseudoindex theory.
Annali di Matematica Pura (to appear)
%
\bibitem[Ra1]{rab}
Rabinowitz, P.:
On subharmonic solutions of a Hamiltonian system.
Comm. Pure Appl. Math. {\bf 33} (1980) 609--633
\end{thebibliography}
%
\subsubsection*{Number-Only System.}
%
For this preferred system do not use the optional argument
in the \verb|\bibitem| command: then, only numbers will
appear for the citations in the text (enclosed in square brackets)
as well as for the marks in your
bibliography (here the number is only end-punctuated without
square brackets).
\begin{verbatim}
\begin{thebibliography}{1}
\bibitem {clar:eke}
Clarke, F., Ekeland, I.:
Nonlinear oscillations and boundary-value problems for
Hamiltonian systems.
Arch. Rat. Mech. Anal. {\bf 78} (1982) 315--333
\end{thebibliography}
\end{verbatim}
%
\subsection{Author-Year System}
%
References are cited in the text by name and year in parentheses
and should look as follows:
(Smith 1970, 1980), (Ekeland et al. 1985, Theorem 2), (Jones and Jaffe
1986; Farrow 1988, Chap.\,2). If the name is part of the sentence
only the year may appear in parentheses,
e.g.\ Ekeland et al. (1985, Sect.\,2.1)
The reference list should contain all citations occurring in the text,
ordered alphabetically by surname (with initials following). If there
are several works by the same author(s) the references should be listed
in the appropriate order indicated below:
\begin{alpherate}
\setlength{\hfuzz}{5pt}
\item
One author: list works chronologically;
\item
Author and same co-author(s): list works chronologically;
\item
Author and different co-authors: list works alphabetically
according to co-authors.
\end{alpherate}
If there are several works by the same author(s) and in the same year,
but which are cited separately, they should be distinguished by the use
of ``a'', ``b'' etc., e.g.\ (Smith 1982a), (Ekeland et al. 1982b).
%
\subsubsection*{How to Code Author-Year System.}
%
If you want to use this system you have to specify in
\verb|documentstyle| the option \verb|[citeauthoryear]|, like:
\begin{verbatim}
\documentstyle[citeauthoryear]{llncs}
\end{verbatim}
Write your citations in the text explicitly except for the year, leaving
that up to \LaTeX{} with the \verb|\cite| command. Then give only the
appropriate year as the optional argument (i.e. the label in square
brackets) with the \verb|\bibitem| command(s).\\[2mm]
{\it Sample Input}
\begin{verbatim}
The results in this section are a refined version
of Clarke and Ekeland (\cite{clar:eke}); the minimality result of
Proposition~14 was the first of its kind.
\end{verbatim}
The above input produces the citation: ``\dots\ refined version of
Clarke and Ekeland (1982); the minimality\dots''. Then the
\verb|\bibitem|
entry of the \verb|thebibliography| environment should read:
\begin{verbatim}
\begin{thebibliography}{}  % (do not forget {})
.
.
\bibitem[1982]{clar:eke}
Clarke, F., Ekeland, I.:
Nonlinear oscillations and boundary-value problems for
Hamiltonian systems.
Arch. Rat. Mech. Anal. {\bf 78} (1982) 315--333
.
.
\end{thebibliography}
\end{verbatim}
{\it Sample Output}
\bibauthoryear
%
\end{document}
