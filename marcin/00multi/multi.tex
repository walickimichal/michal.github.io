\documentstyle[12pt,marcin]{article}

% arbitrary institution

\newcommand{\Mod}{\mbox{\rm\bf Mod}}
\newcommand{\Sign}{\mbox{\rm\bf Sign}}
\newcommand{\Sen}{\mbox{\rm\bf Sen}}
\newcommand{\Set}{\mbox{\rm\bf Set}}
\newcommand{\Cat}{\mbox{\rm\bf Cat}}

% particular categories

\newcommand{\MAlg}{\mbox{\rm\bf MAlg}}
\newcommand{\Alg}{\mbox{\rm\bf Alg}}
\newcommand{\AlgSig}{\mbox{\rm\bf AlgSig}}

\newcommand{\MA}[2]{\MAlg(#1,#2)}

\newcommand{\Nat}{{\sf Nat}}

% homomorphisms & multCats

\newcommand{\SSs}{{\rm SS}}
\newcommand{\PS}{{\rm PS}}
\newcommand{\PP}{{\rm PP}}
\newcommand{\XX}{{\rm XX}}

\newcommand{\mode}[2]{#1(#2)}

% specifications

\newcommand{\SMod}{\mbox{\sf Mod}}
\newcommand{\SSig}{\mbox{\sf Sig}}
\newcommand{\SP}{\mbox{\rm SP}}

% symbols

\newcommand{\reach}[3]{{\sf Reach}^{#1}_{#3}(#2)}

\newcommand{\tinc}{\stackrel{\rm T}{\rightharpoonup}}
\newcommand{\sinc}{\stackrel{\rm S}{\rightharpoonup}}
\newcommand{\teq}{\stackrel{\rm T}{\rightleftharpoons}}
\newcommand{\seq}{\stackrel{\rm S}{\rightleftharpoons}}

\newcommand{\<}{\langle}
\renewcommand{\>}{\rangle}

\newcommand{\ovr}[1]{\overline {#1}}
\newcommand{\under}[1]{\underline {#1}}

\newcommand{\quot}[2]{#1\!{\textstyle /}\!_#2}
\newcommand{\prequot}[2]{#1\!{\textstyle /}\!\!{\textstyle /}\!_#2}

\newcommand{\Comp}[2]{#1;#2}
\newcommand{\impl}{\Rightarrow}
\newcommand{\into}{\rightarrow}
\newcommand{\Card}[1]{|#1|}
\newcommand{\PSet}{{\cal P}} 

\newcommand{\Terms}{{\cal T}}
\newcommand{\GTerms}{\Terms_{\Sigma}}
\newcommand{\XTerms}{\Terms_{\Sigma}(X)}

% Partial orders

\newcommand{\po}{\leq}
\newcommand{\rpo}{\geq}
\newcommand{\spo}{<}
\newcommand{\posup}{\bigvee}
\newcommand{\fposup}{\vee} % finite \posup
% ponizej powinno byc \bigsqcap
\newcommand{\poinf}{\bigwedge}
\newcommand{\fpoinf}{\wedge}

\newcommand{\setsup}{\bigcup}
\newcommand{\fsetsup}{\cup}
\newcommand{\setinf}{\bigcap}
\newcommand{\fsetinf}{\cap}

% other

\newcommand{\Det}{\mbox{\sf Det}}
\newcommand{\PDet}{\mbox{\sf PDet}}
\newcommand{\CDet}{\mbox{\sf CDet}}
\newcommand{\Mquo}{\mbox{$/\!\!/$}}
\newcommand{\implem}{\mbox{$\sim\!\leadsto$}}

\newcommand{\re}[1]{(\ref{#1})}

% environments

\newtheorem{Definition}{Definition}[section]
\newtheorem{Fact}{Fact}[section]
\newtheorem{Lemma}{Lemma}[section]
\newtheorem{Example}{Example}[section]
\newtheorem{Theorem}{Theorem}[section]
\newtheorem{Corollary}{Corollary}[section]
\newenvironment{Proof}
               {\setlength{\parskip}{5pt}
                \setlength{\parindent}{0pt}
                {\sc Proof:}}{\hfill{\sc qed}\vspace{1.5ex}}
%                {\sc Proof:}}{\hfill{\sc qed}$\:$\raisebox{-1pt}{$\Box$}}

\newenvironment{REMARK}[1]{\vspace{1ex}\par\noindent{\bf #1}}
            {\vspace{1ex}\par\noindent\ignorespaces}
% and numbered version
\newcounter{CLAIM}[section]
\newenvironment{REMARKno}[1]{\refstepcounter{CLAIM}
       \vspace{1ex}\par\noindent{\bf #1\ \theCLAIM .}}
       {\vspace{1ex}\par\noindent\ignorespaces}

% small remark
\newenvironment{SREMARK}[1]{\vspace{1ex}\par\noindent\small{\bf #1}}
            {\vspace{1ex}\par\noindent\normalsize\ignorespaces}
\newenvironment{SREMARKno}[1]{\refstepcounter{CLAIM}
       \vspace{1ex}\par\noindent\small {\bf #1\ \theCLAIM .}}
       {\vspace{1ex}\par\noindent\normalsize\ignorespaces}


% co-operation (sic!) 

\newcommand{\marcin}[1]{\vspace{1ex}{\sf [][][ MB: #1 ][][]}\vspace{1ex}}
\newcommand{\michal}[1]{\vspace{1ex}{\sf [][][ MW: #1 ][][]}\vspace{1ex}}


%==============================================================================
%
%   The Document

\input xypic

\begin{document} 

\

\vspace{2cm}

\begin{center}
{\Large\bf Universal Multialgebra}\\[0.8cm]
{\Large Marcin Bia\l{}asik \hspace{3cm} Micha\l{} Walicki}\\[0.4cm] 
Draft: \today 
\end{center}
\medskip

%==============================================================================
%
%
\section{Introduction}

Nondeterministic algebras seem to be paid increasingly more attention
by theoretical computer science community. This is especially true
among those interested in formal program specifications based on the
algebraic paradigm.  These formalisms are widely recognized as a
framework providing for a very natural program development
methodology. Many concepts important in the area, such as modules and
abstract data types, seem to be represented in a very intuitive yet
formal fashion.

In this report, however, we do not necessarily aim at proposing yet
another methodology. We focus on particular models of nondeterminism,
namely multialgebras, being essentially ``ordinary'' multi-sorted
algebras with set-valued functions reflecting possible
nondeterministic choices, in an attempt to gather as many mathematical
facts concerning these as possible, and to find correspondences with
other approaches and notions. In fact, multialgebras seem to lack a
common theory similar to that offered by the universal algebra and
hence, if they are to be used, there is an urgent need for the
development of such strong mathematical foundations.

We have embarked at this rather ambitious (as it has turned out) task
and by no means we claim to have completed it. This version of the
report presents mainly old concepts known from ordinary algebra and
redefined for multialgebras. Many problems are posed by the number of
possible defitions and the necessity to analise each one's value and
applicability.

%==============================================================================
%
%
\section{Preliminary Definitions}

An {\em algebraic signature} $\Sigma$ is a pair $[S,F]$ where $S$ is a
set of sort names and $F$ is a family of sets $\{F_{w,s}\}_{w \in
S^\ast, s \in S}$ of sort names. We write $[f : w \rightarrow s] \in
F$ to denote $w \in S^\ast, s \in S$ and $f \in F_{w,s}$. An {\em
algebraic signature morphism} $\sigma : [S.F] \rightarrow [S',F']$ is
a pair $[\sigma_S,\sigma_F]$ where $\sigma_S : S \rightarrow S'$ and
$\sigma_F$ is a family of maps $\{\sigma_{w,s} : F_{w,s} \rightarrow
F'_{\sigma^\ast_S(w),\sigma_S(s)}\}_{w \in S^\ast, s \in S}$ where
$\sigma^\ast_S(s_1,\ldots,s_n)$ denotes
$\sigma_S(s_1),\ldots,\sigma_S(s_n)$. We will write $\sigma(s)$ for
$\sigma_S(s)$, $\sigma(w)$ for $\sigma^\ast_S(w)$ and $\sigma(f)$ for
$\sigma_{w,s}(f)$ where $f \in F_{w,s}$.  The category of all
algebraic signatures will be denoted \AlgSig. It has algebraic
signatures as objects and algebraic signature morphisms as morphisms.

Let $\Sigma = [S,F]$ be an algebraic signature. A (deterministic)
$\Sigma$-algebra $A$ consists of an $S$-indexed family of carrier sets
$\{A_s\}_{s \in S}$ and for each $[f : s_1 \times \ldots \times s_n
\rightarrow s] \in F$ a function $f^A : A_{s_1} \times \ldots \times
A_{s_n} \rightarrow A_s$.  A $\Sigma$-homomorphism $\phi$ from a
$\Sigma$-algebra $A$ to a $\Sigma$-algebra $B$ is a family of mappings
$\{\phi_s\}_{s \in S}, \phi_s: A_s \rightarrow B_s$ such that for all
$[f:s_1 \times s_2 \times \ldots \times s_n \rightarrow s] \in F$ and
for all $a_i \in A_{s_i}, i \in \{1,\ldots,n\}$ the following
condition holds:
\[ 
\phi_s(f^A(a_1,\ldots,a_n)) =
f^B(\phi_{s_1}(a_1),\ldots,\phi_{s_n}(a_n))
\]
We denote the category of $\Sigma$-algebras by $\Alg(\Sigma)$, and for
the rest of the paper we will assume an arbitrary but fixed signature
$\Sigma$.  

\subsection{Multialgebras}

The above classical definition can be generalized to set structures
depending on the way we introduce sets. Given sets $X$ and $Y$, there
are three basic modes for a function $f$ going from $\PSet(X)$ to
$\PSet(Y)$:
\begin{itemize} 
\item[\PP.] $f$ is {\em induced\/} by some function $f':X \into Y$
which maps elements of $X$ to elements of $Y$ (Point-Point);
\item[\PS.] $f$ is induced by some $f':X \into \PSet(Y)$ which maps
elements of $X$ to sets of elements of $Y$ (Point-Set);
\item[\SSs.] there is no function inducing $f$ (Set-Set).
\end{itemize}
where we say that $f$ is induced by $f'$ iff for any $A \in \PSet(X)$
holds $f(A) = \setsup_{a \in A} f'(a)$ (note that this is not the only
possible definition). 

Given some set $X$ and its power set $\PSet(X)$ we will not
distinguish between elements of $X$ and one-element sets in
$\PSet(X)$, but we use capital letters to denote sets. Similarly, we
will not distinguish between the operations returning individual
elements and the ones returning one-element sets. In addition, we will
consider only total case, i.e.\ the operations do not return the empty
set and we exclude the set from the powerset.

Writing $\XX(f)$ to indicate that $f$ has the $\XX$ mode, we have that
$\PP(f)$ implies $\PS(f)$, and $\PS(f)$ implies $\SSs(f)$. These
implications may be reversed under obvious conditions. That is, for
all $f$, we have:
\begin{eqnarray}%{lcl}
\SSs(f) \impl \PS(f) & {\rm iff} 
        & \mbox{for any } X: f(X)=\setsup_{x\in X}f(x) \\
\PS(f) \impl \PP(f)  & {\rm iff} 
        & \mbox{for any } X: |X|=1\impl |f(X)|=1
\end{eqnarray}
The first property is called {\em additivity\/} of an \SSs\ operation,
and it implies monotonicity wrt.\ set inclusion.

Each of the three notions can be applied to operations within an
algebra and to homomorphisms between algebras. Thus, we obtain 9
different categories of $\Sigma$ multialgebras -- $\MA{ops}{hom}$
denotes the category of multilagebras with operations of mode $ops$
and homomorphisms of mode $hom$. (We may occasionally put $\Sigma$ in
some of these positions to indicate that a statement applies to all
classes, e.g., $\MA\Sigma\SS$ will denote any of the three categories
of \SS, \PS\ or \SSs\ algebras with \SSs\ homomorphisms.) Table
\ref{ta:cats} contains the list of all possible combinations.

\begin{table}
\[
\begin{array}{c||c@{\ \ \subset\ \ }c@{\ \ \subset\ \ }c}
\ \ \ hom & \PP & \PS & \SSs \\
\multicolumn{1}{l||}{ops} & \multicolumn{1}{c}{}& \multicolumn{1}{c}{} & \\ \hline\hline
\PP & \MA\PP\PP & \MA\PP\PS & \MA\PP\SSs \\
\bigcap & \multicolumn{1}{c}{\bigcap} & \multicolumn{1}{c}{\bigcap} & \multicolumn{1}{c}{\bigcap} \\
\PS & \MA\PS\PP & \MA\PS\PS & \MA\PS\SSs \\
\bigcap & \multicolumn{1}{c}{\bigcap} & \multicolumn{1}{c}{\bigcap} & \multicolumn{1}{c}{\bigcap} \\
\SSs & \MA\SSs\PP & \MA\SSs\PS & \MA\SSs\SSs 
\end{array}
\]
\caption{Combinations of operations and homomorphisms.}\label{ta:cats}
\end{table}

$\MA\PP\PP$ is the standard category $\Alg$. The objects of the
categories from the first row are standard algebras and it seems
little relevant to consider the two more general homomorphisms for
this case. In the two other rows, targets of operations in an algebra
are power sets, i.e.\ terms are interpreted as sets. This poses the
next problem of choosing an appropriate extension of the homomorphism
condition. The three basic possibilities for a homomorphism
$\phi:M\into N$ are:
\begin{itemize}
\item[L)]  $\phi(f^M(\ovr a)) \subseteq f^N(\phi(\ovr a))$ - loose;
\item[C)]  $\phi(f^M(\ovr a)) \supseteq f^N(\phi(\ovr a))$ - closed;
\item[T)]  $\phi(f^M(\ovr a)) = f^N(\phi(\ovr a))$ - tight;
\end{itemize}
For the first two columns of table \ref{ta:cats}, $\ovr a$ is a list
of elements in $\Card M$, while for the third it is a list of sets,
i.e.\ elements of $\PSet{\Card M}$.

Obviously, $T\iff (L\land C)$. These notions can be combined with (the
last two rows of) the table \ref{ta:cats}, yielding two copies -- one
for the loose, and one for the closed homomorphisms.  However, closed
homomorphisms are not very relevant.  In Hussmann's work only loose
and tight \PS-homomorphisms appear.  Closed {\SSs} and {\PS}
homo\-mor\-phisms are not interesting because they always exist,
simply by letting $\phi(a) = \Card N$. However, we will consider not
the full power sets but their subsets, in which case closed
homomorphisms are no longer trivial.
We will indicate the mode of the homomorphisms by superscript, e.g., $\MA\PS{\PS^T},
\MA\PS{\SSs^C}$.

%Since $\MA\SSs\SSs$ is the most general category, we will attemt to
%prove various theorems for this case, and hope that they will
%specialize to the more specific subcategories. This is only our
%general strategy. Such a specialization will not always be possible
%while, on the other hand, some results may be obtained only for the
%specific subcategories of $\MA\SSs\SSs$.
%

\subsubsection{The Underlying Set and the Carrier}
%
Let $\po$ be any transitive relation on a set $M$. We write $\under M$
for the set of minimal elements of $M$, i.e., $\under M=\{x\in
M:\forall y\in M: y= x\lor \neg(y\po x)\}$.  For an $r\in M$, $\under
r$ denotes the set of minima which are below $r$, i.e., $\under r
=\{m\in\under M : m\po r\}$. In general, we will consider $\po$ which
are partial orders but occasionally, we may need the above notion for
other relations.
\begin{Definition}
%Let $\under M$ be a set. 
A partial order $\<M,\po\>$ is said to be
{\em generated} by (based on) the set $\under M$ iff
%\begin{enumerate}
%\item $Min(M)=\under M$ ($\under M$ is the set of minima in $M$);
%\item $\forall p\in M\ \exists! m\subseteq\under M : p = \posup m$
%(each element of $M$ is supremum of a unique maximal subset of minimal
%elements, denoted $Min(p)$ or $\under p$);
%\item 
$\forall p,r\in M: p\po r \iff \under p\subseteq \under r$.
%\end{enumerate}
\end{Definition}

In particular, each element $p$ of a generated $M$ is a unique least
upper bound of its members, $p=\posup \under p$. Notice that for a
subset $m\subseteq M$, the lub $\posup m$ need not exist in $M$ (we do
not require existence of arbitrary joins -- sums, or meets --
intersections). Using the notation $\posup m$ in a generated $M$ we
will always mean ``least upper bound of $m$ if it exists in M''.

\michal{OLD formulations: The unique $m$ in the second point refers
to the maximal set of minima such that $p=\posup m$. I.e., if there is
another set $n\subseteq \under M$, such that $p=\posup n$ then
$n\subseteq m$.  Notice that there may be several PO's generated by
the same $\under M$. The intention of this definition is to capture
the intuition that the carrier of a multialgebra (which will be a PO
generated by some underlying set $\under M$) is a {\em subset} of the
power set of $\under M$. The partial order $\po$ represents set
inclusion $\subseteq$, and each element (set) of the carrier is just a
collection of some minimal elements (individuals). Uniqueness of such
a set, together with the last condition, mean extensionality.}

Since the generating minima are all in the PO,
the carriers will be always closed under underlying set. Although one
may imagine that an
\SSs\ algebra need not such a closure, we do impose it in this case as
well because otherwise an
\SSs\ algebra would be just a common (partially) ordered algebra (or
even a standard algebra, if we did not insist on monotonicity of the
operations within it). We want to deal with subsets of power sets and
not arbitrary ordered algebras.

If $y,Y$ are elements of a PO $M$ generated by $\under M$, we will use
the notation $y\in Y$ as an abbreviation for $y\po Y\land y\in\under
M$ (or $y\in\under Y$). In general, whenever using the set operations,
we implicitly assume that they are applied to the underlying sets (of
minima). E.g., $X\setminus Y$ means $\under X\setminus\under Y$,
$X\cap Y$ means $\under X\cap\under Y$, etc. Also, we will ot
distinguish between an element $x\in\under M$ of the underlying set,
and the one-element set $\{x\}$ containing such an element.

The least upper, respectively greatest lower, bounds of a collection of
elements $m_i\in M$, if they exist, will be denoted by $\posup m_i$
and $\poinf m_i$, respectively.

The (set) least upper, respectively greatest lower, bounds will be
defined not only relatively to the PO, but also to the set of
minima. For $m_i\in M$, the (set) least upper bound $\setsup m_i$, if
it exists, is an $X\in M$ such that $\under X = \setsup\under
m_i$. Similarly, $\setinf m_i = X : \under X = \setinf \under m_i$. If
 set lub/glb exist, we will have $\posup m_i =
\setsup m_i$ and $\poinf m_i = \setinf m_i$.

\begin{REMARK}{Remark.}
Considering arbitrary subsets of a power set may look innocent but has
some slightly unexpected consequences.  The definition of a generated
PO requires only that each element corresponds to a collection of
unique individuals. In this respect, we are not in the standard set
theory but in its variant where the basic entities are not sets but (a
set of) underlying individuals -- ``urelements''. Set theory with
urelements goes back to \cite{K} and \cite{P}; \cite{KP} contains a
good review. 

Furthermore, unlike the the theory with ``urelements'' of \cite{KP},
the definiton does not require the existence of arbitrary joins or
meets, and in the case they exist, does not require their
uniqueness. This brings us into a rather non-standard setting.  For
instance, let $\under M=\{1,2,3,4\}$. Then $M=\under M\cup
\{\{1,2,3\},\{2,3,4\}\}$ is generated and both sets are upper bounds
of 2 and 3, without the two having a unique least upper bound which
would require inclusion of $\{2,3\}$ into $M$. Also, there is no upper
bound (union) of the two sets in $M$.

Even worse, let $\under M$
be the set of natural numbers with two additional elements $\Nat\cup
\{a,b\}$, and let $X_n=\{x\in\Nat :x\geq n\}\cup\{a,b\}$. 
Choose $M=\under M\cup \{X_n:n\in\Nat\}$. Then the partial order in
$M$ is not well-founded, since we have $X_n\rpo X_{n+1}$, and $a,b$
have no least upper bound, since all $X_n$ are upper bounds for $a,b$.

These deviations should not worry us to much because, for us,
the crucial possibility is this of relating sets to their
elements and of defining the behaviour of operations on sets in terms
of their behaviour on the elements. We will not deal with induction
along the ordering relation $\po$ -- only with induction along $\in$,
i.e., from elements to sets. Similarly, non-uniqueness of joins and
meets is not a problem in this setting as long as we are able to
uniquely identify the elements belonging to a given set.
\end{REMARK}
There are various equivalent ways of representing generated PO's and,
occasionally, we will use alternative representations. For instance,
$M=\{1,2,3,4,\{1,2\},\{1,2,3\},\{1,2,4\},\{1,2,3,4\}\}$ corresponds to
the diagram to the left in figure \ref{fi:setin} -- we call this {\em flat-set representation}.
\begin{figure}[hbt]
\[
\spreaddiagramrows{-0.5pc}
\spreaddiagramcolumns{-1pc}
\def\objectstyle{\scriptstyle}
\diagram %compileto{setin}
%& & \save\Drop{\{1,2,3,4\}}\restore &   & & & & & \save\Drop{\{\{\{1,2\},3\}, \{\{1,2\},4\} \}}\restore & \\
& & \{1,2,3,4\} &   & & & & & \{\{\{1,2\},3\}, \{\{1,2\},4\} \} & \\
 & \{1,2,3\} \xline[ur] & & \{2,3,4\} \xline[ul] & & &  & \{\{1,2\},3\} \xline[ur]& & \{\{1,2\},4\} \xline[ul] \\
& \{1,2\} \xline[u] \xline[urr] & & & M\ \rrto<0.2pc>^{\_^\in} & & \llto<0.2pc>^{sp} \ M^\in  & \{1,2\} \xline[u] \xline[urr] \\
1 \xline[ur] & 2 \xline[u] & 3 \xline[uul] & 4 \xline[uu]  & & &
                                1 \xline[ur] & 2 \xline[u] & 3 \xline[uul] & 4 \xline[uu]
\enddiagram
\]
\caption{}\label{fi:setin}
\end{figure}

But we may interpret the relations indicated in the diagram as $\in$
instead of $\subseteq$. In this case, $M$ corresponds to
$M^\in=\{1,2,3,4, \{1,2\}, \{\{1,2\},3\},
\{\{1,2\},4\},\{\{\{1,2\},3\},\{\{1,2\},4\} \}\}$.  We call this {\em power-set representation}. 

We have the obvious function $\_^\in:M\into M^\in$ which sends each
$X\in M$ onto $X^\in=Max\{Y\in M:Y\subseteq X\lor Y\in X\cap\under
M\}$ in $M^\in$. (Actually, we are confusing here minimal $Y$ with
$\{Y\}$ -- for the minimal elements we pretend that $Y=\{Y\}$ and
hence $Y\in X\iff Y\subseteq X$ and $\bigcup Y=Y$. These are to be
understood merely as abbreviations simplifying the formulations: we do
not want to split all the proofs into two cases -- one for the
individuals and one for sets -- but we do not want to dwell on the
situtations with $Y\in Y$ either.) We are taking only maximal elements
below $X:$ thus, if $M=\{1,2,3,4,\{1,2\},\{1,2,3\},\{1,2,3,4\}\}$ then
$\{1,2,3\}^in=\{\{1,2\},3\}$ and $\{1,2,3,4\}^in = \{\{\{1,2\},3\},4\}$.

The support function $sp:M^\in\into M$ defined as
\begin{equation}
sp(X)=
\left\{\begin{array}{l} X\ {\rm if}\ X =\{X\}\ (i.e.,\ X\in\under{M^\in}) \\
  \bigcup_{Y\in X} sp(Y)\ {\rm otherwise} \end{array}\right.
\end{equation}
allows us to return to the flat set-representation yielding set of
all individuals of the argument $X^\in$.

Two significant properties of the power-set representation are:

\begin{Fact}
$M^\in$ is 
\begin{enumerate}
\item transitive, i.e., $\forall Y,X: Y\in X\land X\in M^\in\impl Y\in
M^\in$ (or $X\in M^\in\impl X\subseteq M^\in$) ;
\item  strongly extensional, i.e., $\forall X,Y\in M^\in: X=Y\iff sp(X)=sp(Y)$.
\end{enumerate}
\end{Fact}

The second property excludes the possibility of having, for instance,
both $\{\{1,2\},3\}$ and $\{1,\{2,3\}\}$ in $m^\in$. Notice that usual
extensionality would still make these two sets different. It will also
force $\{\{1,2\}\}=\{1,2\}$ and, since by the first property
$\{\{1,2\}\}\in M^\in \impl \{1,2\}\in M^\in$, will simply ``reduce
the number of parantheses $\{...\}$ to the necessary minimum''. (The
identification $Y=\{Y\}$ for the minimal elements migt be viewed as a
special case of this condition.)
%
%(In fact, using this representation we
%might use only the maximal elements to represent the whole $M$.)

The first property implies closure wrt. the underlying
set. It also ensures a kind of ``well-definedness'' of the
structure. For instance, it disallows the set
$M^\in=\{1,2,3,\{1,2\},\{1,\{2,3\}\}$, which would say that $\{2,3\}
both is in the structure -- since it is in $\{1,\{2,3\}\}$, and is not
in it -- since it is not in $M$).

%
%
\subsubsection{Algebras and Homomorphisms of Various Modes}
%
We now spell out the definitions of the introduced classes of
multi-algebras.
%We let the carriers of multialgebras correspond to the
%mode of the source of the operations, and variables range over
%elements of the carriers (even if operations may have target which has
%different mode).

\begin{Definition}
Let $\Sigma = [S,F]$ be a signature. A $\Sigma$-multialgebra $M$ has a
 carrier $\Card M=\setsup_{s\in S}M_s$, where for each $s\in S$, $M_s$
 is a PO generated by some $\under M_s$. Interpretation of each
 operation $f:s1\times...\times sn\into s\in F$ is a function
 $f^M:M_{s1}\times...\times M_{sn}\into M_s$ satisfying some
 conditions which determine the mode of the multialgebra as follows:
\begin{itemize}
\item for $M$ of mode \SSs : \\
 -- $f^M$ is monotone, i.e.\ $\forall X,Y\in \Card M: X\po Y\impl f^M(X)\po f^M(Y)$;
\item for $M$ of mode \PS :\\ %\ is an $M$ of mode \SSs\ where :\\
%a set $M_s$ for each $s\in S$ (in this case we let
% $\under M_s=M_s$); \\
 -- $f^M$ is additive, i.e.\ $f^M(X_1,...,X_n) = \setsup_{x_i\in X_i} f^M(x_1,...,x_n)$.
\item $M$ of mode \PP :\\ %\ is an $M$ of mode \PS\ where: \\
 -- $f^M$ is deterministic, i.e.,\ for all $x_i\in\under M : f^M(x_1,...,x_n)\in \under M$.
\end{itemize}
\end{Definition}
Obviously, $\mode\PP{f}\impl \mode\PS{f}\impl \mode\SSs{f}$, so a \PP\
algebra is a \PS\ algebra is an \SSs\ algebra.
\begin{Definition}
A homomorphism  $\phi:M\into N$ for pair of algebras $M,N$ of the same mode is a
sort compatible mapping $\Card M\into \Card N$ such that $\phi(f^M(\ovr a))
\otimes f^N(\phi(\ovr a))$ (for $\otimes\in\{\po,\rpo,=\}$), satisfying additional conditions for the
different modes: \\
\begin{tabular}{l@{\ \ -\ \ }l}
\SSs & is monotone $\forall X,Y\in\Card M: X\po Y \impl \phi(X)\po\phi(Y)$\\
\PS & is additive and induced by $\under M\into \Card N$ \\
\PP & is additive and induced by $\under M\into \under N$ 
\end{tabular}
\end{Definition}
Notice that the \SSs-algebras are usual partially ordered algebras
\cite{Moll} which ``just happen'' to have a (subset of) power set as
the carrier. 
%The different elements of such carriers are not related
%by any additional relation. 
Also when defining the homomorphisms
we utilize the relation $\po$ which is not, strictly speaking,
a part of the algebra. We may later introduce explicitly the predicate corresponding to $\po$ into the signature (as it was done, for instance, for unified algebras in \cite{UA}), but for the moment we follow the more traditional treatement of ordered algebras \cite{Moll},\cite{Reg}.

%Observe also that, although the definitions of homomorphisms are
%uniform (for various modes of algebras), they do not necessarily
%coincide. For instance, let $\Sigma=[\{s\},\{c:\into s\}]$ and let
%$M,N\in\MAlg(\SSs)$ be such that $\Card M=\{\{1,2\}\}, \Card
%N=\{\{a,b\}\}$ with the only possible interpretation of $c$.  There is
%only one homomorphism $\phi:M\into N$. We can look at $\phi$ as a
%\PP-homomorphism, since it is induced by the mapping
%$\phi_1=\{1\mapsto a, 2\mapsto b\}$, as well as by $\phi_1=\{1\mapsto
%b, 2\mapsto a\}$.  If, on the other hand, $M$ and $N$ are considered
%as \PS\ algebras (with the same interpretation of $c$), their carriers
%must contain all the elements, i.e.\ $\Card M=\{\{1,2\},\{1\},\{2\}\}$
%and $\Card N = \{\{a,b\},\{a\},\{b\}\}$, (we still have that $\under
%M=\{1,2\},\ \under N=\{a,b\}$).  Thus now $\phi_1$ and $\phi_2$ are
%two different \PP-morphisms.
%
\begin{Definition}
For a multi-algebra $M$ and a set of variables $X$, an {\em assignment} is
a function $\alpha:X\into\Card M$. For an algebra of mode \PS, we
additionally require this mapping to go $X\into \under M$.
\end{Definition}
(I.e., variables
are assigned individual elements in the mode \PS, while sets in the mode \SSs.)

%====================================
%
%

\section{General Properties of various $\MAlg(\Sigma)$}
\subsection{Isomorphisms etc.}

\begin{Lemma}\label{le:isotight}
Let $M, N$ be two multialgebras of the same mode. If $\phi:M\into N$
and $\psi:N\into M$ are isomorphisms (both of the same, but arbitrary
mode), then they are tight.
\end{Lemma}

\begin{Proof}
Let us consider the case when both $\phi$ and $\psi$ are loose. Then
for all $X\in \Card M, Y\in \Card N: 1)\ \phi(f^M(X))\po
f^N(\phi(X))$ and 2) $\psi(f^N(Y))\po f^M(\psi(Y))$.
Substituting $\phi(X)$ for $Y$ in 2), we get 
\[
\begin{array}{rlr}
\multicolumn{2}{l}{\psi(f^N(\phi(X))) \po f^M(\psi(\phi(X)))} & \phi\ monotoe\\
\impl & \phi(\psi(f^N(\phi(X)))) \po \phi(f^M(\psi(\phi(X)))) & isomorphism \\
\impl & f^N(\phi(X)) \po \phi(f^M(X))
\end{array} \]
Together with 1) this means that $\phi(f^M(X)) = f^N(\phi(X))$, and
the same for $\psi$ follows by a symmetric argument.

The case when $\phi$ and $\psi$ are closed is entirely analogous.
\end{Proof}

Notice that this lemma depends only on the assumption of
loosenes/closedness but not on the mode of homomorphisms.
This mode is inessential as the next lemma shows:

\begin{Lemma}
Let $M,N,\phi,\psi$ be isomorphic as in the previous lemma
(\ref{le:isotight}). Then
$\mode\SSs{\phi/\psi}\impl\mode\PS{\phi/\psi}\impl\mode\PP{\phi/\psi}$.
\end{Lemma}
\begin{Proof}
First, we show that $\phi$ (and $\psi$) must map underlying sets onto
underlying sets.  Suppose not, i.e.\ $\exists x\in\under M :
\phi(x)=Y\not\in\under N$. Then $Y=\posup_i y_i$ for some set of
indices $i$ with all $y_i\in\under N$. Isomorphism condition requires
then $\psi(Y)=x$, and monotonicity of $\psi: \psi(y_i)=x$. But then
$\phi(\psi(y_i))=Y\not=y_i$, so $\phi, \psi$ would not be an isomorphism.

Now, consider $\mode\PS{\phi,\psi}$, and assume that
$\neg\mode\PP\phi$, i.e.\ $\exists X\in\Card M : X=\posup_i x_i \land
\phi(X)=Y\not=\posup_i \phi(x_i)$. Then $\exists y_j\in Y:y_j\not\in\posup_i \phi(x_i)$.
By isomorphism condition $\psi(Y)=X$, and by monotonicity of
homomorphisms $\psi(y_j)\in X$.(Monotonicity forces $\psi(y_j)\po X$
and the first part of the proof makes this $\po$ into $\psi(y_j)\in
X$.) But this means that there is an $x_i\in X$ such that
$x_i=\psi(y_j)$ and we get $\phi(\psi(y_j) = \phi(x_i) \in\posup_i
\phi(x_i)$. Since $y_j$ was assumed to be in
$\under Y\setminus\under{\posup_i\phi(x_i)}$, we would then have
$\phi(\psi(y_j))\not= y_j$, contradicting the assumption that $\phi,
\psi$ are isomorphisms.

Entirely
analogous argument as above shows a contradiction under the assumption that
 $\mode\SSs{\phi,\psi}$ and  $\neg\mode\PS\phi$. 
\end{Proof}

Thus isomorphisms in all different categories are actually \PP-isomorphisms.

\subsection{Reachability and Subalgebras}
One of the reasons for restricting the carriers of multi-algebras to
subsets of power sets is the wish to define reachability in a natural
way. As usual, we can replace $=$ with inclusion going either
way.

\begin{Definition}\label{de:genpo}
Let $M$ be a $\Sigma$-multialgebra and $p\in\Card M$ and $m\subseteq\Card M$. 
The element $p$ is reachable from $m$:

\begin{tabular}{lcl}
tightly  & iff & $\exists x\in m : p=x$ \\
loosely  & iff & $\exists x\in m : p\po x$ \\
closedly  & iff & $\exists x\in m : p\rpo x$ 
\end{tabular}
\end{Definition}
We will write $\reach XmM$, for the part of the set $M$ (or algebra
$M$) which is X-reachable from $m$. Occasionally, we may drop the $M$
when it is clear from the context. (Loose reachability was used in
\cite{Tapsoft} as a counterpart of inductive reasoning.) Picking only
the T- or C-reachable elements does not necessarily yield a legal
carrier.  For instance:
\[
\diagram
\spreaddiagramcolumns{-1pc}
\spreaddiagramrows{-1pc}
& abc & 
 & & & abc \\
ab \xline[ur] & ac \xline[u] & bc \xline[ul] 
 & & ab \xline[ur] & ac \xline[u] & bc \xline[ul] 
 & & &   ac  & bc  \\
a \xline[u] \xline[ur] & b \xline[ul] \xline[ur] & c \xline[ul] \xline[u]
 & &  & b \xline[ul] \xline[ur] & c \xline[ul] \xline[u]
 & & a  \xline[ur] & b  \xline[ur] & c \xline[ul] \xline[u] \\
 & M & & & & \save\Drop{\reach C{b,c}M}\restore & & & & \save\Drop{\reach L{ac,bc}M}\restore & 
\enddiagram
\]
$\reach C{b,c}M$ is not a PO generated by $b, c$, since for instance,
$Min(abc)=\{b,c\}=Min(bc)$ while the two are distinct. Obviously $\reach
TmM = m$ which can be an arbitrary subset of $M$, and hence need not
satisfy the extensionality property.

$\reach L{ac,bc}M$, on the other hand, does yield a legal carrier.
 In fact, taking $\reach LmM$ for any $m$ yields a
subset of $M$ which satisfies the conditions of definition \ref{de:genpo}.

\begin{Fact}
If $M$ is a PO generated by $\under M$ and $m\subseteq M$, then
$R=\reach LmM$ is a PO generated by $\under R=\under M\cap R$.
\end{Fact}
\begin{Proof}
Let $X$ be an arbitrary element in $R\subseteq M$. Since $X$ is
L-reachable, all elements $x\in Min(X)$ will be also
in $R$, and in $\under R$. Thus, for $X\in R : Min^R(X)=Min^M(X)$.

This last equation proves also uniquness, since if $X\not= Y$ then this holds in $M$
and so $Min^R(X)=Min^M(X)\not= Min^M(Y)=Min^R(Y)$.
\end{Proof}

L-reachability is ``propagated downwards'', C-reachability ``upwards''
and T-reachability is not ``propagated'' at all. Thus, if we want the
whole carrier $M$ to be reachable, the sufficient and necessary conditions
on the set $m$ are as follows:
\begin{eqnarray}
M = \reach LmM & \iff & Max(M)\subseteq m \\
M = \reach CmM & \iff & Min(M) \subseteq m \\
M = \reach TmM & \iff & M = m
\end{eqnarray} 
In particular, reachability of the whole underlying set $\under M$
implies only C-reachability (but not L or T) of the whole carrier $M$.
We also have the obvious observation that
\begin{eqnarray}
Max(\reach LmM) & \subseteq & \reach TmM \\
Min(\reach CmM) & \subseteq & \reach TmM \\
\reach TmM & \subseteq & \reach LmM \cap \reach CmM
\end{eqnarray}
We will write $\reach X{\GTerms}M$ for the part of $M$ which is
X-reachable from the interpretation of the ground terms, i.e., from
$m=\{t^M:t\in\GTerms\}$.

\subsubsection{Subalgebras}

The first requirement for a standard subalgebra $N$ of $M$ is that
$\Card N\subseteq\Card M$. Since our carriers are PO's, we will also
require that the relation $\subseteq$ preserves the PO, i.e.,
$X,Y\in\Card N \land X\po^M Y\impl X\po^N Y$.
As usual,
we have three notions instead of one:
\begin{Definition}
$N$ is a
subalgebra of $M$ iff $\Card N\subseteq \Card M$ and 

\begin{tabular}{lcl}
tight & iff & $\forall f\ \forall x_i\in\Card N : f^N(x_i)=f^M(x_i)$ \\
loose & iff & $\forall f\ \forall x_i\in\Card N : f^N(x_i)\po f^M(x_i)$ \\
closed & iff & $\forall f\ \forall x_i\in\Card N : f^N(x_i)\rpo f^M(x_i)$ 
\end{tabular}
\end{Definition}

In other words, $N$ is a T/L/C subalgebra of $M$ if $\Card N\subseteq
\Card M$ and inclusion of the carrier $\iota:\Card
N\hookrightarrow\Card M$ is a T/L/C homomorphism.

When is the homomorphic image $\phi[N]$ a subalgebra of $M$? Since
only loosely reachable part yields a leagal carrier, let us consider
this question for the special case: when is $\reach L{\phi[N]}M$
=$\{X\in\Card M : \exists Y\in\Card N:X\po\phi(Y)\}$ a subalgebra --
and of what kind -- of $M$?

\begin{Fact}\label{fa:TCsub}
If $\phi:N\into M$ is tight (closed) then $\reach L{\phi[N]}{M}$
is a tight subalgebra of $M$.
\end{Fact}

\begin{Proof}
Consider first the case of tight $\phi$.  Let us abbreviate the
obtained algebra as $[N]$. Its carrier is $\reach L{\phi[N]}{}$, and
for each $X$ in this carrier, we let $f^{[N]}(X) = f^M(X)$. We have to
show that the carrier is closed under all operations.

Let $X$ be such that there exists $Y\in\Card N:X = \phi(Y)$.  Since
$\phi$ is tight, we have that 
\begin{equation}\label{eq:isin}
f^{[N]}(X) = f^M(X) = f^M(\phi(Y))
= \phi(f^N(Y))
\end{equation}
so these applications remain within $\reach L{\phi[N]}{}$, i.e., the
carrier of $[N]$.  This yields a well-defined $f^{[N]}$ because
if there exist two different $Y\not=Z\in\Card N: \phi(Y)=X=\phi(Z)$,
then, by the tightness of $\phi:$
\begin{equation}\label{eq:well}
 \phi(f^N(Y))=f^M(X)=\phi(f^N(Z))
\end{equation}
%
Let $X\in\reach L{\phi[N]}{}\setminus\phi[N]$, i.e.,  $X$ is not an image of any
$Y\in\Card N$. Then $X$ must be below some element which is in the
image  $\phi[N]$, i.e., there exists $Y\in\Card N:
X\spo\phi(Y)$. But then also, by monotonicity of all the opeartions in
$M:$
\begin{equation}\label{eq:notin}
 f^{[N]}(X)=f^M(X)\po f^M(\phi(Y))=\phi(f^N(Y))
\end{equation}
and hence
also these applications remain within the carrier of $[N]$. Hence
$[N]$ is a tight subalgebra of $M$.

If $\phi$ is closed then the last equalities in \re{eq:isin} and
\re{eq:notin} become $\po$. Thus the applications remain within the
carrier $\reach L{\phi[N]}{}$. Well-definedness requires a bit closer
look.  We may have the situation where \re{eq:well} does not hold,
i.e., for $Y\not= Z$, with $\phi(Y)=X=\phi(Z):
\phi(f^N(Y))\not=\phi(f^N(Z))$.  However, by closedness of $\phi$, we
then know that both $f^M(X)\po\phi(f^N(Y))$, and
$f^M(X)\po\phi(f^N(Z))$. Hence $f^{[N]}$ is still well-defined,
and $[N]$ is again a tight subalgebra of $M$.
%
\end{Proof}

\michal{If $\phi$ is loose, the equalities in \re{eq:isin} and \re{eq:notin}
turn into $\rpo$, and so we are not guaranteed that the applications
$f^{[N]}(X)=f^M(X)$ are within the carrier $\reach L{\phi[N]}{}$.
Furthermore, \re{eq:well} need not hold, and we may have that
$\phi(f^N(Y))\po f^M(X)$ and $\phi(f^N(Z))\po f^M(X)$. That $f^M(X)$
is the common upper bound, does not mean both will have a least
upper bound, and we cannot simply re-define the operations to yield such
lub's. This would require their existence and extension of the carrier with these new
elements, and everything becomes pretty messy.\\ Alternatively, we
could require for each $f,\ S\subseteq\Card N: (\forall Y\in
S:\phi(Y)=X) \impl \posup\phi(f^N(Y)) \po f^M(X))$ (to be able to
choose the value $f^{[N]}(X)$ as such a lub). In addition,
would still have to say what to do with $X$ which are not in the image
of $\phi$.\\
It does not help to restrict anything to \PS, or even to \PP-homomorphisms.}

\begin{Fact}
If $\phi:N\into M$ is loose then $\reach L{\reach C{\phi[N]}{}}{}$ is
a tight subalgebra of $M$.
\end{Fact}
\begin{Proof}
For any $X\in\Card{[N]}=\reach L{\reach C{\phi[N]}{}}{}$, we let $f^{[N]}(X)=f^M(X)$. 

If $X\in\phi[N]$, i.e., $\exists Y\in\Card N: X=\phi(Y)$, then, by
looseness of $\phi: \phi(f^N(Y))\po
f^M(\phi(Y))=f^M(X)=f^{[N]}(X)$. So $f^{[N]}(X)\in \reach
C{\phi[N]}{}$, and hence also in $\Card{[N]}$.

If no such $Y$ exists, i.e., $X\in\Card{[N]}\setminus \phi[N]$, then
there exist $Y\in\Card N, Z\in \Card{[N]}: \phi(Y)\po Z \land X\po
Z$. By looseness of $\phi$ and then by monotonicity of $f^M$, we have
$\phi(f^N(Y))\po f^M(\phi(Y))\po f^M(Z)$ and $f^M(X)\po f^M(Z)$. The
first of these inequalities implies $f^M(Z)\in\reach C{\phi[N]}{}$,
and then the second one $f^M(X)\in\reach L{\reach C{\phi[N]}{}}{}$.
\end{Proof}

Now, let us suppose that $\phi[N]$ is a legal carrier, i.e., we need
not close it as we did above. It does not necessarily mean that
$\phi[N]=\reach L{\phi[N]}{}$, since $\phi[N]$ may contain fewer
elements than all those which are L-reachable in $M$ from $\phi[N]$.
We now try to define the subalgebra $[N]$ by
$f^{[N]}(\phi(Y))=\phi(f^N(Y))$.  Obviously, this goeas fine when
$\phi$ is tight, since then $\phi(f^N(Y))=f^M(\phi(Y))$, and we are in
the first case of fact \ref{fa:TCsub}. 

If $\phi$ is loose, however, we
have the problem with well-definedness: if $Z\not= Y$ and
$\phi(f^N(Z))\not=\phi(f^N(Y))$ while $\phi(Z)=X=\phi(Y)$. The
sufficient and necessary condition for obtaining a well-defined
subalgebra $[N]$ is then that the images $\phi(f^N(Y_i))$ over all
$Y_i$ such that $\phi(Y_i)=X$ be ordered in a way admitting a
reasonable definition of $f^{[N]}(X)$. The two natural possibilities
would be to require that there exists an $j$ such that either
$\phi(f^N(Y_j))=\posup_i \phi(f^N(Y_i))$, or else
$\phi(f^N(Y_j))=\poinf_i \phi(f^N(Y_i))$. In both cases we can then take
$f^{[N]}(X)=\phi(f^N(Y_j))$. The construction will give a loose subalgebra of $M$.

Quite analogous condition and requirement allows us to obtain a closed
subalgebra of $M$ when $\phi$ is closed.

\michal{
Now, shall we look at reachable images obtained from $T_\Sigma$ ?}

\subsection{Initial and terminal models}
The following definition defines the word multistructure uniformly for
 all $\Sigma$-multialgebras:
\begin{Definition}
The underlying set $\under T_\Sigma$ is the set of ground $\Sigma$-terms,
 $\GTerms$, and the carrier $\Card{T_\Sigma}$ is simply $\under T_\Sigma$.
%$\Card{T_\Sigma}=\{\{t\}:t\in\GTerms\}\subset \PSet{\under T_\Sigma}$.
The interpretation of operations is defined in the obvious way.
\end{Definition}
We have the obvious
\begin{Fact}
 $T_\Sigma$ is initial in $\MA\Sigma{\SSs^T}$.
\end{Fact}
Since $T_\Sigma$ is essentially determinisitic the unique tight \SSs-morphism into
any other structure is actually a \PS-morphism, so that $T_\Sigma$ is initial
in $\MA\Sigma{\PS^T}$ as well. This result does not depend on the mode of the
algebras considered, so we have that
the word multistructure is an initial structure in the following cases:
\[
\begin{array}{|c@{\ \ }|@{\ \ }c@{\ \ }|@{\ \ }c@{\ \ }|@{\ \ \ }c@{\ \ \ }|}
hom & \MAlg(\PP) & \MAlg(\PS) & \MAlg(\SSs) \\ \hline
\PP  & T_\Sigma & --  & -- \\
\PS^T  & T_\Sigma & T_\Sigma & T_\Sigma \\
\SSs^T & T_\Sigma & T_\Sigma & T_\Sigma \\ \hline
\end{array} 
\]
For loose (or closed) homomorphisms, there is no
corresponding initiality result: the requirement
$\phi(f^{{T_\Sigma}}(x))\po f^M(\phi(x))$ means that, for a
$t\in\GTerms$, the image $\phi(t^{{T_\Sigma}})$ may be any point
$p\in\Card M:p\po t^M$. Thus the homomorphisms from the word
multistructure are no longer unique. The only thing we can say at the moment is:

\begin{Fact}
If $\phi:T_\Sigma\into M$ is loose (closed) then
$\phi[T_\Sigma]\subseteq \reach L{\GTerms}M$ ($\reach C{\GTerms}M$).
\end{Fact}

In all nine cases, the usual collapsing of all the elements yields a
terminal model. Unlike initiality, this applies to all T, C, and L-homomorphisms.

\subsection{Epi-mono factorisation}

\subsubsection{Quotients}
Given a PO $M$, any equivalence relation $\sim\ \subseteq M\times M$
allows us to construct a quotient $\quot M\sim$. If we want to ensure
that taking a quotient of a generated PO yields a generated PO, we
have to take more precautions. Consider the following examples in
which we enforce $A\sim B:$
\[
\spreaddiagramrows{-1.2pc}
\spreaddiagramcolumns{-1.2pc}
\def\objectstyle{\scriptstyle}
\diagramcompileto{ex1}
& A & {\textstyle \sim} & B &  & & & &    & & [A,B] &  \\
    & & &   &  & & {\textstyle  \Longrightarrow} \\
a1 \xline[uur] &  a2 \xline[uu] & &  b1 \xline[uu] &  b2 \xline[uul]  & & & & 
 a1 \xline[uurr] & a2 \xline[uur] & & b1 \xline[uul] & b2 \xline[uull]
\enddiagram 
\]
In this case quotient gives a legal carrier, where the equivalence
classes correspond to taking the union of its members.
\[
\spreaddiagramrows{-1.2pc}
\spreaddiagramcolumns{-1.2pc}
\def\objectstyle{\scriptstyle}
\diagramcompileto{ex2}
 & A\fposup X & &  B\fposup X & & & & & & A\fposup X \xdotted[rr] & & B\fposup X \\
& A \xline[u] & {\textstyle \sim} & B \xline[u] & X \xline[ulll] \xline[ul] 
   & & {\textstyle  \Longrightarrow}
     & & & & [A,B] \xline[ul] \xline[ur] & & X \xline[ulll] \xline[ul] \\
a1 \xline[ur] & a2 \xline[u] & & b1 \xline[u] &  b2 \xline[ul] &
     & & & a1 \xline[urr] & a2 \xline[ur] & & b1 \xline[ul] & b2 \xline[ull]
\enddiagram 
\]
Here, the result is not a carrier because we have that $\under
{A\fposup X} = \under {B\fposup X}$, while $A\fposup X \not= B\fposup
X$. The dotted line indicates the necessary identification to obtain a
carrier.
\[
\spreaddiagramrows{-1.2pc}
\spreaddiagramcolumns{-1.2pc}
\def\objectstyle{\scriptstyle}
\diagramcompileto{ex3}
& A & {\textstyle \sim} & B &  & & &    & & [A,B] \\
& &  X & &   &  &  {\textstyle \Longrightarrow} & & & X \xdotted[u] \\
a1 \xline[uur] & a2 \xline[uu] \xline[ur] & &  b1 \xline[uu] \xline[ul] 
     & b2 \xline[uul]  & & & 
     a1 \xline[uurr] & a2 \xline[uur] \xline[ur] 
        & & b1 \xline[uul] \xline[ul] & b2 \xline[uull]
\enddiagram 
\]
Here, the result is a carrier with the exception that $\under
X\subseteq \under{[A,B]}$, while the relation $X\po [A,B]$ is not
induced by the partial order before taking the quotient.

The examples illustrate the requirements we have to impose on the
relation $\sim$ and the resulting quotient in order to obtain a legal
carrier.

\begin{Definition}
Let $\sim$ be an arbitrary equivalence relation on a partial order
$P=\<M,\po\>$.  The {\em pre-quotient} $\prequot P\sim$ is $\<\prequot
M\sim,\preceq\>$ where
\begin{enumerate}
\item $\prequot M\sim = \{[X]:X\in M\}$, $[X]=\{Y\in M:Y\sim X\}$.
\item $ [X]\preceq [Y] \iff \under{[X]}\subseteq \under{[Y]}$
\end{enumerate}
\end{Definition}
We call $\prequot P\sim$ pre-quotient because, in general, it is not a
partial order but only a pre-order. To obtain a partial order, we have
to enforce antisymmetry, i.e., make further identifications:
$[X]\preceq [Y]\land [Y]\preceq [X]\impl [X]=[Y]$. This construction
yields then the quotient $\quot P\sim=\<\quot M\sim,\preceq\>$ which
is a generated partial order.

\michal{I do not like anything what follows in this subsection!!!}

\begin{Fact}
For any equivalence $\sim\ \subseteq M\times M$, $\<\quot
M\sim,\preceq\>$ is a partial order generated by $\under{\prequot
M\sim}=\under{\quot M\sim}$ and the following two ways of constructing
the quotient are equivalent:
\[
\def\labelstyle{\scriptscriptstyle}
\diagram
& 1:\<\prequot M\sim,\po\> 
    \xto[rrr]^{[X]=[Y]\iff}_{[X]\po[Y]\land [Y]\po [X]} 
     & & & 2:\<{\quot M\sim}',\po\>
     \drto|{\under{[X]}\subseteq\under{[Y]}\impl [X]\preceq [Y]} \\
P=\<M,\po\> \urto|{X\po Y\impl [X]\po [Y]} \drto|{[X]\preceq
        [Y]\iff \under{[X]} \subseteq\under{[Y]}} & & & & & % & &
        \<\quot M\sim,\preceq\>=\quot P\sim \\ 
& 3:\<\prequot M\sim,\preceq\>
        \xto '[rrr]^>>>>>>{[X]=[Y]\iff}_>>>>>{[X]\preceq [Y] \land [Y]\preceq
        [X]}[urrrr] & & &  & % &
%& & 1:\<\prequot M\sim,\po\> 
%    \xto[rrr]^{[X]=[Y]\iff}_{[X]\po[Y]\land [Y]\po [X]} 
%     & & & 2:\<{\quot M\sim}',\po\>
%     \drrto|{\under{[X]}\subseteq\under{[Y]}\impl [X]\preceq [Y]} \\
%P=\<M,\po\> \urrto|{X\po Y\impl [X]\po [Y]} \drrto|{[X]\preceq
%        [Y]\iff \under{[X]} \subseteq\under{[Y]}} & & & & & & &
%        \<\quot M\sim,\preceq\>=\quot P\sim \\ 
%& & 3:\<\prequot M\sim,\preceq\>
%        \xto '[rrr]^>>>>>>{[X]=[Y]\iff}_>>>>>{[X]\preceq [Y] \land [Y]\preceq
%        [X]}[urrrrr] & & &  & & 
\enddiagram
\]
\end{Fact}

\begin{Proof}
Following the lower path, it is easy to see that $\under{\prequot
M\sim}=\under{\quot M\sim}:$ the transition $3\into\quot P\sim$ will
identify two elements iff both $[X]\preceq [Y]$ and $[Y]\preceq [X]$,
in particular no minimal elements will be identified.  Similar remark
shows that no minimal elements are identified in the transition
$1\into 2$. We have to show that also transition $2\into\quot P\sim$
does not identify any minimal elements. (Other elements may be
identified here -- therefore we wrote ${\quot M\sim}'$ in 2.) But this
is obvious since if $[X]$ is minimal in 2, then there is no other
$[Y]$ such that $[Y]\po [X]$ in 2, and hence $\under{[X]}=[X]$. Thus,
for all $[Y]\in{\quot M\sim}'$ and all minimal $[X]\in\under{{\quot
M\sim}'}: [X]\preceq [Y]\iff [X]\po[Y]$. The identifications implied
by the transition $2\into \quot P\sim$ are only of the kind
illustrated in the second example above. Additionally, it will enforce
new relations like the one from the third example.

It is also easy to see that in 2 we have $[X]\po [Y]\impl
\under{[X]}\subseteq\under{[Y]}$. Thus going both ways, we obtain the underlying set already in the first step (constructing the pre-order $\prequot M\sim$), and in the last step we obtain, in both cases, the same partial order generated by this underlying set.
\end{Proof}

The upper path is closer to the construction on (partial) orders as
such, without invoking the set inclusions on the (subsets of) minimal
elements as it is done in the lower path. The last transition $2\into
\quot P\sim$ can be expressed equivalently as saying that we add the
relation $[X]\preceq [Y]$ if $\forall X_i\in[X]\ \forall x_i\in\under
X_i\ \exists Y_i\in[Y]: x_i\po Y_i$ in $P$. This, however, is just a
round-about way of expressing the condition
$\under{[X]}\subseteq\under{[Y]}$.

Notice also that, because we have to add such relations, we are not
able to express the conditions for $\quot P\sim$ to be a generated PO
in terms of the conditions on $\sim$ only.  We could say that: 
An equivalence relation $\sim$ on  $\<M,\po\>$ is {\em
$\posup$-closed} iff $\forall A,B\in M, X\subseteq M : A\sim B \impl
(A\posup X)\sim (B\posup X)$. 
Considering only such relations
would fix the problems like in the second example. But then, we would
still have to add the relations $\preceq$ as illustrated in the third
example, which are hardly expressible in terms of $\sim$.

Of particular interest, there will be equivalence relations on the
underlying sets or, more precisely, relations on $M$ induced by a
relation on $\under M$.  The intention of such an inducing is to
ensure that if two individuals are put in the relation $a\sim b$, then
so are all sets $X\cup{a}\sim X\cup{b}$, i.e., to consider only
$\posup$-closed relations. However, even in this particular case, we
cannot dispense with the extension of the resulting partial order.
If $P$ is
$\{1,2,3,4,5,\{1,2,3\},\{4,5\}\}$ and we identify $2\sim 4$ and $3\sim
5$, we obtain $\{1,24,35,\{1,23,45\},\{23,45\}\}$. The last set is now
included in $\{1,23,45\}$ and this inclusion must be added to the
resulting partial order.
Thus, only the following fact is obvious from the construction in the last fact.

\begin{Fact}
Let $P=\<M,\po\>$ be a PO generated by $\under M$. 
%If $\sim\ \subseteq \under M\times\under M$ is $\posup$-closed then 
$\quot P\sim =\<\quot M\sim,\po\>$ is a PO generated by $\quot {\under M}\sim$.
\end{Fact}

\michal{Thus, $\posup$-closedness seems a redundant notion even in this case. In short: starting with an arbitrary generated PO $P$, there are no conditions on $\sim$ which would guarantee that $\quot P\sim$ is generated.}

%\begin{Definition}
%Let $\Sigma = [S,F]$ be a signature. A (total) $\Sigma$-multialgebra
%is a tuple $M = [S^M,F^M]$ where $S^M = \{M_s\}_{s \in S}$ is a family
%of carrier sets for each sort $s \in S$ and $F^M = \{f^M\}_{f \in F}$
%is a family of set-valued functions such that 
%\[ 
%f^M : M_{s_1} \times \ldots \times M_{s_n} \rightarrow P^+(M_s) 
%\] 
%for each $[f : s_1 \times \ldots s_n \rightarrow s] \in F$ 
%\hfill$\Box$
%\end{Definition} 
%
%The class of all $\Sigma$-multialgebras will be denoted 
%$\MAlg(\Sigma)$.

\subsection{Existence of products (limits) and co-products (co-limits)}


%==============================================================================
%
%
%\section{Multialgebras and their homomorphisms}
%
%The number of various notions of a homomorphism between multialgebras
%might be suprising for an unprepared reader. Nipkow \cite{Nip86}
%introduces four kinds of a homomorphism whilst Hussmann
%\cite{Hus91,Hus92} puts forward only two. The difference between them
%is that the latter ones map elements of carriers of an algebra to
%subsets of carriers of another algebra. We will refer to such
%homomorphisms as P-homomorphisms (``powerset'' homomorphisms). Below
%we give definitions of the former:
%
%\begin{Definition}
%Let $\Sigma = [S,F],\ M,N \in \MAlg(\Sigma)$. A tight
%$\Sigma$-homomorphism from $M$ to $N$ is a fimily of mappings $\phi =
%\{\phi_s\}_{s \in S},\ \phi_s: |M|_s \rightarrow |N|_s$ such that for
%all $[f : s_1 \times s_2 \times \ldots \times s_n \rightarrow s] \in
%F$ and for all $a_i \in M_{s_i}, i \in \{1,\ldots,n\}$ the following
%condition holds:
%\[ 
%\phi_s(f^M(a_1,\ldots,a_n)) = f^N(\phi_{s_1}(a_1),\ldots,\phi_{s_n}(a_n)) 
%\]
%\hfill$\Box$
%\end{Definition}
%
%The tight homomorphism is the most obvious extention of the original
%notion for deterministic algebras. However, for many applications it
%turns out to be too strong. In the next definition we cover the
%remaining ones. For the simplicity of presentation we restrict
%ourselves to single-argument functions.
%
%


%==============================================================================
%
%
\section{Standard Notions and Theorems}

One of the goals we would like to score here, is a more general view
on various kinds of homomorphisms between multialgebras. We cannot
proceed without other related notions, such as congruence and a
quotient of a multialgebra:

\begin{Definition}
Let $\Sigma = [S,F],\ M \in \MAlg(\Sigma)$ and let $\sim$ be an
($S$-sorted) equivalence relation on $M$.  If for all $[f : s'
\rightarrow s] \in F$ and for any $a,b \in |M|_{s'}$:
\begin{itemize}
\item{the relation $\sim$ is a heavy congruence if
      $a \sim b$ implies $f^M(a) \times f^M(b) \ \subseteq \ \sim$ }
\item{the relation $\sim$ is a tight congruence if \\
      $a \sim b$ implies
      $\forall_{a' \in f^M(a)}\ \exists_{b' \in f^M(b)}\ a' \sim b'$}
\item{the relation $\sim$ is a light congruence if 
      $a \sim b$ implies
      $\exists_{a' \in f^M(a)}\ \exists_{b' \in f^M(b)}\ a' \sim b'$}
\item{the relation $\sim$ is a collapsing congruence if 
      $\ \forall_{a \in M_{s'}}\ |\{ [a']_\sim : a' \in f^M(a) \}| = 1$ 
      \hfill$\Box$}
\end{itemize}
\end{Definition}

\begin{Fact}
If $\sim$ is both a collapsing and a light congruence then it is a
heavy congruence.
\hfill$\Box$
\end{Fact} 

Each homomorphism $\phi: M \rightarrow N$ gives rise to a relation on
its domain. As usuall, we called it a {\em kernel\/} of the
homomorphism.

\begin{Definition}
Let $M,N \in \MAlg(\Sigma)$ and let $\phi$ be a homomorphism from $M$
to $N$. Then the kernel of $\phi$ is a relation $\sim_\phi$ on $M$
given by:
\[ a \sim_\phi a' \quad {\rm iff}\quad \phi(a)=\phi(a') \]
\hfill$\Box$
\end{Definition} 

We can now state two basic observations on dependence between types of
homomorphisms and the induced congruence relations.
    
\begin{Fact}
Let $M,N \in \MAlg(\Sigma),\ \phi$ be a homomorphism from $M$ to $N$
and $\sim_\phi$ its kernel.
\begin{itemize}
  \item{$\sim_\phi$ is a tight congruence iff $\phi$ is a tight homomorphism;}
  \item{if $\phi$ is a closed homomorphism then $\sim_\phi$ is a light 
        congruence. \hfill$\Box$}
\end{itemize}
\end{Fact}

\begin{Proof} We only prove the second part. Assume that $a \sim
b$. Hence we need to show that $\phi(f^M(a)) \cap \phi(f^M(b)) \not=
\emptyset$.  Since $\phi(a) = \phi(b)$ we get $f^N(\phi(a)) =
f^N(\phi(b))$.  By the definition of the closed homomorphism
$\phi(f^M(a)) \supseteq f^N(\phi(a)) = f^N(\phi(b)) \subseteq
\phi(f^M(b))$.  Thus $\phi(f^M(a)) \cap \phi(f^M(b)) \supseteq
f^N(\phi(a)) = f^N(\phi(b)) \not= \emptyset$.
\end{Proof}

\begin{Definition}
Let $\Sigma = [S,F],\ M \in \MAlg(\Sigma)$ and let $\sim_M$ be an
equivalence relation on $M$. Then a quotient of the multialgebra $M$
wrt.\ $\sim_M$ (denoted $M\Mquo_\sim$) is a multialgebra $N$ such
that:
\[
\begin{array}{l}
|N|_s = \{ [a]_\sim : a \in M_s \} \\ 
f^N([a_1]_\sim,\ldots,[a_n]_\sim) = 
\{ [b]_\sim : b \in f^M(a'_1,\ldots,a'_n), a'_i \in [a_i]_\sim \}
\end{array}
\]
where $[f : s_1 \times s_2 \times \ldots \times s_n \rightarrow s] \in
F$ and $a_i \in M_{s_i}$ for $i \in \{1,\ldots,n\}$; $[a]_\sim$
denotes the equivalence class of $a$.  
\hfill$\Box$
\end{Definition}

The most important aspect of this definition, is that the relation
$\sim$ does not need to be a congruence and this highlights the
flexibility of multialgebras for many applications. A very special
case occurs when the multialgebra $M$ is deterministic and $\sim$ is
not a congruence and leads to a concept of a {\em nondeterminisation}
of an algebra.

\begin{Fact}
If $\sim$ is a heavy congruence then $M\Mquo_\sim$ is a deterministic
algebra.
\hfill$\Box$
\end{Fact}

\begin{Lemma}
Let $\Sigma$ be a signature and let $M \in \MAlg(\Sigma)$ and let
$\sim$ be an equivalence relation on $M$. Then a mapping $\phi: A
\longrightarrow M\Mquo_\sim$ such that $a \mapsto [a]_\sim$ is a loose
homomorphism.
\hfill$\Box$
\end{Lemma}

\begin{Proof} 
$f^N(\phi(a)) = f^N([a]_\sim) = \{[b]_\sim : b \in f^M(a'), a' \in
[a]_\sim\} \supseteq \{[b]_\sim : b \in f^M(a)\} = \phi(f^M(a))$
\end{Proof}

\begin{Theorem}
Let $\Sigma$ be a signature and $M,N \in \MAlg(\Sigma)$.  Let $\phi: M
\longrightarrow N$ be an onto homomorphism and $\sim$ the
$\phi$-induced equivalence relation on $M$.  Define $\psi_1: M
\longrightarrow M\Mquo_\sim$ and $\psi_2: M\Mquo_\sim \longrightarrow
N$ as follows: $\psi_1(a) = [a]_\sim$ and $\psi_2([a]_\sim) =
\phi(a)$.
\begin{enumerate}
\item{if $\phi$ is tight then both $\psi_1$ and $\psi_2$ are tight;}
\item{if $\phi$ is closed then $\psi_1$ is loose and $\psi_2$ is closed;}
\item{if $\phi$ is very strong then $\psi_1$ is loose and $\psi_2$ is closed.}
\end{enumerate}
\end{Theorem}

\begin{Proof}
(1) We have to show that $\psi_1(f^M(a)) = f^C(\psi_1(a))$ and
$\psi_2(f^C([a]_\sim)) = f^N(\psi_2([a]_\sim))$. First consider
$\psi_1$: $f^C(\psi_1(a)) = f^C([a]_\sim) = \{[b]_\sim : b \in
f^M(a'), a' \in [a]_\sim\}$.  Since $\sim$ is tight i.e.\ if $a \sim
a'$ then $\forall_{b \in f^M(a)}\ \exists_{b' \in f^M(a')}\ b \sim
b'$, we have $\{[b]_\sim : b \in f^M(a'), a' \in [a]_\sim\} =
\{[b]_\sim : b \in f^M(a)\} = \psi_1(f^M(a))$. \\ Now consider
$\phi_2$: $\phi_2(f^C([a]_\sim)) = \psi_2(\{[b]_\sim : b \in f^M(a'),
a' \in [a]_\sim\}) =$ (since $\sim$ is a tight congruence) $=
\psi_2(\{[b]_\sim : b \in f^M(a)\}) = \phi(\{b: b \in f^M(a)\}) =
\phi(f^M(a)) = f^N(\phi(a)) = f^N(\psi_2([a]_\sim)).$

(2) $\psi_1$ is loose by previous lemma. Now, note that $[a]_\sim =
\phi^{-1}\phi(a)$. Hence $\psi_2(f^C([a]_\sim)) = \psi_2(\{[b]_\sim :
b \in f^M(a'), a' \in [a]_\sim\}) = \{\phi(b): b \in f^M(a'), a' \in
[a]_\sim\} = \phi(f^M(\phi^{-1}(\phi(a)))) \supseteq \phi(f^M(a))
\supseteq f^N(\phi(a)) = f^N(\psi_2([a]_\sim))$.

(3) Similarly, $\psi_1$ is loose by lemma. We ought to prove that
$\psi_2(f^C([a]_\sim)) \supseteq f^N(\psi_2([a]_\sim))$.
$\psi_2(f^C([a]_\sim)) = \phi(f^M(\phi^{-1}(\phi(a)))) \supseteq
f^N(\phi(a)) = f^N(\psi_2([a]_\sim))$.
\end{Proof}
\medskip

The above theorem is an equivalent of the well-known Homomorphism Theorem.


\marcin{We are also to consider: power algebras, power-homomorphisms
and usual algebras and try to relate these, i.e.\ multi-homomorphisms
as special cases of power-homomorphisms, etc. In addition, we are to
look at relational algebras.}

%==============================================================================
%
%
\section{Relations between multialgebras}

The relations we are to consider basically come from concurrency
theory. 
%Now we can present some relations:
\begin{Definition}
Let $\Sigma = [S,F],\ M,N \in \MAlg(\Sigma)$ and let $\phi$ be a
mapping from $M$ to $N$ (i.e.\ a fimily of mappings $\phi =
\{\phi_s\}_{s \in S},\ \phi_s: |M|_s \rightarrow |N|_s$). If for all
$[f : s' \rightarrow s] \in F$ and for all $a \in |M|_{s'}$:
\begin{itemize}
\item{ $\phi_s^{-1}(f^N(\phi_{s'}(a))) \subseteq
       \phi^{-1}_s(\phi_s(f^M(\phi^{-1}_{s'}(\phi_{s'}(a)))))$
      then $\phi$ is a strong $\Sigma$-homomorphism; }
\item{ $f^N(\phi_{s'}(a)) \subseteq
       \phi_s(f^M(\phi^{-1}_{s'}(\phi_{s'}(a))))$
       then $\phi$ is a very strong $\Sigma$-homomorphism; \hfill$\Box$}
\end{itemize}
\end{Definition}


\begin{Definition}
6 notions ++ ?
\end{Definition}

\michal{Valis' suggestion: specialization of L for \PS\ $\forall y\in f^M(x):\ 1)\
\phi(y)\subseteq f^N(\phi(x))\ \ \land\ \ 2)\ (f^{N})^{-1}(\phi(y))\supseteq \phi(x)$.}

\begin{Definition}
Let $M,N \in \MAlg(\Sigma)$. A simulation of $M$ by $N$, denoted by
$N \sinc M$, is an $S$-sorted relation $\sinc_s\ \subseteq\ N_s
\times M_s$ such that: for all $[f : w \rightarrow s] \in F$, for all
$b \in N_w, a \in M_w, b' \in f^N(b)$ if $a \sinc b$ then there
exists $a' \in f^M(a)$ and $b' \sinc a'$. In other ``words'':
\[
\forall_{[f : w \rightarrow s] \in F} 
\ \forall_{b \in N_w,\ a \in M_w,\ b' \in f^N(b)}
\quad b \sinc a \Rightarrow \exists_{a' \in f^M(a)}\ b' \sinc a' 
\]
\hfill$\Box$
\end{Definition}
%
We say that two $\Sigma$-multialgebras $M$ and $N$ are {\em similar\/}
(or simulation equivalent, denoted $M \seq N$) iff there exist two
simulations $\sinc_1$ and $\sinc_2$ such that $M \sinc_1 N$ and $N
\sinc_2 M$.


A useful concept in defining some relations, is that of a {\em
determinisation\/} of a multialgebra:

\begin{Definition}
Let $M,N \in \MAlg(\Sigma)$. We say that $N$ is a determinisation of
$M$ iff $N$ is a loose subalgebra of $M$. Moreover, $N$ is a complete
determinisation if it is deterministic i.e.\ for any $[f : s_1 \times
s_2 \times \ldots \times s_n \rightarrow s] \in F$ and for all $a_i
\in N_{s_i},\ i \in \{1,\ldots,n\}$ we have $|f^N(a_1,\ldots,a_n)| =
1.$ 
\hfill$\Box$
\end{Definition}
%
We adopt notations $\Det(M)$ and $\CDet(M)$ for respective classes of
all determinizations and complete determinisations of a given
$\Sigma$-multialgebra $M$. 

\begin{Definition}
Let $M,N \in \MAlg(\Sigma)$. We say that $N$ is trace included in $M$
(written $N \tinc M$ iff for any $B \in \CDet(N)$ there is $A \in
\CDet(M)$ such that $A \cong B$.  
\hfill$\Box$
\end{Definition}
%
$M$ and $N$ are {\em trace equivalent\/} iff $N \tinc M$ and $M
\tinc N$. We then write $M \teq N$.

%==============================================================================
%
%
\section{Logics}

\marcin{Closure of model classes (varieties, quasi-varieties),
preservation of satisfaction of formulae.}

%==============================================================================
%
%
\section{Categories}

\marcin{Self explanatory.}

%==============================================================================
%
%
\section{Applications}

\marcin{Notions of a refinement and implementation.} 

%==============================================================================
%
%
\begin{thebibliography}{Nip 86}

   \bibitem[Bar 75]{KP} J.~Barwise: {\em Admissible Sets and Structures.}
       Series Perspective in Mathematical Logic, Springer-Verlag, 1975

   \bibitem[BT 95]{Reg} M.~Bidoit, A.~Tarlecki: Regular Algebras -- a
       Framework for Observational Specifications with Recursive Definitions.

   \bibitem[Hes 88]{Hes88} W.\ Hesselink: A Mathematical Approach to
       Nondeterminism in Data Types. {\em ACM ToPLaS},
       10, 1988, pp.\ 87--117.

   \bibitem[Hus 92]{Hus92} H.\ Hussmann: {\em Nondeterministic Algebraic
       Specifications and Nonconfluent Term rewriting.} J.\ Logic Programming
       12, 1992, pp.\ 237--255.

   \bibitem[Hus 91]{Hus91} H.\ Hussmann: {\em Nondeterministic Algebraic
       Specifications.} Technische Universit\"{a}t M\"{u}nchen, TUM-I0104,
       March 1991.

   \bibitem[Kri 64]{K} S.~Kripke: Transfinite recursion on admissible ordinals. 
       {\em Journal of Symbolic Logic}, 29, 1964.

   \bibitem[Mol 85]{Moll} B.~M\"{o}ller: On the Algebraic Specification of
   Infinite Objects: Ordered and Continuous Models of Algebraic Types. {\em
   Acta Informatica}, 22, 1985, pp. 537--578.

   \bibitem[Mos ]{UA} P.~Mosses: Unified Algebras and Institutions

   \bibitem[Nip 86]{Nip86} T.\ Nipkow: Non-deterministic Data Types: 
       Models and Implementations. {\em Acta Informatica}, 22, 1986, pp.\ 629--661.

   \bibitem[Nip 87]{Nip87} T.\ Nipkow: {\em Observing Nondeterministic Data
       Types.} LNCS 332.

   \bibitem[Pla 66]{P} R.~Platek: {\em Foundations of Recursion Theory.} PhD Thesis,
       Stanford University, USA, 1966.

   \bibitem[Sch 87]{Sch87} O.\ Schoett: {\em Data Abstraction and the 
       Correctness of Modular Programming.} Ph.D.\ Thesis, University of
       Edinburgh, 1987.

   \bibitem[Wal 93]{Wal93} M.\ Walicki: {\em Algebraic Specifications of 
       Nondeterminism.} Ph.D.\ Thesis, University of Bergen, 1993.

   \bibitem[WM 95]{Tapsoft} M.~Walicki, S.~Meldal: Generated Models and the $\omega$-Rule:
       The Nondeterministic Case. {\em Proceedings of TAPSOFT'95,} LNCS vol.~915, 1995.
       
\end{thebibliography}


\end{document}


