\documentstyle[12pt,marcin]{article}

% arbitrary institution

\newcommand{\Mod}{\mbox{\rm\bf Mod}}
\newcommand{\Sign}{\mbox{\rm\bf Sign}}
\newcommand{\Sen}{\mbox{\rm\bf Sen}}
\newcommand{\Set}{\mbox{\rm\bf Set}}
\newcommand{\Cat}{\mbox{\rm\bf Cat}}

% particular categories

\newcommand{\MAlg}{\mbox{\rm\bf MAlg}}
\newcommand{\Alg}{\mbox{\rm\bf Alg}}
\newcommand{\AlgSig}{\mbox{\rm\bf AlgSig}}

\newcommand{\MA}[2]{\MAlg(#1,#2)}

\newcommand{\Nat}{{\sf Nat}}

% homomorphisms & multCats

\newcommand{\SSs}{{\rm SS}}
\newcommand{\PS}{{\rm PS}}
\newcommand{\PP}{{\rm PP}}
\newcommand{\XX}{{\rm XX}}

\newcommand{\mode}[2]{#1(#2)}

% specifications

\newcommand{\SMod}{\mbox{\sf Mod}}
\newcommand{\SSig}{\mbox{\sf Sig}}
\newcommand{\SP}{\mbox{\rm SP}}

% symbols

\newcommand{\reach}[3]{{\sf Reach}^{#1}_{#3}(#2)}

\newcommand{\tinc}{\stackrel{\rm T}{\rightharpoonup}}
\newcommand{\sinc}{\stackrel{\rm S}{\rightharpoonup}}
\newcommand{\teq}{\stackrel{\rm T}{\rightleftharpoons}}
\newcommand{\seq}{\stackrel{\rm S}{\rightleftharpoons}}

\newcommand{\<}{\langle}
\renewcommand{\>}{\rangle}

\newcommand{\ovr}[1]{\overline {#1}}
\newcommand{\under}[1]{\underline {#1}}

\newcommand{\quot}[2]{#1\!{\textstyle /}\!_#2}
\newcommand{\prequot}[2]{#1\!{\textstyle /}\!\!{\textstyle /}\!_#2}

\newcommand{\Comp}[2]{#1;#2}
\newcommand{\impl}{\Rightarrow}
\newcommand{\into}{\rightarrow}
\newcommand{\Card}[1]{|#1|}
\newcommand{\PSet}{{\cal P}} 

\newcommand{\Terms}{{\cal T}}
\newcommand{\GTerms}{\Terms_{\Sigma}}
\newcommand{\XTerms}{\Terms_{\Sigma}(X)}

% Partial orders

\newcommand{\po}{\leq}
\newcommand{\rpo}{\geq}
\newcommand{\spo}{<}
\newcommand{\posup}{\bigvee}
\newcommand{\fposup}{\vee} % finite \posup
% ponizej powinno byc \bigsqcap
\newcommand{\poinf}{\bigwedge}
\newcommand{\fpoinf}{\wedge}

\newcommand{\setsup}{\bigcup}
\newcommand{\fsetsup}{\cup}
\newcommand{\setinf}{\bigcap}
\newcommand{\fsetinf}{\cap}

% other

\newcommand{\Det}{\mbox{\sf Det}}
\newcommand{\PDet}{\mbox{\sf PDet}}
\newcommand{\CDet}{\mbox{\sf CDet}}
\newcommand{\Mquo}{\mbox{$/\!\!/$}}
\newcommand{\implem}{\mbox{$\sim\!\leadsto$}}

\newcommand{\re}[1]{(\ref{#1})}

% environments

\newtheorem{Definition}{Definition}[section]
\newtheorem{Fact}{Fact}[section]
\newtheorem{Lemma}{Lemma}[section]
\newtheorem{Example}{Example}[section]
\newtheorem{Theorem}{Theorem}[section]
\newtheorem{Corollary}{Corollary}[section]
\newenvironment{Proof}
               {\setlength{\parskip}{5pt}
                \setlength{\parindent}{0pt}
                {\sc Proof:}}{\hfill{\sc qed}\vspace{1.5ex}}
%                {\sc Proof:}}{\hfill{\sc qed}$\:$\raisebox{-1pt}{$\Box$}}

\newenvironment{REMARK}[1]{\vspace{1ex}\par\noindent{\bf #1}}
            {\vspace{1ex}\par\noindent\ignorespaces}
% and numbered version
\newcounter{CLAIM}[section]
\newenvironment{REMARKno}[1]{\refstepcounter{CLAIM}
       \vspace{1ex}\par\noindent{\bf #1\ \theCLAIM .}}
       {\vspace{1ex}\par\noindent\ignorespaces}

% small remark
\newenvironment{SREMARK}[1]{\vspace{1ex}\par\noindent\small{\bf #1}}
            {\vspace{1ex}\par\noindent\normalsize\ignorespaces}
\newenvironment{SREMARKno}[1]{\refstepcounter{CLAIM}
       \vspace{1ex}\par\noindent\small {\bf #1\ \theCLAIM .}}
       {\vspace{1ex}\par\noindent\normalsize\ignorespaces}


% co-operation (sic!) 

\newcommand{\marcin}[1]{\vspace{1ex}{\sf [][][ MB: #1 ][][]}\vspace{1ex}}
\newcommand{\michal}[1]{\vspace{1ex}{\sf [][][ MW: #1 ][][]}\vspace{1ex}}


%==============================================================================
%
%   The Document

\input xypic

\begin{document} 

There are various equivalent ways of representing generated PO's and,
occasionally, we will use alternative representations. For instance,
$M=\{1,2,3,4,\{1,2\},\{1,2,3\},\{1,2,4\},\{1,2,3,4\}\}$ corresponds to the diagram to the left:

\spreaddiagramrows{-0.5pc}
\spreaddiagramcolumns{-1pc}
\def\objectstyle{\scriptstyle}
\diagram
%& & \save\Drop{\{1,2,3,4\}}\restore &   & & & & & \save\Drop{\{\{\{1,2\},3\}, \{\{1,2\},4\} \}}\restore & \\
& & \{1,2,3,4\} &   & & & & & \{\{\{1,2\},3\}, \{\{1,2\},4\} \} & \\
 & \{1,2,3\} \xline[ur] & & \{2,3,4\} \xline[ul] & & &  & \{\{1,2\},3\} \xline[ur]& & \{\{1,2\},4\} \xline[ul] \\
& \{1,2\} \xline[u] \xline[urr] & & & M\ \rrto<0.2pc>^{\_\in} & & \llto<0.2pc>^{sp} \ M^\in  & \{1,2\} \xline[u] \xline[urr] \\
1 \xline[ur] & 2 \xline[u] & 3 \xline[uul] & 4 \xline[uu]  & & &
                                1 \xline[ur] & 2 \xline[u] & 3 \xline[uul] & 4 \xline[uu]
\enddiagram


But we may interpret the relations indicated in the diagram as $\in$
instead of $\subseteq$. In this case, $M$ corresponds to
$M^\in=\{1,2,3,4, \{1,2\}, \{\{1,2\},3\}, \{\{1,2\},4\},\{\{\{1,2\},3\},\{\{1,2\},4\} \}\}$.  We have
the obvious function $\_^\in:M\into M^\in$ which sends each $X\in M$ onto
$X^\in=\{Y\in M:Y\subseteq X\lor Y\in X\cap\under M\}$ in
$M^\in$. (Actually, we are confusing here minimal $Y$ with $\{Y\}$.)
Similarly, the support function $sp:M^\in\into M$ sends $X^\in \mapsto
\left\{\begin{array}{l} X\ {\rm if}\ X\in\under{M^\in} \\
  \bigcup_{Y\in X} sp(Y)\ {\rm otherwise} \end{array}\right.$.



\end{document}