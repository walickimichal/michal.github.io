%\documentclass[10pt]{article}
%\makeatletter
%
\ifcase \@ptsize
    % mods for 10 pt
    \oddsidemargin  0.15 in     %   Left margin on odd-numbered pages.
    \evensidemargin 0.35 in     %   Left margin on even-numbered pages.
    \marginparwidth 1 in        %   Width of marginal notes.
    \oddsidemargin 0.25 in      %   Note that \oddsidemargin = \evensidemargin
    \evensidemargin 0.25 in
    \marginparwidth 0.75 in
    \textwidth 5.875 in % Width of text line.
\or % mods for 11 pt
    \oddsidemargin 0.1 in      %   Left margin on odd-numbered pages.
    \evensidemargin 0.15 in    %   Left margin on even-numbered pages.
    \marginparwidth 1 in       %   Width of marginal notes.
    \oddsidemargin 0.125 in    %   Note that \oddsidemargin = \evensidemargin
    \evensidemargin 0.125 in
    \marginparwidth 0.75 in
    \textwidth 6.125 in % Width of text line.
\or % mods for 12 pt
    \oddsidemargin -10 pt      %   Left margin on odd-numbered pages.
    \evensidemargin 10 pt      %   Left margin on even-numbered pages.
    \marginparwidth 1 in       %   Width of marginal notes.
    \oddsidemargin 0 in      %   Note that \oddsidemargin = \evensidemargin
    \evensidemargin 0 in
    \marginparwidth 0.75 in
    \textwidth 6.375 true in % Width of text line.
\fi

\voffset -2cm
\textheight 22.5cm

%\makeatother
\documentstyle{llncs}

%\makeatletter
%\show\
%\makeatother
\newcommand{\ite}[1]{\item[{\bf #1.}]}
\newcommand{\app}{\mathrel{\scriptscriptstyle{\vdash}}}
\newcommand{\estr}{\varepsilon}
\newcommand{\PSet}[1]{{\cal P}(#1)}
\newcommand{\ch}{\sqcup}
\newcommand{\into}{\to}
\newcommand{\Iff}{\Leftrightarrow}
\renewcommand{\iff}{\leftrightarrow}
\newcommand{\prI}{\vdash_I}
\newcommand{\pr}{\vdash}
\newcommand{\ovr}[1]{\overline{#1}}

\newcommand{\cp}{{\cal O}}

% update function/set
%\newcommand{\upd}[3]{#1\!\Rsh^{#2}_{\!\!#3}} % AMS
\newcommand{\upd}[3]{#1^{\raisebox{.5ex}{\mbox{${\scriptscriptstyle{\leftarrow}}\scriptstyle{#3}$}}}_{{\scriptscriptstyle{\rightarrow}}{#2}}} 
\newcommand{\rem}[2]{\upd{#1}{#2}{\bullet}}
\newcommand{\add}[2]{\upd {#1}{\bullet}{#2}}
%\newcommand{\mv}[3]{{#1}\!\Rsh_{\!\!#3}{#2}}
\newcommand{\mv}[3]{{#1}\:\raisebox{-.5ex}{$\stackrel{\displaystyle\curvearrowright}{\scriptstyle{#3}}$}\:{#2}}

\newcommand{\leads}{\rightsquigarrow} %AMS

\newenvironment{ites}{\vspace*{1ex}\par\noindent 
   \begin{tabular}{r@{\ \ }rcl}}{\vspace*{1ex}\end{tabular}\par\noindent}
\newcommand{\itt}[3]{{\bf #1.} & $#2$ & $\impl$ & $#3$ \\[1ex]}
\newcommand{\itte}[3]{{\bf #1.} & $#2$ & $\impl$ & $#3$ }
\newcommand{\itteq}[3]{\hline {\bf #1} & & & $#2=#3$ }
\newcommand{\itteqc}[3]{\hline {\bf #1} &  &  & $#2=#3$ \\[.5ex]}
\newcommand{\itteqq}[3]{{\bf #1} &  &  & $#2=#3$ }
\newcommand{\itc}[2]{{\bf #1.} & $#2$ &    \\[.5ex]}
\newcommand{\itcs}[3]{{\bf #1.} & $#2$ & $\impl$ & $#3$  \\[.5ex] }
\newcommand{\itco}[3]{   & $#1$ & $#2$  & $#3$ \\[1ex]}
\newcommand{\itcoe}[3]{   & $#1$ & $#2$  & $#3$}
\newcommand{\bit}{\begin{ites}}
\newcommand{\eit}{\end{ites}}
\newcommand{\na}[1]{{\bf #1.}}
\newenvironment{iten}{\begin{tabular}[t]{r@{\ }rcl}}{\end{tabular}}
\newcommand{\ass}[1]{& \multicolumn{3}{l}{\hspace*{-1em}{\small{[{\em Assuming:} #1]}}}}

%%%%%%%%% nested comp's
\newenvironment{itess}{\vspace*{1ex}\par\noindent 
   \begin{tabular}{r@{\ \ }lllcl}}{\vspace*{1ex}\end{tabular}\par\noindent}
\newcommand{\bitn}{\begin{itess}}
\newcommand{\eitn}{\end{itess}}
\newcommand{\comA}[2]{{\bf #1}& $#2$ \\ }
\newcommand{\comB}[3]{{\bf #1}& $#2$ & $#3$\\ }
\newcommand{\com}[3]{{\bf #1}& & & $#2$ & $\impl$ & $#3$\\[.5ex] }

\newcommand{\comS}[5]{{\bf #1} 
   & $#2$ & $#3$ & $#4$ & $\impl$ & $#5$\\[.5ex] }

%%%%%%%%%%%%%%%%
\newtheorem{CLAIM}{Proposition}[section]
\newtheorem{COROLLARY}[CLAIM]{Corollary}
\newtheorem{THEOREM}[CLAIM]{Theorem}
\newtheorem{LEMMA}[CLAIM]{Lemma}
\newcommand{\MyLPar}{\parsep -.2ex plus.2ex minus.2ex\itemsep\parsep
   \vspace{-\topsep}\vspace{.5ex}}
\newcommand{\MyNumEnv}[1]{\trivlist\refstepcounter{CLAIM}\item[\hskip
   \labelsep{\bf #1\ \theCLAIM\ }]\sf\ignorespaces}
\newenvironment{DEFINITION}{\MyNumEnv{Definition}}{\par\addvspace{0.5ex}}
\newenvironment{EXAMPLE}{\MyNumEnv{Example}}{\nopagebreak\finish}
\newenvironment{PROOF}{{\bf Proof.}}{\nopagebreak\finish}
\newcommand{\finish}{\hspace*{\fill}\nopagebreak 
     \raisebox{-1ex}{$\Box$}\hspace*{1em}\par\addvspace{1ex}}
\renewcommand{\abstract}[1]{ \begin{quote}\noindent \small {\bf Abstract.} #1
    \end{quote}}
\newcommand{\B}[1]{{\rm I\hspace{-.2em}#1}}
\newcommand{\Nat}{{\B N}}
\newcommand{\bool}{{\cal B}{\rm ool}}
\renewcommand{\c}[1]{{\cal #1}}
\newcommand{\Funcs}{{\cal F}}
%\newcommand{\Terms}{{\cal T}(\Funcs,\Vars)}
\newcommand{\Terms}[1]{{\cal T}(#1)}
\newcommand{\Vars}{{\cal V}}
\newcommand{\Incl}{\mathbin{\prec}}
\newcommand{\Cont}{\mathbin{\succ}}
\newcommand{\Int}{\mathbin{\frown}}
\newcommand{\Seteq}{\mathbin{\asymp}}
\newcommand{\Eq}{\mathbin{\approx}}
\newcommand{\notEq}{\mathbin{\Not\approx}}
\newcommand{\notIncl}{\mathbin{\Not\prec}}
\newcommand{\notCont}{\mathbin{\Not\succ}}
\newcommand{\notInt}{\mathbin{\Not\frown}}
\newcommand{\Seq}{\mathrel{\mapsto}}
\newcommand{\Ord}{\mathbin{\rightarrow}}
\newcommand{\M}[1]{\mathbin{\mathord{#1}^m}}
\newcommand{\Mset}[1]{{\cal M}(#1)}
\newcommand{\interpret}[1]{[\![#1]\!]^{A}_{\rho}}
\newcommand{\Interpret}[1]{[\![#1]\!]^{A}}
%\newcommand{\Comp}[2]{\mbox{\rm Comp}(#1,#2)}
\newcommand{\Comp}[2]{#1\diamond#2}
\newcommand{\Repl}[2]{\mbox{\rm Repl}(#1,#2)}
%\newcommand\SS[1]{{\cal S}^{#1}}
\newcommand{\To}[1]{\mathbin{\stackrel{#1}{\longrightarrow}}}
\newcommand{\TTo}[1]{\mathbin{\stackrel{#1}{\Longrightarrow}}}
\newcommand{\oT}[1]{\mathbin{\stackrel{#1}{\longleftarrow}}}
\newcommand{\oTT}[1]{\mathbin{\stackrel{#1}{\Longleftarrow}}}
\newcommand{\es}{\emptyset}
\newcommand{\C}[1]{\mbox{$\cal #1$}}
\newcommand{\Mb}[1]{\mbox{#1}}
\newcommand{\<}{\langle}
\renewcommand{\>}{\rangle}
\newcommand{\Def}{\mathrel{\stackrel{\mbox{\tiny def}}{=}}}
\newcommand{\impl}{\mathrel\Rightarrow}
\newcommand{\then}{\mathrel\Rightarrow}
\newfont{\msym}{msxm10}

\newcommand{\false}{\bot}
\newcommand{\true}{\top}

\newcommand{\restrict}{\mathbin{\mbox{\msym\symbol{22}}}}
\newcommand{\List}[3]{#1_{1}#3\ldots#3#1_{#2}}
\newcommand{\col}[1]{\renewcommand{\arraystretch}{0.4} \begin{array}[t]{c} #1
  \end{array}}
\newcommand{\prule}[2]{{\displaystyle #1 \over \displaystyle#2}}
\newcounter{ITEM}
\newcommand{\newITEM}[1]{\gdef\ITEMlabel{ITEM:#1-}\setcounter{ITEM}{0}}
\makeatletter
\newcommand{\Not}[1]{\mathbin {\mathpalette\c@ncel#1}}
\def\LabeL#1$#2{\edef\@currentlabel{#2}\label{#1}}
\newcommand{\ITEM}[2]{\par\addvspace{.7ex}\noindent
   \refstepcounter{ITEM}\expandafter\LabeL\ITEMlabel#1${(\roman{ITEM})}%
   {\advance\linewidth-2em \hskip2em %
   \parbox{\linewidth}{\hskip-2em {\rm\bf \@currentlabel\
   }\ignorespaces #2}}\par \addvspace{.7ex}\noindent\ignorespaces}
\def\R@f#1${\ref{#1}}
\newcommand{\?}[1]{\expandafter\R@f\ITEMlabel#1$}
\makeatother
\newcommand{\PROOFRULE}[2]{\trivlist\item[\hskip\labelsep {\bf #1}]#2\par
  \addvspace{1ex}\noindent\ignorespaces}
\newcommand{\PRULE}[2]{\displaystyle#1 \strut \over \strut \displaystyle#2}
%\setlength{\clauselength}{6cm}
%% \newcommand{\clause}[3]{\par\addvspace{.7ex}\noindent\LabeL#2${{\rm\bf #1}}%
%%   {\advance\linewidth-3em \hskip 3em
%%    \parbox{\linewidth}{\hskip-3em \parbox{3em}{\rm\bf#1.}#3}}\par 
%%    \addvspace{.7ex}\noindent\ignorespaces}
\newcommand{\clause}[3]{\par\addvspace{.7ex}\noindent
  {\advance\linewidth-3em \hskip 3em
   \parbox{\linewidth}{\hskip-3em \parbox{3em}{\rm\bf#1.}#3}}\par 
   \addvspace{.7ex}\noindent\ignorespaces}
\newcommand{\Cs}{\varepsilon}
\newcommand{\const}[3]{\Cs_{\scriptscriptstyle#2}(#1,#3)}
\newcommand{\Ein}{\sqsubset}%
\newcommand{\Eineq}{\sqsubseteq}%


\voffset -1cm


\newcommand{\see}[1]{\vspace*{-2ex}\section{#1}\vspace*{-1.5ex}}
\newcommand{\subsec}[1]{\vspace*{-1ex}\subsection{#1}\vspace*{-1ex}}
%\newenvironment{tabb}{\vspace*{-2ex}\begin{table}}{\vspace*{-2ex}\end{table}}
\input xypic

\begin{document}

\

\begin{center}
{ 
\LARGE 
Categories of Relational Structures
}
\vspace{2ex}

\begin{tabular}{c@{\hspace*{2em}}c}
{\large Micha\l{} Walicki} & {\large Marcin Bia\l{}asik} \\[0.5ex]
Institute of Informatics & Institute of Computer Science \\
University of Bergen & Polish Academy of Sciences \\
{\tt michal@ii.uib.no} & {\tt marcinb@ipipan.waw.pl} 
\end{tabular}
\end{center}

\vspace{2ex}
\noindent
\hspace*{-.4em}{\small{ {\bf Abstract.}
We characterise compositional homomorphims of relational
structures. A study of three categories of such structures --
viewed as multialgebras -- reveals the one with the most desirable
properties.  We study also analogous categories with
homomorphisms mapping elements to sets (thus being relations).
Finally, we indicate some consequences of our results for partial
algebras which are special case of multialgebras.
}}


%=====================================================================

\see{Introduction}

In the study of universal algebra, the central place occupies the pair
of ``dual'' notions of congruence and homomorphism: every congruence
on an algebra induces a homomorphism into a quotient and every
homomorphism induces a congruence on the source algebra.  Categorical
approach attempts to express {\em all} (internal) properties of
algebras in (external) terms of homomorphisms.  When
passing to relational structures, however, the close correspondence of
these internal and external aspects seems to get lost.

%These (and many other convenient) facts, depending slightly on the definition 
%of congruence and relational homomorphism, do not hold for relations.  

The most common generalisation of the definition of
homomorphism to relational structures says that a set function
$\phi:\under A\into \under B$, where both sets are equipped with
respective relations $R^A\subseteq \under A^n$ and $R^B\subseteq
\under B^n$, is a (weak) homomorphism 
% from the relational structure $A$ to $B$ 
iff \vspace*{-1ex}
\eq{ \<x_1...x_n\>\in R^A\ \impl
\<\phi(x_1)...\phi(x_n)\>\in R^B
\label{eq:weak} %\vspace*{-.5ex}
} 
Now {\em any} equivalence on $\under A$ gives rise
to a weak homomorphism and, conversely, a weak homomorphism induces,
in general, only an equivalence relation on $\under A$.  Hence this
homomorphism does not capture the notion of congruence and this is
just one example of an internal property of relational structures that
cannot be accounted for by relational homomorphisms (in various
variants).  Probably for this reason, the early literature on
homomorphisms of relations is extremely meagre \cite{Most,c:101} and
most work on relations concerns the study of relation algebras,
various relational operators and their axiomatizations.  
Although in recent years several authors begun studying relational structures 
and their homomorphisms in various contexts, 
a general treatement of relational homomorphisms is still missing.
This growing interest is reflected in numerous
suggestions on how the definition of relational homomorphism could be
specialized  to obtain a more useful notion.  This issue is
our main objective.

In a more concise, relational notation, \refp{eq:weak} is written as
$R^A;\phi\subseteq \phi;R^B$. This somehow presupposes that $R$ is a
binary relation (of course, a homomorphism is such a relation,
too) since composition has a standard definition only for
binary relations.  There seems to be no generally accepted definition
of composition of relations of arbitrary arities. In the following,
we will compose arbitrary relations (within
the structures), like $R$ above, with binary relations (obtained from
homomorphisms between the structures).  We choose to define the
composition of relations $R^A\subseteq \under A^{n+1}$, resp.
$R^B\subseteq \under B^{n+1}$ with a binary relation $\phi\subseteq
\under A\times\under B$ as the relations on $\under A^n\times\under
B$, as follows: %\vspace*{-.5ex}
%\begin{eqnarray}
\eq{\begin{array}{rcl}
\<a_1...a_n,b\>\in R^A;\phi &\Iff&
\exists a\in\under A: 
\<a_1...a_n,a\>\in R^A\ \land\ \<a,b\>\in\phi
\\ 
\<a_1...a_n,b\>\in \phi;R^B &\Iff& 
\exists b_1...b_n\in\under B:
\<b_1...b_n,b\>\in R^B\ \land\ \<a_i,b_i\>\in\phi 
% \nonumber \end{eqnarray}
\end{array}
\label{eq:com}%\vspace*{-.5ex}
}
%
This definition is certainly not the only possible one.\footnote{It
can be seen as standard composition of binary
relations if we view a tuple $\<a_1...a_n,x\>$ as a pair $\<\ovr
a,x\>$ and let $\<\ovr a,\ovr b\>\in\phi \Iff \<a_i,b_i\>\in\phi$ for
all $1\leq i\leq n$.}  The reason for this choice (which, hopefully,
will become convincing later) is our intension to treat relations in
an algebraic way. 
It allows us to view relations as set-valued functions and turns
relational structures into algebraic ones
({\em algebras of complexes} from \cite{JT1,JT2}). In particular, it admits
composition of relations of arbitrary
arities (analogous to composition of functions) %. (For the moment let
%it suffice that it is associative, i.e.\
% $(\psi;R);\phi=\psi;(R;\phi)$, $(R;\psi);\phi=R;(\psi;\phi)$ 
with binary relations as a special case.

Now, table~\ref{de:relhomsde} presents
a sample of proposed definitions of relational homomorphisms gathered from 
\cite{Most,Gratzer,Cohn,Los:relhoms,Pattison,relhoms,c:93,c:94}. It uses binary
relations but with the above definition (\ref{eq:com}) it may be used for relations
$R$ of arbitrary arity (notation is explained at the end of this Introduction).
The names are taken from the articles introducing the respective
definitions and they themselves should suffice to ilustrate the
existing confusion.  \vspace*{-6ex}

{\small{
\begin{table}%[hbt]
\[\hspace*{-1em}\begin{array}{rr@{\ \ }|rcl@{\ \ }|@{\ \ }rcl}
\multicolumn{2}{r@{\ \ }|}{{\rm homomorphism}\ \phi} & \multicolumn{3}{c}{{\rm relational\ def.}} 
     & \multicolumn{3}{c}{{\rm logical\ def.}\ \forall x,y:} 
  \\ \hline
1.& {\rm weak} & \Comp{\phi^-}{\Comp {R^A}\phi} & \subseteq & R^B
            & R^A(x,y) & \Rightarrow & R^B(\phi(x),\phi(y)) \\
2. & {\rm loose} & \Comp{R^A}\phi & \subseteq & \Comp\phi{R^B} & & 1. & \\
3.& {\rm full} & \Comp{\phi^-}{\Comp {R^A}\phi} & = &
            \Comp{\Comp{\Comp{\phi^-}\phi}{R^B}}{\Comp{\phi^-}\phi}
           & \exists x',y':R^A(x',y') & \Iff & R^B(\phi(x),\phi(y)) \\
4.& {\rm `strong'} & \Comp{\phi^-}{\Comp {R^A}\phi} & \supseteq &
            \Comp{\Comp{\Comp{\phi^-}\phi}{R^B}}{\Comp{\phi^-}\phi} 
           & \exists x',y':R^A(x',y') & \Leftarrow & R^B(\phi(x),\phi(y)) \\
5.& {\rm outdegree} & \Comp{R^A}{\phi} &= &\Comp{\Comp{\phi}{R^B}}{\Comp{\phi^-}\phi}
           & \exists x':R^A(x',y) & \Iff & R^B(\phi(x),\phi(y)) \\
6.& {\rm indegree} & \Comp{\phi^-}{R^A} & = & \Comp{\Comp{\Comp{\phi^-}\phi}{R^B}}{\phi^-}
             & \exists y':R^A(x,y') & \Iff & R^B(\phi(x),\phi(y)) \\
7.& {\rm `very\ strong'} & \Comp{\Comp\phi{\phi^-}}{\Comp{R^A}{\phi}} & \supseteq &
            \Comp{\phi}{R^B} 
           & \exists x',y':R^A(x',y') & \Leftarrow & R^B(\phi(x),y) \\
8.& {\rm regular} & 5. & \& & 6. & 5. & \& & 6. \\
9.& {\rm closed} & \Comp{R^A}\phi&  \supseteq & \Comp\phi{R^B} 
           & \exists y':R^A(x,y') & \Leftarrow & R^B(\phi(x),y) \\
10.& {\rm strong} & R^A&  = & \Comp\phi{\Comp{R^B}{\phi^-}} 
           &  R^A(x,y) & \Iff & R^B(\phi(x),\phi(y)) \\
11.& {\rm tight} & \Comp{R^A}\phi & = & \Comp\phi{R^B} 
            & 2. & \& & 9. \\
\multicolumn{8}{c}{} \\[-.5ex]
\multicolumn{8}{l}{{\rm -\ primed\ symbol}\ z'\ {\rm denotes\ some\ element\ such\ that\ } \phi(z')=\phi(z)} 
%\\ \multicolumn{8}{l}{{\rm -\ for}\ \phi\subseteq\under A\times\under B,\ \phi^-\ 
% {\rm denotes\ the\ inverse\ } \<b,a\>\in\phi^-\Iff \<a,b\>\in\phi} 
\vspace*{-3ex}
\end{array}
\]
\caption{Some definitions of relational homomorphisms} 
\label{de:relhomsde}
\end{table} \vspace*{-5ex}
}\normalsize}

\noindent
This paper is an attempt to bring some order into this
chaos. 
% We certainly do not want to suggest that we are going to
%settle the issue. But i
Given the combinatorial possibilities of defining homomorphisms of relational
structures, a complete classification seems hardly possible. 
The very issue of the ``criteria of usefulness'' of various definitions, depending 
on the intended applications, may be debatable.
Nevertheless, 
we hope that approaching the problem from a more algebraic perspective
may bring at least some clarification.
Instead of listing and defending new definitions we have chosen compositionality
and the elementary properties of the resulting categories as the basis for comparison.
We believe these to be important properties and our results should be useful at least to
those who share this belief.
 
 Section~\ref{se:comp}
addresses the question of composition of homomorphisms of relations. 
We show that there are exactly 9 such
homomorphisms which are closed under composition and
give their characterization  -- in fact, most of
the suggested defintions, like most of those in
table~\ref{de:relhomsde}, do {\em not} enjoy this property which we
believe is crucial. The section ends with some remarks on the weaker notions
of congruence related to these compositional homomorphisms.
%
In section~\ref{se:mult} we introduce {\em
multialgebras} which are relational structures with composition of
relations of arbitrary arities defined in the way (\ref{eq:com}) 
reflecting the traditional algebraic way of
composing functions.  
We study epi-mono factorisation, (finite) completeness and co-completeness of
%in detail 
three categories of such structures. 
%with, respectively, weak, tight and closed homomorphisms.
 Two of them
turn out to have extremely poor structure indicating that the respective
notions of homomorphisms may be less usefull than the remaining one.
Then, in section~\ref{se:horel} we study the analogous
categories with homomorphisms being themselves relations. Such notions have
been occassionally introduced in the literature but our results here  do not 
look promising for their further use.
Section~\ref{se:pa} contains some remarks on
the consequences of the results from section \ref{se:mult} for partial
algebras which are special cases of multialgebras.

Most proofs are rather straightforward, involving
basic facts from relational calculus and category theory.
% In section~\ref{se:comp} 
%they involve applications of simple facts from relational calculus. 
The paper presents a comprehensive view of several results and 
space limitations do not allow us to include even some counter-examples, not to
mention the full proofs.
%We include only few and indicate the basic ideas of some others. 
The details can be found in \cite{BW97}. \vspace*{-4ex}

\subsubsection*{Notation.}
In addition to the standard notions of algebraic signature, structure, etc., 
we will use their analogues for relational structures. A {\em relational signature\/}
 is a pair $\<\Sorts,\Rels \>$ where $\Sorts$ is a set (of sort symbols)
and $\Rels$ is a set of relation symbols with given arities (also called
{\em type} of relation).
% An {\em algebraic signature\/} is a pair $\<\Sorts,\Funcs \>$ where 
% $\Funcs$ is a set of function symbols with associated arities.
A {\em relational $\Sigma$-structure\/} 
is a pair $A= \<|A|, \Rels^A\>$, where $|A|$ is
an $\Sorts$-sorted set called a {\em carrier\/} and $\Rels^A$ is a set of
relations, such that for each $[R_i : s_1\times...\times s_n]\in \Rels$
there is $R_i^A \subseteq |A|_{s_1}\times...\times |A|_{s_n}$.~\footnote{To simplify 
the notation, we limit ourselves to single sorted structures
claiming the results carry over to the multi-sorted case.}
%
% An {\em algebra\/} over a signature $\Sigma=\<\Sorts,\Funcs\>$ 
% is again a pair $A= \<|A|, \Funcs^A\>$, where $|A|$ is
% an $\Sorts$-sorted set carrier, and $\Funcs^A$ is a set of
% functions, with a function $f^A : |A|_{s_1}\times...\times |A|_{s_n} \into
% |A|_s$ for each $[f : s_1\times...\times s_n\into s]\in 
% \Funcs$.

We will study extensively algebras whose carriers are
power sets. 
For such an algebra $A$ with $|A|=\PSet(\under A)$, the
set $\under A$ is called the {\em underlying set}. 
% Given a (sub)set $X\subseteq\PSet(\under A)$, 
% $\under X$ denotes the set $\bigcup X\subseteq\under A$. 
Given a function $f:\under A\into B$, we will use its additive pointwise extension
without making it explicit in the notation -- for any $X\subseteq \under A$, 
$f(X)$ means $\bigcup_{x\in X} f(x)$.
We do not make explicit the distinction
between elements and 1-element sets: if $|A|=\PSet(\under
A)$ and $a\in\under A$, we write $a\in A$ meaning $\{a\}\in A$.

Readers should be wary about the notion of a homomorphism.
% between structures and algebras. 
For relational structures,
it is a mapping between carriers with some additional
constraints. It is just the same for ordinary
algebras. Confusion may arise in connection with 
power set algebras, where various types of mappings are possible,
including the ones between underlying sets or carriers (power
sets). We will usually discuss them separately but they do come
together on some occasions. 

The source of a homomorphism $\psi :
A \into B$ between two structures will be the underlying set $\under A$ (rather than the
carrier $|A|$), and its {\em kernel\/}  a relation
$\congo \subseteq \under A\times\under A$ such that
$a\congo b$ iff $\psi(a)=\psi(b)$. The equivalence classes under
an equivalence $\sim$ are denoted $[x]$.
% _\sim$, or just $[x]$ if $\sim$ is clear from the context. 
% If $X$ is a set, we may write $[X]$ for the set $\{[x]:x\in X\}$. 
Composition is written in diagrammatic order
as $f;g$ for $g(f(\_))$. For a binary relation/function $\phi$, $\phi^-$
 denotes its inverse $=\{\<y,x\>:\<x,y\>\in\phi\}$


%=====================================================================

\see{Homomorphisms of relational structures}
\label{se:comp}

% Aiming at an algebraic treatement of relations we will later
% (sections~\ref{se:cats}, \ref{se:horel}) study some categories where
% objects are relational structures.  

Confronted with a tremendous number of possible definitions of a
homomorphism, we believe that the property of being compositional may
serve as an important feature distinguishing the more ``relevant''
ones. Theorem~\ref{th:comp}, which is the main result of this section, 
gives an exhaustive characterization of
compositional definitions.  
We give only one counter-example showing non-compositionality of full
homomorphisms
%begin by giving counter-examples for compositionality of two of the homomorphisms 
from table~\ref{de:relhomsde} which
%
%\paragraph{Full homomorphisms}
were considered in \cite{Most,Los:relhoms} as {\em
the} homomorphisms between relations. In a more special form, they also appear 
in the study of partial algebras~\cite{Burmeister}. 
% In the context of partial algebras,
%  it is known that these homomorphisms do not compose. But it is not clear if all the
% authors were aware of this fact in the general setting. 

%In both, $\phi,\psi$ are full while $\Comp\phi\psi$ is not:

\begin{Example}\label{ex:noncomp}
Let $A,B,C$ be structures with one relation $R$.
% -- a) presents a 
%many-sorted counter-example and b) the single-sorted case.

\[ \maly{1.3}{1}
\diagram
{\rm a)} & A & \rrto^{\phi} &&   & B & \rrto^{\psi} && & C &    \\
%%  &&&&& {\rm b)} &  A\rrto^{\phi} & & B\rrto^{\psi} & &  C \\
\save\go[drr]\merge\framed<5pt>\restore a & b & c &&    
\save\go[drr]\merge\framed<5pt>\restore a & b & c &&    
\save\go[drr]\merge\framed<5pt>\restore a & b & c   \\
%%  &&&&&& \save\go[dd]\merge\framed<5pt>\restore 0 \xline[rr]^\phi|\tip & & 
%%      \save\go[dd]\merge\framed<5pt>\restore 0 \dto \xline[rr]|\tip & &
%%      \save\go[dd]\merge\framed<5pt>\restore 0 \todr \\
& 0\xto[lu] \xline`d[dr][drr] & 1 \xto[lu] \xto`d[dr]`[rrr]|{\ \phi\ }[rrr]
    && &  0\xto[u]\xto[ul] \xline[dr] & 1\xto[u] \xline[d] 
    &&& 0\xto[u]\xto[ur]\xto[ul] & 1  \\
%%  &&&&&&  1 \xline[rrd]|\tip & & 1 \xline[urr]^\psi|\tip & & 1 \\
&&&&&& \psi \xto`r[rrru][rrru]&& & & %\\
%%  &&&&&&  2 \xline[rr]^\phi|\tip & & 2 \xline[rr]^\psi|\tip & & 2
\enddiagram
\]
%
Both $\phi$ and $\psi$ are full but %due to non-surjectivity of $\phi$, 
$\Comp\phi\psi$ is not.
Although $\Comp\phi\psi(0)=0$ and $\<0,c\>\in R^C$, %resp. $\<0,0\>\in R^C$ 
there is no $x\in A$ in the pre-image of $c$, %resp. $0$ 
such that $\<0,x\>\in R^A$.
\end{Example}
%
%\paragraph{In- and outdegree}
% homomorphisms \cite{Pattison,relhoms} are not as common as the full
% ones. But they are used while authors seem to be unaware of the fact
% that they do not compose either:
% 
% \begin{Example}
% \label{ex:degree} 
% This is essentially the same example as~\ref{ex:noncomp}.b).
% \[ 
% \maly{0}{0.5}
% \diagram
%  A \rto^\phi & B  \rto^\psi & C \\
%  x \rto^\phi & \save\go[d]\merge\framed<5pt>\restore y\dto|<{\rotate\tip} \rto & z \todr \\
%  & y' \urto^\psi
% \enddiagram
% \]
% Both $\phi$ and $\psi$ are obviously outdegree (and indegree, hence
% also regular).
% % We have $\equiv_\phi\ = id_{R_{1}}$ while $\equiv_\psi\ =\{(a,a'),(b,b')\}$.
% However, their composition is not, for while $\Comp{R^A}\rho = \es$,
% where $\rho=\Comp\phi\psi$, we have
% $\Comp{\Comp\rho{R^C}}{\Comp{\rho^-}\rho}=\{\<x,z\>\}$.
% \end{Example}
% 
% \noindent
% These counter-examples are used in the proof of 
% theorem~\ref{th:comp} since many cases involved there reduce to one of these two.
% 
%----------------------------------------------------------------------------
%  \subsec{Compositional homomorphisms}\label{sub:comp}
%
We assume a fixed relational signature, with $R$ ranging over all relation symbols, 
and consider definitions  of homomorphisms $\phi:A\into B$ of the form
\eq{
\Delta[\phi]\ \ \Iff\ \ \ \ l_1[\phi];R^A;r_1[\phi]\ \ \any\ \ l_2[\phi];R^B;r_2[\phi] 
\label{de:compde}
}
where $l[\_]$'s and $r[\_]$'s are relational expressions
(using only composition and inverse), %, parameterized by $\_$), 
and $\any$ is one of the set-relations
$\{=,\subseteq,\supseteq\}$. A definition is {\em compositional} iff
for all $\phi:A\into B$, $\psi:B\into C$, we have 
$\Delta[\phi]\ \&\ \Delta[\psi]\impl \Delta[\phi;\psi]$, i.e.:
\begin{eqnarray}
l_1[\phi];R^A;r_1[\phi] & \any & l_2[\phi];R^B;r_2[\phi]\ \ \& \nonumber \\
l_1[\psi];R^B;r_1[\psi] & \any & l_2[\psi];R^C;r_2[\psi] \label{de:ie}\\
  & \impl & l_1[\phi;\psi];R^A;r_1[\phi;\psi]\ \ \any\ \
    l_2[\phi;\psi];R^C;r_2[\phi;\psi] \nonumber
\end{eqnarray}

%% \michal{Leave out complementation for the moment}

\begin{Theorem}\label{th:comp}
A definition is compositional iff it is equivalent to one of the
following forms (where $\any\ \in\{=,\subseteq,\supseteq\}$ and
$\almost\in\{=,\supseteq\}$): 
\eq{
\begin{array}{l@{)\ \ \ }rcl@{\ \ \ \ \ \ \ \ \ \ \ \ \ }l@{)\ \ \ }rcl}
%
1 & R^A;\phi & \any & \phi;R^B    & 
2 & \phi^-;R^A;\phi & \almost & R^B  \\ 
3 &\phi^-;R^A & \almost & R^B;\phi^-  & 
4 & R^A & \almost & \phi;R^B;\phi^- 
%
% 5 & A;\phi &\subseteq &\phi;B;\phi^-;\phi & 6 &\phi^-;A & \subseteq 
%   & \phi^-;\phi;B;\phi\\
% 7 & \phi^-;A;\phi & \subseteq & B;\phi^-;\phi & 
% 8 & \phi^-;A;\phi & \subseteq & \phi^-;\phi;B     \\
% 9 & \phi^-;A;\phi & \subseteq & \phi^-;\phi;B;\phi^-;\phi  
%
\end{array}
\label{eq:conj}
} 
\end{Theorem}

\noindent
The proof of the ``if''part is an easy check that 1)--4) satisfy 
(\ref{de:ie}). %yield compositional definitions. 
In fact, this %the ``if'' part
holds for {\em any transitive} set-relation $\any$.
%, but for $\subseteq$  cases 2)--4) 
% reduce to case 1) -- see Fact~\ref{fa:allincl}.
E.g., for 3) we verify:
\[ 
\begin{array}{rclcrcl} 
\phi^-;R^A &\any & R^B;\phi^- & 
         \& &  \psi^-;R^B & \any & R^C;\psi^- \\
\impl\ \ \psi^-;\phi^-;R^A & \any & \psi^-;R^B;\phi^- & 
         \& & \psi^-;R^B;\phi^- &\any &R^C;\psi^-;\phi^-   \\%[.5ex]
& & & \impl & (\phi;\psi)^-;R^A & \any & R^C;(\phi;\psi)^-
\end{array}
\]
The ``only if'' part is more tedious but amounts to a simple
induction on the complexity of the
expressions $l_i[\_]$ and $r_i[\_]$ using
% keeping in mind that both sides
%must have the same type. 
% We show that any expression either reduces to one of the forms
% (\ref{eq:conj}), or else yields a non-compositional
% definition.
the following facts. % are utilized heavily in this proof.
Since homomorphisms are functions we have
\eq{
%\begin{array}{crcl}
a)\ \ \phi^-;\phi;\phi^-  =  \phi^-\ \ \ \ \ \ \ 
b)\ \ \phi;\phi^-;\phi  =  \phi \ \ \ \ \ \ \ 
c)\ \ \phi^-;\phi  =  id_{\phi[\under A]} 
%\end{array}
\label{eq:facts}
} 
%
% \begin{Fact}
% \label{fa:facts}
% Let $\any\ \in\{=,\subseteq,\supseteq\}$, and $\phi:A\into B$. Then
% \[
% \begin{array}{r@{)\ \ \ }r@{\ \ \any\ \ }l@{\ \Iff\ }r@{\ \ \any\ \ }l}
% 1 & R^A;\phi;\phi^- & \phi;R^B;\phi^- & R^A;\phi & \phi;R^B;\phi^-;\phi \\
% 2 & \phi^-;R^A;\phi & R^B;\phi^-;\phi &\phi^-;R^A;\phi;\phi^- & R^B;\phi^- \\
% 3 & \phi^-;R^A;\phi & \phi^-;\phi;R^B  & \phi;\phi^-;R^A;\phi &\phi;R^B \\
% 4 & \phi^-;R^A;\phi;\phi^-  & \phi^-;\phi;R^B;\phi^- &
%         \phi^-;R^A;\phi & \phi^-;\phi;R^B;\phi^-;\phi \\
% 5 & \phi;\phi^-;R^A & \phi;R^B;\phi^- & 
%         \phi^-;R^A & \phi^-;\phi;R^B;\phi^- \\
% 6 &\phi;\phi^-;R^A;\phi & \phi;R^B & 
%         \phi^-;R^A;\phi & \phi^-;\phi;R^B \\
% 7 &\phi;\phi^-;R^A;\phi & \phi;R^B;\phi^-;\phi & 
%         \phi^-;R^A;\phi & \phi^-;\phi;R^B;\phi^-;\phi \\
% 8 &\phi;\phi^-;R^A;\phi;\phi^- & \phi;R^B;\phi^- &
%         \phi^-;R^A;\phi;\phi^- & \phi^-;\phi;R^B;\phi^-
% \end{array}
% \]
% \end{Fact}
%
% \begin{Proof} 
% All statements are proved essentially in the same way using equalities
% a) and b) from (\ref{eq:facts}). We illustrate it on the first case
% only.
% 
% 1) \ $R^A;\phi;\phi^-\any \phi;R^B;\phi^- \ \ 
%    \impl \ \ R^A;\phi;\phi^-;\phi \any \phi;R^B;\phi^-;\phi \ \ 
%    \Iff \ \ R^A;\phi \any \phi;R^B;\phi^-;\phi$,\ \ 
%    while from the right side we get \ 
%    $\impl \ R^A;\phi;\phi^- \any \phi;R^B;\phi^-;\phi;\phi^- \ \ 
%    \Iff \ \ R^A;\phi;\phi^-\any \phi;R^B;\phi^-$.
%% 
%% 2) $\phi^-;A;\phi \any B;\phi^-;\phi\impl 
%%   \phi^-;A;\phi;\phi^-\any \phi^-;B;\phi^-;\phi;\phi^- \Iff
%% \phi^-;A;\phi;\phi^- \any B;\phi^-$, and vice versa
%% $\phi^-;A;\phi;\phi^- \any B;\phi^- \impl \phi^-;A;\phi;\phi^-;\phi \any B;\phi^-;\phi 
%% \Iff \phi^-;A;\phi \any B;\phi^-;\phi$.
%% 
%% 3) Entirely analogous to 2).
% \end{Proof}

\noindent
One of the three possibilities for
case 1) in the Theorem~\ref{th:comp} is
$R^A;\phi\subseteq\phi;R^B$.  In fact, this %This is a very powerful
case subsumes (or better, is equivalent to) {\em all} other
definitions using $\subseteq$, also those not conforming to the
compositional formats from the theorem! 
%
\begin{Prop}
\label{fa:allincl}
For any definition of the form
\(\Gamma[\phi]\ \Iff\ \La;R^A;\Ra\ \subseteq\ \Lb;R^B;\Rb,\ \)
%where $l_i[\_]$, $r_i[\_]$ are arbitrary relational expressions (over 
%composition and inverse)
%yielding relations of the same type on both sides of $\subseteq$.
we have $\Gamma[\phi] %\Iff \Phi[\phi]$, where $\Phi[\phi] 
\Iff R^A;\phi\subseteq\phi;R^B$.
\end{Prop}
%
%% \begin{Proof}
%% In addition to (\ref{eq:facts}), we use the following facts:
%% \eq{ 
%% R^A\subseteq R^A;\phi;\phi^-\ \ \ \ \ \ \ \ 
%% R^A\subseteq \phi;\phi^-;R^A\ \ \ \ \ \ \ \ 
%% \phi^-;\phi;R^B\subseteq R^B\ \ \ \ \ \ \ \ 
%% R^B;\phi^-;\phi\subseteq R^B
%% \label{fa:incl}
%% } 
%% We verify $\Gamma[\phi]\Iff \Phi[\phi]$ for 
%% all the possible cases for $l_i[\_]$ and $r_i[\_]$ in $\Gamma[\_]$. 
%% For instance, the
%% equivalence with 2) from the theorem and one other case 
%% is shown as follows: 
%% %\hspace*{-1em}\parbox{15cm}{
%% \begin{enumerate}\MyLPar \setcounter{enumi}{1}
%% \item 
%% $\Gamma[\phi]\ \Iff\ \phi^-;R^A;\phi\subseteq R^B:$ \\
%% $\Phi \impl\ \phi^-;R^A;\phi\subseteq\phi^-;\phi;R^B\subseteq R^B$ \\
%% $\Gamma \impl\ \phi;\phi^-;R^A;\phi;\phi^-;\phi \subseteq
%%     \phi;R^B;\phi^-;\phi$, \ 
%%     i.e.\ $R^A;\phi\subseteq \phi;\phi^-;R^A;\phi \subseteq 
%%     \phi;R^B;\phi^-;\phi \subseteq \phi;R^B$
%% \item[$\bullet$)] 
%% $\Gamma[\phi]\ \Iff\ \phi^-;R^A;\phi\subseteq R^B;\phi^-;\phi:$ \\
%% $\Phi \impl \phi^-;R^A;\phi\subseteq \phi^-;\phi;R^B 
%%     \impl \phi^-;R^A;\phi;\phi^-;\phi\subseteq\phi^-;\phi;R^B;\phi^-;\phi
%%     \impl \phi^-;R^A;\phi\subseteq R^B;\phi^-;\phi$ \\
%% $\Gamma \impl\ \phi;\phi^-;R^A;\phi\subseteq\phi;R^B;\phi^-;\phi$\ 
%%     from which (\ref{fa:incl}) yields $\Phi[\phi]$
%% \end{enumerate} 
%% \vspace*{-4ex}
%% \end{Proof}

\noindent
Thus, there are
9 basic compositional definitions (more can be obtained by
their conjunctions). 
Inspecting the table~\ref{de:relhomsde}, we 
see that 1.\ and 2.\ define the same notion, and the only other
compositional definitions are 9., 10. and 11.

Although we have used a particular definition of
relational composition (\ref{eq:com}), all counter-examples involved in 
the proof (like \ref{ex:noncomp})
use only binary relations. Thus, even if composition of relations were defined
differently, as long as it subsumes the composition of binary
relations, the theorem gives the maximal number of compositional
definitions of homomorphisms.

On the other hand, one might probably come up with other forms of
homomorphism definition  that are not covered by (\ref{de:compde}), e.g., allowing
complementation. However, all
% it seems that the majority (if not all) of 
commonly used forms do
conform to this format. Some authors consider certain
modifications of the definitions from table~\ref{de:relhomsde},
for example, requiring surjectivity. With this restriction, full, outdegree
and indegree homomorphisms (3,5,6) %with the extra surjectivity requirement 
do compose.  But this is just a round-about way of
enforcing the equality $\phi^-;\phi=id_B$ (instead of the more limited
case c) of (\ref{eq:facts})), and leads, respectively, to special cases of 2),
1) and 3) from the Theorem~\ref{th:comp}.


%------------------------------------------------------------------------

\subsec{Congruences on relational structures}
\label{sub:congrel}

Congruences of relational and power structures %(see next section) 
were studied %investigated to some extent
in \cite{relhoms, brink93, brink97}. The latter works study primarily %the issue of 
lifting of various properties of a structure to its power structure
without focusing on the notion of homomorphism.

As observed
before, any equivalence gives rise to a (weak) homomorphism.
However, the more specific definitions from theorem~\ref{th:comp} may
lead to more specific relations. We consider first equational
definitions from the theorem, i.e.: 
\eq{
\begin{array}{l@{)\ \ \ }rcl@{\ \ \ \ \ \ \ \ \ \ \ \ \ }l@{)\ \ \ }rcl}
1 & R^A;\phi  & = & \phi;R^B    &   2 & \phi^-;R^A;\phi & = & R^B  \\  
3 &\phi^-;R^A & = & R^B;\phi^-  &   4 & R^A & = & \phi;R^B;\phi^- 
\end{array}
%1)\ R^A;\phi  =  \phi;R^B \ \ \  2)\ \phi^-;R^A;\phi  =  R^B  \ \ \   
%3)\ \phi^-;R^A  =  R^B;\phi^-\ \ \    4)\ R^A  =  \phi;R^B;\phi^- 
\label{eq:eq}
} 
and characterize these kernels which turn out to be not merely
equivalences but congruences of a sort. First, we define relational quotient structures.
% To proceed further we need a
% notion of a quotient of a relational structure:

\begin{Definition}
\label{de:quo}
Given a relational structure $A=\<\under A,R_1^A,R_2^A...\>$ and an equivalence
 $\congo\ \subseteq \under A\times\under A$, a
quotient structure $A/_{\congo} = Q$ is defined by
$\under Q = \{[a] : a\in\under A\}$ and $R_i^Q = \phi^-;R_i^A;\phi$,
where $\phi:\under A\into \under Q$ maps $a\mapsto [a]$.
\end{Definition}
%
% Thus $R^Q(\clis a{}[]) \Iff \exists\lis{a'}: a'_i\sim a_i\land R^A(\lis {a'})$;
% $\phi(x)=\phi(y)$ iff $x\congo y$ and $\congo\ = \phi;\phi^-$.
%
\begin{Prop}
\label{pro:cong}\label{le:canonichom}
Let $\congo$ be an equivalence on $A$ and $Q,\phi$ be
as in definition~\ref{de:quo}.
\[
\begin{array}{r@{\ \ \ \ }rcl@{\ \ }|@{\ \ }rcl}
& \multicolumn{3}{l}{{\rm if\ }\congo\ {\rm satisfies}} & \multicolumn{2}{c}{{\rm then}} \\ \hline
2) &   & &                               & \phi^-;R^A;\phi &=& R^Q \\
1) & \congo;R^A;\congo & = & R^A;\congo  & R^A;\phi &=& \phi;R^Q \\
3) & \congo;R^A;\congo & = & \congo;R^A  & \phi^-;R^A &=& R^Q;\phi^- \\
4) & \congo;R^A;\congo & = & R^A         & R^A &=& \phi;R^Q;\phi^-
\end{array}
\]
\end{Prop}

\noindent
In 1), 3) and 4) the relation $\congo$ %is not just an arbitrary equivalence but 
has a flavour of a congruence:
\begin{itemize}\MyLPar
\item[1)] called a {\em tight} congruence,
can be stated as: 
% $\forall \lis a,b: (\exists b'\congo b,a'_1\congo
% a_1...a'_n\congo a_n:R^A(\lis{a'},b')) \Iff (\exists b'\congo b: R^A(\lis a,b'))$,
% \\ which gives: \\
$\forall \lis a,b, \lis{a'}\ \exists b'\congo b:$ \\
$ R^A(\lis a,b)\land a_1\congo a'_1...a_n\congo a'_n \impl R^A(\lis{a'},b')$ 
% if $\ovr a\congo \ovr a'$ then 
% $\forall b:[\ R^A(a_i,b) \impl \exists b':b'\congo b\land R^A(a_i',b')\ ]$.
\item[3)]
yields a dual condition: $\forall \lis a, b,b'\ \exists a'_1\congo a_1...a'_n\congo a_n :$\\  
$R^A(\lis a,b) \land b'\congo b \impl  R^A(\lis{a'},b')$;
%
\item[4)] 
is strongest: $\forall \lis a,b,\lis{a'},b':$ \\ $R^A(\lis a,b)\land
 a'_1\congo a_1...a'_n\congo a_n \land b'\congo b\impl R^A(\lis{a'},b')$. 
\end{itemize}
%
%% In this special situation of a homomorphism into a quotient (which is
%% surjective and hence satisfies (\ref{eq:sur})), the four kinds of
%% homomorphisms are ordered by strength: 
%% \eq{ 
%% \raisebox{.8ex}{4}
%% \stackrel{\displaystyle{\nearrow}}\searrow \
%% \stackrel{{\displaystyle{\raisebox{2.5ex}{1}} }}
%% {\displaystyle{\raisebox{-1ex}{3}}} \
%% \stackrel{\displaystyle{\searrow}}\nearrow \raisebox{.8ex}{2}
%% \label{eq:homimpl}
%% }
%
On the other hand, for any (at least weak) homomorphism we have the
converse:

\begin{Prop}\label{pro:congrev} 
Given a homomorphism $\phi:A\into B$, let $\congo$ be the kernel of
$\phi$

\[
\begin{array}{r@{\ \ \ \ }rcl@{\ \ }|@{\ \ }rcl}
& \multicolumn{3}{l}{{\rm if\ }\phi:A\into B{\rm\ satisfies}} & \multicolumn{3}{c}{{\rm then\ }\congo\ 
{\rm is\ an\ equivalence\ and} } \\ \hline
%2) & \phi^-;R^A;\phi &=& R^B    &   & &  \\
1) & R^A;\phi &=& \phi;R^B      & \congo;R^A;\congo & = & R^A;\congo   \\
3) & \phi^-;R^A &=& R^B;\phi^-  & \congo;R^A;\congo & = & \congo;R^A \\
4) & R^A &=& \phi;R^B;\phi^-    & \congo;R^A;\congo & = & R^A    
\end{array}
\]
\end{Prop}

\noindent
There is no line for condition 2) since $\phi^-;R^A;\phi=R^B$
obviously implies that $\congo$ is an equivalence but, in fact, this
follows for any function $\phi$.

This isn't the strongest formulation. % of this fact.
% For $\congo$ being
%an equivalence it suffices, of course, that $\phi$ is a weak homomorphism.  
For instance, 1) implies
$R^A;\phi=\phi;R^B;\phi^-;\phi$ which is sufficient to get the
respective property of $\congo$. In general, since $\congo$ is induced
from the image of $A$ under $\phi$, restricting the
homomorphisms on the $R^B$-side to this image (by
$\phi^-;\phi$) will yield the same properties of $\congo$.

Similar results do not follow for homomorphisms defined by $\supseteq$
in place of $=$ in (\ref{eq:eq}). %, in particular for the closed ones.
We can uniformly replace $=$ by $\supseteq$ in proposition
\ref{pro:congrev}, but then the statements in the right column are trivial for
any mapping $\phi$.  If the target algebra is total then the kernel
may retain the flavor of congruence. However, in general such
homomorphisms induce only an equivalence relation.


%=====================================================================

\see{Multialgebras}\label{se:mult}
Multialgebras can be described as relational structures with
a specific composition
of relations of arbitrary arities. This issue offers several
choices leading to possibly general and complicated solutions (see
e.g.\ \cite{glenn,Topen:93}).  Definition \refp{eq:com} was motivated by
the wish to view relations as set-valued functions where the last,
$n$-th argument of an $n$-ary relation corresponds to an element of
the result set obtained by applying the set-valued function to the first $n-1$
arguments.  This view appears in \cite{c:101}, was elaborated in
\cite{JT1,JT2}, then in \cite{c:102} and re-emerged recently in
the algebraic approaches to nondeterminism
\cite{c:64,c:59,c:Bia1,c:130,survey}. 
It is based on the obvious isomorphism between the set-valued operations
and relations:
% observation that any (set-valued) operation
% $f:A_1\times...\times A_n\into \PSet(A)$ determines a relation
% $R_f\subseteq A_1\times...\times A_n\times A$ and vice versa, via the isomorphism: 
%
\eq{
\label{eq:isomr} 
\begin{array}{r@{\ \ \iso\ \ }l@{\ \ }l}
[A_1\times...\times A_n\into \PSet(A)] & [A_1\times...\times A_n\into
[A\into Bool]] & \iso  \\  
\ [A_1\times...\times A_n\times A\into Bool]  &
\PSet(A_1\times...\times A_n\times A)
\end{array} 
}
%
Thus composition of relations becomes
naturally the composition of the respective set-valued functions given
by additive extension to sets.
% Given an $n$-ary $f$ and $n$ $m_i$-ary $g_i$'s, their composition
% $h=f\<g_1...g_n\>$ is a $(\sum_1^n m_i)$-ary function given 
% by:\footnote{As it is often the case, one draws here an advantage from
% the functional notation which, unlike the relational notation, allows
% one to write composition in a natural way (it was called
% ``superposition'' in \cite{JT1}).}
%
%\eq{
% f(g_1(x^1_1...x^1_{m_{1}})...g_n(x^n_1...x^n_{m_{n}}))\ \ = 
% \bigcup_{x_{i}\in g_{i}(x^{i}_{1}...x^{i}_{m_{i}})}f(x_1...x_n) 
% }
We may now talk about similarity types, or signatures, for
multialgebras in the way entirely analogous to standard algebras. 
Another consequence of this change of perspective is 
that the structure of the derived operators ceases to be simply
a Boolean algebra relatively independent from the actual
multialgebra.\footnote{Cohn~\cite{Cohn}, p.204 sees this as {\em the} reason
for the lacking interest in relational homomorphisms.} 
In fact, they become related to the signature in the
same way as in classical universal algebra.
Some universal algebraic aspects of power structures 
are addressed in \cite{GW84,goldblatt89,brink93, adBrink93, brink97}.

%----------------------------------------------------------------------

%\subsec{Categories of Multialgebras}
%\label{sub:cats}
%
\begin{Definition}\label{de:multi}
Let $\Sigma = \pair \Sorts\Funcs$ be a signature. A $\Sigma$-multialgebra $A$
is given by:
\begin{itemize}
\item 
a carrier $\Card A=\{\Card{A}_s\}_{s\in \Sorts}$, where for each $s\in
\Sorts$, $\Card{A}_s$ = $\PSet(\under A_s)$ of some {\em underlying set}
$\under A_s$, with the obvious embedding $\under A_s\incl \PSet(\under
A_s)$;
\item 
a function $f^A:\under A_{s_1}\times...\times \under A_{s_n}\into
\PSet(\under A_s)$ for each $[f:s_1\times...\times s_n\into s]\in \Funcs$, 
with composition defined through additive extension to sets, i.e.\
 $f^A(X_1,...,X_n) = \setsup_{x_i\in X_i} f^A(x_1,...,x_n)$.
\end{itemize}
\end{Definition}
%
%(Again, in what follows we treat the single-sorted case.)
Although the carrier of a multialgebra is a power set, and hence a Boolean
algebra, this isn't reflected in the signature containing only the 
``declared'' operations. This does not
change the actual structures but has some implications for the
possible homomorphisms (which do not have to preserve the set operations).
This also distinguishes multialgebras from {\em algebras of complexes} 
from \cite{JT1}.


We'll consider categories of multialgebras
with homomorphisms corresponding to the condition 1) of
theorem~\ref{th:comp}. 
Their multialgebraic form is as follows.%of these definitions is as follows.

\begin{Definition}
A homomorphism $\phi:A\into B$ is a mapping $\phi: \under
A\into \under B$ of one of the three modes \vspace*{-1ex}
\[
\begin{array}{r@{\ \ -\ \ {\rm when\ \ }}l} %@{,\ \ {\rm in\ particular\ \ }}l}
{\sl weak} & \phi(f^A(\lis a))\subseteq f^B(\clis a\phi())\\ % & \phi(c^A)\subseteq c^B \\
% -- when $\otimes$ is $\po$,
{\sl closed}& \phi(f^A(\lis a))\supseteq f^B(\clis a\phi())\\ %  & \phi(c^A)\supseteq c^B \\
% -- when $\otimes$ is $\rpo$,
{\sl tight} & \phi(f^A(\lis a))= f^B(\clis a\phi()) \vspace*{-1ex} %  & \phi(c^A)= c^B 
% -- when $\otimes$ is $=$.
\end{array} 
\]
\end{Definition}
%
% We will refer (ambiguously) to the carrier of a multialgebra, $\Card
% A$ as the set $\under A$ or $\PSet(\under A)$. Having the obvious
% embedding of $\under A$ into $\Card A$, we will often treat $\under A$
% as a subset of $\Card A$.  Also, for $X\subseteq\PSet(Y)$, $\under X$
% stands for $\bigcup X\subseteq Y$.  The definitions are given
% generally for the many-sorted case. For the sake of simplicity and
% readability we will ignore sorting and, for the most, give arguments
% for one-sorted case. Their generalization to many-sortedness is
% straightforward and all the results reported here remain valid after
% such a generalization.
%
% Any operation $f:B\into\PSet(A)$ has a unique pointwise extension to
% the operation $\PSet(B) \into\PSet(A)$, and
% composition of operations in a multialgebra is defined by such a
% pointwise extension.

\noindent
Multialgebras are ``partial'' in the sense that operations may return
empty set of values. By the pointwise extension of operations, they
are strict in all arguments.  Notice also that we allow empty
carriers, i.e.\ $\PSet(\es)=\{\emptyset\}$ is a possible carrier of a
multialgebra. Thus we will often refer to three special multialgebras,
namely: ``empty'' with carrier $\PSet(\es)$, ``unit'' with carrier $\PSet(\{\bullet\})$ and
all operations returning $\{\bullet\}$, and ``e-unit'' with carrier $\PSet(\{\bullet\})$ and
all operations returning $\es$.
%
% \begin{center}
% \begin{tabular}{r|r@{\ =\ }l@{\ \ }c}
% algebra     & \multicolumn{2}{c}{with carrier} & and all operations returning \\ \hline
% ``empty''   & $\PSet(\es)$ & $\{\es\} $ & $\es$ \\[.5ex]
% ``unit''    & $\PSet(\{\bullet\})$ & $\{\es,\{\bullet\}\}$ & $\{\bullet\}$  \\[.5ex]
% ``e-unit''  & $\PSet(\{\bullet\})$ & $\{\es,\{\bullet\}\}$ & $\es$
% \end{tabular} 
% \end{center}
%  \noindent

Obviously, a homomorphism $\phi:A\into B$ maps not only the underlying
sets $\phi:\under A\into \under B$ but also the whole carriers
(through its unique extension), i.e.\  $\phi:\PSet(\under
A)\into\PSet(\under B)$.  When the source algebra $A$ is ``more partial'' than
the target $B$ (meaning the operations more often return $\es$), there
will be neither a tight nor a closed homomorphism, but often a weak
one.

An arbitray category of $\Sigma$-multialgebras is denoted $\MAS$, possibly with
a subscript $\__{{\sf W}}$, $\__{{\sf T}}$, $\__{{\sf C}}$ indicating the mode.
% when the mode
%of homomorphisms does not matter. When it does, we use $\MASw, \MASt,
%\MASc$, respectively. 
These three cannot be mixed 
%into one (meaningful) category 
-- e.g.\ a composition of a weak and a closed
homomorphism may yield a mapping which is of none of the three
modes.
%First we register a few facts 
The following fact is common to all three categories.
%
\begin{Prop}
\label{fa:epimono}\label{le:isotight}
Let $\phi$ be a homomorphism in any category $\MAS$
\begin{enumerate}\MyLPar
\item $\phi$ is epi (mono) iff it
is surjective (injective) on the underlying set. 
% \item if $\phi$ is isomorphism then it is tight.
\item $\phi$ is an isomorphism iff it is tight,
surjective and injective.
\end{enumerate}
\end{Prop}

\noindent
%The proof of 1) is actually rather involved.
That the definition of tight homomorphisms is closest to the classical
definition, together with 2), might suggest 
 that $\MASt$ is {\em the} appropriate category to work with. 
However, as we will see, this category is much poorer then $\MASl$.
%----------------------------------------------------------------------

\subsec{Epi-mono factorisation}

We show that all three categories have
epi-mono factorisations and construct factorisation systems for
$\MASl$ and $\MASt$.
% these two categories. $\MASc$ does not enjoy this property.

A {\em quotient\/} of a multialgebra $A$ wrt.\ an {\em equivalence}
$\congo$ (denoted $A/_\sim$) was defined in~\ref{de:quo}: 
it is a multialgebra $Q$ such that $\under Q
= \{ [a] : a \in \under A \}$ and $f^Q(\clis a{}[]) =
\bigcup_{a'_i \in [a_i]} [f^A(\lis{a'})]$ for any function
$f$.\footnote{Observe that, if we perform this construction on a usual
algebra with $\sim$ being an equivalence but not a congruence, we
obtain a well-defined multialgebra.}
Recall also that a kernel of a (weak) homomorphism is an equivalence
and, more importantly, the mapping $\phi: A \into A/_\congo$ given by
$\phi(a) = [a]$ is a weak homomorphism. We also have the following
implication for equivalences $\sim_1\subseteq\sim_2$ which, however, cannot be reversed:
\begin{Prop}
\label{fa:subeq}
%For equivalences %on $A$ with 
If $\sim_1\
\subseteq\ \sim_2\ \subseteq \under A\times\under A$, there is a weak morphism
$\phi:A/_{\sim_{1}}\into A/_{\sim_{2}}$.
\end{Prop}

\noindent
In multialgebraic setting  a tight congruence $\sim$ %\ \subseteq\under A\times\under A$  
is an equivalence such that for all
$f\in\Sigma$ we have $\forall \lis a,\lis b:a_i\sim b_i \impl f^A(\lis
a)\sim f^A(\lis b)$, with $\sim$ on sets defined by the
Egli-Milner extension $:A\sim B\Iff \forall a\in A\ \exists b\in
B:a\sim b\ \land\ \forall b\in B\ \exists a\in A:a\sim b$.  We then
have the familiar facts: if $\sim$ is a tight congruence on $A$ then
$\phi:A\into\quot A{\sim}$ defined by $m\mapsto [m]$ is a tight
homomorphism; and kernel of any tight homomorphism is a tight
congruence.
%
%The following theorem is a counterpart of the  Homomorphism Theorem.
%
\begin{Theorem}
\label{th:epimono} {\bf [of homomorphisms]}
Given  a homomorphism $\phi: A \into B$, its kernel
$\congo$, and $Q = A/_\congo$, define
$\psi_1: A \into Q$ and $\psi_2: Q \into B$ as follows: $\psi_1(a) =
[a]$ and $\psi_2([a]) = \phi(a)$. Then $\psi_1$ is epi, $\psi_2$ is
mono, and
\begin{enumerate}\MyLPar
\item{if $\phi$ is weak then both $\psi_1$ and $\psi_2$ are weak.}
\item{if $\phi$ is tight then both $\psi_1$ and $\psi_2$ are tight;}
\item{if $\phi$ is closed then $\psi_1$ is weak and $\psi_2$ is closed;}
\end{enumerate}
\end{Theorem}
%
To obtain epi-mono factorisation for closed
homomorphisms we need a different notion of quotient: a {\em closed
quotient\/} of a multialgebra $A$ wrt.\ an equivalence $\congo$
(denoted $A/^c_\congo$) is a multialgebra $Q$ such that $\under Q = \{
[a] : a \in \under A \}$ and $f^Q(\clis a{}[]) = \bigcap_{a'_i
\in [a_i]} [f^A(\lis{a'})]$ for any function $f$.

\begin{Theorem}
With the notation from theorem~\ref{th:epimono}, except for $Q$ which
now denotes $A/^c_\congo$: if $\phi$ is closed then
$\psi_1$ is a closed epi and $\psi_2$ is a closed mono.
\end{Theorem}
%
In order to obtain factorisation systems we need precise
characterization of the homomorphisms into a quotient. This presents
some problems since the actual construction can be subsumed by many
different definitions --  mere surjectivity, for example, is not
sufficient. Similar difficulties emerge with partial algebras
which are special cases of multialgebras. Two possible definitions are
given below:

\begin{Definition}
\label{de:jtight}
We say that  $\eta:A\into B$ is
\begin{itemize}\MyLPar
\item
{\em full} iff $\forall f \in \Sigma$,
$\lis a\in\under A : f^B(\clis a\phi()) \cap \phi[A] = 
   \bigcup_{a'_{i}\in [a_{i}]} \phi(f^A(\lis{a'}))$.
\item
{\em fully-tight} iff $\forall f\in\Sigma$,
$\lis b\in\under B : f^B(\lis b) = \bigcup_{a_{i}\in \phi^{-}(b_{i})} \phi(f^A(\lis a))$.
\end{itemize}
\end{Definition}
%
Obviously, tight $\impl$ fully-tight $\impl$ full, while (full and
surjective) $\impl$ fully-tight, but none of these implications can be
reversed.  Full are not compositional and to repair this, one might,
in addition, require surjectivity.  However, fully-tight can be
defined by $\phi^-;R^A_f;\phi=R^B_f$ ($R_f$ is the relation for
$f$), and so by theorem~\ref{th:comp}, are
compositional. %\footnote{A non-compositional variant, used for the same
%purpose as here and defined by $\forall \lis a\in\under A:f^B(\clis
%a\phi()) = \bigcup_{a'_{i}\in [a_{i}]} \phi(f^A(\lis{a'}))$, was
%called ``jointly-tight'' in \cite{c:132}.}  
 We suggest that this is
{\em the} right definition -- being compositional, it also yields the
left factor of a factorisation system in $\MASl$.

\begin{Theorem}
\label{le:FSl}\label{le:FSt}
$\<\epi,\mono\>$ is a factorisation system for 
\begin{itemize}\MyLPar
\item $\MASl$ with $\mono$ all monomorphisms and $\epi$ all full
epimorphisms.
\item
$\MASt$ with  $\mono$ all monomorphisms and $\epi$ all epimorphisms.
\end{itemize}
\end{Theorem}
%
In $\MASc$ homomorphisms into the {\em closed} quotient can be
characterised dually to fully-tight morphisms (with intesection
instead of union) but this class is not closed under composition, and
hence does not lead to a factorisation system.


%------------------------------------------------------------------------

\subsec{Subobjects}
\label{sub:sub}

Defining, as usual, a {\em subobject\/} to be an equivalence class of
monomorphisms, we obtain three different notions of weak, tight and
closed subobjects in the respective categories.  In plain multialgebraic terms:
%we have: 
$B$ is a {\em
tight} ({\em weak, closed\/}) subalgebra of $A$ %(denoted $B\suba A$)
iff $\Card B\subseteq\Card A$ and $\forall f\ \forall \lis b\in\Card B :
f^B(\lis b)=f^A(\lis b)$ (with $\subseteq$ and $\supseteq$, respectively).
%Or else, $B$ is a tight (weak, closed) subalgebra of $A$ iff
%$\under B\subseteq\under A$ and the inclusion $\iota:B\incl A$ is a
%tight (weak, closed) homomorphism.

\begin{Prop}
\label{le:subalgs}
A homomorphism $\phi:B\into A$ induces a subalgebra $[B]$ of $A$,
which is of the same mode as $\phi$.
\end{Prop}

\noindent
The construction is straightforward in $\MASt$ but in the other two cases
%it seems more arbitrary and 
involves taking unions, resp., intersections of
the images of the result sets in order to endow the image with the 
apropriate algebraic structure.

The different modes of homomorphisms lead also to different
constructions of minimal subalgebras or, more generally, of
intersections of subobjects.

\begin{Prop}
\label{le:MASlintersect}
For any $A$, any set $\{\nu_i: B_i \incl A\}$ of $A$'s subobjects in
$\MASl$ (resp. $\MASt$) has an intersection $(B,\nu)=\bigcap
(B_i,\nu_i)$.
\end{Prop}

\noindent
% In particular, for any $\Sigma$, the minimal subobject of any algebra
% $A$ in $\MASl$ will be the empty algebra. For void $\Sigma$, this
% will also be the case in $\MASt$ and $\MASc$.
%
% If $\Sigma$ is void (i.e.\ has no ground terms), the same will be
% the case in $\MASt$. But for non-void $\Sigma$ (with some ground 
% terms of each sort), the minimal subobject of an $A\in\MASt$ will 
% be the tight subalgebra $B$ of $A$ with $\under B$ containing exactly
% the elements which are in the sets interpreting ground
% $\Sigma$-terms in $A$.
%
We do not have a counterpart of this fact for $\MASc$ where subalgebras
can be constructed, so to speak, in two opposite directions.
%In $\MASc$ the concept of a subobject is rather strange.  
On the one hand, empty subalgebra will be a
closed subobject of a given algebra $A$. On the other hand, if we keep
the same carrier $\Card B=\Card A$, and define all operations to
return the full carrier, %e.g., for any $\lis a\in\under B$
%and any $f\in\Sigma$, we let $f^B(\lis a)=\under B$, 
we get a closed subalgebra $B$ of $A$.

% Thus, for a non-void $\Sigma$, a minimal subobject of an $A$ will be
% algebra $B$ with the carrier as in $\MASt$ (all elements contained 
% in the interpretation of the ground terms in $A$) but with 
% the operations returning, for all arguments, the full carrier of 
% the result sort.
%% 
%% \begin{Example}
%% \label{ex:nointersect}
%% Take $\Sigma$ with one constant $c$, algebra $A$ with $\under
%% A=\{1,2\}$ and $c^A=\{1\}$.  The algebras $B_1$ with
%% $\under{B_1}=\{1\}=c^{B_{1}}$ and $B_2$ with
%% $\under{B_2}=\{1,2\}=c^{B_{2}}$ are both (closed) subalgebras of $A$
%% with the obvious injections. However neither is a subobject of
%% another, nor is $A$ a (closed) subobject of any of them. %In fact, t
%% The
%% intersection of $B_1$ and $B_2$ in $\MASc$ will be the empty algebra.
%% \end{Example}
%% 
%% \noindent
%% These two directions of constructing subobject seem somehow in conflict 
%% with each other and make
%% this concept in $\MASc$ rather far from the intended
%% one, even if formally acceptable.

\subsec{Limits and co-limits in $\MAS$.}
\label{sub:limit}

$\MASl$ has all (finite) limits and co-limits. The categories $\MASt$
and $\MASc$ are neither (finitely) complete nor co-complete. Since
these statements involve long lists of specific results, we only
summarize them in table~\ref{tab:A}. \vspace*{-5ex}

\begin{table}%[htb]
\begin{center}
\begin{tabular}{|r|c|c|c|}
\hline 
                   & $\MASl$ & $\MASt$ & $\MASc$ \\ \hline
products           & +  & -- & -- \\
equalizers         & +  & -- & -- \\
{\bf finite completeness} & +  & -- & --  \\ \hline
initial objects   & +  & -- & -- \\
% (of interest)  & (--) & (--) & (--) \\
co-products        & +  & --  &  -- \\
co-equalizers      & +  & +  &  \\
{\bf finite co-completeness} & +  & -- & -- \\ \hline
epi-mono factorisation & +  & +  & + \\ 
factorisation system & + & + &  \\ \hline 
\end{tabular} \vspace*{-2ex}
\end{center}
\caption{Properties of various $\MAS$.}\label{tab:A} \vspace*{-6ex}
\end{table}

\noindent
The initial object in $\MASl$ is not (any form of) term structure but the empty algebra. 
%and for this reason
% we had indicated that it is not ``of interest''. But at least this
% category has initial objects as well as several other desirable basic
% properties -- the two other categories possess almost none of them.
Nevertheless, the category $\MASl$ possesses many desirable
properties not possessed by the other two categories.
It has perhaps been a prevailing opinion among mathematicians
interested in the question that weak homomorphisms of relational
structures provide the most useful notion.  The above results justify
and demonstrate this opinion. We have thus studied all the cases of
one type of compositional homomorphisms -- case 1) from
theorem~\ref{th:comp}. We conjecture that the negative results about
the tight and closed homomorphisms of this type will also occur for
the tight and closed variants of the remaining three types.

%====================================================================


\see{Homomorphisms as Relations}
\label{se:horel}

Since the carrier of a multialgebra is a power set, one might attempt
to define a multialgebra as a usual algebra over such a carrier.  Such
a generality, however, reduces to the considerations of standard
algebras. In order to capture the intended (power-)set structure, some
additional restrictions have to be imposed, the least of which is
monotonicity of the operations within the algebra. (Such a variant,
called power-algebras, was used, for instance in \cite{c:130}, for
modelling a kind of ``call-by-name'' passing of nondeterministic
parameters.)

We will not consider this alternative here and, instead, will look
closer at the alternative notions of homomorphisms between
multialgebras. There are three
immediate possibilities here; a homomorphism $\phi:A\into B$ may be of
three kinds:
$\PP: \under A \into\under B$ -- point-point, 
$\PS: \under A \into \PSet(\under B)$ -- point-set, or
$\SSs:\PSet(\under A) \into \PSet(\under B)$ -- set-set.
%% 
%% \[ \begin{array}{rr@{\ \into\ }ll}
%% \PP: & \under A & \under B & {\rm point-point} \\
%% \PS: & \under A & \PSet(\under B) & {\rm point-set} \\
%% \SSs:& \PSet(\under A) & \PSet(\under B) & {\rm set-set}
%% \end{array}
%% \]
%
Various modes of the first kind have been considered so far.  The last
kind, when taken without additional restrictions, treats multialgebras
%analogously as in power-algebras, 
as standard algebras which only ``by accident'' happen to have power
sets as carriers.  Of course, a \PP-homomorphism induces a \PS\ one by
pointwise extension, and a \PS-homomorphism induces an \SSs\ one by
additive extension.  I.e., modulo the obvious type-conversion, we
have: $\mode\PP\phi \impl \mode\PS\phi \impl \mode\SSs\phi$. All three
kinds can be further combined with the three modes -- tight, weak, or
closed -- leading to a taxonomy %of various possibilities 
which is
simple in structure but messy in interrelations.

\PS-homomorphisms seem a natural counterpart of the set-valued functions within
algebras and have received sporadical attention in the literature. 
For these reasons, we cast a closer look at 
the categories $\MAPS$, highlighting some points
illustrating the lack of equally pleasing categorical properties as
those possessed by the category $\MASl$.  

The only positive result, and
the one not present in $\MAS$, is that initial objects
in $\MAPSt$ resemble the classical term algebras
(theorem~\ref{le:MAPStADJ}). This seems the only reason for which they 
attracted some attention, e.g., in \cite{c:57,c:59}.
 
Before proceeding further, we observe %...\vspace*{-3ex}
%\subsubsection*{... a general fact about isomorphisms}
the following general fact. % about isomorphisms
Claim~\ref{le:isotight} showed that isomorphism of any mode
is actually tight. Although this was stated for the 
\PP-homomorphisms, it does not depend on the kind
of homomorphisms:
\begin{Prop}
\label{le:isopp}
If $\phi:A\into B$ is  an isomorphism then $\SSs(\phi)\impl \PS(\phi)\impl \PP(\phi)$.
\vspace*{-2ex}
\end{Prop}
%% 
%% \noindent
%% Thus isomorphisms in all different categories
%% are actually point-to-point mappings from $\MASt$.



%-------------------------------------------------------------------

\subsec{The category \PSt }

As a technical tool, we introduce the category \PSt. It is a special
case of any of the three $\MAPS$ categories where $\Sigma$ contains
only one sort symbol. Thus the negative results for \PSt\ can be
immediately applied to $\MAPS$.  Characterization of epis and monos,
as well as non-existence of epi-mono factorisations for $\MAPS$ are
obtained from the results in this subsection.

\begin{Definition}
The objects of \PSt\ are power sets, i.e.\ for each object $S$ of
\Set, $\PSet(S)$ is an object of \PSt.  A morphism %in \PSt,
$\under\phi:\PSet(S)\into \PSet(T)$ is a \Set\ morphism
$\under\phi:S\into\PSet(T)$ extended pointwise, i.e.\ for
$X\in\PSet(S) : \under\phi(X) = \bigcup_{x\in X}\under\phi(x)$.
\end{Definition}

\begin{Prop}
\label{le:adjSPS}
Let $I:\PSt\into\Set$ be the inclusion and $\PSet:\Set\into\PSt$ send
each set to its power set $S\mapsto\PSet(S)$, and each morphism to its
pointwise extension to power set. These functors are adjoint
$\PSet\adj I$.
\end{Prop}

\noindent
In particular, the initial object (preserved by $\PSet$) in $\PSt$
will be $\PSet(\es)=\{\emptyset\}$.  It is initial in the same way as $\emptyset$
is initial in \Set.  This is also the terminal object in $\PSt$, 
%n't $\{\bullet\}$ (which isn't an object in $\PSt$) nor $\PSet(\{\bullet\})$ but,
%actually, the same $\{\emptyset\}$. It is terminal 
because
morphisms send elements of the source to subsets of the target, and
hence the unique morphism from any other set will map all 
elements to $\emptyset$.

\begin{Prop}
\label{le:PStcompl}\label{le:PSnocoeq}
\PSt\ is complete but not co-complete.
\end{Prop}

\noindent
Another negative fact of importance is the lack of epi-mono factorisation
shown by a counter-example using the following characterization.
%First we characterize epis and monos in \PSt.

\begin{Lemma}
\label{le:charepimono}
A morphism $\phi:S\into \PSet(T)$ in $\PSt$ is:\\
%\begin{enumerate}\MyLPar
%\item 
mono iff pointwise extension is injective:
$\forall X,Y\subseteq S: X\not=Y \impl \phi(X)\not=\phi(Y)$; \\
%\item 
epi iff it is surjective on $T$, i.e.\ $\forall t\in T \exists s\in S:
\phi(s)=\{t\}$.
%\end{enumerate}
% \\ \marcin{dla tight: if $\phi$ is injective}
\end{Lemma}
%% Using this characterisation, it is easy to construct a morphism for which there
%% is {\em no} factoring epi-mono pair.
\begin{Prop}
\label{le:noepi-mono}
$\PSt$ does not have epi-mono factorisation.
\end{Prop}
%% 
%% \begin{Proof}
%% Let $S=\{a,b,c\}$ and $T=\{1,2,3\}$ and $\phi:S\into\PSet(T)$ be
%% $\phi(a)=\{1,2\},\ \phi(b)=\{1,3\},\ \phi(c)=\{2,3\}$. $\phi$ is
%% neither epi nor mono.  Assume it has an epi-mono factorisation
%% $\phi=\Comp\eta\mu$, with an epi $\eta:S\into\PSet(Q)$ and a mono
%% $\mu:Q\into\PSet(T)$. $Q$ must have at least three elements which
%% $\mu$ sends, respectively, $\mu(x)=\{1,2\},\ \mu(y)=\{1,3\},\
%% \mu(z)=\{2,3\}$ in order to yield $\phi$ in composition with
%% $\eta$. But this means that $\mu$ cannot be mono, since we will have
%% $\mu(\{x,y\})=\mu(\{y,z\})$.
%% \end{Proof}
%% 
%% \noindent
%% Notice that it is not uniqueness that fails here -- there is simply {\em no}
%% epi-mono pair factoring $\phi$. Consequently, we do not have to spend time
%% looking for a factorisation system.



%%%%%%%%%%%%%%%% unnecessary Claim
%%%% \begin{Claim}
%%%% $\PSt$ does not have intersections (for arbitrary class of subobjects).
%%%% \end{Claim}
%%%% \begin{Proof}
%%%% Lemma~\ref{le:noepi-mono} implies that $\PSt$ does not have strong
%%%% epi-mono factorisation.  Since it is (finitely) complete
%%%% (lemma~\ref{le:PStcompl}), it can not have arbitrary intersections.
%%%% \end{Proof}


%----------------------------------------------------------------------
\vspace*{-3ex}
\subsec{The categories $\MAPS$}

Empty algebra is terminal in each $\MAPS$ category.
It is also initial in $\MAPSl$.
For $\MAPSt$, we have a more general result.
%When homomorphisms are tight, the initial algebra begins to be interesting.

\begin{Theorem}
\label{le:MAPStADJ}
Let $U:\MAPSt\into\PSt$ be the forgetful functor given by
$U(A)\Def\PSet(\under A)$.  There is an adjunction $\GTerms\adj U$.
\end{Theorem}
%% 
%% \begin{Proof}
%% For any object $\PSet(X)$, we let $\GTerms:\PSet(X)\mapsto \GTerms(X)$
%% -- the free multialgebra over $X$. Its underlying set $\under
%% {\GTerms(X)}$ comprises all 1-element (term) sets:
%% 
%% \begin{itemize}\MyLPar
%% \item 
%% $X\cup\GTerms\subseteq \under{\GTerms(X)}$ 
%% \item 
%% if $f\in\Sigma$ and $x_1...x_n\in\under{\GTerms(X)}$ then
%% $f(x_1...x_n)\in\under{\GTerms(X)}$,
%% \end{itemize}
%
%% and operations are $f^{\GTerms(X)}(x_1...x_n)\Def \{f(x_1...x_n)\}$.
%% A $\PSt$-morphism $\under\phi:\PSet(X)\into\PSet(Y)$, i.e.\ 
%% $X\into\PSet(Y)$ is mapped obviously on the morphism
%% $\phi:\GTerms(X)\into\GTerms(Y)$ given by $\phi(x)=\under\phi(x)$ for
%% all $x\in X$ and $\phi(f^{\GTerms(X)}(x_1...x_n))=
%% \bigcup_{y_{i}\in\phi(x_{i})}f^{\GTerms(Y)}(y_1...y_n)$.
%% 
%% Let $A$ be a $\Sigma$ multialgebra, and $\under\phi:X\into U(A)$
%% a morphism in $\PSt$. Then we define tight homomorphism $\phi
%% :\GTerms(X)\into A$ by $\phi(x)\Def\under\phi(x)$, for $x\in X$,
%% $\phi(\{t\})\Def t^A$ for $t\in\GTerms$, and
%% $\phi(f^{\GTerms(X)}(x_1...x_n))\Def f^A(\phi(x_1)...\phi(x_n))$.
%% 
%% The unit $\iota:\PSet(X)\into U(\GTerms(X))$ is simple
%% inclusion. Obviously, we have that $\under\phi =
%% \Comp{\iota}{U(\phi)}$. Also, $\phi$ is unique, for if there is
%% another $\psi:\GTerms(X)\into A$, satisfying this last equality, then
%% we must have $\psi(x)=\phi(x)$ for all $x\in X$. Furthermore,
%% $\psi(t)=t^A$ for all $t\in\GTerms$, and
%% $\psi(f^{\GTerms(X)}(x_1...x_n))=f^A(\psi(x_1)...\psi(x_n))$ since
%% $\psi$ is a tight homomorphism. But this means $\psi=\phi$.
%% \end{Proof}
%% \noindent
Unlike for $\MASl$ where an initial object was the empty algebra, here
we obtain essentially standard term algebra $\GTerms(X)$ for each $\PSet(X)$, 
and initial term algebra $\GTerms$.
%quite a natural initial algebra. 
Its initiality depends heavily on the fact that we have \PS\ homomorphisms --
they allow us to send a single element $t\in\GTerms$ to a {\em
set} of elements $t^A$ in the target algebra $A$. In particular, we may have
that %such a $t$ can be sent onto empty set in cases when 
$t^A=\es$.
The construction does not work for $\MAPSl$ or $\MAPSc$.\footnote{Almost
the same construction would give us an adjunction between $\Set$ and
$\MAPSt$. A specific subcategory of $\MAPSt$ (with all
ground terms determinisitc in each algebra) was used in
\cite{eli:nondetaut} for studying nondeterministic automata and
recognizable sets. The initiality result for this subcategory was also
mentioned there.}

Both $U$ and $\GTerms$ are 
faithful (but not full) and so we obtain, as a corollary and using 
lemma~\ref{le:charepimono},
the following characterization of epis and monos.

\begin{Prop}
Let $\phi:M\into N$ be in $\MAPSt$ and $\under\phi=U(\phi)$ be its
image in $\PSt$. Then
% \begin{enumerate}\MyLPar
% \item 
1) $\phi$ is mono iff $\under\phi$ is mono,
% \item 
and 2) $\phi$ is epi if $\under\phi$ is epi. %\vspace*{-1ex}
% \end{enumerate}
\end{Prop}
% (I.e., iff the pointwise
% extension of the mapping $\under\phi:\under M\into
% \PSet(\under N)$ to $\PSet(\under M)$ is injective.)

\noindent
The ``only if'' in 2) does not hold but, to our surprise, 
we were not able to find a necessary condition for epis in $\MAPSt$.

Table~\ref{tab:B} summarizes the properties of the three \PS-categories.\vspace*{-4ex}

\begin{table}%[htb] 
\begin{center}
\begin{tabular}{|r|c|c|c|}
\hline
    & $\MAPSl$ & $\MAPSt$ & $\MAPSc$ \\ \hline
products       &+ & +   & + \\
equalizers     &+ & +   & -- \\
{\bf completeness} &+ & +   & -- \\ \hline
initial objects  &+  & +   & -- \\
co-products    &+ & --  & -- \\
co-equalizers  &-- & --  & -- \\
{\bf co-completeness}&-- & -- & -- \\ \hline
epi-mono factorisation   &-- & --   & -- \\ \hline
\end{tabular}
\caption{Properties of $\MAPS$.}\label{tab:B}
\end{center} \vspace*{-5ex}
\end{table} 


%========================================================================
%========================================================================

\see{Partial Algebras}
\label{se:pa}

We model a partial function $f : X \into Y$ by a %can be modelled by a
multifunction $\mof f$ returning an appropriate singleton set whenever
$f$ is defined, and the empty set
otherwise. 
%% However, there is another possibility for modelling
%% undefinedness by $\mof f$
%% returning every value, i.e.\ the set $Y$ -- this seems to
%% offer more flexibility in treating strictness and error-handling than
%%  partial algebras admit.
%% 
%% We do not attempt to develop the second approach here, but consider the first one
%% and present some relationships with our results.
Given a partial algebra $A$, let $\mof A$ be its
multialgebraic representation with functions given below ($\dom(f^A)$ is
a domain on which $f^A$ is defined):  
\eq{
\begin{array}{rcl}
\lis a\in\dom(f^A) & \Iff & f^{\mof A}(\lis a)\not=\es \\
\lis a\in\dom(f^A) & \Iff & f^{\mof A}(\lis a) = \{f^A(\lis a)\}
\end{array}
\label{mof}}
We list a few relevant definitions concerning partial algebras.
\begin{Definition}\label{de:PAs}
$\phi:A\into B$ %between two partial algebras 
is a {\em weak} homomorphism iff
$\phi(\dom(f^A)) \subseteq \dom(f^B)$ and 
$\forall a_i\in \dom(f^A): \phi(f^A(a_i))= f^B(\phi(a_i))$. 
%A weak homomorphism $\phi$ 
It is also \footnote{The literature 
on partial algebras calls tight
homomorphisms {\em closed} and the full injective ones
{\em  relative-injective}. `Full injective' is equivalent to 
`full-{\em and}-injective'.}: 

%\noindent
\begin{tabular}{r@{\ \ if\ \ }l} %@{\hspace*{-.1em}}
{\em tight} %\footnote[6]{}  
  &  $\Comp{f^A}{\phi}=\Comp{\phi}{f^B}$ \ i.e.\  $\dom(f^A)= \phi^{-}(\dom(f^B))$ \\[.5ex]
{\em full} & $\forall a_i\in A:$ %\\  & & 
$f^B(\phi(a_i))\in \phi[A]
\impl \exists a'_i\in \dom(f^A):\phi(a'_i)=\phi(a_i)$ \\[.5ex]
{\em full injective} & 
 it is injective, and $f^A= \Comp{\Comp{\phi}{f^B}}{\phi^{-}}$
\end{tabular}
\end{Definition}
%% 
%% \addtocounter{footnote}{1}\footnotetext{The literature 
%% on partial algebras calls tight
%% homomorphisms {\em closed} and the full injective ones
%% {\em  relative-injective}. `Full injective' is equivalent to 
%% `full-{\em and}-injective'.}

\noindent
Let $\PAw$ and $\PAt$ be the categories of partial algebras with weak and
tight homomorphisms, respectively. Closed homomorphisms rarely appear
in literature so we do not consider them here.  The functor $\mof
\_:\PAS\into\MAS$ sends any partial algebra to a multialgebra $\mof A$
as in (\ref{mof}); and a morphism $\phi:A\into B$ to the morphism
$\mof\phi:\mof A\into\mof B$ given by $\mof\phi(a)=\phi(a)$. Thus
$\mof\_$ is essentially an inclusion (disregarding the conversion of
elements into 1-element sets). $\mof{\PASw}$ and $\mof{\PASt}$ are
full subcategories of $\MASl$, resp. $\MASt$ with objects being
deterministic multialgebras.  It is hardly surprising that:

\begin{Prop}
\label{fa:faith} 
Functors $\mof\_:\PASw\into\MASl$ and $\mof\_:\PASt\into\MASt$ 
are full and faithful.
\end{Prop}

\noindent
Thus $\mof\_$ reflect limits and co-limits and many properties of the
two categories of partial algebras follow now by revisiting their
multialgebraic proofs and checking whether the constructions, when
applied to the deterministic multialgebras, yield deterministic
objects.  One such consequence is that any set of subobjects of an
algebra in $\PASw$, resp.\ $\PASt$, has an intersection, in particular,
the intersection of a set of tight subalgebras is a tight subalgebra.

%----------------------------------------------------------------------

Yet another consequence is that factorisation systems for $\MASl$, and
$\MASt$, specialise to factorisation systems for $\PASw$ and $\PASt$.
The definition~\ref{de:PAs} of full homomorphisms is a special case of
the relational definition and the multialgebraic one \ref{de:jtight}. 
Thus, there is a one-to-one correspondence:
%between the full homomorphisms which also are surjective and the full(y-tight)
%epis, i.e.\ 
$\phi$ is full-and-surjective iff $\mof\phi$ is full(y-tight) epi.  
Both kinds compose.
%This fact is interesting because full-surjective homomorphisms do
%compose, and so do fully-tight epis.

As argued after definition~\ref{de:jtight}, the simplest
characterisation of compositional homomorphisms into quotients is
$\phi^-;R^A;\phi=R^B$.  It specialises to partial algebras yielding
a notion stronger than full and weaker than full-and-surjective. 
%% : \[
%\eq{ 
%% \forall \lis b,b:\ f^B(\lis b) =b\ \Iff\ \exists \lis a,a:
%% \phi(a)=b \land \phi(a_i)=b_i \land f^A(\lis a)=a\ 
%\label{eq:ftsPA} } 
%% \]
%% Full-and-surjective homomorphisms 
The latter, used as the left factor of
factorisation system, satisfy this condition and can be safely replaced
by such homomorphisms.  Similarly, the constructions of quotients
and associated congruences from $\MASw$ and $\MASt$ specialise to
respective partial algebraic notions.  Nevertheless, $\MAS$ being more
general do not contain all the information relevant in the study of
partial algebras. Various characterizations, numerous factorisation
systems, etc.\ for the latter cannot be obtained from the former.

%% Compositionality and correspondences of various homomorphisms are
%% illustrated in table \ref{ta:homs}.
%% 
%% \begin{table}[hbt]
%% \begin{center}
%% \begin{tabular}{cr|c|lc}
%% $\Comp{\_\ }\_$ & $\phi\in\PASw$\ \ \  & 
%%  & \ \ \ $\mof{\phi}\in\mof{\PASw}$ & $\Comp{\_\ }\_$ \\ \hline
%% + & weak & $\Iff$ & weak & + \\
%% + & tight & $\Iff$  & tight & +\\
%% -- & full & $\Iff$ & full  & -- \\
%% + & (\ref{eq:ftsPA}) & $\Iff$ & fully-tight & + \\
%% + & full-surjective & $\Iff$ & full(y-tight) epi  & +\\
%% +& full-injective & $\Iff$ & full  mono &  + \\
%% & (relative injective) 
%% \end{tabular}
%% \end{center}
%% \caption{Homomorphisms in $\PASw$ and $\mof{\PASw}$.}\label{ta:homs}
%% \end{table}

%=====================================================================

%\see{Some Concluding Observations}

\see{Informal Summary}

According to theorem~\ref{th:comp} there are only 9 compositional
homomorphisms of relations: 1 weak (subset), 4 tight (equality) and 4
closed (superset).  A closer study of three of them (together with the
3 respective \PS-homomorphisms) shows that closed homomorphisms yield
categories with very poor structural properties and one
may expect that this to be the case for the remaining three closed
homomorphisms. Thus, we conjecture that, accepting compositionality and 
the categorical properties
we have studied among the adequate criteria for choice of the homomorphisms, 
 there are at most 5 ``useful''
possibilities. Among them, the 4 tight ones can be expected to have
significant weaknesses, exemplified by the one case, $\MASt$, which has been
given a detailed treatement here.  But even if one still does not want to
work with weak homomorphisms, the results of this paper
% the full characterization of compositional definitions which we have given 
can provide a useful tool
preventing one from looking for new, idiosyncratic notions serving
only very peculiar purposes.

The suggested definition of composition of relations of arbitrary
arities turns relational structures into algebras -- namely,
multialgebras.  These provide a convenient way for algebraic study of
power set structures, relational structures, as well as for modeling
phenomena like partiality and nondeterminism. \\[.5ex]
%
% \begin{LS}
% \item
1) Multialgebras do not give one a grasp on the individual elements --
ground terms may denote sets. This causes trouble in many situations,
in particular, when constructing unique homomorphisms. Two
homomorphisms may reflect the $\Sigma$-algebraic structure in the same
way but still be different.  E.g., algebras with carriers $\under
A=c^A=\{1,2\}=c^B=\under B$, even with tight homomorphisms, will have
``only one'' homomorphism in the sense that $\phi(c^A)=c^B$ but,
actually, there are two $\phi_1 = id$ and $\phi_2(1)=2,\ \phi_2(2)=1$.
\\[.5ex]
%\item
2) Many (counter-)examples are based on empty signatures or signatures
with only one constant.  It seems that some nice results follow only
if $\Sigma$ is non-void (like relevant initial objects in $\MAPSt$),
while others only if it is void.
%\begin{LSA}
%\item
\\[.5ex]
2.1) Weak homomorphisms seem to work best in combining these two poles --
$\MASl$ and $\MAPSl$ had most desirable properties. However, this
happens at the cost that many categorical constructions simply yield
degenerated cases -- empty algebras or algebras where (all) operations
return empty set.
%\item
\\[.5ex]
2.2)
The mere look at the tables~\ref{tab:A}-\ref{tab:B} could suggest that
the weak categories $\MASl$, resp. $\MAPSl$, are most interesting. In
fact, the corresponding tight categories $\MASt$, resp. $\MAPSt$, lack
many of the desirable properties possessed by these weak ones.

However, consulting the actual proofs and constructions we observe
that, in the situations when some construction works in the tight
categories, it often yields a more natural -- less degenerate and
closer to the classical -- result than the corresponding construction
in the weak categories. Nonexistence of some constructions in the
tight categories is often a side-effect of excluding the degenerate
cases admissibile in the weak categories.
%\end{LSA}
\\[.5ex]
%\item
3) There is a certain sense of correspondence between the categories
$\MASl$ and $\MAPSt$.  The weak $\PP$-homomorphisms allow the target
to have ``larger result sets'' than the source. Similar effect is
achieved in $\PS$-homomorphisms in that a single element from the
source can be mapped on a set of elements in the target. It all
depends on what homomorphisms are supposed to be used for, but
assuming that this kind of ``reducing the result sets'' in the source
is intended, there seems to be little point in combining
\PS-homomorphisms (which do that) with the weakness requirement (which
does the same again).

In fact, $\MAPSt$ seems to be the most interesting among the
\PS-categories, in particular, due to the natural homomorphism
condition (involving set equality) and the existence of
interesting initial objects (not present in any other category we have
studied). However, it is not only easier to work with
\PP-homomorphisms -- the category $\MASl$ had, in addition to initial
objects, factorisation system and all (finite) co-limits. These may be
strengths making it the most attractive of all the structures we have
reviewed.
%\end{LS}


\bibliographystyle{bibNo}
\bibliography{my}


\end{document}