\documentstyle[12pt]{article}
\textheight9in \headsep 0in \headheight0in \topmargin0in \textwidth 
6.5in \oddsidemargin0in
\setlength{\unitlength}{1cm}

\def\D{\displaystyle}
\def\NI{\noindent}

\newtheorem{thm}{Theorem}
\newtheorem{lem}{Lemma}
\newtheorem{DEFN}{Definition}
\def\qed{\rule{2mm}{2mm}}
\def\SOF{split opfibration}
\def\BC#1{{\bf \cal #1}}
\def\MON{\hookrightarrow}
\def\COP#1{{\bf \cal #1}^{\rm op}}
\def\OP{\mbox{\small op}}
\def\AMC#1#2#3{{\BC #1}_{{#2},{\BC #3}}} \def\ACT#1#2#3{{\BC 
#1}_{{\BC #2},#3}}

%%% 4 Oct 94


\begin{document}

\title{Identification of certain structures as split opfibrations 
over $\Delta^{\rm op}$}

\author{Paul G. Glenn\\Department of Mathematics\\Catholic University 
of America\\Washington, DC 20064\\E-mail: glenn\symbol{'100}cua.edu} 

\date{October 3, 1994}

\maketitle
\begin{abstract}
The purpose of this paper is to show how certain interesting 
structures, models of coherent first-order theories on the one hand 
and Grothendieck topologies on the other, may be viewed as split 
opfibrations.

In the course of doing so, two general classes of \SOF s will be 
defined of which models and topologies comprise two special examples. 
\end{abstract}
\tableofcontents

\section{Introduction}

The purpose of this paper is to show how certain interesting 
structures, models of coherent first-order theories on the one hand 
and Grothendieck topologies on the other, may be viewed as split 
opfibrations (equivalently, as wreath products or as coalgebras for a 
certain kind of cotriple on {\bf Cat}).

In the course of doing so, two general classes of \SOF s will be 
defined of which models and topologies comprise two special examples. 

The two general classes of \SOF s are related and arise as follows. 
Let ${P: {\cal D} \to {\cal C}}$ and ${K : \BC{D} \to 
{\BC{M}}^{\mbox{op}}}$ be functors where $K$ is a \SOF. Form the 
category $\BC A$ whose objects are monics $i : A \MON P(D)$, denoted 
by $\{A,D \}$. In the case where $P$ is contravariant, a map of $\BC 
A$, $\{A,D \} \to \{A',D' \}$ is a commutative square (in $\BC C$) 

\begin{center}
\begin{picture}(6,2)
\put(1,0){\makebox(0,0){$A$}}
\put(1,2){\makebox(0,0){$A'$}}
\put(4,0){\makebox(0,0){$P(D)$}}
\put(4,2){\makebox(0,0){$P(D')$}}
\put(1,1.5){\vector(0,-1){1}}
\put(4,1.5){\vector(0,-1){1}}
\put(1.5,0){\vector(1,0){1.8}}
%\put(1.5,0.35){\oval(.4,.3)[l]}
\put(1.5,2){\vector(1,0){1.8}}
\put(.5,1){$u$}
\put(4.2,1){$P(f)$}
\end{picture}
\end{center}

\NI where $f : D \to D'$ is in $\BC D$.

In the case where $P$ is covariant, a map $\{A,D \} \to \{A',D' \}$ 
is a commutative square like the one above except that the vertical 
maps go in the opposite direction.

The category $\BC A$ possesses a \SOF\ structure (over $\BC M$) 
provided that $\BC C$ satisfies certain hypotheses, different in each 
case.

For Grothendieck topologies, the objects of $\BC A$ are sieves $ A 
\MON \BC{M}(-,m)$ where $P:\BC{M} \to \BC{S}^{M^{\mbox{\small op}}}$ 
is the contravariant Yoneda embedding.

For models of coherent first-order theories, an object of $\BC A$ is 
a monic $A \MON \prod_{i=0}^n M(\tau_i)$ where $\tau_0,\ldots,\tau_n 
$ are sorts of the theory and $M$ is a model of the theory in the 
category $\BC C$. Here the descriptions of $P$ and $\BC D$ are longer 
and will be given in later sections.

\section{Notations}

\subsection{Split opfibrations}

Here is the notation which will be used in connection with the \SOF\ 
$K : \BC{A} \to \COP M $. For each $A \in \mbox{Ob } \BC A$ over $n 
\in \COP M$ and for for each map $g : m \to n$ in $\BC M$, we'll 
denote the opcartesian map over $g$ by $\lambda(A,g) : A \to A^g$. 
The axioms for split opfibrations require: 

\begin{enumerate}

\item $A^1=A$ and $\lambda(A,1)=1_A$.
\item Whenever $gh$ is defined, then $(A)^{gh}={(A^g)}^h$ and 
$\lambda(A^g,h)\lambda(A,g)=\lambda(A,gh)$. 

\end{enumerate}
That $\lambda(A,g)$ is opcartesian means that for each $u : A \to B$ 
in $\BC A$ over $g : m \to n$, there is a unique map $\iota(u) : A^g 
\to B$ over $1_m$ such that $u = \iota(u) \lambda(A,g)$. 

All this summarized in the diagram:

\begin{center}
\begin{picture}(7,4)
\put(1,3){\makebox(0,0){$A$}}
\put(4,3){\makebox(0,0){$A^g$}}
\put(7,3){\makebox(0,0){B}}
\put(1,1){\makebox(0,0){n}}
\put(4,1){\makebox(0,0){m}}
\put(7,1){\makebox(0,0){m}}
\put(2,2.5){$\lambda(A,g)$}
\put(5,2.5){$\iota(u)$}
\put(2.5,0.5){$g^{\mbox{\small op}}$}
\put(5.5,0.5){$1$}
\put(1.5,3){\vector(1,0){2}}
\put(4.5,3){\vector(1,0){2}}
\put(1.5,1){\vector(1,0){2}}
\put(4.5,1){\vector(1,0){2}}
\put(1,2.5){\vector(0,-1){1}}
\put(.9,2.5){\line(1,0){.2}}
\put(4,2.5){\vector(0,-1){1}}
\put(3.9,2.5){\line(1,0){.2}}
\put(7,2.5){\vector(0,-1){1}}
\put(6.9,2.5){\line(1,0){.2}}
\put(1,3.5){\line(0,1){.5}}
\put(1,4){\line(1,0){6}}
\put(7,4){\vector(0,-1){.5}}
\put(4,4.2){$u$}
\put(0,1){\makebox(0,0){$\COP M:$}}
\put(0,3){\makebox(0,0){$\BC A:$}}
\end{picture}
\end{center}

\section{Structures based on pullbacks } 

\subsection{The general construction}

In this section, we work with fixed small categories $\BC M$, $\BC D$ 
and $\BC C$ and with the functors $P : \COP D \to \BC C$ and $K : \BC 
D \to \COP M$.

Form a new category $\BC A$ as follows. For each $A \in \mbox{Ob } 
\BC C$, $D \in \mbox{Ob } \BC D$ and each monic $i : A \MON P(D)$, 
the data comprised of $i$, $A$ and $D$ is an object of $\BC A$. We 
will denote such an object simply by $\{ A,D \}$. 

A map of $\BC A$ from $\{ A,D \}$ to $\{ A',D' \}$ consists of a pair 
of maps $u : A' \to A$ and $f : D \to D'$ such that $P(f) i' = i u$. 
We will denote such a map by $(u,f)$ where $(u_1,f_1)=(u_2,f_2)$ if 
and only if $f_1=f_2$. (Note that $f_1=f_2$ and $i$ being monic imply 
$u_1=u_2$).

\begin{center}
\begin{picture}(4,3.3)
\put(1,3){\makebox(0,0){$A'$}}
\put(3,3){\makebox(0,0){$P(D')$}}
\put(1,1){\makebox(0,0){$A$}}
\put(3,1){\makebox(0,0){$P(D)$}}
\put(1,2.7){\vector(0,-1){1.2}}
\put(3,2.7){\vector(0,-1){1.2}}
\put(1.3,1){\vector(1,0){1}}
\put(1.3,3){\vector(1,0){1}}
\put(2,3.2){$i'$}
\put(2,.5){$i$}
\put(.7,2){$u$}
\put(3.2,2){$P(f)$}
\end{picture}
\end{center}

\begin{thm} Assume that $P$ is faithful and that $K$ is a \SOF . 
Suppose that $\BC C$ has the property that for each monic $i : A \MON 
P(D)$ in $\BC C$ and all $f : D \to D'$ in $\BC D$, the pullback of 
$i$ along $P(f)$ exists. Define the functor $F : \BC A \to \BC D$ by 
$F(\{A,D\}) = D$ and $F(u,f) = f$. Then $H = KF : \BC A \to \BC D \to 
\COP M$ is a \SOF.
\end{thm}

\NI {\bf Proof}: First, suppose $H(\{A,D\})=n$ and $g^{\mbox{\small 
op}} : n \to m$ is given in $\COP M$. Then, since $K$ is a fibration, 
one has a map $D \to D^g$ in $\BC D$ over $g^{\mbox{\small op}}$. We 
then form the pullback (whose existence follows from the hypothesis 
on $\BC C$):

\begin{center}
\begin{picture}(6,4)
\put(1,3){\makebox(0,0){$A^g$}}
\put(4,3){\makebox(0,0){$P(D^g)$}}
\put(1,1){\makebox(0,0){$A$}}
\put(4,1){\makebox(0,0){$P(D)$}}
\put(1,2.7){\vector(0,-1){1.2}}
\put(4,2.7){\vector(0,-1){1.2}}
\put(1.3,1){\vector(1,0){2}}
\put(1.3,3){\vector(1,0){2}}
\put(2.5,3.3){\makebox(0,0){$i^g$}}
\put(2.5,.5){\makebox(0,0){$i$}}
\put(.7,2){\makebox(0,0){$u^g$}}
\put(5.2,2){\makebox(0,0){$P(\lambda(D,g))$}} \end{picture}
\end{center}

We define $\{A,D\}^g$ to be $i^g : A^g \MON P(D^g)$ and define 
$\lambda(\{A,D\},g)$ in $\BC A$ to be the pullback square 
$(u^g,\lambda(D,g))$ above. Then $H(u^g,\lambda(D,g)) = 
K(\lambda(D,g)) =g$ and the equational conditions for $\lambda$ (for 
the functor $H$) follow immediately. As for the $\iota$ for $H$, let 
$(u,f) : \{A,D\} \to \{A',D'\}$ be any map in $\BC A$ over $g^{\OP}$. 
Consider the following diagram

\begin{center}
\begin{picture}(10,8)
\put(1,7){\makebox(0,0){$A'$}}
\put(1,1){\makebox(0,0){$A$}}
\put(3,4){\makebox(0,0){$A^g$}}
\put(6,4){\makebox(0,0){$P(D^g)$}}
\put(8,7){\makebox(0,0){$P(D')$}}
\put(8,1){\makebox(0,0){$P(D)$}}
\put(1,6){\vector(0,-1){4}}
\put(8,6){\vector(0,-1){4}}
\put(2,7){\vector(1,0){5}}
\put(2,1){\vector(1,0){5}}
\put(4,4){\vector(1,0){1}}
\put(2.5,3.5){\vector(-2,-3){1}}
\put(6.5,3.5){\vector(2,-3){1}}
\put(1.5,6.5){\vector(2,-3){1}}
\put(7.5,6.5){\vector(-2,-3){1}}
\put(5.8,2.5){\makebox(0,0){$P (\lambda(D,g))$}} 
\put(2.3,2.5){\makebox(0,0){$u^g$}}
\put(6,6){\makebox(0,0){$P (\iota(f))$}} 
\put(8.5,4){\makebox(0,0){$P(f)$}}
\put(.5,4){\makebox(0,0){$u$}}
\put(2,6){$u'$}
\end{picture}
\end{center}

\NI The lower square is $\lambda(\{A,D\},g)$. Since the lower square 
is a pullback, a unique map $u' : A' \to A^g$ exists making the upper 
square commute. We define $\iota(u,f)$ to be $(u',\iota(f))$. The map 
$(u',\iota(f))$ is unique since $P$ is faithful. It follows 
immediately that $K(\iota(u,f))=1_m$ and also that 
$\lambda(\{A,D\},g)$ is opcartesian. \qed 

\subsection{Example of sieves and Grothendieck topologies} 

Fix a small category $\BC M$. For the functor $P : \COP D \to \BC C$ 
we take $\BC D = \COP M$, $\BC C = S^{\COP M}$ and $P$ to be the 
Yoneda embedding. The \SOF\ $K : \BC D \to \COP M$ is the identity 
functor. The construction given in the previous section yields a 
\SOF\ $\BC A_+$. A typical object of $\BC A_+$ is a sieve $i : R \MON 
\BC M (-,n)$. The maps of $\BC A_+$ are commutative squares in 
$S^{\COP M}$:

\begin{center}
\begin{picture}(4,3.5)
\put(1,3){\makebox(0,0){$S$}}
\put(3,3){\makebox(0,0){$(-,m)$}}
\put(1,1){\makebox(0,0){$R$}}
\put(3,1){\makebox(0,0){$(-,n)$}}
\put(1,2.7){\vector(0,-1){1.2}}
\put(3,2.7){\vector(0,-1){1.2}}
\put(1.3,1){\vector(1,0){1}}
\put(1.3,3){\vector(1,0){1}}
%\put(2,3.3){\makebox(0,0){$a$}}
%\put(2,.5){\makebox(0,0){$b$}}
\put(.7,2){\makebox(0,0){$u$}}
\put(3.3,2){\makebox(0,0){$g_*$}}
\end{picture}
\end{center}

Any family of sieves on $\BC M$ which is closed under pullback along 
all $g_*$ is clearly a \SOF\ subcategory of $\BC A_+$. This holds in 
particular for any Grothendieck topology on $\BC M$. Further, it 
turns out that topologies may be characterized specifically in terms 
of a closure property on \SOF\ subcategories of $\BC A_+$.

\begin{lem}
Suppose $R \MON (-,n)$ and $R' \MON (-,n)$ are given objects of $\BC 
A_+$. If there is a map $(q,g) : \{R,n\} \to \{(-,m),m\}$ then there 
is also a map $(u,g) : \{R' \cap R,n\} \to \{R' \times_n (-,m),m\}$
\end{lem}

\NI {\bf Proof}: The map $(q,g)$ is the commutative square 
\begin{center}
\begin{picture}(4,3.5)
\put(1,3){\makebox(0,0){$(-,m)$}}
\put(3,3){\makebox(0,0){$(-,m)$}}
\put(1,1){\makebox(0,0){$R$}}
\put(3,1){\makebox(0,0){$(-,n)$}}
\put(1,2.7){\vector(0,-1){1.2}}
\put(3,2.7){\vector(0,-1){1.2}}
\put(1.3,1){\vector(1,0){1}}
\put(1.6,3){\vector(1,0){.7}}
\put(2,3.3){\makebox(0,0){$1$}}
%\put(2,.5){\makebox(0,0){$b$}}
\put(.7,2){\makebox(0,0){$u$}}
\put(3.3,2){\makebox(0,0){$g_*$}}
\end{picture}
\end{center}

\NI and simply indicates that $g_* : (-,m) \to (-,n)$ factors through 
$R$. Thus we have a composite of pullbacks: \begin{center} 
\begin{picture}(11,5)
\put(2,4.5){\makebox(0,0){$R' \times_n (-,m)$}} 
\put(2,2.5){\makebox(0,0){$(-,m)$}}
\put(6.5,4.5){\makebox(0,0){$R' \cap R$}} 
\put(6.5,2.5){\makebox(0,0){$R$}}
\put(10,4.5){\makebox(0,0){$R'$}}
\put(10,2.5){\makebox(0,0){$(-,n)$}}
\put(4,4.5){\vector(1,0){1.2}}
\put(8,4.5){\vector(1,0){1.5}}
\put(2,4){\vector(0,-1){1}}
\put(6.5,4){\vector(0,-1){1}}
\put(10,4){\vector(0,-1){1}}
\put(3,2.5){\vector(1,0){2.8}}
\put(7,2.5){\vector(1,0){2.3}}
\put(2,1.7){\line(0,-1){.7}}
\put(2,1){\line(1,0){8}}
\put(10,1){\vector(0,1){.7}}
\put(4.5,5){\makebox(0,0){proj}}
\put(4.5,2){\makebox(0,0){$q$}}
\put(6,.5){\makebox(0,0){$g_*$}}
\end{picture}
\end{center}

\NI from which one obtains $(u,g)$ as \medskip 

\begin{center} \begin{picture}(8.5,3)
\put(2,3){\makebox(0,0){$R' \times_n (-,m)$}} 
\put(2,1){\makebox(0,0){$R' \cap R$}}
\put(5,1){\makebox(0,0){$R$}}
\put(8,1){\makebox(0,0){$(-,n)$}}
\put(8,3){\makebox(0,0){$(-,m)$}}
\put(3,2){\makebox(0,0){$u=proj$}}
\put(8.4,2){\makebox(0,0){$g_*$}}
\put(4,3){\vector(1,0){3}}
\put(2,2.5){\vector(0,-1){1}}
\put(8,2.5){\vector(0,-1){1}}
\put(3,1){\vector(1,0){1.5}}
\put(5.5,1){\vector(1,0){1.8}}
\end{picture} \end{center}
\qed

\begin{thm}
Let $\BC A$ be a \SOF\ subcategory of $\BC A_+$. Then ob $\BC A$ is a 
Grothendieck topology on $\BC M$ if the following conditions hold. 
\begin{enumerate} \item For each $n \in \mbox{ob } \BC M$, $\{(-,n)\} 
\in \mbox{ob } \BC A$.

\item For all $\{R,n\}$ in $\BC A$ and $\{R',n\}$ in $\BC A_+$, if 
for all $m$ such that a map $(q,g) : \{R,n\} \to \{(-,m),m\}$ exists, 
one has $\{R' \times_n (-,m),m\}$ in $\BC A$, then $\{R',n\}$ belongs 
to $\BC A$ also.

\end{enumerate}
\end{thm}

\NI {\bf Proof}: Condition 1 states that all $1_n : (-,n) \to (-,n)$ 
belong to $\BC A$. Since $\BC A$ is a \SOF\ subcategory of $\BC A_+$, 
then ob $\BC A$ is stable under pullback. Next: the lemma above shows 
that the existence of $(q,g) : \{R,n\} \to \{(-,m),m\}$ yields the 
composite of pullback squares shown above. Thus condition 2 is a 
precise rephrasing of the local character axiom for a Grothendieck 
topology. \qed

\section{Structures defined with images} 

\subsection{Notations and review of definitions} 
\label{IMAGES}
For the reader's convenience, we review some definitions. See 
\cite{FS} and \cite{regcats}.

The {\em image} of a map $f : X \to Y$ in a category $\BC C$ is the 
smallest subobject (if there is one), denoted $\mbox{im } f \MON Y$ 
through which $f$ factors. If $\mbox{im } f=Y$ then $f$ is said to be 
{\em surjective}. (Elsewhere, \cite{Borceux}, $f$ is called an 
extremal epimorphism).

>From now on, we assume $\BC C$ satisfies the following conditions. 
\begin{DEFN}
\label{IMCAT}
\begin{enumerate}
\item All maps of $\BC C$ have images.
\item $\BC C$ has pullbacks.
\item Images are stable under pullbacks. \end{enumerate}
\end{DEFN}

\NI All regular categories, for example, satisfy these hypotheses. 

The following lemma concerning $\BC C$ is elementary. 

\begin{lem} (1) Monic surjective maps are isomorphisms. (2) The 
composition of surjective maps is surjective. (3) Surjective maps are 
epimorphisms.
\end{lem}

\subsection{Split opfibrations defined with images: general $\BC M$} 
\label{IMAGE_SOF}
In this section, we work with fixed small categories and functors $P 
: \BC D \to \BC C$ and $K : \BC D \to \COP M$. 

Form the new category $\BC A$ as follows. For each $A \in \mbox{ob } 
\BC A$, $D \in \mbox{ob } \BC D$ and each monic $i : A \MON P(D)$, 
the data comprised of $i$, $A$ and $D$ is an object of $\BC A$. We'll 
denote this object by $\{A,D\}$.

A map $(u,f) : \{A,D\} \to \{A',D'\}$ is a pair of maps $u : A \to 
A'$ in $\BC C$ and $f : D \to D'$ in $\BC D$ such that $P(f) i = i'u$.
\begin{center}
\begin{picture}(6,2.5)
\put(1,0){\makebox(0,0){$A'$}}
\put(1,2){\makebox(0,0){$A$}}
\put(4,0){\makebox(0,0){$P(D')$}}
\put(4,2){\makebox(0,0){$P(D)$}}
\put(1,1.5){\vector(0,-1){1}}
\put(4,1.5){\vector(0,-1){1}}
\put(1.5,0){\vector(1,0){1.8}}
\put(1.5,2){\vector(1,0){1.8}}
\put(.5,1){$u$}
\put(4.2,1){$P(f)$}
\put(2.5,.4){$i'$}
\put(2.5,2.4){$i$}
\end{picture}
\end{center}

\NI As before, $(u_1,f_1)=(u_2,f_2)$ only if $f_1=f_2$ and 
(redundantly, since $i'$ is monic) $u_1=u_2$. 

\begin{thm}
If $\BC C$ satisfies the conditions given in definition \ref{IMCAT}, 
$K : \BC D \to \COP M$ is a \SOF\ and $P$ is faithful, then $\BC A$ 
has a \SOF\ structure over $\COP M$ given by $H : \BC A \to \COP M$ 
where $H=KF : \BC A \to \BC D \to \COP M$ with $F$ defined by 
$F(\{A,D\}) = D$ and $F(u,f)=f$.
\end{thm}

\NI {\bf Proof}: Let $(u,f) : \{A,D\} \to \{A',D'\}$ be given. Since 
$K$ is a \SOF\ then $f : D \to D'$ factors as $\iota(f) \lambda(D,g) 
: D \to D^g \to D'$ where $g=K(f)$. 

Define $i^g : A^g \MON P(D^g)$ to be $im(A \stackrel{i}{\MON} P(D) 
\stackrel{P(\lambda(D,g))}{\longrightarrow} P(D^g))$ with $u^g : A 
\to A^g$ the corresponding surjection.

Let $A'' \MON P(D')$ be $im(A^g \stackrel{i^g}{\to} P(D^g) 
\stackrel{P(\iota(f))}{\longrightarrow} P(D'))$. 

Since $A \to P(D) \to P(D') = A \stackrel{u^g}{\to} A^g \to A'' \MON 
P(D')$, then $A'' \MON P(D') = im(P(f)i) = im(i'u)$, and therefore 
$A''$ is a subobject of $A'$. This is summarized in the following 
diagram.
\begin{center}
\begin{picture}(13,6)
\put(1,4){\makebox(0,0){$P(D)$}}
\put(1,2){\makebox(0,0){$A$}}
\put(5,4){\makebox(0,0){$P(D^g)$}}
\put(5,2){\makebox(0,0){$A^g$}}
\put(9,2){\makebox(0,0){$A''$}}
\put(12,4){\makebox(0,0){$P(D')$}}
\put(12,2){\makebox(0,0){$A'$}}
\put(1.2,3){\makebox(0,0){$i$}}
\put(5.3,3){\makebox(0,0){$i^g$}}
\put(12.3,3){\makebox(0,0){$i'$}}
\put(2.8,4.3){\makebox(0,0){$P(\lambda(D,g))$}} 
\put(8,4.3){\makebox(0,0){$P(\iota(f))$}} 
\put(2.5,1.7){\makebox(0,0){$u^g$}}
\put(2.5,2.2){\makebox(0,0){{\footnotesize surj}}} 
\put(6.5,2.2){\makebox(0,0){{\footnotesize surj}}} 
\put(10.25,2.2){\makebox(0,0){{\footnotesize monic}}} 
\put(6,1.2){\makebox(0,0){$u$}}
\put(6,5.3){\makebox(0,0){$f$}}
\put(2,4){\vector(1,0){2}}
\put(6,4){\vector(1,0){5}}
\put(1.5,2){\vector(1,0){3}}
\put(5.5,2){\vector(1,0){3}}
\put(9.5,2){\vector(1,0){2.2}}
\put(1,2.5){\vector(0,1){1}}
\put(5,2.5){\vector(0,1){1}}
\put(12,2.5){\vector(0,1){1}}
\put(9.5,2.4){\vector(3,2){1.7}}
\put(12,5){\vector(0,-1){.5}}
\put(12,1){\vector(0,1){.5}}
\put(1,4.5){\line(0,1){.5}}
\put(1,5){\line(1,0){11}}
\put(1,1.5){\line(0,-1){.5}}
\put(1,1){\line(1,0){11}}
\end{picture}
\end{center}

We define $\lambda(\{A,D\},g)$ to be $(u^g,\lambda(D,g))$ and define 
$\iota(u,f)$ to be $(u^\#,\iota(f))$ where $u^\# = A^g \to A'' \MON 
A'$. The uniqueness of $\iota(u,f) = (u^\#,\iota(f))$ follows from 
$P$ being faithful.

To complete the proof:
$H(\lambda(\{A,D\},g))=H(u,\lambda(D,g))=K(\lambda(D,g) )=g$ and 
$H(\iota(u,f))=K(\iota(f))=1$. The equational conditions for 
$\lambda$ (of $H$) follow immediately, and the condition that 
$\lambda(\{A,D\},g)$ is opcartesian follows from the uniqueness of 
$\iota(u,f)$. \qed

\subsection{Split opfibrations defined with images; $\BC M= \Delta$}

\subsubsection{Notations} For each $n \ge 0$, $[n]$ denotes the 
linearly ordered set $\{0,\ldots,n\}$. The objects of the simplicial 
category $\Delta$ are all $[n]$ and the maps are all non-decreasing 
functions.

Given any ordinal number $\kappa$ and any integer $n \ge 0$, we will 
consider various maps $[n] \to \kappa$ in the sections below. All of 
these maps will be {\em assumed non-decreasing} without further 
mention.

\subsubsection{An ordinal mapped \SOF}
\label{ORD_SOF}
Let $\BC C$ be any category. Fix an infinite ordinal $\kappa$ and a 
function $T : \kappa \to \mbox{ob } \BC C$, which we will refer to as 
an {\em ordinal mapping} of $\BC C$. Also fix a subcategory $\BC Q$ 
of $\BC C$ which contains every object in the image of $T$. 

>From these data we form a new category ${\BC C}_{T,\BC Q}$ as 
follows. 
For all $n \ge 0$ and all maps $x : [n] \to \kappa$, the composite 
map $Tx$ is an object of ${\BC C}_{T,\BC Q}$. Let $x : [n] \to 
\kappa$ and $y : [m] \to \kappa$ be given. A map $Tx \to Ty$ in ${\BC 
C}_{T,\BC Q}$ is given by the following data: 

\begin{enumerate}
\item A map $g : [m] \to [n]$ in $\Delta$. \item A sequence of maps 
$q_j : Txg(j) \to Ty(j)$ for $0 \le j \le m$ where each $q_j$ belongs 
in the subcategory $\BC Q$. \end{enumerate}

\NI The map will be denoted $(g;q_\bullet)$. It might be visualized as

\begin{center}
\begin{picture}(14,4)
\put(2,4){\makebox(0,0){$Tx:$}}
\put(2,1){\makebox(0,0){$Ty:$}}
\put(4,1){\makebox(0,0){$Ty(0)$}}
\put(6,4){\makebox(0,0){$Txg(0)$}}
\put(8,1){\makebox(0,0){$Ty(j)$}}
\put(9,4){\makebox(0,0){$Txg(j)$}}
\put(12,4){\makebox(0,0){$Txg(m)$}}
\put(14,1){\makebox(0,0){$Ty(m)$}}
\put(1,2.5){\makebox(0,0){$(g;q_\bullet)$}} 
\put(6,2.5){\makebox(0,0){$q_0$}}
\put(9,2.5){\makebox(0,0){$q_j$}}
\put(13.5,2.5){\makebox(0,0){$q_m$}}
\put(4,4){\makebox(0,0){$\cdots$}}
\put(6,1){\makebox(0,0){$\cdots$}}
\put(6.75,2.5){\makebox(0,0){$\cdots$}}
\put(7.5,4){\makebox(0,0){$\cdots$}}
\put(10.5,4){\makebox(0,0){$\cdots$}}
\put(10.75,2.5){\makebox(0,0){$\cdots$}} 
\put(11,1){\makebox(0,0){$\cdots$}}
\put(14,4){\makebox(0,0){$\cdots$}}
\put(6,3.5){\vector(-2,-3){1.3}}
\put(2,3.5){\vector(0,-1){2}}
\put(9,3.5){\vector(-1,-2){1}}
\put(12,3.5){\vector(2,-3){1.3}}
\end{picture}
\end{center}

Given $[k] \stackrel{f}{\to} [m] \stackrel{g}{\to} [n]$ in 
$\Delta$,and $z : [k] \to \kappa$, the composite $Tx 
\stackrel{(g;q_\bullet)}{\longrightarrow} Ty 
\stackrel{(f;r_\bullet)}{\longrightarrow} Tz$ is $(gf;s_\bullet)$ 
where $s_i = r_i q_{f(i)} : Txgf(i) \to Tyf(i) \to Tz(i)$ and $0 \le 
i \le k$.
\bigskip

\NI {\bf Remark}: An alternative to the definition above of ${\BC 
C}_{T,\BC Q}$ is to form the category $\BC T_\kappa C$ which has the 
same objects as $\BC C$. A map $A \to B$ in $\BC T_\kappa C$ 
corresponds to $i \le j < \kappa$ such that $T(i)=A$ and $T(j)=B$. 
Then $x : [n] \to \kappa$ determines a functor $\bar x : [n] \to \BC 
T_\kappa C$ given by $\bar x (i) =Tx(i)$. With these definitions, we 
may view ${\BC C}_{T,\BC Q}$ as the category whose objects are all 
functors of the form $\bar x : [n] \to \BC T_\kappa C$ for all $n \ge 
0$ and where a map from $\bar x : [n]\to \BC T_\kappa C$ to $\bar y : 
[m] \to \BC T_\kappa C$ consists of $(g;q_\bullet)$ defined as above 
where $q_\bullet$ is a natural transformation $\bar x g \to \bar y$ 
all of whose components are required to belong to $\BC Q$. 

\begin{thm} \label{CTQ_SOF}
The functor $K : {\BC C}_{T,\BC Q} \to \Delta^{op}$ is a \SOF, where 
$K(Tx)=[n]$ if dom$(x)=[n]$ and $K(g;q_\bullet)=g$. \end{thm}

\NI {\bf Proof}: Let $x : [n] \to \kappa$ and $g : [m] \to [n]$ be 
given. Suppose there is a map $(g;q_\bullet) : Tx \to Ty$. Then 
$(g;q_\bullet)$ factors as $(1_{[m]};q_\bullet) (g;1_\bullet)$. 
Define $(Tx)^g$ to be $Txg$, $\lambda(Tx,g) : Tx \to Txg$ to be 
$(g;1_\bullet)$, and $\iota(g;q_\bullet)$ to be $(1_{[m]};q_\bullet) 
: Txg \to Ty$. The \SOF\ conditions follow easily. \qed 

\subsection{Split opfibrations of relations} \label{REL_SOF}

In this section, assume that $\BC C$ satisfies definition \ref{IMCAT} 
in section \ref{IMAGES} {\em and} has all finite products. Choose an 
ordinal mapping $T : \kappa \to \mbox{ob } \BC C$ and a subcategory 
$\BC Q$ of $\BC C$ containing the image of $T$, as in the previous 
section.

Define the functor $P : {\BC C}_{T,\BC Q} \to \BC C$ as follows. 
Given $x : [n] \to \kappa$, define $P(Tx) = \prod_{i=0}^n Tx(i)$. 
Given $(g;q_\bullet) : Tx \to Ty$, we define $P(g;q_\bullet)$ to be 
the composite $$\prod_{i=0}^n Tx(i) \to \prod_{j=0}^m Txg(j) \to 
\prod_{j=0}^m Ty(j)$$ where the $j$'th component of the composite is 
$q(j) \mbox{pr}_{g(j)}$. It is clear that $P$ is faithful. 

We may now construct a \SOF\ $\BC A$ as described in section 
\ref{IMAGE_SOF} using $P : {\BC C}_{T,\BC Q} \to \BC C$ and the \SOF\ 
$K : {\BC C}_{T,\BC Q} \to \Delta^{op}$. An object of $\BC A$, 
$\{A,Tx\}$ is a monic $i : A \MON P(Tx)=\prod_{i=0}^n Tx$, that is: a 
relation on $Tx(0) \times \cdots \times Tx(n)$. A map of $\BC A$ is a 
``homomorphism of relations''. The fibration functor $H : \BC A \to 
\Delta^{op}$ sends $\{A,Tx\}$ to dom($x$) and sends the map 
$(u,(g;q_\bullet))$ to $g$.

\section{Split opfibrations arising from models} 

Let $\cal T$ be a coherent first-order theory and let $M : \cal T \to 
\BC C$ be a model of $\cal T$ in a logical category $\BC C$. Then a 
formula $\phi$ of $\cal T$ with free variables $t_0,\ldots,t_n $ is 
associated by $M$ to a monic $M(\phi) \MON \prod_{i=0}^n M(\tau_i)$ 
in ${\BC C}$ where $\tau_i$ is the sort of the variable $t_i$. Our 
goal is to set out how this situation may be described in terms of 
split opfibrations.

The reader is referred to \cite{MR} for background information on 
coherent first-order theories and their models. However, for the 
reader's convenience, the next section gives a brief outline of some 
of that information.

\subsection{Background on coherent first-order theories} \label{CFOT}
We will consider many-sorted theories-with-equality with connectives 
$\wedge$, $\vee$ and the existential quantifier $\exists$. The 
presentation of a particular theory provides the operation and 
predicate symbols used to form atomic terms and atomic formulas. 
Formulas are then built up from the terms and atomic formulas in the 
standard fashion.

Assertions (called {\em sequents}) of the theory take the form $\Phi 
\Rightarrow \Psi$ where $\Phi$ and $\Psi$ are sets of formulas. The 
intended meaning of $\Phi \Rightarrow \Psi$ is: ``the conjunction of 
the formulas in $\Phi$ implies the disjunction of the formulas in 
$\Psi$''. There is a set of logical axioms and rules of inference 
concerning ``$\Rightarrow$'' together with whatever proper axioms may 
belong to the particular theory.

The interpretation of a theory $\cal T$ in a suitable category $\BC 
C$ follows closely the details of interpretation of theories in 
sets-with-structure. For each sort $\tau$ of the theory, an 
interpretation $M$ designates an object $M(\tau)$ of $\BC C$. 
Compound sorts, denoted $\tau_0 \times \cdots \times \tau_n$, are 
interpreted by $M$ as products $M(\tau_0) \times \cdots \times 
M(\tau_n)$ in $\BC C$. Thus, $\BC C$ must admit at least those 
products. Operation symbols are interpreted as maps between such 
products, and predicate symbols are interpreted as subobjects of 
products of $M(\tau)$'s.

If $\phi$ is a formula with free variables $t_0,\ldots,t_n$ of sorts 
$\tau_0,\ldots,\tau_n$ then $\phi$ is interpreted by $M$ as a 
subobject of $M(\tau_0) \times \cdots \times M(\tau_n)$. There is a 
precise correspondence between building up formulas using connectives 
and $\exists$ in the language of $\cal T$ and defining objects and 
maps in $\BC C$ using certain operations and properties of $\BC C$. 
$\BC C$ must have equalizers (in order to interpret formulas 
involving equality), and therefore all finite limits. In order to 
interpret formulas of the form $\exists \phi$, $\BC C$ must have 
images and satisfy the conditions in section \ref{IMAGES}. In order 
to interpret disjunctions, $\BC C$ must have suprema of arbitrary 
finite sets of subobjects. That is, if $\{A_i \MON X\}_{i \in I}$ is 
a finite set of monics in $\BC C$, then there is a smallest monic 
through which they all factor. Categories having all these properties 
are called {\em logical} categories in \cite{MR} 

We'll denote an interpretation $M$ of $\cal T$ in $\BC C$ by $M : 
\cal T \to \BC C$. In order for M to be a {\em model} of $\cal T$, 
$M$ must associate to every $\cal T$-provable assertion $\Phi 
\Rightarrow \Psi$ of $\cal T$ a monic $M(\bigwedge \Phi) \MON 
M(\bigvee \Psi))$. 

Given a coherent first-order theory $\cal T$, one may form a logical 
category $\BC C (\cal T)$ in which the theory has a canonical model 
$M_0 : \cal T \to \BC C (\cal T)$. The category $\BC C (\cal T)$ is 
the ``classifying logical category'' of $\cal T$ in the sense that 
every model of $\cal T$ factors though $M_0$. 

The objects of $\BC C (\cal T)$ are equivalence classes of formulas 
of the theory where two formulas belong to the same equivalence class 
if they are $\cal T$-provably equivalent. The maps of $\BC C (\cal 
T)$ are equivalence classes of formulas which are $\cal T$-provably 
the graphs of maps. (The graph of a map in any category with products 
is a subobject of a product satisfying the usual conditions; these 
are expressible as sequents. A formula is provably a graph if these 
sequents are provable in $\cal T$).

The canonical model $M_0$ of $\cal T$ in $\BC C (\cal T)$ associates 
to each formula its equivalence class.

The construction of a logical category from a first-order theory has 
(nearly) an inverse in the formation of a theory $\cal T (\BC C)$ 
from a given small logical category $\BC C$, such that $\BC C$ is the 
classifying logical category of $\cal T (\BC C)$. The sorts of $\cal 
T (\BC C)$ are the objects of $\BC C$ and the operation symbols of 
$\cal T (\BC C)$ are the maps of $\BC C$. The $\wedge \vee 
\exists$-language with equality is sufficient to express 
axiomatically the properties of a category being logical. These, plus 
axioms corresponding to identity maps and commutative triangles, 
comprise the proper axioms of $\cal T (\BC C)$. 

\subsection{Split opfibration corresponding to a model} 
\label{MODEL_SOF}
Let $M : \cal T \to \BC C$ be a model of the coherent first-order 
theory $\cal T$ in the logical category $\BC C$. Choose and fix an 
indexing of the set of all basic and compound sorts of $\cal T$ by an 
ordinal $\kappa$. Relative to this indexing, define the function $T : 
\kappa \to \mbox{ob } \BC C$ by setting $T(i)$ to be the image under 
$M$ of the $i$'th sort so indexed. Note that each $T(i)$ is a finite 
product in $\BC C$:
$$
i \stackrel{\mbox{\small indexing}}{\longmapsto} \tau_{i(0)} \times 
\cdots \times \tau_{i(n)} \longmapsto \prod_{k=0}^n M(\tau_{i(k)}) $$
where $\tau_{i(0)} \times \cdots \times \tau_{i(n)}$ is the sort 
associated to $i$ by the indexing.

Next, we must designate a subcategory $\BC Q$ of $\BC C$ which is 
appropriately related to $\cal T$ and the model $M$. 

\begin{DEFN}
Given $M : {\cal T} \to {\BC C}$ and $T : \kappa \to \mbox{ob } {\BC 
C}$, let $\BC Q$ be the smallest subcategory of $\BC C$ satisfying 
the following conditions.
\begin{enumerate}
\item For each $i \in \kappa$, $T(i)$ is an object of $\BC Q$. 

\item For all $i \in \kappa$, $\BC Q$ contains every projection map 
with domain $T(i)$.

\item For each operation symbol $f$ in the language of $\cal T$, $\BC 
Q$ contains the corresponding map $M(f)$. 

\item If $\BC Q$ contains the maps $u_k : T(i) \to T(j_k)$ for 
$k=0,\ldots,n$, then $\BC Q$ also contains the map $(u_0,\ldots,u_n) 
: T(i) \to \prod_{k=0}^n T(j_k)$
\end{enumerate}
\end{DEFN}

Note the inductive nature of the definition of $\BC Q$ in the last 
condition of the definition. Note also that every map of $\BC Q$ is 
one of the four types listed in the definition. 

Once given $T$ and $\BC Q$ there is an ordinal indexed \SOF\ ${\BC 
C}_{T,\BC Q}$ as defined in section \ref{ORD_SOF}. 

Let $\BC A (M)$ denote the \SOF\ arising from $P : {\BC C}_{T, \BC Q} 
\to \BC C$ and the \SOF\ $K : {\BC C}_{T, \BC Q} \to \Delta$. A 
typical object of $\BC A (M)$ is a monic $\{A,Tx\} : A \MON 
\prod_{k=0}^n Tx(k)$.\medskip

Now suppose $\phi$ is a formula of the theory $\cal T$ with free 
variables $t_0,\ldots,t_n$ of sorts $\tau_0,\ldots,\tau_n$. We may 
assume that the sorts are listed according to the indexing so that 
there is an $x : [n] \to \kappa$ such that $\tau_k=Tx(k)$ for each 
$k=0,\ldots,n$. Then $M$ associates to $\phi$ a monic $M(\phi) \MON 
\prod_{k=0}^n M(\tau_k) = PTx$ which is an object $\{M(\phi),Tx\}$ of 
$\BC A (M)$.

Note that $n$ is a part of the specification of the object 
$\{M(\phi),Tx\}$. If $\phi$ were such that one could group some of 
its free variables into a single compound free variable (thus 
changing $n$), the result would be a {\em different} object of ${\BC 
A}(M)$.

\subsection{Split opfibration subcategories of $\BC A (M)$} 
\label{SOF_ATM}

In general, there will be many objects $\{A,Tx\}$ in $\BC A (M)$ such 
that $A \neq M(\phi)$ for any formula $\phi$. However, we may 
identify a \SOF\ subcategory of $\BC A (M)$ whose objects do come 
from the theory via $M$.

First we will say what it means for a map of $\BC C$ to ``come from'' 
the theory. Recall that in a category with products, the {\em graph} 
of a map $f : A \to B$ is the monic $1 \times f : A \MON A \times B$, 
or any subobject of $A \times B$ isomorphic to $1 \times f$. 
Subobjects of $A \times B$ which are graphs are identified by the 
following elementary lemma.

\begin{lem} Let $g : R \MON A \times B$ be given such that 
$\mbox{pr}_A g : R \MON A \times B \to A$ is an isomorphism. Let 
$h={(\mbox{pr}_A g)}^{-1}$. Then $gh = 1 \times f$ where $f = 
\mbox{pr}_B g h : A \stackrel{h}{\to} R \stackrel{g}{\to} A \times B 
\stackrel{{\rm pr}_B}{\to} B$.
\qed
\end{lem}

The following definition and lemma are from \cite{MR} 

\begin{DEFN} Let $\phi(x)$, $\psi(y)$ and $\theta(x,y)$ be formulas 
of the theory $\cal T$ where $x$ and $y$ are the free variables of 
$\phi$ and $\psi$ respectively. Then $\theta(x,y)$ is {\em provably 
the graph of a function from $\phi$ to $\psi$} if the following 
sequents are provable in $\cal T$: (1) $\theta(x,y) \Rightarrow 
\phi(x) \wedge \psi(y)$. (2) $\theta(x,y_1) \wedge \theta(x,y_2) 
\Rightarrow y_1=y_2$. (3) $\phi(x) \Rightarrow \exists y \, 
\theta(x,y)$.
\end{DEFN}

\begin{lem}
If $\theta$ is provably the graph of a function from $\phi$ to $\psi$ 
and $M : \cal T \to \BC C$ is a model of $\cal T$ in the logical 
category $\BC C$, then $M(\theta) \MON M(\phi) \times M(\psi)$ is the 
graph of a map in $\BC C$. \qed \end{lem}

We may now define what it means for a map in $\BC C$ to ``come from'' 
$\cal T$ via $M$.

\begin{DEFN}
Let $M : \cal T \to \BC C$ be a model of a coherent first-order 
theory in a logical category, and let $g : R \MON A \times B$ be the 
graph of $f : A \to B$ in $\BC C$. Then we say that $f$ is an {\em 
$M$-map} if there are formulas $\phi(x)$, $\psi(y)$ and $\theta(x,y)$ 
of $\cal T$ such that $\theta$ is provably the graph of a function 
from $\phi$ to $\psi$ and $M(\theta) \MON M(\phi) \times M(\psi)$ is 
$g$. \end{DEFN}

\NI {\bf Remark}: Recall from section \ref{CFOT} that a coherent 
first-order theory $\cal T$ has a classifying logical category $\BC C 
(\cal T)$ whose objects are equivalence classes of formulas and where 
two formulas are in the same class if they are $\cal T$-provably 
equivalent. The maps are equivalence classes of graphs of functions 
(as in the definition above). The lemma above is part of the proof 
that the model $M : \cal T \to \BC C$ determines a logical functor 
$\BC C (\cal T) \to \BC C$. A map is an $M$-map if it is in the image 
of the functor ${\BC C}({\cal T}) \to \BC C$. 

\begin{lem}
Let $M : \cal T \to \BC C$ be a model of a coherent first-order 
theory in a logical category and let $\BC Q$ be the subcategory 
defined in section \ref{MODEL_SOF}. Then all maps of $\BC Q$ are 
$M$-maps \end{lem}

\NI {\bf Proof}: The maps of $\BC Q$ are all identity maps, all 
projections $\prod_{k=0}^n Tx(k) \to Tx(j)$, all maps $M(f)$ where 
$f$ is an operation symbol of $\cal T$ and all maps of the form 
$(u_0,\ldots,u_n) : A \to \prod_{k=0}^n B_k$ arising from maps $u_k : 
A \to B_k$ all belonging to $\BC Q$. We will give appropriate $\phi$, 
$\psi$ and $\theta$ for each of these types of maps but omit the 
details.

(Note that by the Completeness Theorem in \cite{MR}, it is sufficient 
to interpret the formulas in set-valued models. In sets, the formulas 
below are elementary).

\begin{enumerate}
\item Identity maps: Take $\theta(x,y)$ to be the formula `$x=y$'. 
\item Projection maps: $\mbox{pr}_j : \prod_{k=0}^n Tx(k) \to Tx(j)$. 
Let $t_i$ be a variable of sort $\tau_i$ where $M(\tau_i)=Tx(i)$. 
Take $\phi(t_0,\ldots,t_n)$ to be the formula `$(t_0=t_0) \wedge 
\cdots \wedge (t_n=t_n)$', $\psi(y)$ to be the formula `$y=y$' where 
$y$ is of sort $\tau_j$. Then let $\theta(t_0,\ldots,t_n,y)$ be the 
formula `$\phi(t_0,\ldots,t_n) \wedge (t_j=y)$. 

\item Suppose $f : \tau_0 \times \cdots \tau_n \to \tau_0' \times 
\cdots \times \tau_m'$ is an operation symbol of $\cal T$. Let $t_i$ 
be a variable of sort $\tau_i$, $t'_j$ of sort $\tau_j'$, and let 
$t=(t_0,\ldots,t_n)$ and $t'=(t_0',\ldots,t_m')$ be the corresponding 
compound variables. To represent $M(f)$ as an $M$-map, take $\phi(t)$ 
to be the formula `$(t_0=t_0) \wedge \cdots \wedge (t_n=t_n)$', 
$\psi(t')$ to be the formula `$(t_0'=t_0') \wedge \cdots \wedge 
(t_m'=t_m')$' and take $\theta(t,t')$ to be `$t'=f(t)$'. 

\item If $u_k : P(Tx) \to P(Ty(k))$ are given for $k=0,\ldots,n$ 
where each $u_k$ is an $M$-map with graph $M(\theta_k(t,t'_k))$, then 
the map $(u_0,\ldots,u_n)$ has graph $M(\theta)$ where $\theta(t,t')$ 
is the formula `$\theta_0(t,t'_0) \wedge \cdots \wedge 
\theta_n(t,t'_n)$'.
\end{enumerate} \qed

\begin{lem} With the same hypotheses of the previous lemma: 
\begin{enumerate}
\item Composites of $M$-maps are $M$-maps. \item Let $f : A \to B$ be 
given in $\BC C$ with image $i : C \MON B$ and surjection factor $q : 
A \to C$. If $f$ is an $M$-map then so are $q$ and $i$.
\item Let $(g;q_\bullet) : Tx \to Ty$ be given in ${\BC C}_{T,\BC 
Q}$. Then $P(g;q_\bullet)$ is an $M$-map.
\end{enumerate}
\end{lem}

\NI {\bf Proof}: (1) Suppose $\theta_1$ is provably the graph of a 
function from $\phi_1(t)$ to $\phi_2(t')$ and $\theta_2$ is provably 
the graph of a function from $\phi_2(t')$ to $\phi_3(t'')$. Then 
$\theta_3$ is provably the graph of a function from $\phi_1(t)$ to 
$\phi_3(t'')$ where $\theta_3$ is the formula: `$\, \exists t' \, 
(\theta_1(t,t') \wedge \theta_2(t',t''))\,$'. 

(2) Suppose $\theta(t,t')$ is provably the graph of a function from 
$\phi(t)$ to $\psi(t')$ with $A=M(\phi(t))$, $B=M(\psi(t'))$ and 
$R=M(\theta(t,t'))$, where $1 \times f : R \MON A \times B$ is the 
graph of $f$. Then $C=M(\exists t \, \theta(t,t'))$. The graph of $q 
: A \to C$ is also $M(\theta)$ and the graph of $i : C \to B$ is 
$M(\bar \theta (t_1',t_2'))$ where $\bar \theta$ is the formula `$\, 
\exists t \, \theta(t,t_1') \wedge (t_1'=t_2')\,$'. 

(3) The map $(g;q_\bullet)$, where $g : [m] \to [n]$, factors as 
$(1;q_\bullet)(g;1_\bullet) : Tx \to Txg \to Ty$. (See section 
\ref{ORD_SOF}). $P(g;1_\bullet) =
(\mbox{pr}_{xg(0)},\ldots,\mbox{pr}_{xg(m)})$. Each component of 
$P(g;1_\bullet)$ is an $M$-map since it is a projection, hence so is 
$P(g;1_\bullet)$ by the previous lemma. Next, each component of 
$P(1;q_\bullet) : \prod_{k=0}^m Txg(k) \to \prod_{k=0}^m Ty(k)$ is 
$q_j\, \mbox{pr}_j$. This is an $M$-map, by part 1, since each factor 
is. Hence so is $P(1;q_\bullet)$ (previous lemma) and 
$P(g;q_\bullet)$ (part 1). \qed \medskip 

Given a model $M : \cal T \to \BC C$ as above with its \SOF\ $\BC A 
(M)$ (section \ref{MODEL_SOF}), we may define a subcategory $\AMC 
AMC$ as follows.

The objects of $\AMC AMC$ are all $\{A,Tx\} : A \MON P(Tx)$ which are 
$M$-maps of $\BC C$. The maps of $\AMC AMC$ are all 
$(u,(g;q_\bullet)) : \{A,Tx\} \to \{B,Ty\}$ such that $u$ is an 
$M$-map of $\BC C$.

It follows immediately that every formula $\phi$ of $\cal T$ gives 
rise to an object of $\AMC AMC$, namely $\{M(\phi), Tx \}$ where $x : 
[n] \to \kappa$ indexes the compound sort of the free variables of 
$\phi$. Of course, every object of $\AMC AMC$ is of this form. 

\begin{thm}
$\AMC AMC$ is a \SOF\ subcategory of $\BC A (M)$. \end{thm}

\NI {\bf Proof}: Define $\bar K$ to be $\AMC AMC \stackrel{{\rm full 
\, incl}}{\longrightarrow} \BC A (M) \stackrel{K}{\longrightarrow} 
\Delta^{\rm op}$. Given any map $(u,(g;q_\bullet)) : \{a,Tx \} \to 
\{B,Ty\}$ in $\AMC AMC$, where $g : [m] \to [n]$,
it suffices to show that $\lambda(\{A,Tx\},g) : \{A,Tx\} \to 
\{A^g,Txg\}$ and $\iota(u,(g;q_\bullet)) : \{A^g,Txg\} \to \{B,Ty\}$ 
are maps of $\AMC AMC$.

The map $(u,(g;q_\bullet))$ factors as $\iota(u,(g;q_\bullet)) 
\lambda(\{A,Tx\},g)$ as shown in the diagram below. \medskip 

\begin{center}
\begin{picture}(13,4)
\put(1,4){\makebox(0,0){$P(Tx)$}}
\put(1,2){\makebox(0,0){$A$}}
\put(5,4){\makebox(0,0){$P(Txg)$}}
\put(5,2){\makebox(0,0){$A^g$}}
\put(9,2){\makebox(0,0){$A''$}}
\put(12,4){\makebox(0,0){$P(Ty)$}}
\put(12,2){\makebox(0,0){$B$}}
\put(1.4,3){\makebox(0,0){$i_A$}}
\put(5.3,3){\makebox(0,0){$i^g$}}
\put(12.3,3){\makebox(0,0){$i_B$}}
%\put(2.8,4.3){\makebox(0,0){$P(\lambda(D,g))$}} 
%\put(8,4.3){\makebox(0,0){$P(\iota(f))$}} 
%\put(2.5,1.7){\makebox(0,0){$u^g$}}
\put(2.5,2.2){\makebox(0,0){{\footnotesize surj}}} 
\put(6.5,2.2){\makebox(0,0){{\footnotesize surj}}} 
\put(10.25,2.2){\makebox(0,0){{\footnotesize monic}}} 
\put(9.3,3){\makebox(0,0){{\footnotesize monic}}} 
\put(6,1.2){\makebox(0,0){$u$}}
\put(2,4){\vector(1,0){2}}
\put(6,4){\vector(1,0){5}}
\put(1.5,2){\vector(1,0){3}}
\put(5.5,2){\vector(1,0){3}}
\put(9.5,2){\vector(1,0){2.2}}
\put(1,2.5){\vector(0,1){1}}
\put(5,2.5){\vector(0,1){1}}
\put(12,2.5){\vector(0,1){1}}
\put(9.5,2.4){\vector(3,2){1.7}}
\put(12,1){\vector(0,1){.5}}
\put(1,1.5){\line(0,-1){.5}}
\put(1,1){\line(1,0){11}}
\end{picture}
\end{center}

\NI In this diagram, $A'' \MON P(Ty)$ is the image of $A^g \to P(Txg) 
\to P(Ty)$.

The composite $A \MON P(Tx) \to P(Txg)$ is an $M$-map since it is the 
composite of $M$-maps. Thus $A^g \in \mbox{ob } \AMC AMC$ and both $A 
\to A^g$ and $A^g \MON P(Txg)$ are $M$-maps. Therefore, 
$\lambda(\{A,Tx\},g)$ is a map in $\AMC AMC$ 

By similar reasoning, $A^g \MON P(Txg) \to P(Ty)$ is also an 
$M$-map.Thus both $A^g \to A''$ and $A'' \MON P(Ty)$ are $M$-maps. 
Now $A'' \MON B$ is also an $M$-map since it is the image of the 
$M$-map $u : A \to B$. Therefore $\iota(u,(g;q_\bullet))$ is also a 
map in $\AMC AMC $ \qed

\subsection{Simplicial structure}

Any \SOF\ $K : \BC A \to \Delta^{\rm op}$ determines a simplicial set 
structure on ob $\BC A$ in which $A \in \mbox{ob } \BC A$ is an 
$n$-simplex if $K(A)=n$. If $g : [m] \to [n]=K(A)$ then one has 
$\lambda(A,g) : A \to A^g$ where the $m$-simplex $A^g$ is the 
simplicial image of $A$ along $g$.

In particular, $d_i(A)=$ the $i$'th face of $A$ is $A^{\partial_i}$ 
where $\partial_i : [n-1] \to [n]$ is the map whose image omits $i$. 
The $i$'th degeneracy of $A$ is $s_i(A)=A^{\sigma_i}$ where $\sigma_i 
: [n+1] \to [n]$ is the map whose image repeats $i$. 

If the \SOF\ $\BC A$ arises from a model of a coherent first-order 
theory then there are particularly simple interpretations of faces 
and degeneracies.

Let $M : \cal T \to \BC C$ be a model and let $\BC A = \AMC AMC$ be 
the \SOF\ as defined in section \ref{SOF_ATM}. Suppose $\{A,Tx\}$ is 
an $n$-simplex where $A=M(\phi)$, the free variables of $\phi$ are 
$t_0,\ldots,t_n$ of sorts $\tau_0,\ldots,\tau_n$, and $x : [n] \to 
\kappa$ indexes these sorts ($Tx(k)=\tau_k$). Given
$\partial_i : [n-1] \to [n]$ as above, then $P(\partial_i;1_\bullet) 
: P(Tx) \to P(Tx\partial_i)$ is just the projection $\prod_{k=0}^n 
M(\tau_k) \to \prod_{k \neq i} M(\tau_k)$. The map 
$\lambda(\{A,Tx\},\partial_i) : \{A,Tx\} \to \{d_i(A),Tx\partial_i\}$ 
is

\begin{center}
\begin{picture}(6,3.3)
\put(1,3){\makebox(0,0){$A$}}
\put(3,3){\makebox(0,0){$P(Tx)$}}
\put(4.5,3){\makebox(0,0){$=$}}
\put(6,3){\makebox(0,0){$\prod_{k=0}^n M(\tau_k)$}} 
\put(1,1){\makebox(0,0){$d_i(A)$}}
\put(3,1){\makebox(0,0){$P(Tx\partial_i)$}} 
\put(4.5,1){\makebox(0,0){$=$}}
\put(6,1){\makebox(0,0){$\prod_{k \neq i}^n M(\tau_k)$}} 
\put(1,2.5){\vector(0,-1){1}}
\put(3,2.5){\vector(0,-1){1}}
\put(1.7,1){\vector(1,0){.5}}
\put(1.5,3){\vector(1,0){.8}}
\put(.5,2){$u$}
\put(3.3,2){$P(\partial_i;1_\bullet)$}

\end{picture}
\end{center}

\NI Thus $d_i(A) = d_i(M(\phi)) = M(\exists t_i \, \phi)$. 

Next, consider the degeneracy operator $\sigma_i : [n+1] \to [n]$. 
Since $(\partial_i;1_\bullet) (\sigma_i;1_\bullet) = 
\mbox{identity}$, then $P(\sigma_i;1_\bullet)$ is monic so that the 
image of $A \MON P(Tx) \MON P(Tx\sigma_i)$ is $A \MON P(Tx\sigma_i)$. 
According to the interpretation of the language of $\cal T$ in a 
logical category, as a subobject of $P(Tx \sigma_i)$, $A$ is 
$s_i(M(\phi)) = M(\phi \wedge (t_i'=t_i')) \MON P(Tx \sigma_i)$. 

\subsection{Functors relating models and \SOF s} 

Let $F : {\BC C}_1 \to {\BC C}_2$ be a ``logical'' functor (see 
below) between logical categories, and let $H : M_1 \to M_2$ be a 
homomorphism of models of $\cal T$ in ${\BC C}_1$. We will review how 
``composition of $M_1$ with $F$'' yields a model, denoted $FM_1$, of 
$\cal T$ in ${\BC C}_2$ and show how this determines an opcartesian 
functor of the \SOF s $\BC A (M_1) \to \BC A (FM_1)$ which restricts 
to $\AMC A{M_1}{C_1} \to \AMC A{FM_1}{C_2}$. 

We will also show how the homomorphism $H$ determines an opcartesian 
functor $\AMC A{M_1}C \to \AMC A{M_2}C$. Finally, we will apply these 
results to form a category $\mbox{Mod}_{\rm all}{\cal T}$ of all 
models of $\cal T$ in all logical categories and show that there is a 
functor $\mbox{Mod}_{\rm all}{\cal T} \to \mbox{SOF}({\Delta^{\rm 
op}})$, the category of \SOF s over $\Delta^{\rm op}$.

\subsubsection{Review of definitions}
\label{REVDEF}

\NI {\bf Logical functors, composition with a model}. Let $M : \cal T 
\to \BC C$ be a model and let $F : \BC C \to \BC C'$ be a functor of 
logical categories. Assume that $F$ is {\em logical} (\cite{MR}), 
that is: it preserves finite inverse limits, suprema of finite sets 
of subobjects, and images. Then evaluation along $F$ determines a 
model of $\cal T$ in $\BC C'$. The sort $\tau$ is interpreted as 
$F(M(\tau))$, the operation symbol $f : \tau \to \tau'$ is 
interpreted as $F(M(f)) : F(M(\tau)) \to F(M(\tau'))$, and the 
predicate symbol $R$ of sort $\tau$ is interpreted as $F(M(R)) \MON 
F(M(\tau))$. \medskip

\NI {\bf Homomorphisms of models}. Suppose $M_1$ and $M_2$ are models 
of $\cal T$ in $\BC C$. A {\em homomorphism} $H : M_1 \to M_2$ is a 
family of maps $H(\tau) : M_1(\tau) \to M_2(\tau)$ , (for each 
$\tau=$ a basic or compound sort of $\cal T$), satisfying the 
following commutativity properties.
\begin{enumerate}

\item For each operation symbol $f : \tau \to \tau'$, $H(\tau') 
M_1(f) = M_2(f) H(\tau)$.

\item For each predicate $R$ of sort $\tau$, there is a map $H(R) : 
M_1(R) \to M_2(R)$ such that $M_1(R) \MON M_1(\tau) 
\stackrel{H(\tau)}{\longrightarrow} M_2(\tau) = M_1(R) 
\stackrel{H(R)}{\longrightarrow} M_2(\phi) \MON M_2(\tau)$ 

\end{enumerate}

\NI It follows by a standard induction argument for the 
interpretation of well-formed formulas that if $H$ is a homomorphism 
then for each formula $\phi(t)$ where $t$ is of sort $\tau$, there is 
a map $H(\phi) : M_1(\phi) \to M_2(\phi)$ such that $M_1(\phi) \MON 
M_1(\tau) \stackrel{H(\tau)}{\longrightarrow} M_2(\tau) = M_1(\phi) 
\stackrel{H(\phi)}{\longrightarrow} M_2(\phi) \MON M_2(\tau)$. In 
effect, $H(\phi)$ is built up in a precise way from the $H(\tau)$'s 
and $H(R)$'s analogous to the interpretation of $\phi$ by $M$. 

Since the composition of $M$-maps and the equality of two $M$-maps 
can be expressed in the language of $\cal T$, then the commutativity 
of any finite diagram $\cal F : {\BC D} \to {\BC C}$ of $M$-maps can 
be expressed in the language of $\cal T$. It follows that if $H : M_1 
\to M_2$ is a homomorphism of models, and if $\cal F : {\BC D} \to 
{\BC C}$ is a finite diagram of $M$-maps in $\BC C$, then there is a 
corresponding finite diagram $\cal F' : {\BC D} \to {\BC C}$ of 
$M'$-maps obtained by applying $M'$ to the formulas for the objects 
and maps of the diagram $\cal F$. Since the formulas expressing the 
commutativity of the diagram $\cal F$ are theorems of $\cal T$, the 
diagram $\cal F'$ obtained via $M'$ has the corresponding 
commutativity. That is: $H$ determines a natural transformation $\cal 
F \to \cal F'$. 

In the simple case when the diagram is simply an $M$-map $u : M(\phi) 
\to M(\psi)$, with $u' : M'(\phi) \to M'(\psi)$ the corresponding 
$M'$-map, then there is a commutative square with $u' H(\phi) = 
H(\psi) u$.

Similarly, if one has a commutative square of $M$-maps then there is 
a corresponding commutative square of $M'$-maps. $H$ then yields a 
commutative cube diagram in $\BC C$. These remarks will apply to the 
proof of Theorem \ref{HOM_THM} below.\medskip 

\NI{\bf Opcartesian functors}. Finally, recall (\cite{BarrWells} 
\cite{Grothen}) that if $K : \BC A \to \BC X$ and $K' : \BC A' \to 
\BC X$ are \SOF s, then an {\em opcartesian functor} $F : \BC A \to 
\BC A'$ is a functor such that $K' F = K$ and for all $A \in \mbox{ob 
} \BC A$ and all $g : m \to n = K(A)$ in $\BC X$, $F(\lambda(A,g)) = 
\lambda(F(A),g) : F(A) \to F(A^g) = (F(A))^g$. 

\subsubsection{The opcartesian functor determined by a logical 
functor} \label{F_OF_FM}

Let $M: \cal T \to \BC C$ be a model and let $F : \BC C \to \BC C'$ 
be a logical functor. Let $M'=FM : \cal T \to \BC C'$ be the model 
resulting from composition with $F$ as just described. Given a 
$\kappa$-indexing of the sorts of $\cal T$ where $T : \kappa \to 
\mbox{ob }\BC C$ is the corresponding function (section 
\ref{MODEL_SOF}), define $T' : \kappa \to \mbox{ob }\BC C'$ to be 
$FT$. Let $\BC Q$ and $\BC Q'$ be the subcategories of $\BC C$ and 
$\BC C'$ respectively for $M$ and $M'$. Note that $F$ restricted to 
$\BC Q$ has values in $\BC Q'$.

\begin{thm} \label{FM_THM}
\begin{enumerate}
\item $F$ determines an opcartesian functor $F^+ : {\BC C}_{{\cal 
T},\BC Q} \to {\BC C'}_{{\cal T},\BC Q'}$. \item $F$ determines an 
opcartesian functor $F^M : \BC A (M) \to \BC A (M')$.
\item $F$ preserves $M$-maps and $F^M$ restricts to an opcartesian 
functor $\AMC AMC \to \AMC A{M'}{C'}$.
\end{enumerate}
\end{thm}

\NI {\bf Proof}:
\begin{enumerate}
\item The definition of $F^+$ is: $F^+(Tx) = T'x = FTx$ and if 
$(g;q_\bullet) : Tx \to Ty$ then
$F^+ (g;q_\bullet) = (g;F(q_\bullet))$. Now $\lambda(Tx,g) = 
(g;1_\bullet)$ and so $F^+(\lambda(Tx,g)) = (g;F(1_\bullet)) = 
\lambda(T'x,g)$. Thus $F^+$ is opcartesian. 

\item $F^M$ is defined by applying $F$ as shown in the following 
diagram.
\begin{center}
\begin{picture}(13,4)
\put(1,3){\makebox(0,0){$A$}}
\put(1,1){\makebox(0,0){$B$}}
\put(3,3){\makebox(0,0){$P(Tx)$}}
\put(3,1){\makebox(0,0){$P(Ty)$}}
\put(8,3){\makebox(0,0){$F(A)$}}
\put(8,1){\makebox(0,0){$F(B)$}}
\put(11,3){\makebox(0,0){$P(T'x)$}}
\put(11,1){\makebox(0,0){$P(T'y)$}}
\put(.5,2){\makebox(0,0){$u$}}
\put(2,3.5){\makebox(0,0){$i_A$}}
\put(2,.5){\makebox(0,0){$i_B$}}
\put(4,2){\makebox(0,0){$P(g;q_\bullet)$}} 
\put(7.3,2){\makebox(0,0){$F(u)$}}
\put(12.5,2){\makebox(0,0){$P(F^+(g;q_\bullet))$}} 
\put(9.5,3.5){\makebox(0,0){$F(i_A)$}}
\put(9.5,.5){\makebox(0,0){$F(i_B)$}}
\put(1,2.5){\vector(0,-1){1}}
\put(3,2.5){\vector(0,-1){1}}
\put(8,2.5){\vector(0,-1){1}}
\put(11,2.5){\vector(0,-1){1}}
\put(1.5,3){\vector(1,0){.8}}
\put(1.5,1){\vector(1,0){.8}}
\put(8.7,3){\vector(1,0){1.5}}
\put(8.7,1){\vector(1,0){1.5}}
\put(5,2){\vector(1,0){1}}
\put(5,1.8){\line(0,1){.4}}
\end{picture}
\end{center}

\NI Recall that $\lambda(\{A,Tx\},g) = (v,(g;1_\bullet)) : \{A,Tx\} 
\to \{A^g,Txg\}$ where $i^g : A^g \MON P(Txg)$ is the image of $A 
\stackrel{i_a}{\MON} P(Tx) \to P(Txg)$ and $v : A \to A^g$ is the 
corresponding surjection. Thus $F^M(\lambda (\{A,Tx\},g)) = 
(F(v),F^M((g;1_\bullet))) = (F(v),(g;F(1_\bullet)))$ which is 
$\lambda(\{F(A),T'x\},g)$. Thus $F^M$ is opcartesian. 

\item If $f : A \to B$ in $\BC C$ is an $M$-map with graph $1 \times 
f : R \MON A \times B$ where
$A=M(\phi(t))$, $B=M(\psi(t'))$ and $R = M(\theta(t,t'))$ then $F(1 
\times f) = 1 \times F(f) : F(R) \MON F(A) \times F(B)$ which is 
$M'(\theta(t,t')) \MON M'(\phi(t)) \times M'(\psi(t'))$. That is, 
$F(f)$ is an $M'$-map.

If $(u,(g;q_\bullet)) : \{A,Tx\} \to \{B,Ty\}$ is a map in $\AMC AMC$

\begin{center}
\begin{picture}(5,4)
\put(1,3){\makebox(0,0){$A$}}
\put(1,1){\makebox(0,0){$B$}}
\put(3,3){\makebox(0,0){$P(Tx)$}}
\put(3,1){\makebox(0,0){$P(Ty)$}}
\put(.5,2){\makebox(0,0){$u$}}
\put(2,3.5){\makebox(0,0){$i_A$}}
\put(2,.5){\makebox(0,0){$i_B$}}
\put(4,2){\makebox(0,0){$P(g;q_\bullet)$}} 
\put(1,2.5){\vector(0,-1){1}}
\put(3,2.5){\vector(0,-1){1}}
\put(1.5,3){\vector(1,0){.8}}
\put(1.5,1){\vector(1,0){.8}}
\end{picture}
\end{center}

\NI then all the maps are $M$-maps and the image of this diagram 
under $F$ is $F^M(u,(g;q_\bullet))$ which consists of $M'$-maps. Thus 
$F^M$ restricts to an opcartesian functor $\AMC AMC \to \AMC A{M'}{C' 
}$. \qed

\end{enumerate}

\subsubsection{The opcartesian functor determined by a homomorphism 
of models} \label{F_OF_H}

Let $H : M \to M'$ be a homomorphism of models of $\cal T$ in the 
logical category $\BC C$. Given a $\kappa$-indexing of the sorts of 
$\cal T$, we have two functions $T, T' : \kappa \to \mbox{ob } \BC C$ 
where, for each $i < \kappa$, $T(i)=M(i \mbox{'th sort indexed})$ and 
$T'(i)=M'(i \mbox{'th sort indexed})$. $M$ and $M'$ determine the 
subcategories $\BC Q$ and $\BC Q'$ as described in section 
\ref{MODEL_SOF}.

\begin{thm}
\label{HOM_THM}
The homomorphism $H : M \to M'$ determines an opcartesian functor 
$F^H : \AMC AMC \to \AMC A{M'}C$.
\end{thm}

\NI {\bf Proof}:
First we define a functor $H^T : {\BC C}_{{\cal T},{\BC Q }} \to {\BC 
C}_{{\cal T},{\BC Q'}}$. Set $H^T(Tx) = T'x$. Given the map 
$(g;q_\bullet) : Tx \to Ty$ with $g : [m] \to [n]$ in $\Delta$, then 
$H^T(g;q_\bullet) = (g;q'_\bullet) : T'x \to T'y$ where for each 
$i=0,\ldots,m$, we must define $q'_i : T'xg(i) \to T'y(i)$. 

Let $\tau_{xg(i)}$ and $\tau_{y(i)}$ denote the sorts indexed by 
$xg(i)$ and $y(i)$ respectively. By definition, 
$Txg(i)=M(\tau_{xg(i)})$, $Ty(i)=M(\tau_{y(i)})$ and $q_i : Txg(i) 
\to Ty(i)$ is an $M$-map whose graph is $M(\theta_i ) \MON 
M(\tau_{xg(i)}) \times M(\tau_{y(i)})$ for some formula $\theta_i$. 
We define $q_i'$ to be the function whose graph is $M'(\theta_i ) 
\MON M'(\tau_{xg(i)}) \times M'(\tau_{y(i)})$. It follows from this 
definition that the following diagram commutes for each 
$i=0,\ldots,m$.

\begin{center}
\begin{picture}(6,3.6)
%\put(){\makebox(0,0){$$}}

\put(2,1){\makebox(0,0){$M'(\tau_{xg(i)})$}} 
\put(2,3){\makebox(0,0){$M(\tau_{xg(i)})$}} 
\put(5,1){\makebox(0,0){$M'(\tau_{y(i)})$}} 
\put(5,3){\makebox(0,0){$M(\tau_{y(i)})$}} 
\put(3.5,3.3){\makebox(0,0){$q_i$}}
\put(3.5,.5){\makebox(0,0){$q_i'$}}
\put(0.8,2){\makebox(0,0){$H(\tau_{xg(i)})$}} 
\put(6,2){\makebox(0,0){$H(\tau_{y(i)})$}} 
\put(2,2.5){\vector(0,-1){1}}
\put(5,2.5){\vector(0,-1){1}}
\put(3,1){\vector(1,0){1}}
\put(3,3){\vector(1,0){1}}
\end{picture}
\end{center}

Next we define $F^H$. Let $\{A,Tx\} = i_A : A \MON P(Tx)$ 
be an object of $\AMC AMC$
where $A=M(\phi(t))$, $t$ is the compound sort $(t_0,\ldots,t_n)$ 
with $t_i$ of sort $\tau_{x(i)}$ and $i_A$ is an $M$-map. Then $H$ 
being a homomorphism implies the following diagram commutes. 

\begin{center}
\begin{picture}(6,3.6)
%\put(){\makebox(0,0){$$}}
\put(2,1){\makebox(0,0){$M'(\phi(t))$}}
\put(2,3){\makebox(0,0){$M(\phi(t))$}}
\put(5,1){\makebox(0,0){$\prod M'(\tau_{x(i)})$}} 
\put(5,3){\makebox(0,0){$\prod M(\tau_{x(i)})$}} 
\put(3.5,3.3){\makebox(0,0){\small monic}} 
\put(3.5,.5){\makebox(0,0){\small monic}} 
%\put(0.8,2){\makebox(0,0){$H(\tau_{xg(i)})$}} 
\put(6,2){\makebox(0,0){$ \prod H(\tau_{x(i)})$}} 
\put(2,2.5){\vector(0,-1){1}}
\put(5,2.5){\vector(0,-1){1}}
\put(3,1){\vector(1,0){1}}
\put(3,3){\vector(1,0){1}}
\end{picture}
\end{center}

\NI Define $F^H(\{A,Tx\})$ to be the monic $M'(\phi(t)) \MON \prod 
M'(\tau_{x(i)}) = i_{A'}:A' \MON T'x$. The earlier observation about 
homomorphisms in section \ref{REVDEF} establishes the functoriality 
of $F^H$.

Finally, to show that $F^H$ is opcartesian, we have to show that 
$F^H(\lambda(\{A,Tx\},g)) = \lambda(F^H\{A,Tx\},g)$. Now 
$\lambda(\{A,Tx\},g)$ is the commutative square of $M$-maps 

\begin{center}
\begin{picture}(5,3.5)
%\put(){\makebox(0,0){$$}}
\put(1,3){\makebox(0,0){$A$}}
\put(1,1){\makebox(0,0){$A^g$}}
\put(4,3){\makebox(0,0){$P(Tx)$}}
\put(4,1){\makebox(0,0){$P(Txg)$}}
\put(1,2.5){\vector(0,-1){1}}
\put(4,2.5){\vector(0,-1){1}}
\put(1.5,1){\vector(1,0){1.7}}
\put(1.5,3){\vector(1,0){1.7}}
\put(1.5,2){\makebox(0,0){\small surj}}
\put(4.5,2){\makebox(0,0){\small proj}}
\put(2.5,3.3){\makebox(0,0){\small monic}} 
\put(2.5,.5){\makebox(0,0){\small monic}} \end{picture}
\end{center}

\NI where if $A=M(\phi)$ then $A^g = M(\exists t^{\neg g} \, \phi)$ 
where $t^{\neg g}$ is the list of those free variables $t_k$ where $0 
\le k \le n$ and $k$ is not in the image of $g : [m] \to [n]$.

Similarly, $\lambda(F^H\{A,Tx\},g)$ is the commutative square of 
$M'$-maps

\begin{center}
\begin{picture}(5,3.5)
\put(1,3){\makebox(0,0){$A'$}}
\put(1,1){\makebox(0,0){${(A')}^g$}}
\put(4,3){\makebox(0,0){$P(T'x)$}}
\put(4,1){\makebox(0,0){$P(T'xg)$}}
\put(1,2.5){\vector(0,-1){1}}
\put(4,2.5){\vector(0,-1){1}}
\put(1.5,1){\vector(1,0){1.7}}
\put(1.5,3){\vector(1,0){1.7}}
\put(1.5,2){\makebox(0,0){\small surj}}
\put(4.5,2){\makebox(0,0){\small proj}}
\put(2.5,3.3){\makebox(0,0){\small monic}} 
\put(2.5,.5){\makebox(0,0){\small monic}} \end{picture}
\end{center}

\NI where $A'=M'(\phi)$ and ${(A')}^g = M'(\exists t^{\neg g} \, 
\phi) = (A^g)'$. Thus $F^H(\lambda(\{A,Tx\},g)) = 
\lambda(F^H\{A,Tx\},g)$. \qed

\subsubsection{The category of all models of $\cal T$} 

Fix a coherent first-order theory $\cal T$. We will denote each model 
$M$ of $\cal T$ in the logical category $\BC C$ by $(M,{\BC C})$. The 
class of all models of $\cal T$ in all small logical categories is 
the class of objects of a category denoted $\mbox{Mod}_{\rm all} \cal 
T$. A map $(H,F) : (M,{\BC C}) \to (M',{\BC C'})$ consists of a 
logical functor $F : {\BC C} \to {\BC C'}$ and a homomorphism of 
models $H : FM \to M'$. ($FM$ is the model $\cal T \stackrel{M }{\to} 
{\BC C} \stackrel{F}{\to} {\BC C'}$ defined in section \ref{F_OF_FM}).

\begin{center}
\begin{picture}(5,2.5)
\setlength{\unitlength}{1mm}
%\put(){\makebox(0,0){$$}}
\put(0,20){\makebox(0,0){$\BC C$}}
\put(20,0){\makebox(0,0){$\cal T$}}
\put(40,20){\makebox(0,0){$\BC C'$}}
\put(5,5){\makebox(0,0){$M$}}
\put(35,5){\makebox(0,0){$M'$}}
\put(20,13){\makebox(0,0){$H$}}
\put(20,23){\makebox(0,0){$F$}}
\put(5,20){\vector(1,0){30}}
\put(17,5){\vector(-1,1){12}}
\put(23,5){\vector(1,1){12}}
\put(15,10){\vector(1,0){10}}
\end{picture}
\end{center}

The composite $(M,{\BC C}) \stackrel{(H,F)}{\to} (M',{\BC C'}) 
\stackrel{(H',F')}{\to} (M'',{\BC C''})$ is $(H'',F'F)$ where for 
each sort $\tau$, $H''(\tau)= F'F(M(\tau)) \stackrel{F'(H(\tau))}{\to 
} F'(M'(\tau)) \stackrel{H'(\tau)}{\to} M''(\tau)$. For each 
operation symbol $f$ and each predicate $R$, the maps $H''(f)$ and 
$H''(R)$ are defined similarly.\medskip

\NI {\bf Remark}: Let ${\mbox{\bf Cat}}_{\rm log}$ denote the 
category of all logical categories and logical functors. The functor 
$\mbox{Mod}_{\rm all} \cal T \to {\mbox{\bf Cat}}_{\rm log}$ defined 
by $(M,{\BC C}) \mapsto {\BC C}$ and $(H,F) \mapsto F$ is a \SOF\ 
where $\lambda((M,{\BC C}),F) = (1,F) : (M,{\BC C}) \to (FM,{\BC 
C'})$ and $\iota(H,F) = (H,1_{\BC C}) : (FM,{\BC C'}) \to (M'{\BC 
C'})$. \medskip

Theorems \ref{FM_THM} and \ref{HOM_THM} combine to establish the 
following result.

\begin{thm} There is a faithful functor $W : \mbox{Mod}_{\rm all} 
{\cal T} \to \mbox{SOF}(\Delta^{\rm op})$ where $W(M,{\BC C}) = 
K_{M,{\BC C}} : {\AMC AMC} \to \Delta^{\rm op}$. If $(H,F) : (M,{\BC 
C}) \to (M',{\BC C'})$ is a map of models then the opcartesian 
functor for $W(H,F)$ is ${\AMC AMC} \stackrel{F^M}{\to} {\AMC A {FM} 
{C'}} \stackrel{F^H}{\to} {\AMC A {M'} {C'}}$. \qed
\end{thm}

\subsubsection{A ``purely categorical'' result} 

There is a ``purely categorical'' theorem which shadows the previous 
theorem. It is based on the coherent first order theory ${\cal 
T}({\BC C})$ determined by a small logical category $\BC C$. Recall 
(from \ref{CFOT}) that the sorts of ${\cal T}({\BC C})$ are the 
objects of $\BC C$ and the operation symbols of ${\cal T}({\BC C})$ 
are the maps of $\BC C$. The canonical model $M_0 : {\cal T}({\BC C 
}) \to {\BC C}$ interprets each sort $C$ as the object $C$, each 
operation symbol $f : C \to C'$ as the map $f$, each predicate $R 
\MON C_0 \times \cdots \times C_n$ as the subobject $R$ and so on. 

Given an ordinal $\kappa$ and a function $T : \kappa \to \mbox{ob 
}{\BC C}$, there is an associated \SOF\ defined as follows. One 
defines the category ${\BC C}_T$ to be ${\BC C}_{T,{\BC Q}}$ as in 
section \ref{ORD_SOF} by taking ${\BC Q} = {\BC C}$. We define the 
functor $P : {\BC C}_T \to {\BC C}$ as before: $P(Tx) = \prod_{i=0}^n 
Tx(i)$ if $\mbox{dom}(x)=[n]$. This functor together with the \SOF\ 
$K : {\BC C}_T \to \Delta^{\rm op}$ (theorem \ref{CTQ_SOF}) define a 
\SOF\ of relations $\ACT ACT$ as in section \ref{REL_SOF}. 

Each logical functor $F : {\BC C} \to {\BC C'}$ yields a model $FM_0 
: {\cal T}({\BC C}) \to {\BC C'}$ obtained from the canonical model 
$M_0 : {\cal T}({\BC C }) \to {\BC C}$ by composition with $F$. By 
theorem \ref{FM_THM}, there is an opcartesian functor ${\ACT ACT} \to 
{\ACT A{C'}{T'}}$ where $T' : \kappa \to \mbox{ob }{\BC C'}$ is $FT : 
\kappa \to \mbox{ob }{\BC C} \to \mbox{ob }{\BC C'}$.

Let us call a logical category together with a function $T : \kappa 
\to \mbox{ob }{\BC C}$ an {\em ordinal-mapped logical category} and 
denote it $({\BC C},T)$. A morphism $({\BC C},T) \to ({\BC C'},T')$ 
is a logical functor $F : {\BC C} \to {\BC C'}$ where $T'=FT$. 

Let ${\mbox{Cat}}^{\rm ind}_{\rm log}$ be the category of all 
ordinal-mapped logical categories and logical functors. 

We then obtain:

\begin{thm} There is a faithful functor ${\mbox{Cat}}^{\rm ind}_{\rm 
log} \to {SOF}(\Delta^{\rm op})$ \qed \end{thm} 


\bibliographystyle{plain}
%\bibliography{abibl}
\begin{thebibliography}{1}

\bibitem{BarrWells}
Michael Barr and Charles Wells.
\newblock {\em Category Theory for Computing Science}. \newblock Prentice-Hall, New York, 1990. 

\bibitem{Borceux}
Francis Borceux.
\newblock {\em A Handbook of Categorical Algebra. Volume 1. Basic Category 
Theory}.
\newblock Cambridge University Press, Cambridge, 1994. 

\bibitem{FS}
Peter Freyd and Andre Scedrov.
\newblock {\em Categories, Allegories}.
\newblock North-Holland, Amsterdam - New York, 1990. 

\bibitem{regcats}
Pierre~Antoine Grillet.
\newblock Regular categories.
\newblock In Michael Barr, P.~A. Grillet, and D.~H.~Van Osdol, editors, {\em 
Exact Categories and Categories of sheaves}, Lecture Notes in Mathematics no. 236. Springer-Verlag, New York, 1971.

\bibitem{Grothen}
Alexandre Grothendieck.
\newblock Cat\'{e}gories fibr\'{e}es et descente. \newblock In Alexandre Grothendieck and M.~Reynaud, editors, {\em 
Rev\^{e}tements \'{E}tales et Groupe Fondamental}, Lecture Notes in Mathematics no. 224. Springer-Verlag, New York, 1971. 

\bibitem{MR}
Michael Makkai and Gonzalo~E. Reyes.
\newblock {\em First Order Categorical Logic}. \newblock Lecture Notes in Mathematics no. 611. Springer-Verlag, New York, 
1977.

\end{thebibliography}

\end{document}