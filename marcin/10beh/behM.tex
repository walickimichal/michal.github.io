\documentstyle[10pt]{article}
%\documentstyle[a4,fullpage,12pt,oldlfont]{article}

\makeatletter
\input{a4wide}
\makeatother


\newtheorem{example}{Example}[section]
\newtheorem{definition}[example]{Definition}
\newtheorem{theorem}[example]{Theorem}
\newtheorem{lemma}[example]{Lemma}
\newtheorem{fact}[example]{Fact}
\newtheorem{proposition}[example]{Proposition}
\newtheorem{corollary}[example]{Corollary}

%equalities
\newcommand{\eequal}{\stackrel{e}{=}}%existential equality
\newcommand{\sequal}{\stackrel{s}{=}}%strong equality
\newcommand{\wequal}{\stackrel{w}{=}}%weak equality
\newcommand{\eleq}{\mathrel{\dot{=} }}%element equality
\newcommand{\seq}{\mathrel{\asymp}}% set equality 

\newcommand{\exist}{\downarrow}% defindnes in partial algebra
\newcommand{\st}{\ast}% extra star element


%different Categorical, arrows...
\newcommand{\embd}{\hookrightarrow}% embedding of institutions 
\newcommand{\pfunc}{\hookrightarrow}% partial function
\newcommand{\ring}{\circ}% composition of arrows

\newlength{\listskip}\setlength{\listskip}{1ex plus .2ex minus .5ex}
\newlength{\eqpskip}\setlength{\eqpskip}{.5ex plus .2ex minus .2ex}
\newenvironment{eqp}{\par %\topsep\eqpskip \parskip 0ex
   \begin{tabbing}
   \quad\=\+\quad\=$\{$\ \=\kill\(}{\)\end{tabbing}
   \addvspace{-1ex plus.2ex minus.3ex}} 
\newcommand{\comment}[2]{\)\-\\[\eqpskip]$#1$\>\>$\{$\>\+\+\ignorespaces#2$\}$\-\\[\eqpskip]\(}
\newcommand{\nline}{\)\\\(\mbox{}}

%%%%%%% MICHAL

%%% comments to be removed eventually
\newcommand{\fix}{\par\noindent\hspace*{-2em}{\large\bf fix} $\uparrow$ \dotfill}
\newcommand{\fixx}[1]{\par\noindent\hspace*{-2em}{\large\bf fix} $\uparrow$ {\small
#1} \dotfill}
\newcommand{\fixd}[1]{\par\noindent\hspace*{-2em}{\large\bf fix} $\downarrow$ {\small
#1} \dotfill}
\newcommand{\isit}[1]{\par\noindent\hspace*{-2em}{\small{{\bf [is it right?} #1{\bf ]}}}\par\noindent}
\newcommand{\how}[1]{\par\noindent\hspace*{-2em}{\small{{\bf [how?} #1{\bf
]}}}\par\noindent}
\newcommand{\todo}[1]{\par\noindent\hspace*{-2em}{\small{{\bf [To Do:} #1{\bf ]}}}\par\noindent}
\newcommand{\noo}[1]{\par\noindent\hspace*{-2em}{\large\bf NO} $\uparrow$ {\small
#1} \dotfill}
%%% environments
\newtheorem{schem}[example]{Schema}
\newcommand{\MyLPar}{\parsep -.2ex plus.2ex minus.2ex\itemsep\parsep
   \vspace{-\topsep}\vspace{.5ex}}
\newenvironment{PROOF}{{\bf Proof.}}{\nopagebreak\finish}
\newenvironment{PROOFs}{{\bf Proof (sketch)}}{\nopagebreak\finish}
\newcommand{\finish}{\hspace*{\fill}\nopagebreak 
     \raisebox{-1pt}{$\Box$}\par\addvspace{1.5ex}\noindent}

%%% Specs:
\newcommand{\spec}[1]{\begin{array}[t]{rrl}#1\end{array}\vspace*{1ex}}
\newcommand{\tit}[1]{\multicolumn{3}{l}{#1}}

%%% symbols
\newcommand{\eeq}{\eequal}
\newcommand{\PSet}[1]{{\mathcal{P}}(#1)}
\newcommand{\ovr}[1]{\overline{#1}}
\newcommand{\To}{\Rightarrow}
\newcommand{\Tod}{\Leftarrow}
\newcommand{\Iff}{\Longleftrightarrow}
\newcommand{\hviss}{\Iff}
\newcommand{\ok}[1]{d_{#1}} % OK part of a sort #1

\newcommand{\by}[2]{\stackrel{#1}{#2}}
\newcommand{\Toby}[1]{\ \by{#1}{\Longrightarrow}\ }

\newcommand{\adj}{\mathrel{\small\dashv}} %adjunction


%%% use these for signatures
\newcommand{\Sorts}{{S}} 
\newcommand{\Ops}{\Omega}
\newcommand{\ndc}{\Pi}
\newcommand{\err}{E}
\newcommand{\sign}{(\Sorts,\Ops,\ndc)}
\newcommand{\POps}{P\Omega}
\newcommand{\Pops}{\POps}
\newcommand{\Terms}[1]{T(#1)}
\newcommand{\TermsS}{\Terms\Sigma}
\newcommand{\TermsSX}{\Terms{\Sigma,X}}
\newcommand{\dom}{{\bf dom}}

%%% general categorical concepts
\newcommand{\cat}[1]{{\bf #1}} %put it around any category 
\newcommand{\inst}[1]{{\mathcal{#1}}} %put it around any institution: needs $_$
\newcommand{\fu}[1]{{\sf {#1}}} %put it around any functor
\newcommand{\thr}[1]{{\bf #1}} %put it around any Theory
\newcommand{\obj}[1]{|#1|} %objects of a category
\newcommand{\natt}{\mathrel{\Longrightarrow}} %natural transformation
%%% Categories
%multialgebras
\newcommand{\MA}[1]{\cat{MAlg}_{#1}}
\newcommand{\MAS}{\cat{MAlg}_{\Sigma}} %partial multialgebras
\newcommand{\PMA}[1]{\cat{PMAlg}_{#1}}
\newcommand{\PMAS}{\cat{PMAlg}_{\Sigma}} %partial algebras
\newcommand{\PAl}[1]{\cat{PAlg}_{#1}}
%used little but still...
\newcommand{\PAlo}{\PAl{(\Sorts,\Ops)}}
\newcommand{\MAlo}{\MA{(\Sorts,\Ops,\emptyset)}}
%standard
\newcommand{\Sign}{\cat{Sign}}
\newcommand{\Set}{\cat{Set}}

%%% Functors
\newcommand{\Mod}{\fu{Mod}}
\newcommand{\Sen}{\fu{Sen}}
\newcommand{\sen}{\Sen}

%%% Institutions
\newcommand{\MAH}{\inst{MAH}}
\newcommand{\MAC}{\inst{MAC}}
%\newcommand{\MAPC}{\inst{MAPC}}
\newcommand{\MAP}{\inst{MAP}}%partial multialgebras
\newcommand{\PA}{\inst{PA}}%partial algebras

%%% Equivalences 
\newcommand{\quot}{\sim} % ekvivalens tegn
\newcommand{\kernel}[1]{\quot_{#1}} %% ekvivalens med subskript
\newcommand{\qu}[2]{#1/\!_{#2}}

%%% some special functions used in specifications
\newcommand{\ite}[3]{\mathit{if}\ #1\ \mathit{then}\ #2\ \mathit{else}\ #3}%if then else
% and
\newcommand{\band}{\mathrel{\mathit{and}}}% ``and''
\newcommand{\choice}{\sqcup} % Nondeterministic choice



% arbitrary institution
%% \newcommand{\Mod}{\mbox{\rm\bf Mod}}
%% \newcommand{\Sign}{\mbox{\rm\bf Sign}}
%% \newcommand{\Sen}{\mbox{\rm\bf Sen}}
%% \newcommand{\Set}{\mbox{\rm\bf Set}}
%% \newcommand{\Cat}{\mbox{\rm\bf Cat}}

% particular institutions
%% \newcommand{\MAlg}{\mbox{\rm\bf MAlg}}
%% \newcommand{\Alg}{\mbox{\rm\bf Alg}}
%% \newcommand{\AlgSig}{\mbox{\rm\bf AlgSig}}

% specifications

%% \newcommand{\SMod}{\mbox{\sf Mod}}
%% \newcommand{\SSig}{\mbox{\sf Sig}}
%% \newcommand{\SP}{\mbox{SP}}
\newcommand{\abs}[2]{{\rm \bf abstract}\ \mbox{#1}\ {\rm \bf wrt}\ #2}
\newcommand{\beh}[2]{{\rm \bf behaviour}\ \mbox{#1}\ {\rm \bf wrt}\ #2}
\newcommand{\deter}[1]{{\rm \bf determinize}\ \mbox{#1}}
\newcommand{\derive}[2]{{\rm \bf derive}\ \mbox{#1}\ {\rm \bf by}\ #2}
\newcommand{\der}[2]{{\rm \bf derive}\ \mbox{#1}\ {\rm \bf by}\ #2}
\newcommand{\trans}[2]{{\rm \bf translate}\ \mbox{#1}\ {\rm \bf along}\ #2}
\newcommand{\unite}[2]{{\rm \bf unite}\ \mbox{#1}\ {\rm \bf and}\ \mbox{#2}}

%
\newcommand{\Sort}{\mbox{\it Sorts}}
\newcommand{\Op}{\mbox{\it Ops}}

%--------------------------------------------------------------------
% symbols

\newcommand{\simH}{\by\sim H}
\newcommand{\simV}{\by\sim F}

%% \newcommand{\Det}{\mbox{\sf Det}}
%% \newcommand{\CDet}{\mbox{\sf CDet}}
\newcommand{\Beh}{\mbox{Beh}}
\newcommand{\Abs}{\mbox{Abs}}
%% \newcommand{\Mquo}{\mbox{$/\!\!/$}}
\newcommand{\Eq}{\mbox{Eq}}
\newcommand{\imp}{\mbox{$\sim\!\leadsto$}}
\newcommand{\val}[1]{\mbox{$\overline{#1}$}}
\newcommand{\Obs}{\rm Obs}
\newcommand{\nlet}[3]{[{\sf let}\ #1 := #2\ {\sf in}\ #3\ {\sf ni}]}
\newcommand{\ra}{\rightarrow}
\newcommand{\lc}{\langle}
\newcommand{\rc}{\rangle}
\newcommand{\ec}{\sim_c}
\newcommand{\er}{\sim_r}
\newcommand{\ep}{\sim_p}
\newcommand{\obs}{\equiv_{\Obs}}
\newcommand{\sqs}{\sqsubseteq}
\newcommand{\close}[1]{|\!|#1|\!|}
\newcommand{\smodels}{\:|\!\!\!\approx}
\newcommand{\kap}{\kappa}

%--------------------------------------------------------------------
% pretty definitions

\newcommand{\givedef}[2]
            {\begin{tabular}{p{1.25cm}p{0.35cm}p{13.75cm}}
                    #1 & {\rm iff} & #2 
             \end{tabular} \\[2mm] }

%--------------------------------------------------------------------
% environments

%% \newenvironment{Proof}
%%                {\setlength{\parskip}{5pt}
%%                 \setlength{\parindent}{0pt}
%%                 {\sc Proof:}}{\hfill{\sc qed}$\:$\raisebox{-1pt}{$\Box$}}

\newcommand{\Dybdy}[1]{\begin{Definition} \ \\
                       \indent {\rm #1 } \hfill$\Box$ 
                       \end{Definition} }

\newcommand{\support}{This research was supported by the Polish
Research Committee and by the Norwegian Research Council.}

\input{xypic}

%--------------------------------------------------------------------
% the document

\begin{document} 

\

\vspace{2cm}

%% \begin{center}
%% {\Large\bf Behavioural Satisfaction and Equivalence \\
%%         in Nondeterministic Algebras\footnote{\support} }\\[0.8cm] 
%% {\large Marcin Bia\l{}asik }\\[0.4cm] 
%%         Institute of Computer Science \\
%%         Polish Academy of Sciences \\
%%         ul.\ Ordona 21, Warsaw, Poland \\
%%         {\tt marcinb@ipipan.waw.pl} \\
\today 
%% \end{center}
%% \medskip

%==============================================================================
%


\section{Introduction}

In this paper we present some results concerning behavioural
equivalence and satisfaction in multialgebras.  One of the principal
claims made in the introduction is that multialgebras should be viewed
as primary models of a specification. This necessitates for a
behavioural abstraction at the nondeterministic level. We must however
keep in mind that our final goal is a deterministic model and hence
the abstracting equivalence relation on multialgebras should
correspond to some equivalence on deterministic algebras.

The paper is organised as follows. In Section 2 we present basic
notions related to algebraic specification and program
development. This covers the deterministic case only. In Section 3 we
introduce nondeterministic algebras (multialgebras). Interpretation of
terms in such algebras and logics to specify the multialgebras are
discussed in Section 4. Section 5 deals with algebraic specifications
that allow for nondeterministic models and introduces anew such
notions as abstracting (behavioural) equivalence and implementation.
Section 6 is entirely devoted to an example of program
development. Results concerning the abstracting equivalence and
behavioural satisfaction are discussed in Section 7.

\section{Algebras and algebraic specifications}

An {\em algebraic signature} $\Sigma$ is a pair $[S,F]$ where $S$ is a
set of sort names and $F$ is a family of sets $\{F_{w,s}\}_{w \in
S^\ast, s \in S}$ of sort names. We write $[f : w \rightarrow s] \in
F$ to denote $w \in S^\ast, s \in S$ and $f \in F_{w,s}$. An {\em
algebraic signature morphism} $\sigma : [S,F] \rightarrow [S',F']$ is
a pair $[\sigma_S,\sigma_F]$ where $\sigma_S : S \rightarrow S'$ and
$\sigma_F$ is a family of maps $\{\sigma_{w,s} : F_{w,s} \rightarrow
F'_{\sigma^\ast_S(w),\sigma_S(s)}\}_{w \in S^\ast, s \in S}$ where
$\sigma^\ast_S(s_1,\ldots,s_n)$ denotes
$\sigma_S(s_1),\ldots,\sigma_S(s_n)$. We will write $\sigma(s)$ for
$\sigma_S(s)$, $\sigma(w)$ for $\sigma^\ast_S(w)$ and $\sigma(f)$ for
$\sigma_{w,s}(f)$ where $f \in F_{w,s}$.  The category of all
algebraic signatures will be denoted \AlgSig. It has algebraic
signatures as objects and algebraic signature morphisms as morphisms.

Let $\Sigma = [S,F]$ be an algebraic signature. A (deterministic)
$\Sigma$-algebra $A$ consists of an $S$-indexed family of carrier sets
$\{A_s\}_{s \in S}$ and for each $[f : s_1 \times \ldots \times s_n
\rightarrow s] \in F$ a function $f^A : A_{s_1} \times \ldots \times
A_{s_n} \rightarrow A_s$.  A $\Sigma$-homomorphism $\phi$ from a
$\Sigma$-algebra $A$ to a $\Sigma$-algebra $B$ is a family of mappings
$\{\phi_s\}_{s \in S}, \phi_s: A_s \rightarrow B_s$ such that for all
$[f:s_1 \times s_2 \times \ldots \times s_n \rightarrow s] \in F$ and
for all $a_i \in A_{s_i}, i \in \{1,\ldots,n\}$ the following
condition holds:
\[ 
\phi_s(f^A(a_1,\ldots,a_n)) =
f^B(\phi_{s_1}(a_1),\ldots,\phi_{s_n}(a_n))
\]

\[
\begin{array}{lcl}
   \SSig[\lc\Sigma,\Psi\rc] & = & \Sigma \\
   \SMod[\lc\Sigma,\Psi\rc] & = & \{ A \in \MAlg(\Sigma): 
                                     A \models \Psi\}
\end{array}
\]

Flat specifications, however, do not allow to structure our
specifications -- the resulting one is a lengthy list of axioms
difficult to manage and to reason with. Also, it requires us to work
over a single signature.

The following specification-building operations were proposed in the
literature (for an overview see XXX or XXX):

\begin{itemize}
\item Union of $\Sigma$-specifications. \\
$
\begin{array}{lcl}
   \SSig[\unite{\SP}{\SP'}] & = & \Sigma \\
   \SMod[\lc\Sigma,\Psi\rc] & = & \SMod[\SP] \cap \SMod[\SP']
\end{array}
$

\item Translation of a $\Sigma$-specification $\SP$ along some
signature morphism $\sigma : \Sigma \ra \Sigma'$. \\ 
$
\begin{array}{lcl}
   \SSig[\trans{\SP}{\sigma}] & = & \Sigma' \\
   \SMod[\lc\Sigma,\Psi\rc] & = & \{ A' \in \MAlg(\Sigma'): 
                                     A'\mid_\sigma \in \SMod[\SP]\}
\end{array}
$

\item Derival of $\Sigma'$-specification from a $\Sigma$-specification
by a morphism $\sigma : \Sigma' \ra \Sigma$. This operation allows to
hide details whilst preserving the collection of models. \\
$
\begin{array}{lcl}
   \SSig[\derive{\SP}{\sigma}] & = & \Sigma' \\
   \SMod[\lc\Sigma,\Psi\rc] & = & \{ A'\mid_\sigma: 
                                     A' \in \SMod[\SP]\}
\end{array}
$
\end{itemize}




%===========================================================================
%
\section{Multialgebras}

We finally come to the definition of a multialgebra: 

\begin{Definition}
Let $\Sigma = [S,F]$ be a signature. A $\Sigma$-multialgebra is a
tuple $M = [S^M,F^M]$ where $S^M = \{M_s\}_{s \in S}$ is a family of
carrier sets for each sort $s \in S$ and $F^M = \{f^M\}_{f \in F}$ is
a family of set-valued functions such that
\[ 
f^M : M_{s_1} \times \ldots \times M_{s_n} \rightarrow P^+(M_s) 
\]
for each $[f : s_1 \times \ldots s_n \rightarrow s] \in F$
\hfill$\Box$
\end{Definition} 
%
Since we are going to discuss deterministic and nondeterministic
algebras in parallel, in order to avoid ambiguity we shall always use
letters $A,B,\ldots$ for the former ones and $M,N,\ldots$ for the
latter ones. The class of all $\Sigma$-algebras and all
$\Sigma$-multialgebras will be denoted $\Alg(\Sigma)$ and
$\MAlg(\Sigma)$, respectively.

Below we present a definition of a tight multihomomorphism which is
the most obvious extension of the original notion for deterministic
algebras.

\begin{Definition}
Let $\Sigma = [S,F]$ be a signature, $M,N \in \MAlg(\Sigma)$.  A tight
$\Sigma$-multi\-homo\-mor\-phism $\phi$ from $M$ to $N$ is a family of
mappings $\phi = \{\phi_s\}_{s \in S}, \phi_s: M_s \rightarrow N_s$
such that for all $[f : s_1 \times s_2 \times \ldots \times s_n
\rightarrow s] \in F$ and for all $a_i \in M_{s_i}, i \in
\{1,\ldots,n\}$ the following condition holds:
\[ 
\phi_s(f^M(a_1,\ldots,a_n)) =
f^N(\phi_{s_1}(a_1),\ldots,\phi_{s_n}(a_n)) 
\]
\hfill$\Box$
\end{Definition}
%
For many applications, however, it turns out to be too strong. In the
next definition we cover other possible notions of a homomorphism that
will come up in the discussion. For simplicity of presentation we
restrict ourselves to single-argument functions.

\begin{Definition}
Let $\Sigma = [S,F]$ be a signature, $M,N \in \MAlg(\Sigma)$ and let
$\phi$ be a mapping from $M$ to $N$ (i.e.\ a fimily of mappings 
$\phi = \{\phi_s\}_{s \in S}, \phi_s: M_s \rightarrow N_s$).
If for all $[f : s' \rightarrow s] \in F$ and for all $a \in M_{s'}$: 
\begin{itemize}
\item{ $\phi_s(f^M(a)) \subseteq f^N(\phi_{s'}(a))$ 
       then $\phi$ is a loose $\Sigma$-homomorphism; }
\item{ $\phi_s(f^M(a)) \supseteq f^N(\phi_{s'}(a))$ 
       then $\phi$ is a closed $\Sigma$-homomorphism; }
\item{ $\phi_s^{-1}(f^N(\phi_{s'}(a))) \subseteq
       \phi^{-1}_s(\phi_s(f^M(\phi^{-1}_{s'}(\phi_{s'}(a)))))$
      then $\phi$ is a strong $\Sigma$-homomorphism; }
\item{ $f^N(\phi_{s'}(a)) \subseteq
       \phi_s(f^M(\phi^{-1}_{s'}(\phi_{s'}(a))))$
       then $\phi$ is a very strong $\Sigma$-homomorphism; \hfill$\Box$}
\end{itemize}
\end{Definition}
%
A {\em tight (loose, closed,...) subalgebra} $N$ of $M$ is a
multialgebra such that $|N| \subseteq M$ the injective morphism
$\imath : N \hookrightarrow M$ (i.e.\ the identity) is of respective
type. We can now describe a class of deterministic algebras defined by
a given multialgebra.

\begin{Definition}
Let $M,N \in \MAlg(\Sigma)$. We say that $N$ is a determinisation of
$M$ iff $N$ is a loose subalgebra of $M$. Moreover, $N$ is a complete
determinisation if it is deterministic i.e.\ for any $[f : s_1 \times
s_2 \times \ldots \times s_n \rightarrow s] \in F$ and for all $a_i
\in N_{s_i},\ i \in \{1,\ldots,n\}$ we have $|f^N(a_1,\ldots,a_n)| =
1.$ 
\hfill$\Box$
\end{Definition}
%
We adopt notations $\Det(M)$ and $\CDet(M)$ for respective classes of
all determinizations and complete determinisations of a given
$\Sigma$-multialgebra $M$. Also, unless otherwise specified, a term
``deterministic algebra'' will denote a multialgebra $M$ such that $M
\in \CDet(M)$.

We need yet another notion -- a quotient of a multialgebra. Unlike in
the standard one, multialgebras allow us to drop the requirement that
the relation is a congruence. It suffices to have an equivalence.

\begin{Definition}
Let $M \in \MAlg(\Sigma)$ and let $\sim_M$ be an equivalence relation
on $M$. A quotient of the multialgebra $M$ w.r.t. $\sim_M$ (denoted
$A\Mquo_{\sim_M}$) is a multialgebra $N$ such that:
\[
\begin{array}{l}
N_s = \{ [a]_{\sim_M} : a \in M_s \} \\
f^N([a_1]_{\sim_M},\ldots,[a_n]_{\sim_M}) = 
\{ [b]_{\sim_M} : b \in f^M(a'_1,\ldots,a'_n), a'_i \in [a_i]_{\sim_M} \}
\end{array}
\]
where $[f : s_1 \times s_2 \times \ldots \times s_n \rightarrow s] \in F$
and $a_i \in A_{s_i}$ for $i \in \{1,\ldots,n\}$; $[a]_{\sim_M}$ denotes
the equivalence class of $a$.
\hfill$\Box$
\end{Definition}
%
For any subset of $P \subseteq |M|$ we denote by $\close{P}_{\sim_M}$,
or simply $\close{P}$ the closure of $P$ wrt.\ $\sim_M$. Note that a
very special case occurs when the multialgebra $A$ is deterministic
and $\sim$ is not a congruence and it leads to a concept of a {\em
nondeterminisation} of an algebra.

\section{Nondeterministic terms and formulae}

The logic will be tailored to an arbitrarily chosen, but fixed,
signature and, as usual, the formulae will be constructed over a set
of terms. However, our set of terms will be an extension of the
typical set of algebraic terms through an addition of a special ``{\sf
let}'' construct, which is used to restrict the nondeterminism and
henceforth to increase the expressive power.  We shall explain this in
detail.

Let $\Sigma = [S,F]$ be an arbitrary algebraic signature, to be fixed
for the sequel, and let $X=\{X_s\}_{s\in S}$, where $X_s$ is an 
enumerable set of variables of sort $s$. We shall use a short
notation of the form $x\in X$ to denote the fact that $x\in X_s$ for
some $s\in S$.

\begin{Definition} 
The set of nondeterministic $\Sigma$-terms (n-terms) over an $S$-sorted 
set of variables $X$, denoted $NT_\Sigma(X)$, the least set of expressions 
such that:
\begin{itemize} 
\item{each variable $x \in X_s$ is an n-term of sort $s$;} 
\item{if $t_i$ are n-terms of sort $s_i$ and 
      $[f:s_1 \times \ldots \times s_n \rightarrow s] \in F$, 
      then $f(t_1, \ldots, t_n)$ is an n-term of sort s;} 
\item{if $t,t'$ are n-terms and $x$ is a variable of the same sort 
      as the term $t'$, then $\nlet{x}{t'}{t}$ is an n-term, iff $x$ is 
      free in $t$;}
\end{itemize}
where a variable $x$ is free in an n-term $t$ iff it is not bound by 
another {\sf let} construct in  $t$.
\hfill$\Box$ 
\end{Definition}
%
For the sake of simplicity, whenever possible we shall omit explicit
references to sorts of variables, functions, terms and objects; the
implicit sorts are always assumed to be such that the considered
expression (object) is well-formed (well-defined).

The interpretation of n-terms in a given $\Sigma$-multialgebra $A$
will be given by multi-sorted valuation of variables. We have already
mentioned that n-terms will be interpreted as sets. However, the
valuation of variables is still deterministic and assigns to each $x
\in X$ a single value from appropriate carrier sets of $A$.

\begin{Definition}
For any multialgebra $A\in \MAlg(\Sigma)$, by a (multi-sorted)
valuation of variables in $X$ in $A$ we mean any mapping $v: X
\rightarrow |A|$, with $v(x)\in s^{A}$ for every $s\in S$ and every
$x\in X_s$.
\hfill$\Box$ 
\end{Definition}
%
The semantics of n-terms in a multialgebra $M$ is obtained by
extending the valuation $v$ in a standard way, with two special
prerequisites. The value of a set-valued function symbol $f$ on a set
of elements will be interpreted as taking a set-theoretic union of the
values of $f^M$ for all the elements of this set. In case of the
$\nlet{x}{t'}{t}$ construct we proceed as follows: first, we pick a
particular value $a$ of $t'$ in $M$ and substitute it for all the
occurrences of $x$ in $t$ i.e. take the value of the n-term $t[a/x]$;
the value of the ``{\sf let}'' term is a sum of values of all such
terms formed for any value of $t'$ in $M$. Formally, we define the
semantics as follows:

\begin{Definition} Let $M$ be a multialgebra in $\MAlg(\Sigma)$, and
let $v: X \rightarrow M$ be a multi-sorted valuation of variables
from $X$ in $M$. For any $v:X\ra |M|, x\in X, a\in |M|$, we define
\[
v[a/y](x) = \left\{
            \begin{array}{ll} 
                   a    & \mbox{iff $x=y$}, \\
                   v(x) & \mbox{otherwise}
            \end{array}
            \right.
\]
The interpretation of n-terms in $A$ induced by $v$ is a function 
$\val{v}: NT_{\Sigma}(X) \rightarrow P(|M|)$ defined as follows:
\begin{itemize} 
\item{$\val{v}(x)$ is the singleton set $\{v(x)\}$}
\item{$\val{v}(f(t_1,\ldots,t_n))$ is the set $\bigcup_{a_i \in
      \val{v}(t_i)} f^{A}(a_1,\ldots,a_n)$}
\item{$\val{v}(\nlet{x}{t'}{t})$ is the set 
      $\bigcup_{a \in \val{v}(t')} \val{v[a/y]}(t)$}
\end{itemize}
\hfill$\Box$ 
\end{Definition}
%
Often, if the valuation is not important or obvious in the context, we
refer to a value of a term $t$ in $M$ by $t^M$ instead of
$\val{v}(t)$. Now, in order to give the reader more insight into a
difference between the usual substitution and the ``{\sf let}''
substitution, we present the following example.
\medskip

\noindent
{\bf Example:} Consider a signature $ \Sigma_r = [S_r,F_r]$ where $S_r
= \{OhOne, AB\}$ and $F_r = \{g: AB \times AB \rightarrow OhOne; c:
\rightarrow AB; One: \rightarrow OhOne\}$ and a
$\Sigma_r$-multialgebra $A$ below:
\[ 
\begin{array}{lcll} 
       OhOne  & = & \{0,1\} & \\ AB & = & \{a,b\} & \\
       g(x,y) & = & \left\{\begin{array}{ll} 1 & \mbox{if $x = y$} \\ 
                                              0 & \mbox{otherwise} 
                           \end{array} \right. & \\ 
       c      & = & \{a,b\} & \mbox{(a nondeterministic constant)} \\ 
       One    & = & 1 & \mbox{(a deterministic constant)}
\end{array} 
\]
Now, for any valuation $v$ we have $\nlet{x}{c}{g(x,x)}^{A} = \{1\}$, and
$g(x,x)[c/x]^{A} = \{0,1\}$. \\. \hfill$\Box$ 

\medskip
However, we have defined n-terms over {\em all} variables. In a usual
case, if we restrict the set of variables to some sorts only, the
definition is just obvious. Here, we must keep in mind that we need to
restrict only the free variables whilst the bound ones may range over
the initial set. Hence, for any signature $\Sigma = [S,F]$ and $\Obs
\subseteq S$ the set $NT_{\Sigma}(X_{\Obs})_{\Obs}$ is the set of terms
$t_s \in NT_{\Sigma}(X)_s$ such that $s \in \Obs$ and ${\rm Free}(t_s)
\subseteq X_{\Obs}$.

In addition we shall need {\em contexts}. For a given signature
$\Sigma = [S,F]$ a set of observable contexts $C(X_{\Obs})_{\Obs}$ is
just the set $\bigcup_{s \not\in \Obs} NT(X_{\Obs} \cup 
\{z_s\})_{\Obs}$ where $z_s$ is a variable of sort $s$ such that $z_s
\not\in X$. We write $c\lc a \rc$ meaning the term obtained by
substituting a value $a$ for $z$. 

Finally, the formulae of our logic will be built over n-terms with a
``rewrite'' symbol, i.e. ``$\ra$''.

\begin{Definition}
A formula over $\Sigma$ is any expression of the form $t \rightarrow
t'$, where $t, t'\in NT_{\Sigma}(X)$ are n-terms of the same sort. \\
The set of all $\Sigma$-formulae will be denoted by ${\cal
F}_\Sigma$. Moreover, we write ${\cal F}_{\Sigma,\Obs}$ for
the set of formulae built over terms from
$NT_{\Sigma}(X_{\Obs})_{\Obs}$.
\hfill$\Box$ 
\end{Definition}
%
We read a formula $t \ra t'$ as {\em the n-term $t$ rewrites to $t'$}
and this is supposed the capture the intuitive meaning that values of
$t'$ are also the values of $t$. In other words, the rewrite operator
``$\ra$'' is just the inclusion of appropriate sets of values of terms
and it is reflected in the notion of satisfaction of a formula in a
given model.

\begin{Definition}
The satisfaction of a formula $t\rightarrow t'$ in a multialgebra $M$
under valuation $v$ is defined by
\[ 
M,v \models t\rightarrow t' \quad {\rm iff} \quad \val{v}(t)
\supseteq \val{v}(t')
\] 
\hfill$\Box$ 
\end{Definition}

Last we introduce a concept of a {\em reachable subalgebra} of $M \in
\MAlg(\Sigma)$ i.e.\ a subalgebra where every element of the carrier
is in the set of results of some ground term $t \in T_{\Sigma}$ (this
is one of the many definitions of reachability in nondeterministic
algebras, cf.\ \cite{WM95} for details). We will also speak of a
subalgebra {\em reachable from sorts \Obs,} where $\Obs$ is some subset
of sorts in $\Sigma$ -- this means that carrier elements are in values
of some term in $T_\Sigma(X_{\Obs})$ for some valuation of
variables. Since the latter concept will be more often used, we denote
such a subalgebra by $\val{M}$ assuming the set $\Obs$ is specified in
the context.

At the end we can state a basic fact concerning a relationship between
a multialgebra and an abstracting equivalence among its
determinisations.

\begin{Lemma}
Let $\Sigma = [S,F]$ be a signature. For any $\Sigma$-multialgebra $M$ and
any two algebras $A,B \in \CDet(M)$ there holds $A \equiv_W B$ where $W$ is
a set of terms deterministic in $M$.
\hfill$\Box$
\end{Lemma}   

\begin{Proof} 
Note that deterministic terms are the only ones whose
value cannot change during the determinisation. Hence any equality
between these terms is preserved.  
\end{Proof}


%========================================================================
%
%  MULTIALGEBRAS AND ALGEBRAIC SPECS

\section{Multialgebras and algebraic specifications}



\begin{Definition} 
Given two multialgebras $M,N \in \MAlg(\Sigma)$ we say they are
observationally equivalent (written $M \obs N$) iff there exists a set
of variables $Y_{\Obs}$ and surjective valuations $v_M: Y_{\Obs} \ra
|M|_{\Obs}, v_M: Y_{\Obs} \ra |M|_{\Obs}$ such that for any $t,t' \in
NT(Y_{\Obs})_{\Obs}$ holds
\[
M,v_M \models t \ra t' \quad {\rm iff} \quad N,v_N \models t \ra t'
\]
\hfill$\Box$
\end{Definition}
%
We will refer to the above equivalence on $\MAlg(\Sigma)$ as the
observational abstraction. 


%=========================================================================
%
%  AN EXAMPLE

\section{An examples}


In this section we would like to concentrate on some examples of
nondeterministic specification and their use in program development.
First, let us consider a simple specification of sets of natural
numbers. We assume that {\sf Nat} and {\sf Bool} are the only
observable sorts.

\medskip 
\begin{tabular}{ll}
{\bf spec:}   & {\sf Spec-Set} \\
              & {\bf using:} {\sf Spec-Nat, Spec-Bool}  \\
{\bf sorts:}  & {\sf Set} \\
{\bf opns:}   & $\emptyset: \rightarrow${\sf Set} \\
              & {\sf ins: Nat $\times$ Set $\rightarrow$ Set} \\
              & {\sf del: Nat $\times$ Set $\rightarrow$ Set} \\
              & {\sf $\in$: Nat $\times$ Set $\rightarrow$ Bool} \\
{\bf axioms:} & {\sf ins(x,ins(x,S)) $=$ ins(x,S)} \\
              & {\sf ins(x,ins(y,S)) $=$ ins(y,ins(x,S))} \\
              & {\sf del(x,$\emptyset$) $= \emptyset$} \\
              & {\sf del(x,ins(x,S)) $=$ del(x,S)} \\
              & {\sf x $\not=$ y $\Rightarrow$ del(x,ins(y,S)) $=$ 
                 ins(y,del(x,S))} \\
              & {\sf x $\in \emptyset =$ false} \\
              & {\sf x $\in$ ins(x,S) $=$ true} \\
              & {\sf x $\not=$ y $\Rightarrow$ x $\in$ ins(y,S) $=$ 
                 x $\in$ S} \\  
{\bf end;}
\end{tabular} 

We may wish to implement sets as lists, i.e.\ by the following
specification (we omitt the axioms):

\medskip 
\begin{tabular}{ll}
{\bf spec:}   & {\sf Spec-List} \\
              & {\bf using:} {\sf Spec-Nat, Spec-Bool} \\
{\bf sorts:}  & {\sf List}\\
{\bf opns:}   & {\sf nil: $\rightarrow$ List} \\
              & {\sf cons: Nat $\times$ List $\rightarrow$ List} \\
              & {\sf tail: List $\rightarrow$ List} \\
              & {\sf head: List $\rightarrow$ Nat} \\
              & {\sf member: Nat $\times$ List $\rightarrow$ Bool} \\
              & {\sf delete: Nat $\times$ List $\rightarrow$ List} \\
{\bf axioms:} & {\sf [...]} \\
{\bf end;}
\end{tabular}

\medskip
Now, we have that:
\[
\abs{\sf Spec-Set}{\sf Nat, Bool} \quad \imp \quad
\derive{\sf Spec-List}{\sigma}
\]
where $\sigma$ is the a signature morphism defined by: $\{{\sf Set
\mapsto List, add \mapsto cons, \ \in\ \mapsto member,...}\}$ We need
to close the class of models with respect to the observational
equivalence since there may be many lists with the same set of
elements.

Now, let us enrich {\sf Spec-Set} with an additional operation {\sf
choose}: 

\medskip 
\begin{tabular}{ll}
{\bf spec:}   & {\sf Spec-Set-Choose} \\
              & {\bf using:} {\sf Spec-Set}  \\
{\bf sorts:}  & \\
{\bf opns:}   & {\sf choose: Set $\rightarrow$ Nat} \\
{\bf axioms:} & {\sf S $\not= \emptyset \Rightarrow$ choose(S) $\in$ S
                 $=$ true} \\
{\bf end;}
\end{tabular} 

\medskip 
\noindent
Consider now the following enrichment of {\sf Spec-List} with {\sf
choose} defined as the head of a list.

\medskip 
\begin{tabular}{ll}
{\bf spec:}   & {\sf Spec-List-Head} \\
              & {\bf using:} {\sf Spec-List} \\
{\bf sorts:}  & \\
{\bf opns:}   & {\sf choose: List $\rightarrow$ Nat} \\
{\bf axioms:} & {\sf choose(L) $=$ head(L)} \\
{\bf end;}
\end{tabular}

\medskip
\noindent
However, in a deterministic setting it is not true that
\[
\abs{\sf Spec-Set-Choose}{\sf Nat} \quad \imp \quad
\derive{\sf Spec-List-Head}{\sigma_h}
\]
where $\sigma_h = \sigma \cup \{{\sf choose \mapsto choose}\}$.  It is
because from {\sf Spec-Set-Choose} it follows that ${\sf
choose(add(n,add(m,A))) = choose(add(m,add(n,A)))}$ and, since this
equation is observable, it must also hold for {\sf Spec-List-Head}.
Obviously, it does not.  We may still implement {\sf Spec-Set-Choose}
with lists but {\sf choose} has to be defined more carefully, e.g.\ as
the maximal element of the list.

What happens if we allow multialgebras? Note that {\sf
Spec-Set-Choose} has (among many others) a model where all operations
but {\sf choose} are determininistic and {\sf choose} itself is fully
nondeterministic i.e.\ it may return any element of the set. We can
implement {\sf Spec-Set-Choose} with lists where {\sf choose} returns
a random element of a given list:

\medskip 
\begin{tabular}{ll}
{\bf spec:}   & {\sf Spec-List-Random} \\
              & {\bf using:} {\sf Spec-List} \\
{\bf sorts:}  & \\
{\bf opns:}   & {\sf choose: List $\rightarrow$ Nat} \\
{\bf axioms:} & {\sf L $\not=$ nil $\Rightarrow$ 
                member(choose(L),L) = true }\\
{\bf end;}
\end{tabular}

\[
\abs{\sf Spec-Set-Choose}{\sf Nat} \quad \imp \quad
\derive{\sf Spec-List-Random}{\sigma_h}
\]

{\sf Spec-List-Random} has among its determinisations {\sf
Spec-List-Head}:

\[
\deter{\sf Spec-List-Random} \quad \imp \quad {\sf Spec-List-Head}
\]


%=========================================================================
%
%  BEHAVIOURAL SATISFACTION AND EQUIVALENCE

\section{Behavioural satisfaction}

In this section we shall consider a generalisation of the standard
satisfaction relation.

\begin{Definition}
Given a multialgebra $M \in \MAlg(\Sigma)$ an equivalence relation
$\sim$ on $M$ is an observational equivalence iff for any $t,t' \in
T_\Sigma(X_{\Obs})_{\Obs}$:
\[
M \models t \ra t' \quad {\rm iff} \quad 
M\Mquo_{\sim} \models t \ra t'
\]
\hfill$\Box$
\end{Definition}
%
We write $\Eq(M)$ for a class of observational equivalences on $M$ (we
leave $\Obs$ implicit). Also, note the above definition implies that
an observational equivalence is an identity on observable sorts. 
However, unlike standard algebras we have that:

\begin{Fact}
In general, for a given multialgebra $M$ there is no largest element
in $\Eq(M)$. There are maximal ones. 
\hfill$\Box$
\end{Fact}
\begin{figure}
\caption{\label{no-largest}Two maximal observational equivalences.}
\begin{center}
\begin{picture}(470,130)
%
% lewa algebra, kropki
%
\multiput(115,10)(60,0){2}{\circle*{3}}
\multiput(100,55)(30,0){4}{\circle*{3}}
\multiput(100,100)(45,0){3}{\circle*{3}}
%
% srodkowa algebra, kropki
%
\multiput(265,10)(30,0){2}{\circle*{3}}
\multiput(250,55)(30,0){3}{\circle*{3}}
\multiput(250,100)(30,0){3}{\circle*{3}}
%
% prawa algebra, kropki
%
\multiput(375,10)(30,0){2}{\circle*{3}}
\multiput(375,55)(30,0){2}{\circle*{3}}
\multiput(360,100)(30,0){3}{\circle*{3}}
%
% lewa algebra, wektory
%
\multiput(115,10)(60,0){2}{\vector(1,3){14}}
\multiput(115,10)(60,0){2}{\vector(-1,3){14}}
\put(130,55){\vector(1,3){14}}
\put(160,55){\vector(-1,3){14}}
\multiput(100,55)(90,0){2}{\vector(0,1){43}}
%
% srodkowa algebra, wektory
%
\multiput(265,10)(30,0){2}{\vector(1,3){14}}
\multiput(265,10)(30,0){2}{\vector(-1,3){14}}
\multiput(250,55)(30,0){3}{\vector(0,1){43}}
%
% prawa algebra, wektory
%
\multiput(375,55)(30,0){2}{\vector(1,3){14}}
\multiput(375,55)(30,0){2}{\vector(-1,3){14}}
\multiput(375,10)(30,0){2}{\vector(0,1){43}}
%
% oddzielenie i nazwanie sortow oraz funkcji
%
\put(70,32){\line(1,0){400}}
\put(70,77){\line(1,0){400}}
\put(0,8){In}
\put(0,53){Hidden}
\put(0,98){Out}
\put(0,30){$f: {\rm In} \ra {\rm H}$}
\put(0,75){$g: {\rm H} \ra {\rm Out}$}
%
% pokazanie relacji, lewa algebra
%
\put(139,53){$\sim_1$}
\multiput(109,53)(60,0){2}{$\sim_2$}
%
% nazwanie kropek, lewa algebra
%
\put(112,0){$a$}
\put(172,0){$c$}
\put(98,107){$0$}
\put(142,107){$1$}
\put(187,107){$2$}
%
% nazwanie kropek, srodkowa algebra
%
\put(262,0){$a$}
\put(292,0){$c$}
\put(248,107){$0$}
\put(277,107){$1$}
\put(307,107){$2$}
%
% nazwanie kropek, prawa algebra
%
\put(372,0){$a$}
\put(402,0){$c$}
\put(358,107){$0$}
\put(387,107){$1$}
\put(417,107){$2$}
%
% nazwanie algebr
%
\put(140,130){$M$}
\put(250,130){$N_1 = M\Mquo_{\sim_1}$}
\put(360,130){$N_2 = M\Mquo_{\sim_2}$}
%
\end{picture}
\end{center}
\end{figure}
%
%


\begin{Proof} 
See example on Fig~\ref{no-largest}. Both $\sim_1$ and $\sim_2$ are
maximal observational equivalences but there is no largest one. The
existence of maximal elements follows from Kuratowski-Zorn lemma:

Given a multialgebra $M \in \MAlg(\Sigma)$, consider a chain of
observational equivalences on $M$: $\{ \sim_i \}_{i \in \omega}$ such
that $\sim_0\ \subseteq \ldots \subseteq\ \sim_i\ \subseteq\
\sim_{i+1}\ \subseteq \ldots$ and denote $\sim\: = \bigcup_{i \in
\omega} \sim_i$. Obviously, $\sim$ is an identity on observable sorts
since all $\sim_i$'s were. Now, we have to show that $\sim\ \in
\Eq(M)$. Assume it does not, i.e.\ for some $t \ra t'$ we have $M
\models t \ra t'$ but $M\Mquo_\sim \not\models t \ra t'$. Since $\sim$
is an identity on $\Obs$ then the above means that for some observable
$t$, a valuation $v$ and $\pi : a \mapsto [a]_\sim$ we have
\[ 
\val{v}(t) \not= \val{v;\pi}(t) \quad\quad\quad (\ast) 
\]
The term $t$ must be of the form $l\lc r\rc $ where $r$ is of some
non-observable sort $s$. By the definition of the quotient and
$(\ast)$ we know that $\val{v}(l\lc r\rc ) \not=
\val{v}(l\lc \close{r}_\sim\rc )$. Now, by the definition of $\sim$ there is
some $\sim_n$ such that $\val{v}(l\lc \close{r}_\sim\rc ) =
\val{v}(l\lc \close{r}_{\sim_n}\rc )$ and this means that $\sim_n$ is not an
observational equivalence which leads to contradiction.
\end{Proof}
\medskip

Next we proceed to define observational satisfaction:

\begin{Definition}
Given a multialgebra $M \in \MAlg(\Sigma)$ and an observational
equivalence $\sim_M$ define an observational satisfaction of a formula
$t \ra t'$ in $M$ under a valuation $v$:
\[
M,v \models_{\sim_M} t \ra t' \quad {\rm iff} 
                \quad M\Mquo_{\sim_M},v;\pi \models t \ra t' 
\]
where $\pi: M \ra M\Mquo_\sim$ is the mapping $a \mapsto [a]_\sim$
\hfill$\Box$
\end{Definition}
%
There is an alternative definition that would require inclusion of
appropriate equivalence classes instead of standard satisfaction in
the quotient i.e.:
\[
M,v \models^{'}_\sim t \ra t' \quad {\rm iff} 
                \quad [\val{v}(t)]_\sim \supseteq [\val{v}(t')]_\sim
\]
Note the above means that $[\val{v}(t)]_\sim \supseteq
\val{v}(t')$. The latter definition is somewhat closer to the
deterministic observational satisfaction but there the two definitions
are equivalent. In our framework, the definitions are equivalent for
observable formulae but in general it may not be so for unobservable
ones. It is because $\val{v};\pi \subseteq \val{v;\pi}$ and need not
$\val{v};\pi = \val{v;\pi}$.
% A jak dla maksymalnych ?  
Therefore, we may want to find conditions on $\sim$ to imply equality
there.

\begin{Fact}
\label{f-tight}
Given $M \in \MAlg(\Sigma)$, $\sim\ \in \Eq(M)$, a valuation $v$ and
$\pi: M \ra M\Mquo_\sim$, if $\sim$ is a tight congruence on $M$ then
$\val{v};\pi = \val{v;\pi}$.
\hfill$\Box$
\end{Fact}

Finally, we want to find a definition of observational equivalence
similar to context indistinguishability. In multialgebras this is just
one of the three equivalences induced by observable sorts that we are
to consider here. To give the reader some intuitions, we discuss the
equivalences informally:
\begin{itemize}
\item Context equivalence is the obvious extension of the
deterministic notion, i.e.\ two values are c-equivalent iff any
observable context applied to them yields the same set of results.
\item Result equivalence -- two values are r-equivalent iff they are
results of the same terms of observable domain. This conveys an
intuition that if some values always come up together during
``execution'' of observable terms then we can ``glue'' them without
affecting the observable behaviour of a multialgebra.
\item Path equivalence is the most complex one and it is a further
extension of the c-equivalence. Note that in a deterministic case each
value reachable from $\Obs$ could be pointed at by some term and hence
the c-equivalence could be rephrased as follows: a value $a$ is
equivalent to $b$ iff for any terms $r,r'$ such that $a = r,\ b = r'$
and any observable context $c$ we have $c\lc r\rc  = c\lc r'\rc $. Now, the
p-equivalence is just that except that we require the compared values
to be included in (and not necessarily equal to) the results of terms
$r$ and $r'$.
\end{itemize}

\begin{Definition}
Let $\Sigma = [S,F]$ and $\Obs \subseteq S$. Given $M \in
\MAlg(\Sigma)$ define the following relations on $M$: \\ 
\givedef
{$a \ec b$} 
{for any $v: X_{\Obs} \ra |M|$, and any $c \in
C_{\Sigma}(X_{\Obs})_{\Obs}$ holds $(c\lc a \rc)^M = (c\lc b \rc)^M$}
\givedef
{$a \er b$} 
{for any $v: X_{\Obs} \ra |M|$, and any $r \in T_{\Sigma}(X_{\Obs})$
holds $a \in r^M$ iff $b \in r^M$}
\givedef
{$a \ep' b$}
{for any $v: X_{\Obs} \ra |M|$, any $r,r' \in
T_{\Sigma}(X_{\Obs})$ such that $a \in r^M, b \in r'^M$, and any
$c \in C_{\Sigma}(X_{\Obs})_{\Obs}$ holds $(c\lc r \rc)^M =
(c\lc r' \rc)^M$}
\ \hfill$\Box$
\end{Definition}
%
Note that, in a deterministic case, $\er$ is the identity whilst $\ec$
and $\ep$ are both equivalent (to context indistiguishability).  In
the nondeterministic case, $\ec$ and $\er$ are equivalences but $\ep'$
need not be reflexive$\:$!\ \ Hence we are going to consider an
equivalence $\ep$ defined as the reflexive closure of $\ep'$:
\[
\ep\ = (\ep')^\ast
\]
In general, the equivalences defined above are not tight and we cannot
use Fact~\ref{f-tight} to prove they are observational. However, the
condition that the equivalence must be tight is sufficient but not
necessary:

\begin{Lemma}
For any $M \in \MAlg(\Sigma)$ the equivalences $\ec, \er$ and $\ep$ are
observational. 
\hfill$\Box$
\end{Lemma}

\begin{Proof} We have to show for $\sim\ \in \{\ec,\er,\ep\}$ that given
any valuation $v: X_{\Obs} \ra |M_{\Obs}|$ and observable $t,t'$:
\[
M,v \models t \ra t' \quad {\rm iff} \quad 
                         M\Mquo_{\sim} ,\pi;v \models t \ra t'
\]
where $\pi: a \mapsto [a]_{\sim}$. Therefore we need to prove
$\val{v;\pi} = \val{v};\pi$ and we know that $\sim$ is an identity on
$\Obs$.
\begin{itemize}
\item[($\ec$)] Assume it is not observational. Then there exists an
observable term $t = l\lc r\rc $ for which the equality does not hold. It
means, that for a subterm $r$ (of some sort $s$) there exist $a_s \in
\val{v}(r), b_s \not\in \val{v}(r)$ such that $a \ec b$ and
$\val{v}(l\lc a_s\rc ) \not= \val{v}(l\lc b_s\rc )$. But this leads to
contradiction, since we can easily construct a context to distinguish
between $a_s$ and $b_s$ simply by replacing the subterm $r$ with a
variable $z$ in $t$.
\item[($\er$)] 
Simply note that for any $t$ we have $\close{\val{v}(t)}_{\er} =
\val{v}(t)$.
\item[($\ep$)] The reasoning is similar to the case with
$(\ec)$. If $\ep$ is not observational then there is an observable
term $t = l\lc r\rc $ such that $\val{v}(l\lc r\rc ) \not= 
\val{v}(l\lc \close{r}_{\ep}\rc )$. But this means there are  $a_s \in
\val{v}(r), b_s \not\in \val{v}(r)$ such that $a \ep b$. By the
definition of $\ep$ for any $r'$ with $b_s \in \val{v}(r')$ we have
$\val{v}(l\lc r\rc ) = \val{v}(l\lc r'\rc )$ -- a contradiction.
\end{itemize}
\end{Proof}
A reader should be aware that a transitive sum of observational
equivalences is not, in general, an observational equivalence.

\begin{Definition}
A multialgebra $M \in \MAlg(\Sigma)$ is said to be fully abstract iff
any observational equivalence on $M$ is an identity.
\hfill$\Box$
\end{Definition}



%=========================================================================
%
%  ABSTRACTING EQUIVALENCE

\section{Abstracting equivalence}

\begin{Definition} 
Given two multialgebras $M,N \in \MAlg(\Sigma)$ we say they are
observationally equivalent (written $M \obs N$) iff there exists a set
of variables $Y_{\Obs}$ and surjective valuations $v_M: Y_{\Obs} \ra
|M|_{\Obs}, v_M: Y_{\Obs} \ra |M|_{\Obs}$ such that for any $t,t' \in
NT(Y_{\Obs})_{\Obs}$ holds
\[
M,v_M \models t \ra t' \quad {\rm iff} \quad N,v_N \models t \ra t'
\]
\hfill$\Box$
\end{Definition}
%
We will refer to the above equivalence on $\MAlg(\Sigma)$ as the
observational abstraction. 

\begin{Definition}
An abstraction $\equiv$ on $\MAlg(\Sigma)$ is said to be factorisable
iff for any $M,N \in \MAlg(\Sigma)$ such that $M \equiv N$ there exist
two equivalences $\sim_M, \sim_N$ on $M$ and $N$ respectively, such
that:
\[
M\Mquo_{\sim_M} \cong N\Mquo_{\sim_N}
\]
that is, the quotients are isomorphic.
\hfill$\Box$
\end{Definition}

In a classical deterministic case, the observational abstraction is
factorisable, i.e.\ if $A \obs B$ then $A/_{\sim_A} \cong B/_{\sim_B}$
where $\sim_A, \sim_B$ are context indistinguishability congruences on
$A$ and $B$, respectively. However, in multialgebraic setting the
observational abstraction is not factorisable:

\begin{Lemma}
Observational abstraction is not factorisable.
\hfill$\Box$
\end{Lemma}  

\begin{Proof}
See multialgebras $N_1$ and $N_2$ on Fig~\ref{no-largest}. They are
observationally equivalent, fully abstract and not isomorphic.
\end{Proof}

\begin{Lemma}
Let $M,N \in \MAlg(\Sigma)$ and $M \obs N$. Then for any $A \in
\CDet(M)$ there exists $B \in \CDet(N)$ such that $A \obs B$.
\hfill$\Box$
\end{Lemma}

\begin{Corollary}
Let $M,N \in \MAlg(\Sigma)$ and $M \obs N$. Then $\Beh(M) = \Beh(N)$
where $\Beh(M) = \{A/_{\ec} : A \in \CDet(M)\}$.
\hfill$\Box$
\end{Corollary}

\begin{Theorem}
Let $M,N \in \MAlg(\Sigma)$ and $M \obs N$. Then there exists $K \in
MAlg(\Sigma)$ such that:
\begin{itemize}
\item $M \sqs K,\ N \sqs K$,
\item $M \obs K,\ N \obs K$,
\item $K$ is fully abstract. \hfill$\Box$
\end{itemize}
\end{Theorem}

\begin{Proof}
Let $M \obs N$. We know they have same deterministic behaviours and we
construct $K$ of that set. Denote ${\cal B} = \Beh(M) = \Beh(N)$.
\[
\begin{array}{l}
      |K| = \bigcup \{|A| : A \in {\cal B}\} \\
      f^K(a_1,...,a_n) = \bigcup \{f^A(a_1,...,a_n) : A \in {\cal B}\}
\end{array}
\]
This construction, however, does not cope with multi-argument
functions -- $K$, as it is defined above, may be partial. 
\end{Proof}





\begin{thebibliography}{Nip 86}

   \bibitem[BBK 92a]{BBK92a} G.\ Bernot, M.\ Bidoit, T.\ Knapik: {\em
       Towards an Adequate Notion of Observation.}  report
       LIENS--92--2, 1992.

   \bibitem[BBK 92b]{BBK92b} G.\ Bernot, M.\ Bidoit, T.\ Knapik: {\em
       Observational Specifications and Indistinguishability
       Assumption.}  report LIENS--92--3, 1992.

   \bibitem[BHW 95]{BHW95} M.\ Bidoit, R.\ Hennicker, M.\ Wirsing:
       {\em Behavioural and Abstractor Specifications.}

   \bibitem[Gla 90]{Gla90} R.J.\ van Glabbeek: {\em Comparative
       Concurrency Semantics and Refinement of Actions.} Ph.D.\
       Thesis, Vrije Universiteit te Amsterdam, 1990.

   \bibitem[Hes 88]{Hes88} W.\ Hesselink: {\em A Mathematical Approach
       to Nondeterminism in Data Types.} ACM Trans.\ on Programming
       Lan.\ and Systems 10, 1988, pp.\ 87--117.

   \bibitem[Hus 92]{Hus92} H.\ Hussmann: {\em Nondeterministic
       Algebraic Specifications and Nonconfluent Term rewriting.} J.\
       Logic Programming 12, 1992, pp.\ 237--255.

   \bibitem[Hus 91]{Hus91} H.\ Hussmann: {\em Nondeterministic
       Algebraic Specifications.} Technische Universit\"{a}t
       M\"{u}nchen, TUM-I0104, March 1991.

   \bibitem[Kna 92]{Kna92} T.\ Knapik: {\em Specifications with
       Observable Formulae and Observational Satisfaction Relation.}
       Recent Trends in Data Type Specification '91, LNCS 655, pp.\
       271--291; Also: report LIENS--92--14, 1992.
   
   \bibitem[Nip 86]{Nip86} T.\ Nipkow: {\em Non-deterministic Data
       Types: Models and Implementations.} Acta Informatica 22, 1986,
       pp.\ 629--661.

   \bibitem[Nip 87]{Nip87} T.\ Nipkow: {\em Observing Nondeterministic
       Data Types.} LNCS 332.

   \bibitem[NO 87]{NO87} P.\ Nivela, F.\ Orejas: {\em Initial
       Behaviour Semantics for Algebraic Specification.} LNCS 332.

   \bibitem[NOS 87]{NOS87} P.\ Nivela, F.\ Orejas, A.\ S\'{a}nchez:
       {\em Implementation and Behaviour Equivalence: A Survey.}
       Recent Trends in Data Type Specification '91, LNCS 655, pp.\
       93--125.

   \bibitem[ST 88a]{ST88a} D.\ Sannella, A.\ Tarlecki: {\em
       Specifications in an Arbitrary Institution.} Information and
       Computation 76, 1988.

   \bibitem[ST 88b]{ST88b} D.\ Sannella, A.\ Tarlecki: {\em Towards
       Formal Development of Programs from Algebraic Specifications:
       Implementations Revisited.} Acta Informatics 25, 1988, pp.\
       233--281.

   \bibitem[ST 92]{ST92} D.\ Sannella, A.\ Tarlecki: {\em Towards
       Formal Development of Programs from Algebraic Specifications:
       Model--Theoretic Foundations.} ICALP'92, LNCS; Also: report
       ECS-LFCS-92-204.

   \bibitem[Sch 87]{Sch87} O.\ Schoett: {\em Data Abstraction and the
       Correctness of Modular Programming.} Ph.D.\ Thesis, University
       of Edinburgh, 1987.

   \bibitem[Wal 93]{Wal93} M.\ Walicki: {\em Algebraic Specifications
       of Nondeterminism.} Ph.D.\ Thesis, University of Bergen, 1993.

   \bibitem[WM 95]{WM95} M.\ Walicki, S.\ Meldal: {\em Generated
       models and the $\omega$-rule: the nondeterministic case.} 
       TAPSOFT'95, LNCS 915, pp.\ 424--438. 
       
\end{thebibliography}


\end{document}


