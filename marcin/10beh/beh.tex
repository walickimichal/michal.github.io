\documentclass[10pt]{article}
%\documentstyle[a4,fullpage,12pt,oldlfont]{article}

%\usepackage{amssymb}

\makeatletter
\input{a4wide}
\makeatother


\newtheorem{example}{Example}[section]
\newtheorem{definition}[example]{Definition}
\newtheorem{theorem}[example]{Theorem}
\newtheorem{lemma}[example]{Lemma}
\newtheorem{fact}[example]{Fact}
\newtheorem{proposition}[example]{Proposition}
\newtheorem{corollary}[example]{Corollary}

%equalities
\newcommand{\eequal}{\stackrel{e}{=}}%existential equality
\newcommand{\sequal}{\stackrel{s}{=}}%strong equality
\newcommand{\wequal}{\stackrel{w}{=}}%weak equality
\newcommand{\eleq}{\mathrel{\dot{=} }}%element equality
\newcommand{\seq}{\mathrel{\asymp}}% set equality 

\newcommand{\exist}{\downarrow}% defindnes in partial algebra
\newcommand{\st}{\ast}% extra star element


%different Categorical, arrows...
\newcommand{\embd}{\hookrightarrow}% embedding of institutions 
\newcommand{\pfunc}{\hookrightarrow}% partial function
\newcommand{\ring}{\circ}% composition of arrows

\newlength{\listskip}\setlength{\listskip}{1ex plus .2ex minus .5ex}
\newlength{\eqpskip}\setlength{\eqpskip}{.5ex plus .2ex minus .2ex}
\newenvironment{eqp}{\par %\topsep\eqpskip \parskip 0ex
   \begin{tabbing}
   \quad\=\+\quad\=$\{$\ \=\kill\(}{\)\end{tabbing}
   \addvspace{-1ex plus.2ex minus.3ex}} 
\newcommand{\comment}[2]{\)\-\\[\eqpskip]$#1$\>\>$\{$\>\+\+\ignorespaces#2$\}$\-\\[\eqpskip]\(}
\newcommand{\nline}{\)\\\(\mbox{}}

%%%%%%% MICHAL

%%% comments to be removed eventually
\newcommand{\fix}{\par\noindent\hspace*{-2em}{\large\bf fix} $\uparrow$ \dotfill}
\newcommand{\fixx}[1]{\par\noindent\hspace*{-2em}{\large\bf fix} $\uparrow$ {\small
#1} \dotfill}
\newcommand{\fixd}[1]{\par\noindent\hspace*{-2em}{\large\bf fix} $\downarrow$ {\small
#1} \dotfill}
\newcommand{\isit}[1]{\par\noindent\hspace*{-2em}{\small{{\bf [is it right?} #1{\bf ]}}}\par\noindent}
\newcommand{\how}[1]{\par\noindent\hspace*{-2em}{\small{{\bf [how?} #1{\bf
]}}}\par\noindent}
\newcommand{\todo}[1]{\par\noindent\hspace*{-2em}{\small{{\bf [To Do:} #1{\bf ]}}}\par\noindent}
\newcommand{\noo}[1]{\par\noindent\hspace*{-2em}{\large\bf NO} $\uparrow$ {\small
#1} \dotfill}
%%% environments
\newtheorem{schem}[example]{Schema}
\newcommand{\MyLPar}{\parsep -.2ex plus.2ex minus.2ex\itemsep\parsep
   \vspace{-\topsep}\vspace{.5ex}}
\newenvironment{PROOF}{{\bf Proof.}}{\nopagebreak\finish}
\newenvironment{PROOFs}{{\bf Proof (sketch)}}{\nopagebreak\finish}
\newcommand{\finish}{\hspace*{\fill}\nopagebreak 
     \raisebox{-1pt}{$\Box$}\par\addvspace{1.5ex}\noindent}

%%% Specs:
\newcommand{\spec}[1]{\begin{array}[t]{rrl}#1\end{array}\vspace*{1ex}}
\newcommand{\tit}[1]{\multicolumn{3}{l}{#1}}

%%% symbols
\newcommand{\eeq}{\eequal}
\newcommand{\PSet}[1]{{\mathcal{P}}(#1)}
\newcommand{\ovr}[1]{\overline{#1}}
\newcommand{\To}{\Rightarrow}
\newcommand{\Tod}{\Leftarrow}
\newcommand{\Iff}{\Longleftrightarrow}
\newcommand{\hviss}{\Iff}
\newcommand{\ok}[1]{d_{#1}} % OK part of a sort #1

\newcommand{\by}[2]{\stackrel{#1}{#2}}
\newcommand{\Toby}[1]{\ \by{#1}{\Longrightarrow}\ }

\newcommand{\adj}{\mathrel{\small\dashv}} %adjunction


%%% use these for signatures
\newcommand{\Sorts}{{S}} 
\newcommand{\Ops}{\Omega}
\newcommand{\ndc}{\Pi}
\newcommand{\err}{E}
\newcommand{\sign}{(\Sorts,\Ops,\ndc)}
\newcommand{\POps}{P\Omega}
\newcommand{\Pops}{\POps}
\newcommand{\Terms}[1]{T(#1)}
\newcommand{\TermsS}{\Terms\Sigma}
\newcommand{\TermsSX}{\Terms{\Sigma,X}}
\newcommand{\dom}{{\bf dom}}

%%% general categorical concepts
\newcommand{\cat}[1]{{\bf #1}} %put it around any category 
\newcommand{\inst}[1]{{\mathcal{#1}}} %put it around any institution: needs $_$
\newcommand{\fu}[1]{{\sf {#1}}} %put it around any functor
\newcommand{\thr}[1]{{\bf #1}} %put it around any Theory
\newcommand{\obj}[1]{|#1|} %objects of a category
\newcommand{\natt}{\mathrel{\Longrightarrow}} %natural transformation
%%% Categories
%multialgebras
\newcommand{\MA}[1]{\cat{MAlg}_{#1}}
\newcommand{\MAS}{\cat{MAlg}_{\Sigma}} %partial multialgebras
\newcommand{\PMA}[1]{\cat{PMAlg}_{#1}}
\newcommand{\PMAS}{\cat{PMAlg}_{\Sigma}} %partial algebras
\newcommand{\PAl}[1]{\cat{PAlg}_{#1}}
%used little but still...
\newcommand{\PAlo}{\PAl{(\Sorts,\Ops)}}
\newcommand{\MAlo}{\MA{(\Sorts,\Ops,\emptyset)}}
%standard
\newcommand{\Sign}{\cat{Sign}}
\newcommand{\Set}{\cat{Set}}

%%% Functors
\newcommand{\Mod}{\fu{Mod}}
\newcommand{\Sen}{\fu{Sen}}
\newcommand{\sen}{\Sen}

%%% Institutions
\newcommand{\MAH}{\inst{MAH}}
\newcommand{\MAC}{\inst{MAC}}
%\newcommand{\MAPC}{\inst{MAPC}}
\newcommand{\MAP}{\inst{MAP}}%partial multialgebras
\newcommand{\PA}{\inst{PA}}%partial algebras

%%% Equivalences 
\newcommand{\quot}{\sim} % ekvivalens tegn
\newcommand{\kernel}[1]{\quot_{#1}} %% ekvivalens med subskript
\newcommand{\qu}[2]{#1/\!_{#2}}

%%% some special functions used in specifications
\newcommand{\ite}[3]{\mathit{if}\ #1\ \mathit{then}\ #2\ \mathit{else}\ #3}%if then else
% and
\newcommand{\band}{\mathrel{\mathit{and}}}% ``and''
\newcommand{\choice}{\sqcup} % Nondeterministic choice



\newcounter{COMMENT}
\def\margCom{2em}
\newenvironment{comm}{\refstepcounter{COMMENT}\begin{list}{\normalsize}
  {\leftmargin\margCom \rightmargin\margCom}}{\end{list}}
\newcommand{\com}[1]{\begin{comm}\item[{\large\bf ?\ }] 
   {\small\bf{\theCOMMENT.#1}} \dotfill{{\large\bf ?}} \end{comm}}

\newcounter{CONT}
% specifications

\newcommand{\abs}[2]{{\rm \bf abs}\ \mbox{#1}\ {\rm \bf wrt}\ #2}
\newcommand{\beh}[2]{{\rm \bf beh}\ \mbox{#1}\ {\rm \bf wrt}\ #2}
\newcommand{\deter}[1]{{\rm \bf det}\ \mbox{#1}}
\newcommand{\derive}[2]{{\rm \bf derive}\ \mbox{#1}\ {\rm \bf by}\ #2}
\newcommand{\der}[2]{{\rm \bf derive}\ \mbox{#1}\ {\rm \bf by}\ #2}
\newcommand{\trans}[2]{{\rm \bf translate}\ \mbox{#1}\ {\rm \bf along}\ #2}
\newcommand{\unite}[2]{{\rm \bf unite}\ \mbox{#1}\ {\rm \bf and}\ \mbox{#2}}

\newcommand{\Sort}{\mbox{\it Sorts}}
\newcommand{\Op}{\mbox{\it Ops}}

%--------------------------------------------------------------------
% symbols

\newcommand{\simH}{\by\sim H}
\newcommand{\simV}{\by\sim F}
\newcommand{\simC}{\by\sim C}
\newcommand{\simP}{\by\sim P}

\newcommand{\nsimP}{\by{\nsim} P}
\newcommand{\nsimC}{\by{\nsim} C}

\newcommand{\vv}[1]{{\cal V}_{#1}}
\newcommand{\vm}{\vv M}
\newcommand{\vn}{\vv N}
\newcommand{\va}{\vv A}

%% \newcommand{\Det}{\mbox{\sf Det}}
%% \newcommand{\CDet}{\mbox{\sf CDet}}
\newcommand{\Beh}[1]{\mbox{\sf Beh}(#1)}
\newcommand{\Abs}[1]{\mbox{\sf Abs}(#1)}
%% \newcommand{\Mquo}{\mbox{$/\!\!/$}}
\newcommand{\Eq}{\mbox{Eq}}
\newcommand{\imp}{\mbox{$\sim\!\leadsto$}}
\newcommand{\val}[1]{\mbox{$\overline{#1}$}}
\newcommand{\Obs}{\rm Obs}
\newcommand{\nlet}[3]{[{\sf let}\ #1 := #2\ {\sf in}\ #3\ {\sf ni}]}
\newcommand{\ra}{\rightarrow}
\newcommand{\lc}{\langle}
\newcommand{\rc}{\rangle}
\newcommand{\ec}{\sim_c}
\newcommand{\er}{\sim_r}
\newcommand{\ep}{\sim_p}
\newcommand{\obs}{\equiv_{\Obs}}
\newcommand{\sqs}{\sqsubseteq}
\newcommand{\close}[1]{|\!|#1|\!|}
\newcommand{\smodels}{\:|\!\!\!\approx}
\newcommand{\kap}{\kappa}

%--------------------------------------------------------------------
% pretty definitions

\newcommand{\givedef}[2]
            {\begin{tabular}{p{1.25cm}p{0.35cm}p{13.75cm}}
                    #1 & {\rm iff} & #2 
             \end{tabular} \\[2mm] }

%--------------------------------------------------------------------
% environments

%% \newenvironment{Proof}
%%                {\setlength{\parskip}{5pt}
%%                 \setlength{\parindent}{0pt}
%%                 {\sc Proof:}}{\hfill{\sc qed}$\:$\raisebox{-1pt}{$\Box$}}


\newcommand{\support}{This research was supported by the Polish
Research Committee and by the Norwegian Research Council.}

\input{xypic}
%%% \xyoption{arc}
\definemorphism{none}{}{}{} 
 % use as \rnone|= \lunone|\sim to get just =,\sim


%--------------------------------------------------------------------
% the document

\begin{document} 

%%\

%%\vspace{2cm}

%% \begin{center}
%% {\Large\bf Behavioural Satisfaction and Equivalence \\
%%         in Nondeterministic Algebras\footnote{\support} }\\[0.8cm] 
%% {\large Marcin Bia\l{}asik }\\[0.4cm] 
%%         Institute of Computer Science \\
%%         Polish Academy of Sciences \\
%%         ul.\ Ordona 21, Warsaw, Poland \\
%%         {\tt marcinb@ipipan.waw.pl} \\
\today 
%% \end{center}
%% \medskip

%==============================================================================
%
%

\section{Abs $>$ ND, and so ND+Abs $>$ ND}
This does follow from the theorem 4.1 from \cite{behImpos}. Although
the theorem there was proved for a restricted first-order language (no
predicate symbols except equality), its proof, based on the
Ultrafilter Theorem, can be applied directly to a first-order language involving
also predicate symbols, in particular, a language of ND specifications.

\section{assuming for determinism:}
\[\begin{array}{rcl}
\SMod(\beh{SP}\sim) & = & \SMod(\abs{SP}\equiv)\ =\ \Abs {SP}\\
 \{A: A/_\sim\in\SMod(SP)\}  & = & \{A: \exists B\in\SMod(SP): A\equiv
B\} \\
 & = & \SMod(\{t=s:SP\vdash t=s:
t,s\in T(X_{V})_V\})
\end{array}
\]
and 
\eq{\label{de:viF}
\begin{array}{rcl}
A\equiv B & \iff & {\rm exists\ a\ set\ }X\ 
         {\rm of\ }\vis{\rm -variables,\ and} \\ 
 & & \vis{\rm -surjective\ valuations\ }
      \alpha:X\sur \vis^A,\ \beta:X\sur \vis^B, \\
 & & {\rm such\ that,\ for\ all\ } s,t\in(\XTerms)_\vis: 
% \\ & & 
A\models_\alpha s\odot t \Iff B\models_\beta s\odot t
\end{array}
}
One might use a more general form admitting some formulae $\phi$ instead of
$s\odot t$. In the deterministic case $\odot$ is simply $=$, but later we
may use some other relation.

\section{three (++?) levels of definition:}
Sorts of a signature are partitioned into $\vis$isible and $\hid$idden. A {\em visible term}
is a term with no variables of $\hid$ sorts returning a value of a $\vis$ sort. 
A {\em visible context} is a visible term $\cont$ with one ``variable'' $\dum$ 
which may be of a $\hid$ sort.
\begin{enum}
\item\label{it:h} $\simH$ defined by obervations on $\hid$-values:  
 context indistinguishability
  \begin{enumerate}\MyLPar
   \item any 1-step transition -- full bisimilarity (all contexts)
   \item any transition into $\vis$ -- visible bisimilarity (visible contexts), see Fig.~\ref{fi:Vbisim},
  \end{enumerate}
\item\label{it:vf} $\simV$ defined by preservation of $\vis$-formulae: observational
 wrt. $\phi$'s iff $A\models\phi\Iff A/_{\sim}\models\phi$ --
 too rough, but... 
\item\label{it:ext} external $M\equiv N$ 
\end{enum}
Consider:
%
\begin{ite}
\item [$\neg$\ref{it:h})] $\simH$ vs. $\simV$ (formulae for \ref{it:vf})
\begin{figure}[ht]
 \[ \maly{1.5}{0.5}
 \diagram
 V:           &&&&& 0 & & 1                      &&&& 0 & & 1 
     &&&& 0 && 1 \\
 H: \xto[u]^f &&&&& x\xto[u] &\simV & y\xto[u]   &&&& & xy \xto[ul] \xto[ur] & 
     &&&& x \uto && y \ullto \uto \\
 V: \xto[u]^g &&&&& A& a \xline[ul] \xline[ur]&  &&&& B& a \xline[u] &
     &&&&  C & a \xline[ul] \xline[ur]
 \enddiagram
\]
\caption{}\label{fi:Vbisim}
\end{figure}

No matter what, we cannot have $x\simH y$ in $A$, 
since $f^A(x)\not=f^A(y)$. But in all three we do have
$f(g(a))=\{0,1\}$ -- the same $\vis$-formulae are satisfied (unless we admit bindings).
\item[?\ref{it:vf})] Ok, here I do not say what I said before....but what 
formulae, what terms? -- will get them from paths.
%
\item[?\ref{it:ext})] 
$M\equiv N \by{\Iff}{?} \CDet(M)\match\CDet(N)$ (implies \ref{it:vf}?)
\end{ite}

\section{Perhaps: hypergraphs?}
For \ref{it:ext} -- external obs-equivalence, but also for internal (?).
We will define the relation $\CDet(M)\simeq\CDet(N)$ as equality of 
observable path-patterns, $\viP(M)=\viP(N)$.

Given a multi $M$ and a term 
$h(g(v_1,v_2),v_3)$, with 
\begin{itemize}\MyLPar
\item $h(a,v_3)^{M}= \{0,1\}$, $h(b,v_3)^{M}= \{1,2,3\}$ and
$g(v_1,v_2)^{M}=\{a,b\}$,
\end{itemize} 
its hypergraph $\gr M = \ngr{ \hypa{a,v_{3}}{h}{0,1} \\ 
\hypa{{b,v_{3}}}{h}{{1,2,3}} \\ \hypa{{v_{1},v_{2}}}{g}{{a,b}} \\
\hypa{{v_{2},v_{1}}}{g}{{a,b}} }$ 

\noindent
is shown on the left
\[\maly{.5}{1}
\diagram
M:& 0 & 1 & 2 & 3   &&&&&N:&  0 & 1 & 2 & 3 \\
& \frl h \uto \xdotted[ur]|>{\tip} & & 
\frl h \ulto \uto \urto &  &&&&&&
 & & \frl h \uto \xdotted[ul]|>{\tip}  \ullto \urto \\
& a \xdotted[u]|1 & &  b \xline[u]|1 &  &&&&&& 
 & & c \xdotted[u]|1 \\
& & \frl g \xdotted[ul]|>{\tip} \urto  & &   &&&&&&  
 & \frl g \xdotted[ur]|>{\tip} \\
& v_1 \xdotted[ur] & & v_2 \xdotted[ul] &  v_3 \xdotted[uuulll]|>>>2 
\xline[uuul]|2  &&&&&&
 v_1 \xdotted[ur] & & v_2 \xdotted[ul] &  v_3 \xdotted[uuul]|2
\enddiagram
\]
On the right, there is a hypergraph for another multi $N$.

A {\em path} for a term, e.g. $h(g(v_1,v_2),v_3)$, is indicated with 
dotted arrows. It amounts to picking one unique element from the 
target of each hyperedge of the term in a consistent way.
The indicated paths on both graphs are the first two below; the last 
one is yet another path in $\gr M:$
\[\begin{array}{l@{\hspace*{2em}}l@{\hspace*{2em}}l}
\ngr{ \hypa{{v_{1},v_{2}}}{g}{{a}} \\ \hypa{a,v_{3}}{h}{1} } & 
\ngr{  \hypa{{v_{1},v_{2}}}{g}{c} \\ \hypa{c,v_{3}}{h}{1}  } &
\ngr{ \hypa{{v_{1},v_{2}}}{g}{{b}} \\ \hypa{b,v_{3}}{h}{1} } 
\end{array}
\]
The point is that, modulo the hidden values $a,b,c$, through which 
these paths pass, they are equivalent and all correspond to a
determinization of the term $h(g(v_1,v_2),v_3)$ which returns $1$.
%
\Deff{
A (directed) hypergraph is a tuple $\<N,E,arg,res\>$ where:
\begin{ite}
\item $N$ is a set of vertices and $E$ a set of edges
\item $arg:E\into N^{*}$ -- the sequence of {\em arguments} for each edge
\item $res:E\into \PSet(N)$ -- the set of {\em results} for each edge
\end{ite}
A labelled hypergraph has, in addition, a set 
of symbols $S$ and a function $lab:E\into S$. 
}
An edge is written 
$\<arg,lab,res\>$ or $\hype{arg}{lab}{res}$.
Notice that, since $arg$ returns a sequence, we allow repetitions.
%
\Deff{
The hypergraph $\gr M$ for a multi $M$ over a signature $\Sigma=\<\Sorts,\Funcs\>$ 
is given by:
\begin{ite}
\item $N=\bigcup_{S\in\Sorts}S^{M}\cup\{\cdot\}$
\item for each
$m\in\Card M,\ E(m)=\<\<\cdot\>,{\es},m\>$ 
\item for each constant $[c:\ \into S]\in\Funcs$, an edge 
$E(c)=\<\cdot,c,c^{M}\>$
\item for each function $[f:S_{1}\times\ldots\times S_{n}\into 
S]\in\Funcs$ and each sequence $\lis x$ with $x_{i}\in 
S^{M}_{i}$ such that $f^M(\lis x)\not=\es$, 
an edge $E(\lis x,f) = \<\<\lis x\>,f, f^{M}(\lis x)\>$ \vspace*{-4ex}
\end{ite}
} 
For an edge $e=\<\<\lis x\>,f, res\>$ we will often write
$arg_{i}(e)$ for its $i$-th argument. A $det$-edge is an edge $d$ with
$|res(d)|=1$ -- we write $d\din e$ if $lab(d)=lab(e)$, $arg(d)=arg(e)$
and $res(d)\in res(e)$. (In a given $\gr M$ no two edges are so
related, but we will use this notion for expressing determinizations.)
%
\Deff{\label{de:path}
The set $\Paths(M)$ of {\em paths} for a given $\Sigma$-multi $M$, is
defined inductively (for each path we define the sequence of its
$arg$uments and the $res$ult): 
\begin{enum}
\item for each
$m\in\Card M,\ P(m)=\hypd{\cdot}{\es}{m}\in\Paths(M)$ \hfill --
$arg(P(m))=\<m\>$, $res(P(m))=m$; 
\item for each constant $c$ and
 $m\in c^M$, $P=\hypd{\cdot}cm\in\Paths(M)$ \hfill -- $arg(P)=\<m\>$, $res(P)=m$; 
\item if $P_1...P_n\in\Paths(M)$, with
$m_i=res(P_i)\in S_i^M$, and $[f:S_1\times...\times S_n\into
S]\in\Funcs$, then $P=[P_1...P_n,\hypd{m_1...m_n}fm]\in\Paths(M)$ for
each $m\in f^M(m_1...m_n)$, \\ -- $arg(P)=\<arg(P_1)...arg(P_n)\>$ and
$res(P)=m$. %\vspace*{-4ex} 
\end{enum} 
%The set $\Paths_\Sigma(M)$ of {\em ground paths} 
%contains all paths without edges of type 1).  
The edges  with label $\es$ are called {\em variable}; $var(P)$ are
all variable edges in a path $P$.
} 
%
We can trivially extract from a given path $P$ the corresponding term $T(P)$, possibly,
with some elements $res(e)$ for each variable edge $e\in P$. On the other hand, 
given a term $t$ and an assignment $\alpha:var(t)\into M$, a path $\path t\alpha$ is
any path with $T(P)=t$ and respective variable edges $\hypd\cdot\es{\alpha(x)}$ for
each $x\in var(t)$.
%% \Deff{
%% Given a term $t\in\XTerms$, and an 
%% assignment $\alpha:X\into M$, a {\em path for} $t$ and $\alpha$, $\path t{\alpha}$, 
%% is a path defined inductively 
%% \begin{enumerate}\MyLPar
%%  \item for $x\in X: \path x\alpha = \hypd{\cdot}{\es}{\alpha(x)}$;
%% \item for a constant $c$,  $\path c\alpha$ is any $e\in E(c)$;
%% \item for $t=f(\lis t)$, a path $\path{t}\alpha$ is 
%% $[\ppath{P_{1}}{t_{1}}\alpha...\ppath{P_{n}}{t_{n}}\alpha,e]$ where 
%%  for each $i$, $\ppath{P_{i}}{t_{i}}\alpha$ is a path for $t_{i}$, 
%%  $arg_{i}(e)=res(\ppath{P_{i}}{t_{i}}\alpha)$ and 
%%  $e\din E(arg_{1}(e)...arg_{n}(e),f)$
%% \end{enumerate} \vspace*{-4ex}
%% }
%

Consider a function $c:V\into H$ and $f:H\times H\into V_1$ in multialgebras
$L$ and $R$ as shown in Fig.~\ref{fi:let} (we let here $f(x,y)=f(y,x)$).
\begin{figure}[ht]
\[ \maly{0.8}{1}
\diagram
L& 1 &     &&&&   R& 1 &   &&&&     & K &\\
& 0 &      &&&&   & 0 &    &&&&    0 && 1\\
\frl f \xdotted[uur]|>{\tip} & \frl f \xto[u] & \frl f \xto[uul] &&&& 
  \frl f \xto[uur]\xdotted[ur]|>{\tip} & \frl f \xto[u] \uutor & \frl f \xto[uul] \xto[ul]  &&&&  & \frl f \ulto \urto & \\
a \xdotted[u]<0.5ex> \xdotted[u]<-0.5ex> \xline[ur] & & 
 b \xline[u]<0.5ex> \xline[u]<-0.5ex> \xline[ul]  &&&&
      a \xdotted[u]<0.5ex> \xdotted[u]<-0.5ex> \xline[ur] & & 
       b \xline[u]<0.5ex> \xline[u]<-0.5ex> \xline[ul] 
 &&&&   & a \xline[u]<0.5ex> \xline[u]<-0.5ex> & \\
& \frl c \xdotted[ul]|>{\tip} \xto[ur] &  &&&&
  & \frl c \xdotted[ul]|>{\tip} \xto[ur] &  &&&&   & \frl c \uto & \\
& v \xline[u] &  &&&& & v \xline[u] &  &&&&       & v\xline[u] &
\enddiagram
\]
\caption{}\label{fi:let}
\end{figure}

The double lines correspond to repeating arguments, e.g., $f(x,x)$ which, in $L$
will return $1$, while in $R$ either $1$ or $0$. 

Given a partition of sorts into $\vis$ and $\hid$, we will abstract
the hidden elements from a path. This amounts to replacing {\em all}
hidden elements occurring on the path by a dummy placeholder $\dum$.
Two (different) paths for $f(v,v)$ are indicated in Fig.~\ref{fi:let}
with dotted lines, namely:
\[ L1:\ [\hypd vca,\hypd vca, \hypd{a,a}f1]\ \ \ \ \ and\ \ \ \ \ 
   R1:\ [\hypd vca,\hypd vca, \hypd{a,a}f0]
\]
If we consider
$H$ to be the only $\hid$-sort, the visible parts of these two paths will look like
\[ V(L1)=\ [\hypd vc{\dum},\hypd vc{\dum}, \hypd{\dum,\dum}f1]\ \ \ \ \ and\ \ \ \ \ 
   V(R1)=\ [\hypd vc{\dum},\hypd vc{\dum}, \hypd{\dum,\dum}f0]
\]
They differ in that $res(V(L1))=1$ while $res(V(R1))=0$. However,
there is a path in $L$, namely $L2: 
[\hypd vca,\hypd vcb, \hypd{a,b}f0]$ with visible part $V(L2)=\
[\hypd vc{\dum},\hypd vc{\dum}, \hypd{\dum,\dum}f0]\ = V(R1)$.  In fact, if we
let $\viP(M)$ denote all visible
parts of all paths, we will here obtain $\viP(L)=\viP(R)$.
\Deff{\label{de:vip}
Given  a set $\vis\subseteq\Sorts$ 
\begin{enum}
\item the {\em visible part}, $V(P)$, of a path $P$
is obtained by replacing in all edges $e\in P$, all elements 
$x\in arg(e)$ and $x\in res(e)$ of sort $S\not\in \vis$ by $\dum$.
\item
a {\em visible pattern} is $V(P)$ for any path $P$ with $var(P)$ and $res(P)$ 
of visible sorts; the set of all visible patterns for a multi $M$ is denoted $\viP(M)$.
\item
two multi $M,N$ are behaviourally-equivalent wrt. $\vis$, $M\equiv_\vis N$
iff for all $V\in\vis: V^M\iso V^N$ and $\viP(M)=\viP(N)$.  \vspace*{-4ex}
\end{enum}
}
The last equality is actually an abuse of notation. It should rather be
written as a relation $\viP(M)\match\viP(N)$ denoting lifting of the 
bijection between 
the $\vis$-sorts to visible patterns. If $\nu:V^M\leftrightarrow V^N:\mu$
is the bijection for each $V\in\vis$
then, for any $P\in\viP(M)$, $\nu(P)$ is a visible pattern obtained by 
applying $\nu(x)$ to all $x$ occurring in $P$. Thus, $\viP(M)\match\viP(N)$
indicates that this extension yields double isomorphism: 
1) $\forall P\in\viP(M): P\iso \nu(P)$, and $\forall P\in\viP(N): P\iso \mu(P)$, 
and  2) $\nu\circ \mu= id_{\viP(M)}$ and $\mu\circ \nu= id_{\viP(N)}$.
We will use the notation $\viP(M)=\viP(N)$ or $\viP(M)\subseteq\viP(N)$ as
an abbreviation for this more elaborate formulation.
\com{
This might be relaxed so that we only require $\viP(M)\equiv_\vis \viP(N)$ 
(Egli-Milner).
}
%
Notice that $L$ and $R$ from fig.~\ref{fi:let} are equivalent also to the 
multialgebra $K$.
%
\Pro{
If $M\equiv N$ then, for any visible term $t:\lis V\into V$ the following 
diagram commutes (in all directions, $\nu:\vis^M\leftrightarrow \vis^N:\mu$ is 
the bijection on visible sorts):
\[ %\maly{0}{0.5}
\diagram
M \dnone|{\equiv} & \lis{V^M} \dto<-0.5ex>_\nu \rrto^{t^M} & & V^M \dto<-0.5ex>_\nu \\
N  & \lis{V^N} \uto<-0.5ex>_\mu \rrto^{t^N} & & V^N \uto<-0.5ex>_\mu \\
\enddiagram
\]
%and assignment
%$\alpha:X\into \under M: \nu(t^M(\clis x\alpha()))=t^N(\clis x{\nu(\alpha}({))})$, and vice versa.
}
\Cor{
If $M\equiv N$ then (\ref{de:viF}) holds with $\subseteq$ for $\odot$.
}
\com{
This is not trace-equivalence:
\begin{enum}
\item we have many-argument functions
\item (an `action' $f$ from a value $x$ may lead to different results)
\item we have $\vis$ and $\hid$ sorts and 
\item observe values of these sorts and not actions
\end{enum}
It is ako. trace-equiv if all funcs 1) have 1 argument, 2) are det
and 3) all sorts are hidden.
}
We will mostly use examples with 1-argument operations and in such cases we 
will not draw explicitly the ``label vertices'' but rather draw several 
appropriately annotated edges.

\subsection{Factorizability}
In general, there is no largest path-equivalence on an $M$ (Claim~\ref{cl:maxP})
but the class  $\Abs M$, in fact also
$\CDet(\Abs M)$, can be represented by
something like an initial object,
not necessarily ``the most nondeterministic'' one, but ...admitting most 
determinizations and where all $\hid$-operations are as deterministic as 
possible.
Instead of the usual factorizability, we will have an analogous result
with another construction of the canonical representative $\vm$ for
$\Abs M$.

Given $\viP(M)$ construct a multi $\vm$ by taking the $\vis^M$ and then, 
for each $V(P)\in\viP(M):$
\com{ OLD: replace each occurrence of $\dum$ with a new element $x_i$ --
this is $\under \vm$, and functions are defined by the obtained paths.} 
\begin{enum}
\item
for each edge $e\in V(P)$ replace all $\dum\in arg(e)$ of the same sort with the same, 
new element $z_e$ (e.g. $\hypd{x_1,\dum,x_2,\dum}f\dum$ becomes
$\hypd{x_1,z_e,x_2,z_e}f\dum$). Call the resulting pattern $ZV(P)$.
\item 
Let $ZV(P)= [\clis P{ZV}(),\hypd{\lis x}fv]$. Since $v\in\vis$ so
 $v\not=\dum$, and $x_i\not=\dum$ by 1). For any $i$ such that $res(ZV(P_{i}))=\dum$
put $res(ZV(P_{i}))=x_i$.
\item Proceeding recursively down $ZV(P)$, we will replace all occurrences
of $\dum$ by some elements (possibly some new $z_e$'s). The result is a path $\vm(V(P))$.
\end{enum}
The new elements $z_e$ will constitute the carriers of $\hid$-sorts.
Whenever, after that, some function remains undefined for some $x_i$'s, let 
$f^{\vm}(..x_i..)=\es$. 
Following should be easy to show :
\begin{enum}
\item $\vm\equiv M$, 
\item $M\equiv N\Iff \vm\iso \vn$, 
%\item $M\equiv N\Leftarrow \vm\iso \vn$, 
\item[$\vm$] \begin{ite}\MyLPar
 \item is partial; 
 \item removes ``redundant nondeterminism'':
There is only one visible pattern in $\viP(M):\hypd vg\dum\hypcont f0$, 
so $\vm$ has only one $z$. 
\begin{figure}[ht]
 \[ \maly{1}{1}
 \diagram
 V &&& \vm:&   0  &&&  M:&& 0 && \\
 H \uto_f &&&& z_1 \uto &&& a\urrto & b\urto \rrnone|\ldots && x\ulto & y\ullto \\
 V \uto_g &&&& v \uto   &&&         & & v\ullto \ulto \urto \urrto
 \enddiagram
\]
\caption{}\label{fi:nonu}
\end{figure}
\item
removes all hidden nondeterminism -- operations taking only 
$\hid$-arguments will be deterministic, the only nondeterminism remains at 
the operations with $\vis$-arguments. In fig.~\ref{fi:nmax}, for instance,
the construction -- for any of the three algebras, in particular, for
$\quot M{\sim_{2}}$ -- will yield $\vm\iso M$.
 \end{ite}
\item 
see claim~\ref{cl:Pabs}
\end{enum}


\section{Internal  $\simH$ in multis}\label{se:MCD}

Henceforth ``pattern''
means a visible pattern. 
Let's say an equivalence on values of $\hid$ sorts
\eq{\label{eq:pathobs}
\sim\ {\rm is\ a\ {\mbox{\rm path-equivalence}}\ }\Iff \viP(A)=\viP(\quot A{\sim})
}
For $x\in V$ we won't distinguish $x$ from $[x]=\{x\}$.

\Pro{\label{cl:maxP}
In general, there does not exist a largest path-equivalence $\sim$ on a multialgebra $M$.
}
\begin{Proof}
%This follows from claim~\ref{cl:max} but w
We have the ``famous'' counter-example given in fig.~\ref{fi:nmax}.
\begin{figure}[hbt]
\[ \maly{2}{0.5}
\diagram
&&&&&& &&& M &&&     &&&&&&&&&
  && \quot M{\sim_{1}} &&  &&&&&&&&&
  && \quot M{\sim_{2}} && \\
V&&&&&& 0 &&& 1 &&& 2  &&&&&&&&&
  0 && 1 && 2     &&&&&&&&& 
  0  && 1 && 2 \\
H\uto^g&&&&&& a \uto & \sim_1 & b\urto & \sim_2 & c\ulto &\sim_1 & d \uto &&&&&&&&&
  a \uto &  & bc\uto  &   & d \uto &&&&&&&&&
  & ab \ulto\urto && cd\ulto\urto & \\
V\uto^f&&&&&& & v \ulto\urto &&&& w\ulto\urto &  &&&&&&&&&
  & v \ulto\urto && w\ulto\urto &  &&&&&&&&&
  & v \uto && w\uto &  &&&
\enddiagram
\]
\caption{}\label{fi:nmax}
\end{figure}

\end{Proof}

\noindent
This is natural. Since nondeterminism on the $\hid$ sorts 
is abstracted away, one can expect that many non-isomorphic multis, with very
different operations, in particular varying degree of nondeterminism, 
on $\hid$ sorts, will display the same observable behaviour.

\com{This lack is not, perhaps, so dangerous as it may seem.
All quotients by any, also maximal path-equivalence $\sim$ will, by definition, have
the same classes of visible paths...}
%
\Pro{\label{cl:hom}
If $h:M\into N$ (weak) homomorphism then 
$\viP(M)\subseteq\viP(N)$.
}
\begin{Proo}
If $P$ is a path in $M$ then $h(P)$ is a path in $N$ 
obtained from $P$ by replacing each element $m$ by $h(m)$. But $V(P)=V(h(P))$.
\end{Proo}

\Cor{\label{co:vpmsub}
For any $M$ and any equivalence $\sim\ :\viP(M)\subseteq\viP(\quot M\sim)$.
}
%
The opposite inclusion does not hold in general, as illustrated by the example
in fig.~\ref{fi:cp}.a: in $C$ there is no path giving rise to the visible pattern
$[\hypd wf{\dum},\hypd{\dum}g0]$ which does exist in $\viP(\quot C\sim)$.
\begin{figure}[hbt]
\[ \maly{1.5}{1}
\diagram
0 & 1 &&& 0 && 1 &&&&&  V &&&&&
0 & 1 & 2             &&& 0 & 1 & 2 \\
a\uto \rnone|{\sim} & b\uto &C \rrto&&\quot C\sim & ab\ulto\urto & &&&&& H\uto^g &&&&&
a\uto &b \uto &c \uto &&& a\uto & bc\uto \urto \\
v\uto & w\uto &&& v\urto && w\ulto &&&&& V\uto^f &&&&&
v \uto &   & A         &&& v \uto & & B \\
 & && {\txt{a)}}& &&   &&&&&&&&&&      & &           & {\txt{b)}} &&  &&
\enddiagram
\]
\caption{}\label{fi:cp}
\end{figure}

\noindent
Path-equivalence in multi does not
have much to do with context indistinguishability as illustrated in
fig.~\ref{fi:Vbisim} or~\ref{fi:nmax}.
\com{
But I think that the former fits our intuition about observability 
better than the latter. I do not want to distinguish the three algebras
in fig.~\ref{fi:nmax}, nor the three from fig.~\ref{fi:ndmax}.
The nice thing is that, except for the ``patological'' cases of unreachable
hidden values, the two notions coincide in the deterministic case.
}
In fact, generalizing the notion of visible contexts for multi, 
we should be able to 
get some relations between context indistinguishability and path-equivalence.
We let $\cont$ range over visible contexts and write $\simC$ for context 
indistinguishability, i.e.
$a\simC b \Iff \forall \cont: \contA a=\contA b$. In multis, this last equality
denotes equality of the respective result sets.
Nevertheless, even if correct, the following fact is not very interesting -- 
$\simC$ is the largest path-equivalence only because
the assumption of reachability of all hidden values prevents us from identifying
a lot of things which, typically (without such an assumption), could be safely
identified.
\com{
\Cla{For any $M:\viP(\quot M{\simC})\subseteq\viP(M)$.
}
\begin{Proof}
Now, again, I believe it. But do it directly, showing that nothing can go wrong...
By reference to terms? By induction on the path-length, that all paths are reflected? 
0 nad 1 are excluded, for 2 we have 
\[ [\ [\ P_i:\hypd{\lisn {v_{i}}k}{f_i}{[x_i]}\ ]_1^n,\ \hypd{\clis x{}[]}gv\ ]
\]
What can go wrong?
\end{Proof}
}
%
\Cla{\label{cl:MPisC}
Let $M$ be a multi such that for any $H\in\hid$ and
any $h\in H^M$, there is a sequence of visible elements $v$ and
a term $t_h$ such that $t_h^M(v)=\{h\}$ (possibly,  
a constant $c^M=\{h\}$). Then 
\begin{enum}
\item\label{it:PC} for any path-equivalence on $M:\ \sim\ \subseteq\ \simC$;
\item\label{it:MsimCP} $\viP(M)=\viP(\quot M{\simC})$
\com{ -- this is now previous lemma?;}
\item\label{it:MPisC} $\simC$ is the largest path-equivalence on $M$.
\end{enum}
}
\begin{Proof}
\ref{it:PC})
Assume $a\sim b$, $t_a^M=a, t_b^M=b$ and let $\cont$ be a visible context.
If $\contA a\not=\contA b$, we can, without loss of generality, 
let $v\in \contA a\setminus \contA b$. That is, in $M$ there is a path $P_a$ for
$\contA{t_a}$ with $res(P_a)=v$ but there is no path $P_b$ for $\contA{t_b}$ 
with $res(P_b)=v$.
Since $a\sim b$, we have that $t_a^{\quot M{\sim}}=t_b^{\quot M{\sim}}$ and hence
there is a path $P_b$ in $\quot M{\sim}$ for $\contA{t_b}$ with $res(P_b)=v$. But then 
$\viP(M)\not=\viP(\quot M\sim)$, so $\sim$ is not a path-equivalence contrary to the assumption.

\ref{it:MsimCP}) ??? -- $\simC$ is NOT necessarily a tight congruence: 
in fig.~\ref{fi:ntig}, $b\simC b_1$
but $g(b)\nsimC g(b_1)$; possible $t_i\into H$ would not change anything.
\begin{figure}[hbt]
\[ \maly{1}{0.8}
\diagram
V&&& 0 & & 1 \\
H\uto_f &&& a \uto \urrto & & a_1 \ullto  & a_2 \ulto \\
H\uto_g &&& b \uto & &   b_1 \uto \urto  &
%\\  H\uto_h &&& c \uto & & c_1 \uto
\enddiagram
\]
\caption{}\label{fi:ntig}
\end{figure}


\ref{it:MPisC}) Directly from \ref{it:PC} and \ref{it:MsimCP}.
\end{Proof}

\noindent
Fig.~\ref{fi:nmax} shows that weakening the reachability assumption to
 -- for any hidden element $h$, there is 
an operation such that $h\in f^M(v)$ -- does not suffice for the existence of a 
largest path-equivalence $\sim$.~\footnote{By the way, the figure 
also shows that as long as we 
want to consider these three multis equivalent, there is no chance to get
factorisability.
% Furthermore, it indicates that it may be fruitless to search for
% a ``correspondence'', not to speak about homomorphism, which
% characterizes $\equiv$.
}


\section{Internal $\simH$ in  deterministic algebras}
The definition (\ref{eq:pathobs}) of an internal path-equivalence 
applies to deterministic algebras. 
%$\sim$ in a multi.

The three algebras in fig.~\ref{fi:vpH} are $\equiv$ having the same patterns
$[\hypd vh{\dum},\hypd{\dum}{f^{n}}{\dum},\hypd{\dum}g1]$, for each $n\geq 0$.
We have, for $a\sim b:\quot A{\sim} = C = \quot B{\sim}$
\begin{figure}[ht]
\[ \maly{1}{0.5}
\diagram
 & 1 & A    &&& B & 1 &    &&& C & 1 \\
a \urto^g \tolu^f && b \ulto \llto_f  &&& a \urto^g \rrto^f && b \ulto  \toru_f
     &&& & c \uto_g\todl_f \\
& v \ulto^h &  &&& & v \urto_h &  &&& & v \uto_h
\enddiagram
\]
\caption{}\label{fi:vpH}
\end{figure}
\com{
This actually makes sense:
\begin{enum}
\item First, for any
$P\in\viP(A)$, all $\dum$ must occur for the first time as a result
of some edge -- hidden values must be somehow ``produced'', before
they are propagated (otherwise they would be among the variables of
the pattern; see def.~\ref{de:path} and~\ref{de:vip}.2). This is the main
difference from context indistinguishability, $\simC$, which allows us
to probe hidden values directly. Here, if there is a $\hid$ sort with no
operations producing $\hid$ values, all its elements can be safely made
equivalent $x\sim y$. Then the quotient may turn out to be a multi,
but this does not seem to be problematic -- no visible test will ever
reach this nondeterminsm.
%
\com{I think this is nice, since we should not run into particular 
problems with unreachable hidden values -- they are treated uniformly.}
%
\item It seems relevant that in a deterministic algebra we do not need $\dum$ -- 
a visible pattern with $\dum$ and without is the same, since any
edge $\hypd vf{\dum}$ determines unique element $x:f(v)=x$. 
In other words, requiring visibility of our
patterns amounts to forbidding variables of $\hid$ sorts in our observation
(terms).
\item
If $g:V\into H$, $f:H\into V1$ and, say $g^A(v)=a$ while $f^A(a)=1$
and $f^A(b)=1$, checking the pattern equivalence will amount to varying
the possible result of $g^A(v)$ - somehow, nondeterminizing $A$. 
If we get $a\sim b$ (here we do),
this means that we may change our algebra so that $g^A(v)=blatex beh
$ or even
$g^A(v)=\{a,b\}$.  This seems to be the meaning of $a\simC b$ -- it
does not matter which value we pick, the end-results will be always
the same.
\end{enum}
}
Corollary~\ref{co:vpmsub} carries over to deterministic algebras.
\Cor{\label{co:vpsub}
For any $A$ and any equivalence $\sim\ :\viP(A)\subseteq\viP(\quot A\sim)$.
}
Unlike for multis, path-equivalence in deterministic algebras is closely 
related to context indistinguishability.
\Pro{\label{cl:simCP}
For any $\Sigma$-algebra $A: \viP(A)=\viP(\quot A{\simC})$
}
\begin{Proof}
Let $P$ be a path in $\quot A\simC$ such that $V(P)\in\viP(\quot A\simC)$. 
By induction on $P$ we construct a path $P'$ in $A$ with $V(P')=V(P).$
\[\begin{array}{r@{\ \ }|@{\ \ }l@{\ \ \ \ }l}
 P\ \ \ \ & \ \ \ \ P' \\ \hline
\hypd\cdot\es{[x]} & \hypd\cdot\es x &  \\
\hypd\cdot c{[x]} & \hypd\cdot c a & :{\rm arbitrary}\ a\in [x] \\
\lis P,\hypd{\clis x{}[]}f{[x]} & \lis {P'},\hypd{\lis a}fa &
:a_i=res(P'_i)\in[x_i],\  a=f^A(\lis a) \\
\end{array}
\]
Variable edges must all be visible, hence in 1. $[x]=\{x\}$.
Step 3. yields a desired result because $\simC$ is a congruence --
for any edge $f^{\quot{A}{\simC}}(\clis x{}[])=[x]$ and
any $a_i\in[x_i]$, there is an $a\in[x]$ such that $f^A(\lis a)=a$.
Trivially, $V(P')=V(P).$
 Thus $\viP(\quot A\simC)\subseteq\viP(A)$.
\end{Proof}

\noindent
Thus context indistinguishability is a special case of our path-equivalence relation.
The opposite implication does not hold. There exist path-equivalences which may
identify context-distinguishable elements.
A typical example is given in fig.~\ref{fi:cp}.b.

\Pro{\label{cl:max}
In general, there does not exist a largest path-equivalence $\sim$ on 
an $A$.
}
\begin{Proof} Both $\sim_1$ and $\sim_2$ are maximal path-equivalences.
\begin{figure}[hbt]
\[ \maly{1}{0.5}
%\definemorphism{none}{}{}{} 
 % use as \rnone|= \lunone|\sim to get just =,\sim
\diagram
0 &  & 1             &&& 0 &  1              &&& 0 &  1 \\
a\uto \rnone|{\sim_1}&b \rnone|{\sim_2}\ulto &c \uto &&& ab\uto & c\uto     &&& a\uto & bc\uto \ulto \\
v \uto &   & A    &&& v \uto & \quot A{\sim_{1}}  &&& v \uto & \quot A{\sim_{2}}
\enddiagram
\]
\caption{}\label{fi:ndmax}
\end{figure}

\end{Proof}

\noindent
In fact, the only reason for this are unreachable $\hid$-elements. 
\Pro{\label{cl:PisC}
Let $A$ be a (deterministic) algebra such that for any $H\in\hid$ and
any $h\in H^A$, there is a sequence of visible elements $v$ and
a term $t$ such that $t^A(v)=h$ (possibly,  
a constant $c^A=h$). Then there is a
largest path-equivalence, in fact, $\simC$.
}
\begin{Proof}
From claim~\ref{cl:simCP} we know that $\simC$ is a path-equivalence.
Assume that there is another equivalence, $\sim\ \not\subseteq\ \simC$,
such that 
for some $a,b\in H^A:a\sim b \land a\nsimC b$. 
The latter conjunct means that there
is a visible context $\cont$ with $\contA a\not=\contA b$. 
By assumption, let $t(v)=a$ and 
let $P$ be the path for
$\contA{t(v)}$, in particular, $res(V(P))=res(P)=\contA a$ in $A$.
Due to determinacy of $A$ it is the only result of $V(P)$.
But in $\quot A\sim$ there are at least two visible patterns 
for $\contA{t(v)}$, one with $res(V(P_a))=[\contA a]$ and another with 
$res(V(P_b))= [\contA b]$.
 Hence $\viP(\quot A\sim)\not=\viP(A)$.
\end{Proof}

\noindent
We can carry out the construction of $\va$ in the deterministic case 
but, typically, the result will be a partial algebra. This is not
necessarily a vice -- in fact, all our concepts can be applied to partial
algebras.

\section{Multis and $\CDet$'s}
Thus, what we mean by $\CDet(M)\simeq\CDet(N)$ is $M\equiv N$. Notice that this
does not mean $\CDet(M)=\CDet(N)$. In fig.~\ref{fi:nmax}, for instance, 
an $A$ with $f^A(v)=b$ and $f^A(w)=c$ is in $\CDet(M)$ but $A\not\in\CDet(\quot M{\sim_1})$. $A$ is not even isomorphic to any $B\in\CDet(\quot M{\sim_{1}})$.
%
\begin{enum}
\item\label{it:aa}
$A\in\CDet(M) \land A\equiv B \impl \exists N: N\equiv M\land B\in\CDet(N)$
%% \setcounter{CONT}{\theenumi}
%% \end{enumerate}
%% and:
%% \begin{enumerate}\MyLPar \setcounter{enumi}{\theCONT}
\item\label{it:ee}
$A\in\CDet(N) \land N\equiv M \impl \exists B\in\CDet(\vm):B\iso A$ 
  and\\
  $A\in\CDet(\vm) \impl \exists N\equiv M : A\in\CDet(N)$
\item 
from \ref{it:aa} and \ref{it:ee} we should get:
 $A\in\CDet(\vm) \land A\equiv B \impl B\in\CDet(\vm)$
\item $A\in\CDet(M) \impl \Card A\subseteq \under M$ ($\Card A=\under M$ -- no) \\
$A\in\CDet(M) \Iff \forall f,\lis a\in\Card A :
E^A(\lis a,f)\din E^M(\lis a,f)$,
in particular, {\em any} selection of $det$-edges among the edges of 
$\gr M$ yields an $A\in\CDet(M)$.
\item
Obviously $A,B\in\CDet(M)\not\impl A\equiv B$.
\item $A\in\CDet(M)\land A\equiv B \not\impl B\in\CDet(M)$.
\item 
$M\equiv N \not\impl \CDet(M)\equiv\CDet(N)$, i.e., \\
$\forall A\in\CDet(M)\exists B\in\CDet(N):A\equiv B\ (\viP(A)=\viP(B))\ \land$ 
vice versa.
\begin{figure}[hbt]
\[ \maly{1}{0.8}
\diagram
V&&&      M  & 0 & 1 &   &&&& 0 && 1 & N\\
H\uto_f &&& x_1 \urto & x_2\urto & x_3 \ulto & x_4\ulto  &&&&  & x\ulto\urto & \\
V\uto_g &&& & v \uto\ulto  &  w\uto\urto & &&&&  v \urto && w \ulto
\enddiagram
\]
\caption{}\label{fi:MCDet}
\end{figure}

\noindent
In fig.~\ref{fi:MCDet}, $\forall B\in\CDet(N):B\models f(g(v))=f(g(w))$ -- 
$\CDet(N)=\{B_1,B_2\}$ with $f^{B_{1}}(x)=0$ and $f^{B_{2}}(x)=1$. None of these
is $\equiv A\in\CDet(M)$ where $g^A(v)=x_1$, $g^A(w)=x_4$.
%
\end{enum}

\section{Wrong, undesirable or uncertain ...}
\subsection{Context indistinguishability (ci)}
\begin{enum}
\item
Fine, we observe hidden values but where do they come from? In (ci)
I just have some access to them, no matter what they are. But in an actual program
they must be somehow produced -- by the constructors returning values of $\hid$-sorts.
Thus, given $c_1,c_2:\ \into H$, I may say that their results are distinguishable
iff for some context $\cont:\contA{c_1}\not=\contA{c_2}$, but I cannot say that 
$x$ and $y$ are distinguishable if they are arbitrary, in particular, unreachable
values of $\hid$-sort. Or: whatever is unreachable is indistinguishable.

Saying that two terms $t_1,t_2:V\into H$ are $\simC$ iff for each context $\cont:
\contA{t_1}=\contA{t_2}$ is just a special case of (ci) but it is too strong for
nondeterminism.
\item
Is too strong disabling identification in fig.~\ref{fi:Vbisim}.
\item
In fig.~\ref{fi:let}, in $L:a\simC b$, for the only contexts are $f(c(v),\dum),
f(\dum,c(v))$ which yield $\{0,1\}$, and $f(\dum,\dum)$ yielding $0$.
In $R$ all these yield $\{0,1\}$. Thus $b\in L$ is distinguishable from $b\in R$
by the context $f(\dum,\dum)$.
\end{enum}
%
\subsection{another one}
$x\sim y\in H$ iff they are returned by the same terms and are 
contex-indistinguishable, i.e. 
\[\forall t\in(\VTerms{X_{\vis}})_H\ \forall v : x\in t^M(v)\Iff y\in t^M(v)
\land
\forall \cont\in(\VTerms{X_{\vis},\dum})_\vis\ \forall v: \contA x=\contA y
\]
Already context indistinguishability is too strong disabling identification 
in fig.~\ref{fi:Vbisim}.
\subsubsection*{I suspect it to be the same as this:}
\Deff{
Given an $m\in \under M$, define paths {\em passing through} $m$, 
$P(m)\subseteq P(M)$:
\begin{enum}
\item $[\hypd \cdot\es x] \in P(m) \Iff x=m$
\item $[\hypd \cdot cx] \in P(m) \Iff x=m$
\item $[\lis P, \hypd{\lis m}fx]\in P(m)  \Iff (P_i\in P(m)$ for some 
$1\leq i\leq n$ or $x=m)$.
\end{enum}
Then $\viP(m) = \{V(P):P\in P(m)\}$ and $x\simP y \Iff \viP(x)=\viP(y)$.
}
We should have:\ 
\Cla{
$\simP$ is a path-equivalence (obvious?), i.e., 
$\viP(M) = \viP(\quot M{\simP})$.
}
In fig.~\ref{fi:nmax} $\sim_2\ =\ \simP$, in particular, we cannot identify
$a\sim_1 b$ since they do have different visible paths passing through them. 
In fig.~\ref{fi:Vbisim}, I cannot identify $x$ and $y$.
[This, I think, makes possible only these identifications which do not change
anything in nondeterminism. (Is it a congruence?)]
\Cla{\label{cl:Pabs}
$\vm\iso\quot{\vm}{\simP}$, i.e. $\forall x,y\in\under\vm: x\nsimP y$
-- depends on $\vm$.}
%
$\vm$ is not initial in 
$\{\quot N{\simP}: N\equiv M\}$ (fig.~\ref{fi:nonu}: $\quot M{\simP}\iso \vm$).
In fig.~\ref{fi:Vbisim} $\vm\iso A$ and $C\iso\quot C{\simP}$ since $x\nsimP y$.
But there are two homs $A\into C$, the other one sending both $x$ and $y$ to $y$.


\subsection{Other junk}
\subsubsection*{-- hierarchy of observations: }
$\{t(x), let, \in\}\times\{\subseteq,
\frown,\ldots\}$

\subsubsection*{-- does any of the above classes have largest observational
equivalence?}
Is a rather irrelevant question. I do want the algebras from fig.~\ref{fi:nmax}
to be equivalent.

However ?
\[
\forall A\in\MAlg(\Sigma) \exists {\rm largest}\ \sim_A\  \impl 
 (A \equiv B \Iff A/_{\sim_{A}} \simeq B/_{\sim_{B}})
\]

\subsubsection*{? wish:}
$\not\sim$ monototone wrt. the number of Obs-ervations; and
\[\begin{array}{c@{\ }|@{\ }c@{\ \ }l}
{\rm rel.} & {\rm  preserved\ under\ }Det & \\ \hline
\sim &  no & f(x)\seteq f(y) \\
\not\sim & no & f(x)\cap f(y)\not=\es \\ \hline
\subseteq, \seteq, \frown & no \\
\not\frown, ',\deteq & yes & 
\end{array}
\]




\bibliographystyle{bibNo}
\bibliography{my}

\end{document}


