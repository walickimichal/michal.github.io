\documentclass[10pt]{article}
%\documentstyle[a4,fullpage,12pt,oldlfont]{article}

\makeatletter
\input{a4wide}
\makeatother


\newtheorem{example}{Example}[section]
\newtheorem{definition}[example]{Definition}
\newtheorem{theorem}[example]{Theorem}
\newtheorem{lemma}[example]{Lemma}
\newtheorem{fact}[example]{Fact}
\newtheorem{proposition}[example]{Proposition}
\newtheorem{corollary}[example]{Corollary}

%equalities
\newcommand{\eequal}{\stackrel{e}{=}}%existential equality
\newcommand{\sequal}{\stackrel{s}{=}}%strong equality
\newcommand{\wequal}{\stackrel{w}{=}}%weak equality
\newcommand{\eleq}{\mathrel{\dot{=} }}%element equality
\newcommand{\seq}{\mathrel{\asymp}}% set equality 

\newcommand{\exist}{\downarrow}% defindnes in partial algebra
\newcommand{\st}{\ast}% extra star element


%different Categorical, arrows...
\newcommand{\embd}{\hookrightarrow}% embedding of institutions 
\newcommand{\pfunc}{\hookrightarrow}% partial function
\newcommand{\ring}{\circ}% composition of arrows

\newlength{\listskip}\setlength{\listskip}{1ex plus .2ex minus .5ex}
\newlength{\eqpskip}\setlength{\eqpskip}{.5ex plus .2ex minus .2ex}
\newenvironment{eqp}{\par %\topsep\eqpskip \parskip 0ex
   \begin{tabbing}
   \quad\=\+\quad\=$\{$\ \=\kill\(}{\)\end{tabbing}
   \addvspace{-1ex plus.2ex minus.3ex}} 
\newcommand{\comment}[2]{\)\-\\[\eqpskip]$#1$\>\>$\{$\>\+\+\ignorespaces#2$\}$\-\\[\eqpskip]\(}
\newcommand{\nline}{\)\\\(\mbox{}}

%%%%%%% MICHAL

%%% comments to be removed eventually
\newcommand{\fix}{\par\noindent\hspace*{-2em}{\large\bf fix} $\uparrow$ \dotfill}
\newcommand{\fixx}[1]{\par\noindent\hspace*{-2em}{\large\bf fix} $\uparrow$ {\small
#1} \dotfill}
\newcommand{\fixd}[1]{\par\noindent\hspace*{-2em}{\large\bf fix} $\downarrow$ {\small
#1} \dotfill}
\newcommand{\isit}[1]{\par\noindent\hspace*{-2em}{\small{{\bf [is it right?} #1{\bf ]}}}\par\noindent}
\newcommand{\how}[1]{\par\noindent\hspace*{-2em}{\small{{\bf [how?} #1{\bf
]}}}\par\noindent}
\newcommand{\todo}[1]{\par\noindent\hspace*{-2em}{\small{{\bf [To Do:} #1{\bf ]}}}\par\noindent}
\newcommand{\noo}[1]{\par\noindent\hspace*{-2em}{\large\bf NO} $\uparrow$ {\small
#1} \dotfill}
%%% environments
\newtheorem{schem}[example]{Schema}
\newcommand{\MyLPar}{\parsep -.2ex plus.2ex minus.2ex\itemsep\parsep
   \vspace{-\topsep}\vspace{.5ex}}
\newenvironment{PROOF}{{\bf Proof.}}{\nopagebreak\finish}
\newenvironment{PROOFs}{{\bf Proof (sketch)}}{\nopagebreak\finish}
\newcommand{\finish}{\hspace*{\fill}\nopagebreak 
     \raisebox{-1pt}{$\Box$}\par\addvspace{1.5ex}\noindent}

%%% Specs:
\newcommand{\spec}[1]{\begin{array}[t]{rrl}#1\end{array}\vspace*{1ex}}
\newcommand{\tit}[1]{\multicolumn{3}{l}{#1}}

%%% symbols
\newcommand{\eeq}{\eequal}
\newcommand{\PSet}[1]{{\mathcal{P}}(#1)}
\newcommand{\ovr}[1]{\overline{#1}}
\newcommand{\To}{\Rightarrow}
\newcommand{\Tod}{\Leftarrow}
\newcommand{\Iff}{\Longleftrightarrow}
\newcommand{\hviss}{\Iff}
\newcommand{\ok}[1]{d_{#1}} % OK part of a sort #1

\newcommand{\by}[2]{\stackrel{#1}{#2}}
\newcommand{\Toby}[1]{\ \by{#1}{\Longrightarrow}\ }

\newcommand{\adj}{\mathrel{\small\dashv}} %adjunction


%%% use these for signatures
\newcommand{\Sorts}{{S}} 
\newcommand{\Ops}{\Omega}
\newcommand{\ndc}{\Pi}
\newcommand{\err}{E}
\newcommand{\sign}{(\Sorts,\Ops,\ndc)}
\newcommand{\POps}{P\Omega}
\newcommand{\Pops}{\POps}
\newcommand{\Terms}[1]{T(#1)}
\newcommand{\TermsS}{\Terms\Sigma}
\newcommand{\TermsSX}{\Terms{\Sigma,X}}
\newcommand{\dom}{{\bf dom}}

%%% general categorical concepts
\newcommand{\cat}[1]{{\bf #1}} %put it around any category 
\newcommand{\inst}[1]{{\mathcal{#1}}} %put it around any institution: needs $_$
\newcommand{\fu}[1]{{\sf {#1}}} %put it around any functor
\newcommand{\thr}[1]{{\bf #1}} %put it around any Theory
\newcommand{\obj}[1]{|#1|} %objects of a category
\newcommand{\natt}{\mathrel{\Longrightarrow}} %natural transformation
%%% Categories
%multialgebras
\newcommand{\MA}[1]{\cat{MAlg}_{#1}}
\newcommand{\MAS}{\cat{MAlg}_{\Sigma}} %partial multialgebras
\newcommand{\PMA}[1]{\cat{PMAlg}_{#1}}
\newcommand{\PMAS}{\cat{PMAlg}_{\Sigma}} %partial algebras
\newcommand{\PAl}[1]{\cat{PAlg}_{#1}}
%used little but still...
\newcommand{\PAlo}{\PAl{(\Sorts,\Ops)}}
\newcommand{\MAlo}{\MA{(\Sorts,\Ops,\emptyset)}}
%standard
\newcommand{\Sign}{\cat{Sign}}
\newcommand{\Set}{\cat{Set}}

%%% Functors
\newcommand{\Mod}{\fu{Mod}}
\newcommand{\Sen}{\fu{Sen}}
\newcommand{\sen}{\Sen}

%%% Institutions
\newcommand{\MAH}{\inst{MAH}}
\newcommand{\MAC}{\inst{MAC}}
%\newcommand{\MAPC}{\inst{MAPC}}
\newcommand{\MAP}{\inst{MAP}}%partial multialgebras
\newcommand{\PA}{\inst{PA}}%partial algebras

%%% Equivalences 
\newcommand{\quot}{\sim} % ekvivalens tegn
\newcommand{\kernel}[1]{\quot_{#1}} %% ekvivalens med subskript
\newcommand{\qu}[2]{#1/\!_{#2}}

%%% some special functions used in specifications
\newcommand{\ite}[3]{\mathit{if}\ #1\ \mathit{then}\ #2\ \mathit{else}\ #3}%if then else
% and
\newcommand{\band}{\mathrel{\mathit{and}}}% ``and''
\newcommand{\choice}{\sqcup} % Nondeterministic choice



% specifications

\newcommand{\abs}[2]{{\rm \bf abstract}\ \mbox{#1}\ {\rm \bf wrt}\ #2}
\newcommand{\beh}[2]{{\rm \bf behaviour}\ \mbox{#1}\ {\rm \bf wrt}\ #2}
\newcommand{\deter}[1]{{\rm \bf determinize}\ \mbox{#1}}
\newcommand{\derive}[2]{{\rm \bf derive}\ \mbox{#1}\ {\rm \bf by}\ #2}
\newcommand{\der}[2]{{\rm \bf derive}\ \mbox{#1}\ {\rm \bf by}\ #2}
\newcommand{\trans}[2]{{\rm \bf translate}\ \mbox{#1}\ {\rm \bf along}\ #2}
\newcommand{\unite}[2]{{\rm \bf unite}\ \mbox{#1}\ {\rm \bf and}\ \mbox{#2}}

\newcommand{\Sort}{\mbox{\it Sorts}}
\newcommand{\Op}{\mbox{\it Ops}}

%--------------------------------------------------------------------
% symbols

\newcommand{\simH}{\by\sim H}
\newcommand{\simV}{\by\sim F}

%% \newcommand{\Det}{\mbox{\sf Det}}
%% \newcommand{\CDet}{\mbox{\sf CDet}}
\newcommand{\Beh}{\mbox{Beh}}
\newcommand{\Abs}{\mbox{Abs}}
%% \newcommand{\Mquo}{\mbox{$/\!\!/$}}
\newcommand{\Eq}{\mbox{Eq}}
\newcommand{\imp}{\mbox{$\sim\!\leadsto$}}
\newcommand{\val}[1]{\mbox{$\overline{#1}$}}
\newcommand{\Obs}{\rm Obs}
\newcommand{\nlet}[3]{[{\sf let}\ #1 := #2\ {\sf in}\ #3\ {\sf ni}]}
\newcommand{\ra}{\rightarrow}
\newcommand{\lc}{\langle}
\newcommand{\rc}{\rangle}
\newcommand{\ec}{\sim_c}
\newcommand{\er}{\sim_r}
\newcommand{\ep}{\sim_p}
\newcommand{\obs}{\equiv_{\Obs}}
\newcommand{\sqs}{\sqsubseteq}
\newcommand{\close}[1]{|\!|#1|\!|}
\newcommand{\smodels}{\:|\!\!\!\approx}
\newcommand{\kap}{\kappa}

%--------------------------------------------------------------------
% pretty definitions

\newcommand{\givedef}[2]
            {\begin{tabular}{p{1.25cm}p{0.35cm}p{13.75cm}}
                    #1 & {\rm iff} & #2 
             \end{tabular} \\[2mm] }

%--------------------------------------------------------------------
% environments

%% \newenvironment{Proof}
%%                {\setlength{\parskip}{5pt}
%%                 \setlength{\parindent}{0pt}
%%                 {\sc Proof:}}{\hfill{\sc qed}$\:$\raisebox{-1pt}{$\Box$}}

\newcommand{\Dybdy}[1]{\begin{Definition} \ \\
                       \indent {\rm #1 } \hfill$\Box$ 
                       \end{Definition} }

\newcommand{\support}{This research was supported by the Polish
Research Committee and by the Norwegian Research Council.}

\input{xypic}
%\xyoption{curve}
\xyoption{arc}
%--------------------------------------------------------------------
% the document

\begin{document} 

%%\

%%\vspace{2cm}

%% \begin{center}
%% {\Large\bf Behavioural Satisfaction and Equivalence \\
%%         in Nondeterministic Algebras\footnote{\support} }\\[0.8cm] 
%% {\large Marcin Bia\l{}asik }\\[0.4cm] 
%%         Institute of Computer Science \\
%%         Polish Academy of Sciences \\
%%         ul.\ Ordona 21, Warsaw, Poland \\
%%         {\tt marcinb@ipipan.waw.pl} \\
\today 
%% \end{center}
%% \medskip

%==============================================================================
%
%

\section{Abs $>$ ND, and so ND+Abs $>$ ND}

\section{assuming for determinism:}
\[\begin{array}{rcl}
\SMod(\beh{SP}\sim) & = & \SMod(\abs{SP}\equiv) \\
 \{A: A/_\sim\in\SMod(SP)\}  & = & \{A: \exists B\in\SMod(SP): A\equiv
B\} \\
 & = & \SMod(\{t=s:SP\vdash t=s:
t,s\in T(X_{V})_V\})
\end{array}
\]

\section{three (++?) levels of definition:}
 
\begin{enumerate}\MyLPar 
\item\label{it:h} $\simH$ defined by obervations on $H$-values
  \begin{enumerate}\MyLPar
   \item any 1-step transition -- full bisimilarity
   \item any transition into $V$ -- visible bisimilarity (see Fig.~\ref{fi:Vbisim}),
   context indistinguishability
  \end{enumerate}
\item\label{it:vf} $\simV$ defined by preservation of $V$-formulae: observational
 wrt. $\alpha$'s iff $A\models\alpha\Iff A/_{\sim}\models\alpha$ --
 too rough, but... 
\item\label{it:ext} external $M\equiv N$ 
\end{enumerate}
Consider:
%
\begin{itemize}
\item [$\neg$3.a] \ref{it:h} vs. formulae for \ref{it:vf}.
\begin{figure}[ht]
 \[ \maly{1.5}{0.5}
 \diagram
 V: &&&&& 0 & & 1 &&& 0 & & 1 \\
 H: \ar[u]^f &&&&& x\ar[u] &\simV & y\ar[u] &&& & xy \ar[ul] \ar[ur] & \\
 V: \ar[u]^g &&&&& & a \ar @{-}[ul] \ar @{-}[ur] &  &&& & a \ar @{-}[u]
 \enddiagram
\]
\caption{}\label{fi:Vbisim}
\end{figure}

No matter what, we cannot have $x\simH y$, since $f(x)\not=f(y)$. But in both we do have
$f(g(a))=\{0,1\}$ -- the same $V$-formulae are satisfied (unless we admit bindings).
\item[$\neg$3.b]
On the other hand, \ref{it:vf}: $s^M=\es\subseteq \{3\}=t^M$ and 
 $s^N=\{1,2\}\subseteq \{1,2,3\}=t^N$ -- should it $\impl M\equiv 
 N$ ? Notice that this can be achieved using $H$ on fully abstract 
 multis, with $s=f(g(a))$ and $t=h(g(a)):$
\begin{figure}[ht]
\[ \maly{1.5}{0.5}
\diagram
V: &&&&&   1 & 2& 3             
  &&&&& 1 & 2 %{\ellipse(7,3){.}} 
                & 3  \\
H: \ar^{f}_{h}[u]   &&&&& & x \ar@{{}{.}{>}}_{h}[ur] &
   &M&\by{\equiv}{?}&N&&   & x \ar^{f}[ul] \ar[u]\ar @{{}{.}{>}}_{h}[ur] &     \\
 V: \uto^g &&&&& & a \ar @{-}[u] &
   &&&&&   & a \ar @{-}[u] &     
\enddiagram
\]
\caption{}\label{fi:noVF}
\end{figure}

This also shows that even if $M$ and $N$ are fully abstract, \\
$M\equiv 
N\not\impl\forall B\in\CDet(N)\exists A\in\CDet(M):A\equiv B$.
\item[?3.c] $M\equiv N \by{\Iff}{?} \CDet(M)\iso\CDet(N)$ (implies \ref{it:vf}?)
\end{itemize}

\section{? wish:}
% \begin{itemize}\MyLPar
% \item 
$\not\sim$ monototone wrt. the number of Obs-ervations; and
\[\begin{array}{c@{\ }|@{\ }c@{\ \ }l}
{\rm rel.} & {\rm  preserved\ under\ }Det & \\ \hline
\sim &  no & f(x)\seteq f(y) \\
\not\sim & no & f(x)\cap f(y)\not=\es \\ \hline
\subseteq, \seteq, \frown & no \\
\not\frown, ',\deteq & yes & 
\end{array}
\]

\section{hierarchy of observations: }
$\{t(x), let, \in\}\times\{\subseteq,
\frown,\ldots\}$

\section{does any of the above classes have largest observational
equivalence?}
$\forall A\in\MAlg(\Sigma) \exists {\rm largest}\ \sim_A\  \impl 
 (A \equiv B \Iff A/_{\sim_{A}} \simeq B/_{\sim_{B}})$ ?

\section{Perhaps: for \ref{it:ext}}
Hypergraphs for terms: Given a multi $M$ and a term 
$h(g(v_1,v_2),v_3)$, with 
\begin{itemize}\MyLPar
\item $h(a,v_3)^{M}= \{0,1\}$, $h(b,v_3)^{M}= \{1,2,3\}$ and
$g(v_1,v_2)^{M}=\{a,b\}$,
\end{itemize} 
its hypergraph $\gr M = \ngr{ \hypa{a,v_{3}}{h}{0,1} \\ 
\hypa{{b,v_{3}}}{h}{{1,2,3}} \\ \hypa{{v_{1},v_{2}}}{g}{{a,b}} }$ 

is shown on the left
\[\maly{.5}{1}
\diagram
M:& 0 & 1 & 2 & 3   &&&&&N:&  0 & 1 & 2 & 3 \\
& \frl h \uto \ar @{{}{.}>}[ur] & & 
\frl h \ulto \uto \urto &  &&&&&&
 & & \frl h \uto \ar @{{}{.}>}[ul]  \ullto \urto \\
& a \ar @{.}[u]|1 & &  b \ar@{-}[u]|1 &  &&&&&& 
 & & c \ar @{.}[u]|1 \\
& & \frl g \ar @{{}{.}>}[ul] \urto  & &   &&&&&&  
 & \frl g \ar @{{}{.}>}[ur] \\
& v_1 \ar @{.}[ur]|1 & & v_2 \ar @{.}[ul]|2 &  v_3 \ar @{.}[uuulll]|>>>2 
\ar@{-}[uuul]|2  &&&&&&
 v_1 \ar @{.}[ur]|1 & & v_2 \ar @{.}[ul]|2 &  v_3 \ar @{.}[uuul]|2
\enddiagram
\]
On the right, there is a hypergraph for another multi $N$.

A {\em path} for a term, e.g. $h(g(v_1,v_2),v_3)$, is indicated with 
dotted arrows. It amounts to picking one unique element from the 
target of each hyperedge of the term in a consistent way.
The indicated paths on both graphs are the first two below; the last 
one is yet another path in $\gr M:$
\[\begin{array}{l@{\hspace*{2em}}l@{\hspace*{2em}}l}
\ngr{ \hypa{{v_{1},v_{2}}}{g}{{a}} \\ \hypa{a,v_{3}}{h}{1} } & 
\ngr{  \hypa{{v_{1},v_{2}}}{g}{c} \\ \hypa{c,v_{3}}{h}{1}  } &
\ngr{ \hypa{{v_{1},v_{2}}}{g}{{b}} \\ \hypa{b,v_{3}}{h}{1} } 
\end{array}
\]
The point is that, modulo the hidden values $a,b,c$, through which 
these paths pass, they are equivalent and all correspond to a
determinization of the term $h(g(v_1,v_2),v_3)$ which returns $1$.
%
\Dybdy{
A (directed) hypergraph is a tuple $\<V,E,arg,res\>$ where:
\begin{itemize}\MyLPar
\item $V$ is a set of vertices and $E$ a set of edges
\item $arg:E\into V^{*}$ -- the sequence of {\em arguments} for each edge
\item $res:E\into \PSet(V)$ -- the set of {\em results} for each edge
\end{itemize}
A labelled hypergraph has, in addition, a set 
of symbols $S$ and a function $lab:E\into S$. An edge is written 
$\<arg,lab,res\>$.
} %\end{Definition}
Notice that, since $arg$ returns a sequence, we allow repetitions.
%
\begin{Definition}
Given a signature $\Sigma=\<\Sorts,\Funcs\>$ and a $\Sigma$-multi $M$, its hypergraph 
$\gr M$ is given by:
\begin{itemize}\MyLPar
\item $V=\bigcup_{S\in\Sorts}S^{M}\cup\{\cdot\}$
\item for each constant $[c:\ \into S]\in\Funcs$, an edge 
$E(c)=\<\cdot,c,c^{M}\>$
\item for each function $[f:S_{1}\times\ldots\times S_{n}\into 
S]\in\Funcs$ and each sequence $\lis x$ with $x_{i}\in 
S^{M}_{i}$, an edge $E(\lis x,f) = \<\<\lis x\>,f, f^{M}(\lis x)\>$
\end{itemize}
\end{Definition}
For an edge $e=\<\<\lis x\>,f, res\>$ we will often write 
$arg_{i}(e)$ for its $i$-th argument. A $det$-edge is an edge $d$ 
with $|res(d)|=1$ -- we write $d\in e$ if $lab(d)=lab(e)$, 
$arg(d)=arg(e)$ and $res(d)\in res(e)$.
%
\begin{Definition}
Given a term $t\in\XTerms$, and an 
assignment $\alpha:X\into M$, a {\em path} $\path t{\alpha}$ for $t,\alpha$ 
is a hypergraph with only $det$-edges defined 
inductively:
\begin{itemize}\MyLPar
\item for $x\in X: \path x\alpha = \<\cdot,x,\alpha(x)\>$, with 
$arg(\path x\alpha)=x$ and $res(\path x\alpha)=\alpha(x)$.
\item for a constant $c$, a $\path c\alpha = e\in E(c)$, with 
$arg(\path c\alpha)=c$ and $res(\path c\alpha)=res(e)$.
\item for $t=f(\lis t)$, a path $\path{t}\alpha$ is 
$[\ppath{P_{1}}{t_{1}}\alpha...\ppath{P_{n}}{t_{n}}\alpha,e]$ where 
 \begin{itemize}
  \item for each $i$, $\ppath{P_{i}}{t_{i}}\alpha$ is a path for $t_{i}$
  \item $arg_{i}(e)=res(\ppath{P_{i}}{t_{i}}\alpha$ and 
  \item $e\in E(arg_{1}(e)...arg_{n}(e),f)$
 \end{itemize} 
We then let $arg(\path{t}\alpha)=\bigcup_{i} 
arg(\ppath{P_{i}}{t_{i}}\alpha$ and 
$res(\path{t}\alpha = res(e)$.
\end{itemize}
\end{Definition}


 Given 
$A,B\in\CDet$ with $V^{A}=V^{B}$, we define $A\simeq B$ iff 


\end{document}


