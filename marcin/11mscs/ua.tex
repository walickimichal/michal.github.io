
\documentclass[10pt]{article}

\makeatletter

\ifcase \@ptsize
    % mods for 10 pt
    \oddsidemargin  0.15 in     %   Left margin on odd-numbered pages.
    \evensidemargin 0.35 in     %   Left margin on even-numbered pages.
    \marginparwidth 1 in        %   Width of marginal notes.
    \oddsidemargin 0.25 in      %   Note that \oddsidemargin = \evensidemargin
    \evensidemargin 0.25 in
    \marginparwidth 0.75 in
    \textwidth 5.875 in % Width of text line.
\or % mods for 11 pt
    \oddsidemargin 0.1 in      %   Left margin on odd-numbered pages.
    \evensidemargin 0.15 in    %   Left margin on even-numbered pages.
    \marginparwidth 1 in       %   Width of marginal notes.
    \oddsidemargin 0.125 in    %   Note that \oddsidemargin = \evensidemargin
    \evensidemargin 0.125 in
    \marginparwidth 0.75 in
    \textwidth 6.125 in % Width of text line.
\or % mods for 12 pt
    \oddsidemargin -10 pt      %   Left margin on odd-numbered pages.
    \evensidemargin 10 pt      %   Left margin on even-numbered pages.
    \marginparwidth 1 in       %   Width of marginal notes.
    \oddsidemargin 0 in      %   Note that \oddsidemargin = \evensidemargin
    \evensidemargin 0 in
    \marginparwidth 0.75 in
    \textwidth 6.375 true in % Width of text line.
\fi

\voffset -2cm
\textheight 22.5cm

\makeatother

%\makeatletter
%\show\
%\makeatother
\newcommand{\ite}[1]{\item[{\bf #1.}]}
\newcommand{\app}{\mathrel{\scriptscriptstyle{\vdash}}}
\newcommand{\estr}{\varepsilon}
\newcommand{\PSet}[1]{{\cal P}(#1)}
\newcommand{\ch}{\sqcup}
\newcommand{\into}{\to}
\newcommand{\Iff}{\Leftrightarrow}
\renewcommand{\iff}{\leftrightarrow}
\newcommand{\prI}{\vdash_I}
\newcommand{\pr}{\vdash}
\newcommand{\ovr}[1]{\overline{#1}}

\newcommand{\cp}{{\cal O}}

% update function/set
%\newcommand{\upd}[3]{#1\!\Rsh^{#2}_{\!\!#3}} % AMS
\newcommand{\upd}[3]{#1^{\raisebox{.5ex}{\mbox{${\scriptscriptstyle{\leftarrow}}\scriptstyle{#3}$}}}_{{\scriptscriptstyle{\rightarrow}}{#2}}} 
\newcommand{\rem}[2]{\upd{#1}{#2}{\bullet}}
\newcommand{\add}[2]{\upd {#1}{\bullet}{#2}}
%\newcommand{\mv}[3]{{#1}\!\Rsh_{\!\!#3}{#2}}
\newcommand{\mv}[3]{{#1}\:\raisebox{-.5ex}{$\stackrel{\displaystyle\curvearrowright}{\scriptstyle{#3}}$}\:{#2}}

\newcommand{\leads}{\rightsquigarrow} %AMS

\newenvironment{ites}{\vspace*{1ex}\par\noindent 
   \begin{tabular}{r@{\ \ }rcl}}{\vspace*{1ex}\end{tabular}\par\noindent}
\newcommand{\itt}[3]{{\bf #1.} & $#2$ & $\impl$ & $#3$ \\[1ex]}
\newcommand{\itte}[3]{{\bf #1.} & $#2$ & $\impl$ & $#3$ }
\newcommand{\itteq}[3]{\hline {\bf #1} & & & $#2=#3$ }
\newcommand{\itteqc}[3]{\hline {\bf #1} &  &  & $#2=#3$ \\[.5ex]}
\newcommand{\itteqq}[3]{{\bf #1} &  &  & $#2=#3$ }
\newcommand{\itc}[2]{{\bf #1.} & $#2$ &    \\[.5ex]}
\newcommand{\itcs}[3]{{\bf #1.} & $#2$ & $\impl$ & $#3$  \\[.5ex] }
\newcommand{\itco}[3]{   & $#1$ & $#2$  & $#3$ \\[1ex]}
\newcommand{\itcoe}[3]{   & $#1$ & $#2$  & $#3$}
\newcommand{\bit}{\begin{ites}}
\newcommand{\eit}{\end{ites}}
\newcommand{\na}[1]{{\bf #1.}}
\newenvironment{iten}{\begin{tabular}[t]{r@{\ }rcl}}{\end{tabular}}
\newcommand{\ass}[1]{& \multicolumn{3}{l}{\hspace*{-1em}{\small{[{\em Assuming:} #1]}}}}

%%%%%%%%% nested comp's
\newenvironment{itess}{\vspace*{1ex}\par\noindent 
   \begin{tabular}{r@{\ \ }lllcl}}{\vspace*{1ex}\end{tabular}\par\noindent}
\newcommand{\bitn}{\begin{itess}}
\newcommand{\eitn}{\end{itess}}
\newcommand{\comA}[2]{{\bf #1}& $#2$ \\ }
\newcommand{\comB}[3]{{\bf #1}& $#2$ & $#3$\\ }
\newcommand{\com}[3]{{\bf #1}& & & $#2$ & $\impl$ & $#3$\\[.5ex] }

\newcommand{\comS}[5]{{\bf #1} 
   & $#2$ & $#3$ & $#4$ & $\impl$ & $#5$\\[.5ex] }

%%%%%%%%%%%%%%%%
\newtheorem{CLAIM}{Proposition}[section]
\newtheorem{COROLLARY}[CLAIM]{Corollary}
\newtheorem{THEOREM}[CLAIM]{Theorem}
\newtheorem{LEMMA}[CLAIM]{Lemma}
\newcommand{\MyLPar}{\parsep -.2ex plus.2ex minus.2ex\itemsep\parsep
   \vspace{-\topsep}\vspace{.5ex}}
\newcommand{\MyNumEnv}[1]{\trivlist\refstepcounter{CLAIM}\item[\hskip
   \labelsep{\bf #1\ \theCLAIM\ }]\sf\ignorespaces}
\newenvironment{DEFINITION}{\MyNumEnv{Definition}}{\par\addvspace{0.5ex}}
\newenvironment{EXAMPLE}{\MyNumEnv{Example}}{\nopagebreak\finish}
\newenvironment{PROOF}{{\bf Proof.}}{\nopagebreak\finish}
\newcommand{\finish}{\hspace*{\fill}\nopagebreak 
     \raisebox{-1ex}{$\Box$}\hspace*{1em}\par\addvspace{1ex}}
\renewcommand{\abstract}[1]{ \begin{quote}\noindent \small {\bf Abstract.} #1
    \end{quote}}
\newcommand{\B}[1]{{\rm I\hspace{-.2em}#1}}
\newcommand{\Nat}{{\B N}}
\newcommand{\bool}{{\cal B}{\rm ool}}
\renewcommand{\c}[1]{{\cal #1}}
\newcommand{\Funcs}{{\cal F}}
%\newcommand{\Terms}{{\cal T}(\Funcs,\Vars)}
\newcommand{\Terms}[1]{{\cal T}(#1)}
\newcommand{\Vars}{{\cal V}}
\newcommand{\Incl}{\mathbin{\prec}}
\newcommand{\Cont}{\mathbin{\succ}}
\newcommand{\Int}{\mathbin{\frown}}
\newcommand{\Seteq}{\mathbin{\asymp}}
\newcommand{\Eq}{\mathbin{\approx}}
\newcommand{\notEq}{\mathbin{\Not\approx}}
\newcommand{\notIncl}{\mathbin{\Not\prec}}
\newcommand{\notCont}{\mathbin{\Not\succ}}
\newcommand{\notInt}{\mathbin{\Not\frown}}
\newcommand{\Seq}{\mathrel{\mapsto}}
\newcommand{\Ord}{\mathbin{\rightarrow}}
\newcommand{\M}[1]{\mathbin{\mathord{#1}^m}}
\newcommand{\Mset}[1]{{\cal M}(#1)}
\newcommand{\interpret}[1]{[\![#1]\!]^{A}_{\rho}}
\newcommand{\Interpret}[1]{[\![#1]\!]^{A}}
%\newcommand{\Comp}[2]{\mbox{\rm Comp}(#1,#2)}
\newcommand{\Comp}[2]{#1\diamond#2}
\newcommand{\Repl}[2]{\mbox{\rm Repl}(#1,#2)}
%\newcommand\SS[1]{{\cal S}^{#1}}
\newcommand{\To}[1]{\mathbin{\stackrel{#1}{\longrightarrow}}}
\newcommand{\TTo}[1]{\mathbin{\stackrel{#1}{\Longrightarrow}}}
\newcommand{\oT}[1]{\mathbin{\stackrel{#1}{\longleftarrow}}}
\newcommand{\oTT}[1]{\mathbin{\stackrel{#1}{\Longleftarrow}}}
\newcommand{\es}{\emptyset}
\newcommand{\C}[1]{\mbox{$\cal #1$}}
\newcommand{\Mb}[1]{\mbox{#1}}
\newcommand{\<}{\langle}
\renewcommand{\>}{\rangle}
\newcommand{\Def}{\mathrel{\stackrel{\mbox{\tiny def}}{=}}}
\newcommand{\impl}{\mathrel\Rightarrow}
\newcommand{\then}{\mathrel\Rightarrow}
\newfont{\msym}{msxm10}

\newcommand{\false}{\bot}
\newcommand{\true}{\top}

\newcommand{\restrict}{\mathbin{\mbox{\msym\symbol{22}}}}
\newcommand{\List}[3]{#1_{1}#3\ldots#3#1_{#2}}
\newcommand{\col}[1]{\renewcommand{\arraystretch}{0.4} \begin{array}[t]{c} #1
  \end{array}}
\newcommand{\prule}[2]{{\displaystyle #1 \over \displaystyle#2}}
\newcounter{ITEM}
\newcommand{\newITEM}[1]{\gdef\ITEMlabel{ITEM:#1-}\setcounter{ITEM}{0}}
\makeatletter
\newcommand{\Not}[1]{\mathbin {\mathpalette\c@ncel#1}}
\def\LabeL#1$#2{\edef\@currentlabel{#2}\label{#1}}
\newcommand{\ITEM}[2]{\par\addvspace{.7ex}\noindent
   \refstepcounter{ITEM}\expandafter\LabeL\ITEMlabel#1${(\roman{ITEM})}%
   {\advance\linewidth-2em \hskip2em %
   \parbox{\linewidth}{\hskip-2em {\rm\bf \@currentlabel\
   }\ignorespaces #2}}\par \addvspace{.7ex}\noindent\ignorespaces}
\def\R@f#1${\ref{#1}}
\newcommand{\?}[1]{\expandafter\R@f\ITEMlabel#1$}
\makeatother
\newcommand{\PROOFRULE}[2]{\trivlist\item[\hskip\labelsep {\bf #1}]#2\par
  \addvspace{1ex}\noindent\ignorespaces}
\newcommand{\PRULE}[2]{\displaystyle#1 \strut \over \strut \displaystyle#2}
%\setlength{\clauselength}{6cm}
%% \newcommand{\clause}[3]{\par\addvspace{.7ex}\noindent\LabeL#2${{\rm\bf #1}}%
%%   {\advance\linewidth-3em \hskip 3em
%%    \parbox{\linewidth}{\hskip-3em \parbox{3em}{\rm\bf#1.}#3}}\par 
%%    \addvspace{.7ex}\noindent\ignorespaces}
\newcommand{\clause}[3]{\par\addvspace{.7ex}\noindent
  {\advance\linewidth-3em \hskip 3em
   \parbox{\linewidth}{\hskip-3em \parbox{3em}{\rm\bf#1.}#3}}\par 
   \addvspace{.7ex}\noindent\ignorespaces}
\newcommand{\Cs}{\varepsilon}
\newcommand{\const}[3]{\Cs_{\scriptscriptstyle#2}(#1,#3)}
\newcommand{\Ein}{\sqsubset}%
\newcommand{\Eineq}{\sqsubseteq}%


\voffset -1cm


\input xypic


\title{Homomorphisms of Power Structures}
\author{{Marcin Bia{\l}asik} \and {Micha{\l} Walicki}}
{\small{\date{Preliminary version: \today} }}


\begin{document} 

% \maketitle
\

\begin{center}
{ 
\LARGE 
Homomorphisms of Power Structures
}
\vspace{3ex}

\begin{tabular}{c@{\hspace*{5em}}c}
{\large Micha\l{} Walicki} & {\large Marcin Bia\l{}asik} \\[0.5ex]
Institute of Informatics & Institute of Computer Science \\
University of Bergen & Polish Academy of Sciences \\
{\tt michal@ii.uib.no} & {\tt marcinb@ipipan.waw.pl} 
\end{tabular}
\end{center}

{\abstract{The paper attempts a systematic study of homomorphisms of relational
structures. Such structures are compared 
% -- with respect to the possible homomorphisms -- 
to algebras of complexes and power structures (called here multialgebras), 
and arguments are given for the choice of multialgebras.
The paper begins by showing that, under reasonable restrictions on the form of 
definition of homomorphism, there are exactly nine 
compositional homomorphisms of relational structures.  
A detailed study of three among the resulting categories  
reveals one, $\MASw$ with weak homomorphisms, possessing most structural properties. 
A conjecture is proposed that, accepting
compositionality of homomorphisms and the properties of the resulting categories as
the basis of comparison, the weak homomorphisms of $\MASw$ offer the most convenient
notion among the nine possible.
The three analogous homomorphisms of multialgebras which map
elements to sets (thus, being themselves relations), yield categories with poor
structural properties.
Finally, (some) obtained results are transfered to partial algebras which are
special cases of multialgebras. A short version of this paper, summarizing its 
main results, appeared in \cite{WB97}.
}}


{\small{\tableofcontents}}\normalsize

\section{Introduction}

In the study of universal algebra, the central place occupies the pair
of ``dual'' notions of congruence and homomorphism: every congruence
on an algebra induces a homomorphism into a quotient and every
homomorphism induces a congruence on the source algebra.  Categorical
approach attempts to express {\em all} (internal) properties of
algebras in (external) terms of homomorphisms.  When
passing to relational structures or power set structures, however, the 
close correspondence of
these internal and external aspects seems to get lost.

%These (and many other convenient) facts, depending slightly on the definition 
%of congruence and relational homomorphism, do not hold for relations.  

The most common, and natural, generalisation of the definition of
homomorphism to relational structures says that a set function
$\phi:\under A\into \under B$,\footnote{Underlying sets will be used extensively to
indicate the ``bare, unstructured sets'' as opposed to power sets or other sets with
structure. For the moment, one may ignorte this notational convention.} 
where both sets are equipped with
respective relations $R^A\subseteq \under A^n$ and $R^B\subseteq
\under B^n$, is a (weak) homomorphism 
% from the relational structure $A$ to $B$ 
iff 
\eq{ \<x_1...x_n\>\in R^A\ \impl
\<\phi(x_1)...\phi(x_n)\>\in R^B
\label{eq:weak}
} 
With this definition {\em any} equivalence on $\under A$ gives rise
to a weak homomorphism and, conversely, a weak homomorphism induces,
in general, only an equivalence relation on $\under A$.  Hence this
homomorphism does not capture the notion of congruence and this is
just one example of an internal property of relational structures that
cannot be accounted for by relational homomorphisms (in various
variants).  
Probably for this reason, the early literature on
homomorphisms of relations is extremely meagre \cite{Most,c:101} and
most work on relations concerns the study of relation algebras,
various relational operators and their axiomatizations. 
Although in recent years 
several authors begun studying relational structures and their homomorphisms
 in various contexts, a general
treatement of relational homomorphisms is still missing.
This growing interest is reflected in numerous
suggestions on how the definition of such a homomorphism could be
specialized  to obtain a more useful notion.  This issue is
our main objective.

Another internal property not reflected by relational homomorphisms is
the structure of derived operators. Relations are predicates and one
may consider derived predicates built on top of a relational
structure. They form a Boolean algebra over basis relations and thus
their structure is quite independent from the actual relations
underneath.
%
Cohn \cite{Cohn} observes here that, unlike for algebras, weak
relational homomorphisms need not preserve derived predicates and
claims that this {\em ``accounts for the fact that homomorphisms play
a subordinate role in relational structures''} (p.~204).\footnote{However, it
should be stressed that the statement is made relative to the derived
predicates with complementation, and so it does not seem so convincing
when one wants to consider only positive derived predicates. In fact,
\cite{Cohn} gives a generalization of Lyndon's theorem for classes of
algebras (equivalence of axiomatization by positive sentences and
closure under homomorphisms) to relational structures (p.~238 --
equivalence of axiomatization by universal positive sentences and
closure under weak relational homomorphisms).}

This claim depends on way the notion of derived operator is applied in
relational context, in particular, how relations are composed --
Boolean combinations is but one possibility.
In a more concise, relational notation, \refp{eq:weak} is written as
$R^A;\phi\subseteq \phi;R^B$. This somehow presupposes that $R$ is a
binary relation (of course, a homomorphism is such a relation,
too) since composition $\_;\_$ has a standard definition only for
binary relations.  There seems to be no generally accepted definition
of composition of relations of arbitrary arities. In the following
study of homomorphisms, we will compose arbitrary relations (within
the structures), like $R$ above, with binary relations (obtained from
homomorphisms between the structures).  We choose to define the
composition of relations $R^A\subseteq \under A^{n+1}$, resp.
$R^B\subseteq \under B^{n+1}$ with a binary relation $\phi\subseteq
\under A\times\under B$ as the relations on $\under A^n\times\under
B$, as follows:

%\begin{eqnarray}
\eq{\begin{array}{rcl}
\<a_1...a_n,b\>\in R^A;\phi &\Iff&
\exists a\in\under A: 
\<a_1...a_n,a\>\in R^A\ \land\ \<a,b\>\in\phi
\\ 
\<a_1...a_n,b\>\in \phi;R^B &\Iff& 
\exists b_1...b_n\in\under B:
\<b_1...b_n,b\>\in R^B\ \land\ \<a_i,b_i\>\in\phi 
% \nonumber \end{eqnarray}
\end{array}
\label{eq:com}
}
%
This definition is certainly not the only possible one -- \cite{Topen:93, glenn}
contain more general suggestions.\footnote{It
can be seen as a straightforward application of composition of binary
relations if we view a tuple $\<a_1...a_n,x\>$ as a pair $\<\ovr
a,x\>$ and let $\<\ovr a,\ovr b\>\in\phi \Iff \<a_i,b_i\>\in\phi$ for
all $1\leq i\leq n$.}  The reason for this choice (which, hopefully,
will become convincing later) is our intension to treat relations in
an algebraic way. 
It allows us to view relations as set-valued functions and turns
relational structures into algebraic ones
({\em algebras of complexes} from \cite{JT1,JT2}). In particular, it admits
composition of relations of arbitrary
arities analogous to composition of functions. (For the moment let
it suffice that it is associative, i.e.\
$(\psi;R);\phi=\psi;(R;\phi)$, $(R;\psi);\phi=R;(\psi;\phi)$ and
subsumes composition of binary relations as a special case.)

Now, table~\ref{de:relhomsde} presents
a sample of proposed definitions of relational homomorphisms gathered from 
\cite{Most,Gratzer,Cohn,Los:relhoms,Pattison,relhoms,c:93,c:94}. It uses binary
relations but with the above definition (\ref{eq:com}) it may be used for relations
$R$ of arbitrary arity.
The names are taken from the articles introducing the respective
definitions and they themselves should suffice to ilustrate the
existing confusion.  
%\vspace*{-3ex}

{\small{
\begin{table}[hbt]
\[\begin{array}{rr@{\ \ }|rcl@{\ \ }|@{\ \ }rcl}
\multicolumn{2}{r@{\ \ }|}{{\rm homomorphism}\ \phi} & \multicolumn{3}{c}{{\rm relational\ def.}} 
     & \multicolumn{3}{c}{{\rm logical\ def.}\ \forall x,y:} 
  \\ \hline
1.& {\rm weak} & \Comp{\phi^-}{\Comp {R^A}\phi} & \subseteq & R^B
            & R^A(x,y) & \Rightarrow & R^B(\phi(x),\phi(y)) \\
2. & {\rm loose} & \Comp{R^A}\phi & \subseteq & \Comp\phi{R^B} & & 1. & \\
3.& {\rm full} & \Comp{\phi^-}{\Comp {R^A}\phi} & = &
            \Comp{\Comp{\Comp{\phi^-}\phi}{R^B}}{\Comp{\phi^-}\phi}
           & \exists x',y':R^A(x',y') & \Iff & R^B(\phi(x),\phi(y)) \\
4.& {\rm `strong'} & \Comp{\phi^-}{\Comp {R^A}\phi} & \supseteq &
            \Comp{\Comp{\Comp{\phi^-}\phi}{R^B}}{\Comp{\phi^-}\phi} 
           & \exists x',y':R^A(x',y') & \Leftarrow & R^B(\phi(x),\phi(y)) \\
5.& {\rm outdegree} & \Comp{R^A}{\phi} &= &\Comp{\Comp{\phi}{R^B}}{\Comp{\phi^-}\phi}
           & \exists x':R^A(x',y) & \Iff & R^B(\phi(x),\phi(y)) \\
6.& {\rm indegree} & \Comp{\phi^-}{R^A} & = & \Comp{\Comp{\Comp{\phi^-}\phi}{R^B}}{\phi^-}
             & \exists y':R^A(x,y') & \Iff & R^B(\phi(x),\phi(y)) \\
7.& {\rm `very\ strong'} & \Comp{\Comp\phi{\phi^-}}{\Comp{R^A}{\phi}} & \supseteq &
            \Comp{\phi}{R^B} 
           & \exists x',y':R^A(x',y') & \Leftarrow & R^B(\phi(x),y) \\
8.& {\rm regular} & 5. & \& & 6. & 5. & \& & 6. \\
9.& {\rm closed} & \Comp{R^A}\phi&  \supseteq & \Comp\phi{R^B} 
           & \exists y':R^A(x,y') & \Leftarrow & R^B(\phi(x),y) \\
10.& {\rm strong} & R^A&  = & \Comp\phi{\Comp{R^B}{\phi^-}} 
           &  R^A(x,y) & \Iff & R^B(\phi(x),\phi(y)) \\
11.& {\rm tight} & \Comp{R^A}\phi & = & \Comp\phi{R^B} 
            & 2. & \& & 9. \\
\multicolumn{8}{c}{} \\[-.5ex]
\multicolumn{8}{l}{{\rm -\ primed\ symbol}\ z'\ {\rm denotes\ some\ element\ such\ that\ } \phi(z')=\phi(z)} \\
\multicolumn{8}{l}{{\rm -\ for}\ \phi\subseteq\under A\times\under B,\ \phi^-\ 
 {\rm denotes\ the\ inverse\ } \<b,a\>\in\phi^-\Iff \<a,b\>\in\phi} \vspace*{-2ex}
\end{array}
\]
\caption{Some definitions of relational homomorphisms} 
\label{de:relhomsde}
\end{table}
}\normalsize}

This paper is an attempt to bring some order into this
chaos. 
Given the combinatorial possibilities of defining homomorphisms of relational 
structures, a complete classification seems hardly possible. But the very issue of
the ``criteria of usefulness'', depending on the intended applications, may be debatable.
We only hope that approaching the problem from a more algebraic perspective
may bring at least some clarification. 
Instead of listing and defending new definitions, we have chosen compositionality
of homomorphisms and the elementary properties of the resulting categories as the basis
for comparison. We believe these to be important properties and our results should be 
useful at least to those sharing this belief.

Section~\ref{se:comp}
addresses the question of composition of homomorphisms.  In
subsection~\ref{sub:comp} we give a characterization of relational
homomorphisms which are closed under composition -- in fact, most of
the suggested defintions, like most of those in
table~\ref{de:relhomsde}, do {\em not} enjoy this property which we
believe is crucial.  We also characterize equivalences associated with
various compositional homomorphisms. 
Then, in section~\ref{se:mult} we introduce {\em
multialgebras} which are relational structures with composition of
relations reflecting the traditional way of
composing functions (\ref{eq:com}).  Subsection~\ref{sub:bao} sketches the relation
between multialgebras and their quotients by (congruence) ideals as well as
Boolean algebras with operators from \cite{JT1} of which both
relational algebras \cite{JT2} and our multialgebras are special
cases.  In subsections~\ref{sub:cats}--\ref{sub:limit} we study in
detail three categories of multialgebras with three variants of
homomorphism. This rather lengthy series of proofs is summarized in
table~\ref{tab:A} on page~\pageref{tab:A} which an impatient reader
may visit directly.  In section~\ref{se:horel} we study the analogous
categories with homomorphisms being themselves relations -- again,
summarizing rather cumbersome proofs in the table~\ref{tab:B} on
page~\pageref{tab:B}.  Section~\ref{se:pa} contains some remarks on
the consequences of the results from \ref{se:mult} for partial
algebras which are special cases of multialgebras.

\subsubsection*{Preliminary definitions and notation}

A {\em relational signature\/}
$\Sigma$ is a pair $\<\Sorts,\Rels \>$ where $\Sorts$ is a set (of sorts)
and $\Rels$ is a set of relation symbols with given arities (also called
{\em type} of relation), written
$[R_i:s_1\times...\times s_n]\in \Rels$. Similarly, {\em an algebraic
signature\/} is a pair $\<\Sorts,\Funcs \>$ where 
$\Funcs$ is a set of function symbols with associated sorts and arities,
written $[f:s_1\times...\times s_n\into s]\in \Funcs$. 
A {\em relational structure\/} over a signature $\Sigma=\<\Sorts,\Rels\>$
%% or a {\em $\Sigma$-structure} 
is a pair $A= \<|A|, \Rels^A\>$, where $|A|$ is
an $\Sorts$-sorted set called a {\em carrier\/} and $\Rels^A$ is a set of
relations, such that for each $[R_i : s_1\times...\times s_n]\in \Rels$
there is $R_i^A \subseteq |A|_{s_1}\times...\times |A|_{s_n}$. 
An {\em algebra\/} over a signature $\Sigma=\<\Sorts,\Funcs\>$ 
%% or a {\em $\Sigma$-structure} 
is again a pair $A= \<|A|, \Funcs^A\>$, where $|A|$ is
an $\Sorts$-sorted set called the {\em carrier}, and $\Funcs^A$ is a set of
functions, with a function $f^A : |A|_{s_1}\times...\times |A|_{s_n} \into
|A|_s$ for each $[f : s_1\times...\times s_n\into s]\in \Funcs$.

In order to reduce the complexity of notation, we shall limit
ourselves to single sorted structures and algebras 
($\Sorts$ has only  one element)
claiming that the results carry over to a multi-sorted case. 


We will study extensively structures (algebras) over carriers being
power sets. 
For such a structure $A$ with $|A|=\PSet(\under A)$, the
set $\under A$ will be called the {\em underlying set}. 
(In case of relational structures the terms ``carrier'' and ``underlying
set'' are used to describe the same object, and so does the respective
notation.)
In general, given a (sub)set $X$ of a power set $\PSet(\under A)$, 
we will write $\under X$ for the set $\bigcup X\subseteq\under A$. 
Given a function $f:\under A\into \under B$, we will often use additive pointwise extension
without making it explicit in the notation -- for any $X\subset \under A$, we write
$f(X)$ meaning $\bigcup_{x\in X} f(x)$.
Finally, we do not make explicit the distinction
between elements and single element sets, and if $|A|=\PSet(\under
A)$, $a\in\under A$ we write $a\in A$ meaning $\{a\}\in A$.

Structures and algebras are denoted with capital letters $A,B,C$
or $M,N$.
% (sometimes we overload this notation to stand also
%for the respective carrier sets).
Relations will be denoted $R$ or $P$;
functions with $f,g,h$; morphisms between algebras or structures with
Greek letters $\phi,\psi,...$. Lower case letters $a,b,c,... m,n,... 
x,y,...$ are used for elements of carrier sets and (!) underlying sets. 

A big pain is the notation for sequences. Typically, the fact that we consider
an $n$-ary function or relation is insignificant -- a proof for 
the one-argument function/relation applies equally to many arguments. 
Since notation suffers already from many indices and superscripts, we hope that 
writing $x,a,b,m,n$ for sequences (whenever these {\em possibly may} be sequences)
won't cause any confusion.
In particular, given a function symbol $f$ with a fixed (but
not necessarily specified)
arity, the notation $f(x)$ implicitly assumes that $x$ is an appropriate sequence.
%In such cases, $x_i$ is understood as the $i$-th element of the sequence $x$.
Occassionally, we may use the explicit notation $\lis x$.
Given a sequence $x$ and a one argument function (typically, a homomorphism
 $\phi$), we are tempted to write $\phi(x)$ for the sequence
$\clis x\phi()$, and similarly $[x]$ for the sequence $\clis x{}[]$, 
$x\in A$ for $\lis x\in A$ (i.e., $x_1\in A\land...\land x_n\in A$).
Also, given a binary relation $\sim$, 
we may write $x\sim y$ for $x_1\sim y_1 \land...\land x_n\sim y_n$.

Readers should be also wary about the notion of a homomorphism between
structures and algebras. For relational structures, as it was already
presented, it is a mapping between carriers with some additional
constraints imposed. It is just the same for ``ordinary''
algebras. Confusion may arise when we come to consider algebras
of a power set structure, where various types of mappings are possible,
including the ones between underlying sets or carriers (power
sets). We will usually discuss them separately but they do come
together on some occasions. 

Given two structures (algebras) $A$ and $B$ and a homomorphism $\psi :
A \into B$, its {\em kernel\/} is typically a relation
$\congo_\psi \subset \under A\times\under A$ 
(or $\congo_\psi \subset |A|\times |A|$, depending on the type of the 
homomorphism) such that
$a\congo_\psi b$ iff $\psi(a)=\psi(b)$. The equivalence classes under
equivalence relation $\sim$ are denoted $[x]_\sim$, or just $[x]$ if 
the the relation is clear from the context.

Composition of relations/functions is written in diagrammatic order
as $f;g$ for $g(f(\_))$. For a binary relation/function $\phi$, $\phi^-$
 denotes its inverse $=\{\<y,x\>:\<x,y\>\in\phi\}$


%=====================================================================
%=====================================================================

\section{Relational structures and compositionality}
\label{se:comp}

% Aiming at an algebraic treatement of relations we will later
% (sections~\ref{se:cats}, \ref{se:horel}) study some categories where
% objects are relational structures.  


Confronted with a tremendous number of possible definitions of a
homomorphism, we believe that the property of being compositional may
serve as an important feature distinguishing the more ``relevant''
ones. Theorem~\ref{th:comp}, which is the main result of this section, 
gives an exhaustive characterization of
compositional definitions.  We
begin by giving counter-examples for compositionality of two of the
homomorphisms from table~\ref{de:relhomsde}. 
%
%\subsubsection*
\paragraph{Full homomorphisms}
%Full homomorphisms 
were considered in \cite{Most,Los:relhoms} as {\em
the} homomorphisms between relations. In a more special form, they also appear 
in the study of partial algebras~\cite{Burmeister}. In the context of partial algebras,
 it is known that these homomorphisms do not compose. But it is not clear if all the
authors were aware of this fact in the general setting. 

%In both, $\phi,\psi$ are full while $\Comp\phi\psi$ is not:

\begin{Example}\label{ex:noncomp}
Let $A,B,C$ be structures with one relation $R$ -- a) presents a 
many-sorted counter-example and b) the single-sorted case.

\[ \maly{1.3}{1}
\diagram
{\rm a)} & A & \rrto^{\phi} &&   & B & \rrto^{\psi} && & C &    %\\
  &&&&& {\rm b)} &  A\rrto^{\phi} & & B\rrto^{\psi} & &  C \\
\save\go[drr]\merge\framed<5pt>\restore a & b & c &&    
\save\go[drr]\merge\framed<5pt>\restore a & b & c &&    
\save\go[drr]\merge\framed<5pt>\restore a & b & c   %\\
  &&&&&& \save\go[dd]\merge\framed<5pt>\restore 0 \xline[rr]^\phi|\tip & & 
      \save\go[dd]\merge\framed<5pt>\restore 0 \dto \xline[rr]|\tip & &
      \save\go[dd]\merge\framed<5pt>\restore 0 \todr \\
& 0\xto[lu] \xline`d[dr][drr] & 1 \xto[lu] \xto`d[dr]`[rrr]|{\ \phi\ }[rrr]
    && &  0\xto[u]\xto[ul] \xline[dr] & 1\xto[u] \xline[d] 
    &&& 0\xto[u]\xto[ur]\xto[ul] & 1 % \\
  &&&&&&  1 \xline[rrd]|\tip & & 1 \xline[urr]^\psi|\tip & & 1 \\
%%  &&&&&&   && 1  \xline[urr]^\psi|\tip  &&  \\
&&&&&& \psi \xto`r[rrru][rrru]&& & & %\\
  &&&&&&  2 \xline[rr]^\phi|\tip & & 2 \xline[rr]^\psi|\tip & & 2
%%  &&&&&&   &&  &&
\enddiagram
\]
%
Both $\phi$ and $\psi$ are full homomorphisms. However, due to
non-surjectivity of $\phi$, the composition $\Comp\phi\psi$ is not
full.  Although $\Comp\phi\psi(0)=0$ and $\<0,c\>$, resp. 
$\<0,0\>\in R^C$ there is no
element $x\in A$ in the pre-image of $c$, resp. $0$ for which 
we would have $\<0,x\>\in R^A$.
\end{Example}
%
%\subsubsection*
\paragraph{In- and outdegree}
homomorphisms \cite{Pattison,relhoms} are not as common as the full
ones. But they are used while authors seem to be unaware of the fact
that they do not compose either:

\begin{Example}
\label{ex:degree} 
This is essentially the same example as~\ref{ex:noncomp}.b).
\[ 
\maly{0}{0.5}
\diagram
 A \rto^\phi & B  \rto^\psi & C \\
 x \rto^\phi & \save\go[d]\merge\framed<5pt>\restore y\dto|<{\rotate\tip} \rto & z \todr \\
 & y' \urto^\psi
\enddiagram
\]
Both $\phi$ and $\psi$ are obviously outdegree (and indegree, hence
also regular).
% We have $\equiv_\phi\ = id_{R_{1}}$ while $\equiv_\psi\ =\{(a,a'),(b,b')\}$.
However, their composition is not, for while $\Comp{R^A}\rho = \es$,
where $\rho=\Comp\phi\psi$, we have
$\Comp{\Comp\rho{R^C}}{\Comp{\rho^-}\rho}=\{\<x,z\>\}$.
\end{Example}

\noindent
These counter-examples are used in the proof of 
theorem~\ref{th:comp} since many cases involved there reduce to one of these two.

%----------------------------------------------------------------------------

\subsection{Compositional Homomorphisms}
\label{sub:comp}
We assume a fixed relational signature, with $R$ ranging over all relation symbols, 
and consider definitions  of homomorphisms $\phi:A\into B$ of the form
\eq{
\Delta[\phi]\ \ \Iff\ \ \ \ l_1[\phi];R^A;r_1[\phi]\ \ \any\ \ l_2[\phi];R^B;r_2[\phi] 
\label{de:compde}
}
where $l[\_]$'s and $r[\_]$'s are relational expressions
(using only relational composition
and inverse and parameterized by $\_$), and $\any$ is one of the set-relations
$\{=,\subseteq,\supseteq\}$. A definition is {\em compositional} iff
for all $\phi:A\into B$, $\psi:B\into C$, we have 
$\Delta[\phi]\ \&\ \Delta[\psi]\impl \Delta[\phi;\psi]$, i.e.:
\begin{eqnarray}
l_1[\phi];R^A;r_1[\phi] & \any & l_2[\phi];R^B;r_2[\phi]\ \ \& \nonumber \\
l_1[\psi];R^B;r_1[\psi] & \any & l_2[\psi];R^C;r_2[\psi] \\
  & \impl & l_1[\phi;\psi];R^A;r_1[\phi;\psi]\ \ \any\ \
    l_2[\phi;\psi];R^C;r_2[\phi;\psi] \nonumber
\end{eqnarray}

%% \michal{Leave out complementation for the moment}

\begin{Theorem}\label{th:comp}
A definition is compositional iff it is equivalent to one of the
following forms: 
\eq{
\begin{array}{l@{)\ \ \ }rcl@{\ \ \ \ \ \ \ \ \ \ \ \ \ }l@{)\ \ \ }rcl}
%
1 & R^A;\phi & \any & \phi;R^B    & 
2 & \phi^-;R^A;\phi & \almost & R^B  \\ 
3 &\phi^-;R^A & \almost & R^B;\phi^-  & 
4 & R^A & \almost & \phi;R^B;\phi^- 
%
% 5 & A;\phi &\subseteq &\phi;B;\phi^-;\phi & 6 &\phi^-;A & \subseteq 
%   & \phi^-;\phi;B;\phi\\
% 7 & \phi^-;A;\phi & \subseteq & B;\phi^-;\phi & 
% 8 & \phi^-;A;\phi & \subseteq & \phi^-;\phi;B     \\
% 9 & \phi^-;A;\phi & \subseteq & \phi^-;\phi;B;\phi^-;\phi  
%
\end{array}
\label{eq:conj}
} 
where $\any\ \in\{=,\subseteq,\supseteq\}$ and
$\almost\in\{=,\supseteq\}$.
\end{Theorem}

\noindent
This leaves us with 9 basic compositional definitions (more can be
obtained by their conjunctions). Inspecting the
table~\ref{de:relhomsde}, we can see that 1. and 2. define actually the same
notion, and the only other compositional definitions are 9., 10. and 11.

To prove the ``if'' part, one easily checks that 1)--4) do yield
compositional definitions. In fact, the theorem holds for {\em
any} transitive set-relation $\any$ in cases 2)--4) but for
$\antialmost$ they reduce to case 1) -- see Fact~\ref{fa:allincl}.
For instance, for 3) we verify:
\[ 
\begin{array}{rclcrcl} 
\phi^-;R^A &\any & R^B;\phi^- & 
         \& &  \psi^-;R^B & \any & R^C;\psi^- \\
\impl\ \ \psi^-;\phi^-;R^A & \any & \psi^-;R^B;\phi^- & 
         \& & \psi^-;R^B;\phi^- &\any &R^C;\psi^-;\phi^-   \\%[.5ex]
& & & \impl & (\phi;\psi)^-;R^A & \any & R^C;(\phi;\psi)^-
\end{array}
\]
Furthermore, since homomorphisms are functions we have
\eq{
\begin{array}{crcl}
a) & \phi^-;\phi;\phi^- & = & \phi^-\\ 
b) & \phi;\phi^-;\phi & = & \phi \\
c) & \phi^-;\phi & = & id_{\phi[\under A]} \\
\end{array}
\label{eq:facts}
} 

Thus, if in (\ref{eq:conj}), instead of simple $\phi$ we tried to use
some arbitrary (but well typed) relational expression
$e[\phi]\subseteq \under A\times\under B$ built from $\phi$, the
expression $e[\phi]$ would reduce just to $\phi$~!
\medskip

To restrict the number of possibilities in the proof of ``only if'',
the following observations are useful.

\begin{Fact}
\label{fa:facts}
Let $\any\ \in\{=,\subseteq,\supseteq\}$, and $\phi:A\into B$. Then
\[
\begin{array}{r@{)\ \ \ }r@{\ \ \any\ \ }l@{\ \iff\ }r@{\ \ \any\ \ }l}
1 & R^A;\phi;\phi^- & \phi;R^B;\phi^- & R^A;\phi & \phi;R^B;\phi^-;\phi \\
2 & \phi^-;R^A;\phi & R^B;\phi^-;\phi &\phi^-;R^A;\phi;\phi^- & R^B;\phi^- \\
3 & \phi^-;R^A;\phi & \phi^-;\phi;R^B  & \phi;\phi^-;R^A;\phi &\phi;R^B \\
4 & \phi^-;R^A;\phi;\phi^-  & \phi^-;\phi;R^B;\phi^- &
        \phi^-;R^A;\phi & \phi^-;\phi;R^B;\phi^-;\phi \\
5 & \phi;\phi^-;R^A & \phi;R^B;\phi^- & 
        \phi^-;R^A & \phi^-;\phi;R^B;\phi^- \\
6 &\phi;\phi^-;R^A;\phi & \phi;R^B & 
        \phi^-;R^A;\phi & \phi^-;\phi;R^B \\
7 &\phi;\phi^-;R^A;\phi & \phi;R^B;\phi^-;\phi & 
        \phi^-;R^A;\phi & \phi^-;\phi;R^B;\phi^-;\phi \\
8 &\phi;\phi^-;R^A;\phi;\phi^- & \phi;R^B;\phi^- &
        \phi^-;R^A;\phi;\phi^- & \phi^-;\phi;R^B;\phi^-
\end{array}
\]
\end{Fact}
%
\begin{Proof} 
All statements are proved essentially in the same way using equalities
a) and b) from (\ref{eq:facts}). We illustrate it on the first case
only.

1) \ $R^A;\phi;\phi^-\any \phi;R^B;\phi^- \ \ 
   \impl \ \ R^A;\phi;\phi^-;\phi \any \phi;R^B;\phi^-;\phi \ \ 
   \Iff \ \ R^A;\phi \any \phi;R^B;\phi^-;\phi$,\ \ 
   while from the right side we get \ 
   $\impl \ R^A;\phi;\phi^- \any \phi;R^B;\phi^-;\phi;\phi^- \ \ 
   \Iff \ \ R^A;\phi;\phi^-\any \phi;R^B;\phi^-$.
%% 
%% 2) $\phi^-;A;\phi \any B;\phi^-;\phi\impl 
%%   \phi^-;A;\phi;\phi^-\any \phi^-;B;\phi^-;\phi;\phi^- \Iff
%% \phi^-;A;\phi;\phi^- \any B;\phi^-$, and vice versa
%% $\phi^-;A;\phi;\phi^- \any B;\phi^- \impl \phi^-;A;\phi;\phi^-;\phi \any B;\phi^-;\phi 
%% \Iff \phi^-;A;\phi \any B;\phi^-;\phi$.
%% 
%% 3) Entirely analogous to 2).
\end{Proof}

\noindent
A special case of case 1) in the Theorem~\ref{th:comp} is
$R^A;\phi\subseteq\phi;R^B$.  As a matter of fact, this is a very powerful
case since it subsumes (or better, is equivalent to) {\em all} other
definitions using $\subseteq$!  The following fact will eliminate all
these alternatives in the subsequent proof.
%
\begin{Fact}
\label{fa:allincl}
%Let $\Delta$ stand for the following form of definition 
Consider the definitions of the form
%\[
\(\Delta:\ \ \ \La;R^A;\Ra\ \subseteq\ \Lb;R^B;\Rb\ \ \ 
\), %\]
where $l_i[\_]$, $r_i[\_]$ are arbitrary relational expressions (over 
composition and inverse)
yielding relations of the same type on both sides of $\subseteq$.
For any such definition we have $\Delta\ \Iff\
R^A;\phi\subseteq\phi;R^B$.
\end{Fact}
%
\begin{Proof}
In addition to (\ref{eq:facts}), we will use the following facts,
without mentioning which one: 
\eq{ 
R^A\subseteq R^A;\phi;\phi^-\ \ \ \ \ \ \ \ 
R^A\subseteq \phi;\phi^-;R^A\ \ \ \ \ \ \ \ 
\phi^-;\phi;R^B\subseteq R^B\ \ \ \ \ \ \ \ 
R^B;\phi^-;\phi\subseteq R^B
\label{fa:incl}
} 
Just verify that $\Phi\Iff \Delta$ for $\Phi=R^A;\phi\subseteq\phi;R^B$ and $\Delta$
with all the possible cases for $l_i[\_]$ and $r_i[\_]$. First, we
show equivalence with the other three alternatives from the theorem.
\begin{enumerate}\MyLPar \setcounter{enumi}{1}
\item 
$\Delta\ =\ \phi^-;R^A;\phi\subseteq R^B:$ \\
$\Phi\ \impl\ \phi^-;R^A;\phi\subseteq\phi^-;\phi;R^B\subseteq R^B$. \\
$\Delta\ \impl\ \phi;\phi^-;R^A;\phi;\phi^-;\phi \subseteq
    \phi;R^B;\phi^-;\phi$, \ \ 
    i.e.\ $R^A;\phi\subseteq \phi;\phi^-;R^A;\phi \subseteq 
    \phi;R^B;\phi^-;\phi \subseteq \phi;R^B$. 
\item  
$\Delta\ =\ \phi^-;R^A\subseteq R^B;\phi^-:$ \\
$\Phi\ \impl\ \phi^-;R^A;\phi\subseteq\phi^-;\phi;R^B\subseteq R^B 
  \ \impl\ \phi^-;R^A;\phi;\phi^-\subseteq R^B\phi^-\ 
\by\impl{(\ref{fa:incl})}\ \phi^-;R^A\subseteq R^B;\phi^-$. \\ 
$\Delta\ \impl\ \phi;\phi^-;R^A;\phi\subseteq \phi;R^B;\phi^-;\phi\ 
\by\impl{(\ref{fa:incl})}\  R^A;\phi\subseteq \phi;R^B$.
\item 
$\Delta\ =\ R^A\subseteq \phi;R^B;\phi^-:$ \\
$\Phi\ \impl\ R^A;\phi;\phi^-\subseteq \phi;R^B;\phi^-$ and (\ref{fa:incl}). \\ 
$\Delta\ \impl\ R^A;\phi\subseteq\phi;R^B;\phi^-;\phi\subseteq\phi;R^B$.
\end{enumerate}
%
All the rest is proved in the same vein, sometimes using
fact~\ref{fa:facts}. We just show one more case:
%(perhaps, the most complex one).
\begin{itemize}\MyLPar
\item
$\Delta\ =\ \phi^-;R^A;\phi\subseteq R^B;\phi^-;\phi:$ \\
$\Phi\ \impl\ \phi^-;R^A;\phi\subseteq \phi^-;\phi;R^B\ 
    \impl\ \phi^-;R^A;\phi;\phi^-;\phi\subseteq\phi^-;\phi;R^B;\phi^-;\phi\
    \impl\ \phi^-;R^A;\phi\subseteq R^B;\phi^-;\phi$. \\
$\Delta\ \impl\ \phi;\phi^-;R^A;\phi\subseteq\phi;R^B;\phi^-;\phi$\ 
    from which (\ref{fa:incl}) yields $F$.
\end{itemize}\vspace*{-4ex}
\end{Proof}

%% \michal{If, by $\co\phi$, we denote the complement relation of
%% $\phi$, i.e. $\{\<x,y\>:\<x,y\>\not\in\phi\}$ then, due to the 
%% fact that none of the set-relations $=$, $\subseteq$, $\supseteq$ 
%% holds between $\co\phi;\co\psi$ and $\co{\phi;\psi}$, we
%% can quickly conclude that this relational operation does not yield 
%% new compositional definitions.}

\noindent
The ``only if'' part is more tedious but still quite simple in view of
(\ref{eq:facts}) and above fact.  We show it by induction on the complexity of the
expressions $l_i[\_]$ and $r_i[\_]$, keeping in mind that both sides
must be well-typed, i.e.\ must yield relations with the same source
and target on both sides of $\any$. 
Instead of writing $R\subseteq \under X^n\times \under Y$,
we indicate the typing of relational expressions by $X\into Y$.  We
show that any expression either reduces to one of the forms
(\ref{eq:conj}) from the theorem, or else yields a non-compositional
definition. Due to equalities a) and b) from (\ref{eq:facts}), the
induction steps are actually trivial. The cases marked IH indicate
such a reduction of more complex expressions to some previous case
using these equalities.

%\newpage
\begin{Proof} 
``{\sc Only If}'' of theorem~\ref{th:comp}
\begin{enumerate}\MyLPar
\item 
$\La;R^A;\Rb \subseteq \Lb;R^B;\Rb$ reduces to 
$R^A;\phi \subseteq \phi;R^B$ which is case 1) of
Theorem~\ref{th:comp}. Hence there is no need to consider
inclusion in this direction.
\item 
$R^A;\Ra\ \almost\ldots$
  \begin{enumerate}\MyLPar
   \item $R^A \almost\ldots : A\into A$
    \begin{enumerate}\MyLPar
     \item $\phi;R^B;\phi^-$ -- case 4) of Theorem~\ref{th:comp}
     \item -- IH
     %induction step will yield $\Lb=\phi;\phi^-;\phi$ which, by c) of 
     %(\ref{eq:facts}), is the same as $\phi$, and similarly for 
     % $\Rb=\phi^-;\phi;\phi^-$.
    \end{enumerate}
   \item\label{ENb} $R^A;\phi \almost\ldots : A\into B$
    \begin{enumerate}\MyLPar
     \item $\phi;R^B$ -- case 1) of Theorem~\ref{th:comp}
     \item $\phi;R^B;\phi^-;\phi$ -- see counter-example~\ref{ex:degree}.
     \item IH
    \end{enumerate}
   \item $R^A;\phi;\phi^-\almost\ldots : A\into A$
    \begin{enumerate}\MyLPar
     \item $\phi;R^B;\phi^-$ -- this is the same as \ref{ENb}.ii 
           by fact~\ref{fa:facts}.1).
     \item IH
    \end{enumerate}
   \item IH
  \end{enumerate}
\item $\phi^-;R^A;\Ra\ \almost\ldots$ 
  \begin{enumerate}\MyLPar
   \item\label{aa} $\phi^-;R^A \almost\ldots : B\into A$
    \begin{enumerate}\MyLPar
      \item $R^B;\phi^-$ -- case 2) of Theorem~\ref{th:comp}
      \item $\phi^-;\phi;R^B;\phi^-$ -- see counter-example~\ref{ex:degree}. 
      \item IH
    \end{enumerate}
   \item\label{bbb} $\phi^-;R^A;\phi \almost\ldots :B\into B$
    \begin{enumerate}\MyLPar
     \item $R^B$ -- case 3) of Theorem~\ref{th:comp}
     \item $R^B;\phi^-;\phi$ -- see counter-example~\ref{ex:bb1}.
     \item $\phi^-;\phi;R^B$ -- to construct a counter-example just
           reverse relations in example~\ref{ex:bb1}.
     \item $\phi^-;\phi;R^B;\phi^-;\phi$ -- see counter-examples~\ref{ex:bb1} 
           or \ref{ex:noncomp}.
     \item IH
    \end{enumerate}
   \item\label{ccc} $\phi^-;R^A;\phi;\phi^-\almost\ldots :B\into A$
    \begin{enumerate}\MyLPar
     \item $R^B;\phi^-$ -- the same as \ref{bbb}.ii by fact~\ref{fa:facts}.2).
     \item $\phi^-;\phi;R^B;\phi^-$ -- the same as \ref{bbb}.iv by 
           fact~\ref{fa:facts}.4).
     \item IH
    \end{enumerate}
   \item IH
  \end{enumerate}
\item $\phi;\phi^-;R^A;\Ra\ \almost\ldots$
  \begin{enumerate}\MyLPar
   \item $\phi;\phi^-;R^A \almost \ldots :A\into A$
    \begin{enumerate}\MyLPar
      \item $\phi;R^B;\phi^-$ -- the same as \ref{aa}.ii 
              by fact~\ref{fa:facts}.5).
%     \item $\phi^-;\phi;R^B;\phi^-$
      \item IH
    \end{enumerate}
   \item $\phi;\phi^-;R^A;\phi \almost\ldots :A\into B$
    \begin{enumerate}\MyLPar
     \item $\phi;R^B$ -- the same as \ref{bbb}.iii by fact~\ref{fa:facts}.6).
     \item $\phi;R^B;\phi^-;\phi$ -- the same as \ref{bbb}.iv by 
           fact~\ref{fa:facts}.7).
     \item IH
    \end{enumerate}
   \item $\phi;\phi^-;R^A;\phi;\phi^-\almost\ldots :A\into A$
    \begin{enumerate}\MyLPar
     \item $\phi;R^B;\phi^-$ -- the same as \ref{ccc}.ii by 
           fact~\ref{fa:facts}.8).
     \item IH
    \end{enumerate}
   \item IH
  \end{enumerate}
\item IH
\end{enumerate} \vspace*{-4ex}
\end{Proof}

\begin{Example}\label{ex:bb1}
We let $\phi(x_i)=y_i$, $\psi(y_i)=z_i$ for $i=1,2$ and $\psi(y_3)=z_1$.
\[ \maly{1}{1}
\diagram
A     &\rrto^\phi &&& B   & \rrto^\psi &&& C \\
x_2              &&&& y_2 \ddrto      &&&& z_2 \\
                 &&&&                 &&&&  \\   
x_1\uuto         &&&& y_1\uuto & y_3   &&& z_1\uuto|<{\rotate\tip}
\enddiagram
\]
%
We have that $\phi^-;R^A;\phi \almost \phi^-;\phi;R^B;\phi^-;\phi$ and
similarly for $\psi$.  Yet, $\rho^-;R^A;\rho \almost
\rho^-;\rho;R^C;\rho^-;\rho$ does not hold (where $\rho=\phi;\psi$),
for the left hand side yields $\<z_1,z_2\>$ while the right hand side,
in addition, $\<z_2,z_1\>$.
\end{Example}

\noindent
Notice that, although we have used a rather special definition of
relational composition, all counter-examples involve only binary
relations. Thus, even if one defined composition of relations
differently, as long as it subsumes the composition of binary
relations, the theorem gives the maximal number of compositional
definitions of homomorphisms.

On the other hand, one might probably come up with other forms of
defining homomorphisms that are not covered by our theorem (e.g., using
complementation in addition to composition and inverse). However,
it seems that the majority (if not all) of commonly used forms do
conform to this format. Occasionally, some authors consider certain
modifications of the definitions from table~\ref{de:relhomsde} and,
for example, also require surjectivity. For instance, full outdegree
and indegree homomorphisms (3,5,6) with the extra surjectivity
requirement do compose.  This restriction is but a round-about way of
enforcing the equality $\phi^-;\phi=id_B$ (instead of the more limited
case c) of (\ref{eq:facts})), and leads to a special case of cases 2),
1) and 3) from the Theorem~\ref{th:comp}, respectively.


%------------------------------------------------------------------------

\subsection{Congruences on relational structures}
\label{sub:congrel}
Congruences of relational and power structures were studied, for instance, in
\cite{relhoms, koh1, brink93, brink97}. The latter works study primarily 
lifting of various properties of a structure to its power structure
without focusing on the notion of homomorphism.

As observed
before, any equivalence gives a rise to a (weak) homomorphism.
However, the more specific definitions from theorem~\ref{th:comp} may
lead to more specific relations. We consider first equational
definitions from the theorem, i.e.: 
\eq{
\begin{array}{l@{)\ \ \ }rcl@{\ \ \ \ \ \ \ \ \ \ \ \ \ }l@{)\ \ \ }rcl}
1 & R^A;\phi  & = & \phi;R^B    &   2 & \phi^-;R^A;\phi & = & R^B  \\  
3 &\phi^-;R^A & = & R^B;\phi^-  &   4 & R^A & = & \phi;R^B;\phi^- 
\end{array}
\label{eq:eq}
} 
and characterize these kernels which turn out to be not merely
equivalences but congruences of a sort. To proceed further we need a
notion of a quotient:

\begin{Definition}
\label{de:quo}
Given a structure $A=\<\under A,R_1^A,R_2^A...\>$ and an equivalence
relation $\congo\ \subseteq \under A\times\under A$, we define a
quotient structure $A/_{\congo} = Q$ as follows:
\begin{eqnarray*}
\under Q & = & \{[x] : x\in\under A\} \\
R_i^Q    & = & \phi^-;R_i^A;\phi
\end{eqnarray*}
where $[x] = \{y\in\under A:x\congo y\}$, and $\phi:\under A\into
\under Q$ is defined by $\phi(x)=[x]$
\end{Definition}
%
Obviously, $\phi(x)=\phi(y)$ iff $x\congo y$. In particular, we have
that $\congo\ = \phi;\phi^-$.

\begin{Prop}
\label{pro:cong}\label{le:canonichom}
Let $\congo$ be an equivalence on $A$. Moreover, let $Q$ and $\phi$ be
as in definition~\ref{de:quo}.
\[
\begin{array}{r@{\ \ \ \ }rcl@{\ \ }|@{\ \ }rcl}
& \multicolumn{3}{l}{{\rm if\ }\congo\ {\rm satisfies}} & \multicolumn{2}{c}{{\rm then}} \\ \hline
2) &   & &                               & \phi^-;R^A;\phi &=& R^Q \\
1) & \congo;R^A;\congo & = & R^A;\congo  & R^A;\phi &=& \phi;R^Q \\
3) & \congo;R^A;\congo & = & \congo;R^A  & \phi^-;R^A &=& R^Q;\phi^- \\
4) & \congo;R^A;\congo & = & R^A         & R^A &=& \phi;R^Q;\phi^-
\end{array}
\]
\end{Prop}

\begin{Proof}
2) is obvious from the definition of $Q$. For the rest, we use the
fact that $\phi$ is surjective, i.e.\ $\phi^-;\phi=id_Q$, and so 
\eq{
\phi^-;\phi;R^Q=R^Q;\phi^-;\phi=R^Q.
\label{eq:sur}
} 
We start from the definition $\phi^-;R^A;\phi=R^Q$ which implies
$\phi;\phi^-;R^A;\phi;\phi^- = \phi;R^Q;\phi^-$ and then

1) $\impl R^A;\phi;\phi^-;\phi= \phi;R^Q;\phi^-;\phi \ 
   \impl \ R^A;\phi=\phi;R^Q$.

3) $\impl \phi;\phi^-;R^A = \phi;R^Q;\phi^- \ 
   \impl \ \phi^-;\phi;\phi^-;R^A = \phi^-;\phi;R^Q;\phi^- \ 
   \impl \ \phi^-;R^A = R^Q;\phi^-$.

4) $\impl \phi;R^Q;\phi^-=\phi;\phi^-;R^A;\phi;\phi^- = R^A$.
\end{Proof}

\noindent
In 1), 3) and 4) the relation $\congo$ is not just an arbitrary
equivalence but has a flavour of a congruence:

\begin{itemize}\MyLPar
\item[1)]
can be stated as: \\
$\forall \lis a,b: (\exists b'\congo b,a'_1\congo
a_1...a'_n\congo a_n:R^A(\lis{a'},b')) \Iff (\exists b'\congo b: R^A(\lis a,b'))$,
\\ which gives: \\
$\forall \lis a,b, \lis{a'}: R^A(\lis a,b)\land a_1\congo a'_1...a_n\congo a'_n 
\impl \exists
b'\congo b:R^A(\lis{a'},b')$ \\ -- we call it a {\em tight} congruence;
% if $\ovr a\congo \ovr a'$ then 
% $\forall b:[\ R^A(a_i,b) \impl \exists b':b'\congo b\land R^A(a_i',b')\ ]$.

\item[3)]
yields a dual condition: \\ $\forall \lis a, b,b': R^A(\lis a,b) \land
b'\congo b \impl \exists a'_1\congo a_1...a'_n\congo a_n : R^A(\lis{a'},b')$;

\item[4)] 
is strongest: $\forall \lis a,b,\lis{a'},b': R^A(\lis a,b)\land
 a'_1\congo a_1...a'_n\congo a_n \land b'\congo b\impl R^A(\lis{a'},b')$. 
\end{itemize}
%
In this special situation of a homomorphism into a quotient (which is
surjective and hence satisfies (\ref{eq:sur})), the four kinds of
homomorphisms are ordered by strength: 
\eq{ 
\raisebox{.8ex}{4}
\stackrel{\displaystyle{\nearrow}}\searrow \
\stackrel{{\displaystyle{\raisebox{2.5ex}{1}} }}
{\displaystyle{\raisebox{-1ex}{3}}} \
\stackrel{\displaystyle{\searrow}}\nearrow \raisebox{.8ex}{2}
\label{eq:homimpl}
}

On the other hand, for any (at least weak) homomorphism we have the
converse:

\begin{Prop}\label{pro:congrev} 
Given a homomorphism $\phi:A\into B$, let $\congo$ be the kernel of
$\phi$

\[
\begin{array}{r@{\ \ \ \ }rcl@{\ \ }|@{\ \ }rcl}
& \multicolumn{3}{l}{{\rm if\ }\phi:A\into B{\rm\ satisfies}} & \multicolumn{3}{c}{{\rm then\ }\congo\ 
{\rm is\ an\ equivalence\ and} } \\ \hline
%2) & \phi^-;R^A;\phi &=& R^B    &   & &  \\
1) & R^A;\phi &=& \phi;R^B      & \congo;R^A;\congo & = & R^A;\congo   \\
3) & \phi^-;R^A &=& R^B;\phi^-  & \congo;R^A;\congo & = & \congo;R^A \\
4) & R^A &=& \phi;R^B;\phi^-    & \congo;R^A;\congo & = & R^A    
\end{array}
\]
\end{Prop}
\begin{Proof}

1) $\impl R^A;\phi;\phi^-=\phi;R^B;\phi^-$ which yields
$\phi;\phi^-;R^A;\phi;\phi^-=\phi;\phi^-;\phi;R^B;\phi^-$.
But $\phi;\phi^-;\phi=\phi$, and thus 
$\phi;\phi^-;R^A;\phi;\phi^-=R^A;\phi;\phi^-$, 
i.e.\ $\congo;R^A;\congo\ = R^A;\congo$.

The proofs for 3) and 4) are entirely analogous.
\end{Proof}

\noindent
There is no line for condition 2) since $\phi^-;R^A;\phi=R^B$
obviously implies that $\congo$ is an equivalence but, in fact, this
follows for any mapping $\phi$.

This is not the strongest formulation of this fact. For $\congo$ being
an equivalence it suffices, of course, that $\phi$ is a weak
homomorphism.  Furthermore, for instance 1) implies
$R^A;\phi=\phi;R^B;\phi^-;\phi$ which is sufficient to establish the
respective property of $\congo$. In general, since $\congo$ is induced
only from the image of $A$ under $\phi$, restricting the
homomorphisms' definitions on the $R^B$-side to this image (i.e.\ by
$\phi^-;\phi$) will yield the same properties of $\congo$.

Since the left column of the table expresses the only restrictions on
$\phi$, we do not have here any relationships like those in
(\ref{eq:homimpl}).

Similar results do not follow for homomorphisms defined by $\supseteq$
in place of $=$ in (\ref{eq:eq}), in particular for the closed ones.
We can uniformly replace $=$ by $\supseteq$ in proposition
\ref{pro:congrev}, but then the statements in the right column are trivial for
any mapping $\phi$.  If the target algebra is total then the kernel
may retain the flavor of congruence. However, in general such
homomorphisms induce only an equivalence relation.


%=====================================================================
%=====================================================================

\section{Multialgebras}
\label{se:mult}
Our interest in relational structures originates actually from earlier studies
of multialgebras. These can be described as relational structures with
a specific composition
of relations of arbitrary arities. This issue offers several
choices leading to possibly general and complicated solutions (see
e.g.\ \cite{glenn,Topen:93}).  Our definition \refp{eq:com} was motivated by
the wish to view relations as set-valued functions where the last,
$n$-th argument of an $n$-ary relation corresponds to an element of
the result set obtained by applying the function to the first $n-1$
arguments.  This view appears in \cite{c:101}, was elaborated in
\cite{JT1,JT2}, then in \cite{c:102} and re-emerged in recent years in
the algebraic approaches to nondeterminism
\cite{c:64,c:59,c:Bia1,c:130,survey}.  It is based on the
simple observation that any (set-valued) operation
$f:A_1\times...\times A_n\into \PSet(A)$ determines a relation
$R_f\subseteq A_1\times...\times A_n\times A$ and vice versa, via the
isomorphism: 
%
\eq{
\label{eq:isomr} 
\begin{array}{r@{\ \ \ \iso\ \ \ }l}
[A_1\times...\times A_n\into \PSet(A)] & [A_1\times...\times A_n\into
[A\into Bool]] \\ & [A_1\times...\times A_n\times A\into Bool] \\ &
\PSet(A_1\times...\times A_n\times A)
\end{array} 
}
%
In \cite{JT1,BJ} one introduces the following notions.

\begin{Definition}
\label{de:cmplx}
Given a relational structure $\<\under A,R_0,R_1,...\>$ where each
$R_i\subseteq \under A^{n_{i}}$, the structure:
%
\begin{enumerate}\MyLPar
\item 
$A'=\<\under A,\hat{f_1},\hat{f_2},...\>$ is a poly-algebra,
where each $\hat{f_i}$ corresponds to $R_i$ as in \refp{eq:isomr};
\item 
$A=\<\PSet(\under A),f_1,f_2,...\>$ is an algebra of complexes of
$A'$, where $\PSet(\under A)$ is Boolean algebra of subsets of $\under
A$ and for all $f_i:$ $f_i(X_1...X_n)=\bigcup_{x_{k}\in X_{k}}
\hat{f_i}(x_1...x_n)$.
\end{enumerate}
\end{Definition}
%
In this setting, composition of relations becomes
naturally the composition of the respective set-valued functions given
by additive extension to sets, i.e.\ for $f:\under A^n\into\under A$
and $g:\under A^m\into\under A: f(x_1...g(y_1...y_m)...x_n) =
\setsup_{z\in g(y_{1}...y_{m})} f(x_1...z...x_n)$.  In general, given
an $n$-ary $f$ and $n$ $m_i$-ary $g_i$'s, their composition
$f\<g_1...g_n\>$ is a $(\sum_1^n m_i)$-ary function $h$ given 
by:\footnote{As it is often the case, one draws here an advantage from
the functional notation which, unlike the relational notation, allows
one to write composition in a natural way (it was called
``superposition'' in \cite{JT1}).}
%
\eq{
h(\ovr x_1...\ovr x_n)\ \ =\ \ 
f(g_1(\ovr x_1)...g_n(\ovr x_n))\ \ = \bigcup_{x_{i}\in g_{i}(\ovr
x_{i})}f(x_1...x_n) 
}
where $\ovr x_i$ is an $m_i$-tuple.
 
One may now talk about similarity types, or signatures, 
in the way entirely analogous to standard algebras. 
% In particular, the above mentioned works study many-sorted multialgebras
% but we, as it was stated in the introduction concentrate on the
% one-sorted case which does not introduce any essential restrictions
% while makes notation simpler.
%
% where signatures are pairs
% $\Sigma=\<\Sorts,\Funcs\>$ listing the sort symbols $\Sorts$ and
% assigning sorted arities to the function symbols $\Funcs$. In the
% following, we will rather concentrate on the one-sorted case which
% does not seem to introduce any essential restrictions while makes
% notation simpler.
An important consequence of this change of perspective is the
fact that the structure of the derived operators ceases to be simply
a Boolean algebra relatively independent from the actual
multialgebra. In fact, they become related to the signature in the
same way as in classical universal algebra.

Algebras of complexes are what we will call {\em multialgebras\/} 
-- with one proviso: the carrier of a multialgebra is a power set but Boolean
operations are not part of the multialgebraic signature. This seemingly negligible
differece turns out to have significant consequences, since signatures
determine homomorphisms.
We pause for a moment at the notion of algebra of complexes in
order to illustrate these consequences -- subsection~\ref{sub:bao} quotes 
relevant definitions and results and shows that viewing multialgebras as 
algebras of complexes results in a rather limited notion of homomorphism.
This may be an indication of why multialgebras deserve a separate
study which we are proposing and carrying out in the subsequent subsections.

%---------------------------------------------------------------------

\subsection{Boolean algebras with operators}
\label{sub:bao}
Definition~\ref{de:cmplx} makes algebras of complexes a special case of the
so-called Boolean algebras with operators.
In the context of homomorphisms, the choice of the
signature for the Boolean algebra does matter quite a lot: e.g., is 0 an
interpretation of the 
constant or an element returned by the meet operation? In the latter case, 
a homomorphism must merely preserve meets, in the former also 0. 


\begin{Definition}[\cite{JT1}]
A Boolean algebra (BA) $B$ is a familiar structure $\<\under
B,\join,0,\meet,1,-\>$. \\ 
A \emphas{Boolean algebra with operators} (BAO) $B$ is a structure
$\<B,f_1,f_2...\>$ where $B$ is a Boolean algebra\footnote{We hope
that the overloading of notation (symbol $B$) will not cause any
confusion} and each $f_i$ is an operator, i.e.\ it is
\begin{enumerate}\MyLPar
\item \emphas{normal} (strict) $:x=0\impl f_i(..x..)=0$, and
\item \emphas{additive}, i.e.\ it preserves finite joins: 
      $f_i(..,x\join y,..) = f_i(..,x,..)\join f_i(..,y,..)$
\end{enumerate}
An operator preserving all joins is \emphas{completely additive}.  A BA
is \emphas{atomic} if each element is a join of some atoms and it is 
\emphas{complete} if it is a complete lattice.
\end{Definition}

\noindent
The theorem below is taken from \cite{JT1} and it states the relations
between multialgebras and BAO's.  Just like any BA is isomorphic to a
{\em subset} of some power set which itself need not be a power set
\cite{Stone1,Stone2}, so any BAO is guaranteed to be isomorphic merely
to a {\em subalgebra} of some multialgebra which itself need not be a
multialgebra -- \ref{it:bao}).  In order to obtain an isomorphism from
BA to a power set, one has to require the BA to be complete and
completely distributive. Similarly, an isomorphism from BAO to a
multialgebra requires additional conditions as in \ref{it:baoat}).  We
are glossing over details of the definition of regular subalgebra and
perfect extension which are rather involved (see \cite{JT1}). In the
present context it is only relevant that perfect extension of an BAO
is atomic and additive, and so isomorphic to a multialgebra by
\ref{it:baoat}).

\begin{Theorem}[\cite{JT1}]
\label{th:bao}
Let $B$ be some BAO:
\begin{enumerate}\MyLPar
\item
\label{it:bao} 
If $B$ is normal it is isomorphic to a regular subalgebra of some
multialgebra. 
\item
\label{it:baoat} 
If $B$ is normal, atomic and completely additive, it is isomorphic to
a multialgebra over the underlying set of its atoms
\item\label{it:baoex} Any $B$ has a unique (up to isomorphism) 
BAO perfect extension. 
%\cite{JT1}, Th.2.15
\end{enumerate}
\end{Theorem}

\noindent
The fact that, in general, only \ref{it:bao}) but not \ref{it:baoat}) holds,
is for us a legitimate reason for a separate study of
multialgebras. One might use \ref{it:baoex}) to argue that we may
still concern ourselves exclusively with BAO's. But this would require
taking perfect extensions each time some BAO-construction does not
yield a multialgebra. We find such a solution rather cumbersome. In
particular, homomorphic image of a multialgebra (seen as an atomic,
normal and completely additive BAO) need not be a multialgebra unless
we consider only quotients by ideals, that is, unless we severly
restrict the notion of homomorphism.

\begin{Definition} A {\em congruence ideal} of a BAO $B$ is a subset 
$N\subseteq B$ satisfying: 
%which is a BA ideal (1-2) satisfying also 3):
\begin{enumerate}\MyLPar
\item $a,b\in N\impl a\join b\in N$
\item $a\in N\ \&\ b\leq a\ \impl b\in N$
\item $x\in N\impl f_i(..x..)\in N$ for all operators $f_i\in\Sigma$.
\end{enumerate}
\end{Definition}

\noindent
(Congruence) ideals play the same role for BA(O)'s as congruences for
algebras. Any congruence relation $\congo$ on a BAO is completely
determined by any one of its blocks, in particular by the ideal
$[0]_\congo$.  The mathematically elegant and well-know construction
of a quotient $Q$ of a multialgebra $A$ (seen as a BAO) by an ideal
does yield a (structure isomorphic to a) multialgebra.  But oddly (or
interestingly) enough, this results in a subalgebra obtained by
``removing'' the elements of $N$!  Even worse, it does not correspond
to a homomorphism $\phi:A\into Q$ seen as a function between the
respective underlying sets but to a mapping $\under A\into
\PSet(\under Q)$!  We close these considerations by illustrating this
fact.

For the moment, by a ``multialgebra $A$'' we mean a (normal, atomic and
completely additive) BAO over the underlying set $\under A$, i.e.\
with the Boolean part being the power set algebra $\<\PSet(\under
A),\cup,\es,\cap,\under A, \under A\!\setminus\!\!\_\,\>$.  Given a
congruence ideal $N$ on $A$, it is easy to verify that the relation
$x\congo_N y \iff \exists n\in N: x\cup n= y\cup n$ is a congruence on
$A$\footnote{$\congo_N$ is trivially reflexive and symmetric, and it
is transitive since $N$ is an ideal. Moreover, since $N$ is a
conguence ideal then if $n \in N$ then $f(...n...) \in N$. Hence if
$x\congo_N y$ then $f(...x...)\cup f(...n...) = f(...x\cup n...) =  
f(...y\cup n...) = f(...x...)\cup f(...n...)$ and we get $f(...x...)
\congo_N f(...y...)$
% Its definition is equivalent to:
% $x\congo_N y \iff x\div y\in N$ (where $\div$ is the symmetric
% difference). In particular, it implies that $(x\cup n)\setminus n = s
% = (y\cup n)\setminus n$ for some $s$, i.e.\ $x\cup n=s\cup n=y\cup n$.
% So, if $x\congo_N y$, then $f(...x...)\cup f(...n...)=f(...x\cup n...)
% = f(...s\cup n...) = f(...s...)\cup f(...n...) = f(...s\cup n...)=
% f(...y\cup n...)= f(...y...)\cup f(...n...)$.  Since $N$ is a
% congruence ideal, if $n\in N$ then $f(...n...)\in N$, i.e.\
% $f(...x...)\congo_N f(...s...)\congo_N f(...y...)$.

It is also trivially congruence wrt. the Boolean operations $\cup$ and
$\cap$. For if $x_1\congo y_1$ and $x_2\congo y_2$, i.e.\ $x_1\cup
n_1=s_1\cup n_1=y_1\cup n_1$ and $x_2\cup n_2=s_2\cup n_2=y_2\cup
n_2$, then obviously $(x_1\cup x_2)\cup (n_1\cup n_2)=(s_1\cup
s_2)\cup (n_1\cup n_2)= (y_1\cup y_2)\cup (n_1\cup n_2)$, and
similarly for $\cap$.}
%
(also note that $x\congo_N y$ iff $x\div y \in N$ where $\div$ is the
symmetric difference). Thus, the mapping $\phi:A\into Q$ defined by
$\phi(x)= [x]_\congo$ is a BAO homomorphism and the quotient $Q=\quot
A{{\congo_{N}}}$ is a BAO, where
%
\begin{itemize}\MyLPar
\item 
$\under Q=\{[x]:x\in \PSet(\under A)\}$ 
\item 
$\<\under Q,\join,[0],\meet,[1]\>$ is given by
$[x]\join [y] = [x\cup y]$, $[x]\meet[y]=[x\cap y]$ and 
\item 
$f^Q([x])=[f^A(x)]$;
\end{itemize}

\begin{Fact}
Given a multialgebra $A$ and a congruence ideal $N$. 
Then $\quot A{{\congo_{N}}}=Q\iso V$ where 
%
\begin{itemize}\MyLPar
\item 
$\under V=\under A\setminus\under N$, where $\under N=\bigcup N$
and all operations are given by
\item 
$f^V(v)=f^A(v)\cap\under V$.
\end{itemize}
\end{Fact}

\begin{Proof}
First we have to prove the existence of an ``ordinary''
algebraic isomorphism, i.e. a bijection on the carrier sets, and then that
it satisfies the algebraic homomorphism condition.

We do have $[x]=[x\cap\under V]$ (since $x\congo_N (x\cap\under V)$)
which obviously implies the set-isomorphism $\iota: \PSet(\under V)
\iso Q$ defined by $\iota:v\mapsto [v]$. First we show it is
injective: take $v_1,v_2\in\PSet(\under V)$. If $v_1\not= v_2$ then
also $v_1\div v_2\not\in N$ since $v_1,v_2\subseteq\under V$, and thus
$[v_1]\not=[v_2]$. Surjectivity: any $[x] = [x\cap \under V]$ where
$x\cap\under V \in \PSet(\under V)$ hence $\iota(x\cap\under V) = [x]$.

Let us consider the inverse mapping $\iota^- : [x] \mapsto \bigcup [x]
\cap \under V$. In order to show that $\iota^-;\iota([x]) = [x]$ we
need $\bigcup [x]\cap\under V = x\cap\under V$. The inclusion $\supseteq$
 is obvious whilst the opposite we prove by contradiction: there
must be an {\em element\/} $e \in \under V$ such that $e \in (\bigcup
[x]\cap\under V) \setminus (x\cap\under V)$ which implies existence of an $x'
\in [x]$ with $e \in x'\setminus x$. But then $e \in (x'\div x) \not\in N$ and
$x' \not\congo_N [x]$. 

Now take $\iota^-;\iota([x]) = [\bigcup [x]\cap \under V] =
[x\cap\under V] = [x]$. Similarly one can show $\iota;\iota^-(x)=x$
(although since we have shown $\iota$ to be a bijection the above is
not necessary).

Next we show that $\iota$ preserves the ordering: if $v_1\subseteq
v_2$, i.e.\ $v_1\cup v_2 = v_2$ then $[v_1]\vee [v_2]= [v_1\cup
v_2]=[v_2]$, i.e.\ $[v_1]\leq [v_2]$. In particular,
$\iota(\es)=[\es]=0^Q$ and $\iota(\under V)=[\under V]=1^Q$.  It
reflects ordering: $[x_1]\leq [x_2]\iff [x_1]\vee [x_2]=[x_2]\iff
[v_1]\vee[v_2]=[v_2]$ for appropriate $v_1,v_2\subseteq\under V$ and
so $\iff [v_1\cup v_2]=[v_2]\iff v_1\cup v_2\cup n = v_2\cup n$ for
some $n\in N$ which holds iff $v_1\cup v_2=v_2$ since $\under V\cap
n=\es$.  Thus, $\iota$ is an order-isomorphism of $Q$ and $V$ seen as
Boolean algebras, and hence it is a Boolean algebra isomorphism.

Finally $\iota, \iota^-$ are (algebraic) homomorphisms: For any
operation $f$ and $v\in\under V: f^V(v)=f^A(v)\cap \under V$.  So
$\iota(f^V(v))=[f^V(v)]=[f^A(v)\cap\under V]=
[f^A(v)]=f^Q([v])=f^Q(\iota(v))$.  Conversely,
$\iota^-(f^Q([x]))=\iota^-(f^Q([v]))= \iota^-([f^A(v)])=
\iota^-([f^A(v)\cap\under V]) = \iota^-([f^V(v)])=
f^V(v)=f^V(\iota^-([v])) = f^V(\iota^-([x]))$.
\end{Proof}

Translating this fact into the concrete language of multialgebras we
see that a congruence ideal on $A$ is the power set of any subset $\under
N$ of the underlying set $\under A$.  The above quotient construction
amounts to removing $\under N$ from the underlying set, i.e.\ $\under
V=\under A\setminus\under N$ and letting all the operations on $\under
V$ be given by $f^V(v) = f^A(v)\cap \under V$, i.e.\ taking
relative subalgebra of $A$ (see def.~\ref{de:relative}) on the set
$\under V$, albeit in a very round-about way.

It is certainly worth focusing here on the composition of the quotient
homomorphism $\phi: A \into Q$ with $\iota^-$ -- we have that
$\phi;\iota^- : x \mapsto x\cap\under V$. It follows that for $a
\in\under A$, $\phi;\iota^-(a) = \{a\}$ iff $a\in\under V$ and
$\phi;\iota^-(a) = \emptyset$ otherwise. Thus it is not a mapping on
the level of underlying sets, but sends the underlying set into a
power set. We return to such homomorphisms in Section~\ref{se:horel}.

The whole exercise gives us a hint that using Boolean algebras and
ideals for talking about homomorphisms of multialgebras may not be the
most flexible way of approaching the problem. One might certainly play
with the congruence condition on the ideals but this would not change
the fact that quotients by ideals are just subobjects (in the ordinary
algebraic sense). 


%----------------------------------------------------------------------

\subsection{Categories of Multialgebras}
\label{sub:cats}
As announced before, 
multialgebras are algebras of complexes, defined in \ref{de:cmplx}, with one
difference -- we do not take into account the signature of Boolean algebra
which is a part of the signature of an algebra of complexes. 
% This does not
% change the actual structures but does have some implications for the
% possible homomorphisms.
%
\begin{Definition}\label{de:multi}
Let $\Sigma = \pair \Sorts\Funcs$ be an algebraic signature. A $\Sigma$-multialgebra $M$
is given by:
\begin{itemize}
\item 
a carrier $\Card M=\{\Card{M}_s\}_{s\in \Sorts}$, where for each $s\in
\Sorts$, $\Card{M}_s$ = $\PSet(\under M_s)$ of some {\em underlying set}
$\under M_s$, with the obvious embedding $\under M_s\incl \PSet(\under
M_s)$;
\item 
a function $f^M:\under M_{s_1}\times...\times \under M_{s_n}\into
\PSet(\under M_s)$ for each $f:s_1\times...\times s_n\into s\in \Funcs$,
with composition defined through additive extension to sets, i.e.\
 $f^M(X_1,...,X_n) = \setsup_{x_i\in X_i} f^M(x_1,...,x_n)$.
\end{itemize}
\end{Definition}
%
As mentioned before, we treat the single-sorted case.
In this section we will consider three categories of multialgebras
with homomorphisms corresponding to condition 1) of
theorem~\ref{th:comp}. Writing these definitions in multialgebraic
form we have:

\begin{Definition}
A homomorphism $\phi:M\into N$ is a mapping $\phi: \under
M\into \under N$ of one of the three modes 
\[
\begin{array}{r@{\ \ -\ \ {\rm when\ \ }}l}
{\sl weak} & \phi(f^M( m))\subseteq f^N(\phi( m)) \\
{\sl closed}& \phi(f^M( m))\supseteq f^N(\phi( m)) \\
{\sl tight} & \phi(f^M( m))= f^N(\phi( m)) 
\end{array} 
%%%\begin{array}{r@{\ \ -\ \ {\rm when\ \ }}l@{,\ \ {\rm in\ particular\ \ }}l}
%%%{\sl weak} & \phi(f^M( m))\subseteq f^N(\phi( m)) & \phi(c^M)\subseteq c^N \\
% -- when $\otimes$ is $\po$,
%%%{\sl closed}& \phi(f^M( m))\supseteq f^N(\phi( m))  & \phi(c^M)\supseteq c^N \\
% -- when $\otimes$ is $\rpo$,
%%%{\sl tight} & \phi(f^M( m))= f^N(\phi( m))  & \phi(c^M)= c^N 
% -- when $\otimes$ is $=$.
%%%\end{array} 
\]
\end{Definition}
(In particular, respective relations hold for constants $c^M$ and $c^N$.)

Multialgebras are ``partial'' in the sense that operations may return
empty set of values. By the pointwise extension of operations, they
are strict in all arguments.  Notice also that we allow empty
carriers, i.e.\ $\PSet(\es)=\{\emptyset\}$ is a possible carrier of a
multialgebra. Thus we will often refer to three special multialgebras,
namely:

\begin{center}
\begin{tabular}{r|r@{\ =\ }l@{\ \ }c}
algebra     & \multicolumn{2}{c}{with carrier} & and all operations returning \\ \hline
``empty''   & $\PSet(\es)$ & $\{\es\} $ & $\es$ \\[.5ex]
``unit''    & $\PSet(\{\bullet\})$ & $\{\es,\{\bullet\}\}$ & $\{\bullet\}$  \\[.5ex]
``e-unit''  & $\PSet(\{\bullet\})$ & $\{\es,\{\bullet\}\}$ & $\es$
\end{tabular} 
\end{center}

\noindent
% Also an operation $\phi:N\into M$ has a unique pointwise extension
% to the operation $\PSet(N) \into\PSet(M)$, and

Obviously, a homomorphism $\phi:M\into N$ maps not only the underlying
sets $\under M\into \under N$ but also the whole carriers (through its
unique extension) i.e.\ also $\PSet(M)\into\PSet(N)$.  When the source
$M$ is ``more partial'' than the target $N$ (meaning the operations
more often return $\es$), there will be neither a tight nor a closed
homomorphism, but often a weak one $N\into M$.

We will write $\MASw, \MASt,
\MASc$ for categories of multialgebras with weak, tight, and closed
homomorphisms, respectively. 
When the mode of homomorphisms does not matter, we write $\MAS$ -- meaning
any of these categories.
The three cannot be mixed into one
(meaningful) category -- e.g.\ a composition of a weak and a closed
homomorphism may yield a transformation which is of none of the three
modes. \vspace*{1ex}

\noindent
First we register a few facts common to all three categories:
%
\begin{Prop}
\label{le:isotight}
Let $M, N\in\MAS$. If $\phi:M\into N$ and $\psi:N\into M$ are
isomorphisms, then they are tight.
\end{Prop}

\begin{Proof}
Consider the case when both $\phi$ and $\psi$ are weak. Then
for all $x\in \Card M, y\in \Card N$ we have (1) $\phi(f^M(x))\po
f^N(\phi(x))$ and (2) $\psi(f^N(y))\po f^M(\psi(y))$.  Substituting
$\phi(x)$ for $y$ in (2), we get
\[
\begin{array}{rcl}
 {\psi(f^N(\phi(x))) \po f^M(\psi(\phi(x)))} \\
\phi\ {\rm monotone} & \impl & \phi(\psi(f^N(\phi(x)))) \po \phi(f^M(\psi(\phi(x)))) \\
{\rm isomorphism} & 
\impl & f^N(\phi(x)) \po \phi(f^M(x))
\end{array} 
\]
Together with (1) this means that $\phi(f^M(x)) = f^N(\phi(x))$, and
the same follows for $\psi$ by a symmetric argument.

The case when $\phi$ and $\psi$ are closed is entirely analogous.
\end{Proof}

\noindent
We can trivially extend this fact to: if $\phi$ is iso then it is
tight, surjective and injective. This characterizes isomorphisms:

\begin{Prop}
A homomorphism $\phi:M\into N$ in $\MAS$ is an isomorphism iff it is 
surjective, injective and tight.
\end{Prop}

\begin{Proof}
We have just shown the ``only if''. For the ``if'', define
$\psi:N\into M$ by $\psi(m)=n$ iff $\phi(n)=m$. This makes
$\phi;\psi=id_M$ and $\psi;\phi=id_N$, and we only have to verify that
$\psi$ is a tight homomorphism. Take an arbitrary $n\in\under N$ and
let $m=\psi(n)$, i.e.\ $\phi(m)=n$. Tightness of $\phi: \phi(f^M(m))=
f^N(\phi(m))$ yields then $\psi(\phi(f^M(m)))= \psi(f^N(\phi(m)))
\impl f^M(m)=\psi(f^N(n)) \impl f^M(\psi(n))=\psi(f^N(n))$.
\end{Proof}

\noindent
Another common fact is the following.

\begin{Prop}
\label{fa:epimono}
A homomorphism $\phi : M \into N$ in $\MAS$ is epi (mono) iff it
is surjective (injective) on the underlying set. 
\end{Prop}

\begin{Proof} 
The fact is trivial but the proof not quite. We consider all three
categories jointly, marking the places (the ``only if'' parts) where
the constructions differ.

\begin{enumerate}\MyLPar
\item 
Epi iff surjective

\begin{enumerate}\MyLPar

\item %1a) 

Assume $\phi$ is surjective. \\ 
If $\psi_i:N\into T$ for $i=1,2$ are different, then there is an
$n\in\under N: \psi_1(n)\not=\psi_2(n)$.  By surjectivity of $\phi$
there is an $m\in\under M:\phi(m)=n$, and so
$\Comp\phi{\psi_1}\not=\Comp\phi{\psi_2}$.

\item %1b)

If $\phi$ is not surjective, 

$\bullet$\ Let us consider first $\MASt$ and $\MASc$.\\
The set $N_1=\under N\setminus\phi[\under M]$ is non-empty (since
$\phi$ is not surjective), and let $N_2\iso N_1$ be a set disjoint
from $\under N$ with bijections $\iota_{21}:N_2\leftrightarrow
N_1:\iota_{12}$. Extending these to bijections $\iota_{21}:\phi[\under
M]\cup N_2\leftrightarrow \phi[\under M]\cup N_1:\iota_{12}$ by making
them identities on $\phi[\under M]$ ($\forall x\in\phi[\under
M]:\iota_{12}(x)=\iota_{21}(x)=x$), we define algebra $T$ on the
underlying set $\under T=N_1\cup\phi[\under M]\cup N_2$ as follows:
\[
f^T(x) = 
\left\{ \begin{array}{ll}
    f^N(x) & 
        {\rm iff\ } x\in N_1\cup\phi[\under M]\ =\under N \\
   \iota_{12}(f^N(\iota_{21}(x))) & 
        {\rm iff\ } x\in N_2\cup\phi[\under M]\
        \land\ x\not\in\phi[\under M] \\
   \es & 
        {\rm otherwise,\ i.e.\ iff\ } x\in N_2\cup N_1\ 
        \land\ x\not\in N_1\ \land\ x\not\in N_2 \\
\end{array}\right.
\]
%
The set $\phi[\under M]$ is closed under operations in $N$, i.e.\ 
$x\in\phi[\under M]\impl f^N(x)\in\phi[\under
M]$ -- cf.  proposition~\ref{le:subalgs}.  In particular, all ground
terms are interpreted in $T$ as in $N$.  We then have two tight
homomorphisms $\psi_i:N\into T$ given by $\psi_1(x)=x$ for all
$x\in\under N$, and $\psi_2(x)=x$ for $x\in\phi[\under M]$ and
$\psi_2(x)=\iota_{12}(x)$ for $x\in N_1$.  Thus $\psi_1\not=\psi_2$
but $\Comp\phi{\psi_1}=\Comp\phi{\psi_2}$, i.e.\ $\phi$ is not epi.

$\bullet$\ In $\MASl$ \\ The construction is different here. Assuming
non-surjectivity of $\phi$, pick an $n_1\in\under N$ such that
$\forall m\in\under M:n_1\not=\phi(m)$. Let $n_2\not\in\under N$ be a
new element and define an algebra $T$ on the underlying set $\under
T=\under N\cup\{n_2\}$ as follows. Any set $f^N(x)$ containing $n_1$
is extended with $n_2$, i.e.\ if $n_1\in f^N(x) \impl f^T(x)\Def
f^N(x)\cup\{n_2\}$, and then all functions are defined on $n_2$
as they are on $n_1$, i.e.\ $f^T(..n_2..)\Def f^T(..n_1..)$.  The
mapping $\psi_1(n)\Def n$ is obviously a weak homomorphism $N\into
T$. So is $\psi_2$ defined below:
\[
\psi_2(n) \Def \left\{\begin{array}{ll} n_2 & {\rm for\ }n=n_1 \\ 
                                        n & {\rm otherwise}
                      \end{array}\right.
\] 
We have $\psi_2(c^N)\subseteq c^T$ and for any $n:
\psi_2(f^N(n))\subseteq f^T(\psi_2(n))$. Thus
$\psi_1\not=\psi_2$ but $\Comp\phi{\psi_1}=\Comp\phi{\psi_2}$, i.e.\
$\phi$ is not epi.

\end{enumerate}

\item 
Mono iff injective

\begin{enumerate}\MyLPar

\item %2a) 

If $\phi$ is injective \\
and $\psi_1,\psi_2:T\into M$ are different, then there is a
$t\in\under T:\psi_1(t)\not=\psi_2(t)$. But then, since $\phi$ is
injective, $\phi(\psi_1(t))\not=\phi(\psi_2(t))$, so that $\phi$ is
mono.
%
\item %2b) 

If $\phi$ is not injective, 

$\bullet$\ we consider $\MASl$ and $\MASc$ first.\\
There are (at least) two $m_1\not=m_2\in\under M$ such that
$\phi(m_1)=\phi(m_2)$.  Take an $z\not\in\under M$ and define algebra
$T$ on the underlying set $\under M\cup\{z\}$ extending $M$ by
\[
f^{T}(x_1...,z,... x_n)=
\left\{\begin{array}{ll} \es      & {\rm in\ }\MASl\\
                         \under M & {\rm in\ } \MASc
       \end{array} \right.
\]
Let $\psi_1, \psi_2:T\into M$ be defined by $\psi_i(x)=x$ if
$x\in\under M$, $\psi_1(z)=m_1$ and $\psi_2(z)=m_2$. Obviously, they
are appropriate homomorphisms, the compositions of $\psi_i$ with
$\phi$ are identical, but $\psi_1\not=\psi_2$. Hence $\phi$ is not
mono.

$\bullet$\ In $\MASt$, \\
the construction of $T$ is more involved. First, observe that for any
$f\in\Sigma$ (actually, any term $t$), and any $x_1...x_n, y_1...y_n$
such that $\phi(x_i)=\phi(y_i)$ we will have 
\eq{
\label{eq:tightEq}
\phi(t^M(x_1...x_n)) = t^N(\phi(x_1)...\phi(x_n)) =
t^N(\phi(y_1)...\phi(y_n)) = \phi(t^M(y_1...y_n)) 
} 
Let $\under T=\{\<x,y\>:x,y\in\under M\land \phi(x)=\phi(y)\}$ (in
particular $\{\<x,x\>:x\in\under M\}\subset\under T$). Define the
algebra $T$ by letting, for all constants $c$, all function symbols
$f$ and arguments $\<x_1,y_1\>...\<x_n,y_n\>\in\under T:$
%
\begin{eqnarray*}
c^T & \Def & \{\<x,y\>\in c^M\times c^M : \phi(x)=\phi(y)\} \\
f^T(\<x_1,y_1\>...\<x_n,y_n\>) & \Def & 
\{\<x,y\>\in f^M(x_1...x_n)\times f^M(y_1...y_n):\phi(x)=\phi(y)\}
\end{eqnarray*}
%
Let $\pi_1,\pi_2:T\into M$ be the respective projections, i.e.\
$\pi_1(\<x,y\>)=x$ and $\pi_2(\<x,y\>)=y$. These will be tight
homomorphisms: for constants we have: 
\[
\pi_1(c^T)= c^M\cup \{x:x\in
c^M\land \exists y\not= x:y\in c^M\land \phi(x)=\phi(y)\}=c^M
\] 
and for other functions: 
\[
\begin{array}{lcl}
\pi_1(f^T(\<x_1,y_1\>...\<x_n,y_n\>)) 
& = &
\{x:x\in f^M(x_1...x_n)\land \exists y\in
f^M(y_1...y_n):\phi(x)=\phi(y)\} \\
& = & 
f^M(x_1...x_n)
\end{array}
\]
where the last equality follows from \refp{eq:tightEq}, since
$\phi(f^M(x_1...x_n))=\phi(f^M(y_1...y_n))$ means exactly that
$\forall x\exists y: x\in f^M(x_1...x_n)\impl (y\in f^M(y_1...y_n)
\land \phi(x)=\phi(y))$, and vice versa (i.e.\ the argument for
$\pi_2$ is entirely symmetric).

Since, by assumption $\phi(m_1)\not=\phi(m_2)$, we have
that$\<m_1,m_2\>\in\under T$ and so $\pi_1\not=\pi_2$. But trivially,
$\Comp{\pi_1}\phi=\Comp{\pi_2}\phi$, so $\phi$ is not mono.
\end{enumerate}
\end{enumerate}
\end{Proof}


%----------------------------------------------------------------------

\subsection{Epi-mono factorisation}

We show that all three categories have
epi-mono factorisations and construct factorisation systems for
$\MASl$ and $\MASt$.

The following definitions and results are just multialgebraic
reformulations of the respective facts for relations 
(subsection~\ref{sub:congrel}). A {\em
quotient\/} of a multialgebra $M$ wrt.\ an equivalence relation
$\congo$ (denoted $M/_\sim$) is a multialgebra $N$ such that $\under N
= \{ [a]_\congo : a \in \under M \}$ and, for any function $f:$
\eq{
\label{eq:weakquo}
f^N([m]) = \{ [n] : n
\in f^M(m'), m'\in [m]\} } 
Recall also that a kernel of a weak homomorphism is an
equivalence and, more importantly, the mapping $\phi: M \longrightarrow
M/_\congo$ defined by $\phi(a)=[a]$ is a weak homomorphism. We also
have the following implication which, unfortunately, cannot be reversed:

\begin{Fact}
\label{fa:subeq}
Let $\sim_1$ and $\sim_2$ be two equivalences on $M$. If $\sim_1\
\subseteq\ \sim_2$ then there is a weak homomorphism
$\phi:M/_{\sim_{1}}\into M/_{\sim_{2}}$.
\end{Fact}

\begin{Proof}
We have two weak homomorphisms $\phi_i:M\into N_i$, for $i=1,2$ and
where $N_i=M/_{\congo_i}$. Define $\phi: N_1 \into N_2$ by
$\phi([n]_1)\mapsto \phi_2(n)=[n]_2$, where $[m]_i$ is $[m]_{{\sim_i}}$.
It is a weak homomorphism since 
$\phi(f^{N_1}([n]_1)) = \phi(\{[m]_1 : m\in f^M(n'), n'\in [n]_1\}) 
\subseteq \phi(\{[m]_1  : m\in f^M(n'), n'\in [n]_2\}) =
          \{\phi([m]_1) : m\in f^M(n'), n'\in [n]_2\} = 
          \{[m]_2 : m\in f^M(n'), n'\in [n]_2\} = f^{M_2}([n]_2) = 
          f^{M_2}(\phi([n]_1))$.
\end{Proof}

\noindent
The definition of a tight congruence is accommodated to multilagebras:
$\sim\ \subseteq \under M\times\under M$ is a {\em tight congruence}
iff it is equivalence and for all $f\in\Sigma$ we have $\forall
a,b:a\sim  b \impl f^M(a)\sim f^M(b)$, where $\sim$ on sets
is defined by (Egli-Milner extension): 
$A\sim B\iff \forall a\in A\ \exists b\in B:a\sim b\
\land\ \forall b\in B\ \exists a\in A:a\sim b$.  We then have the
familiar facts: if $\sim$ is a tight congruence on $M$ then
$\phi:M\into\quot M{\sim}$ defined by $m\mapsto [m]_\sim$ is a tight
homomorphism; and kernel of any tight homomorphism is a tight
congruence.

The following theorem is a counterpart of the well-known Homomorphism
Theorem.

\begin{Theorem}
\label{th:epimono}
Let $M,N \in \MAS$.  Let $\phi: M \into N$ be a homomorphism and
$\congo$ its kernel. Define $\psi_1: M \into M/_\congo$ and $\psi_2:
M/_\congo \into N$ as follows: $\psi_1(a) = [a]_\congo$ and
$\psi_2([a]_\congo) = \phi(a)$.
\begin{enumerate}\MyLPar
\item{if $\phi$ is tight then both $\psi_1$ and $\psi_2$ are tight;}
\item{if $\phi$ is closed then $\psi_1$ is weak and $\psi_2$ is closed;}
\item{if $\phi$ is weak then both $\psi_1$ and $\psi_2$ are weak.}
\end{enumerate}
\end{Theorem}

\begin{Proof}
1) We have to show that $\psi_1(f^M(a)) = f^C(\psi_1(a))$ and
$\psi_2(f^C([a]_\congo)) = f^N(\psi_2([a]_\congo))$. First consider
$\psi_1$: $f^C(\psi_1(a)) = f^C([a]_\congo) = \{[b]_\congo : b \in
f^M(a'), a' \in [a]_\congo\}$.  Since $\sim$ is tight i.e.\ if $a \sim
a'$ then $\forall {b \in f^M(a)}\ \exists {b' \in f^M(a')}\ b \sim
b'$, we have $\{[b]_\congo : b \in f^M(a'), a' \in [a]_\congo\} =
\{[b]_\congo : b \in f^M(a)\} = \psi_1(f^M(a))$.

Now consider $\phi_2$: $\phi_2(f^C([a]_\congo)) = \psi_2(\{[b]_\congo
: b \in f^M(a'), a' \in [a]_\congo\}) =$ (since $\sim$ is a tight
congruence) $= \psi_2(\{[b]_\congo : b \in f^M(a)\}) = \phi(\{b: b \in
f^M(a)\}) = \phi(f^M(a)) = f^N(\phi(a)) = f^N(\psi_2([a]_\congo)).$

2) $\psi_1$ is weak by previous lemma. Now, note that $[a]_\congo =
\phi^{-1}\phi(a)$. Hence $\psi_2(f^C([a]_\congo)) =
\psi_2(\{[b]_\congo : b \in f^M(a'), a' \in [a]_\congo\}) = \{\phi(b):
b \in f^M(a'), a' \in [a]_\congo\} = \phi(f^M(\phi^{-1}(\phi(a))))
\supseteq \phi(f^M(a)) \supseteq f^N(\phi(a)) =
f^N(\psi_2([a]_\congo))$.

3) Similarly, $\psi_1$ is weak by lemma. We ought to prove that
$\psi_2(f^C([a]_\congo)) \subseteq f^N(\psi_2([a]_\congo))$.
$\psi_2(f^C([a]_\congo)) = \phi(f^M(\phi^{-1}(\phi(a)))) \subseteq
f^N(\phi(\phi^{-1}(\phi(a)))) = f^N(\phi(a)) =
f^N(\psi_2([a]_\congo))$.
\end{Proof}

\noindent
This does not mean that only $\MASl$ and $\MASt$ have epi-mono
factorisations. We have singled out the quotient construction from
definition (\ref{eq:weakquo}) because it is the one we are going to
use most often. However, modifying it appropriately yields
corresponding construction for $\MASc$.
% \eq{
% \label{eq:quots}
% \begin{array}{rr|@{\ }l}
% & & [b] \in f^Q([a_1]...[a_n]) \\ \hline
% 1.& {\rm weak} & \exists a'_1\in[a_1]...a'_n\in[a_n] \exists b'\congo b: 
%   b'\in  f^A(a'_1...a'_n) \\
% 2.& {\rm closed} & \forall a'_1\in[a_1]...a'_n\in[a_n] \exists b'\congo b:
%   b'\in   f^A(a'_1...a'_n) \\
% 3. & {\rm tight} &
% \end{array}
% }
% Given an equivalence on $A$ and constructing its quotient $Q$
% according to this latter, closed definition yields the respective
% theorem for $\MASc$. First we observe that natural homomorphism into
% such a quotient is clsoed:
% 
\begin{Fact}\label{fa:closed}
Given an equivalence $\congo$ on $A$, define $Q$ 
by 
\begin{itemize}\MyLPar
\item $\under Q=\{[a]:a\in\under A\}$, 
\item for any constant $c:[a]\in c^Q \Iff \exists a'\in[a]:a'\in c^A$,
\item for any function $f: [a]\in f^Q([a_1]...[a_n]) \Iff
 \forall a'_1\in[a_1]...a'_n\in[a_n] \exists a'\congo a:
  a'\in   f^A(a'_1...a'_n)$.
\end{itemize}
Mapping $\eta:A\into Q$ given by $\eta(a)=[a]$ is a closed epimorphism.
\end{Fact}

\begin{Proof}
It is obviously surjective. We show that it is closed.
If $[a]\in c^Q$ then\\  $\exists a'\in[a]:a'\in c^A\impl [a]=[a']\land
 [a']\in\eta(c^A)=\{[a']:a'\in c^A\}\impl [a]\in\eta(c^A)$.

Assume $[a]\in f^Q([a_1]...[a_n])$, and pick any representatives
$a_i\in[a_i]$. From definition we have that $\exists a'\congo a:
a'\in f^A(a_1...a_n)$ and thus $[a]=[a']\in\{[b]:b\in f^A(a_1...a_n)\}
= \eta(f^A(a_1...a_n))$. Thus we have 
$\eta(f^A(a_1...a_n))\supseteq f^Q(\eta(a_1)...\eta(a_n))$, i.e.,
$\eta$ is closed.
\end{Proof}

\begin{Prop}
\label{th:epimonoC}
Let $A,B \in \MASc$ and $\phi: A \into B$ be a closed homomorphism
with kernel $\congo$. Define $Q=\quot A\congo$ as in fact \ref{fa:closed},
and $\eta: A \into Q$ and $\mu:
Q \into B$ as follows: $\eta(a) = [a]$ and
$\mu([a]) = \phi(a)$. Then $\eta$ is epi, $\mu$ is mono
and both are closed.
\end{Prop}

\begin{Proof}
Fact~\ref{fa:closed} gives the required properties of $\eta$.
$\mu$ is mono, because  $\mu([a])=\mu([b]) \impl
\phi(a)=\mu([a])=\mu([b])=\phi(b)$ and $\phi(a)=\phi(b)\impl
[a]=[b]$.

To see that $\mu$ is closed: $x\in c^B\impl x\in \phi(c^A)$ since
$\phi$ is closed, that is $x=\phi(a)$ for some $a\in c^A$. But
$\phi(a)=\mu([a])$ and, from the definition of $Q:a\in c^A \impl
[a]\in c^Q$. Thus $\phi(c^Q)\supseteq c^B$.
Similarly, $x\in
f^B(\mu([a_1])...\mu([a_n]))=f^B(\phi(a_1)...\phi(a_n))
\impl x\in \phi(f^A(a'_1...a'_n))$ for all $a'_i\in[a_i]$. 
In
particular, $x=\phi(a)$ for some $a\in f^A(a'_1...a'_n)$, and
$x=\mu([a])$. 
Thus, $\forall a'_i\in[a_i]\exists a'\in[a]:a'\in f^A(a'_1...a'_n)$,
so $[a]\in f^Q([a_1]...[a_n])$ and $x=\mu([a])\in \mu(f^Q([a_1]...[a_n]))$.
\end{Proof}

\subsubsection*{Factorisation systems}
We construct factorisation systems for the categories $\MASl$ and $\MASt$
and observe that analogous construction does not yield a factorisation
system in $\MASc$.

For an $\eta:M\into N$ and $m\in\under M$ let $[m] \Def \{m_i\in\under
M:\eta(m_i)=\eta(m)\}$, and $\eta[M]=\{n\in\under N:\exists m\in\under
M: n=\eta(m)\}$.

\begin{Definition}
\label{de:jtight}
We say that  $\phi:A\into B$ is
\begin{itemize}\MyLPar
\item
{\em full} iff for all $f\in\Sigma$ and
$a\in\under A : f^B(\phi(b)) \cap \phi[A] = \bigcup_{a_i\in [a]} \phi(f^A(a_i))$.
\item
{\em fully-tight} iff for all $f\in\Sigma$ and
$b\in\under B : f^B(b) = \bigcup_{a\in \phi^{-}(b)} \phi(f^A(a))$.
\end{itemize}
\end{Definition}
%
Obviously, tight $\impl$ fully-tight $\impl$ full, while full-and-epi 
$\impl$ fully-tight, but none of these implications can be
reversed.~\footnote{Fully-tight are the same as satisfying
$\phi^-;R^A;\phi=R^B$ (where $R$ is the relation for $f$). A
non-compositional variant, defined by $\forall a:f^B(\phi(a)) =
\bigcup_{a_i\in [a]} \phi(f^A(a_i))$, was called ``jointly-tight'' in
\cite{c:132}.}  The above definition is needed to obtain a
factorization system for $\MASl$.

\begin{Theorem}
\label{le:FSl}
$\<\epi,\mono\>$ is a factorisation system for $\MASl$, where
$\mu\in\mono$ iff it is injective, and $\eta\in\epi$ iff it is a full
epi (and hence fully-tight epi).
\end{Theorem}

\begin{Proof}
Isomorphisms (are tight and) are both in $\mono$ and $\epi$ (since
tightness implies full tightness). Also both classes
are obviously closed under composition (weak injective are obvious, while
full epis, being fully-tight epis, compose by the remark after theorem~\ref{th:comp}).
% \footnote{\label{ft:jtepi}For $\epi$ it is crucial that we take
% epi's -- fully-tight homomorphisms which are not epi are not closed under
% composition. E.g., let $\under M=\under N=\under
% P=\<\{0,1\},\{a,b,c\}\>$ and
% $f^M(0)=a, f^M(1)=b, f^N(0)=\{a,b\}, f^N(1)=c, f^P(0)=\{a,b,c\}$ and
% $\phi:M\into N$ be given by $\phi(0)=\phi(1)=0$, and
% $\psi:N\into P$ be given by $\psi(0)=\psi(1)=0$, and 
% $\phi(n)=\psi(n)=n$ for $n\in\{a,b,c\}$. Both $\phi$ and $\psi$ are 
% fully-tight, but $f^P(\psi(\phi(\{0,1\}))) = f^P(0)=\{a,b,c\} \not= 
% \{a,b\}=\psi(\phi(f^M(\{0,1\})))$. The general fact is that if 
% $\phi:M\into N$ is a fully-tight epi and $\psi:N\into P$ is
% fully-tight, then $\Comp\phi\psi$ is fully-tight.}
By theorem~\ref{th:epimono}, each morphism $\phi$ in $\MASl$ can be
factored as $\phi = \Comp\eta\mu$ where $\mu\in\mono$ and
$\eta\in\epi$ (since the epi into a quotient is fully-tight).

We have to show that any $\eta\in\epi$ and $\mu\in\mono$ are
orthogonal.
\[
\diagram A \rto^{\eta} \dto_\alpha & B \dto^\beta
\morphism\dotted\tip{}[ld]|{\exists!\uni} \\ A' \rto_\mu & B'
\enddiagram
\]
Assume $\eta:A\into B$ and $\mu:A'\into B'$ and $\alpha:A\into A'$,
$\beta:B\into B'$ are such that $\Comp\alpha\mu =
\Comp\eta\beta$. Since $\mu$ is injective (mono), we can define
$\uni:B\into A'$ as $\uni(b)\Def \mu^-(\beta(b))$. We show that (c) it
is a weak homomorphism satisfying (a) $\Comp\eta\uni=\alpha$ and (b)
$\Comp\uni\mu=\beta$.

(a) From $\uni(\eta(a)) = \mu^-(\beta(\eta(a)))$ and
$\Comp\alpha\mu=\Comp\eta\beta$ we get
$\uni(\eta(a))=\mu^-(\mu(\alpha(a)))$ and since $\mu^-(\mu(a'))=a'$,
this implies $\uni(\eta(a))=\alpha(a)$.

(b) From $\uni(\eta(a)) = \mu^-(\beta(\eta(a)))$ we get
$\mu(\uni(\eta(a))) = \beta(\eta(a))$.  Since $\eta$ is epi it is
surjective, so for any $b\in\under B$ there is an $a\in\under A$ such
that $b=\eta(a)$. Thus the equality $\mu(\uni(b))=\beta(b)$ holds for
all $b\in\under B$, and so $\Comp\uni\mu=\beta$.

(c) We have to verify that for all $b\in\under B:
\uni(f^B(b))\subseteq f^{A'}(\uni(b))$. Writing
$[a]=\{a':\eta(a')=\eta(a)\}=\eta^{-}(\eta(a))$, we have by (a) that
for all $a'\in[a]:\uni(b) = \alpha(a')$, and so
$\uni(b)=\alpha(\eta^{-}(b))=\alpha([a])$.  Hence
$f^{A'}(\uni(b))=f^{A'}(\alpha([a]))$. Since $\alpha$ is a
weak homomorphism $\alpha(f^{A}([a_{i}]))\subseteq
f^{A'}(\alpha([a]))$, and so by (a) we have
\[
\uni(\eta(f^{A}([a]))) = \alpha(f^{A}([a]))\subseteq
f^{A'}(\alpha([a])).
\]
Since $\eta$ is fully-tight $f^{B}(b) =
\bigcup\eta(f^{A}([a]))$, where $b = \eta(a)$. Thus
\[
\uni(f^{B}(b)) = \uni (\bigcup \eta(f^{A}([a]))) = 
\bigcup \uni(\eta(f^{A}([a]))) \subseteq 
f^{A'}(\alpha([a])) = f^{A'}(\uni(b)).
\]
Uniqueness of $\uni$ follows from the fact that $\eta$ is epi (or
$\mu$ is mono).
\end{Proof}

\begin{Theorem}
\label{le:FSt}
$\<\epi,\mono\>$ is a factorization system for $\MASt$ where $\epi$
are all epi and $\mono$ are all mono.
\end{Theorem}

\begin{Proof}
The proof is an exact repretition of the proof of theorem~\ref{le:FSl}
where in (c) all inclusions are replaced by set equalities.
\end{Proof}

\noindent
To construct a factorisation system
 in the category $\MASc$, we might attempt to restrict the
class of epis to those used in fact~\ref{fa:closed},  defining them
dually to fully-tight epis from definition~\ref{de:jtight}.

\begin{Definition}\label{de:fc} A homomorphism $\phi:A\into B$ is 
{\em fully-closed} iff for all $f\in\Sigma$ and
$b\in\under B : f^B(b) = \bigcap_{a\in \phi^{-}(b)} \phi(f^A(a))$.
\end{Definition}
%
This, however, does not work -- in relational terms this definition
amounts to the requirement that $R^A_f;\phi\supseteq \phi;R^B_f$ {\em
and} $R^B_f$ is a maximal such relation on $B$. This is not a
compositional definition:

\begin{Example}
\[ \maly{1}{.8}
\diagram
  &   & A &  \xto[rrr]^\phi        &&&    & B & \xto[rr]^\psi &&& C & \\
\save\go[drrr]\merge\framed<5pt>\restore 1 & 2 & 3 & 4
   &&& \save\go[drr]\merge\framed<5pt>\restore 1 & 23 & 4     
   &&& \save\go[dr]\merge\framed<5pt>\restore 23 & 14 \\
  & a \ulto \uto & b \uto \urto &  &&&    & ab \uto &  &&&  ab \uto &
\enddiagram
\]
Both $\phi$ and $\psi$ satisfy definition~\ref{de:fc} but $\rho=\phi;\psi$
does not -- it would require that $f^C(ab) = 
\rho(f^A(a))\cap\rho(f^A(b)) =
\rho(\{1,2\})\cap\rho(\{3,4\}) = \{23,14\}\cap\{23,14\}$.
\end{Example}
%
Perhaps, there are some factorisation systems in $\MASc$ but we are
not inquiring into this issue any longer.

%------------------------------------------------------------------------

\subsection{Subobjects}
\label{sub:sub}

Defining, as usual, a {\em subobject\/} to be an equivalence class of
monomorphisms, we obtain three different notions of weak, tight and
closed subobjects in the three respective categories.  In terms of
classical algebra, for the respective categories $\MASt$, $\MASl$ and
$\MASc$, we obtain the following notions: $N$ is a {\em
tight} ({\em weak, closed\/}) subalgebra of $M$ (denoted $N\suba M$)
iff $\Card N\subseteq\Card M$ and $\forall f\ \forall x\in\Card N$
holds $f^N(x)=f^M(x)$ ($f^N(x)\po f^M(x)$, $f^N(x)\rpo
f^M(x)$).

In other words, $N$ is a tight (weak, closed) subalgebra of $M$ iff
$\under N\subseteq\under M$ and the inclusion $\iota:N\incl M$ is a
tight (weak, closed) homomorphism.

For a homomorphism $\phi:N\into M$, let $\phi[N]$ denote the power set
of the image of $\phi$ on the underlying set, i.e.\ $\PSet(\phi[\under
N])$ where $\phi[\under N]=\{\phi(n):n\in\under N\}$.

% Since $\phi$ maps $\under N\into\under M$, this set is actually a subset of
% the power set of some subset $\phi[\under N]\subseteq \under M$.

\begin{Prop}
\label{le:subalgs}
A homomorphism $\phi:N\into M$ induces a subalgebra $[N]$ of $M$,
which is of the same mode as $\phi$, and also a tight subalgebra $T$ of
$M$.
\end{Prop}

\begin{Proof}
The underlying set of the carrier of $[N]$ is the set
$\under{[N]}=\phi[\under N]$. For $x\in\under{[N]}$, we let
$x^-=\{n\in\under N: \phi(n)=x\}$.

1) $\phi$ is tight. Defining $f^{[N]}(x)=\phi(f^N(x^-))=f^M(x)$ for
all $x\in\under{[N]}$ obviously yields a tight subalgebra of $M$. Here
we have $[N]=T$. 

2) $\phi$ is closed. Definition as above yields a tight subalgebra
$T$.  On the other hand, defining the operations by
$f^{[N]}(x)=\bigcap_{n\in x^-}\phi(f^N(n))$, will yield a closed
subalgebra of $M$. Here $\phi$ is closed, so for any $n\in x^-:
\phi(f^N(n))\supseteq f^M(\phi(n))=f^M(x)$, and hence
$f^{[N]}(x)\supseteq f^M(x)$.  

3) When $\phi$ is weak we cannot, in general, define the induced
operations as in $M$, since these result sets may be ``too
large''. Instead, for any $x\in\phi[\under N]$ define
$f^{[N]}(x)=\bigcup_{n\in x^-}\phi(f^N(n))$.  Since $\phi$ is weak, we
have that for any $n\in x^-:\phi(f^N(n))\subseteq
f^M(\phi(n))=f^M(x)$, and hence $f^{[N]}(x)\subseteq f^M(x)$. Thus,
$[N]$ is a weak subalgebra of $M$. 

3a) To define $T$, we simply close the set $\phi[\under N]$ under all
the operations of $M$. This can be defined as $\under
T=\bigcap\{\under T_i:\phi[\under N]\subseteq\under T_i \land T_i\
{\rm is\ a\ tight\ }T_i\suba M\}$. (That this yields a tight
subalgebra of $M$ follows from proposition~\ref{le:MASlintersect}.)
\end{Proof}

\noindent
When $\phi$ is mono, the set $x^-$ will have at most one element, and
so the three definitions of the operations will have the same form:
$f^{[N]}(x)=\phi(f^N(x^-))$, yielding the subalgebra $[N]$ of the
respective mode.

% In particular, subobjects in $\MASl$ will correspond to weak subalgebras.

The different modes of homomorphisms lead also to different
constructions of minimal subalgebras or, more generally, of
intersections of subobjects.

\begin{Prop}
\label{le:MASlintersect}
For any $M$, any set $(N_i,\nu_i)$ of $M$'s subobjects in $\MASl$
(resp. $\MASt$) has an intersection $(N,\nu)=\bigcap (N_i,\nu_i)$.
\end{Prop}

\begin{Proof}
First we show the claim for $\MASl$.

1) Each $\nu_i:N_i\into M$ is a mono, so let for each $m\in\under M$,
$m^{-i}=n\in \under N_i$ such that $\nu_i(n)=m$. Each $\nu_i[N_i]$ is
a weak subalgebra of $M$.  We construct the intersection $N$ by taking
$\under {N}\Def \bigcap_i \nu_i[\under{N_i}]$ and for $m\in\under{N}:
f^N(m) \Def \bigcap_i \nu_i(f^{N_i}(m^{-i}))$.

2) The inclusion $\nu: N \incl M$ makes $(N, \nu)$ a subobject of
$M$. For each $i$, $(N, \mu_i)$ is a subobject of $N_i$, where
$\mu_i:N\into N_i$ is defined by $\mu_i(m) = m^{-i}$, and obviously
$\nu = \Comp{\mu_i}{\nu_i}$ for each $i$.

3) To see that $N$ is infimum of $(N_i,\nu_i)$, assume that there is a
subobject $(K,\kappa)$ of $M$ with mono $\kappa_i:K\into N_i$ for each
$i$, making $\kappa=\Comp{\kappa_i}{\nu_i}$.  For each $k\in\under K$
we must have (*) $\kappa(k)=\nu_i(\kappa_i(k))$,
% i.e.\ each $\kappa_i(k)=(\kappa(k))^{-i}$ 
which means that $\kappa(k)\in\bigcap_i \nu_i[\under N_i]=\under N$,
i.e.\ $\kappa[\under K]\subseteq \under N$.  (*) implies also that
$\kappa(k)^{-i}=\kappa_i(k)$. We thus obtain for each
$i:\kappa(f^K(k)) = \nu_i(\kappa_i(f^K(k))) \subseteq
\nu_i(f^{N_i}(\kappa_i(k))) = \nu_i(f^{N_i}(\kappa(k)^{-i}))$.  Thus
for each $k:\kappa(f^K(k))\subseteq f^N(\kappa(k))$ which,
together with (*), means that $\kappa[K]$ is a weak subalgebra of
$N$. Hence $\kappa':K\into N$ defined exactly as $\kappa$ is a weak
monomorphism, and so $(K,\kappa')$ is a subobject of $N$. 
%\\[1ex]

%\medskip

In $\MASt$, 

1) the carrier of $N$ is defined in the same way, while the operations
can be defined simply as $f^N(m)=f^M(m)$ (which actually
coincides with the construction in $\MASl$, since here all monos are
tight).

2) Then $(N,\nu)$ is a subobject of $M$ and $(N,\mu_i)$ a subobject of
$N_i$ by the same argument as before. 

3) Finally, tightness of $\kappa_i, \nu_i$ and $\kappa$ can be used to
show that $(N,\nu)$ is an infimum tight subobject of all $(N_i,\nu_i)$
in the same way as before.
\end{Proof}

\noindent
In particular, for any $\Sigma$, the minimal subobject of any algebra
$M$ in any $\MASl$ will be the empty algebra. 
% For void $\Sigma$, this will also be the case in $\MASt$ and $\MASc$.

% If $\Sigma$ is void (i.e.\ has no ground terms), the same will be
% the case in $\MASt$. But for non-void $\Sigma$ (with some ground 
% terms of each sort), the minimal subobject of an $M\in\MASt$ will 
% be the tight subalgebra $N$ of $M$ with $\under N$ containing exactly
% the elements which are in the sets interpreting ground
% $\Sigma$-terms in $M$.

In $\MASc$ the concept of a subobject is rather strange.  We may
restrict the carrier by removing some elements, but only those which
are not returned by any functions applied to some elements which
remained in the carrier.  On the one hand, empty subalgebra will be a
closed subobject of a given algebra $M$. On the other hand, if we keep
the same carrier $\Card N=\Card M$, and define all operations to
return full carrier, e.g., for any $x\in M_{\ovr s}=N_{\ovr s}$
and any $f:{\ovr s}\into t\ \in\Sigma$, we let $f^N(x)=N_t$, we
get a closed subalgebra $N$ of $M$.

% Thus, for a non-void $\Sigma$, a minimal subobject of an $M$ will be
% algebra $N$ with the carrier as in $\MASt$ (all elements contained 
% in the interpretation of the ground terms in $M$) but with 
% the operations returning, for all arguments, the full carrier of 
% the result sort.

These two directions of constructing subobject are incompatible making
the concept of a subobject in $\MASc$ rather far from the intended
one.

\begin{Example}
\label{ex:nointersect}
Take a signature with one constant $c$, algebra $M$ with $\under
M=\{1,2\}$ and $c^M=\{1\}$.  The algebras $N_1$ with
$\under{N_1}=\{1\}=c^{N_{1}}$ and $N_2$ with
$\under{N_2}=\{1,2\}=c^{N_{2}}$ are both (closed) subalgebras of $M$
with the obvious injections. However neither is a subobject of
another, nor is $M$ a (closed) subobject of any of them. In fact, the
intersection of $N_1$ and $N_2$ in $\MASc$ will be the empty algebra.
\end{Example}
%
Finally, we only mention another notion of subalgebra which
does not follow from the categorical considerations. It is a special
case of weak subalgebra and is used in study of partial algebras (cf.~\ref{se:pa}).

\begin{Definition}\label{de:relative}
$N$ is a {\em relative subalgebra} of $M$, $N\suba_r M$, iff $\Card
N\subseteq\Card M$ and $\forall f, x\in\under N:
f^N(x)=f^M(x)\cap\under N$.
\end{Definition}
%
The intension is that given an algebra $M$ and an arbitrary subset
$X\subseteq\under M$, the operations in $M$ determine uniquely a
relative subalgebra on this subset. We may denote this subalgebra
$M[X]$.  The operations in $M[X]$ approximate the operations in $M$
``as far as possible'' -- it is the {\em maximal} subalgebra of $M$
over the carrier $X$.

%---------------------------------------------------------------------

\subsubsection{A few remarks on Generatedness}

Example~\ref{ex:nointersect}, demonstrating the ``unnatural'' character of
intersections -- and hence, of unique minimal subobjects -- in
$\MASc$, restricts the following remarks to the two other
categories.  Proposition~\ref{le:MASlintersect}, stating the existence
of intersection in $\MASl$ and $\MASt$, implies the existence of
generated subalgebras in these two categories.

\begin{Definition}
\label{de:gen}
For $X\subseteq\Card M$, $M\genby X \Def \bigcap \{N:X\subseteq\Card
N\land N\suba M\}$.
\end{Definition}

\noindent
By proposition \ref{le:MASlintersect}, $M\genby X$ is a weak
(resp. tight) subalgebra of $M$.

We will write $M\genbyL X$, resp. $M\genbyT X$ for this construction
in the respective categories, and $M\genby X$ when stating a fact that
applies to both of them (like the above definition).

We have $M\genby X=M\genby{\under X}$ and $\Card M=\PSet(\under M)$,
so in particular $\under X\subseteq\under{M\genby
X}=\under{M\genby{\under X}}$.  By the proof of
proposition~\ref{le:MASlintersect}, in $\MASl$ and $\MASt$, $M\genby
X$ is a minimal subalgebra of $M$ containing $X$.


In $\MASt$, we will have for $T=M\genbyT{\Cons^M}:$\footnote{$\Cons$ denotes
the set of constant symbols in $\Sigma$ and $\GTerms$ the set of ground $\Sigma$-terms.
The superscript $\_^M$ indicates the interpretation of the respective (sets of) symbols
in structure $M$.}

\begin{eqnarray}
\forall f\in\Sigma\ \forall x\in\under T:\ f^T(x) 
& = & f^M(x) \label{eq:Tb} \\ 
{\rm in\ particular:\ \ }\forall t\in\GTerms:\ t^T 
& = & t^M \nonumber \\ 
{\rm and\ \ }M\genbyT{\Cons^M} 
& = & M\genbyT{\GTerms^M} \label{eq:Ta} \\
M\genbyT{\Cons^M} 
& = & M\genbyT\es \label{eq:Tc}
\end{eqnarray}

Thus, in $\MASt$, the definition~\ref{de:gen} yields the concept of
generated subalgebra which is close to the classical one.

However, things are not so nice in $\MASl$. For the first, \re{eq:Tc}
does not hold here. Since empty algebra is a subalgebra of any other
algebra, the classical notion of minimality becomes rather vacuous:
for any $M$, we have trivially $\es=M\genby\es\suba M$. In particular,
$M\genby\es\not= M\genby{\Cons^M}$.  Thus, unlike in the classical
algebra, we would have to insist that for generated algebras
$M\not=M\genby\es$.  Furthermore, neither of the equalities \re{eq:Tb}
and \re{eq:Ta} will hold. Let $\Sigma$ have one constant $c$, $M$ be
such that $\under M=c^M=\{1,2,3\}$, and $L=M\genbyL{c^M}$. Although
$\under{L}=\{1,2,3\}$, we have that $c^L=\es$! In fact, for any
$X\subseteq M$ we will have for $L = M\genbyL X$:

\begin{eqnarray}
\forall f\in\Sigma\ \forall x\in\under L :\ f^L(x) 
& = & \es \label{eq:Lb} \\
{\rm in\ particular:\ \ }\forall t\in\GTerms :\ t^L 
& = & \es \nonumber \\
\under L 
& = & \under X \label{eq:La} \\ 
{\rm in\ particular:\ \ }M\genbyL{\Cons^M} 
& \not= & M\genbyL{\GTerms^M}  \nonumber \\
M\genbyL{\Cons^M} 
& \not= & M\genbyL{\es}  \nonumber 
\end{eqnarray}

Thus, in $\MASl$, the definition~\ref{de:gen} restricts trivially the
carrier of the subalgebra and degenerates the operations according to
\re{eq:Lb}.

In order not to prolong this discussion about definitions, we would
postulate that the notion of a tight subalgebra is the most natural
generalization of the classical definition of generatedness to
multialgebra, i.e.: \eq{\label{de:genT} M{\rm\ is\ generated\ iff\
}M=M\genbyT{\es} } Notice that, since tight homomorphisms are special
cases of the weak ones, this definition may be used not only when
working in $\MASt$ but also in $\MASl$.

Definition \re{de:genT} deviates from the classical one in the
following way.  Since (ground) terms may be interpreted as sets rather
than individuals, we have to decide, in general, which individuals are
to be called ``reachable''.  Consider two possible definitions: an
$m\in\under M$ is ``$\Sigma$-reachable'' iff

\begin{eqnarray}
\exists t\in\GTerms:\ \ \{m\} & = & t^M \label{eq:reachA} \\
{\rm or\ \ \ }\exists t\in\GTerms:\ \ m & \in & t^M \label{eq:reachB}
\end{eqnarray}

The suggested notion \re{de:genT} admits generated algebras $M$ with
some elements $m\in\under M$ such that \re{eq:reachA} does not hold,
but it requires \re{eq:reachB} to hold. It merely implies that all
individuals of the carier are among the possible results returned by
some ground term.  \re{eq:reachA}, on the other hand, would require
that for each element of the carrier there is a {\em deterministically
interpreted} term $t$ denoting this element.  As a matter of fact,
``reachability'' of kind \re{eq:reachB} was introduced in \cite{c:133}
as the semantic counterpart of sound and complete inductive reasoning
with nondeterministic specifications. It was also shown there that a
natural extension of induction principle, like for \re{eq:reachB},
does not work for \re{eq:reachA}.

In the following subsection we will see that empty algebra is initial
in $\MASl$ while $\MASt$ does not, in general, have initial
objects. Consequently, no matter which of the above notions of
generatedness we choose, the classical fact about deterministic (also
partial) algebras, namely, that $A$ is generated iff the unique
homomorphism from an initial object is epi, does not hold for
multialgebras (cf. fact~\ref{fa:epimono}).

%----------------------------------------------------------------------

\subsection{Limits and co-limits in $\MAS$.}
\label{sub:limit}

We show that $\MASl$ has all (finite) limits and co-limits. The other
two categories, $\MASt$ and $\MASc$, are neither (finitely) complete
nor co-complete.  In the first subsection we study $\MASl$ and in the
next one the other two categories.

\subsubsection{Completeness and Co-completeness of $\MASl$.}

To begin with, unit algebra is obviously terminal in $\MASl$.

\begin{Fact}\label{le:init}
Unit algebra is terminal in $\MASl$.
\end{Fact}

\begin{Fact}
Empty algebra is initial in $\MASl$.
\end{Fact}

\begin{Proof}
Since homomorphisms are induced from their definition on the
underlying set, when this set is empty, there is a unique weak
function from it into any other set. Thus, the ampty algebra is an
initial (albeit, hardly interesting) object in $\MASl$.
\end{Proof}

\noindent
In particular, the $\Sigma$ word multistructure $T_\Sigma$ given by:
%
\begin{itemize}\MyLPar
\item 
the underlying set $\under T_\Sigma$ which is the set of ground
$\Sigma$-terms, $\GTerms$,

\item 
interpretation of the operations defined in the obvious way:
$t^{T_{\Sigma}}=\{t\}$.
\end{itemize}
%
is not initial in $\MAS$.

\begin{Lemma}
$\MASl$ has all binary products.
\end{Lemma}

\begin{Proof}
A product $M_1\times M_2$ of multialgebras $M_1$ and $M_2$ can be
given by the carrier $\carrier{\under M_1\times \under M_2}$ and
operations defined by $f=f^{M_1\times M_2}(\pair {m_1}{m_2}) =
f^{M_1}(m_1)\times f^{M_2}(m_2)$.  This is a well-defined multialgebra
with composition given by $\Comp gf(\pair{m_1}{m_2}) =
g(f^{M_1}(m_1)\times f^{M_2}(m_2)) =
\Comp{g^{M_1}}{f^{M_1}}(m_1)\times \Comp{g^{M_2}}{f^{M_2}}(m_2)$, and
$\pi_i(\pair{m_1}{m_2})\mapsto m_i$ are the obvious projections.

For any $N$ with homomorphisms $h_i:N\into M_i$, the unique morphism
$u:N\into M_1\times M_2$ making $h_i=\Comp u{\pi_i}$ is given by
$u(n)=\pair{h_1(n)}{h_2(n)}$. It is a weak homomorphism since:
$u(f^N(n)) = \{\pair{h_1(a)}{h_2(a)}:a\in f^N(n)\} \subseteq
h_1(f^N(n))\times h_2(f^N(n)) \subseteq$ (since $h_i$ are weak)
$f^{M_1}(h_1(n))\times f^{M_2}(h_2(n)) = f(\pair{h_1(n)}{h_2(n)}) =
f(u(n))$.
\end{Proof}

\noindent
Notice that the first inclusion will be strict when
$\Card{f^N(n)}>1$. Hence, this construction does not work in $\MASt$
or $\MASc$. Also, $S\times\es=\es$, i.e.\ if we have
$f^{M_{1}}(m_1)=S$ and $f^{M_{2}}(m_2)=\es$, we will get
$f^{M_{1}\times M_{2}}(\<m_1,m_2\>)=\es$.

\begin{Lemma}
$\MASl$ has equalizers.
\end{Lemma}

\begin{Proof}
As an equalizer of a pair $h_1,h_2:M\into N$, we take a multialgebra
over underlying set $\under E = \{m\in\under M:h_1(m)=h_2(m)\}$, with
operations defined in the obvious way: $f^E(m)=f^M(m)|_{\Card E}$, and
with the inclusion $j:\under E \incl \under M$.  Obviously, $\Comp
j{h_1}=\Comp j{h_2}$ and $j$ is a weak homomorphism.

If $g:T\into M$ is such that $\Comp g{h_1}=\Comp g{h_2}$, then define
$u:T\into E$ by $u(t)=g(t)$. It is necessarily unique making $g=\Comp
uj$. It is also well-defined since if there is a $t$ such that
$g(t)\not\in \under E$, then $\Comp g{h_1}(t)\not=\Comp g{h_2}(t)$.
\end{Proof}

\noindent
In the construction, both $\under E$ and $f^E(m)$ may happen to be
empty. Thus, $j$ is typically a strict inclusion and so the
construction does not work in $\MASt$ or $\MASc$. Also, if $\under E$
is empty, there is no morphism $u:T\into E$. However, then there is no
morphism $g:T\into M$, which would make $\Comp g{h_1}=\Comp g{h_2}$,
so there is no need to construct such a $u$.

% \begin{Lemma}
% $\MASt, \MASc$ do not have equalizers.
% \end{Lemma}
% \begin{Proof}
% \end{Proof}

As an immediate corollary we have

\begin{Theorem}\label{th:MASlcmpl}
$\MASl$ has all finite limits.
\end{Theorem}

% \noindent
% Combined with lemma~\ref{le:MASlintersect} we obtain {\em strong} epi-mono
% factorization strengthening theorem~\ref{th:epimono}.3.
% 
% \begin{Corollary}\label{co:strongepimono}
% $\MASl$ has strong epi-mono factorization.
% \end{Corollary}
% \begin{Proof}
% This is the general result stating that in a finitely complete
% category (th.~\ref{th:MASlcmpl}) with all intersections
% (not only over {\em sets} like in le.~\ref{le:MASlintersect}), 
% each morphism factors through a strong epi and a mono.
% \end{Proof}

\begin{Lemma}
\label{le:coeql}
$\MASl$ has co-equalizers.
\end{Lemma}

\begin{Proof}
For a pair $h_1,h_2:M\into N$ let $\sim$ be the least equivalence on
$\under N$ induced by the relation $\{\pair{h_1(m)}{h_2(m)}:m\in\under
M\}$. Take $C=\quot M\sim$ and the cannonical (weak) homomorphism 
$k:N\into C$. Obviously,
$\Comp{h_1}k=\Comp{h_2}k$.

If $g:N\into Z$ is such that $\Comp{h_1}g=\Comp{h_2}g$ then certainly
the kernel of $g$, $\sim_g$, must include $\sim$. The mapping
$u:C\into Z$ defined by $u([n]_\congo)=g(n)$ is unique one making
$\Comp ku=g$. By fact~\ref{fa:subeq} it is a weak homomorphism.
\end{Proof}

\begin{Lemma}
\label{le:MASlcopr}
$\MASl$ has co-products.
\end{Lemma}

\begin{Proof}
As a co-product $M=M_1+M_2$ take a multialgebra over the underlying
set being the disjoint union $\under M_1 \uplus \under M_2$.  Each
constant $c$ is interpreted as $c^M=c^{M_1}\uplus c^{M_2}$.  If
$m \in M_1$, let
$f^M(m)=f^{M_1}(m)$ and similarly if all the
arguments $m\in M_2$. For the mixture of arguments from both
$m_1\in M_1$ and $m_2\in M_2$, let
$f^M(m_1,m_2)=\emptyset$. The inclusions $\iota_i:M_i\incl M$
are the obvious injections. They are tight except for constants for
which they are, in general, weak.

Given any pair $\phi_i:M_i\into N$, the homomorphism $\uni:M\into N$
defined by $\uni(\iota_1(m))=\phi_1(m)$ and
$\uni(\iota_2(m))=\phi_2(m)$ is unique one making
$\phi_i=\Comp{\iota_i}\uni$, for $i=1,2$. It is of the same mode as
both $\phi_i$, i.e.\ weak if $\phi$'s are weak, tight if they are
tight, and closed if they are closed.

Thus, if $\Sigma$ does not contian any constants, the construction
works for all three categories but in the general case we have the
result only for $\MASl$.
\end{Proof}

\noindent
Having initial object, co-equalizers and co-products in $\MASl$, we
obtain the corollary

\begin{Theorem}
\label{th:cocomp}
$\MASl$ has all finite co-limits.
\end{Theorem}


\subsubsection{The categories $\MASt$ and $\MASc$.}

Empty algebra won't be initial in $\MASc$ or $\MASt$. In fact:

% (If $\Sigma$ contains one constant $c$ and $c^M=\es$ there is no 
% tight nor closed homomorphism

\begin{Fact}
\label{le:noinit}
$\MASc$ and $\MASt$ do not have initial objects.
\end{Fact}

\begin{Proof}
Let us consider $\MASt$ and a $\Sigma$ with one constant $c$. Assuming
that $I$ is initial in $\MASt$, let $S=c^I$ be the set interpreting
$c$ in $I$.  Let $A$ be another algebra with
$c^A=\PSet(S)$. Obviously, there is no tight homomorphism $I\into A$,
so $I$ is not initial after all. The same argument works for $\MASc$.
\end{Proof}

\noindent  
Since algebras may have empty carriers, and operations may return
empty sets, unit algebra will not be terminal in $\MASt$ or
$\MASc$. It would be, if we considered only non-empty carriers and
total multialgebras (all operations returning non-empty sets of
values). Since a given function $f$ may in some algebras return empty
and in other non-empty set, e-unit algebra won't be terminal in
$\MASt$.  However, it will be terminal in $\MASc$.

\begin{Fact}
E-unit algebra is terminal in $\MASc$ but not in  $\MASt$.
\end{Fact}


\begin{Fact}
$\MASt$ and $\MASc$ do not have equalizers.
\end{Fact}

\begin{Proof}
Let $\Sigma$ have one constant $c$, $\under M=c^M=\{a,b\}$, $\under
N=c^N=\{1,2\}$ and consider two tight (and closed) homomorphisms
$\phi_1(a)=1,\ \phi_1(b)=2$, and $\phi_2(a)=2,\ \phi_2(b)=1$. An
equalizer $E$ in $\MASt$ (or $\MASc$) must have at least two elements
with $\{x,y\}\subseteq c^E$, in order to have a tight (closed)
homomorphism into $M$. However, any definition of a homomorphism
$\psi:E\into M$ will then violate the requirement
$\Comp\psi{\phi_1}=\Comp\psi{\phi_2}$, so that this pair of arrows
does not have any equalizer in $\MASt$ (or $\MASc$).
\end{Proof}

\begin{Fact}\label{le:noprod}
$\MASt, \MASc$ do not have products.
\end{Fact}

\begin{Proof}
% Take two isomorphic $M_1$ and $M_2$ over $\Sigma$ with only one sort
% symbol. It is set, so their product must be the cartesian product of
% their carriers. 
% Now, let $\Sigma$ have in addition one constant symbol $c$. Carrier of
% the product must be ``something like'' the product of the carriers (by
% the first paragraph). Then there are isomorphisms from $M_1$ to both
% $M_1$ and $M_2$ but no tight morphism into their product where $c$
% will have greater cardinality than in $M_1$.
Consider $\MASt$, and take $\Sigma$ with one sort $S$ and one constant
symbol $c$.  Let $M_S=\under M=\{1,2\}$, $N_S=\under N=\{a,b\}$ and
$c^{M}=M_S$, $c^{N}=N_S$.  Suppose that $P$ is their product. Let
$K_S=\under K=\{x,y,z\}$ and $c^K=K_S$, and define $m:K\into M$,
$n:K\into N$ by: $m(x)=1, m(y)=1, m(z)=2$ and $n(x)=a, n(y)=b,
n(z)=b$. 
% under var det et trykkfeil paapekt av Adis:
Assuming unique $\uni:K\into P$ making $\Comp\uni{\pi_M}=m$
and $\Comp\uni{\pi_N}=n$, we must have following values of $\pi_M,
\pi_N:$ 
{\small { \[ \begin{array}{c|cc} & \pi_M & \pi_N \\ \hline
\uni(x) & 1 & a \\ \uni(y) & 1 & b \\ \uni(z) & 2 & b \end{array} \] }
} which means that $\under P$ must have at least three elements
$x_1=\uni(x), y_1=\uni(y), z_1=\uni(z)$.  Since $\uni$ is tight, we
must have that $c^P=\{x_1,y_1,z_1\}$.

Now take $id:M\into M$ and $n:M\into N$ (e.g., $n(1)=a,
n(2)=b$). Since $P=M\times N$, we thus must have a unique tight
homomorphism $h:M\into P$ (whose compositions with the projections
yield the respective morphisms $id$ and $n$). But there is no such
tight morphism, since $|c^M|=2 < |c^P|$.

The same argument works for $\MASc$.
\end{Proof}

\noindent
Since, by lemma~\ref{le:canonichom} the canonical homomorphism into a
quotient, $N\into \quot N\sim$, is weak, the construction of
co-equlizers from lemma~\ref{le:coeql} does not seem to yield
co-equalizers for $\MASc$ and $\MASt$.  However, we can show that in
$\MASt$, for $h_1,h_2:M\into N$, the equivalence $\sim$ defined as
before (i.e.\ the least equivalence induced by
$\{\pair{h_1(m)}{h_2(m)}:m\in\under M\}$) will lead to the tight
natural homomorphism $k:N\into \quot N\sim$.

\begin{Fact}
$\MASt$ has co-equalizers.
\end{Fact}

\begin{Proof}
The construction, argument (and notation) are the same as in
lemma~\ref{le:coeql}, but we have to verify that $k:N\into
\quot N\sim$ will be tight when $h_1,h_2$ are tight.

1) We show first that \eq{\label{eq:cong} n_1\sim n'_1\ldots n_k\sim
n'_k\ \ \impl\ \ f^N(n_1...n_k)\sim f^N(n'_1...n'_k) } where for
$A,B\subseteq\under N$, the notation $A\sim B$ means $(\forall a\in
A\exists b\in B:a\sim b)\ \land\ (\forall b\in B\exists a\in A:a\sim
b)$. (In 1a), 1b) wew write $n,n'$ for the sequences.)
% To simplify notation, let us write everything for one argument functions.

1a) If $n\sim n'$ because there is an $m:n=h_1(m)\land n'=h_2(m)$,
then we get, by tightness of $h_1$, $h_2$ that
$f^N(n)=f^N(h_1(m))=h_1(f^M(m))$ and
$f^N(n')=f^N(h_2(m))=h_2(f^M(m))$. The set $f^M(m)=\{m_1...m_p\}$
gives then rise to the pairwise equivalence $h_1(m_i)\sim h_2(m_i)$
which yields the required $f^N(n)\sim f^N(n')$.

1b) If $n\sim n'$ but there is no such $m$, % such that $n=h_1(m)$ and
$n'=h_2(m)$, then there is a ``chain'' $n=n_1\sim n_2\sim...\sim
n_z=n'$ with $n_i=h_1(m_i)$ and $n_{i+1}=h_2(m_i)$.
%$n=h_1(m_1), n_2=h_2(m_1), n_2=h_1(m_2), n_3=h_2(m_2),..., n_z=h_1(m_z), n'=h_2(m_z)$
%or else $...n_z=h_2(m_z), n'=h_1(m_z)$. 
Applying the argument from 1a) for each consequtive pair $n_i\sim
n_{i+1}$ gives us, by transitivity, the conclusion $f^N(n)\sim
f^N(n')$.

2) The claim that $k:N\into \quot N\sim$ is tight amounts to saying
that for any representatives $n'_1\in[n_1]...n'_k\in[n_k]$ and for any
$[n]\in f^{N\Mquo_{\sim}}([n_1]...[n_k])$, some element from $[n]$ is
in $f^N(n'_1...n'_k)$, i.e.\ $[n]\cap f^N(n'_1...n'_k)\not=\es$.  But
this follows immediately from \re{eq:cong} -- for any representatives
from the equivalence classes $[n_1]...[n_k]$, the result set
$f^N(n'_1...n'_k)$ contains at least one representative from each
equivalence class which is in the result set of
$f^{N\Mquo_{\sim}}([n_1]...[n_k])$.
\end{Proof}


\begin{Fact}
$\MASt$ and $\MASc$ do not have co-products.
\end{Fact}

\begin{Proof}
Consider a signature with one constant $c$ and two multialgebras with
$c^N=\{1,2\}$ and $c^M=\{1,2,3\}$. Assuming that they have a
co-product $P$ we must have two tight injections $\iota_M:M\into P$
and $\iota_N:N\into P$. Obviously, $c^P$ cannot have exactly one
element (for then the universal property would fail for an algebra $Z$
with $c^Z=\{a,b\}$ and morphisms $\phi_M,\phi_N$ defined, e.g.,
$\phi_N(1)=a=\phi_M(1),\ \phi_N(2)=b=\phi_M(2)$, and $\phi_M(3)=a$).

Hence $c^P$ must have exactly two elements (otherwise there is no
tight $\iota_N$).  Without loss of generality, assume that
$c^P=\{a,b\}$ and $\iota_N(\{1,2\})=c^P=\iota_M(\{1,2\})$, while
$\iota_M(3)=\iota_M(2)$.  Considering $\phi_M:M\into P$ defined as
$\iota_M$ on $1$, $2$ with $\phi_M(3)=\phi_M(1)$, we see that there is
no tigth homomorphism $\uni:P\into P$ which would make
$\Comp{\iota_M}\uni=\phi_M$. So, there is no co-product of $M$ and $N$
in $\MASt$.

Essentially the same argument works for $\MASc$ (we only have to
consider, in addition, the possibility of a strictly closed
$\iota_M$).
\end{Proof}

\subsubsection*{Thus ...} 

we have that $\MASl$ is (finitely) co-complete and complete, 
while $\MASt$ and $\MASc$ are neither.
The following table summarizes the results of this section:

\begin{table}[htb]
\begin{center}
\begin{tabular}{|r|c|c|c|}
\hline 
                   & $\MASl$ & $\MASt$ & $\MASc$ \\ \hline
products           & +  & -- & -- \\
equalizers         & +  & -- & -- \\
{\bf completeness} & +  & -- & --  \\ \hline
initial objects    & +  & -- & -- \\
% (of interest)  & (--) & (--) & (--) \\
co-products        & +  & --  &  -- \\
co-equalizers      & +  & +  &  \\
{\bf co-completeness} & +  & -- & -- \\ \hline
epi-mono factorisation & +  & +  & + \\ 
factorisation system & +  & +  &  \\ \hline %\vspace*{-2ex}
\end{tabular}
\end{center}
\caption{Properties of various $\MAS$.}\label{tab:A} \vspace*{-3ex}
\end{table}

Although the initial object in $\MASl$ -- the empty algebra -- isn't
particulary exciting, we see that this category possesses many desirable
properties not possessed by the other two categories.
It has perhaps been a prevailing opinion among mathematicians
interested in the question that weak homomorphisms of relational
structures provide the most useful notion.  The above results justify
and demonstrate this opinion. We have thus studied all the cases of
one type of compositional homomorphisms -- case 1) from
theorem~\ref{th:comp}. We conjecture that the negative results about
the tight ($=$) and closed ($\supseteq$) homomorphisms of this type will also occur for
the tight and closed variants of the remaining three types.


%====================================================================
%====================================================================


\section{Homomorphisms as Relations}
\label{se:horel}

Since carrier of a multialgebra is a power set, one might attempt to
define a multialgebra as a usual algebra over such a carrier.  Such a
generality, however, reduces to the considerations of standard
algebras. In order to capture the intended (power) set structure, some
additional restrictions have to be imposed, the least of which is
monotonicity of the operations within the algebra. (Such a variant,
called power algebras, was used, for instance in \cite{c:130}, for
modelling a kind of ``call-by-name'' passing of nondeterministic
parameters.)

We will not consider this alternative here and, instead, will look
closer at the alternative notions of homomorphisms between
multialgebras. There are three
immediate possibilities here; a homomorphism $\phi:M\into N$ may be of
three kinds:

\[ \begin{array}{rll}
\PP: & \under M \into \under N & {\rm point-point} \\
\PS: & \under M \into \PSet(\under N) & {\rm point-set} \\
\SSs:& \PSet(\under M)\into \PSet(\under N) & {\rm set-set}
\end{array}
\]
%
The first one has been considered so far, while the last one, when
taken without additional restrictions, treats the multialgebras,
analogously as in power algebras, as standard algebras which only ``by
accident'' happen to have power sets as carriers.  Of course, a
\PP-homomorphism induces a \PS\ one by pointwise extension, and a
\PS-homomorphism induces an \SSs\ one by additive extension.  I.e.,
modulo the obvious type-conversion, we have: $\mode\PP\phi \impl
\mode\PS\phi \impl \mode\SSs\phi$. All three kinds can be further
combined with the three modes -- tight, weak, or closed -- leading to
a taxonomy of various possibilities which is simple in structure but
messy in interrelations.  We write $\MASS$ and $\MAPS$ for the
respective categories (with \SSs\ and \PS\ homomorphisms) with
appropriate subscripts when necessary.  We will consider in some
details three categories $\MAPS$, highlighting merely some points
illustrating the lack of equally pleasing categorical properties as
those possesed by the category $\MASl$.  The three categories to be considered now
correspond to the three considered in previous section, where a morphism 
$\phi:A\into B$ is given by
\[
\begin{array}{rrcl}
\MAPSl & R^A; \phi & \subseteq & \phi; R^B \\
\MAPSt & R^A; \phi & = & \phi; R^B \\
\MAPSc & R^A; \phi & \supseteq & \phi; R^B 
\end{array}
\]
The difference is that here $\phi$ is a {\em relation}. (This generalization of
the notion of homomorphism to relations has been studied,
with a different emphasis, in \cite{koh2}.) It is easy to verify that this generalization
still yields compositional definitions.
The only positive result, and
the one not present in $\MAS$, is the construction of initial objects
in $\MAPSt$ which do resemble the classical term algebras
(proposition~\ref{le:MAPStADJ}). This seems the only reason for which this category
attracted some attention, e.g., in \cite{c:57,c:59}.
Before proceeding further, we mention ...
\subsubsection*{... a general fact about isomorphisms}
In proposition~\ref{le:isotight} we concluded that isomorphisms of any mode
are actually tight homomorphisms. The lemma was given for the 
\PP-homomorphisms but, as it turns out, it does not depend on the kind
of homomorphisms:

\begin{Prop}
\label{le:isopp}
Let $\phi:M\leftrightarrow N:\psi$ be an isomorphism. Then
$\mode\SSs{\phi/\psi}\impl\mode\PS{\phi/\psi}\impl\mode\PP{\phi/\psi}$.
\end{Prop}

\begin{Proof}
First, we show that $\phi$ (and $\psi$) must map underlying sets onto
underlying sets.  Suppose not, i.e.\ $\exists x\in\under M :
\phi(x)=Y\not\in\under N$. Then $Y=\posup_i y_i$ for some set of
indices $i$ with all $y_i\in\under N$. Isomorphism condition requires
then $\psi(Y)=x$, and monotonicity of $\psi: \psi(y_i)=x$. But then
$\phi(\psi(y_i))=Y\not=y_i$, so $\phi, \psi$ would not be an
isomorphism.

Now, consider $\mode\PS{\phi,\psi}$, and assume that
$\neg\mode\PP\phi$, i.e.\ $\exists X\in\Card M : X=\posup_i x_i \land
\phi(X)=Y\not=\posup_i \phi(x_i)$. Then $\exists y_j\in
Y:y_j\not\in\posup_i \phi(x_i)$.  By isomorphism condition
$\psi(Y)=X$, and by monotonicity of homomorphisms $\psi(y_j)\in
X$.(Monotonicity forces $\psi(y_j)\po X$ and the first part of the
proof makes this $\po$ into $\psi(y_j)\in X$.) But this means that
there is an $x_i\in X$ such that $x_i=\psi(y_j)$ and we get
$\phi(\psi(y_j) = \phi(x_i) \in\posup_i \phi(x_i)$. Since $y_j$ was
assumed to be in $Y\setminus{\posup_i\phi(x_i)}$, we would then have
$\phi(\psi(y_j))\not= y_j$, contradicting the assumption that $\phi,
\psi$ are isomorphisms.

Entirely analogous argument as above shows a contradiction under the
assumption that $\mode\SSs{\phi/\psi}$ and $\neg\mode\PS\phi$.
\end{Proof}

% Thus isomorphisms in all different categories
% $\MAlg^{\XX}_{x}(\Sigma)$ are actually the \PP-isomorphisms from
% $\MASt$.

Thus isomorphisms in all different categories $\MAPS$ and $\MASS$
are actually the \PP-isomorphisms from $\MASt$.



%-------------------------------------------------------------------

\subsection{The category \PSt }

As a technical tool, in this subsection we introduce the category
\PSt, which is a closer counterpart of $\MAPSt$ than \Set. In fact,
\PSt\ is a special case of any of the three $\MAPS$ categories, when
$\Sigma$ contains only one sort symbol. Thus the negative results for
\PSt\ can be immediately applied to $\MAPS$.  Characterization of epis
and monos, as well as non-existence of epi-mono factorizations for
$\MAPS$ will be obtained from the results in this subsection.

\begin{Definition}
The objects of \PSt\ are power sets, i.e.\ for each object $S$ of
\Set, $\PSet(S)$ is an object of \PSt.  A morphism in \PSt,
$\under\phi:\PSet(S)\into \PSet(T)$ is a \Set\ morphism
$\under\phi:S\into\PSet(T)$ extended pointwise, i.e.\ for
$X\in\PSet(S) : \under\phi(X) = \bigcup_{x\in X}\under\phi(x)$.
\end{Definition}

\begin{Prop}
\label{le:adjSPS}
Let $I:\PSt\into\Set$ be the inclusion and $\PSet:\Set\into\PSt$ send
each set to its power set $S\mapsto\PSet(S)$, and each morphism to its
pointwise extension to power set. These functors are adjoint
$\PSet\adj I$.
\end{Prop}

\begin{Proof}
Let $\under\phi:A\into\PSet(B)$ be a morphism in $\Set$ and
$\phi:\PSet(A)\into\PSet(B)$ be its pointwise extension. The unit of
adjunction is embedding $\iota:A\into\PSet(A)$ and we have the obvious
commutativity $\under\phi = \Comp\iota{I(\phi)}$. Uniqueness of $\phi$
follows from the fact that it must coincide with $\under\phi$ on all
$a\in A$ and that it must be extended pointwise to be a morphism in
$\PSt$.
\end{Proof}

% \begin{Claim}\label{le:cocomp}
% \PSt\ is co-complete.
% \end{Claim}
% \begin{Proof}
%  \michal{This proof is really wrong -- $\PSet$ is surjective on the 
% objects but it messes up with morphisms: in $\PSt$ we get co-limits/co-equalizers
% only for morphisms which come from \Set, i.e.\ only for \PP-morphisms.}
% By lemma~\ref{le:adjSPS}, $\PSet$ preserves co-limits.  Hence,
% since \Set\ is co-complete and all objects of $\PSt$ are in the image
% of $\PSet$, so $\PSt$ is co-complete.\footnote{However, $I$ isn't full,
% and so \PSt\ isn't a reflective subcategory of \Set.}
% \end{Proof}
%

\noindent
In particular, the initial object (preserved by $\PSet$) in $\PSt$
will be $\PSet(\es)=\{\emptyset\}$.  It is initial in the same way as $\emptyset$
is initial in \Set.  This is also a terminal object in $\PSt$ because
morphisms send elements of the source to subsets of the target, and
hence the unique morphism from any other set will send all its
elements to $\emptyset$.

\begin{Fact}
\label{le:PStcompl}
\PSt\ is complete.
\end{Fact}

\begin{Proof}
1) It has equalizers. For $\psi_1,\psi_2:S\into\PSet(T)$, let $E\Def
\{X\in\PSet(S)\land\psi_1(X)=\psi_2(X)\}$, and $\phi:E\into\PSet(S)$
be defined as $\phi(\{X\})\Def X$. Obviously,
$\Comp\phi{\psi_1}=\Comp\phi{\psi_2}$. Whenever $\eta:A\into\PSet(S)$
satisfies the same (last) equality, we can define a unique
$\uni:A\into\PSet(E)$ making $\eta=\Comp\uni\phi$ by
$\uni(a)\Def\{\eta(a)\}$.

2) It has products. For $\PSet(A_i)$, let $E\Def \bigdcup_i A_i$
(disjoint union), and $\pi_i:E\into\PSet(A_i)$ be given by
$\pi_i(e)\Def\{e\}$ if $e\in A_i$ and $\pi_i(e)\Def\emptyset$
otherwise.
% Then, for each $i$, we define an inverse mapping $\_^{-i}:
% \PSet(A_i)\into \PSet(E)$ by 
% $X^{-i}= \left \{ \begin{array}{cl}
%                   \es & {\rm if\ } X=\es \\
%                   \{e\in E:\pi_i(e)\in X\} & {\rm otherwise}
%     \end{array} \right . $

For any object $S$ and arrows $\psi_i:S\into A_i$, the morphism
$\uni:S\into\PSet(E)$ given by $\uni(s) \Def \bigdcup_i(\psi_i(s))$
obviously makes $\psi_i=\Comp\uni{\pi_i}$. Uniqueness of $\uni$
follows from the fact that if for some $s$, $\rho:S\into\PSet(E)$ is
such that $\rho(s)\not=\bigdcup_i(\psi_i(s))$, then there is an $i$
such that $\Comp\rho{\pi_i}(s)\not =\psi_i(s)$. Thus
$(\PSet(E),\pi_i)$ is a product of $\PSet(A_i)$.

3) $\PSt$ has also terminal object -- the empty algebra -- and so it
has all limits.
\end{Proof}

%%% 
%%% \michal{ I have a slight problem: \\
%%% (1) Using the adjunction from the lemma~\ref{le:adjSPS}: $I$
%%% preserves limits, hence a product $\PSet(X)$ of (two) objects
%%% $\PSet(A_1)$ and $\PSet(A_2)$ in \PSt\ must be isomorphic (in $\Set$)
%%% to their (cartesian) product $\PSet(A_1)\times\PSet(A_2)$.
%%% \\
%%% (2) By lemma~\ref{le:isopp} a $\PSt$
%%% isomorphism $\PSet(A)\iso \PSet(B)$ is the same as a $\Set$ isomorphism
%%% $A\iso B$.
%%% \\
%%% (3) I cannot figure out how (or what) to make isomorphic here.
%%% }

\noindent
However, perhaps a bit surprisingly, the category $\PSt$ is not co-complete.

\begin{Lemma}
\label{le:PSnocoeq}
$\PSt$ does not have co-equalizers.
\end{Lemma}

\begin{Proof}
Let $A=\{a\}$ and $B=\{1,2,3,4\}$ and consider two morphisms
$\phi_1(a)=\{1,2\}$ and $\phi_2(a)=\{3,4\}$. We will consider two
morphisms $\psi_i:B\into Z$ with $Z=\{x,y\}$ defined by
\[
\begin{array}{rcccl@{\ \ \ \ \ \ \ \ \ \ }rcccl}
\psi_1(1) & = & \psi_1(3) & =& x &  \psi_2(1) & = & \psi_2(4) & = & x \\
\psi_1(2) & = & \psi_1(4) & =& y &   \psi_2(2) & = &\psi_2(3) & = & y \\
\end{array} 
\]

1) For the first, assuming that there is a co-equalizer $\nu:B\into
C$, we must have that $\Comp{\phi_1}\nu=\Comp{\phi_2}\nu$, which means
$\nu(\phi_1(a))=\nu(\{1,2\})=\nu(\{3,4\})=\nu(\phi_2(a))$, i.e.\ it
must be possible to partition this image $\nu(B)=N\subseteq C$ in two
ways as $\nu(1)\cup\nu(2)=N=\nu(3)\cup\nu(4)$.

2) On the other hand, we have for $i=1,2:
\psi_i(\phi_1(a))=\psi_i(\phi_2(a))$, so there must exist unique
$\uni_i:C\into Z$ making $\Comp{\nu}{\uni_i}=\psi_i$ for $i=1,2$. We
show that no such $\uni_i$ exist.

3) In order to make $\Comp\nu{\uni_1}=\psi_1$ we must have
$\nu(1)\not=\nu(2)$, in fact, $\nu(1)\cap\nu(2)=\es$, for otherwise,
we would get $\uni_1(\nu(1))\cap\uni_1(\nu(2))\not=\es$ and so at
least one of these would be different from $\psi_1$. The same argument
forces $\nu(1)\cap\nu(4)=\es$, $\nu(3)\cap\nu(2)=\es$ and
$\nu(3)\cap\nu(4)=\es$.  Similarly, considering $\psi_2$ we must have
$\nu(1)\cap\nu(3)=\es$ and $\nu(2)\cap\nu(4)=\es$.  Together, these
observations force all $\nu(n)$ for $n=1,2,3,4$ to be mutually
disjoint!

4) Combining this result with the conclusion of point 1. we get an
impossibility. Point 1. implied two partitions of the set $N$ into,
respectively, $\nu(1)\cup \nu(2)$ and $\nu(3)\cup\nu(4)$. But there do
not exist two such partitions with all parts $\nu(n)$ mutually
disjoint.
\end{Proof}

\begin{Lemma}
\label{le:charepimono}
A morphism $\phi:S\into \PSet(T)$ in $\PSt$ is:

\begin{enumerate}\MyLPar
\item 
mono iff its pointwise extension to $\PSet(S)$ is injective:
$\forall X,Y\subseteq S: X\not=Y \impl \phi(X)\not=\phi(Y)$;
\item 
epi iff it is surjective on $T$, i.e.\ $\forall t\in T \exists s\in S:
\phi(s)=\{t\}$.
\end{enumerate}

% \\ \marcin{dla tight: if $\phi$ is injective}
\end{Lemma}

\begin{Proof}
1) Assume that the extension of $\phi$ is injective, i.e.\ for any
$X_1\not= X_2\in\PSet(S): \phi(X_1)\not=\phi(X_2)$ and there is an $A$
and $\psi_1\not=\psi_2:A\into \PSet(S)$.  Then there is an $a\in\under
A$ such that $X_1=\psi_1(a)\not=\psi_2(a)=X_2$. But then also
$\phi(\psi_1(a))\not=\phi(\psi_2(a))$.

For the opposite assume there are $X_1\not=X_2 \in\PSet(S)$ such that
$\phi(X_1)=\phi(X_2)$.  Take $\PSet(\{a\})$ and
$\psi_i:A\into\PSet(S)$ defined by $\psi_1(a)\Def X_1$ and
$\psi_2(a)\Def X_2$.  Although $\psi_1\not=\psi_2$, their compositions
with $\phi$ will be equal. Hence $\phi$ isn't mono. 

2) Assume $\exists t\in T: \forall s\in S: \phi(s)\not=\{t\}$.  Take
$Z=\{z_1,z_2\}$ and define $\psi_1,\psi_2:T\into \PSet(Z)$ by
$\psi_i(t)\Def\{z_i\}$ and $\psi_i(t')\Def \{z_1,z_2\}$ for all
$t'\not=t$. Although $\psi_1\not=\psi_2$, we have that
$\Comp\phi{\psi_1}=\Comp\phi{\psi_2}$.

For the opposite, assume $\phi$ is surjective on $T$, and there are
$\psi_1\not=\psi_2:T\into \PSet(Z)$. Then there is a $t\in T$ such
that $\psi_1(t)\not=\psi_2(t)$. But since there is an $s\in S$ such
that $\phi(s)=\{t\}$, we will then have that also
$\Comp\phi{\psi_1}\not=\Comp\phi{\psi_2}$.
\end{Proof}

\begin{Lemma}
\label{le:noepi-mono}
$\PSt$ does not have epi-mono factorisation.
\end{Lemma}

\begin{Proof}
Let $S=\{a,b,c\}$ and $T=\{1,2,3\}$ and $\phi:S\into\PSet(T)$ be
$\phi(a)=\{1,2\},\ \phi(b)=\{1,3\},\ \phi(c)=\{2,3\}$. $\phi$ is
neither epi nor mono.  Assume it has an epi-mono factorisation
$\phi=\Comp\eta\mu$, with an epi $\eta:S\into\PSet(Q)$ and a mono
$\mu:Q\into\PSet(T)$. $Q$ must have at least three elements which
$\mu$ sends, respectively, $\mu(x)=\{1,2\},\ \mu(y)=\{1,3\},\
\mu(z)=\{2,3\}$ in order to yield $\phi$ in composition with
$\eta$. But this means that $\mu$ cannot be mono, since we will have
$\mu(\{x,y\})=\mu(\{y,z\})$.
\end{Proof}

%%%%%%%%%%%%%%%% unnecessary Claim
%%%% \begin{Claim}
%%%% $\PSt$ does not have intersections (for arbitrary class of subobjects).
%%%% \end{Claim}
%%%% \begin{Proof}
%%%% Lemma~\ref{le:noepi-mono} implies that $\PSt$ does not have strong
%%%% epi-mono factorization.  Since it is (finitely) complete
%%%% (lemma~\ref{le:PStcompl}), it can not have arbitrary intersections.
%%%% \end{Proof}


%----------------------------------------------------------------------

\subsection{The categories $\MAPS$}

\begin{Fact}
\label{le:terMAPS}
Empty algebra is terminal in each $\MAPS$ category.
\end{Fact}

\begin{Proof}
%One-point algebra is obviously terminal in any $\MAPS$ with non-empty
%carriers. If empty carriers are allowed, t
% with empty carrier, i.e.\ $\<\emptyset,\{\emptyset\}\>$.
Since \PS\ homomorphisms send points on sets, the unique -- tight --
morphism from any algebra $M$ to the one with empty carrier will send
all elements of $M$ onto $\es$.
\end{Proof}


\noindent
With weak \PS\ homomorphisms empty algebra will also be initial.
When homomorphisms are tight, the initial algebra begins to be interesting.

\begin{Prop}
\label{le:MAPStADJ}
Let $U:\MAPSt\into\PSt$ be the forgetful functor given by
$U(M)\Def\PSet(\under M)$.  There is an adjunction $\GTerms\adj U$.
\end{Prop}

\begin{Proof}
For any object $\PSet(X)$, we let $\GTerms:\PSet(X)\mapsto \GTerms(X)$
-- the free multialgebra over $X$. Its underlying set $\under
{\GTerms(X)}$ is defined inductively as:

\begin{itemize}\MyLPar
\item 
$X\cup\GTerms\subseteq \under{\GTerms(X)}$ 
\item 
if $f\in\Sigma$ and $x_1...x_n\in\under{\GTerms(X)}$ then
$f(x_1...x_n)\in\under{\GTerms(X)}$,
\end{itemize}

and operations are $f^{\GTerms(X)}(x_1...x_n)\Def \{f(x_1...x_n)\}$.
A $\PSt$-morphism $\under\phi:\PSet(X)\into\PSet(Y)$, i.e.\ 
$X\into\PSet(Y)$ is mapped obviously on the morphism
$\phi:\GTerms(X)\into\GTerms(Y)$ given by $\phi(x)=\under\phi(x)$ for
all $x\in X$ and $\phi(f^{\GTerms(X)}(x_1...x_n))=
\bigcup_{y_{i}\in\phi(x_{i})}f^{\GTerms(Y)}(y_1...y_n)$.

Let $M$ be another $\Sigma$ multialgebra, and $\under\phi:X\into U(M)$
a morphism in $\PSt$. Then we define tight homomorphism $\phi
:\GTerms(X)\into M$ by $\phi(x)\Def\under\phi(x)$, for $x\in X$,
$\phi(\{t\})\Def t^M$ for $t\in\GTerms$, and
$\phi(f^{\GTerms(X)}(x_1...x_n))\Def f^M(\phi(x_1)...\phi(x_n))$.

The unit $\iota:\PSet(X)\into U(\GTerms(X))$ is simple
inclusion. Obviously, we have that $\under\phi =
\Comp{\iota}{U(\phi)}$. Also, $\phi$ is unique, for if there is
another $\psi:\GTerms(X)\into M$, satisfying this last equality, then
we must have $\psi(x)=\phi(x)$ for all $x\in X$. Furthermore,
$\psi(t)=t^M$ for all $t\in\GTerms$, and
$\psi(f^{\GTerms(X)}(x_1...x_n))=f^M(\psi(x_1)...\psi(x_n))$ since
$\psi$ is a tight homomorphism. But this means $\psi=\phi$.
\end{Proof}

\noindent
Notice that this result (especially, uniqueness of $\phi$) depends
crucially on the tightness of homomorphisms. The construction does not
work for $\MAPSl$ or $\MAPSc$.

The functor $\GTerms$ will send a \PSt\ morphism
$\under\phi:X\into\PSet(Y)$ onto its homomorphic extension in
$\MAPSt$, $\phi:\GTerms(X)\into \GTerms(Y)$. $\GTerms$ is obviously
faithful (but not full).\footnote{It isn't full, for take for example
$\Sigma=\<\{S,T\},\{c:\into S, f:S\into T\}\>$, and $X_S=\{x\},\
Y_S=\{y\}$. There are two \PSt\ morphisms $X_S\into Y_S$ sending $x$,
respectively, on $\{y\}$ or $\es$.  But there is an additional
morphism $\GTerms(X_S)\into \GTerms(Y_S)$ sending $x$ onto $\{c\}$.

Similarly, $U$ isn't full, and so $\MAPSt$ isn't (equivalent to) a
reflective subcategory of \PSt.}  As a matter of fact, almost the same
construction would give us an adjunction between $\Set$ and
$\MAPSt$.~\footnote{A specific subcategory of $\MAPSt$ (where all
ground terms are determinisitc in each algebra) was used in
\cite{eli:nondetaut} for studying nondeterministic automata and
recognizable sets. The initiality result for this subcategory was also
mentioned there.}

Unlike for $\MASl$ where an initial object was the empty algebra, here
we obtain quite a natural initial algebra. Initiality of this term
algebra depends heavily on the fact that we have \PS\ homomorphisms --
they allow us to send a single element $t$ of term algebra to a {\em
set} of elements $t^M$ in the target algebra $M$. In particular, such
a $t$ can be sent onto empty set in cases when $t^M=\es$.

\begin{Fact}
Initial objects:
\begin{enumerate}\MyLPar
\item
$\MAPSt$ has initial objects.
\item 
Empty algebra is initial in $\MAPSl$.
\item
$\MAPSc$ does not have initial objects.
\end{enumerate}
\end{Fact}

\begin{Proof}
1) is a corollary of proposition~\ref{le:MAPStADJ} and 2) is obvious.  For
$\MAPSc$ consider a signature with one constant symbol $c$, and
suppose that $I$ is initial. Then $c^I\not=\es$ because there will be
algebras where $c$ is non-empty and we must have a closed homomorphism
from $I$ to such algebras. Let $A$ have the (underlying) carrier
$\{1,2,3,4\}$ and $c^A=\{1,2\}$. No matter what $c^I$ is, there will
be more than one closed \PS-homomorphisms to $A$. (If
$c^I=\{\bullet\}$ then there will be four.) Thus there is no initial
algebra here.
\end{Proof}

\begin{Theorem}
\label{th:MAPStcmpl}
$\MAPSt$ is complete.
\end{Theorem}

\begin{Proof}
We use constructions analogous to the ones from
fact~\ref{le:PStcompl}.

1) It has equalizers. For $\psi_1,\psi_2:M\into N$, let $\under E\Def
\{X\in\PSet(\under M)\land\psi_1(X)=\psi_2(X)\}$. Interpretation of
the operations in $E$ is then given in the obvious way: for all
constants $c\in\Sigma: c^E\Def \{c^M\}$, and $f^E(\{X_1\}...\{X_n\})
\Def \{f^M(X_1...X_n)\}$.  This is well-defined because, for all
ground terms $t$, we must have $\psi_1(t^M)=\psi_2(t^M)$ -- since
$\psi$'s are tight -- and hence $\{t^M\}\in\under E$.  Furthermore, if
$\{X_i\}\in\under E$ (i.e.\ $\psi_1(X_i)=\psi_2(X_i)\}$) then, for all
$f\in\Sigma$ we have
$\psi_1(f^M(X_i))=f^N(\psi_1(X_i))=f^N(\psi_2(X_i))=\psi_2(f^M(X_i))$,
i.e.\ $\{f^M(X_i)\}\in \under E$.

Define $\phi:E\into M$ by $\phi(\{X\})\Def X$. Obviously,
$\Comp\phi{\psi_1}=\Comp\phi{\psi_2}$. Whenever $\eta:A\into M$
satisfies the same (last) equality, we can define a unique
$\uni:A\into E$ making $\eta=\Comp\uni\phi$ by
$\uni(a)\Def\{\eta(a)\}$.

2) It has products. Given $N_i$, we construct $N$ by taking $\under
N\Def \bigdcup_i \under N_i$ (disjoint union), and defining operations
by: for all constants $c:c^N\Def \bigdcup_i c^{N_i}$, and for other
$f\in\Sigma:$
\[ 
f^N(x_1,...,x_n)\Def 
\left \{ \begin{array}{cl}
         f^{N_i}(x_1,...,x_n) 
         & {\rm if\ all\ } x_1...x_n\in N_i {\rm \ for\ some\ }i \\ 
         \es 
         & {\rm otherwise} 
         \end{array} 
      \right.
\]
Note that this makes 
\eq{
\label{eq:tight} 
f^N({\textstyle \bigdcup_i} X^1_i,...,{\textstyle \bigdcup_i} X^n_i) =
{\textstyle \bigdcup_i} f^{N_i}(X^1_i,...,X^n_i) 
}
The projections $\pi_i:N\into N_i$ are given by 
\[
\pi_i(n)\Def \left \{ \begin{array}{cl} \{n\} & {\rm if\ } n\in N_i \\
\emptyset & {\rm otherwise} \end{array} \right.
\]
Each $\pi_i$ is a tight homomorphism: (a) $\pi_i(c^N)=c^{N_i}$, (b) if
$x_1...x_n\in N_i$ then
$\pi_i(f^N(x_1...x_n))=f^{N_i}(\pi_i(x_1)...\pi_i(x_n))$, and (c) if
some $x\not\in N_i$ then $\pi_i(x)=\es$ so
$\pi_i(f^N(..,x,..))=\pi_i(\es)=\es= f^{N_i}(..,\es,..)=
f^{N_i}(..,\pi_i(x),..)$ (since operations are strict!)

% For each $i$, define an inverse mapping (embedding) $\_^{-i}:
% \under N_i\into \PSet(\under N)$ by $x^{-i} \Def \{x\}$, i.e.\ 
% $\{x:\pi_i(x)=\{x\}\}$, with the obvious extension to $\PSet(\under
% N_i)$. 

For any object $A$ and arrows $\psi_i:A\into N_i$, the morphism
$\uni:A\into N$ given by $\uni(a) \Def \bigdcup_i(\psi_i(a))$ %^{-i}$
obviously makes $\psi_i=\Comp\uni{\pi_i}$.  It is a tight
homomorphism: for any $a_1...a_n\in\under A$, tightness of all
$\psi_i$ gives us the second equation \re{eq:bul}, and \re{eq:tight}
gives the third one \re{eq:star} of the following equalities:

\begin{eqnarray}
\uni(f^A(a_1,...,a_n)) 
& = &  
{\textstyle{\bigdcup_i}}(\psi_i(f^{A}(a_1,...,a_n))) \nonumber 
\\ \label{eq:bul} 
& = &  
{\textstyle{\bigdcup_i}} (f^{A_i}(\psi_i(a_1),...,\psi_i(a_n))) 
\\ \label{eq:star}
& = &  
f^N({\textstyle{\bigdcup_i}}\psi_i(a_1),
  ...,{\textstyle{\bigdcup_i}}\psi_i(a_n)) \\ %[.5ex] 
& = & 
f^N(\uni(a_1),...,\uni(a_n)) \nonumber
\end{eqnarray}

%% \[ \begin{array}{r@{\ \ =\ \ }l}
%% \uni(f^A(a_1,...,a_n)) 
%% &   \bigdcup_i (\psi_i(f^{A}(a_1,...,a_n))) \\[.5ex] 
%% (\bullet) &   \bigdcup_i (f^{A_i}(\psi_i(a_1),...,\psi_i(a_n))) \\[.5ex] 
%% (*) &   f^N(\bigdcup_i\psi_i(a_1),...,\bigdcup_i\psi_i(a_n)) \\[.5ex] 
%%  & f^N(\uni(a_1),...,\uni(a_n)) 
%% \end{array} \]

Uniqueness of $\uni$ follows from the fact that if $\rho:A\into N$ is
such that for some $a: \rho(a)\not=\bigdcup_i(\psi_i(a))$, then there
is an $i$ such that $\Comp\rho{\pi_i}(a)\not =\psi_i(a)$. Thus
$(N,\pi_i)$ is a product of $N_i$.

3) $\MAPSt$ has also terminal object -- the empty algebra -- and so it
has all limits.
\end{Proof}

\begin{Fact}
$\MAPSl$ and $\MAPSc$ have products.
\end{Fact}

\begin{Proof}
The same construction as in the proof of theorem~\ref{th:MAPStcmpl},
point (2).  For $\MAPSl$, the equality \re{eq:bul} in the last chain
of equations will be replaced by $\subseteq$, and for $\MAPSc$ by
$\supseteq$.
\end{Proof}

\begin{Fact}
$\MAPSl$ has equalizers.
\end{Fact}

\begin{Proof}
The construction is similar to that in the proof of
theorem~\ref{th:MAPStcmpl}.(1), except for the definition of the
algebra on $\under E$.  For $\psi_1,\psi_2:M\into N$, let $\under
E\Def \{X\in\PSet(\under M)\land\psi_1(X)=\psi_2(X)\}$.  Given a set
$S\subseteq\under M$, let $sup^E(S)\Def \{X\}\in\under E : X\subseteq
S\land (\forall \{Y\}\in E: Y\subseteq S\impl Y\subseteq
X)$.~\footnote{This is well defined, for the ordering on $\under E$
defined by $\{X\}\leq\{Y\}\iff X\subseteq Y$ is an upper semilattice
-- if for all $i: \{X_i\}\in\under E$ then $\{\bigcup_i X_i\}\in\under
E$.  If, in addition, $X_i\subseteq S$ for all $i$, then also
$\bigcup_i X_i\subseteq S$.}  Interpretation of the operations in $E$
is then defined by: for all constants $c\in\Sigma: c^E\Def
sup^E(c^M)$, and $f^E(\{X_1\}...\{X_n\}) \Def sup^E(f^M(X_1...X_n)).$

% This is well-defined because, for all ground terms $t$, we must have 
% $\psi_1(t^M)=\psi_2(t^M)$ -- since $\psi$'s are tight -- and hence
% $\{t^M\}\in\under E$.  Furthermore, if $\{X_i\}\in\under E$ (i.e.\ 
% $\psi_1(X_i)=\psi_2(X_i)\}$) then, for all $f\in\Sigma$ we have 
% $\psi_1(f^M(X_i))=f^N(\psi_1(X_i))=f^N(\psi_2(X_i))=\psi_2(f^M(X_i))$, 
% i.e.\ $\{f^M(X_i)\}\in \under E$.

Let $\phi:E\into M$ be defined by $\phi(\{X\})\Def X$. It is a weak
\PS-homomorphism: $\phi(f^E(\{X_1\}...\{X_n\})) =
\phi(sup^E(f^M(X_1...X_n))) \subseteq f^M(X_1...X_n)$, where the last
inclusion follows from the definition of $sup^E$ (and $\phi$).  And
obviously $\Comp\phi{\psi_1}=\Comp\phi{\psi_2}$.  Whenever
$\eta:A\into M$ satisfies the same (last) equality, then for any
$a\in\under A$ we must have $\{\eta(a)\}\in\under E$. Thus we can
define a unique $\uni:A\into E$ making $\eta=\Comp\uni\phi$ by
$\uni(a)\Def\{\eta(a)\}$.
\end{Proof}

\noindent
Finally, we have the following negative results about completeness of
$\MAPSc$ and co-completeness of all three categories $\MAPS$.

\begin{Fact}
$\MAPSc$ does not have equalizers.
\end{Fact}

\begin{Proof}
Let $\Sigma$ have one constant $c$, $\under M=c^M=\{\bullet\}$,
$\under N=\{1,2\}$ and $c^N=\{1\}$, and $\psi_i:M\into N$ be given by
$\psi_1(\bullet)=\{1\}$ and $\psi_2(\bullet)=\{1,2\}$. There is no
algebra $E$ with a closed \PS-homomorphism $\phi:E\into M$ which would
make $\Comp\phi{\psi_1}=\Comp\phi{\psi_2}$, so there is no equalizer
of this pair of arrows in $\MAPSc$.
\end{Proof}

\begin{Corollary}
$\MAPS$ do not have epi-mono factorisation.
\end{Corollary}

\begin{Proof}
The negative result of lemma~\ref{le:noepi-mono} carries over to all
three categories $\MAPS$.
\end{Proof} 

% \michal{ Does it mean that
% there is no factorization system? In any case, such a system won't
% factor morphisms into epis and monos.}

\begin{Corollary}
$\MAPS$ do not have co-equalizers.
\end{Corollary}

\begin{Proof}
The negative result of lemma~\ref{le:PSnocoeq} carries over to all
three categories $\MAPS$.
\end{Proof}

\begin{Fact}
$\MAPSt$ and $\MAPSc$ do not have co-products.
\end{Fact}

\begin{Proof}
Let $\Sigma$ have only one constant $c$, and consider two
$\Sigma$-algebras, $M_1,M_2$ with the (underlying) carriers
interpreting $c$, resp., $c^{M_{1}}=\{a_1,b_1\}$ and
$c^{M_{2}}=\{a_2,b_2\}$. Consider first $\MAPSt$.

Let $M$ be an isomorphic algebra with $c^M=\{a,b\}$ and
$\phi_i(a_i)\Def\{a\}, \phi_i(b_i)\Def\{b\}$ be two homomorphisms from
$M_i$ to $M$.  If $C$ is a co-product $M_1+M_2$, then the respective
injections must coincide, i.e.\ \eq{ \label{eq:coinc}
\iota_1(a_1)=\iota_2(a_2)\ \ \ \ {\rm and}\ \ \ \
\iota_1(b_1)=\iota_2(b_2) } For if not, then there is an $x\in c^C$
such that $x\not\in\iota_1(a_1)$ and $x\in\iota_2(a_2)$. But then we
must have $x\in\iota_1(b_1)$ since $\iota_1$ is tight. Also
$x\not\in\iota_2(b_2)$, for otherwise
$x\in\iota_2(a_2)\cap\iota_2(b_2)$ and so, for any $\uni:C\into M$,
$\uni(\iota_2(a_2))\cap\uni(\iota_2(b_2))\not=\es$ -- then, since
$\phi_2(a_2)\cap\phi_2(b_2)=\es$, there would be no $\uni$ making
$\Comp{\iota_2}\uni=\phi_2$. So, the set $c^C$ must be divided into
two disjoint parts $c^C= c_a\cup c_b$ with
$c_a=\iota_1(a_1)=\iota_2(a_2)$ and $c_b=\iota(b_1)=\iota_2(b_2)$.

Take now $N$ with $c^N=\{x_1,y_1,x_2,y_2\}$ and two homomorphisms
$\psi_i:M_i\into N$ defined by:
%
\[
\begin{array}{r@{\ \Def\ }l@{\ \ \ \ \ \ \ \ \ \ \ \ }r@{\ \Def\ }l}
\multicolumn{2}{l}{\ \ \psi_1:M_1\into N} & \multicolumn{2}{l}{\psi_2:M_2\into N} \\[.5ex]
\psi_1(a_1) & \{x_1,x_2\} & \psi_2(a_2) & \{x_1,y_1\} \\
\psi_1(b_1) & \{y_1,y_2\} & \psi_2(b_2) & \{x_2,y_2\}
\end{array}
\]
%
If $C$ is a co-product we must have a unique $\uni:C\into N$ making
$\Comp {\iota_i}\uni=\psi_i$. In particular,
$\uni(\iota_1(a_1))=\{x_1,x_2\} \not= \{x_1,y_1\} =
\uni(\iota_2(a_2))$. But this would require
$\iota_1(a_1)\not=\iota_2(a_2)$ which contradicts \re{eq:coinc}.
% Hence there can be no $C$ satisfying the co-product definition.

The same counter-example shows the lemma for $\MAPSc$.
\end{Proof}

Notice that this proof assumes presence of some constants in the
signature. If there are no constants, the same proof as in
lemma~\ref{le:MASlcopr} can be used to show existence of
co-products. In general, however, we only have

\begin{Fact}
$\MAPSl$ has co-products.
\end{Fact}

\begin{Proof}
The same construction (and proof) as in lemma~\ref{le:MASlcopr}.
\end{Proof}

\noindent
The following characterization is a corollary of \ref{le:charepimono}
and \ref{le:MAPStADJ}. Functor $U$ is as in proposition~\ref{le:MAPStADJ}.

\begin{Prop}
Let $\phi:M\into N$ be in $\MAPSt$ and $\under\phi=U(\phi)$ be its
image in $\PSt$.
\begin{enumerate}\MyLPar
\item $\phi$ is mono iff $\under\phi$ is mono,
\item $\phi$ is epi if $\under\phi$ is epi.
\end{enumerate}

% (I.e., iff the pointwise
% extension of the mapping $\under\phi:\under M\into
% \PSet(\under N)$ to $\PSet(\under M)$ is injective.)
\end{Prop}

\begin{Proof}
$\phi$ mono $\impl\under\phi$ mono follows from
proposition~\ref{le:MAPStADJ} and \ref{le:charepimono}, since the right
adjoint $U$ preserves monomorphisms.

%% \michal{$U$ preserves limits, hence it preserves pullbacks, hence it
%% preserves monos [prop.~2.9.3 in Borceux]}

The two other implications ($\under\phi$ mono $\impl \phi$ mono; and
$\under\phi$ epi $\impl \phi$ epi) follow since
$U$ is faithful and hence reflects monos and epis.
\end{Proof}
Combined with the characterization from lemma~\ref{le:charepimono},
this means that monos in $\MAPSt$ are exactly the morphisms which
remain injective when extended pointwise to the power set of the
source, while surjectivity on the underlying set of the target implies
that a morphism is epi.
The ``only if'' in 2) does not hold but, to our surprise, we were not able
to find a necessary condition for epis in $\MAPSt$.


\subsubsection*{Thus ...}
we can summarize the properties of the three \PS-categories:

\begin{table}[htb] 
\begin{center}
\begin{tabular}{|r|c|c|c|}
\hline
    & $\MAPSl$ & $\MAPSt$ & $\MAPSc$ \\ \hline
products       &+ & +   & + \\
equalizers     &+ & +   & -- \\
{\bf completeness} &+ & +   & -- \\ \hline
initial objects  &+  & +  & -- \\
co-products    &+ & --  & -- \\
co-equalizers  &-- & --  & -- \\
{\bf co-completeness}&-- & -- & -- \\ \hline
epi-mono factorization   &-- & --   & -- \\ \hline
\end{tabular}
\caption{Properties of $\MAPS$.}\label{tab:B}
\end{center}
\end{table}


%========================================================================
%========================================================================

\section{Partial Algebras}
\label{se:pa}

Any partial function $f : X \into Y$ can be modelled by a
multifunction $\mof f$ returning an appropriate singleton set whenever
$f$ is defined, and the empty set
otherwise. Given a partial algebra $A$, let $\mof A$ be its
multialgebraic representation with functions given below ($\dom(f^A)$ is
a domain on which $f^A$ is defined):  
\eq{
\begin{array}{rcl}
\lis a\in\dom(f^A) & \Iff & f^{\mof A}(\lis a)\not=\es \\
\lis a\in\dom(f^A) & \Iff & f^{\mof A}(\lis a) = \{f^A(\lis a)\}
\end{array}
\label{mof}}


\subsection{Partial algebras -- basics}

We list some basic definitions and facts concering partial algebras.

%% \begin{Fact}{\bf [Generally]}
%% For any partial maps $f:A\into B$, $g:B\into C$, $f^-(Y)=\{a\in
%%  A:f(a)\in Y\}$ we have
%% \begin{enumerate}
%% \item
%% $\dom(f)= f^-(B)$ and $\dom(\Comp fg) = f^-(\dom(g)) = f^-(g^-(C))$.
%% \item
%% $\forall Y\subseteq B: f(f^-(Y))\subseteq Y$ 
%% ($\not=$ if $Y\setminus f[A]\not=\es$) \\
%% $\forall X\subseteq A: f^-(f(X))\supseteq X$ 
%% ($\not=$ if $\exists z\not\in X:f(z)\in f(X)$)
%% \end{enumerate}
%% \end{Fact}

\begin{Definition}\label{de:PAs}
A homomorphism between partial algebras $A$ and $B$ is a mapping 
$\phi:\Card A\into\Card B$ satisfying \ (*) $\forall a\in \dom(f^A):
\phi(f^A(a))= f^B(\phi(a))$ and

\begin{enumerate} \MyLPar
\item\label{it:paw}
{\em weak :} $\Comp{f^A}{\phi}\klee\Comp{\phi}{f^B}$ \ \ \ \
i.e.\ $\phi(\dom(f^A)) \subseteq \dom(f^B)$
\item\label{it:pat}
{\em tight\footnote{In literature on partial algebras this kind of a
homomorphism is called {\em closed}.} :}  
$\Comp{f^A}{\phi}=\Comp{\phi}{f^B}$ \ \ \ \ 
i.e.\  $\dom(f^A)= \phi^{-}(\dom(f^B))$
\item 
{\em full :} $\forall a_1...a_n\in\Card A: f^B(\phi(a_1)...\phi(a_n))\in \phi[A]
\impl \exists a'_1...a'_n\in \dom(f^A):\phi(a'_i)=\phi(a_i)$
%  $\phi(\dom(f^A)) = \dom(f^B)$
\item 
{\em full-injective \footnote{This
homomorphism is usually called {\em  relative-injective}. The
definition is equivalent to `full {\em and} injective'.} :}
\begin{enumerate}\MyLPar
 \item injective, and
 \item $f^A= \Comp{\Comp{\phi}{f^B}}{\phi^{-}}$
 \end{enumerate} 
\end{enumerate} 
\end{Definition}
%
\begin{Fact}
$\phi$ is iso $\iff$ it is bijective and closed
$\iff$ it is surjective and relative-injective!
\end{Fact}
%
\begin{Definition}
$A$ is a subalgebra of $B$ iff :\ 
  (*) $A\subseteq B$ and $\forall a\in \dom(f^A): f^A(a)= f^B(a)$ and
\begin{enumerate}\MyLPar
\item\label{weak} {\em weak}, $A\suba_w B$, iff : 
   $\dom(f^A) \subseteq \dom(f^B)$;
\item\label{rel} {\em relative}, $A\suba_r B$, iff : 
   $\dom(f^A) = A \cap (f^B)^{-}(A)$;
\item\label{tight} {\em tight}, $A\suba_t B$, iff : 
   $\dom(f^A) = A \cap \dom(f^B)$.
\end{enumerate}
\end{Definition}
%
\begin{Fact}
$A\suba_t B \impl A\suba_r B \impl A\suba_w B$.
%$\ref{tight} \impl \ref{rel} \impl \ref{weak}$.
\end{Fact}
%
\begin{Fact}
$A\suba_r B \iff f^A =\ \Comp{\Comp{\subseteq}{f^B}}{\subseteq^-}$, i.e.\ iff
$\subseteq$ is a full (injective) hom.
\end{Fact}
%
\begin{Fact}
$\phi:A\into B$ in $\PAw$ is 
\begin{itemize}\MyLPar
\item mono iff it is injective;
\item epi iff the closure of its image $\phi[A]$ in $B$ is the whole
$B: {\cal C}_B(\phi[A])=B$.
\end{itemize}
\end{Fact}


%-----------------------------------------------------------------------

\subsection{Some relationships}

We let $\PAS$ denote a category of partial algebras (with some
homomorphisms). To indicate a specific category, we use subscript --
$\PAw$ is the category with weak and $\PAt$ with tight homomorphisms.
$\mof \_:\PAS\into\MAS$ is the mapping sending an algebra $A\in\PAS$
to a multialgebra $\mof A$ with the underlying set $\under{\mof
A}=\Card A$ and all operations defined on sets by pointwise extension.

\begin{Fact}
\label{fa:faith} 
$\mof\_$ is a faithful functor:
\begin{enumerate}\MyLPar
\item $\PASw\into\MASl$ 
\item $\PASt\into\MASt$
\end{enumerate}
\end{Fact}

\begin{Proof}
Let $A,B$ be arbitrary algebras in $\PAS$ with a homomorphism
$\phi:A\into B$.  We use generally the fact that $ x\in\dom(f^A)
\iff f^{\mof A}(x)\not=\es$.  Homomorphisms are mapped directly
onto the (same) homomorphisms in $\MAS$.

1) $\phi$ is weak: $\phi(\dom(f^A))\subseteq\dom(f^B)$, i.e.\
$x\in\dom(f^A)\impl \phi(x)\in\dom(f^B)$.  Thus either a) there is an
$a$ such that $f^{A}(x) = a$, and so $\phi(a)=\phi(f^A(x))=
f^B(\phi(x))$, i.e.\ $\mof\phi(f^{\mof A}(x))=f^{\mof
B}(\mof\phi(x))$, or else b) $f^{\mof A}(x)=\es$ and
$\mof\phi(\es)=\es\subseteq f^{\mof B}(\mof\phi(x))$.

2) $\phi$ is tight: $\dom(f^A)=\phi^-(\dom(f^B))$. This gives
immediately $\phi(\dom(f^A))\subseteq\dom(f^B)$, i.e.\ $\mof\phi$ is
weak. But tight homomorphisms not only preserve but also reflect definedness,
i.e.\ $f^{\mof A}(x)=\es\iff f^{\mof B}(\mof\phi(x))=\es$ and so, also
in the case when $ x\not\in\dom(f^A)$ we now get that
$\mof\phi(f^{\mof A}(x))= f^{\mof B}(\mof\phi(x))$.

That $\mof\_$ is faithful is obvious since it is ``essentially
identity'' on the morphisms.
\end{Proof}

As a matter of fact, $\mof\_$ is essentially an inclusion since,
disregarding some obvious ``type conversion'' of elements into
1-element sets and adding the power set on the top of the underlying
set, it does not do anything interesting with the objects. Therefore
we won't write it explicitly unless really necessary.

The dual fact to ~\ref{fa:faith} is:

\begin{Fact}
\label{fa:full} 
Given full subcategories $\Cat{D}_w\subset\MASl$ and
$\Cat{D}_t\subset\MASt$ with deterministic multialgebras, then each
morphism
\begin{enumerate}\MyLPar
\item in $\Cat{D}_w$ is weak (according to def.~\ref{de:PAs}.\ref{it:paw});
\item in $\Cat{D}_t$ is tight (according to def.~\ref{de:PAs}.\ref{it:pat}).
\end{enumerate}
\end{Fact}
\begin{Proof}
Let $\phi:A\into B$ be an arbitrary morphism (in the appropriate
subcategory).

1) Since all result sets are either empty or have exactly 1 element,
the weakness condition means that $\phi(x)\not\in\dom(f^B)\iff
f^B(\phi(x))=\es \impl f^A(x)=\es \iff x\not\in\dom(f^A)$, which is
exactly the weak condition: $x\in\dom(f^A)\impl \phi(x)\in\dom(f^B)$.

2) Instead of implication as above we have equivalence here which
defines tight homomorphisms.
\end{Proof}
%
This means that $\mof{\PASw}=\Cat{D}_w$ and
$\mof{\PASt}=\Cat{D}_t$, and so we obtain as a corollary:

\begin{Prop}\label{co:fufa}
The functor $\mof\_$ from fact~\ref{fa:faith} if full and faithful.
\end{Prop}
%
Thus $\mof\_$ reflects limits and co-limits and many properties of the
two categories of partial algebras follow now by revisiting the proofs
of the respective properties in multialgebras and checking whether the
constructions, when applied to the deterministic multialgebras, yield
deterministic objects.  One such consequence is, for instance, that
any set of subobjects of any algebra in $\PASw$, resp. $\PASt$, has
intersection, i.e.\ in particular that intersection of a set of tight
subalgebras is a tight subalgebra.

%----------------------------------------------------------------------

\subsection{Full and fully-tight homomorphisms.}\label{sub:fjt}

Another consequence which can be obtained from proposition~\ref{co:fufa}
are some factorization systems for $\PASw$, resp. $\PASt$, which
specialize the more general result for $\MASl$, resp. $\MASt$. To see
this we have to consider the more special notions of homomorphisms.

The definition~\ref{de:PAs} of full homomorphisms is a special case of the
relational definition and so corresponds to the multialgebraic one
\ref{de:jtight}.  Thus, there is also a one-to-one correspondence
between the full homomorphisms which also are surjective and the full(y-tight)
epis, i.e.\ $\phi$ is full-surjective iff $\mof\phi$ is full(y-tight)
epi.  This fact is interesting because full-surjective homomorphisms do
compose, and so do fully-tight epis.
%Recalling that equation (\ref{eq:fts}) from remark~\ref{re:useful} 
%gave compositional homomorphisms, w

We may specialize the compositional definition $\phi^-;R^A;\phi=R^B$
for partial algebras to the definition which falls again between full
and full-surjective, while it is compositional and yields the left
factor of a factorization system: 
\eq
{ \forall \lis b,b:\ f^B(\lis b) =b\ \Iff\ [\exists \lis a,a:
\phi(a)=b \land \phi(a_i)=b_i \land f^A(\lis a)=a\ ]
%% { \forall b,b':\ \ f^B(b) =b'\
%% \iff\ [\exists a,a': \phi(a)=b \land \phi(a')=b'
%%   \land f^A(a)=a'\ ]
\label{eq:ftsPA}
}
Closure under composition and correspondences of various homomorphisms are
illustrated in table \ref{ta:homs}.

\begin{table}[hbt]
\begin{center}
\begin{tabular}{cr|c|lc}
$\Comp{\_\ }\_$ & $\phi\in\PASw$\ \ \  & 
 & \ \ \ $\mof{\phi}\in\Cat{D}_w\subseteq\MASl$ & $\Comp{\_\ }\_$ \\ \hline
+ & weak & $\iff$ & weak & + \\
+ & tight & $\iff$  & tight & +\\
-- & full & $\iff$ & full  & -- \\
+ & (\ref{eq:ftsPA}) & $\iff$ & fully-tight & + \\
+ & full-surjective & $\iff$ & full(y-tight) epi  & +\\
+& full-injective & $\iff$ & full  mono &  + \\
& (relative injective) 
\end{tabular}
\end{center}
\caption{Homs in $\PASw$ and $\Cat{D}_w=\mof{\PASw}$.}\label{ta:homs}
\end{table}

%% In general, the opposite implication in \refp{eq:fulljt} will hold
%% if $\phi$ is surjective.  

\noindent
1) Full(y-tight) epis constitute left factor of a factorization system
in $\MASl$, the right factor being monos. Monos in $\MASl$ are exactly
the injective morphisms and these are also monos in $\PASw$. Thus, as
simple consequence of the theorem about factorization system in
$\MASl$,we obtain the well-know result, namely, that $\<\epi,\mono\>$
is a factorization system for $\PASw$, where $\epi$ are injective and
$\mono$ are full-surjective, or (\ref{eq:ftsPA}), homomorphisms.  We
have given an ``internal'' construction of the quotient
(multi)algebras as images of fully-tight epis under arbitrary
equivalence. In fact, this construction, when performed within
$\Cat{D}_w$, resp.  $\Cat{D}_t$ coincides with the respective quotient
construction for partial algebras in $\PAw$, resp. $\PAt$.  In the
case of $\PAw$, the congruence condition is modified to ``if $a \sim
b$ and {\em both} $a\in\dom(f^A)$ {\em and} $b\in\dom(f^A)$,
then $f^A(a)\sim f^A(b)$''. In the quotient, $[a]\in\dom(f) \iff
\exists a'\in\Card A:a'\in\dom(f^A)$, which corresponds to the
condition of fully-tightness. 

\medskip

2) In $\PAt$, one uses the notion of a {\em tight} congruence which
has to satisfy a stronger condition that for any congruent (sequence
of) elements $a\sim b$ we have $a\in \dom(f^A)\iff
b\in\dom(f^A)$. This notion corresponds then exactly to the tight
congruences.  Thus, as another corollary we obtain the factorization
system for $\PAt$ where $\epi$ are just surjective homomorphisms
(since tightness implies full-tightness) while $\mono$ are just
injective ones. (Notice that this sytem in $\MASt$ comprises all epis
while in $\PASt$ only a subset of epis.) \vspace*{1ex}

There are other, well-known factorization systems for $\PASw$ which
cannot be obtained from $\MASl$. For instance, a system where $\epi$
are epimorphisms and $\mono$ are tight-injective is a factorization
system for $\PASw$. But since epis in this category have much more
specific characterization than in $\MASl$ this is not a system for
$\MASl$.  Similarly, a system with surjective homomorphisms as $\epi$ and
full-injective as $\mono$ is a factorization system for $\PASw$. We do
not have an exact correspondance in table~\ref{ta:homs} between
full-injective and fully-tight monos and the respective combination
does not yield a factorization system for $\MASl$.


%=====================================================================
%=====================================================================

\section{Some Concluding Observations}


According to theorem~\ref{th:comp} there are only 9 compositional
homomorphisms of relations: 1 weak (subset), 4 tight (equality) and 4
closed (superset).  A closer study of three of them (together with the
3 respective \PS-homomorphisms) shows that closed homomorphisms yield
categories with very poor structural properties and one
may expect that this to be the case for the remaining three closed
homomorphisms. Thus, we conjecture that, accepting compositionality and 
the categorical properties
we have studied among the adequate criteria for choice of the homomorphisms, 
 there are at most 5 ``useful''
possibilities. Among them, the 4 tight ones can be expected to have
significant weaknesses, exemplified by the one case, $\MASt$, which has been
given a detailed treatement here.  But even if one still does not want to
work with weak homomorphisms, the results of this paper
% the full characterization of compositional definitions which we have given 
can provide a useful tool
preventing one from looking for new, idiosyncratic notions serving
only very peculiar purposes.

The suggested definition of composition of relations of arbitrary
arities turns relational structures into algebras -- namely,
multialgebras.  These provide a convenient way for algebraic study of
power set structures, relational structures, as well as for modeling
phenomena like partiality and nondeterminism. The price for the
flexibility and generality of multialgebras is the loosened relation
between the algebraic structure and homomorphisms.

\begin{enumerate}
\item
Multialgebras do not give one a grasp on the individual elements --
ground terms may denote sets. This causes trouble in many situations,
in particular, when constructing unique homomorphisms. Two
homomorphisms may reflect the $\Sigma$-algebraic structure in the same
way but still be different.  E.g., algebras with carriers $\under
M=c^M=\{1,2\}=c^N=\under N$, even with tight homomorphisms, will have
``only one'' homomorphism in the sense that $\phi(c^M)=c^N$ but,
actually, there are two $\phi_1 = id$ and $\phi_2(1)=2,\ \phi_2(2)=1$.

\item
Many (counter-)examples are based on empty signatures or signatures
with only one constant.  It seems that some nice results follow only
if $\Sigma$ is non-void (like relevant initial objects in $\MAPSt$),
while others only if it is void.
\begin{enumerate}
\item
Weak homomorphisms seem to work best in combining these two poles --
$\MASl$ and $\MAPSl$ had most desirable properties. However, this
happens for the cost that many categorical constructions simply yield
degenerated cases -- empty algebras or algebras where (all) operations
return empty set.
\item
The mere look at the tables~\ref{tab:A}-\ref{tab:B} could suggest that
the weak categories $\MASl$, resp. $\MAPSl$, are most interesting. In
fact, the corresponding tight categories $\MASt$, resp. $\MAPSt$, lack
many of the desirable properties possessed by these weak ones.

However, consulting the actual proofs and constructions we observe
that, in the situations when some construction works in the tight
categories, it often yields a more natural -- less degenerate and
closer to the classical -- result than the corresponding construction
in the weak categories. Nonexistence of some constructions in the
tight categories is often a side-effect of excluding the degenerate
cases admissibile in the weak categories.
\end{enumerate}

\item
There is a certain sense of correspondence between the categories
$\MASl$ and $\MAPSt$.  The weak $\PP$-homomorphisms allow the target
to have ``larger result sets'' than the source. Similar effect is
achieved in $\PS$-homomorphisms in that a single element from the
source can be mapped on a set of elements in the target. It all
depends on what homomorphisms are supposed to be used for, but
assuming that this kind of ``reducing the result sets'' in the source
is intended, there seems to be little point in combining
\PS-homomorphisms (which do that) with the weakness requirement (which
does the same again).

In fact, $\MAPSt$ seems to be the most interesting among the
\PS-categories, in particular, due to the natural homomorphism
condition (involving set equality) and, furthermore, the existence of
interesting initial objects (not present in any other category we have
studied). However, it is not only easier to work with
\PP-homomorphisms -- the category $\MASl$ had, in addition to initial
objects, factorization system and all (finite) co-limits. These may be
strengths making it the most attractive of all the structures we have
reviewed.
\end{enumerate}


\bibliographystyle{bibNo}
\bibliography{my}


\end{document}