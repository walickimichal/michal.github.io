\documentstyle[amssymb,a4wide]{article}
%\show\max
\input defs
\input xypic

\begin{document}

\title{Refining}

\author{{\it Micha{\l} Walicki\( ^*\)}\\
\small Department of Computer Science\\
\small University of Bergen\\
\small Bergen, Norway\\
\footnotesize michal@ii.uib.no}

%\maketitle

\section{Refinement Notions}
Let's say that, at the abstract level, we can make (nondeterministically) two different
transitions $\trans cb$ and $\trans cd$. There are various notions as to how
such a system can be refined. Simulation, for instance, requires that, for each refinement
 $c'$ of $c$ \footnote{As far as possible, we will use the convention that
 the primed
symbols refer to the refinements of the unprimed ones} there exists refinements
 $b'$, $d'$ of $b$ and $d$ respectively, with the concrete transitions
 $\trans {c'}{b'}$ and $\trans {c'}{d'}$. This is a very strong notion
 because it forces each refinement of the source to reflect the
 nondeterminism of the abstract level. In particular, it excludes the
 possibility of refining the source to more deterministic components, e.g.,
 $c'$ and $c''$ such that only $\trans {c'}{b'}$ and $\trans {c''}{d'}$ but
 not $\trans {c'}{d'}$.

In the multialgebraic formulation, the two transitions from $c$ will
correspond to a set-valued function $f$ such that $f(c)=\{b,d\}$. We will use
this latter notation and, to emphasize the analogy, talk occassionally about
transition $f:\trans cd$, meaning $d\in f(c)$.
At the moment, we do not assume anything about a possible mapping $h$ from
the concrete to the abstract structure but will sometimes write $h(d')=d$ (or
 $d'\in h^{-1}(d)$ to make it explicit that $d'$ is a refinement of $d$.

Just playing with quantifiers (and guarding the universally quantified
variables in the antecedent and existential ones in the consequent), we
obtain six refinement notions which restrict the concrete relation $f'$
simulating the abstract $f$ in various ways.
\begin{DEFINITION}{}\label{de:ref}
 $A'$ is an {\em $n$-refinement} (for $n$=1...6) of $A$ iff:\\[1ex]
\hspace*{4em} $\begin{array}{|rrrcl|}
\hline
& \multicolumn{4}{l|}{\forall c\ \forall d : d\in f(c) \impl}\\
\hline
1. &\exists c'\ \exists d' :& & d'\in f'(c') & \land\ h(c')=c \land h(d')=d\\
2. & \forall d'\ \exists c' : & h(d')=d \impl & d'\in f'(c')
&  \land\ h(c')=c \\
3. & \exists c'\ \forall d' : & h(d')=d \impl & d'\in f'(c') &
 \land\ h(c')=c \\
\hline
4. & \forall c'\ \exists d' :& h(c')=c \impl & d' \in f'(c')
   &  \land\ h(d')=d\\
5. & \exists d'\ \forall c' : & h(c')=c \impl & d'\in f'(c')
   &  \land\ h(d')=d \\
6. & \forall c'\ \forall d' : &  h(c')=c \land h(d')=d \impl & d'\in f'(c') & \\
\hline
\end{array}$
\end{DEFINITION}
\noindent
The division between lines 3. and 4. corresponds to the distinction between
the refinements which allow more determinisitic refinements, in the sense
that a refinement $c'$ of source $c$ need not refine the full nondeterminism
of the abstract transitions from $c$ (1-3), and those which do not (4-6).
As is easily seen, it corresponds to the distinction between these
notions where the refinement $c'$ of the source  is quantified existentially
vs. universally.

Rephrasing the above formulae in natural language means: 
%
\begin{enumerate}\MyLPar
\item $\exists c' \into \exists d':$   {\em (At least) each transition is simulated.} \\
It was introduced in \cite{MeBroy} and in \cite{Wir} (as {\em transition
refinement}) exactly for the purpose of determinisation in the refinement
process: nondeterminism at the abstract level may be refined to several
deterministic agents acting on different representations of the same abstract
value.
\item $\forall d' \leftarrow \exists c':$  {\em Each $d'$ is reachable from (at least) some $c'$.} \\
This is like a backward simulation requiring  the possibility of a reverse
simulation from each refinement of a target to some refinement of the
source. Let us therefore call it {\em cosimulation}. 
\item $\exists c' \into \forall d':$ {\em Exists a $c'$ simulating (at
least) the whole $\trans cd$.} \\ 
This is a stronger version of 2. It requires that not only each
refinement of the target be reachable from some refinement of the source but,
in addition, that {\em all} refinements of a target be reachable from one
refinement of the source.
\item $\forall c' \into \exists d' :$ {\em Each $c'$ simulates (at least)   $\trans cd$.} \\
This is usual {\em simulation}.
\item $\exists d' \leftarrow \forall c' :$ {\em Exists a $d'$ reachable from
(at least) each $c'$.} \\
This means that all refinements $c'$ of the source lead to a common ``meeting point''
 $d'$ which, for each $c'$,  simulates the abstract transition. 
\item $\forall c' \into \forall d' :$ {\em Each $c'$ simulates (at
least) the whole $\trans cd$, and each $d'$ is reachable from (at least) each $c'$.} \\
It is the strongest (and, probably, least useful) notion. Arbitrary
refinement of the  source has a transition to arbitrary refinement of any
target. In a sense, picking any representatives among $c'$s and $d'$s gives a
copy of the abstract structure.
\end{enumerate}
\noindent
The ``at least'' phrases indicate that there may be other transitions at the
concrete level. Such redundancies dissapear when we require the mapping $h$
to be a homomorphism. Also, universal quantification over the abstract
transitions forces the concrete structure to display the full range of
nondeterminism present in the abstract one (modulo the fact that 1.-3. allow for
determinisation). Thinking in terms of transitions this may be plausible, but
viewing nondeterminism as a means of abstraction, one would like to admit
refinements which are more deterministic. This, too, will be possible by the
requirement of $h$ being a homomorphism.
%
\begin{SREMARK}{Remark.}
The conditions would be different if, instead of restricting the relation
 $f'$ we attempted to restrict $h$. This would amount to swapping $h(d')=d$ and
 $d'\in f'(c')$. In the cases of existentially quantified target $d'$ (1.,
 4. and 5.) the result would be the same, while
 for the other three cases we would obtain \\[1ex]
\hspace*{6em} $\begin{array}{|rrrcl|}
\hline
& \multicolumn{4}{l|}{\forall c\ \forall d : d\in f(c) \impl}\\
\hline
2h & \forall d'\ \exists c' : & d'\in f'(c') \impl & h(d')=d
&  \land\ h(c')=c \\
3h & \exists c'\ \forall d' : & d'\in f'(c') \impl & h(d')=d &
 \land\ h(c')=c \\
6h & \forall c'\ \forall d' : &  h(c')=c \land d'\in f'(c') \impl & h(d')=d & \\
\hline
\end{array}$
\\[1ex]
\noindent
3h ensures the existence of a $c'$ such that all its transitions $f':\trans {c'}{d'}$ 
simulate only the transition  $f:\trans cd$. The other two cases are really
degenerate. 2h claims that whenever a $d'$ is reachable from some $c'$ by a 
transition $f':\trans{c'}{d'}$, then this transition is actually
simulating the transition $f:\trans cd$. 6h means that all transitions
 $f':\trans {c'}{d'}$ from all refinements $c'$ simulate $f:\trans cd$, i.e., that actually
 there is only one such a transition at the abstract level. Although somebody
 may find these cases relevant, we do not see any
 point in considering them.
\end{SREMARK}
%
The obvious implications between the six notions give the lattice structure 
\begin{figure}[hbt]
\spcol{-1.4}\sprow{-1.2}
\hspace*{15em}\diagramcompileto{Lat}
 & 6. \forall c'\forall d' \dldouble|>\hole|>\tip \drdouble|>\hole|>\tip & \\
5. \exists d' \forall c' \ddouble|>\hole|>\tip  & & 3. \exists c' \forall d'\ddouble|>\hole|>\tip  \\
4. \forall c' \exists d' \drdouble|>\hole|>\tip  & & 2. \forall d' \exists c' \dldouble|>\hole|>\tip \\
 & 1. \exists c' \exists d'
\enddiagram 
\caption{Relative strength of refinements.}\label{fi:lat}
\end{figure}


\section{Homomorphisms Do Not Suffice}

\begin{DEFINITION}{}\label{de:homs}
A multihomomorphism $h:A\into B$ is a mapping $\carrier A \into \carrier B$
satisfying
\begin{itemize}\MyLPar
\item[{\em (H)}] $h(f^A(x))\subseteq f^B(h(x))$
\end{itemize}
\noindent
Other, more specific kinds, presuppose {\em (H)}:
\begin{itemize}\MyLPar
\item[{\em (T)}] $h(f^A(x))\supseteq f^B(h(x))$;
\item[{\em (JT)}] $h(f^A([x]))\supseteq f^B(h(x))$;
\item[{\em (S)}] $h^{-1}(f^B(h(x))) \subseteq h^{-1}(h(f^A(h^{-1}(h(x)))))$;
\item[{\em (V)}] $f^B(h(x)) \subseteq h(f^A(h^{-1}(h(x))))$.
\end{itemize}
\end{DEFINITION}
\noindent
All operations applied to sets refer to pointwise extension, i.e.,
$f([x])=\{f(y):y\in[x]\}$.

We have (T) $\impl$ (JT). Nipkow \cite{Nip} introduces also (S) and
(V) (with (T) $\impl$ (V) $\impl$ (S)) but they seem of minor importance in
the present context. 

Typically, one would require that equivalence is a congruence
wrt. deterministic operations. But doing semantics, we may turn it around and
say that deterministic operations are the ones for which (each allowed) equivalence is a congruence.
\begin{DEFINITION}{}\label{de:quot}
Let $A$ be a multialgebra and $\equiv$ an equivalence on $\carrier A$ ($[x]$ denotes
the equivalence class of $x$). The {\em quotient multialgebra}, $\quot
A\equiv$ is given by:
\begin{enumerate}\MyLPar
\item $\carrier {\quot A\equiv} = \{[x]:x\in \carrier A\}$;
\item $f^{\quot A\equiv}([x]) = [f^A([y])]$.
\end{enumerate}
\end{DEFINITION}
\noindent
As for operations, the notation is extended pointwise to sets of equivalence 
classes $[f([x])] = \{[d]:d\in f([x])\}$.
%
\begin{LEMMA}\label{le:qoutHomo}
 $h:A\into \quot A\equiv$ defined by $h(x)=[x]$ is a multihomomorphism.
\end{LEMMA}
\begin{PROOF}
Obvious: $d\in f^A(x) \impl [d]\in f^{\quot A\equiv}([x])$, i.e.,
 $h(f^A(x))\subseteq f^{\quot A\equiv}(h(x))$.
\end{PROOF}
%
\begin{LEMMA}\label{le:epi-mono}
Each multihomomorphism $g:A\into B$ has a unique epi-mono factorisation
 $\Comp hi$, with $h:A\twoheadrightarrow \quot A\equiv$, $i:\quot A\equiv \inj B$.
\end{LEMMA}
\begin{PROOF}
Let $\equiv$ be $ker(g) = \{\<a_1,a_2\>: g(a_1)=g(a_2)\}$, and let $h$ be the
canonical homomorphism $A\into \quot A\equiv$.  $i([x])\Def g(x)$ gives a
well-defined homomorphism $\quot A\equiv \into B$ with $\Comp hi=g$.
\end{PROOF}
\noindent
One would like to associate various refinement notions with various kinds of
homomorphisms and, preferably, some constructions yielding the abstract
structure from a concrete one. As it is easy to check, none of the notions
from definition \ref{de:homs} characterizes any of the refinement
notions. Furthermore, $A$ is a 1-refinement of $\quot A\equiv$,
 which, being the weakest notion, does not tell us which
kind of refinement we have obtained. In order to distiguish between them, we
look closer at the conditions which must be imposed on the equivalence
relation $\equiv$.

\section{Equivalences}
We assume that the abstract structure $B=\quot A\equiv$ is obtained from $A$
by the quotient construction (def. \ref{de:quot}). Thus we have that
 $[x]=h^{-1}(h(x))$,\footnote{We are here treating $[x]$ as a subset of
 $\carrier A$, rather then as an element of $\carrier {\quot A\equiv}$.}
 where $h$ is the cannonical homomorphism.
In order to ensure that $A$ is an $n$-refinement, the equivalence $\equiv$ must
 satisfy the following additional conditions:

\begin{figure}[hbt]
\hspace*{10em}$\begin{array}{r|l}
 & \forall c,d \in \carrier A \\
\hline
1. & ---  \\
2. & d \in f([c]) \impl [d] \subseteq f([c]) \\[.4ex]
3. & d \in f([c]) \impl \exists x\in [c] : [d]\subseteq f(x) \\[.4ex]
4. & d \in f([c]) \impl \forall x\in [c] : ([d]\cap f(x)) \Not= \es \\[.4ex]
5. & d \in f([c]) \impl  ([d] \cap {\displaystyle \bigcap_{x\in [c]}}f(x)) \Not= \es \\
6. & a)\ \forall x,y\in [c] : f(x) \Seteq f(y) \\[.2ex]
   & b)\ d \in f(c) \impl [d]\subseteq f(c)
\end{array}$
\caption{$n$-equivalences corresponding to the $n$-refinements from def. \ref{de:ref}.}\label{fi:equiv}
\end{figure}

\noindent
Naturally enough, we obtain the same implications for these equivalences as for the
refinements from figure \ref{fi:lat}, where the left branch ($\impl 5. \impl 4.$) 
 originates from 6a. and the right one ($\impl 3. \impl 2.$) from 6b.
%
\begin{CLAIM}
 $A$ is an $n$-refinement of $\quot A\equiv \ifff\ \equiv$ is an $n$-equivalence
 from figure \ref{fi:equiv}. 
\end{CLAIM}


\begin{thebibliography}{MeRef}\MyLPar
\bibitem[Nip86]{Nip} T.~Nipkow. Non-deterministic Data Types: Models and Implementations.
  {\em Acta Informatica}, 22 (1986).
\bibitem[WB94]{MeBroy}
\bibitem[Wir94]{Wir}

\end{thebibliography}

\end{document}
