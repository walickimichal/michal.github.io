\paragraph{Overview.}
This report describes the {\em Ada style checker}, which was designed and 
constructed in Winter and Spring 1989--90. The style checker is based on
the Stanford Anna Tools and has been annotated using
Anna \cite{luckham:anna-lrm}.
The style checker examines Ada~\cite{*:ada-lrm}
programs for ``correct style'' which 
is defined in a {\em style specification
language (SSL)}.  Hence, the Ada style checker takes as input an Ada program
and a set of style specifications and produces as output a list of
style violations in the Ada program.  An example of a style specification
is:

\begin{quote}\it
Use clauses may not be used
\end{quote}

\noindent
The output of the style checker for this example will be an error message
corresponding to each occurrence of a use clause in the input Ada program.
Figure~\ref{fig:black-box1} illustrates the operation of the Ada style
checker.

\begin{figure}[htbp]

\vspace{1.55in}

\special{psfile=intro1.ps}

\caption{The I/O specification of the style checker}
\label{fig:black-box1}

\end{figure}

\paragraph{Use of the style checker.}
Style checking of programs is a typical activity in software houses
where large volumes of software are written regularly.  Administrators
or managers in these software houses set out programming style
guidelines for their programmers.  Typically, these guidelines are
English documents.

We refer to the writer of the style guidelines as the {\em system
manager}, someone with a high-level understanding
of software development, but who does not necessarily write any
software themselves.

\paragraph{The style specification language.}
Since we are automating the process of style checking, the style
specifications have to be written out in some formal, machine
processable manner.  Also, given the qualifications of the system
manager, we require a high-level approach to writing style
specifications.  Our style specification language satisfies both these
requirements.  The style specification language provides the
capability of defining style guidelines using a predefined vocabulary
of {\em concepts}.  The style specification of use clauses shown
earlier can be written out in our style specification language as:

\begin{quote}\it
count(use\_clause) {\rm = 0}
\end{quote}

\noindent
or as:

\begin{quote}\it
no(use\_clause)
\end{quote}

Details on interpreting such specifications are provided later.

\paragraph{The style checker generator.}
We are faced with the often expensive task of processing the
high-level style specification into an internal form before an Ada
program can be compared to it.  Given the typical scenario in which
the style checker is used, we can assume that the style specifications
are set forth by the system manager once and for all, and that based
on them, many programs are then written and checked for style
violations.  The original style specifications are seldom changed.
Hence, our approach is to process the style specifications only when
they are changed and to generate a style checker based on these
specifications.  This generated style checker is then used to examine
Ada programs for style violations.  Given a fixed set of style
specifications, this process allows for a much faster style checker.
Figure~\ref{fig:black-box2} illustrates this approach.  A Style
Specification is first written and given as input to the SSL Parser and
then to the Style Checker Generator.  This produces the executable
Style Checker.  The Style Checker can now be used to check Ada programs
for style violations.

\begin{figure}[htbp]

\vspace{2in}

\special{psfile=intro2.ps}

\caption{The style checker generator}
\label{fig:black-box2}

\end{figure}

The Style Checker Generator converts the high-level style specification
into an Ada program with Anna constraints.  This generated Ada program
takes the
Ada program on which style guidelines need to be checked as input.  The
input program is parsed and semanticized, resulting in a
DIANA~\cite{evans:diana-lrm} {\it Abstract Syntax Tree} (AST) whose
nodes are decorated with semantic information.  The generated Ada
program then invokes a routine that traverses the complete DIANA AST.

The generated Anna constraints are annotations on this tree traversal
routine.  They specify what the tree traversal routine is to expect
while traversing the DIANA AST.  Anything unexpected corresponds to a
style guideline violation.

The final step in generating the style checker is to apply the Anna
Transformer~\cite{sankar:thesis} to the Anna/Ada program generated by
the Style Checker Generator.  This results in an Ada program where the
Anna constraints are transformed into Ada checks on the Ada program.
Violation of these checks cause an Anna constraint violation to be
reported.  Hence, when this transformed program is executed on an
input Ada program, Anna constraint violations occur every time a
style guideline violation is detected.  The Anna constraint violations
are handled in a manner appropriate for reporting style guideline
violations, hence the user of the style checker is unaware of the
Anna constraint violations.

The Anna constraints may refer to various parts of the Ada program by
using predefined {\em language objects} ({\em e.g.}, {\it use\_clause}).
Furthermore, calls to various predefined {\em operations} on these
language objects are allowed ({\em e.g.}, {\it count}, {\it no}). The
language objects and the operations are parts of the SSL {\em
vocabulary}.

The language objects and operations are represented in Ada as discrete
types and functions, which are contained in a {\em vocabulary
package}.  The Anna constraints thus make use of these types and
contain calls to the functions.

The style checker may be activated from the command line (using the
command line interface provided by the system) or from another Ada
program (using the programmatic interface). The user specifies the
name of an input Ada program to be checked when activating the style
checker.  When the input program has been completed checked, the list
of style guideline violations is returned to the calling routine.

\paragraph{Implementation status.}
The Style Checker Generator has not been implemented.  However, a
set of style specifications based on the SPC software
guide~\cite{SPC88} have been hand-translated to produce a style
checker for these specifications.  This version is called the
{\em Stanford Ada Style Checker}.  This style checker has been used
on many practical Ada programs.

\paragraph{Organization of this report.}

Chapter~\ref{ssl} is a description of the SSL.
Chapter~\ref{init-system} describes the overall structure of the system.
A user manual for the present version of the style checker can be found in
Chapter~\ref{usage} and \ref{prog-interf}.  Chapter~\ref{usage} describes
the command line interface for invoking the style checker, and
Chapter~\ref{prog-interf} describes the programmatic interface.
Chapter~\ref{future} concludes with suggestions for future work ---
the next steps in the development of components such as
the SSL compiler and the style checker generator.

Appendix~\ref{stanford} is a detailed description of the style guidelines
inserted in the current version of the style checker.
A maintenance guide can be found in Appendix~\ref{maint}.

\section{Acknowledgements}

We wish to thank Prof.~David C. Luckham for his inspiration and support.
We are also grateful to Steve Sherman and Barry Schiff of the Lockheed
Missiles and Space Corporation for introducing us to the idea of
a general framework for style checking, and for sharing their experiences
as software managers at Lockheed.  We are thankful to members of
Stanford University's Program Analysis and Verification Group, and
especially to Geoff Mendal for his assistance in adapting the Anna
tools for the style checker.
