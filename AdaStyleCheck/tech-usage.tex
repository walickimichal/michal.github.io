In this chapter, we will show how the style checker can be activated 
at the command line and provide an example. We assume,
that the style checker has been installed in the correct directory, 
search paths set up, etc.

The style checker is activated by running the program named 'style'.

In order to run, the user has to specify the name of the Ada file to
be style checked.

For instance, typing at the command line:

\begin{center} {\tt style test} \end{center}

will cause the style checker to check the compilation unit 'test.a', using the
default option values (see below).

However, the user is also allowed to provide options when activating
the style checker. The command line format is as follows:


\begin{center}
style \{-[c$|$e$|$v$|$s$|$l$|$n\#]\} $<$file name $>$
\end{center}

where:

\begin{itemize}
\item[] {\bf -c :} {\it Current compilation unit only, NOT with'ed packages (default
false).} If this option is not specified (false), all the packages with'ed in the
compilation unit will be checked before the current compilation unit. Note, that
this may cause many compilation units to be checked by the style checker, since the with'ed 
compilation units may also have with'ed packages themselves.
 {\bf This option is not 
implemented in the current version. In the version implemented by the date
of this publication, only the current
compilation unit will be checked}.
\item[] {\bf -e :} {\it Suppress explanation in output (default false).} 
If the option is not specified (false), the style violation will only
appear with a source position and a name, both on the screen and in the 
{\tt .vio}-file (see below). The explanation is currently hard-coded into the
program. However, the planned style checker generator is intended to include 
the option of allowing the system manager to specify the explanation himself.
\item[] {\bf -v :} {\it Verbose mode. Log to screen. (default false).} If this 
switch is specified (true), output will be sent to the standard output during the style
checking, {\em i.e.}, the violations that are detected will be sent to the output device. 
\item[] {\bf -s :} {\it Continue on semantic error (default is to abort style
checking when semantic error occurs).} Semantic checking
of the input program is performed before the style checking takes place. If a
semantic error occurs, the style checking will halt, since all semantic 
information needed to perform the style checking properly might not be present.
However, some semantic errors might not influence the style checking. A warning will be
displayed, if style checking continues despite a semantic error.
\item[] {\bf -l} {\it : Suppress .vio log file.} The style checker will
by default dump a list of violations detected into a file with the postfix `{\tt .vio}'. The user can suppress this by providing this option at the command line.
\item[] {\bf -n\#} {\it  : Stop at violation number \# (default 0).} If not specified, 
style checking will be performed until the end of the input file is reached. 
However, if the user for some reason would like the style checking to stop after
a certain number of style violations, this can be specified using this flag. 
The number 0 will cause the style checker to report all violations detected. 
\item[] {\bf $<$file name$>$ :} {\it File name with .a omitted.}
\end{itemize}

Activating the style checker without any parameters at all will produce a listing
of the command line format and the options allowed.

\section{Example}

Consider the following package stored in the the file `ex1.a':\\
\apebegin\bhinge
\Package Example1 \Is
\End Example1;

\Package \Body Example1 \Is
  x: Integer;
  b: Boolean;
\Begin
  b := b \And (x = 0);
\End Example1;
\apeend

Using the style rules, that 
have been included in the style checker so far (see Appendix~\ref{stanford}), 
activation of the style checker by the command {\tt style -v ex1}
yields the following output on the screen:

\begin{verbatim}
Stanford Ada Style Checker Version 1.20, Rev. 4/30/90

Source file: ex1.a
List   file: ex1.vio
Output file: ex1.sty

Loading parse table . . . . . . parse tables loaded
Parsing source file . . . . . . parsing complete. No syntax errors.
Checking semantics  . . . . . . semantic OK - output in .sem file
Checking source file for style guideline conformance . . .
*** ERROR: STYLE VIOLATION Line  1, Column  1:
Package Specifications And Bodies [SPC-88 5.6.2]
This compilation unit contains specification and body for the same package.
Separate them in different files

*** ERROR: STYLE VIOLATION Line  5, Column  6:
Types Integer, Natural and Positive [SPC-88 6.3.2]
Do not use predefined types (Integer, Natural and Positive)

*** ERROR: STYLE VIOLATION Line  8, Column  10:
Short Circuit Control Forms [SPC-88 4.2.1]
And not allowed. Use short-circuit forms instead (and then, etc.)

----------------------------------

 3 errors detected
 0 warnings detected
\end{verbatim}

The option '{\tt -v}' (verbose mode) specifies that we want output from
the style checker on the standard output while it is executing.

Two files have now been created: A '{\tt .vio}' file and a '{\tt .sty}'
file. The {\tt .vio}-file contains a dump of the violations, 
corresponding to the output shown above, and the {\tt .sty}-file is a
copy of the original source code with the style violations inserted.

The file {\tt ex1.sty} now has the following contents after the style
checker has been activated:

\begin{verbatim}
package Example1 is
^
-- *** ERROR: STYLE VIOLATION Line  1, Column  1:
-- Package Specifications And Bodies [SPC-88 5.6.2]
-- This compilation unit contains specification and body for the same package.
-- Separate them in different files

end Example1;

package body Example1 is
  x: Integer;
-----^
-- *** ERROR: STYLE VIOLATION Line  5, Column  6:
-- Types Integer, Natural and Positive [SPC-88 6.3.2]
-- Do not use predefined types (Integer, Natural and Positive)

  b: Boolean;
begin
     b := b and (x = 0);
----------^
-- *** ERROR: STYLE VIOLATION Line  8, Column  10:
-- Short Circuit Control Forms [SPC-88 4.2.1]
-- And not allowed. Use short-circuit forms instead (and then, etc.)

end Example1;

----------------------------------

--  3 errors detected
--  0 warnings detected
\end{verbatim}

Note that a reference to SPC's Ada Style Guide is provided for each 
style violation. For instance, the style rule causing the first style
violation shown above is described in chapter 5.6.2 in the style guide.

