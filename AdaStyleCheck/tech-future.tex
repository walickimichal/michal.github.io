The current version, as it is presented in this report, is not fully
implemented. More work has to be done in order to complete the system.

The following chapter lists, in order of priority, the different areas where
future work would be beneficial.

\section{The Style Checker Generator}
The style checker is working, using the rules that
we have included and supported in the Stanford version of the
vocabulary package.
However, the style checker generator, which should be able to produce the
style rules package based on style rules specified in SSL, is not yet
implemented.

Using the syntax described in Section~\ref{ssl}, ALEX can be used to
generate a parser for the style specification language.

AYACC can then be used to generate the style checker generator, such that
SSL style rules directly can be translated into
Anna annotations. The output of the style checker generator should be in the
file named {\tt `style\_rules.Anna'}. The package in this file contains the
 {\bf Style\_Rules} package, which contains the 2 files: {\bf Init\_Style\_Rules}
and {\bf Apply\_Style\_Rules}. 

Examples of how conversion of SSL statements to Anna annotations should
 be performed is given in appendix~\ref{stanford}.

\section{Using the Mitre Primitives}
As described in the systems overview chapter, because
 the time frame for the style checker project has been relatively narrow,
it has been difficult to apply the division of the vocabulary package into
several abstraction layers. 
However,
research at Mitre has produced an Ada query language described
in~\cite{MITRE},
which includes an implementation of 350 primitives to describe Ada programs.
Including these primitives as lower level operations in the style checker
would allow the functions available to the system manager to use these
lower level functions, and thereby enforce the division of the vocabulary 
package into different layers of abstraction.

However, this will require a modification of the higher level functions,
since they do not make use of these primitives.

\section{Turning Style Rules Off}
Since the user might not want all of the style rules to be invoked, when
the style checker is activated, it would be convenient for the user, if
the system allowed him to specify the rules, which should be turned
off. This should include both the command line interface and the 
programmatic interface. 

In order to implement this feature in the programmatic interface, another
list has to be declared. This list could hold the style rule IDs for the 
style rules, that the user would like to deactivate. 

The command line interface should allow the user to either specify
the number of the style rule or the ID at the command line.

\section{With'ed Packages}
An Ada compilation unit can refer to other packages by including a
`with'-statement. The current version of the style checker only
checks the input program specified by the user. However, future versions
of the style checker should support the possibility of checking all
with'ed compilation units, before the input unit is checked.

Switches for the command line interface and arguments for the programmatic
interface are already provided in the current version.

\section{X-Windows Environment}
Providing the user with an X-Windows Environment would greatly enhance
the user-friendliness of the system. The environment could allow
incremental style checking, and on-line changes of the Ada program.

\section{Using Webster's Dictionary}
By providing an interface to the on-line Webster dictionary, the system
manager will be able to write style specifications that pertain to the
English-ness of names used in Ada programs.  For example, the system
manager may specify that portions of names between underscores have to
be English words.
