As described in Chapter~\ref{ssl}, the
system manager can specify the style guidelines using language objects
(which denote some language construct) and operations on these objects.
In contrast to the syntax of SSL, the collections of 
objects and operations are considered to be dynamic entities, which can
be modified several times during the life cycle of the style checker. This can
occur for instance
when new operations are needed in order to express a new style guideline.

This chapter gives the reader a brief overview of the system
and shows how it is affected by modifying the objects and the operations.

\section{The System Structure}
	
\begin{figure}[htbp]

\vspace{4.1in}

\special{psfile=system.ps}

\caption{System Architecture}\label{System}

\end{figure}

The following is a brief high-level description of each element of the system.
The numbers refer to the numbers on Figure~\ref{System}.

\begin{enumerate}

\item
The system manager creates a style specification, using the SSL (1).

\item
The parser (2) parses the system manager's style specification based on the 
BNF grammar described in Chapter~\ref{ssl} and an executable version of style
checker is constructed (5,6).


\item
This executable image (7) will then take Ada programs as input and 
produce violation indications as output.
This program will depend heavily on the set of Anna Tools used for parsing,
semanticizing and manipulating the DIANA representation of the input program
(see~\cite{evans:diana-lrm}).

\end{enumerate}

\section{Modifications}
We consider the collection of objects and the collection of operations to
be dynamic entities, {\em i.e.}, they are subject to change during the lifetime of
the system. Because the system structure must allow modifications to take place 
easily, our system
allows modification of the collection of 
objects or the collection of operations simply 
by modification of the vocabulary package (3) and the mapping (4).
For example, adding an operation can be done in the following way: a function is 
declared and formally specified in the visible part of the vocabulary package
and implemented in the body of the vocabulary package. This will of course 
require some 
knowledge of the implementation. A mapping is also defined between the 
actual function/procedure in the vocabulary package and the operation
in the system manager's view. The vocabulary package is then transformed
using the Anna tools and is compiled using the Ada compiler to a 
separate module. At this point, the function is ready
for use by the system manager in his style specification.

The functions and procedures in the package can be divided into several
layers of abstraction, {\em i.e.}, functions on higher levels of abstraction
can use lower level functions in their implementation. This approach will
ease the modifications of the package.

\begin{figure}[htbp]

\vspace{1.9in}

\special{psfile=rings.ps}

\caption{Layers of the Vocabulary Package}\label{Layers}

\end{figure}

The system manager and the people maintaining the functions at different
levels of abstraction all have a different view (see Figure~\ref{Layers}). The
system manager will view the functions and procedures as operations on
the Ada language entities, without knowing anything about the implementation
of the system. The system manager's style specification will only be based 
on knowledge of the Ada language ({\em e.g.}, notions of nested language 
constructs, if-then-else statements, package bodies and specifications, and the
like).

The person maintaining the the functions and procedures at the highest level 
of abstraction must have some knowledge about the underlying abstract
syntax tree, but need not know the details. A higher level function can
be implemented using the low level functions. A system manager would
 typically get one of his software engineers to maintain these higher level
functions and procedures, while
maintaining the lower level functions and procedures will require an intimate
knowledge of the underlying DIANA tree. An example of people maintaining
these is members of the Stanford Anna group or others familiar with the Anna
Tools.
