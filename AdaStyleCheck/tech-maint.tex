The following describes the process of producing a new stand-alone version of
 the style checker. We assume that an Anna library has been created by 
invoking {\tt Anna.mklib}, and that all the files necessary to create the style
checker are present in the directory.

The Stanford version of the style checker is currently found in the directory:
\begin{center}
\tt /anna/lockheed/style
\end{center}

As described earlier, the system consists of many different packages, which
are all linked to form a 1.8 to 2MB executable file (The Stanford
version including 20 rules takes approximately 1.84MB in version \version).

The style checker is divided into several packages:

\begin{itemize}
\item {\bf List\_Package Package:} The list package specification is given in {\tt list.Anna} 
and the body in {\tt list\_.Anna}.

\item {\bf Voc Package:} The package specification is found in
{\tt voc\_v.Anna}, and the body in {\tt voc\_b.Anna}. 

\item {\bf Style\_Rules Package:} The package is found in {\tt style\_rules.Anna}.

\item {\bf Style\_Checker Package:} The specification of the package is found
in {\tt style\_v.Anna} and the body in {\tt style\_b.Anna}.

\item {\bf Style\_Driver Procedure:} The package is found in {\tt
style\_driver.Anna}.

\item {\bf Dummy Packages:} In order to run the Anna transformer,
several dummy packages/stubs have been made to cover packages that the 
Anna transformer should not transform, {\em e.g.}, Text\_IO, Command\_Line, SYSTEM, etc.
These dummy packages are found in the files {\tt dummies1.Anna} and {\tt
dummies2.Anna}.
\end{itemize}

A 'makefile' has been made to ease the transformation, compiling and linking.

The following files are required to recompile the complete style checker from
scratch:
{\tt list.Anna},
{\tt list\_.Anna},
{\tt dummies1.Anna},
{\tt dummies2.Anna},
{\tt voc\_v.Anna},
{\tt voc\_v.a},
{\tt voc\_b.Anna},
{\tt voc\_b.a},
{\tt style\_v.Anna},
{\tt style\_rules.Anna},
{\tt style\_b.Anna},
{\tt style\_driver.Anna},
the correct make file ({\tt Makefile}).

Furthermore, all paths needed to execute the Ada compiler should be set up.

The following stages are needed in order to produce the executable file:

\begin{itemize}
\item [1.] Transforming the Anna files into Ada files by running the
 Stanford Anna Transformer on the {\tt .Anna} file. Note that this does not
apply to voc\_v.Anna and voc\_b.Anna. These files are only used as dummy 'stubs',
{\em i.e.}, they contain only the information necessary to make the Anna Transformer
transform the subsequent packages correctly. {\bf Thus, care should be taken to
save a copy of the vocabulary package, since running the transformer causes
voc\_?.a to be overwritten.} We have saved a copy in the files voc\_?.back.

\item [2.] Compiling each package using the Ada compiler. (Now remember to
compile the correct copies of voc\_v.a and voc\_b.a.)
\item [3.] Repeating step 1 and 2 until all packages have been compiled,
and then
\item [4.] Linking the packages and producing the executable file.
\end{itemize}

The end result is now an executable file, which can be activated from the
command line (See Chapter~\ref{usage}).

The whole process of transforming, compiling, and linking is then invoked by
issuing the following command line call:

{\center {\tt make style}}

As shown above, the order of processing should be as follows:

\begin{itemize}
\item [1.] {\tt list.Anna}
\item [2.] {\tt list\_.Anna}
\item [3.] {\tt dummies1.Anna}
\item [4.] {\tt dummies2.Anna}
\item [5.] {\tt voc\_v.Anna}
\item [6.] {\tt voc\_b.Anna}
\item [7.] {\tt voc\_b.Anna}
\item [8.] {\tt style\_v.Anna}
\item [9.] {\tt style\_rules.Anna}
\item [10.] {\tt style\_b.Anna}
\item [11.] {\tt style\_driver.Anna}
\item [12.] Linking using {\tt a.ld}
\end{itemize}

If only the programmatic interface is needed, the file {\tt style\_driver.Anna}
is substituted with the new file and everything is compiled and linked
as before.
