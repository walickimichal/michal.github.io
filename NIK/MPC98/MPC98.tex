\documentclass[11pt]{article}
\usepackage{amssymb}

\makeatletter
%\voffset -2cm
\input{a4wide}
\makeatother

\input defs
\newcommand\der{{\,\Rightarrow\,}}
\newcommand{\togo}[1]{{\tiny{#1}}\normalsize}
\newcommand{\leadto}{\mathrel{\rightsquigarrow}}

\newcounter{BIB}
\newcommand{\bibyear}[1]{#1}
\newcommand{\bibemph}[1]{{\em #1}}
\newcommand{\bibemphic}[1]{{\em #1}}
\newcommand{\bibsc}[1]{{\sc{#1}}}
\newcommand{\key}[1]{{\bf #1}}

\input xypic
%\xyoption{graph}
\xyoption{frame}

\title{Nondeterminism, Underspecification, \\ Initiality and \ldots Junk}
\author{Micha{\l} Walicki
\and Sigurd Meldal}


\begin{document}

\maketitle
\abstract{\noindent
This paper indicates how nondeterministic operators supply a useful (and
even {\em necessary\/}) abstraction mechanism, and how the common
alternative, {\em underspecification\/} is insufficient even in quite
simple examples.  As an example we show various approaches to specifying
unordered structures and iteration over such structures as sets.

Having established the utility (and {\em need\/}) for nondeterministic
operators, we go on to discuss the challenges of retaining the benefits of
initial semantics  (no ``junk'', no ``confusion'') in the
presence of nondeterministic operators. We show how nondeterminism is
{\em no worse\/} than underspecification in that regard, but that they both
result in the introduction of ``intuitive junk'' into the models. In
conclusion we point out how {\em quasi-\/}initial semantics retain the
desirable attributes of initial semantics while still allowing for
nondeterminism (and underspecification).

}


\section{Introduction}
{\em Initial models\/} have established themselves as a key ingredient in
any algebraic specification toolkit, and there is no call for an argument
in opposition to its plausibility.  Initial semantics has immediate
intuitive appeal in that it establishes a very close relation between the
syntax of a specification and its semantics -- initial semantics is
essentially syntactic, as reflected in the names ``term model'' or ``word
model''.  Initiality implies that the semantic changes induced by syntactic
changes in a specification are minimal, and so reflect the latter in the
natural and expected manner.  For instance, adding a new sort and a
constant of this sort to an existing specification have no exotic influence
on the model.  The model will be extended with a new sort and just the one
element needed for interpretation of the new constant.  Furthermore, from a
methodological point of view, when writing distinct terms one usually
intends them to be distinct entities unless they are explicitly equated
with other terms.

The two slogans summarizing these properties of initial models are ``{\em no
junk\/}'' and ``{\em no confusion\/}'' -- no element appears in the
model whose
existence is not required by the interpretation of terms introduced in the
specification, and distinct terms are interpreted as distinct elements
unless their identity are required by the equations of the specification.  In
short, initiality is nice because it results in intuitive models, and the
two slogans make the notion of ``intuitive'' more precise.

A limitation of the initial approach is that not all theories have initial
models.  It is well known \cite{c:76, c:78, c:122} %[10,8,14]
that Horn theories are the most general theories that can be guaranteed
always to possess initial models.  As the interest in initial semantics was
carried over to the domain of {\em non\/}deterministic
specifications, so was the
interest in Horn theories.

In the first part of the paper we present the argument motivating the use
of nondeterminism in specification, even when the systems (datastructures,
algorithms) specified are intended to be implemented deterministically.  In
particular, we emphasize the difference between {\em nondeterminism\/} and
{\em underspecification\/}.  Roughly, the example used to argue the case
boils down to the observation that unordered structures (in particular {\em
sets\/}) are very common and often very convenient abstract data structures
-- both in specification and programming.  Equally common is the need for
an {\em iteration\/} mechanism over such structures.  But (deterministic)
iteration introduces implicit ordering!  We show by means of an example
that the correct iterator over unordered structures should be specified as
nondeterministic choice and that iteration along some implicitly assumed
ordering, in spite of apparent plausibility, can easily lead to unintended
results, collapsing otherwise non-equal data structures into identity.

As a vehicle for exploring these issues we shall make use of a choice
operator $\ch$. Its intuitive meaning is simple: The result of $a \ch b$ is
either $a$ or $b$ -- no more, no less. In general, $\ch(S)$ should
denote {\em some\/}
member of $S$, and we do not care which one.  This should allow us to
discuss whether $c = a \ch b$ is valid or not, the consequences of
introducing the equation as part of a specification, etc. Since the notion
of {\em choice\/} is such a simple and intuitive one, we are striving for a
semantic system which allows us to state its properties (and consequences
of use) in a similarly straightforward and uncomplicated manner.

Having argued the utility and necessity of nondeterministic operators as an
abstraction mechanism, we go on to illustrate the consequences of applying
the theory of initiality to the domain of nondeterminism.  We present two
general ways
of introducing nondeterminism into specification and exemplify them by means of
multialgebras \cite{c:53, c:Hes, c:58, c:59, c:64} %[3,4,7]
and unified algebras \cite{c:91, c:92}. %[11,12].
A simple, yet illustrative, example is worked out in these formalisms
leading to some general observations\footnote{The example itself was used
in \cite{c:91}, but has certainly been contemplated by anybody devoting a
few minutes' thought to nondeterminism.  We are particularly thankful to
Vladimir Antimirov for his comments and discussion of it in the context of
unified algebras.}: since nondeterminism subsumes underspecification, the
initial models of nondeterminism introduce ``intuitive {\em junk\/}'' and,
furthermore, that treating nondeterministic specifications not merely as
underspecification but as specification of (truly) nondeterministic
operations does not make the situation worse.  As a possible alternative to
initial semantics we suggest {\em quasi-initial\/} semantics of nondeterminism.

The presentation focuses on the concepts and general phenomena rather than
technicalities.  We give the definitions of multialgebraic and unified
algebra semantics but the paper aims at rather informal presentation.  More
technical details can be found in \cite{survey,BW97}.  In particular, the
issue of formal reasoning with the specifications of nondeterminsm is not
addressed at all -- the reader is referred to \cite{c:130,c:69, c:69a}.
% Then we suggest that the way to avoid the
%implausibilities is simply to allow the specifier to say what he
%intends to say.

\section{Why Nondeterminism?}\label{se:why}
One of the aims of specifying a program is to express its functionality at a
proper abstraction level, avoiding complications more properly
addressed at a later
stage in the operational and implementation refinement process.
Abstract properties of the intended structure may (and should) allow different
implementations, and one should avoid overspecifying the problems as this may
exclude some, otherwise acceptable, attractive or even optimal,
implementations.

The traditional means of abstraction in algebraic specifications is the use
of {\em under\/}specification, which is often confused with nondeterminism
or else meant to be a sufficient surrogate.  We'll open with a few remarks on
this issue and then proceed to an example illustrating how
nondeterminism may be an appropriate abstraction mechanism in writing a
specification, and where underspecification cannot achieve the same (and
reasonable) goals.

%\section{Underspecification vs. nondeterminism}
\subsection{Underspecification vs. nondeterminism}\label{sub:underA}
% Limiting a language to Horn
% formulae may restrict the possibility of assuming such a ``don't care''
%attitude.
The choice operation may serve as a (paradigmatic) example: assume that we
want an
operation $\ch:E\times E\into E$ which, for arguments $x,y$, should equal
either $x$ or $y$. A possible underspecification might be:
%The only way
%to capture this intention without using disjunction is to leave $c$ partly
specified
%or unspecified. For instance:
\[\begin{array}{rrrr@{\ \into\ }l}
SP= & \Sorts: & \multicolumn{2}{l}{E} \\
 & \Funcs: &  a,b: & & E \\
 &   & \ch: & E\times E & E \\
 & \Ax: & \multicolumn{3}{l}{
 \begin{array}[t]{rrcl}
1.& x\ch x& = & x \\
2.& x\ch y& = & y\ch x \\
3.& x\ch (y\ch z)& = & (x\ch y)\ch z
\end{array} }
\end{array}
\]
where $a, b$ are the two intended elements of sort $E$.  Axioms 1.-3.
assert the general properties of $\ch$ as a join operator.  The
underspecification
results in several nonisomorphic models for the specification, and the
collection of models is taken as being the (``loose'') semantics of
the specification.

$SP$ will allow models where $a\ch b = a$, and others where $a\ch b = b$.
In general, loose semantics will allow structures which violate the
demands of ``no junk'' and ``no confusion.''  Not only will $SP$ admit
models where $a=b$, but also such where $a\ch b$ returns {\em
neither\/} $a$ nor
$b$.

As specified, the operation $\ch$ has only a remote resemblance to
the choice operation. To get something closer to choice, one might add
the axiom $(x\ch y= x)\lor (x\ch y= y)$. However, that would exclude the
possibility of defining initial semantics (we shall address this issue in the
subsequent section).

But there are other problems arising from disallowing nondeterminism
and replacing it
with mere underspecification.  Consider for instance an abstract datatype
{\em set,\/} with a choice operation. A natural implementation of such a
datatype would be to represent sets as {\em lists\/}, and let the choice
operation be implemented as the head function. However, in  \cite{c:132}
it was shown how such an implementation would violate even the most
minimal of set specifications, and how the problem finds a natural
resolution by allowing for nondeterminism.  In this paper we will
illustrate another general point indicating the insufficiency of
deterministic specifications when dealing with sets.  To do that, we first
introduce a few basic concepts from the multialgebraic framework for
specification of nondeterminism.

The challenge in allowing nondeterminism consists in essence in defining
precisely the relationship between the arguments $x$ and $y$ and the result
of $x\ch y$ -- the latter, to be a choice, {\em must} equal one of the two,
and should certainly not introduce new elements into the semantic models.

This leads to multiagebras, where nondeterministic
operations are modeled by set-valued function.


\subsection{Multialgebras}\label{sub:multA}
A {\em multialgebra\/} (over some signature $\Sigma$) is a
$\Sigma$-algebra where
operations return sets (possibly singleton) of elements.
Following \cite{c:127, c:126}, we introduce two primitive predicates
for building
atomic formulae: $s\Eq t$ (element identity -- {\em equality\/})
requires that $s$ and $t$ denote the same, {\em
one-element} set,
and $s\Incl t$ ({\em inclusion\/}) that the set interpreting $s$ is
included in that
interpreting $t$.
\begin{DEFINITION}\label{de:mult}
A $\Sigma$-multialgebra $A$ is given by:
\begin{enumerate}\MyLPar
\item for each sort symbol $S$, the carrier set $S^A$
\item for each operation symbol $f:S_1...S_n \into S$, an operation
$f^A:S_1^S...S_N^A\into \PSet(S^A)$
\end{enumerate}
Given a $\Sigma$-term $t$ (with variables $X$), any assignment
$\alpha:X\into A$
of {\em individual} elements from the
carrier of $A$ to $X$, induces a unique interpretation $\ovl\alpha(t)$ of
$t$ in $A$.

The formulae of specifications are constructed from the atomic equalities and
inclusions, and
are of the form $a_1,...,a_n\Seq b_1,...,b_m$, with atomic $a_i$ and $b_j$.
$A$ satisfies a formula $\phi$ under assignment $\alpha$
\begin{enumerate}\setcounter{enumi}{2}\MyLPar
\item $A\models_\alpha s\Eq t$ iff $\ovl\alpha(s)=\ovl\alpha(t)=\{e\}$ for some
$e\in A$;
\item $A\models_\alpha s\Incl t$ iff $\ovl\alpha(s)\subseteq\ovl\alpha(t)$;
\item $A\models_\alpha a_1,...,a_n \Impl b_1,...,b_m$ iff $\exists
1\leq i\leq n : A\not\models a_i$ or $\exists 1\leq j\leq m:A\models b_m$.
\end{enumerate}
$A\models\phi$ iff $A\models_\alpha\phi$ for all $\alpha$.
\end{DEFINITION}

\noindent
$\PSet(X)$ denotes the set of non-empty subsets of $X$.  2.  makes it
possible for a constant $c:\ \into S$ to denote a {\em set} of elements
from $S^A$.  Also, according to 3.  $s\Eq t$ holds iff both terms denote
the same 1-element set and not just the same set.  Variables are assigned
{\em individual} and not sets.\footnote{Motivations for these choices can
be found in \cite{c:132, c:126}.}
Now, consider the following specification
\[\begin{array}{rrrr@{\ \into\ }l}
SPN= & \Sorts: &\multicolumn{2}{l}{E} \\
 & \Funcs: & a,b,c: & & E \\
 &   & \ch: & E\times E & E \\
 & \Ax: & \multicolumn{3}{l}{
 \begin{array}[t]{rrcl@{\hspace*{5em}}rrcl}
1.& z\Incl x\ch y & \Impl & z\Eq x \lor z\Eq y  &  3.& a & \Eq & a  \\
2.& c & \Incl & a\ch b  &  4.& b & \Eq & b  \\
&  &  &  &  5.& c & \Eq & c
\end{array} }
\end{array}
\]
3.-5.  merely ensure
that the constants, in particular $c$, are deterministic, i.e., denote
individual elements.  Note the distinctions between $c$ and $z$.  
The operation $x\ch y$ is nondeterministic and
returns either $x$ or $y$: whenever $z$ is a result of $x\ch y$, then
{\em either\/} $z\Eq x$ {\em or\/} $z\Eq y$.  However, %On the other hand
$c$ is {\em deterministic\/} (axiom 5), but {\em underspecified\/}
-- in any model $c$ will denote one element, either the same as denoted by
$a$ or the one denoted by $b$.


\subsection{Nondeterministic choice is the iterator for sets}\label{sub:iter}
Consider {\em graphs\/} as an example of an eminently practical
abstraction.  Graphs are typically presented as sets of vertices and sets
of edges, and graph algorithms are rife with expressions such as ``for each
edge $e$ from a vertex $v$ do \dots''. The algorithms usually do {\em not\/}
depend on the order in which the set of edges are traversed, leaving this
choice to the implementor of either the set abstraction or the graph
abstraction.

As an example, let us consider the problem of constructing the {\em
depth-first traversal tree\/} of nodes reachable from a particular node of
a directed, irreflexive graph.  We assume that a (connected) graph $g$ is
represented as a {\em set\/} $G$ of edges, each edge being a pair $\<x,y\>$
identifying the source and target vertices.  (The set $V$ of all vertices
in a graph can be easily obtained from $G$.)
%as a pair $\<V,N\>$, where $V$ is a set of vertices and
Assume further that given a graph $g$, the function $N:V\into\PSet(V)$
returns, for every vertex $v$ the set of its neighbours in $g$.  Working
with sets, we also assume some standard specification thereof, in
particular, operations {\em empty set\/}, $\es:\ \into S$, {\em adding an
element\/} to a set, $+:S\times E\into S$ and {\em removing\/} an element
$\setminus: S\times E\into S$.
%(We will allow ourselves to apply these to single
%elements, treating them as 1-element sets.)

A textbook presentation of the {\em DFS\/} algorithm is the following pair
of procedures:
%%%%%%%%
%%
%% Michal <- Sigurd:
%% This needs formatting (bold for keywords, indents etc.)
%% The correct symbols for EmptySet etc.
%% I left it as verbatim to maintain the structure I intended for
%% the final result (and as a fallback in case we runshor on time)
%%
%%%%%%%%
\begin{center}
\begin{tabular}{l@{\hspace*{5em}}l}
%\begin{verbatim}
   DFS ($G, v$) \key{is}	 &            Traverse ($G, v, T$) \key{is}   \\
   \key{begin}                  &           \key{begin} \\
   \ \ \    $T$ := $\es$;   &      \ \ \         mark $v$;  \\
   \ \ \    Traverse ($G, v, T$); &     \ \ \     \key{for all} edges $\<v, x\>$ \key{do}  \\
   \ \ \    \key{return} $T$;  &  \ \ \ \ \ \  \key{if} $x$ is unmarked \key{then} \\
   \key{end};   &    \ \ \ \ \ \ \ \ \         $T$ := $T+\<v,x\>$;\ \ Traverse ($G, x, T$); \\
%                          &           \ \ \ \ \ \ \ \ \           $T$ := $T+\<v,x\>$;  \\
	                  &           \ \ \ \ \ \       \key{end} if;  \\
	                  &           \ \ \    \key{end} for; \\
	                  &           \ \ \    \key{return} $T$; \\
                          &           \key{end} Traverse;
%\end{verbatim} 
\end{tabular}
\end{center}
\noindent
The $DFS$ algorithm as stated is simple, and we should hope for a similar
simplicity and level of abstraction when we give an algebraic specification
for it. (Note in particular the use of the set iterator in the loop of
{\em Traverse\/}.)

The following attempts to give an algebraic version of the specification of
the standard algorithm for $DFS(g,v)$
starting from the vertex $v$ in a graph $g$ and constructing the DFS-tree
by recursive calls to the operation $Tr:Gr\times V\times Gr\into Gr$, where
the second $Gr$-argument is a set of edges representing the tree (graph)
constructed so far.
\[\begin{array}{rrcl}
1.& & & DFS(g,v) = Tr(g,v,\es) \\
2.& N(g,v)=\es & \Impl & Tr(g,v,t)=t \\
3.& N(v)=S+x, x\in |t| & \Impl & Tr(g,v,t)= Tr(g\setminus\<v,x\>,v,t) \\
4.& N(v)=S+x, x\not\in |t| & \Impl & Tr(g,v,t)=
  Tr(g\setminus\<v,x\>,v,\  Tr(g\setminus\<v,x\>,x,t+\<v,x\>)\ )
\end{array}
\]
The construction starts with the empty tree (axiom 1).  $Tr$
backtracks if $v$ has no
neighbours (either because all were visited (marked by removing them from
$g$) or because $v$ is a leaf) (axiom 2).  If the set
$N(v)$ is not empty, i.e., it is $S+x$ for some set $S$ and vertex $x$,
then there are two possibilities for each element of this set.  If $v$'s
neighbour $x$ is marked (indicated by $x\in |t|$, i.e., $x$ is among the
vertices included in $t$) then the edge $\<v,x\>$ is simply removed
from further
consideration (axiom 3).  If, on the other hand the neighbour $x$ was not
visited before, then the edge $\<v,x\>$ added to $t$ as well as being
removed from $g$, and the construction proceeds recursively down through
$x$ (axiom 4).


This hopefully corresponds to the intuitive understanding of the algorithm and
may look plausible -- but only as long as we do not inspect the definition
of the sort of sets.
Sets are unordered and consequently adding elements to a set is
always considered a
commutative operation:
\begin{equation}
 S+x+y  =  S+y + x
\end{equation}
is a standard axiom of set specifications.

However, from this and the specification of the $DFS$ and $Tr$ functions
we obtain the following:
\begin{equation}
Tr(g+\<v,b\>+\<v,a\>, v,\es) = Tr(g+\<v,a\>+\<v,b\>, v, \es)
\end{equation}
In other words, for a graph:

$g = \vcenter{\xymatrix @R=2ex @C=1ex{
% & \save*{v}*\cir<5pt>{}\restore \ar[dl]\ar[dr] \\
& v \ar[dl]\ar[dr] \\  a\ar[rr] & &  b}}$
we have that the specification of the $DFS$ and $Tr$ functions result in
the collapse of (intuitively) distinct trees into one:
$\vcenter{\xymatrix @R=2ex @C=1ex{ & v \ar[dl]\ar[dr] \\ a & &  b}}\ =\
 \vcenter{\xymatrix @R=2ex @C=1ex{ & v \ar[dl] \\ a\ar[rr] & &  b}}$.

\noindent
The problem is that the internal structure of the set values intrudes into
the specification.  The definition of $Tr$ in terms of adding elements to a
set tries, on the one hand, to respect the representation of graphs in
terms of sets (of neighbours of each vertex) and, on the other hand, use
the ordering of sets, implicit in their definition by means of adding
elements.  One might avoid the problem by choosing a {\em particular set
representation\/} when specifying the $DFS$ function, e.g., letting the
neighbours $N(v)$ be represented as a sequence instead of as a set.
However, this is quite contrary to the central tenet of (algebraic)
specifications -- it forces us into making representational or operational
decisions when there is no conceptual need to do so: our abstract
specification formalism forces us to abandon a key abstraction!

An abstract definition of $DFS$ should {\em not} assume any such particular
representation of the sets of neighbours.  It also should {\em not} force
the implementor to construct a  {\em particular\/} DFS-tree but, on the
contrary, determine only the necessary requirements, leaving the choice of a
particular strategy (here particular way of ordering a set) to the
implementor.

This objective can be achieved by the following, nondeterministic variant
of the above specification.  We assume the same representation of graphs
and specification of sets as before.  In addition, we assume a
nondeterministic choice operation
$\ch:S\into E$. There are a number of ways to specify choice, for
instance, by an axiom
$z\Incl \ch.(S+x)\Impl (z\Eq x)\lor (z\Incl \ch.S)$, or, perhaps,
$S\Eq\es\lor\ch(S)\in S$. The former can be read as ``If we add an element
$x$ to a set $S$, then applying the choice operator to the resulting set we
either get $x$ or we get one of the values possible when applying choice
to $S$.''
\[\begin{array}{rrcl}
1.& & & DFS(g,v) \Incl Tr(g,v,\es) \\
2.& N(g,v)\Eq \es & \Impl & Tr(g,v,t)\Eq t \\
3.& x\Incl \ch(N(v)), x\in |t| & \Impl & Tr(g,v,t)\Incl
Tr(g\setminus\<v,x\>,v,t) \\
4.& x\Incl \ch(N(v)), x\not\in |t| & \Impl & Tr(g,v,t)\Incl
  Tr(g\setminus\<v,x\>,v,\  Tr(g\setminus\<v,x\>,x,t+\<v,x\>)\ )
\end{array}
\]
Axiom 1. indicates that $DFS$ will return one of the trees $Tr$ is capable
of returning. Axiom 2. takes care of the backtracking case (which is
deterministic). Axioms 3. and 4. picks an arbitrary neighbour, and requires
that $Tr$ must allow for the edge to that neighbour being the next one to
traverse. But the specification does not {\em require\/} that particular
neighbour to be the next (except if it is the {\em only\/} neighbour, of
course).

The operation $Tr$ is now allowed to return any (all) DFS-trees -- depending
on the choice of $x$ from the set $N(v)$ at each stage (axiom 4).
Instead of the
unintended collapse of different trees, we will now have that:
\[
Tr( \vcenter{\xymatrix @R=2ex @C=1ex{
& v \ar[dl]\ar[dr] \\  a\ar[rr] & &  b}}, v,\es)\ \Incl\ \{
\vcenter{\xymatrix @R=2ex @C=1ex{ & v \ar[dl]\ar[dr] \\ a & &  b}},\
 \vcenter{\xymatrix @R=2ex @C=1ex{ & v \ar[dl] \\ a\ar[rr] & &  b}}\ \}
\]
i.e., the set of possible results of applying $Tr$ to the graph is
constrained by the set of two trees on the right-hand side of the
inclusion.

The example illustrates the point that deterministic specifications may,
unavoidably, amount to {\em overspecification} thus violating the central
objective of algebraic specifications. As we said, this is just one
example of a situation which is likely to occur whenever the involved
data structures are inherently unordered.

\section{Initiality + Nondeterminism\ $\leadto$\ Junk}\label{se:junk}
Having argued the utility of nondeterministic operators, what are the
consequences for our models when we attempt to carry over the notion of
initiality. In particular, what happens to the ``no junk'' slogan which
is at the basis of the argument in favor of initial models in standard
algebraic specifications of deterministic operators?

A conservative approach would use nondeterminism merely for the purpose of
underspecification and hope that initiality in classes of models of such
specifications would retain the purity expected of traditional initial
models.  However, we will demonstrate that from the point of view of the
initial semantics the sole problem of nondeterminism is caused by the fact
that {\em nondeterminism subsumes underspecification\/}.  Nondeterminism in
itself does not cause much harm -- it is the fact that deterministic
operations can be underspecified by means of nondeterministic ones that
leads to some challenges in the initial models.

The initial model $\ini$ of $SP$ from \ref{sub:underA} will contain
\[\begin{array}{c@{\ }|@{\ \ }c@{\ \ \ \ }c}
x       & \ini(x) \\ \hline
a       & a   \\
b       & b   \\
a\ch b	& a\ch b  \\
a\ch a	& a   \\
\multicolumn{2}{c}{\ldots}
\end{array}
\]
The problematic point is the presence of the element $a\ch b$.  Formally,
this is quite correct since the corresponding term exists in the
specification and is not provably equal to $a$ or $b$.  However, with
respect to the original intention it represents {\em junk\/}, since
$a$ and $b$
were supposed to generate all the elements of the sort $S$.  We may call
this ``intuitive junk'' to distinguish it from unreachable elements
corresponding to (formal) junk.  It may appear in a carelessly written
specification, but here it appears because of the insufficiently expressive
language.

Initial semantics requires one to restrict the language to Horn clauses.
Hence, one cannot add the axiom $(x\ch y = x) \lor (x\ch y= y)$ to get rid
of the unintended element $a\ch b$.  Thus, due to the limitations of the
language, underspecification gives us a choice between an imprecisely
defined class of models or intuitive junk, both of which are rather
unsatisfactory.

Nondeterminism can be introduced into a language basically in two ways: either
by means of some new primitive relation/predicate (e.g., set membership,
inclusion) allowing specification of nondeterministic operations, or else
by including some primitive operation (e.g., choice) with a
predefined semantics
reflecting the intended nondeterminism.  The former is exemplified by
multialgebras and in \ref{sub:multB} we make some further remarks on this
approach.  As an example of the latter, we have chosen unified algebras and
present them in \ref{sub:ua}.


\subsection{Multialgebras}\label{sub:multB}
If we want to be sure of always obtaining initial models then we are
restricted to the language to conditional formulae (or, if
necessary, Horn clauses), unlike in
definition~\ref{de:mult}. Hu{\ss}mann \cite{c:58, c:59} investigated the constraints
to be satisfied in order to obtain initial models for nondeterministic
operators -- we will merely summarize those points of this approach
required to make the relevant observations.

The specification
$SPN$ from \ref{sub:multA} used disjunction to specify the choice
operation.  If we remove the constant $c$ and axioms 2.  and 5., then the
specification will have an initial multialgebraic model. {\em But
this is just a
lucky coincidence.\/}  In general, the presence of disjunction does not admit
initial models.  Let us therefore attempt to do without disjunctions --
a corresponding specification might be as follows:
\[\begin{array}{rrrr@{\ \into\ }l}
SPNI= & \Sorts: &\multicolumn{2}{l}{E} \\
 & \Funcs: & a,b,c: & & E \\
 &   & \ch: & E\times E & E \\
 & \Ax: & \multicolumn{3}{l}{
 \begin{array}[t]{rrcl@{\hspace*{5em}}rrcl}
1.& x &\Incl & x\ch y  &  3.& a & \Eq & a  \\
2.& x &\Incl & x\ch y   &  4.& b & \Eq & b  \\
3.& c & \Incl & a\ch b &  5.& c & \Eq & c
\end{array} }
\end{array}
\]
It has an initial model, constructed from the term multistructure $T_\Sigma:$
\[\begin{array}{r@{\ =\ }l@{\hspace*{4em}}r@{\ =\ }l}
 \ini(a) & \{a\}  & \ini(a\ch b) & \{a,b,c\} \\
\ini(b) & \{b\} & \ini(a\ch c)& \{a,c\} \\
\ini(c) & \{c\} & \ini(b\ch c) & \{b,c\} \\
{\rm for\ other\ }s,t\in T_\Sigma: \ini(s\ch t) & \ini(s)\cup\ini(t)
\end{array}
\]
As we can see, $\{c\}$ appears in the initial model as an additional
element.  As far as choice is concerned, the axioms 1.  and 2.  do make it
into set union (in the initial model), but the underspecified $c$ gives
rise to the ``intuitive junk''.  This phenomenon is by no means limited
to this particular way of using multialgebras -- it is the {\em unavoidable\/}
side-effect of marrying nondeterminsm (with the possibility of
underspecification) with initiality.

In fact, to guarantee the existence of initial multimodels the restriction
to Horn clauses is not sufficient.  Hu{\ss}mann \cite{c:59} specified
other, sufficient (and slightly restrictive but not unreasonable)
conditions.  Since they do not make the `'intuitive junk'' disappear, we
will not elaborate on them here.

\subsection{Unified algebras}\label{sub:ua}
Unified algebras have been introduced by Peter Mosses in
\cite{c:92, c:91}. %[11,12].
In the following discussion the notation for unified algebras
 has been modified to be consistent with the rest of the paper.

Every unified signature $\Sigma$ contains the subsignature $\Omega$  with the
operations
% $\{{\sf nothing}, \_|\_, \_\&\_\}$ and predicates
% $\{\_=\_, \_\leq\_, \_:\_\}$. For the sake of
% notational compatibility we will use the notation
$\{\bot, \_\sqcap\_, \_\ch\_\}$ and binary predicates $\{=, \Incl, :\}$.\footnote{Since 
unified algebras have only one sort there is no need to specify profiles
of the operations, except for the number of arguments they take.}
Formulae of the specifications can be (universal) Horn clauses. A unified
$\Sigma$-algebra $A$ is a structure (with one sort) such that:
\begin{itemize}\MyLPar
\item $A$ is a distributive lattice with $\sqcap^A$ as join, $\ch^A$ as meet,
   and $\bot^A$ as bottom.
\item There is a distinguished set $E^A\subseteq A$ - the {\em
individuals\/}  of $A$.
\item $=^A$ is the identity on the elements of the lattice (not only on the
individuals).
\item $\Incl^A$ is the partial order of the lattice, i.e.,
   $x\Incl^A y\Iff x\ch^Ay=^Ay$.
\item For every $f\in\Sigma$, $f^A$ is monotone w.r.t. $\Incl$.
\item $x:^Ay$ holds iff $x\in E^A$ and $x\Incl^A y$
\end{itemize}
``Unified'' refers to the fact that sorts (just like individuals) are
elements of the lattice (and not sets of the elements as in the case of
multialgebras).  The partial order of the lattice corresponds to set
inclusion, and nondeterministic choice is interpreted as joins.  Thus
nondeterministic choice is here a predefined operation -- join -- whose
semantics is built into the framework.

The operations from $\Omega$ can be specified using Horn clauses alone, and
it is shown
that specifications with Horn clauses always possess initial unified models.
The example specification becomes:
\[\begin{array}{rrrr@{\ \into\ }l}
SPU= & \Funcs: & \multicolumn{3}{l}{a,b,c} \\
 & \Ax: & \multicolumn{3}{l}{
 \begin{array}[t]{rrcl@{\hspace*{5em}}rrcl}
1.& a & : & a  \\
2.& b & : & b   \\
3.& c & : & a\ch b
\end{array} }
\end{array}
\]
The first axioms make $a$ and $b$ individuals.
The third one makes $c$ an individual which, in addition, lies below the join
$a\ch b$. The intended meaning of this is to have two individuals,
$a$ and $b$, and to let $c$ be a result of choosing non-deterministically
between
the two.
But again, having $c$ as a constant in the signature, one may expect to
find it also
in the initial model. However, since any model must be a (distributive)
lattice,
there are more surprises than that. The initial unified algebra for
$SPU$ is given in the following figure:
\[
\xymatrix @R=2.5ex @C=2ex {
 & & a\ch b \ar@{-}[dl] \ar@{-}[dr] \ar@{-}[dd]\\
& a\ch c \ar@{-}[dl] \ar@{-}[dr] & & b\ch c \ar@{-}[dl] \ar@{-}[dr]  \\
a \ar@{-}[d] & & % \save*{(a\meet c)\ch(b\meet c)}*\frm<5pt>{-}\restore
(a\ch c)\meet(b\ch c)
  \ar@{-}[dll]\ar@{-}[drr]\ar@{-}[dl]\ar@{-}[dd]
     & & b \ar@{-}[d] \\
a\meet(b\ch c) \ar@{-}[dr]\ar@{-}[drr] & c\ar@{-}[d]\ar@{-}[drr] &  & &
     (a\ch c)\meet b \ar@{-}[dl] \ar@{-}[dll] \\
& a\meet c \ar@{-}[dr] & a\meet b \ar@{-}[d] & b\meet c \ar@{-}[dl] & \\
& & a\meet b\meet c \ar@{-}[d] & & \\
& & \bot & &
}
\]
Most of the elements of this model are surprises which we would not expect, and
definitely did not intend, when writing the specification $SPU$.
 They have to appear because the lattice model requires all joins and meets
to be
present and they are irreducible with the lattice axioms
alone.\footnote{For the
purpose of modeling nondeterminism the meets are not needed and the upper
semi-lattice would provide sufficient structure. This would give a
significantly more
intuitive model - see (the upper part of) the next figure. Nevertheless,
even for
such a simple example, junky $c$ and its joins would remain.}
We can get rid of some of such ``vacuous sorts'' by introducing additional
axioms.
In the present case, the axiom
\begin{equation}
a\meet b =\bot  \label{eq:addua}
\end{equation}  will identify most of the
redundant elements leading to the following, more intuitive model:
\[
\xymatrix @R=2.5ex @C=2ex {
 & & a\ch b \ar@{-}[dl] \ar@{-}[dr] \ar@{-}[dd]\\
& a\ch c \ar@{-}[dl] \ar@{-}[dr] & & b\ch c \ar@{-}[dl] \ar@{-}[dr]  \\
a \ar@{-}[dr] & & [c]  \ar@{-}[dl]\ar@{-}[dr]\ar@{-}[dl]\ar@{-}[dd]
     & & b \ar@{-}[dl] \\
& [a\meet c] \ar@{-}[dr] &  & [b\meet c] \ar@{-}[dl] & \\
& & [\bot] & &
}
\]
Here, $c$ is still a separate element which cannot be set equal to $a$ nor
$b$.
Consequently, the respective joins and meets must be present as well, although
we intended that one from each pair will actually collapse (e.g., if $c$ is
equal to $a$
then $a\ch c=a\meet c=a$).\footnote{We can give some plausibility to the
presence of
joins -- not knowing
whether $c$ is $a$ or $b$, we do not know whether $a\ch c$ is going to be
$a$ or $a\ch b$, i.e., we
have to keep both. It is more difficult to give a natural interpretation of
meets, in
particular, meets of individuals.}
Thus, in general, even after spending time and effort on
finding out which additional axioms are needed to remove irrelevant
elements, we
will be left with some redundant objects in the initial model.
As in the case of multialgebras, the removal of the axiom $c:a\ch b$
underspecifying $c$ would lead to
the intended  initial model with the elements $\{\bot, a,b,a\ch b\}$ (but
it would remove the intended use of choice).

There is another aspect of the ``redundancy'' in the context of unified
algebras.
Even if we do not specify the choice operator explicitly, it will be
present in the
model. In the above example we only said that $c$ should be a result of
$a\ch b$ and
the elements corresponding to $a\ch c$ and $b\ch c$ have been included in
the model,
so to speak, automatically. This is, of course, a consequence of the fact that
choice is a primitive operator and  appears in all the approaches we have seen
where its semantics is built into the general framework.

\section{Comparison}
We emphasize some points concerning the relations between the models we
have examined. One should keep in mind that these particular models were
chosen
merely as examples of two ways of approaching nondeterminism, the one
insisting
on the interpretation in terms of sets and their elements (multialgebras),
the other
attempting to fix the semantics of choice in an appropriate mathematical
structure
(unified algebras). The remarks apply to these respective classes.

It is easy to see that the upper part of the second lattice
from~\ref{sub:ua} is
essentially the same as the initial multimodel from~\ref{sub:multB},
where the partial order is specialized to set inclusion.
In particular, the ``redundant'' elements $c, \{a,c\}, \{b,c\}$ are present in
the multimodel as they are in the unified algebra. Certainly, the advantage
of the multimodel is that it is obtained directly, while unified algebras
required an additional axiom (\ref{eq:addua}).
On the other hand, unified algebras guarantee the existence of initial
models for
all specifications with Horn clauses, while only a subclass of Horn
specifications
possesses initial multimodels.
Nevertheless, even these ``natural'' models involve elements which are
superfluous
when compared to the intentions of the specifier.

Treating choice as a primitive of the language with a predefined semantics is
plausible in
so far as choice is a very natural nondeterministic operation which will
typically
be present in specifications using nondeterminism. Although in the initial
multimodel
of $SPNI$ choice happens to be join, this will not be the case in general.
In other
(non-generated) multimodels $a\ch b$ may return all other kinds of elements
besides
 $a$ and $b$.  Thus, all models of $SPNI$ will satisfy
$a\ch c\Incl a\ch (a\ch b)$, but not necessarily $a\ch c\Incl a\ch b$.
The latter simply does not hold in the
models where $a\ch c$ returns some elements not returned by $a\ch b$.
This, of course, does not happen in the unified algebra models for which
nondeterminism cannot be (re-)defined by the user.

Nevertheless, it is precisely this
predefined meaning of $\ch$ in unified algebras which is responsible for the
occurrence of ``junk'' which, to some extent, can be eliminated by
additional axioms.
Multialgebras, on the other hand, do not restrict the user in this respect
but give him the possibility of explicitly specifying the join operator if
there
is a need for it. This gives increased flexibility
since each nondeterministic operator can be specified directly - in
particular, when no such operations are present, multialgebra reduces
essentially to a standard algebra. In unified algebras, on the other hand,
all nondeterminism must be referred back to the primitive choice operator
which is always present in the model.

The superfluous objects may or may not have some intuitive plausibility but
are not
among the objects usually intended by the writer of the specification.
This is only an empirical statement and we do not claim to have proved that
all
initial models which can possibly be produced in the future are doomed to
contain
``junk''. But, as a matter of fact, all approaches of this kind, including
unified
algebras, rewriting logic \cite{c:87} and other models based on partial
orders, e.g. \cite{c:77},
do share this feature.
% It is reasonable to expect that a high degree of sophistication
% may be needed in
% order to cope with the problems indicated by the above examples. 
%Even if one succeeds
% in eliminating junk, the result may lack the simplicity of the 
% original initial semantics of deterministic specifications.
Concluding our examples we can just repeat the sequence of ``implications'':
\begin{center}
nondeterminism $+$ initiality\  $\leadto$\  underspecification $+$
initiality\ $\leadto$\  junk.
\end{center}
The examples indicate that the amount of redundancy is less in
multialgebras than in
the other models. However, the existence of initial multialgebras is
guarded by
 restrictive conditions. Unified algebras specified with Horn formulae always
 possess initial models, but the occurrence, and the character, of redundant
elements
in such models is difficult for a human being to determine from the
specification
text alone.
Initiality is certainly a virtue, but an even greater virtue is the
adequacy of
the model. It seems that, in the case of nondeterminism, the intuitive
appeal of the ``no junk'' property
of initial models gets lost. Even worse --
initiality becomes a powerful ally of intuitive junk in the semantics of
nondeterministic specifications.

\section{Disjunction \& Quasi-Initiality}
The constructions motivated by the search for the initial semantics attempt
to indicate that $c$ is equal to $a$ or $b$, as is the intention of the
specification,
without actually identifying $c$ with any of the two: the source of
``intuitive junk''
in all the
examples was the underspecified element $c$. One wants to model the fact that
$c=a$ or $c=b$ without
spelling it clearly out.  This fear originates, of course, in the knowledge
that
disjunctive equations do not, in general, allow initial models.

The source of evil lies in the fact that nondeterminism allows us to
underspecify $c$, while one still insists on using initial models of such
specifications.
As exemplified by the initial model of $SP$ in the beginning of section
\ref{se:junk}, deterministic
underspecifications give rise to similar problems. However, in the
deterministic
case underspecification is an optional tool for handling exceptional
situations rather
than the standard procedure. Typically, one tries to apply some strategy,
such as
generator induction \cite{c:28, c:44,c:60}, %[1,2,5],
resulting in specifications with intended initial models.
Nondeterminism, on the other hand, implies disjunction and, with it,
underspecification.

Another drawback is that the nondeterministic operations themselves
 are not adequately specified. In all  cases, underspecification of the
 $c$ element, combined with initiality,
destroyed $a\ch b$ as binary choice and turned it into an operation
capable of
returning, besides $a$ and $b$, some $c$. These observations make us argue
that making a
clear distinction between nondeterminism and underspecification and admitting
disjunctive formulae, such as in $SPN$ in \ref{sub:multA}, to handle the
latter is
more advantageous than it is dangerous.
Above all, it allows us to restore the intended simplicity of the model.

Introducing the lattice structure in unified algebras, one avoids
disjunction and
models it with the help of the join operation. It might seem that, since
disjunction
is the join operation (in the Boolean algebra of propositions), the two
will lead to
similar results. However, the obvious differences are quite significant:
\begin{center}
\begin{tabular}{c@{\ \ }|@{\ \ }c}
Lattice/Joins & Disjunctive specification \\ \hline
all joins must be present & only relevant disjunctions are included \\
introduces ``junk'' & eliminates ``junk'' \\
allows initial models & disallows initial models
\end{tabular}
\end{center}
%\noindent
The first point refers to the fact that in the lattice model join is an
operation (choice) with predefined semantics.  If there are several
constants in the specification $SPU$, say $d$ besides $a, b$ and $c$, then
unified algebra is forced to interpret the element $a\ch d$ as join.  But
if we are interested in a binary operation which is only some variation of
choice, e.g., chooses nondeterministically merely between $a$ and $b$, and
for other arguments behaves as, say first projection, then we are forced to
relate it to the primitive choice operation.  $a\ch d$ will always be
there.  Disjunction allows us to specify different kinds of
nondeterministic operations without any reference to some particular,
basic form of
nondeterminism.

So, we may treat disjunction as a replacement of the join operation. It
will not
lead to the lattice models but, instead, to the standard algebraic models,
that is, {\em multimodels\/}. The most
important advantage of this is that it actually removes the unintended
``junk''
and produces the models (in the initial coverings) which have an intuitive
clarity
comparable to initial models of deterministic specifications. We could
(under)specify
our example as it was done in $SPN$ in \ref{sub:multA}.
To make a final comment on the issue of nondeterminism vs.
underspecification, let's
modify it slightly
\[\begin{array}{rrrr@{\ \into\ }l}
SPN2= & \Sorts: &\multicolumn{2}{l}{E} \\
 & \Funcs: & a,b,c: & & E \\
 &   & \ch: & E\times E & E \\
 & \Ax: & \multicolumn{3}{l}{
 \begin{array}[t]{rrcl@{\hspace*{4em}}rrcl}
1.& z\Incl x\ch y & \Impl & z\Eq x \lor z\Eq y  &  3.& a & \Eq & a  \\
2.& && c\Eq a  \lor  c\Eq b &  4.& b & \Eq & b  \\
&  &  &  &                5.& c & \Eq & c
\end{array} }
\end{array}
\]
% (Compare this to the axiom 2. from $SP$.)
% Substituting $d, a, b$ for $z,x,y$ and using axiom 1. we will get that
% $d$ equals $a$  or $b$.
Axiom 2. does not make $c$ a nondeterministic operation but merely
underspecifies it to be either $a$ or $b$.
Axiom 1., on the other hand, specifies binary choice by underspecifying
each particular {\em application}
of it. It reads: if $z$ is a result of (some) application of $x\ch y$ then
$z$ equals
$x$ or $y$.
The difference between an underspecified $c$ and a nondeterministic
operation $\ch$
is that the former has a unique value in every model, but these values  may
vary
from one model to another. The latter, on the other hand, may also return
different
values in one model.
$a\ch b$ may be thought of as a series of distinct applications, each
returning
either $a$ or $b$. That is, $c$ is underspecified, so to speak, once for
all, while it is every single {\em application} of $a\ch b$ that
is underspecified! Thus
\begin{center}
\begin{tabular}{r@{\ \ }c@{\ \ }l}
nondeterminism & $\not=$ & underspecification \\
nondeterminism & $=$  & underspecification of each application
\end{tabular}
\end{center}
% The difference between $d$ and $c$ is that $c$ is underspecified
% independently from
% $\ch$, while possible interpretations of $d$ in a given model are
% determined by the
% interpretation of $\ch$.
%Notice that the choice operator is actually not needed at all. Disjunction
% allows
%us to distinguish c as underspecified in SPD from d as a result returned by
%nondeterministic choice in SPD1 below (although we are not going into the
%details here - see [15]).
%
The price is, of course, that $SPN2$ no longer has an  initial model.
 Nevertheless, disjunction does not spoil
initiality completely. We have shown that, in the presence of mild
restrictions,
disjunctive specifications do possess quasi-initial multialgebra semantics
\cite{c:127, c:133}, %[15].
namely:
\begin{CLAIM}
Given a specification $SP$ as in definition~\ref{de:mult}.
If for every ground term $t$, there is a deterministic ground term $s$ such
that
$SP\models s\Incl t$, then $SP$ has a quasi-initial semantics.
\end{CLAIM}
Quasi-initiality generalizes initiality to situations involving
``either ... or ...'' \cite{c:137}. For $SPN2$, such a semantics can be
summarized by saying
that the model class will consist of two parts: the models where $c$ equals
$a$ and those where $c$ equals $b$; $a\ch b$ will always contain,
at least, both $a$ and $b$. The quasi-initial semantics of $SPN2$ is given
by the
following two multimodels which are  initial in the
respective components of the model class:
\[\begin{array}{c|cc}
 & A & B  \\ \hline
a & a & a  \\
b & b & b  \\
a\ch b & \{a,b\} & \{a,b\} \\
c  & a & b
\end{array}
\]

\section{Conclusions}
We have argued that nondeterminism provides adequate means for specifying
operations on inherently unordered structures. In particular,
nondeterministic choice is the natural operation for performing iteration
over sets. In such context, deterministic specifications are bound to
overspecify the operations and nondeterminism is simply the adequate
abstraction mechanism.

The traditional abstraction mechanism -- underspecification -- is
different from nondeterminism and cannot perform its function.
The former amounts to underspecifying a given operation requiring,
nevertheless, its result to be unique in every model. Nondeterminism, on
the other hand, allows one operation to return different results within
the same model. It can be thought of as underspecification not of an
operation itself but of its particular applications.

We argued that the suggested initial semantics of
nondeterminism, although resulting possibly in elegant mathematical
models, suffer from intuitive flaws -- in the context of
nondeterminsm such a semantics does not have intuitive purity and appeal
which made it so convincing in the case of deterministic specifications.
In particular, it loses the connection between the argument set and the
results which are supposed to come from this set. This real, and
intuitive, connection is the crux of nondeterminism and it seems to be
most adequately represented by multialgebras. The price is that one can
no longer obtain initial semantics in general situation. As a possible
alternative, we indicated the quasi-initial semantics.

\bibliographystyle{bibNo}
\bibliography{my}


\end{document}

REFERENCES
[1]	Dahl, O.J., Verifiable Programming,  Prentice Hall, 1992.
[2]	Guttag, J.V., The Specification and Application to Programming of
ADT, Tech. Rep., CSR6-59, Computer Systems Research Group, University of
Toronto 1975.
[3]	Hesselink, W.H., ``A Mathematical Approach to Nondeterminism in
Data Types'', ACM: Transactions on Programming Languages and Systems, vol.
10, 1988.
[4]	Hu�mann, H., Nondeterministic Algebraic Specifications, Ph.D.
thesis, Fakult�t f�r Mathematik und Informatik, Universit�t Passau, 1990.
[5]	Huet, G., Hullot, J., ``Proofs by Induction in Equational Theories
with Constructors'', JCSS, vol. 25, 239-266, 1981.
[6]	Kaplan, S., ``Rewriting with a Nondeterministic Choice Operator'',
Theoretical Computer Science, vol. 56, 37-57, 1988.
[7]	Kapur, D., Towards a theory of abstract data types, Ph.D. thesis,
Laboratory for CS, MIT, 1980.
[8]	Mahr, B., Makowsky, J.A., ``Characterizing specification languages
which admit initial semantics'', in Proc. 8th CAAP, vol. 159, LNCS,
Springer, 1983, pp. 300-316.
[9]	Maibaum, T.S.E., The Semantics of Nondeterminism, Tech. Rep.,
CS-77-30, University of Waterloo, Ontario, Canada December 1977.
[10]	Makowsky, J.A., ``Why Horn Formulas Matter in Computer Science'',
Journal of Computer and System Science, vol. 34, 266-292, 1987.
[11]	Mosses, P.D., ``Unified Algebras and Action Semantics'', in
STACS'89, vol. 349, LNCS, Springer, 1989.
[12]	Mosses, P.D., Unified Algebras and Institutions, Tech. Rep., DAIMI
PB-274, CS Dep., Aarhus University 1989.
[13]	Subrahmanyam, P.A., ``Nondeterminism in Abstract Data Types'', in
Automata, Languages and Programming, vol. 115, LNCS, Springer, 1981.
[14]	Tarlecki, A., ``Free constructions in algebraic institutions'', in
Mathematical Foundation of Computer Science '84, vol. 176, LNCS, Springer,
1984.
[15]	Walicki, M., Meldal, S., ``Equality in Specifications of
Nondeterminism'' [submitted for publication]; also available in Walicki,
M., Calculii for nondeterministic specifications: three completeness
results, Tech. Rep., 75, Institutt for Informatikk, Universitetet i Bergen
December 1992.



\end{document}






