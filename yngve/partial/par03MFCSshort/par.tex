\documentclass[10pt]{article}
\usepackage[all]{xy}
\usepackage{latexsym}

\makeatletter

\ifcase \@ptsize
    % mods for 10 pt
    \oddsidemargin  0.15 in     %   Left margin on odd-numbered pages.
    \evensidemargin 0.35 in     %   Left margin on even-numbered pages.
    \marginparwidth 1 in        %   Width of marginal notes.
    \oddsidemargin 0.25 in      %   Note that \oddsidemargin = \evensidemargin
    \evensidemargin 0.25 in
    \marginparwidth 0.75 in
    \textwidth 5.875 in % Width of text line.
\or % mods for 11 pt
    \oddsidemargin 0.1 in      %   Left margin on odd-numbered pages.
    \evensidemargin 0.15 in    %   Left margin on even-numbered pages.
    \marginparwidth 1 in       %   Width of marginal notes.
    \oddsidemargin 0.125 in    %   Note that \oddsidemargin = \evensidemargin
    \evensidemargin 0.125 in
    \marginparwidth 0.75 in
    \textwidth 6.125 in % Width of text line.
\or % mods for 12 pt
    \oddsidemargin -10 pt      %   Left margin on odd-numbered pages.
    \evensidemargin 10 pt      %   Left margin on even-numbered pages.
    \marginparwidth 1 in       %   Width of marginal notes.
    \oddsidemargin 0 in      %   Note that \oddsidemargin = \evensidemargin
    \evensidemargin 0 in
    \marginparwidth 0.75 in
    \textwidth 6.375 true in % Width of text line.
\fi

\voffset -2cm
\textheight 22.5cm

\makeatother

%\makeatletter
%\show\
%\makeatother
\newcommand{\ite}[1]{\item[{\bf #1.}]}
\newcommand{\app}{\mathrel{\scriptscriptstyle{\vdash}}}
\newcommand{\estr}{\varepsilon}
\newcommand{\PSet}[1]{{\cal P}(#1)}
\newcommand{\ch}{\sqcup}
\newcommand{\into}{\to}
\newcommand{\Iff}{\Leftrightarrow}
\renewcommand{\iff}{\leftrightarrow}
\newcommand{\prI}{\vdash_I}
\newcommand{\pr}{\vdash}
\newcommand{\ovr}[1]{\overline{#1}}

\newcommand{\cp}{{\cal O}}

% update function/set
%\newcommand{\upd}[3]{#1\!\Rsh^{#2}_{\!\!#3}} % AMS
\newcommand{\upd}[3]{#1^{\raisebox{.5ex}{\mbox{${\scriptscriptstyle{\leftarrow}}\scriptstyle{#3}$}}}_{{\scriptscriptstyle{\rightarrow}}{#2}}} 
\newcommand{\rem}[2]{\upd{#1}{#2}{\bullet}}
\newcommand{\add}[2]{\upd {#1}{\bullet}{#2}}
%\newcommand{\mv}[3]{{#1}\!\Rsh_{\!\!#3}{#2}}
\newcommand{\mv}[3]{{#1}\:\raisebox{-.5ex}{$\stackrel{\displaystyle\curvearrowright}{\scriptstyle{#3}}$}\:{#2}}

\newcommand{\leads}{\rightsquigarrow} %AMS

\newenvironment{ites}{\vspace*{1ex}\par\noindent 
   \begin{tabular}{r@{\ \ }rcl}}{\vspace*{1ex}\end{tabular}\par\noindent}
\newcommand{\itt}[3]{{\bf #1.} & $#2$ & $\impl$ & $#3$ \\[1ex]}
\newcommand{\itte}[3]{{\bf #1.} & $#2$ & $\impl$ & $#3$ }
\newcommand{\itteq}[3]{\hline {\bf #1} & & & $#2=#3$ }
\newcommand{\itteqc}[3]{\hline {\bf #1} &  &  & $#2=#3$ \\[.5ex]}
\newcommand{\itteqq}[3]{{\bf #1} &  &  & $#2=#3$ }
\newcommand{\itc}[2]{{\bf #1.} & $#2$ &    \\[.5ex]}
\newcommand{\itcs}[3]{{\bf #1.} & $#2$ & $\impl$ & $#3$  \\[.5ex] }
\newcommand{\itco}[3]{   & $#1$ & $#2$  & $#3$ \\[1ex]}
\newcommand{\itcoe}[3]{   & $#1$ & $#2$  & $#3$}
\newcommand{\bit}{\begin{ites}}
\newcommand{\eit}{\end{ites}}
\newcommand{\na}[1]{{\bf #1.}}
\newenvironment{iten}{\begin{tabular}[t]{r@{\ }rcl}}{\end{tabular}}
\newcommand{\ass}[1]{& \multicolumn{3}{l}{\hspace*{-1em}{\small{[{\em Assuming:} #1]}}}}

%%%%%%%%% nested comp's
\newenvironment{itess}{\vspace*{1ex}\par\noindent 
   \begin{tabular}{r@{\ \ }lllcl}}{\vspace*{1ex}\end{tabular}\par\noindent}
\newcommand{\bitn}{\begin{itess}}
\newcommand{\eitn}{\end{itess}}
\newcommand{\comA}[2]{{\bf #1}& $#2$ \\ }
\newcommand{\comB}[3]{{\bf #1}& $#2$ & $#3$\\ }
\newcommand{\com}[3]{{\bf #1}& & & $#2$ & $\impl$ & $#3$\\[.5ex] }

\newcommand{\comS}[5]{{\bf #1} 
   & $#2$ & $#3$ & $#4$ & $\impl$ & $#5$\\[.5ex] }

%%%%%%%%%%%%%%%%
\newtheorem{CLAIM}{Proposition}[section]
\newtheorem{COROLLARY}[CLAIM]{Corollary}
\newtheorem{THEOREM}[CLAIM]{Theorem}
\newtheorem{LEMMA}[CLAIM]{Lemma}
\newcommand{\MyLPar}{\parsep -.2ex plus.2ex minus.2ex\itemsep\parsep
   \vspace{-\topsep}\vspace{.5ex}}
\newcommand{\MyNumEnv}[1]{\trivlist\refstepcounter{CLAIM}\item[\hskip
   \labelsep{\bf #1\ \theCLAIM\ }]\sf\ignorespaces}
\newenvironment{DEFINITION}{\MyNumEnv{Definition}}{\par\addvspace{0.5ex}}
\newenvironment{EXAMPLE}{\MyNumEnv{Example}}{\nopagebreak\finish}
\newenvironment{PROOF}{{\bf Proof.}}{\nopagebreak\finish}
\newcommand{\finish}{\hspace*{\fill}\nopagebreak 
     \raisebox{-1ex}{$\Box$}\hspace*{1em}\par\addvspace{1ex}}
\renewcommand{\abstract}[1]{ \begin{quote}\noindent \small {\bf Abstract.} #1
    \end{quote}}
\newcommand{\B}[1]{{\rm I\hspace{-.2em}#1}}
\newcommand{\Nat}{{\B N}}
\newcommand{\bool}{{\cal B}{\rm ool}}
\renewcommand{\c}[1]{{\cal #1}}
\newcommand{\Funcs}{{\cal F}}
%\newcommand{\Terms}{{\cal T}(\Funcs,\Vars)}
\newcommand{\Terms}[1]{{\cal T}(#1)}
\newcommand{\Vars}{{\cal V}}
\newcommand{\Incl}{\mathbin{\prec}}
\newcommand{\Cont}{\mathbin{\succ}}
\newcommand{\Int}{\mathbin{\frown}}
\newcommand{\Seteq}{\mathbin{\asymp}}
\newcommand{\Eq}{\mathbin{\approx}}
\newcommand{\notEq}{\mathbin{\Not\approx}}
\newcommand{\notIncl}{\mathbin{\Not\prec}}
\newcommand{\notCont}{\mathbin{\Not\succ}}
\newcommand{\notInt}{\mathbin{\Not\frown}}
\newcommand{\Seq}{\mathrel{\mapsto}}
\newcommand{\Ord}{\mathbin{\rightarrow}}
\newcommand{\M}[1]{\mathbin{\mathord{#1}^m}}
\newcommand{\Mset}[1]{{\cal M}(#1)}
\newcommand{\interpret}[1]{[\![#1]\!]^{A}_{\rho}}
\newcommand{\Interpret}[1]{[\![#1]\!]^{A}}
%\newcommand{\Comp}[2]{\mbox{\rm Comp}(#1,#2)}
\newcommand{\Comp}[2]{#1\diamond#2}
\newcommand{\Repl}[2]{\mbox{\rm Repl}(#1,#2)}
%\newcommand\SS[1]{{\cal S}^{#1}}
\newcommand{\To}[1]{\mathbin{\stackrel{#1}{\longrightarrow}}}
\newcommand{\TTo}[1]{\mathbin{\stackrel{#1}{\Longrightarrow}}}
\newcommand{\oT}[1]{\mathbin{\stackrel{#1}{\longleftarrow}}}
\newcommand{\oTT}[1]{\mathbin{\stackrel{#1}{\Longleftarrow}}}
\newcommand{\es}{\emptyset}
\newcommand{\C}[1]{\mbox{$\cal #1$}}
\newcommand{\Mb}[1]{\mbox{#1}}
\newcommand{\<}{\langle}
\renewcommand{\>}{\rangle}
\newcommand{\Def}{\mathrel{\stackrel{\mbox{\tiny def}}{=}}}
\newcommand{\impl}{\mathrel\Rightarrow}
\newcommand{\then}{\mathrel\Rightarrow}
\newfont{\msym}{msxm10}

\newcommand{\false}{\bot}
\newcommand{\true}{\top}

\newcommand{\restrict}{\mathbin{\mbox{\msym\symbol{22}}}}
\newcommand{\List}[3]{#1_{1}#3\ldots#3#1_{#2}}
\newcommand{\col}[1]{\renewcommand{\arraystretch}{0.4} \begin{array}[t]{c} #1
  \end{array}}
\newcommand{\prule}[2]{{\displaystyle #1 \over \displaystyle#2}}
\newcounter{ITEM}
\newcommand{\newITEM}[1]{\gdef\ITEMlabel{ITEM:#1-}\setcounter{ITEM}{0}}
\makeatletter
\newcommand{\Not}[1]{\mathbin {\mathpalette\c@ncel#1}}
\def\LabeL#1$#2{\edef\@currentlabel{#2}\label{#1}}
\newcommand{\ITEM}[2]{\par\addvspace{.7ex}\noindent
   \refstepcounter{ITEM}\expandafter\LabeL\ITEMlabel#1${(\roman{ITEM})}%
   {\advance\linewidth-2em \hskip2em %
   \parbox{\linewidth}{\hskip-2em {\rm\bf \@currentlabel\
   }\ignorespaces #2}}\par \addvspace{.7ex}\noindent\ignorespaces}
\def\R@f#1${\ref{#1}}
\newcommand{\?}[1]{\expandafter\R@f\ITEMlabel#1$}
\makeatother
\newcommand{\PROOFRULE}[2]{\trivlist\item[\hskip\labelsep {\bf #1}]#2\par
  \addvspace{1ex}\noindent\ignorespaces}
\newcommand{\PRULE}[2]{\displaystyle#1 \strut \over \strut \displaystyle#2}
%\setlength{\clauselength}{6cm}
%% \newcommand{\clause}[3]{\par\addvspace{.7ex}\noindent\LabeL#2${{\rm\bf #1}}%
%%   {\advance\linewidth-3em \hskip 3em
%%    \parbox{\linewidth}{\hskip-3em \parbox{3em}{\rm\bf#1.}#3}}\par 
%%    \addvspace{.7ex}\noindent\ignorespaces}
\newcommand{\clause}[3]{\par\addvspace{.7ex}\noindent
  {\advance\linewidth-3em \hskip 3em
   \parbox{\linewidth}{\hskip-3em \parbox{3em}{\rm\bf#1.}#3}}\par 
   \addvspace{.7ex}\noindent\ignorespaces}
\newcommand{\Cs}{\varepsilon}
\newcommand{\const}[3]{\Cs_{\scriptscriptstyle#2}(#1,#3)}
\newcommand{\Ein}{\sqsubset}%
\newcommand{\Eineq}{\sqsubseteq}%


\voffset -1cm


\begin{document}

\title{Partiality as nondeterminism}
\author{Yngve Lamo \and Micha{\l} Walicki}
%\date{{\small{Department of Informatics\\University of Bergen\\5020 Bergen,
%Norway\\$\{$yngvel,michal$\}$@ii.uib.no}}}
\date{{}}
\maketitle


\section{Introduction}
The problem of partial operations is of fundamental importance in specifying
and deriving programs. At the abstract level, one would like to avoid
detailed treatement of error situations.  However, at the level close to
actual implementation, it will often be mandatory to address explicitly
possible error situations and to describe the program's behaviour in such
situations. 

In the tradition of algebraic specifications, there are two main
approaches to describing partiality: the partial algebras
on the one hand \cite{burm,reichel,state}, and total algebras with explicit
definition domains and error elements,
on the other. This later approach comprises manifold variations, including
error-algebras \cite{error}, labeled algebras~\cite{label}, 
order-sorted algebras~\cite{order1}, techniques using predicates \cite{member} or
functions \cite{guard} to specify definition domains of operations.
%For a closer
%discussion of the traditional solutions see \cite{sort}.% and \cite{survey}.

In
partial algebras, an undefined term has no interpretation in the
carrier. During the refinement process, one makes terms gradually more
defined, by adding values for undefined terms. Terms which remain undefiend
til the very end of the specification process are then understood as errors whose
handling is left for the implementation. Partial algebras offer an
abstract model due to strictness assumption
which releases the specifier from the need to explicitly treat error
situations. This, however, turns out to be a drawback when the specification
approaches the implementation level. Strictness of all operations makes
explicit error handling 
difficult, if at all possible. Extensions of partial algebras to
handle this level of specification are indicated in \cite{state,lazy}. Typically,
they involve a translation of partial algebra
specifications into some total framework. % \cite{state}.

The approaches based on total algebras explore the possibilities of explicit
error handling, by using specific values as error elements. They are well
suited to describe detailed error handling. Their main drawback is that one
is from the very beginning forced to treat explicitly error elements.

We thus have, on the one hand, an abstract framework of partial algebras
and, on the other hand, the total algebra approaches forcing one to specify
error situations explicitly from the very start.
We are proposing a single framework capable of addressing both these
aspects. Following the idea originally expressed and formalised in \cite{calc},
we use nondeterminism in specification process as a means of {\em
abstraction} -- here, from the unknown results returned by partial operations.
We view undefinedness as nondeterminism -- a term whithout a
well-defined value may result in any value. At the early stages of
development, this expresses our ignorance or disinterest in
what exactly will happen in a given -- error -- situation. This preserves
the intuition that any operation in any actual run of a program actually
produces some -- even if uninteded and unexpected -- result. At the later development
stages, such nondeterminism may be narrowed leading, eventually, to explicit
error values and their treatement.

As a model of nondeterminism, we use multialgebras \cite{calc,multi}.
These are summarized in section~\ref{se:pre} which also gives the basic
definitions of mappings of institutions \cite{inst}.
Section~\ref{se:meth} shows on a simple example how the problems of partiality can be
addressed in $\inst{MA}$ -- the institution of multialgebras.  Since
multialgebras offer a non-strict (or rather, not necessarily strict)
framework, we also show how to deal with strictness, if it is desirable for
some reasons.  Section~\ref{se:partial} shows two particular mappings of the
institution of partial algebras into $\inst{MA}$ formalizing also the
intuition of reusing partial algebra specifications for furhter development
in a non-strict context. Similar possibilities of embedding are indicated for
other frameworks thus suggesting that $\inst{MA}$ may be a useful setting for
both {\em comparing} and {\em combining} advantages of different formalisms
for partiality. More detailed discussion as well as proofs can be found in \cite{TR}.

  
\section{Preliminaries}\label{se:pre}
Categories are written with the bold font $\cat{Cat}$,  functors with
Sans Serif $\fu{Func}$, and institutions with script $\inst{I}$.
$|A|$ denotes the carrier of an algebra $A$. Signature is usually written
$\Sigma$ and denotes a pair of sets
$(\Sorts,\Ops)$ with sort- and operation-symbols.
Sequences $s_1, \ldots, s_k$ are usually written
as $\overline{s}$. Application of functions are then understood to {\em not}
distribute over the elements, i.e., $f(\ovr s)$ denotes the term
$f(s_1,\ldots, s_k)$.  
%%%Occasionally, a sequence $s_1 \ldots s_k$ may be denoted by $s^*$ -- 
%%%applications of functions are then understood to
%%%distribute over the elements, i.e., $f(s^*)$ denotes the sequence
%%%$(f(s_1),\ldots, f(s_k))$. $f^A$ is the interpretation of the symbol $f$ in the (multi)algebra $A$

\subsubsection*{Institutions}

\begin{definition}
An {\em institution} \cite{inst} is a quadruple $\inst{I} = (\Sign,\Sen,\Mod,\models)$, where:
	\begin{itemize}\MyLPar
\item $\Sign$ is a category of signatures.  
\item $\Sen: \Sign \to \Set$ is a functor which associates a set of {\it sentences} to each signature.
\item $\Mod:\Sign^{op} \to \cat{Cat}$ is a functor which associates a
category of {\it models}, whose morphisms are called $\Sigma$-morphisms, to
each signature $\Sigma$
\item $\models$ is a satisfaction relation -- for each signature $\Sigma$, a relation
 $\models_{\Sigma} \subseteq |\Mod(\Sigma)| \times
\sen(\Sigma)$, such that the {\it satisfaction condition} holds:
for any $M' \in \Mod(\Sigma'), \mu: \Sigma \to \Sigma' , \phi \in \sen(\Sigma)$\vspace*{-1ex}
  \[ M' \models_{\Sigma'} \Sen(\mu)(\phi) \mbox{ iff } (\Mod(\mu))(M') \models_{\Sigma} \phi\]
	\end{itemize}
\end{definition}
%
We write: $\Gamma \models_{\Sigma}
 \varphi \mbox{ iff } \forall M \in \Mod(\Sigma): M \models_{\Sigma} \Gamma
 \Rightarrow M \models_{\Sigma} \varphi $.  
 $\Gamma^{\bullet}$ denotes the semantical consequences of $\Gamma$
 i.e. $\Gamma^{\bullet} = \{ \varphi: \Gamma \models \varphi \}$
 
A {\em theory} (specification) in an institution is a pair
$\thr{Th}=(\Sigma,\Gamma)$ where $\Sigma\in {\Sign}$ and $\Gamma\subseteq
\sen(\Sigma)$. For a given institution $\inst{I}$, we have the corresponding
category of theories $\cat{Th_{\inst{I}}}$ with theories as objects and
theory morphisms $\mu:(\Sigma,\Gamma) \to (\Sigma',\Gamma')$, where
$\mu:\Sigma \to \Sigma'$, is a signature morphism such that: $\Gamma'
\models_{\Sigma'} \sen(\mu)(\Gamma)$.  
There are
a canonic projection functor $\fu{sign}:\thr{Th} \to \cat{Sign}$ and 
an embedding functor $\fu{th}: \cat{Sign} \to \thr{Th}$ defined by
$\fu{th}(\Sigma)= (\Sigma,\emptyset)$.
The models for the theory
$\thr{Th}=(\Sigma,\Gamma)$ is the full subcategory
$\Mod(\Sigma,\Gamma)$ of $\Mod(\Sigma)$ where $M \in
\Mod(\Sigma,\Gamma) \mbox{ iff } M \models_{\Sigma} \varphi,
\forall \varphi \in \Gamma$. 
%, we will write $\Mod(\Sigma,\Gamma)$ instead of
%$\Mod_{\models}(\Sigma,\Gamma)$.  
%%The satisfaction condition gives that
%%$\Mod(\mu)(\Mod(\Sigma',\Gamma')) \subseteq$ % \thr{Th}
%%$(\Mod(\Sigma,\Gamma)$, for each morphism $\mu:(\Sigma,\Gamma) \to
%%(\Sigma',\Gamma') \in \thr{Th}$. This means that the functor $\Mod$ can be
%%extended to a functor $\Mod_{\models}:\cat{Th}^{op} \to \cat{Cat}$. 
 A theory morphism $\mu:(\Sigma,\Gamma)
\to (\Sigma',\Gamma')$ is {\em axiom conserving} if $\mu(\Gamma) \subseteq
\Gamma'$. This defines the subcategory $\thr{Th_0}$ with theories as objects
and axiom conserving theory morphisms as morphisms.

\begin{definition}
Let $\inst{I}= ({\Sign},\sen,\Mod,\models)$ and
$\inst{I'}=({\Sign'},\sen',\Mod',\models')$ be two institutions.
\begin{enumerate}\MyLPar
\item
Given a natural transformation $\alpha:\fu{Sen}
\to \fu{Sen'} \circ \fu{\Phi}$ and a functor $\fu{\Phi}:\cat{Sign} \to
\thr{Th_0'}$
an $\alpha$-extension to theories of $\fu{\Phi}$ is a functor $\fu{\Phi}:\thr{Th_0} \to \thr{Th_0'}$ mapping a
theory $\thr{Th}=(\Sigma,\Gamma)$ to the theory $\fu{\Phi}(\thr{Th})$ with
signature $\fu{\Phi}(\Sigma)$ and with axioms $\fu{\Phi}(\Sigma) \cup
\alpha_{\Sigma}(\Gamma)$
\item
Given a functor $\fu{\Phi}:\thr{Th_0} \to \thr{Th_0'}$ and a natural
transformation $\alpha: \sen \Rightarrow \sen' \circ \fu{\Phi}$, $\fu{\Phi}$ is
$\alpha$-sensible iff:
\begin{itemize}\MyLPar
	\item There is a functor $\fu{\Phi^{\diamond}}:\fu{Sign} \to \fu{Sign'}$ such that $\fu{sign'} \circ \fu{\Phi} =\fu{\Phi^\diamond}  \circ  \fu{sign}$
	\item $(\Gamma')^\bullet = (\emptyset'_{\Sigma} \cup \alpha_{\Sigma}(\Gamma))^{\bullet}$
\end{itemize}
where the set of axioms induced by $\fu{\Phi}(\Sigma)$ is denoted by
$\emptyset'_{\Sigma}$.
\end{enumerate}
\end{definition}
%
To relate different institutions we will use various maps of institutions.
\begin{definition}\label{de:maps}
Let $\inst{I}= ({\Sign},\sen,\Mod,\models)$ and
$\inst{I'}=({\Sign'},\sen',\Mod',\models')$ be two institutions.
\begin{enumerate}\MyLPar
\item 
A {\em map of institutions} \cite{general} is a triple  $(\fu{\Phi},\alpha,\beta): \inst{I} \to \inst{I'}$ where:
\begin{itemize}\MyLPar
	\item $\alpha: \sen \natt \sen' \circ \fu{\Phi}$ is a natural transformation.
	\item $\fu{\Phi}: \thr{Th_0} \to \thr{Th_0'}$ is an $\alpha$-sensible functor %, where
\item $\beta: \Mod' \circ \fu{\Phi}^{op} \natt \Mod$ is a natural transformation 
\end{itemize}
such that for each $\phi \in \sen(\Sigma)$ and $ M' \in
\Mod'(\fu{\Phi}(\Sigma,\emptyset))$ the following condition holds:\vspace*{-1ex}
\[ M' \models_{sign'(\fu{\Phi}(\Sigma,\emptyset))} \alpha_{\Sigma}(\phi)
\mbox{ iff } \beta_{(\Sigma,\emptyset)} (M') \models_{\Sigma} \phi \vspace*{-1ex}\]
\item
An {\em embedding of institutions} \cite{member} is a map of institutions, written
$(\fu{\Phi},\alpha,\beta): \inst{I} \embd \inst{I'}$, where the functor $\beta_T
:\Mod'(\fu{\Phi}(T)) \to \Mod(T)$ is an equivalence of categories for each $T \in
\cat{Th_{\inst{I}}}$. 
\item
A map of institutions $(\fu{\Phi},\alpha,\beta)$ is:
	\begin{itemize}\MyLPar
		\item  $(\alpha)$ {\em simple} iff $\fu{\Phi}$ is an $\alpha$-extension to theories of a functor $\fu{F}:\cat{Sign} \to \thr{Th_0'}$, i.e. $\fu{\Phi}$ maps axioms to axioms.
		\item $(\alpha)$ {\em plain} iff $\fu{\Phi}$ is an $\alpha$-extension to theories of a functor $\fu{F}:\cat{Sign} \to \thr{Th_0'}$ that maps $\Sigma$ to $(\Sigma',\emptyset)$, i.e. $\fu{\Phi}$ maps signatures to signatures.
	\end{itemize}
\item
A {\em subinstitution} \cite{general} is a map of institutions
$(\fu{\Phi},\alpha,\beta):\inst I \to \inst I'$ that is plain, with $\fu{\Phi}$
faithful and injective on objects, $\alpha$ injective and with $\beta$ a
natural isomorphism.
\item\label{it:trans}
An {\em institution transformation} \cite{system} is a triple  $(\fu{\Phi},\alpha,\beta): \inst{I} \to \inst{I'}$ where:
\begin{itemize}\MyLPar
	\item $\alpha: \sen \natt \sen' \circ \fu{\Phi}$ is a natural transformation.
	\item $\fu{\Phi}: \thr{Th_0} \to \thr{Th_0'}$ is an $\alpha$ plain, $\alpha$-sensible functor %, where
%	$\thr{Th_0}=(\Sigma,\Gamma),\thr{Th'_0}=(\Sigma',\Gamma')$ 
\item a natural transformation $\beta: \Mod \natt \Mod' \circ \fu{\Phi}^{op}$
\end{itemize}
such that for each $\phi \in \sen(\Sigma)$ and $ M \in
\Mod(\Sigma,\emptyset)$ the following condition holds:
\[ \beta_{(\Sigma,\emptyset)}(M) \models_{sign'(\fu{\Phi}(\Sigma,\emptyset))}
\alpha_{\Sigma}(\phi) \mbox{ iff } M \models_{\Sigma} \phi \]
\end{enumerate}
\end{definition}

\subsubsection*{Multialgebras}
For an overview of multialgebras see \cite{multi,catrel}). 
Multialgebraic signatures  (and signature morphisms) are the same as in the
classical case and form the category $\Sign$.
Also terms, $\TermsSX$, are defined in the usual way. 
Signature morphisms are extended to terms in the canonical way.
A multialgebra is an algebra where operations may be
set-valued. ${\mathcal{P}}(y)$ denotes the powerset of set $y$.

\begin{definition}\label{def:ma}
Given a signature $\Sigma=(\Sorts,\Ops)$, a $\Sigma$-{\em multialgebra} $A$ is given by:
\begin{itemize}\MyLPar
\item  a set $s^A$, the carrier set, for each sort symbol $s\in\Sorts$
\item  a subset $ c^A \in {\mathcal{P}}(s^A)$, for each constant, $c:\to s$
\item an operation $\omega^A : s_1^A \times \cdots \times s_k^A \to {\mathcal{P}}(s^A)$
	for each symbol $\omega : s_1\times\cdots\times s_k \to s \in \Ops$
\end{itemize}
Operations are defined on sets by pointwise extension.
\end{definition}
%%One sometimes demands that constants and operations are total
%%\cite{calc,toplas}, i.e. never return empty set and take values only in
%%${\mathcal{P}}^+(s^A)$, the nonempty subsets of $s^A$. We will not make this
%%assumption.% unless else is stated.
Note that for a constant (or any ground term) $c$, $c^A$ denotes a subset of the
carrier $s^A$. This allows us to use constants as predicates and to refine
such predicates to individual values (denoted by terms) in a way not requiring any additional
transition between the two. 

As homomorphisms we use weak homomorphisms (see
\cite{catrel} for alternative notions).
\begin{definition}Given two $\Sigma$-multialgebras $A$ and $B$, 
a function $h: |A| \to |B|$ is a (weak) homomorphism if:
\begin{enumerate}\MyLPar
\item $h(c^A) \subseteq c^B$, for each constant $c: \to s$ 
\item $h(\omega^A(a_1, \ldots ,a_n)) \subseteq\omega^B(h(a_1), \ldots ,h(a_n))$,
for each operation  $\omega:s_1\times\cdots\times s_n\to s\in\Ops$ and for all $a_i \in s_{i}^A$.
\end{enumerate}
\end{definition}
Multialgebraic specifications are written using the following formulae:
%
\begin{definition}
Formulae of multialgebraic specifications are of the following forms ($t,t'\in\TermsSX$):
%{\bf{Atomic formula}}:
\begin{enumerate}\MyLPar
\item Atomic formulae:
\begin{itemize}\MyLPar
\item $t \eleq  t'$ (equality), $t$ and $t'$ denote the same one-element set.
\item $t \prec t'$ (inclusion), the set interpreting $t$ is included in
the set interpreting $t'$.
\end{itemize}
\item $a_1, \ldots ,a_n \To b_1, \ldots ,b_m$, where either $n>0$ or $m>0$ and
each $a_i$ and $b_j$ is atomic.
\end{enumerate}
%The first two are called atomic formulae.
\end{definition}
%
Given a set of variables $X$, an assignment is a function $\alpha: X \to |A|$ 
assigning {\em individual\/} elements of the carrier to variables. It
induces a unique
interpretation ${\alpha}: T(\Sigma(X)) \to A$ of terms
(with variables from $X$) in $A$. 

\begin{definition}\label{de:sat}
[Satisfaction] Given an assignment $\alpha:X\to|A|$:
\begin{enumerate}\MyLPar
\item $A \models_\alpha t \eleq  t'\ {\rm iff\ } {\alpha}(t)
={\alpha}(t') = \{e\},\ {for\ some\ }{\rm individual\ value\ }e \in |A|$
\item $A \models_\alpha t \prec t'\ {\rm iff\ } {\alpha}(t) \subseteq {\alpha}(t')$
\item $ A \models_\alpha a_1, \ldots ,a_n \To b_1, \ldots ,b_m\ {\rm iff\ }
\exists i: 1 \leq i \leq n : A \not\models_\alpha a_i\ {\rm or\ } \exists j: 1 \leq j \leq m : A \models_\alpha b_j$
\item $A \models \varphi\ {\rm iff\ } A \models_\alpha \varphi\ {\rm for\ all\ } \alpha$
\end{enumerate}
\end{definition}
Notice that $\eleq$ denotes {\em deterministic} equality -- the two sets
must contain exactly one (the same) element.
%
Multialgebras with weak homomorphismsm form an institution
$\inst{MA}$. 
There exist sound and complete calculii for
multialgebraic specifications: for multialgebras without operations returning
empty set \cite{toplas,calc}, and for the ones admitting empty result sets
(like in def.~\ref{def:ma}) \cite{BK}.


%\setcounter{section}{2}

\section{Developing specifications in $\inst{MA}$}
\label{se:meth}
We illustrate the possibilities of development in $\inst{MA}$
from abstract specifications analogous to partial-algebra specifications,
through a series of refinement steps, to specifications with explicit error
handling. At the intial, most abstract level, \ref{subsub:pa1}, error situations are not
addressed at all. Operations known to be total are specified as
deterministic, while others remain underspecified which allows, in
particular, for their nondeterministic interpretation. At the next stage, \ref{subsub:er2},
error situations are identified and we indicate several possible ways to do
that. Then one can begin explict error handling, first, \ref{subsub:er3}, by specifying the
behaviour of other operations in error situations and, eventually,
\ref{subsub:er4}, by
introducing explicit error values. A great flexibility of error treatement is
offered enabling one to introduce error values, exceptions and various ways
of reacting to them.

We emphasize that the whole
process is merely a gradual refinement of the initial specification by
extending its signature and the set of axioms. ({\bf det} abbreviates axioms
of the form $t\eleq t$.)

\subsection{Initial specification}\label{subsub:pa1}
We start with a standard specification not addressing any
error situations explicitly.
\[ \spec{
	\tit{\mbox{\bf spec\ Stack}^{\inst{MA}}=} \\
		{\bf include}
			&& {\bf Nat}\\
		\Sorts: && Stack\\
		\Ops:   && empty: \to Stack\\
			&& top:Stack \to Nat\\
			&& pop:Stack \to Stack\\
			&& push:Nat \times Stack \to Stack\\
		{\bf det:} && empty \\
		{\bf axioms:}			
			&1.& push(x,s) \eleq push(x,s) \To top(push(x,s)) \eleq x\\
			&2.& push(x,s) \eleq push(x,s) \To pop(push(x,s)) \eleq s
}
\]
At this level, only $empty$ is explicitly specified to be
deterministic.  Axioms 1. and 2. are the ``usual'' stack axioms. They
illustrate the general strategy of
guarding which requires well-defined (deterministic) results only on
arguments which are well-defined (deterministic).
%%This is the way we will do it in general -- the outermost operation in a
%%composition of functions will be guarded by a deterministic assertion. The
%%way axiom 2. should be understood is: if a stack $s'$ is constructed from $push(x,s)$
%%and $s'$ is defined, then  $pop(s')$ is defined and equal to $s$.
Since variables only range over individual elements, we need no
additional guards of the form $s\eleq s$. Also, 
the models for the specification may display flexible behaviour on nondeterministic values
resulting from $push$. 

Notice
that this is essentially the same as a partial
algebra specification -- just replace the sign $\eleq$ by $\eeq$. 
Reuse of partial algebra specifications in our framework is
discussed in section~\ref{se:partial}.

\subsection{Error situations}\label{subsub:er2}
Error situations may be identified and marked by appropriate error constants:
\vspace*{1ex}

%\hspace*{1.2em}
\(
1. \spec{	\tit{\mbox{\bf spec\ ErrStack1}^{\inst{MA}}=} \\
	\tit{\mbox{\bf enrich\ Stack\  by:}} \\
	    	\Ops:	&& errStack: \to Stack\\
		        && errNat: \to Nat \\
	 	{\bf axioms:}
			&1.& pop(empty) \prec errStack\\
			&2.& top(empty) \prec errNat
}
\) 

\noindent
Such error constants do not have any
direct influence on the semantics; they function mainly as labels visualizing
the special situations:
$pop(empty)$ may still be deterministic or not -- it has only been marked as a
special term ``of type'' $errStack$. 

Instead of introducing explicit error constants, one may indicate error
situations by forcing respective terms to be nondeterministic:
\vspace*{1ex}

\(
2. \spec{	\tit{\mbox{\bf spec\ ErrStack2}^{\inst{MA}}=} \\
	\tit{\mbox{\bf enrich\ Stack\  by:}} \\
	 	{\bf axioms:}
			&1.& pop(empty) \eleq pop(empty) \To\\
			&2.& top(empty) \eleq top(empty) \To
}
\)  

\noindent
The inequalities ensure that the error terms can't be interpreted as
individuals of the carrier; they must be sets -- possibly empty. (The
associated logic prevents then one from substituting such terms for
variables.) This precludes later treatement of particular error
situations by means of deterministic error constants.  Still, it offers
significant flexibility to be illustrated
in~\ref{subsub:er4}.

Further refinement of these two possibilities will take full advantage of
non-strict semantics. Notice that, so far, no specific error treatement has
been described -- $pop(empty)$ is merely
labeled in the first case and made nondeterministic in the second.

Finally, one can follow the order-sorted approach with constants for
appropriate subsorts:
\vspace*{1ex}

\( 
3. \spec{
	\tit{\mbox{\bf spec\ OsStack}^{\inst{MA}}=} \\
	\tit{\mbox{\bf enrich\ Stack\  by:}} \\
	    	\Ops:	&& nonempty: \to Stack\\
	 	{\bf axioms:}
			&1.& s \prec nonempty \To top(s) \eleq top(s)\\
			&2.& s \prec nonempty \To pop(s) \eleq pop(s)\\
			&3.& empty \prec nonempty \To	\\
			&4.& push(x,s)\eleq push(x,s) \To push(x,s)\prec nonempty
}
\)

\noindent
The new constant $nonempty$ is used as a subsort of non-empty stacks (by
axiom 3. $empty$ does not belong to this subsort) for
which $pop$ and $top$ are defined. 
%Note that adding ``strictness' axioms like:
%$pop(s)\eleq pop(s)\To s\prec nonempty$, would preclude the possibility of
%deterministic error recovery at a later stage -- the axiom requires the result
%of $pop(empty)$ to be nondeterministic (cf. {\bf ErrStack2} above).

%The order-sorted approach may be naturally combined with the explicit error
%constants, e.g., this specification may extend {\bf ErrStack1}.

\subsection{Behaviour on errors}\label{subsub:er3}
Error constants, like those introduced in {\bf ErrStack1}, may be used for a
uniform specification of behaviour of other operations on all elements (error
values) included in the constants: \vspace*{1ex}

%\hspace*{1.2em}
\(
1. \spec{	\tit{\mbox{\bf spec\ ErrStack1a}^{\inst{MA}}=} \\
	\tit{\mbox{\bf enrich\ ErrStack1\  by:}} \\
	 	{\bf axioms:}
			&1.& pop(push(x,errStack))\eleq empty \\
			&2.& top(push(x,errStack))\eleq x \\
			&3.& pop(push(errNat,s))\prec pop(s), s\prec errStack \\
			&4.& top(push(errNat,s))\prec top(s), s\prec errStack 
}
\) %\vspace*{1ex}

\noindent
In the first two axioms, the prescribed results are always well-defined. If
the first argument happens to be $errNat$, axiom 1. will give $empty$ and
axiom 2. will result in $errNat$ (whether it happens to be determnistic or
not).
The last two axioms use $\prec$ and not $\eleq$. This is so
because the terms on the right-hand-side may, possibly, be error terms (when
$s$ is $empty$). The alternatives give precedence to
$errStack$ over $errNat$. If both arguments are $err$, axiom 1.,
respectively 2., will be applied -- instead of axiom 3., axiom 1. will yield $empty$
as the result of $pop$. Similarly in axiom 4.

Another possibility is to treat each error situation separately. The
following specification refines {\bf ErrStack2} but it might as well be a
direct refinement of {\bf ErrStack1}: \vspace*{1ex}

\(
2. \spec{	\tit{\mbox{\bf spec\ ErrStack2a}^{\inst{MA}}=} \\
	\tit{\mbox{\bf enrich\ ErrStack2\  by:}} \\
	 	{\bf axioms:}
			&1.& pop(push(x,pop(empty)))\eleq empty \\
			&2.& top(push(x,pop(empty)))\eleq x \\
			&3.& pop(push(top(empty),s))\prec pop(s), s\prec pop(empty) \\
			&4.& top(push(top(empty),s))\prec top(s), s\prec pop(empty) 
}
\)

\noindent
At this stage we have still not determined any specific error values but only 
the behaviour of other operations applied in situations identified so far as
errors 
-- $errStack$ may later be specified to be a particular value or else remain nondeterministic.

The most dramatic possibility is to delegate all responsibility for error
treatement to the implementation. Unlike in approaches with implicit
assumptions about (non)strictness, here this is a decision to be made explicitly by the
specifier. 
Explicit specification which excludes further
description of error makes the result of respective error terms empty (this
can be read as a requirement of raising a run-time exception): \vspace*{1ex}

\(
3. \spec{	\tit{\mbox{\bf spec\ ErrStack3}^{\inst{MA}}=} \\
	\tit{\mbox{\bf enrich\ ErrStack2\  by:}} \\
	 	{\bf axioms:}
			&1.& s \prec pop(empty) \To  \\
			&2.& x \prec top(empty) \To \\
			&3.& s \prec push(errNat,s') \To
}
\)

\noindent
This implies strictness of other operations applied to these
arguments, since empty argument will always lead to empty result.  (For
instance, $push(x,pop(empty))$ will now return the empty set, too.) Making
analogous extension of {\bf ErrStack1} is possible, though it does not seem
quite purposeful -- labeling error situtations will, typically, involve later
their explicit treatement.

\subsection{Error values}\label{subsub:er4}
We now arrive at the lowest level and introduce explicit error values.
%\hspace*{1.2em}

\(
1. \spec{	\tit{\mbox{\bf spec\ ErrValStack1}^{\inst{MA}}=} \\
	\tit{\mbox{\bf enrich\ ErrStack1a\  by:}} \\
	 	{\bf axioms:}
			&1.& pop(empty) \eleq empty \\
			&2.& top(empty) \eleq errNat \\
			&3.& push(errNat,s) \eleq s
}
\) 
%\vspace*{2ex}

\noindent
This is apparently consistent with the intention of the
behaviour from {\bf ErrStack1a} (according to first two axioms from {\bf
ErrStack1a}, $errStack$ behaves as $empty$, and according to the last two,
pushing $errNat$ on $s$ behaves then as $s$). However, there are serious
problems with this refinement.

A counter-intuitive consequence of this enirchment is that $empty\prec
errStack$ by axiom 1. from {\bf ErrStack1}. Now axiom 1. says that, no
matter what was earlier said about the error si\-tu\-at\-ion $pop(empty)$, it
can be disregarded and that we do, instead, immediate error recovery. 
Since $pop(empty)$ has been earlier
identified as an error $errStack$, a more plausible refinement would be to
identify this
error value which can take care of a possible indication of the error
situation.

%%%Another problem might occur from making $errNat$ deterministic. If it is a
%%%constant introduced in this specification (as it happened in our example),
%%%this might be ok. However, if this error constant comes from another
%%%specification {\bf Nat}, we shouldn't force additional restriction since
%%%these may interfere with its specification elsewhere. 

The really serious problem is caused by the last axiom. It makes $push(errNat,s)$
deterministic. Thus, since $errNat$ is deterministic (axiom 2), we can substitute it into
axioms from {\bf Stack} and may conclude that
$top(push(errNat,s))\eleq errNat$. However, according to  {\bf ErrStack1a},
if $s\not\prec errStack$ we have that $top(push(errNat,s))\prec
top(s)$. Thus, this may lead to collapsing the sort of elements (here $Nat$).

Forcing some error terms to be deterministic requires revisiting
earlier specifications and checking for such unintended
coincidences. The uniform way of introducing errors, which is safe, is to
force errors to be sets. The following specification makes $pop(empty)$ a new
deterministic constants (this is ok in this example), but illustrates this
generally recommended way by axiom 3.
%\vspace*{1ex}

\(
2. \spec{	\tit{\mbox{\bf spec\ ErrValStack2}^{\inst{MA}}=} \\
	\tit{\mbox{\bf enrich\ ErrStack1a\  by:}} \\
	    	\Ops:	&& popEmpty: \to Stack\\
	 	{\bf axioms:}
			&1.& pop(empty) \eleq popEmpty \\
			&2.& top(empty) \prec errNat \\
			&3.& push(errNat,s)\eleq push(errNat,s) \To
}
\)

\noindent
We are forcing $push(errNat,s)$ to be a set in order to avoid interference
with the axioms of {\bf Stack}. The axioms from {\bf ErrStack1a} prescribe
recovery from this situation. This strategy does {\em not} require revisiting
earlier specifications -- it ensures that newly identified error sets will
not interfere with earlier assuumptions about results prodused on defined
(deterministic) values.

The last remaining question concerns the actual value to be returned by
$push(errNat,s)$. We  specify it is a set including:
some
error value, which does not have any influence on the other operations (axiom
2.), and
the value to be used for the recovery purpose (axiom 1.):

\(
3. \spec{	\tit{\mbox{\bf spec\ ErrValStack3}^{\inst{MA}}=} \\
	\tit{\mbox{\bf enrich\ ErrStack2\  by:}} \\
	    	\Ops:	&& pushErr: \to Stack\\
		{\bf det:} && pushErr \\
	 	{\bf ax:}
			&1.& s\prec push(errNat,s) \\
			&2.& pushErr\prec push(errNat,s) \\
			&3.& s\prec pop(pushErr) \To \\
			&4.& x\prec top(pushErr) \To \\
			&5.& s\prec push(x,pushErr) \To
}
\)

\noindent
These axioms ensure the desired behaviour (from {\bf
ErrStack1a}).  Axioms 3.-5. make the results of other operations applied to
$pushErr$ empty, so that they will only consider the recovery
value.\footnote{The intention here is to interpret the result set, at least $\{s,pushErr\}$, not as
a nondeterminisitc choice but as an actual union of the elements.
%Any refinement of this specification -- eventually,
%also the implementation -- must conform to this. 
The only difference between
further refinement and eventual implementation is that the
former may further restric the range of the respective sets, while the latter
may interpret elements of the set as ``simultaneously present'', e.g., as a
recovery value and an error label/message.}
$pushErr$ can be naturally interpreted as a side effect of
$push(errNat,s)$, and implemented as, say, sending an error
message to the user. On the other hand, the presence of such an error value
in the result set can be interpreted as an exception. The axioms
3.-5. represent immediate catching this exception. Replacing, e.g., the
axiom 3. by $pushErr \prec pop(pushErr)$ will then correspond to throwing
this exception also from an application of $pop$ until one arrives at a
situation where other operations ignore this error value (specified with
axioms corresponding to 3.-5.)
% which correspond to catching the exception and performing
%the further action only on the remaining values in the set -- here on $s$).

At any previous development step we could have extended the stack
specification to a specification of bounded stack. This can be also done now
as a refinement of the last specification: %\vspace*{1ex}

\(
4. \spec{	\tit{\mbox{\bf spec\ BoundedStack}^{\inst{MA}}=} \\
	\tit{\mbox{\bf enrich\ ErrValStack3\  by:}} \\
	    	\Ops:	&& max: \to Nat\\
		        && hgh: Stack \to Nat \\
		{\bf det:} && max, hgh \\
	 	{\bf axioms:}
			&1.& hgh(empty) \eleq 0 \\
			&2.& push(x,s)\eleq push(x,s) \To hgh(push(s,x))\eleq
			succ(hgh(s)) \\
			&3.& hgh(s)\eleq max \To hgh(push(x,s))\eleq
			succ(max) \\
			&4.& hgh(s)<max \To push(x,s)\eleq push(x,s) \\
			&5.& hgh(s)\geq max \To push(s,x)\prec errStack
}
\)

\noindent
Notice that the last axiom makes the error resulting from exceeding the bound
behave as $empty$ according to {\bf ErrStack1a}. If this is not desirbale,
one would have to identify it as a new error type with appropriate axioms.

%%%Also, when $hgh(s)\eleq max$, the result of $push(x,s)$ will yield a
%%%possibly nondeterministic
%%%$errStack$. Nevertheless, its height is defined uniquely and uniformly for
%%%all such possible results by axiom 3. 


\section{Partial algebras and multialgebras}\label{se:partial}
For a survey of partial algebras, the reader is referred to
\cite{burm,state,mossa}. A partial algebra $A$ is an
algebra where each operation $\omega^A:s_1^A\times...\times s_n^A\to s^A$ may
be partial with a subset of the argument as the domain of defintion, 
$\dom(\omega^A)\subseteq s_1^A\times...\times s_n^A$. (Weak) homomorphisms
are homomorphsims preserving the defintion domains, i.e.,
$\phi(\dom(\omega^A))\subseteq \dom(\omega^B)$. The atomic formulae are {\em
existential equations}, written $t\eequal t'$, which are satisfied in an
algebra $A$ iff both $t^A$ and $t'^A$ are defined and equal. In particular,
the equation $t\eequal t$, forces $t$ to be defiend. 
The institution
of partial algebras, $\inst{PA}$, has as sentences conditional existential
equations. 

$\inst{PA}$ specification of stacks will look exactly as the one given in
subsection~\ref{subsub:pa1} with all $\eleq$ replaced by $\eequal$.
%({\bf det}
%axiom there was an abbreviation for $empty\eleq empty$, which would now
%become the existential equation $empty \eequal empty$). 
Conversely, replacing
all $\eequal$ by $\eleq$ in an
arbitrary $\inst{PA}$ specification, yields
an $\inst{MA}$ specification with intuitively the same meaning but
with a larger model class, where the development down to explicit error
handling as illustrated in section ~\ref{se:meth} can take place. This section
makes this intuitive correspondance precise. First we consider a simple
embedding of $\inst{PA}$ into $\inst{MA}$.

%\subsection{Embedding $\inst{PA}$ into $\inst{MA}$.}\label{sub:emb}
Any $\Sigma$-partial algebra can be trivially converted into a
$\Sigma$-multialgebra by making all undefined operations return the empty
set. 
Since operations in multialgebra are strict on the empty set, the
implicit strictness assumption from partial algebras, will be enforced
automatically.
%%%%%%%%
\begin{definition}\label{de:MP}
The functor $\fu{\beta}^-:\Mod_{\inst{PA}}(\Sigma)\to\Mod_{\inst{MA}}(\Sigma)$
maps a partial algebra $A$ to a multialgebra given by:
\begin{itemize}\MyLPar
\item $|\fu{\beta}^-(A)|=|A|$
\item for all $\overline x\in |\fu{\beta}^-(A)|$ and $f\in\Omega$:
$f(\overline x)^{\fu{\beta}^{-}(A)}= \left\{\begin{array}{ll}\{f(\overline x)^A\} & - {\rm
if\ it\ is\ defined}\\
\emptyset & - {\rm otherwise} \end{array}\right.$
\end{itemize}
For a homomorphism:  $h \in \Mod(\fu{\Psi}(\Sigma,\Gamma))$, we define $\fu{\beta}^-(h) = h$.

For a multialgebra $M$ where all nondeterministic operations return empty
set, $\fu{\beta}(M)$ will denote the
corresponding partial algebra, i.e., $\fu{\beta}^-(\fu{\beta}(M))=M$.
\end{definition}
%
%%Saying that a multialgebra and a partial
%%algebra are ``essenitally the same'', we will
%%mean that they are obtained from each other by means of $\fu{\beta}(\_)$,
%%resp. $\fu{\beta}(\_)$.
%
The embedding of $\inst{PA}$ into $\inst{MA}$ is now obtained by 
augmenting the
partial algebra specification with additional axioms forcing all operations
to return either a unique element or the empty set. This is the underlying
model in partial algebras which in multialgebraic context need explicit
axioms. The required axioms are given in the (proof aketch of) the following proposition.
%%For an operation $f(\overline x)$, the axiom forcing it to be empty or
%%deterministic is of the form $y\prec f(\overline x)\To f(\overline x)\eleq
%%f(\overline x)$, where $y$ is a fresh variable. 
%
\begin{proposition}\label{prop:PaMa}
There is a (simple map) embedding $(\fu{\Psi},\alpha, \beta): \inst{PA}\embd \inst{MA}$.
\end{proposition}
%
\begin{PROOFs} We merely indicate the construction.
\begin{itemize}\MyLPar
\item The functor $\fu{\Psi}: \Sign_{\inst{PA}} \to \thr{Th}_{0\inst{MA}}$ is given
by: $\fu{\Psi}(S,\Omega)= ((S,\Omega),\emptyset_{\Sigma})$, where
$\emptyset_{\Sigma}$ contains an axiom $y \prec f(\overline{x}) \to
f(\overline{x}) \eleq f(\overline{x})$ for each $f\in\Omega$. For  morphisms
$\fu{\Psi}(\mu_S,\mu_{\Omega})$ is the identity.  
\item The natural transformation
$\alpha: \Sen_{\inst{PA}} \to \Sen_{\inst{MA}} \circ \fu{\Psi}$ is given by:
   \begin{enumerate}\MyLPar 
    \item $\alpha(t \eequal t') \equiv t \eleq t'\mbox{ for
each axiom }t \eequal t'$ 
    \item $\alpha(a_1 \wedge \cdots \wedge a_n
\To a) \equiv \alpha(a_1) \wedge \cdots \wedge \alpha(a_n)
\To \alpha(a)\mbox{ for each clause }a_1 \wedge \cdots \wedge
a_n \To a$ 
   \end{enumerate}
$\fu{\Psi}$ is extended to a functor 
$\fu{\Psi}: \thr{Th}_{0\inst{PA}} \to \thr{Th}_{0\inst{MA}}$ by letting
$\fu{\Psi}(\Sigma,\Gamma) = (\Sigma,\emptyset_\Sigma\cup\alpha_\Sigma(\Gamma))$.
%
\item 
The components of the natural transformation $\beta: Mod_{\inst{MA}}
	\circ \fu{\Psi}^{op} \to Mod_{\inst{PA}}$ are $\fu{\beta}$'s from definition~\ref{de:MP}, i.e.:
\begin{itemize}\MyLPar
	\item $|\beta_{\Sigma}(M')| = |M'|$
	\item $f(x_1, \ldots ,x_n)^{\beta_{\Sigma}(M')} = \left\{\begin{array}{l}
	  undefined\ {\rm if}\ f(x_1, \ldots ,x_n)^{M'}=\emptyset \\
          x\ {\rm such\ that}\ f(x_1, \ldots ,x_n)^{M'}=\{x\}\ {\rm otherwise} \end{array}\right.$
    \end{itemize}
%This is well defined partial algebra by lemma~\ref{le:esspa}.
For a homomorphism:  $h \in \Mod_{\inst{MA}}(\fu{\Psi}(\Sigma,\Gamma))$, we define $\beta_\Sigma(h) = h$.
\end{itemize}\vspace*{-3ex}
\end{PROOFs}
We also have an immediate consequence of the above proof:
\begin{fact}\label{fa:iso}
For a $\inst{PA}$ theory $(\Sigma,\Gamma)$, 
the functor $\beta_{(\Sigma,\Gamma)}$ is an equivalence (in fact, an
isomorphism) of categories $\Mod_{\inst{MA}}(\fu{\Psi}(\Sigma,\Gamma))$ and
$\Mod_{\inst{PA}}(\Sigma,\Gamma)$. 
\end{fact}
%
As showed in \cite{mossa}, $\inst{PA}$ allows to
specify exactly the finitely locally presentable categories \cite{locally},
i.e. we have identified the subinstitution of $\inst{MA}$ allowing to specify
these classes of models. 

%\subsection{Extending the model class of $\inst{PA}$ specifications}\label{sub:import}
Now, the intention of passing from a partial to a multilagebra specification
is, on the one hand, to reuse specifications written in the former
framework and, on the other, to allow for their furhter development with
explicit error handling. The embedding from
proposition~\ref{prop:PaMa} does not address this latter issue since it
yields essentially the same model class. 

The desired generalization is made by importing $\inst{PA}$ specifications
{\em without} augmenting them with additional axioms. 
This results in a larger model class where, in
addition to essentially the same partial models, we also have the models
with non-strict operations.
Further development can now take
place in the multialgebraic framework, allowing one to refine the
specification to the level of explicit error treatement.
The following proposition, stating the existence of {\em institution
transformation}, def.~\ref{de:maps}.\ref{it:trans}, formalizes the possibility of such a
scenario. 
%%The isomorphism of respective model classes stated in fact \ref{fa:iso} gives
%%us the proof of the following proposition.
\begin{proposition}\label{prop:PaMaT}
There is an institution transformation $(\fu{\Psi}^*,\alpha,
\beta^{-}):\inst{PA} \to \inst{MA}$.
\end{proposition}
%
\begin{PROOFs} The definition is as in proposition~\ref{prop:PaMa}
except that $\fu{\Psi}^*:\Sign_{\inst{PA}} \to
\thr{Th}_{0\inst{MA}}$ does not add any new axioms, while the components of
the natural transformation $\beta^-:\Mod_{\inst{PA}} \to
\Mod_{\inst{MA}} \circ (\fu{\Psi}^*)^{op}$ are $\beta^-$s from
definition~\ref{de:MP}.
%%We merely indicate the construction.
%%\begin{itemize}\MyLPar
%%\item The functor $\fu{\Psi}^*: \Sign_{\inst{PA}} \to \thr{Th}_{0\inst{MA}}$ is given
%%by: $\fu{\Psi}^*(S,\Omega)= (S,\Omega,\emptyset)$. For  morphisms
%%$\fu{\Psi}^*(\mu_S,\mu_{\Omega})$ is the identity.  
%%\item The natural transformation
%%$\alpha: \Sen_{\inst{PA}} \to \Sen_{\inst{MA}} \circ \fu{\Psi}^*$ is given by:
%%   \begin{enumerate}\MyLPar 
%%    \item $\alpha(t \eequal t') \equiv t \eleq t'\mbox{ for
%%each axiom }t \eequal t'$ 
%%    \item $\alpha(a_1 \wedge \cdots \wedge a_n
%%\To a) \equiv \alpha(a_1) \wedge \cdots \wedge \alpha(a_n)
%%\To \alpha(a)\mbox{ for each clause }a_1 \wedge \cdots \wedge
%%a_n \To a$ 
%%   \end{enumerate}
%%$\fu{\Psi}^*$ is extended to a functor 
%%$\fu{\Psi}^*: \thr{Th}_{0\inst{PA}} \to \thr{Th}_{0\inst{MA}}$ by letting
%%$\fu{\Psi}^*(\Sigma,\Gamma) = (\Sigma,\alpha_\Sigma(\Gamma))$.
%%%
%%\item The components of the natural transformation $\beta^{-}: \Mod_{\inst{PA}} \to
%%\Mod_{\inst{MA}} \circ (\fu{\Psi}^*)^{op}$ are $\beta^-$ from
%%definition~\ref{de:MP}.
%%\end{itemize}\vspace*{-4ex}
\end{PROOFs}\vspace*{-2ex}
%

\section{Conclusions and future work}
We have presented a multialgebra based approach to developing specifications
with partial operations. The novelty of the proposed framwork lies in
thinking about and modeling undefined operations by nondeterministic ones --
an operation applied to an argument outside its definition domain may result
in an unexpected and initially unknown value. This view leads actually
to the combination of various features of several earlier approaches. It
allows one to start with high level specifications, where error
situations can be dealt with at the same level of abstraction as in partial
algebras. Narrowing the range of nondeterminism modeling undefinedness, one
can refine such specifications to a low level error handling. 
We have illustrated by examples a wide range of possibilities for error
handling admitted by the proposed framework. In particular, utilizing sets to
model error situations allows a function to return both a marking that such a
situation occurred and relevant recovery values.

From the methodological perspective, we have shown the possiblity of
reusing partial algebra specifications without the necessity to perform any
translation (except for the trivial replacement of $\eeq$ by $\eleq$.)
Thus,
we believe the proposed framework may be more useful, and in any case easier
to apply, when extending partial algebra specifications to explicit error
handling, than the frameworks based on translation of such specifications into
deterministic specifications with predicates.

We have also indicated (proofs are to be found in \cite{TR}) that other
frameworks and institutions can be embedded into the institution of
multialgebras $\inst{MA}$ (e.g., memebership algebras, order-sorted
algebras). 
We thus suggest that $\inst{MA}$ may
provide an adequate framework for both {\em comparing} and {\em combining} the advantages
of earlier approaches within a unified framework.
For this purpose, it would be desirable to restrict the 
generality of $\inst{MA}$. Looking for a suitable subinstitution may
start, for instance, by trying to identify the conditions for $\inst{MA}$
specifications ensuring the existence of initial models. 

Another, though probably, less problematic issue concerns the reasoning
system. The earlier systems need to be adjusted to the present context: the
complete logic from \cite{toplas,calc} did not allow for empty result sets in a model,
while the one from \cite{BK} used only inclusions $\prec$, but not the element
equality $\eleq$.


\bibliography{biblo}
\bibliographystyle{plain}

\end{document}
