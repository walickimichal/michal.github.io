\section{Composition and refinement}\label{se:compref}
%
We will now review various ways of composing specifications of parameterized
data types. We will discuss the classical vertical and horisontal
composition, showing the counterparts of the standard compositionality
theorems. The main difference will concern the fact that, in general,
stepwise application of constructions will not yield the same as a direct
construction along the respective composition, but a refinement of the latter.
We will discuss it in details ...

%
We recall that, given a parameter passing diagram (like 1. below in
proposition~\ref{prop:vertcomp}), by fact~\ref{fa:paramtoo},
$\mu':\thr{Y_{\rho(\weak)}}\to \thr{P[Y]_\weak}$ is a parameterization
morphism, and hence, in particular (by fact~\ref{fa:weakparamor}), $\mu':\thr
Y_\weak\to\thr{P[Y]_\cons}$ is a specification morphism.


\subsection{Vertical composition}
Given two actual prameter passing morphisms $\nu:\thr X_\weak\to \thr
Y_{\nu(\weak)}$ and $\rho:\thr Y_\weak\to \thr Z_{\rho(\weak)}$,
(as indicated in the diagrams 1. and 2. in Figure~\ref{fi:vertcomp}), 
we would like to compose them vertically, i.e.,
we want to show that also $(\nu;\rho):\thr X_\weak \to \thr Z_{(\nu;\rho)(-)}$ is
an actual parameter passing.%\vspace*{-1ex}

\begin{figure}[hbt]
\[\xymatrix@R=0.5cm@C=1.5cm{
\thr X_\weak \ar[r]^\mu \ar[dd]_\nu   \ar@(l,l)[dddddd]_{\nu;\rho}  \ar@{}[ddr]|{1.}
  & \thr{P[X]_\cons} \ar[dd]^{\nu'} \ar@(r,r)[dddddd]^{(\nu;\rho)'} \\ \\ 
\thr Y_{\nu(\weak)} \ar[r]^{\mu'} & \thr{P[Y]_\cons} \\ 
%
\thr Y_\weak \ar[r]^{\mu'} \ar[dd]_\rho \ar@{}[ddr]|{2.}
  & \thr{P[Y]_\cons} \ar[dd]^{\rho'} \\ \\ 
\thr Z_{\rho(\weak)} \ar[r]^{\mu''} & \thr{P[Z]_\cons} \\ 
\thr Z_{(\nu;\rho)(\weak)} \ar[r]^{\mu'''} & \thr{P[Z]'_\cons} \\ 
}
\]
\caption{}\label{fi:vertcomp}\vspace*{-1ex}
\end{figure}
%

\noindent
The notation from this figure will be used throughout the whole subsection.

In general, the specifications $\thr{Z_{\rho(\weak)}}$ and
$\thr{Z_{(\nu;\rho)(\weak)}}$ need not be the same -- the latter may have
fewer global guards than the former.
\begin{fact}\label{fa:triv}
Given $\mu,\nu$ and $\rho$ as in the Figure~\ref{fi:vertcomp}:
\begin{enumerate}\MyLPar
\item $\thr{Z_{(\nu;\rho)(\weak)}}\models \thr{Z_{\rho(\weak)}}$.
\item If $\nu$ is surjective on the sorts, then $\thr{Z_{\rho(\weak)}}\models
\thr{Z_{(\nu;\rho)(\weak)}}$.
\item If $\nu$ is
surjective on the sorts, then $\thr{Z_{(\nu;\rho)(\weak)}} = \thr{Z_{\rho(\weak)}}$.
\end{enumerate}
\end{fact}
\begin{PROOF}
Direct from definition~\ref{def:weakalong}. Obviously, both specifications
have the same signature. 
\begin{enumerate}\MyLPar
\item  All sorts which are in the
image of $(\nu;\rho)$ are also in the image of $\rho$, so the global guards
dropped in $\thr{Z_{(\nu;\rho)(\weak)}}$ are also dropped in
$\thr{Z_{\rho(\weak)}}$. 
\item If $\nu$ is surjective on the sorts then if a sort is in the image of
$\rho$ it will also be in the image of $(\nu;\rho)$. Hence all global guards
from $\thr{Z_{(\nu;\rho)(\weak)}}$ will also be present in $\thr{Z_{\rho(\weak)}}$.
\item If $\nu$ is surjective on the sorts, then the equality follows from the
two points above. \vspace*{-2ex}
\end{enumerate}
\end{PROOF}
Notice that, in points 2. and 3., surjectivity of $\nu$ on sorts is sufficient but
not necessary condition. It is sufficient and necessary that for any sort $s\in\Sigma(\thr{Y})$
which is not in the image of $\nu$, there is a sort  $s'\in\Sigma(\thr{Y})$ which
is in the image of $\nu$ and such that $\rho(s)=\rho(s')$.
%
\begin{proposition}\label{prop:vertcomp}
If 
$\nu:\thr X_\weak\to \thr Y_{\nu(\weak)}$ and $\rho:\thr Y_\weak\to \thr Z_{\rho(\weak)}$
 are actual parameter passing
morphisms, then so is $(\nu;\rho):\thr X_\weak \to \thr Z_{(\nu;\rho)(\weak)}$
(see the diagram in Figure~\ref{fi:vertcomp}). 
\end{proposition}
\begin{PROOF}
We have both $\nu(\cons_s)=\cons_{\nu(s)}$ and
$\rho(\cons_{s'})=\cons_{\rho(s')}$ for all sort symbols $s\in\Sigma(\thr X)$
and $s'\in\Sigma(\thr Y)$, and thus
$(\nu;\rho)(\cons_s)=\cons_{(\nu;\rho)(s)}$. We show that
$(\nu;\rho):\thr X_\weak \to \thr Z_{(\nu;\rho)(\weak)}$ is
a specification morphism, i.e., $\thr Z_{(\nu;\rho)(\weak)} \models
(\nu;\rho)(\thr X_\weak)$.

%For a specification $\thr S$ and any signature morphism
%$\sigma:\Sigma\to\Sigma(\thr S)$ we have that $\thr S \supseteq \thr
%S_{\sigma(\weak)} \supseteq \thr S_\weak$,
%\eq{
\begin{eqnarray}%{rcl}
&{\rm fact\ \ref{fa:triv}} :& \thr Z_{(\nu;\rho)(\weak)}\models \thr Z_{\rho(\weak)} \nonumber \\
&\rho\ {\rm is\ a\ specification\ morphism}\ 2.& \thr Z_{\rho(\weak)}\models\rho(\thr Y_\weak) \nonumber \\
& \To & \thr Z_{(\nu;\rho)(\weak)}\models\rho(\thr Y_\weak) \label{eq:mod}
\end{eqnarray}
%}
%From 1. we have that $\thr Y_{\nu(\weak)}\models \nu(X_\weak)$, 
%
%$\thr Z_{(\nu;\rho)(\weak)}\models \rho(\thr Y_\weak)$
%
Now, the axioms of $\thr Y_{\nu(\weak)} =  (\Phi,\Gamma')$, where $\Gamma'$
is the subset of global guards from $\thr Y_\cons$ which (whose sort symbols)
are not in the image
of $\nu$. To complete the proof we have to show that $\thr
Z_{(\nu;\rho)(\weak)}\models \rho(\Gamma')$.
But this follows directly from definition~\ref{def:weakalong}. For any
global guard $\gamma\in\Gamma'$ is {\em not} in the image of $\nu$ and hence
it will {\em not} be in the image of $\nu;\rho$. Consequently, if
$\gamma\in\Gamma'$ then
$\rho(\gamma)\in\thr Z_{(\nu;\rho)(\weak)}$ (though not necessarily 
 $\rho(\gamma)\in \thr Z_{\rho(\weak)}$!!).

Thus $\thr Z_{(\nu;\rho)(\weak)}\models \rho(\Gamma')$ which together with (\ref{eq:mod})
%$\thr Z_{(\nu;\rho)(\weak)}\models \rho(\thr Y_\weak)$, i.e., 
yields
\eq{
\thr Z_{(\nu;\rho)(\weak)}\models \rho(\thr Y_{\nu(\weak)}).
\label{eq:modB}
}
In diagram 1. $\nu$ is a parameter passing, so $\thr
Y_{\nu(\weak)}\models \nu(\thr X_\weak)$ which implies  
$\rho(\thr Y_{\nu(\weak)})\models (\nu;\rho)(\thr X_\weak)$.
This, together with (\ref{eq:modB}) give the conclusion: 
$\thr Z_{(\nu;\rho)(\weak)}\models (\nu;\rho)(\thr X_{\weak}))$.
\end{PROOF}
%\noindent
In general, the specifications $\thr{P[Z]_\cons}$ and
$\thr{P[Z]'_\cons}$ may be different. In the classical case, this is merely a
consequence of their definition by pushout (which is unique only up to
isomorphism). In our case, however, the difference may be more significant,
since we also may drop and/or add some global guards on the way. As in
fact~\ref{fa:triv}, 
the only difference may concern the presence/absence of global guards (since
all other axioms are involved in the pushout construction), so these are the only
axioms we mention in the following example.
\begin{example}\label{ex:difpush}
Consider the following instantiations. (Two lines in $\thr{Y_{\nu(\weak)}}$,
$\thr{Y_\weak}$, etc. represent two disitinc sorts which are identified by
the second instantiation $\rho$.)
\[\xymatrix@R=0.5cm@C=0.7cm{
&\thr{X_\weak}&\cons  \save[].[]*\frm{.}\restore  \ar[rr]^{\mu} \ar[dd]_{\nu} && 
    c \save[].[rrrr]*\frm{.}\restore \ar[dd] & & \cons \ar[dd] && x\prec
  \cons \ar@{.>}[dd] & \thr{P[X]_\cons}\\ 
&& && && && \\
\thr{Y}_{\nu(\weak)}& & \cons \ar[rr]^{\mu'} && c  && \cons && x \prec\cons
    \save[].[dllll]*\frm{.}\restore  & \thr{P[Y]_\cons}\\
 & y\prec \cons_1\ \ \ \cons_1 \save[].[ru]*\frm{.}\restore
    \ar[rrrr]^{\mu'} &&&& 
   \cons_1 &&  y\prec\cons_1 &  \\ \\
%%% next inst
\thr{Y_\weak} & \cons_1 \save[].[dr]*\frm{.}\restore \ar[dddr]_{\rho}\ar[rrrr]_{\mu'}
&&&&  \cons_1 \ar[dddl] && y\prec \cons_1 \ar@{.>}[ddd] &  & \thr{P[Y]_\cons}\\
 & & \cons \ar[dd]_\rho  \ar[rr]_{\mu'} && c \ar[dd] && \cons \ar[dd] 
  && x\prec\cons \save[].[ullll]*\frm{.}\restore  \ar@{.>}[dd]\\ 
& & && && && \\
 & \thr{Z_{\rho(\weak)}}& \cons' \save[].[]*\frm{.}\restore \ar[rr]^{\mu''} && 
  \cons' \save[].[rrrr]*\frm{.}\restore && \cons & y\prec\cons' & x \prec
 \cons & \thr{P[Z]_\cons}
}
\]
And now a direct instantiation along $\comp{\nu}{\rho}$:
\[\xymatrix@R=0.3cm@C=0.8cm{
\thr{X_\weak} & \cons \save[].[]*\frm{.}\restore \ar[ddd]_{(\nu;\rho)} \ar[rr]^\mu && 
  c \save[].[rrrr]*\frm{.}\restore \ar[ddd] 
   && \cons \ar[ddd] && x\prec\cons \ar@{.>}[ddd] & \thr{P[X]_\cons} \\
 & && &&&&& \\  & && &&&&& \\
\thr{Z_{(\nu;\rho)(\weak)}} & \cons' \save[].[]*\frm{.}\restore \ar[rr]^{\mu'''} && 
  c  \save[].[rrrr]*\frm{.}\restore
   && \cons  && x\prec\cons  & \thr{P[Z]'_\cons}
}
\]
Both $\thr{P[Z]_\cons}$ and $\thr{P[Z]'_\cons}$  have isomorphic
signatures: in the former  $\allcons = \{\cons,\cons'\}$, while in the
latter ${\allcons}'=\{\cons,c\}$. 

The significant difference consists in that  $\thr{P[Z]_\cons}$  has the
global guard for $\mu''(\cons)$, namely $y\prec\cons'$ originating from
$\thr{P[Y]_\cons}$. (Thus here $\Cons=\{\cons,\cons'\}$ and $\subcons=\emptyset$.)
In $\thr{P[Z]'_\cons}$, on the other hand, this guard is
not present. (So here $\Cons'= \{\cons\}$, while ${\subcons}'=\{c\}$.)

Thus, the pdt
$\pdtsimple{\mu''}{whoever}{\thr{Z_\cons}}{\thr{P[Z]_\cons}}$ would forbid
extending the carrier of $\cons'$, while
$\pdtsimple{\mu'''}{whoever}{\thr{Z_\cons}}{\thr{P[Z]'_\cons}}$ would not.
\end{example}
%
So, in general, $\thr{P[Z]_\cons}$
and $\thr{P[Z]'_\cons}$ are not isomorphic. We have the following fact.
\begin{fact}\label{fa:pzpz}
With the notation from Figure~\ref{fi:vertcomp} and
example~\ref{ex:difpush}:
\begin{enumerate}\MyLPar
\item $\thr{P[Z]_\cons}\models \thr{P[Z]'_\cons}$.
\item if $\thr{P[Z]'_\cons}\not\models \thr{P[Z]_\cons}$, then it is only
because for some constant(s) $c: \thr{P[Z]_\cons}\models x\prec c$ and
$\thr{P[Z]'_\cons}\not\models x\prec c$. 
\end{enumerate}
\end{fact}
\begin{PROOF}
The signatures of both specifications will be isomorphic, so we assume that they are
identical. All axioms except global guards are involved in the pushout
constructions, so their presence (or satisfaction) follows from the standard
isomorphism of pushout objects. The difference may concern only some constants which are in $\Cons$ but
not in $\Cons'$ (only in ${\subcons}'$, as in the
example~\ref{ex:difpush}). This justifies point 2.
For point 1. we show that if $\cons\in\Cons'$ then
$\cons\in\Cons$, that is if $\thr{P[Z]'_\cons}\models x\prec\cons$ then
$\thr{P[Z]_\cons}\models x\prec\cons$, which will yield the conclusion.

This follows trivially. Any global guard $x\prec\cons$ in $\thr{P[Z]'_\cons}$
is an image of a respective global guard either from
$\thr{Z_{(\nu;\rho)(\weak)}}$ or from $\thr{P[X]_\cons}$. In the latter case,
it will also be present in $\thr{P[Y]_\cons}$ and hence also in
$\thr{P[Z]_\cons}$. 

In the former case, if this guard is also in $\thr{Z_{\rho(\weak)}}$ it will
be present in $\thr{P[Z]_\cons}$. If it does not belong to
$\thr{Z_{\rho(\weak)}}$, this means that it (its sort) is in the image of
$\rho$ (and therefore was dropped). But then, its $\rho$ pre-image must be in
$\thr{Y_\cons}$, that is, must be present in $\thr{P[Y]_\cons}$. But then it
is also present in $\thr{P[Z]_\cons}$ as the result of pushout construction.
\end{PROOF}
The fact that stepwise instantiation ($\thr X_\cons$ by $\thr Y_\cons$ and then by
$\thr Z_\cons$ leading to $\thr{P[Z]_\cons}$) yields a different result than the direct instantiation
($\thr X_\cons$ by $\thr{Z_\cons}$ leading to $\thr{P[Z]'_\cons}$) may look like a severe weakness of our
setting. After all, equality of these two indicates the desirable
compositionality which would be expected by anybody familiar with the
traditional, pushout based theory of parameterized specifications. 

However,
we are not developing a theory of parameterized specifications but of
specification of parameterized data types. This means, we are interested in
constructions allowing us to obtain new data types (algebras) from others. In
this setting, performing different series of constructions or, as in the case
of vertical composition, performing constructions in different ways, may be
expected to yield different results. 

Our point is that stepwise instantaition, first along $\nu$ and then along
$\rho$ represents a slightly different construction that direct instantition
along $\eta=\nu;\rho$. In fact, we sugest to think of the former as a refinement
of the latter. The latter is a one step construction along $\eta$. In this
sense, splitting this construction in two steps, first along $\nu$ and then
$\rho$, is a more detailed, refined construction which may introduce new
aspects. We certainly want the result of this refined construction to be
``compatible'' with the results prescribed by the more rough, one step
construction. This is the meaning of one construction refining another which
corresponds to the classical concept of refinement by model class
inclusion. This is indicated by the fact~\ref{fa:pzpz} and we now
proceed to illustrate the semantic aspect of this refinement.

\subsection{Vertical composition -- semantics}
%
As noted in section~\ref{sub:appsem}, we can view the semantics of
instantiation from two angles: on the one hand, as a new pdt with a class of
its semantic functors and, on the other hand, as an actualisation: a functor
for the resulting pdt induced by a
particular functor for the instantiated pdt. We now apply this distinction in
the discussion of the semantics of vertical composition.
\subsubsection{Vertical composition as a refinement of pdt}\label{sub:vertref}
%\begin{definition}
We postpone the general definition of refinement to a later section and for
the moment take it intuitively to mean:
a pdt $\thr P'=\pdtsp{\mu'}{\delta'}{\thr{X'_\cons}}{\thr{P[X]'_\cons}}$ is a {\em
refinement} of a pdt $\thr P=\pdt$, $\thr P\leadsto \thr P'$, if any semantic
functor for $\thr P'$ can be used for obtaining 
a semantic functor for $\thr P$.
%\end{definition}

A trivial, though by no means only, example of such a refinement is when
$\thr{P[X]_\cons}\leadsto\thr{P[X]_\cons}$, i.e.,
$\Mod(\thr{P[X]_\cons})\supseteq \Mod(\thr{P[X]'_\cons})$, while other
components are equal. This is, in fact, the case with the results of vertical
composition. If we view $\thr{P[Z]_\cons}$ and $\thr{P[Z]'_\cons}$ as two
independent pdts (i.e., ``forget'' that they both originate from instantiation
of the same pdt), we see that, 
by fact~\ref{fa:pzpz}, $\thr{P[Z]_\cons}\models
\thr{P[Z]'_\cons}$, i.e., we have an inclusion (functor)
$\fu i:\Mod(\thr{P[Z]_\cons})\subseteq\Mod(\thr{P[Z]'_\cons})$. 
Thus any semantic functor $\fu F$ for $\thr P= \pdtsp{\mu''}{\delta''}{\thr
Z_\cons}{\thr{P[Z]_\cons}}$ 
gives a semantic functor for $\thr P'= \pdtsp{\mu'''}{\delta'''}{\thr
Z_\cons}{\thr{P[Z]'_\cons}}$, simply 
by composing $\comp{\fu F}{\fu i}$.
The other components of both pdts are (essentially) the same, and so we get
%
\begin{fact}
Given $\thr P= \pdtsp{\mu''}{\delta''}{\thr
Z_\cons}{\thr{P[Z]_\cons}}$ and  $\thr P'= \pdtsp{\mu'''}{\delta'''}{\thr
Z_\cons}{\thr{P[Z]'_\cons}}$ (as in Figure~\ref{fi:vertcomp}), $\thr P'\leadsto \thr P$.
\end{fact}
%
Refinement amounts in this case to the situation illustrated in
example~\ref{ex:difpush}, namely, that while $\thr P'$ may allow extension of
some carriers (corresponding to $\cons'$ in the example), $\thr P$ may forbid
it by introducing additional global guards. Thus, in general, all semantic
functors for $\thr P$ are also semantic functors for $\thr P'$, but there may
be some functors for $\thr P'$ which are not valid semantic functors for
$\thr P$.

\subsubsection{Vertical composition as an actualisation of a particular semantic
functor}
There is, however, a more specific relation between the stepwise
instantiation and the direct one. According to
proposition~\ref{prop:inducedapp}, any semantic functor $\fu F_X$ for $\pdt$
induces a semantic functor $\fu F_Y$ for any instantiation of formal
parameter $\thr X_\cons$ by an actual parameter $\thr Y_\cons$. 
If we now consider the results of respective actualisations, i.e., functors
$\fu F_Z$ (obtained by stepwise actualisation through $\thr Y_\cons$ first
along $\nu$ and then $\rho$) and
$\fu F'_Z$ (obtained by direct actualisation along $(\nu;\rho)$) which are
both induced starting from the same, given $\fu F_X$, then it turns out that
the semantics is fully compositional. 

We discuss it in more detail.
The semantic counterpart of the diagram from Figure~\ref{fi:vertcomp} is
shown below.
\[
\hspace*{-3em}\xymatrix@C=0.8cm@R=1cm{
&&&& \Mod(\thr X_\cons) \ar[dr]^{\fu F_X} \ar[d]^{\iota_X} & \\
&&& \Mod(\thr Y_\cons)  \ar[dr]^>>>{\fu F_Y} \ar[d]^{\iota_{Y1}}
\ar[ur]^{|_{\nu}} \ar@{.}[dl]|{=} 
     & \Mod(\thr X_\weak) & \Mod(\thr{P[X]_\cons}) \ar[l]_{|_{\mu}} \\
&& \Mod(\thr Y_\cons) \ar[dr]^>>>{\fu F'_Y} \ar[d]^{\iota_Y} 
     & \Mod(\thr Y_{\nu(\weak)}) \ar[ur]^>>>>>{|_{\nu}} \ar@{^{(}.>}[dl]
     & \Mod(\thr{P[Y]_\cons}) \ar[l]^{|_{\mu'}} \ar[ur]^{|_{\nu'}}
     \ar@{.}[dl]|{=} \\
& \Mod(\thr Z_\cons) \ar[ur]^{|_\rho} \ar[d]^{\iota_Z} \ar[dr]^>>>{\fu F_Z} & \Mod(\thr Y_\weak) &
       \Mod(\thr{P[Y]_\cons}) \ar[l]_{|_{\mu'}} \\
\Mod(\thr Z_\cons) \ar[d]_{\iota_{Z1}} \ar[dr]^>>>{\fu F'_Z} \ar@{.}[ur]|{=}\ar@/^3pc/[uuuurrrr]^{|_{(\nu;\rho)}}
     & \Mod(\thr Z_{\rho(\weak)}) \ar[ur]^>>>>>{|_{\rho}} &
       \Mod(\thr{P[Z]_\cons}) \ar[l]^{|_{\mu''}} \ar[ur]_{|_{\rho'}}
       \ar@{^{(}.>}[dl]^{\fu i} \\
\Mod(\thr Z_{(\nu;\rho)(\weak)}) \ar@{^{(}.>}[ur]
    & \Mod(\thr{P[Z]'_\cons}) \ar[l]^{|_{\mu'''}}   \ar@/_3pc/[uuuurrrr]_{|_{(\nu;\rho)'}}
}
\]
Given a semantic functor $\fu F_X$ (in the uppermost diagram),
proposition~\ref{prop:inducedapp} allows us to construct a functor $\fu F_Y$,
and similarly, an $\fu F_Z$ can be constructed given an arbitrary $\fu F'_Y$. Thus, using
$\fu F_Y$ obtained from the instantiation along $\nu$ for $\fu F'_Y$, we can
construct an $\fu F_Z$ from a given $\fu F_X$. Notice that the associated
$\iota_Z$ guarantees the image of $\Mod(\thr Z_\cons)$ to be
included in $\Mod(\thr Z_{\rho(\weak)})$. 

For the direct instantiation, we can obtain $\fu F'_Z$ from a given $\fu
F_X$ by proposition~\ref{prop:inducedapp}. On the other hand,
by fact~\ref{fa:pzpz}, we also have the inclusion (functor) 
$\fu i:\Mod(\thr{P[Z]_\cons})\subseteq\Mod(\thr{P[Z]'_\cons})$.  
Hence, composing we obtain $\comp{\fu
F_Z}{\fu i}:\Mod(\thr{Z_\cons})\to\Mod(\thr{P[Z]'_\cons})$, which gives a
possible semantic functor $\fu F'_Z$ for the pdt
$\thr P'=\pdtsp{\mu'''}{\delta'''}{\thr{Z_\cons}}{\thr{P[Z]'_\cons}}$. 
Compositionality of actualisation is expressed in the following proposition.
%
\begin{proposition}
With the notation from the diagram above: $\fu F_Z;\fu i = \fu F'_Z$.
\end{proposition}
\begin{PROOF}
The discussion above shows that $\fu F_z;\fu i$ is a possible semantic
functor for $\thr P'$. To show the equality to $\fu F'_Z$ induced by a direct
instantiation, there remains a couple of tedious  details. 
\\[1ex]
I. Firstly, $\iota_{Z1}$, associated with the functor $\fu F'_Z$ obtained from the direct
instantiation, will include $\Mod(\thr Z_\cons)$ 
in $\Mod(\thr Z_{(\nu;\rho)(\weak)})$, while $\iota_Z$ associated with $\fu
F_Z$ obtained through the stepwise instantiation guarantees only inclusion in
$\Mod(\thr{Z_{\rho(\weak)}})$. 
%Both these classes are included in
%$\Mod(\thr{Z_\weak})$, so $\comp{\fu F_Z}{\fu i}$ is certainly a possible
%semantic functor for $\thr P'$, but w
We show that it actually is a special case
of a functor obtained from a direct instantiation, i.e., that actually $\comp{\fu
F_Z}{|_{\mu''}}:\Mod(\thr{Z_\cons})\to \Mod(\thr{Z_{(\nu;\rho)(\weak)}})$ --
the lowest square (with two inclusions and reducts) commutes.
\\[1ex]
I.1. This is unproblematic when $\thr Z_{\rho(\weak)}=\thr Z_{(\nu;\rho)(\weak)}$,
so let us consider the case when they are not equal. Then there is a global
guard $x\prec c_s$ (of sort $s$) which is included in $\thr Z_{(\nu;\rho)(\weak)}$ but not in
$\thr Z_{\rho(\weak)}$. This means that for any algebra
$A\in\Mod(\thr{Z_\cons})$, $(\fu F'_Z(A))|_{\mu'''}\models x\prec c$ -- in
other words, $\fu F'_Z$ does not extend the carrier of $s$. 
\\[1ex]
I.2. In principle,
from the diagram, it might look that $\fu F_Z$ might extend this carrier
since we may have $x\prec c_s \not\in \thr Z_{\rho(\weak)}$. However, this
last fact holds only if $s$ is in the image of $\rho$ (which makes the
respective global guards disappear from $\thr Z_{\rho(\weak)}$). At the same
time, since the guard is present in $\thr Z_{(\nu;\rho)(\weak)}$, it means
that $s$ is not in the imahge of $(\nu;\rho)$ -- hence its $\rho$ pre-image $s'$ must not be in the
image of $\nu$. 
\\[1ex]
I.3. This means that the respective guard $x\prec c_{s'}\in \thr
Y_{\nu(\weak)}$ and, by the pushout construction, 
$x\prec c_{s'}\in \thr{P[Y]_\cons}$. But then the respective guard $x\prec
\rho'(c_{s'})=c_s$ will also appear in $\thr{P[Z]_\cons}$. Finally, since $s'$ is
not in the image of $\nu$, we get $\mu'(c_{s'})=c_{s'}$, which implies that
also $\mu''(c_{s})=c_{s}$. In short $\fu F_Z$ will not, after all, extend
the carrier of sort $s$, and hence $(\fu F_Z(A))|_{\mu''} \in \Mod(\thr
Z_{(\nu;\rho)(\weak)})$. 
\\[1ex]
II. To prove the main claim, we need to look at the details of definitions of
induced functors. What we have to show is that the following two are equal
for any $A\in\Mod(\thr Z_\cons)$ (cf. definition~\ref{def:actinstfunct}):
\begin{enumerate}\MyLPar
\item $\fu F'_Z(A) = \iota_{Z1}(A) \amalgam_{\iota_{Z1}(A)|_{(\nu;\rho)}} \fu
F_X(A|_{(\nu;\rho)})$ -- direct actualisation, and
\item $\fu F_Z(A) = \iota_{Z}(A) \amalgam_{\iota_{Z}(A)|_{\rho}} \fu
F_Y(A|_{\rho})$, where 
$\fu F_Y(A|_\rho) = \iota_{Y1}(A|_\rho) \amalgam_{\iota_{Y1}(A|_\rho)|_{\nu}} \fu F_X((A|_{\rho})|_\nu)$.
\end{enumerate}
The problem here might possibly originate from the situation as in
fact~\ref{fa:pzpz}.2 which was illustrated in example~\ref{ex:difpush}, i.e.,
that $\fu F'_Z(A)$ yields an algebra which does not satisfy the global guard
$x\prec c$ satisfied by all algebras in $\Mod(\thr{P[Z]_\cons})$. Showing
equality of 1. and 2. we show, in particular, that such a situation does not occur.

{\Large{ Yngve... does this prop. hold??? I think so!!! Prove or give counter-example!!!}}
\end{PROOF}

%\pagebreak

%%%%%%%%%%% horisontal
%\subsection{Horisontal composition -- parameterized formal
%parameter}\label{sec:semfunccomp}
\subsection{Horisontal composition}\label{sec:semfunccomp}
Horisontal composition of pdts is defined in the standard way.
\begin{definition}\label{def:horcomp}
For parameterized data type specifications
$\pdtsp{\mu}{\delta}{\thr{X_\cons}}{\thr{P[X]_\cons}}$ and
$\pdtsp{\mu'}{\delta'}{\thr{P[X]_\cons}}{\thr{W[P[X]]_\cons}}$,
we define their composition to be $\pdtsp{(\mu;\mu')}{(\delta;\delta')}{\thr
X_\cons}{\thr{W[P[X]]_\cons}}$. 
\end{definition}
%
\begin{proposition}\label{prop:horcomp}
The composition as defined in~\ref{def:horcomp} is (isomorphic to) a pdt. (In
the sense that there exists a $\thr{W[P[X]]'_\cons}\iso\thr{W[P[X]]_\cons}$
such that $\pdtsp{(\mu;\mu')}{(\delta;\delta')}{\thr
X_\cons}{\thr{W[P[X]]'_\cons}}$ is a pdt).
\end{proposition}
\begin{PROOF}
The first 4 points of definition~\ref{def:parametersyntax} are trivially
satisfied. We have to verify point~\ref{it:rel}. 
\begin{itemize}
\item[\ref{it:sat}]
If, for some $\cons:\mu'(\mu(\cons))\not=\delta'(\delta(\cons))$ then either 1)
$\mu(\cons)\not=\delta(\cons)$ or 2)
%$\mu(\cons_s)=\delta(\cons_s)=\cons_s$ and
$\mu'(\cons)\not=\delta'(\cons)$. 

Let us start with 1). By \ref{it:sat} of
definition~\ref{def:parametersyntax}, $\thr{P[Y]_\cons}$ contains the axiom
$\mu(\cons)\prec\delta(\cons)$. If both
$c_1=\mu(\cons)\not=\cons\not=\delta(\cons)=c_2$, then this axiom is actually
$c_1\prec c_2$ and,  by point~\ref{it:unguard}, 
$\mu'(c_1)\prec\mu'(c_2)\in\thr{W[P[X]]_\cons}$. But then also both $\delta'$ and
$\mu'$ are identities on $c_1,c_2$, i.e., this last axiom is in fact
$\mu'(\mu(\cons))\prec \delta'(\delta(\cons))$. 

If either $\delta(\cons)=\cons$ or $\mu(\cons)=\cons$, then we must have that
$\delta(\cons)=\cons$ since, if  $\mu(\cons)=\cons$, then by the presence of
$\mu(\cons)\prec\delta(\cons)$ in $\thr{P[X]_\cons}$, we would have to have also
$\delta(\cons)=\cons$. By point~\ref{it:unguard}, we have then
$\mu'(\mu(\cons))\prec \mu'(\cons)\in\thr{W[P[X]]_\cons}$, while by
\ref{it:sat}, $\mu'(\cons)\prec \delta'(\cons)=\delta'(\delta(\cons))\in\thr{W[P[X]]_\cons}$. But
then, adding the axiom $\mu'(\mu(\cons))\prec\delta'(\delta(\cons))$ yields
an isomorphic specification. 

Let us now consider the case 2). Having verified case 1), we can now assume that
$\mu(\cons)=\delta(\cons)$. If $\mu(\cons)=\delta(\cons)=c\not=\cons$, then
$\mu',\delta'$ are identities on $c$, which contradicts the assumption of
this case. I.e., $\delta(\cons)=\mu(\cons)=\cons$. But then by \ref{it:sat},
we have $\mu'(\cons)\prec\delta'(\cons)\in\thr{W[P[X]]_\cons}$, which is the
required axiom $\mu'(\mu(\cons))\prec\delta'(\delta(\cons))$. 
%
\item[\ref{it:unguard}] This follows trivially: by~\ref{it:unguard},
$\mu(\phi)\in\thr{P[X]_\cons}$ and then, by the same point,
$\mu'(\mu(\phi))\in\thr{W[P[X]]_\cons}$. 
%
\item[\ref{it:corax}] Follows equally easily. Let 
$x_1\prec\cons_1, \ldots ,x_m\prec\cons_m, \ovr{a} \to \ovr{b}$ be a fully
guarded axiom from $\thr{X_\cons}$. Then, by \ref{it:corax}, 
$x_1\prec\delta(\cons_1), \ldots ,x_m\prec\delta(\cons_m), \ovr{a} \to
\ovr{b}$ is in $\thr{P[X]_\cons}$. But then, by the same point, 
$x_1\prec\delta'(\delta(\cons_1)), \ldots ,x_m\prec\delta'(\delta(\cons_m)),
\ovr{a} \to \ovr{b}$ is in $\thr{W[P[X]]_\cons}$. 
\end{itemize}
\end{PROOF}

\subsection{Horisontal composition -- semantics}
As was the case with vertical composition, horisontal composition of pdts
gives us a more structured specification. 
According to
proposition~\ref{prop:horcomp}, composing horisontally two pdts, we obtain a
new pdt with the associated class of semantic functors. 
We show that semantics of a stepwise, horisontally
composed pdt, which can be written as
$\pdt;\pdtsp{\mu'}{\delta'}{\thr{P[X]_\cons}}{\thr{W[P[X]]_\cons}}$, is a
refinement of the seamntics of the respective composed pdt 
$\pdtsp{(\mu;\mu')}{(\delta;\delta')}{\thr{X_\cons}}{\thr{W[P[X]]_\cons}}$ --
the former, possessing more structure in the form of the intermediary stage
$\thr{P[X]_\cons}$, may put additional restrictions on the admissible
functors. Yet, composition of such funtors will always yield a functor for
the composed pdt.

\subsubsection{Horisontal composition as a refinement of pdt}\label{sub:horref}
%\subsubsection{Horisontal composition as an actualisation of a particular
%	semantic functor}\label{sub:horact}
\begin{proposition}
Given pdts $\pdt$ and
$\pdtsp{\mu'}{\delta'}{\thr{P[X]_\cons}}{\thr{W[P[X]]_\cons}}$, 
with the semantic functors
$\fu{F_\thr{X}}: \Mod(\thr{X_\cons}) \to 
\Mod(\thr{P[X]_\cons})$ and $\fu{F_\thr{P[X]}}: \Mod(\thr{P[X]_\cons}) \to
\Mod(\thr{W[P[X]]_\cons})$. The composition
%$\comp{\fu{F_\thr{X}}}{\fu{F_\thr{P[X]}}}$ 
$\comp{\fu{F_\thr{X}}}{\fu{F_\thr{P[X]}}}:\thr{X_\cons} \to
\thr{W[P[X]]_\cons}$, is a semantic functor for
$\pdtsp{(\mu;\mu')}{(\delta;\delta')}{\thr{X_\cons}}{\thr{W[P[X]]_\cons}}$. 
\end{proposition} 
The propoisition means that all the loops in the following digram commute:
\[ \hspace*{-2em}\xymatrix@C=0.3cm{
&& \Mod(\thr{X_\cons}) \ar[dd]^{\iota_{\thr{X}}} \ar@(l,l)[dddd]_{\iota'_{\thr{X}}} \ar[rrdd]_{\fu{F_\thr{X}}} \ar@(u,u)[rrrrdddd]^{\comp{\fu{F_\thr{X}}}{\fu{F_\thr{P[X]}}}}\\
\\
	&& \Mod(\thr{X_{\weak}})
	&& \Mod(\thr{P[X]_\cons}) \ar[ll]_{|_{\mu}} \ar[dd]_{\iota_{\thr{P[X]}}} \ar[rrdd]_{\fu{F_\thr{P[X]}}}\\
\\
	&&\Mod(\thr{X_{\weak}}) 
	&& \Mod(\thr{P[X]_{\weak}}) 
	&& \Mod(\thr{W[P[X]]_\cons}) \ar[ll]_{|_{\mu'}} \ar@(d,d)[llll]^{|_{\comp{\mu}{\mu'}}}\\
\\
}
\]
\begin{PROOF}
 $\iota'_{\thr{X}}:\Mod(\thr{X_\cons})\to\Mod(\thr{X_\weak})$ is defined by:
\begin{itemize}
\item $s^{\iota'(A)} = s^{\comp{\fu{F_\thr{X}}}{\fu{F_\thr{P[X]}}}(A)}$, for
sorts $s \in \Sigma(\thr{X_\cons})$ 
\item $\omega^{\iota'(A)} =
\omega^{\comp{\fu{F_\thr{X}}}{\fu{F_\thr{P[X]}}}(A)}$, for operations $\omega
\in \Sigma_\weak(\thr{X_\cons})$ 
\item $\cons_s^{\iota'(A)} =
{(\comp{\mu}{\mu'})(\cons_s)}^{\comp{\fu{F_\thr{X}}}{\fu{F_\thr{P[X]}}}(A)}$. 
\end{itemize}
%%Since $\iota_{\thr{X}}$ and $\iota_{\thr{P[X]}}$ are tight monomorphisms, so 
%% $\iota'_{\thr{X}}$ is a tight
%% $\Sigma_\weak(\thr{X})$-monomorphism, too.
{\Large{Yngve!!! This needs a real proof!!!}}
\end{PROOF}
The following example illustrates the refinement obtained by introducing an
intermediary parameter.
%
\begin{example}
We start with the specification of stacks from example~\ref{ex:stackelSP} and
refine it by introducing an intermediary parameter of non-empty stacks. 
To do that, we first refine the specification
$\thr{Stack[El]_\cons}\leadsto\thr{Stack[El]'_\cons}$, by adding constant
$nEs$ for non-empty stack and operation $new$ for constructing a basic
non-empty stack:
\[\xymatrix{
\frame{\txt{$\spec{
	\tit{\mbox{\bf spec\ El}_\cons^{{}}=} \\
		\spSort{El}
		\spOps{\allcons}{ \cons_{El}: \to El}
		\Gamma:
			& x \prec \cons_{EL}
}$}}
\ar[rrr]^<<<<<<<<<<<<{\txt{$\large{\mu(\cons_{El})=ok}$}}_<<<<<<<<<<<<{\txt{$\large{\delta(\cons_{El})={ok}}$}} &&&
\frame{\txt{$\spec{
\tit{\mbox{\bf spec\ Stack[El]}'_\cons=} \\
	\spSorts{\Sorts'}{Stack,El}
	\spOps{\Ops'}{empty: && Stack\\
		top: & Stack & El\\
		pop: & Stack & Stack\\
		push: & El \times Stack & Stack \\
		new: & El & Stack}
	\spOps{{\allcons}'}{	\cons_{El}, ok: && El \\
		\cons_{Stack}, nEs: && Stack }
	\Phi':
		&1.& && empty \eleq empty \\
		&2.& x \prec ok%, s\prec \cons_{Stack}%, push(x,s) \eleq push(x,s) 
			& \To & top(push(x,s)) \eleq x\\
		&3.& x \prec ok%, s\prec \cons_{Stack}%, push(x,s) \eleq push(x,s)
			 & \To & pop(push(x,s)) \eleq s \\
		&4.& x\prec ok & \To & new(x)\eleq push(x,empty) \\
		&5.& x\prec ok &\To & push(x,s)\prec nEs \\
	\Gamma':&6.& && x \prec \cons_{EL}\\
		&7.& && s \prec \cons_{Stack} \\
}$}}
}
\]
The intermediary parameter $\thr{nEstack[El]_\cons}$ is given by:
\[
\spec{
\tit{\thr{nEstack[El]_\cons}=}\\
	\spSorts{\Sorts''}{Stack,El}
	\spOps{\Ops''}{new: &El & Stack\\
		top: & Stack & El\\
		push: & El \times Stack & Stack}
	\spOps{{\allcons}''}{	\cons_{El}: && El \\
		\cons_{Stack}, nEs: && Stack }
	\Phi'':
		&2.&&  & top(push(x,s)) \eleq x\\
		&4.&&  & new(x)\eleq push(x,empty) \\
		&5.&&  & push(x,s)\prec nEs \\
	\Gamma'':&6.& && x \prec \cons_{EL}\\
		&7.& && s \prec \cons_{Stack} \\
		&7.& && s \prec nEs \\
}
\]
{\Large{Yngve!!! This example has to be fixed and ... it has to illustrate
that composed spec may yield fewer functors!!!}}
\end{example}

\subsection{Refinement}\label{sub:ref}
We now summarise the concept of refinement of pdt. As we have emphasised, it
amounts not only to the simple model class inclusion but, primarily, to
introduction of additional {\em structure} on the pdts. The following
definition captures the general concept

\begin{definition}
Given two pdts, $\thr P=\pdt$ and $\thr
P'=\pdtsp{\mu'}{\delta'}{\thr{X'_\cons}}{\thr{P[X]'_\cons}}$, then $\thr P'$
refines $\thr P$, $\thr P\leadsto\thr P'$, if there exist functors
$\reff X:\Mod(\thr  X_\cons)\to\Mod(\thr X'_\cons)$ and
$\reff{P[X]}:\Mod(\thr {P[X]'_\cons})\to\Mod(\thr{P[X]_\cons})$, such that
for any semantic functor $\fu F'$ for $\thr P'$, the functor $\reff X;\fu
F';\reff{P[X]}$ is a semantic funtor for $\thr P$.
\end{definition}
The following diagram illustrates the requirement:
\[ \xymatrix{
&& \Mod(\thr{X_\cons})  \ar[dd]_{\iota} \ar@{.>}[rrdd]^{\reff{X};\fu{F}';\reff{P[X]}} \ar[lldd]_{\reff{X}}\\
&& && \\
\Mod(\thr{X'_\cons}) \ar[dd]_{\iota'} \ar[rrrdd]^{\fu{F'}} %\ar[rruu]^{|_{\nu}}
	&& \Mod(\thr{X_{\weak}}) \ar@{}[ur]|{1.}
		&& \Mod(\thr{P[X]_\cons}) \ar[ll]_{|_{\mu}} \\
&& && \\
\Mod(\thr{X'_{\weak}}) \ar@{}[ur]|{2.} %\ar[uurr]_{|_{\nu}}  \ar@{}[uuuurr]|{2.}
	&&& \Mod(\thr{P[X]'_\cons}) \ar[lll]^{|_{\mu'}} \ar[uur]_{\reff{P[X]}} %\ar@{}[luu]|{3.}	
}
\]
A simple case is when $\thr{P[X]_\cons}\leadsto\thr{P[X]'_\cons}$, i.e., when
$\reff{P[X]}$ is a model class inclusion $\Mod(\thr{P[X]'_\cons})\subseteq
\Mod(\thr{P[X]_\cons})$ (cf. subsection~\ref{sub:vertref}).

The contravariance on the parameter side may seem a bit unusual since,
following the idea of refinement as the model class inclusion, one might
expect the refinement relation to hold also when
$\thr{X_\cons}\leadsto\thr{X'_\cons}$. However, one should keep in mind that
we are talking about design specifications of actual structure of data
types/programs. A pdt with source $\thr{X'_\cons}$ could not, in general,
replace a pdt with the source $\thr{X_\cons}\leadsto\thr{X'_\cons}$. 

Finally, we give an example showing an even more particular case of
refinement by adding structure, where the formal parameter of a pdt is itself
refined to a pdt.

\begin{example}
Let $\thr
P=\pdtsp{\mu}{\delta}{\thr{Set}_\cons}{\thr{\choice[Set]_\cons}}$ be the
following pdt which extends a specification of sets with a nondeterministic
binary choice.
\[\xymatrix{
\frame{\txt{$\spec{
\tit{\mbox{\bf spec\ Set}_\cons^{{}}=} \\
	\spSort{Set,El}
	\spOps{\Ops}{
	\emptyset: && Set \\
	\prep\ : & El\times Set & Set 
	}
	\spOps{\allcons}{ \cons_{El}: \to El \\ \cons_{Set}:\to Set}
	\Phi: & 1.& x\prep (y\prep S) & \eleq & y\prep (x\prep S) \\
	      & 2.& x\prep (x\prep S) & \eleq & x\prep S \\
	\Gamma:
	   & & x &\prec &\cons_{EL} \\ 
	   & & S &\prec &\cons_{Set}
}$}}
%\ar[rrr]^<<<<<<<<<<<{\txt{$\large{\mu(\cons)=\cons}$}}_<<<<<<<<<<<{\txt{$\large{\delta(\cons)=\cons}$}}
%	&&&
\ar@(u,r)[dr]^{\txt{$\large{\hspace*{1em}\mu(\cons_{El})=\delta(\cons_{El})=ok}$}}_{\txt{$\large{\mu(\cons_{Set})=\delta(\cons_{Set})=
	\cons_{Set}\ \ }$}}
\\ &
\hspace*{-4em}\frame{\txt{$\spec{
\tit{\thr{\choice[Set]}_\cons=} \\
	\spSort{Set,El}
	\spOps{\Ops}{
	\emptyset: && Set \\
	\prep\ : & El\times Set & Set \\
	\choice : & Set & El
	}
	\spOps{\allcons}{ ok,\cons_{El}: \to El \\ \cons_{Set}:\to Set}
	\Phi: & 1.& x\prec ok,y\prec ok &\To & x\prep (y\prep S)  \eleq  y\prep (x\prep S) \\
	      & 2.& x\prec ok & \To & x\prep (x\prep S)  \eleq  x\prep S \\
	      & 3.& z\prec \choice(x\prep S) & \To & z\eleq x, z\prec \choice(S)\\
	\Gamma:
   	    & &&& x \prec \cons_{EL} \\ 
	    & &&& S\prec \cons_{Set}
}$}}
}
\]
We let here $\mu(\cons_{El})=\delta(\cons_{El})=ok$, and
$\mu(\cons_{Set})=\delta(\cons_{Set})=\cons_{Set}$. The former is motivated
by the possibility of $\choice(\emptyset)$ returning a new, ``error'' element.

A possible semantic functor $\fu F$ may send a $\thr{Set}_\cons$ algebra $A$ on the
same algebra where, for any nonepty set $S$, the operation $\choice^{\fu
F(A)}(S)$ returns all the elements of the set $S$.

Obviously, the specification $\thr{Set_\cons}$ can be naturally parameterized
by elements, i.e., we ``extract'' from it a parameter (sub)specification. We
obtain then $\thr P'=\pdtsp{\mu'}{\delta'}{\thr{El}_\cons}{\thr{Set[El]_\cons}}$,
where $\thr{El_\cons}$ contains merely the sort $El$ and the global guard
$x\prec \cons_{El}$, while $\thr{Set[El]_\cons}$ is exactly the same as
$\thr{Set_\cons}$. $\mu'$ and $\delta'$ are identities on $\cons_{El}$. 

The point is now that the composed pdt $\thr P';\thr P$ is a refinement of
$\thr P$. 
\[\xymatrix@C=0.6cm@R=1.5cm{
\Mod(\thr{El_\cons}) \ar[drr]^{\fu F'} \ar@{^{(}.>}[d] && &&
  \Mod(\thr{Set_\cons})  \ar@{^{(}.>}[d]
    \ar[llll]_{|_{\mu'}\ =\ \reff{Set_\cons}}  \ar@{.>}[drr]^{\fu G} \ar@{.}[dll]|{id} \\
\Mod(\thr{El_\weak}) && \Mod(\thr{Set[El]_\cons})  \ar@{^{(}.>}[d] \ar[drr]^{\fu F} \ar[ll]_{|_{\mu'}}
   && \Mod(\thr{Set_\weak}) && \Mod(\thr{\choice[Set]_\cons})
   \ar@{.}[dll]|{id\ =\ \reff{\choice[Set]}} \ar[ll]_{|_{\mu}}\\ 
&& \Mod(\thr{Set[El]_\weak}) && \Mod(\thr{\choice[Set[El]]_\cons}) \ar[ll]_{|_{\mu}}
}
\]
%% The functor $\reff{\choice[Set[El]]}$ is the identity, while
%% $\reff{Set}:\Mod(\thr{Set_\cons})\to\Mod(\thr{El_\cons})$ is the reduct
%% functor $|_{\mu'}$.
Since $\mu'(\cons_{El})=\cons_{El}$ the $|_{\mu'}$ reduct will, actually,
return a $\thr{El_\cons}$-algebra.

Given $\fu F'$ and $\fu F$, the resulting functor $\fu G$ %obtained from this
							  %refinement 
will be, as a matter of
fact, the same as the functor $\fu F$. 

{\Large{Yngve!!! Check this!!!}}

However, the refinement consists in requiring a more structured data type
which has to consist of the composition of two functors $\fu F';\fu F$. In
this sense, it is reasonable to call $\thr P';\thr P$ a refinement of $\thr
P$.

{\Large{Yngve!!! Show this!!!}}
\end{example}
