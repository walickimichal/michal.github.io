
\subsection{Signatures with sort constants}\label{sub:gsp}
We start by modifiying the concept of signature and specification. The idea
is that each signature may have, in addition to the standard set of sort and
operation symbols, a (possibly empty) set of distinguished (sort and subsort) constant
symbols $\allcons = \Cons\cup\subcons$. 
The set $\Cons$ %=\{\cons_s:\to s: s\in\Sorts\}$ 
contains  constant symbols
$\cons_s$ for various sort symbols $s$ -- the intention of $\cons_s$ is to denote all the elements of the respective
sort $s$. 
The set $\subcons$  may contain additional constants
which will represent various subsorts -- the constants from this set are
called ``subsort constants''. (Usually, the distinction between $\Cons$ and
$\subcons$ does not matter and then we will write ``(sub)sort constants''.)
\begin{definition}
\label{def:modsign}
A signature with sort constants, $\Sigma_\cons$, is a triple $\Sigma_\cons =
 (\Sorts,\Ops,\allcons)$, where $\Sigma=(\Sorts,\Ops)$ is an ordinary
 signature and $\allcons=\Cons\cup\subcons$ is a (possibly empty) set of additional constants,
 $\allcons\cap\Ops=\emptyset$ and $\Cons\cap\subcons=\emptyset$. 
\end{definition}
%
Signatures with sort constants will be used merely as a syntactic representation of 
ordinary signatures. This is possible in multialgebraic setting since
constants may denote sets of elements. (In a more traditional setting, one would
have to represent the sort constants, for instance, by predicates or (sub)sort symbols.)

We allow the set $\allcons$ to be empty. Also, we allow the set
$\Cons$ to contain several distinct constants of the same sort (although
their intended meaning will be the same).
The technical reasons for that come up in relating construction of co-limits
(proposition~\ref{prop:finco}, especially, lemma~\ref{fa:coeq}) and, in particular, pushouts
(subsection~\ref{sub:canpush}). For the most, we think of the set
$\Cons$ as containing one constant for each sort, i.e., as
$\Cons=\{\cons_s:\to s : s\in\Sorts\}$. Most relevant constructions will
involve and yield such signatures. (in general, we use the symbol $\cons_s$
for an arbitrary constant from $\Cons$ of sort $s$, e.g., for a signature
morphism $\mu$, 
$\mu(c)\not=\cons_s$ means the same as $\mu(c)\not\in\Cons$.)


We will use the following operations relating the signatures with sort
 constants to ordinary signatures.

\begin{definition}\label{def:sigops}
Given a signature with sort constants $\Sigma_\cons=(\Sorts,\Ops,\allcons)$ we let:
\begin{enumerate}\MyLPar
\item $\under{\Sigma_\cons}=(\Sorts,\Ops \cup \allcons)$
i.e. $(\Sorts, \Ops \cup \Cons \cup \subcons)$ -- the underlying signature
\item ${\Sigma_\weak}=(\Sorts,\Ops,\subcons)$ i.e. $\Sigma_\cons
\setminus \Cons$ -- the reduced signature
\item ${\Sigma}=(\Sorts,\Ops)$ i.e. $(\Sigma_\cons
\setminus \allcons)$ -- the standard (part of the) signature
\end{enumerate}
The other way around, given an ordinary signature $\Sigma=(\Sorts,\Ops)$, we let
\begin{enumerate}\MyLPar\setcounter{enumi}{3}
\item ${\Sigma_\cons}=(\Sorts,\Ops,\Cons)$
, where   $\Cons=\{\cons_s:\to s:s\in\Sorts\}$ and $\Cons\cap\Ops=\emptyset$
-- the corresponding guarded signature with sort constants.
\item\label{it:ins} $\ins\Sigma=(\Sorts,\Ops,\emptyset)$
, -- the included signature (with sort constants).
\end{enumerate}
\end{definition}
%
Unless stated otherwise, the signatures considered will always be signatures with sort constants. 
Given an arbitrary specifcation $\thr{SP}$ we will sometimes write
$\Sigma(\thr{SP})$ to denote its signature.

\begin{definition}
A morphism between signatures with sort constants $\mu:\Sigma_\cons \to
\Sigma'_{\cons}$ is a signature morphism between the underlying
signatures $\mu:\under{\Sigma_\cons}\to\under{\Sigma'_\cons}$, sending
$\allcons$ to ${\allcons}'$ and $\Sigma$ to $\Sigma'$.
\end{definition}
In other words, the $\Cons$-constants need not be sent to $\Cons'$-constants
but may be mapped to subsort constants ${\subcons}'$, as well.

\begin{fact}
The signatures with sort constants form a category $\cat{Sign_\cons}$, with
the identity function as identity and function composition as composition.
\end{fact}
%
Since the signatures with sort constants essentially use underlying signature
morphism, the transformation from the former to the latter
%signatures with sort constants to
%underlaying signatures 
can be extended to a functor $\fu{_-}:\cat{Sign_\cons} \to
\cat{Sign}$, which is the (pointwise) identity on the signature morphisms. 

The other way arround the inclusion of signatures (point~\ref{it:ins} of
def.~\ref{def:sigops}) can also be extended to a
functor $\fu{\ins\ }:\cat{Sign} \to \cat{Sign_\cons}$, which sends $\Sigma$
to $\ins\Sigma$ and is the identity on morphisms. The following fact says
that $\cat{Sign}$ can be treated as a full subcategory of $\cat{Sign_\cons}$.

\begin{fact}
$\ins\ :\cat{Sign} \to \cat{Sign_\cons}$ is full and faithfull.
\end{fact}
We have that for any $\Sigma\in\cat{Sign}:\Sigma=\under{(\ins\Sigma)}$ but,
in general, for $\Sigma_\cons\in\cat{Sign_\cons}: \Sigma_\cons\not\iso
\ins{(\under{\Sigma_\cons})}$. This is because for an isomorphism in
$\cat{Sign_\cons}$ we must have isomorphism between the respective sets of
(sub)sort constants, but while $\allcons$ in $\Sigma_\cons$ may be
non-empty, it is always empty in $\ins{(\under{\Sigma_\cons})}$.

%%\begin{lemma}
%%$\fu{_-}:\cat{Sign_\cons} \to \cat{Sign}$ is a left adjoint to $\ins\ :\cat{Sign} \to \cat{Sign_\cons}$.
%%\end{lemma}
%%\begin{PROOF}
%%Suppose that $\sigma: \Sigma_\cons \to \ins{(\Sigma')}$ is a
%%$\cat{Sign_\cons}$ morphism, then $\under{\sigma}: \under{\Sigma_\cons} \to
%%\under{\ins{(\Sigma')}}$ (note that $\under{\ins{(\Sigma')}} = \Sigma'$) is
%%the unique factorization morphism, since $\ins{\under{\sigma}}=
%%\under{\sigma}$ and $\ins\ :\cat{Sign} \to \cat{Sign_\cons}$ is full and
%%faithfull. 
%%\end{PROOF}
%%
%%\begin{corollary}
%%The functor $\fu{_-}$ preserves co-limits.
%%\end{corollary}
%
The following proposition shows that
 that (finite) co-limits in $\cat{Sign}_\cons$
can be obtained from the respective co-limits in $\cat{Sign}$. 

\begin{proposition}\label{prop:finco}
The functor $\fu{_-}:\cat{Sign_\cons} \to \cat{Sign}$ reflects (and
preserves) finite co-limits.
\end{proposition}
%
To prove the proposition we show that the functor reflects initial object, sums and co-equlizers.

\begin{lemma}
The functor $\under{\ }$ reflects (and preserves) initial objects.
\end{lemma}
%
\begin{PROOF}
The empty signature $\Sigma^\emptyset=(\emptyset,\emptyset)$ is {\em the}
initial object in 
$\cat{Sign}$, and the empty signature
$\Sigma_\cons^\emptyset=(\emptyset,\emptyset,\emptyset)$ is {\em the} initial
object in
$\cat{Sign_\cons}$.
But $\Sigma^\emptyset=\under{\Sigma_\cons^\emptyset}$, so initial object is
both reflected and preserved.
\end{PROOF}%\vspace*{-2ex}
%
\begin{lemma} The functor $\under{\ }$ reflects (and preserves) sums (binary co-products).
\end{lemma}
%
\begin{PROOF}
Let $\Sigma_{\cons 1}=(\Sorts_1,\Ops_1,\allcons_1)$ and 
$\Sigma_{\cons 2}=(\Sorts_2,\Ops_2,\allcons_2)$ be $\cat{Sign_\cons}$ objects and
$T$ be a co-cone 
%$\iota_1:\Sigma_{\cons 1}\to\Sigma_\cons$, $\iota_2:\Sigma_{\cons 2}\to\Sigma_\cons$ 
in $\cat{Sign_\cons}$ as in the left diagram. Assume that its image $\under
T$ (as in the right diagram) 
%, i.e., $\under{\iota_1}:\under{\Sigma_{\cons 1}}\to\under{\Sigma_\cons}$,
%$\under{\iota_2}:\under{\Sigma_{\cons 2}}\to\under{\Sigma_\cons}$  
is a
co-limit (sum) in $\cat{Sign}$. We have to show that $T$ is a sum in
$\cat{Sign_\cons}$.
\[\xymatrix@R=0.6cm@C=0.5cm{
T  & \Sigma_\cons &   &&& \under T & \under{\Sigma_\cons}\iso
\under{\Sigma_{\cons 1}}+\under{\Sigma_{\cons 2}}\\
\Sigma_{\cons 1} \ar[ur]_{\iota_1} && \Sigma_{\cons 2} \ar[ul]^{\iota_2} 
  &&&
 \under{\Sigma_{\cons 1}} \ar[ur]_{\under{\iota_1}} &&
    \under{\Sigma_{\cons 2}} \ar[ul]^{\under{\iota_2}}
}
\]
By the standard construction $\under{\Sigma_\cons}\iso \under{\Sigma_{\cons
1}} + \under{\Sigma_{\cons 2}}$, where the latter denotes disjoint union (of
sort and operation symbols from both signatures). To simplify the notation,
let us assume, without loss of geenrality, that we have equality here. Then
both $\under{\iota_i}$'s are injections,

Let $C$ be any co-cone $\mu_1:\Sigma_{\cons 1}\to\Sigma'_\cons$, $\mu_2:\Sigma_{\cons
2}\to\Sigma'_\cons$  in $\cat{Sign_\cons}$. Since $\under
T$ is a sum, we have a unique mediator to $\under C$, 
$u_{\under{\mu_{1}},\under{\mu_{2}}}:
\under{\Sigma_\cons}\to\under{\Sigma'_\cons}$, such that
$\under{\mu_i}=\under{\iota_i};u_{\under{\mu_{1}},\under{\mu_{2}}}$ for
$i\in\{1,2\}$.
It is given by: for any symbol
$x\in\under{\Sigma_\cons}:
u_{\under{\mu_{1}},\under{\mu_{2}}}(x)=\left\{\begin{array}{ll} 
 \under{\mu_1}(x) & if\ x\in\under{\Sigma_{\cons 1}} \\
 \under{\mu_2}(x) & if\ x\in\under{\Sigma_{\cons 2}}
\end{array}\right.$.
\[\xymatrix@R=0.8cm@C=0.7cm{
C & \Sigma'_\cons &  &&& \under C & \under{\Sigma'_\cons} \\
& \Sigma_\cons \ar@{.>}[u]|{u_{\mu_{1},\mu_{2}}}&  &&&  
  & \under{\Sigma_{\cons 1}}+\under{\Sigma_{\cons 2}} \ar@{.>}[u]|{u_{\under{\mu_{1}},\under{\mu_{2}}}}\\
\Sigma_{\cons 1} \ar[ur]_{\iota_1} \ar@(u,l)[uur]^{\mu_1} 
  && \Sigma_{\cons 2} \ar[ul]^{\iota_2}  \ar@(u,r)[uul]_{\mu_2} 
  &&&
 \under{\Sigma_{\cons 1}} \ar[ur]_{\under{\iota_1}} \ar@(u,l)[uur]^{\under{\mu_1}}
  &&  \under{\Sigma_{\cons 2}} \ar[ul]^{\under{\iota_2}}  \ar@(u,r)[uul]_{\under{\mu_2}}
}
\]
The claim is that there is also a unique mediator
$u_{\mu_{1},\mu_{2}}:\Sigma_\cons\to\Sigma'_\cons$. Indeed, let it be given
by $u_{\mu_{1},\mu_{2}}(x)= u_{\under{\mu_{1}},\under{\mu_{2}}}(x)$ for all
$x\in\Sigma_\cons$ (then $\under{u_{\mu_{1},\mu_{2}}}=
u_{\under{\mu_{1}},\under{\mu_{2}}}$). 
 It obviously
makes $\mu_i=\iota_i;u_{\mu_{1},\mu_{2}}$, for $i\in\{1,2\}$. 

It is also a
$\cat{Sign_\cons}$ morphism because all $\iota_i$'s and $\mu_i$'s
are: the (sub)sort constants $\allcons_1, \allcons_2$ from $\Sigma_{\cons 1},
\Sigma_{\cons 2}$, respectively, are mapped by $\iota_1,\iota_2$ to (sub)sort constants
${\allcons}$ in $\Sigma_\cons$. Their $\iota_i$ images are also {\em all} the (sub)sort
constants ${\allcons}$, since the other symbols in $\Sigma_\cons$ are images
of $\Sigma_1$, resp., $\Sigma_2$ symbols (i.e., of those symbols from
$\Sigma_{\cons 1}, \Sigma_{\cons 2}$ which are {\em
not} (sub)sort constants). Then, since also all the images under $\mu_i$ of (sub)sort constants
from $\Sigma_{\cons i}$ are (sub)sort constants in $\Sigma'_\cons$, it
follows from the definition of $u_{\mu_{1},\mu_{2}}$ that it, too, maps
(sub)sort constants -- of the form $c=\iota_i(c)$ -- to (sub)sort constants,
since $u_{\mu_{1},\mu_{2}}(c)=\mu_i(c)$, for respective $i$'s.

Finally, this $u_{\mu_{1},\mu_{2}}$ is unique making
$\mu_i=\iota_i;u_{\mu_{1},\mu_{2}}$. For if there is another $u\not=
u_{\mu_{1},\mu_{2}}$, such that $\mu_i=\iota_i;u$, then it would also be the
case that $\under u\not= u_{\under{\mu_{1}},\under{\mu_{2}}}$ and,
furthermore, that
$\under{\mu_i}=\under{\iota_i};\under u$, contradicting the
uniqueness of $u_{\under{\mu_{1}},\under{\mu_{2}}}$. 

Preservation of sums follows now easily. A sum $\Sigma_{\cons
1}+\Sigma_{\cons 2}$ in $\cat{Sign_\cons}$ must be
isomorphic to a sum as given above, i.e., a disjoint union of all the symbols
from both signatures: 
%i.e., for 
%$\Sigma_{\cons 1}=(\Sorts_1,\Ops_1,\allcons_1)$ and 
%$\Sigma_{\cons 2}=(\Sorts_2,\Ops_2,\allcons_2)$, 
%we have 
$\Sigma_{\cons 1}+\Sigma_{\cons 2}\iso
(\Sorts_1\uplus\Sorts_2,\Ops_1\uplus\Ops_2,\allcons_1\uplus\allcons_2)$ where
also $(\Ops_1\uplus\Ops_2)\cap(\allcons_1\uplus\allcons_2)=\emptyset$. 
But
then $\under{\Sigma_{\cons 1}+\Sigma_{\cons 2}} \iso
(\Sorts_1\uplus\Sorts_2,(\Ops_1\cup\allcons_1)\uplus(\Ops_2\cup\allcons_2))\iso
\under{\Sigma_{\cons 1}} + \under{\Sigma_{\cons 2}}$.
\end{PROOF}
%
\begin{lemma}\label{fa:coeq}
The functor $\under{\ }$ reflects (and preserves) co-equalizers. 
\end{lemma}
%
\begin{PROOF}
Let $\Sigma_\cons=(\Sorts,\Ops,{\allcons})$,
$\Sigma'_\cons=(\Sorts',\Ops',{\allcons}')$,
$\Sigma''_\cons=(\Sorts'',\Ops'',{\allcons}'')$ be objects in
$\cat{Sign_\cons}$ and
suppose that we have a co-cone in $\cat{Sign_\cons}$ as on the left diagram
(with $\mu_1;\sigma=\mu_2;\sigma$), 
%%$\mu,\nu:\Sigma_\cons \to \Sigma'_\cons, \sigma:\Sigma'_\cons
%%\to \Sigma''_\cons$, is a co-cone in $\cat{Sign_\cons}$ and that
%%$\under{\mu},\under{\nu}:\under{\Sigma_\cons} \to \under{\Sigma'_\cons},
%%\under{\sigma}:\under{\Sigma'_\cons} \to \under{\Sigma''_\cons}$, 
and that its image (on the right diagram)
is a co-limit (co-equalizer) in $\cat{Sign}$. We have to show that the
original co-cone (on the left) is a co-limit in $\cat{Sign_\cons}$.
\[\xymatrix@C=1.2cm{
\Sigma_\cons \ar@<0.7ex>[r]^{\mu_1} \ar@<-0.7ex>[r]_{\mu_1} 
  & \Sigma'_\cons \ar[r]^{\sigma} 
  & \Sigma''_\cons
&&&
\under{\Sigma_\cons} \ar@<0.7ex>[r]^{\under{\mu_1}}
\ar@<-0.7ex>[r]_{\under{\mu_1}} 
  & \under{\Sigma'_\cons} \ar[r]^{\under{\sigma}} 
  & \under{\Sigma''_\cons} \iso \qu{\under{\Sigma'_\cons}}{\approx}
}
\]
By the standard construction in $\cat{Sign}$, we have an isomorphism $\under{\Sigma''_\cons} \iso
\qu{\under{\Sigma'_\cons}}{\approx}$, where
$\qu{\under{\Sigma'_\cons}}{\approx}=(\qu{\Sorts'}{\approx},\qu{\Ops'}{\approx})$, is the standard choice of
co-equalizer i.e. the quotient by  the least equivalence $\approx$ on
$\under{\Sigma'_\cons}$ induced by the 
relation with the following components:
 \begin{enumerate}\MyLPar
\item Sorts: $\approx_{\Sorts'}= \{ \langle \under{\mu_1}(s),\under{\mu_2}(s)
\rangle: s \in \Sorts \}$,
%\under{\Sigma_\cons} \}$,
\item Operations: $\approx_{\Ops'\cup{\allcons}'}= \{ \langle \under{\mu_1}(\omega), \under{\mu_2}(\omega)
\rangle : \omega \in \Ops\cup{\allcons} \}$
%\under{\Sigma_\cons} \}$
\end{enumerate}
To simplify the notation we will assume, without loss of generality, that, in
fact,  $\under{\Sigma''_\cons} = \qu{\under{\Sigma'_\cons}}{\approx}$. 

Since $\mu_1,\mu_2$ are $\cat{Sign_\cons}$-morphisms, the equivalence
$\approx$ above can be viewed (is the same) as the least equivalence $\sim$ 
on $\Sigma'_\cons$ induced by the following components:
%(from the original $\Sigma'_\cons$) 
 \begin{enumerate}\MyLPar\setcounter{enumi}{2}
\item\label{it:s1} Sorts: $\sim_{\Sorts'}= \{ \langle \mu_1(s),\mu_2(s)
\rangle: s \in \Sorts \}$,
%{\Sigma_\cons} \}$,
\item Operations: $\sim_{\Ops'}= \{ \langle \mu_1(\omega), \mu_2(\omega)
\rangle : \omega \in \Ops \}$
\item\label{it:sub1} (sub)sort constants: $\sim_{{\allcons}'}= \{ \langle 
\mu_1(c), \mu_2(c) \rangle : c \in \allcons \}$.
\end{enumerate}
%
We then have
$\under{\qu{\Sigma'_\cons}{\sim}}=\qu{\under{\Sigma'_\cons}}{\approx}$, so we
let $\Sigma''_\cons=\qu{\Sigma'_\cons}{\sim}$.

Let $\mu_1,\mu_2,\gamma$ be an arbitrary co-cone as shown on the left diagram.
\[\xymatrix@C=1.2cm@R=0.7cm{
\Sigma_\cons \ar@<0.7ex>[r]^{\mu_1} \ar@<-0.7ex>[r]_{\mu_1} 
  & \Sigma'_\cons \ar[r]^{\sigma} \ar[dr]_\gamma
  & \Sigma''_\cons = \qu{\Sigma'_\cons}{\sim} \ar@{.>}[d]^{u_\gamma}
&&&
\under{\Sigma_\cons} \ar@<0.7ex>[r]^{\under{\mu_1}}
\ar@<-0.7ex>[r]_{\under{\mu_1}} 
  & \under{\Sigma'_\cons} \ar[r]^{\under{\sigma}} \ar[dr]_{\under{\gamma}}
  & \qu{\under{\Sigma'_\cons}}{\approx} \ar@{.>}[d]^{u_{\under\gamma}}
\\
&& \Sigma'''_\cons &&& && \under{\Sigma'''_\cons}
}
\]
Since the image $\under\sigma,\
\under{\Sigma''_\cons}=\under{\qu{\Sigma'_\cons}{\sim}}=\qu{\under{\Sigma'_\cons}}{\approx}$ is
co-equalizer, we have a unique mediator $u_{\under\gamma}$ making
$\under\gamma=\under\sigma;u_{\under\gamma}$. We show that $u_\gamma:\Sigma''_\cons\to\Sigma'''_\cons$, given
by $u_\gamma(x)=u_{\under\gamma}(x)$ for all symbols $x\in\Sigma''_\cons$ (in
particular, $\under{u_\gamma}=u_{\under\gamma}$) is
a unique mediator in $\cat{Sign_\cons}$. It obviously makes
$\gamma=\sigma;u_\gamma$. 

It is also a morphism in $\cat{Sign_\cons}$. Since both $\sigma,\gamma$ are
morphisms in $\cat{Sign_\cons}$ they send (sub)sort constants ${\allcons}'$
to the (sub)sort constants ${\allcons}''$, respectively,
${\allcons}'''$. Since $\under\sigma$ is surjective, then so is $\sigma$, and
thus the (sub)sort constants in $\Sigma''_\cons$ are exactly the
$\sigma$-images of (sub)sort constants ${\allcons}'$ from $\Sigma'_\cons$. By
definition of $u_\gamma$ and the fact that $\gamma=\sigma;u_\gamma$, this
means that for every (sub)sort constant $\sigma(c)=[c]\in{\allcons}''$,
$u_\gamma([c])=\gamma(c)\in{\allcons}'''$. 

Finally, $u_\gamma$ is a unique mediator. For if there was another
$u\not=u_\gamma$ making $\gamma=\sigma;u$, then we would also have $\under
u\not=u_{\under\gamma}$ and $\under\gamma=\under\sigma;\under u$,
contradicting the uniqueness of $u_{\under\gamma}$.

The fact that co-equalizers are also preserved by the functor $\under{\ }$
follows now easily. A co-equalizer of $\mu_1,\mu_2$ must be isomorphic to the
quotient $\qu{\Sigma'_\cons}{\sim}$, with $\sim$ defined by the points
\ref{it:s1}.-\ref{it:sub1}. above. But then its image
$\under{\qu{\Sigma'_\cons}{\sim}}$ is trivially isomorphic
to the co-equalizer $\qu{\under{\Sigma'_\cons}}{\approx}$ in
$\cat{Sign}$. 
\end{PROOF}
%
Since we can create all finite co-limits by initial objects, sums and
co-equlizers the proposition~\ref{prop:finco} follows from the above lemmata.
The concrete way of doing this is to construct a co-limit in $\cat{Sign}$ and
then make an appropriate choice of the (sub)sort constants,
according to the prescriptions given in the proofs above.