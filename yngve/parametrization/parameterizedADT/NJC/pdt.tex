\section{Introduction}
The important distinction between parameterized specifications and specifications
of parameterized programs has been originally pointed out in \cite{para}. An 
important difference concerns the objects which the two kinds of parameterization
allow one to reuse. 

A {\em parameterized specification}, ``PSP'', e.g., of stacks parameterized
by elements, \thr{Stack(El)}, allows
one, under some conditions, to instantiate
the formal parameter specification \thr{El} with an actual parameter specification, say \thr{Nat}, to
obtain a new specification \thr{Stack(Nat)}. What is reused is the {\em text} of
the parameterized specification. Except for that, the model classes of \thr{Stack(El)}
and \thr{Stack(Nat)} are just model
classes of two different specifications, related merely by the existence of a
reduct functor from the latter to the former, 
e.g. \cite{Alge}.

{\em Specification of a parameterized data type}, ``PDT'' \cite{para,para1}, on the other hand, is a
specification which requires a reusable {\em implementation}. E.g.,
specification of a data type stack parameterized by elements, \thr{Stack[El]},
requires an {\em implementation of a data type with a parameter}, i.e., one capable
of taking any implementation of \thr{El} and resulting in an implementation of
\thr{Stack[El]}. The model class of such a specification should be thus seen as
consisting of some -- perhaps all -- functors sending models of the formal
parameter specification \thr{X} to models of the parameterized specification $\thr{P[X]}:$
\begin{equation}
\FMod(\thr{P[X]}) \subseteq \{F:\Mod(\thr{X})\to\Mod(\thr{P[X]})\}. \label{eq:fsem}
\end{equation}
There has been surprisingly little work done wrt. to this latter kind of
parameterization (recent work on CASL, \cite{CASL}, is a valuable
exception). ``Surprisingly'' because the idea of specifying actual 
architecture of designed software should appeal to one's sense of potential
applicability of specifications. 
We mention a few possible reasons of this negligence which also indicate the
approach of this paper.

Study of PSPs has long tended in the direction of PDTs \cite{paramADJ,Alge,paramGanz,CATs}. 
%also: Ganzinger,Poigne
One of the problems
is that, while the former continued the tradition of working with classes
axiomatized by (possibly conditional) equations, the latter require a precise
grasp on individual algebras (which, for modeling purposes, can be
identified with programs): a program $P$ taking as a parameter another program
$X$ cannot change $X$ -- $X$ functions in the context of $P$, that is in $P[X]$, in the
same way as it would in isolation. This intuition of ``preserving actual
parameter'' has been identified as one of the semantic requirements on PSP in
form of the persistency requirement on the functors from (\ref{eq:fsem}),
e.g., \cite{paramGanz,paramPers,Alge}.
However, in the purely equational context, there was hardly any syntactic
counterpart of this semantic requirement. Thus, no syntactic/logical means
were available for reasoning about correctness of such implementations.

Even worse, persistency turns out to be all too strong a requirement
eliminating many (if not most) interesting examples of PDTs as
``illegal''. For instance, a functor which takes an \thr{El}-algebra of elements and
adds a new element (intended, e.g., as the ``error value'' resulting from inspecting the top of an empty
stack in the specification \thr{Stack[El]}) is {\em not} persistent. In
general, most free functors are not persistent, since they involve,
typically, generation of new elements. Partial algebras admit more free
functors than the total ones (since ``error'' elements remain simply
undefined), but they still exclude the possibility of adding ``new'' elements
and, in particular (as pointed out e.g. in \cite{partial}), of explicit error treatment.

We introduce a more adequate framework for specifying PDTs. The first
thing is a generalization of the classical concept of persistent functors, so
that our semantic functors can add new elements to the parameter algebras. 
This is the basic idea (going back to the short paper~\cite{paramPoigne})
underlying our approach.

Although the semantic definition of the desirable functors is simple, 
a lot of book-keeping is required on the side of the syntax. 
Section \ref{se:paraADT} begins by introducing the syntactic preliminaries
needed for specification of PDTs, and then defines the syntax and semantics
of PDTs. Section \ref{se:app} discusses syntax and semantics of actual
parameter passing. Section \ref{se:compref} shows the counterparts of the
classical, vertical and horizontal composition theorems. In this connection,
we also encounter the concept of refinement of PDTs. Since PDTs correspond
more to the design-, and not only to the requirement-specifications, thier
refinement reflects more specific design decisions. Unlike the classical concept of
model class inclusion, refinement of PDTs amounts to introduction of additional
structure. For instance, identifying a part of a (flat)
specification as a parameter, amounts to requiring a structured, i.e.,
parameterized (rather than a flat) implementation. This section leads to a
general concept of such a refinement, exemplifying the idea of ``constructor
specifications'' from \cite{para1}, which is summarized in section \ref{sub:ref}.
Section \ref{se:conc} contains some concluding remarks.

The framework is presented using multialgebras \cite{calc,multi} and thus, in particular, allowing
unrestricted use of nondeterministic data types. However, we do not focus on the issue
of nondeterminism here and multialgebras are chosen merely as an example of a
framework capable of expressing the intended concepts. 
Entirely analogous definitions and constructions to those
presented here can be done in any institution where the category of
signatures has pushouts, the model functor is (finitely) continuous
(or else, with amalgamation property) and where the signatures
can express predicates. Thus our  results can be used in a
straightforward way, for instance, for membership algebras \cite{member}, 
for total/partial algebras with predicates, etc. This is one of the reasons
why we did not attempt to define a new institution (of PDTs), but introduced the
required concepts ``on the top'' of the institution of multialgebras. 

The
reader is assumed to have elementary knowledge of category theory and
algebraic specifications. Some
familiarity with the concept of institution would be advantageous but isn't necessary. To make
the paper (almost) self-contained, we
begin now, in section~\ref{se:ma}, by reviewing the relevant concepts from
institutions and, in particular, the institution of multialgebras.

Most of the proofs are omitted here -- interested reader may find them in \cite{TRparam}.

\section{Preliminaries -- institution of multialgebras}\label{se:ma}
We list here only relevant definitions and facts. For more detailed
discussion of these concepts and proofs of the theorems concerning
institutions, the reader is referred, for instance, to
\cite{stateinst}, while for those
concerning the institution of multialgebras to \cite{MAinst}.
We start with the general concepts related to institutions.
The reader completely unfamiliar with  institutions may
go directly to subsection~\ref{sub:ma} and inspect subsection~\ref{sub:inst}
only if necessary later on.


\subsection{Institutions}\label{sub:inst}
An institution \cite{inst,stateinst} is an abstract form for a logical framework. Showing that a particular
specification formalism forms an institution allows one to apply some 
facts, properties and constructions which have been demonstrated generally for all, or some
particular institutions.
%
\begin{definition}\label{de:inst}
An institution is a quadruple $\inst{I} = (\Sign,\Sen,\Mod,\models)$, where:
\begin{itemize}\MyLPar
\item $\Sign$ is a category of signatures.  
\item $\Sen: \Sign \to \Set$ is a functor which associates a set of {\it sentences} to each signature.
\item $\Mod:\Sign^{op} \to \cat{Cat}$ is a functor which associates a
category of {\it models}, whose morphisms are called $\Sigma$-morphisms, to
each signature $\Sigma$
\item $\models$ is a satisfaction relation -- for each signature $\Sigma$, a relation
 $\models_{\Sigma} \subseteq |\Mod(\Sigma)| \times
\sen(\Sigma)$, such that the following {\it satisfaction condition} holds:
for any $M' \in \Mod(\Sigma'), \mu: \Sigma \to \Sigma' , \phi \in \sen(\Sigma)$
  \[ M' \models_{\Sigma'} \Sen(\mu)(\phi)\ \  \mbox{ iff }\ \ \Mod(\mu)(M') \models_{\Sigma} \phi\]
	\end{itemize}
\end{definition}
%
Given an institution, one defines the category of theories (or
specifications), $\cat{Th}$, with objects being pairs $(\Sigma,\Gamma)$ of a
signature $\Sigma$ and a set of $\Sigma$-sentences $\Gamma$. Models of a
theory are obtained by restricting the $\Mod$ functor, i.e.,
$\Mod(\Sigma,\Gamma)=\{M\in\Mod(\Sigma):M\models_\Sigma\Gamma\}$. A morphism
in $\cat{Th}$, $\mu:(\Sigma,\Gamma)\to(\Sigma',\Gamma')$, 
is a {\em specification morphism}, that is, a signature morphism
$\mu:\Sigma\to\Sigma'$ such that $\Gamma'\models\mu(\Gamma)$. 

Putting  theories together is done by
constructing their co-limits. The following institution independent result ensures that the category of
specifications has all co-limits if the signature category has.  This
result is used to create co-limits of specifications by first creating the
co-limit for the corresponding signatures. The functor $\fu{Sign}$ forgets the
axioms and retains the signature of a specification.
%
\begin{proposition} %\cite{inst}
\label{pro:reflects}
The functor  $\fu{Sign}: \thr{Th} \to \thr{Sign}$ reflects co-limits, in any institution $\inst{I}$.
\end{proposition}
%
In particular, passing an actual parameter specification to a
parameterized specification corresponds to constructing a pushout. The
desirable property is that such a construction at the level of syntax be
reflected at the level of model classes. This is ensured in a particular kind
of institutions, which will be of importance in this paper, 
the {\em exact} institutions.
%
\begin{definition} An institution $\inst{I}$ is 
\begin{enumerate}\MyLPar
\item {\em semi-exact} iff $\Sign$
has pushouts and $\Mod$ sends pushouts in $\Sign$ to pullbacks in
$\cat{Cat}$,
\item {\em exact} iff $\Sign$ has finite co-limits and $\Mod$ sends
finite co-limits in $\Sign$ to limits in $\cat{Cat}$.
\end{enumerate}
\end{definition}
Although the weaker property of semi-exactness is sufficient for treatment
of parameterization and actualization, we will work
with an exact institution. 

Semi-exactness is of uttermost
relevance for treatment of PDTs because it implies the
following, more specific property.
%
\begin{lemma}\label{le:amalg}(Amalgamation Lemma).\\
In any semi-exact institution $\inst{I}$, for every pushout of
signatures (on the left):
\[\xymatrix@R=0.3cm@C=0.6cm{
\Sigma \ar[dd]_{\mu_{2}} \ar[rr]^{\mu_1} && {\Sigma_1} \ar[dd]^{{\mu'}_2}  &&&& 
  \Mod(\Sigma) && \Mod(\Sigma_1) \ar[ll]_{\_|_{\mu_{1}}}  \\
&&  &\ar@{=>}[rr]^{\Mod} &&&\\
{\Sigma_2} \ar[rr]_{{\mu'}_1}	&& {\Sigma'} &&&&
   \Mod(\Sigma_2) \ar[uu]^{\_|_{\mu_{2}}} && \Mod(\Sigma')
   \ar[uu]_{\_|_{\mu'_{2}}} \ar[ll]^{\_|_{\mu'_{1}}}  
}
\]
we have that: for any two models $M_1 \in \Mod({\Sigma_1})$ and $M_2 \in
\Mod({\Sigma_2})$ satisfying $M_1|_{{\mu}_1} = M_2|_{{\mu}_2}$, there exists 
a unique model $M' \in \Mod({\Sigma'})$, such that $M'|_{{\mu'}_1} = M_2$ and
$M'|_{{\mu'}_2} = M_1$.
\end{lemma}
%
%The corresponding amalgamation property holds also for homomorphisms. 
(The amalgamation lemma holds then also for pushouts of
specifications, since these are constructed from pushouts of signatures by proposition~\ref{pro:reflects}.)
The amalgamation lemma tells that the model class of a
pushout $\Sigma'$ of signatures along $\mu_1,\mu_2$ is a pullback (in $\cat{Cat}$) of the respective
morphisms $\_|_{\mu_{1}}$, $\_|_{\mu_{2}}$. 
The model $M'$, denoted $M_1\amalgam_M M_2$ (where $M=M_1|_{\mu_{1}}=M_2|_{\mu_{2}}$), is called the {\em
amalgamated sum} of $M_1$ and $M_2$. This property allows one to give a
uniform definition of functorial semantics of actual parameter passing, as we
will see in a later section.


\subsection{Multialgebras}\label{sub:ma}
Signatures for multialgebras are the standard algebraic signatures, i.e.,
pairs $(\Sorts,\Ops)$ where $\Sorts$ is a set of sort names, and $\Ops$ a
set of operation names with argument sorts and result sort from
$\Sorts$. Also terms (over a signature $\Sigma$, with variables $X$, denoted
$T(\Sigma,X)$) are built in the usual way. 
Unlike in standard algebras, operation in a multialgebra may return
(possibly empty) sets of elements.
%
\begin{definition}\label{def:ma}
A {\em multialgebra} $A$ for a signature $\Sigma=(\Sorts,\Ops)$ is given by:
\begin{itemize}\MyLPar
\item  a set $s^A$, the carrier set, for each sort symbol $s\in\Sorts$
\item  a subset $ c^A \in {\mathcal{P}}(s^A)$, for each constant, $c:\to s$
\item an additive operation $\omega^A : s_1^A \times \cdots \times s_k^A \to {\mathcal{P}}(s^A)$
	for each symbol $\omega : s_1\times\cdots\times s_k \to s \in \Ops$
\end{itemize}
\end{definition}
``Additivity'' means that operations are composed pointwise by collecting the
results of applications to all elements of argument sets, i.e., for sets of elements
$t_1,\ldots,t_n$ (of appropriate sorts) and operation $\omega:
\omega^A(t_1 , \ldots ,t_n)) = \bigcup_{a_{i}\in t_{i}}\omega^A(a_1 , \ldots
,a_n)$. 
%One sometimes demands that constants and operations are total
%\cite{multi,toplas}, i.e. never return empty set and take values only in
%${\mathcal{P}}^+ (s^A)$, the nonempty subsets of $s^A$. We will not make this
%assumption.% unless else is stated.

Note that for a constant $c \in \Ops$, $c^A$ denotes possibly a subset of the
carrier $s^A$. This allows one to use constants as predicates as was done,
for instance, in \cite{partial}. This will be the main aspect of
multialgebras used in this paper.

As homomorphisms of multialgebras, we use weak homomorphisms (see
\cite{catrel} for alternatives).
\begin{definition}Given two multialgebras $A$ and $B$, 
a function $h: |A| \to |B|$ is a (weak) homomorphism if:
\begin{enumerate}\MyLPar
\item $h(c^A) \subseteq c^B$, for each constant $c: \to s$ 
\item $h(\omega^A(a_1 , \ldots , a_n)) \subseteq\omega^B(h(a_1), \ldots ,h(a_n))$,
for each operation  $\omega:s_1\times\cdots\times s_n\to s\in\Ops$ and for all $a_i \in s_{i}^A$.
\end{enumerate}
\end{definition}
Saying ``homomorphism'' we will always mean weak homomorphism.
The collection of all $\Sigma$-multialgebras with $\Sigma$-homomorphisms
forms a category. 

Multialgebraic specifications are written using the following formulae:
%
\begin{definition}\label{def:masen}
Formulae of multialgebraic specifications are of the following forms:
%{\bf{Atomic formula}}:
\begin{enumerate}\MyLPar
\item Atomic formulae, for terms $t, t' \in \TermsSX$:
\begin{itemize}\MyLPar
\item $t \eleq  t'$ (equality), $t$ and $t'$ denote the same one-element set.
\item $t \prec t'$ (inclusion), the set interpreting $t$ is included in
the set interpreting $t'$.
\end{itemize}
\item $a_1 , \ldots ,a_n \To b_1 , \ldots ,b_m$, where either $n>0$ or $m>0$ and
each $a_i$ and $b_j$ is atomic.
\end{enumerate}
\end{definition}
%
Given a set of variables $X$, an assignment is a function $\alpha: X \to |A|$ 
assigning {\em individual} elements (not sets!) from the carrier of $A$ to
variables. In the standard way, it induces a unique
interpretation $\overline{\alpha}: T(\Sigma,X) \to A$ of every term $t$
(with variables from $X$) in
$A$.
 
Satisfaction of formulae in a multialgebra is defined as follows:
\begin{definition}\label{de:sat}
Given an assignment $\alpha:X\to|A|$:
\begin{enumerate}\MyLPar
\item $A \models_\alpha t \eleq  t'\ {\rm iff\ } \ovr{\alpha}(t)
=\overline{\alpha}(t') = \{e\},\ {\rm for\ some\ element\ } e \in |A|$
\item $A \models_\alpha t \prec t'\ {\rm iff\ } \ovr{\alpha}(t) \subseteq \overline{\alpha}(t')$
\item $ A \models_\alpha a_1 , \ldots ,a_n \To  b_1 , \ldots ,b_m\ {\rm iff\ }
\exists i: 1 \leq i \leq n : A \not\models_\alpha a_i\ {\rm or\ } \exists j: 1 \leq j \leq m : A \models_\alpha b_j$
\item $A \models \varphi\ {\rm iff\ } A \models_\alpha \varphi\ {\rm for\ all\ } \alpha$.
\end{enumerate}
\end{definition}
%
Multialgebraic specifications contain, as a special case, all usual equational
specifications. 
The following example should give the flavor of some additional aspects
of multialgebraic specifications.
\begin{example}
A (sketchy) specification of the integers.

\( 
	\spec{
	\tit{\mbox{\bf spec\ Int}=} \\
		\spSort{Int}
		\spOp{ zero: && Int\\
			succ: & Int & Int\\
			 pred: &Int & Int \\
			 pos: & & Int}
		\Phi:
			&1.& && zero \eleq zero \\
			&2.& && succ(x) \eleq succ(x)\\
			&3.& && pred(succ(x)) \eleq x\\
			&4.& && succ(zero) \prec pos\\
			&5.& x\prec pos &\To& succ(x) \prec pos\\
	}
\)

\noindent
Axioms 1. and 2. state merely that $zero$ and $succ$ are deterministic
operations. (Such axioms can be naturally abbreviated, e.g., by ${\bf det:}\ 
zero,succ$.)
Axiom 3. states a property of $pred$ applied to $succ$. 
Unlike the constant
$zero$, the constant $pos$ can
be ``nondeterministic'', i.e., it can, in general, denote a set of elements
of sort $Int$  -- its
intention here is to represent the subsort of positive integers.
The last two axioms specify the least contents of this subsort.
\end{example}
Of course, a complete specification would include more axioms, but we merely
wanted to indicate the special features of multialgebraic specifications. 

%
Putting the definitions from this subsection together, we obtain the institution of 
multialgebras $\inst{MA}$.
The property of $\inst{MA}$ which is of particular relevance for this paper
follows from the following proposition by lemma~\ref{le:amalg}.
\begin{proposition}
The model functor $\fu{Mod}:\cat{Sign}^{op}\to\cat{Cat}$ in the institution
$\inst{MA}$ is finitely continuous -- $\inst{MA}$ is an exact institution.
\end{proposition}

\section{Specifications of parameterized data types}\label{se:paraADT}
To specify parameterized data types we will use a restricted syntax for
specifications. All specifications can be naturally viewed as standard
multialgebraic specifications. Also, all semantic constructions take place in
the category of standard multialgebras. Subsection~\ref{sub:gsp} introduces
merely convenient syntactic abbreviations. 


\subsection{Signatures with sort constants}\label{sub:gsp}
We start by modifying the concept of signature and specification. The idea
is that each signature may have, in addition to the standard set of sort and
operation symbols, a (possibly empty) set of distinguished (sort and subsort) constant
symbols $\allcons = \Cons\cup\subcons$. 
The set $\Cons$ %=\{\cons_s:\to s: s\in\Sorts\}$ 
contains  constant symbols
$\cons_s$ for various sort symbols $s$ -- the intention of $\cons_s$ is to denote all the elements of the respective
sort $s$. 
The set $\subcons$  may contain additional constants
which will represent various subsorts -- the constants from this set are
called ``subsort constants''. (Usually, the distinction between $\Cons$ and
$\subcons$ does not matter and then we will write ``(sub)sort constants''.)
\begin{definition}
\label{def:modsign}
A signature with sort constants, $\Sigma_\cons$, is a triple $\Sigma_\cons =
 (\Sorts,\Ops,\allcons)$, where $\Sigma=(\Sorts,\Ops)$ is an ordinary
 signature and $\allcons=\Cons\cup\subcons$ is a (possibly empty) set of additional constants,
 $\allcons\cap\Ops=\emptyset$ and $\Cons\cap\subcons=\emptyset$. 
\end{definition}
%
Signatures with sort constants will be used merely as a syntactic representation of 
ordinary signatures. This is possible in multialgebraic setting since
constants may denote sets of elements. (In a more traditional setting, one would
have to represent the sort constants, for instance, by predicates or (sub)sort symbols.)

We allow the set $\allcons$ to be empty. Also, we allow the set
$\Cons$ to contain several distinct constants of the same sort (although
their intended meaning will be the same).
The technical reasons for that come up in relating construction of co-limits
%(proposition~\ref{prop:finco}, especially, lemma~\ref{fa:coeq}) 
and, in particular, pushouts
(subsection~\ref{sub:canpush}). For the most, we think of the set
$\Cons$ as containing one constant for each sort, i.e., as
$\Cons=\{\cons_s:\to s : s\in\Sorts\}$. Most relevant constructions will
involve and yield such signatures. (In general, we use the symbol $\cons_s$
for an arbitrary constant from $\Cons$ of sort $s$, e.g., for a signature
morphism $\mu$, 
$\mu(c)\not=\cons_s$ means the same as $\mu(c)\not\in\Cons$.)


We will use the following operations relating the signatures with sort
 constants to ordinary signatures.

\begin{definition}\label{def:sigops}
Given a signature with sort constants $\Sigma_\cons=(\Sorts,\Ops,\allcons)$ we let:
\begin{enumerate}\MyLPar
\item $\under{\Sigma_\cons}=(\Sorts,\Ops \cup \allcons)$
i.e. $(\Sorts, \Ops \cup \Cons \cup \subcons)$ -- the underlying signature
\item ${\Sigma_\weak}=(\Sorts,\Ops,\subcons)$ i.e. $\Sigma_\cons
\setminus \Cons$ -- the reduced signature
\item ${\Sigma}=(\Sorts,\Ops)$ i.e. $(\Sigma_\cons
\setminus \allcons)$ -- the standard (part of the) signature
\end{enumerate}
Conversely, given an ordinary signature $\Sigma=(\Sorts,\Ops)$, we let
${\Sigma_\cons}=(\Sorts,\Ops,\Cons)$, 
 where   $\Cons=\{\cons_s:\to s:s\in\Sorts\}$ and $\Cons\cap\Ops=\emptyset$.
%%\begin{enumerate}\MyLPar\setcounter{enumi}{3}
%%\item ${\Sigma_\cons}=(\Sorts,\Ops,\Cons)$
%%, where   $\Cons=\{\cons_s:\to s:s\in\Sorts\}$ and $\Cons\cap\Ops=\emptyset$
%%-- the corresponding guarded signature with sort constants.
%%\item\label{it:ins} $\ins\Sigma=(\Sorts,\Ops,\emptyset)$
%%, -- the included signature (with sort constants).
%%\end{enumerate}
\end{definition}
%
Unless stated otherwise, the signatures considered will always be signatures with sort constants. 
Given an arbitrary specification $\thr{SP}$, we will sometimes write
$\Sigma(\thr{SP})$ to denote its signature.

\begin{definition}\label{def:sigmor}
A morphism between signatures with sort constants $\mu:\Sigma_\cons \to
\Sigma'_{\cons}$ is a signature morphism between the underlying
signatures $\mu:\under{\Sigma_\cons}\to\under{\Sigma'_\cons}$, sending
$\allcons$ to ${\allcons}'$ and $\Sigma$ to $\Sigma'$.
\end{definition}
In other words, the $\Cons$-constants need not be sent to $\Cons'$-constants
but may be mapped to subsort constants ${\subcons}'$, as well.
The signatures with sort constants and morphisms from def.~\ref{def:sigmor}
form a category $\cat{Sign_\cons}$. 
Since the morphisms are essentially the morphisms of the underlying
signatures,  the transformation from $\cat{Sign_\cons}$ to $\cat{Sign}$
can be extended to a functor $\fu{_-}:\cat{Sign_\cons} \to
\cat{Sign}$, which is the (pointwise) identity on the signature morphisms. 
%%
%%The other way around the inclusion of signatures (point~\ref{it:ins} of
%%def.~\ref{def:sigops}) can also be extended to a
%%functor $\fu{\ins\ }:\cat{Sign} \to \cat{Sign_\cons}$, which sends $\Sigma$
%%to $\ins\Sigma$ and is the identity on morphisms. The following fact says
%%that $\cat{Sign}$ can be treated as a full subcategory of $\cat{Sign_\cons}$.
%%
%%\begin{fact}
%%$\ins\ :\cat{Sign} \to \cat{Sign_\cons}$ is full and faithfull.
%%\end{fact}
%%We have that for any $\Sigma\in\cat{Sign}:\Sigma=\under{(\ins\Sigma)}$ but,
%%in general, for $\Sigma_\cons\in\cat{Sign_\cons}: \Sigma_\cons\not\iso
%%\ins{(\under{\Sigma_\cons})}$. This is because for an isomorphism in
%%$\cat{Sign_\cons}$ we must have isomorphism between the respective sets of
%%(sub)sort constants, but while $\allcons$ in $\Sigma_\cons$ may be
%%non-empty, it is always empty in $\ins{(\under{\Sigma_\cons})}$.
%%
The following proposition shows that
 that (finite) co-limits in $\cat{Sign}_\cons$
can be obtained from the respective co-limits in $\cat{Sign}$. 

\begin{proposition}\label{prop:finco}
The functor $\fu{_-}:\cat{Sign_\cons} \to \cat{Sign}$ reflects (and
preserves) finite co-limits.
\end{proposition}
%
More concretely, to obtain a co-limit in $\cat{Sign_\cons}$, one can construct a co-limit in $\cat{Sign}$ and
then make a straightforward choice of the (sub)sort constants.

\subsection{Guarded specifications}
To write specifications of parameterized data types we will use guarded
axioms. In general, one only requires that the axioms from the parameter specification
hold only for the elements in the parameter algebras, and guards are needed to mark these elements.

\begin{definition}\label{def:guf}
Given a signature  $\Sigma_\cons$ (with sort constants $\allcons$):
\begin{enumerate}\MyLPar
\item\label{it:guard} a {\em guard} $\gamma$ is an atom of the form $x \prec c$, where
$x$ is a variable and $c \in \allcons$;
\item a {\em guarded formula} is of the form $\phi_\cons = \guard^{\ast},
	\ovr{a} \To \ovr{b}$ where $\ovr{a},\ovr{b}$ are (sequences of)
	${\Sigma_\weak}$ atoms and 
	$\guard^\ast$ is a (non-empty) sequence of guards $\guard_i =
	x_i\prec {c_{i}}$ for some variables $x_i$ occuring in the
	atoms $\ovr{a},\ovr{b}$ -- $\gamma_i$'s in such conditional
	formulae are called {\em local guards};
\item\label{it:fullguard} a {\em fully guarded} formula is a guarded formula where there is a
guard $\guard_i$ for each variable $x_i$ occurring in the formula;
\item\label{it:ung} an {\em unguarded formula} is any formula over $\Sigma_\cons$
with no guards.
\end{enumerate}
\end{definition}
The only places where $\Cons$ may occur are in guards
-- for conditional axioms in the premises. (We will also allow unconditional
axioms of form~\ref{it:guard} and \ref{it:ung}.). Note that guards $x\prec\cons$ are only special
(strongest) cases -- in general, $x\prec c$, where $c\in\allcons$, is a
guard. 

A formula containing only ground $\Sigma_\cons$ terms (a ground
$\Sigma_\cons$ formula) can be seen either as a fully guarded formula
(according to point~\ref{it:fullguard}.), or else as an unguarded formula
(according to point~\ref{it:ung}.). Saying ``a fully guarded formula'' we
will typically mean also such formulae.

We will use only a restricted form of the specifications, namely, 
guarded  specifications.

\begin{definition}\label{def:guardedspec}
A {\em guarded specification} is a triple $\thr{SP_\cons} = (\Sigma_\cons,\Phi_\cons,\Modax_{\Sigma})$ where:
\begin{itemize}\MyLPar
\item $\Sigma_\cons$ is a signature with sort constants
\item Each $\phi_\cons \in \Phi_\cons$ is a guarded or unguarded formula
\item $\Modax_\Sigma \subseteq \{ x \prec \cons_s : \cons_s \in \Cons \}$ is the
set of axioms called {\em global guards}.
\end{itemize}
If $\Cons$ contains at least one constant of each sort and there is an equality in the last point, 
we call a specification {\em
	globally guarded}. If, in addition, all axioms in $\Phi_\cons$ are
	fully guarded (possibly, ground unguarded $\Sigma_\cons$ formulae), the specification is {\em fully guarded}.
\end{definition}
Note that all the local guards of the form $x\prec\cons\To...$ in a globally
guarded specification are trivially satisfied 
due to the presence of the global guards $\Modax$.

\begin{example}\label{ex:Nat}
Fully guarded specification of the natural numbers.

\( 
	\spec{
	\tit{\mbox{\bf spec\ Nat}_\cons^{{}}=} \\
		\spSort{Nat}
		\spOp{ zero: && Nat\\
			succ: & Nat & Nat\\
			 pred: &Nat & Nat}
		\spOps{\allcons}{ \cons_{Nat}: && Nat }
		\Phi:
			&1.& && zero \eleq zero \\
			&2.& x \prec \cons_{Nat} & \To & succ(x) \eleq succ(x)\\
			&3.& x \prec \cons_{Nat} & \To & pred(succ(x)) \eleq x\\
		\Gamma:	&4.& && x \prec \cons_{Nat}\\
	}
\)
\end{example}
Except for the constant $\cons_{Nat}$ which
comprises all the elements of sort $Nat$, 
there is no real difference between this and the 
specification with all the guards are dropped.

\begin{definition}\label{def:specOps}
Given a guarded specification $\thr{SP_\cons}
= (\Sigma_\cons,\Phi_\cons,\Modax_{\Sigma})$, 
\begin{enumerate}\MyLPar
\item its {\em weakening} is a
specification $\thr{SP_\weak} = (\Sigma_\cons,\Phi_\cons)$.
\item its {\em underlying} specification is $\under{\thr{SP_\cons}} = (\under{\Sigma_\cons},
\Phi_\cons \cup \Modax_\Sigma)$.
\end{enumerate}
Conversely, for an ordinary specification $\thr{SP}=((\Sorts,\Ops),\Phi)$,
$\thr{SP_\cons}$ denotes the fully guarded specification
$(\Sigma_\cons,\Phi_\cons,\Modax_{\Sigma})$, where $\Cons$ is as in
def.~\ref{def:sigops}, 
$\Sigma_\cons=(\Sorts,\Ops,\Cons)$ and
$\Modax_\Sigma=\{x\prec \cons:\cons\in\Cons\}$.
\end{definition}
% 
$\thr{SP}_\cons$ denotes, in general, a guarded specification with all,
some or none axioms guarded. However, when $\thr{SP}$ is an ordinary specification,
then $\thr{SP_\cons}$ denotes the fully guarded version. This latter case will be
used less frequently than the former one, and we will indicate it explicitly.

Keep in mind that, given a specification $\thr{SP_\cons}$ with signature
$\Sigma_\cons$, the signature of its weakened version $\thr{SP_\weak}$ is
still $\Sigma_\cons$ and not $\Sigma_\weak$.

As models for guarded specifications we use ordinary multialgebras.
\begin{definition}
The model class of a guarded specification $\thr{SP_\cons}$ is the
model class of its underlying specification: $\Mod(\thr{SP_\cons}) = \Mod(\under{\thr{SP_\cons}})$.
\end{definition}
In particular, for a given ordinary specification $\thr{SP}$, there is obvious equivalence of
model categories between the unguarded $\Mod(\thr{SP})$ and fully guarded $\Mod(\thr{SP_\cons})$.
Also, for a given guarded specification $\thr{SP_\cons}$ there is the obvious
inclusion functor
\eq{\label{eq:idcons}
\fu{id_\weak}:\Mod(\thr{SP_\cons}) \to \Mod(\thr{SP_\weak})
}
which sends each algebra in the first class to itself.


\subsection{Syntax of parameterized data type specification}
\begin{definition}\label{def:parametersyntax}
A parameterized data type specification (a PDT) is a
 quadruple $\pdt$,
where
\begin{enumerate}\MyLPar
\item 
$\thr{X_\cons}=(\Sigma_\cons,\Phi_\cons, \Modax_\Sigma)$ and $\thr{P[X]_\cons} =
 (\Sigma'_{\cons},\Phi'_{\cons}, \Modax'_{\Sigma'})$ are fully guarded specifications,
\item 
$\Sigma_\cons=(\Sorts,\Ops,\allcons)\subseteq
(\Sorts',\Ops',S^{\cons'})=\Sigma'_\cons$ are signatures from $\cat{Sign_\cons}$, 
\item 
$\mu:\Sigma_\cons\to \Sigma'_\cons$ is a signature morphism, called the {\em
``parameterization morphism''}, which is identity on all symbols with the
possible exception of $\Cons$, and such that if $\mu(\cons_s)\not=\cons_s$
then $\mu(\cons_s)\not\in\Sigma_\cons$ (i.e., $\mu(\cons_s)$ is then a fresh
constant from ${\allcons}'$),
\item 
$\delta:\Sigma_\cons\to\Sigma'_\cons$, called the {\em ``local guard mapping''}, is 
a signature morphism, which is identity on all symbols with the
possible exception of $\Cons$,
\item\label{it:rel}
 the two specifications, $\mu$ and $\delta$ are such that:
\begin{enumerate}
\item\label{it:sat} if $\mu(\cons_s)\not=\delta(\cons_s)$ then the axiom
$\mu(\cons_s)\prec\delta(\cons_s)$ is among $\Phi'_\cons$
\item\label{it:unguard} for every unguarded axiom $\phi\in\Phi_\cons:\mu(\phi)\in\Phi'_\cons$
\item\label{it:corax} for every guarded axiom  $\phi_\cons\in\Phi_\cons:
\delta(\phi_\cons)\in\Phi'_\cons$.
\end{enumerate}
\end{enumerate}
\end{definition}
%
We write $\delta$ at the end of the tuple because in many constructions it
plays no role. In such cases, we will often drop it from the notation and
write a PDT simply as $\pdtmu$. Although it is defined as a signature morphism,
its only role is to map the local guards. Being  the identity on most
symbols, it is essentially a mapping
$\delta:\Cons\to{\allcons}'$. The point~\ref{it:corax} can be stated in more
detail as follows: for every guarded axiom $\phi_\cons\in\Phi_\cons:$ \\
\hspace*{2em} $\phi_\cons =x_1\prec\cons_1, \ldots ,x_m\prec\cons_m, \ovr{a}
 \To \ovr{b}$, \\ where
$x_1\prec\cons_1,\ldots,x_m\prec\cons_m$ are all the local guards in
$\phi_\cons$ with $\cons_i\in\Cons$ (and $\ovr{a}, \ovr{b}$ sequences of $\Sigma_\weak$
atoms), 
$\delta(\phi_\cons)\in\Phi'_\cons$,
where $\delta(\phi_\cons)$ is the corresponding axiom:\\
\hspace*{1.5em}$\delta(\phi_\cons)=x_1\prec\delta(\cons_1),
\ldots ,x_m\prec\delta(\cons_m),\ovr a \To \ovr b$.
\\
Similarly, $\mu$ is essentially a mapping $\mu:\Cons\to{\allcons}'$. 
Although the definition is rather liberal, for all practical purposes we can
think of the syntax as given by a $\mu$ which maps $\Cons$ to ${\allcons}'$,
and by a $\delta$ with
$\delta(\cons_s)=\mu(\cons_s)$ or else
$\delta(\cons_s)=\cons_s$ (see the remarks below). This, in fact, covers most natural situations and
will be the case in all our examples.

Axioms of the form \ref{it:sat} are needed to ensure that the guarded axioms
from the parameter specification will still apply, at least, to the elements originating
from the parameter algebras. Since $\thr{X_\cons}$ is fully guarded, axioms
mentioned in \ref{it:unguard} are actually ground formulae, typically,
involving only (sub)sort constants from $\allcons$. The
restriction~\ref{it:unguard} has significant effect if, e.g., $\phi$ is
$c\prec\cons$ and $\mu(\cons)=c'\not=\cons$, in which case, we require
$\mu(\phi)=c\prec c'$ to be included in $\Phi'_\cons$.
This requirement ensures that elements contained in
$c$ (in a parameter algebra) will remain within $\mu(\cons)$ (we will see it
shortly when discussing the semantics).

The definition discriminates against some
isomorphic (wrt. semantics) specifications, since $\mu$ has to be identity on
$\Sigma_\weak$. 
%Like $\delta$, this mapping is essentially a function $\Cons\to{\allcons}'$.
It will be the main element in defining the
semantics of PDTs.
The image
under $\mu$ of the constants $\Cons$ can be twofold -- 
$\mu(\cons_s)$ has to be $\cons_s$ itself or a
fresh constant from ${\subcons}'$.  
\begin{itemize}\MyLPar
\item[1)] The former situation, $\mu(\cons_s)=\cons_s$, corresponds to the
classical case of persistency, i.e., to  non-extending the carrier.
\item[2)] Otherwise, $\mu(\cons_s)\not=\cons_s$, we
will obtain a distinction between the elements of sort $s$ originating from
the parameter, $\mu(\cons_s)$, and the ones originating from the parameterized specification
itself, the ``new'' $\cons_s$. 
\end{itemize}
These two cases can be marked by the respective keywords: 1) {\bf
non-extending carrier} $s$, or 2) (possibly) {\bf extending carrier} $s$.

The mapping $\delta$ is introduced to allow more flexibility in
parameterized data type specifications. If the carrier of sort $s$ is not
extended, case 1) above, $\delta$ has no effect -- according to
\ref{it:sat}, it has to be $\delta(\cons_s)=\mu(\cons_s)=\cons_s$ (or else
another subsort constant $c$ but such that $\thr{P[X]_\cons}\models
\cons_s\prec c$ which, together with the global guard $x\prec \cons_s$ says
that the two denote the same set, i.e., the whole carrier of sort $s$). But if the carrier of $s$
is (possibly) extended, case 2) above, $\delta$ allows to
either 
\begin{itemize}\MyLPar
\item[2a)] restrict the local guards from the formal parameter, when
$\delta(\cons_s)=\mu(\cons_s)$ -- in this case the axioms from
the parameter specification are required to hold only for the elements from the
parameter algebra, 
\item[2b)] 
or else extend the local guards -- in which case the axioms from the parameter
specification have to hold also for (possibly new) elements from
$\delta(\cons_s)$; the presence of axioms~\ref{it:sat}, $\mu(\cons_s) \prec
\delta(\cons_s)$, ensures that the old axioms still hold at least for the old elements.
\end{itemize}
Thus, in the case 2) of {\bf extending carrier} $s$, 
$\delta$ allows us either 2a) {\bf restricting (local) guards} $s$,
or 2b) {\bf extending (local) guards} $s$.

\begin{example}\label{ex:groups}
Specification of groups parameterized by monoids.

First, we take a standard (deterministic multialgebraic) specification of
groups, with multiplication $\cdot$, unit $e$ and inverse $(\_)^-$:

\[ 	\spec{
	\tit{{\mbox{\bf spec\ Group}}^{{}}=} \\
		\spSorts{\Sorts'}{S}
		\spOps{\Ops'}{ \cdot: & S \times S & S \\
			e: &&  S \\
			(\_)^-: & S & S}
\Phi':  	&1.&  e\cdot x & \eleq &  x \cdot e \\
		&2.&  e \cdot x & \eleq &  x \\
		&3.&  x \cdot(y\cdot z)& \eleq & (x\cdot y)\cdot z  \\
		&4.& e & \eleq & e \\
		&5.& x\cdot y & \eleq &  x\cdot y \\
		&6.& x\cdot x^- & \eleq &  e \\
		&7.& x^-  & \eleq & x^-  
}
\]
This does not conform to the required format -- we  add the sort constant
$\cons_S$, a global guard
$x \prec \cons_S$, and then a new constant $mono$ to mark the elements
originating from the monoid specification. The result is the
PDT on the right, with the formal parameter on
the left:
\[ \xymatrix{
\frame{\txt{${\spec{\tit{\mbox{\bf spec\ Monoid}_\cons^{{}}=} \\
		\spSort{S}
		\spOp{ \cdot : &S \times S& S \\
		       e: && S }
	     \spOps{\allcons}{\cons_S: && S}
		{\Phi:}
	&1.&  e\cdot x & \eleq &  x \cdot e \\
		&2.&  e \cdot x & \eleq &  x \\
		&3.&  x \cdot(y\cdot z)& \eleq & (x\cdot y)\cdot z  \\
		&4.& e & \eleq & e \\
		&5.& x\cdot y & \eleq &  x\cdot y \\
	\Gamma:	&6.&  x & \prec & \cons_S 
}}$}} \ar[rrr]^{\txt{$\large{\mu(\cons_{S})=mono}$}}_{\txt{$\large{\delta(\cons_{S})=\cons_{S}}$}} &&&
\frame{\txt{${\spec{
	\tit{{\mbox{\bf spec\ Group[Monoid]}}_\cons^{{}}=} \\
		\spSorts{\Sorts'}{S}
		\spOps{\Ops'}{ \cdot: & S \times S & S \\
			e: &&  S \\
			(\_)^-: & S & S}
		\spOps{{\allcons}'}{\cons_S: &&  S \\
			mono: &&  S }
\Phi':  	&1.&  e\cdot x & \eleq &  x \cdot e \\
		&2.&  e \cdot x & \eleq &  x \\
		&3.&  x \cdot(y\cdot z)& \eleq & (x\cdot y)\cdot z  \\
		&4.& e & \eleq & e \\
		&5.& x\cdot y & \eleq &  x\cdot y \\
		&6.& x\cdot x^- & \eleq &  e \\
		&7.& x^-  & \eleq & x^-  \\
\Gamma':
	&8.&  x & \prec & \cons_S
}}$}}
}
\]
The parameterization morphism sends
$\mu(\cons_S)=mono$. This is so because constructing a {\bf Group}-algebra out of a given
{\bf Monoid}-algebra, one may need to add new elements, {\bf extending
carrier}, of sort $S$. The (dummy)
constant $mono$ keeps the track of the old elements. 

On the other hand, even if new elements appear in the resulting algebra, 
we certainly want the {\bf Monoid}-axioms to hold not only for the old but
also for these new elements. 
Therefore, we let $\delta(\cons_S)=\cons_S$. 
\end{example}
In the presentation above, we
have dropped all the local guards $x\prec\cons_S\To...$ in both specifications, since they will
be trivially satisfied due to the presence of the global guards $\Gamma$,
resp. $\Gamma'$. We have also
dropped the axiom $mono\prec\cons_S$ (required by
definition~\ref{def:parametersyntax}, point~\ref{it:sat}), since it follows from the
global guard $\Gamma'$. The respective specifications are equivalent (i.e.,
isomorphic in $\cat{Th}_{\inst{MA}}$).
We will often use such abbreviations in order to simplify the examples. (The
syntactic presence of all guards will be of significance first in considering
composition in section~\ref{se:compref}.)

It should be observed that the flexibility offered by $\delta$ is highly
desirable -- in many other situations, one would like
to restrict the local guards from the formal parameter to hold only for the
elements originating from it but not for the new ones added by the
parameterized specification. The classical example of such a situation is a
specification of 
stacks parameterized by elements.
%
\begin{example}\label{ex:stackelSP}
As in the previous example, we drop the local guards $s\prec\cons_{Stack}$
and the axiom
$ok\prec\cons_{El}$ from $\thr{Stack[El]_\cons}$.
\[\xymatrix{
\frame{\txt{$\spec{
	\tit{\mbox{\bf spec\ El}_\cons^{{}}=} \\
		\spSort{El}
		\spOps{\allcons}{ \cons_{El}: \to El}
		\Gamma:
			& x \prec \cons_{EL}
}$}}
\ar[rrrr]^{\txt{$\large{\mu(\cons_{El})=ok}$}}_{\txt{$\large{\delta(\cons_{El})={ok}}$}} &&&&
\frame{\txt{$\spec{
	\tit{\mbox{\bf spec\ Stack[El]}_\cons^{{}}=} \\
		\spSorts{\Sorts'}{Stack,El}
		\spOps{\Ops'}{empty: && Stack\\
			top: & Stack & El\\
			pop: & Stack & Stack\\
			push: & El \times Stack & Stack}
		\spOps{{\allcons}'}{	\cons_{El}: && El \\
			\cons_{Stack}: && Stack \\
			ok:  && El}
		\Phi':
			&1.& && empty \eleq empty \\
			&2.& x \prec ok
				& \To & top(push(x,s)) \eleq x\\
			&3.& x \prec ok
				 & \To & pop(push(x,s)) \eleq s \\
		\Gamma':	&4.& && x \prec \cons_{EL}\\
			&5.& && s \prec \cons_{Stack} \\
}$}}
}
\]
The parameterization morphism with $\mu(\cons_{El}) = ok$, allows {\bf extending
carrier} of sort $El$. The local guard mapping $\delta$ coincides here
with $\mu$, i.e., $\delta(\cons_{El})=ok$, thus {\bf restricting (local)
guards} of sort $El$ -- the potentially new
elements of sort $El$ arising in {\bf Stack}, like $top(empty)$, are not
intended to behave as the ``ordinary'' elements from the 
actual parameter sort but to function merely as, say, ``error'' elements. 
\end{example}
The effect of $\delta$ may be little visible in the above example since the
parameter $\thr{El_\cons}$ does not contain any proper axioms. However, it
should be clear that a PDT where the local guards in $\thr{Stack[El]_\cons}$
axioms 2. and 3. were replaced by $x\prec\cons_{El}$ would be very
different. Also, it should be easy to imagine replacing the formal parameter
$\thr{El_\cons}$ with, say $\thr{Nat_\cons}$ from example~\ref{ex:Nat}, in which axioms 
should be guarded. For instance, the axiom  $x\prec \cons_{Nat}\To
pred(succ(x))\eleq x$ from $\thr{Nat_\cons}$
would correspond to $x\prec ok \To pred(succ(x))\eleq x$ in 
$\thr{Stack[Nat]_\cons}$, since we do not necessarily want this
axiom to apply to the result of $top(empty)$.

We register a simple fact about the parameterization morphisms:
\begin{fact}\label{fa:weakparamo}
If $\pdt$ is a PDT then:
\begin{enumerate}\MyLPar
\item $\mu: \thr{X_\cons} \to \thr{P[X]_\cons}$ may not be a specification morphism, but
\item $\mu: \thr{X_\weak} \to \thr{P[X]_\weak}$ is a specification morphism,
and hence also
\item $\mu: \thr{X_\weak} \to \thr{P[X]_\cons}$ is a specification morphism.
\end{enumerate}
\end{fact}
%%\begin{PROOF}
%%1. A trivial counter-example is given by $\thr{X_\cons}$ with the only global
%%   guard axiom $x\prec\cons$, $\thr{P[X]_\cons}$ with a subsort constant $c$ and
%%   the only global guard   axiom $x\prec\cons$, and
%%   $\mu(\cons)=\delta(\cons)=c$. Obviously $\thr{P[X]_\cons}\not\models
%%   \mu(x\prec\cons)= x\prec c$.
%%\\[1ex]
%%2. follows from
%%definition~\ref{def:parametersyntax}. $\mu$-translations of all unguarded
%%axioms from $\thr{X_\weak}$ are included in $\thr{P[X]_\cons}$ by
%%point~\ref{it:unguard}. For any guarded axiom $x_i\prec\cons_i,\ovr a\To\ovr
%%b$ from $\thr{X_\weak}$, the corresponding axiom
%%$x_i\prec\delta(\cons_i),\ovr a\To\ovr b$ is in $\thr{P[X]_\cons}$. But then
%%we also have the axioms $\mu(\cons_i)\prec\delta(\cons_i)$ in
%%$\thr{P[X]_\cons}$, and these together imply that 
%%$\thr{P[X]_\cons}\models x_i\prec\mu(\cons_i),\ovr a\To\ovr b$, i.e., 
%%$\thr{P[X]_\cons}\models \mu(x_i\prec \cons_i,\ovr a\To\ovr b)$.
%%\\[1ex]
%%3. follows from 2., since $\thr{P[X]_\cons}\models\thr{P[X]_\weak}$.
%%\end{PROOF}

\subsection{Semantics of parameterized data type specification}
It is, of course, possible to use ordinary (loose) semantics for our PDTs,
i.e., to treat them as simple parameterized specifications. 
But the trouble we have taken with the
syntactic operations and restrictions on the specifications was meant to
provide the possibility to define the semantics as 
parameterized data types, i.e., data types consisting of algebras parameterized
by algebras. 

To define the semantics for PDTs we will use a special case of the general (weak)
homomorphisms of multialgebras.
\begin{definition}
A {\em tight} homomorphism $h:A \to B$ is a homomorphism satisfying:
	\[ h(\omega^A(x_1,...,x_n)) = \omega^B(h(x_1),...,h(x_n)) \]
\end{definition}
The following is a logical characterization of tight subalgebra (an algebra
$A$ with a tight monomorphism to $B$).
\begin{proposition}\label{prop:satiff}
If $h:A\to B$ is a tight $\Sigma_\cons$-monomorphism
and $\phi$ is an arbitrary, fully
guarded $\Sigma_\cons$ formula, then $A\models\phi\Iff B\models\phi$.
\end{proposition}
%
It is necessary to restrict the attention to fully guarded formulae (possibly
ground formulae), because assignments to ``new'' elements in $B$ which are
not in the image of $h[A]$ may actually
falsify some formulae which hold in $A$.

Now, given a PDT $(\mu,\thr{X_\cons},\thr{P[X]_\cons})$
and a functor $\fu{F}:\Mod(\thr{X_\cons})\to \Mod(\thr{P[X]_\cons})$, we
obtain two functors: 
\begin{itemize}\MyLPar
\item 
$\fu{id}_\weak:\Mod(\thr{X_\cons})\to \Mod(\thr{X_\weak})$ defined in
 (\ref{eq:idcons}) at the end of section~\ref{sub:gsp}, and
\item  the composition
$\comp{\fu{F}}{\_|_\mu}:\Mod(\thr{X_\cons})\to \Mod(\thr{X_\weak})$. 
\end{itemize}
The latter has the target $\Mod(\thr{X_\weak})$ and not $\Mod(\thr{X_\cons})$
because, in the case when $\mu(\cons_s)=c\not=\cons_s$ for some $s$, the
reduct $A|_\mu$ of an algebra $A\in\Mod(\thr{P[X]_\cons})$ may contain more
elements in the sort $s^{A|_{\mu}}$ than those in $\cons_s^{A|_{\mu}}$, i.e.,
it may fail to satisfy the global guard $x\prec \cons_s$. This captures the
intention that the parameterized algebra may actually add new elements to the
sorts of the parameter algebra. We define (loose) semantics of parameterized
data type specifications by putting some restrictions on the functors
$\fu{F}:\Mod(\thr{X_\cons})\to \Mod(\thr{P[X]_\cons})$. Notice that the
effect of (or requirements put by) $\delta$ are present in the actual axioms
of both specifications, so that we do not have to consider $\delta$ explicitly
here. 
%
\begin{definition}
\label{def:parametersemantic}
The semantics of the
parameterized data type specification $\thr P = (\mu,\thr{X_\cons},\thr{P[X]_\cons})$
is the class of all functors $\fu{F}:
\Mod(\thr{X_\cons}) \to \Mod(\thr{P[X]_\cons})$, such that there exists a
natural transformation $\iota: \fu{id_\weak} \natt
\comp{\fu{F}}{\_|_{\mu}}$, where for each $A\in\Mod(\thr{X_\cons})$ the
component $\iota_A$ is a tight ${\Sigma}(\thr{X_\cons})$-monomorphism.
\end{definition}
A functor with this property is
called a {\em semantic functor} for the PDT \thr{P},
 and $\PMod(\thr{P})$ will denote the class of all semantic functors for a parameterized
data type.
%%
%%First, let us observe that although this semantics $\PMod(\thr P)$ is
%%rather liberal, it does capture the structuring aspect at least in the sense
%%that the algebras which may result from it are not all possible models of the
%%flat specification. More precisely, a PDT $\thr P = \pdtmu$ can be seen as
%%defining two classes of algebras: 
%%\begin{enumerate}\MyLPar
%%\item the one is simply the model class
%%$\Mod(\thr{P[X]_\cons})$ with $\thr{P[X]_\cons}$ viewed as a flat specification, and 
%%\item the
%%other is obtained from the semantic functors of the PDT, namely, the class
%%$\FMod(\thr P)=\{ \fu F(X): X\in\Mod(\thr{X_\cons}), \fu F\in\PMod(\thr P)\}$ 
%%of all $\thr{P[X]_\cons}$ algebras which can be obtained as an image of some $\thr X_\cons$
%%algebra under some semantic functor $\fu F$.
%%\end{enumerate}
%%Obviously, the latter class is contained in the former. 
%%\begin{fact} 
%%In general $\FMod(\thr{P}) \not= \Mod(\thr{P[X]_\cons})$.
%%\end{fact}
%%\begin{PROOF}
%%Consider following specification $\thr P = \pdtmu$:
%%
%%\( 
%%\xymatrix{
%%	{\txt{\spec{
%%	\tit{\mbox{\bf spec\ X}_\cons^{{}}=} \\
%%		\spSorts{\Sorts}{s}
%%		\spOps{\Ops}{c:		&& s}
%%		\spOps{\allcons}{\cons_{s}:	&& s}
%%		\Gamma:
%%			&1.& && x \prec \cons_{s} \\
%%	} }}
%%\ar[rrr]^{\mu(\cons_s)=d}
%%&&& 
%%	{\txt{ \spec{
%%	\tit{\mbox{\bf spec\ P[X]}_\cons^{{}}=} \\
%%		\spSorts{\Sorts'}{s}
%%		\spOps{\Ops'}{c:&& s}
%%		\spOps{{\allcons}'}{d,\cons_{s}:	&& s}
%%		\Gamma':
%%			&1.& && x \prec \cons_{s} \\
%%	}  }}
%%}
%%\)
%%
%%\noindent
%%The algebra $A$ given by: carrier $|A|=\{c\}$ and $c^A = \cons_s^A = c, d^A =
%%\emptyset$, is in $\Mod(\thr{P[X]_\cons})$.
%%However, the reduct $A|_\mu$ has no tight $\Sigma_\cons$-subalgebra in
%%$\Mod(\thr X_\cons)$, because $\cons_s^{A|_\mu}=\emptyset$ while
%%$c^{A|_\mu}=c$. Hence $\FMod(\thr{X}) \not= \Mod(\thr{P[X]})$.
%%\end{PROOF}
%
%
The following fact is an alternative formulation of the above definition~\ref{def:parametersemantic}.

\begin{fact}\label{fa:parsemAlt}
A functor $\fu{F}: \Mod(\thr{X_\cons}) \to \Mod(\thr{P[X]_\cons})$ is a
semantic functor of a parameterized data type specification
$(\mu,\thr{X_\cons},\thr{P[X]_\cons})$ iff:
\begin{enumerate}\MyLPar
\item there exists a functor $\fu{\iota}:\Mod(\thr{X_\cons}) \to
\Mod(\thr{X_{\weak}})$ such that for every algebra $A \in
\Mod(\thr{X_\cons})$ there is a tight $\Sigma_\cons$-monomorphism $\fu{\iota}_A: A \to \iota(A)$
\item For every $A\in\Mod(\thr{X_\cons}): \iota(A) =
(\fu{F}(A))|_\mu$, i.e., 
the following diagram commutes:
\[ \xymatrix@C=0.5cm@R=0.5cm{
	\Mod(\thr{X_\cons}) \ar[rr]^{\fu{F}} \ar[rd]_{\iota}
			&& \Mod(\thr{P[X]_\cons}) \ar[ld]^{|_\mu} \\
		& \Mod(\thr{X_{\weak}}) \\
								}
\]
	\end{enumerate}
\end{fact}
%In the following we will drop the application of $\fu{id_\weak}$ from the
%notation. 
(Although the $\iota$ here is not the same as in
definition~\ref{def:parametersemantic}, its role is essentially the same and
there is no real danger in confusing the two.)
The requirement of $\iota_A$ being a monomorphism implies that it must
be injective \cite{catrel}. The tightness requirement ensures that
$\iota_A(\cons^A)= \mu(\cons)^{\fu{F}(A) |_{\mu}}$, i.e., the carrier
$s^A=\cons_s^A$ is injectively embedded into the carrier $s^{\fu{F}(A)}$ as the subset
$\mu(\cons_s)^{\fu{F}(A)}$. Together, the requirements mean that 
$A$ is a (tight) subalgebra of
$\fu{F}(A)|_{\mu}$ and the carrier of this subobject corresponds bijectively in $\fu{F}(A)$ to
$\mu(\cons_s)^{\fu{F}(A)}$ -- thus ensuring
protection of the parameter algebra. This is a generalization of the
requirement that $\fu{F}$ has to be a persistent functor. The
classical persistency is obtained as the special case when $\mu(\cons_s)= \cons_s$, for all $s$.


\begin{example}\label{ex:stackelSem}
For the specification of stacks from example~\ref{ex:stackelSP} we may, for
instance, define the following semantic functor
$\fu{F}:\Mod(\thr{El}_\cons)\to\Mod(\thr{Stack[El]_\cons})$: 
\begin{itemize}\MyLPar
\item objects: a given $A\in\Mod(\thr{El}_\cons)$ is mapped to
$\fu{F}(A)\in\Mod(\thr{Stack[El]_\cons})$ given by:
\begin{itemize}\MyLPar
 \item $El^{\fu{F}(A)} = El^A\cup\{\bot\}$, where $\bot\not\in El^A$, and \\
  $Stack^{\fu{F}(A)} = (El^{A})^*$ -- finite strings of elements from $A$
 \item $empty^{\fu{F}(A)}=\varepsilon$ -- the empty string
 \item $push^{\fu{F}(A)}(x,s) = \left\{\begin{array}{ll}s & if\ x=\bot\\ xs &
  otherwise\end{array}\right.$
 \item $pop^{\fu{F}(A)}(xs) = s$ and $pop^{\fu{F}(A)}(\varepsilon)=\varepsilon$
 \item $top^{\fu{F}(A)}(xs) = x$ and $top^{\fu{F}(A)}(\varepsilon)=\bot$
 \item $ok^{\fu{F}(A)} = El^{A}$, $\cons_{El}^{\fu{F}(A)} = El^{\fu{F}(A)}$
  and $\cons_{Stack}^{\fu{F}(A)} = Stack^{\fu{F}(A)}$.
\end{itemize}
\item morphisms: $h:A\to B$ is mapped to $\fu{F}(h):\fu{F}(A)\to \fu{F}(B)$
given by: 
 \begin{itemize}\MyLPar
 \item for $x\in El^{\fu{F}(A)}: \fu{F}(h)(x) = \left\{\begin{array}{ll} h(x) 
 & {\rm if\ }x\not= \bot \\ \bot & {\rm otherwise}
 \end{array}\right.$
 \item  for $s\in Stack^{\fu{F}(A)}: \fu{F}(h)(s)$ is the pointwise 
 application of $\fu{F}(h)(x)$ to successive $x$'s in $s$. 
 \end{itemize} 
\end{itemize} 
Of course, one might attempt a more specific error treatment but we are here
merely illustrating the basic idea. 
We check that $\fu{F}$ satisfies 
definition \ref{def:parametersemantic} by verifying the
conditions of fact~\ref{fa:parsemAlt}. 

The functor $\iota:\Mod(\thr{El_\cons}) \to \Mod(\thr{El_\weak})$ is given by
$\iota(A)= \fu{F}(A)|_\mu$, which obviously makes the diagram in point
2. (fact~\ref{fa:parsemAlt}) commute. For each $A\in\Mod({\thr{El_\cons}})$, $\iota_A$ is the embedding of $A$
into $\fu{F}(A)|_\mu$ ($\forall x\in|A|:\iota_A(x)=x$), i.e., it is
monomorphism. It is tight (mainly) because we have that $\iota_A(\cons^A) = El^A=
(ok^{\fu{F}(A)})|_{\mu} = \mu(\cons)^{\fu{F}(A)|_{\mu}}$. (To obtain this last equality it is essential
that the target is required merely to be a model of $\thr{EL_\weak}$ and not
of $\thr{El_\cons}$. Verification of tightness for other operations is easy.) Thus, although the reduct
${\fu{F}(A)|_{\mu}}$ contains the additional element $\bot$ in its
carrier, this element does not interfere with the requirement of $A$ being a
tight subalgebra of this reduct.
\end{example}
%
Notice that, thanks to the local guards $x\prec ok\To...$ in axioms 2. and
3. from $\thr{Stack[El]_\cons}$, we can actually obtain extension of the
carrier with a new $El$ement $\bot$, and yet ignore it when $push$ing on
stacks. 
The above functor illustrates just one possibility of the functor semantics
from definition~\ref{def:parametersemantic}. There are, of course, more.

A functor sending each
$\thr{El_\cons}$ algebra $A$ to the standard stack algebra where $top(empty)$
returns the empty set would be another possibility. This would be, in fact,
the solution analogous to the free-persistent functor semantics with partial
algebras (the local guards $x\prec ok$ could then be dropped in axioms 2. and
3. in $\thr{Stack[El]_\cons}$). It is known that partial algebras admit more free persistent functors
than the total algebras do, and we can capture this extensions since
undefinedness of partial algebra terms can be modeled using empty set
(see~\cite{partial}). 

But our extension is much more powerful since, in general, it will
admit free functor semantics whenever $\thr{P[X]_\cons}$ does not force any new
identifications of $\thr X_\cons$ elements (which aren't already forced by
$\thr X_\cons$). The example illustrates the possibility 
of extending the carrier of parameter algebra. If our specifications are deterministic (or, more
generally, allow free extensions), we can always choose the free functor
semantics (as long as no new identifications are implied). Such a functor can
be chosen as the semantics of the 
specification in this example, and also in example~\ref{ex:groups}. This cannot be done with partial
algebras alone as long as one uses the classical definition of persistent functor.

Of course, the definition~\ref{def:parametersyntax} of the syntax of PDT does
not guarantee the existence of a semantic 
functor. As in the case of persistency, we would have to show that the
parameterized theory $\thr{P[X]_\cons}$ is a conservative extension of the
parameter $\thr{X_\weak}$, and this problem is undecidable. It is a possible
topic for further work to identify syntactic restrictions on the
specifications ensuring the existence of a semantic functor.

\section{Actual parameter passing}\label{se:app}
Since we specify PDTs, it might seem that passing an
actual parameter amounts merely to applying a semantic functor to the actual (parameter)
algebra in order to obtain a new algebra. This, however, is only what happens
at the level of programs implementing our specifications. Remaining still at
the specification level, we want to allow passing other specifications as
actual parameters to a specification of parameterized data type -- 
such an operation should yield a specification of a {\em new} PDT.
That is, instantiating the formal parameter of a PDT 
by a specification allows one to reuse the PDTs. This will
be addressed mostly by the syntactic considerations below and in
subsection~\ref{sub:appsyn}. In subsection~\ref{sub:appsem} we will show that
instantiation at the level of specifications can be reflected at the level of
semantic functors. In fact, given a semantic functor $\fu F$ for a PDT $\pdtmu$ and
an instantiation of the formal parameter $\thr{X_\cons}$ by an actual
parameter $\thr{Y_\cons}$, we can actualize $\fu F$, i.e., construct a semantic functor for the
resulting specification in a canonical way. As in the classical approach to
PDTs based on free-persistent functor semantics, this implies the possibility of
reusing not only specifications of PDTs but also their actual implementations. 

We start with an example of the intended construction. We use
the specification of stacks parameterized by elements from example~\ref{ex:stackelSP}.
As the actual parameter we want to pass the specification of natural
numbers from example~\ref{ex:Nat}.
(Recall that, given a standard specification {\bf Nat}, we can obtain the
desired form ${\bf Nat}_\cons$ by applying the syntactic operation $\__\cons$
from definition~\ref{def:specOps}.)

Let the actual parameter passing morphism -- a specification morphism
$\nu:{\bf El}_\cons\to{\bf Nat}_\cons$ -- be given by $\nu(El)= Nat$
and $\nu(\cons_{El}) = \cons_{Nat}$. The result of instantiation is the following specification
\[
	\spec{
	\tit{\mbox{\bf spec\ Stack[Nat]}_\cons^{{}}=} \\
		\spSorts{\Sorts} {Stack,Nat}
		\spOps{\Ops} { empty: && Stack\\
			 top: & Stack & Nat\\
			 pop: & Stack & Stack\\
			 push: & Nat \times Stack & Stack \\
			 zero: && Nat \\
			 succ: & Nat & Nat }
		\spOps{{\allcons}}{	 \cons_{Nat}: &&  Nat \\
			 \cons_{Stack}: && Stack \\
			 ok: && Nat }
		\Phi:
			&1.& && zero \eleq zero \\
			&2.& x \prec ok & \To & succ(x) \eleq succ(x)\\
			&3.& x \prec ok & \To & pred(succ(x)) \eleq x\\
			&4.& && empty \eleq empty \\
			&5.& x \prec ok
				 & \To & top(push(x,s)) \eleq x\\
			&6.& x \prec ok
				& \To & pop(push(x,s)) \eleq s \\
		\Gamma:	&7.& && x \prec \cons_{Nat}\\
			&8.& && s \prec \cons_{Stack} \\
	}
\]
%
The first three axioms come from $\thr{Nat}_\cons$.
The thing to observe is that the guards $x\prec \cons_{Nat}$ in axioms 2. and 3. from ${\bf
Nat}_\cons$ have changed to $x\prec ok$. This happens because the
parameterization morphism was defined by $\mu(\cons_{El}) = ok$, and
$\delta(\cons_{El})=ok$ prescribed {\bf restricting (local) guards}.
The
above specification is obtained as a pushout (in the category of
multialgebraic specifications $\ThMA$ -- see section~\ref{se:ma}) of $\mu$ and $\nu$:
\eq{\label{eq:pushoutA} %\begin{array}[c]{l}
\xymatrix@R=0.5cm@C=0.5cm{
		\thr{El_\weak} \ar[rr]^{\mu} \ar[dd]_{\nu}
			&& \thr{Stack[El]_\cons} \ar[dd]^{\nu'}	\\ \ \\
	\thr{Nat_{\weak}} \ar[rr]_{\mu'}
			&& \thr{Stack[Nat]_\cons}
}
%\end{array}
}
Notice that the formal and actual parameter specifications in this diagram
are weakened (to ${\bf El}_\weak$, resp. ${\bf Nat}_{\weak}$) by removing
their global guards. The intention of this weakening is to remove the translation
$x\prec ok$ of the global guard $x\prec \cons_{Nat}$ along $\mu'$ from the resulting
specification -- we want to keep this global guard there, and not the
stronger (but not intended) $x\prec \mu'(\cons_{Nat})=ok$. The global guard
$x\prec\cons_{Nat}$ enters the result along
$\nu'$ from $\thr{Stack[El]_\cons}$.

The result is  a new PDT
$\pdtsp{\mu'}{\delta'}{\thr{Nat_\cons}}{\thr{Stack[Nat]_\cons}}$, where
$\mu'$ and $\delta'$ are identities except for 
$\mu'(\cons_{Nat}) = ok =\delta'(\cons_{Nat})$.
%%\begin{itemize}\MyLPar
%%\item $\mu'(\cons_{Nat}) = ok$
%%\item $\delta'(\cons_{Nat}) = ok$.
%%\end{itemize}
The latter equality reflects {\bf restricting (local) guards} in the axioms from ${\bf
 Nat}_{\weak}$ (2., 3.) according to $\mu$.

\subsection{Syntax of actual parameter passing}\label{sub:appsyn}
In order to ensure the existence of the pushout as illustrated in
diagram~(\ref{eq:pushoutA}) above, we have to ensure that the involved morphisms $\mu$
and $\nu$ from ${\bf El}_\weak$ are actually specification morphisms. The
former is so by fact~\ref{fa:weakparamo}. As to the latter, we have to
take into account a more general situation than in diagram
(\ref{eq:pushoutA}), namely, the possibility that 
the actual parameter contains sorts which are not in the image of $\nu$, i.e., 
that the passing morphism is not surjective on the sorts.

The global guards of sorts $s'$ which are in the image of $\nu$ should be
dropped from the actual parameter in order not to be translated by $\mu'$.
%%!), while the local ones treated
%%according to the prescription given by $\delta(\cons_{s'})$.
But if $\thr Y_\cons$ contains a sort $s$ which is not in the image of $\nu$, its
global guard $x\prec \cons_s$ should be preserved in $\thr{P[Y]_\cons}$,
while the local ones left unchanged (since parameterization does not affect
these sorts at all).
The treatment of global guards in these two cases calls for the 
definition~\ref{def:weakalong} of weakening a specification
$\thr{Y}_\cons$ not to $\thr{Y}_\weak$ but only to $\thr{Y}_{\nu(-)}$ by removing
only the global guards which are in the image of a given (signature)
morphism $\nu$. 

\begin{definition}\label{def:weakalong}
Given signatures $\Sigma_\cons=(\Sorts,\Ops,\cons)$ and
$\Sigma'=(\Sorts',\Ops',\cons')$, guarded specifications
$\thr{SP_\cons}=(\Sigma_\cons,\Phi_\cons,\Modax_\Sigma)$,
$\thr{SP'_\cons}=(\Sigma'_\cons,\Phi'_\cons,\Modax_{\Sigma'})$ and a
signature morphism $\nu:\Sigma_\cons\to\Sigma'_\cons$, the {\em weakening of
$\thr{SP'_\cons}$ along $\nu$} is the specification $\thr{SP'_{\nu(\weak)}} =
(\Sigma'_\cons,\Phi'_\cons,\Modax_{\Sigma'}\setminus\{x\prec\nu(\cons_s):s\in
S\})$.
\end{definition}
This weakening removes only these global guards from the target specification
which are in the image of global guards from the source specification. 
%%In
%%particular, if $\nu(\cons_s)\not=\cons_{\nu(s)}$, the global guard $x\prec
%%\cons_{\nu(s)}$ will not be removed, but $x\prec \nu(\cons_s)$ will be.
This definition will be used only in conjunction with the next one.

\begin{definition}
An {\em actual parameter passing} is a specification morphism:
$\nu:\thr{X_\weak} \to \thr{Y_{\nu(\weak)}}$ satisfying
$\nu(\cons_s) = \cons_{\nu(s)}$, for all $s \in \Sigma(\thr{X_\weak})$.
\end{definition}
The intuition behind the extra requirement $\nu(\cons_s) = \cons_{\nu(s)}$
should be clear: $\cons_s$ in the formal parameter stands for all elements of
the carrier of sort $s$ and  $\cons_{\nu(s)}$ does the same with respect to the
elements of the carrier $\nu(s)$ in the actual parameter. Thus $\nu$ should
identify the two --  the restrictions induced on and expressed using
$\cons_s$ should now be transferred to $\cons_{\nu(s)}$.

We register the following simple fact to be of relevance for defining
semantics of instantiation and composition.

\begin{lemma}\label{le:id-^-1}
Given specifications $\thr X_\cons$ and $\thr Y_\cons$, 
if $\nu:\thr{X_\weak}\to \thr{Y_{\nu(-)}}$ is an actual parameter passing,
then $\nu:\thr X_\cons\to \thr Y_\cons$ is a specification morphism.
\end{lemma}
%%\begin{PROOF}
%%Let $\thr{X_\weak}=(\Sigma,\Phi_\cons)$,
%%$\thr{Y_{\nu(\weak)}}=(\Sigma',\Phi'_\cons,\Modax'_\weak)$, and $\nu:\thr
%%{X_\weak}\to \thr{Y_{\nu(-)}}$ be an actual parameter passing.
%%Suppose that $A \in \Mod(\thr{Y_\cons})$. Then
%%\begin{enumerate}\MyLPar
%%\item First, $A\in \Mod(\thr{Y_{\nu(\weak)}})$ since
%%$\thr{Y_{\nu(-)}}\subseteq \thr{Y_\cons}$. Then, since 
%%$\nu:\thr{X_\weak}\to \thr{Y_{\nu(-)}}$ is
%%	a specification morphism we get that $A|_\nu \in
%%	\Mod(\thr{X_\weak})$, so $A|_\nu \models \Phi_\cons$.
%%
%%\item Since $A\in\Mod(\thr Y_\cons)$, we have that $A\models x\prec
%%\cons_{s'}$ for all $s'\in\Sigma'$, in particular, $A \models x
%%	\prec \cons_{\nu(s)}$ for all $s \in \Sigma$. 
%%Since $\nu(\cons_s) = \cons_{\nu(s)}$,  we obtain
%%	$A|_\nu \models x \prec \cons_s$ for all $s \in \Sigma$,
%%	i.e., $A|_\nu \models \Modax_\Sigma$.  
%%\end{enumerate} 
%%1. and 2. mean that $A|_\nu \in \Mod(\thr{X_\cons})$, and since $A$ was
%%arbitrary, the claim follows. 
%%\end{PROOF}

\begin{definition}\label{def:actparPush}
Given a PDT $\pdt$ 
and an actual parameter passing $\nu:\thr{X_\weak}
\to 
\thr{Y_{\nu(\weak)}}$, the result of instantiation (the actual parameter passing)
 is a specification $\thr{P[Y]_\cons}$ obtained by pushout (in $\ThMA$) of
$\nu$ and $\mu$:
\[ \xymatrix@R=0.6cm@C=0.6cm{
	&& \thr{X_\weak} \ar[rr]^{\mu}_\delta \ar[dd]_{\nu}
			&& \thr{P[X]_\cons} \ar[dd]^{\nu'}	\\
	\\
			&& \thr{Y_{\nu(\weak)}} \ar[rr]^{\mu'}%_{\delta'}
			&& \thr{P[Y]_\cons}
								}
\]
\end{definition}
%
Now, it is not clear that the result will be a PDT. In fact, since pushout is
defined only up to isomorphism, this need not be the case. To ensure that it
is, we will make a canonical choice of the pushout object. 

\subsubsection{The canonical pushout object}\label{sub:canpush}
The examples and technicalities in this subsection culminate in
lemma~\ref{le:weakmu} and proposition~\ref{prop:paramtoo}. Impatient reader
may only register these facts and continue reading at section~\ref{sub:appsem}.

In the definition~\ref{def:parametersyntax} of PDT
we demand that the parameterization
morphism  be an inclusion on the reduced signature of the
(formal)parameter. Thus $\mu'$ has to be the identity on
$\Sigma_\weak(\thr{Y_{\nu(\weak)}})$, so the names in the pushout object
$\thr{P[Y]_\cons}$ should be chosen appropriately.

A more intricate question concerns the choice of the sets $\Cons$ and
$\subcons$ in the resulting specification. Since morphisms in
$\cat{Sign_\cons}$ are the same as in $\cat{Sign}$, we may have two different
signatures $\Sigma_\cons\not=\Sigma'_\cons$, with
$\under{\Sigma_\cons}=\Sigma=\under{\Sigma'_\cons}$, which differ merely by
the fact that a constant $c$ in $\Sigma_\cons$ belongs to $\subcons$ while in
$\Sigma'_\cons$ to $\Cons'$. 
We want the resulting
specification to possess sort constant $\cons_s$ for each sort symbol $s$
(assuming that the parameterized specification and the actual parameter do),
and also to be fully guarded (assuming that the parameterized
specification and the actual parameter are). The idea here is to include
among the sort constants $\Cons_{\thr{P[Y]_{\cons}}}$ of the resulting specification all the
(sub)sort constants $c$, such that the specification contains also the
respective global guard $x\prec c$. 
We illustrate it by the following examples.

\begin{example}\label{ex:sigchoice}
Let $\thr{X}_\cons$ contain two sorts $S_1,S_2$ and $\mu$ send the respective
$\mu(\cons_i)=c_i$ in  $\thr{P[X]_\cons}$. Let $\nu$ identify these two
sorts, i.e., $\nu(S_1)=\nu(S_2)=S$. The resulting pushout $\thr{P[Y]_\cons}$ is shown in the
rightmost bottom corner:
\[
\xymatrix@R=0.5cm@C=0.5cm{
\thr X_\weak  \ar[ddd]_\nu & S_1: & {\cons_1} \save[].[dr]*\frm{.}\restore \ar[dddr] \ar[rrrr]^\mu && &&
           c_1 \ar[dddl] & & \cons_1 \ar[dddl]  & x_1\prec
           \cons_1  \ar@{.>}[dddl] & \thr{P[X]_\cons} \ar[ddd]^{\nu'} \\
 & S_2: & & \cons_2 \ar[rr]^\mu \ar[dd]  && c_2 \save[].[urrrr]*\frm{.}\restore 
               \ar[dd] & & \cons_2 \ar[ddl] &  & x_2\prec
           \cons_2  \ar@{.>}[dd]\\ 
& & &&&&&&  \\
\thr{Y_{\nu(\weak)}} & S: & & \cons \save[].[]*\frm{.}\restore \ar[rr]^{\mu'} && 
    c \save[].[rrrr]*\frm{.}\restore & \cons_2 & \cons_1 &   x\prec \cons_1 &
    x\prec \cons_2 & \thr{P[Y]_\cons}
}
\]
The resulting specification  $\thr{P[Y]_\cons}$ contains two constants $\cons_1$ and $\cons_2$,
both of the same sort $S$ and it also contains both global guards $x\prec
\cons_1$ and $x\prec \cons_2$ originating from the global guards in
$\thr{P[X]_\cons}$.
\end{example}
%
The example indicates the reason for allowing the set $\Cons$ to contain more
than one constant $\cons$
of each sort. Of course, trivial and automatic manipulation may be performed
to remove such redundant duplicates, but its result, although in a strong
sense equivalent (due to the global guards, the respective model categories
are not only equivalent but contain ``essentially'' the same objects), won't be isomorphic to the
pushout specification (due to the difference
in the signatures).
%
%%\begin{example}\label{ex:sigchoiceB}
%%Let $\thr{X}_\cons$ contain again two sorts $S_1,S_2$ which are identified by
%%$\nu$, i.e., $\nu(S_1)=\nu(S_2)=S$.  
%%But now let $\mu$ send
%%$\mu(\cons_1)=c_1$ while $\mu(\cons_2)=\cons_2$ in  $\thr{P[X]_\cons}$.
%%The resulting pushout $\thr{P[Y]_\cons}$ is shown in the
%%rightmost bottom corner:
%%\[
%%\xymatrix@R=0.5cm@C=0.5cm{
%%\thr X_\weak \ar[ddd]_\nu & S_1: & {\cons_1} \save[].[dr]*\frm{.}\restore \ar[dddr] \ar[rrrr]^\mu && &&
%%           c_1 \ar[dddl] &  \cons_1 \ar[dddl]  & x_1\prec
%%           \cons_1 \ar@{.>}[dddl]  & \thr{P[X]_\cons} \ar[ddd]^{\nu'} \\
%% & S_2: & & \cons_2 \ar[rr]^\mu \ar[dd]  && \cons_2 \save[].[urrr]*\frm{.}\restore 
%%               \ar[dd] &  &  & x_2\prec \cons_2 \ar@{.>}[dd]\\ 
%%& & &&&&&&  \\
%%\thr {Y_{\nu(\weak)}} & S: & & \cons \save[].[]*\frm{.}\restore \ar[rr]^{\mu'} && 
%%    \cons \save[].[rrr]*\frm{.}\restore & \cons_1 &   x\prec \cons_1 & x\prec
%%    \cons  & \thr{P[Y]_\cons}
%%}
%%\]
%%The resulting specification  $\thr{P[Y]_\cons}$ contains again two constants $\cons_1$ and $\cons$,
%%both of the same sort $S$, with the respective global guards $x\prec
%%\cons_1$ and $x\prec \cons$ originating from the global guards in
%%$\thr{P[X]_\cons}$. The point now is that $\nu'$ is sending
%%$\nu'(c_1)=\cons$ or, in other words, that $\mu'(\cons)$ is actually a sort constant
%%$\cons\in\Cons_{\thr{P[Y]_{\cons}}}$ and not merely a subsort constant
%%$c\in\subcons_{\thr{P[Y]_{\cons}}}$. This is because $\mu(\cons_2)=\cons_2$
%%which ``overrides'' the fact that $\mu(\cons_1)=c_1$. 
%%\end{example}
%%The examples illustrate the motivation for the following definition of the
%%canonical choice of the pushout specification and, in particular, its signature.

\begin{definition}\label{def:quasiunique}
In definition~\ref{def:actparPush} we choose the (signature for the)
pushout object $\thr{P[Y]_\cons}$ in the following canonical way. 

Given a pushout signature $\under{\Sigma(\thr{P[Y]_\cons})}$ of $\mu$ and $\nu$ in
$\cat{Sign}$, we choose as the names in $\Sigma(\thr{P[Y]_\cons})$ all the names coming from $\Sigma(\thr
Y_\cons)$ -- the rest of the names are ``inherited'' from $\thr{P[X]_\cons}$. 
(When several subsort constants from $\thr{P[X]_\cons}$ get identified
 (like in example \ref{ex:sigchoice}), 
just choose a fresh name for the resulting subsort constant.)

As the resulting set of sort constants, 
$\Cons_{\thr{P[Y]_{\cons}}}$, we take the images of all constants
$c\in\allcons_{\thr{Y_{\cons}}}\cup\allcons_{\thr{P[X]_{\cons}}}$ for which we also have a
global guard $x\prec c$. (These are 
the images under $\nu'$ of all
$\Cons_{\thr{P[X]_{\cons}}}$ and (the images of) those
$\Cons_{\thr{Y_{\cons}}}$ which are {\em not} in the image of $\mu$.)
% -- cf. example~\ref{ex:sigchoiceB}). 

The subsort constants are all the remaining images of subsort constants from
$\thr{Y_\cons}$ and $\thr{P[X]_\cons}$, i.e., 
\[\subcons_{\thr{P[Y]_{\cons}}} = (\nu'[\allcons_{\thr{P[X]_{\cons}}}] \cup
\mu'[\allcons_{\thr{Y}_{\cons}}]) \setminus \Cons_{\thr{P[Y]_{\cons}}}.
\]
Finally, as the axioms we take the union of the axioms from
$\thr{Y_{\nu(\weak)}}$ (translated along $\mu'$) and $\thr{P[X]_\cons}$ (translated along $\nu'$).
\end{definition}
%
With this choice, we can now state two facts which will
be used in the sequel.
%
\begin{lemma}\label{le:weakmu}
Given a PDT $\pdtmu$ with an actual parameter
passing $\nu: \thr{X_\weak} \to \thr{Y_{\nu(\weak)}}$,
the pushout diagram as in the definition~\ref{def:actparPush} and the
canonical pushout object from definition~\ref{def:quasiunique}:
then $\nu': \thr{P[X]_\weak} \to \thr{P[Y]_{\nu(\weak)}}$ is an actual
parameter passing. 
\end{lemma}
%%\begin{PROOF}
%%Let $\thr{P[X]_\cons}= (\Sigma_\cons',\Phi'_\cons,\Modax_{\Sigma'_{\cons}})$.
%%That $\nu'(\cons_{s'}) = \cons_{\nu'(s')}$ for all $s' \in \Sigma_\cons'$
%%follows from the pushout properties, since if $\cons_{s'} = \mu(\cons_s)$, then
%%$\nu'(\cons_{s'})= \nu(\cons_s)= \cons_{\nu(s)}$ (since $\nu$ is an actual
%%parameter passing), and otherwise $\nu'(\cons_s) =
%%\cons_{\nu'(s')}$. Furthermore, $\nu'$ is a specification morphism, i.e.,
%%$\thr{P[Y]_{\nu(\weak)}} \models \thr{P[X]_\weak}$,  since
%%for every formula $\phi_\cons \in \thr{P[X]_\weak}:\nu'(\phi_\cons) \in
%%\thr{P[Y]_{\nu(\weak)}}$, by the canonical pushout construction.
%%\end{PROOF}
%
\begin{proposition}\label{prop:paramtoo}
With $\thr{P[Y]_\cons}$ and $\mu'$ from definition~\ref{def:quasiunique}, 
$\pdtsp{\mu'}{\mu'}{\thr{Y_\cons}}{\thr{P[Y]_\cons}}$ is a PDT.
\end{proposition} 
%%\begin{PROOF}
%%All global guards from $\thr{Y_{\nu(\weak)}}$ get included in
%%$\thr{P[Y]_\cons}$. 
%%Translations along $\mu$ (and then $\nu'$) of global guards
%%$\Modax_{\Sigma({\thr{X_{\cons}}})}$ originating from $\thr{X_\cons}$ are
%%``forgotten'' (by starting from $\thr{X_\weak}$), but the global guards
%%themselves are passed from $\thr{P[X]_\cons}$ along $\nu'$. 
%%Thus $\thr{P[Y]_\cons}$ is globally guarded (if $\thr{Y_\cons}$ and
%%$\thr{P[X]_\cons}$ are). Also, if
%%$\thr{Y_\cons}$ and $\thr{P[X]_\cons}$ are fully guarded, then so is $\thr{P[Y]_\cons}$.
%%
%%$\mu'$ is identity by the choice of the names in $\thr{P[Y]_\cons}$, possibly
%%with the exception of some $\cons\in\Cons_{\thr Y_{\cons}}$, which are in the image of
%% $\Cons_{\thr X_{\cons}}$ and are mapped along $\mu$
%%(example~\ref{ex:sigchoice}). But these are then mapped to fresh constants,
%%either as by $\mu$, or by the choice of the canonical $\thr{P[Y]_\cons}$.
%%
%%The local guard mapping $\delta'=\mu'$ makes inclusion of the axioms 
%%$\mu'(\cons_{s'})\prec \delta'(\cons_{s'})$ unnecessary. 
%%
%%Finally, by the canonical pushout construction for all
%% the axioms  $\phi_\cons\in\thr{Y_{\nu(\weak)}}$
%%(i.e., except some global guards), we have
%% $\mu'(\phi_\cons)\in\thr{P[Y]_\cons}$, so that the point~\ref{it:rel} of
%%definition~\ref{def:parametersyntax} is satisfied.
%%\end{PROOF}
%
In some cases, it might seem more natural to choose $\delta'$ in a
different way (i.e., $\delta'\not=\mu'$), and
we expect that one can obtain more flexibility in passing actual
parameters, but the details of that need to be postponed to a future work. 
For now, we are satisfied with the proposition~\ref{prop:paramtoo}, and ignore the details
and possibility of more specific choices of the PDTs resulting from instantiation.


\subsection{Semantics of actual parameter passing}\label{sub:appsem}
The first aspect of the semantics of instantiation is expressed in
proposition~\ref{prop:paramtoo} -- it gives a specification of a new parameterized
data type. That is, we can reuse PDTs for constructing new PDTs by 
instantiating the formal parameter. 

There is, however, another semantic aspect which we call ``actualization''.
Given a functor $\fu{F}:\Mod(\thr{X_\cons})\to \Mod(\thr{P[X]_\cons})$
defining semantics of a PDT according to
definition~\ref{def:parametersemantic} and an actual parameter passing
$\nu:\thr{X_\weak}\to\thr{Y_{\nu(-)}}$, we want to define the way of
transforming $\thr{Y_\cons}$ algebras into $\thr{P[Y]_\cons}$ algebras, i.e.,
a functor $\fu{F}':\Mod(\thr{Y_\cons})\to \Mod(\thr{P[Y]_\cons})$ which is
{\em induced} by $\fu F$. In
addition, we want this functor to satisfy the conditions corresponding to
those imposed on the semantics of parameterized data type specifications (fact~\ref{fa:parsemAlt}).

Let us return to the example of the PDT of stacks
from example~\ref{ex:stackelSP}, the chosen semantic functor
$\fu{F}:\Mod(\thr{El_\cons})\to\Mod(\thr{Stack[El]_\cons}$ from 
example~\ref{ex:stackelSem} and the actual parameter passing $\nu:\thr{El_\weak}\to\thr{Nat_\weak}$ from the
beginning of this section. Let $N$ be the standard $\thr{Nat_\cons}$
algebra (i.e., the standard $\thr{Nat}$ algebra with $\cons_{Nat}^N$ being
all the natural numbers). The functor $\fu{F}'$ will simply embed $N$ into
$\fu{F}'(N)$ and mimic the action of $\fu{F}$ with respect to constructing
the rest of $\thr{Stack[Nat]_\cons}$ algebra $\fu{F}'(N)$. 

The important point is that such an actualization of functor $\fu F$ 
to a functor $\fu{F}'$ can be done canonically given a semantic functor
$\fu{F}$ with the corresponding
$\iota:\Mod(\thr{X_\cons})\to\Mod(\thr{X_\weak})$.
This corresponds to the classical case of free-persistent functor semantics of
parameterization in the presence of amalgamation lemma. Notice, however, that
our result is far more general, since we show it for any functor satisfying
definition~\ref{def:parametersemantic}. Thus we 
do not require persistency (but
allow extension of the carrier) and, furthermore, the extension need not be
free, i.e., free functors are only special cases. 
To show this, we will need the following definition.

\begin{definition}\label{def:actparsem}
Given a PDT $(\mu,\thr{X_\cons}, \thr{P[X]_\cons})$,
the semantic functor $\fu{F}:\thr{X_\cons} \to \thr{P[X]_\cons}$, with
corresponding $\iota$ and an actual parameter passing morphism $\nu_\weak:
\thr{X_\weak} \to \thr{Y_{\nu(\weak)}}$. A 
functor $\iota': \Mod(\thr{Y_\cons}) \to
\Mod(\thr{Y_{\nu(\weak)}})$ is {\em induced by} $\iota$ iff for all $Y \in
\Mod(\thr{Y_\cons})$:
\begin{enumerate}\MyLPar
\item there is a tight monomorphism $\iota'_Y:Y\to \iota'(Y)$, and
\item $\iota( Y |_{\nu}) = \iota'(Y)|_{\nu}$.
\end{enumerate}
\end{definition}
The second point uses overloaded notion of $\nu$ which is admissible by
lemma~\ref{le:id-^-1}. It means the commutativity of the leftmost
square (2.)
in the diagram below. The rest of the diagram is referred to 
in the following definition.
\[ \xymatrix@R=0.5cm@C=0.5cm{
&& \Mod(\thr{X_\cons})  \ar[dd]_{\iota} \ar[rrdd]^{\fu{F}}\\
&& && \\
\Mod(\thr{Y_\cons}) \ar[dd]_{\iota'} \ar[rrrdd]^{\fu{F'}} \ar[rruu]^{|_{\nu}}
	&& \Mod(\thr{X_{\weak}}) \ar@{}[ur]|{1.}
		&& \Mod(\thr{P[X]_\cons}) \ar[ll]_{|_{\mu}} \\
&& && \\
\Mod(\thr{Y_{\nu(\weak)}}) \ar[uurr]_{|_{\nu}}  \ar@{}[uuuurr]|{2.}
	&&& \Mod(\thr{P[Y]_\cons}) \ar[lll]^{|_{\mu'}} \ar[uur]_{|_{\nu'}} \ar@{}[luu]|{3.}	
								}
\]
%
Notice that the lower triangle of this diagram is actually a more specific
case of the general requirement, namely, that $\iota'$ is a functor
$\iota':\Mod(\thr{Y_\cons})\to\Mod(\thr{Y_\weak})$, since
$\Mod(\thr{Y_{\nu(\weak)}})\subseteq\Mod(\thr{Y_\weak})$. It corresponds to
the fact that carrier of any sort $s$ from $\thr{Y_\cons}$, which is not in the image of $\nu$, is not
extended in the models of $\thr{P[Y]_\cons}$ (or else, that $\mu'$ maps its
$\cons_s$ and the respective global guard to the same $\cons_s$ and the same
global guard in $\thr{P[Y]_\cons}$).
%
\begin{definition}\label{def:actinstfunct} 
Let $\fu{F}$, $\iota$ be as in the previous definition (\ref{def:actparsem}) and
$\iota'$ be induced by $\iota$. 
The instantiation functor $\fu{F'}: \Mod(\thr{Y_\cons}) \to
\Mod(\thr{P[Y]_\cons})$ is then defined by: 
\begin{itemize} 
\item objects: $\fu{F'}(Y) = \iota'(Y)
	\amalgam_{\iota'(Y)|_{\nu}} \fu{F}(Y|_\nu)$ 
\item homomorphisms:
	$\fu{F'}(h) = \iota'(h) \amalgam_{\iota'(h)|_{\nu}} \fu{F}(h|_\nu)$
\end{itemize}
\end{definition}
The notation $Y\amalgam_X Z$ denotes the amalgamated sum of $Y$ and $Z$ with
respect to the common reduct $X$ (cf. section \ref{sub:inst}).

$\fu{F'}$ is well-defined since, by commutativity of 1. and 2. in the diagram
above, $\iota'(Y)|_{\nu} =
(\fu{F}(Y|_\nu))|_{\mu}$ and $\iota'(h)|_{\nu} = (\fu{F}(h|_\nu))|_{\mu}$.
With this definition of $\fu{F}'$, all the loops
in the diagram  above commute. 
%

By continuity of the $\Mod$-functor (from $\inst{MA}$, section~\ref{se:ma}),
the square 3. in the diagram above is a pullback
(since the corresponding specification was constructed as pushout
according to definition~\ref{def:actparPush}). Hence, by amalgamation
property, $\fu{F}'(Y)$ is indeed guaranteed to belong to
$\Mod(\thr{P[Y]_\cons})$. 

We thus have the construction of the desired actualization functor provided
that we have a functor $\iota'$ induced by $\iota$. The following proposition
shows that such an induced functor can always be obtained providing also a
way to construct it.

\begin{proposition}\label{prop:inducedapp}
Given a functor $\iota: \Mod(\thr{X_\cons}) \to
\Mod(\thr{X_{\weak}})$ (associated with the semantic functor for a PDT)
%parameterized data type specification) 
and an actual parameter passing morphism
$\nu:\thr{X_\weak} \to \thr{Y_{\nu(\weak)}}$, there exist a
functor $\iota': \Mod(\thr{Y_\cons}) \to \Mod(\thr{Y_{\nu(\weak)}})$ induced by
$\iota$.
\end{proposition}
%
\begin{PROOF}
Let the formal parameter specification be $\thr{X_\cons}=
((\Sorts,\Ops),\Phi,\Modax_\Sorts)$ and the actual parameter specification $\thr{Y_\cons}=
((\nu(\Sorts)\cup \Sorts', \nu(\Ops)\cup \Ops'), \nu(\Phi)\cup 
\Phi',\nu(\Modax_\Sorts)\cup \Modax_{\Sorts'})$, (where $\nu(\Sorts) \cap \Sorts' = \nu(\Ops)
\cap \Ops' = \nu(\Modax_\Sorts) \cap \Modax_{\Sorts'} = \emptyset$). We let $\Cons$'s to
be included in respective $\Ops$'s, since they do not require separate
treatment here. $s/\omega$ range over symbols from $\Sorts/\Ops$ and $s'/\omega'$ from
$\Sorts'/\Ops'$. The algebra $V=\iota'(Y)$ is constructed by inheriting the sorts and
operations which are not in the image of $\nu$ directly from $Y$, while those
which are in the image of $\nu$ from $\iota(Y|_\nu)$. 
We define $\fu{\iota'}$ on:
\begin{itemize}\MyLPar
\item algebras:
	for every $Y\in\Mod(\thr{Y_\cons})$ we let
	$\iota'(Y)=V\in\Mod(\thr{Y_{\nu(\weak)}})$ be the following algebra:
\begin{enumerate}\MyLPar
\item sorts:
	\begin{enumerate}\MyLPar
	\item\label{it:sim} for $s \in \Sorts : \nu(s)^{V}= s^{\iota(Y|_{\nu})}$
	\item\label{it:snew} for $s' \in \Sorts' : {s'}^{V}= {s'}^{Y}$
	\end{enumerate}
\item operations:
	\begin{enumerate}\MyLPar
	\item\label{it:opim} for $\omega \in \Ops : \nu(\omega)^{V}= \omega^{\iota(Y|_{\nu})}$, 
	\item\label{it:opnew} for $\omega' \in \Ops' : \omega'(\ovr y)^{V}= 
	  \left\{
	  \begin{array}{ll} \omega'(\ovr y)^{Y} & if\ all\ \ovr y \in |Y| \\
%	\item\label{it:opnewmix} 
            \emptyset & otherwise\ (if\ some\ \ovr y\not\in|Y|)
	  \end{array}\right.$
	\end{enumerate}
\end{enumerate}
\item homomorphisms: $\iota'(h)=h$, i.e., $\iota'(h_{\nu(s)}) = \iota(h_s)$ and $\iota'(h_{s'}) = h_{s'}$.
\end{itemize}
%
We let $\iota'(y)=y$ and show that $\iota'$ satisfies definition~\ref{def:actparsem}.
This defining equation makes it trivially a monomorphism. It is, indeed, a tight
$\Sigma(\thr{Y_\cons})$-homomorphism because: for all $\omega\in\Ops$, i.e.,
$\nu(\omega)\in\nu(\Ops)$, and all $\ovr y\in|Y|:
\nu(\omega)(\ovr y)^V\isby{\ref{it:opim}}
\omega(\ovr y)^{\iota(Y|_{\nu})}=\omega(\ovr y)^{Y|_{\nu}}=\omega(\ovr y)^Y$, where the
middle equality holds since $\iota$ is tight homomorphism, and the last one
by the definition of reduct. For the remaining operation $\omega'\in\Ops'$
and $\ovr y\in|Y|: \omega'(\ovr y)^V\isby{\ref{it:opnew}}\omega'(\ovr y)^Y$.
Furthermore, we have that $V|_{\nu} = \iota(Y|_{\nu})$, by construction.

By construction, i.e., by \ref{it:snew}, $V$  satisfies all the global guards
in $\Modax_{S'}$. 
By proposition~\ref{prop:satiff}, it also satisfies all the fully guarded
axioms of $\thr{Y_\cons}$, since $\iota'$ is a tight mono. Thus
$V\in\Mod(\thr{Y_{\nu(\weak)}})$. 
\end{PROOF}
We have thus shown the possibility of reusing specifications of parameterized
data types not only by syntactic instantiation (def.~\ref{def:actparPush}) but
also by providing the above actualization construction.

In the final example of the whole setting, we use also the possibility
offered by multialgebras to model nondeterminism. 
\begin{example}
We specify a 
generic extension of a deterministic data type with a (binary)
nondeterministic choice: we give a (generic) PDT
and instantiate it for a more specific data type.
We do not extend carrier, i.e, $\mu(\cons_{El})=\cons_{El}$.

\( \xymatrix{
\frame{\txt{\spec{
	\tit{\mbox{\bf spec\ El}_\cons^{{}}=} \\
		\spSorts{\Sorts} {El}
		\spOps{\Ops}{	\cons_{El}: &&  El \\ \ \\ }
		\Gamma:	&1. x \prec \cons_{El}%\\ \ \\ \ \\ \ 
	} } }
\ar[rrr]^<<<<<<<<<<<<<{\txt{\normalsize{$\mu(\cons_{El})=\cons_{El}$}}}_<<<<<<<<<<<<<{\txt{\normalsize{$\delta(\cons_{El})=\cons_{El}$}}}
&&&
\frame{\txt{\spec{
	\tit{\mbox{\bf spec\ $\choice$[El]}_\cons^{{}}=} \\
		\spSorts{\Sorts'} {El}
		\spOps{\Ops'} {
			\choice: & El \times El & El\\
			\cons_{El}: &&  El \\	}
		\Phi':
			&1.& 			&& x \prec x \choice y \\
			&2.& 			&& y \prec x \choice y \\
			&3.& z\prec x\choice y	
					& \To 	 & z \eleq x, z \eleq y\\
		\Gamma':	&4.& && x \prec \cons_{El}\\
	}} }
}
\)

\noindent
As the semantic functor we take the free functor $\fu{F}:\Mod(\thr{El_\cons}) \to
\Mod(\thr{\choice[El]_\cons})$, i.e. for an $\thr{El_\cons}$ algebra $A$, 
$\fu{F}(A)$ is  given by:
\begin{itemize}\MyLPar
\item $|\fu{F}(A)| = |A|$
\item $x \choice^{\fu{F}(A)} y = \{ x,y \}$
\end{itemize}
Any actual instantiation can now be associated with an actualization of this
functor. 
Using a specification of natural numbers as actual parameter we get the
specification of natural numbers with
binary choice as the result. The corresponding semantic functor
$\fu{F'}:\Mod(\thr{Nat_\cons}) \to \Mod(\thr{\choice[Nat]_\cons})$
(induced by
$\fu F$ and the obvious parameter passing $\nu(El)=Nat$), 
will embed an arbitrary $\thr{Nat_\cons}$ 
algebra $A$ into $\fu F'(A)$ by
\begin{itemize}\MyLPar
	\item $|\fu{F}(A)| = |A|$
	\item $x \choice^{\fu{F}(A)} y = \{ x,y \}$
	\item ${\omega}^{\fu{F}(A)} = {\omega}^A$ -- for all operation
	symbols $\omega\in\Sigma(\thr{Nat_\cons})$.
	\end{itemize}
\end{example}
%%One could probably think of a more general means of
%%specifying instantiation mechanisms at the algebra (program) level, that is, mechanisms of taking
%%a PDT and matching an actual parameter {\em algebra}
%%in order to
%%obtain a new data type (and not merely, as we have now, a PDT
%% which can be applied to actual parameter algebras coming only from the model
%%class of the parameter specification). The actualization mechanism described
%%above would be a special case of such a more general instantiation in that
%%the ``matching'' of actual parameter algebras here is expressed by the reduct
%%functor from $\Mod(\thr{Y_\cons})$ to $\Mod(\thr{X_\cons})$.
%%This would require a closer look at
%%the possibilities of describing the semantic functors and we have to leave such considerations for
%%future work.

%\section{Composition and refinement}\label{se:compref}
%
We will now review various ways of composing specifications of parameterized
data types. We will discuss the classical vertical and horisontal
composition, showing the counterparts of the standard compositionality
theorems. The main difference will concern the fact that, in general,
stepwise application of constructions will not yield the same result as a direct
construction along the respective composition, but a refinement of the latter.
Subsections~\ref{sub:vcsyn} and \ref{sub:vcsem} discuss vertical, and
\ref{sub:hcsyn} %and \ref{sub:hcsem} 
horisontal compostion. Section~\ref{sub:ref} will summarize the concept of refinement
which emerges from this section.

We recall that, given a parameter passing diagram (like 1. below in
Figure~\ref{fi:vertcomp}), by proposition~\ref{prop:paramtoo},
$\mu':\thr{Y_{\nu(\weak)}}\to \thr{P[Y]_\cons}$ is a parameterization
morphism, and hence, in particular (by fact~\ref{fa:weakparamo}), $\mu':\thr
Y_\weak\to\thr{P[Y]_\cons}$ is a specification morphism.


\subsection{Vertical composition}\label{sub:vcsyn}
Given two actual prameter passing morphisms $\nu:\thr X_\weak\to \thr
Y_{\nu(\weak)}$ and $\rho:\thr Y_\weak\to \thr Z_{\rho(\weak)}$,
(as indicated in the diagrams 1. and 2. in Figure~\ref{fi:vertcomp}), 
we would like to compose them vertically, i.e.,
we want to show that also $(\nu;\rho):\thr X_\weak \to \thr Z_{(\nu;\rho)(-)}$ is
an actual parameter passing.\vspace*{-2ex}

\begin{figure}[hbt]
\[\xymatrix@R=0.4cm@C=1.5cm{
\thr X_\weak \ar[r]^\mu \ar[dd]_\nu   \ar@(l,l)[dddddd]_{\nu;\rho}  \ar@{}[ddr]|{1.}
  & \thr{P[X]_\cons} \ar[dd]^{\nu'} \ar@(r,r)[dddddd]^{(\nu;\rho)'} \\ \\ 
\thr Y_{\nu(\weak)} \ar[r]^{\mu'} & \thr{P[Y]_\cons} \\ 
%
\thr Y_\weak \ar[r]^{\mu'} \ar[dd]_\rho \ar@{}[ddr]|{2.}
  & \thr{P[Y]_\cons} \ar[dd]^{\rho'} \\ \\ 
\thr Z_{\rho(\weak)} \ar[r]^{\mu''} & \thr{P[Z]_\cons} \\ 
\thr Z_{(\nu;\rho)(\weak)} \ar[r]^{\mu'''} & \thr{P[Z]'_\cons} \\ 
}
\]
\caption{}\label{fi:vertcomp}\vspace*{-1ex}
\end{figure}

\noindent
The notation from this figure will be used throughout this and next
subsection (\ref{sub:vcsyn}, \ref{sub:vcsem}).

In general, the specifications $\thr{Z_{\rho(\weak)}}$ and
$\thr{Z_{(\nu;\rho)(\weak)}}$ need not be the same -- the latter may have
more global guards than the former.
\begin{fact}\label{fa:triv}
Given a PDT $\pdtsimple{\mu}{\delta}{\thr X_\cons}{\thr{P[X]_\cons}}$
and actual parameter passing $\nu$
and $\rho$ as in the Figure~\ref{fi:vertcomp}: 
\begin{enumerate}\MyLPar
\item $\thr{Z_{(\nu;\rho)(\weak)}}\models \thr{Z_{\rho(\weak)}}$.
\item If $\nu$ is surjective on the sorts, then $\thr{Z_{\rho(\weak)}}\models
\thr{Z_{(\nu;\rho)(\weak)}}$.
\item If $\nu$ is
surjective on the sorts, then $\thr{Z_{(\nu;\rho)(\weak)}} \iso \thr{Z_{\rho(\weak)}}$.
\end{enumerate}
\end{fact}
%%\begin{PROOF}
%%Direct from definition~\ref{def:weakalong}. Obviously, both specifications
%%have isomorphic signatures (so, for simplicity, we assume that the are
%%equal). Also, all the axioms except, possibly, some global 
%%guards, are involved in both pushout constructions and will be satisfied by
%%both specifications. The only difference may concern absence in
%%$\thr{Z_{\rho(\weak)}}$  of some global which are present in
%%$\thr{Z_{(\nu;\rho)(\weak)}}$. 
%%\begin{enumerate}\MyLPar
%%\item  All sorts which are in the
%%image of $(\nu;\rho)$ are also in the image of $\rho$, so the global guards
%%dropped in $\thr{Z_{(\nu;\rho)(\weak)}}$ are also dropped in
%%$\thr{Z_{\rho(\weak)}}$. 
%%\item If $\nu$ is surjective on the sorts then if a sort is in the image of
%%$\rho$ it will also be in the image of $(\nu;\rho)$. Hence all global guards
%%from $\thr{Z_{(\nu;\rho)(\weak)}}$ will also be present in $\thr{Z_{\rho(\weak)}}$.
%%\item If $\nu$ is surjective on the sorts, then the ismorphism follows from the
%%two points above. \vspace*{-2ex}
%%\end{enumerate}
%%\end{PROOF}
Notice that, in points 2. and 3., surjectivity of $\nu$ on sorts is sufficient but
not necessary condition. It is sufficient and necessary that for any sort $s\in\Sigma(\thr{Y})$
which is not in the image of $\nu$, there is a sort  $s'\in\Sigma(\thr{Y})$ which
is in the image of $\nu$ and such that $\rho(s)=\rho(s')$.

In spite of the possible differences, the following proposition ensures at
least that
composition of actual paramter passing morphisms is still an actual parameter passing.
\begin{proposition}\label{prop:vertcomp}
If 
$\nu:\thr X_\weak\to \thr Y_{\nu(\weak)}$ and $\rho:\thr Y_\weak\to \thr Z_{\rho(\weak)}$
 are actual parameter passing
morphisms, then so is $(\nu;\rho):\thr X_\weak \to \thr Z_{(\nu;\rho)(\weak)}$
(see the diagram in Figure~\ref{fi:vertcomp}). 
\end{proposition}
%%\begin{PROOF}
%%We have both $\nu(\cons_s)=\cons_{\nu(s)}$ and
%%$\rho(\cons_{s'})=\cons_{\rho(s')}$ for all sort symbols $s\in\Sigma(\thr X)$
%%and $s'\in\Sigma(\thr Y)$, and thus
%%$(\nu;\rho)(\cons_s)=\cons_{(\nu;\rho)(s)}$. We show that
%%$(\nu;\rho):\thr X_\weak \to \thr Z_{(\nu;\rho)(\weak)}$ is
%%a specification morphism, i.e., $\thr Z_{(\nu;\rho)(\weak)} \models
%%(\nu;\rho)(\thr X_\weak)$.
%%\[
%%\begin{array}{rrl}
%%&{\rm fact\ \ref{fa:triv}} :& \thr Z_{(\nu;\rho)(\weak)}\models \thr Z_{\rho(\weak)} \nonumber \\
%%&\rho\ {\rm is\ a\ specification\ morphism} :& \thr Z_{\rho(\weak)}\models\rho(\thr Y_\weak) \nonumber \\
%%& \To & \thr Z_{(\nu;\rho)(\weak)}\models\rho(\thr Y_\weak) \label{eq:mod}
%%\end{array}
%%\]
%%Now, the axioms of $\thr Y_{\nu(\weak)} =  (\Phi,\Gamma')$, where $\Gamma'$
%%is the subset of global guards from $\thr Y_\cons$ which (whose sort symbols)
%%are not in the image
%%of $\nu$. To complete the proof we have to show that $\thr
%%Z_{(\nu;\rho)(\weak)}\models \rho(\Gamma')$.
%%But this follows directly from definition~\ref{def:weakalong}. For any
%%global guard $\gamma\in\Gamma'$ is {\em not} in the image of $\nu$ and hence
%%it will {\em not} be in the image of $\nu;\rho$. Consequently, if
%%$\gamma\in\Gamma'$ then
%%$\rho(\gamma)\in\thr Z_{(\nu;\rho)(\weak)}$ (though not necessarily 
%% $\rho(\gamma)\in \thr Z_{\rho(\weak)}$!!).
%%
%%Thus $\thr Z_{(\nu;\rho)(\weak)}\models \rho(\Gamma')$ which together with (\ref{eq:mod})
%%yields
%%\eq{
%%\thr Z_{(\nu;\rho)(\weak)}\models \rho(\thr Y_{\nu(\weak)}).
%%\label{eq:modB}
%%}
%%In diagram 1. $\nu$ is a parameter passing, so $\thr
%%Y_{\nu(\weak)}\models \nu(\thr X_\weak)$ which implies  
%%$\rho(\thr Y_{\nu(\weak)})\models (\nu;\rho)(\thr X_\weak)$.
%%This, together with (\ref{eq:modB}) give the conclusion: 
%%$\thr Z_{(\nu;\rho)(\weak)}\models (\nu;\rho)(\thr X_{\weak}))$.
%%\end{PROOF}
%
Now, in general, also the resulting specifications $\thr{P[Z]_\cons}$ and
$\thr{P[Z]'_\cons}$ may be different. In the classical case, this is merely a
consequence of their definition by pushout (which is unique only up to
isomorphism). In our case, however, the difference may be more significant,
since we also may drop and/or add some global guards on the way. 
As in fact~\ref{fa:triv}, 
the only difference may concern the presence/absence of global guards (since
all other axioms are involved in the pushout construction), so these are the only
axioms we mention in the following example.
\begin{example}\label{ex:difpush}
Consider first two instantiations $\nu:{\thr
X_\weak}\to\thr{Y_{\nu(\weak)}}$ and 
$\rho:{\thr Y_\weak}\to\thr{Z_{\rho(\weak)}}$. (Two lines in $\thr{Y_{\nu(\weak)}}$,
$\thr{Y_\weak}$, etc. represent two distinct sorts which are identified by
the second instantiation $\rho$.)
\[\xymatrix@R=0.5cm@C=0.7cm{
\thr{X_\weak} \ar[dd]^\nu & &\cons  \save[].[]*\frm{.}\restore  \ar[rr]^{\mu} \ar[dd] && 
    c \save[].[rrrr]*\frm{.}\restore \ar[dd] & & \cons \ar[dd] && x\prec
  \cons \ar@{.>}[dd] & \thr{P[X]_\cons} \ar[dd]_{\nu'}\\ 
&& && && && \\
\thr{Y}_{\nu(\weak)}& & \cons \ar[rr]^{\mu'} && c  && \cons && x \prec\cons
    \save[].[dllll]*\frm{.}\restore  & \thr{P[Y]_\cons}\\
 & y\prec \cons_1\ \ \ \cons_1 \save[].[ru]*\frm{.}\restore
    \ar[rrrr]^{\mu'} &&&& 
   \cons_1 &&  y\prec\cons_1 &  \\ 
%}\]
%\[\xymatrix@R=0.5cm@C=0.7cm{
%%% next inst
\thr{Y_\weak}\ar[ddd]^{\rho} & \cons_1 \save[].[dr]*\frm{.}\restore \ar[dddr]\ar[rrrr]_{\mu'}
&&&&  \cons_1 \ar[dddl] && y\prec \cons_1 \ar@{.>}[ddd] &  & \thr{P[Y]_\cons}
  \ar[ddd]_{\rho'}\\
 & & \cons \ar[dd]  \ar[rr]_{\mu'} && c \ar[dd] && \cons \ar[dd] 
  && x\prec\cons \save[].[ullll]*\frm{.}\restore  \ar@{.>}[dd]\\ 
& & && && && \\
 \thr{Z_{\rho(\weak)}} & & \cons' \save[].[]*\frm{.}\restore \ar[rr]^{\mu''} && 
  \cons' \save[].[rrrr]*\frm{.}\restore && \cons & y\prec\cons' & x \prec
 \cons & \thr{P[Z]_\cons}
}
\]
And now a direct instantiation along $(\comp{\nu}{\rho})$:
\[\xymatrix@R=0.3cm@C=0.9cm{
\thr{X_\weak} \ar[ddd]_{(\nu;\rho)} & \cons \save[].[]*\frm{.}\restore \ar[ddd] \ar[rr]^\mu && 
  c \save[].[rrrr]*\frm{.}\restore \ar[ddd] 
   && \cons \ar[ddd] && x\prec\cons \ar@{.>}[ddd] & \thr{P[X]_\cons} \ar[ddd]_{(\nu;\rho)'}\\
 & && &&&&& \\  & && &&&&& \\
\thr{Z_{(\nu;\rho)(\weak)}} & \cons' \save[].[]*\frm{.}\restore \ar[rr]^{\mu'''} && 
  \cons'  \save[].[rrrr]*\frm{.}\restore
   && \cons  && x\prec\cons  & \thr{P[Z]'_\cons}
}
\]
%Both $\thr{P[Z]_\cons}$ and $\thr{P[Z]'_\cons}$  have isomorphic
%signatures: in the former  $\allcons = \{\cons,\cons'\}$, while in the
%latter ${\allcons}'=\{\cons,c\}$. 
%
The significant difference consists in that  $\thr{P[Z]_\cons}$  has the
global guard for $\mu''(\cons)$, namely $y\prec\cons'$ originating from
$\thr{P[Y]_\cons}$. Thus here $\Cons=\{\cons,\cons'\}$ and $\subcons=\emptyset$.
In $\thr{P[Z]'_\cons}$, on the other hand, this guard is
not present, so here $\Cons'= \{\cons\}$, while ${\subcons}'=\{\cons'\}$.

Thus, the PDT
$\pdtsimple{\mu''}{whoever}{\thr{Z_\cons}}{\thr{P[Z]_\cons}}$ would forbid
extending the carrier of $\cons'$, while
$\pdtsimple{\mu'''}{whoever}{\thr{Z_\cons}}{\thr{P[Z]'_\cons}}$ would not.
\end{example}
%
So, in general, $\thr{P[Z]_\cons}$
and $\thr{P[Z]'_\cons}$ are not isomorphic. We have the following fact.
\begin{fact}\label{fa:pzpz}
With the notation from Figure~\ref{fi:vertcomp} and
example~\ref{ex:difpush}:
\begin{enumerate}\MyLPar
\item $\thr{P[Z]_\cons}\models \thr{P[Z]'_\cons}$.
\item if $\thr{P[Z]'_\cons}\not\models \thr{P[Z]_\cons}$, then it is only
because for some sort constant(s) $c: \thr{P[Z]_\cons}\models x\prec c$ and
$\thr{P[Z]'_\cons}\not\models x\prec c$. 
\end{enumerate}
\end{fact}
%%\begin{PROOF}
%%The signatures of both specifications will be isomorphic, so we assume that they are
%%identical. All axioms except global guards are involved in the pushout
%%constructions, so their presence (or satisfaction) follows from the standard
%%isomorphism of pushout objects. The difference may concern only some constants which are in $\Cons$ but
%%not in $\Cons'$ (only in ${\subcons}'$, as in the
%%example~\ref{ex:difpush}). This justifies point 2.
%%For point 1. we show that if $\cons\in\Cons'$ then
%%$\cons\in\Cons$, that is if $\thr{P[Z]'_\cons}\models x\prec\cons$ then
%%$\thr{P[Z]_\cons}\models x\prec\cons$, which will yield the conclusion.
%%
%%This follows trivially. Any global guard $x\prec\cons$ in $\thr{P[Z]'_\cons}$
%%is an image of a respective global guard either from
%%$\thr{Z_{(\nu;\rho)(\weak)}}$ or from $\thr{P[X]_\cons}$. In the latter case,
%%it will also be present in $\thr{P[Y]_\cons}$ and hence also in
%%$\thr{P[Z]_\cons}$. 
%%
%%In the former case, if this guard is also in $\thr{Z_{\rho(\weak)}}$ it will
%%be present in $\thr{P[Z]_\cons}$. If it does not belong to
%%$\thr{Z_{\rho(\weak)}}$, this means that it (its sort) is in the image of
%%$\rho$ (and therefore was dropped). But then, its $\rho$ pre-image must be in
%%$\thr{Y_\cons}$, that is, must be present in $\thr{P[Y]_\cons}$. But then it
%%is also present in $\thr{P[Z]_\cons}$ as the result of pushout construction.
%%\end{PROOF}
%
The fact that stepwise instantiation (of $\thr X_\cons$ by $\thr Y_\cons$ and then by
$\thr Z_\cons$ leading to $\thr{P[Z]_\cons}$) yields a different result than the direct instantiation
(of $\thr X_\cons$ by $\thr{Z_\cons}$ leading to $\thr{P[Z]'_\cons}$) may look like a severe weakness of our
setting. After all, equality of these two indicates the desirable
compositionality which would be expected by anybody familiar with the
traditional, pushout based theory of parameterized specifications. 

However,
we are not developing a theory of parameterized specifications but of
specification of parameterized data types. This means, we are interested in
constructions allowing us to obtain new data types (algebras) from others. In
this setting, performing different series of constructions or, as in the case
of vertical composition, performing constructions in different ways, may be
expected to yield different results. 

Our point is that stepwise instantaition, first along $\nu$ and then along
$\rho$ represents a slightly different construction than the direct instantiation
along $\eta=\nu;\rho$. In fact, we sugest to think of the former as a refinement
of the latter. The latter is a one step construction along $\eta$. In this
sense, splitting this construction in two steps, first along $\nu$ and then
$\rho$, is a more detailed, refined construction which may introduce new
aspects. We certainly want the result of this refined construction to be
``compatible'' with the results prescribed by the more rough, one step
construction. This is the meaning of one construction refining another which
corresponds to the classical concept of refinement by model class
inclusion. This is indicated by the fact~\ref{fa:pzpz}.1 and we now
proceed to illustrate the semantic aspect of this refinement.

\subsection{Vertical composition -- semantics}\label{sub:vcsem}
%
As noted in section~\ref{sub:appsem}, we can view the semantics of
instantiation from two angles: on the one hand, as a new PDT with a class of
its semantic functors and, on the other hand, as an actualisation: a functor
for the resulting PDT induced by a
particular functor for the instantiated PDT. We now apply this distinction in
the discussion of the semantics of vertical composition.

\subsubsection{Vertical composition as a refinement of PDT}\label{sub:vertref}
We postpone the general definition of refinement to section~\ref{sub:ref} and for
the moment take it intuitively to mean:
a PDT  $\thr P=\pdt$ is a {\em
refinement} of a PDT  $\thr
P'=\pdtsp{\mu'}{\delta'}{\thr{X'_\cons}}{\thr{P[X]'_\cons}}$, $\thr
P'\leadsto \thr P$, if any semantic 
functor for $\thr P$ can be used for obtaining 
a semantic functor for $\thr P'$.

A trivial, though by no means only, example of such a refinement is when
$\thr{P[X]'_\cons}\leadsto\thr{P[X]_\cons}$, i.e.,
$\Mod(\thr{P[X]'_\cons})\supseteq \Mod(\thr{P[X]_\cons})$, while other
components are equal. This is, in fact, the case with the results of vertical
composition. If we view $\thr{P[Z]_\cons}$ and $\thr{P[Z]'_\cons}$ as two
independent PDTs (i.e., ``forget'' that they both originate from instantiation
of the same PDT), we see that, 
by fact~\ref{fa:pzpz}, $\thr{P[Z]_\cons}\models
\thr{P[Z]'_\cons}$, i.e., we have an inclusion (functor)
$\fu i:\Mod(\thr{P[Z]_\cons})\subseteq\Mod(\thr{P[Z]'_\cons})$. 
Thus any semantic functor $\fu F$ for $\thr P= \pdtsp{\mu''}{\delta''}{\thr
Z_\cons}{\thr{P[Z]_\cons}}$ 
gives a semantic functor for $\thr P'= \pdtsp{\mu'''}{\delta'''}{\thr
Z_\cons}{\thr{P[Z]'_\cons}}$, simply 
by composing $\comp{\fu F}{\fu i}$.
The other components of both PDTs are (essentially) the same, and so we get
%
\begin{fact}
Given $\thr P= \pdtsp{\mu''}{\delta''}{\thr
Z_\cons}{\thr{P[Z]_\cons}}$ and  $\thr P'= \pdtsp{\mu'''}{\delta'''}{\thr
Z_\cons}{\thr{P[Z]'_\cons}}$ (as in Figure~\ref{fi:vertcomp}), $\thr P'\leadsto \thr P$.
\end{fact}
%
Refinement amounts in this case to the situation illustrated in
example~\ref{ex:difpush}, namely, that while $\thr P'$ may allow extension of
some carriers (corresponding to $\cons'$ in the example), $\thr P$ may forbid
it by introducing additional global guards. Thus, in general, all semantic
functors for $\thr P$ are also semantic functors for $\thr P'$, but there may
be some functors for $\thr P'$ which are not valid semantic functors for
$\thr P$.

\subsubsection{Vertical composition as an actualisation of a particular semantic
functor}
There is, however, a more specific relation between the stepwise
instantiation and the direct one. According to
proposition~\ref{prop:inducedapp}, any semantic functor $\fu F_X$ for $\pdt$
induces a semantic functor $\fu F_Y$ for any instantiation of formal
parameter $\thr X_\cons$ by an actual parameter $\thr Y_\cons$. 
If we now consider the results of respective actualisations, i.e., functors
$\fu F_Z$ (obtained by stepwise actualisation through $\thr Y_\cons$ first
along $\nu$ and then $\rho$) and
$\fu F'_Z$ (obtained by direct actualisation along $(\nu;\rho)$) which are
both induced starting from the same, given $\fu F_X$, then it turns out that
the semantics is fully compositional, i.e., both functors are equal.

We discuss it in more detail.
The semantic counterpart of the diagram from Figure~\ref{fi:vertcomp} is
shown below.
\[
\hspace*{-3em}\xymatrix@C=0.8cm@R=1cm{
&&&& \Mod(\thr X_\cons) \ar[dr]^{\fu F_X} \ar[d]^{\iota_X} & \\
&&& \Mod(\thr Y_\cons)  \ar[dr]^>>>{\fu F_Y} \ar[d]^{\iota_{Y1}}
\ar[ur]^{|_{\nu}} \ar@{.}[dl]|{=} 
     & \Mod(\thr X_\weak) & \Mod(\thr{P[X]_\cons}) \ar[l]_{|_{\mu}} \\
&& \Mod(\thr Y_\cons) \ar[dr]^>>>{\fu F'_Y} \ar[d]^{\iota_Y} 
     & \Mod(\thr Y_{\nu(\weak)}) \ar[ur]^>>>>>{|_{\nu}} \ar@{^{(}.>}[dl]
     & \Mod(\thr{P[Y]_\cons}) \ar[l]^{|_{\mu'}} \ar[ur]^{|_{\nu'}}
     \ar@{.}[dl]|{=} \\
& \Mod(\thr Z_\cons) \ar[ur]^{|_\rho} \ar[d]^{\iota_Z} \ar[dr]^>>>{\fu F_Z} & \Mod(\thr Y_\weak) &
       \Mod(\thr{P[Y]_\cons}) \ar[l]_{|_{\mu'}} \\
\Mod(\thr Z_\cons) \ar[d]_{\iota_{Z1}} \ar[dr]^>>>{\fu F'_Z} \ar@{.}[ur]|{=}\ar@/^3pc/[uuuurrrr]^{|_{(\nu;\rho)}}
     & \Mod(\thr Z_{\rho(\weak)}) \ar[ur]^>>>>>{|_{\rho}} &
       \Mod(\thr{P[Z]_\cons}) \ar[l]^{|_{\mu''}} \ar[ur]_{|_{\rho'}}
       \ar@{^{(}.>}[dl]^{\fu i} \\
\Mod(\thr Z_{(\nu;\rho)(\weak)}) \ar@{^{(}.>}[ur]
    & \Mod(\thr{P[Z]'_\cons}) \ar[l]^{|_{\mu'''}}   \ar@/_3pc/[uuuurrrr]_{|_{(\nu;\rho)'}}
}
\]
Given a semantic functor $\fu F_X$ (in the uppermost diagram),
proposition~\ref{prop:inducedapp} allows us to construct a functor $\fu F_Y$,
and similarly, an $\fu F_Z$ can be constructed given an arbitrary $\fu F'_Y$. Thus, using
$\fu F_Y$ obtained from the actualisation along $\nu$ for $\fu F'_Y$, we can
construct an $\fu F_Z$ from a given $\fu F_X$. Notice that the associated
$\iota_Z$ guarantees the image of $\Mod(\thr Z_\cons)$ to be
included in $\Mod(\thr Z_{\rho(\weak)})$. 

For the direct actualisation, we can obtain $\fu F'_Z$ from a given $\fu
F_X$ by proposition~\ref{prop:inducedapp}. On the other hand,
by fact~\ref{fa:pzpz}, we also have the inclusion (functor) 
$\fu i:\Mod(\thr{P[Z]_\cons})\subseteq\Mod(\thr{P[Z]'_\cons})$.  
Hence, composing we obtain $\comp{\fu
F_Z}{\fu i}:\Mod(\thr{Z_\cons})\to\Mod(\thr{P[Z]'_\cons})$, which gives a
possible semantic functor $\fu F'_Z$ for the PDT
$\thr P'=\pdtsp{\mu'''}{\delta'''}{\thr{Z_\cons}}{\thr{P[Z]'_\cons}}$. 
Compositionality of actualisation is expressed in the following proposition.
%
\begin{proposition}
With the notation from the diagram above, where all functors are induced by
$\fu F_X$ (in particular, $\fu F_Y=\fu F'_Y$ and $\iota_Y=\iota_{Y1}$), we have:
$\fu F_Z;\fu i = \fu F'_Z$.
\end{proposition}
%
\subsection{Horisontal composition}\label{sec:semfunccomp}\label{sub:hcsyn}
Horisontal composition of PDTs is defined in the standard and expected way.
\begin{definition}\label{def:horcomp}
For parameterized data type specifications
$\pdtsp{\mu}{\delta}{\thr{X_\cons}}{\thr{P[X]_\cons}}$ and
$\pdtsp{\mu'}{\delta'}{\thr{P[X]_\cons}}{\thr{W[P[X]]_\cons}}$,
we define their horisontal composition to be $\pdtsp{(\mu;\mu')}{(\delta;\delta')}{\thr
X_\cons}{\thr{W[P[X]]_\cons}}$. 
\end{definition}
%
Composing horisontally PDTs yields a new PDT, as expressed in the following proposition.
\begin{proposition}\label{prop:horcomp}
The composition as defined in~\ref{def:horcomp} is (isomorphic to) a
PDT. (I.e., there exists a $\thr{W[P[X]]'_\cons}\iso\thr{W[P[X]]_\cons}$
such that $\pdtsp{(\mu;\mu')}{(\delta;\delta')}{\thr
X_\cons}{\thr{W[P[X]]'_\cons}}$ is a PDT).
\end{proposition}
The first 4 points of definition~\ref{def:parametersyntax} are trivially
satisfied and verification of point~\ref{it:rel} is based, primarily, on the 
presence of axioms of the form~\ref{it:sat}.

On the semantic side,
the following proposition states that, given semantic functors for the (horisontally composed)
component PDTs, their composition yields a semantic functor for the composed
PDT. 
%\newpage
\begin{proposition}
Given PDTs $\pdt$ and
$\pdtsp{\mu'}{\delta'}{\thr{P[X]_\cons}}{\thr{W[P[X]]_\cons}}$, 
with the semantic functors
$\fu{F_\thr{X}}: \Mod(\thr{X_\cons}) \to 
\Mod(\thr{P[X]_\cons})$ and $\fu{F_\thr{P[X]}}: \Mod(\thr{P[X]_\cons}) \to
\Mod(\thr{W[P[X]]_\cons})$. The composition
$\comp{\fu{F_\thr{X}}}{\fu{F_\thr{P[X]}}}:\thr{X_\cons} \to
\thr{W[P[X]]_\cons}$, is a semantic functor for
$\pdtsp{(\mu;\mu')}{(\delta;\delta')}{\thr{X_\cons}}{\thr{W[P[X]]_\cons}}$. 
\end{proposition}
The proposition means that all the loops in the following diagram commute:
\[
\xymatrix@C=0.4cm@R=0.5cm{
&& \Mod(\thr{X_\cons}) \ar[dd]^{\iota_{\thr{X}}}
\ar@(l,l)[dddd]_{\iota'} \ar[rrdd]^{\fu{F_\thr{X}}}
\ar@(u,r)[rrrrdddd]^{\comp{\fu{F_\thr{X}}}{\fu{F_\thr{P[X]}}}}\\ 
\\
&& \Mod(\thr{X_{\weak}}) \ar@{.>}[dd]^{\iota_{\thr{P[X]}}|_\mu}
&& \Mod(\thr{P[X]_\cons}) \ar[ll]_{|_{\mu}} \ar[dd]^{\iota_{\thr{P[X]}}} \ar[rrdd]^{\fu{F_\thr{P[X]}}}\\
\\
&&\Mod(\thr{X_{\weak}})
&& \Mod(\thr{P[X]_{\weak}}) \ar@{.>}[ll]_{|_{\mu}}
&& \Mod(\thr{W[P[X]]_\cons}) \ar[ll]_{|_{\mu'}} \ar@(d,d)[llll]_{|_{\comp{\mu}{\mu'}}}\\
\\
}
\]
In the proof one verifies that the composition
$\iota'=\iota_{\thr{X}};(\iota_{\thr{P[X]}}|_\mu)$ provides the required tight
monomorphism for each algebra $A\in\Mod(\thr{X_\cons})$.

As was the case with the vertical composition, horisontal composition of PDTs
gives us a more structured specification. 
According to
proposition~\ref{prop:horcomp}, composing horisontally two PDTs, we obtain a
new PDT with the associated class of semantic functors. 
However, the semantics of a PDT obtained by a stepwise, horisontal
composition of PDTs $\thr P=\pdt$ and $\thr
P'=\pdtsp{\mu'}{\delta'}{\thr{P[X]_\cons}}{\thr{W[P[X]]_\cons}}$, 
which can be written as $\thr P;\thr P'$, 
%$\pdt;\pdtsp{\mu'}{\delta'}{\thr{P[X]_\cons}}{\thr{W[P[X]]_\cons}}$, 
is a
refinement of the semantics of the respective composed PDT 
$\pdtsp{(\mu;\mu')}{(\delta;\delta')}{\thr{X_\cons}}{\thr{W[P[X]]_\cons}}$ --
the former, possessing more structure in the form of the intermediary stage
$\thr{P[X]_\cons}$, may put additional restrictions on the admissible
functors.

The following example illustrates this fact -- that horisontal composition, introducing an
 intermediary parameter, can actually be a strict refinement of the composed
 PDT, i.e., that some functors admissible as a semantics for the composed PDT
 may no longer be obtained as a composition of the semantic functors for the
 component PDTs.

\begin{example}\label{ex:abstrref}
The following PDT $\thr P=\pdt$ requires extension of the parameter algebra $A$ with a new
function $f$ and allows extending $A$'s carrier with
new elements (one of which may be $d$). 
\[\xymatrix{
\frame{\txt{$\spec{\tit{\mbox{\bf \spec\ X}_\cons=} \\
		\spSort{El}
		\spOps{\allcons}{ \cons: \to El}
		\Gamma:
			& x \prec \cons
}$}}
\ar[rrr]^<<<<<<<<<<<<{\txt{$\large{\mu(\cons)= ok}$}}_<<<<<<<<<<<<{\txt{$\large{\delta(\cons)={\cons}}$}} 
&&& \frame{\txt{$\spec{\tit{\mbox{\bf P[X]}_\cons=} \\
	\spSorts{\Sorts'}{El}
	\spOps{\Ops'}{
		d:	&& El \\
		f: & El & El }
	\spOps{{\allcons}'}{	\cons, ok: && El}
	\Phi':
		&1.& && f(d) \eleq d\\
		&2.&x\prec \cons &\To& f(x) \eleq f(x)\\
		&3.& && ok\prec\cons \\
	\Gamma':&4.& && x \prec \cons\\
}$}}
}
\]
Let the semantic functor $\fu{F}: \Mod(\thr{X_\cons}) \to
	\Mod(\thr{P[X]_\cons})$ send an $A\in\Mod(\thr{X_\cons})$ to $\fu
	F(A)$ given by:
\begin{itemize}\MyLPar
\item $|\fu{F}(A)| = |A| \uplus d$, i.e. $d$ is a new element added to the carrier of $A$,
\item $f^{\fu{F}(A)}(x)= d$, for all $x\in|\fu F(A)|$,
\item $ok^{\fu{F}(A)} = |A|$, by the semantic functor requirement,
\item $\cons^{\fu{F}(A)} = |\fu{F}(A)|$, by default.
\end{itemize}
Let's now introduce $\thr{W[X]_\cons}$ as an intermediary parameter, i.e., we
now have two PDTs $\thr
P'=\pdtsp{\mu'}{\delta'}{\thr{X_\cons}}{\thr{W[X]_\cons}}$ and
$\thr P''=\pdtsp{\mu''}{\delta''}{\thr{W[X]_\cons}}{\thr{P[W[X]]_\cons}}$:
\[ \xymatrix{
\thr{X_\cons}
\ar[rr]^<<<<<<<<<<{\txt{$\large{\mu'(\cons)=\cons}$}}_<<<<<<<<<<{\txt{$\large{\delta'(\cons)=\cons}$}}
&& \frame{\txt{$\spec{
\tit{\thr{W[X]_\cons}=}\\
	\spSorts{\Sorts''}{El}
	\spOps{\Ops''}{f: &El & El}
	\spOps{{\allcons}''}{\cons: && El}
	\Phi'':
		 &2.& x\prec\cons&\To& f(x) \eleq f(x)\\
	\Gamma'':&4.& && x \prec \cons_{EL}\\
}$}}
\ar[rr]^<<<<<<<<<<{\txt{$\large{\mu''(\cons)=ok}$}}_<<<<<<<<<<{\txt{$\large{\delta''(\cons)=\cons}$}}  
&& \thr{P[X]_\cons}=\thr{P[W[X]]_\cons}
}
\]
Obviously, we have that $\thr P=\pdtsp{(\mu';\mu'')}{(\delta';\delta'')}{\thr
X_\cons}{\thr{P[W[X]]_\cons}}$. But the refinement $\thr P\leadsto \thr
P';\thr P''$ is strict -- e.g., the functor $\fu F$ for the former cannot be obtained
by composing any two functors for the latter two.

For any semantic functor $\fu{F'}: \Mod(\thr{X_\cons}) \to\Mod(\thr{W[X]_\cons})$ 
can't extend the carrier of any $A\in \Mod(\thr{X_\cons})$, but 
merely adds a deterministic function $f$. Furthermore, any semantic functor 
$\fu{F''}: \Mod(\thr{W[X]_\cons}) \to\Mod(\thr{P[W[X]]_\cons})$ may add a
new element $d$ to the carrier of a parameter algebra $B\in\Mod(\thr{W[X]_\cons})$ and force
$f(d)\eleq d$. However, $\fu{F''}$ has to ``preserve'' the parameter algebra
$B$, i.e., $B$ must be a tight subalgebra of
$\fu{F''(B)}$. This means that function $f^{\fu F''(B)}$ applied to the elements
from the carrier of $B$ (i.e., from $ok^{\fu F''(B)}$) has to return elements
from the same carrier (as it did in $B$). If $d$ is a new element (as was the
case for $\fu F$), it will never be ``reachable by $f$'' from these old
elements. This illustrates the impossibility
of obtaining the original functor $\fu F$ as a composition of any $\fu F'$
and $\fu F''$.
\end{example}
%
\section{Refinement}\label{sub:ref}
We now summarise the concept of refinement of PDTs. As we have emphasised, it
amounts not only to the simple model class inclusion but, primarily, to
introduction of additional {\em structure} on the PDTs. The following
definition captures the general concept

\begin{definition}\label{def:ref}
A PDT $\thr
P'=\pdtsp{\mu'}{\delta'}{\thr{X'_\cons}}{\thr{P[X]'_\cons}}$
refines a PDT $\thr P=\pdt$, 
$\thr P\leadsto\thr P'$, if there exist functors
$\reff X:\Mod(\thr  X_\cons)\to\Mod(\thr X'_\cons)$ and
$\reff{P[X]}:\Mod(\thr {P[X]'_\cons})\to\Mod(\thr{P[X]_\cons})$, such that
for any semantic functor $\fu F'$ for $\thr P'$, the functor $\reff X;\fu
F';\reff{P[X]}$ is a semantic funtor for $\thr P$.
\end{definition}
The following diagram illustrates the requirement:
\[ \xymatrix@R=0.5cm@C=0.4cm{
&& \Mod(\thr{X_\cons})  \ar[dd]_{\iota} \ar@{.>}[rrdd]^{\reff{X};\fu{F}';\reff{P[X]}} \ar[lldd]_{\reff{X}}\\
&& && \\
\Mod(\thr{X'_\cons}) \ar[dd]_{\iota'} \ar[rrrdd]^{\fu{F'}}
	&& \Mod(\thr{X_{\weak}}) \ar@{}[ur]|{1.}
		&& \Mod(\thr{P[X]_\cons}) \ar[ll]_{|_{\mu}} \\
&& && \\
\Mod(\thr{X'_{\weak}}) \ar@{}[ur]|{2.}
	&&& \Mod(\thr{P[X]'_\cons}) \ar[lll]^{|_{\mu'}} \ar[uur]_{\reff{P[X]}}
}
\]
The relation is trivially transitive, i.e., $\thr P\leadsto\thr
P'\leadsto\thr P'' \To \thr P\leadsto \thr P''$.

The contravariance of $\reff{X}$ on the parameter side may seem a bit unusual since,
following the idea of refinement as the model class inclusion, one might
expect the refinement relation to hold also when
$\thr{X_\cons}\leadsto\thr{X'_\cons}$, i.e., $\Mod(\thr
X_\cons)\supseteq\Mod(\thr X'_\cons)$. However, one should keep in mind that
we are talking about design specifications of actual structure of data
types/programs. A PDT with source $\thr{X'_\cons}$ could not, in general,
replace a PDT with the source $\thr{X_\cons}$ if the refinement
$\thr{X_\cons}\leadsto\thr{X'_\cons}$ were strict.

A simple example is when $\thr{P[X]_\cons}\leadsto\thr{P[X]'_\cons}$, i.e., when
$\reff{X}$ is identity and
$\reff{P[X]}$ is a model class inclusion $\Mod(\thr{P[X]'_\cons})\subseteq
\Mod(\thr{P[X]_\cons})$ (as was the case of vertical composition in section~\ref{sub:vertref}). 
Another example was \ref{ex:abstrref} %and \ref{ex:stackref},
% insection~\ref{sub:hcsem}, 
where both $\reff{X}$ and $\reff{P[X]}$ were
identities but where intermediary parameter forced additional requirements
which did not (necessarily) follow from the original, refined PDT.

Finally, we give an example showing an even more particular case of
refinement by adding structure, where the formal parameter of a PDT is itself
refined to a PDT.

\begin{example}
Let $\thr
P=\pdtsp{\mu}{\delta}{\thr{Set}_\cons}{\thr{\choice[Set]_\cons}}$ be the
following PDT which extends a specification of sets with a nondeterministic
choice.
\[\xymatrix{
\frame{\txt{$\spec{
\tit{\mbox{\bf spec\ Set}_\cons^{{}}=} \\
	\spSort{Set,El}
	\spOps{\Ops}{
	\emptyset: && Set \\
	\prep\ : & El\times Set & Set 
	}
	\spOps{\allcons}{ \cons_{El}: \to El \\ \cons_{Set}:\to Set}
	\Phi: & 1.& x\prep (y\prep S) & \eleq & y\prep (x\prep S) \\
	      & 2.& x\prep (x\prep S) & \eleq & x\prep S \\
	\Gamma:
	   &3. & x &\prec &\cons_{EL} \\ 
	   &4. & S &\prec &\cons_{Set}
}$}}
\ar@(u,r)[dr]^{\txt{$\large{\hspace*{1em}\mu(\cons_{El})=\delta(\cons_{El})=ok_{El}}$}}_{\txt{$\large{\mu(\cons_{Set})=\delta(\cons_{Set})=
	\cons_{Set}\ \ }$}}
\\ &
\hspace*{-4em}\frame{\txt{$\spec{
\tit{\thr{\choice[Set]}_\cons=} \\
	\spSort{Set,El}
	\spOps{\Ops}{
	\emptyset: && Set \\
	\prep\ : & El\times Set & Set \\
	\choice : & Set & El
	}
	\spOps{\allcons}{ ok,\cons_{El}: \to El \\ \cons_{Set}:\to Set}
	\Phi: & 1.& x\prec ok,y\prec ok &\To & x\prep (y\prep S)  \eleq  y\prep (x\prep S) \\
	      & 2.& x\prec ok & \To & x\prep (x\prep S)  \eleq  x\prep S \\
	      & 3.& \multicolumn{3}{l}{x\prec ok, z\prec ok, z\prec \choice(x\prep S) }\\
	            && & \To & z\eleq x, z\prec \choice(S)\\
	\Gamma:
   	    &4. &&& x \prec \cons_{EL} \\ 
	    &5. &&& S\prec \cons_{Set}
}$}}
}
\]
We admit here {\bf extending carrier} $El$, $\mu(\cons_{El})=ok$, which is
motivated by the possible need of a new, ``error'' element to be returned by
 $\choice(\emptyset)$.

A possible semantic functor $\fu F$ may send a $\thr{Set}_\cons$ algebra $A$ on the
algebra $\fu F(A)$ where, for any nonempty set $S$, the operation 
$\choice^{\fu F(A)}(S)$ returns all the elements of the set
$S$. $\choice^{\fu F(A)}(\emptyset)$ may return a new, ``error'' element $\bot$,
added to the carrier of $A$. Adding this element to a set, $\bot\prep^{\fu F(A)}S$ may then result in
the emptyset $\emptyset^{\fu F(A)}$.

Obviously, the specification $\thr{Set_\cons}$ can be naturally parameterized
by elements, i.e., we ``extract'' from it a parameter (sub)specification. We
obtain then $\thr P'=\pdtsp{\mu'}{\delta'}{\thr{El}_\cons}{\thr{Set[El]_\cons}}$,
where $\thr{El_\cons}$ contains merely the sort $El$ and the global guard
$x\prec \cons_{El}$, while $\thr{Set[El]_\cons}$ is exactly the same as
$\thr{Set_\cons}$. $\mu'$ and $\delta'$ are identities on $\cons_{El}$. 

The point is now that the composed PDT $\thr P';\thr P$ is a refinement of
$\thr P$ according to definition~\ref{def:ref}. 
\[\xymatrix@C=0.6cm@R=1cm{
\Mod(\thr{El_\cons}) \ar[drr]^{\fu F'} \ar@{.>}[d] && &&
  \Mod(\thr{Set_\cons})  \ar@{.>}[d]
    \ar[llll]_{|_{\mu'}\ =\ \reff{Set}}  \ar@{.>}[drr]^{\fu G} \ar@{.}[dll]|{id} \\
\Mod(\thr{El_\weak}) && \Mod(\thr{Set[El]_\cons})  \ar@{.>}[d] \ar[drr]^{\fu F} \ar[ll]_{|_{\mu'}}
   && \Mod(\thr{Set_\weak}) && \Mod(\thr{\choice[Set]_\cons})
   \ar@{.}[dll]|{id\ =\ \reff{\choice[Set]}} \ar[ll]_{|_{\mu}}\\ 
&& \Mod(\thr{Set[El]_\weak}) && \Mod(\thr{\choice[Set[El]]_\cons}) \ar[ll]_{|_{\mu}}
}
\]
Since $\mu'(\cons_{El})=\cons_{El}$ the  reduct $|_{\mu'}$ will
return an $\thr{El_\cons}$-algebra (and not only an $\thr{El_\weak}$-algebra).

Now, given $\fu F'$ and $\fu F$, the functor $\fu G$ can be trivially chosen
to be $\fu F$, if we let $\reff{Set}$ be identity. More generally, if $\fu F'$ were such that $|_{\mu'};\fu
F'=\fu{id}_{\thr{Set_{\cons}}}$, we will have that $\fu G=\reff{Set};\fu
F';\fu F;\reff{\choice[Set]}=\fu F$. 

However, even in such cases, the refinement consists in requiring a more
structured data type, which consists of building first an algebra of sets
over a given algebra of elements, and then an algebra with choice (i.e., 
composing two functors $\fu F';\fu F$). In
this sense, it is reasonable to call $\thr P';\thr P$ a refinement of $\thr
P$.

 Moreover, the functor $\fu F'$ will not, in general, be surjective on the
 objects, i.e., it may choose only a subclass of all $\thr{Set[El]_\cons}$
 algebras. In this case, the application of the composition to all models of
 $\thr{El_\cons}$, $\fu F';\fu F(\Mod(\thr{El_\cons}))$ may result in fewer
 $\thr{\choice[Set[El]]_\cons}$ algebras than $\fu F(\Mod(\thr{Set[El]_\cons}))$,
 which is another reason for viewing this composition as a refinement of the
 original PDT.
\end{example}
%
There is yet another possibility of viewing the above as an example of
refinement. Suppose that we have implemented the PDT
$\pdtsp{\mu}{\delta}{\thr{Set_\cons}}{\thr{\choice[Set]_\cons}}$, i.e., we
have a functor $\fu G$.  Then, having implemented also a functor $\fu F'$, we
can compose it with $\fu G$. Since $\fu F'$ will not, typically, be
surjective on the objects, this composition will, in general, yield a smaller
subclass of $\Mod(\thr{\choice[Set]_\cons})$ than the image of $\fu G$.

In any case, we can view the above process as a gradual development of a
design for the flat specification $\thr{\choice[Set]_\cons}$. In the first
step, we extract from it the parameter $\thr{Set_\cons}$ which prescribes a
more specific, structured implementation. (This was the PDT {\thr{P}} with
which we started the above example.) In the second step, we again extract
the parameter $\thr{El_\cons}$, requiring even more structure. Viewed in this
way, our setting gives a concrete specialisation of the general concept of
``constructor implementations'' from \cite{para1}. A more detailed study of
the methodological possibilities offered by our PDTs is left for a future work.
\section{Composition and refinement}\label{se:compref}
%
We will now review the classical vertical and horizontal
composition, showing the counterparts of the standard compositionality
theorems. The main difference will concern the fact that, in general,
stepwise application of constructions will not yield the same result as a direct
construction along the respective composition, but a refinement of the latter.
Subsections~\ref{sub:vcsyn} and \ref{sub:vcsem} discuss vertical, and
\ref{sub:hcsyn} %and \ref{sub:hcsem} 
horizontal composition. Section~\ref{sub:ref} will summarize the concept of refinement
which emerges from this section.

We recall that, given a parameter passing diagram (like 1. below in
Figure~\ref{fi:vertcomp}), by proposition~\ref{prop:paramtoo},
$\mu':\thr{Y_{\nu(\weak)}}\to \thr{P[Y]_\cons}$ is a parameterization
morphism, and hence, in particular (by fact~\ref{fa:weakparamo}), $\mu':\thr
Y_\weak\to\thr{P[Y]_\cons}$ is a specification morphism.


\subsection{Vertical composition}\label{sub:vcsyn}
Given two actual parameter passing morphisms $\nu:\thr X_\weak\to \thr
Y_{\nu(\weak)}$ and $\rho:\thr Y_\weak\to \thr Z_{\rho(\weak)}$,
(as indicated in the diagrams 1. and 2. in Figure~\ref{fi:vertcomp}), 
we would like to compose them vertically, i.e.,
we want to show that also $(\nu;\rho):\thr X_\weak \to \thr Z_{(\nu;\rho)(-)}$ is
an actual parameter passing.\vspace*{-3ex}

\begin{figure}[hbt]
\[\xymatrix@R=0.4cm@C=1.5cm{
\thr X_\weak \ar[r]^\mu \ar[dd]_\nu   \ar@(l,l)[dddddd]_{\nu;\rho}  \ar@{}[ddr]|{1.}
  & \thr{P[X]_\cons} \ar[dd]^{\nu'} \ar@(r,r)[dddddd]^{(\nu;\rho)'} \\ \\ 
\thr Y_{\nu(\weak)} \ar[r]^{\mu'} & \thr{P[Y]_\cons} \\ 
%
\thr Y_\weak \ar[r]^{\mu'} \ar[dd]_\rho \ar@{}[ddr]|{2.}
  & \thr{P[Y]_\cons} \ar[dd]^{\rho'} \\ \\ 
\thr Z_{\rho(\weak)} \ar[r]^{\mu''} & \thr{P[Z]_\cons} \\ 
\thr Z_{(\nu;\rho)(\weak)} \ar[r]^{\mu'''} & \thr{P[Z]'_\cons} \\ 
}\vspace*{-2ex}
\]
\caption{}\label{fi:vertcomp}\vspace*{-1ex}
\end{figure}

\noindent
The notation from this figure will be used throughout this and next
subsection (\ref{sub:vcsyn}, \ref{sub:vcsem}).
 
The specifications $\thr{Z_{\rho(\weak)}}$ and
$\thr{Z_{(\nu;\rho)(\weak)}}$ need not be the same -- the latter may have
more global guards than the former. In general, we only have that 
$\thr{Z_{(\nu;\rho)(\weak)}}\models \thr{Z_{\rho(\weak)}}$.
%%\begin{fact}\label{fa:triv}
%%Given a PDT $\pdtsimple{\mu}{\delta}{\thr X_\cons}{\thr{P[X]_\cons}}$
%%and actual parameter passing $\nu$
%%and $\rho$ as in the Figure~\ref{fi:vertcomp}: 
%%\begin{enumerate}\MyLPar
%%\item $\thr{Z_{(\nu;\rho)(\weak)}}\models \thr{Z_{\rho(\weak)}}$.
%%\item If $\nu$ is surjective on the sorts, then $\thr{Z_{\rho(\weak)}}\models
%%\thr{Z_{(\nu;\rho)(\weak)}}$.
%%\item If $\nu$ is
%%surjective on the sorts, then $\thr{Z_{(\nu;\rho)(\weak)}} \iso \thr{Z_{\rho(\weak)}}$.
%%\end{enumerate}
%%\end{fact}
%%Notice that, in points 2. and 3., surjectivity of $\nu$ on sorts is sufficient but
%%not necessary condition. It is sufficient and necessary that for any sort $s\in\Sigma(\thr{Y})$
%%which is not in the image of $\nu$, there is a sort  $s'\in\Sigma(\thr{Y})$ which
%%is in the image of $\nu$ and such that $\rho(s)=\rho(s')$.
%%
In spite of the possible differences, the following proposition ensures at
least that
composition of actual parameter passing morphisms is still an actual parameter passing.
\begin{proposition}\label{prop:vertcomp}
If 
$\nu:\thr X_\weak\to \thr Y_{\nu(\weak)}$ and $\rho:\thr Y_\weak\to \thr Z_{\rho(\weak)}$
 are actual parameter passing
morphisms, then so is $(\nu;\rho):\thr X_\weak \to \thr Z_{(\nu;\rho)(\weak)}$
(see the diagram in Figure~\ref{fi:vertcomp}). 
\end{proposition}
%
Now, in general, also the resulting specifications $\thr{P[Z]_\cons}$ and
$\thr{P[Z]'_\cons}$ may be different. In the classical case, this is merely a
consequence of their definition by pushout (which is unique only up to
isomorphism). In our case, however, the difference may be more significant,
since we also may drop and/or add some global guards on the way. 
As in the case of $\thr{Z_{\rho(\weak)}}$ and $\thr{Z_{(\nu;\rho)(\weak)}}$, 
%fact~\ref{fa:triv}, 
the only difference may concern the presence/absence of global guards (since
all other axioms are involved in the pushout construction), so these are the only
axioms we mention in the following example.
\begin{example}\label{ex:difpush}
Consider first two instantiations $\nu:{\thr
X_\weak}\to\thr{Y_{\nu(\weak)}}$ and 
$\rho:{\thr Y_\weak}\to\thr{Z_{\rho(\weak)}}$. (Two lines in $\thr{Y_{\nu(\weak)}}$,
$\thr{Y_\weak}$, etc. represent two distinct sorts which are identified by
the second instantiation $\rho$.)
\[\xymatrix@R=0.4cm@C=0.7cm{
\thr{X_\weak} \ar[dd]^\nu & &\cons  \save[].[]*\frm{.}\restore  \ar[rr]^{\mu} \ar[dd] && 
    c \save[].[rrrr]*\frm{.}\restore \ar[dd] & & \cons \ar[dd] && x\prec
  \cons \ar@{.>}[dd] & \thr{P[X]_\cons} \ar[dd]_{\nu'}\\ 
&& && && && \\
\thr{Y}_{\nu(\weak)}& & \cons \ar[rr]^{\mu'} && c  && \cons && x \prec\cons
    \save[].[dllll]*\frm{.}\restore  & \thr{P[Y]_\cons}\\
 & y\prec \cons_1\ \ \ \cons_1 \save[].[ru]*\frm{.}\restore
    \ar[rrrr]^{\mu'} &&&& 
   \cons_1 &&  y\prec\cons_1 &  \\ 
%}\]
%\[\xymatrix@R=0.4cm@C=0.7cm{
%%% next inst
\thr{Y_\weak}\ar[ddd]^{\rho} & \cons_1 \save[].[dr]*\frm{.}\restore \ar[dddr]\ar[rrrr]_{\mu'}
&&&&  \cons_1 \ar[dddl] && y\prec \cons_1 \ar@{.>}[ddd] &  & \thr{P[Y]_\cons}
  \ar[ddd]_{\rho'}\\
 & & \cons \ar[dd]  \ar[rr]_{\mu'} && c \ar[dd] && \cons \ar[dd] 
  && x\prec\cons \save[].[ullll]*\frm{.}\restore  \ar@{.>}[dd]\\ 
& & && && && \\
 \thr{Z_{\rho(\weak)}} & & \cons' \save[].[]*\frm{.}\restore \ar[rr]^{\mu''} && 
  \cons' \save[].[rrrr]*\frm{.}\restore && \cons & y\prec\cons' & x \prec
 \cons & \thr{P[Z]_\cons}
}
\]
And now a direct instantiation along $(\comp{\nu}{\rho})$:
\[\xymatrix@R=0.2cm@C=0.9cm{
\thr{X_\weak} \ar[ddd]_{(\nu;\rho)} & \cons \save[].[]*\frm{.}\restore \ar[ddd] \ar[rr]^\mu && 
  c \save[].[rrrr]*\frm{.}\restore \ar[ddd] 
   && \cons \ar[ddd] && x\prec\cons \ar@{.>}[ddd] & \thr{P[X]_\cons} \ar[ddd]_{(\nu;\rho)'}\\
 & && &&&&& \\  & && &&&&& \\
\thr{Z_{(\nu;\rho)(\weak)}} & \cons' \save[].[]*\frm{.}\restore \ar[rr]^{\mu'''} && 
  \cons'  \save[].[rrrr]*\frm{.}\restore
   && \cons  && x\prec\cons  & \thr{P[Z]'_\cons}
}
\]
%Both $\thr{P[Z]_\cons}$ and $\thr{P[Z]'_\cons}$  have isomorphic
%signatures: in the former  $\allcons = \{\cons,\cons'\}$, while in the
%latter ${\allcons}'=\{\cons,c\}$. 
%
The significant difference consists in that  $\thr{P[Z]_\cons}$  has the
global guard for $\mu''(\cons)$, namely $y\prec\cons'$ originating from
$\thr{P[Y]_\cons}$. Thus here $\Cons=\{\cons,\cons'\}$ and $\subcons=\emptyset$.
In $\thr{P[Z]'_\cons}$, on the other hand, this guard is
not present, so here $\Cons'= \{\cons\}$, while ${\subcons}'=\{\cons'\}$.

Thus,$\pdtsimple{\mu''}{whoever}{\thr{Z_\cons}}{\thr{P[Z]_\cons}}$ forbids
extending the carrier of $\cons'$, while
$\pdtsimple{\mu'''}{whoever}{\thr{Z_\cons}}{\thr{P[Z]'_\cons}}$ does not.
\end{example}
%
So, in general, $\thr{P[Z]_\cons}$
and $\thr{P[Z]'_\cons}$ are not even isomorphic. We have the following fact.
\begin{fact}\label{fa:pzpz}
With the notation from Figure~\ref{fi:vertcomp} and
example~\ref{ex:difpush}:
\begin{enumerate}\MyLPar
\item $\thr{P[Z]_\cons}\models \thr{P[Z]'_\cons}$.
\item if $\thr{P[Z]'_\cons}\not\models \thr{P[Z]_\cons}$, then it is only
because for some sort constant(s) $c: \thr{P[Z]_\cons}\models x\prec c$ and
$\thr{P[Z]'_\cons}\not\models x\prec c$. 
\end{enumerate}
\end{fact}
%%\begin{PROOF}
%%The signatures of both specifications will be isomorphic, so we assume that they are
%%identical. All axioms except global guards are involved in the pushout
%%constructions, so their presence (or satisfaction) follows from the standard
%%isomorphism of pushout objects. The difference may concern only some constants which are in $\Cons$ but
%%not in $\Cons'$ (only in ${\subcons}'$, as in the
%%example~\ref{ex:difpush}). This justifies point 2.
%%For point 1. we show that if $\cons\in\Cons'$ then
%%$\cons\in\Cons$, that is if $\thr{P[Z]'_\cons}\models x\prec\cons$ then
%%$\thr{P[Z]_\cons}\models x\prec\cons$, which will yield the conclusion.
%%
%%This follows trivially. Any global guard $x\prec\cons$ in $\thr{P[Z]'_\cons}$
%%is an image of a respective global guard either from
%%$\thr{Z_{(\nu;\rho)(\weak)}}$ or from $\thr{P[X]_\cons}$. In the latter case,
%%it will also be present in $\thr{P[Y]_\cons}$ and hence also in
%%$\thr{P[Z]_\cons}$. 
%%
%%In the former case, if this guard is also in $\thr{Z_{\rho(\weak)}}$ it will
%%be present in $\thr{P[Z]_\cons}$. If it does not belong to
%%$\thr{Z_{\rho(\weak)}}$, this means that it (its sort) is in the image of
%%$\rho$ (and therefore was dropped). But then, its $\rho$ pre-image must be in
%%$\thr{Y_\cons}$, that is, must be present in $\thr{P[Y]_\cons}$. But then it
%%is also present in $\thr{P[Z]_\cons}$ as the result of pushout construction.
%%\end{PROOF}
%
This fact, that stepwise instantiation (of $\thr X_\cons$ along $\nu$ by
$\thr Y_\cons$ and then along $\rho$ by
$\thr Z_\cons$ leading to $\thr{P[Z]_\cons}$) yields a different result than the direct instantiation
(along $(\nu;\rho)$ leading to $\thr{P[Z]'_\cons}$) may look like a severe weakness of our
setting. After all, equality of these two indicates the desirable
compositionality which would be expected by anybody familiar with the
traditional, pushout based theory of parameterized specifications. 

However,
we are not developing a theory of parameterized specifications but of
specification of parameterized data types. Hence we are interested in
constructing new data types (algebras) from others. 
Performing different series of constructions or, as in the case
of vertical composition, performing constructions in different ways, may be
expected to yield different results. 

Our point is that stepwise instantiation, first along $\nu$ and then along
$\rho$, represents a slightly different construction than the direct instantiation
along $\eta=\nu;\rho$. In fact, we suggest to think of the former as a refinement
of the latter. The latter is a one step construction along $\eta$. In this
sense, splitting this construction in two steps, first along $\nu$ and then
$\rho$, is a more detailed, refined construction which may introduce new
aspects. We certainly want the result of this refined construction to be
``compatible'' with the results prescribed by the more rough, one step
construction. This is the meaning of one construction refining another which
corresponds to the classical concept of refinement by model class
inclusion. This is indicated by the fact~\ref{fa:pzpz}.1 and we now
proceed to illustrate the semantic aspect of this refinement.

\subsection{Vertical composition -- semantics}\label{sub:vcsem}
%
As noted in section~\ref{sub:appsem}, we can view the semantics of
instantiation from two angles: on the one hand, as a new PDT with a class of
its semantic functors and, on the other hand, as an actualization: a functor
for the resulting PDT induced by a
particular functor for the instantiated PDT. We now apply this distinction in
the discussion of the semantics of vertical composition.

\subsubsection{Vertical composition as a refinement of PDT}\label{sub:vertref}
We postpone the general definition of refinement to section~\ref{sub:ref} and for
the moment take it intuitively to mean:
a PDT  $\thr P=\pdt$ is a {\em
refinement} of a PDT  $\thr
P'=\pdtsp{\mu'}{\delta'}{\thr{X'_\cons}}{\thr{P[X]'_\cons}}$, $\thr
P'\leadsto \thr P$, if any semantic 
functor for $\thr P$ can be used for obtaining 
a semantic functor for $\thr P'$.

A trivial, though by no means only, example of such a refinement is when
$\thr{P[X]'_\cons}\leadsto\thr{P[X]_\cons}$, i.e.,
$\Mod(\thr{P[X]'_\cons})\supseteq \Mod(\thr{P[X]_\cons})$, while other
components are equal. This is, in fact, the case with the results of vertical
composition. If we view $\thr{P[Z]_\cons}$ and $\thr{P[Z]'_\cons}$ as two
independent PDTs (i.e., ``forget'' that they both originate from instantiation
of the same PDT), we see that, 
by fact~\ref{fa:pzpz}, $\thr{P[Z]_\cons}\models
\thr{P[Z]'_\cons}$, i.e., we have an inclusion (functor)
$\fu i:\Mod(\thr{P[Z]_\cons})\subseteq\Mod(\thr{P[Z]'_\cons})$. 
Thus any semantic functor $\fu F$ for $\thr P= \pdtsp{\mu''}{\delta''}{\thr
Z_\cons}{\thr{P[Z]_\cons}}$ 
gives a semantic functor for $\thr P'= \pdtsp{\mu'''}{\delta'''}{\thr
Z_\cons}{\thr{P[Z]'_\cons}}$, simply 
by composing $\comp{\fu F}{\fu i}$.
The other components of both PDTs are (essentially) the same, and so we get
%
\begin{fact}
Given $\thr P= \pdtsp{\mu''}{\delta''}{\thr
Z_\cons}{\thr{P[Z]_\cons}}$ and  $\thr P'= \pdtsp{\mu'''}{\delta'''}{\thr
Z_\cons}{\thr{P[Z]'_\cons}}$ (as in Figure~\ref{fi:vertcomp}), $\thr P'\leadsto \thr P$.
\end{fact}
%
This refinement amounts to the situation from
example~\ref{ex:difpush}, namely, that while $\thr P'$ may allow extension of
some carriers (corresponding to $\cons'$ in the example), $\thr P$ may forbid
it by introducing additional global guards. Thus, in general, all semantic
functors for $\thr P$ are also semantic functors for $\thr P'$, but there may
be some functors for $\thr P'$ which are not valid semantic functors for
$\thr P$.

\subsubsection{Vertical composition as an actualization of a particular semantic
functor}
There is, however, a more specific relation between the stepwise
instantiation and the direct one. According to
proposition~\ref{prop:inducedapp}, any semantic functor $\fu F_X$ for $\pdt$
induces a semantic functor $\fu F_Y$ for any instantiation of the formal
parameter $\thr X_\cons$ by an actual parameter $\thr Y_\cons$. 
If we now consider the results of respective actualizations, i.e., functors
$\fu F_Z$ (obtained by stepwise actualization through $\thr Y_\cons$ first
along $\nu$ and then $\rho$) and
$\fu F'_Z$ (obtained by direct actualization along $(\nu;\rho)$) which are
both induced starting from the same, given $\fu F_X$, then it turns out that
the semantics is fully compositional, i.e., both functors are equal.

In more detail, the semantic counterpart of the diagram from Figure~\ref{fi:vertcomp} is
shown below.
\[
\hspace*{-2em}\xymatrix@C=0.7cm@R=0.7cm{
&&&& \Mod(\thr X_\cons) \ar[dr]^{\fu F_X} \ar[d]^{\iota_X} & \\
&&& \Mod(\thr Y_\cons)  \ar[dr]^>>>{\fu F_Y} \ar[d]^{\iota_{Y1}}
\ar[ur]^{|_{\nu}} \ar@{.}[dl]|{=} 
     & \Mod(\thr X_\weak) & \Mod(\thr{P[X]_\cons}) \ar[l]_{|_{\mu}} \\
&& \Mod(\thr Y_\cons) \ar[dr]^>>>{\fu F'_Y} \ar[d]^{\iota_Y} 
     & \Mod(\thr Y_{\nu(\weak)}) \ar[ur]^>>>>>{|_{\nu}} \ar@{_{(}.>}[dl]
     & \Mod(\thr{P[Y]_\cons}) \ar[l]^{|_{\mu'}} \ar[ur]^{|_{\nu'}}
     \ar@{.}[dl]|{=} \\
& \Mod(\thr Z_\cons) \ar[ur]^{|_\rho} \ar[d]^{\iota_Z} \ar[dr]^>>>{\fu F_Z} & \Mod(\thr Y_\weak) &
       \Mod(\thr{P[Y]_\cons}) \ar[l]_{|_{\mu'}} \\
\Mod(\thr Z_\cons) \ar[d]_{\iota_{Z1}} \ar[dr]^>>>{\fu F'_Z} \ar@{.}[ur]|{=}\ar@/^3pc/[uuuurrrr]^{|_{(\nu;\rho)}}
     & \Mod(\thr Z_{\rho(\weak)}) \ar[ur]^>>>>>{|_{\rho}} &
       \Mod(\thr{P[Z]_\cons}) \ar[l]^{|_{\mu''}} \ar[ur]_{|_{\rho'}}
       \ar@{^{(}.>}[dl]^{\fu i} \\
\Mod(\thr Z_{(\nu;\rho)(\weak)}) \ar@{^{(}.>}[ur]
    & \Mod(\thr{P[Z]'_\cons}) \ar[l]^{|_{\mu'''}}   \ar@/_3pc/[uuuurrrr]_{|_{(\nu;\rho)'}}
}
\]
Given a semantic functor $\fu F_X$ (in the uppermost diagram),
proposition~\ref{prop:inducedapp} allows us to construct a functor $\fu F_Y$,
and similarly, an $\fu F_Z$ can be constructed given an arbitrary $\fu F'_Y$. Thus, using
$\fu F_Y$ obtained from the actualization along $\nu$ for $\fu F'_Y$, we can
construct an $\fu F_Z$ from a given $\fu F_X$. Notice that the associated
$\iota_Z$ guarantees the image of $\Mod(\thr Z_\cons)$ to be
included in $\Mod(\thr Z_{\rho(\weak)})$. 

For the direct actualization, we can obtain $\fu F'_Z$ from a given $\fu
F_X$ by proposition~\ref{prop:inducedapp}. On the other hand,
by fact~\ref{fa:pzpz}, we also have the inclusion (functor) 
$\fu i:\Mod(\thr{P[Z]_\cons})\subseteq\Mod(\thr{P[Z]'_\cons})$.  
Hence, composing we obtain $\comp{\fu
F_Z}{\fu i}:\Mod(\thr{Z_\cons})\to\Mod(\thr{P[Z]'_\cons})$, which gives a
possible semantic functor $\fu F'_Z$ for the PDT
$\thr P'=\pdtsp{\mu'''}{\delta'''}{\thr{Z_\cons}}{\thr{P[Z]'_\cons}}$. 
Compositionality of actualization is expressed in the following
proposition. 
%
\begin{proposition}
With the notation from the diagram above, where all functors are induced by
$\fu F_X$ (in particular, $\fu F_Y=\fu F'_Y$ and $\iota_Y=\iota_{Y1}$), we have:
$\fu F_Z;\fu i = \fu F'_Z$.
\end{proposition}
%
%\newpage
\subsection{Horizontal composition}\label{sec:semfunccomp}\label{sub:hcsyn}
Horizontal composition of PDTs is defined in the standard and expected way.
\begin{definition}\label{def:horcomp}
For PDTs %parameterized data type specifications
$\pdtsp{\mu}{\delta}{\thr{X_\cons}}{\thr{P[X]_\cons}}$ and
$\pdtsp{\mu'}{\delta'}{\thr{P[X]_\cons}}{\thr{W[P[X]]_\cons}}$,
we define their horizontal composition to be $\pdtsp{(\mu;\mu')}{(\delta;\delta')}{\thr
X_\cons}{\thr{W[P[X]]_\cons}}$. 
\end{definition}
%
Composing horizontally PDTs yields a new PDT, as expressed in the following proposition.
\begin{proposition}\label{prop:horcomp}
The composition as defined in~\ref{def:horcomp} is (isomorphic to) a
PDT. (I.e., there exists a $\thr{W[P[X]]'_\cons}\iso\thr{W[P[X]]_\cons}$
such that $\pdtsp{(\mu;\mu')}{(\delta;\delta')}{\thr
X_\cons}{\thr{W[P[X]]'_\cons}}$ is a PDT).
\end{proposition}
The first 4 points of definition~\ref{def:parametersyntax} are trivially
satisfied and verification of point~\ref{it:rel} is based, primarily, on the 
presence of axioms of the form~\ref{it:sat}.

On the semantic side,
the following proposition states that, given semantic functors for the (horizontally composed)
component PDTs, their composition yields a semantic functor for the composed
PDT. 
%\newpage
\begin{proposition}
Given PDTs $\pdt$ and
$\pdtsp{\mu'}{\delta'}{\thr{P[X]_\cons}}{\thr{W[P[X]]_\cons}}$, 
with the semantic functors
$\fu{F_\thr{X}}: \Mod(\thr{X_\cons}) \to 
\Mod(\thr{P[X]_\cons})$ and $\fu{F_\thr{P[X]}}: \Mod(\thr{P[X]_\cons}) \to
\Mod(\thr{W[P[X]]_\cons})$. The composition
$\comp{\fu{F_\thr{X}}}{\fu{F_\thr{P[X]}}}:\thr{X_\cons} \to
\thr{W[P[X]]_\cons}$, is a semantic functor for
$\pdtsp{(\mu;\mu')}{(\delta;\delta')}{\thr{X_\cons}}{\thr{W[P[X]]_\cons}}$. 
\end{proposition}
The proposition means that all the loops in the following diagram commute:
\[
\xymatrix@C=0.4cm@R=0.5cm{
&& \Mod(\thr{X_\cons}) \ar[dd]^{\iota_{\thr{X}}}
\ar@(l,l)[dddd]_{\iota'} \ar[rrdd]^{\fu{F_\thr{X}}}
\ar@(u,r)[rrrrdddd]^{\comp{\fu{F_\thr{X}}}{\fu{F_\thr{P[X]}}}}\\ 
\\
&& \Mod(\thr{X_{\weak}}) \ar@{.>}[dd]^{\iota_{\thr{P[X]}}|_\mu}
&& \Mod(\thr{P[X]_\cons}) \ar[ll]_{|_{\mu}} \ar[dd]^{\iota_{\thr{P[X]}}} \ar[rrdd]^{\fu{F_\thr{P[X]}}}\\
\\
&&\Mod(\thr{X_{\weak}})
&& \Mod(\thr{P[X]_{\weak}}) \ar@{.>}[ll]_{|_{\mu}}
&& \Mod(\thr{W[P[X]]_\cons}) \ar[ll]_{|_{\mu'}} \ar@(d,d)[llll]_{|_{\comp{\mu}{\mu'}}}\\
\\
}
\]
In the proof one verifies that the composition
$\iota'=\iota_{\thr{X}};(\iota_{\thr{P[X]}}|_\mu)$ provides the required tight
monomorphism for each algebra $A\in\Mod(\thr{X_\cons})$.

As was the case with the vertical composition, horizontal composition of PDTs
gives us a more structured specification. 
According to
proposition~\ref{prop:horcomp}, composing horizontally two PDTs, we obtain a
new PDT with the associated class of semantic functors. 
However, the semantics of a PDT obtained by a stepwise, horizontal
composition of PDTs $\thr P=\pdt$ and $\thr
P'=\pdtsp{\mu'}{\delta'}{\thr{P[X]_\cons}}{\thr{W[P[X]]_\cons}}$, 
which can be written as $\thr P;\thr P'$, 
%$\pdt;\pdtsp{\mu'}{\delta'}{\thr{P[X]_\cons}}{\thr{W[P[X]]_\cons}}$, 
is a
refinement of the semantics of the respective composed PDT 
$\pdtsp{(\mu;\mu')}{(\delta;\delta')}{\thr{X_\cons}}{\thr{W[P[X]]_\cons}}$ --
the former, possessing more structure in the form of the intermediary stage
$\thr{P[X]_\cons}$, may put additional restrictions on the admissible
functors.

The following example illustrates this fact -- that horizontal composition, introducing an
 intermediary parameter, can actually be a strict refinement of the composed
 PDT, i.e., that some functors admissible as a semantics for the composed PDT
 may no longer be obtained as a composition of the semantic functors for the
 component PDTs.

\begin{example}\label{ex:abstrref}
The following PDT $\thr P=\pdt$ requires extension of the parameter algebra $A$ with a new
function $f$ and allows extending $A$'s carrier with
new elements (one of which may be $d$). 
\[\xymatrix{
\frame{\txt{$\spec{\tit{\mbox{\bf \spec\ X}_\cons=} \\
		\spSort{El}
		\spOps{\allcons}{ \cons: \to El}
		\Gamma:
			& x \prec \cons
}$}}
\ar[rrr]^<<<<<<<<<<<<{\txt{$\large{\mu(\cons)= ok}$}}_<<<<<<<<<<<<{\txt{$\large{\delta(\cons)={\cons}}$}} 
&&& \frame{\txt{$\spec{\tit{\mbox{\bf P[X]}_\cons=} \\
	\spSorts{\Sorts'}{El}
	\spOps{\Ops'}{
		d:	&& El \\
		f: & El & El }
	\spOps{{\allcons}'}{	\cons, ok: && El}
	\Phi':
		&1.& && f(d) \eleq d\\
		&2.&x\prec \cons &\To& f(x) \eleq f(x)\\
		&3.& && ok\prec\cons \\
	\Gamma':&4.& && x \prec \cons\\
}$}}
}
\]
Let the semantic functor $\fu{F}: \Mod(\thr{X_\cons}) \to
	\Mod(\thr{P[X]_\cons})$ send an $A\in\Mod(\thr{X_\cons})$ to $\fu
	F(A)$ given by:
\begin{itemize}\MyLPar
\item $|\fu{F}(A)| = |A| \uplus d$, i.e. $d$ is a new element added to the carrier of $A$,
\item $f^{\fu{F}(A)}(x)= d$, for all $x\in|\fu F(A)|$,
\item $ok^{\fu{F}(A)} = |A|$, by the semantic functor requirement,
\item $\cons^{\fu{F}(A)} = |\fu{F}(A)|$, by default.
\end{itemize}
Let's now introduce $\thr{W[X]_\cons}$ as an intermediary parameter, i.e., we
now have two PDTs $\thr
P'=\pdtsp{\mu'}{\delta'}{\thr{X_\cons}}{\thr{W[X]_\cons}}$ and
$\thr P''=\pdtsp{\mu''}{\delta''}{\thr{W[X]_\cons}}{\thr{P[W[X]]_\cons}}$:
\[ \xymatrix{
\thr{X_\cons}
\ar[rr]^<<<<<<<<<<{\txt{$\large{\mu'(\cons)=\cons}$}}_<<<<<<<<<<{\txt{$\large{\delta'(\cons)=\cons}$}}
&& \frame{\txt{$\spec{
\tit{\thr{W[X]_\cons}=}\\
	\spSorts{\Sorts''}{El}
	\spOps{\Ops''}{f: &El & El}
	\spOps{{\allcons}''}{\cons: && El}
	\Phi'':
		 &2.& x\prec\cons&\To& f(x) \eleq f(x)\\
	\Gamma'':&4.& && x \prec \cons_{EL}\\
}$}}
\ar[rr]^<<<<<<<<<<{\txt{$\large{\mu''(\cons)=ok}$}}_<<<<<<<<<<{\txt{$\large{\delta''(\cons)=\cons}$}}  
&& \thr{P[X]_\cons}=\thr{P[W[X]]_\cons}
}
\]
Obviously, we have that $\thr P=\pdtsp{(\mu';\mu'')}{(\delta';\delta'')}{\thr
X_\cons}{\thr{P[W[X]]_\cons}}$. But the refinement $\thr P\leadsto \thr
P';\thr P''$ is strict -- e.g., the functor $\fu F$ for the former cannot be obtained
by composing any two functors for the latter two.

For any semantic functor $\fu{F'}: \Mod(\thr{X_\cons}) \to\Mod(\thr{W[X]_\cons})$ 
can't extend the carrier of any $A\in \Mod(\thr{X_\cons})$, but 
merely adds a deterministic function $f$. Furthermore, any semantic functor 
$\fu{F''}: \Mod(\thr{W[X]_\cons}) \to\Mod(\thr{P[W[X]]_\cons})$ may add a
new element $d$ to the carrier of a parameter algebra $B\in\Mod(\thr{W[X]_\cons})$ and force
$f(d)\eleq d$. However, $\fu{F''}$ has to ``preserve'' the parameter algebra
$B$, i.e., $B$ must be a tight subalgebra of
$\fu{F''(B)}$. This means that function $f^{\fu F''(B)}$ applied to the elements
from the carrier of $B$ (i.e., from $ok^{\fu F''(B)}$) has to return elements
from the same carrier (as it did in $B$). If $d$ is a new element (as was the
case for $\fu F$), it will never be ``reachable by $f$'' from these old
elements. This illustrates the impossibility
of obtaining the original functor $\fu F$ as a composition of any $\fu F'$
and $\fu F''$.
\end{example}
%
\section{Refinement}\label{sub:ref}
The general concept of refinement of PDTs
% As we have emphasised, it amounts, primarily, to
amounts to 
introduction of additional {\em structure}.% on the PDTs. 
%The following definition captures the general concept

\begin{definition}\label{def:ref}
A PDT $\thr
P'=\pdtsp{\mu'}{\delta'}{\thr{X'_\cons}}{\thr{P[X]'_\cons}}$
refines a PDT $\thr P=\pdt$, 
$\thr P\leadsto\thr P'$, if there exist functors
$\reff X:\Mod(\thr  X_\cons)\to\Mod(\thr X'_\cons)$ and
$\reff{P[X]}:\Mod(\thr {P[X]'_\cons})\to\Mod(\thr{P[X]_\cons})$, such that
for any semantic functor $\fu F'$ for $\thr P'$, the functor $\reff X;\fu
F';\reff{P[X]}$ is a semantic functor for $\thr P$.
\[ \xymatrix@R=0.4cm@C=0.4cm{
&& \Mod(\thr{X_\cons})  \ar[dd]_{\iota} \ar@{.>}[rrdd]^{\reff{X};\fu{F}';\reff{P[X]}} \ar[lldd]_{\reff{X}}\\
&& && \\
\Mod(\thr{X'_\cons}) \ar[dd]_{\iota'} \ar[rrrdd]^{\fu{F'}}
	&& \Mod(\thr{X_{\weak}}) %\ar@{}[ur]|{1.}
		&& \Mod(\thr{P[X]_\cons}) \ar[ll]_{|_{\mu}} \\
&& && \\
\Mod(\thr{X'_{\weak}}) %\ar@{}[ur]|{2.}
	&&& \Mod(\thr{P[X]'_\cons}) \ar[lll]^{|_{\mu'}} \ar[uur]_{\reff{P[X]}}
}\vspace*{-2ex}
\]
\end{definition}
%The following diagram illustrates the requirement:
The relation is trivially transitive, i.e., $\thr P\leadsto\thr
P'\leadsto\thr P'' \To \thr P\leadsto \thr P''$.

The contravariance of $\reff{X}$ on the parameter side may seem a bit unusual since,
following the idea of refinement as the model class inclusion, one might
expect the refinement relation to hold also when
$\thr{X_\cons}\leadsto\thr{X'_\cons}$, i.e., $\Mod(\thr
X_\cons)\supseteq\Mod(\thr X'_\cons)$. However, if this refinement were
strict, i.e., if $\Mod(\thr X_\cons)\supset\Mod(\thr X'_\cons)$, 
a semantic functor $\fu F'$ with source $\Mod(\thr{X'_\cons})$ could not, in
general, be used in places were one assumes a functor with source $\Mod(\thr{X_\cons})$.
%%one should keep in mind that
%%we are talking about design specifications of actual structure of data
%%types/programs. A PDT with source $\thr{X'_\cons}$ could not, in general,
%%replace a PDT with the source $\thr{X_\cons}$ if the refinement
%%$\thr{X_\cons}\leadsto\thr{X'_\cons}$ were strict.

A simple example is when $\thr{P[X]_\cons}\leadsto\thr{P[X]'_\cons}$, i.e., when
$\reff{X}$ is identity and
$\reff{P[X]}$ is a model class inclusion $\Mod(\thr{P[X]'_\cons})\subseteq
\Mod(\thr{P[X]_\cons})$ (as was the case of vertical composition in section~\ref{sub:vertref}). 
Another example was \ref{ex:abstrref} %and \ref{ex:stackref},
% insection~\ref{sub:hcsem}, 
where both $\reff{X}$ and $\reff{P[X]}$ were
identities but where intermediary parameter forced additional structure
not (necessarily) required by the original, refined PDT.

Finally, we give an example showing an even more particular case of
refinement by adding structure, where the formal parameter of a PDT is itself
refined to a PDT.

\begin{example}
%Let 
The PDT $\thr
P=\pdtsp{\mu}{\delta}{\thr{Set}_\cons}{\thr{\choice[Set]_\cons}}$ 
%be the PDT given in Figure~\ref{fi:setch} which 
extends a specification of sets with a nondeterministic
choice.
%
%\begin{figure}[hbt]
\[%\hspace*{-1em}
\xymatrix{
\frame{\txt{$\spec{
\tit{\mbox{\bf spec\ Set}_\cons^{{}}=\ \Sorts: Set,El} \\
%	\spSort{Set,El}
	\spOps{\Ops}{
	\emptyset: && Set \\
	\prep\ : & El\times Set & Set 
	}
	\spOps{\allcons}{ \cons_{El}: \to El,\ \ \  \cons_{Set}:\to Set}
	\Phi: & 1.& x\prep (y\prep S) & \eleq & y\prep (x\prep S) \\
	      & 2.& x\prep (x\prep S) & \eleq & x\prep S \\
	\Gamma:
	   &3. & x &\prec &\cons_{EL} \\ 
	   &4. & S &\prec &\cons_{Set}
}$}}
\ar@(u,r)[dr]^{\txt{\normalsize{\hspace*{1.5em}$\mu(\cons_{El})=\delta(\cons_{El})=ok_{El}$}}}_{\txt{\normalsize${\mu(\cons_{Set})=\delta(\cons_{Set})=
	\cons_{Set}\ \ \ }$}} 
\\ &
\hspace*{-8em}\frame{\txt{$\spec{
\tit{\thr{spec \choice[Set]}_\cons=\ \Sorts: Set, El} \\
%	\spSort{Set,El}
	\spOps{\Ops}{
	\emptyset: && Set \\
	\prep\ : & El\times Set & Set \\
	\choice : & Set & El
	}
	\spOps{\allcons}{ ok,\cons_{El}: \to El,\ \ \  \cons_{Set}:\to Set}
	\Phi: & 1.& x\prec ok,y\prec ok &\To & x\prep (y\prep S)  \eleq  y\prep (x\prep S) \\
	      & 2.& x\prec ok & \To & x\prep (x\prep S)  \eleq  x\prep S \\
	      & 3.& x\prec ok & \To & \choice(x\prep S)\prec ok \\
	      & 4.& x\prec ok, z\prec \choice(x\prep S) 
	            & \To & z\eleq x, z\prec \choice(S)\\
	\Gamma:
   	    &5. &&& x \prec \cons_{EL} \\ 
	    &6. &&& S\prec \cons_{Set}
}$}}
}%\vspace*{-2ex}
\]
%\caption{Sets with nondeterministc choice parameterized by sets.}\label{fi:setch}
%\end{figure}
%
We admit here {\bf extending carrier} $El$, $\mu(\cons_{El})=ok$, which is
motivated by the possible need of a new, ``error'' element to be returned by
 $\choice(\emptyset)$.

A possible semantic functor $\fu F$ may send a $\thr{Set}_\cons$ algebra $A$ on the
algebra $\fu F(A)$ where, for any nonempty set $S$, the operation 
$\choice^{\fu F(A)}(S)$ returns all the elements of the set
$S$. $\choice^{\fu F(A)}(\emptyset)$ may return a new, ``error'' element $\bot$,
added to the carrier of $A$. Adding this element to a set, $\bot\prep^{\fu F(A)}S$ may then result in
the empty set $\emptyset^{\fu F(A)}$ (or else, leave $S$ unchanged).

Obviously, the specification $\thr{Set_\cons}$ can be naturally parameterized
by elements, i.e., we ``extract'' from it a parameter (sub)specification. We
obtain then $\thr P'=\pdtsp{\mu'}{\delta'}{\thr{El}_\cons}{\thr{Set[El]_\cons}}$,
where $\thr{El_\cons}$ contains merely the sort $El$ and the global guard
$x\prec \cons_{El}$, while $\thr{Set[El]_\cons}$ is exactly the same as
$\thr{Set_\cons}$. $\mu'$ and $\delta'$ are identities on $\cons_{El}$. 

The point is now that the composed PDT $\thr P';\thr P$ is a refinement of
$\thr P$ according to definition~\ref{def:ref}. 
\[\xymatrix@C=0.6cm@R=1cm{
\Mod(\thr{El_\cons}) \ar[drr]^{\fu F'} \ar@{.>}[d] && &&
  \Mod(\thr{Set_\cons})  \ar@{.>}[d]
    \ar[llll]_{|_{\mu'}\ =\ \reff{Set}}  \ar@{.>}[drr]^{\fu G} \ar@{.}[dll]|{id} \\
\Mod(\thr{El_\weak}) && \Mod(\thr{Set[El]_\cons})  \ar@{.>}[d] \ar[drr]^{\fu F} \ar[ll]_{|_{\mu'}}
   && \Mod(\thr{Set_\weak}) && \Mod(\thr{\choice[Set]_\cons})
   \ar@{.}[dll]|{id\ =\ \reff{\choice[Set]}} \ar[ll]_{|_{\mu}}\\ 
&& \Mod(\thr{Set[El]_\weak}) && \Mod(\thr{\choice[Set[El]]_\cons}) \ar[ll]_{|_{\mu}}
}
\]
Since $\mu'(\cons_{El})=\cons_{El}$ the  reduct $|_{\mu'}$ will
return an $\thr{El_\cons}$-algebra (and not only an
$\thr{El_\weak}$-algebra), so $\fu F'$ must be here persistent in the
classical sense.

Now, given $\fu F'$ and $\fu F$, the functor $\fu G$ can be trivially chosen
to be $\fu F$, if we let $\reff{Set}$ be identity. More generally, if $\fu F'$ were such that $|_{\mu'};\fu
F'=\fu{id}_{\thr{Set_{\cons}}}$, we will have that $\fu G=\reff{Set};\fu
F';\fu F;\reff{\choice[Set]}=\fu F$. 

However, even in such cases, the refinement consists in requiring a more
structured data type, which consists of building first an algebra of sets
over a given algebra of elements, and then an algebra with choice (i.e., 
composing $\fu F';\fu F$). 
%In this sense, it is reasonable to call $\thr P';\thr P$ a refinement of $\thr
%P$.

 Moreover, the functor $\fu F'$ will not, in general, be surjective on the
 objects, i.e., it may choose only a subclass of all $\thr{Set[El]_\cons}$
 algebras. In this case, the application of the composition to all models of
 $\thr{El_\cons}$, $\fu F';\fu F(\Mod(\thr{El_\cons}))$ may result in fewer
 $\thr{\choice[Set[El]]_\cons}$ algebras than $\fu F(\Mod(\thr{Set[El]_\cons}))$,
 which is another reason for viewing this composition as a refinement of the
 original PDT.
\end{example}
%
There is yet another possibility of viewing the above as an example of
refinement. Suppose that we have implemented the PDT
$\pdtsp{\mu}{\delta}{\thr{Set_\cons}}{\thr{\choice[Set]_\cons}}$, i.e., we
have a functor $\fu G$.  Then, having implemented also a functor $\fu F'$, we
can compose it with $\fu G$. Since $\fu F'$ will not, typically, be
surjective on the objects, this composition will, in general, yield a smaller
subclass of $\Mod(\thr{\choice[Set]_\cons})$ than the image of $\fu G$.

All these scenarios are different examples of a process of gradual
refinement by introduction of additional structure into the original
specification. If we start with a flat specification
$\thr{\choice[Set]_\cons}$, in the first
step we extract from it the parameter $\thr{Set_\cons}$ which prescribes a
more specific, structured implementation. (This was the PDT {\thr{P}} with
which we started the above example.) In the second step, we again extract
the parameter $\thr{El_\cons}$, requiring even more structure. Viewed in this
way, our setting gives a concrete specialization of the general concept of
``constructor implementations'' from \cite{para1}. A more detailed study of
the methodological possibilities offered by our PDTs is left for a future work.

\section{Conclusions}\label{se:conc}
We have presented a framework for specifying parameterized data types. The
syntax of PDTs is defined by a series of restrictions on the
syntax of parameterized specifications, with the additional means for indicating
the possibility of extending the carrier of the parameter algebras as well as
restricting the axioms of the parameter specification (to apply only  to the ``old''
elements). 

Semantics of PDTs is defined by a class of functors which satisfy a
generalization of the classical persistency requirement -- the parameter
has to be a (tight) subalgebra of its image under the semantic functor. This
generalization leads to a great flexibility in specifying PDTs which was
illustrated by a series of examples. 

We have re-stated and proved the theorems of vertical and horizontal
composition of PDTs in our setting. An important concept emerging from these
theorems concerns refinement of PDTs. We have given a general definition of
such a refinement which is reflected in concrete examples, primarily, as introduction of
additional structure into the specified PDT.

We view PDTs as design
specifications which put requirements not only on the abstract (input-output)
properties of the implementations but also on the actual structure of the
implementation. In this way, our PDTs give a concrete realization of a more
general concept of `constructor specifications' from \cite{para1} which has
recently been included into CASL \cite{CASL} as `architectural
specifications'. Indeed, the suggested refinement of PDTs can be naturally seen as an
example of program development based on constructor specifications where
successive stages amount to splitting the original, loose specification into
smaller pieces, 
according to the desired structure of the intended implementation.

Although we have presented the whole setting using the institution of multialgebras, it should
be easy to recognize the generic aspect of our definitions and
constructions. Entirely analogous extensions can be made on the top of any
semi-exact institution where signatures contain means of expressing predicates or
subsorts. We see a straightforward way of doing that, for instance, for
partial algebras or partial membership algebras. Since our constructions
involved occasionally empty set as a default value, one might perhaps expect
that generalization to total algebras could be more involved, but we expect
that this, too, can be achieved.
In any case, we suggest quite a general way of applying our setting which
certainly is not restricted to multialgebras. It is thus a significant
extension of similar ideas applied in \cite{paramPoigne} to 
very specific context of order sorted algebras.

A point which certainly requires a further study concerns reasoning about
PDTs. We expect that addition of generic axiom schemata expressing closure of the
parameter algebras (i.e., that operations applied to the ``old'' elements
return ``old'' elements) will lead to a complete axiomatization but this issue
needs to be investigated.

On the other hand, we would like to use PDTs for study and, perhaps, design
of more specific structuring mechanisms at the level of implementations.
We believe that the current work can provide a useful starting point for designing more
detailed constructs, for instance, for architectural specifications in languages like CASL.


\bibliography{biblo}
\bibliographystyle{plain}


\end{document}


