
\section{Introduction}
The concept of institutions \cite{inst} has become the standard framework for presenting
model-theoretic aproaches to logic and, in particular, to algebraic
specification. 

In this paper we show that the multialgebras form an exact institution
$\inst{MA}$.  We prove that $\inst{MA}$ is an institution in
section~\ref{se:instMA}.  In section~\ref{se:exact} we show that the model
functor for multialgebras $\fu{Mod}_{\inst{MA}}$ sends finite co-limits in
$\cat{Th}$ (category of specifications) to limits in $\cat{Cat}$,
i.e. $\inst{MA}$ is an exact institution \cite{statestruct} (called
institution with composable signatures in \cite{stateinst}). We also mention
the well known fact that every exact institution satisfies the amalgamation
lemma. The results are not used in this paper but they form a basis for
subsequent work on structuring multialgebras and their specifications, in
particular, on amalgamation and parameterized multialgebras. 
In study of
parameterizated specifications the co-limits (actually pushouts)
are used for defining parameter instantiation. The exactness of the model
functor (actually the amalgamation lemma) ensures that corresponding
instantiation can be performed at the semantic level.

We begin, in section~\ref{se:pre}, by presenting the background definitions of
multilagebras and collecting the relevant definitions and results about institutions.

\section{Preliminaries}\label{se:pre}

\subsection{Notation}
We use the notation $|${\cat{C}}$|$ to denote the objects of a category
\cat{C}. The same notation is used to denote the carrier $|A|$ of an algebra
$A$. (This shouldn't cause any confusion.)  Institutions are written with
the script font $\inst{I}$, categories with bold $\cat{Cat}$, and functors with
Sans Serif $\fu{Func}$.

Since specifications is categories we write $\thr{Spec}$ for ordinary specifications.
%, $\thr{Par[X]}$ denote a specification parametrized by specification$\thr{X}$, note that we us $[]$ as parentheses.
 Sequences $s_1, \ldots, s_k$ will be often denoted
by $\overline{s}$. Application of functions are then understood to not
distribute over the elements, i.e., $f(\ovr s)$ denotes the term
$f(s_1,\ldots, s_k)$.  
Occasionally, a sequence $s_1 \ldots s_k$ may be denoted by $s^*$ -- 
applications of functions are then understood to
distribute over the elements, i.e., $f(s^*)$ denotes the sequence
$(f(s_1),\ldots, f(s_k))$. We will denote the disjoint union of sets $A,B$ by $A \uplus B$.

\subsection{Algebraic Signatures}

 % In the following sections, we will identify and use various
%sub institutions of $\inst{MA}$.
%\fixx{only sub institution?}\\
Signatures for multialgebras are the same as classical {\it algebraic signatures}.
%
\begin{definition}\label{de:Sign}
The category of signatures $\Sign$ has:
\begin{itemize}\MyLPar
\item signatures as objects: a signature
$\Sigma$ is a pair of sets $(\Sorts,\Ops)$ of symbols for names of sorts and
operations. Each operation symbol $\omega\in\Ops$ is a (k+2)-tuple:
$\omega : s_1 \times \cdots \times s_k \to s$,
where {$s_1, \ldots , s_k,s \in S$ and $k \geq 0$}.  $\omega$ is the
{\it name} of the operation and $s_1 \times \ldots \times s_k \to s$ its {\it
arity}. If $ k=0$ then an operation $c: \to s$ is called a {\it constant} of
sort $s$.
\item  signature morphisms as arrows: 
a signature morphism $\mu: \Sigma \to \Sigma'$ is a pair $\mu = (\mu_S,
\mu_{\Omega})$ of (total) functions:
$\mu_S: S \to S'$, $\mu_{\Omega}: \Ops \to \Ops'$, 
such that 
$ \mu_{\Omega}(\omega:s_1 \times \cdots \times s_n \to s) =
\omega':\mu_S(s_1) \times \cdots \times \mu_S(s_n) \to \mu_S(s)$
\item
Identities are the identity
signature morphisms and morphism are composed component-wise.
\end{itemize}
\end{definition}
%
In the standard way, 
we extend the signature morphism $\mu : \Sigma \to \Sigma'$ to terms, we use the notation $\TermsSX$ for the $\Sigma$ terms with $X$ as variables.

\begin{definition} Extension of a
signature morphism $\mu$ to terms ${\tilde{\mu}} : \TermsSX \to \Terms{\Sigma',X'}$ is defined by:
\begin{itemize}\MyLPar
\item ${\tilde{\mu}} (x_s) = x_{\mu (s)}$, for each variable $x_s \in X_s$
\item ${\tilde{\mu}} (c) = \mu(c)$
\item ${\tilde{\mu}} (\omega(t_1, \ldots ,t_n )) = \mu(\omega)({\tilde{\mu}}(t_1), \ldots , {\tilde{\mu}}(t_n))$
\end{itemize}
\end{definition}
In general variables can be renamed too, but (without loss of generality) we simplify the presentation. 
%The extra subscript for variables prevents name clash. 
We will write $\mu(t)$
instead of ${\tilde{\mu}}(t)$, for terms $t \in \TermsSX$.

\subsubsection{Algebraic signatures have all finite co-limits}\label{sub:sign}
It is well known that the category of algebraic signatures is co-complete,
see e.g. \cite{fundamental1}. We are merely restating here this standard fact.
 Limiting our
attention to finite co-limits, 
it is sufficient to consider the existence of initial object, co-products (sums) and co-equlizers
(see e.g. \cite{cat}).
Since
multialgebras use algebraic signatures all these results will also hold
for multialgebric signatures, $\cat{Sign_{\inst{MA}}}$.

\begin{fact}
The empty signature, $\Sigma_{\emptyset}$ is initial in $\cat{Sign}$
\end{fact}

\begin{fact}
The sum of two signatures; $\Sigma + \Sigma'$ is the disjoint union (of sorts
and operations), with the natural injections.
\end{fact}

\begin{fact}
Given two signature morphims $\mu,\nu:\Sigma \to \Sigma'$, let $\kernel{}$ be the least equivalence on $\Sigma'$ induced by the relation with components:
	\begin{itemize}
	\item Sorts: $\kernel{S'}= \{ \langle \mu(s),\nu(s) \rangle: s \in \Sigma \}$,
	\item Operations $\kernel{\Omega'}= \{ \langle \mu(\omega), \nu(\omega) \rangle : \omega \in \Omega \}$
	\end{itemize} 

Then $\qu{\Sigma'}{\kernel{}}$ is a co-equalizer object, with canonical
signature morphim $\iota: \Sigma' \to \qu{\Sigma'}{\kernel{}}$,
and we have that $\comp{\mu}{\iota} =
\comp{\nu}{\iota}$, by construction.

Note that if $\sigma:\Sigma' \to Z$ is a signature morphism such that
$\comp{\mu}{\sigma}= \comp{\nu}{\sigma}$, then the kernel of $\sigma$ has to
include $\kernel{}$, so the signature morphism $u_{\sigma}:
\qu{\Sigma'}{\kernel{}} \to Z$, defined by $u_{\sigma}([s']_{\kernel{}})=
\sigma(s')$ and $u_{\sigma}([\omega']_{\kernel{}})= \sigma(\omega')$ is the
unique factorization arrow.
\end{fact}

\begin{fact} \cite{fundamental1}
The category $\cat{Sign}$ of algebraic signatures  has all (finite) co-limits.
\end{fact}

\subsection{Multialgebras}
We will now summarize the relevant notions about multialgebras (for an
overview, see \cite{multi,catrel}).

A multialgebra for a signature $\Sigma$ is an algebra where operations may be
set-valued. ${\mathcal{P}}(y)$ denotes the power set of set $y$.

\begin{definition}\label{def:ma}
(Multialgebra) A multialgebra $A$ for $\Sigma$ is given by:
\begin{itemize}
\item  a set $s^A$, the carrier set, for each sort symbol $s\in\Sorts$
\item  a subset $ c^A \in {\mathcal{P}}(s^A)$, for each constant, $c:\to s$
\item an operation $\omega^A : s_1^A \times \cdots \times s_k^A \to {\mathcal{P}}(s^A)$
	for each symbol $\omega : s_1\times\cdots\times s_k \to s \in \Ops$
\end{itemize}
\end{definition}
One sometimes demands that constants and operations are total
\cite{multi,toplas}, i.e. never return empty set and take values only in
${\mathcal{P}}^+ (s^A)$, the nonempty subsets of $s^A$. We will not make this
assumption.% unless else is stated.

Note that for a constant $c \in \Ops$, $c^A$ denotes a (sub)set of the
carrier $s^A$. This allows one to use constants as predicates as was done,
for instance, in \cite{partial}. 

As homomorphisms of multialgebras, we will use weak homomorphisms (see
\cite{catrel} for alternative notions).
\begin{definition}Given two multialgebras $A$ and $B$, 
a function $h: |A| \to |B|$ is a (weak) homomorphism if:
\begin{enumerate}
\item $h(c^A) \subseteq c^B$, for each constant $c: \to s$ 
\item $h(\omega^A(a_1 , \ldots , a_n)) \subseteq\omega^B(h(a_1), \ldots ,h(a_n))$,
for each operation  $\omega:s_1\times\cdots\times s_n\to s\in\Ops$ and for all $a_i \in s_{i}^A$.
\end{enumerate}
\end{definition}
Saying ``homomorphism'' we will always mean weak homomorphism.
%, unless something else is stated.

\begin{definition} 
The category of $\Sigma$-multialgebras, $\MAS$, has $\Sigma$-multialgebras as
objects and homomorphisms as arrows.  The identity arrows are the identity
homomorphisms and composition of arrows is obvious composition of
homomorphisms.
\end{definition} 
Multialgebraic specifications are written using the following formulae:
%
\begin{definition}\label{def:masen}
Formulae of multialgebraic specifications are of the following forms:
%{\bf{Atomic formula}}:
\begin{enumerate}\MyLPar
\item Atomic formulae, suppose $t, t' \in \TermsSX$:
\begin{itemize}\MyLPar
\item $t \eleq  t'$ (equality), $t$ and $t'$ denote the same one-element set.
\item $t \prec t'$ (inclusion), the set interpreting $t$ is included in
the set interpreting $t'$.
\end{itemize}
\item $a_1 , \ldots ,a_n \To b_1 , \ldots ,b_m$, where either $n>0$ or $m>0$ and
each $a_i$ and $b_j$ is atomic.
\end{enumerate}
%The first two are called atomic formulae.
\end{definition}
%
Given a set of variables $X$, an assignment is a function $\alpha: X \to |A|$ 
assigning {\em individual} elements of the carrier of $A$ to variables. It induces a unique
interpretation $\overline{\alpha}: T(\Sigma(X)) \to A$ of every term $t$
(with variables from $X$) in
$A$.
\begin{definition} Given a $\Sigma$-multialgebra $A$, an {\it assignment} to 
variables $X$ is a function $\alpha:X \to |A|$. An assignment induces a
unique interpretation $\overline{\alpha}(t)$ in $A$ of any term $t \in \TermsSX$ as
follows:
\begin{itemize}\MyLPar
\item $\ovr\alpha(x)=\{\alpha(x)\}$
\item $\ovr\alpha(c)=c^A$
\item $\ovr\alpha(\omega(t_1 , \ldots ,t_n)) = \bigcup_{a_{i}\in\ovr\alpha(t_{i})}\omega^A(a_1 , \ldots ,a_n)$
\end{itemize}
\end{definition}
Keep in mind that variables are assigned not sets but individual
elements of the carrier. 
We will write $\alpha(t)$ instead of $\ovr\alpha(t)$.
 
Satisfaction of formulae in a multialgebra is defined as follows:
\begin{definition}\label{de:sat}
Given an assignment $\alpha:X\to|A|$:
\begin{enumerate}
\item $A \models_\alpha t \eleq  t'\ {\rm iff\ } \ovr{\alpha}(t)
=\overline{\alpha}(t') = \{e\},\ {\rm fore\ some element\ } e \in |A|$
\item $A \models_\alpha t \prec t'\ {\rm iff\ } \ovr{\alpha}(t) \subseteq \overline{\alpha}(t')$
\item $ A \models_\alpha a_1 , \ldots ,a_n \To  b_1 , \ldots ,b_m\ {\rm iff\ }
\exists i: 1 \leq i \leq n : A \not\models_\alpha a_i\ {\rm or\ } \exists j: 1 \leq j \leq m : A \models_\alpha b_j$
\item $A \models \varphi\ {\rm iff\ } A \models_\alpha \varphi\ {\rm for\ all\ } \alpha$
\end{enumerate}
\end{definition}
%
Putting these definitions together, we show (in section~\ref{se:instMA}) that
the multialgebras form an institution $\inst{MA}$. Before that we recall the
standard concepts of institution and some results which will be relevant for
us in investigating the institution of multialgebras.

 

\subsection{Institutions}\label{se:inst}

\begin{definition}\label{de:inst} \cite{inst} 
An institution is a quadruple $\inst{I} = (\Sign,\Sen,\Mod,\models)$, where:
	\begin{itemize}
\item $\Sign$ is a category of signatures.  
\item $\Sen: \Sign \to \Set$ is a functor which associates a set of {\it sentences} to each signature.
\item $\Mod:\Sign^{op} \to \cat{Cat}$ is a functor which associates a
category of {\it models}, whose morphisms are called $\Sigma$-morphisms, to
each signature $\Sigma$
\item $\models$ is a satisfaction relation -- for each signature $\Sigma$, a relation
 $\models_{\Sigma} \subseteq |\Mod(\Sigma)| \times
\sen(\Sigma)$, such that the following {\it satisfaction condition} holds:
for any $M' \in \Mod(\Sigma'), \mu: \Sigma \to \Sigma' , \phi \in \sen(\Sigma)$
 
  \[ M' \models_{\Sigma'} \Sen(\mu)(\phi) \mbox{ iff } \Mod(\mu)(M') \models_{\Sigma} \phi\]
	\end{itemize}
\end{definition}
%
The definition can be represented as the following diagram:

\[\xymatrix{
	\Sigma \ar[d]_{\mu}
		& {\Mod}(\Sigma)
			& \models_{\Sigma}
				& \Sen(\Sigma) \ar[d]^{\Sen(\mu)}	\\
	\Sigma'
		& {\Mod}(\Sigma') \ar[u]_{{\Mod}(\mu)}
			& \models_{\Sigma'}
				& \Sen(\Sigma')			\\
								}
\]

The following subsections review
institution independend concepts and results which will be
used in the later section.



\subsubsection{Category of specifications}\label{sub:catspec}
Based on the above satisfaction relation for institutions (definition~\ref{de:inst}) we write: $\Gamma \models_{\Sigma}
 \varphi \mbox{ iff } \forall M \in \Mod(\Sigma): M \models_{\Sigma} \Gamma
 \Rightarrow M \models_{\Sigma} \varphi $.  With this in mind we write
 $\Gamma^{\bullet}$ for the semantical consequences of $\Gamma$
 i.e. $\Gamma^{\bullet} = \{ \varphi: \Gamma \models \varphi \}$
 
A {\em theory} (specification) in an institution is any pair
${Th}=(\Sigma,\Gamma)$ where $\Sigma\in\obj{\Sign}$ and $\Gamma\subseteq
\sen(\Sigma)$. For a given institution $\inst{I}$, we have the corresponding
category of theories $\cat{Th}$ (possibly indexed by the institution,
$\cat{Th_{\inst{I}}}$) with theories as objects and theory morphisms
$\mu:(\Sigma,\Gamma) \to (\Sigma',\Gamma')$, where $\mu:\Sigma \to \Sigma'$,
is a signature morphism such that: $\Gamma'
\models_{\Sigma'} \sen(\mu)(\Gamma)$.  The models for the theory
${Th}=(\Sigma,\Gamma)$ is the full sub category
$\Mod_{\models}(\Sigma,\Gamma)$ of $\Mod(\Sigma)$ where $M \in
\Mod_{\models}(\Sigma,\Gamma) \mbox{ iff } M \models_{\Sigma} \varphi,
\forall \varphi \in \Gamma$, we will write $\Mod(\Sigma,\Gamma)$ instead of
$\Mod_{\models}(\Sigma,\Gamma)$.  The satisfaction condition gives that
$\Mod(\mu)(\Mod(\Sigma',\Gamma')) \subseteq$ % \thr{Th}
$\Mod(\Sigma,\Gamma)$, for each morphism $\mu:(\Sigma,\Gamma) \to (\Sigma',\Gamma') \in \thr{Th}$. This means that the functor $\Mod$ can be
extended to a functor $\Mod_{\models}:\cat{Th}^{op} \to \cat{Cat}$. There is
a canonic projection (forgetful) functor $\fu{Sign}:\thr{Th} \to \cat{Sign}$ and there is
an embedding functor $\fu{th}: \cat{Sign} \to \thr{Th}$ defined by
$\fu{th}(\Sigma)= (\Sigma,\emptyset)$.
% A theory morphism $\mu:(\Sigma,\Gamma) \to (\Sigma',\Gamma')$ is called axiom preserving if $\mu(\Gamma) \subseteq \Gamma'$. This defines the sub category $\thr{Th_0}$ with theories as objects and axiom preserving theory morphisms as morphisms.

%\begin{definition}
%Given a functor $\fu{\Phi}:\thr{Th_0} \to \thr{Th_0'}$ and a natural
%transformation $\alpha: \fu{Sen} \Rightarrow \fu{Sen'} \circ \fu{\Phi}$, $\fu{\Phi}$ is
%$\alpha$-sensible iff:
%\begin{itemize}
%	\item There is a functor $\fu{\Phi^{\diamond}}:\fu{Sign} \to \fu{Sign'}$ such that $\fu{sign'} \circ \fu{\Phi} =\fu{\Phi^\diamond}  \circ  \fu{sign}$
%	\item $(\Gamma')^\bullet = (\emptyset'_{\Sigma} \cup \alpha_{\Sigma}(\Gamma))^{\bullet}$
%\end{itemize}
%Where we denote the set of axioms induced by $\fu{\Phi}(\Sigma)$ by $\emptyset'_{\Sigma}$.
%\end{definition}
%

\subsubsection{Construction of co-limits of specifications}
The following institution independent result ensures that the category of
specifications has all co-limits if the signature category has.  This
result is used to create co-limits of specifications by first creating the
co-limit for the coresponding signatures.


\begin{theorem} \cite{inst}
\label{teo:reflects}
The functor  $\fu{Sign}: \thr{Th} \to \thr{Sign}$ reflects co-limits, in any institution $\inst{I}$.
\end{theorem}
As a particular case, the theorem means that:
Given specifications $\thr{X}=(\Sigma,\Phi)$, $\thr{X^1}=(\Sigma^1,\Phi^1)$, $\thr{X^2}=(\Sigma^2,\Phi^2)$, and specification morphisms $\mu_1:\thr{X} \to \thr{X^1}$ and $\mu_2:\thr{X} \to \thr{X^2}$.
If the  diagram to the left is a pushout of signatures than the diagram to
the right is a pushout of specifications:
\[\xymatrix@R=0.3cm{%\label{di:signpushout}
\Sigma \ar[dd]_{\mu_{2}} \ar[rr]^{\mu_1}&& \Sigma^1 \ar[dd]^{{\mu'}_2} &&&&
   \thr{X} \ar[dd]_{\mu_{2}} \ar[rr]^{\mu_1} && \thr{X^1} \ar[dd]^{{\mu'}_2}
   \\
      && &&& \ar@{=>}[ll]_{\fu{Sign}} &\\
\Sigma^2 \ar[rr]_{{\mu'}_1}		&& \Sigma' &&&&
\thr{X^2} \ar[rr]_{{\mu'}_1}		&& \thr{X'} \\
}
\]
where $\thr{X'}=
(\Sigma',\Phi')$ and $\Phi'=  {\mu'}_{1}(\Phi^2) \cup
{\mu'}_{2}(\Phi^1)$.



\subsubsection{Continuity of \fu{Mod} and amalgamation}
Construction on specifications can be ``carried over'' to the respective
constructions on their model classes provided that the \fu{Mod} functor has
some desired properties. Typical constructions on specifications are co-limits
and the desired property of \fu{Mod} is that it transforms co-limits in
$\cat{Th}$ to limits in $\cat{Cat}$.

\begin{definition} An institution $\inst{I}$ is 
\begin{enumerate}
\item {\em semi exact} iff $\Sign$
has pushouts and $\Mod$ sends pushouts in $\Sign$ to pullbacks in
$\cat{Cat}$,
\item {\em exact} iff $\Sign$ has finite co-limits and $\Mod$ sends
finite co-limits in $\Sign$ to limits in $\cat{Cat}$.
\end{enumerate}
\end{definition}
Of course, any exact institution is also semi exact. The importance of this
notion is exemplified by the following lemma which indicates the construction
for instantiation of parameterized specifications.
%\begin{definition} An institution $\inst{I}$ is {\em exact} iff $\Sign$ has finite co-limits and $\Mod$ transforms finite co-limits in $\Sign$ into limits in $\cat{Cat}$
%\end{definition}
%
%
%
\begin{lemma}(Amalgamation Lemma).\\
In any semi exact institution $\inst{I}$, for every pushout of
signatures (on the left):
\[\xymatrix@R=0.3cm@C=0.6cm{
%\label{di:specpushout}
\Sigma \ar[dd]_{\mu_{2}} \ar[rr]^{\mu_1} && {\Sigma^1} \ar[dd]^{{\mu'}_2}  &&&& 
  \Mod(\Sigma) && \Mod(\Sigma^1) \ar[ll]_{\_|_{\mu_{1}}}  \\
&&  &\ar@{=>}[rr]^{\Mod} &&&\\
{\Sigma^2} \ar[rr]_{{\mu'}_1}	&& {\Sigma'} &&&&
   \Mod(\Sigma_2) \ar[uu]^{\_|_{\mu_{2}}} && \Mod(\Sigma')
   \ar[uu]_{\_|_{\mu'_{2}}} \ar[ll]^{\_|_{\mu'_{1}}}  
}
\]
we have that: for any two models $M_1 \in \Mod({\Sigma^1})$ and $M_2 \in
\Mod({\Sigma^2})$ satisfying $M_1|_{{\mu}_1} = M_2|_{{\mu}_2}$, there exists 
a unique model $M' \in \Mod({\Sigma'})$, such that $M'|_{{\mu'}_1} = M_2$ and
$M'|_{{\mu'}_2} = M_1$.
\end{lemma}
%
The corresponding amalgamation property holds also for homomorphisms. 
In fact, the amlgamation lemma tells that the model class of a
pushout $\Sigma'$ of signatures along $\mu_1,\mu_2$ is a pullback (in $\cat{Cat}$) of the respective
morphisms $\_|_{\mu_{1}}$, $\_|_{\mu_{2}}$. The model $M'$ is the {\em
amalgamated union} of $M_1$ and $M_2$. 

By
theorem~\ref{teo:reflects}, the amalgamation lemma holds then also for pushouts of
specifications, since these are constructed from pushouts of signatures.

\section{Institution of multialgebras, $\inst{MA}$}
\label{sec:multi}
We now define and prove that the multialgebras form
an institution $\inst{MA}$ (subsection~\ref{se:instMA}) and that this
institution is exact (subsection~\ref{se:exact}).

\subsection{Multialgebras form an institution}\label{se:instMA}
%To relate algebras of different specifications one have the reduct construction, the reduct is the semantical ``inverse'' of a signature morphism.
First we apply the standard concept of reduct to multialgebras.
\begin{definition}
Let $\mu: \Sigma \to \Sigma'$ be a signature morphism.\\
Reduct of an algebra: The $\mu$-reduct
of a $\Sigma'$-multialgebra $A'$, is the $\Sigma$-multialgebra $A'|_\mu$
defined by:
\[ \begin{array}{ll} 
	s^{{A'}|_\mu} = \mu(s)^{A'} 		&\mbox{, for all $s \in S$,} \\
 	\omega^{{A'}|_\mu} = \mu(\omega)^{A'} 	&\mbox{, for all $\omega \in \Omega$,}\\
\end{array} \]
Reduct of assignment:
For a set of variables $X$, $t$ a $\Sigma'$ term, $A'$ a $\Sigma'$ algebra
and $\alpha': \mu(X) \to A'$ an assignment for $A'$, the $\mu$-reduct of
$\alpha'$, $\alpha' |_\mu:X\to A'|_\mu$ is defined by:
\[	\begin{array}{l}
%	\alpha' |_\mu : X \to A' |_\mu \mbox{ and}\\  
	(\alpha' |_\mu)_{s}(x) = \alpha'_{\mu(s)}(\mu(x))
	\end{array}
\]
Reduct of a homomorphism: The $\mu$ reduct of a weak $\Sigma'$ homomorphism
$h':A' \to B'$, is the weak $\Sigma$ homomorphism $h'|_\mu: A'|_\mu \to  B'|_\mu$ defined by:
\[ \begin{array}{l} 
%	h'|_\mu : A'|_\mu \to  B'|_\mu \mbox{ and}\\
 	(h'|_\mu)_s = h'_{\mu(s)}
\end{array} \]
\end{definition}
If one wants to allow possible renamings of variables $X$ along the signature
morphisms, the definition would be entirely analogous -- we omit this technicality.
%
\begin{fact} The reduct of a multialgebra is a multialgebra, the reduct of an
assignment is an  assignment to the reduct algebra, the reduct of a weak
homomorphism is a weak homomorphism between the reduct of two algebras.%, likewise is the reduct of a tight
%homomorphism a tight homomorphism between the reduct of two algebras.
\end{fact}
The proof of this fact is analogous to the classical case. 

We are now ready to define the model functor $\fu{Mod_{\inst{MA}}}:\Sign^{op}\to\cat{Cat}$ which
maps each $\Sigma\in\obj{\Sign}$ to the category of all $\Sigma$-multialgebras $\MAS$.
 
\begin{definition}\label{de:MAlg}
The functor $\fu{Mod_{\inst{MA}}}: \Sign^{op} \to \cat{Cat}$ is defined by:
\begin{itemize}\MyLPar
\item objects: $\fu{Mod_{\inst{MA}}}(\Sigma) = \MAS$
\item arrows: $\fu{Mod_{\inst{MA}}}(\mu:\Sigma \to \Sigma') = \fu{Mod_{\inst{MA}}}_\mu :
\MA{\Sigma'} \to \MAS$, where the functor $\fu{Mod_{\inst{MA}}}_\mu$ is given by:
	\begin{enumerate}\MyLPar
	\item $\fu{Mod_{\inst{MA}}}_\mu(A') = A'|_\mu$
	\item $\fu{Mod_{\inst{MA}}}_\mu(h') = h'|_\mu$
	\end{enumerate}
\end{itemize}
\end{definition}

\begin{lemma}(Reduct theorem)
If $\mu:\Sigma \to \Sigma'$ is a signature morphism, $X$ a set of variables, $t$ a $\Sigma'$ term, $A'$ a $\Sigma'$ algebra and $\alpha': \mu(X) \to A'$ an assignment for $A'$ then we have that:
\[ \alpha' |_{\mu}(t) ^{{A'} |_{\mu}} = \alpha'(\mu(t))^{A'} \]
\end{lemma}
%{\bf proof}
\begin{PROOF}
The proof goes by induction on the complexity of the term $t$.
\begin{enumerate}
\item $t = x \in X_s$.
	\begin{eqp}
%	\comment{}{assumption}
		\alpha' |_{\mu}(x)^{{A'} |_\mu} 
	\comment{=}{assignment}
		(\alpha' |_{\mu})_s(x)
	\comment{=}{def. $\alpha' |_\mu$}
		(\alpha' _{\mu(s)})(\mu(x))
	\comment{=}{assignment}
		\alpha'(\mu(x))^{A'}
	\end{eqp}
\item $t = c, (c: \to s)$
	\begin{eqp}
%	\comment{}{assumption}
		\alpha' |_{\mu}(c)^{{A'} |_\mu}
	\comment{=}{no assignment for constants}
		c^{{A'} |_\mu}
	\comment{=}{def. $A' |_\mu$}
		\mu(c)^{A'}
	\comment{=}{no assignment, for constants}
		\alpha'(\mu(c))^{A'}
	\end{eqp}
\item $t = \omega(t_1, \ldots ,t_n), (\omega: s_1 \times \cdots \times s_n \to s)$
	\begin{eqp}
%	\comment{}{assumption}
		\alpha' |_{\mu}(\omega(t_1, \ldots ,t_n)^{{A'} |_\mu}
	\comment{=}{assignment on function}
		\bigcup_{a_{i}\in\alpha' |_{\mu}(t_{i})} \omega^{{A'} |_\mu}(a_1, \ldots ,a_n)
	\comment{=}{ind. hyp. and def reduct}
		\bigcup_{a_{i}\in\alpha'({\mu}(t_{i}))} \mu(\omega)^{A'}(a_1, \ldots ,a_n)
	\comment{=}{assignment on function}
		\alpha'(\mu(\omega)(\mu(t_1), \ldots ,\mu(t_n))^{A'}
	\end{eqp}
\end{enumerate}
\end{PROOF}
%{\bf end proof}
The following lemma leads immediately to the satisfaction condition.
\begin{lemma}\label{le:masat}
For any signature morphism $\mu: \Sigma \to \Sigma'$ and $\Sigma'$ algebra
$A'$, given assignment
\begin{itemize}\MyLPar
\item $\alpha:X \to A'|_\mu$, define  $\alpha':\mu(X) \to A'$ by:
 $\alpha'_{\mu(s)}(\mu(x)) = \alpha_s(x)$, and
\item $\alpha':\mu(X) \to A'$, define $\alpha:X \to A' |_\mu$ by:
 			$\alpha_s(x) = \alpha'_{\mu(s)}(\mu(x))$, i.e. $\alpha=\alpha'|_\mu$
\end{itemize}
Then for any $\Sigma$-formula
$\varphi$ and for any $\Sigma'$ multialgebra $A'$ we have that:
\[ (A' |_\mu) \models_\alpha \varphi \Iff A' \models_{\alpha'} \mu(\varphi) \]
\end{lemma}
\begin{PROOF}
By induction on the formulas:
\begin{enumerate}
 
\item For atomic formulae $t_1 \eleq  t_2$ and $t_1\prec t_2$, the reduct theorem
gives $\alpha'|_\mu(t_i)^{A'|_{\mu}}=\alpha'(\mu(t_i))^{A'}$, which yields
the result.

\item $a_1, \ldots ,a_n \rightarrow b_1, \ldots b_m$ \label{it:conditional}\\
There are two cases:\\
a) for some $a_i: (A' |_\mu) \not\models_\alpha a_i \by{IH}{\Iff}
A' \not\models_{\alpha'} \mu(a_i)$, or\\
b) for some $b_k: (A' |_\mu) \models_\alpha b_k
  \by{IH}{\Iff} A' \models_{\alpha'} \mu(b_k)$
\end{enumerate}
\end{PROOF}
%
\begin{lemma}\label{th:masat}(Satisfaction condition)
The satisfaction condition is fullfiled for multialgebras, i.e. for any
signature morphism $\mu: \Sigma \to \Sigma'$, for any $\Sigma$-formula
$\varphi$ and for any $\Sigma'$ multialgebra $A'$ we have that:
\[ (A' |_\mu) \models_\Sigma \varphi \Iff A' \models_{\Sigma'} \mu(\varphi) \]
\end{lemma}
\begin{PROOF} Let $\varphi$ be an arbitrary formula.\\
``$\Leftarrow$'': let $\alpha:X \to A'|_\mu$ be arbitrary and let $\alpha'$
 be as in lemma~\ref{le:masat}. Then, by assumption, $A' \models_{\Sigma'}
 \mu(\varphi)$ and, in particular, $A' \models_{\alpha'} \mu(\varphi)$. By
 lemma~\ref{le:masat}, $(A' |_\mu) \models_\alpha \varphi$. Since $\alpha$
 was arbitrary, we obtain $(A' |_\mu) \models_\Sigma \varphi$.\\
``$\Rightarrow$'': let $\alpha':\mu(X) \to A'$ be arbitrary, and let $\alpha$
 be as in lemma~\ref{le:masat}. By assumption, $(A' |_\mu) \models_\Sigma
 \varphi$, in particular, $(A' |_\mu) \models_\alpha \varphi$. By
 lemma~\ref{le:masat}, $A' \models_{\alpha'} \mu(\varphi)$. Since $\alpha'$
 was arbitrary, we obtain $A' \models_{\Sigma'} \mu(\varphi)$.
\end{PROOF}
%\fixx{All the details in this proof}\\
%
Finally, the functor assigning to each signature the set of sentences is
defined as in definition~\ref{def:masen}:
\begin{definition}\label{de:form}
The sentences functor $\fu{Sen_{\inst{MA}}}: \Sign\to\Set$ is given by:
\begin{itemize}
\item objects: $\fu{Sen_{\inst{MA}}}(\Sigma) =$ {the set of all $\Sigma$
formulae} (def.~\ref{def:masen})
\item $\fu{Sen_{\inst{MA}}}(\mu: \Sigma \to \Sigma') = \fu{Sen_{\inst{MA}}}_{\mu}:\fu{Sen_{\inst{MA}}} \to \fu{Sen_{\inst{MA}}}(\Sigma')$ { defined by}:
		\begin{enumerate}
		\item $\fu{Sen_{\inst{MA}}}_{\mu}(t \eleq  t') = {\tilde{\mu}}(t) \eleq {\tilde{\mu}}(t')$
		\item $\fu{Sen_{\inst{MA}}}_{\mu}(t \prec  t') = {\tilde{\mu}}(t) \prec {\tilde{\mu}}(t')$
		\item $\fu{Sen_{\inst{MA}}}_{\mu}(a_1, \ldots ,a_n \to b_1, \ldots ,b_n) \\ = 
 \fu{Sen_{\inst{MA}}}_{\mu}(a_1), \ldots ,\fu{Sen_{\inst{MA}}}_{\mu}(a_n) \to \fu{Sen_{\inst{MA}}}_{\mu}(b_1), \ldots ,\fu{Sen_{\inst{MA}}}_{\mu}(b_n)$
		\end{enumerate}
	\end{itemize}
\end{definition}
%
With these definitions, lemma \ref{th:masat} yields the following:
\begin{fact}\label{fa:mainst}
The multialgebras form the institution $\inst{MA}$ with:
	\begin{itemize}
	\item the category \Sign\ as signatures (def.~\ref{de:Sign}),
	\item the model functor $\fu{Mod_{\inst{MA}}}$ (def.~\ref{de:MAlg}),
	\item the sentence functor $\fu{Sen_{\inst{MA}}}$ (def.~\ref{de:form}),
	\item $\models$ from def.~\ref{de:sat} as the satisfaction relation.
	\end{itemize}
	\end{fact}


\subsection{$\inst{MA}$ is an exact institution}
\label{se:exact}
It is well known that the category of algebraic signatures is co-complete,
see e.g. \cite{fundamental1}, where it also is proved (by use of comma
categories) that the model functor for classical total algebras is
continuous. 
Using the constructions of the required
co-limits of algebraic signatures from \ref{sub:sign}, we now show that
that $\inst{MA}$ is an exact institution -- that
 the model functor for multialgebras
$\fu{Mod_{\inst{MA}}}:\cat{Sign}^{op} \to \cat{Cat}$ is continuous, i.e. it
maps finite co-limits in $\cat{Sign}$ into limits in $\cat{Cat}$. (Note that this
is different from showing that the category $\Mod_{\inst{MA}}(\Sigma)$ has all
limits (resp. co-limits), which is proved in \cite{catrel}.)
%
\begin{lemma}
\label{le:initialtofinal}
The model of the empty signature is the unit category, that is final in $\cat{Cat}$.
\end{lemma}
\begin{PROOF}
The model of the empty signature is an algebra $\emptyset$ with no carrier (and no
operations), i.e. $\Mod_{\inst{MA}}(\Sigma_{\emptyset}) = \{ \emptyset \}$.
There is only one function (homomorphism) $h: \emptyset \to \emptyset$ -- the identity
homomorphism. This is obviously a final object in \cat{Cat}.
\end{PROOF}


\begin{lemma}\label{le:sumtoprod}
\Mod\ sends sums to products:
\[\xymatrix@C=0.5cm@R=0.3cm{
\Sigma \ar[ddr]_{\iota_{\Sigma}} && \Sigma' \ar[ddl]^{\iota_{\Sigma'}} &&&&
   \Mod_{\inst{MA}}(\Sigma) && \Mod_{\inst{MA}}(\Sigma')	\\ && &\ar@{=>}[rrr]^{\Mod}&&& \\
& {\Sigma + \Sigma'} & &&&&	
   & \Mod_{\inst{MA}}(\Sigma + \Sigma') \ar[uul]^{|_{\iota_{\Sigma}}} \ar[uur]_{|_{\iota_{\Sigma'}}}	
								}
\]
\end{lemma}
\begin{PROOF}
Since the reduct $\_|_\mu$ is a functor for any signature morphism $\mu$ the
diagram to the right is a cone in \cat{Cat} -- 
%\[\xymatrix{
%& \Mod_{\inst{MA}}(\Sigma + \Sigma') \ar[dl]_{|_{\iota_{\Sigma}}} \ar[dr]^{|_{\iota_{\Sigma'}}}	\\
%\Mod_{\inst{MA}}(\Sigma)&& \Mod_{\inst{MA}}(\Sigma')	\\
%}
%\]
we have to show that it %$\Mod_{\inst{MA}}(\Sigma + \Sigma')$ 
is a product
cone.
% of $\Mod_{\inst{MA}}(\Sigma)$ and $\Mod_{\inst{MA}}(\Sigma')$ in $\cat{Cat}$. 

Suppose that $(\cat{C}, \fu{F}:\cat{C} \to
\Mod_{\inst{MA}}(\Sigma),\fu{G}:\cat{C} \to \Mod_{\inst{MA}}(\Sigma'))$ is a
cone in $\cat{Cat}$.  
\[\xymatrix{
& \cat{C} \ar@/_/[ddl]_{\fu{F}} \ar@/^/[ddr]^{\fu{G}} \ar@{-->}[d]^{!u_{(\fu{F},\fu{G})}} \\
& \Mod_{\inst{MA}}(\Sigma + \Sigma') \ar[dl]^{|_{\iota_{\Sigma}}} \ar[dr]_{|_{\iota_{\Sigma'}}}	\\
\Mod_{\inst{MA}}(\Sigma) && \Mod_{\inst{MA}}(\Sigma')
}
\]
Given two multialgebras $A\in\Mod_{\inst{MA}}(\Sigma)$
and  $A'\in\Mod_{\inst{MA}}(\Sigma')$, we get an algebra $A\oplus A'\in
\Mod_{\inst{MA}}(\Sigma+\Sigma')$ by taking the $\Sigma$-part from $A$ and
the $\Sigma'$-part from $A'$ -- it is defined as follows: 
 for any sort symbol $s\in\Sigma:s^{A\oplus A'}=s^A$ (and $s\in\Sigma':s^{A\oplus
 A'}=s^{A'}$), and  for any operation symbol $f\in\Sigma:f^{A\oplus A'}=f^A$ (and $f\in\Sigma':f^{A\oplus
A'}=f^{A'}$). This works because $\Sigma+\Sigma'$ is disjoint union.

Likewise, given a $\Sigma$-homomorphism $h:A \to B$ and a
$\Sigma'$-homomorphism  $h':A' \to B'$,  the $\Sigma \oplus \Sigma'$-homomorphism $h \oplus
h': A \oplus A' \to B \oplus B'$, is defined by $(h \oplus h')_s = h_s \mbox{ if
} s \in \Sigma$, and $h'_s$ otherwise (when $s\in\Sigma'$). 
%%
%%Taking the disjoint union (of the carrier sets and the
%%interpretation of the operations) of a $\Mod_{\inst{MA}}(\Sigma)$ algebra $A$
%%and a $\Mod_{\inst{MA}}(\Sigma')$ algebra $A'$ we get a
%%$\Mod_{\inst{MA}}(\Sigma + \Sigma')$ algebra $A \oplus A'$. 
%%Likewise the
%%disjoint union of a $\Sigma$ homomorphism $h:A \to B$ and a $\Sigma'$
%%homomorphism $h':A' \to B'$ is a $\Sigma \oplus \Sigma'$ homomorphism $h \oplus
%%h': A \oplus A' \to B \oplus B'$, defined by $(h \oplus h')_s = h_s \mbox{ if
%%} s \in \Sigma$, and $h'_s$ otherwise (when $s\in\Sigma'$). 
Then $\oplus$ yields the unique objects/morphisms satisfying:
\eq{\begin{array}{ccc}(A\oplus A')|_{\iota_{\Sigma}}=A & and & (A\oplus
A')|_{\iota_{\Sigma'}}=A'\\
(h\oplus h')|_{\iota_{\Sigma}}=h & and & (h\oplus
h')|_{\iota_{\Sigma'}}=h'
\end{array}
\label{eq:prod}
}
We define the functor $u_{(\fu{F},\fu{C})}: \cat{C} \to
\Mod_{\inst{MA}}(\Sigma + \Sigma')$ by $u_{(\fu{F},\fu{C})}({C})
= \fu{F}({C}) \oplus \fu{G}({C})$ (and analogously for morphisms in
\cat{C}). It
is a factorization, i.e., $\comp{u_{(\fu F,\fu
G)}}{|_{\iota_{\Sigma}}}=\fu F$, similarly for $|_{\iota_{\Sigma'}}$ and
$\fu G$.

$u_{(\fu F, \fu G)}$ is unique since each pair of algebras
(resp. homomorphisms)  
$A\in\Mod_{\inst{MA}}(\Sigma)$ and $A'\in\Mod_{\inst{MA}}(\Sigma')$, has a unique preimage
$(A\oplus A')\in\Mod_{\inst{MA}}(\Sigma + \Sigma')$, satisfying (\ref{eq:prod}).

Thus the image of a co-product diagram from \cat{Sign} is a product diagram in \cat{Cat}.
%I.e. a cone; $\cat{C}$ with functors $\fu{F}:\cat{C} \to \Mod(\Sigma),\fu{G}:\cat{C} \to \Mod(\Sigma')$, the arrow $u_{(\fu{F},\fu{C})}: \cat{C} \to \Mod(\Sigma) \times \Mod(\Sigma')$, defined by $u_{(\fu{F},\fu{C})}(\cat{C}) = \fu{F}(\cat{C}) \times \fu{G}(\cat{C})$ is the unique factorization arrow.
\end{PROOF}

%\vfill
%\pagebreak

\begin{lemma}\label{le:coequtoequ}
\Mod\ sends co-equalizers to equalizers.
\end{lemma}
\begin{PROOF}
Let $\mu,\nu:\Sigma \to \Sigma'$ be two morphisms in \cat{Sign} and  $\qu{\Sigma'}{\kernel{}}$,
	$\iota: \Sigma' \to \qu{\Sigma'}{\kernel{}}$ their
	co-equalizer. 
We have to show that
	$\Mod_{\inst{MA}}(\qu{\Sigma'}{\kernel{}})$, $|_{\iota}$
	is an equlizer for $|_{\mu}$ and $|_{\nu}$, in $\cat{Cat}$.
%If the following diagram is commuting for any signature $\Sigma''$ and signature morphism $\sigma''$,
So assume that for any $\sigma''$ such that
$\comp\mu{\sigma''}=\comp\nu{\sigma''}$, there is a unique $u_{\sigma''}$
with $\sigma''=\comp{\iota}{u_{\sigma''}}:$
\[\xymatrix@C=1.8cm{ %\label{di:signcoequlizer}
\Sigma \ar@/_/[r]_{\nu} \ar@/^/[r]^{\mu}
	& {\Sigma'} \ar[r]^{\iota} \ar[dr]_{\sigma''}
	& \qu{\Sigma'}{\kernel{}} \ar@{-->}[d]^{u_{\sigma''}} \\
&& \Sigma''
}
\]
Let $\cat{C}$ and $\fu{F}:\cat{C}\to\Mod_{\inst{MA}}(\Sigma')$ be arbitrary
in \cat{Cat} such that $\comp{\fu F}{|_\mu}=\comp{\fu{F}}{|_\nu}$. We have to show the
existence of a unique $u_{\fu{F}}$ satisfying
$\comp{u_{\fu{F}}}{|_{\iota}}=\fu F:$
\[\xymatrix@C=1.5cm{
\label{di:modequlizer}
	\cat{C} \ar[dr]^{\fu{F}}  \ar@{-->}[d]^{!u_{\fu{F}}} \\
	\Mod_{\inst{MA}}(\qu{\Sigma'}{\kernel{}}) \ar[r]_{|_{\iota}}
		& \Mod_{\inst{MA}}(\Sigma') \ar@/_/[r]_{|_\nu} \ar@/^/[r]^{|_\mu}
			& \Mod_{\inst{MA}}(\Sigma) 	\\
							}
\]
1. First, we show that $\comp{|_\iota}{|_\mu}=\comp{|_\iota}{|_\nu}$, i.e. $\Mod_{\inst{MA}}(\qu{\Sigma'}{\kernel{}})$ is a cone.

Let $A''\in\Mod_{\inst{MA}}(\qu{\Sigma'}{\kernel{}})$ be arbitrary and let $s$ be any
sort symbol in $\qu{\Sigma'}{\kernel{}}$. We have $s^{(A''|_{\iota})|_{\mu}}
= \mu(s)^{(A''|_{\iota})}= \iota(\mu(s))^{A''}$, and similarly $s^{(A''|_{\iota})|_{\nu}}
= \nu(s)^{(A''|_{\iota})}= \iota(\nu(s))^{A''}$. But since $\iota$ is
equalizing $\mu$ and $\nu$, we have $ \iota(\mu(s))= \iota(\nu(s))$, so that
$\iota(\mu(s))^{A''} =  \iota(\nu(s))^{A''}$. In the same way, we show the
equality $\iota(\mu(\omega))^{A''} =  \iota(\nu(\omega))^{A''}$ for any
operation symbol $\omega\in\qu{\Sigma'}{\kernel{}}$. 
\\[1ex]
2. We now show that $\Mod_{\inst{MA}}(\qu{\Sigma'}{\kernel{}})$ is a limit cone.
\\[1ex]
a. Given an $A'\in\Mod_{\inst{MA}}(\Sigma')$ with $A'|_\mu=A'|_\nu$ we construct an
$A^-\in\Mod_{\inst{MA}}(\qu{\Sigma'}{\kernel{}})$ satisfying $A^-|_\iota=A'$.

Since $\iota$ is surjective, any symbol $x''\in\qu{\Sigma'}{\kernel{}}$ is in
its image, i.e., $x''=\iota(x')$ for some $x'\in\Sigma'$. We then let
$\iota(x')^{A^{-}}=x'^{A'}$. This is in fact well defined algebra. For suppose
that there are two different symbols $s'\not= t'\in\Sigma'$ such that
$\iota(s')=\iota(t')=x''$. Then, from the construction of $\iota$ and
$\qu{\Sigma'}{\kernel{}}$, we know that there exists an $x\in\Sigma$ such that
$s'=\mu(x)$ and $t'=\nu(x)$. But then, since 
$A'|_\mu=A'|_\nu$, we obtain the middle of the following equalities:
$s'^{A'}=x^{A'|_{\mu}}=x^{A'|_{\nu}}=t'^{A'}$. Thus
$\iota(s')=\iota(t')\To s'^{A'}=t'^{A'}$, and $A^{-}$ is well defined.

Obviously, $A^{-}|_\iota = A'$, since for each symbol $x'\in\Sigma'$ we have
$x'^{A^{-}|_{\iota}} = \iota(x')^{A^{-}} = x'^{A'}$.
\\[1ex]
b. In fact $A^{-}$ is the unique $\qu{\Sigma'}{\kernel{}}$-algebra satisfying
   $A^{-}|_\iota=A'$. 

For if $B''\in\Mod_{\inst{MA}}(\qu{\Sigma'}{\kernel{}})$ is such that
   $B''|_\iota=A'$ then we must have for any symbol
   $\iota(x')=x''\in\qu{\Sigma'}{\kernel{}}: x''^{B''}=\iota(x')^{B''} =
   x'^{B|_{\iota}} = x'^{A'} = x'^{A^{-}|_{\iota}} = \iota(x')^{A^{-}}=
   x''^{A^{-}}$, i.e., $B''=A^{-}$.
\\[1ex]
c. We extend the definition of $A^-$ from 2a. to homomorphisms between the
respective objects. That is, for a (homo)morphism $h':A'\to B'$ in $\Mod_{\inst{MA}}(\Sigma')$ where both
$A'|_\mu=A'|_\nu$ and $B'|_\mu=B'|_\nu$, $h^-:A^-\to B^-$ is a (homo)morphism
in $\Mod_{\inst{MA}}(\qu{\Sigma'}{\kernel{}})$ defined by $h^-_{\iota(s')}(a)=h'_{s'}(a)$
for all sort symbols $\iota(s')=s''\in\qu{\Sigma'}{\kernel{}}$ and $a\in
s'^{A'}$. Obviously, this is a unique $h^-$ such that $h|_\iota = h'$.
\\[1ex]
d. Now, given an $\fu F:\cat C\to \Mod_{\inst{MA}}(\Sigma')$ satisfying $\comp{\fu
F}{|_\mu}=\comp{\fu F}{|_\nu}$, we define $u_{\fu F}:\cat
C\to \Mod_{\inst{MA}}(\qu{\Sigma'}{\kernel{}})$:
\begin{itemize}\MyLPar
\item for any object $C\in\cat C:u_{\fu F}(C) = (\fu F(C))^-$ (as in point 2a.)
\item for any morphism $h:C\to D\in \cat C: u_{\fu F}(h) = (\fu F(h))^-$ (as
in point 2c.)
\end{itemize}
It is trivial to verify that $u_{\fu F}$ is a functor and, indeed, one that
makes $\comp{u_{\fu F}}{|_\iota}=\fu F$. By uniqueness of $A^-$ and $h^-$,
this is also a unique functor satisfying this equality.
\end{PROOF}
%
%%%%%%% OLD and insufficient:
%%%First we show that the objects of $\Mod_{\inst{MA}}(\qu{\Sigma'}{\kernel{}})
%%%\simeq \{ A' \in \Mod_{\inst{MA}}(\Sigma'): A'|_{\mu}= A'|_{\nu} \}$.
%%%Suppose that $A' \in
%%%\Mod_{\inst{MA}}(\Sigma')$ and $A'|_{\mu}= A'|_{\nu}$. 
%%%This means that $\mu(s)^{A'} = \nu(s)^{A'}$, for all
%%%$s \in \Sigma$ and that $\mu(\omega)^{A'} = \nu(\omega)^{A'}$, for all $\omega \in
%%%\Sigma$. 
%%%
%%%We construct an $A''\in \Mod_{\inst{MA}}(\qu{\Sigma'}{\kernel{}})$ by
%%%letting: 
%%%$\iota(s')^{{A''}}= s'^{A'}$, for all $s'
%%%\in \Sigma'$ and ${\iota}(\omega')^{{A''}}=
%%%%%%\omega'^{A'}$, for all $\omega' \in \Sigma'$. %, where
%%%Since $\iota$ is surjective, this indeed defines a
%%%$\qu{\Sigma'}{\kernel{}}$-algebra. 
%%%
%%So $A'$ is isomorphic to a multialgebra
%%${A''}$ in $\Mod_{\inst{MA}}(\qu{\Sigma'}{\kernel{}})$, where
%%
%%${\iota}[s']^{{A''}}= s'^{A'}$, for all $s'
%%\in \Sigma'$ and ${\iota}[\omega']^{{A''}}=
%%\omega'^{A'}$, for all $\omega' \in \Sigma'$. %, where
%%${\iota}(\nu(s))^{{A''}}= \nu(s)^{A'} =
%%\mu(s)^{A'} = {\iota}(\mu(s))^{{A''}}$ and
%%${\iota}(\nu(\omega))^{{A''}}=
%%\nu(\omega)^{A'} = \mu(\omega)^{A'} =
%%{\iota}(\mu(\omega))^{{A''}}$.
%%
%%The other way around; suppose that $\qu{A'}{\kernel{}} \in
%%\Mod_{\inst{MA}}(\qu{\Sigma'}{\kernel{}})$, then is
%%$\qu{A'}{\kernel{}}|_{\iota_{\kernel{}}} \in \Mod_{\inst{MA}}(\Sigma')$ and
%%since ${\iota_{\kernel{}}}(\mu(s)) = {\iota_{\kernel{}}}(\nu(s))$, for all $s
%%\in S$ is $\nu(s)^{\qu{A'}{\kernel{}}|_{\iota_{\kernel{}}}} =
%%\mu(s)^{\qu{A'}{\kernel{}}|_{\iota_{\kernel{}}}}$, for all $s$ and simmilary
%%we get that $\nu(\omega)^{\qu{A'}{\kernel{}}|_{\iota_{\kernel{}}}} =
%%\mu(\omega)^{\qu{A'}{\kernel{}}|_{\iota_{\kernel{}}}}$ for all $\omega \in
%%\Omega$, i.e. $\qu{A'}{\kernel{}}|_{(\comp{\mu}\iota_{\kernel{}})} =
%%\qu{A'}{\kernel{}}|_{(\comp{\nu}{\iota_{\kernel{}}})}$
%%
%%We have to show that the morphisms of
%%$\Mod_{\inst{MA}}(\qu{\Sigma'}{\kernel{}}) \simeq \{ h' \in
%%\Mod_{\inst{MA}}(\Sigma'): |_{\mu}(h')= |_{\nu}(h') \}$. Suppose that $h': A'
%%\to B'$ and that $h'|_{\mu} = h'|_{\nu}$, it means that $h'_{\mu(s)} =
%%h'_{\nu(s)}$, for all $s$, so $h'$ is isomorph to a homomorphism
%%$\qu{h'}{\iota_{\kernel{}}}:\qu{A'}{\iota_{\kernel{}}} \to
%%\qu{B'}{\iota_{\kernel{}}}$
%%
%%The other way is similar as for algebras.
%%
%%So given a cone $\cat{C},\fu{F}:\cat{C} \to \Mod_{\inst{MA}}(\Sigma')$ we
%%	define the factorization arrow $\fu{u_{F}}:\cat{C} \to
%%	\Mod_{\inst{MA}}(\qu{\Sigma'}{\kernel{}})$ by: \begin{itemize} \item
%%	Objects: $\fu{u_{F}}(O)= \fu{{F}}(O)$ \item Morphisms:
%%	$\fu{u_{F}}(M)= \fu{{F}}(M)$ \end{itemize} Since
%%	$\comp{\fu{F}}{|_{\mu}} = \comp{\fu{F}}{|_{\nu}}$ by asumption we get
%%	by definition that $\fu{{F}} =
%%	\comp{\fu{u_{F}}}{|_{\iota_{\kernel{}}}}$. $\fu{u_{F}}$ is unique
%%	since $\Mod_{\inst{MA}}(\qu{\Sigma'}{\kernel{}}) \simeq \{ A' \in
%%	\Mod_{\inst{MA}}(\Sigma'): |_{\mu}(A')= |_{\nu}(A') \}$.
%
Summing upp we get the following result.
\begin{proposition}
$\Mod_{\inst{MA}}$ is a finitely continuous functor.
\end{proposition}
\begin{PROOF}
We have that $\Mod_{\inst{MA}}$ is finitely continous on signatures by
lemma~\ref{le:initialtofinal}, lemma~\ref{le:sumtoprod} and
lemma~\ref{le:coequtoequ}. So the result follows by teorem~\ref{teo:reflects}.
\end{PROOF}


\begin{corollary}
$\inst{MA}$ is an exact institution.
\end{corollary}


\begin{corollary}
The amalgamation lemma holds for $\inst{MA}$.
\end{corollary}