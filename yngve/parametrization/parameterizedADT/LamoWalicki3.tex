\documentclass{llncs}
\begin{document}
\title{Specification of parameterized data types}

\author{Yngve Lamo\inst{1}
\and Micha{\l} Walicki\inst{2,3}}
\institute{Department of Engineering, Bergen University College\\ N-5020 Bergen, Norway\\
\email{yla@hib.no}
\and
 Department of Informatics, University of Bergen\\ N-5020 Bergen, Norway\\
	\email{michal@ii.uib.no}\\*[1ex]
\and
The authors gratefully acknowledges
the financial support received from\\
the Norwegian Research Council, NFR.
}
\maketitle

We revisit the concept of persistent functor (used in \cite{paramADJ,Alge,paramGanz,CATs}) pointing out its (well-known)
limitations (see e.g. \cite{para}) for the purpose of describing the semantics of specifications of
parameterized data types (PDTs). We introduce a more general notion of a semantic 
functor which requires only that the parameter algebra is a
subalgebra of
its image. We illustrate the flexibility and advantages of the proposed
construction by examples.

The main part of the presentation concerns then the syntactic restrictions on
the specifications which allows one to define the semantics of PDTs in this
way. One obtains the possibilities to:
\begin{enumerate}
\item preserve the carriers of the parameter algebras (corresponding to
classical persistency), or
\item extending (some) carriers of the parameter algebras.
\end{enumerate}
The later situation means that one can, typically, use free functor
semantics, (see \cite{Alge}). In this case, one also has two further options:
\begin{itemize}
\item[2a.] either to restrict the validity of the axioms from the parameter
specification to apply only to the ``old'' elements (from the carriers of the
parameter algebras),
\item[2b.] or to extend them to apply also to the ``new'' elements added to
the carrier.
\end{itemize}
The former case applies typically in situations when  a (carrier of a) data
type is extended
with special kind of elements (like ``error'' values), while the latter applies in
situations where the added elements are ``essentially'' of the same
kind (e.g., when extending a monoid to a group by adding necessary inverse
elements).

We show the counterparts of the classical vertical and horisontal composition
theorems, \cite{Alge} and identify a general concept of refinement of PDTs which amounts
not to
model class inclusion but to introduction of additional structure into the
specified data types. We thus obtain a specific case of ``architectural
specifications'' from CASL, \cite{CASL} and of ``constructor implementations'', \cite{para,para1}.
We suggest possible advantages of specializing
these  general concepts in the presented way.

A brief comparison with related work is given. A similar idea of
allowing extension of the parameter algebras was
introduced by Axel Poigne in \cite{paramPoigne}. The differences
concern the fact that \cite{paramPoigne} uses order sorted
algebras to identify the parameter part, while we use predicates for the same purpose.
This gives more flexibility -- for instance,
the case 2b above is excluded in \cite{paramPoigne}.
Our framework is more general, in
fact, the
sufficient conditions for application of the presented ideas are that one
is working in a
semi-exact institution (the presence of amalgamation lemma, \cite{Alge}) and that
signatures can express predicates (plus minor technicalities).
We also discuss our work with CASL, \cite{CASL}
specifications, but emphasize the possibilities for applications under
these general assumptions.


\bibliography{biblo}
\bibliographystyle{plain}
%If you want to include some references in the abstract then do
% include all that are relevant - not only one! (Given the text,
% it would be natural to include at least references to Sannella/Tarlecki,
% Ganzinger, Erig (persistent functors) -- you may pick most of them
%from the NJC paper (or TechRep. om parametrisering?)
%
\end{document}, 