
\setcounter{section}{3}
{\Large{Yngve!}} -- read what you have written, ask yourself what is wrong,
try to fix it, and then read the (final?!?!) version which is in TRparam.tex
(with sub31NEW.tex).

\subsection{Signatures with sort constants}\label{sub:gsp}
We start by modifiying the concept of signature and specification. The idea
is that each signature may have, in addition to the standard set of sort and
operation symbols, a (possibly empty) set of distinguished (sort and subsort) constant
symbols $\allcons = \Cons\cup\subcons$. 
The set $\Cons$ %=\{\cons_s:\to s: s\in\Sorts\}$ 
contains  constant symbols
$\cons_s$ for various sort symbols $s$ -- the intention of $\cons_s$ is to denote all the elements of the respective
sort $s$. 
The set $\subcons$  may contain additional constants
which will represent various subsorts -- the constants from this set are
called ``subsort constants''.
\begin{definition}
\label{def:modsign}
A signature with sort constants, $\Sigma_\cons$, is a triple $\Sigma_\cons =
 (\Sorts,\Ops,\allcons)$, where $\Sigma=(\Sorts,\Ops)$ is an ordinary
 signature and $\allcons=\Cons\cup\subcons$ is a (possibly empty) set of additional constants,
 $\allcons\cap\Ops=\emptyset$ and $\Cons\cap\subcons=\emptyset$. 
\end{definition}
%
Signatures with sort constants will be used merely as a syntactic representation of 
ordinary signatures. This is possible in multialgebraic setting since
constants may denote sets of elements. (In a traditional setting, one would
have to represent the sort constants, for instance, by predicates or (sub)sort symbols.)

We allow the set $\allcons$ to be empty. Also, we allow the set
$\Cons$ to contain several distinct constants of the same sort (although
their intended meaning will be the same).
The technical reasons for that come up in relating construction of co-limits
(proposition~\ref{prop:finco}, especially, lemma~\ref{fa:coeq}) and, in particular, pushouts
(subsection~\ref{sub:canpush}). For the most, we think of the set
$\Cons$ as containing one constant for each sort, i.e., as
$\Cons=\{\cons_s:\to s : s\in\Sorts\}$. Most relevant constructions will
involve and yield such signatures. (in general, we use the symbol $\cons_s$
for an arbitrary constant from $\Cons$ of sort $s$, e.g., for a signature
morphism $\mu$, 
$\mu(c)\not=\cons_s$ means the same as $\mu(c)\not\in\Cons$.)


We will use the following operations relating the signatures with sort
 constants to ordinary signatures.

\begin{definition}\label{def:sigops}
Given a signature with sort constants $\Sigma_\cons=(\Sorts,\Ops,\allcons)$ we let:
\begin{enumerate}\MyLPar
\item $\under{\Sigma_\cons}=(\Sorts,\Ops \cup \allcons)$
i.e. $(\Sorts, \Ops \cup \Cons \cup \subcons)$ -- the underlying signature
\item ${\Sigma_\weak}=(\Sorts,\Ops,\subcons)$ i.e. $\Sigma_\cons
\setminus \Cons$ -- the reduced signature
\item ${\Sigma}=(\Sorts,\Ops)$ i.e. $(\Sigma_\cons
\setminus \allcons)$ -- the standard (part of the) signature
\end{enumerate}
The other way around, given an ordinary signature $\Sigma=(\Sorts,\Ops)$, we let
\begin{enumerate}\MyLPar\setcounter{enumi}{3}
\item ${\Sigma_\cons}=(\Sorts,\Ops,\Cons)$
, where   $\Cons=\{\cons_s:\to s:s\in\Sorts\}$ and $\Cons\cap\Ops=\emptyset$
-- the corresponding guarded signature with sort constants.
\item\label{it:ins} $\ins\Sigma=(\Sorts,\Ops,\emptyset)$
, -- the included signature (with sort constants).
\end{enumerate}
\end{definition}
%
Unless stated otherwise, the signatures considered will always be signatures with sort constants. 
Given an arbitrary specifcation $\thr{SP}$ we will sometimes write
$\Sigma(\thr{SP})$ to denote its signature.

\begin{definition}
A morphism between signatures with sort constants $\mu:\Sigma_\cons \to
\Sigma'_{\cons}$ is a signature morphism between the underlying
signatures $\mu:\under{\Sigma_\cons}\to\under{\Sigma'_\cons}$, sending
$\allcons$ to ${\allcons}'$ and $\Sigma$ to $\Sigma'$.
\end{definition}
In other words, the $\Cons$-constants need not be sent to $\Cons'$-constants
but may be mapped to subsort constants ${\subcons}'$, as well.

\begin{fact}
The signatures with sort constants form a category $\cat{Sign_\cons}$, with
the identity function as identity and function composition as composition.
\end{fact}
%
Since the signatures with sort constants essentially use underlying signature
morphism, the transformation from the former to the latter
%signatures with sort constants to
%underlaying signatures 
can be extended to a functor $\fu{_-}:\cat{Sign_\cons} \to
\cat{Sign}$, which is the (pointwise) identity on the signature morphisms. 

The other way arround the inclusion of signatures (point~\ref{it:ins} of
def.~\ref{def:sigops}) can also be extended to a
functor $\fu{\ins\ }:\cat{Sign} \to \cat{Sign_\cons}$, which sends $\Sigma$
to $\ins\Sigma$ and is the identity on morphisms. The following fact says
that $\cat{Sign}$ can be treated as a full subcategory of $\cat{Sign_\cons}$.

\begin{fact}\label{fa:fullsub}
$\ins\ :\cat{Sign} \to \cat{Sign_\cons}$ is full and faithfull.
\end{fact}
We have that for any $\Sigma\in\cat{Sign}:\Sigma=\under{(\ins\Sigma)}$ but,
in general, for $\Sigma_\cons\in\cat{Sign_\cons}: \Sigma_\cons\not\iso \ins{(\under{\Sigma_\cons})}$.
\begin{lemma}
$\fu{_-}:\cat{Sign_\cons} \to \cat{Sign}$ is a left adjoint to $\ins\ :\cat{Sign} \to \cat{Sign_\cons}$.
\end{lemma}
\begin{PROOF}
Suppose that $\sigma: \Sigma_\cons \to \ins{(\Sigma')}$ is a
$\cat{Sign_\cons}$ morphism, then $\under{\sigma}: \under{\Sigma_\cons} \to
\under{\ins{(\Sigma')}}$ (note that $\under{\ins{(\Sigma')}} = \Sigma'$) is
the unique factorization morphism, since $\ins{\under{\sigma}}=
\under{\sigma}$ and $\ins\ :\cat{Sign} \to \cat{Sign_\cons}$ is full and
faithfull. 

{\Large{WRONG!!!}} -- 
What is the unit of this supposed adjunction?? 
%If $\Sigma_\cons$ has
%non-empty set of (sub)sort constants $\allcons$, then there is no
%$\cat{Sign_\cons}$-morphism $\Sigma_\cons\to\ins{(\under{\Sigma_\cons})}$,
%because the latter has no (sub)sort constants!!!
\end{PROOF}
{\Large{Also}}, there is no adjunction other way around, i.e., neither
$\ins{\ }$ is left adjoint to $\under{\ }$.
Perhaps, the fact that $\cat{Sign}$ is a full subcategory of
$\cat{Sign_\cons}$ (fact~\ref{fa:fullsub}), might be used, but I don't know...

\begin{corollary}
The functor $\fu{_-}$ preserves co-limits.
\end{corollary}
%
The following proposition shows that
 that (finite) co-limits in $\cat{Sign}_\cons$
can be obtained from the respective co-limits in $\cat{Sign}$. 

\begin{proposition}\label{prop:finco}
The functor $\fu{_-}:\cat{Sign_\cons} \to \cat{Sign}$ reflects finite co-limits.
\end{proposition}
%
To prove the proposition we show that the functor reflects initial object, sums and co-equlizers.

\begin{lemma}
The functor $\under{\ }$ reflects and preserves initial objects.
\end{lemma}
%
\begin{PROOF}
The empty signature $\Sigma^\emptyset=(\emptyset,\emptyset)$ is {\em the}
initial object in 
$\cat{Sign}$, and the empty signature
$\Sigma_\cons^\emptyset=(\emptyset,\emptyset,\emptyset)$ is {\em the} initial
object in
$\cat{Sign_\cons}$.
But $\Sigma^\emptyset=\under{\Sigma_\cons^\emptyset}$, so initial object is
both reflected and preserved.
\end{PROOF}%\vspace*{-2ex}
%
\begin{lemma} The functor $\under{\ }$ reflects and preserves sums (binary co-products).
\end{lemma}
%
\begin{PROOF}
Supose that $\sigma:\Sigma_\cons \to \Sigma''_\cons, \sigma':\Sigma'_\cons
\to \Sigma''_\cons$ is a co-cone in $\cat{Sign_\cons}$, if
$\under{\sigma}:\under{\Sigma_\cons}\to \under{\Sigma''_\cons},
\under{\sigma'}:\under{\Sigma'_\cons} \to \under{\Sigma''_\cons}$ is a
co-limit (sum) of the co-cone in $\cat{Sign}$, we have to show that
${\Sigma''_\cons}$ is a co-limit of the co-cone in $\cat{Sign_\cons}$. 

We have that there is an isomorphism $\tau:\under{\Sigma''_\cons} \to
\under{\Sigma_\cons} + \under{\Sigma'_\cons}$, where $\under{\Sigma_\cons} +
\under{\Sigma'_\cons}$ is the standard choiche of sum object for algebraic
signatures, i.e. the disjoint union of sorts and operations of
$\under{\Sigma_\cons}$ and $\under{\Sigma'_\cons}$, this mean that for every
other co-cone $\nu:\Sigma_\cons \to \Sigma'''_\cons, \nu':\Sigma'_\cons \to
\Sigma'''_\cons$ in $\cat{Sign_\cons}$, can we define the unique
factorization arrow $\mu_{\nu,\nu'}: \under{\Sigma''_\cons} \to
\under{\Sigma'''_\cons}$ by: $\mu_{\nu,\nu'}= \comp{\tau}{u_{\nu,\nu'}}$,
where $u_{\nu,\nu'}: \under{\Sigma_\cons} + \under{\Sigma'_\cons} \to 
\under{\Sigma'''_\cons}$ is (the standard mediator) defined by: for any symbol
$x\in\under{\Sigma_\cons} + \under{\Sigma'_\cons}: u_{\nu,\nu'}(x) = \left\{\begin{array}{ll}
		\nu(x) & {\rm if\ } x \in \under{\Sigma_\cons} \\ 
		\nu'(x) & {\rm if\ } x \in \under{\Sigma'_\cons}\\
 \end{array}\right.$.

Since $\nu,\nu'$ is $\cat{Sign_\cons}$ morphisms is the compositions
$\sigma;\tau;u_{\nu,\nu'}$ and $\sigma';\tau;u_{\nu,\nu'}$ $\cat{Sign_\cons}$
morphisms. But since both $\sigma$ and $u_{\nu,\nu'}$ is $\cat{Sign_\cons}$
morphism has also $\tau$ to be a $\cat{Sign_\cons}$ morphisms. 
\end{PROOF}
%	 
%
\begin{lemma}\label{fa:coeq}
The functor $\under{\ }$ reflects and preserves co-equalizers. 
\end{lemma}
%

\begin{PROOF}
Suppose that $\mu,\nu:\Sigma_\cons \to \Sigma'_\cons, \sigma:\Sigma'_\cons
\to \Sigma''_\cons$, is a co-cone in $\cat{Sign_\cons}$ and that
$\under{\mu},\under{\nu}:\under{\Sigma_\cons} \to \under{\Sigma'_\cons},
\under{\sigma}:\under{\Sigma'_\cons} \to \under{\Sigma''_\cons}$, is a
co-limit (co-equalizer) of the cone in $\cat{Sign}$. 

We have that there is an isomorphism $\tau:\under{\Sigma''_\cons} \to
\qu{\under{\Sigma'_\cons}}{\kernel{}}$, where
$\qu{\under{\Sigma'_\cons}}{\kernel{}}$, is the standard choiche of
co-equalizer i.e. ${\kernel{}}$ is the least equivalence on
$\under{\Sigma'_\cons}$ induced by the 
relation with the following components:
 \begin{itemize}\MyLPar
\item Sorts: $\kernel{S'}= \{ \langle \mu(s),\nu(s) \rangle: s \in \under{\Sigma_\cons} \}$,
\item Operations $\kernel{\Omega'}= \{ \langle \mu(\omega), \nu(\omega)
\rangle : \omega \in \under{\Sigma_\cons} \}$\\	 
Specially this mean that for (Sub)sorts $\kernel{\allcons}= \{ \langle
\mu(s^\cons), \nu(s^\cons) \rangle : s^\cons \in \allcons \}$, since
$\under{{\allcons}} \subseteq \Omega'$, where  $\Omega'$ is the operations of
$\under{\Sigma_\cons}$ 
\end{itemize} 
So for every other co-cone $\mu,\nu:\Sigma_\cons \to \Sigma'_\cons,
\gamma:\Sigma'_\cons \to \Sigma'''_\cons$, in $\cat{Sign_\cons}$, we can
define the unique factorization arrow $\rho_\gamma$ by $\rho_\gamma=
\sigma;\tau;u_\gamma$, where $u_\gamma$ is (the standard mediator) defined
by:$u_\gamma([s'])=\gamma(s')$ and $u_\gamma([\omega'])=\gamma(\omega')$ and
by the same argument as in the proof for co-products we get that $\tau$ is a
$\cat{Sign_\cons}$ morphism 
\end{PROOF}
{\Large{Try to write these two proofs}} -- making them sufficient and, not
least, correct in english, before seeing the version in TRparam...


Since we can create all finite co-limits by initial objects, sums and
co-equlizers the proposition~\ref{prop:finco} follows from the above lemmata.
