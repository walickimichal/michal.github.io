\documentclass[12pt]{article}
%\documentstyle[a4,fullpage,12pt,oldlfont]{article}

%\usepackage{amssymb}

\makeatletter

\ifcase \@ptsize
    % mods for 10 pt
    \oddsidemargin  0.15 in     %   Left margin on odd-numbered pages.
    \evensidemargin 0.35 in     %   Left margin on even-numbered pages.
    \marginparwidth 1 in        %   Width of marginal notes.
    \oddsidemargin 0.25 in      %   Note that \oddsidemargin = \evensidemargin
    \evensidemargin 0.25 in
    \marginparwidth 0.75 in
    \textwidth 5.875 in % Width of text line.
\or % mods for 11 pt
    \oddsidemargin 0.1 in      %   Left margin on odd-numbered pages.
    \evensidemargin 0.15 in    %   Left margin on even-numbered pages.
    \marginparwidth 1 in       %   Width of marginal notes.
    \oddsidemargin 0.125 in    %   Note that \oddsidemargin = \evensidemargin
    \evensidemargin 0.125 in
    \marginparwidth 0.75 in
    \textwidth 6.125 in % Width of text line.
\or % mods for 12 pt
    \oddsidemargin -10 pt      %   Left margin on odd-numbered pages.
    \evensidemargin 10 pt      %   Left margin on even-numbered pages.
    \marginparwidth 1 in       %   Width of marginal notes.
    \oddsidemargin 0 in      %   Note that \oddsidemargin = \evensidemargin
    \evensidemargin 0 in
    \marginparwidth 0.75 in
    \textwidth 6.375 true in % Width of text line.
\fi

\voffset -2cm
\textheight 22.5cm

\makeatother

%\makeatletter
%\show\
%\makeatother
\newcommand{\ite}[1]{\item[{\bf #1.}]}
\newcommand{\app}{\mathrel{\scriptscriptstyle{\vdash}}}
\newcommand{\estr}{\varepsilon}
\newcommand{\PSet}[1]{{\cal P}(#1)}
\newcommand{\ch}{\sqcup}
\newcommand{\into}{\to}
\newcommand{\Iff}{\Leftrightarrow}
\renewcommand{\iff}{\leftrightarrow}
\newcommand{\prI}{\vdash_I}
\newcommand{\pr}{\vdash}
\newcommand{\ovr}[1]{\overline{#1}}

\newcommand{\cp}{{\cal O}}

% update function/set
%\newcommand{\upd}[3]{#1\!\Rsh^{#2}_{\!\!#3}} % AMS
\newcommand{\upd}[3]{#1^{\raisebox{.5ex}{\mbox{${\scriptscriptstyle{\leftarrow}}\scriptstyle{#3}$}}}_{{\scriptscriptstyle{\rightarrow}}{#2}}} 
\newcommand{\rem}[2]{\upd{#1}{#2}{\bullet}}
\newcommand{\add}[2]{\upd {#1}{\bullet}{#2}}
%\newcommand{\mv}[3]{{#1}\!\Rsh_{\!\!#3}{#2}}
\newcommand{\mv}[3]{{#1}\:\raisebox{-.5ex}{$\stackrel{\displaystyle\curvearrowright}{\scriptstyle{#3}}$}\:{#2}}

\newcommand{\leads}{\rightsquigarrow} %AMS

\newenvironment{ites}{\vspace*{1ex}\par\noindent 
   \begin{tabular}{r@{\ \ }rcl}}{\vspace*{1ex}\end{tabular}\par\noindent}
\newcommand{\itt}[3]{{\bf #1.} & $#2$ & $\impl$ & $#3$ \\[1ex]}
\newcommand{\itte}[3]{{\bf #1.} & $#2$ & $\impl$ & $#3$ }
\newcommand{\itteq}[3]{\hline {\bf #1} & & & $#2=#3$ }
\newcommand{\itteqc}[3]{\hline {\bf #1} &  &  & $#2=#3$ \\[.5ex]}
\newcommand{\itteqq}[3]{{\bf #1} &  &  & $#2=#3$ }
\newcommand{\itc}[2]{{\bf #1.} & $#2$ &    \\[.5ex]}
\newcommand{\itcs}[3]{{\bf #1.} & $#2$ & $\impl$ & $#3$  \\[.5ex] }
\newcommand{\itco}[3]{   & $#1$ & $#2$  & $#3$ \\[1ex]}
\newcommand{\itcoe}[3]{   & $#1$ & $#2$  & $#3$}
\newcommand{\bit}{\begin{ites}}
\newcommand{\eit}{\end{ites}}
\newcommand{\na}[1]{{\bf #1.}}
\newenvironment{iten}{\begin{tabular}[t]{r@{\ }rcl}}{\end{tabular}}
\newcommand{\ass}[1]{& \multicolumn{3}{l}{\hspace*{-1em}{\small{[{\em Assuming:} #1]}}}}

%%%%%%%%% nested comp's
\newenvironment{itess}{\vspace*{1ex}\par\noindent 
   \begin{tabular}{r@{\ \ }lllcl}}{\vspace*{1ex}\end{tabular}\par\noindent}
\newcommand{\bitn}{\begin{itess}}
\newcommand{\eitn}{\end{itess}}
\newcommand{\comA}[2]{{\bf #1}& $#2$ \\ }
\newcommand{\comB}[3]{{\bf #1}& $#2$ & $#3$\\ }
\newcommand{\com}[3]{{\bf #1}& & & $#2$ & $\impl$ & $#3$\\[.5ex] }

\newcommand{\comS}[5]{{\bf #1} 
   & $#2$ & $#3$ & $#4$ & $\impl$ & $#5$\\[.5ex] }

%%%%%%%%%%%%%%%%
\newtheorem{CLAIM}{Proposition}[section]
\newtheorem{COROLLARY}[CLAIM]{Corollary}
\newtheorem{THEOREM}[CLAIM]{Theorem}
\newtheorem{LEMMA}[CLAIM]{Lemma}
\newcommand{\MyLPar}{\parsep -.2ex plus.2ex minus.2ex\itemsep\parsep
   \vspace{-\topsep}\vspace{.5ex}}
\newcommand{\MyNumEnv}[1]{\trivlist\refstepcounter{CLAIM}\item[\hskip
   \labelsep{\bf #1\ \theCLAIM\ }]\sf\ignorespaces}
\newenvironment{DEFINITION}{\MyNumEnv{Definition}}{\par\addvspace{0.5ex}}
\newenvironment{EXAMPLE}{\MyNumEnv{Example}}{\nopagebreak\finish}
\newenvironment{PROOF}{{\bf Proof.}}{\nopagebreak\finish}
\newcommand{\finish}{\hspace*{\fill}\nopagebreak 
     \raisebox{-1ex}{$\Box$}\hspace*{1em}\par\addvspace{1ex}}
\renewcommand{\abstract}[1]{ \begin{quote}\noindent \small {\bf Abstract.} #1
    \end{quote}}
\newcommand{\B}[1]{{\rm I\hspace{-.2em}#1}}
\newcommand{\Nat}{{\B N}}
\newcommand{\bool}{{\cal B}{\rm ool}}
\renewcommand{\c}[1]{{\cal #1}}
\newcommand{\Funcs}{{\cal F}}
%\newcommand{\Terms}{{\cal T}(\Funcs,\Vars)}
\newcommand{\Terms}[1]{{\cal T}(#1)}
\newcommand{\Vars}{{\cal V}}
\newcommand{\Incl}{\mathbin{\prec}}
\newcommand{\Cont}{\mathbin{\succ}}
\newcommand{\Int}{\mathbin{\frown}}
\newcommand{\Seteq}{\mathbin{\asymp}}
\newcommand{\Eq}{\mathbin{\approx}}
\newcommand{\notEq}{\mathbin{\Not\approx}}
\newcommand{\notIncl}{\mathbin{\Not\prec}}
\newcommand{\notCont}{\mathbin{\Not\succ}}
\newcommand{\notInt}{\mathbin{\Not\frown}}
\newcommand{\Seq}{\mathrel{\mapsto}}
\newcommand{\Ord}{\mathbin{\rightarrow}}
\newcommand{\M}[1]{\mathbin{\mathord{#1}^m}}
\newcommand{\Mset}[1]{{\cal M}(#1)}
\newcommand{\interpret}[1]{[\![#1]\!]^{A}_{\rho}}
\newcommand{\Interpret}[1]{[\![#1]\!]^{A}}
%\newcommand{\Comp}[2]{\mbox{\rm Comp}(#1,#2)}
\newcommand{\Comp}[2]{#1\diamond#2}
\newcommand{\Repl}[2]{\mbox{\rm Repl}(#1,#2)}
%\newcommand\SS[1]{{\cal S}^{#1}}
\newcommand{\To}[1]{\mathbin{\stackrel{#1}{\longrightarrow}}}
\newcommand{\TTo}[1]{\mathbin{\stackrel{#1}{\Longrightarrow}}}
\newcommand{\oT}[1]{\mathbin{\stackrel{#1}{\longleftarrow}}}
\newcommand{\oTT}[1]{\mathbin{\stackrel{#1}{\Longleftarrow}}}
\newcommand{\es}{\emptyset}
\newcommand{\C}[1]{\mbox{$\cal #1$}}
\newcommand{\Mb}[1]{\mbox{#1}}
\newcommand{\<}{\langle}
\renewcommand{\>}{\rangle}
\newcommand{\Def}{\mathrel{\stackrel{\mbox{\tiny def}}{=}}}
\newcommand{\impl}{\mathrel\Rightarrow}
\newcommand{\then}{\mathrel\Rightarrow}
\newfont{\msym}{msxm10}

\newcommand{\false}{\bot}
\newcommand{\true}{\top}

\newcommand{\restrict}{\mathbin{\mbox{\msym\symbol{22}}}}
\newcommand{\List}[3]{#1_{1}#3\ldots#3#1_{#2}}
\newcommand{\col}[1]{\renewcommand{\arraystretch}{0.4} \begin{array}[t]{c} #1
  \end{array}}
\newcommand{\prule}[2]{{\displaystyle #1 \over \displaystyle#2}}
\newcounter{ITEM}
\newcommand{\newITEM}[1]{\gdef\ITEMlabel{ITEM:#1-}\setcounter{ITEM}{0}}
\makeatletter
\newcommand{\Not}[1]{\mathbin {\mathpalette\c@ncel#1}}
\def\LabeL#1$#2{\edef\@currentlabel{#2}\label{#1}}
\newcommand{\ITEM}[2]{\par\addvspace{.7ex}\noindent
   \refstepcounter{ITEM}\expandafter\LabeL\ITEMlabel#1${(\roman{ITEM})}%
   {\advance\linewidth-2em \hskip2em %
   \parbox{\linewidth}{\hskip-2em {\rm\bf \@currentlabel\
   }\ignorespaces #2}}\par \addvspace{.7ex}\noindent\ignorespaces}
\def\R@f#1${\ref{#1}}
\newcommand{\?}[1]{\expandafter\R@f\ITEMlabel#1$}
\makeatother
\newcommand{\PROOFRULE}[2]{\trivlist\item[\hskip\labelsep {\bf #1}]#2\par
  \addvspace{1ex}\noindent\ignorespaces}
\newcommand{\PRULE}[2]{\displaystyle#1 \strut \over \strut \displaystyle#2}
%\setlength{\clauselength}{6cm}
%% \newcommand{\clause}[3]{\par\addvspace{.7ex}\noindent\LabeL#2${{\rm\bf #1}}%
%%   {\advance\linewidth-3em \hskip 3em
%%    \parbox{\linewidth}{\hskip-3em \parbox{3em}{\rm\bf#1.}#3}}\par 
%%    \addvspace{.7ex}\noindent\ignorespaces}
\newcommand{\clause}[3]{\par\addvspace{.7ex}\noindent
  {\advance\linewidth-3em \hskip 3em
   \parbox{\linewidth}{\hskip-3em \parbox{3em}{\rm\bf#1.}#3}}\par 
   \addvspace{.7ex}\noindent\ignorespaces}
\newcommand{\Cs}{\varepsilon}
\newcommand{\const}[3]{\Cs_{\scriptscriptstyle#2}(#1,#3)}
\newcommand{\Ein}{\sqsubset}%
\newcommand{\Eineq}{\sqsubseteq}%


\voffset -1cm

\input{xypic}
\xyoption{curve}

%--------------------------------------------------------------------
% the document

\begin{document} 

\hfill\today
% 
\section{Initiality in arbitrary languages}
\begin{Definition}
Atomic language $\langa$ is given by a set of binary relation symbols
$L_1,...,L_k$. Given a signature $\Sigma$ and a set of variables $X$, the
atomic formulae of $\langa$ relatively to $\Sigma,X$ form the set
$\langsx= \{L_i(t_1,t_2): L_i\in \langa\land t_1,t_2\in\XTerms\}$. 
% Equality is always implicitly in any $\langa$ but not among the 
% explicitly listed $L_1,...,L_k$.
\end{Definition}
$\langa$ comprises the set of {\em logical} (or primitive) relations with
uniform interpretation in all structures. If they are
axiomatisable, all structures have to satisfy the same axioms. But they may
be not axiomatisable, e.g., identity, in which case they are interpreted
uniformly in all $\langa$-structures. We do not want to deal with intricacies
of definability and will simply  assume that $\langa$ is given
an interpretation which is fixed in one way or another. 
%(We will state some necessary conditions for such interpretations but 
%not in terms of definability.) 
What makes an $A$ an $\langa$ structure is that it conforms to
this fixed interpretation.\\
. \comt{no $L_i\in\langa$ is definable from others in $\langa$ using
relational operators $\neg,\cap,\cup$...} 

Thus, $\Stra$ is a subset of standard (arbitrarily interpreted)
$\langa$-structures: tuples  $\<\car A,L_1^A,...,L_k^A\>$ where $\car
A$ is a (carrier) set (possibly empty), and each $L_i^A\subseteq\car A\times \car A$
satisfies the assumed interpretation. 
We always require $\Stra$ to be closed under $\langa$-isomorphisms, i.e.,
bijective mappings $h:A\into B$ such that $L^A(a_1,a_2) \iff L^B(h(a_1),h(a_2))$.
Equivalently, modeling relations as multifunctions 
$[L\subseteq X\times Y]\iso [L:X\into\PSet Y]$,  $A=\{
\<a,L_1(a)...L_k(a)\>:a\in\car A\}$ where $b \in L_i(a) \iff L_i^A(a,b)$.
By ``a language $\langa$'' we mean its fixed interpretation, i.e., the corresponding class $\Stra$.

An
$\langs$-structure is an $\langa$-structure which, in addition, contains
an interpretation $\Sigma^A$ of all symbols from $\Sigma$.
$\Strs$ denotes the class of all $\langs$-structures.

\begin{Definition}
Given $\Sigma$, $\lla S\subseteq \langa$ and two $\langs$-structures $A,B$, an
$\langss S$-homomorphism $\phi:A\into B$ is a total function
$\phi:|A|\into|B|$ satisfying:
\begin{enum}
\item $\phi(f^A(a_1...a_k)) = f^B(\phi(a_1)...\phi(a_k))$ for any
$a_i\in\car A$
\item for all $L_i\in \lla S$ and $a\in\car A: \phi(L_i^A(a)) \subseteq
L_i^B(\phi(a))$.
\end{enum}
\end{Definition}
Using only a subset $\lla S\subset \langa$ allows us to treat homomorphisms which do
not necessarily preserve all logical relations from
$\langa$. $\Sigma$-homomorphism is a $\langss\emptyset$-homomorphism. If
$\phi$ is an $\langs$-homomorphism, then it is also $\langss S$-homomorphism
for any $\lla S\subset \langa$.

\begin{Definition}
Given $\langa=\{L_1...L_k\}$ and $\Sigma$ define the term structure $\tsl$:
\begin{enum}
\item the carrier $\car{\tsl}=\GTerms$ -- the ground $\Sigma$-terms
\item for each $f\in\Funcs$ and $t_1...t_k\in\car{\tsl}:
f^{\scr\tsl}(t_1...t_k) = f(t_1...t_k)$
\item for each $t\in\car\tsl$ and $L_i\in\langa: L_i^{\scr\tsl}(t)=\emptyset$.
\end{enum}
\end{Definition}
$\tsl$ is initial {\em for} $\Strs$ with $\langs$-homomorphisms (in fact, any
$\langss S$-homomorphisms). $SP\subseteq\langsx$ is {\em consistent} iff 
$\Mod{SP}=\setsuch{A\in\Strs}{A\models SP}\not=\emptyset$.
%
\begin{Definition}
Given a consistent $SP\subseteq\langsx$, define $\tsp$ as follows:
\begin{enum}
\item the carrier $\car{\tsp} = \GTerms$ 
%$\quot{\tsl}\equiv$, where $t\equiv s\Iff \Mod{SP}\models s=t$
\item for each $f\in\Funcs$ and $t_1...t_k\in\car{\tsp}:
f^{\scr\tsp}(t_1...t_k) = f(t_1...t_k)$
\item for each $t\in\car{\tsp}$ and $L_i\in\langa:
L_i^{\scr\tsp}(t) = \{s:\Mod{SP}\models L_i(t,s)\}$.
\end{enum}
\end{Definition}
Notice that $\tsp$ is not the standard construction since it does not involve
quotient. 
But $\tsp$ need not be initial in $\Mod{SP}$ because it may happen that
$\tsp\not\in\Strs$ for other reasons (besides intended identity
interpretation). We first investigate these other reasons. 
To ensure $\tsp\in\Strs$ we need to require all relations to
be {\em strictly immediate}. 
Intuitively, a relation $R\subseteq X\times Y$ is {\em immediate} iff 
$R(x,y)$ never requires existence of other elements than $x$ and
$y$, or else iff $\{x\},\{y\}$ can be made into a structure
satisfying $R(x,y)$. 
Given a strict ordering $<$, the relation $R_{sep}(x_1,x_2)\Iff
\exists x: x_1<x\land x<x_2$ is not immediate. Similarly, the directedness relation
$R_{dir}(x_1,x_2)\Iff \exists x: x_1<x\land x_2<x$ is not immediate. 
The ordering $<$ itself, equivalence, membership, set-inclusion, on
the other hand, are immediate.

\begin{Definition} Define constructions on structures --
substructure, intersection, inverse: 
\begin{enumerate}
\item
A $\Sigma$-substructure of an $A\in\Strs$ is a structure $B=\<\car
B, L_1^B,...,L_k^B,\Sigma^B\>$ where $\car B\subseteq\car A$ is
$\Sigma$-closed,  $\forall f\in\Funcs, b_i\in\car B: f^B(b_i)=f^A(b_i)$, and $L_i^B=L_i^A\cap
\car B\times\car B$. We write $B\sqsubseteq_\Sigma A$ (possibly dropping
subscript.) \hfill -- $\Diamond(B\not\in\Stra)$.
\item
Given a set of $\langs$-structures $\{A_n:n\in I\}$ for some $I$, we 
let $\bigcap A_n$ denote the structure $\<\bigcap_n{\car
{A_n}},\bigcap_nL_1^{A_{n}},..., \bigcap_n L_k^{A_{n}},
\bigcap_n\Sigma^{A_{n}}\>$ \hfill -- $\Diamond(\bigcap A_n\not\in\Stra\cup\Str\Sigma)$.
\item
For a set $X$, $A\in\Stra$ and a function $f:X\into\car A$ define the
structure $\inv fA=\<X, L_1^{\inv fA},...,L_k^{\inv fA}\>$ by
$L_i^{\inv fA}(x_1,x_2)\iff L_i^{A}(f(x_1),f(x_2))$ \hfill -- $\Diamond(\inv fA\not\in\Stra)$.
\end{enumerate}
\end{Definition}
%For an $A\in\Strs$, a $B\sqsubseteq_\Sigma A$ may fail to be an
%$\langa$-structure. Also $\bigcap A_n$ may happen to be neither $\Sigma$- nor
%$\langa$-structure.
The first point below captures the intuition of immediacy but we will also need a
stronger property from point \ref{si}:
%\newpage
\begin{Definition}\label{def:immed}
\ 
\begin{enum}
\item\label{i:immed}
A language $\langa$ is {\em immediate}, \imm, iff for  any
 collection of $A_n\in\Stra: \bigcap A_n \in \Stra$.
\item\label{si}
A language $\langa$ is {\em strictly immediate}, \simm, iff for  any set $X$, collection of $A_n\in\Stra$
 and  functions $f_n:X\into \car{A_n} : \bigcap \inv{f_{n}}{A_{n}}\in\Stra$.
%\<X, \bigcap_n \langa^{\inv{f_{n}}{A_{n}}}\>\in\Stra$.
\end{enum}
\end{Definition}
%
\comt{Strict immediacy of $\langa$ is NOT more than strict immediacy of each of its
components.}\\
For instance, a (quasi, partial, total)  ordering is immediate but a
dense ordering $R_{dns}$ is not. To see this take two $R_{dns}$-structures: as $A_1$ the dense ordering of
rationals in the closed interval $\car{A_1}=[0,1]$ and as $A_2$ the same
ordering on this interval
but with all rationals within the open inteval $(0,1)$ replaced by new distinct
elements. $A_1\cap A_2=\<\{0,1\},\{\<0,0\>,\<0,1\>,\<1,1\>\}\>$ is still (quasi, partial, total) ordering but it
is not dense, i.e., it is not an $R_{dns}$-structure.
%
\begin{Prop}\label{cl:iff}
%If $\langa$ is immediate then
\begin{enum}
%$\{A_n\}\subseteq\Stra \impl \bigcap A_n\in\Stra$.
%\item for any set $\{X_i=\<X,\langa^{X_{i}}\>\}\subseteq\Stra:\bigcap X_i\in\Stra$.
\item\label{eachL} $\langa$ is immediate iff each $L\in\langa$ is immediate.\comt{}
\item\label{injective} $\langa$ is immediate iff
for  any collection of $A_n\in\Stra$
 and  {\em injective} functions $f_n:X\into \car{A_n} : \<X, \bigcap_n
 \langa^{\inv{f_{n}}{A_{n}}}\>\in\Stra$.
\item\label{ifSigma} $\langa$ is immediate iff
for any $\Sigma: \{A_n\}\subseteq\Strs \land
%$\bigcap A_n= \<\bigcap_n\car{A_n},\bigcap_n L_k^{A_{n}},\bigcap_n\Sigma^{A_{n}}\>$ is
\bigcap A_n\in\Str\Sigma \impl \bigcap A_n\in\Stra$.
\item\label{ifsub} $\langa$ is immediate iff
for  any $\Sigma: A\in\Strs \land 
B\sqsubseteq_\Sigma A \impl B\in \Stra$.
\item\label{samecar} If $\langa$ is immediate then:
%intersection of $\langa$-structures on the same set is again an $\langa$-structure:
$\{A_n\}\subseteq\Stra\land\car {A_n}=X \impl \bigcap A_n\in\Stra$.
\item\label{subset} If $\langa$ is immediate then: 
%restriction of an $\langa$-structure to its subset is an $\langa$-structure: 
for any  set $X: A\in \Stra \impl \<\car A\cap X,
\langa^A\cap X\times X\>\in \Stra$.
\item\label{sii} If $\langa$ is strictly immediate then it is immediate.
\end{enum}
\end{Prop}
\begin{Proof} 
\ref{eachL}. Let all $A_n\in\Stra$. Since each $L_i\in\langa$ is immediate,
$\bigcap A_n\in\Str {L_i}$ -- but this means $\bigcap A_n\in\Stra$. [???]
Conversely, assume $\langa$ is immediate. [if each $L_i$-structure can be
extended to an $\langa$-structure ???] Let $A_n\in\Str{L_i}$ and $A'_n$ be an
extension of $A_n$ to an $\langa$-structure. $\bigcap A'_n\in\Stra$ and so,
in particular, $\bigcap A'_n\in\Str{L_i}$. \\[1ex]
\ref{injective}. Let $\langa$ be immediate. For an $A_n\in\Stra$ and an injective
$f_n:X\into\car{A_n}$, the structure $A'_n$ obtained by replacing each
$f_n(x)$ by $x$ and preserving all relations will be in $\Stra$. Then also
for each $L\in\langa:L^{\inv{f_{n}}{A'_{n}}}(x,y) \iff L^{A'_{n}}(x,y)$. We thus
get $\<X,\bigcap_n \langa^{\inv{f_{n}}{A'_{n}}}\> = \<\bigcap_n\car{A'_n}, \bigcap_n
\langa^{A'_{n}}\>\in\Stra$ since $\langa$ is immediate. 
Conversely, $\car{\bigcap A_n}= \bigcap_n \car{A_n}$ yields
an injective (identity) function $f_n:\car{\bigcap A_n}\into \car{A_n}$,
for each $A_n$. 
Then for each $L\in\langa:L^{\inv{f_{n}}{A_{n}}}(x,y) \iff L^{A_{n}}(x,y)$, and so
$\bigcap A_n = \<\bigcap_n\car{A_n},\bigcap_n
\langa^{\inv{f_{n}}{A_{n}}}\>\in\Stra$ by assumption.
\\[1ex]
\ref{ifSigma}. 
If $\bigcap A_n\in\Stra$, then, in particular, if
$\bigcap A_n\in\Str\Sigma$, then $\bigcap A_n\in\Stra$.
Conversely: for empty $\Sigma$, any $\bigcap A_n\in\Str\Sigma$,
hence $\bigcap A_n\in\Stra$ by \ref{ifSigma}. \\[1ex]
%
\ref{ifsub}. 
Let $B\sqsubseteq_\Sigma A$. Construct $A_1\in\Strs$ from $A$ by
replacing each element $a\in \car A\setminus\car B$ by
a new element $n_a$ and preserving interpretation of all the symbols (i.e.,
$L_i^{A_1}(n_a) = \{n_b: b\in L_i^A(a)\}$ etc.).  Then $B=A\cap A_1\in\Str\Sigma$
and, by~\ref{ifSigma}., $B\in\Stra$.
Conversely: if $\bigcap A_n\in\Str\Sigma$ then $\bigcap A_n\sqsubseteq_\Sigma
A_n$ for any $A_n$ and, by \ref{ifsub}., $\bigcap A_n\in \Stra$. \\[1ex]
\ref{samecar}. 
is a special case of \refb{def:immed}.\ref{i:immed} (or of \ref{ifSigma}).\\[1ex]
%
\ref{subset}. 
follows from \ref{injective} with injection $id:X\cap\car{A}\into\car A$, $id(a)=a$
(or from \ref{ifsub}. with $\Sigma=\emptyset$). \\[1ex]
%
\ref{sii}. follows from \ref{injective} (which is a special case of definition~\refb{def:immed}.\ref{si}).
\end{Proof}
\ref{subset}. means that $\Stra$ for an immediate $\langa$ is closed under
substructures but this condition is weaker than immediacy. Closure under
substructures means that given $\{A_{n}\}_{n\in I}\subseteq\Stra$ with
$\car A=\bigcap_{n\in I}\car{A_n}$ then, for any
$A_i\in\{A_n\}: \<\car A, L^{A_{i}}\cap \car{A}\times
\car{A}\>\in\Stra$ which does not imply that $\bigcap A_n=\<\car{A},
\bigcap_{n\in I} L^{A_{n}}\>\in\Stra$. The conclusion of \ref{subset}.,
namely, that {\em any}
subset of an $\langa$-structure is already an $\langa$-structure (with
inherited relations), is weaker than immediacy.
\begin{Example}
Here is a language closed under substructures which is not immediate...
\end{Example}
What distinguishes strict immediacy from immediacy is only identity.
For instance, identity, non-identity, weak PO are immediate but not strictly immediate. 
\begin{Definition}\label{def:exp}
1. Given an $A\in\Stra$, a {\em pre-quotient} of $A$ is any structure $B$ with
$|B|=\bigcup_{a\in|A|} [a]$ where, for each $a\in|A|$, $[a]$ is a non-empty
set such that $a\not=b \impl [a]\cap [b]=\emptyset$ and where, for each
$L\in\langa: L^B(x_a,x_b)$
iff $x_a\in [a],x_b\in [b]$ and $L^A(a,b)$.

\noindent
2. $\langa$ {\em has no reference to identity} iff $\Stra$ is closed under pre-quotients.
\end{Definition}
%
\comt{Having a reference to identity is a more general claim than being
identity. For instance, antisymmetry ($L(x,y)\land L(y,x)\imp x=y$) has a
reference to identity, but this does not mean that any $L$-structure must
involve identity. A $B$ where $L^B$ is irreflexive does not force any identities.
But there exists an 
$A$ where an element can not be ``splitt'', namely for an $x: L^A(x,x)$,
$L^{\ext Ax}$ will not be antisymmetric.}

Notice also that a reference to non-identity is not a reference to
identity. If $R$ requires $x\not= y$, then any $R$-structure $A$ may be
extended to an $R$-structure $\ext Aa$.
\begin{Prop}\label{cl:simIff}
$\langa$ is
strictly immediate $\iff$ $\langa$ is immediate and has no reference to identity.
\end{Prop}
\begin{Proof} $\impl$) implication to immediacy is \refb{cl:iff}.\ref{sii}. So
assume that $\langa$ has reference to identity and let $B$ be a pre-quotient
of an $A\in\Stra$ witnessing to that, i.e., $B\not\in\Stra$. Let $f:\car
B\into \car A$ be given by $\forall b\in [a]: f(b) = a$ (notation from
def.~\ref{def:exp}). Then $B= \inv fA$, but since $B\not\in\Stra$, so
$\langa$ is not strictly immediate.
%$A$, $[a]$ be
%witnesses to that. Let $X=\car{A}\cup\{[a]\}$ and $f:X\into\car{A}$ be
%identity on $\car A$ and $f([a])=a$. Then $\ext Aa\not\in\Stra$ but $\ext
%Aa=\inv fA$, so $\langa$ is not strictly immediate. 
\\[1ex]
$\Leftarrow$) 
Let all $A_n\in\Stra$ and $f_n:X\into \car{A_n}$ be arbitrary functions. Each
$f_n$ factors through $\quot X{\equiv_{n}}$, i.e., $f_n= g_n;h_n$ where $g_n$
is surjective and $h_n$ injective:
\[\xymatrix@R=.4cm@C=1.5cm{
& & \car{A_1} \\
& \quot X{\equiv_{1}} \ar@{^{(}->}[ur]_{h_1} & \\
X \ar@{>>}[ur]_{g_1} \ar@{>>}[dr]^{g_n} \ar@/^0.6cm/[uurr]^{f_1} \ar@/_0.6cm/[ddrr]_{f_n} & ... &
\\
& \quot X{\equiv_{n}} \ar@{^{(}->}[dr]^{h_n} & \\
& & \car{A_n}
}\]
Since $\langa$ is immediate and all $h_i$ are injective, all $\quot
X{\equiv_{i}}\in\Stra$. 
For each $g_i$ let $X_i$ be the pre-quotient of $\quot X{\equiv_i}$ on the
carrier $X$ with $[x]=\{x'\in X:g_i(x')=x\}$. Since $\langa$ has no reference
to identity, all $X_i\in\Stra$. Finally, by immediacy of $\langa$,
$\bigcap_nX_n\in\Stra$ -- but each $X_i=\inv{f_i}{A_i}$, so $\langa$ is
strictly immediate.
%
%For any given
%$\quot X{\equiv_{n}}$, replace each $[x]$ by an arbitrary representative
%$x\in[x]$ -- the resulting structure is in $\Stra$. Now expand all
%$L\in\langa$ by adding all remaining $y\in [x]\setminus\{x\}$ according to
%def.~\refb{def:exp}.\comt{Need {\em set} in \ref{def:exp}} 
%\\ Since $\langa$ has no reference to identity, the
%resulting structure -- $X_n$ with carrier $\car{X_n}=X$ -- is in $\Stra$. 
%Thus, by immediacy of $\langa$ (def.\refb{def:immed}), $\bigcap X_n\in\Stra$. But $X_n=\inv
%{f_{n}}{A_{n}}$, so $\langa$ is strictly immediate.
\end{Proof}
\begin{Example}
A weak PO (reflexive, transitive, antisymmetric) is immediate but not strictly immediate -- the reference to
identity in the antitransitivity axiom provides a simple
counter-example: one element PO $B=\{b;b\leq b\}$ and a function from two
elements set $f:\{0,1\}\into \{b\}$ gives $\inv fB =
\{\{0,1\};0\leq 0,0\leq 1,1\leq 0,1\leq 1\}$, so we have $0\leq 1\land 1\leq
0$ but $0\not=1$.

A strict PO (irreflexive, transitive) has no reference to identity -- it is
strictly immediate. 
\end{Example}
\begin{Prop}
If $\langa$  is strictly immediate and $SP\subseteq\langsx$ is consistent then $\tsp\in\Strs$.
\end{Prop}
\begin{Proof}
We do have $\tsp\in\Str\Sigma$.
The assumptions ensure that all relations $L_i$ are correctly
interpreted in $\tsp$. 
For each $A_n\in\Mod{SP}\subseteq\Strs$ let $f_n:\GTerms\into \car{A_n}$ be given by
$f_n(t)=t^{A_{n}}$. Since $\langa$ is strictly immediate,
$\bigcap \inv{f_{n}}{A_{n}}\in\Stra$. But this is precisely $\tsp$.
%%
%%But each $f_n$ factors through $\car{\tsp}$, i.e., 
%%\[\xymatrix{
%% & \GTerms \ar[ld]_{g_n} \ar[rd]^{f_n} \\
%%\car{\tsp} \ar[rr]^{h_n} & & \car{A_n}
%%}
%%\]
%%with $f_n=g_n;h_n$. Thus we have that also $\bigcap \inv{h_{n}}{A_{n}}\in\Stra$, 
%%but this structure is just
%%$\tsp\in\Stra$.
\end{Proof}

\begin{Prop}
If $\langa$ is strictly immediate then:

\begin{tabular}{rrl}
%\begin{enum}\item 
1. & for any ground $\phi\in \langs:$ & $\tsp\models\phi\Iff
\Mod{SP}\models\phi$ \\
%\item 
2. & for any  $\phi\in \langsx:$ & $\tsp\models\phi\Longleftarrow \Mod{SP}\models\phi$.
%\end{enum}
\end{tabular}
\end{Prop}

\begin{Prop}
If $\langa$ is strictly immediate then $\tsp$ is initial in ${\Mod{SP}}$  wrt. $\langs$-homomorphisms.
\end{Prop}
%
\noindent\hrulefill

Now, we treat the case of immediate but not strictly. We need to add
quotient...

\begin{Definition}
Given a language $\langa=\{L_1...L_n\}$, an $\langa$-congruence $\equiv$ is an
equivalence relation such that in any $A\in\Stra$ and for all $L\in\langa:$

\hspace*{\fill}{$\forall a_1,a_2,a'_1,a'_2\in\car A:
a_1\equiv a'_1 \land a_2\equiv a'_2 \land L^A(a_1,a_2) \impl L^A(a'_1,a'_2)$.}
\end{Definition}
Unless stated otherwise, $\equiv$ will denote an arbitrary $\langa$-congruence.
We will write $\lequiv$ for the language $\langa\cup\{\equiv\}$ where $\equiv$ is a designated
$\langa$-congruence.
For an $a\in A\in\Str{\lequiv}$ we let $[a]=\{a'\in A: a\equiv^A
a'\}$. $\lequ$ is a language (with reference to identity) where the
designated congruence is identity. We obviously have 
\eq{
\Str{\lequ}\subset\Str{\lequiv} \label{eq:sub}
}
The inclusion is proper by \refb{cl:simIff}.
\begin{Fact}
If $\langa$ is strictly immediate then so is $\lequiv$.
\end{Fact}
\begin{Proof}
Let $\{A_i\}\subseteq \Str{\lequiv}$ and $f_i:X\into A_i$ be as on the
diagram below.
\[\xymatrix@R=0.7cm@C=1.5cm{
 & \car{A_1} \\
X \ar[ur]^{f_1} \ar[dr]^{f_n} \ar[r]^{f_i}  & ... \\
 & \car{A_n}
}\]
We show that 1) for every $i:\inv{f_i}{A_i}\in\Str{\lequiv}$ and then
that 2) $\bigcap \inv{f_i}{A_i} \in\Str{\lequiv}$.

1) We only have to show that inverse of $\equiv_i$ is an $\langa$-congruence
   on $X$ with inverse of $\langa^{A_{i}}$. So, assume that $L^{\inv
   {f_{i}}{A_{i}}}(x_1,y)$ and $\equiv^{\inv{f_{i}}{A_{i}}}(x_1,x_2)$. This
   means that $L^{A_{i}}(f_i(x_1),f_i(y))$ and
   $\equiv^{A_{i}}(f_i(x_1),f_i(x_2))$. Since $\equiv^{A_{i}}$ is an
   $\langa$-congruence on $A_i$, we get that $L^{A_{i}}(f_i(x_2),f_i(y))$,
   and thus $L^{\inv   {f_{i}}{A_{i}}}(x_2,y)$. 

2) Let $A$ abbreviate $\bigcap \inv{f_i}{A_i}$. Since $\langa$ is
   (strictly) immediate, $A\in\Stra$. We have to show that $\bigcap
   \equiv^{\inv{f_{i}}{A_{i}}}$ is a congruence on $A$. So assume that
   $L^A(x_1,y)$ and $\equiv^A(x_1,x_2)$. Then both these
   relations hold in all $\inv{f_i}{A_i}$ and, by 1) we also get
   $L^{\inv{f_{i}}{A_{i}}}(x_2,y)$ for all $\inv{f_i}{A_i}$. But then also $L^A(x_2,y)$.
\end{Proof}
\begin{Definition}
Given an $A\in\Stra$ and an $\langa$-congruence (on $A$) $\equiv$, the
quotient $\quot A\equiv$ is the structure given by: $\car{\quot A\equiv} =
\{[a]:a\in\car A\}$ and for all $L\in\langa: L^{\quot
A{\equiv}}([a],[a'])\iff L^A(a,a')$.
\end{Definition}
The relations $L^{\quot{A}\equiv}$ are well defined because $\equiv$ is an
$\langa$-congruence and hence the choice of representatives does not matter.

As in the classical case, quotients are actually (isomorphic to) substructures.
\begin{Fact}\label{fa:quotsub}
Given an $\langa$ and an $\langa$-congruence $\equiv$, for any $A\in\Stra$
there exists a set
$X\subseteq\car A: \quot A\equiv \iso
\<\car A\cap X, \langa^A\cap X\times X\>$.
\end{Fact}
\begin{Proof}
From each $[a]$ select a unique element $a\in [a]$ and let $X$ be the set of
these. Since $\equiv$ is an $\langa$-congruence the relations $L^X(a,a')\iff
L^A(a,a')$ are well defined (do not depend on the choice of $a$'s). Obviously
$\quot A\equiv \iso \<\car A\cap X, \langa^A\cap X\times X\>$.
\end{Proof}
%
From now on we will often refer to a pair of languages, $\lequiv$ and $\lequ$
where the former is obtained from the latter by replacing all references to
identity by an $\langa$-congruence (and vice versa). If $\lequ$ is axiomatized
by a set of formulae $\Phi^=$ then $\lequiv$ is axiomatized by the set
$\Phi^\equiv$ (each occurrence of $=$ in $\Phi^=$ is replaced by $\equiv$) and
the congruence axioms for $\equiv$.
%
\begin{Example} If $\lequ$
is the language for partial orderings as shown on the left, then $\lequiv$ is
as shown on the right -- $=$ in the 3rd axiom replaced by $\equiv$ and
congruence axioms for $\equiv:$
\[\begin{array}{rr|rr}
& \lequ = \{\leq,=\} & & \lequiv = \{\leq,\equiv\} \\ \hline
\Phi^=:& x\leq x & \Phi^\equiv: &   x\leq x \\
& x\leq y \land y\leq z \impl x\leq z &&  x\leq y \land y\leq z \impl x\leq z \\
& x\leq y\land y\leq x\impl x=y && x\leq y\land y\leq x\impl x \equiv y \\ \cline{1-4}
&& \multicolumn{2}{l}{\equiv\ {\rm is\ an\ equivalence\ }:} \\ %\cline{3-4}
& && x\equiv x,\ x\equiv y \impl y\equiv x, \\
& &&  x\equiv y\land y\equiv x\impl x\equiv z \\ %\cline{3-4}
&& \multicolumn{2}{l}{{\rm and\ a\ congruence\ wrt.\ }\leq:} \\ %\cline{3-4}
& && x\leq y \land x\equiv x'\land y\equiv y'\impl x'\leq y'
\end{array}
\]
\end{Example}
%
\begin{Fact}\label{fa:then}
For any pair $\lequiv$ and $\lequ$ (closed under isos):
\begin{enumerate}\MyLPar
\item\label{it:quo} $A\in\Str{\lequiv}\impl \quot A\equiv\in\Str{\lequ}$
\item\label{it:pre} $A\in\Str{\lequ} \land \quot B\equiv \iso A \impl B\in\Str{\lequiv}$.
\end{enumerate}
\end{Fact}
\begin{Proof}
1. follows since $\equiv^{\quot A\equiv}$ is identity.
2.: $A\in\Str{\lequ}\by{(\ref{eq:sub})}\impl A\in\Str{\lequiv}$ and since $\lequiv$
has no reference to identity then $B\in\Str{\lequiv}$.
\end{Proof}
%
\begin{Prop}Let $\equiv$ stand for an arbitrary $\langa$-congruence:
\begin{enumerate}\MyLPar
\item\label{cl:quot}
If $\langa$ is immediate then $\Stra$ is closed under quotients: $A\in\Stra
\impl \quot A\equiv\in\Stra$.
\item\label{fa:simcl}
If $\langa$ is strictly immediate then $\Str{\langa}$ is closed under
quotients and pre-quotients.
\item
If $\lequiv$ is strictly immediate and $A\in\Str{\lequiv}$  then
$\quot A\equiv\in \Str{\lequ}\subset\Str{\lequiv}$.
\item If $\lequiv$ is strictly immediate and $\quot A\equiv\in\Str{\lequ}$  then
$A\in\Str{\lequiv}$.
\end{enumerate}
\end{Prop}
\begin{Proof}
%Let $f:\car{\quot A\equiv}\into \car A$ be injective function sending $[a]$
%onto some $a\in[a]$. Then $\quot A\equiv \iso \inv
%fA\in\Str{\lequiv}$ since $\lequiv$ is immediate (b
\begin{enumerate}\MyLPar
\item 
By~\refb{cl:iff}.\ref{subset} $\Stra$ is closed under $\langa$-substructures,
so the claim follows by closure under isos and \refb{fa:quotsub}.
\item If $\langa$ is strictly immediate then, by claim~\refb{cl:simIff}, it has no
reference to identity, thus $\Stra$ is closed under pre-quotients. Since
$\langa$ is also immediate, closure under quotients follows from \ref{cl:quot}. 
\item If $\langa$ is strictly immediate then it is immediate, so
$A\in\Str{\lequiv} \by{\ref{cl:quot}}\impl \quot A\equiv\in\Str{\lequiv}$,
but $\equiv^{\quot A\equiv}$ is identity, and so $\quot
A\equiv\in\Str{\lequ}$.
\item $A$ is (isomorphic to) a pre-quotient of $\quot A\equiv$. Hence 
$\quot A\equiv\in\Str{\lequ} \by{(\ref{eq:sub})}\impl \quot
A\equiv\in\Str{\lequiv} \by{\ref{fa:simcl}}\impl A\in\Str{\lequiv}$.
\end{enumerate}
\end{Proof}
%
%
\begin{Theorem}
$\lequ$ is immediate iff $\lequiv$ is strictly immediate.
\end{Theorem}
\begin{Proof}
$\Leftarrow$) Let $\{A_i\}\subseteq \Str{\lequ}$ -- we have to show that
$\bigcap_i A_i \in\Str{\lequ}$. By (\ref{eq:sub})
$\{A_i\}\subset\Str{\lequiv}$ and, since $\lequiv$ is immediate, $A=\bigcap_i
\car{A_i}\in\Str{\lequiv}$. Now $\equiv^A= \bigcap_i \equiv^{A_i}$ and each
$\equiv^{A_i}$ is actually identity $=^{A_i}$, so $\equiv^A$ is identity on
$A$, $=^A$. But this means that $A\in\Str{\lequ}$, so $\lequ$ is immediate.
\\[1ex]
$\impl$) Let $\{A_i\}\subseteq\Str{\lequiv}$ and $f_i:X\into \car{A_i}$ be a
collection of functions. We have to show that
$A=\bigcap\inv{f_i}{A_i}\in\Str{\lequiv}$. Each $f_i$ factors through a
surjective $g_i$ and injective $h_i$ as shown on the diagram to the left:
\[\xymatrix@R=.4cm@C=1.5cm{
& \Str{\lequiv} &  && \Str{\lequ} \\
& & \car{A_1} 
  &&& \car{\quot{A_1}{\equiv^{A_{1}}}}  \\
& \quot X{\equiv_{1}} \ar@{^{(}->}[ur]_{h_1} & 
  && \quot{(\quot X{\equiv_1})}{\equiv'_1} \ar@{^{(}->}[ur]_{h'_1}\\
X \ar@{>>}[ur]_{g_1} \ar@{>>}[dr]^{g_n} \ar@/^0.6cm/[uurr]^{f_1} \ar@/_0.6cm/[ddrr]_{f_n} & ... &
   && ... \\
& \quot X{\equiv_{n}} \ar@{^{(}->}[dr]^{h_n} & 
  && \quot{(\quot X{\equiv_{n}})}{\equiv'_n} \ar@{^{(}->}[dr]_{h'_n} & \\
& & \car{A_n}
  &&& \car{\quot{A_n}{\equiv^{A_{n}}}} 
}\]
For each $A_i$ let $q_i:A_i\into\quot{A_i}{\equiv^{A_i}}$ be the quotient
mapping, and $\equiv'_i$ the congruence on $\quot X{\equiv_i}$ defined by
$[x]\equiv'_i[y]\Iff q_i(h_i([x])) = q_i(h_i([y]))$. \comt{Is it congruence?}
Then $h'_i:\quot{(\quot
X{\equiv_i})}{\equiv'_i} \into \car{\quot{A_i}{\equiv^{A_i}}}$ is injective
and, by immediacy of  $\lequ$ (claim~\refb{cl:iff}.\ref{injective}), 
$\quot{(\quot X{\equiv_i})}{\equiv'_i}=
\inv{h'_i}{\quot{A_{i}}{\equiv^{A_{i}}}}\in\Str{\lequ}$. By
fact~\refb{fa:then}.\ref{it:pre} we have then $\quot
X{\equiv_i}\in\Str{\lequiv}$. 
Since $\lequiv$ has no reference to identity

\end{Proof}

\noindent\hrulefill

If $\langa$ is not immediate, we risk that $\GTerms$ do not provide enough
witnesses. In such a case, we can still provide a canonical construction,
although it won't necessarily yield an initial model...

%\end{document}
%%%%%%%%
\newpage
\comt{OLD: intersection}\\
$\tsp$ need not be initial in $\Mod{SP}$ because it may happen that
$\tsp\not\in\Strs$.
To ensure this does not happen we need to require all relations to
be {\em immediate}. 
Intuitively, a relation $R\subseteq X\times Y$ is {\em immediate} iff 
$R(x,y)$ never requires existence of other elements than $x$ and
$y$, or else iff $\{x\},\{y\}$ can be made into a structure
satisfying $R(x,y)$. 
Given a strict ordering $<$, the relation $R_{sep}(x_1,x_2)\Iff
\exists x: x_1<x\land x<x_2$ is not immediate. Similarly, the directedness relation
$R_{dir}(x_1,x_2)\Iff \exists x: x_1<x\land x_2<x$ is not immediate. 
The ordering $<$ itself, identity, membership, set-inclusion, on
the other hand, are immediate.

\begin{Definition}
A $\Sigma$-substructure of an $A\in\Strs$ is a structure $B=\<\car
B, L_1^B,...,L_k^B,\Sigma^B\>$ where $\car B\subseteq\car A$ is
$\Sigma$-closed,  $\forall f\in\Funcs, b_i\in\car B: f^B(b_i)=f^A(b_i)$, and $L_i^B=L_i^A\cap
\car B\times\car B$. We write $B\sqsubseteq_\Sigma A$ (possibly dropping subscript.)

Given a set of $\langs$-structures $\{A_i:i\in I\}$ for some $I$, we 
let $\bigcap A_i$ denote the structure $\<\bigcap_i{\car
A_i},\bigcap_iL_1^{A_{i}},..., \bigcap_iL_k^{A_{i}},
\bigcap_i\Sigma^{A_{i}}\>$.
\end{Definition}
For an $A\in\Strs$, a $B\sqsubseteq_\Sigma A$ may fail to be an
$\langa$-structure. Also $\bigcap A_i$ may happen to be neither $\Sigma$- nor $\langa$-structure.
\begin{Definition}\label{Odef:immed}
A relation $R$ is {\em immediate} iff for  any set of
$R$-structures $\{A_i\}$, 
$\bigcap A_i= \<\bigcap_i\car{A_i},\bigcap_i R\>$ 
is also an $R$-structure.
{\em Mediate} relation is one that is not
immediate. 
$\langa$ (or its ``interpretation'') is {\em immediate} iff all $L_i\in\langa$ are immediate.
\end{Definition}
For instance, a (quasi, partial, total)  ordering is immediate but a
dense ordering $R_{dns}$ is not. To see this take two $R_{dns}$-structures: as $A_1$ the dense ordering of
rationals in the closed interval $\car{A_1}=[0,1]$ and as $A_2$ the same
ordering on this interval
but with all rationals within the open inteval $(0,1)$ replaced by new distinct
elements. $A_1\cap A_2=\<\{0,1\},\{\<0,0\>,\<0,1\>,\<1,1\>\}\>$ is still (quasi, partial, total) ordering but it
is not dense, i.e., it is not an $R_{dns}$-structure.

We note some equivalent formulations of immediate relations.
\begin{Prop}
$\langa$ is immediate 
\begin{enum}
%\item for any set $\{X_i=\<X,\langa^{X_{i}}\>\}\subseteq\Stra:\bigcap X_i\in\Stra$.
\item\label{OifSigma}$\iff$ 
for any $\Sigma: \{A_i\}\subseteq\Strs \land
%$\bigcap A_i= \<\bigcap_i\car{A_i},\bigcap_i L_k^{A_{i}},\bigcap_i\Sigma^{A_{i}}\>$ is
\bigcap A_i\in\Str\Sigma \impl \bigcap A_i\in\Stra$.
\item\label{Oifsub}$\iff$ 
for  any $\Sigma: A\in\Strs \land 
B\sqsubseteq_\Sigma A \impl B\in \Stra$.
\item\label{Osamecar}$\impl$ 
%intersection of $\langa$-structures on the same set is again an $\langa$-structure:
$\{A_i\}\subseteq\Stra\land\car {A_i}=X \impl \bigcap A_i\in\Stra$.
\item\label{Osubset}$\impl$ 
%restriction of an $\langa$-structure to its subset is an $\langa$-structure: 
for any set $X: \<\car A,\langa^A\>\in \Stra \impl \<\car A\cap X,
\langa^A\cap X\times X\>\in \Stra$.
\end{enum}
\end{Prop}
\begin{Proof} 
%1. It is just a special case of the definition of immediate
%relation. Conversely, assume 1. and let $A_i\in\Stra$ and
%$X=\bigcap_i\car{A_i}$ \\[1ex]
\ref{OifSigma}. If $\langa$ is immediate, then $\bigcap A_i\in\Stra$. In particular, if
$\bigcap A_i\in\Str\Sigma$, then $\bigcap A_i\in\Stra$.
Conversely, for empty $\Sigma$, any $\bigcap A_i\in\Str\Sigma$,
hence $\bigcap A_i\in\Stra$ by \ref{OifSigma}. \\[1ex]
%
\ref{Oifsub}. 
Let $B\sqsubseteq_\Sigma A$. Construct $A_1\in\Strs$ from $A$ by
replacing each element $a\in \car A\setminus\car B$ by
a new element $n_a$ and preserving interpretation of all the symbols (i.e.,
$L_i^{A_1}(n_a) = \{n_b: b\in L_i^A(a)\}$ etc.).  Then $B=A\cap A_1\in\Str\Sigma$
and, since $\langa$ is immediate, $B\in\Stra$.
Conversely, if $\bigcap A_i\in\Str\Sigma$ then $\bigcap A_i\sqsubseteq_\Sigma
A_i$ for any $A_i$ and, by \ref{Oifsub}., $\bigcap A_i\in \Stra$. \\[1ex]
%
\ref{Osamecar}. is a special case of definition \ref{Odef:immed}.\\[1ex]
%
\ref{Osubset}. follows from \ref{Oifsub}. with $\Sigma=\emptyset$. Conversely,
   does not seem to go...
%%assume 4. and
%%   let $\{A_i\}\subseteq\Stra$. Then for each $A_k\in \{A_i\}$ we have
%%   $A'_k=\<\car {A_k}\cap \bigcap_i\car{A_i}, R^{A_{k}}\cap
%%   \bigcap_i\car{A_i}\times \bigcap_i \car
%%   {A_i}\>\in\Stra$. [??? By \ref{samecar}. we obtain that $\bigcap A'_i\in \Stra$.???]
\end{Proof}
\comt{OLD: intersection}
%%%%%%%
%%%%%%
\subsection{Horn extension of $\langa$ ...}
To allow Horn extensions of $\langsx$ we need to exclude
mediate but also contrary relations.
\begin{Definition}
Interpretation of relation symbols $L,R\subseteq X\times Y$ is {\em contrary}
iff in any $L,R$-structure $A$, $\exists x,y\in A:
((x,y)\in R^A\lor (x,y)\in L^A) \land ((x,y)\not\in R^A\lor (x,y)\not\in
L^A)$. \\ .
\comt{$\ \Iff\forall A:\exists x,y:(x,y)\not\in R^A \iff (x,y)\in L^A
 \Iff \forall A:L^A\not\subseteq R^A\land R^A\not\subseteq L^A$} \\
non-c= \comt{$\exists A:\forall x,y: (x,y)\in L^A\cap
R^A\lor (x,y)\in \neg L^A\cap \neg R^A \Iff \exists A:L^A\subseteq R^A\lor
R^A\subseteq L^A$}
\end{Definition}

\begin{Example}
Let $L_1,L_2$ be contrary, and let
$SP:$ 1. $L_1(a,b)\impl R_1$, 2. $L_2(a,b)\impl R_2$. In an initial
model we would let all relations not hold. However, due to contrariness, we
have choices:

$\begin{array}{lll}
a) & \neg L_1(a,b) \land L_2(a,b) \land R_2\ (\land \neg R_1) & 
   since\ \neg L_1(a,b)\impl L_2(a,b)\\
b) & \neg L_2(a,b) \land L_1(a,b) \land R_1\ (\land \neg R_2) & 
   since\ \neg L_2(a,b)\impl L_1(a,b)
\end{array}
$
\end{Example}

\newpage
\ 

\newpage
\section{Labeling...}
For a $\Sigma$-structure $A$, its underlying set $\under A$ is possibly
different from its carrier $\car A$, but $\under A\subseteq\car A$.
\begin{Definition}
For a signature $\Sigma=(\Sorts,\Funcs)$ and a set $X$,
$\Sigma$-diagram of $X$, $\diag X$ is given:
\begin{enum}
\item $\GTerms \subseteq\diag X$
\item for each $x\in X\setminus\GTerms$,  a new name $n(x)\in\diag X$ (typically, $n(x)=x$)
\item for all $n_1...n_k\in\diag X$ and $k$-ary $f\in\Funcs: f(n_1...n_k)\in\diag X$.
\end{enum}
Thus $\diag\emptyset=\GTerms=\diag{{\scriptsize{\GTerms}}}$. For a $\Sigma$-structure
$A$, $\diag{A}$ is the $\Sigma$-diagram of the set
$\{a\in\car A:\forall t\in\GTerms\ t^A\not=a\}$.
The interpretation of $\diag A$ in $A$ is obious: $n(a)^A=a$,
$f(n_1...n_k)^A= f^A(n_1^A...n_k^A)$. 
\end{Definition}
There are infinitely many different (but isomorphic) diagrams for each $A$ but we work with
{\em the} diagram -- an arbitrary representative. For $t\in\GTerms$, we let $n(t)=t$.

In general, $f(n(a_1)...n(a_k)) \not= n(a)$ even if $f^A(a_1...a_k) = a$. The
equivalence (congruence) $\equiv\ \subseteq\ \diag A\times\diag A$ is defined
by: $n_1\equiv n_2 \Iff n_1^A=n_2^A$. By $[n]$ (possibly $[n]^A$) we denote the $\equiv$-equivalence
class of $n\in\diag A$, and their
collection $\quot{\diag A}{\equiv}$, by $\diagE A$.
%%% Moved to 1.
%%\begin{Definition}
%%Atomic language $L$ is given by a set of binary relation symbols
%%$L_1,...,L_k$. Given a signature $\Sigma$ and a set of variables $X$, the
%%atomic formulae of $L$ relatively to $\Sigma,X$ form the set
%%$L_{\Sigma,X}= \{L_i(t_1,t_2): L_i\in L\land t_1,t_2\in\XTerms\}$.
%%\end{Definition}
%%
\begin{Definition}
Given  an atomic language $L_{\Sigma,X}$ and a $\Sigma$-structure $A$ with
(primitive) interpretation $L_i^A$ of all $L_i\in L$, we define the
lifting, or labeling, of $A$ wrt. $S\subseteq L$, $\lift{A,S}$ (we write it 
$\lift A$ when $S$ and $L$ are implicitly given by the context): 
\begin{enum}
\item For  any $a\in\car A$, its lifting is a tuple $l(a)=
\<a,[n(a)],L_1(a),...,L_k(a)\>$,\\ where $\{L_1,...,L_k\}=S$ and $L_i(a) =\{[n]:n\in\diag A\land
L_i^A(a,n)\}$
\item $\car {\lift A} = \{l(a):a\in\car A\}$
\item For $l(a_1)...l(a_k)\in\car{\lift A}$ and a $k$-ary $f\in\Funcs$,
$f^{\lift A}(l(a_1)...l(a_k)) = l(f^A(a_1...a_k))$.
\end{enum}
If $x\in\car{\lift A}:el(x)\in\car A$ is s.t.
$l(el(x))=x$, $n(x)= [n(el(x))]$  and $L_i(x)=L_i(el(x))$.
\\
Given an assignment $\alpha:X\into\car{\lift A}$:\comt{$\into\under A\subseteq\car A$}
\begin{enum}\setcounter{enumi}{3}
\item $\lift A\models_\alpha L_i(x,y) \Iff n(\alpha(y))\in L_i(\alpha(x))$.
\item $\lift A\models_\alpha x=y \Iff n(\alpha(x))\in\{n(\alpha(y)\}$, i.e., $n(\alpha(x))= n(\alpha(y))$
\end{enum}
\end{Definition}

\begin{Claim} $\xymatrix@R=.3cm{[X\into\car A]  \ar@<.5ex>[r]^l &
[X\into\car{\lift A}] \ar@<.5ex>[l]^{el}}$. 
$A\models_\alpha L_i(x,y) \Iff \lift A\models_{\alpha;l} L_i(x,y)$. \vspace*{-1ex}
\end{Claim}
%
\begin{Definition}
Given $\Sigma$ and $L=\{L_1,...,L_k\}$, a labeled
$\Sigma,L$-structure $A$ is given by
\begin{enum}
\item unlabeled carrier -- a set $\carr{A}$
\item a congruence $\equiv$ on $\diag{\carr{A}}$ such that $\equiv\cap\ 
\carr A\times \carr A = id_{\carr A}$
\item the carrier $\car A$ containing, for each $[t]\in\diagE{\carr{A}}$ a unique tuple
$\l a=\<a,[t],L'_1...L'_k\>$ where $a\in\carr{A}$ and each
$L'_i\subseteq\diagE{\carr{A}}$ ($n(\l a)=[t]$ and $L_i(\l a)=L'_i$)
\item for $\l x_1...\l x_k\in\car A: f^A(\l x_1...\l x_k)=\l x$
s.t. $f(n(\l x_1)...n(\l x_k))\in n(\l x)$.
\end{enum}
For $\alpha:X\into\car A$, $A\models_\alpha L_i(x,y)\Iff n(\alpha(y))\in L_i(\alpha(x))$.
\end{Definition}
%
\begin{Definition}
Given $\Sigma$, $S\subseteq L$ and two $\Sigma,L$-structures $A,B$. A
$\Sigma,S$-homomorphism $\phi:A\into B$ is a total function
$\phi:|A|\into|B|$ given by $\phi:\diagEL A\into\diagEL B$ satisfying:
\begin{enum}
\item $\phi([f(n_1...n_k)]^A) = [f(n'_1...n'_k)]^B$ for any
$n'_i\in\phi([n_i]^A)$
\item for all $L_i\in S$ and $n\in\diagEL A: \phi(L_i^A(n)) \subseteq
L_i^B(\phi(n))$.
\end{enum}
\end{Definition}
Using only a subset $S\subset L$ allows us to treat homomorphisms which do
not necessarily preserve all logical relations from
$L$. $\Sigma$-homomorphism is a $\Sigma,\emptyset$-homomorphism. A {\em weak}
$\Sigma,S$-homomorphism is like above but with $\subseteq$ instead of $=$ in
1 -- assuming $el(\car B)\subseteq \PSet{\under B}$.

\section{Initiality and labeling...}
\begin{Definition}
Given $L=\{L_1...L_k\}$ and $\Sigma$ define the labeled structure $\init_{\Sigma,L}$:
\begin{enum}
\item the unlabeled carrier $= \GTerms$
\item the congruence $\equiv$ on $\diag{\scriptsize{\GTerms}}=\GTerms$ is identity
\item the carrier
$\car{\init_{\Sigma,L}}=\{\<t,t,L'_1...L'_k\>:t\in\GTerms\land
L'_i=\emptyset\}$
\item for each $f\in\Funcs$ and $t_1...t_k\in\car{\init_{\Sigma,L}}:
f^{\scriptsize\init_{\Sigma,L}}(t_1...t_k) = t$ with $n(t)=f(n(t_1)...n(t_k))$.
\end{enum}
\end{Definition}
Obviously, $\init_{\Sigma,L}$ is initial in $\LM{\Sigma,L}$ with any
$\Sigma,S$-homomorphisms (for any $S\subseteq L$).

Given a language $L$, with a logical (primitive) interpretation of all
$L_i\in L$ as immediate relations,
a specification $SP\subseteq L_{\Sigma,X}$ induces a model class $\Mod{SP}$. We
lift it to the class of labeled models $\lift{\Mod{SP}}$, 
construct a labeled model $\init_{SP}\models SP$ and show that it is initial
in $\liftp{\Mod{SP}} = \lift{\Mod{SP}}\cup\{\init_{SP}\}$. Also,
$\liftp{\Mod{SP}}\models\phi\Iff \Mod{SP}\models \phi$.

\begin{Definition}
Given $SP\subseteq L_{\Sigma,X}$, with the logical (primitive) interpretation
of all $L_i\in L$ as immediate relations. Assuming that $\Mod{SP}\not=\emptyset$, we define labeled model
$\init_{SP}$: % starting with $\init_{\Sigma,L}$ and 
\begin{enum}
\item let $\equiv\ \subseteq\ \car{\init_{\Sigma,L}}\times
\car{\init_{\Sigma,L}}$ be given by $\l t_1\equiv \l t_2\Iff \Mod{SP}\models
n(\l t_1)=n(\l t_2)$
\item $\init_{\equiv}=\quot{\init_{\Sigma,L}}\equiv$
\item $\init_{SP}$ is obtained from $\init_{\equiv}$ by letting for each $\l s, \l t
\in\car{\init_{\equiv}}$ and $L_i\in L:$ \\
$\l s\in L_i(\l t)\Iff \Mod{SP}\models
L_i(t',s')$, for some $s'\in n(\l s), t'\in n(\l t)$.
\end{enum}
\end{Definition}

\begin{Prop}
For any (ground!) $\phi\in L_{\Sigma}: \init_{SP}\models\phi\Iff \Mod{SP}\models\phi$.
\end{Prop}

\begin{Prop}
$\init_{SP}$ is initial in $\liftp{\Mod{SP}}$.
\end{Prop}

\begin{Prop}
For all formulae $\phi$ over atoms $L_{\Sigma,X}:\liftp{\Mod{SP}}\models\phi\Iff \Mod{SP}\models \phi$.
\end{Prop}
%
To allow Horn extensions of atomic language we will need to exclude not only
mediate but also contrary relations.
\begin{Definition}
Interpretation of relation symbols $L,R\subseteq X\times Y$ is {\em contrary}
iff in any $L,R$-structure $A$, $\exists x,y\in A:
((x,y)\in R^A\lor (x,y)\in L^A) \land ((x,y)\not\in R^A\lor (x,y)\not\in
L^A)$. \\ .
\comt{$\ \Iff\exists x,y\in A:(x,y)\not\in R^A \iff (x,y)\in L^A$}\\
. \comt{$\ \Iff L^A\not\subseteq R^A\land R^A\not\subseteq L^A$}
\end{Definition}

\begin{Example}
Let $L_1,L_2$ be contrary, and let
$SP:$ 1. $L_1(a,b)\impl R_1$, 2. $L_2(a,b)\impl R_2$. In an initial
model we would let all relations not hold. However, due to contrariness, we
have choices:

$\begin{array}{lll}
a) & \neg L_1(a,b) \land L_2(a,b) \land R_2\ (\land \neg R_1) & 
   since\ \neg L_1(a,b)\impl L_2(a,b)\\
b) & \neg L_2(a,b) \land L_1(a,b) \land R_1\ (\land \neg R_2) & 
   since\ \neg L_2(a,b)\impl L_1(a,b)
\end{array}
$
\end{Example}


%\bibliographystyle{bibNo}
%\bibliography{my}

\end{document}


