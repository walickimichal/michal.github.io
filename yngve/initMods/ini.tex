\documentclass[10pt]{article}
%\documentstyle[a4,fullpage,12pt,oldlfont]{article}

%\usepackage{amssymb}

\makeatletter

\ifcase \@ptsize
    % mods for 10 pt
    \oddsidemargin  0.15 in     %   Left margin on odd-numbered pages.
    \evensidemargin 0.35 in     %   Left margin on even-numbered pages.
    \marginparwidth 1 in        %   Width of marginal notes.
    \oddsidemargin 0.25 in      %   Note that \oddsidemargin = \evensidemargin
    \evensidemargin 0.25 in
    \marginparwidth 0.75 in
    \textwidth 5.875 in % Width of text line.
\or % mods for 11 pt
    \oddsidemargin 0.1 in      %   Left margin on odd-numbered pages.
    \evensidemargin 0.15 in    %   Left margin on even-numbered pages.
    \marginparwidth 1 in       %   Width of marginal notes.
    \oddsidemargin 0.125 in    %   Note that \oddsidemargin = \evensidemargin
    \evensidemargin 0.125 in
    \marginparwidth 0.75 in
    \textwidth 6.125 in % Width of text line.
\or % mods for 12 pt
    \oddsidemargin -10 pt      %   Left margin on odd-numbered pages.
    \evensidemargin 10 pt      %   Left margin on even-numbered pages.
    \marginparwidth 1 in       %   Width of marginal notes.
    \oddsidemargin 0 in      %   Note that \oddsidemargin = \evensidemargin
    \evensidemargin 0 in
    \marginparwidth 0.75 in
    \textwidth 6.375 true in % Width of text line.
\fi

\voffset -2cm
\textheight 22.5cm

\makeatother

%\makeatletter
%\show\
%\makeatother
\newcommand{\ite}[1]{\item[{\bf #1.}]}
\newcommand{\app}{\mathrel{\scriptscriptstyle{\vdash}}}
\newcommand{\estr}{\varepsilon}
\newcommand{\PSet}[1]{{\cal P}(#1)}
\newcommand{\ch}{\sqcup}
\newcommand{\into}{\to}
\newcommand{\Iff}{\Leftrightarrow}
\renewcommand{\iff}{\leftrightarrow}
\newcommand{\prI}{\vdash_I}
\newcommand{\pr}{\vdash}
\newcommand{\ovr}[1]{\overline{#1}}

\newcommand{\cp}{{\cal O}}

% update function/set
%\newcommand{\upd}[3]{#1\!\Rsh^{#2}_{\!\!#3}} % AMS
\newcommand{\upd}[3]{#1^{\raisebox{.5ex}{\mbox{${\scriptscriptstyle{\leftarrow}}\scriptstyle{#3}$}}}_{{\scriptscriptstyle{\rightarrow}}{#2}}} 
\newcommand{\rem}[2]{\upd{#1}{#2}{\bullet}}
\newcommand{\add}[2]{\upd {#1}{\bullet}{#2}}
%\newcommand{\mv}[3]{{#1}\!\Rsh_{\!\!#3}{#2}}
\newcommand{\mv}[3]{{#1}\:\raisebox{-.5ex}{$\stackrel{\displaystyle\curvearrowright}{\scriptstyle{#3}}$}\:{#2}}

\newcommand{\leads}{\rightsquigarrow} %AMS

\newenvironment{ites}{\vspace*{1ex}\par\noindent 
   \begin{tabular}{r@{\ \ }rcl}}{\vspace*{1ex}\end{tabular}\par\noindent}
\newcommand{\itt}[3]{{\bf #1.} & $#2$ & $\impl$ & $#3$ \\[1ex]}
\newcommand{\itte}[3]{{\bf #1.} & $#2$ & $\impl$ & $#3$ }
\newcommand{\itteq}[3]{\hline {\bf #1} & & & $#2=#3$ }
\newcommand{\itteqc}[3]{\hline {\bf #1} &  &  & $#2=#3$ \\[.5ex]}
\newcommand{\itteqq}[3]{{\bf #1} &  &  & $#2=#3$ }
\newcommand{\itc}[2]{{\bf #1.} & $#2$ &    \\[.5ex]}
\newcommand{\itcs}[3]{{\bf #1.} & $#2$ & $\impl$ & $#3$  \\[.5ex] }
\newcommand{\itco}[3]{   & $#1$ & $#2$  & $#3$ \\[1ex]}
\newcommand{\itcoe}[3]{   & $#1$ & $#2$  & $#3$}
\newcommand{\bit}{\begin{ites}}
\newcommand{\eit}{\end{ites}}
\newcommand{\na}[1]{{\bf #1.}}
\newenvironment{iten}{\begin{tabular}[t]{r@{\ }rcl}}{\end{tabular}}
\newcommand{\ass}[1]{& \multicolumn{3}{l}{\hspace*{-1em}{\small{[{\em Assuming:} #1]}}}}

%%%%%%%%% nested comp's
\newenvironment{itess}{\vspace*{1ex}\par\noindent 
   \begin{tabular}{r@{\ \ }lllcl}}{\vspace*{1ex}\end{tabular}\par\noindent}
\newcommand{\bitn}{\begin{itess}}
\newcommand{\eitn}{\end{itess}}
\newcommand{\comA}[2]{{\bf #1}& $#2$ \\ }
\newcommand{\comB}[3]{{\bf #1}& $#2$ & $#3$\\ }
\newcommand{\com}[3]{{\bf #1}& & & $#2$ & $\impl$ & $#3$\\[.5ex] }

\newcommand{\comS}[5]{{\bf #1} 
   & $#2$ & $#3$ & $#4$ & $\impl$ & $#5$\\[.5ex] }

%%%%%%%%%%%%%%%%
\newtheorem{CLAIM}{Proposition}[section]
\newtheorem{COROLLARY}[CLAIM]{Corollary}
\newtheorem{THEOREM}[CLAIM]{Theorem}
\newtheorem{LEMMA}[CLAIM]{Lemma}
\newcommand{\MyLPar}{\parsep -.2ex plus.2ex minus.2ex\itemsep\parsep
   \vspace{-\topsep}\vspace{.5ex}}
\newcommand{\MyNumEnv}[1]{\trivlist\refstepcounter{CLAIM}\item[\hskip
   \labelsep{\bf #1\ \theCLAIM\ }]\sf\ignorespaces}
\newenvironment{DEFINITION}{\MyNumEnv{Definition}}{\par\addvspace{0.5ex}}
\newenvironment{EXAMPLE}{\MyNumEnv{Example}}{\nopagebreak\finish}
\newenvironment{PROOF}{{\bf Proof.}}{\nopagebreak\finish}
\newcommand{\finish}{\hspace*{\fill}\nopagebreak 
     \raisebox{-1ex}{$\Box$}\hspace*{1em}\par\addvspace{1ex}}
\renewcommand{\abstract}[1]{ \begin{quote}\noindent \small {\bf Abstract.} #1
    \end{quote}}
\newcommand{\B}[1]{{\rm I\hspace{-.2em}#1}}
\newcommand{\Nat}{{\B N}}
\newcommand{\bool}{{\cal B}{\rm ool}}
\renewcommand{\c}[1]{{\cal #1}}
\newcommand{\Funcs}{{\cal F}}
%\newcommand{\Terms}{{\cal T}(\Funcs,\Vars)}
\newcommand{\Terms}[1]{{\cal T}(#1)}
\newcommand{\Vars}{{\cal V}}
\newcommand{\Incl}{\mathbin{\prec}}
\newcommand{\Cont}{\mathbin{\succ}}
\newcommand{\Int}{\mathbin{\frown}}
\newcommand{\Seteq}{\mathbin{\asymp}}
\newcommand{\Eq}{\mathbin{\approx}}
\newcommand{\notEq}{\mathbin{\Not\approx}}
\newcommand{\notIncl}{\mathbin{\Not\prec}}
\newcommand{\notCont}{\mathbin{\Not\succ}}
\newcommand{\notInt}{\mathbin{\Not\frown}}
\newcommand{\Seq}{\mathrel{\mapsto}}
\newcommand{\Ord}{\mathbin{\rightarrow}}
\newcommand{\M}[1]{\mathbin{\mathord{#1}^m}}
\newcommand{\Mset}[1]{{\cal M}(#1)}
\newcommand{\interpret}[1]{[\![#1]\!]^{A}_{\rho}}
\newcommand{\Interpret}[1]{[\![#1]\!]^{A}}
%\newcommand{\Comp}[2]{\mbox{\rm Comp}(#1,#2)}
\newcommand{\Comp}[2]{#1\diamond#2}
\newcommand{\Repl}[2]{\mbox{\rm Repl}(#1,#2)}
%\newcommand\SS[1]{{\cal S}^{#1}}
\newcommand{\To}[1]{\mathbin{\stackrel{#1}{\longrightarrow}}}
\newcommand{\TTo}[1]{\mathbin{\stackrel{#1}{\Longrightarrow}}}
\newcommand{\oT}[1]{\mathbin{\stackrel{#1}{\longleftarrow}}}
\newcommand{\oTT}[1]{\mathbin{\stackrel{#1}{\Longleftarrow}}}
\newcommand{\es}{\emptyset}
\newcommand{\C}[1]{\mbox{$\cal #1$}}
\newcommand{\Mb}[1]{\mbox{#1}}
\newcommand{\<}{\langle}
\renewcommand{\>}{\rangle}
\newcommand{\Def}{\mathrel{\stackrel{\mbox{\tiny def}}{=}}}
\newcommand{\impl}{\mathrel\Rightarrow}
\newcommand{\then}{\mathrel\Rightarrow}
\newfont{\msym}{msxm10}

\newcommand{\false}{\bot}
\newcommand{\true}{\top}

\newcommand{\restrict}{\mathbin{\mbox{\msym\symbol{22}}}}
\newcommand{\List}[3]{#1_{1}#3\ldots#3#1_{#2}}
\newcommand{\col}[1]{\renewcommand{\arraystretch}{0.4} \begin{array}[t]{c} #1
  \end{array}}
\newcommand{\prule}[2]{{\displaystyle #1 \over \displaystyle#2}}
\newcounter{ITEM}
\newcommand{\newITEM}[1]{\gdef\ITEMlabel{ITEM:#1-}\setcounter{ITEM}{0}}
\makeatletter
\newcommand{\Not}[1]{\mathbin {\mathpalette\c@ncel#1}}
\def\LabeL#1$#2{\edef\@currentlabel{#2}\label{#1}}
\newcommand{\ITEM}[2]{\par\addvspace{.7ex}\noindent
   \refstepcounter{ITEM}\expandafter\LabeL\ITEMlabel#1${(\roman{ITEM})}%
   {\advance\linewidth-2em \hskip2em %
   \parbox{\linewidth}{\hskip-2em {\rm\bf \@currentlabel\
   }\ignorespaces #2}}\par \addvspace{.7ex}\noindent\ignorespaces}
\def\R@f#1${\ref{#1}}
\newcommand{\?}[1]{\expandafter\R@f\ITEMlabel#1$}
\makeatother
\newcommand{\PROOFRULE}[2]{\trivlist\item[\hskip\labelsep {\bf #1}]#2\par
  \addvspace{1ex}\noindent\ignorespaces}
\newcommand{\PRULE}[2]{\displaystyle#1 \strut \over \strut \displaystyle#2}
%\setlength{\clauselength}{6cm}
%% \newcommand{\clause}[3]{\par\addvspace{.7ex}\noindent\LabeL#2${{\rm\bf #1}}%
%%   {\advance\linewidth-3em \hskip 3em
%%    \parbox{\linewidth}{\hskip-3em \parbox{3em}{\rm\bf#1.}#3}}\par 
%%    \addvspace{.7ex}\noindent\ignorespaces}
\newcommand{\clause}[3]{\par\addvspace{.7ex}\noindent
  {\advance\linewidth-3em \hskip 3em
   \parbox{\linewidth}{\hskip-3em \parbox{3em}{\rm\bf#1.}#3}}\par 
   \addvspace{.7ex}\noindent\ignorespaces}
\newcommand{\Cs}{\varepsilon}
\newcommand{\const}[3]{\Cs_{\scriptscriptstyle#2}(#1,#3)}
\newcommand{\Ein}{\sqsubset}%
\newcommand{\Eineq}{\sqsubseteq}%


\voffset -1cm


\newcommand{\comt}[1]{... \ \hfill{\small{[#1 {\bf !}]}}}
%--------------------------------------------------------------------
% the document

\begin{document} 

\hfill\today
% 
\section{Labeled multialgebras...}

\begin{Definition}
Given a specification $SP=\<\Sigma,\Phi\>$, we define two relations on
$\GTerms\times\GTerms$ \vspace*{-1ex}
\[ s\incleq t \iff SP\models s\prec t\ \ \  and\ \ \ 
   s\seteq t\iff s\incleq t\land t\incleq s \vspace*{-1ex} \]
\end{Definition}
Obviously, $\incleq$ is an ordering relation while $\seteq$ a
$\Sigma$-congruence on $\GTerms$. Equivalence classes wrt. $\seteq$ will be
denoted by $[t]=\{s\in\GTerms:s\seteq t\}$, and the whole class
$\quot{\GTerms}{\seteq}$ by $\spTerms$. We will consider also other congruences on
ground terms satisfying $\seteq\ \subseteq\ \congrA$ -- the respective equivalence
classes will be denoted $[t]^\congrA$, and their collection
$\quot{\GTerms}{\congrA}$ by $\spTTerms\congrA$.

\begin{Definition}\label{de:labm}
A labeled $\Sigma$-multialgebra $A$ has the carrier $|A|=\under A$, for some
set $\under A$ which, however, has a structure described by the triple
$\<\under A,\G A, \congrA\>$ where:
\begin{ite}
\item $\congrA$ is a congruence on $\GTerms$ such that $\seteq\ \subseteq\ \congrA$ 
\item $\G A\subseteq\PSet {\under A}\times \PSetp{\aTerms}\times \PSet{\aTerms}$, with projections
$set:\G A\into\PSet{\under A}$, $lab_1:\G A\into\PSetp{\aTerms}$,  $lab_2:\G
A\into\PSet{\aTerms}$. (We write $lab(X)=lab_1(X)\cup lab_2(X)$.)
\end{ite}
$\G A$ contains all and only labeled sets interpreting ground terms: for each
$t\in\GTerms$, its interpretation in $A$, $t^A\in\G A$,
satisfies the following conditions ($X,Y\in\G A$):
\begin{enum}
\item $[t]^\congrA\in lab_1(t^A)$
\item $[t]^\congrA\in lab(X) \impl set(t^A)\subseteq set(X)$
\item $[t]^\congrA\in lab(X) \land s\incleq t \impl [s]^\congrA\in lab(X)$
\item $lab_2(X)\not=\emptyset \impl |set(X)|=1 \land |lab_1(X)|=1$
\item $lab_1(X)=lab_1(Y)\land set(X)=set(Y) \impl X=Y\ \  (lab_2(X)=lab_2(Y))$
\item $lab(X)\subseteq lab(Y) \impl set(X)\subseteq set(Y)$
\item $set(X)=\{e\}=set(Y) \land |lab(X)|=1=|lab(Y)| \impl lab(X)=lab(Y)$.\\ 
\comt{to avoid $\{e\}^s \not\eleq \{e\}^t$, while $m(z)=\{e\}\incl s, t$.}
\end{enum}
We furthermore distinguishe the following sets:
\begin{ite}
\item $\Gone A= \{ X\in\G A : |set(X)|=1\}$ -- labeled sets
with 1 element
%$\Gone A= \bigcup \{ set(X):X\in\G A\land |set(X)|=1\}\subseteq\under A$ 
%containing only elements of 1-element sets from $\G A$. 
\item $\Gtwo A = \{X\in\G A:|set(X)|\not=1\}$ -- labeled sets with no or at least two elements
\item $\restA=\under A\setminus\Gone A$
\end{ite}
We won't distinguish between $\Gone A$ and the set of its elements, i.e., the
set $\bigcup\{set(X):X\in\Gone A\}$. For instance, 
writing $a\in\Gone A$ for an $a\in\under A$, we mean that there exists a
labeled set $X\in\Gone A$ such that $set(X)=\{a\}$. This justifies the
notation in the last point. 

\comt{In general $\G A \not= \PSet{\Gone A}$, and even $\bigcup\G
A\not\subseteq\Gone A$.} 
%

Operations are  (multi)functions
$|A|\into \PSet{\under A}\uplus\G A$ such that:
\begin{ite}
\item $f^A(a_1...a_n) \in\left\{\begin{array}{ll}
   \G A & if\ all\ a_i\in\Gone A\\
   \PSet{\under A} & otherwise \end{array}\right.$
\item In the first case, each $\{a_i\}=t^A_i$ for some $t_i\in\GTerms$ and
$f^A(a_1...a_i) = f(t_1...t_n)^A$
%% \item In the second case, if for all $a_i\not\in\G A: \{a_i\}\in\Gone A$,
%% then $f^A(a_1...a_n) = \emptyset$. 
%%
%% \comt{$=set(f^A(\{a_1\}...\{a_n\}))$}
\end{ite}
For convenience, given a non-ground
term $t(x)$ and $a\in |A|$, we write $set(t(a))^A$ for 
the set $f(a)^A$.
\end{Definition}

\begin{Definition}\label{de:sat}
Assignment is a function $\alpha:X\into|A|$.
Labeled multialgebra $A\models_\alpha \phi$:
\begin{ite}
\item $A\models_\alpha t\eleq t:$ iff $ t\in\G A\ \land\
|set(t^A)|=1=|lab_1(t^A)|,$ or $t\not\in \G A\ \land\ |set(t^{A,\alpha})|=1$

%$\begin{array}[t]{ll}
%   t\in\G A\ \land\ |set(t^A)|=1=|lab_1(t^A)|,\ or \\
%   t\not\in \G A\ \land\ |set(t^{A,\alpha})|=1   \end{array}$
\item $A\models_\alpha s\eleq t:$ iff $A\models_\alpha s\eleq s\land t\eleq
t$, $set(s^{A,\alpha})=set(t^{A,\alpha})$ and if $s^A,t^A\in\G A$ then $lab_1(s^A)=lab_1(t^A)$.
%$\begin{array}[t]{ll}
%set(s^A)=\{e\}=set(t^A)\ \land\ 
%lab_1(s^A)= \{[s]\}=lab_1(t^A) & if\ s^A,t^A\in\Gone A\\
%   false & if\ s^A\ or\ t^A\in\Gtwo A \\
%set(s^{A,\alpha})=\{e\}=set(t^{A,\alpha}) & otherwise \end{array}$.\com
\item $A\models_\alpha s\incl t:$ iff $\begin{array}[t]{ll}
set(s^A)\subseteq set(t^A)\land lab(s^A)\subseteq lab(t^A) & if\
s^A,t^A\in\G A\\
set(s^{A,\alpha})\subseteq set(t^{A,\alpha}) & otherwise \end{array}$. \\
\comt{The first clause describes also the case when $s,t\in\GTerms$ and $set(s^A)=set(t^A)=\emptyset$.}
\end{ite}
\end{Definition}

\begin{Prop}
Given a specification $SP$: for every $A\in\MMS$ there is an $L\in\LMS$ such
that for every formula $\phi:A\models\phi\iff L\models \phi$.
\end{Prop}
\begin{Proof}
Given an $A\in\MMS$, we construct an $L\in\LMS$ by endowing appropriate sets
with adequate labels constructed from $\spTTerms\congrA$. The carrier
$|L|=|A|$ and for each ground
term $t\in\GTerms$, we define its interpretation  $t^L$ by
$set(t^L)=t^A$ and labeling these sets as follows (all $s\in\GTerms$):
\begin{enum}
\item if $t^A=\emptyset$ then $lab_1(t^L)=\{[t]^\congrA\}=\{[s]^\congrA:s^A=\emptyset\}$ and $lab_2(t^L)=\emptyset$.
\item if $|t^A|=1$ then  $lab_1(t^L)=\{[t]^\congrA\}$ and
$lab_2(t^L)=\{[s]^\congrA:A\models s\prec t\}\setminus\{[t]^\congrA\}$. In
fact, $lab_2(t^L) = \{[s]^\congrA\}$ where $s^A=\emptyset$, or $lab_2(t^L)=\emptyset$ if no such
$s\in\GTerms$ exists.
\item otherwise $lab_1(t^L)=\{[s]^\congrA:s\in\GTerms\land A\models
s\prec t\}$ and $lab_2(t^L)=\emptyset$.
%\item[$\bullet$] PROBLEM with labeling of $t$ if $A\models t\eleq t$ and $A\models s\prec t$, because $s^A=\emptyset$!
\end{enum}
All operations are interpreted as in $A$, with the exception that applied to
ground terms, they return sets in $\G L$ rather than in $\PSet{L}$. It is
trivial to check that $L$ is a labeled multialgebra according to definition~\ref{de:labm}.

For any ground term $t:set(t^L) = t^A$, thus the only thing which
might change satisfaction of some formulae is the labeling affecting
relations between ground terms. However, for all $s,t\in\GTerms:$
\begin{ite}
\item if $A\models s\eleq t$ then $[s]^\congrA=[t]^\congrA$; by the point
2. we will have $set(s^L)=set(t^L)$ and
$lab_1(s^L)=\{[s]^\congrA\}=lab_1(t^L)$, i.e., $L\models s\eleq t$,
\item if $A\not\models t\eleq t$ then $|t^A|\not=1$ and hence $L\not\models
t\eleq t$
\item if $A\not\models s\eleq t$ while both $A\models s\eleq s$ and
$A\models t\eleq t$, then $s^A\not= t^A$, and hence $L\not\models s\eleq t$.
\end{ite}
Thus $L\models s\eleq t\iff A\models s\eleq t$, for all $s,t\in\GTerms$.
Similarly, for ground inclusions $s\prec t$, we have:
\begin{ite}
\item if $A\models s\prec t$ then $s^A\subseteq t^A$. But then also for any
other ground term $k:A\models k\prec s \impl A\models k\prec t$, and so
$lab(s^L)\subseteq lab(t^L)$,
\item if $A\not\models s\prec t$ then $s^A\not\subseteq t^A$ and so
$L\not\models s\prec t$.\\
\comt{if $A\models t\eleq t$ and $A\models s\prec t$, because $set(s^A)=\emptyset$,
we get $L\models s\prec t$ because of 1. and 2.}
\end{ite}
Hence $L\models s\prec t\iff A\models s\prec t$ for all $s,t\in\GTerms$. Since $set(t^L)=t^A$ for all
ground terms and interpretation of all operations returns the same sets in
$L$ as it did in $A$, we will have that for any assignment $\alpha:X\into
|L|=|A|$, $set(t^{L,\alpha})=t^{A,\alpha}$. Thus $L\models \phi\iff
M\models\phi$ for all formulae.
\end{Proof}

\begin{Corollary}
$\LMS\models\phi\ \impl\ \MMS\models\phi$
\end{Corollary}
However, the opposite implication does {\em not} hold.
\begin{Example}
There are {\em more} labeled models than multialgebras. The specification:

$\begin{array}{ll}
1. & a\eleq a\ \land\  b\eleq b\ \land \\
2. & s\prec a\ \land\  s\prec b\ \land \\
3. & a\not\eleq b\ \ \land \\
4. & s\not\prec c
\end{array}$

\noindent
has no multialgebraic models! 1., 2. and 3. force in any multialgebra $A:
s^A=\emptyset$, which is inconsistent with 4. But it has a labeled model:

$\begin{array}{l@{\ \mapsto\ }ll}
a & \{a\}^{\{a\},\{s\}} & (or\ X^{\{a\},\{s\}},\ for\ any\ X\ s.t.\ |X|=1) \\[.5ex]
b & \{b\}^{\{b\},\{s\}} & (or\ X^{\{b\},\{s\}},\ for\ any\ X\ s.t.\ |X|=1) \\[.5ex]
c & \emptyset^{\{c\}} & (or\ X^{\{c\}},\ for\ any\ X) \\[.5ex]
s & \emptyset^{\{s\}}
\end{array}$
\end{Example}

\section*{Initial models}
%
\begin{Claim}
Need: for (the) initial structure $T(SP)$ to be a model, for every ground
atom: $T(SP)\models atom\iff SP\models
atom$ ($s\eleq t$ or $s\prec t$, with $s,t\in\GTerms$).
\end{Claim}
\begin{Definition}
Given $SP=\<\Sigma,\Gamma\>$, we define the structure $T(SP)$:
\begin{ite}
\item $|T(SP)| = \{[t]:SP\models t\eleq t\}$, (sorted by $s\in S$).
\item $f^{T(SP)}([t_1]...[t_n])=X$, where
$set(X) = \{ [p]: p\in\GTerms\land SP\models p\eleq
p\land SP\models p\prec f(t_1...t_n)\}$ and (we let $f(t_1...t_n)=t$):
\begin{ite}
\item[a)] if $SP\models t\eleq t$ then: 
$lab_1(X) = \{[t]\}$ and $lab_2(X) 
= \bigcup\{lab(p^{T(SP)}):SP\models p\prec t\}\setminus\{[t]\}$,\vspace*{.5ex}
\item[b)] if $SP\not\models t\eleq t$ then: 
$lab_1(X) = \{[t]\}\cup \bigcup\{lab(p^{T(SP)}):SP\models p\prec
t\}$, and $lab_2(X)=\emptyset$.
\end{ite}
\end{ite}
\end{Definition}

\[\begin{array}{r|l|l}
L\models & t^L & s^L\\ \hline\hline
t\eleq t &  \{\bullet\}^{\{[t]\},\emptyset} & \\ 
t\eleq t \land s\prec t \land s\not\eleq s &  \{\bullet\}^{\{[t]\},\{[s]\}} &
\emptyset^{\{[s]\},\emptyset}\\
t\eleq t \land s\prec t \land s\eleq s &  \{\bullet\}^{\{[t]=[s]\},\emptyset} & \{\bullet\}^{\{[s]=[t]\},\emptyset} \\
t\not\eleq t \land s\prec t \land s\not\eleq s&  \{\bullet\}^{\{[t],[s]\},\emptyset} & 
\emptyset^{\{[s]\},\emptyset} \\
t\not\eleq t \land s\prec t\land s\eleq s &  \{\bullet\}^{\{[t],[s]\},\emptyset} & \{\bullet\}^{\{[s]\},\emptyset}
\end{array}
\]
$T(SP)$ is obviously a labeled {\em multi}algebra (additive). 

\begin{Definition}
A homomorphism of labeled multialgebras $\phi:A\into B$ is a total function
$h:\under A\into \under B$ such that for all $a_i\in\under
A:\phi(set(f^A(a_1...a_n))) \subseteq set(f^B(h(a_1)...h(a_n)))$.
\end{Definition}

%\bibliographystyle{bibNo}
%\bibliography{my}

\end{document}


