\documentstyle[a4wide,leqno,11pt]{article}
%\makeatletter
%\show\label
%\makeatother

\newtheorem{example}{Example}[section]
\newtheorem{definition}[example]{Definition}
\newtheorem{theorem}[example]{Theorem}
\newtheorem{lemma}[example]{Lemma}
\newtheorem{fact}[example]{Fact}
\newtheorem{proposition}[example]{Proposition}
\newtheorem{corollary}[example]{Corollary}

%equalities
\newcommand{\eequal}{\stackrel{e}{=}}%existential equality
\newcommand{\sequal}{\stackrel{s}{=}}%strong equality
\newcommand{\wequal}{\stackrel{w}{=}}%weak equality
\newcommand{\eleq}{\mathrel{\dot{=} }}%element equality
\newcommand{\seq}{\mathrel{\asymp}}% set equality 

\newcommand{\exist}{\downarrow}% defindnes in partial algebra
\newcommand{\st}{\ast}% extra star element


%different Categorical, arrows...
\newcommand{\embd}{\hookrightarrow}% embedding of institutions 
\newcommand{\pfunc}{\hookrightarrow}% partial function
\newcommand{\ring}{\circ}% composition of arrows

\newlength{\listskip}\setlength{\listskip}{1ex plus .2ex minus .5ex}
\newlength{\eqpskip}\setlength{\eqpskip}{.5ex plus .2ex minus .2ex}
\newenvironment{eqp}{\par %\topsep\eqpskip \parskip 0ex
   \begin{tabbing}
   \quad\=\+\quad\=$\{$\ \=\kill\(}{\)\end{tabbing}
   \addvspace{-1ex plus.2ex minus.3ex}} 
\newcommand{\comment}[2]{\)\-\\[\eqpskip]$#1$\>\>$\{$\>\+\+\ignorespaces#2$\}$\-\\[\eqpskip]\(}
\newcommand{\nline}{\)\\\(\mbox{}}

%%%%%%% MICHAL

%%% comments to be removed eventually
\newcommand{\fix}{\par\noindent\hspace*{-2em}{\large\bf fix} $\uparrow$ \dotfill}
\newcommand{\fixx}[1]{\par\noindent\hspace*{-2em}{\large\bf fix} $\uparrow$ {\small
#1} \dotfill}
\newcommand{\fixd}[1]{\par\noindent\hspace*{-2em}{\large\bf fix} $\downarrow$ {\small
#1} \dotfill}
\newcommand{\isit}[1]{\par\noindent\hspace*{-2em}{\small{{\bf [is it right?} #1{\bf ]}}}\par\noindent}
\newcommand{\how}[1]{\par\noindent\hspace*{-2em}{\small{{\bf [how?} #1{\bf
]}}}\par\noindent}
\newcommand{\todo}[1]{\par\noindent\hspace*{-2em}{\small{{\bf [To Do:} #1{\bf ]}}}\par\noindent}
\newcommand{\noo}[1]{\par\noindent\hspace*{-2em}{\large\bf NO} $\uparrow$ {\small
#1} \dotfill}
%%% environments
\newtheorem{schem}[example]{Schema}
\newcommand{\MyLPar}{\parsep -.2ex plus.2ex minus.2ex\itemsep\parsep
   \vspace{-\topsep}\vspace{.5ex}}
\newenvironment{PROOF}{{\bf Proof.}}{\nopagebreak\finish}
\newenvironment{PROOFs}{{\bf Proof (sketch)}}{\nopagebreak\finish}
\newcommand{\finish}{\hspace*{\fill}\nopagebreak 
     \raisebox{-1pt}{$\Box$}\par\addvspace{1.5ex}\noindent}

%%% Specs:
\newcommand{\spec}[1]{\begin{array}[t]{rrl}#1\end{array}\vspace*{1ex}}
\newcommand{\tit}[1]{\multicolumn{3}{l}{#1}}

%%% symbols
\newcommand{\eeq}{\eequal}
\newcommand{\PSet}[1]{{\mathcal{P}}(#1)}
\newcommand{\ovr}[1]{\overline{#1}}
\newcommand{\To}{\Rightarrow}
\newcommand{\Tod}{\Leftarrow}
\newcommand{\Iff}{\Longleftrightarrow}
\newcommand{\hviss}{\Iff}
\newcommand{\ok}[1]{d_{#1}} % OK part of a sort #1

\newcommand{\by}[2]{\stackrel{#1}{#2}}
\newcommand{\Toby}[1]{\ \by{#1}{\Longrightarrow}\ }

\newcommand{\adj}{\mathrel{\small\dashv}} %adjunction


%%% use these for signatures
\newcommand{\Sorts}{{S}} 
\newcommand{\Ops}{\Omega}
\newcommand{\ndc}{\Pi}
\newcommand{\err}{E}
\newcommand{\sign}{(\Sorts,\Ops,\ndc)}
\newcommand{\POps}{P\Omega}
\newcommand{\Pops}{\POps}
\newcommand{\Terms}[1]{T(#1)}
\newcommand{\TermsS}{\Terms\Sigma}
\newcommand{\TermsSX}{\Terms{\Sigma,X}}
\newcommand{\dom}{{\bf dom}}

%%% general categorical concepts
\newcommand{\cat}[1]{{\bf #1}} %put it around any category 
\newcommand{\inst}[1]{{\mathcal{#1}}} %put it around any institution: needs $_$
\newcommand{\fu}[1]{{\sf {#1}}} %put it around any functor
\newcommand{\thr}[1]{{\bf #1}} %put it around any Theory
\newcommand{\obj}[1]{|#1|} %objects of a category
\newcommand{\natt}{\mathrel{\Longrightarrow}} %natural transformation
%%% Categories
%multialgebras
\newcommand{\MA}[1]{\cat{MAlg}_{#1}}
\newcommand{\MAS}{\cat{MAlg}_{\Sigma}} %partial multialgebras
\newcommand{\PMA}[1]{\cat{PMAlg}_{#1}}
\newcommand{\PMAS}{\cat{PMAlg}_{\Sigma}} %partial algebras
\newcommand{\PAl}[1]{\cat{PAlg}_{#1}}
%used little but still...
\newcommand{\PAlo}{\PAl{(\Sorts,\Ops)}}
\newcommand{\MAlo}{\MA{(\Sorts,\Ops,\emptyset)}}
%standard
\newcommand{\Sign}{\cat{Sign}}
\newcommand{\Set}{\cat{Set}}

%%% Functors
\newcommand{\Mod}{\fu{Mod}}
\newcommand{\Sen}{\fu{Sen}}
\newcommand{\sen}{\Sen}

%%% Institutions
\newcommand{\MAH}{\inst{MAH}}
\newcommand{\MAC}{\inst{MAC}}
%\newcommand{\MAPC}{\inst{MAPC}}
\newcommand{\MAP}{\inst{MAP}}%partial multialgebras
\newcommand{\PA}{\inst{PA}}%partial algebras

%%% Equivalences 
\newcommand{\quot}{\sim} % ekvivalens tegn
\newcommand{\kernel}[1]{\quot_{#1}} %% ekvivalens med subskript
\newcommand{\qu}[2]{#1/\!_{#2}}

%%% some special functions used in specifications
\newcommand{\ite}[3]{\mathit{if}\ #1\ \mathit{then}\ #2\ \mathit{else}\ #3}%if then else
% and
\newcommand{\band}{\mathrel{\mathit{and}}}% ``and''
\newcommand{\choice}{\sqcup} % Nondeterministic choice


\begin{document}

\section*{Specification of depth-first search in directed reflexive graphs}

Specification of graph theory problems and algorithms is an interesting
application area for nondeterminism. One reason for this is
that graphs are defined as set, while their presentation in computer memory
uses deterministic data structures, such as lists. To abstract from this
inessential determinism we need some ``set-oriented''
specification language, like the presented one. The problem in the example
consists in finding traverse order through all vertices of a given graph
following the well known, in the algorithmic graph theory, procedure called {\em
depth-first search}. The presented specification is very abstract and it is
not easy to see (without comments) that a tree (defined by a function)
$T$ satisfying the specification is actually a DFS-tree.

The signature of the  specification contains
\begin{description}\MyLPar
\item[\(\cal V\)] --- the sort of vertices of graphs;
\item[\(V:\ \to \cal V\)] --- a nondeterministic constant, defining the set of
vertices of the given graph; 
\item[\(A:\cal V\to\cal V\)] --- a nondeterministic function defining sets
of neigbours of vertices, \(u\Incl A(v)\) means that the pair \(\<v,u\>\) is
an arc in the given graph;
\item[\(V_0:\ \to\cal V\)] --- a deterministic constant denoting the vertex
where the search begins;
\item[\(M:\cal V\to\cal V\)] --- a function defining vertices which were
visited before a vertex is visited. This corresponds to marking of graph
vertices, usually used in this procedure; 
\item[\(T:\cal V\to\cal V\)] --- a function defining a resulting DFS-tree,
like $A$ defines the given graph.
\end{description}

The specification of initial data contains such clauses:
\clause{G1}{ax:graph}{$A(V)\Incl V$,} 
   -- the pair \(\<V,A\>\) is a directed graph;
\clause{G2}{ax:refl-A}{$v\Incl A(v)$,} 
   -- reflexivity of the graph; it ensures both
   non-emptness of \(A(v)\) for each \(v\Incl V\), and totality of \(A(v)\) on
the sort \(\cal V\); 
\clause{V1} {ax:det-V0} {$V_0\Eq V_0$,} 
   -- uniqueness of \(V_0\);
\clause{V2}{ax:vert-V0}{\(V_0\Incl V\),}
   -- \(V_0\) is a vertex of the graph.

The following clauses determine some properties of the marking function $M$. This
function linearly orders all vertices because marked vertices never become
unmarked. The order relation ``a vertex $x$ is visited before $y$ 
(or \(x\Eq y\)'' is
written as \(x\Incl M(y)\). Clauses M1-M4 express the well-known properties
of linear order.
\clause{M1}{ax:refl-M}{\(v\Incl M(v)\) \quad (reflexivity);}
\clause{M2}{ax:trans-M}{\(x\notIncl M(y),y\notIncl M(z),x\Incl M(z)\) \quad
  (transitivity); }
\clause{M3}{ax:asym-M}{\(x\notIncl M(y),y\notIncl M(x),x\Eq y\) \quad
  (antisymmetricity);}
\clause{M4}{ax:lin-M}{\(x\Incl M(y),y\Incl M(x)\)\quad (linearity).}

The following three clauses relate $M$ to a given graph structure.
\clause{M5}{ax:M-V0}{\(M(V_0)\Eq V_0\) \quad (the search begins from $V_0$);}
\clause{M6}{ax:M-V}{\(M(V)\Incl V\) \quad (everything remains inside of $V$);}
\clause{M7}{ax:M-A}{\(u\Incl A(M(v)),A(M(v))\Incl M(u)\),}
   -- expresses some locality of $M$: if one reaches a vertex $u$ and
there are some neigbours of visited vertices which have not
been visited yet, then $u$ is also
one of such vertices -- one cannot leave a part of the graph before
all accessible vertices have been visited.

The following clauses describe $T$ and a complex interaction between $T$ and
$M$.   
\clause{T1}{ax:TinA}{\(T(v)\Incl A(v)\),}
  -- resulting tree is a subset of the graph's arcs;
\clause{T2}{ax:Tsuf}{\(A(M(v))\Incl T(M(v))\),}
  -- $T$ has enough arcs to reach all the neighbours of marked vertices;
%  \ref{ax:M-A};
\clause{T3}{ax:Tleaf}{\(A(v)\Incl M(v), T(v)\notIncl v\),}
  -- describes the conditions for $v$ to be a leaf (i.e., \(T(v)\Eq v\)) of the
  tree $T$: all neighbours of $v$ must be marked before $v$;
\clause{T4}{ax:Tnew}{\(u\notIncl T(v), v\Incl M(u)\),}
  -- monotonicity of $M$ along arcs of $T$; together with
  \ref{ax:asym-M} it says that $T(v)$ and $M(v)$ have only one common value
  $v$;
\clause{T5}{ax:Tmin}{\(u\notIncl T(v), z\Incl M(u),z\Incl M(v), A(z)\Incl
  M(u)\).} Refines the locality condition \ref{ax:M-A} to an optimality
requirement on $v$ and $u$:
all neighbours of any $z$ between them (in the
order defined by $M$) are visited before $u$.  

We will define an {\em ancestor} function $C$, which allows easy
formulation of the properties we want to express. The function $C$ is a
transitive closure of \(T^{-1}\). The following clauses relate $C$ and $T$.

\clause{C1}{ax:C-V0}{\(C(V_0)\Eq V_0\),}
   -- again, everything begins from $V_0$;
\clause{C2}{ax:C-T}{\(v\notIncl T(u),C(u)\Incl C(v)\),}
   -- monotonicity of $C$;
\clause{C3}{ax:tight-C}{\(u\notIncl T(v), z\notIncl C(u), z\Incl C(v), z\Eq u\),}
   -- $C$ cannot be ``too large''.

We should prove now that $C$ is transitive and antisymmetric:
\clause{C4}{th:trans-C}{\(x\notIncl C(y),y\notIncl C(z),x\Incl C(z)\),}
\clause{C5}{th:asym-C}{\(x\notIncl C(y),y\notIncl C(x),x\Eq y\).}

The more interesting property is that $C$ defines a tree-order
\clause{C6}{th:tree-C}{\(x\notIncl C(v), y\notIncl(v), x\Incl C(y), y\Incl
C(x)\),} i.e., $T$ defines a forest, not only an acyclic graph
(acyclicity is the same as antisymetricity).

For the case of non-oriented graphs, i.e., when the condition of symmetricity
for $A$
\clause{NO}{ax:sym-A}{\(x\notIncl A(y),y\Incl A(x)\)}
holds, the characteristic property of DFS-trees (forests, in general) can
be formulated as
\clause{CH}{th:DFS}{\(x\notIncl A(y),x\Incl T(y),x\Incl C(y),y\Incl C(x)\).}
It says that any edge from $A$ not used in $T$ connects some vertex and its
ancestor. 

Unfortunately, we can not prove either \ref{th:trans-C} or
\ref{th:asym-C} (nor some other properties). What we get in the proof search are
infinitely growing formulas. Properties like \ref{ax:C-T} or
\ref{ax:tight-C} localy relate $T$ and $C$ but do not say enough about the global
properties of $C$.  
\end{document}
