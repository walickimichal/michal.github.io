\documentclass[leqno]{article}
\makeatletter
%\show\label
\input{a4wide}
\makeatother

\usepackage{amssymb}

\newtheorem{example}{Example}[section]
\newtheorem{definition}[example]{Definition}
\newtheorem{theorem}[example]{Theorem}
\newtheorem{lemma}[example]{Lemma}
\newtheorem{fact}[example]{Fact}
\newtheorem{proposition}[example]{Proposition}
\newtheorem{corollary}[example]{Corollary}

%equalities
\newcommand{\eequal}{\stackrel{e}{=}}%existential equality
\newcommand{\sequal}{\stackrel{s}{=}}%strong equality
\newcommand{\wequal}{\stackrel{w}{=}}%weak equality
\newcommand{\eleq}{\mathrel{\dot{=} }}%element equality
\newcommand{\seq}{\mathrel{\asymp}}% set equality 

\newcommand{\exist}{\downarrow}% defindnes in partial algebra
\newcommand{\st}{\ast}% extra star element


%different Categorical, arrows...
\newcommand{\embd}{\hookrightarrow}% embedding of institutions 
\newcommand{\pfunc}{\hookrightarrow}% partial function
\newcommand{\ring}{\circ}% composition of arrows

\newlength{\listskip}\setlength{\listskip}{1ex plus .2ex minus .5ex}
\newlength{\eqpskip}\setlength{\eqpskip}{.5ex plus .2ex minus .2ex}
\newenvironment{eqp}{\par %\topsep\eqpskip \parskip 0ex
   \begin{tabbing}
   \quad\=\+\quad\=$\{$\ \=\kill\(}{\)\end{tabbing}
   \addvspace{-1ex plus.2ex minus.3ex}} 
\newcommand{\comment}[2]{\)\-\\[\eqpskip]$#1$\>\>$\{$\>\+\+\ignorespaces#2$\}$\-\\[\eqpskip]\(}
\newcommand{\nline}{\)\\\(\mbox{}}

%%%%%%% MICHAL

%%% comments to be removed eventually
\newcommand{\fix}{\par\noindent\hspace*{-2em}{\large\bf fix} $\uparrow$ \dotfill}
\newcommand{\fixx}[1]{\par\noindent\hspace*{-2em}{\large\bf fix} $\uparrow$ {\small
#1} \dotfill}
\newcommand{\fixd}[1]{\par\noindent\hspace*{-2em}{\large\bf fix} $\downarrow$ {\small
#1} \dotfill}
\newcommand{\isit}[1]{\par\noindent\hspace*{-2em}{\small{{\bf [is it right?} #1{\bf ]}}}\par\noindent}
\newcommand{\how}[1]{\par\noindent\hspace*{-2em}{\small{{\bf [how?} #1{\bf
]}}}\par\noindent}
\newcommand{\todo}[1]{\par\noindent\hspace*{-2em}{\small{{\bf [To Do:} #1{\bf ]}}}\par\noindent}
\newcommand{\noo}[1]{\par\noindent\hspace*{-2em}{\large\bf NO} $\uparrow$ {\small
#1} \dotfill}
%%% environments
\newtheorem{schem}[example]{Schema}
\newcommand{\MyLPar}{\parsep -.2ex plus.2ex minus.2ex\itemsep\parsep
   \vspace{-\topsep}\vspace{.5ex}}
\newenvironment{PROOF}{{\bf Proof.}}{\nopagebreak\finish}
\newenvironment{PROOFs}{{\bf Proof (sketch)}}{\nopagebreak\finish}
\newcommand{\finish}{\hspace*{\fill}\nopagebreak 
     \raisebox{-1pt}{$\Box$}\par\addvspace{1.5ex}\noindent}

%%% Specs:
\newcommand{\spec}[1]{\begin{array}[t]{rrl}#1\end{array}\vspace*{1ex}}
\newcommand{\tit}[1]{\multicolumn{3}{l}{#1}}

%%% symbols
\newcommand{\eeq}{\eequal}
\newcommand{\PSet}[1]{{\mathcal{P}}(#1)}
\newcommand{\ovr}[1]{\overline{#1}}
\newcommand{\To}{\Rightarrow}
\newcommand{\Tod}{\Leftarrow}
\newcommand{\Iff}{\Longleftrightarrow}
\newcommand{\hviss}{\Iff}
\newcommand{\ok}[1]{d_{#1}} % OK part of a sort #1

\newcommand{\by}[2]{\stackrel{#1}{#2}}
\newcommand{\Toby}[1]{\ \by{#1}{\Longrightarrow}\ }

\newcommand{\adj}{\mathrel{\small\dashv}} %adjunction


%%% use these for signatures
\newcommand{\Sorts}{{S}} 
\newcommand{\Ops}{\Omega}
\newcommand{\ndc}{\Pi}
\newcommand{\err}{E}
\newcommand{\sign}{(\Sorts,\Ops,\ndc)}
\newcommand{\POps}{P\Omega}
\newcommand{\Pops}{\POps}
\newcommand{\Terms}[1]{T(#1)}
\newcommand{\TermsS}{\Terms\Sigma}
\newcommand{\TermsSX}{\Terms{\Sigma,X}}
\newcommand{\dom}{{\bf dom}}

%%% general categorical concepts
\newcommand{\cat}[1]{{\bf #1}} %put it around any category 
\newcommand{\inst}[1]{{\mathcal{#1}}} %put it around any institution: needs $_$
\newcommand{\fu}[1]{{\sf {#1}}} %put it around any functor
\newcommand{\thr}[1]{{\bf #1}} %put it around any Theory
\newcommand{\obj}[1]{|#1|} %objects of a category
\newcommand{\natt}{\mathrel{\Longrightarrow}} %natural transformation
%%% Categories
%multialgebras
\newcommand{\MA}[1]{\cat{MAlg}_{#1}}
\newcommand{\MAS}{\cat{MAlg}_{\Sigma}} %partial multialgebras
\newcommand{\PMA}[1]{\cat{PMAlg}_{#1}}
\newcommand{\PMAS}{\cat{PMAlg}_{\Sigma}} %partial algebras
\newcommand{\PAl}[1]{\cat{PAlg}_{#1}}
%used little but still...
\newcommand{\PAlo}{\PAl{(\Sorts,\Ops)}}
\newcommand{\MAlo}{\MA{(\Sorts,\Ops,\emptyset)}}
%standard
\newcommand{\Sign}{\cat{Sign}}
\newcommand{\Set}{\cat{Set}}

%%% Functors
\newcommand{\Mod}{\fu{Mod}}
\newcommand{\Sen}{\fu{Sen}}
\newcommand{\sen}{\Sen}

%%% Institutions
\newcommand{\MAH}{\inst{MAH}}
\newcommand{\MAC}{\inst{MAC}}
%\newcommand{\MAPC}{\inst{MAPC}}
\newcommand{\MAP}{\inst{MAP}}%partial multialgebras
\newcommand{\PA}{\inst{PA}}%partial algebras

%%% Equivalences 
\newcommand{\quot}{\sim} % ekvivalens tegn
\newcommand{\kernel}[1]{\quot_{#1}} %% ekvivalens med subskript
\newcommand{\qu}[2]{#1/\!_{#2}}

%%% some special functions used in specifications
\newcommand{\ite}[3]{\mathit{if}\ #1\ \mathit{then}\ #2\ \mathit{else}\ #3}%if then else
% and
\newcommand{\band}{\mathrel{\mathit{and}}}% ``and''
\newcommand{\choice}{\sqcup} % Nondeterministic choice



\begin{document}

\section{Specification of depth-first search in directed reflexive graphs}

Specification of graph theory problems and algorithms is an interesting
application area for nondeterminism. One reason for this is
that graphs are defined as set, while their presentation in computer memory
uses deterministic data structures, such as lists. To abstract from this
inessential determinism we need some ``set-oriented''
specification language, like the presented one. The problem in the example
consists in finding traverse order through all vertices of a given graph
following the well known, in the algorithmic graph theory, procedure called {\em
depth-first search}. The presented specification is very abstract and it is
not easy to see (without comments) that a tree (defined by a function)
$T$ satisfying the specification is actually a DFS-tree.
%
The signature of the  specification contains
\begin{description}\MyLPar
\item[\(\cal V\)] --- the sort of vertices of graphs;
\item[\(V:\ \to \cal V\)] --- a nondeterministic constant, defining the set of
vertices of the given graph; 
\item[\(A:\cal V\to\cal V\)] --- a nondeterministic function defining sets
of neigbours of vertices, \(u\Incl A(v)\) means that the pair \(\<v,u\>\) is
an arc in the given graph;
\item[\(V_0:\ \to\cal V\)] --- a deterministic constant denoting the vertex
where the search begins;
\item[\(M:\cal V\to\cal V\)] --- a function defining vertices which were
visited before a vertex is visited. This corresponds to marking of graph
vertices, usually used in this procedure; 
\item[\(T:\cal V\to\cal V\)] --- a function defining a resulting DFS-tree,
like $A$ defines the given graph.
\end{description}
The specification of initial data contains such clauses:
\clause{G1}{ax:graph}{$A(V)\Incl V$ 
   -- the pair \(\<V,A\>\) is a directed graph; }
\clause{G2}{ax:refl-A}{$v\Incl A(v)$,} 
   -- reflexivity of the graph; it ensures both
   non-emptness of \(A(v)\) for each \(v\Incl V\), and totality of \(A(v)\) on
the sort \(\cal V\); 
\clause{V1} {ax:det-V0} {$V_0\Eq V_0$,
   -- uniqueness of \(V_0\); }
\clause{V2}{ax:vert-V0}{\(V_0\Incl V\),
   -- \(V_0\) is a vertex of the graph. }

The following clauses determine some properties of the marking function $M$. This
function linearly orders all vertices because marked vertices never become
unmarked. The order relation ``a vertex $x$ is visited before $y$ 
(or \(x\Eq y\)'' is
written as \(x\Incl M(y)\). Clauses M1-M4 express the well-known properties
of linear order.
\clause{M1}{ax:refl-M}{\(v\Incl M(v)\) \quad (reflexivity);}
\clause{M2}{ax:trans-M}{\(x\notIncl M(y),y\notIncl M(z),x\Incl M(z)\) \quad
  (transitivity); }
\clause{M3}{ax:asym-M}{\(x\notIncl M(y),y\notIncl M(x),x\Eq y\) \quad
  (antisymmetricity);}
\clause{M4}{ax:lin-M}{\(x\Incl M(y),y\Incl M(x)\)\quad (linearity).}

The following three clauses relate $M$ to a given graph structure.
\clause{M5}{ax:M-V0}{\(M(V_0)\Eq V_0\) \quad (the search begins from $V_0$);}
\clause{M6}{ax:M-V}{\(M(V)\Incl V\) \quad (everything remains inside of $V$);}
\clause{M7}{ax:M-A}{\(u\Incl A(M(v)),A(M(v))\Incl M(u)\),}
   -- expresses some locality of $M$: if one reaches a vertex $u$ and
there are some neigbours of visited vertices which have not
been visited yet, then $u$ is also
one of such vertices -- one cannot leave a part of the graph before
all accessible vertices have been visited.

The following clauses describe $T$ and a complex interaction between $T$ and
$M$.   
\clause{T1}{ax:TinA}{\(T(v)\Incl A(v)\),}
  -- resulting tree is a subset of the graph's arcs;
\clause{T2}{ax:Tsuf}{\(A(M(v))\Incl T(M(v))\),}
  -- $T$ has enough arcs to reach all the neighbours of marked vertices;
%  \ref{ax:M-A};
\clause{T3}{ax:Tleaf}{\(A(v)\Incl M(v), T(v)\notIncl v\),}
  -- describes the conditions for $v$ to be a leaf (i.e., \(T(v)\Eq v\)) of the
  tree $T$: all neighbours of $v$ must be marked before $v$;
\clause{T4}{ax:Tnew}{\(u\notIncl T(v), v\Incl M(u)\),}
  -- monotonicity of $M$ along arcs of $T$; together with {\bf M3} %\ref{ax:asym-M} 
it says that $T(v)$ and $M(v)$ have only one common value
  $v$;
\clause{T5}{ax:Tmin}{\(u\notIncl T(v), z\Incl M(u),z\Incl M(v), A(z)\Incl
  M(u)\).} Refines the locality condition {\bf M7} %\ref{ax:M-A} 
to an optimality requirement on $v$ and $u$:
all neighbours of any $z$ between them (in the
order defined by $M$) are visited before $u$.  

We will define an {\em ancestor} function $C$, which allows easy
formulation of the properties we want to express. The function $C$ is a
transitive closure of \(T^{-1}\). The following clauses relate $C$ and $T$.

\clause{C1}{ax:C-V0}{\(C(V_0)\Eq V_0\),}
   -- again, everything begins from $V_0$;
\clause{C2}{ax:C-T}{\(v\notIncl T(u),C(u)\Incl C(v)\),}
   -- monotonicity of $C$;
\clause{C3}{ax:tight-C}{\(u\notIncl T(v), z\notIncl C(u), z\Incl C(v), z\Eq u\),}
   -- $C$ cannot be ``too large''.

We should prove now that $C$ is transitive and antisymmetric:
\clause{C4}{th:trans-C}{\(x\notIncl C(y),y\notIncl C(z),x\Incl C(z)\),}
\clause{C5}{th:asym-C}{\(x\notIncl C(y),y\notIncl C(x),x\Eq y\).}

The more interesting property is that $C$ defines a tree-order
\clause{C6}{th:tree-C}{\(x\notIncl C(v), y\notIncl(v), x\Incl C(y), y\Incl
C(x)\),} i.e., $T$ defines a forest, not only an acyclic graph
(acyclicity is the same as antisymetricity).

For the case of non-oriented graphs, i.e., when the condition of symmetricity
for $A$
\clause{NO}{ax:sym-A}{\(x\notIncl A(y),y\Incl A(x)\)}
holds, the characteristic property of DFS-trees (forests, in general) can
be formulated as
\clause{CH}{th:DFS}{\(x\notIncl A(y),x\Incl T(y),x\Incl C(y),y\Incl C(x)\).}
It says that any edge from $A$ not used in $T$ connects some vertex and its
ancestor. 

Unfortunately, we can not prove either {\bf C4} or {\bf C5} %\ref{th:trans-C} or \ref{th:asym-C} 
(nor some other properties). What we get in the proof search are
infinitely growing formulas. Properties like {\bf C2} and {\bf C3} %\ref{ax:C-T} or \ref{ax:tight-C} 
localy relate $T$ and $C$ but do not say enough about the global
properties of $C$.  

\section{Concrete algorithms ...}
\begin{itemize}\MyLPar
\item Does the above specify an algorithm OR its result?
\item Does it specify explicitly a data structure OR just relay on sets? And works only
because sets are the convenient data structure for this particular problem?
\item It should be checked agains constraint programming\ldots
\end{itemize}
In general, we do need an explicit data structure \ldots
%\begin{description}\MyLPar
%\item[
$\es: \to S$, 
%] -- constructs an empty set; the sort/data structure: sets of $\cal V$
%%\item[$\cup:S\times E\to S$] -- a set constructor
%%\item[$\setminus:S\times E\to S$] -- set minus
%\item[
$\ch:S\to E$ -- iterator (choice), 
%\item[
$|\_|:G\to S$ -- the set of all vertices in a graph \ldots
%\item[\ldots]
%\end{description}
Most (if not all) operations from the previous section work on the {\em defined} sets!
$g$ is the actually traversed graph and $t$ the DFS-tree constructed so far.
\bit %egin{itemize}
\itte{B}{}{B(g,v)\Eq A(g,v)\setminus v} -- auxiliary \\[.5ex]
\itt{D0}{}{D(g,v_0)\Incl Tr(g,v_0,\es)} 
% & \multicolumn{3}{l}{-- start constructing $T$ from empty graph} \\[.5ex]
 or & & & $T(g,v_0)\Seteq Tr(g,v_0,\es)$ \\[.5ex]
\itt{D1}{(B(g,v)\setminus |t|)\Eq\es}{Tr(g,v,t)\Incl t}
  & \multicolumn{3}{l}{-- backtrack if all neighbours were visited (leaf is a special case)} \\[.5ex]
\itt{D2}{x\Incl \ch.(B(g,v)\setminus |t|)} 
  {Tr(g,v,t)\Incl Tr(g,v, Tr(g,x,t\cup \<v,x\>))}
% & \multicolumn{3}{l}{-- go recursively to all the neighbours} \\[.5ex]
 or& & & $Tr(g,v,t)\Seteq TTr(g,v, t, B(g,v)\setminus |t|)$ 
\eit %nd{itemize}
where $TTr(g,v,t,s)\Seteq \bigcup_{x\Incl\ch.s}Tr(g,v, Tr(g,x,t\cup \<v,x\>))$ 
In those with $\Incl$, this relation expresses naturally the {\em progress} of the 
computation. Alternatively, keep explicitly the ``marked'' vertices:
\bit
\itt{D0}{}{D(g,v_0)\Incl Tr(g,v_0,\es,\{v_0\})|_1} 
\itt{D1}{(B(g,v)\setminus s)\Eq\es}{Tr(g,v,t,s)\Incl [t,s]}
\itte{D2}{x\Incl \ch.B(g,v), x\not\in s} 
  {Tr(g,v,t,s)\Incl Tr(g,v, Tr(g,x,t\cup \<v,x\>,s\cup x))}
\eit
% E.g. in D2., first find an $x$ from the RHS of $\Incl$ and then
% ($\impl$) compute LHS by computing RHS. In B., we might (?) write 
% $B(g,v)\Incl A(g,v)\setminus v$ -- if no other axioms have $B(g,v)$ in the LHS, this
% will be the only possibility \ldots

\subsection{Converting a natural $n$ to a binary representation} 
$/$ divides ignoring fractions:
\begin{ites}
\itt{B1}{}{bin(0)\Incl \estr} 
\itte{B2}{n> 0}{bin(n)\Incl bin(n/2) \app (n\ mod\ 2)}
\end{ites}

\subsection{First-fit Bin-packing}
Given a list of $n$ numbers (bins, initially 0, but here whatever), a list of 
$m$ numbers 
(perhaps $x_m\leq U$), we let
$P1:\Nat\times Seq_n\into Seq_n$ and $P:Seq_m\times Seq_n\into Seq_n:$\vspace*{1ex}

\noindent
\begin{iten}
\itt{P1}{}{P1(x,\estr)\Incl {\rm Err}} %\ [\estr\app x\ \ {\rm no\ }l]}
\itt{P2}{x+y\leq U}{P1(x,l\app y)\Incl l\app(x+y)}
\itte{P3}{x+y> U}{P1(x,l\app y)\Incl P1(x,l)\app y}
\end{iten}
\hspace*{.5em}
\begin{iten}
\itt{Pa}{}{P(\estr,l)\Incl l}
\itt{Pb}{}{P(s\app x,l)\Incl P(s,P1(x,l))}
\itte{PS}{}{PS(s,l)\Incl P(SrtDec(s),l)} \\[1ex]
\end{iten}
Replacing in {\bf P1.} Err by $\estr\app x$ will fit
$Seq_m$ into a minimum number of bins. {\bf PS.} is a variant sorting first 
$Seq_m$ in a decreasing order.

\newpage\subsection{Topological sort}
DAG $G$ is a pair $\<E,V\>$ where $E$ is the set of edges $xy$ (i.e., $\<x,y\>$).
(Notation $\upd Eon=E\setminus\{o\}\cup\{n\}$ for a set/function $E$ means: 
replace $o$ by $n$; $\rem Eo$ -- remove $o$; $\add En$ -- add $n$.)
% We write $x\Incl \ch.G$, resp. $yx\Incl \ch.G$, assuming that the choice is
% made for the appropriate set $V$, resp. $E$. Adding/removing edge/vertex affects only the
% respective component of the graph.

Indegree $I:Gr\times V\to\Nat$
\bit
\itt{I1}{}{I(\<\es,V\>,x) \Incl 0}
\itt{I2}{yz\Incl \ch.E, z\Eq x}{I(\<E,V\>,x) \Incl I(\<\rem E{yz},V\>,x) + 1} 
\itt{I3}{yz\Incl \ch.E, z\notEq x}{I(\<E,V\>,x) \Incl I(\<\rem E{yz},V\>,x)}
\itteqc{I?}{I(\<E,V\>,x)} {|\{y\in V:yx\in E\}|} 
\itteqq{}{\cp(I(E,V,x))}{|E|}
\eit
and the set of $0$-vertices $Z:G\into Set(V)$
\bit
\itt{Z1}{}{Z(\<E,\es\>) \Incl \es}
\itt{Z2}{x\Incl \ch.V, I(\<E,V\>,x)\notEq 0}{Z(\<E,V\>)\Incl Z(\<E,\rem Vx\>)}
\itt{Z3}{x\Incl \ch.V, I(\<E,V\>,x)\Eq 0}{Z(\<E,V\>)\Incl Z(\<E,\rem Vx\>)\cup x}
\itteqc{Z?}{Z(G)}{\{x\in V:I(G,x)=0\}} 
\itteqq{}{\cp(Z(E,V))}{|V|*\cp(I(E,V,x))}
\eit
Topological sort $TS: G \into Seq$
\bit
\itt{T1}{Z(G)\Eq\es}{TS(G)\Incl\estr}
\itte{T2}{x\Incl \ch.Z(G)}{TS(G)\Incl TS(U(G,x)) \app x}
\eit
and updating $G$ wrt. $x\in Z(G)$, i.e., removing all edges from $x$ -- $U:G\times V \into G$
\bit
\itt{U1}{}{U(\<\es,V\>,x)\Incl \<\es,\rem Vx\>}
\itt{U2}{ zy\Incl \ch.E, z\Eq x}{U(\<E,V\>,x)\Incl U(\<\rem E{zy},V\>,x)}
\itt{U3}{ zy\Incl \ch.E, z\notEq x}{U(\<E,V\>,x)\Incl U(\<\rem E{zy},V\>,x)\cup \< zy,\es\>}
\itteq{U?}{U(\<E,V\>,x)} {\<E\setminus \{xy\in E:y\in V\},V\setminus x\>}
\eit

\subsection*{Show ?}
If there is a path from $x$ to $y$ in $G$, then $y$ comes $aft$er $x$ in $TS(G)$ (since $G$ is a DAG, 
$pth(x,y,G)\impl x\notEq y$)

\noindent\begin{iten}
\itt{}{pth(x,y,G)}{y\in aft(x,TS(G))}
\itte{i.e}{y\in|D(G,x)|}{y\in aft(x,TS(G))}
\end{iten}
%% \(\begin{array}{rrcl}
%% & pth(x,y,G) & \impl & y\in aft(x,TS(G)) \\[.5ex]
%% or & y\in |D(G,x)| & \impl & y\in aft(x,TS(G))
%% \end{array}\)
\hspace{1em}
\begin{iten}
\itt{p1}{}{aft(x,\estr) \Incl \estr}
\itt{p2}{x\Eq z}{aft(x,l\app z)\Incl l}
\itt{p2}{x\notEq z}{aft(x,l\app z)\Incl aft(x,l)}
\end{iten}

\noindent
Apparently, everything is implicitly generated and induction is everywhere. Show?
\bit
\itt{?1}{y\in|D(G,x)|, x\Incl\ch.Z(G)}{y\notIncl\ch.Z(G)}
\itt{?2}{y\in|D(G,x)|, x\notIncl\ch.Z(G)}{y\notIncl\ch.Z(G)}
\itt{?3}{y\notIncl\ch.Z(G), t\Incl TS(G)}{y\in aft(head(t),t)}
\itte{?4}{y\in|D(G,x)|, x\Incl\ch.Z(G), z\Incl\ch.Z(G), x\notEq z}{y\in|D(U(G,z),x)|}
\eit
?2. takes care of two possibilities: either $x\not\in Z(G)$ or else $x\in Z(G)$ but it wasn't
returned by the actual computation of choice in axiom T2. Similarly ?3.

\newpage
\subsubsection{Top. sort: multiple calls $\iff$ data structures}
We convert a DAG $\<E,V\>$ into a set $G=\{\<x,P\>:x\in V\}$ where for each 
vertex $\<x,P\>$,
$P$ is the set of vertices $y$ such that $\<y,x\>\in E$.
This conversion amounts to initialization of the Indegree's for all vertices. 
The data structure is then $\<Z,G\>$ with $Z$ being the current (sub)set of 
0-vertices, i.e., $x\in Z\iff \<x,\es\>\in G$. 
The algorithm goes  as follows:
\bit
\itt{T1}{G\Incl Conv(\<E,V\>)}{TS(\<E,V\>)\Incl h(Z(G),G)}
\itteqc{}{\cp(TS(\<E,V\>))}{\cp(Conv(\<E,V\>))+\cp(Z(G)) + \cp(h(Z,G))}
\itcoe{}{}{{\rm where\ }G=Conv(\<E,V\>),\ Z=Z(G)}
\eit

\bit
\itt{h1}{}{h(\es,G)\Incl\estr}
\itt{h2}{z\Incl\ch.Z}{h(Z,G)\Incl h(U(\rem Zz,\rem G{\<z,\es\>},\rem G{\<z,\es\>},z))\app z}
\itteq{h?}{\cp(h(Z,G))}{\ ?\ \cp(Z)*\cp(U(Z,G,G))\ {\rm -\ dynamic\ }Z\ ?} \\[-1ex]
\eit
$U:Set(V)\times G\times G\times V\into \<Set(V),G\>$ removes $zx$ from $G$ 
and updates the 0-vertices:
%% \bit
%% \itt{U1}{}{U(Z,\es,G,z)\Incl \<Z,G\>}
%% \itt{U2}{\<x,P\>\Incl\ch.H, \{z\}\Eq P}    {U(Z,H,G,z) \Incl
%%        U(\add Zx, \rem H{\<x,P\>},\rem G{\<x,P\>},z)}
%% \itt{U3}{\<x,P\>\Incl\ch.H, z\in P, \{z\}\notEq P}{U(Z,H,G,z) \Incl 
%%        U(Z,\rem H{\<x,P\>}, \upd G{\<x,P\>}{\<x,P\setminus z\>},z)}
%% \itt{U4}{\<x,P\>\Incl\ch.H, z\not\in P}{U(Z,H,G,z)\Incl 
%%        U(Z,\rem H{\<x,P\>}, G, z)}
%% \ass{The initial call is for: $Z=Z(G)$ and a $z\in Z$.} \\[.5ex] \hline
%% \itt{U?}{U(Z,G,G,z)=\<Z',G'\>}{Z'=Z(G')\ \&\ G'=G\setminus (z\times V)\setminus (Z'\times V)}
%% \itteqq{}{\cp(U(Z,H,G,z))}{|H|*(\cp(\in P)+\cp(\rem H{}))} 
%% \eit

\noindent
$\begin{array}{rccl@{\ \ \impl\ \ }l}
{\bf U1.} &&& & {U(Z,\es,G,z)\Incl \<Z,G\>} \\[.5ex]
{\bf U2.} & \<x,P\>\Incl\ch.H & z\in P & \{z\}\Eq P  &  U(Z,H,G,z) \Incl
       U(\add Zx, \rem H{\<x,P\>},\rem G{\<x,P\>},z) \\[1ex]
{\bf U3.} & \into & \into & \{z\}\notEq P  & U(Z,H,G,z) \Incl 
       U(Z,\rem H{\<x,P\>}, \upd G{\<x,P\>}{\<x,P\setminus z\>},z) \\[1ex]
{\bf U4.} & \into  & z\not\in P & & U(Z,H,G,z)\Incl 
       U(Z,\rem H{\<x,P\>}, G, z) \\[.5ex]
&\multicolumn{4}{l}{[{\rm The\ initial\ call\ is\ for}: Z=Z(G), z\in Z.]} \\[.5ex] \hline
{\bf U?} &\multicolumn{3}{r@{\ \ \impl\ \ }}{U(Z,G,G,z)=\<Z',G'\>} & Z'=Z(G')\ \ 
     \&\ \ G'=G\setminus (z\times V)\setminus (Z'\times V) \\[.5ex]
  &\multicolumn{4}{r}{\cp(U(Z,H,G,z))=|H|*(\cp(\in P)+\cp(\rem H{}))} \\[-1ex] \ & \ 
\end{array}
$

\noindent
Initialization, $Conv:Set(E)\times Set(V)\into G$ goes as follows:
\bit
\itt{Co1}{}{Conv(E,\es)\Incl \es}
\itt{Co2}{x\Incl\ch.V}{Conv(E,V)\Incl Conv(E,\rem Vx)\cup C(E,x,\es)}
\itteqc{Co?}{Conv(E,V)}{\{C(E,x,\es):x\in V\}}
\itteqq{}{\cp(Conv(E,V))}{|V|*\cp(C(E,x,\es))} \\[2ex]
\itt{C1}{}{C(\es,x,P)\Incl \<x,P\>}
\itt{C2}{zy\Incl \ch.E, y\Eq x}{C(E,x,P)\Incl C(\rem E{zy},x,\add Pz)}
\itt{C3}{zy\Incl \ch.E, y\notEq x}{C(E,x,P)\Incl C(\rem E{zy},x,P)}
\itteqc{C?}{C(E,x,\es)}{\<x,\{z:zx\in E\}\>} 
\itteqq{}{\cp(C(E,x,\es))}{|E|*\cp(\rem E{})} \\
\itteq{}{\cp(Conv(E,V))}{|V|*|E|*\cp(\rem E{})}
\eit
Finally, $Z(G)$ is called only once at the start:
\bit
\itt{Z1}{}{Z(\es)\Incl \es}
\itt{Z2}{\<x,P\>\Incl \ch.G, P\notEq\es}{Z(G)\Incl Z(\rem G{\<x,P\>})}
\itt{Z3}{\<x,P\>\Incl \ch.G, P\Eq\es}{Z(G)\Incl Z(\rem G{\<x,P\>})\cup x}
\itteqc{Z?}{Z(G)}{\{x:\<x,\es\>\in G\}}
\itteqq{}{\cp(Z(G))}{|G|*\cp(\rem G{}),\ \ \ |G|=|V|} \\[-1ex]
\eit

\newpage
\subsection{MCST and Kruskal}
We write $xy$ for a pair $\<x,y\>$ -- probably, it may be later 
specified as an ordered pair (directed edge) or unordered set 
(undirected edge).

The data structure is an (irreflexive, connected) graph $\<E,V\>$ where $E$ is a 
set of pairs $xy$ 
for $x,y\in V$, and a function cost $cst:E\into\Nat$, such that 
$cst(xx)=\infty$.
%
\bit \hspace*{-1em}
\itt{MC1}{}{MCST(\es,V)\Incl \es}
\itt{MC2}{xy\Incl \ch.min(E)}{MCST(E,V)\Incl h(E, 
V, \{xy\}, Ne(E, xy)) } \\[-1ex]
\itt{h1}{}{h(E,V,T,\es)\Incl T}
\itt{h2}{xy\Incl \ch.min(N), y\not\in T}{h(E,V,T,N)\Incl h(E,
V, \add T{xy},Ne(E, \add T{xy})) }
\ass{each edge in $N$, i.e. $Ne(E,T)$ goes from $T$ outside it}
\eit 
%
For sets of edges $T\subseteq E$, $Ne(E,T)$ returns the set of minimal edges going 
out of $T$ into $E\setminus T$:
\bit
\itt{Ne}{}{Ne(E,T)\Incl n(E,T,\es)}
\itt{n0}{}{n(\es,T,C)\Incl C}
\itt{n1}{xy\Incl\ch.E, x\in T, y\in T}
      {n(E,T,C)\Incl n(\rem E{xy},T,C)}
\itt{n2}{xy\Incl\ch.E, x\not\in T, y\not\in T}
      {n(E,T,C)\Incl n(\rem E{xy},T,C)}
\itc{n3}{xy\Incl\ch.E, x\in T, y\not\in T, e\Incl Con(C,y)}
  \itco{cst(xy)<cst(e)}{\impl}{n(E,T,C)\Incl n(\rem E{xy}, T, \upd Ce{xy})}
\itc{n4}{xy\Incl\ch.E, x\in T, y\not\in T, e\Incl Con(C,y)}
  \itcoe{cst(xy)\geq cst(e)}{\impl}{n(E,T,C)\Incl n(\rem E{xy}, T, C)}
\eit
$C$ is a set of edges to vertices neighbouring current $T$, and for 
an $y\not\in T$, $Con(C,y)$ returns 
the edge (from $C$) connecting $y$ 
to $T$, or $yy$ if no such edge exists:
\bit
\itt{c1}{}{Con(\es,y)\Incl yy}
\itt{c2}{xy\Incl \ch.C}{Con(C,y)\Incl xy}
\itt{c3}{xz\Incl \ch.C, z\notEq y, x\notEq y}
     {Con(C,y)\Incl Con(\rem C{xz},y)}
\ass{at most one edge to $y$ in $C$ and, for {\bf Ne}, $cst(yy)=\infty$} 
\eit
%
$min(E)$ returns the subset of edges $E$ with minimal cost:
\bit
\itt{min}{}{min(E)\Incl m(E,\es,\infty)}
\itt{m1}{}{m(\es,T,n)\Incl T}
\itt{m2}{e\Incl\ch.E, cst(e) < n}{m(E,T,n)\Incl m(\rem Ee, \{e\},cst(e))}
\itt{m3}{e\Incl\ch.E, cst(e) = n}{m(E,T,n)\Incl m(\rem Ee, \add Te,n)}
 \itt{m4}{e\Incl\ch.E, cst(e) > n}{m(E,T,n)\Incl m(\rem Ee, T,n)}
\eit

\newpage
\subsubsection{Kruskal light}
$F$ is a forest -- initially, just the set of vertices $V$.
\bit
\itt{MC1}{}{MSCT(\es,V)\Incl \es}
\itt{MC2}{}{MCST(E,V)\Incl h(V, srt(E))} %\\[-1ex]
\itteq{}{\cp(MCST(E,V))}{\cp(srt(E) + h(V,E))}
\eit
$srt$ returns the sequence of edges sorted in increasing order according to weights,
\bit
\itt{h1}{}{h(F,\estr)\Incl F}
\itt{h2}{xy\Incl hd(S), DIF(F,xy)\Eq\true}{h(F,S)\Incl h(join(F,xy), tail(S))\cup xy}
\itt{h3}{xy\Incl hd(S), DIF(F,xy)\Eq\false}{h(F,S)\Incl h(F, tail(S))}
\itteq{}{\cp(h(E,V))}{|E|*\cp(DIF(F)+join(F))}
\eit
$DIF(F,e)$ returns true if $e$ connects different trees
from $F$
\bit
\itt{d1}{in(F,x)\Eq in(F,y)}{DIF(F,\es)\Incl \false}
\itt{d2}{in(F,x)\notEq in(F,y)}{DIF(F,\es)\Incl \true}
\itteq{}{\cp(DIF(F))}{2*in(F)}
\eit
where $in(F,x)$ returns the tree from $F$ to which $x$ belongs:
\bit
\itt{i1}{t\Incl \ch.F, x\in t}{in(F,x)\Incl t}
\itt{i2}{t\Incl \ch.F, x\not\in t}{in(F,x)\Incl in(\rem Ft,x)}
\ass{$x$ does belong to some tree in $F$} \\
\itteq{}{\cp(in(F))}{|F|*\_\in t}
\eit
Finally, $join(F,xy)$ joins trees from $F$ connected by $xy$ returning a new forest:
\bit
\itt{j1}{s\Incl in(F,x), t\Incl in(F,y)}
   {join(F,xy)\Incl \rem F{s,t}\cup \{s\cup t\cup xy\}}
\ass{$s\notEq t$ are tees} \\
\itteq{}{\cp(join(F))}{2* in(F) + 2*\rem F{}}
\eit
$\cp(MC) = srt(E) + |E|*(4*in(F)+ 2*\rem F{}) = srt(E)+|E|*(4*|F|*\in t + 2*\rem F{})$.
\subsubsection*{... a bit better:}
\bit
\itc{h2'}{xy\Incl hd(S), s\Incl in(F,x), }
 \itco{t\Incl in(F,y), s\notEq t}{\impl}{h(F,S)\Incl h(join(F,s,t,xy), tail(S))\cup xy}
\itc{h3'}{xy\Incl hd(S), s\Incl in(F,x),}
 \itco{t\Incl in(F,y), s\Eq t}{\impl}{h(F,S)\Incl h(F, tail(S))}
%\itteq{}{\cp(MCST(E,V))}{\cp(\ srt(E) + |E|*((2*in(F))+join(F))\ )}
\hline\multicolumn{4}{r}
{$\cp(MCST(E,V)) = \cp(\ srt(E) + |E|*((2*in(F))+join(F))\ )$}
\eit
and 
\bit
\itt{j1'}{}{join(F,s,t,xy)\Incl \rem F{s,t}\cup \{s\cup t\cup xy\}}
\itteq{}{\cp(join(F))}{2*(\rem F{})}
\eit
$\cp(MC)) = srt(E) + |E|*(2*in(F)+ 2*\rem F{})= srt(E) + |E|*(2*|F|*\in t+ 2*\rem F{})$


Here, a tree is a single vertex or a set of edges and membership $x\in T$ given by:
\bit
\itt{}{}{x\not\in\es}
\itt{}{e\Incl\ch.T, e\Eq x}{x\in T}
\itt{}{e\Incl\ch.T, e\Eq y,y\notEq x}{x\in T\Incl x\in \rem Te}
\itt{}{e\Incl\ch.T, e\Eq zy, z\Eq x}{x\in T}
\itt{}{e\Incl\ch.T, e\Eq zy, y\Eq x}{x\in T}
\itt{}{e\Incl\ch.T, e\Eq zy, z\notEq x, y\notEq x}{x\in T\Incl x\in \rem Te}
\eit

\newpage
\subsubsection*{or worse ...}
\bit
\itt{MC1}{}{MSCT(\es,V)\Incl \es}
\itt{MC1}{}{MCST(E,V)\Incl h(E,V)}\\[-1ex]
\itt{h1}{}{h(\es,F)\Incl F}
\itte{h2}{xy\Incl\ch.min(dif(F,E))}{h(E,F)\Incl h(E\setminus xy,join(F,xy))}
\eit
$min$ is as in {\bf MSCT}, $dif(F,E)$ returns the edges from $E$ connecting different trees
from $F$
\bit
\itt{d1}{}{dif(F,\es)\Incl \es}
\itt{d2}{xy\Incl\ch.E, in(F,x)\Eq in(F,y)}{dif(F,E)\Incl dif(F,E\setminus xy)}
\itte{d3}{xy\Incl\ch.E, in(F,x)\notEq in(F,y)}{dif(F,E)\Incl dif(F,E\setminus xy)\cup xy}
\eit
where $in(F,x)$ returns the tree from $F$ to which $x$ belongs:
\bit
\itt{i1}{t\Incl \ch.F, x\in t}{in(F,x)\Incl t}
\itt{i2}{t\Incl \ch.F, x\not\in t}{in(F,x)\Incl in(F\setminus t,x)}
\ass{$x$ does belong to some tree in $F$}
\eit
Finally, $join(F,xy)$ joins trees from $F$ connected by $xy$ returning a new forest:
\bit
\itt{j1}{s\Incl in(F,x), t\Incl in(F,y)}
   {join(F,xy)\Incl (F\setminus s\setminus t)\cup (s\cup t\cup xy)}
\ass{$s\notEq t$ are tees}
\eit

\newpage\section{Kind of ...}
general schema ?
\[A_1 \impl \ldots \impl A_n\ \ \impl\ \ L\Incl R\]
\begin{itemize}\MyLPar
\item 
To compute $L$ compute $R$, provided that all $A_i$ hold -- or, compute $A_1$, 
then $A_2$, etc., and finally compute $R$.
\item
$A_i$'s must bind all the variables as results of computations, i.e., if an $x$ occurs 
somewhere (not in $L$ but in $R$ or $A_j$), it must be introduced in some $A_i<A_j$ of 
the form $x\Incl C_i$. Then $A_j$ may check if it satisfies some condition $x\Eq C_j$.
\item
Shall we distinguish static conditions from those ``to be computed''? If there are 
several $x\Incl C_i\ldots x\Incl C_j$, how to interpret this?
Certainly, data structures should be specified statically -- not necessarily with the
algorithmic interpretation.
\item
For an algorithm, such restrictions, as well as the order of $A_i$'s matter. 
Also, things like {\bf U2} matter. Logically, they are irrelevant -- instead of {\bf U2}
we might well write 
\[xy\Incl \ch.E\ \impl\ \ U(\<E,V\>,x)\Incl U(\<E\setminus\<x,y,V\>,x) \]
\item Computations, like in {\bf T2} are inside-out.
\item LOGIC: since $\Incl$ is $:=$, its LHSs have 
 \begin{itemize}
  \item variables $:x\Incl\ldots z\Incl\ldots\impl\ldots$, or
  \item function to be computed $:\ldots \impl F(x)\Incl\ldots$ (-- with variables?)
 \end{itemize}
\end{itemize}
Finally ...
\begin{itemize}\MyLPar
\item Isn't it just 
 \begin{itemize}\MyLPar
 \item recursion (show it?): {\tt if} and {\tt while}?
 \item {\tt Prolog}?
 \item rewriting? (What are the differences?)
 \item constructive specifications? (We need not be executable?)
 \item algorithmic logic (dynamic l., etc.)
 \end{itemize}
\item Give it multialgebra/evolving-algebra/imperative semantics ?
\item Generatedness -- any universal assumption of? Shouldn't all stuff involve explicit
distinction destructors vs. constructors -- Ds for observation, Cs for construction? 
E.g., \begin{itemize}
\item[{\bf Z3.}] $x\Incl \ch.V(G),\ I(G,x)\Eq 0\impl Z(G)\Incl Z(RemV(G,x))\cup x$
\item[{\bf U3.}] $ zy\Incl \ch.E(G),\ z\notEq x\impl U(G,x)\Incl 
   AddE(U(RemE(G, zy)),\  zy)$
  \end{itemize}
\item Would like to get, (semi)automatically
 \begin{itemize}\MyLPar
 \item complexity of the algorithm
 \item static properties of functions (as {\bf U?}, {\bf Z?} etc.)
 \item usual logical means to reason about the formulae
 \end{itemize}
\end{itemize}


\end{document}
