% file name: `njcdoc.tex'
\documentstyle{njcarticle}

\titlehead{Instruction to us njc style files}
\authorhead{Raatikainen}

\title{INSTRUCTIONS TO USE NJC STYLE FILES}

\author{KIMMO E.~E. RAATIKAINEN\\
University of Helsinki, Department of Computer Science\\
P.O.~Box 26 (Teollisuuskatu 23), FIN-00014 University of Helsinki, Finland\\
{\tt Kimmo.Raatikainen@cs.Helsinki.FI}}

\usenjcepsmacros
\njccitationstyle
\dates{Received August 1994. Revised September, October 1994.}
\twolevelnumbering{equation,figure}
\draft
\begin{document}

\maketitle

\begin{abstract}
Each paper must include an abstract that summarizes the paper.
Recommended length is at most 150 words.
The abstract should not contain any references or displayed equations.
\LaTeX-environment is \verb|abstract|.
\end{abstract}

\begin{keywords}
List of keywords (4--6) is optional. \LaTeX-environment is \verb|keywords|.
\end{keywords}

\begin{subject}
Each paper should provide the CR Classification
(see the appropriate issue of {\it Computing Reviews\/}).
\LaTeX-environment is \verb|subject|.
\end{subject}

\section{Introduction}

{\sc Nordic Journal of Computing} encourages 
the authors to use \LaTeX\ in preparing the manuscripts with the
\LaTeX\  style file \verb|njcarticle.sty| and the
Bib\TeX\ style file \verb|njcarticle.bst|.
The figures must be either in the \LaTeX\ {\tt picture} environment
or in Encapsuled Postscript.

This document describes how the \LaTeX\ style file {\tt njcarticle.sty}
and the Bib\TeX\ style file {\tt njcarticle.bst} are to be used.
The files can be obtained through anonymous {\tt ftp}, as follows:
\begin{verbatim}
   %ftp ftp.cs.Helsinki.FI
   Name: anonymous
   Password: <your email address>
   ftp> cd pub/Nordic_Journal_of_Computing/Style_Files
   ftp> get njcarticle.sty
   ftp> get njcepsmacros.sty
   ftp> get njcarticle.bst
   ftp> get njcdoc.tex
   ftp> get njcdoc.bib
   ftp> get njclogo.ps
   ftp> quit
\end{verbatim}

The \LaTeX-file of an accepted paper, in the final form, should
be send through email to \verb|njc@cs.Helsinki.FI|.
If the paper contains Encapsuled Postscript files
and/or Bib\TeX\ is used send the files separately and an email
that explains the files sent.

\section{General}

The \LaTeX\ style file {\tt njcarticle.sty} does not accept
any options.
In \verb|\documentstyle|-command all options given in brackets
are simply omitted.
The layout of page should not be changed.

Redundant spaces ought to be minimized
by careful arrangement of tables and figures.
Read your {\tt .log} file carefully.
There should be no \verb!Overfull \hbox! (as here)
and certainly no visible one (more than {\tt 1pt}).
If necessary, reword the text.
The preamble command \verb|\draft| can be used to produce
a visible overfullrulebox in the margin.

The style file provides macros to create running heads.
The \verb!\authorhead!\ contains the authors' surnames as ``Author'',
``Author1, Author2'', and
``Author1, Author2, Author3'' for up to three authors,
and ``Author1 et~al.'' for four or more authors.
The \verb!\titlehead!
contains a short form of the title, not more than 30 characters.

The argument of \verb|title|-command must be written in capitals.
When the paper has more than one author, the authors in the argument of
\verb|author|-command are separated by usual \verb|\and|-command or 
by \verb|\AND|-command that inserts vertical glue between the blocks
of authors'  names, affiliations, etc.

Neither footnotes nor appendices should be used.
However, if appendices are really necessary, their place
is after {\bf Acknowledgements} and before {\bf References}.

\section{Sectioning, numbering, etc.}

\subsection{Sectioning}

The following three \LaTeX-sectioning commands are available:
\verb|\section|, \verb|\subsection|, and \verb|\subsubsection|.

\subsection{Numbering}
\label{numbering}

The numbering of displayed equations, theorems,
figures, tables, and other ``numbered'' environments
follows one of two styles:
either consecutive in each section,
or consecutive through the whole paper (default).
If you prefer the first style, the style file provides
\verb|\twolevelnumbering|-command.
The argument of the command is a list of environments that are
numbered consecutively in each section, e.g.
\begin{verbatim}
    \twolevelnumbering{figure,equation,theorem}
\end{verbatim}

Tables should be referred to ``Table I''.
Equations and figures should be referred to in abbreviated forms:
``\eqnref{eqn:example}'' and ``\figref{fig:example}''.
Use macros \verb|\eqnref{|{\em eqn-label\/}\verb|}|,
\verb|\figref{|{\em fig-label\/}\verb|}|, and
\verb|\tableref{|{\em tab-label\/}\verb|}|
to generate the references.

\subsection{Lists of items}

The depths of \LaTeX-environments \verb|itemize| and \verb|enumerate|
are restricted to two.

\subsection{Spacing before and after environments}

Extra space is added at the top of list if the input file
has a blank line before any list-making environment.
The vertical space after the environment is the same as the
one preceding it.
The list-making environments are: {\tt quote}, {\tt quotation},
{\tt verse}, {\tt itemize}, {\tt enumerate}, {\tt description},
{\tt center}, {\tt flushleft}, and {\tt flushright},
as well as the theorem-like environments.

\subsection{Tabbing environment}

The style file defines \verb|\tabbingstretch| to specify the strut
to be used in the {\tt tabbing} environment. The functionality of
\verb|\tabbingstretch| is the same as that of \verb|\arraystretch|
in {\tt array} and {\tt tabular} environments.
\section{Theorem-like environments}

The style file defines the following theorem-like environments:
\begin{center}
\begin{tabular}{llll}
\verb|theorem|	& \verb|proposition|	& \verb|claim|	& \verb|fact| \\

\verb|lemma|	& \verb|definition|	& \verb|problem|& \verb|remark| \\

\verb|corollary|& \verb|conjecture|	& \verb|example|&  \verb|observation|
\end{tabular}
\end{center}
These are ``numbered'' environments.
The style file also defines the corresponding ``unnumbered''
environments: \verb|theorem*|, ..., \verb|observation*|.
The proofs can be typed within the environment \verb|proof|.

For example, the \LaTeX\ source
\begin{verbatim}
\begin{theorem}[Chebychev's Inequality]
If $X$ is any random variable, then
\begin{equation}
\Pr[|X|\ge a] \le \mbox{\rm E}(X^2)/a^2\;.
\label{eqn:example}
\end{equation}
\end{theorem}
\end{verbatim}
produces
\begin{theorem}[Chebychev's Inequality]
If $X$ is any random variable, then
\begin{equation}
\Pr[|X|\ge a] \le \mbox{\rm E}(X^2)/a^2\;.
\label{eqn:example}
\end{equation}
\end{theorem}


\begin{proof}
If $F(x)$ denotes the distribution function of the random variable $X$,
then
\[
\Pr[|X|\ge a] = \int_{|x|\ge a} \;d\:F(x)\;.
\]
Since in the region of integration $|x|/a\ge1$, it follows that
\[
\int_{|x|\ge a} \;d\:F(x) \le \frac{1}{a^2}
\int_{|x|\ge a} x^2\;d\:F(x)\;.
\]
By extending the integration to all values of $x$,
we merely strengthen the inequality:
\[
\int_{|x|\ge a} \;d\:F(x) \le
\frac{1}{a^2} \int_{|x|\ge a} x^2\;d\:F(x) \le
\frac{1}{a^2} \int x^2\;d\:F(x) = 
\mbox{\rm E}(X^2)/a^2\:.\; \qed
\]
% The following empty line is necessary

\end{proof}

When a proof ends with a displayed equation as above,
the box ``\qed'' should be at the right end of the formula
rather than at the beginning of the next line.
In this case, the box must be inserted through
\verb|\qed| at the end of the equation, and a blank line
must be left between the closing of the equation and the \verb|\end{proof}|.
(See the source of these instructions.)

\section{Figures and tables}

\begin{figure}[t]
\makeepsbox{njclogo.ps}
\caption{An example.}
\label{fig:example}
\end{figure}

Figures and tables are to be inserted in the text nearest their first reference.
They should be arranged so as not to cause an excessive amount of
blank space on the remainder of the page. 

The captions are centered below the figures and above the tables.
If a table needs to extend over to a second page, the continuation of the
table should have a caption: ``Table II {\em (cont.)}''.
Macro \verb|\continued| generates this caption.
Example:
\begin{quote}
   \verb|\begin{table}|\\
   \verb|\continued   % Instead of \caption|\\
   \verb|   ...|\\
   \verb|\end{table}|
\end{quote}

\figref{fig:example} is an example of inserting an Encapsuled Postscript file.
The style file provides the command \verb|\usenjcepsmacros| to read
the option file {\tt njcepsmacros.sty}.
The Encapsuled Postscript file is inserted using the command:
\begin{quotation}\em
\verb|\makeepsbox[|pos\/\verb|]{|filename\/\verb|}[|scaling\verb|]|
\end{quotation}
The optional argument {\em pos\/} tells how the box is placed on the page:\\
\begin{tabular}{cl}
\tt c &  centered (default)\\
\tt l & leftmargin justified \\
\tt r & rightmargin justified
\end{tabular}

The optional argument {\em scaling\/} tells
how the eps-file is to be scaled.
The possible values of the argument are:\\
\begin{tabular}{ll}
\tt scale=<factor> &  scaling factor = {\tt<factor>/1000}\\
\tt width=<dimen> & scaling factor = {\tt<dimen>/<natural\_width>} \\
\tt height=<dimen> & scaling factor = {\tt<dimen>/<natural\_height>} 
\end{tabular}

Since the \TeX-command \verb|\special| is used, all device drivers
may not understand the format.
If your driver has problems, you can use \verb|\epsdraft|-com\-mand
in the preamble.
This command changes the \verb|\makeepsbox|-com\-mand
to omit the \verb|\special|-com\-mand.
In this case the only visible effect is reservation of space.

Observe that an Encapsuled Postscript file must begin with
``\verb|%!|'' and must contain a commentline
\begin{verbatim}
   %%BoundingBox: llx lly urx ury
\end{verbatim}
If the BoundingBox-comment is not found, the file is omitted.

Remember that {\bf previously published material must be accompanied by written
permission from the author and publisher!}

\section{References}
The preferred style of referring to the bibliography is:
``\cite{Raatikainen-1993} proposed to use the Bonferroni inequality
[see \shortcite[pp.~41--43]{Kleijnen-1987};
\shortcite[Ch.~9.4]{Law-Kelton-1991}].''

Note that the
years are in brackets in the running text, but without brackets if the
reference itself is in brackets.
For three or more authors, use {\em et~al.}
Several papers in the same year are distinguished
as ``\shortcite{Raatikainen-1994a}, \shortcite{Raatikainen-1994b}''.

The second possible style is to use numbered references
and the standard \LaTeX-environment \verb|thebibliography|.

The style file provides the command \verb|\njccitationstyle|
to activate the preferred citation style.
The following macros are defined to simplify the preferred style:
\begin{center}\begin{tabular}{ll}
 macro & procudes\\ \hline
 \verb|\cite{bibreflabel}| & ``Author [Year]''\\ 
 \verb|\cite[note]{bibreflabel}| & ``Author [Year, note]''\\ 
 \verb|\shortcite{bibreflabel}| & ``Author Year''\\ 
 \verb|\shortcite[note]{bibreflabel}| & ``Author Year, note''\\ 
 \verb|\citeauthor{bibreflabel}| & ``Author''\\ 
 \verb|\citeyear{bibreflabel}| & ``Year''\\ 
 \verb|\bracketcite{bibreflabel}| & ``[Author Year]''\\ 
 \verb|\bracketcite[note]{bibreflabel}| & ``[Author Year, note]''\\ 
 \verb|\citemul{bibreflabel1,bibreflabel2}| & ``Author [Year1, Year2]''\\ \hline
\end{tabular}\end{center}

The macros above assume that your bibliography items are written as
\begin{verbatim}
   \bibitem[{Author}{Year}]{bibreflabel}
   \bibitem[{Author}{Year1}]{bibreflabel1}
   \bibitem[{Author}{Year2}]{bibreflabel2}
\end{verbatim}
The Bib\TeX\ style file {\tt njcarticle.bst} generates
the bibliography items in the format above.

{\sc Observe} that the \verb|\citemul|-command assumes that
the authors of the references are one person (or same persons):
\citemul{Raatikainen-1993,Raatikainen-1994a,Raatikainen-1994b}.

\section{Use of the style files}

{\sc Nordic Journal of Computing} provides the following files:
\begin{enumerate}
\item
{\tt njcarticle.sty} contains the \LaTeX\ code for producing camera-ready
	output accepted by {\sc Nordic Journal of Computing}.
\item
{\tt njcepsmacros.sty} contains the \LaTeX\ code for including Encapsuled
	Postscript in the camera-ready output.
\item
{\tt njcarticle.bst} contains the Bib\TeX\ code for producing a reference list
suitable for {\sc Nordic Journal of Computing}.
\item
{\tt njcdoc.tex} contains the \LaTeX-source of these instructions.
\item
{\tt njcdoc.bib} contains the Bib\TeX-source of these instructions.
\item
{\tt njclogo.ps} contains the Encapsuled Postscript file used in
	these instructions.
\end{enumerate}

To format the document, type:
\begin{verbatim}
     latex njcdoc 
     bibtex njcdoc
     latex njcdoc
     latex njcdoc
\end{verbatim}
 
\section{Disclaimer}
The macros are not guaranteed to be free of errors.
Any bugs, inconsistencies, suggestions, and other comments
should be reported to {\sc Nordic Journal of Computing} by
email to {\tt njc@cs.Helsinki.FI}.

\begin{acknowledgements}
This section comes before the References and is unnumbered.
\LaTeX-en\-viron\-ment is \verb|acknowledgements|.
\end{acknowledgements}

\bibliographystyle{njcarticle}
\bibliography{njcdoc}

\end{document}

